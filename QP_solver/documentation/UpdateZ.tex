\documentclass[a4paper]{article}
%\usepackage{html}
\usepackage[dvips]{graphics,color,epsfig}
\usepackage{path}
\usepackage{amssymb}
\usepackage{amsfonts}
\usepackage{amsmath}
\usepackage{amsthm}
\usepackage{psfrag}
\newcommand{\N}{\ensuremath{\mathbb{N}}}
\newcommand{\F}{\ensuremath{\mathbb{F}}}
\newcommand{\Z}{\ensuremath{\mathbb{Z}}}
\newcommand{\R}{\ensuremath{\mathbb{R}}}
\newcommand{\Q}{\ensuremath{\mathbb{Q}}}
\newcommand{\C}{\ensuremath{\mathbb{C}}}

\newtheorem{lemma}{Lemma}
\newtheorem{assumption}{Assumption}
\newtheorem{definition}{Definition}


\title{Update $Uz$}
\author{Frans Wessendorp}
\begin{document}
\maketitle
\section{The necessity for an additional update}
Suppose in Ratio Test Step~1 variable $x_{i}$ is leaving and variable $x_{j}$ is
entering. If $M_{B \setminus \{i\}}$ is not regular, we no longer have a unique
solution to $(UQP_{j}^{t}(\hat{B} \setminus \{i\}))$ for each value of $t$.
Currently the algorithm then proceeds with Ratio Test Step~2 with
$\hat{B}:=B \cup \{j\}$ and variable $x_{i}$ leaving in the first ratio test in
Ratio Test Step~2. However, this does not always work, for
$B^{\prime}:=B \setminus \{i\} \cup \{j\}$, $M_{B^{\prime}}$ may be regular
whereas $M_{B \cup \{j\}}$ and $M_{B \setminus \{i\}}$ are both singular, such
that both ways to compute the new basis inverse $M_{B^{\prime}}^{-1}$ from
$M_{B}^{-1}$ by `growing' and `shrinking' updates,
that add or remove one column and row per update,
are blocked. The simplest way
to solve this problem is to have a replacement step in phaseII as well. 

\section{Update type z}
We use an inversion formula described in~\cite{Zielke} which expresses $(M +
UV^{T})^{-1}$ in terms of $M^{-1}$, $U$ and $V$, where $UV^{T}$ has at
most rank 2 and $M$ as well as $M+UV^{T}$ are both symmetric.
We use the general inversion formula
given in~\cite{Zielke}, Equation~(3), and specialize it to our specific
situation in subsequent sections:
\begin{equation}
\label{eq:inversion_formula}
\left(M + UV^{T}\right)^{-1} =
M^{-1} - M^{-1} U \left(I_{r} +
V^{T}M^{-1}U\right)^{-1}V^{T}M^{-1}
\end{equation}
where
\begin{eqnarray}
\label{def:U}
U &:=& \left(u- u_{i}e_{i}, e_{i} \right) \\
\label{def:V}
V &:=& \left(e_{i}, u \right)
\end{eqnarray}
and 
\begin{equation}
\label{def:diff_column}
u:=
\zeta
-
M_{\bullet, i}
\end{equation}
with $\zeta$ being the new column (row).
Thus the correction matrix $C$ has the following form
\begin{equation}
C:=UV^{T}=ue_{i}^{T} + e_{i}u^{T} - u_{i}e_{i}e_{i}^{T}.
\end{equation}
Plugging this into Equation~(\ref{eq:inversion_formula}) yields
\begin{equation}
\label{eq:inversion_formula_spec}
\left(M + UV^{T}\right)^{-1} =
M^{-1} - M^{-1}\left(u - u_{i}e_{i}, e_{i}\right)
Q^{-1}
\left(
\begin{array}{c}
e_{i}^{T} \\
\hline
u^{T}
\end{array}
\right)
M^{-1}
\end{equation}
where 
\begin{eqnarray}
Q &:=& I_{2} +
\left(
\begin{array}{c}
e_{i}^{T}  \\
u^{T}
\end{array}
\right)M^{-1}
\left(u - u_{i}e_{i}, e_{i}\right) \\
&=&
\label{def:Q_explicit}
\left(
\begin{array}{cc}
1 + M^{-1}_{i, \bullet}u - u_{i}M_{i,i}^{-1} &
M_{i,i}^{-1} \\
u^{T}M^{-1}u-u_{i}u^{T}M^{-1}_{\bullet, i} &
1 + u^{T}M^{-1}_{\bullet, i}
\end{array}
\right)
\end{eqnarray}
%and $\beta_{i,i}$ is defined as 
%\begin{equation}
%\label{def:beta}
%\beta_{i,i}:=M^{-1}_{i,i}
%\end{equation}
The inverse of $Q$,
\begin{equation}
Q=
\left(
\begin{array}{cc}
q_{1} & q_{2} \\
q_{3} & q_{4}
\end{array}
\right)
\end{equation}
is given by
\begin{equation}
Q^{-1}=
\frac{1}{\det(Q)}
\left(
\begin{array}{cc}
q_{4}  & -q_{2} \\
-q_{3} &  q_{1}
\end{array}
\right)
\end{equation}
with
\begin{eqnarray}
\label{def:det_Q}
\det(Q)
&=&
q_{1}q_{4}-q_{2}q_{3}
\nonumber \\
&=&
\left(1+M_{i, \bullet}^{-1}u-u_{i}M_{i,i}^{-1}\right)
\left(1 + u^{T}M^{-1}_{\bullet, i}\right)
\nonumber \\
&&
-M_{i,i}^{-1}\left(u^{T}M^{-1}u-u_{i}u^{T}M_{\bullet,i}^{-1}\right)
\nonumber \\
&=&
\left(1+M_{i, \bullet}^{-1}u\right)^{2}
-u_{i}M_{i,i}^{-1}\left(1+u^{T}M_{\bullet,i}^{-1}\right)
-M_{i,i}^{-1}u^{T}M^{-1}u
\nonumber \\
&&
+ u_{i}M_{i,i}^{-1}u^{T}M_{\bullet,i}^{-1}
\nonumber \\
&=& \left(1+M_{i, \bullet}^{-1}u\right)^{2} -
    M_{i,i}^{-1}\left(u_{i}+ u^{T}M^{-1}u\right)
\end{eqnarray}
Collecting terms in Equation~(\ref{eq:inversion_formula_spec}) then yields
\begin{eqnarray}
\label{eq:inversion_formula_spec_explicit}
\lefteqn{\left(M+ UV^{T}\right)^{-1}=M^{-1} -} \nonumber \\
& & \frac{1}{\det(Q)}
\left[
  M^{-1}_{\bullet, i}
  \left(k_{1}v^{T} + k_{2}M^{-1}_{i, \bullet}\right)
  +v\left(k_{3}v^{T} + k_{1}M^{-1}_{i, \bullet}\right)
\right]
\end{eqnarray}
where
\begin{eqnarray}
\label{def:v}
v     &:=& M^{-1}u \\
\label{def:w}
w     &:=& M^{-1}_{i, \bullet}u \\
\label{def:k1}
k_{1} &:=& 1 + w \\
\label{def:k2}
k_{2} &:=& -\left(u^{T}v + u_{i}\right) \\
\label{def:k3}
k_{3} &:=& -M_{i,i}^{-1} 
\end{eqnarray}
With these definitions $\det(Q)$ can then be written in terms of
$k_{1}$, $k_{2}$ and $k_{3}$ as
\begin{equation}
\label{def:det_Q_k}
\det(Q)=k_{1}^{2}-k_{2}k_{3}
\end{equation}
In order to get rid of the difference $u$
in Equation~(\ref{eq:inversion_formula_spec_explicit}) and
Definitions~(\ref{def:v}) to~(\ref{def:k3}) we rewrite the latter ones
\begin{eqnarray}
\label{def:v_zeta}
v
&:=&
M^{-1}u
\nonumber \\
&=& 
M^{-1}\left[\zeta - M_{\bullet, i}\right]
\nonumber \\
&=&
M^{-1}\zeta - e_{i}
\end{eqnarray}
\begin{eqnarray}
\label{def:w_zeta}
w
&:=&
M^{-1}_{i, \bullet}u
\nonumber \\
&=&
M^{-1}_{i, \bullet}\left[\zeta - M_{\bullet, i}\right] 
\nonumber \\
&=&
M^{-1}_{i, \bullet}\zeta - 1
\end{eqnarray}
\begin{eqnarray}
\label{def:k_1_zeta}
k_{1}
&:=&
1 + w
\nonumber \\
&=&
M^{-1}_{i, \bullet}\zeta
\end{eqnarray}
\begin{eqnarray}
k_{2} &:=& -\left(u^{T}v + u_{i}\right) \nonumber \\
      &=& -\left[\left(\zeta-M_{\bullet,i}\right)^{T}
           M^{-1}\left(\zeta-M_{\bullet,i}\right)
	   + \left(\zeta-M_{\bullet,i}\right)_{i}
	   \right]   \nonumber \\
      &=& -\left[\zeta^{T}M^{-1}\zeta
        +M_{i,\bullet}M^{-1}M_{\bullet,i}
	          -2\zeta^{T}M^{-1}M_{\bullet,i} 
         	  +\zeta_{i} - M_{i,i}
	   \right] \nonumber \\
      &=& -\left[\zeta^{T}M^{-1}\zeta + M_{i,i}
                  -2\zeta_{i} + \zeta_{i} - M_{i,i}
            \right] \nonumber \\
      &=& -\zeta^{T}M^{-1}\zeta + \zeta_{i}
\end{eqnarray}

The symmetry of the right-hand side of
Equation~(\ref{eq:inversion_formula_spec_explicit}) becomes obvious when
rewriting it as the sum of three symmetric matrices
\begin{eqnarray*}
\lefteqn{\left(M+ UV^{T}\right)^{-1}=M^{-1}-}  \\
& & \frac{1}{\det(Q)}
\left[
k_{1}\left(
  M^{-1}_{\bullet, i}v^{T}
  +vM^{-1}_{i,\bullet}
\right)
+
k_{2}M^{-1}_{\bullet, i}M^{-1}_{i, \bullet}
+
k_{3}vv^{T}
\right]
\end{eqnarray*}

In case of a regular matrix $M+UV^{T}$ the determinant $\det(Q)$ is
always nonzero as the factorization of 
\begin{equation*}
\left(
\begin{array}{cc}
I & -V^{T} \\
U & M
\end{array}
\right)
\end{equation*}
into the products 
\begin{equation*}
\left(
\begin{array}{cc}
I & 0 \\
U & M + UV^{T}
\end{array}
\right)
\left(
\begin{array}{cc}
I & -V^{T} \\
0 & I
\end{array}
\right)
=
\left(
\begin{array}{cc}
I & -V^{T}  \\
0 & M
\end{array}
\right)
\left(
\begin{array}{cc}
I + V^{T}M^{-1}U & 0 \\
M^{-1}U          & I
\end{array}
\right)
\end{equation*}
shows.
By taking determinants we obtain
\begin{equation}
\label{eq:det_prod}
\det\left(M + UV^{T}\right) =
\det\left(M\right) \det\left(I + V^{T}M^{-1}U\right)
\end{equation}
by assumption $M$ and $M+UV^{T}$ are regular, therefore by the above
Equation
$\det(I + V^{T}M^{-1}U)=\det(Q) \neq 0$.

A determinant identity that will be useful in the following is subsumed in the
following lemma
\begin{lemma}[Jacobi's determinant identity]
\label{lem:Jacobi_det_id}
Let
\begin{equation}
A:=
\left(
\begin{array}{c|c}
B & C \\
\hline
D & E
\end{array}
\right)
, \quad
A^{-1}:=
\left(
\begin{array}{c|c}
W & X \\
\hline
Y & Z
\end{array}
\right)
\end{equation}
where $B$ and $W$ are $k \times k$ matrices then $\det(B)=\det(A)\det(Z)$.
\end{lemma}
The justification for the new additional update is given by the following
lemma.
\begin{lemma}
\label{lem:main}
If $M_{B}$ is regular with entering variable $x_{j}$, $j \not \in B$, and
leaving variable $x_{i}$, $i \in B$, then if $M_{B \setminus \{i\}}$ 
is singular $M_{B \setminus \{i\} \cup \{j\}}$ is regular.
\end{lemma}

\begin{proof}
Suppose $M_{B \setminus \{i\} \cup \{j\}}=M_{B}+UV^{T}$ with
$U$, $V$ defined according to Equations~(\ref{def:U}), (\ref{def:V})
and (\ref{def:diff_column}) to be singular, then according to
Equation~(\ref{eq:det_prod}), the regularity of $M_{B}$ and
Equation~(\ref{def:det_Q})
\begin{equation}
\label{eq:det_Q_eq_det_M_B_replaced}
\det \left(I+V^{T}M_{B}^{-1}U\right) =
\det(Q) = 0
\Longleftrightarrow
\det(M_{B \setminus \{i\} \cup \{j\}})=0
\end{equation}
By Equation~(\ref{def:det_Q_k})
\begin{equation}
\label{def:det_Q_k_proof}
\det(Q) = k_{1}^{2} - k_{2}k_{3}
\end{equation}
where
\begin{equation}
\label{def:ks}
k_{1}:=\left(M_{B}^{-1}\right)_{\beta(i), \bullet}\zeta,
\quad
k_{2}:=\zeta_{\beta(i)} - \zeta^{T}M_{B}^{-1}\zeta,
\quad
k_{3}:=-\left(M_{B}^{-1}\right)_{\beta(i), \beta(i)}
\end{equation}
By the assumption of variable $x_{j}$ entering and variable $x_{i}$ leaving
\begin{equation}
\label{precond:pivot_proof}
\left(M_{B}^{-1}\right)_{\beta(i), \bullet}
\zeta
=
q_{x_{i}}
\neq
0
\end{equation}
If the assumption $\det(M_{B\setminus \{i\}})=0$ holds true
we have, according to Lemma~(\ref{lem:Jacobi_det_id}) for $M_{B}$ and
the regularity of $M_{B}$, 
\begin{equation}
\label{precond:leaving_proof}
\det(M_{B \setminus \{i\}})=0
\Longleftrightarrow
\left(M_{B}^{-1}\right)_{\beta(i),\beta(i)}=0
\end{equation}
such that by Equations~(\ref{def:ks}), (\ref{precond:pivot_proof})
and~(\ref{precond:leaving_proof}) $\det(Q) \neq 0$ in
Equation~(\ref{def:det_Q_k_proof}).
%If on the other hand the assumptions $\det(M_{B \cup \{j\}})=0$ and
%$D_{j,j}=D_{i,j}$ hold true, we have according to \cite{Sven},
%Equation (6.7),
%\begin{equation*}
%\det(M_{B \cup \{j\}}) = 0
%\Longleftrightarrow
%2D_{j,j} -\zeta^{T}M_{B}^{-1}\zeta = 0
%\end{equation*}
%which together with the assumption $D_{i,j}=D_{j,j}$ and
%the fact $\zeta_{\beta(i)}=2D_{i,j}$ is equivalent to
%\begin{equation}
%\label{precond:entering_proof}
%\det(M_{B \cup \{j\}}) = 0
%\Longleftrightarrow
%\zeta_{\beta(i)} -\zeta^{T}M_{B}^{-1}\zeta = 0
%\end{equation}
%such that by Equations~(\ref{def:ks}), (\ref{precond:pivot_proof})
%and~(\ref{precond:entering_proof}) again $\det(Q) \neq 0$
%in Equation~(\ref{def:det_Q_k_proof}).
%For both cases (\ref{eq:det_Q_eq_det_M_B_replaced}) implies the lemma.
\end{proof}

\section{Update type z in the context of reduced bases}
We distinguish 4 different cases with respect to the entering and leaving
variables and the respective changes of the basis headings they entail:
\begin{description}

\item [$\mathbf{U_{Z_{1}}}$:$\quad j \in O \setminus B$ replaces $i \in B_{O}$]
\begin{equation}
\label{update:o_rep_o}
\begin{array}{ccccccc}
C^{\prime}      &:=&  C     &\quad\quad&
S_{B}^{\prime}  &:=&  S_{B}  \\
B_{O}^{\prime}  &:=&  B_{O} \setminus \{i\} \cup \{j\}  &\quad\quad&
B_{S}^{\prime}  &:=&  B_{S}
\end{array}
\end{equation}

\item [$\mathbf{U_{Z_{2}}}$:$\quad j \in S \setminus B$ replaces $i \in B_{O}$:]
\begin{equation}
\label{update:s_rep_o}
\begin{array}{ccccccc}
C^{\prime}      &:=& C \setminus \{\sigma(j)\}   &\quad\quad&
S_{B}^{\prime}  &:=& S_{B} \cup  \{\sigma(j)\}   \\
B_{O}^{\prime}  &:=& B_{O} \setminus \{i\} &\quad\quad&
B_{S}^{\prime}  &:=&B_{S} \cup \{j\}  
\end{array}
\end{equation}

\item [$\mathbf{U_{Z_{3}}}$:$\quad j \in O \setminus B$ replaces $i \in B_{S}$:]
\begin{equation}
\label{update:o_rep_s}
\begin{array}{ccccccc}
C^{\prime}      &:=& C \cup \{\sigma(i)\}   &\quad\quad&
S_{B}^{\prime}  &:=& S_{B} \setminus  \{\sigma(i)\}   \\
B_{O}^{\prime}  &:=& B_{O} \cup \{j\} &\quad\quad&
B_{S}^{\prime}  &:=&B_{S} \setminus \{i\}  
\end{array}
\end{equation}

\item [$\mathbf{U_{Z_{4}}}$:$\quad j \in S \setminus B$ replaces $i \in B_{S}$:]
\begin{equation}
\label{update:s_rep_s}
\begin{array}{ccccccc}
C^{\prime}      &:=& C \setminus \{\sigma(j)\} \cup \{\sigma(i)\}  &\quad\quad&
S_{B}^{\prime}  &:=& S_{B} \setminus  \{\sigma(i)\} \cup \{\sigma(j)\}   \\
B_{O}^{\prime}  &:=& B_{O}                                         &\quad\quad&
B_{S}^{\prime}  &:=&B_{S} \setminus \{i\} \cup \{j\} 
\end{array}
\end{equation}
\end{description} 
Note that with respect to the reduced basis matrix $\check{M}_{B}$
\begin{equation}
\label{def:M_B_red}
\check{M}_{B}:=
\left(
\begin{array}{c|c}
0 & A_{C, B_{O}} \\
\hline
A_{C, B_{O}}^{T} & 2D_{B_{O}, B_{O}}
\end{array}
\right)
\end{equation}
only the Updates~(\ref{update:o_rep_o}) and~(\ref{update:s_rep_s}) are
replacement updates, whereas the Updates~(\ref{update:s_rep_o})
and~(\ref{update:o_rep_s}) are `shrinking' and `growing' updates.
Since the solver uses the reduced basis inverse $\check{M}_{B}^{-1}$ we can
directly apply the update described in the last section only for
Updates~(\ref{update:o_rep_o}) and~(\ref{update:s_rep_s}), although we could
indirectly apply last section's update by expanding the the reduced basis
matrix inverse $\check{M}_{B}^{-1}$ to $M_{B}^{-1}$, apply the update
and shrink back to $\check{M}_{B}^{-1}$ again.
Since we want to work with the reduced basis inverse $\check{M}_{B}^{-1}$
directly we generalize some of the updates described in~\cite{Sven},
Section (6.3.2), in the sense that enlarging the reduced basis matrix by two
rows and columns as well as the removal of two rows and columns of the reduced
basis matrix is possible.

To this end we restate the expression for $M_{B}^{-1}$ in terms of 
$\check{M}_{B}^{-1}$ derived in~\cite{Frans_Deg}, Section~(3.7),
given the basis heading
$\left[C, S_{B}, B_{O}, B_{S}\right]$ as
\begin{equation}
\label{eq:M_B_inv_exp}
M_{B}^{-1}=
\left(\begin{array}{c|c|c|c}
        \left(\check{M}_{B}^{-1}\right)_{C,C} &
	0 &
	\left(\check{M}_{B}^{-1}\right)_{C,B_{O}} &
	 \left(\check{M}_{B}^{-1}\right)_{C, B_{O}}\alpha^{T} \\
	\hline
	0 &
	0 &
	0 &
	A_{S_{B},B_{S}} \\
	\hline
	\left(\check{M}_{B}^{-1}\right)_{B_{O}, C} &
	0 &
	\left(\check{M}_{B}^{-1}\right)_{B_{O}, B_{O}} &
          \left(\check{M}_{B}^{-1}\right)_{B_{O},B_{O}}\alpha^{T} \\
	\hline
	\alpha\left(\check{M}_{B}^{-1}\right)_{B_{O},
	C} &
	A_{S_{B}, B_{S}}^{T} &
	\alpha\left(\check{M}_{B}^{-1}\right)_{B_{O},
	B_{O}} &
	\alpha\left(\check{M}_{B}^{-1}\right)_{B_{O}, B_{O}}\alpha^{T}
      \end{array}
\right)
\end{equation}
where
\begin{equation*}
\alpha=-A_{S_{B}, B_{S}}^{T}A_{S_{B}, B_{O}}
\end{equation*}
Furthermore in order to take into account possible permutations within the
different blocks in the above expressions we introduce the following
permutations:
\begin{equation}
\beta: B_{O} \cup B_{S} \rightarrow B_{O} \cup B_{S} 
\end{equation}
\begin{equation}
\beta_{O}: B_{O}  \rightarrow  B_{O}, \quad
\beta_{S}: B_{S} \rightarrow  B_{S}
\end{equation}
\begin{equation}
\gamma_{C}: E \cup S_{N} \rightarrow  E \cup S_{N}, \quad
\gamma_{S_{B}}: S_{B} \rightarrow  S_{B} 
\end{equation}
where the permutations are defined with respect to the headings in ascending
order.



\subsection{The precondition of the leaving variable}
The precondition of the leaving variable of the update for the basis matrix, 
$\det(M_{B\setminus \{i\}})=0$, is
according to Lemma~(\ref{lem:Jacobi_det_id}) equivalent to 
\begin{equation}
\label{precond:leaving} 
\left(M_{B}^{-1}\right)_{\beta(i),\beta(i)}=0
\end{equation}
For the updates $U_{Z_{1}}$ and $U_{Z_{2}}$ this precondition translates
directly for the reduced basis
matrix $\check{M}_{B}$, for the updates $U_{Z_{3}}$ and $U_{Z_{4}}$ for the
reduced basis matrix $\check{M}_{B}$, however, the translation is a little bit
more involved. Before delving into the details we introduce the following
definitions which will be useful later
\begin{equation}
\label{def:theta_vartheta}
\theta :=
\left(
\begin{array}{c}
A_{C, j} \\
\hline
2D_{B_{O}, j}
\end{array}
\right),
\quad
\vartheta^{T}:=
\left(
\begin{array}{c}
0 \\
\hline
A_{\sigma(i), B_{O}}^{T}
\end{array}
\right)
\end{equation}
\begin{equation}
\label{def:rho_varrho}
\rho:=
\left(\check{M}_{B}^{-1}\right)_{\bullet, \beta_{O}(i)},
\quad
\varrho^{T}:=
\left(\check{M}_{B}^{-1}\right)_{\bullet, \gamma_{C}(\sigma(j))}
\end{equation}
According to Equation~(\ref{eq:M_B_inv_exp})
$\left(M_{B}^{-1}\right)_{\beta(i),\beta(i)}=0$
for $i \in B_{S}$ amounts to
\begin{equation}
\label{eq:precond}
\left(\alpha\left(\check{M}_{B}^{-1}\right)_{B_{O}, B_{O}}
    \alpha^{T}\right)_{\beta_{S}(i), \beta_{S}(i)} = 0
\end{equation}
Simplifying the part involving $\alpha$ in this expression yields
\begin{eqnarray*}
\left(\alpha\right)_{\beta_{S}(i), \bullet} &=&
-\left(A_{S_{B}, B_{S}}^{T}\right)_{\beta_{S}(i), \bullet}A_{S_{B}, B_{O}} \\
&=&
-A_{\sigma(i),i}\left(A_{S_{B},B_{O}}\right)_{\gamma_{S_{B}}(\sigma(i))} \\
&=&
-A_{\sigma(i),i}A_{\sigma(i),B_{O}}
\end{eqnarray*}
such that Equation~(\ref{eq:precond}) becomes 
\begin{equation*}
A_{\sigma(i), B_{O}}\left(\check{M}_{B}^{-1}\right)_{B_{O}, B_{O}}
A_{\sigma(i), B_{O}}^{T} =0
\end{equation*}
Using the Definitions~(\ref{def:theta_vartheta}) this can be written as
\begin{equation}
\label{eq:precond_simplified}
\vartheta\check{M}_{B}^{-1}\vartheta^{T}=0
\end{equation}
Subsuming this for later reference we obtain 
\begin{equation}
\label{table:leaving_precond}
\begin{array}{c|ccc}
 & 
\left(M_{B}^{-1}\right)_{\beta(i), \beta(i)}
&=& 0
 \\
\hline
U_{Z_{1}}
&
\left(\check{M}_{B}^{-1}\right)_{\beta_{O}(i), \beta_{O}(i)}
&=& 0
 \\
U_{Z_{2}}
&
\left(\check{M}_{B}^{-1}\right)_{\beta_{O}(i), \beta_{O}(i)}
&=& 0
 \\
U_{Z_{3}}
&
\vartheta\check{M}_{B}^{-1}
\vartheta^{T}
&=& 0
 \\
U_{Z_{4}}
&
\vartheta\check{M}_{B}^{-1}\vartheta^{T}
&=& 0
\end{array}
\end{equation}

\subsection{The pivoting precondition}
Due to the fact variable $x_{i}$ has been selected to be the leaving variable with $x_{j}$ entering by Ratio Test Step~1 we have
\begin{equation}
\label{precond:pivot}
\left(M_{B}^{-1}\right)_{\beta(i), \bullet}\eta \neq 0
\end{equation}
where $\eta$ denotes the entering column. 
We distinguish according to Equations~(\ref{update:o_rep_o})
to~(\ref{update:s_rep_s}) the following cases:

\paragraph{$\mathbf{U_{Z_{1}}}$: $j \in O \setminus B$ replaces $i \in B_{O}$.}
According to~(\ref{update:o_rep_o}) we have for the entering column $\eta$
\begin{equation}
\eta:=
\left(
\begin{array}{c}
A_{C, j} \\
\hline
A_{S_{B},j} \\
\hline
2D_{B_{O}, j} \\
\hline
0
\end{array}
\right)
\end{equation}
such that the pivoting precondition~(\ref{precond:pivot})
for the basis matrix inverse in terms of
the reduced basis matrix inverse becomes
\begin{eqnarray}
\left(M_{B}^{-1}\right)_{\beta_{O}(i), \bullet} \eta
&=&
\left(\check{M}_{B}^{-1}\right)_{\beta_{O}(i),C}A_{C,j}
+\left(\check{M}_{B}^{-1}\right)_{\beta_{O}(i), B_{O}}2D_{B_{O},j}
\nonumber \\
&=&
\check{M}_{\beta_{O}(i), \bullet}
\left(
\begin{array}{c}
A_{C,j} \\
\hline
2D_{B_{O},j}
\end{array}
\right)
\neq 0,
\nonumber
\end{eqnarray}
using the Definitions~(\ref{def:theta_vartheta})
and~(\ref{def:rho_varrho}) this can be written as
\begin{equation}
\rho^{T}\theta \neq 0
\end{equation}

\paragraph{$\mathbf{U_{Z_{2}}}$: $j \in S \setminus B$ replaces $i \in B_{O}$.}
According to~(\ref{update:s_rep_o}) and the fact that
\begin{equation}
A_{S_{B},j}=0, \quad 2D_{B_{O}, j}=0 \quad j \in S \setminus B
\end{equation}
we have for the entering column $\eta$
\begin{equation}
\eta:=
\left(
\begin{array}{c}
A_{C, j} \\
\hline
0 \\
\hline
0 \\
\hline
0
\end{array}
\right)
\end{equation}
such that the pivoting precondition~(\ref{precond:pivot})
for the basis matrix inverse in terms of
the reduced basis matrix inverse becomes
\begin{eqnarray}
\left(M_{B}^{-1}\right)_{\beta_{O}(i), \bullet} \eta
&=&
\left(\check{M}_{B}^{-1}\right)_{\beta_{O}(i),C}A_{C,j}
\nonumber \\
&=&
A_{\sigma(j),j}
 \left(\check{M}_{B}^{-1}\right)_{\beta_{O}(i), \gamma_{C}(\sigma(j))} \neq 0
\nonumber
\end{eqnarray}
Since $A_{\sigma(j),j} \neq 0$ the last inequality amounts to
\begin{equation}
\left(\check{M}_{B}^{-1}\right)_{\beta_{O}(i), \gamma_{C}(\sigma(j))} \neq 0
\end{equation}

\paragraph{$\mathbf{U_{Z_{3}}}$: $j \in O \setminus B$ replaces $i \in B_{S}$.}
According to~(\ref{update:o_rep_s})
we have for the entering column $\eta$
\begin{equation}
\eta:=
\left(
\begin{array}{c}
A_{C, j} \\
\hline
A_{S_{B},j} \\
\hline
2D_{B_{O},j} \\
\hline
0
\end{array}
\right)
\end{equation}
such that ~(\ref{precond:pivot})
for the basis matrix inverse in terms of
the reduced basis matrix inverse becomes
\begin{eqnarray}
\left(M_{B}^{-1}\right)_{\beta_{S}(i), \bullet} \eta
&=&
\left(\alpha\right)_{\beta_{S}(i), \bullet}
\left(\check{M}_{B}^{-1}\right)_{B_{O},C}A_{C,j}
+\left(A_{S_{B}, B_{S}}^{T}\right)_{\beta_{S}(i), \bullet}A_{S_{B},j}
\nonumber \\
&&
+\left(\alpha\right)_{\beta_{S}(i), \bullet}
 \left(\check{M}_{B}^{-1}\right)_{B_{O}, B_{O}}2D_{B_{O},j}
\nonumber \\
&=&
-A_{\sigma(i),i}A_{\sigma(i),B_{O}}
 \left(\check{M}_{B}^{-1}\right)_{B_{O},C}A_{C,j}
+A_{\sigma(i),i}A_{\sigma(i),j}
\nonumber \\
&&
-A_{\sigma(i),i}A_{\sigma(i),B_{O}}
 \left(\check{M}_{B}^{-1}\right)_{B_{O},B_{O}}2D_{B_{O},j}
\nonumber \\
&=&
A_{\sigma(i),i}
\left[
A_{\sigma(i),j} - A_{\sigma(i), B_{O}}
\left(\check{M}_{B}^{-1}\right)_{B_{O}, \bullet}
\left(
\begin{array}{c}
A_{C,j}  \\
\hline
2D_{B_{O}, j}
\end{array}
\right)
\right]
\neq 0
\nonumber
\end{eqnarray}
Since $A_{\sigma(i), i} \neq 0$ the last inequality above amounts to
\begin{equation}
A_{\sigma(i),j} - A_{\sigma(i), B_{O}}
\left(\check{M}_{B}^{-1}\right)_{B_{O}, \bullet}
\left(
\begin{array}{c}
A_{C,j}  \\
\hline
2D_{B_{O}, j}
\end{array}
\right)
\neq
0,
\nonumber
\end{equation}
using the Definitions~(\ref{def:theta_vartheta}) this can be written as
\begin{equation}
A_{\sigma(i),j} -
\vartheta\check{M}_{B}^{-1}\theta \neq 0
\end{equation}

\paragraph{$\mathbf{U_{Z_{4}}}$: $j \in S \setminus B$ replaces $i \in B_{S}$.}
According to~(\ref{update:o_rep_s})
and the fact that
\begin{equation}
A_{S_{B},j}=0, \quad 2D_{B_{O}, j}=0, \quad j \in S \setminus B
\end{equation}
we have for the entering column $\eta$
\begin{equation}
\eta:=
\left(
\begin{array}{c}
A_{C, j} \\
\hline
0 \\
\hline
0 \\
\hline
0
\end{array}
\right)
\end{equation}
such that the pivoting precondition~(\ref{precond:pivot})
for the basis matrix inverse in terms of
the reduced basis matrix inverse becomes
\begin{eqnarray}
\left(M_{B}^{-1}\right)_{\beta_{S}(i), \bullet} \eta
&=&
\left(\alpha\right)_{\beta_{S}(i), \bullet}
\left(\check{M}_{B}^{-1}\right)_{B_{O},C}A_{C,j}
\nonumber \\
&=&
-A_{\sigma(i),i}A_{\sigma(i),B_{O}}
 \left(\check{M}_{B}^{-1}\right)_{B_{O},C}A_{C,j}
\nonumber \\
&=&
-A_{\sigma(i),i}A_{\sigma(j),j}A_{\sigma(i),B_{O}}
 \left(\check{M}_{B}^{-1}\right)_{B_{O},\gamma_{C}(j)}
 \neq 0
 \nonumber
\end{eqnarray}
Since $A_{\sigma(i), i} \neq 0$ and $A_{\sigma(j),j} \neq 0$
the last inequality above amounts to
\begin{equation}
A_{\sigma(i),B_{O}}
\left(\check{M}_{B}^{-1}\right)_{B_{O},\gamma_{C}(\sigma(j))}
\neq 0,
\nonumber
\end{equation}
using the Definitions~(\ref{def:theta_vartheta})
and~(\ref{def:rho_varrho}) this can be written as
\begin{equation}
\vartheta \varrho^{T} \neq 0
\end{equation}
Subsuming this for later reference we obtain
\begin{equation}
\label{table:pivot_precond}
\begin{array}{c|ccc}
&
\left(M_{B}^{-1}\right)_{\beta(i), \bullet} \eta
&\neq& 0
 \\
\hline
U_{Z_{1}} 
&
\rho^{T} \theta &\neq& 0
 \\
U_{Z_{2}}
&
\left(\check{M}_{B}^{-1}\right)_{\beta_{O}(i), \gamma_{C}(\sigma(j))}
&\neq& 0
 \\
U_{Z_{3}}
&
A_{\sigma(i), j}
-\vartheta\check{M}_{B}^{-1}\theta
&\neq& 0
 \\
U_{Z_{4}}
&
\vartheta \varrho^{T} &\neq& 0
\end{array}
\end{equation}

\section{Updates $U_{Z_{1}} \dots U_{Z_{4}}$}
Since the inverse of a matrix $A$ can be represented as a product of the
cofactor matrix of $A$ and $1/\det(A)$ and we represent $A^{-1}$ in rational
form, we use the following form 
\begin{equation}
\label{def:rat_form}
A^{-1}=\frac{\hat{A}^{-1}}{\left|\det\left(A\right)\right|}
\end{equation}
such that the denominator is by convention positive.

\subsection{Updates $\mathbf{U_{Z_{1}}}$, $\mathbf{U_{Z_{4}}}$ and the rational form}
According to~(\ref{update:o_rep_o}) and~(\ref{update:s_rep_s})
and the Definition~(\ref{def:M_B_red}) of the reduced basis matrix
the update described by Equation~(\ref{eq:inversion_formula_spec_explicit})
can directly be applied to the reduced basis $\check{M}_{B}$
in order to implement
the updates $U_{Z_{1}}$ and $U_{Z_{4}}$. In order to obtain the rational form
in~(\ref{def:rat_form}) for Equation~(\ref{eq:inversion_formula_spec_explicit})
\begin{eqnarray}
\label{def:update_Z1_Z4}
\lefteqn{\left(M+ UV^{T}\right)^{-1}=M^{-1} -} \nonumber \\
& & \frac{1}{\det(Q)}
\left[
  M^{-1}_{\bullet, i}
  \left(k_{1}v^{T} + k_{2}M^{-1}_{i, \bullet}\right)
  +v\left(k_{3}v^{T} + k_{1}M^{-1}_{i, \bullet}\right)
\right]
\end{eqnarray}
where
\begin{equation}
\label{def:update_Z1_Z4_entities}
v:= M^{-1}\zeta-e_{i},
\quad
k_{1}:=M_{i, \bullet}^{-1}\zeta,
\quad
k_{2}:=\zeta_{i} - \zeta^{T}M^{-1}\zeta,
\quad
k_{3}:=-M_{i,i}^{-1}
\end{equation}
and
\begin{equation}
\label{eq:det(Q)}
\det(Q)=k_{1}^{2}-k_{2}k_{3}
\end{equation}
we use the following substitutions
\begin{equation}
M^{-1}=:\frac{\hat{M}^{-1}}{d},
\quad
v=:\frac{\hat{v}}{d},
\quad
k_{1}=:\frac{\hat{k}_{1}}{d},
\quad
k_{2}=:\frac{\hat{k}_{2}}{d},
\quad
k_{3}=:\frac{\hat{k}_{3}}{d}
\end{equation}
where $d$ is defined as
\begin{equation}
\label{def:old_denom_Z1_Z4}
d:=\left|\det(M)\right|
\end{equation}
Substituting in the above equation yields,
if we abbreviate $M+UV^{T}$ by $M_{Z}$,
\begin{eqnarray}
M_{Z}^{-1}&=&\frac{\hat{M}^{-1}}{d}
-
\frac{
\left[
  \hat{M}^{-1}_{\bullet, i}
  \left(\hat{k}_{1}\hat{v}^{T} + \hat{k}_{2}\hat{M}^{-1}_{i, \bullet}\right)
  +\hat{v}\left(\hat{k}_{3}\hat{v}^{T}
  +\hat{k}_{1}\hat{M}^{-1}_{i, \bullet}\right)
\right]
}{\det(Q)d^{3}}
\nonumber \\
&=&
\frac{
\det(Q)d^{2}\hat{M}^{-1}
-
\left[
  \hat{M}^{-1}_{\bullet, i}
  \left(\hat{k}_{1}\hat{v}^{T} + \hat{k}_{2}\hat{M}^{-1}_{i, \bullet}\right)
  +\hat{v}\left(\hat{k}_{3}\hat{v}^{T}
  +\hat{k}_{1}\hat{M}^{-1}_{i, \bullet}\right)
\right]
}{\det(Q)d^{3}}
\nonumber
\end{eqnarray}
Using Definitions~(\ref{def:rat_form})
and~(\ref{def:old_denom_Z1_Z4}) and defining the new denominator $\hat{z}$
according to Equation~(\ref{eq:det_prod})
\begin{equation}
\label{def:new_denom_Z1_Z4}
\hat{z}:=\det(Q)d,
\end{equation}
we finally obtain from the last Equation above
\begin{eqnarray}
\label{def:update_Z1_Z4_denom}
M_{Z}^{-1}
&=&
\frac{\hat{z}d\hat{M}^{-1}
- 
\left[
  \hat{M}^{-1}_{\bullet, i}
  \left(\hat{k}_{1}\hat{v}^{T} + \hat{k}_{2}\hat{M}^{-1}_{i, \bullet}\right)
  +\hat{v}\left(\hat{k}_{3}\hat{v}^{T}
  +\hat{k}_{1}\hat{M}^{-1}_{i, \bullet}\right)
\right]}
{\hat{z}d^{2}}
\nonumber \\
&=&
\frac{sgn(\hat{z})}{\left|\hat{z}\right|}
%\left[
\frac{\hat{z}d\hat{M}^{-1}
- 
\left[
  \hat{M}^{-1}_{\bullet, i}
  \left(\hat{k}_{1}\hat{v}^{T} + \hat{k}_{2}\hat{M}^{-1}_{i, \bullet}\right)
  +\hat{v}\left(\hat{k}_{3}\hat{v}^{T}
  +\hat{k}_{1}\hat{M}^{-1}_{i, \bullet}\right)
\right]}
{d^{2}}
%\right]
\nonumber \\
&=&
\frac{\hat{M}_{Z}^{-1}}{\left|\hat{z}\right|}
\end{eqnarray}

  

\subsection{Update $\mathbf{U_{Z_{1}}}$}
According to~(\ref{update:o_rep_o}) and the
Definitions~(\ref{def:theta_vartheta}) 
and~(\ref{def:rho_varrho}) the
Update~(\ref{def:update_Z1_Z4}) applied to $\check{M}_{B}^{-1}$
leads to the following correspondences
in~(\ref{def:update_Z1_Z4_entities})
\begin{eqnarray}
\label{def:zeta_Z1}
\zeta
&=&
\theta + \left(2D_{j,j}-\theta_{\beta_{O}(i)}\right)e_{\beta_{O}(i)}
= \theta+2\left(D_{j,j}-D_{i,j}\right)e_{\beta_{O}(i)} \\
M_{i, \bullet}^{-1}
&=&
\left(\check{M}_{B}^{-1}\right)_{\beta_{O}(i), \bullet}
= \rho^{T}
\end{eqnarray}
According to table~(\ref{table:leaving_precond})
the leaving variable precondition~(\ref{precond:leaving})
in terms of the reduced basis inverse for $U_{Z_{1}}$ is 
\begin{equation}
\label{precond:leaving_Z1}
\left(\check{M}_{B}^{-1}\right)_{\beta_{O}(i),\beta_{O}(i)}=0 
\end{equation}
which, together with the Definitions~(\ref{def:update_Z1_Z4_entities}),
implies
\begin{equation}
\label{eq:k_3_zero}
k_{3} = 0
\end{equation}
Furthermore, according to table~(\ref{table:pivot_precond}) the
pivoting precondition~(\ref{precond:pivot})
in terms of the reduced basis inverse for $U_{Z_{1}}$ is
\begin{equation*}
\rho^{T}\theta
=
\left(\check{M}_{B}^{-1}\right)_{\beta_{O}(i), \bullet}\theta \neq 0
\end{equation*}
which, together with the leaving variable
precondition~(\ref{precond:leaving_Z1}) and~(\ref{def:zeta_Z1}) is equivalent
to
\begin{equation*}
\rho^{T}\zeta \neq 0,
\end{equation*}  
which in turn together with the Definitions~(\ref{def:update_Z1_Z4_entities})
implies 
\begin{equation}
\label{ineq:Z_1_k_1}
k_{1} \neq 0
\end{equation}
(\ref{ineq:Z_1_k_1}) and~(\ref{eq:k_3_zero}) together with
Equation~(\ref{eq:det(Q)}) imply, in accordance with Lemma~(\ref{lem:main}),  
\begin{equation}
\det(Q)=k_{1}^{2} \neq 0.
\end{equation}
Note that according to the above and
Definition~(\ref{def:new_denom_Z1_Z4}) of the new denominator
$\hat{z}$ we have
\begin{equation}
sgn(\hat{z}) = 1
\end{equation} 
Update~(\ref{def:update_Z1_Z4_denom})
applied to $\check{M}_{B}$ then simplifies to
\begin{eqnarray}
\label{def:update_Z1}
\check{M}_{B^{\prime}}^{-1}
&=&
\frac{1}{\left|\hat{z}\right|}
\frac{
\hat{k}_{1}^{2}\hat{\check{M}}_{B}^{-1}
-\left[
  \hat{\rho}
  \left(\hat{k}_{1}\hat{v}^{T} +\hat{k}_{2}\hat{\rho}^{T}\right)
  +\hat{k}_{1}\hat{v}\hat{\rho}^{T}
 \right]
}{d^{2}}
\nonumber \\
&=&
\frac{1}{\left|\hat{z}\right|}
\frac{
\hat{k}_{1}^{2}\hat{\check{M}}_{B}^{-1}
-\left[
  \hat{k}_{2}\hat{\rho}\hat{\rho}^{T}
  +\hat{k}_{1}\left(\hat{\rho}\hat{v}^{T} +\hat{v}\hat{\rho}^{T}\right)
 \right]
}{d^{2}}
\nonumber \\
&=&
\frac{\hat{\check{M}}_{B^{\prime}}}{\left|\hat{z}\right|}
\end{eqnarray}
where $\hat{z}$ is the new denominator and
\begin{equation}
\label{def:update_Z1_entities}
\begin{array}{rclcrcl}
\hat{\rho}
&:=&
\left(\hat{\check{M}}_{B}^{-1}\right)_{\bullet, \beta_{O}(i)}
&&
\hat{v}
&:=&
\hat{\check{M}}_{B}^{-1}\zeta - d e_{\beta_{O}(i)}
\\
\hat{k}_{1}
&:=&
\hat{\rho}^{T}\zeta
&&
\hat{k}_{2}
&:=&
d\zeta_{\beta_{O}(i)}- \zeta^{T}\hat{\check{M}}_{B}^{-1}\zeta
\\
\hat{z}
&:=&
\hat{k}_{1}^{2}/d
&&
d
&:=&
\left|\det(\hat{\check{M}}_{B})\right|
\end{array}
\end{equation}

\subsection{Update $\mathbf{U_{Z_{4}}}$}
According to~(\ref{update:s_rep_s}) and the
Definitions~(\ref{def:theta_vartheta})
and~(\ref{def:rho_varrho})
the Update~(\ref{def:update_Z1_Z4})
applied to $\check{M}_{B}^{-1}$ leads to the follwing correspondences
in~(\ref{def:update_Z1_Z4_entities})
\begin{eqnarray}
\label{def:zeta_Z4}
\zeta
&=&
\vartheta^{T} 
+ 
\left(
  \left(\check{M}_{B}\right)_{\gamma_{C}(\sigma(i)),\gamma_{C}(\sigma(i))}
  - \vartheta_{\gamma_{C}(\sigma(j))}
\right)
e_{\gamma_{C}(\sigma(j))}^{T}
=
\vartheta^{T} 
\\
M_{i, \bullet}^{-1}
&=&
\left(\check{M}_{B}^{-1}\right)_{\bullet, \gamma_{C}(\sigma(j))}
=
\varrho^{T}
\end{eqnarray}
According to table~(\ref{table:pivot_precond}) the
pivoting precondition~(\ref{precond:pivot})
in terms of the reduced basis inverse for $U_{Z_{4}}$ is together
with~(\ref{def:zeta_Z4})
\begin{equation*}
\vartheta \varrho^{T}= \zeta^{T} \varrho^{T} \neq 0,
\end{equation*}
which together with the Definitions~(\ref{def:update_Z1_Z4_entities}) implies 
\begin{equation}
\label{ineq:Z_4_k_1}
k_{1} \neq 0
\end{equation}
Furthermore according to table~(\ref{table:leaving_precond})
the leaving variable precondition~(\ref{precond:leaving})
in terms of the reduced basis inverse for $U_{Z_{4}}$ is
together with~(\ref{def:zeta_Z4})
\begin{equation*}
\vartheta\check{M}_{B}^{-1}\vartheta^{T}
=
\zeta_{\gamma_{C}(\sigma(j))}-\zeta^{T}\check{M}_{B}^{-1}\zeta= 0 
\end{equation*}
which, again, together with the Definitions~(\ref{def:update_Z1_Z4_entities})
implies
\begin{equation}
\label{eq:k_2_zero}
k_{2} = 0
\end{equation}
(\ref{ineq:Z_4_k_1}) and~(\ref{eq:k_2_zero}) together with
Equation~(\ref{eq:det(Q)}) imply, in accordance with Lemma~(\ref{lem:main}),  
\begin{equation}
\det(Q)=k_{1}^{2} \neq 0
\end{equation}
Note that according to the above and
Definition~(\ref{def:new_denom_Z1_Z4}) of the new denominator
$\hat{z}$ we have
\begin{equation}
sgn(\hat{z}) = 1
\end{equation}  
Update~(\ref{def:update_Z1_Z4_denom})
applied to $\check{M}_{B}$ then simplifies to
\begin{eqnarray}
\label{def:update_Z4}
\check{M}_{B^{\prime}}^{-1}
&=&
\frac{1}{\left|\hat{z}\right|}
\frac{
\hat{k}_{1}^{2}\hat{\check{M}}_{B}^{-1}
-\left[
  \hat{k}_{1}\hat{\varrho}^{T}\hat{v}^{T}
  +\hat{v}\left(\hat{k}_{3}\hat{v}^{T}+\hat{k}_{1}\hat{\varrho}\right)  
 \right]
}{d^{2}}
\nonumber \\
&=&
\frac{1}{\left|\hat{z}\right|}
\frac{
\hat{k}_{1}^{2}\hat{\check{M}}_{B}^{-1}
-\left[
  \hat{k}_{3}\hat{v}\hat{v}^{T}
  +\hat{k}_{1}\left(\hat{\varrho}^{T}\hat{v}^{T}+\hat{v}\hat{\varrho}\right)  
 \right]
}{d^{2}}
\nonumber \\
&=&
\frac{\hat{\check{M}}_{B^{\prime}}^{-1}}{\left|\hat{z}\right|}
\end{eqnarray}
where $\hat{z}$ is the new denominator and
\begin{equation}
\label{def:update_Z4_entities}
\begin{array}{rclcrcl}
\hat{v}
&:=&
\hat{\check{M}}_{B}^{-1}\zeta -de_{\gamma_{C}(\sigma(j))}
&&
\hat{\varrho}^{T}
&:=&
\left(\hat{\check{M}}_{B}^{-1}\right)_{\bullet, \gamma_{C}(\sigma(j))}
\\
\hat{k}_{1}
&:=&
\hat{\varrho}\zeta
&&
\hat{k}_{3}
&:=&
-\left(\hat{\check{M}}_{B}^{-1}\right)_{\gamma_{C}(\sigma(j)),
  \gamma_{C}(\sigma(j))}
\\
\hat{z}
&:=&
\hat{k}_{1}^{2}/d
&&
d
&:=&
\left|\det(\hat{\check{M}}_{B})\right|
\end{array}
\end{equation}


\subsection{Updates $\mathbf{U_{Z_{2}}}$ and $\mathbf{U_{Z_{3}}}$ in general}
According to~(\ref{update:s_rep_o}) and~(\ref{update:o_rep_s}) the Updates
$U_{Z_{2}}$ and $U_{Z_{3}}$ require a basis matrix update that is capable of
growing and shrinking the basis matrix by two columns and rows in one step.
Let 
\begin{equation}
\label{def:U_W}
M_{>>}=
\left(
\begin{array}{c|c}
M & U \\
\hline
U^{T} & W
\end{array}
\right)
\end{equation}
where $(U^{T}, W)$ are the two rows and $(U^{T}, W^{T})^{T}$ the two columns
to be added, with $W$ being the $2 \times 2$ matrix with the entries
they have in common.
\begin{equation}
\label{product}
M_{>>}=
\left(
\begin{array}{ccc|c}
1	&	 &	&        \\
	& \ddots &	& 0      \\
	&	 & 1	&        \\
\hline
	& X^{T}  &      & I_{2}  \\
\end{array}
\right)
\left(
\begin{array}{ccc|c}
	&	 &	&        \\
	& M      &	& 0      \\
	&	 & 	&        \\
\hline
	& 0      &      & Z      \\
\end{array}
\right)
\left(
\begin{array}{ccc|c}
1	&	 &	&        \\
	& \ddots &	& X      \\
	&	 &  1	&        \\
\hline
	& 0      &      & I_{2}  \\
\end{array}
\right)
\end{equation}
where
\begin{equation}
\label{def:X_Z}
X:=M^{-1}U, \quad Z:=W - U^{T}M^{-1}U
\end{equation}
The inverse $M_{>>}^{-1}$ is then given by
\begin{eqnarray}
M_{>>}^{-1}&=&
\left(
\begin{array}{ccc|c}
1	&	 &	&        \\
	& \ddots &	& -X     \\
	&	 &  1	&        \\
\hline
	& 0      &      & I_{2}  \\
\end{array}
\right)
\left(
\begin{array}{ccc|c}
	&	 &	&        \\
	& M^{-1} &	& 0      \\
	&	 & 	&        \\
\hline
	& 0      &      & Z^{-1} \\
\end{array}
\right)
\left(
\begin{array}{ccc|c}
1	&	 &	&        \\
	& \ddots &	& 0      \\
	&	 & 1	&        \\
\hline
	& -X^{T} &      & I_{2}  \\
\end{array}
\right)  
\nonumber \\
&=&
\label{eq:inv_expand_by_2}
\left(
\begin{array}{c|c}
M^{-1}+XZ^{-1}X^{T} & -XZ^{-1}   \\
\hline
-Z^{-1}X^{T}        & Z^{-1}
\end{array}
\right)
\end{eqnarray}
From Jacobi's determinant identity we get 
\begin{equation}
\det
\left(
\begin{array}{c|c}
M & 0 \\
\hline
0 & Z
\end{array}
\right)
=
\det\left(M\right)\det\left(Z\right)
\end{equation}
such that from~(\ref{product}) we obtain
\begin{equation}
\label{eq:det_id_Z2_Z3}
\det\left(M_{>>}\right) = \det\left(M\right)\det\left(Z\right)
\end{equation}

\subsubsection{Update $U_{Z_{3}}$ and the rational form}
In order to obtain the rational form of Definition~(\ref{def:rat_form})
for $M_{>>}^{-1}$ in
Equation~(\ref{eq:inv_expand_by_2}) in terms of $\hat{M}^{-1}$
and its common denominator
\begin{equation}
\label{def:old_denom_Z3}
d:=\left|\det\left(M\right)\right|
\end{equation}
we first rewrite the $2 \times 2$-matrix $Z^{-1}$ in terms of $Z$
\begin{equation}
\label{eq:2_by_2_matrix_inverse}
Z^{-1}=\frac{1}{\det\left(Z\right)}PZP^{T}, \quad
P:=
\left(
\begin{array}{cr}
0 & -1 \\
1 & 0
\end{array}
\right),
\end{equation}
and substitute this in Equation~(\ref{eq:inv_expand_by_2}) yielding
\begin{equation}
\label{eq:inv_expand_by_2_Z}
M_{>>}^{-1}=
\frac{1}{\det\left(Z\right)}
\left(
\begin{array}{c|c}
\det\left(Z\right)M^{-1} + XPZP^{T}X^{T} & -XPZP^{T} \\
\hline
-PZP^{T}X^{T}     & PZP^{T}
\end{array}
\right)
\end{equation}
Using the following substitutions, that seem to be coming out of the blue, 
\begin{eqnarray}
\label{def:X_hat_Z_hat}
X=:\frac{\hat{X}}{d} & & Z=:\frac{\hat{Z}}{d}
\end{eqnarray}
in Equation~(\ref{eq:inv_expand_by_2_Z}) we obtain
\begin{equation}
\label{eq:inv_expand_2_Z3}
M_{>>}^{-1}=
\frac{1}{\det\left(Z\right)d}
\left(
\begin{array}{c|c}
(\det\left(Z\right)d^{3}M^{-1}+\hat{X}P\hat{Z}P^{T}\hat{X}^{T})/ d^{2} &
-\hat{X}P\hat{Z}P^{T}/d  \\
\hline
-P\hat{Z}P^{T}\hat{X}^{T}/d & P\hat{Z}P^{T} 
\end{array}
\right)
\end{equation}
Note that 
\begin{equation}
\label{eq:det_hat_Z}
\det(\hat{Z})=d^{2}\det\left(Z\right).
\end{equation}
Using Definitions~(\ref{def:rat_form}) and~(\ref{def:old_denom_Z3})
and defining the new denominator $\hat{z}$ according to 
Equation~(\ref{eq:det_id_Z2_Z3})
\begin{equation}
\label{def:new_denom_Z3}
\hat{z}:=\det\left(Z\right) d = \det(\hat{Z}) /d
\end{equation}
we finally obtain from Equation~(\ref{eq:inv_expand_2_Z3})
\begin{eqnarray}
M_{>>}^{-1}&=&
\frac{sgn\left(\hat{z}\right)}{\left|\hat{z}\right|}
\left(
\begin{array}{c|c}
(\det(\hat{Z})\hat{M}^{-1}+\hat{X}P\hat{Z}P^{T}\hat{X}^{T})/d^{2} &
-\hat{X}P\hat{Z}P^{T}/d \\
\hline
-P\hat{Z}P^{T}\hat{X}^{T}/d & P\hat{Z}P^{T}
\end{array}
\right) 
\nonumber \\
&=&
\frac{\hat{M}_{>>}^{-1}}{\left|\hat{z}\right|}
\end{eqnarray}
Note that this update works only provided that $\det(Z) \neq 0$.

\subsubsection{Update $U_{Z_{2}}$ and the rational form}
In order to obtain the rational form of Definition~(\ref{def:rat_form})
for $M^{-1}$ in Equation~(\ref{eq:inv_expand_by_2}) in terms of
parts of $\hat{M}_{>>}^{-1}$ and its common denominator
\begin{equation}
\label{def:old_denom_Z2}
d:=\left|\det(M_{>>})\right|
\end{equation}
we multiply Equation~(\ref{eq:inv_expand_by_2}) by the above $d$ 
\begin{equation}
dM_{>>}^{-1}=
\left(
\begin{array}{c|c}
d\left(M^{-1}+XZ^{-1}X^{T}\right) & \hat{\chi}   \\
\hline
\hat{\chi}^{T}                           & \hat{Z}^{-1}
\end{array}
\right)
\end{equation}
where
\begin{equation}
\label{def:chi}
\hat{\chi}:=-dXZ^{-1},
\quad
\hat{Z}^{-1}:=d Z^{-1}
\end{equation} 
and solve for $M^{-1}$ to obtain
\begin{equation}
\label{eq:inv_shrink_by_2_Z}
\left(
\begin{array}{c|c}
M^{-1} & \vdots \\
\hline
\hdots & \bullet
\end{array}
\right)
=
\frac{1}{d}
\left[
\hat{M}_{>>}^{-1}
\left(
\begin{array}{c|c}
dXZ^{-1}X^{T} & \vdots \\
\hline
\hdots        & \bullet
\end{array}
\right)
\right]
\end{equation}
The relevant submatrix $dXZ^{-1}X^{T}$ in the above Equation can be expressed
in terms of the leaving rows and columns $\hat{\chi}$ of the basis inverse
and their intersection $\hat{Z}^{-1}$ using
Equations~(\ref{eq:2_by_2_matrix_inverse}), (\ref{def:chi})
and~(\ref{eq:det_hat_Z})
\begin{eqnarray}
dXZ^{-1}X^{T}
&=&
\left(-dXZ^{-1}\right)\frac{1}{d}Z\left(-dZ^{-1}X^{T}\right) \\
&=&
\frac{1}{\det(Z^{-1})d}
\hat{\chi} PZ^{-1}P^{T}\hat{\chi}^{T} \\
&=&
\frac{1}{\det(\hat{Z}^{-1})}
\hat{\chi} P\hat{Z}^{-1}P^{T}\hat{\chi}^{T}.
\end{eqnarray}
Equation~(\ref{eq:inv_shrink_by_2_Z}) then becomes
\begin{equation}
\label{eq:inv_shrink_by_2_Z2}
\left(
\begin{array}{c|c}
M^{-1}  & \vdots \\
\hline
\hdots  & \bullet
\end{array}
\right)
=
\frac{1}{\det(\hat{Z}^{-1})d}
\left[
\det(\hat{Z}^{-1})\hat{M}_{>>}^{-1}
-
\left(
\begin{array}{c|c}
\hat{\chi} P\hat{Z}^{-1}P^{T} \hat{\chi}^{T} & \vdots \\
\hline
\hdots                           & \bullet
\end{array}
\right)
\right]
\end{equation}
Using Definitions~(\ref{def:rat_form}) and~(\ref{def:old_denom_Z2}) and
defining the new denominator $\hat{z}$ according to
Equation~(\ref{eq:det_id_Z2_Z3})
\begin{equation}
\label{def:new_denom_Z2}
\hat{z}:=\det(Z^{-1})d = \det(\hat{Z}^{-1}) / d
\end{equation}
we finally obtain from Equation~(\ref{eq:inv_shrink_by_2_Z2})
\begin{eqnarray}
\left(
\begin{array}{c|c}
M^{-1}  & \vdots \\
\hline
\hdots  & \bullet
\end{array}
\right)
&=&
\frac{sgn(\hat{z})}{\left|\hat{z}\right|d^{2}}
\left[
\det(\hat{Z}^{-1})\hat{M}_{>>}^{-1}
-
\left(
\begin{array}{c|c}
\hat{\chi} P\hat{Z}^{-1}P^{T} \hat{\chi}^{T} & \vdots \\
\hline
\hdots                           & \bullet	
\end{array}
\right)
\right]
\nonumber \\
&=&
\frac{1}{\left|\hat{z}\right|}
\left(
\begin{array}{c|c}
\hat{M}^{-1} & \vdots \\
\hline
\hdots       & \bullet
\end{array}
\right)
\end{eqnarray}
Note that this update works only provided that $\det(Z^{-1}) \neq 0$.

\subsection{Update $\mathbf{U_{Z_{3}}}$}
According to~(\ref{update:o_rep_s}) and Definition~(\ref{def:U_W})
$U$ and $W$ are defined as
\begin{equation}
\label{def:U2}
U:=
\left(
\begin{array}{c|c}
 0                        & A_{C,j}  \\
\hline
A_{\sigma(i), B_{O}}^{T}  & 2D_{B_{O}, j}
\end{array}
\right)
=
\left(
\begin{array}{c|c}
0                     &  \theta_{C} \\
\hline
\vartheta_{B_{O}}^{T} & \theta_{B_{O}}
\end{array}
\right),
\quad
W:=
\left(
\begin{array}{c|c}
0                & A_{\sigma(i), j} \\
\hline
A_{\sigma(i), j} & 2D_{j,j}
\end{array}
\right)
\end{equation}
such that according to the Definitions~(\ref{def:X_Z}) and
Substitutions~(\ref{def:X_hat_Z_hat}) $\hat{X}$ becomes
\begin{eqnarray}
\hat{X}
&:=&
d\check{M}_{B}^{-1}U
\nonumber \\
&=&
\hat{\check{M}}_{B}^{-1}
\left(
\begin{array}{c|c}
\vartheta^{T} & \theta
\end{array}
\right)
\nonumber \\
&=&
\left(
\begin{array}{c|c}
\hat{\check{M}}_{B}^{-1}\vartheta^{T}
&
\hat{\check{M}}_{B}^{-1}\theta  \\
\end{array}
\right)
\end{eqnarray}
using again the Definition~(\ref{def:X_Z}), the
Substitutions~(\ref{def:X_hat_Z_hat}) and the above yields $\hat{Z}$
\begin{eqnarray}
\hat{Z}
&:=&
d\left(W - U^{T}\check{M}_{B}^{-1}U\right)
\nonumber \\
&=&
dW - U^{T}\hat{X}
\nonumber \\
&=&
dW
-
\left(
\begin{array}{c}
\vartheta \\
\hline
\theta^{T}
\end{array}
\right)
\left(
\begin{array}{c|c}
\hat{\check{M}}_{B}^{-1}\vartheta^{T}
&
\hat{\check{M}}_{B}^{-1}\theta  \\
\end{array}
\right)
\nonumber \\
&=&
dW
-
\left(
\begin{array}{c|c}
\vartheta \hat{\check{M}}_{B}^{-1} \vartheta^{T}
&
\vartheta \hat{\check{M}}_{B}^{-1}\theta
\\
\hline
\theta^{T}\hat{\check{M}}_{B}^{-1}\vartheta^{T}
&
\theta^{T}\hat{\check{M}}_{B}^{-1}\theta
\end{array}
\right)
\nonumber \\
&=&
d
\left(
\begin{array}{c|c}
0                & A_{\sigma(i), j} \\
\hline
A_{\sigma(i), j} & 2D_{j,j}  
\end{array}
\right)
-
\left(
\begin{array}{c|c}
\vartheta \hat{\check{M}}_{B}^{-1} \vartheta^{T}
&
\vartheta \hat{\check{M}}_{B}^{-1}\theta
\\
\hline
\theta^{T}\hat{\check{M}}_{B}^{-1}\vartheta^{T}
&
\theta^{T}\hat{\check{M}}_{B}^{-1}\theta
\end{array}
\right)
\nonumber
\end{eqnarray}
Since $\nu$ is according to \cite{Sven}, Lemma(2.7), defined as
\begin{equation}
\nu:=
2D_{j,j}
-\left(
  A_{C,j}^{T}\left|\right.A_{S_{B}, j}^{T}\left|\right.
  2D_{B_{O},j}^{T} \left|\right. 2D_{B_{S}, j}^{T}
 \right)
M_{B}^{-1}
\left(
\begin{array}{c}
A_{C, j} \\
\hline
A_{S_{B},j} \\
\hline
2D_{B_{O}, j} \\
\hline
2D_{B_{S}, j}
\end{array}
\right) 
\end{equation}
and $D_{B_{S}, j}=0$ we obtain for $\nu$ together with
Definition~(\ref{eq:M_B_inv_exp}) of the basis inverse
and Definitions~(\ref{def:theta_vartheta})
\begin{eqnarray}
\nu
&=&
2D_{j,j}
-\left(
  \theta_{C}^{T}\left|\right.A_{S_{B}, j}^{T}\left|\right.
  \theta_{B_{O}}^{T} \left|\right. 0
 \right)
\left(
\begin{array}{c}
\left(\check{M}_{B}^{-1}\right)_{C,\bullet}\theta
\\
\hline
0
\\
\hline
\left(\check{M}_{B}^{-1}\right)_{B_{O}, \bullet}\theta
\\
\hline
\hdots
\end{array}
\right)
\nonumber \\
&=&
2D_{j,j}- \theta^{T} \check{M}_{B}^{-1} \theta
\end{eqnarray}

With this and the precondition of the leaving variable for $U_{Z_{3}}$
in table~(\ref{table:leaving_precond}) $\hat{Z}$ can then be expressed as
\begin{equation}
\hat{Z}:=
\left(
\begin{array}{c|c}
0
&
dA_{\sigma(i),j}
 -\vartheta\hat{\check{M}}_{B}^{-1}\theta
\\
\hline
dA_{\sigma(i),j}
 -\theta^{T}\hat{\check{M}}_{B}^{-1}\vartheta^{T}
&
d\nu
\end{array}
\right)
\end{equation} 
By the pivoting precondition for $U_{Z_{3}}$ in
table~(\ref{table:pivot_precond}) and the symmetry of $\check{M}_{B}^{-1}$
we have 
\begin{equation}
\det(Z)=-\left(A_{\sigma(i),j}
 -\vartheta\check{M}_{B}^{-1}\theta
\right)^{2} \neq 0
\end{equation}
in accordance with Lemma~(\ref{lem:main}). Note that according to the above and
the Definition~(\ref{def:new_denom_Z3}) of the new denominator $\hat{z}$
we have 
\begin{equation}
\label{eq:sign_of_new_denom_Z3}
sgn(\hat{z})=-1
\end{equation}

If we define
\begin{equation}
\label{def:kappa_hat_nu_hat}
\hat{\kappa}:= dA_{\sigma(i),j}-\vartheta\hat{\check{M}}_{B}^{-1}\theta,
\quad
\hat{\nu}:=d\nu
\end{equation}
\marginpar{check definition of $\kappa$ with last subsection} 
we may write
\begin{eqnarray}
-\hat{X}P\hat{Z}P^{T}
&:=&
-
\left(
\begin{array}{c|c}
\hat{\check{M}}_{B}^{-1}\vartheta^{T}
&
\hat{\check{M}}_{B}^{-1}\theta  \\
\end{array}
\right)
\left(
\begin{array}{c|c}
\hat{\nu}     & -\hat{\kappa} \\
\hline
-\hat{\kappa} & 0
\end{array}
\right)
\nonumber \\
&=&
\left(
\begin{array}{c|c}
\hat{\kappa}\hat{\check{M}}_{B}^{-1}\theta
-\hat{\nu}\hat{\check{M}}_{B}^{-1}\vartheta^{T}
&
\hat{\kappa}\hat{\check{M}}_{B}^{-1}\vartheta^{T}
\end{array}
\right)
\end{eqnarray}
as well as 
\begin{equation}
\det(\hat{Z})=-\hat{\kappa}^{2},
\end{equation}
if we furthermore define
\begin{equation}
\label{def:phi_hat_varphi_hat}
\hat{\phi}:=\left(\hat{\check{M}}_{B}^{-1}\right)\theta,
\quad
\hat{\varphi}:=\left(\hat{\check{M}}_{B}^{-1}\right)\vartheta^{T}
\end{equation}
we may now express $\hat{X}P\hat{Z}P^{T}\hat{X}^{T}$ as
\begin{eqnarray}
\hat{X}P\hat{Z}P^{T}\hat{X}^{T}
&:=&
-
\left(
\begin{array}{c|c}
\hat{\kappa}\hat{\phi}-\hat{\nu}\hat{\varphi}
&
\hat{\kappa}\hat{\varphi}
\end{array}
\right)
\left(
\begin{array}{c}
\hat{\varphi}^{T}  \\
\hline
\hat{\phi}^{T}
\end{array}
\right)
\nonumber \\
&=&
\left(
  \hat{\nu}\hat{\varphi}-\hat{\kappa}\hat{\phi}
\right)
\hat{\varphi}^{T}
-\hat{\kappa}\hat{\varphi}\hat{\phi}^{T}
\nonumber \\
&=&
\hat{\nu}\hat{\varphi}\hat{\varphi}^{T}
-\hat{\kappa}
 \left[
   \hat{\varphi}\hat{\phi}^{T}+\hat{\phi}\hat{\varphi}^{T}
 \right]
\end{eqnarray}

Subsuming we may finally write
\begin{eqnarray}
\label{def:update_Z3}
\hat{\check{M}}_{B^{\prime}}^{-1}
&=&
-
\left(
\begin{array}{c|c|c}
\left[
  -\hat{\kappa}^{2}\hat{\check{M}}_{B}^{-1}
  +\hat{\nu}\hat{\varphi}\hat{\varphi}^{T}
  -\hat{\kappa}
  \left[
    \hat{\varphi}\hat{\phi}^{T}+\hat{\phi}\hat{\varphi}^{T}
  \right]
\right] / d^{2}
&
\left[
  \hat{\kappa}\hat{\phi} - \hat{\nu}\hat{\varphi}
\right] /d
&
  \hat{\kappa}\hat{\varphi} / d
\\
\hline
\left[
  \hat{\kappa}\hat{\phi}^{T} - \hat{\nu}\hat{\varphi}^{T}
\right] /d
&
\hat{\nu}
&
-\hat{\kappa}
\\
\hline
\hat{\kappa}\hat{\varphi}^{T} /d
&
-\hat{\kappa}
&
0
\end{array}
\right)
\nonumber \\
&=&
\left(
\begin{array}{c|c|c}
\left[
  \hat{\kappa}^{2}\hat{\check{M}}_{B}^{-1}
  -\hat{\nu}\hat{\varphi}\hat{\varphi}^{T}
  +\hat{\kappa}
  \left[
    \hat{\varphi}\hat{\phi}^{T}+\hat{\phi}\hat{\varphi}^{T}
  \right]
\right] / d^{2}
&
\left[
  \hat{\nu}\hat{\varphi} - \hat{\kappa}\hat{\phi}
\right] /d
&
-\hat{\kappa}\hat{\varphi} / d
\\
\hline
\left[
  \hat{\nu}\hat{\varphi}^{T} - \hat{\kappa}\hat{\phi}^{T}
\right] /d
&
-\hat{\nu}
&
\hat{\kappa}
\\
\hline
-\hat{\kappa}\hat{\varphi}^{T} /d
&
\hat{\kappa}
&
0
\end{array}
\right)
\end{eqnarray}
where $\hat{z}$ is the new denominator and 
\begin{equation}
\label{def:update_Z3_entities}
\begin{array}{rclcrcl}
\hat{\phi} &:=& \hat{\check{M}}_{B}^{-1}\theta
&&
\hat{\varphi} &:=& \hat{\check{M}}_{B}^{-1}\vartheta^{T}
\\
\hat{\kappa} &:=& dA_{\sigma(i), j}-\vartheta\hat{\check{M}}_{B}^{-1}\theta
&&
\hat{\nu} &:=& 2dD_{j,j}-\theta^{T}\hat{\check{M}}_{B}^{-1}\theta
\\
\hat{z} &:=& \hat{\kappa}^{2} / d
&&
d &:=& \left|\det(\hat{\check{M}}_{B})\right|
\end{array}
\end{equation}

\subsection{Update $\mathbf{U_{Z_{2}}}$}
In analogy with Definitions~(\ref{def:theta_vartheta}) we define
\begin{equation}
\label{def:rho_hat_varrho_hat}
\hat{\rho}:=\left(\hat{\check{M}}_{B}^{-1}\right)_{\bullet, \beta_{O}(i)},
\quad
\hat{\varrho}^{T}
:=
\left(\hat{\check{M}}_{B}^{-1}\right)_{\bullet, \gamma_{C}(\sigma(j))}
\end{equation}
such that
\begin{equation}
\hat{\chi}:=
\left(
\begin{array}{c|c}
\hat{\varrho}_{C^{\prime}}^{T}     & \hat{\rho}_{C^{\prime}} \\
\hline
\hat{\varrho}_{B_{O}^{\prime}}^{T} & \hat{\rho}_{B_{O}^{\prime}}
\end{array}
\right)
\end{equation}
and
\begin{eqnarray}
\hat{Z}^{-1}
&:=&
\left(
\begin{array}{c|c}
\hat{\varrho}_{\gamma_{C}(\sigma(j))}^{T}
&
\hat{\rho}_{\gamma_{C}(\sigma(j))} \\
\hline
\hat{\varrho}_{\beta_{O}(i)}^{T}
&
\hat{\rho}_{\beta_{O}(i)}
\end{array}
\right)
\nonumber \\
&=&
\left(
\begin{array}{c|c}
\left(
  \hat{\check{M}}_{B}^{-1}
\right)_{\gamma_{C}(\sigma(j)), \gamma_{C}(\sigma(j))}
&
\left(\hat{\check{M}}_{B}^{-1}\right)_{\gamma_{C}(\sigma(j)), \beta_{O}(i)}
\\
\hline
\left(\hat{\check{M}}_{B}^{-1}\right)_{\beta_{O}(i), \gamma_{C}(\sigma(j))}
&
\left(\hat{\check{M}}_{B}^{-1}\right)_{\beta_{O}(i), \beta_{O}(i)}
\end{array}
\right) 
\end{eqnarray}
The above is using the precondition of the leaving variable for $U_{Z_{2}}$ in
table~(\ref{table:leaving_precond}) simplified to
\begin{equation}
\hat{Z}^{-1}=
\left(
\begin{array}{c|c}
\left(
  \hat{\check{M}}_{B}^{-1}
\right)_{\gamma_{C}(\sigma(j)), \gamma_{C}(\sigma(j))}
&
\left(\hat{\check{M}}_{B}^{-1}\right)_{\gamma_{C}(\sigma(j)), \beta_{O}(i)}
\\
\hline
\left(\hat{\check{M}}_{B}^{-1}\right)_{\beta_{O}(i), \gamma_{C}(\sigma(j))}
&
0
\end{array}
\right) 
\end{equation}
By the pivoting precondition for $U_{Z_{2}}$ in 
table~(\ref{table:pivot_precond}) we then have
\begin{equation}
\det(Z^{-1}) =
-\left( 
\left(\check{M}_{B}^{-1}\right)_{\beta_{O}(i),\gamma_{C}(\sigma(j))}
\right)^{2}
\neq 0
\end{equation}
again, in accordance with Lemma~(\ref{lem:main}). Note that according to the above and
the Definition~(\ref{def:new_denom_Z2}) of the new denominator $\hat{z}$
we have 
\begin{equation}
\label{eq:sign_of_new_denom_Z2}
sgn(\hat{z})=-1
\end{equation}


If we define 
\begin{equation}
\hat{\kappa}:=
\left(\hat{\check{M}}_{B}^{-1}\right)_{\gamma_{C}(\sigma(j)), \beta_{O}(i)}
\quad
\hat{{\xi}}:=
\left(
  \hat{\check{M}}_{B}^{-1}
\right)_{\gamma_{C}(\sigma(j)), \gamma_{C}(\sigma(j))}
\end{equation}
we may write
\begin{eqnarray}
-\hat{\chi} P \hat{Z}^{-1}P^{T}\hat{\chi}^{T}
&=&
-\left(
\begin{array}{c|c}
\hat{\varrho}_{C^{\prime} \cup B_{O}^{\prime}}^{T}
&
\hat{\rho}_{C^{\prime} \cup B_{O}^{\prime}} \\
\end{array}
\right)
\left(
\begin{array}{c|c}
0              & -\hat{\kappa}  \\
\hline
-\hat{\kappa} & \hat{\xi}
\end{array}
\right)
\left(
\begin{array}{c}
\hat{\varrho}_{C^{\prime} \cup B_{O}^{\prime}}   \\
\hline
\hat{\rho}_{C^{\prime} \cup B_{O}^{\prime}}^{T} 
\end{array}
\right)
\nonumber \\
&=&
\left(
\begin{array}{c|c}
\hat{\kappa}\hat{\rho}_{C^{\prime} \cup B_{O}^{\prime}}
&
\hat{\kappa}\hat{\varrho}_{C^{\prime} \cup B_{O}^{\prime}}^{T}
 -\hat{\xi}\hat{\rho}_{C^{\prime} \cup B_{O}^{\prime}}
\end{array}
\right)
\left(
\begin{array}{c}
\hat{\varrho}_{C^{\prime} \cup B_{O}^{\prime}}   \\
\hline
\hat{\rho}_{C^{\prime} \cup B_{O}^{\prime}}^{T}  
\end{array}
\right)
\nonumber \\
&=&
-\hat{\xi}
 \hat{\rho}_{C^{\prime} \cup B_{O}^{\prime}}
 \hat{\rho}_{C^{\prime} \cup B_{O}^{\prime}}^{T}
+\hat{\kappa}
 \left[
 \hat{\rho}_{C^{\prime} \cup B_{O}^{\prime}}
 \hat{\varrho}_{C^{\prime} \cup B_{O}^{\prime}}
 +\hat{\varrho}_{C^{\prime} \cup B_{O}^{\prime}}^{T}
 \hat{\rho}_{C^{\prime} \cup B_{O}^{\prime}}^{T}
 \right]
\end{eqnarray}
as well as 
\begin{equation}
\det(\hat{Z}^{-1})=-\hat{\kappa}^{2},
\end{equation}


Subsuming this we may finally write
\begin{eqnarray}
\label{def:update_Z2}
\hat{\check{M}}_{B^{\prime}}^{-1}
&=&
-
\left(
\begin{array}{c|c}
\left[
  -\hat{\kappa}^{2}\hat{\check{M}}_{B}^{-1}
  -\hat{\xi}
  \hat{\rho}_{C^{\prime} \cup B_{O}^{\prime}}
  \hat{\rho}_{C^{\prime} \cup B_{O}^{\prime}}^{T}
  +\hat{\kappa}
  \left[
    \hat{\rho}_{C^{\prime} \cup B_{O}^{\prime}}
    \hat{\varrho}_{C^{\prime} \cup B_{O}^{\prime}}
    +\hat{\varrho}_{C^{\prime} \cup B_{O}^{\prime}}^{T}
    \hat{\rho}_{C^{\prime} \cup B_{O}^{\prime}}^{T}
  \right]
\right]/d^{2}
&
\vdots
\\
\hline
\hdots
&
\bullet
\end{array}
\right)
\nonumber \\
&=&
\left(
\begin{array}{c|c}
\left[
  \hat{\kappa}^{2}\hat{\check{M}}_{B}^{-1}
  +\hat{\xi}
  \hat{\rho}_{C^{\prime} \cup B_{O}^{\prime}}
  \hat{\rho}_{C^{\prime} \cup B_{O}^{\prime}}^{T}
  -\hat{\kappa}
  \left[
    \hat{\rho}_{C^{\prime} \cup B_{O}^{\prime}}
    \hat{\varrho}_{C^{\prime} \cup B_{O}^{\prime}}
    +\hat{\varrho}_{C^{\prime} \cup B_{O}^{\prime}}^{T}
    \hat{\rho}_{C^{\prime} \cup B_{O}^{\prime}}^{T}
  \right]
\right]/d^{2}
&
\vdots
\\
\hline
\hdots
&
\bullet
\end{array}
\right)
\end{eqnarray}
where $\hat{z}$ is the new denominator and 
\begin{equation}
\label{def:update_Z2_entities}
\begin{array}{rclcrcl}
\hat{\rho}
&:=&
\left(
  \hat{\check{M}}_{B}^{-1}
\right)_{\bullet, \beta_{O}(i)}
&&
\hat{\varrho}^{T}
&:=&
\left(
  \hat{\check{M}}_{B}^{-1}
\right)_{\bullet, \gamma_{C}(\sigma(j))}
\\
\hat{\kappa}
&:=&
\left(
  \hat{\check{M}}_{B}^{-1}
\right)_{\gamma_{C}(\sigma(j)), \beta_{O}(i)}
&&
\hat{\xi}
&:=&
\left(
  \hat{\check{M}}_{B}^{-1}
\right)_{\gamma_{C}(\sigma(j)), \gamma_{C}(\sigma(j))}
\\
\hat{z}
&:=&
\hat{\kappa}^{2}/d
&&
d
&:=&
\left|\det(\hat{\check{M}}_{B})\right|
\end{array}
\end{equation}

\section{Implementation Issues}
Comparing the Updates for $U_{Z_{1}}$ to $U_{Z_{4}}$ we note that the Updates
$U_{Z_{1}}$ and $U_{Z_{4}}$ are en-bloc replacement updates, that is, the to be
replaced row and column need no special treatment, in contrast to the Updates
$U_{Z_{2}}$ and $U_{Z_{3}}$, which change the dimensions of the reduced basis
inverse by $2$. Nevertheless, all the Updates have a similar update
formula in common for parts of the to be updated basis inverses,
for $U_{Z_{1}}$ and $U_{Z_{4}}$ this part comprises the whole
basis inverse $\check{M}_{B}^{-1}$ and for $U_{Z_{2}}$
and $U_{Z_{3}}$ the corresponding part consists of  
\begin{equation}
\left(
 \check{M}_{B}^{-1}
\right)_{C \cap C^{\prime} \cup B_{O} \cap B_{O}^{\prime}}
\end{equation}
The general update formula for this part for the Updates
$U_{Z_{1}}$ and $U_{Z_{4}}$ can then be written as 
\begin{equation}
\check{M}_{B^{\prime}}^{-1}
=
\frac{
\omega_{0}\check{M}_{B}^{-1}
+\omega_{1}\Psi_{1}\Psi_{1}^{T}
+\omega_{2}\left[\Psi_{1}\Psi_{2}^{T}+\Psi_{2}\Psi_{1}^{T}\right]
}{\omega_{3}},
\end{equation}
The values the above entities assume for the updates $U_{Z_{1}}$ and
$U_{Z_{4}}$ are subsumed in the following table below.
\begin{equation}
\begin{array}{c|cc}
& U_{Z_{1}} & U_{Z_{4}} \\
\hline
C^{\prime}
& C
& C \setminus \{\sigma(j)\} \cup \{\sigma(i)\}
\\
B_{O}^{\prime}
& B_{O} \setminus \{i\} \cup \{j\}
& B_{O}
\\
\Psi_{1}
& \hat{\rho}
& \hat{\check{M}}_{B}^{-1}\zeta-de_{\gamma_{C}(\sigma(j))}
\\
\Psi_{2}
& \hat{\check{M}}_{B}^{-1}\zeta-de_{\beta_{O}(i)}
& \hat{\varrho}^{T}
\\
\omega_{0}
& \left(\hat{\rho}^{T}\zeta\right)^{2}
& \left(\hat{\varrho}\zeta\right)^{2}
\\
\omega_{1}
& -\left(
   d\zeta_{\beta_{O}(i)}-\zeta^{T}\hat{\check{M}}_{B}^{-1}\zeta
  \right) 
& \left(
   \hat{\check{M}}_{B}^{-1}
  \right)_{\gamma_{C}(\sigma(j)), \gamma_{C}(\sigma(j))} 
\\
\omega_{2}
& -\hat{\rho}^{T}\zeta
& -\hat{\varrho}\zeta
\\
\omega_{3}
& d^{2}
& d^{2}
\end{array}
\end{equation}
 
Correspondingly for the Updates $U_{Z_{2}}$ and $U_{Z_{3}}$ as 
\begin{equation}
\left(
\check{M}_{B^{\prime}}^{-1}
\right)_{C \cap C^{\prime} \cup B_{O} \cap B_{O}^{\prime}}
=
\frac{
\omega_{0}
\left(
\check{M}_{B}^{-1}
\right)_{C \cap C^{\prime} \cup B_{O} \cap B_{O}^{\prime}}
+\omega_{1}\Psi_{1}\Psi_{1}^{T}
+\omega_{2}\left[\Psi_{1}\Psi_{2}^{T}+\Psi_{2}\Psi_{1}^{T}\right]
}{\omega_{3}},
\end{equation}
The values the above entities assume for the updates $U_{Z_{2}}$ and
$U_{Z_{3}}$ are subsumed in the following table below.
\begin{equation}
\begin{array}{c|cccc}
& U_{Z_{2}} & U_{Z_{3}} \\
\hline
C^{\prime}
& C \setminus \{\sigma(j)\}
& C \cup \{\sigma(i)\}
\\
B_{O}^{\prime}
& B_{O} \setminus \{i\}
& B_{O} \cup \{j\}
\\
\Psi_{1}
& \hat{\rho}_{C^{\prime} \cup B_{O}^{\prime}}
& \varphi
\\
\Psi_{2}
& \hat{\varrho}_{C^{\prime} \cup B_{O}^{\prime}}^{T}
& \phi
\\
\omega_{0}
& \left(
   \hat{\check{M}}_{B}^{-1}
  \right)_{\gamma_{C}(\sigma(j)), \beta_{O}(i)}^{2}
& \left(
   dA_{\sigma(i),j}-\vartheta\hat{\check{M}}_{B}^{-1}\theta
  \right)^{2} d
\\
\omega_{1}
& \left(
   \hat{\check{M}}_{B}^{-1}
  \right)_{\gamma_{C}(\sigma(j)), \gamma_{C}(\sigma(j))}
& -\hat{\nu}
\\
\omega_{2}
& -\left(
   \hat{\check{M}}_{B}^{-1}
  \right)_{\gamma_{C}(\sigma(j)), \beta_{O}(i)}
& d A_{\sigma(i), j}-\vartheta\hat{\check{M}}_{B}^{-1}\theta
\\
\omega_{3}
& d^{2}
& d^{2}
\end{array}
\end{equation}
The function performing this part of the Updates $U_{Z_{1}} \dots U_{Z_{4}}$
has the following signature:
\begin{tabbing}
\texttt{z\_update\_inplace($\Psi_{1}, \Psi_{2}, \omega_{0}, \omega_{1},
\omega_{2}, \omega_{3}$)}
\end{tabbing}
\subsection{Interfaces}
%In general entities depending only on the constraint matrix $A$ and on the
%objective function matrix $D$ as well as on the indices $j, i$ of the current
%entering and leaving variables are considered to be global constants.  
The QP-solver is represented by the classes \texttt{QP\_solver} and
\texttt{QP\_basis\_inverse} and other classes representing the different
pricing strategies. An instance of the class \texttt{QP\_solver} has
members representing the headings $B_{O},B_{S}, C$ and $S_{B}$,
the constraint matrix $A$, the objective function matrix $D$ as well as a
reference, \texttt{inv\_M\_B}, referring to an instance of the class
\texttt{QP\_basis\_inverse}. An instance of the class
\texttt{QP\_basis\_inverse} however does not have a reference
referring to its ambient solver. This entails that methods of the basis inverse
implementing the Updates $U_{Z_{1}} \dots U_{Z_{4}}$ described in previous
sections can not access these entities kept in the class \texttt{QP\_solver};
we therefore adopt the policy that these entities are passed as parameters of
these methods wherever necessary.   

Some of the entities used in the update formulas~(\ref{def:update_Z1}),
(\ref{def:update_Z4}), (\ref{def:update_Z3}) and~(\ref{def:update_Z2})
are already determined by the Ratio Test Step~1 preceeding the update step~1.
Among other entities the Ratio Test Step~1 computes
\begin{equation}
\label{def:q_hat_nu_hat}
\hat{q}:=\hat{\check{M}}_{B}^{-1}\theta
\quad
\hat{\nu}:=2D_{j,j}d-\theta^{T}\hat{\check{M}}_{B}^{-1}\theta
\end{equation}  


\subsection{Update $U_{Z_{1}}$}
According to the Definitions~(\ref{def:zeta_Z1})
and~(\ref{def:q_hat_nu_hat}), $\hat{v}$ of the 
Definitions~(\ref{def:update_Z1_entities}) can be expressed in terms
of the entering column $\theta$ and $D$ as
\begin{eqnarray*}
\hat{v}
&:=&
\hat{\check{M}}_{B}^{-1}\zeta - de_{\beta_{O}(i)} \\
&=&
\hat{\check{M}}_{B}^{-1}\theta
+2\left(D_{j,j}-D_{i,j}\right)\hat{\check{M}}_{B}^{-1}e_{\beta_{O}(i)}
-de_{\beta_{O}(i)} \\
&=&
\hat{q}+2\left(D_{j,j}-D_{i,j}\right)\hat{\rho}-de_{\beta_{O}(i)}
\end{eqnarray*}
Similarly we obtain for $\hat{k}_{2}$ using Definitions~(\ref{def:zeta_Z1})
and~(\ref{def:q_hat_nu_hat})
\begin{eqnarray*}
\hat{k}_{2}
&:=&
d\zeta_{\beta_{O}(i)}-\zeta^{T}\hat{\check{M}}_{B}^{-1}\zeta \\
&=&
2dD_{j,j}
-\left(\theta+2\left(D_{j,j}-D_{i,j}\right)e_{\beta_{O}(i)}^{T}\right)
\hat{\check{M}}_{B}^{-1}
\left(\theta+2\left(D_{j,j}-D_{i,j}\right)e_{\beta_{O}(i)}\right) \\
&=&
2dD_{j,j}-\theta^{T}\hat{\check{M}}_{B}^{-1}\theta
-4\left(D_{j,j}-D_{i,j}\right)^{2}
e_{\beta_{O}(i)}^{T}\hat{\check{M}}_{B}^{-1}e_{\beta_{O}(i)} \\
&&
-4\left(D_{j,j}-D_{i,j}\right)
e_{\beta_{O}(i)}^{T}\hat{\check{M}}_{B}^{-1}\theta \\
&=&
\hat{\nu}
-4\left(D_{j,j}-D_{i,j}\right)^{2}
\left(\hat{\check{M}}_{B}^{-1}\right)_{\beta_{O}(i), \beta_{O}(i)}
-4\left(D_{j,j}-D_{i,j}\right)\hat{\rho}^{T}\theta \\
&=&
\hat{\nu}
-4\left(D_{j,j}-D_{i,j}\right)\hat{\rho}^{T}\theta \\
&=&
\hat{\nu}
-4\left(D_{j,j}-D_{i,j}\right)\hat{q}_{\beta_{O}(i)}
\end{eqnarray*}
Using the above, the entities used
in~(\ref{def:update_Z1_entities}) can be expressed as
\begin{equation}
\label{def:update_Z1_entities_prep}
\begin{array}{rclcrcl}
\hat{\rho}
&:=&
\left(\hat{\check{M}}_{B}^{-1}\right)_{\bullet, \beta_{O}(i)}
&&
\hat{v}
&:=&
\hat{q}+2\left(D_{j,j}-D_{i,j}\right)\hat{\rho}-de_{\beta_{O}(i)}
\\
\hat{k}_{1}
&:=&
\hat{q}_{\beta_{O}(i)}
&&
\hat{k}_{2}
&:=&
\hat{\nu}
-4\left(D_{j,j}-D_{i,j}\right)\hat{q}_{\beta_{O}(i)}
\\
\hat{z}
&:=&
\hat{k}_{1}^{2}/d
&&
d
&:=&
\left|\det(\hat{\check{M}}_{B})\right|
\end{array}
\end{equation}
such that the part in the class \texttt{QP\_basis\_inverse} of 
Update $U_{Z_{1}}$ can be outlined as
\begin{tabbing}
\texttt{z\_replace\_original\_by\_original($y, s_{\Delta}, s_{\nu}, k_{i}$) \{}
\\
\texttt{  } \= // compute $\hat{v}, \hat{\rho}, \hat{k}_{1}, \hat{k}_{2}$ \\
\> \texttt{$\hat{k}_{1}:=y_{k_{i}}$} \\
\> \texttt{$\hat{\rho}:=
  \left(\hat{\check{M}}_{B}^{-1}\right)_{\bullet, k_{i}}$} \\
\> \texttt{$\hat{v}:=y + s_{\Delta} \hat{\rho} - de_{k_{i}}$} \\
\> \texttt{$\hat{k}_{2}:=s_{\nu} - 2s_{\Delta}\hat{k}_{1}$} \\
\> \texttt{z\_update\_in\_place($\hat{\rho}, \hat{v}, \hat{k}_{1}^{2},
-\hat{k}_{2}, -\hat{k}_{1}, d^{2}$)} \\
\> // store new denominator $\hat{z}$ \\
\> \texttt{$d:=\hat{k}_{1}^{2}/d$} \\
\texttt{\}}
\end{tabbing}
The calling counterpart in the class \texttt{QP\_solver} can be outlined as
\begin{tabbing}
\texttt{z\_replace\_variable\_original\_by\_original() \{} \\
\texttt{  } \= // compute $k_{i}, \hat{k}_{2}$ \\
\> \texttt{$s_{\Delta}:= 2\left(D_{j,j} - D_{i,j}\right)$} \\
\> \texttt{$s_{\nu}:= \nu$} \\
\> \texttt{$k_{i}:=\beta_{O}(i)$} \\
\> // update headings \\
\> // ... \\
\> // update basis inverse \\
\> \texttt{inv\_M\_B.z\_replace\_original\_by\_original($\hat{q},
s_{\Delta}, s_{\nu}, k_{i}$)} \\
\texttt{\}}
\end{tabbing}

\subsection{Update $U_{Z_{4}}$}
According to the Definitions~(\ref{def:zeta_Z4}),
%and~(\ref{def:q_hat_nu_hat}),
$\zeta=\vartheta$, such that the entities $\hat{v}$ and $\hat{k}_{1}$ used
in Definitions~(\ref{def:update_Z4_entities}) can be expressed in terms
of the entering column $\vartheta$ as
\begin{equation}
\label{def:update_Z4_entities_prep}
\begin{array}{rclcrcl}
\hat{v}
&:=&
\hat{\check{M}}_{B}^{-1}\vartheta^{T} -de_{\gamma_{C}(\sigma(j))}
&&
\hat{\varrho}^{T}
&:=&
\left(\hat{\check{M}}_{B}^{-1}\right)_{\bullet, \gamma_{C}(\sigma(j))}
\\
\hat{k}_{1}
&:=&
\hat{\varrho}\vartheta^{T}
&&
\hat{k}_{3}
&:=&
-\left(\hat{\check{M}}_{B}^{-1}\right)_{\gamma_{C}(\sigma(j)),
  \gamma_{C}(\sigma(j))}
\\
\hat{z}
&:=&
\hat{k}_{1}^{2}/d
&&
d
&:=&
\left|\det(\hat{\check{M}}_{B})\right|
\end{array}
\end{equation}
such that, by taking into account that according to
Definition~(\ref{def:theta_vartheta}), $\vartheta_{C}=0$,
the part in the class \texttt{QP\_basis\_inverse} of
Update $U_{Z_{4}}$ can be outlined as
\begin{tabbing}
\texttt{z\_replace\_slack\_by\_slack($u, k_{j}$) \{} \\
\texttt{  } \= // compute $\hat{v}, \hat{\varrho}, \hat{k}_{1}, \hat{k}_{3}$ \\
\> \texttt{$\hat{v}:=
    \left(\hat{\check{M}}_{B}^{-1}\right)_{\bullet, B_{O}}u-de_{k_{j}}$} \\
\> \texttt{$\hat{\varrho}^{T}:=
  \left(\hat{\check{M}}_{B}^{-1}\right)_{\bullet,k_{j}}$} \\
\> \texttt{$\hat{k}_{1}:=\hat{\varrho}_{B_{O}}u$} \\
\> \texttt{$\hat{k}_{3}:=
  -\left(\hat{\check{M}}_{B}^{-1}\right)_{k_{j}, k_{j}}$} \\
\> \texttt{z\_update\_in\_place($\hat{v}, \hat{\varrho}, \hat{k}_{1}^{2},
-\hat{k}_{3}, -\hat{k}_{1}, d^{2}$)} \\
\> // store new denominator $\hat{z}$ \\
\> \texttt{$d:=\hat{k}_{1}^{2}/d$} \\
\texttt{\}}
\end{tabbing}
The calling counterpart in the class \texttt{QP\_solver} can be outlined as
\begin{tabbing}
\texttt{z\_replace\_variable\_slack\_by\_slack() \{} \\
\texttt{  } \= // compute $u, k_{j}$ \\
\> \texttt{$u:=\vartheta_{B_{O}}^{T}$} \\
\> \texttt{$k_{j}:=\gamma_{C}(\sigma(j))$} \\
\> // update headings \\
\> // ... \\
\> // update basis inverse \\
\> \texttt{inv\_M\_B.z\_replace\_slack\_by\_slack($u, k_{j}$)}
\\
\texttt{\}}
\end{tabbing}


\subsection{Update $U_{Z_{2}}$}
Using Definitions~(\ref{def:q_hat_nu_hat}) the entities used
in~(\ref{def:update_Z2_entities}) can be expressed as
\begin{equation}
\label{def:update_Z2_entities_prep}
\begin{array}{rclcrcl}
\hat{\rho}
&:=&
\left(
  \hat{\check{M}}_{B}^{-1}
\right)_{\bullet, \beta_{O}(i)}
&&
\hat{\varrho}^{T}
&:=&
\left(
  \hat{\check{M}}_{B}^{-1}
\right)_{\bullet, \gamma_{C}(\sigma(j))}
\\
\hat{\kappa}
&:=&
\left(
  \hat{\check{M}}_{B}^{-1}
\right)_{\gamma_{C}(\sigma(j)), \beta_{O}(i)}
&&
\hat{\xi}
&:=&
\left(
  \hat{\check{M}}_{B}^{-1}
\right)_{\gamma_{C}(\sigma(j)), \gamma_{C}(\sigma(j))}
\\
\hat{z}
&:=&
\hat{\kappa}^{2}/d
&&
d
&:=&
\left|\det(\hat{\check{M}}_{B})\right|
\end{array}
\end{equation}
such that the part in the class \texttt{QP\_basis\_inverse} of
Update $U_{Z_{2}}$ can be outlined as
\begin{tabbing}
\texttt{z\_replace\_original\_by\_slack() \{} \\
\texttt{  } \= // compute $\hat{\rho}, \hat{\varrho}, \hat{\kappa},
\hat{\xi}$ \\
\> \texttt{$\hat{\rho}:=
  \left(\hat{\check{M}}_{B}^{-1}\right)_{\bullet, k_{i}}$} \\
\> \texttt{$\hat{\varrho}:=
  \left(\hat{\check{M}}_{B}^{-1}\right)_{\bullet, k_{j}}$} \\
\> \texttt{$\hat{\kappa}:=
  \left(\hat{\check{M}}_{B}^{-1}\right)_{k_{j}, k_{i}}$} \\
\> \texttt{$\hat{\xi}:=
  \left(\hat{\check{M}}_{B}^{-1}\right)_{k_{j}, k_{j}}$} \\    
\> \texttt{z\_update\_in\_place($\hat{\rho}, \hat{\varrho}^{T},
\hat{\kappa}^{2}, \hat{\xi}, -\hat{\kappa}, d^{2}$)} \\
\> // store new denominator $\hat{z}$ \\
\> \texttt{$d:=\hat{\kappa}^{2}/d$} \\
\texttt{\}}
\end{tabbing}
The calling counterpart in the class \texttt{QP\_solver} can be outlined as
\begin{tabbing}
\texttt{z\_replace\_variable\_original\_by\_slack() \{} \\
\texttt{  } \= // compute $u, k_{j}$ \\
\> \texttt{$k_{j}:=\gamma_{C}(\sigma(j))$} \\
\> \texttt{$k_{i}:=\beta_{O}(i)$} \\
\> // update headings \\
\> // ... \\
\> // prepare basis inverse \\
\> \texttt{inv\_M\_B.swap\_variable($k_{i}$)} \\
\> \texttt{inv\_M\_B.swap\_constraint($k_{j}$)}\\
\> // update basis inverse \\
\> \texttt{inv\_M\_B.z\_replace\_original\_by\_slack()}
\\
\texttt{\}}
\end{tabbing}



\subsection{Update $U_{Z_{3}}$}
Using Definitions~(\ref{def:q_hat_nu_hat}) 
as well as Definitions~(\ref{def:theta_vartheta}) the entities used
in~(\ref{def:update_Z3_entities}) can be expressed as
\begin{equation}
\label{def:update_Z3_entities_prep}
\begin{array}{rclcrcl}
\hat{\phi} &:=& \hat{q}
&&
\hat{\varphi}
&:=&
\hat{\check{M}}_{B}^{-1}\vartheta^{T}
\\
\hat{\kappa} &:=& dA_{\sigma(i), j}-\vartheta\hat{q}
&&
\hat{\nu} &:=& 2dD_{j,j}-\theta^{T}\hat{\check{M}}_{B}^{-1}\theta
\\
\hat{z} &:=& \hat{\kappa}^{2}/d
&&
d &:=& \left|\det(\hat{\check{M}}_{B})\right|
\end{array}
\end{equation}
such that, by taking into account that according to
Definition~(\ref{def:theta_vartheta}), $\vartheta_{C}=0$,
the part in the class \texttt{QP\_basis\_inverse} of
Update $U_{Z_{3}}$ can be outlined as
\begin{tabbing}
\texttt{z\_replace\_slack\_by\_original($y,u,\kappa,\nu$) \{}
\\
\texttt{  } \= // compute $\hat{\phi}, \hat{\varphi}, \hat{\kappa},
\hat{\nu}$ \\
\> \texttt{$\hat{\phi}:=y$} \\
\> \texttt{$\hat{\varphi}:= 
 \left(\hat{\check{M}}_{B}^{-1}\right)_{\bullet, B_{O}}u$} \\
\> \texttt{$\hat{\kappa}:=\kappa$} \\
\> \texttt{$\hat{\nu}:=\nu$} \\
\> \texttt{z\_update\_in\_place($\hat{\varphi}, \hat{\phi},
\hat{\kappa}^{2}, -\hat{\nu}, \hat{\kappa}, d^{2}$)} \\
\> // append new rows and columns \\
\> $\left(\hat{\check{M}}_{B^{\prime}}\right)_{s, \bullet}:=
\left[
  (\hat{\nu}\hat{\varphi}-\hat{\kappa}\hat{\phi})_{C}/d
  \left|\right. -\hat{\nu}
  \left|\right.(\hat{\nu}\hat{\varphi}-\hat{\kappa}\hat{\phi})_{B_{O}}/d 
  \left|\right. \hat{\kappa}
\right]$ \\
\> $\left(\hat{\check{M}}_{B^{\prime}}\right)_{l+b, \bullet}:=
\left[
  -\hat{\kappa}\hat{\varphi}^{T}_{C}/d
  \left|\right. \hat{\kappa}
  \left|\right. -\hat{\kappa}\hat{\varphi}^{T}_{B_{O}}/d
  \left|\right. 0
\right]$ \\
\> // store new denominator $\hat{z}$ \\
\> \texttt{$d:=\hat{\kappa}^{2}/d$} \\
\texttt{\}}
\end{tabbing}
The calling counterpart in the class \texttt{QP\_solver} can be outlined as
\begin{tabbing}
\texttt{z\_replace\_variable\_slack\_by\_original() \{} \\
\texttt{  } \= // compute $u, k_{j}$ \\
\> \texttt{$y:=\hat{q}$} \\
\> \texttt{$u:=\vartheta_{B_{O}}$} \\
\> \texttt{$\kappa:=dA_{\sigma(i), j}-\vartheta_{B_{O}}\hat{q}_{B_{O}}$} \\
\> \texttt{$\nu:=\hat{\nu}$} \\
\> // update headings \\
\> // ... \\
\> // update basis inverse \\
\> \texttt{inv\_M\_B.z\_replace\_slack\_by\_original($y, u, \kappa, \nu$)}
\\
\texttt{\}}
\end{tabbing}



\section{Verification}
\subsection{Update $U_{Z_{1}}$}
According to Equations  we have
\begin{eqnarray}
\hat{\rho}^{T}\check{M}_{B}=de_{\beta_{O}(i)}
\end{eqnarray}
as well as 
\begin{equation}
\hat{v}^{T}\check{M}_{B}=
d\left(\theta^{T} 
   -\left(\check{M}_{B}\right)_{\beta_{O}(i), \bullet}
 \right)
=
du^{T}
\end{equation}
such that $\hat{\check{M}}_{B^{\prime}}^{-1}\check{M}_{B}$ becomes
\begin{eqnarray}
\hat{\check{M}}_{B^{\prime}}^{-1}
\check{M}_{B}
&=&
\frac{
\hat{k}_{1}^{2}\hat{\check{M}}_{B}^{-1}
-\left[
  \hat{k}_{2}\hat{\rho}\hat{\rho}^{T}
  +\hat{k}_{1}\left(\hat{\rho}\hat{v}^{T} +\hat{v}\hat{\rho}^{T}\right)
 \right]
}{d^{2}}
\check{M}_{B}
\nonumber \\
&=&
\frac{
  d\hat{k}_{1}^{2}I
  -d
  \left[
    \hat{k}_{2}\hat{\rho}e_{\beta_{O}(i)}^{T}
    +\hat{k}_{1}
    \left(
      \hat{\rho}u^{T}
      +\hat{v}e_{\beta_{O}(i)}^{T}
    \right)
  \right]
}{d^{2}}
\nonumber \\
&=&
\hat{z}
-\frac{
  \hat{k}_{2}\hat{\rho}e_{\beta_{O}(i)}^{T}
  +\hat{k}_{1}\hat{\rho}u^{T}
  +\hat{k}_{1}\hat{v}e_{\beta_{O}(i)}^{T}
}{d}
\end{eqnarray}

According to Equations we have
\begin{eqnarray}
\hat{\check{M}}_{B}^{-1}UV^{T}
&=&
\hat{\check{M}}_{B}^{-1}ue_{\beta_{O}(i)}^{T}
+\hat{\check{M}}_{B}^{-1}e_{\beta_{O}(i)}u^{T}
-u_{i}\hat{\check{M}}_{B}^{-1}e_{\beta_{O}(i)}e_{\beta_{O}(i)}^{T}
\nonumber \\
&=&
\hat{\check{M}}_{B}^{T}
\left(\theta-\left(\check{M}_{B}\right)_{\bullet, \beta_{O}(i)}\right)
e_{\beta_{O}(i)}^{T}
+\hat{\rho}u^{T}
-u_{\beta_{O}(i)}\hat{\rho}e_{\beta_{O}(i)}^{T}
\nonumber \\
&=&
\left(
  \hat{\check{M}}_{B}^{-1}\theta -de_{\beta_{O}(i)}
\right)
e_{\beta_{O}(i)}^{T}
+\hat{\rho}u^{T}
-u_{\beta_{O}(i)}\hat{\rho}e_{\beta_{O}(i)}^{T}
\nonumber \\
&=&
\hat{v}e_{\beta_{O}(i)}^{T}
+\hat{\rho}u^{T}
-u_{\beta_{O}(i)}\hat{\rho}e_{\beta_{O}(i)}^{T}
\nonumber \\
&=&
\hat{v}e_{\beta_{O}(i)}^{T}
+\hat{\rho}\left(u^{T}-u_{\beta_{O}(i)}e_{\beta_{O}(i)}^{T}\right)
\end{eqnarray}
and
\begin{eqnarray}
\hat{\rho}^{T}UV^{T}
&=&
\hat{\rho}^{T}ue_{\beta_{O}(i)}^{T}
+\hat{\rho}^{T}e_{\beta_{O}(i)}u^{T}
-u_{\beta_{O}(i)}\hat{\rho}^{T}e_{\beta_{O}(i)}e_{\beta_{O}(i)}^{T}
\nonumber \\
&=&
\hat{\rho}^{T}
\left(\theta-\left(\check{M}_{B}\right)_{\bullet, \beta_{O}(i)}\right)
e_{\beta_{O}(i)}^{T}
\nonumber \\
&=&
\left(\hat{k}_{1}-d\right)e_{\beta_{O}(i)}^{T}
\end{eqnarray}
as well as
\begin{eqnarray}
\hat{v}^{T}UV^{T}
&=&
\hat{v}^{T}ue_{\beta_{O}(i)}^{T}
+\hat{v}^{T}e_{\beta_{O}(i)}u^{T}
-u_{\beta_{O}(i)}\hat{v}^{T}e_{\beta_{O}(i)}e_{\beta_{O}(i)}^{T}
\nonumber \\
&=&
\hat{v}^{T}
\left(\theta-\left(\check{M}_{B}\right)_{\bullet, \beta_{O}(i)}\right)
e_{\beta_{O}(i)}^{T}
+\hat{v}_{\beta_{O}(i)}u^{T}
-u_{\beta_{O}(i)}\hat{v}_{\beta_{O}(i)}e_{\beta_{O}(i)}^{T}
\nonumber \\
&=&
-\hat{k}_{2}e_{\beta_{O}(i)}^{T}
-\hat{v}^{T}
\left(\check{M}_{B}\right)_{\bullet, \beta_{O}(i)}e_{\beta_{O}(i)}^{T}
+\hat{v}_{\beta_{O}(i)}u^{T}
-u_{\beta_{O}(i)}\hat{v}_{\beta_{O}(i)}e_{\beta_{O}(i)}^{T}
\nonumber \\
&=&
-\hat{k}_{2}e_{\beta_{O}(i)}^{T}
-du_{\beta_{O}(i)}e_{\beta_{O}(i)}^{T}
+\hat{v}_{\beta_{O}(i)}u^{T}
-u_{\beta_{O}(i)}\hat{v}_{\beta_{O}(i)}e_{\beta_{O}(i)}^{T}
\nonumber \\
&=&
-\hat{k}_{2}e_{\beta_{O}(i)}^{T}
-u_{\beta_{O}(i)}\left(d+\hat{v}_{\beta_{O}(i)}\right)e_{\beta_{O}(i)}^{T}
+\hat{v}_{\beta_{O}(i)}u^{T}
\nonumber \\
&=&
-\hat{k}_{2}e_{\beta_{O}(i)}^{T}
-u_{\beta_{O}(i)}\hat{k}_{1}e_{\beta_{O}(i)}^{T}
+\left(\hat{k}_{1}-d\right)u^{T}
\nonumber \\
&=&
-\hat{k}_{2}e_{\beta_{O}(i)}^{T}
+\hat{k}_{1}
\left(u^{T}-u_{\beta_{O}(i)}e_{\beta_{O}(i)}^{T}\right)
-du^{T}
\end{eqnarray}
such that $\hat{\check{M}}_{B^{\prime}}^{-1}UV^{T}$ becomes
\begin{eqnarray}
\hat{\check{M}}_{B^{\prime}}^{-1}UV^{T}
&=&
\frac{
\hat{k}_{1}^{2}\hat{\check{M}}_{B}^{-1}
-\left[
  \hat{k}_{2}\hat{\rho}\hat{\rho}^{T}
  +\hat{k}_{1}\left(\hat{\rho}\hat{v}^{T} +\hat{v}\hat{\rho}^{T}\right)
 \right]
}{d^{2}}
UV^{T}
\nonumber \\
&=&
\frac{
\hat{k}_{1}^{2}\hat{v}e_{\beta_{O}(i)}^{T}
+\hat{k}_{1}^{2}
\hat{\rho}\left(u^{T}-u_{\beta_{O}(i)}e_{\beta_{O}(i)}^{T}\right)
-\hat{k}_{2}\left(\hat{k}_{1}-d\right)\hat{\rho}e_{\beta_{O}(i)}^{T}
}{d^{2}}
\nonumber \\
&&
\frac{
-\hat{k}_{1}\hat{\rho}
\left(
  -\hat{k}_{2}e_{\beta_{O}(i)}^{T}
  +\hat{k}_{1}
  \left(u^{T}-u_{\beta_{O}(i)}e_{\beta_{O}(i)}^{T}\right)
  -du^{T}
\right)
}{d^{2}}
\nonumber \\
&&
\frac{
-\hat{k}_{1}\left(\hat{k}_{1}-d\right)\hat{v}e_{\beta_{O}(i)}^{T}
}{d^{2}}
\nonumber \\
&=&
\hat{k}_{2}d\hat{\rho}e_{\beta_{O}(i)}^{T}
+\hat{k}_{1}d\hat{\rho} u^{T}
+\hat{k}_{1}d\hat{v}e_{\beta_{O}(i)}^{T}
\end{eqnarray}
Finally we have 
\begin{equation}
\hat{\check{M}}_{B}^{-1}\left(M_{B}+UV^{T}\right)=\hat{z}
\end{equation}
as expected..
\begin{thebibliography}{99}
\bibitem{Sven} Sven Sch\"{o}nherr. Quadratic Programming in Geometric Optimization:
Theory, Implementation, and Applications, Dissertation, Diss. ETH No 14738, ETH
Z\"{u}rich, Institute of Theoretical Computer Science, 2002.
\bibitem{Chvatal} Va\v{s}ek Chv\'{a}tal. \textit{Linear Programming}. W. H. Freeman and Company,
New York, Chapter 8, 1983 
\bibitem{Zielke} Gerhard Zielke. Inversion of Modified Symmetric Matrices. 
\textit{Journal of the Association for Computing Machinery}, Vol. 15, No. 3,
July 1968, pp. 402-408
\bibitem{Frans_Deg} Degeneracy
\end{thebibliography}
\end{document}
