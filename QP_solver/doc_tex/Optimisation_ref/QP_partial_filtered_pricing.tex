% =============================================================================
% The CGAL Reference Manual
% Chapter: Geometric Optimisation
% Section: Smallest Enclosing Sphere
% =============================================================================

\begin{ccRefClass}{QP_partial_filtered_pricing<Traits, NT, ET2NT>}
\ccIndexSubitem[t]{sphere}{smallest enclosing sphere of spheres}
\ccIndexSubitem[t]{smallest enclosing}{sphere of spheres}
\ccIndexSubitem[t]{bounding volumes}{smallest enclosing sphere of spheres}
\ccIndexSubitemSeeAlso[t]{circle}{smallest enclosing sphere of spheres}

% -----------------------------------------------------------------------------
\ccDefinition
%The pricing step of the solvers algorithm consists of determining  whether the
%current solution is optimal, if not the pricing step then determines the next
%entering variable.
%The pricing
%strategy is the part of the solvers pricing step that selects a variable among the
%nonbasic variables that qualify as entering variable. A nonbasic variable
%$x_{j}$ qualifies as an entering variable if its associated $\mu_{j}$, which
%coincides with the notion of reduced cost in the linear case, is strictly
%negative.

\emph{partial filtered pricing} is the combination of \emph{filtered pricing}
and \emph{partial pricing}. As with partial exact pricing a set $S \subseteq N$
of \emph{active variables} is kept and as with filtered pricing
the nonnegativity of some $\mu_{j}$ is \emph{certified} by comparing
$\tilde{\mu}_{j}$, evaluated with the (fast) floating point number type
\ccc{NT}, and the error bound $b$ on
$\left|\tilde{\mu}_{j} - \mu_{j}\right|$: if $\tilde{\mu}_{j} \geq b$,
$\mu_{j}$ is known to be nonnegative.

The pricing step starts with finding the smallest approximated
$\mu_{j}$ in the active set. If the exact value $\mu_{j_{S}}$,
$j_{S}:=\arg\min_{j \in S}\tilde{\mu}_{j}$, is negative, $x_{j_{S}}$ is the
entering variable. Otherwise, a set $V$ of nonactive and nonbasic variables
whose approximated $\mu_{j}$ is strictly negative is determined,
$V:=\{k \in N \setminus S \left|\right. \tilde{\mu}_{k} < 0 \}$. If $V$ is
nonempty, the variable $x_{j_{V}}$ with the smallest approximated
$\mu_{j}$ in $V$ is determined, that is
$j_{V}:=\arg\min_{j \in V}\tilde{\mu}_{j}$. If $\mu_{j_{V}} < 0$, the set of active variables is
augmented and $x_{j_{V}}$ is the entering variable. If on the other hand $V$ is
empty, that is $\tilde{\mu}_{k} \geq 0$ for $k \in N \setminus S$, the
nonnegativity of $\mu_{j}$ for $j \in N \setminus S$ is either certified by the
use of the floating point filter's error bound, that is
$\tilde{\mu}_{j} \geq b$, or the exact value $\mu_{j}$ is evaluated. If
$\mu_{j} <0$, $x_{j}$ is the entering variable, otherwise if none of
the variables $x_{j}$, $j \in N \setminus S$ qualifies as entering variable the
current feasible solution is optimal.

Partial filtered pricing substantially improves performance for
quadratic programs with high \emph{variables to constraints} ratios
and number types \ccc{NT} that are fast compared
to the internal number type \ccc{ET}. The floating point number type \ccc{NT}
and a functor type providing conversion from the exact number type \ccc{ET} to
the floating point type \ccc{NT} can be configured by means of template
parameters.   

\ccInclude{CGAL/QP_partial_filtered_pricing.h}

\ccInheritsFrom
\ccc{QP__filtered_base<Traits, NT, ET2NT>, QP__partial_base<Traits>}

\ccRequirements
\ccIndexRequirements

The class \ccRefName\ expects a model of the concept
\ccc{QPSolverTraits} as its template argument for the first template parameter,
a model of the \cgal\ concept \ccc{Ring number type} for the second template
parameter and a functor type
providing conversion from the exact number type \ccc{Traits::ET} to \ccc{NT}
for the third template parameter. The second template parameter defaults to
\ccc{double} and the third template parameter defaults to a functor providing
conversion from \ccc{Traits::ET} to \ccc{double}.  
Furthermore, as with \emph{full filtered} pricing, the element types
of $A$, $b$, $c$, and $D$ are required to provide implicit lossless conversion
to \ccc{NT}. Currently the number type $NT$ is supported for \ccc{double} only.

\ccTypes \ccIndexClassTypes


\ccCreation
\ccIndexClassCreation
\ccCreationVariable{pfps}

\ccConstructor{ (ET2NT et2nt = ET2NT());}
{creates a variable of type \ccRefName\ .}


\ccUnchecked

\ccAccessFunctions
\begin{ccIndexMemberFunctions}
\ccIndexMemberFunctionGroup{access}


% -----------------------------------------------------------------------------
%\ccPredicates
%\ccIndexMemberFunctionGroup{predicates}


% -----------------------------------------------------------------------------
\ccModifiers
\ccIndexMemberFunctionGroup{modifiers}

%\ccMemberFunction{int pricing();}{determines the index of the entering
%variable, returns $-1$ if the current solution is optimal.}


% -----------------------------------------------------------------------------
\ccHeading{Validity Check}
\ccIndexMemberFunctionGroup{validity check}

% -----------------------------------------------------------------------------
\ccHeading{Miscellaneous}
\ccIndexMemberFunctionGroup{miscellaneous}

\end{ccIndexMemberFunctions}


% -----------------------------------------------------------------------------
\ccSeeAlso
%
    \ccRefIdfierPage{CGAL::QP_full_filtered_pricing<Traits, NT, ET2NT>}\\
    \ccRefIdfierPage{CGAL::QP_partial_exact_pricing<Traits>}\\
% -----------------------------------------------------------------------------

\ccImplementation
\ccIndexImplementation
%\ccIndexSubitem[t]{incremental algorithm}{\ccc{Min_sphere_of_spheres_d}}
%\ccIndexSubitem[t]{farthest-first heuristic}{\ccc{Min_sphere_of_spheres_d}}


\ccExample
In the following example we demonstrate how an implicitly defined matrix may be
represented.
In the \emph{polytope distance} example given earlier the quadratic part of the
objective function is defined as $D=C^{T}C$, where $C$ is given by the point
sets $P$ and $Q$. We may represent $D$ as follows: if $D_{i,j}$ is accessed,
$C_{\bullet, i}^{T}C_{\bullet, j}$ is computed `on-the-fly' and the result is
returned. 

The matrix $D$ is implicitly represented by the classes
\ccc{PD_D_iterator<T>} and \ccc{PD_D_row_iterator<T>}.
The class \ccc{PD_D_iterator<T>} is initialized with the iterator
\ccc{C.begin()} referring to the first point of $C$. When dereferenced at
position $i$, it returns an object of type \ccc{PD_D_row_iterator<T>}
initialized with the index $i$ and the iterator \ccc{C.begin()}. This object is
another iterator implicitly representing the $i$-th row of $D$. When
dereferenced at position $j$, it accesses the points at positions $i$ and $j$,
computes their inner product, and returns the result. The example is sketched
below.

\ccHeading{Polytope Distance (continued)}
\ccIncludeVerbatim{Optimisation_ref/QP_partial_filtered_pricing_strategy_example.C}

\end{ccRefClass}

% ===== EOF ===================================================================

