\begin{ccRefClass}{Quadratic_program_options}

\ccInclude{CGAL/QP_options.h}

\ccDefinition
This is a class used for passing options to the linear and
quadratic programming solvers. Currently, we support only
options referring to
\begin{enumerate}
\item the verbosity, 
\item the pricing strategy (see \ccc{Quadratic_program_pricing_strategy}),
\item the validation mode (see the Validity section of
\ccc{Quadratic_program_solution})
\end{enumerate}
The idea is that this list grows in the future.

\ccCreation
\ccIndexClassCreation
\ccCreationVariable{options}

\ccConstructor{Quadratic_program_options();}
{constructs an instance of \ccRefName\ where all available options
are at their defaults.}

\ccOperations
Here we just have set/get pairs for any option type.

\ccHeading{Verbosity}

\ccMethod{void set_verbosity (int verbosity);}
{sets the verbosity of the solver to the value \ccc{verbosity} when
\ccVar\ is passed to any of the four solution functions. The provided
value must be a number between $0$ and $5$. Verbosity $0$ is the default and
results in the solver running silently. Verbosity $1$ prints a short
summary of every iteration. Higher verbosity values print more information
about the solution process, but these are mainly for debugging 
purposes and have no effect if you compile with 
\texttt{CGAL\_QP\_NO\_ASSERTIONS} or \texttt{NDEBUG}.}

\ccMethod{int get_verbosity () const;}
{returns the verbosity level of \ccVar.}

\ccHeading{Pricing strategy}

\ccMethod{void set_pricing_strategy
  (Quadratic_program_pricing_strategy pricing_strategy);}
{sets the pricing strategy of the solver to the value \ccc{pricing_strategy}
when \ccVar\ is passed to any of the four solution functions. The pricing
strategy controls how the solver proceeds from any intermediate solution.
The provided value must be \ccc{QP_FULL_EXACT} (the default), or 
\ccc{QP_EXACT_BLAND}. The latter value guarantees that the solver does
not internally cycle.}

\ccMethod{Quadratic_program_pricing_strategy get_pricing_strategy() const;}
{returns the pricing strategy of \ccVar.}

\ccHeading{Validation mode}

\ccMethod{void set_validation_flag(bool validate);}
{sets the validation flag of the solver to the value \ccc{validate}.
The default is \ccc{false}. By providing value \ccc{true} you can
tell the solver to automatically check whether the program has
correctly been solved, see the Validity section of the class
\ccc{Quadratic_program_solution}.}

\ccMethod{bool get_validation_flag() const;}
{returns the validation flag of \ccVar.}

\ccExample
\ccReferToExampleCode{QP_solver/cycling.cpp}

\ccSeeAlso

\ccc{Quadratic_program_solution}\\
\ccc{solve_quadratic_program}\\
\ccc{solve_linear_program}\\
\ccc{solve_nonnegative_quadratic_program}\\
\ccc{solve_nonnegative_linear_program}
\end{ccRefClass}
