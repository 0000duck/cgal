 =============================================================================
\begin{ccRefClass}{QP_solver<Traits>}

% -----------------------------------------------------------------------------
\ccDefinition An object of class \ccRefName\ is a data structure
that represents a point $\tilde{x}$ attaining the minimum of a convex
quadratic objective function $f(x)=c^{T}x+x^{T}Dx$ on
a domain $R \subseteq \R^{n}$ given as the intersection $R=\{ x\in\R^n
\mid \mbox{$Ax\ccTexHtml{\gtreqless}{<=,=,>=} b$, $l \le x \le u$}\}$ of halfspaces and
hyperplanes. In other words, an object of class \ccRefName\ represents
a solution to the convex mathematical program
\begin{eqnarray*}
\mbox{(QP)}&\mbox{minimize} & c^{T}x+x^{T}Dx \\
&\mbox{subject to}   & Ax\ccTexHtml{\gtreqless}{<=,=,>=}  b, \\
&& l \leq x \leq u,
\end{eqnarray*}
over the $n$ variables $x=(x_0,\ldots,x_{n-1})$.

Here, $A \in \R^{m \times n}$, $b \in \R^{m}$, $c \in \R^{n}$, and in
addition, the matrix $D \in \R^{n \times n}$ is positive-semidefinite.
The symbol ``$\ccTexHtml{\gtreqless}{<=,=,>=}$'' in the program's constraint $Ax \ccTexHtml{\gtreqless}{<=,=,>=}
b$ means that any row of this may either be an inequality $(Ax)_i \leq b_i$ or
$(Ax)_i \geq b_i$, or an equality $(Ax)_i = b_i$.  Furthermore, $l\in
(\R\cup\{-\infty\})^n$ and $r\in (\R\cup\{\infty\})^n$, so that each
variable $x_i$, $i\in\{0,\ldots,n-1\}$, can be \emph{free} (if
$l_i=-\infty$ and $r_i=\infty$), \emph{bounded from above} (if
$u_i\not=\infty$) and/or \emph{bounded from below} (if
$l_i\not=-\infty)$.  In case $l$ is the zero vector and all entries of
$u$ are $\infty$, the program (QP) is said to be in \emph{standard
form}.

A point $x \in \R^{n}$ with $x\in R$ is called a \emph{feasible
solution}; a feasible solution $x$ is said to be an \emph{optimal
feasible solution} if $f(x')\geq f(x)$ for all $x'\in R$.  If
$R=\emptyset$, the program (QP) is said to be \emph{infeasible};
it is \emph{feasible} if $\R\not=\emptyset$.  If $f$ is bounded from
below on $R$, the quadratic program (QP) is called \emph{bounded},
otherwise \emph{unbounded}.  A quadratic program that is feasible and
bounded always has an optimal feasible solution (which need not be
unique), and we call such a program \emph{optimally solvable}.

While a feasible solution $\tilde{x}$ can be specified by listing its
entries $\tilde{x}_i$, $i\in\{0,\ldots,n-1\}$, it can also be
described by specifying for each variable $x_i$ which of the
constraints $l_i\leq \tilde{x}_i$ and $\tilde{x}_i\leq u_i$ is tight,
and in case none is, by explicitly stating the value $\tilde{x}_i$.
Along these lines, the algorithm will represent a feasible solution
$\tilde{x}$ by a certain
\begin{itemize}
\item subset $B\subseteq \{0,\ldots,n-1\}$,
\item the values $\tilde{x}_i$ for all $i\in B$, and
\item for all $i\in \{0,\ldots,n-1\}\setminus B$ whether
(a)~$\tilde{x}_i=0$, (b)~the \emph{lower bound} $l_i \leq \tilde{x}_i$
holds with equality, or (c)~the \emph{upper bound} $\tilde{x}_i\leq
u_i$ holds with equality.
\end{itemize}
The variables $x_i$, $i\in B$, are called \emph{basic} and the
remaining variables \emph{nonbasic} (with respect to $\tilde{x}$ and
$B$).  Notice that in case of a problem in standard form, all nonbasic
variables are zero (and therefore need not be queried).

\paragraph{Usage.}
An object of class \ccRefName\ is constructed by specifing $A, b, c,
D, l, u$, and for each constraint from $Ax \ccTexHtml{\gtreqless}{<=,=,>=} b$ whether it
needs to hold with inequality ($\leq$ or $\geq$) or with equality.
The algorithm then determines the object's \emph{status}, which
reflects whether the quadratic program (QP) is infeasible, unbounded,
or optimally solvable.  You can query the status by calling
\ccc{status()}.  If according to this the program is optimally
solvable, an optimal feasible solution $\tilde{x}$ is computed
together with the value $f(\tilde{x})$, which you can query using the
access functions \ccc{solution_numerator()},
\ccc{solution_denominator()},
\ccc{basic_original_variables_numerator_begin()},
\ccc{nonbasic_original_variables_numerator_begin()}, and
\ccc{variables_common_denominator()}.  The value $f(\tilde{x})$ and
entries $\tilde{x}_i$ of the computed optimally
feasible solution $\tilde{x}$ are represented as quotients over the
number type \ccc{Traits::ET}.  The routines
\ccc{solution_numerator()},
\ccc{basic_original_variables_numerator_begin()} and
\ccc{nonbasic_original_variables_numerator_begin()} return the
respective numerators; the denominator of $f(\tilde{x})$ is
\ccc{solution_denominator()}, and all entries $\tilde{x}_i$ have
(identical) denominator \ccc{variables_common_denominator()}.
Alternatively, you can also invoke the routines \ccc{solution()} and
\ccc{variables_value_begin()} to query $f(\tilde{x})$ and $\tilde{x}$;
the result of these routines are quotients, i.e., numbers of type
\ccc{CGAL::Quotient<Traits::ET>}.


In case of a problem (QP) in standard form it suffices to query the
values of the basic variables of $\tilde{x}$ (as returned by
\ccc{basic_original_variables_numerator_begin()}); the values
$\tilde{x}_i$ of all other (nonbasic) variables are, as explained
above, zero in this case.

In case the algorithm determines that the program (QP) is unbounded,
it computes a feasible solution $\tilde{x}$ and an \emph{unbounded
direction} $w\in\R^n$ such that $\tilde{x}-\alpha w$ is a feasible
solution of (QP) for any real $\alpha\geq 0$ with the property that
$g(\alpha):= f(\tilde{x}-\alpha w)$ is not bounded from below on
$\R_+$.  The feasible solution $\tilde{x}$ is queried in the same way
as the optimal feasible solution in the case when the program is
optimally solvable; the unbounded direction $w$ can be accessed using
\ccc{unbounded_direction_begin()}.

When (QP) is either unbounded or optimally solvable, the feasible
solution $\tilde{x}$ computed in this way is called the instance's
\emph{current solution}.

\paragraph{Specializations.}
In order to improve performance, the algorithm can be tailored to
specific variants of the quadratic program.  In particular, if the
program is (at compile time) known to be linear (i.e., $D=0$), or in
standard form, or symmetric (meaning $D^{T}=D$), or if the constraint
matrix $A$ is known to have full row rank and $Ax \ccTexHtml{\gtreqless}{<=,=,>=} b$
consists of equalities only then compile-time flags in the traits
class \ccc{Traits} can be used to specialize the algorithm.  The
resulting code will take advantage of the given additional knowledge
and therefore run (much) faster. The examples below illustrate this.

\paragraph{Exactness of the solution.}
In contrast to many other quadratic programming solvers, the
algorithm's output (\ccc{status()}, \ccc{solution()}, etc.)\ is
\emph{correct} for all inputs.  (In other words, the optimal feasible
solution computed by the algorithm is not a point $\tilde{x}'$ that is
\emph{close}, $\|\tilde{x}'-\tilde{x}\|\le \epsilon$, to an actual
optimal feasible solution $\tilde{x}$ of the input program, but it
computes an optimal feasible solution.  In contrast to this, many
solvers merely compute \emph{floating-point approximations} to optimal
feasible solutions and cannot correctly decide for all possible input
programs (QP) whether they are infeasible or not.)

For this to work, the algorithm requires that the number type
\ccc{Traits::ET}, which is used for internal calculations, is an
\emph{exact number type}, meaning that it correctly models the
rational numbers (in particular, without rounding errors).
(Technically,
\ccc{Number_type_traits<Traits::ET>::Has_exact_ring_operations} must
be \ccc{CGAL::Tag_true}, see the documentation of concept
\ccc{QPSolverTraits}.)

\paragraph{Pricing strategies.}
The solver's underlying algorithm is an \emph{iterative method}.
After having found an initial feasible solution the algorithm steps
from feasible solution to feasible solution, always improving the
solution.  The performance of the algorithm is greatly influenced by
the \emph{pricing strategy} that is used during the so-called
\emph{pricing step}. In this repeatedly occurding phase, the algorithm
determines whether or not the current solution is optimal, and if it
is not, selects one from (usually) many possible \emph{improved
solutions}. The pricing strategy determines \emph{which} improved
solution is taken for the next iteration.

Four ready-to-use pricing strategies are provided: \emph{full exact
pricing}, \emph{full filtered pricing}, \emph{partial exact pricing},
and \emph{partial filtered pricing}.  (Currently, it is not possible
to run the algorithm with a user-defined pricing strategy.)

The two filtered pricing strategy allow for an \emph{inexact} number
type \ccc{Traits::NT} like e.g.\ \ccc{double} to be used during the
pricing step: instead of running expensive exact arithmetic
internally, the algorithm will then use the fast number-type
\ccc{Traits::NT} to determine the next improved solution --- still in
such a way that exact solutions are obtained in the end. If the
inexact number type \ccc{Traits::NT} is considerable faster than
\ccc{Traits::ET}, filtered pricing can result in increased performance
of the algorithm.

The two partial pricing strategies incorporate the techniques of
\emph{random sampling} to speed up the pricing step.  They work best
for quadratic programs with high \emph{variables-to-constraints ratio}
$n/m$.

%Any feasible solution of (QP) can be specified by a
%subset $B\subseteq\{1,\ldots,n\}$ of the variables and their values
%$x_{B}$ in such a way that all variables $\{1,\ldots,n\}\setminus B$
%have value zero; in this case, the variables in $B$ are said to be
%\emph{basic} and the variables in $\{1,\ldots,n\}\setminus B$ are
%called \emph{nonbasic}.  (Nonbasic variables always have value zero; a
%basic variable might have value zero.)

\ccInclude{CGAL/QP_solver.h}

\ccRequirements
\ccIndexRequirements

The class \ccRefName\ expects a model of the concept
\ccc{QPSolverTraits} as its template argument.

\ccTypes \ccIndexClassTypes

\ccNestedType{ET}{a typedef to \ccc{Traits::ET}.}

\ccNestedType{A_iterator}{a typedef to \ccc{Traits::A_iterator}.\\
Note: as described in the documentation of concept
\ccc{QPSolverTraits}, \ccc{A_iterator} iterates over the
\emph{columns} of the matrix $A$.}

\ccNestedType{B_iterator}{a typedef to \ccc{Traits::B_iterator}.}

\ccNestedType{C_iterator}{a typedef to \ccc{Traits::C_iterator}.}

\ccNestedType{D_iterator}{a typedef to \ccc{Traits::D_iterator}. \\
Note: as described in the documentation of concept
\ccc{QPSolverTraits}, \ccc{D_iterator} iterates over the
\emph{rows} of the matrix $D$.}
\ccNestedType{L_iterator}{a typedef to \ccc{Traits::L_iterator}.}
\ccNestedType{U_iterator}{a typedef to \ccc{Traits::U_iterator}.}
\ccNestedType{FL_iterator}{a typedef to \ccc{Traits::FL_iterator}.}
\ccNestedType{FU_iterator}{a typedef to \ccc{Traits::FU_iterator}.}

\ccNestedType{Row_type}{is a typedef to \ccc{Traits::Row_type}.}

\ccNestedType{Row_type_iterator}{is a typedef to
\ccc{Traits::Row_type_iterator}.}

\ccNestedType{Is_linear}{a typedef to \ccc{Traits::Is_linear}.}

\ccNestedType{Is_symmetric}{a typedef to \ccc{Traits::Is_symmetric}.}

\ccNestedType{Has_equalities_only_and_full_rank}{is a typedef to
\ccc{Traits::Has_equalities_only_and_full_rank}.}

\ccNestedType{Is_in_standard_form}{a typedef to \ccc{Traits::Is_in_standard_form}.}

%\ccNestedType{Use_perturbation}{is a typedef to
%\ccc{Traits::Use_perturbation}.}

%\ccNestedType{Pricing_strategy}{a typedef to
%\ccc{QP_pricing_strategy<Traits>}. \ccc{Pricing_strategy} is a base
%class of all four ready-to-use pricing strategies.}

\ccNestedType{Basic_variable_value_iterator}{a STL random access
  iterator with value type \ccc{ET} used to access the numerators of
  the basic variables of a feasible (or optimal feasible) solution,
  see \ccc{basic_original_variables_numerator_begin()} below.}

\ccNestedType{Basic_variable_index_iterator}{a STL random access
  iterator with value type \ccc{int} used to access the indices of the
  basic variables of a feasible (or optimal feasible) solution, see
  \ccc{basic_original_variables_numerator_begin()} below.}

\ccNestedType{Nonbasic_variable_iterator}{a STL random access iterator
  with value type \ccc{ET} used to access the nonbasic variables of a
  feasible (or optimal feasible) solution, see
  \ccc{nonbasic_original_variables_begin()} below.}

\ccNestedType{Nonbasic_variable_index_iterator}{a STL random access
  iterator with value type \ccc{int} used to access the indices of the
  nonbasic variables of a feasible (or optimal feasible) solution, see
  \ccc{nonbasic_original_variables_begin()} below.}

\ccNestedType{Variable_value_iterator}{a STL random access iterator
with value type \ccc{CGAL::Quotient<ET>} used to access all (i.e.,
basic and nonbasic) variables of a feasible (or optimal feasible)
solution, see \ccc{variables_value_begin()} below.}

%KKT
%\ccNestedType{Lambda_value_iterator}{a STL random access iterator with value type}

%The following enum type is provided by the class \ccRefName\ :

\ccEnum{enum Strategy { FULL_EXACT_PRICING, FULL_FILTERED_PRICING,
PARTIAL_EXACT_PRICING, PARTIAL_FILTERED_PRICING };}{enumeration used
to specify one of the four pricing strategies to use.}

\ccEnum{enum Status { INFEASIBLE, UNBOUNDED, OPTIMAL };}{enumeration used to
indicate the status of a quadratic program.}

\ccEnum{enum Bound_identifier { ZERO, LOWER, UPPER };}{enumeration
used to describe the assignment of a nonbasic variable $x_i$ of a
feasible solution to either the value zero, to its lower bound $l_i$,
or to its upper bound $u_i$, see \ccc{nonbasic_variables_begin()}.}

\ccCreation
\ccIndexClassCreation
\ccCreationVariable{qp}

\ccConstructor{( int n, int m, A_iterator a_it, B_iterator b_it,
C_iterator c_it, D_iterator d_it, Row_type_iterator r_it =
CGAL::Const_oneset_iterator<Row_type>( Rep::EQUAL), Strategy
strategy = FULL_EXACT_PRICING)} {creates a variable of type \ccRefName\ and solves the
quadratic program (QP) assuming the vector $l$ is the zero vector and
all entries of $u$ are $\infty$ (i.e., (QP) is assumed to be in
standard form).  The matrix $A$ of (QP) is given by \ccc{a_it}, which
is an iterator over the columns of $A$.  Similarly, \ccc{d_it}
iterates over the rows of $D$. The iterator \ccc{r_it} specifies for
each row of $Ax \ccTexHtml{\gtreqless}{<=,=,>=} b$ whether it is an inequality ($\leq$ or
$\geq$) or an equality constraint.  \ccRequire
\ccc{a_it} iterates over $n$ columns each of which has $m$ elements,
\ccc{d_it} iterates over $n$ rows each of which having $n$ elements,
and similarly \ccc{c_it} and \ccc{r_it} iterate over $n$ elements, and
\ccc{b_it} iterates over $m$ elements.  Furthermore, the matrix $D$
specified by \ccc{d_it} must be positive-semidefinite, and $n>0$.}

\ccConstructor{( int n, int m, A_iterator a_it, B_iterator b_it,
C_iterator c_it, D_iterator d_it, Row_type_iterator r_it, FL_iterator
fl_it, L_iterator lb_it, FU_iterator fu_it, U_iterator ub_it,
Strategy
strategy = FULL_EXACT_PRICING)}
{creates a variable of type \ccRefName\ and solves the quadratic
program (QP).  The matrix $A$ of (QP) is given by \ccc{a_it}, which is
an iterator over the columns of $A$.  Similarly, \ccc{d_it} iterates
over the rows of $D$. The iterator \ccc{r_it} specifies for each row
of $Ax \ccTexHtml{\gtreqless}{<=,=,>=} b$ whether it is an inequality ($\leq$ or $\geq$) or
an equality constraint. \ccc{fl_it} iterates over $n$ Boolean values,
one for each variable $x_i$, $i\in\{0,\ldots,n-1\}$, and specifies
whether $l_i>-\infty$.  If so, the corresponding entry from the
iterator \ccc{lb_it} is taken as $l_i$; otherwise $l_i$ is set to
$-\infty$. Similarly, \ccc{fu_it} iterates over $n$ Boolean values
specifying whether $u_i$ is finite or not; if it is, the corresponding
entry from the iterator \ccc{ub_it} is taken as $u_i$, otherwise
$u_i:=\infty$.  \ccRequire \ccc{a_it} iterates
over $n$ columns each of which has $m$ elements, \ccc{d_it} iterates
over $n$ rows each of which having $n$ elements, and similarly
\ccc{c_it}, \ccc{r_it}, \ccc{fl_it}, \ccc{lb_it}, \ccc{fu_it}, and
\ccc{ub_it} iterate over $n$ elements, and \ccc{b_it} iterates over
$m$ elements.  Furthermore, the matrix $D$ specified by \ccc{d_it}
must be positive-semidefinite, and $n>0$.}

\ccUnchecked

\ccAccessFunctions
\begin{ccIndexMemberFunctions}
\ccIndexMemberFunctionGroup{access}

\ccMemberFunction{Status status() const;}{returns the status of the
quadratic program, that is, whether (QP) is infeasible, unbounded, or
optimally solvable.}

\ccMemberFunction{CGAL::Quotient<ET> solution( ) const;}{ returns the
 value $f(\tilde{x})$ where $\tilde{x}$ is the instance's current
 solution (i.e., some feasible solution in case (QP) is unbounded and
 an optimal feasible solution in case the program is optimally
 solvable).  The number $f(\tilde{x})$ is returned as a quotient over
 \ccc{ET}. \ccPrecond \ccc{status()} returns \ccc{UNBOUNDED} or
 \ccc{OPTIMAL}.}

\ccMemberFunction{Basic_variable_value_iterator
basic_variables_value_begin( ) const;}{returns a const-iterator
referring to the numerator of the first basic variable of the current
solution $\tilde{x}$.  Normally, you use the resulting iterator
together with the iterator returned by
\ccc{basic_variables_index_begin()}; while the former iterates over
the numerators of the basic variables, the latter iterates over the
indices (i.e., numbers in $\{0,\ldots,n-1\}$) of the corresponding
variables. \ccPrecond \ccc{status()} returns \ccc{UNBOUNDED} or \ccc{OPTIMAL}.}

\ccMemberFunction{Basic_variable_value_iterator
basic_variables_value_end( ) const;}{returns the past-the-end
const-iterator corresponding to \ccc{basic_variables_value_begin()}.
\ccPrecond \ccc{status()} returns \ccc{UNBOUNDED} or \ccc{OPTIMAL}.}

\ccMemberFunction{Basic_variable_index_iterator
basic_variables_index_begin( ) const;}{returns a const-iterator referring
to the first index in the list of basic variables.
\ccPrecond \ccc{status()} returns \ccc{UNBOUNDED} or \ccc{OPTIMAL}.}

\ccMemberFunction{Basic_variable_index_iterator
basic_variables_index_end ( ) const;}{returns the past-the-end
const-iterator corresponding to
\ccc{basic_variables_index_begin()}. \ccPrecond \ccc{status()} returns
\ccc{UNBOUNDED} or \ccc{OPTIMAL}.}

\ccMemberFunction{Nonbasic_variable_value_iterator
nonbasic_variables_begin( ) const;}{returns a const-iterator referring
to the first nonbasic variable of the current solution $\tilde{x}$.
The value type of the iterator is \ccc{Bound_identifier}; a value of
\ccc{ZERO} means that the corresponding nonbasic variable $x_i$ is
zero in the current solution $\tilde{x}$, a value of
\ccc{LOWER} means that $\tilde{x}_i=l_i$, and a value of \ccc{UPPER}
implies $\tilde{x}_i=u_i$. Normally, you use the resulting iterator
together with the iterator returned by
\ccc{nonbasic_variables_index_begin()}; while the former iterates over
the nonbasic variables, the latter iterates over the indices (i.e.,
numbers in $\{0,\ldots,n-1\}$) of the corresponding variables. \\
Note: in case of a program (QP) in standard form, this iterator will
have yield \ccc{ZERO} for all nonbasic variables. \ccPrecond
\ccc{status()} returns \ccc{UNBOUNDED} or \ccc{OPTIMAL}.}

\ccMemberFunction{Nonbasic_variable_value_iterator
nonbasic_variables_end( ) const;}{returns the past-the-end
const-iterator corresponding to \ccc{nonbasic_variables_begin()}.
\ccPrecond \ccc{status()} returns \ccc{UNBOUNDED} or \ccc{OPTIMAL}.}

\ccMemberFunction{Nonbasic_variable_index_iterator
nonbasic_variables_index_begin( ) const;}{returns a const-iterator referring
to the first index in the list of nonbasic variables.
\ccPrecond \ccc{status()} returns \ccc{UNBOUNDED} or \ccc{OPTIMAL}.}

\ccMemberFunction{Nonasic_variable_index_iterator
nonbasic_variables_index_end ( ) const;}{returns the past-the-end
const-iterator corresponding to
\ccc{nonbasic_variables_index_begin()}. \ccPrecond \ccc{status()}
returns \ccc{UNBOUNDED} or \ccc{OPTIMAL}.}

\ccMemberFunction{Variable_value_iterator variables_value_begin()
const;}{returns a const-iterator referring to the entry $\tilde{x}_0$
of the current solution $\tilde{x}$. \ccPrecond
\ccc{status()} returns \ccc{UNBOUNDED} or \ccc{OPTIMAL}.}

\ccMemberFunction{Variable_value_iterator variables_value_end()
const;}{returns the past-the-end const-iterator corresponding to
\ccc{variables_value_begin()}. \ccPrecond \ccc{status()} returns
\ccc{UNBOUNDED} or \ccc{OPTIMAL}.}


%KKT
%% \begin{ccAdvanced}    
%% \ccHeading{Access to $\lambda$}

%% \ccMemberFunction{Lambda_value_iterator
%%     lambda_value_begin( ) const;}{returns a const-iterator referring to the
%%     $\lambda$ associated with the first constraint in user defined order of the
%%     constraints.}

%% \ccMemberFunction{Lambda_value_iterator
%%     lambda_value_end( ) const;}{returns the corresponding pas-the-end
%%     const-iterator.}
%% \end{ccAdvanced}
    
\ccHeading{Access to quadratic program}

\ccMemberFunction{int  number_of_variables  ( ) const;}{returns the number of
variables of the quadratic program as specified in the constructor call.}

\ccMemberFunction{int  number_of_constraints( ) const;}{returns the number of
constraints of the quadratic program as specified in the constructor call.}

\ccMemberFunction{A_iterator  a_begin( ) const;}{returns a random access
const-iterator referring to the first column of the matrix $A$.}

\ccMemberFunction{A_iterator  a_end( ) const;}{returns a random access
const-iterator referring to the last column of of the matrix $A$.}

\ccMemberFunction{B_iterator b_begin( ) const;}{returns a random
access const-iterator referring to the first entry of the vector $b$.}

\ccMemberFunction{B_iterator  b_end( ) const;}{returns a random access
const-iterator referring to the last entry of the vector $b$.}

\ccMemberFunction{C_iterator  c_begin( ) const;}{returns a random access
const-iterator referring to the first entry of the vector $c$.}

\ccMemberFunction{C_iterator  c_end( ) const;}{returns a random access
const-iterator referring to the last entry of the vector $c$.}

\ccMemberFunction{D_iterator  d_begin( ) const;}{returns a random access
const-iterator referring to the first row of the matrix $D$.}

\ccMemberFunction{D_iterator  d_end( ) const;}{returns a random access
const-iterator referring to the last row of the matrix $D$.}

\ccMemberFunction{Row_type_iterator  row_type_begin( ) const;}{returns a
random access const-iterator referring to the first entry of the row
type vector (see \ccc{r_it} in the documentation of the constructor).}

\ccMemberFunction{Row_type_iterator  row_type_end( ) const;}{returns a
random access const-iterator referring to the last entry of the row
type vector  (see \ccc{r_it} in the documentation of the constructor).}
% -----------------------------------------------------------------------------
\ccPredicates
\ccIndexMemberFunctionGroup{predicates}

\ccMemberFunction{bool is_basic( int i) const;}{returns \ccc{true} if
and only if variable $x_{i}$, $i \in \{0,\ldots,n-1\}$, is basic in
the current solution.\ccPrecond \ccc{status()} returns \ccc{UNBOUNDED}
or \ccc{OPTIMAL}}

% -----------------------------------------------------------------------------
%\ccModifiers
%\ccIndexMemberFunctionGroup{modifiers}



% -----------------------------------------------------------------------------
\ccHeading{Validity Check}
\ccIndexMemberFunctionGroup{validity check}
%\ccIndexSubitem[t]{validity check}{\ccc{Min_sphere_d}}
An object \ccVar\ is valid if and only if \ccc{qp.status()} returns
one of the following three values and their corresponding requirements
are met.
\begin{itemize}
\item \ccc{OPTIMAL}: The feasible solution $\tilde{x}$ computed by the
algorithm fulfills $f(\tilde{x}) \leq f(x)$ for all $x \in R$.
\item \ccc{INFEASIBLE}: $R = \emptyset$.
\item \ccc{UNBOUNDED}: The feasible solution $\tilde{x}$ and the
unbounded direction $w$ computed by the algorithm fulfill
\begin{itemize}
\item[(i)] $\tilde{x}-\alpha w$ is feasible for all $\alpha\ge 0$, and
\item[(ii)] the function $g(\alpha):= f(\tilde{x}-\alpha w)$ is not
bounded from below on the interval $[0,\infty)$.
\end{itemize}
\end{itemize}

\ccMemberFunction{ bool is_solution_valid() const;}{ returns \ccc{true} iff
  \ccVar\ is valid.}

% -----------------------------------------------------------------------------
%\ccHeading{Miscellaneous}
%\ccIndexMemberFunctionGroup{miscellaneous}


\end{ccIndexMemberFunctions}


% -----------------------------------------------------------------------------
\ccSeeAlso
%
    \ccRefIdfierPage{CGAL::QP_MPS_instance<IT,ET>}
%    \ccRefIdfierPage{CGAL::QP_full_exact_pricing<Traits>}\\
%    \ccRefIdfierPage{CGAL::QP_full_filtered_pricing<Traits, NT, ET2NT>}\\
%    \ccRefIdfierPage{CGAL::QP_partial_exact_pricing<Traits>}\\
%    \ccRefIdfierPage{CGAL::QP_partial_filtered_pricing<Traits, NT, ET2NT>}\\
% -----------------------------------------------------------------------------

\ccImplementation
\ccIndexImplementation
%\ccIndexSubitem[t]{incremental algorithm}{\ccc{Min_sphere_of_spheres_d}}
%\ccIndexSubitem[t]{farthest-first heuristic}{\ccc{Min_sphere_of_spheres_d}}

The current implementation does not guarantee termination for all
instances; in rare cases, the algorithm cycles forever.  (Termination
can always be guaranteed by using symbolic perturbation, and this
feature will be added in a future release.)

\ccExample

The following example solves the quadratic programming problem
\begin{eqnarray*}
\mbox{(QP)}&\mbox{minimize} & x_0^2 + 3x_1 \\
&\mbox{subject to}   & x_0+2x_1=1, \\
&& 0 \le  x_0 \le \infty, \\
&& 0 \le  x_1 \le 1.
\end{eqnarray*}

\ccIncludeExampleCode{QP_solver/qp_solver1.cpp}

Please refer also to the documentation of class
\ccc{CGAL::QP_MPS_instance<IT,ET>}; using it, you can read programs of
the form (QP) from MPS-files.
\end{ccRefClass}

% ===== EOF ===================================================================

