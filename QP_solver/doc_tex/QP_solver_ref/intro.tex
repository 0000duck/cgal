\ccRefChapter{Linear and Quadratic Programming Solver}
\label{chapter:QPsolver}
\ccChapterAuthor{Kaspar Fischer \and Bernd G{\"a}rtner \and Sven Sch{\"o}nherr
\and Frans Wessendorp}

\section{Classified Reference Pages}

\ccHeading{Concepts}
\ccRefConceptPage{QuadraticProgramInterface}\\
$\quad$ (for quadratic programs with variable bounds $l\leq x \leq u$) \\
\ccRefConceptPage{LinearProgramInterface} \\
$\quad$(for linear programs with variable bounds $l\leq x \leq u$)\\
\ccRefConceptPage{NonnegativeQuadraticProgramInterface}\\
$\quad$ (for quadratic programs with variable bounds $x\geq 0$) \\
\ccRefConceptPage{NonnegativeLinearProgramInterface}\\
$\quad$ (for linear programs with variable bounds $x\geq 0$)

\ccRefConceptPage{MPSFormat}\\
$\quad$ (the format used for reading and writing linear and quadratic 
programs)

\ccHeading{Classes}

There is a class that represents the solution of a linear
or quadratic program. An instance of this class is returned by any of 
the solution functions below.

%\ccRefIdfierPage{Quadratic_program_solution}

We offer a number of predefined models for the above program concepts.
The following two are simultaneously models for all four concepts and
are probably the most convenient models; they allow you to construct
linear or quadratic programs entry by entry, or from streams in
\ccc{MPSFormat}. At any time, you can query these programs for
linearity and nonnegativity and thus select the appropriate 
solution function.

\ccc{Quadratic_program<NT>}\\
$\quad$ (for linear and quadratic programs that own their 
data and are built entry-wise)\\
\ccc{Quadratic_program_from_mps<NT>}\\
$\quad$ (for linear and quadratic programs read from an input stream in 
\ccc{MPSFormat}; the constructed program can also be manipulate entry-wise)

Then there are specific models for any of the four program concepts above; 
these are useful if you want to maintain the program data yourself, and
they simply wrap random access iterators over the program data. 

\ccc{Quadratic_program_from_iterators<A_it, B_it,  R_it, FL_it, L_it, FU_it, U_it, D_it, C_it>}\\
$\quad$ (for quadratic programs that wrap given iterators, without copying 
data) \\
\ccc{Quadratic_program_from_pointers<NT>}\\
$\quad$ (for quadratic programs that wrap given pointers, without copying 
data) 

\ccc{Linear_program_from_iterators<A_it, B_it,  R_it, FL_it, L_it, FU_it, U_it, C_it>}\\
$\quad$ (for linear programs that wrap given iterators, without copying 
data) \\
\ccc{Linear_program_from_pointers<NT>}\\
$\quad$ (for linear programs that wrap given pointers, without copying 
data) 

\ccc{Nonnegative_quadratic_program_from_iterators<A_it, B_it, R_it,  D_it, C_it>}\\
$\quad$ (for nonnegative quadratic programs, wrapping given iterators)\\
\ccc{Nonnegative_quadratic_program_from_pointers<NT>}\\
$\quad$ (for nonnegative quadratic programs, wrapping given pointers)

\ccc{Nonnegative_linear_program_from_iterators<A_it, B_it, R_it, C_it>}\\
$\quad$ (for nonnegative linear programs, wrapping given iterators)\\
\ccc{Nonnegative_linear_program_from_pointers<NT>}\\
$\quad$ (for nonnegative linear programs, wrapping given pointers)

\ccHeading{Functions}

In case you want to construct a program from complicated iterators
(whose types you don't know, or simply don't want to bother with), 
we provide four makers.

\ccc{make_quadratic_program_from_iterators}\\
\ccc{make_linear_program_from_iterators}\\
\ccc{make_nonnegative_quadratic_program_from_iterators}\\
\ccc{make_nonnegative_linear_program_from_iterators}

There are four functions to solve a program, one for each program
concept. 

\ccc{solve_quadratic_program}\\
\ccc{solve_linear_program}\\
\ccc{solve_nonnegative_quadratic_program}\\
\ccc{solve_nonnegative_linear_program}

Programs can be written to an output stream in \ccc{MPSFormat}, using
one of the following four functions.

\ccc{print_quadratic_program}\\
\ccc{print_linear_program}\\
\ccc{print_nonnegative_quadratic_program}\\
\ccc{print_nonnegative_linear_program}
