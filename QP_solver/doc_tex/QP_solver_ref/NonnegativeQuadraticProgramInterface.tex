\begin{ccRefConcept}{NonnegativeQuadraticProgramInterface}

\ccDefinition
A model of \ccRefName\ describes a convex quadratic program of the form
%%
\begin{eqnarray*}
\mbox{(QP)}& \mbox{minimize} 
& \qpx^{T}D\qpx+\qpc^{T}\qpx+c_0 \\
&\mbox{subject to}   & A\qpx\qprel \qpb, \\
&                    & \qpx \geq 0
\end{eqnarray*}
%%
in $n$ real variables $\qpx=(x_0,\ldots,x_{n-1})$.
Here, 
\begin{itemize}
\item $A$ is an $m\times n$ matrix (the constraint matrix), 
\item $\qpb$ is an $m$-dimensional vector (the right-hand side),
\item $\qprel$ is an $m$-dimensional vector of relations 
from $\{\leq, =, \geq\}$, 

\item $D$ is a symmetric positive-semidefinite $n\times n$ matrix (the
  quadratic objective function),

\item $\qpc$ is an $n$-dimensional vector (the linear objective
  function), and 
\item $c_0$ is a constant.

\end{itemize}


The description is given by appropriate \emph{random-access} 
iterators over the program data, see below. The program therefore 
comes in \emph{dense} representation which includes zero entries.

\ccHasModels
\ccc{Quadratic_program<NT>}\\
\ccc{Quadratic_program_from_mps<NT>}\\
\ccc{Nonnegative_quadratic_program_from_iterators<A_it, B_it, R_it, FL_it, L_it, FU_it, U_it, D_it, C_it>}\\

\ccTypes

\ccNestedType{A_iterator}{A random access iterator type to go 
  columnwise over the constraint matrix $A$. The value type
  is a random access iterator type for an individual column that
  goes over the entries in that column.}

\ccNestedType{B_iterator}{A random access iterator type to go over 
  the entries of the right-hand side $\qpb$.}

\ccNestedType{R_iterator}{A random access iterator type to go over the
  relations $\qprel$. The value type of \ccc{R_iterator} is
  \ccc{CGAL::Comparison_result}.}


\ccNestedType{D_iterator}{A random access iterator type to go rowwise 
  over the matrix $2D$. The value type
  is a random access iterator type for an individual row that
  goes over the entries in that row, up to (and including) the
  entry on the main diagonal.}


\ccNestedType{C_iterator}{A random access iterator type to go over the
  entries of the linear objective function vector $c$.}


\ccOperations

\ccCreationVariable{qp}

\ccMethod{int get_n() const;}{returns the number $n$ of variables (number
  of columns of $A$) in \ccVar.}

\ccMethod{int get_m() const;}{returns the number $m$ of constraints
  (number of rows of $A$) in \ccVar.}

\ccMethod{A_iterator get_a() const;}{returns an iterator over the columns
  of $A$. The corresponding past-the-end iterator is \ccc{get_a()+get_n()}.
  For $j=0,\ldots,n-1$, $\ccc{*(get_a()+j)}$ is a random access
  iterator for column $j$.}

\ccMethod{B_iterator get_b() const;}{returns an iterator over the entries
  of $\qpb$. The corresponding past-the-end iterator is 
        \ccc{get_b()+get_m()}.}

\ccMethod{R_iterator get_r() const;}{returns an iterator over the entries
  of $\qprel$. The corresponding past-the-end iterator is 
        \ccc{get_r()+get_m()}.
  The value \ccc{CGAL::SMALLER} stands
  for $\leq$, \ccc{CGAL::EQUAL} stands for $=$, and \ccc{CGAL::LARGER}
  stands for $\geq$.}


\ccMethod{D_iterator get_d() const;}{returns an iterator for the rows of
  $2D$. For $i=0,\ldots,n-1$, $\ccc{*(get_d()+i)}$ is a random access
  iterator for the entries in row $i$ \emph{below or on the diagonal}.
  The valid range of this iterator is 
  guaranteed to have length $i+1$ but not more. Values to the right 
  of the diagonal are deduced from the symmetry requirement on $D$.}


\ccMethod{C_iterator get_c() const;}{returns an iterator over the entries
  of $\qpc$.}

\ccMethod{std::iterator_traits<C_iterator>::value_type get_c0() const;}
{returns the constant term $c_0$ of the objective function.}


\ccRequirements

The value types of all iterator types (nested iterator types,
respectively, for \ccc{A_iterator} and \ccc{D_iterator}) must be
convertible to some common \ccc{IntegralDomain} \ccc{ET}.

\ccSeeAlso
The models

\ccc{Quadratic_program<NT>}\\
\ccc{Quadratic_program_from_mps<NT>}\\
\ccc{Nonnegative_quadratic_program_from_iterators<A_it, B_it, R_it, D_it, C_it>}
and the other concepts

\ccc{QuadraticProgramInterface}\\
\ccc{LinearProgramInterface}\\
\ccc{NonnegativeLinearProgramInterface}
\end{ccRefConcept}
