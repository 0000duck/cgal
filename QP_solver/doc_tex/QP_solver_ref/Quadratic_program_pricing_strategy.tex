\begin{ccRefClass}{Quadratic_program_pricing_strategy}

\ccInclude{CGAL/QP_options.h}

\ccDefinition
This is an enumeration type containing the values
\ccc{QP_CHOOSE_DEFAULT}, \ccc{QP_DANTZIG}, 
\ccc{QP_PARTIAL_DANTZIG}, \ccc{QP_FILTERED_DANTZIG},
\ccc{QP_PARTIAL_FILTERED_DANTZIG}, and\ccc{QP_BLAND}. 

It indicates the pricing strategy to be used in
solving a linear or quadratic program. Here we describe when to 
choose which strategy.

\ccHeading{\ccc{QP_CHOOSE_DEFAULT}}
This is the default value of the pricing strategy in
\ccc{Quadratic_program_options}, and it lets the solver choose the
strategy that it thinks is most appropriate for the problem at hand.
There are only few reasons to deviate from this default, but you are
free to experiment, of course.

\ccHeading{\ccc{QP_DANTZIG}} 
If the input type is \textbf{not} \ccc{double}, this is usually the best choice
for `''typical'' sparse linear and quadratic programs of medium size,
where the number of variables has roughly the same order of magnitude 
as the number of constraints. 

\ccHeading{\ccc{QP_PARTIAL_DANTZIG}}
If the input type is \textbf{not} \ccc{double}, this is usually the best choice
for linear and quadratic programs with few (up to 100, say), constraints,
but a possibly large number of variables, or the other way around 
(few variables, but possibly many constraints) 

\ccHeading{\ccc{QP_FILTERED_DANTZIG}} 
If the input type \textbf{is} \ccc{double}, this is usually the best choice
for `''typical'' sparse linear and quadratic programs of medium size,
where the number of variables has roughly the same order of magnitude 
as the number of constraints. 

{\bf Note:} filtered strategies may occasionally fail due to double 
exponent overflows. In this case, the slower fallback option is
the non-filtered variant \ccc{QP_DANTZIG} of this strategy.

\ccHeading{\ccc{QP_PARTIAL_FILTERED_DANTZIG}}
If the input type \textbf{is} \ccc{double}, this is usually the best choice
for linear and quadratic programs with few (up to 100, say), constraints,
but a possibly large number of variables, or the other way around 
(few variables, but possibly many constraints). 

{\bf Note:} filtered strategies may occasionally fail due to double 
exponent overflows. In this case, the slower fallback option is
the non-filtered variant \ccc{QP_PARTIAL_DANTZIG} of this strategy.

\ccHeading{\ccc{QP_BLAND}}
This is hardly ever the most efficient choice, but it is guaranteed
to avoid internal cycling of the solution algorithm.

\ccSeeAlso

\ccc{Quadratic_program_options}
\end{ccRefClass}
