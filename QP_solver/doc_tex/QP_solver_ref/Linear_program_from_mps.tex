\begin{ccRefClass}{Linear_program_from_mps<NT>}

\ccInclude{CGAL/QP_models.h}

\ccDefinition
An object of class \ccRefName\ describes a linear program of the form
%%
\begin{eqnarray*}
\mbox{(QP)}& \mbox{minimize} & c^{T}x+c_0 \\
&\mbox{subject to}   & Ax\qprel b, \\
&                    & l \leq x \leq u
\end{eqnarray*}
%%
in $n$ real variables $x=(x_0,\ldots,x_{n-1})$.
Here, 
\begin{itemize}
\item $A$ is an $m\times n$ matrix (the constraint matrix), 
\item $b$ is an $m$-dimensional vector (the right-hand side),
\item $\qprel$ is an $m$-dimensional vector of relations 
from $\{\leq, =, \geq\}$, 
\item $l$ is an $n$-dimensional vector of lower
bounds for $x$,
\item $u$ is an $n$-dimensional vector of upper bounds for
$x$, 
\item $c$ is an $n$-dimensional vector (the linear objective
  function), and 
\item $c_0$ is a constant.
\end{itemize}

The program data are read from an input stream in \ccc{MPSFormat}. This is
a commonly used format for encoding linear and quadratic programs that
is understood by many solvers. All values are expected to be readable
into type \ccc{NT}.

\textbf{Note:} The space required to store the program is $\Theta(nm +
n^2)$, even if these matrix $A$ is very sparse. This might be
prohibitive. In this case, you may use the model
\ccc{Sparse_linear_program_from_mps<NT>} whose space requirements
are bounded by the number of nonzero entries in the program
description. In this latter model, access to the iterators in
\ccc{LinearProgramInterface} will be a little slower, though.

As a rule of thumb, however, if there is a need for the sparse model,
then \cgal's linear programming solver will probably not be able to 
solve it anyway.

\ccIsModel
\ccc{LinearProgramInterface}

\ccCreation
\ccIndexClassCreation
\ccCreationVariable{lp}

\ccConstructor{Sparse_linear_program_from_mps(std::istream& in)} {reads \ccVar\ from the input stream \ccc{in}.}

\ccOperations

\ccMethod{bool is_valid() const;}{returns \ccc{true} if and only if an
MPS-encoded linear program could be extracted from the input stream.}

\ccMethod{const std::string& name_of_variable (int i) const;} {returns the name of the $i$-th variable.\ccPrecond \ccVar\ccc{.is_valid()}}

\ccMethod{bool is_nonnegative() const;}{returns \ccc{true} if and only if the linear program read into \ccVar\ is a nonnegative program
\ccPrecond \ccVar\ccc{.is_valid()}.}

\ccMethod{const std::string& error() const;}{returns an error message explaining why the input is not in MPS format \ccPrecond \ccc{!} \ccVar\ccc{.is_valid()}}

\ccSeeAlso
\ccc{Sparse_linear_program_from_mps<NT>}\\
\ccc{Linear_program<NT>}\\
\ccc{Linear_program_from_iterators<A_it, B_it, R_it, FL_it, L_it, FU_it, U_it, D_it, C_it>}\\
\ccc{Linear_program_from_pointers<NT>}


\end{ccRefClass}
