
\begin{ccRefClass}{Combination_enumerator<CombinationElement>}

\ccDefinition
The class \ccRefName\ is used to enumerate all fixed-size combinations
(subsets) of a \emph{source range} of elements. For example, it can
enumerate all the combinations of 2 elements from the source range $[3,7)$
($7$ excluded) which gives the enumeration \{3,4\}, \{3,5\}, \{3,6\}, \{4,5\},
\{4,6\}, \{5,6\}. The source range consists of elements of type
\ccc{CombinationElement} and is specified by its first element and the element
just beyond its last one.

The template parameter should be a model of the concept \ccc{CombinationElement}.

Each combination is uniquely represented as an increasing sequence of elements.
Thus, the combinations can be lexicographically ordered. They are enumerated in
that order, so that we can talk about the first or last combination.

%\ccIsModel
%\ccRefConceptPage{ConceptName}

\ccInclude{CGAL/Combination_enumerator.h}

\ccCreation

\ccCreationVariable{c}
\ccConstructor{Combination_enumerator(int k, const CombinationElement & first,
const CombinationElement & beyond);}{This constructor initializes \ccVar\ to
enumerate the combinations of \ccc{k} elements from the source range
\ccc{[first, beyond)}. The current combination is set to the first combination
of the enumeration. \ccPrecond{\ccc{1 <= k <= beyond - first}}}

\ccConstructor{Combination_enumerator(const Combination_enumerator & combi);}%
{The copy constructor.}

\ccHeading{Access to the current combination}

\ccMethod{const CombinationElement & operator[](int i);}{Returns the \ccc{i}-th
element of the current combination. \ccPrecond{\ccc{0 <= i < }
\ccVar.\ccc{number_of_elements()}}}

\ccHeading{Access to the enumeration}

\ccMethod{int number_of_elements();}{Returns the size of the enumerated
combinations (the parameter \ccc{k} from the class' constructor).}

\ccMethod{const CombinationElement & min_element();}{Returns the smallest
element of the source range. (the parameter \ccc{first} of the class'
constructor).}

\ccMethod{const CombinationElement & beyond_element();}{Returns the successor
to the largest element of the source range (the parameter \ccc{beyond} of
the class' constructor).}

\ccMethod{bool finished();}{Returns \ccc{true} if and only if all combinations
have been enumerated.}

\ccOperations

\ccMethod{void reset();}{Resets the enumerator. The current combination is set
to the first one of the enumeration.}

\ccMethod{void operator++();}{Moves \ccVar\ to the next combination.}

\ccMethod{Combination_enumerator operator++(int);}{Post-incrementation. Same as
the pre-incrementation above, but returns the original value of \ccVar.}

%\ccSeeAlso

\end{ccRefClass}
