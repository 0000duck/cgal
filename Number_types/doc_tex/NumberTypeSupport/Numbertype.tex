\section{Introduction}

This chapter gives an overview of the number types supported by
\cgal. Number types must fulfill certain syntactical and semantic
requirements, such that they can be successfully used in \cgal\ code.
In general they are expected to be a model of an algebraic structure
concepts and in case they model a subring of the real numbers they are
also a model of \ccc{RealEmbeddable}. For an overview of the algebraic
structure concepts see Section~\ref{caf_ref::algebraic_structures}.

\section{Built-in Number Types}

The built-in number types \ccc{float}, \ccc{double} and \ccc{long double} have
the required arithmetic and comparison operators. They lack some required
routines though which are automatically included by \cgal .
\footnote{ The functions can be found in the header files 
        {\tt CGAL/int.h}, {\tt CGAL/float.h}, {\tt CGAL/double.h} and
        {\tt CGAL/long\_long.h}.}


All built-in number types of \CC\ can represent a discrete (bounded)
subset of the rational numbers only.  We assume that the
floating-point arithmetic of your machine follows {\sc Ieee}
floating-point standard.  Since the floating-point culture has much
more infrastructural support (hardware, language definition and
compiler) than exact computation, it is very efficient.
Like with all number types with finite precision representation
which are used as approximations to the infinite ranges of 
integers or real numbers, the built-in number types are inherently
potentially inexact.
Be aware of this if you decide to use the efficient built-in 
number types: you have to cope with numerical problems.  
For example, you can compute the intersection point of two lines and 
then check whether this point lies on the two lines. 
%With exact arithmetic, the check will always return \ccc{true}. 
With floating point arithmetic,
roundoff errors may cause the answer of the check to be \ccc{false}. 
With the built-in integer types overflow might occur.

%\newpage
\section[Number Types Provided by CGAL]{Number Types Provided by \cgal}
\label{cgal-nt}

%CGAL::Quotient
\cgal\ provides several number types that can be used for 
exact computation.  These include the \ccc{Quotient} class that can
be used to create, for example, a number type that behaves like a rational
number when parameterized with a number type which can represent integers.

%CGAL::MP_float
The number type \ccc{MP_Float} is able to represent multi-precision floating
point values, a generalization of integers scaled by a (potentially negative)
power of 2.  It allows to deal with ring operations over floating-point values
with requiring rational numbers.  By plugging it in \ccc{Quotient}, one obtains
rational numbers.  Note that \ccc{MP_Float} may not be as efficient as the
integer types provided by \gmp\  or \leda, but it has the advantage
to make more parts of \cgal\ independent on these external libraries for
handling robustness issues.

%CGAL::Lazy_exact_nt
The templated number type \ccc{Lazy_exact_nt<NT>} is able to represent any
number that \ccc{NT} is able to represent, but because it first tries to use an
approximate value to perform computations it can be faster than \ccc{NT}.

%CGAL::Interval_nt
A number type for doing interval arithmetic, \ccc{Interval_nt}, is provided.
This number type helps in doing arithmetic filtering in many places such
as \ccc{Filtered_predicate}.

%CGAL::Root_of_2
\ccc{CGAL::Root_of_2} is a number type that allows to represent algebraic
numbers of degree up to~2 over a \ccc{RingNumberType}.  A generic function
\ccc{CGAL::make_root_of_2} allows to build this type generically.

%CGAL::Number_type_checker 
A debugging helper \ccc{Number_type_checker<NT1,NT2,Comparator>} is also
provided which allows to compare the behavior of operations over two number
types.

\section[Number Types Provided by GMP]{Number Types Provided by \gmp} 
\label{gmp-nt}

\cgal\ provides wrapper classes for number types defined in the {\sc
Gnu} Multiple Precision arithmetic library~\cite{g-ggmpa-}. The file
{\tt  CGAL/Gmpz.h} provides the class \ccc{Gmpz}, a wrapper class for
the arbitrary-precision integer type \ccc{mpz_t}, which is compliant
with the \cgal\ number type requirements. The file {\tt  CGAL/Gmpq.h}
provides the class \ccc{Gmpq}, a wrapper class for the arbitrary-precision
rational type \ccc{mpq_t}, which is compliant with the \cgal\ number
type requirements.

The file {\tt CGAL/Gmpzf.h} provides the class \ccc{Gmpzf}, an exact
arbitrary-precision floating-point type. Hence, It does not support
operators like \verb=/=\ to guarantee exactness of the operations. The
arithmetic operations on this type are restricted to \verb=+=, \verb=-=,
\verb=*=\ and \ccc{CGAL::integral_division}.

The file {\tt CGAL/Gmpfr.h} provides the class \ccc{Gmpfr},
a fixed-precision floating-point number type. Since the precision
(number of bits used to represent the mantissa of the number) is fixed
for each object, the result of each operation is rounded when necessary.
Though not necessary at first, the user will take full advantage of this
number type by understanding the ideas behind floating-point arithmetic,
such as precision and rounding, and understandig the flags set by this
library after each operation.  For more details, the reader should refer
to \cite{cgal:mt-mpfr} and the \ccc{Gmpfr} reference manual.

In addition, it is possible to directly use the C++ number types provided by
\gmp~: \ccc{mpz_class}, \ccc{mpq_class} (note that support for
\ccc{mpf_class} is incomplete).  The file {\tt CGAL/gmpxx.h} provides the
necessary functions to make these classes compliant to the \cgal\ number type
requirements.

To use these classes, \gmp\ and \mpfr\ must be installed.


\section[Number Types Provided by LEDA] {Number Types Provided by \leda}
\label{leda-nt}

\leda\ provides number types that can be used for exact computation 
with both Cartesian and homogeneous representations.  If you are using
homogeneous representation with the built-in integer types
\ccc{short}, \ccc{int}, and \ccc{long} as ring type, exactness of
computations can be guaranteed only if your input data come from a
sufficiently small integral range and the depth of the computations is
sufficiently small.  \leda\ provides the number type \ccc{leda_integer} for
integers of arbitrary length. (Of course the length is
somehow bounded by the resources of your computer.)  It can be used as
ring type in homogeneous kernels and leads to exact
computation as long as all intermediate results are rational.  For the
same kind of problems, Cartesian representation with number type
\ccc{leda_rational} leads to exact computation as well.
The number type \ccc{leda_bigfloat} in \leda\ is a variable precision
floating-point type. Rounding mode and precision (i.e.\ mantissa length) of
\ccc{leda_bigfloat} can be set. 

The most sophisticated number type in \leda\ is the number type called
\ccc{leda_real}. Like in Pascal, where the name \ccc{real} is used for
floating-point numbers, the name \ccc{leda_real} does not describe the
number type precisely, but intentionally.  
\ccc{leda_real}s are a subset of real algebraic
numbers.  Any integer is \ccc{leda_real} and \ccc{leda_real}s are closed under
the operations $+,-,*,/$ and $k$-th root computation. For \leda\ version 5.0 and
or later \ccc{leda_real} is also able to represent real roots of polynomials.  
%\ccTexHtml{$\sqrt[k]{\ }$}{k-th root computation}. 
\ccc{leda_real}s guarantee that
all comparisons between expressions involving \ccc{leda_real}s produce the
exact result.

The files {\tt CGAL/leda\_integer.h}, {\tt CGAL/leda\_rational.h},
{ \tt CGAL/leda\_bigfloat.h} and {\tt CGAL/leda\_real.h} provide the
necessary functions to make these classes compliant to the \cgal\ number type
requirements.

\section[Number Types Provided by CORE]{Number Types Provided by \core}
\label{core-nt}

In principle \core~\cite{klpy-clp-99} provides the same set of number types as \leda. 
The type \ccc{CORE::BigInt} represent integers and \ccc{CORE::BigRat} 
represent rationals of arbitrary length. The number type \ccc{CORE::BigFloat} is 
a variable precision floating-point type. It is also possible to interpret it as an
interval type, since it also carries the error of a computed value.
As for \leda, the most sophisticated number type in \core\ is \ccc{CORE::Expr}, 
which is in its functionality equivalent to \ccc{leda_real}.  
%\ccc{CORE::Expr}s are a subset of real algebraic numbers.  Any integer is a
%\ccc{CORE::Expr} and \ccc{CORE::Expr}s are closed under the operations
%$+,-,\times,/$ and $\sqrt{}$.  \ccc{CORE::Expr}s guarantee that all
%comparisons between expressions involving \ccc{CORE::Expr}s produce the exact
%result.

The files {\tt CGAL/CORE\_BigInt.h}, {\tt CGAL/CORE\_BigRat.h},
{\tt CGAL/CORE\_BigFloat.h} and {\tt CGAL/CORE\_Expr.h} provide the
necessary functions to make these classes compliant to the \cgal\ number type
requirements.

\core\ version 1.7 or later is required.

\section[Interval arithmetic]
{Interval arithmetic}

Interval arithmetic is very important for geometric programming. It
is a fundamental tool for filtering predicates. For many problems,
intervals of machine double-precision numbers are sufficient, but it is
not always enough. For example, one approach for determining the sign of
an expression is to evaluate its sign using interval arithmetic and to
repeatedly increase the precision of the bounds of the intervals until
either the interval does not contain zero or its width is less than the
separation bound of the polynomial.

For intervals of machine double-precision numbers, \cgal~provides
the class \ccc{Interval_nt}. For intervals of floating-point
arbitrary-precision numbers, the class \ccc{Gmpfi} is provided in the
file {\tt CGAL/Gmpfi.h}.

Endpoints of \ccc{Gmpfi} intervals are represented as \ccc{Gmpfr}
numbers.  Each interval has an associated \emph{precision}, which is
the maximum precision (number of bits used to represent the mantissa)
of its endpoints.  The result of the operations is guaranteed to be
always contained in the returned interval. Since the interval arithmetic
is implemented on top of \ccc{Gmpfr}, the global flags and the default
precision are inherited from the \ccc{Gmpfr} interface. See
\cite{cgal:r-mpfi} and the \ccc{Gmpfi} reference manual for details.

To use the \ccc{Gmpfi} class, \mpfi\ must be installed.


\section{User-supplied Number Types}

In order to use your own number type it must be a model of the according
algebraic structure concept, in particular you must provide a 
specialization of \ccc{CGAL::Algebraic_structure_traits} and also of
\ccc{Real_embeddable_traits} in case it is a sub ring of the real numbers. 
If you even want to provide a related ensemble of number types you should also 
provide specializations for \ccc{CGAL::Coercion_traits} in order to 
reflect their interoperability. 
 

\section{Design and Implementation History}

This package was naturally one of the first packages implemented in \cgal.
It initially contained the \ccc{Quotient}, \ccc{Gmpz} and \ccc{Gmpq} classes,
together with the interfaces to the number types provided by \leda, which were
implemented by Stefan Schirra and Andreas Fabri.

Later, around 1998-2002, Sylvain Pion implemented \ccc{Interval_nt}
\ccc{MP_Float} and \ccc{Lazy_exact_nt}, together with the interfaces to
the \ccc{mpz_class} and \ccc{mpq_class} types from \gmp.

Number type concepts were then refined, notably by Lutz Kettner and
Susan Hert, who also contributed utility algorithms.

The \ccc{Root_of_2} class was contributed by Sylvain Pion and Monique Teillaud
around 2005, and was later improved with the help of Pedro Machado Manh\~aes
de Castro.  The class \ccc{Sqrt_extension} was then contributed by Michael
Hemmer and Ron Wein.

The work on concepts was further extended within the Exacus project, and was finally
contributed to \cgal\ by Michael Hemmer in 2006, as what is now the separate
Algebraic Foundations package~\ref{ChapterAlgebraicFoundations}, together with
a rewritten interface to operations on number types.

In 2008-2010, Bernd G\"artner added the \ccc{Gmpzf} class, while
Luis Pe\~naranda and Sylvain Lazard contributed the \ccc{Gmpfi} and
\ccc{Gmpfr} classes.


% This section has been removed since it is replaced by doc in
% Algebraic_foundations

%\section{Utility Routines}
%The number type concepts mentioned in the previous section list all 
%the required functionality.
%For the user of a number type it is handy to have a larger set of
%operations available. \cgal\ defines a number of such operations,
%to compute, for example, the minimum or maximum of two numbers, and the
%absolute value, square, sign or square root of a number.  These are
%available both as global functions and as functors.  See the reference
%manual for more details.
%
%Those routines are implemented using the required operations from the
%number type concepts. They are defined by means of templates, so you do not
%have to supply all those operations when you write  a new number type.
%But if you have a better implementation for any of them, you can provide a 
%corresponding overloading function with the same name for your number types,
%which will get preference over the template functions listed above.
%
%For the number types \ccc{int} and \ccc{double} there is
%also a random numbers generator \ccc{CGAL::Random}.


% \section{Todo}

\subsection{General thoughts}

\begin{itemize}

\item there might be a bug in Simple_interval_root due to its use of
  approximate_interval_width (which uses doubles and so might never
  get small enough).

\item The other trajectory modification events need to be documented
  (Set\_moving\_point). Hmmm, I don't remember what this means. 

\item There is no 2d regular triangulation. This is not too much work,
  but a bit (since I will need to refactor my 2d delaunay)

\item the FunctionKernel is not documented much (the options are never
  explained)

\item The is some confusion about where a KDS should get a
  FunctionKernel. Currently they mostly get them from the Simulator
  (which directly provides the RootStack). I am not sure that this is
  the right solution. The better solution may just be to have the
  FunctionKernel (and RootStack) be fetched from the SimulationTraits.

\item The FunctionKernel numeric solver I implemented is not as good
  as the one provided by GSL which is shipped on most linux boxes. It
  is GPL, so the user does have to make a decision about using it.
  Currently, there is a traits class that the user can select if they
  want to use it (at the cost of reassembling the SimulationTraits).

\item The 3d visualizion uses Coin. It is currently all in headers so
  there is really no problem with the user only using it if they have
  coin. It is pretty simple to port it to any other 3d viewer if CGAL
  picks one.

\item The whole SimulationTraits is a bit funny since it allows you to
  a) fetch a bunch of kernels and b) fetch pointers to the
  ActiveObjectsTable and simulator. I don't see an obvious way to keep
  its simplicity and have it make more sense.

\item the Qt stuff is broken since I am waiting for the proper person
  to update the makefile


\end{itemize}

%%% Local Variables: 
%%% mode: latex
%%% TeX-master: t
%%% End: 

