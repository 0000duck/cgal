\begin{ccRefClass} {Gmpfr}
\label{Gmpfr}

\ccDefinition

An object of the class \ccc{Gmpfr} is a fixed precision floating-point
number, based on the \mpfr\ library. This type is inexact, due to the fact
that the mantissa of each number is represented by a fixed amount of bits
(this amount is called \emph{precision}). If an operation needs more bits
than the precision of the result number, the results are rounded following
different possible criteria (called \emph{rounding modes}).

Currently, \mpfr\ supports four rounding modes: round to nearest,
round toward zero, round down (or toward \(-\infty\)) and round up
(or toward \(+\infty\)).  When not specified explicitly, the
operations use the default rounding mode, which is in practice a
variable local to each execution thread. The default rounding mode
can be set to any of the four rounding modes (initially, it is set
to nearest). To specify rounding modes for operations, the type
used is \ccc{std::float_round_style}.

This type is \ccc{ImplicitInteroperable} with \ccc{Gmpz}, \verb-long-,
\verb-unsigned long-, \verb-int-, \verb-double- and \verb-long double-.

\ccInclude{CGAL/Gmpfr.h}

\ccIsModel
\ccc{FieldWithKthRoot}\\
\ccc{RealEmbeddable}

\ccTypes

\ccNestedType{Precision_type}
                {Type representing the precision (number of bits
                used to represent the mantissa) of a number.}

\ccCreation
\ccCreationVariable{f}

\ccConstructor{Gmpfr();}
        {Creates an uninitialized \ccc{Gmpfr} \ccVar.}

\ccConstructor{Gmpfr(const Gmpfr& n);}
        {Copy constructor. The copied object inherits the precision of
        \ccc{n}, and thus it is not rounded.}

%--------------------------------------------------
% \ccConstructor{Gmpfr(mpfr_srcptr p);}
%         {Creates a \mpfr\ number, initialized with the value of \ccc{p},
%         a \mpfr\ number used in a C implementation. The data structure
%         containing \ccc{p} will not be cleared after object's
%         destruction.}
%--------------------------------------------------

\ccConstructor{Gmpfr(long si);}
        {Creates \ccc{Gmpfr}, initialized with the value of \ccc{si}.}

\ccConstructor{Gmpfr(unsigned long ui);}
        {Creates a \ccc{Gmpfr}, initialized with the value of \ccc{ui}.}

\ccConstructor{Gmpfr(int i);}
        {Creates a \ccc{Gmpfr}, initialized with the value of \ccc{i}.}

\ccConstructor{Gmpfr(double d);}
        {Creates a \ccc{Gmpfr}, initialized with the value of \ccc{d}.}

\ccConstructor{Gmpfr(long double ld);}
        {Creates a \ccc{Gmpfr}, initialized with the value of \ccc{ld}.}

\ccConstructor{Gmpfr(const Gmpz &z);}
        {Creates a \ccc{Gmpfr}, initialized with the value of \ccc{z}.}

\ccConstructor{Gmpfr(const Gmpzf &zf);}
        {Creates a \ccc{Gmpfr}, initialized with the value of \ccc{zf}.}

\ccConstructor{Gmpfr(const std::pair<Gmpz,long> &ie);}
        {Creates a \ccc{Gmpfr}, initialized with the value of
        \( ie.first \times 2^{ie.second} \) .}

Note that all constructors can be called with two optional parameters.
One can specify as second parameter the rounding mode desired for
the conversion from the source number and as a third parameter the
precision with which this \ccc{Gmpfr} will be created.  If only one
optional parameter is specified, it can be either the rounding mode
or the precision. If no optional parameters are specified, the
precision of the created object is chosen in such a way that the
conversion is exact (i.e., no rounding is performed).

These optional parameters, along with other functions which will be
explained below, allow users to control the rounding and precision.
%--------------------------------------------------
% For example, let
% \ccc{i} be an integer with value \(1001\): this number can be represented
% in floating-point with ten bits of mantissa. If we want to work with a
% smaller precision of, say, eight bits, we can not represent exactly its
% value. In this case, we are interested in telling \ccc{Gmpfr} how to handle
% this lack of precision. If we do \verb-Gmpfr(i,8,std::round_to_nearest)-,
% \verb-i- will have the value of \(1000\), because it is the closest value
% to \verb-i- one can represent with eight bits.
%--------------------------------------------------
For example, being \verb-z- \ a \ccc{Gmpz},
\verb-Gmpfr g(z,53,std::round_toward_neg_infinity)-\ will construct a
\ccc{Gmpfr} \verb-g- \ having as value the biggest 53-bit
floating-point number that is equal or smaller than to \verb-z-.


\ccOperations

\ccSetThreeColumns
        {static Rounding\_mode}
        {add( a, b, Rounding\_mode r)}
        {}

%%%%%%%%%%%%%%%%%%%%%%%%%%%%%%%%%%%%%%%%%%%%%%%%%%
% precision
%%%%%%%%%%%%%%%%%%%%%%%%%%%%%%%%%%%%%%%%%%%%%%%%%%

Each Gmpfr object has a precision associated to it. The precision is the
amount of bits needed to represent the mantissa. \mpfr\ has a default
precision value, which can be controlled by static functions of the Gmpfr
class (in practice, this default value is a variable local to each
execution thread). There are also functions to get and set the precision of
each Gmpfr object.

\ccFunction{static Precision_type get_default_precision();}
        {This returns the current precision used in \ccc{Gmpfr}
        creation by default.}

\ccFunction{static Precision_type set_default_precision(Precision_type p);}
        {This function sets the default \mpfr\ precision to p, and returns
        the old one.}

\ccMethod{Precision_type get_precision()const;}
        {Returns the precision of \ccVar.}

\ccMethod{Gmpfr round(Precision_type p, std::float_round_style r)const;}
        {Returns the value of \ccVar, rounded with precision \ccc{p}
        in the direction \ccc{r}.}

%%%%%%%%%%%%%%%%%%%%%%%%%%%%%%%%%%%%%%%%%%%%%%%%%%
% rounding modes
%%%%%%%%%%%%%%%%%%%%%%%%%%%%%%%%%%%%%%%%%%%%%%%%%%

\ccFunction{static std::float_round_style get_default_rndmode();}
        {This function returns the current rounding mode used by \mpfr.}

\ccFunction{static std::float_round_style
            set_default_rndmode(std::float_round_style r);}
        {This function sets the \mpfr\ rounding mode to \ccc{r} and returns
        the old one.}


%%%%%%%%%%%%%%%%%%%%%%%%%%%%%%%%%%%%%%%%%%%%%%%%%%
%% inexact flags handling functions
%%%%%%%%%%%%%%%%%%%%%%%%%%%%%%%%%%%%%%%%%%%%%%%%%%

\mpfr\ provides some flags to know whether performed operations were exact
or not, or they incurred in overflow or underflow, if the exponent is
out of range, or the result was \ccc{NaN} (not-a-number). One can clear the
flags before a set of operations and inspect them afterward, in order to
see if something unexpected happened during the operations. The static
functions used to handle flags are:

\ccFunction{static void clear_flags();}
        {Clears all the flags set by \mpfr (they are not cleared
        automatically).}

\ccFunction{static bool underflow_flag();}
        {Shows whether an operation incurred in underflow.}

\ccFunction{static bool overflow_flag();}
        {Shows whether an operation incurred in overflow.}

\ccFunction{static bool nan_flag();}
        {Shows whether the result of an operation was \ccc{NaN}.}

\ccFunction{static bool inex_flag();}
        {Shows whether an operation was inexact.}

\ccFunction{static bool erange_flag();}
        {Returns \ccc{true} iff a range error occurred. Such an exception
        occurs when some function which does not return a \ccc{Gmpfr}
        has an invalid result. For example, this flag will be set if
        one of the operands of a comparison is \ccc{NaN}.}


%%%%%%%%%%%%%%%%%%%%%%%%%%%%%%%%%%%%%%%%%%%%%%%%%%
% arithmetic operators
%%%%%%%%%%%%%%%%%%%%%%%%%%%%%%%%%%%%%%%%%%%%%%%%%%

Arithmetic operators \verb=+= , \verb=-= , \verb=*= \ and \verb=/= \
are overloaded, but special care must be taken when applying them.
The precision of an operation between two \ccc{Gmpfr}s is defined
as the maximum of the operands precision and the default precision.

The second operand of the former operations can be a \verb=Gmpfr=,
\verb=int=, \verb=long=, \verb=unsigned=, \verb=unsigned long=, or
\verb=Gmpz=. The precision of an operation between a \ccc{Gmpfr}
and a number of another type is defined as the maximum between the
number's precision and the default precision.

To specify the rounding mode and/or the precision to perform an operation,
this class provides the four static functions \verb=add=, \verb=sub=,
\verb=mul=~and \verb=div=. Only one of them is shown here, since their
interfaces are similar:

\ccFunction{static Gmpfr add (const Gmpfr &a,const Gmpfr &b);}{}
\ccFunction{static Gmpfr add (const Gmpfr &a,const Gmpfr &b,
                              std::float_round_style r);}{}
\ccFunction{static Gmpfr add (const Gmpfr &a,const Gmpfr &b,Precision_type p);}
        {}
\ccFunction{static Gmpfr add (const Gmpfr &a,const Gmpfr &b,Precision_type p,
                                std::float_round_style r);}{}

When the precision is not specified in this family of functions, it is
defined as in the overloaded operators. When the rounding mode is not
specified, the default is used.

Other arithmetic functions provided by the class are:

\ccMethod{Gmpfr abs(Precision_type p,
                    std::float_round_style r=get_default_rndmode())const;}
        {Returns the absolute value of \ccVar, rounded with precision
        \ccc{p} in the direction \ccc{r}. If \ccc{p} is not specified,
        the precision used is the maximum between \ccVar 's precision
        and the default.}

\ccMethod{Gmpfr sqrt(Precision_type p,
                     std::float_round_style r=get_default_rndmode())const;}
        {Returns the square root of \ccVar, rounded with precision
        \ccc{p} in the direction \ccc{r}. If \ccc{p} is not specified,
        the precision used is the maximum between \ccVar 's precision
        and the default.}

%\ccMethod{Gmpfr cbrt(Precision_type p,
%                     std::float_round_style r=get_default_rndmode())const;}
%        {Returns the cubic root of \ccVar, rounded with precision
%        \ccc{p} in the direction \ccc{r}. If \ccc{p} is not specified,
%        the precision used is the maximum between \ccVar 's precision
%        and the default.}

\ccMethod{Gmpfr kthroot(int k,
                        Precision_type p,
                        std::float_round_style r=get_default_rndmode())const;}
        {Returns the k-th root of \ccVar, rounded with precision
        \ccc{p} in the direction \ccc{r}. If \ccc{p} is not specified,
        the precision used is the maximum between \ccVar 's precision
        and the default.}

\ccMethod{Gmpfr square(Precision_type p,
                       std::float_round_style r=get_default_rndmode())const;}
        {Returns the square of \ccVar, rounded with precision \ccc{p} in
        the direction \ccc{r}. If \ccc{p} is not specified, the precision
        used is the maximum between \ccVar 's precision and the default.}


%%%%%%%%%%%%%%%%%%%%%%%%%%%%%%%%%%%%%%%%%%%%%%%%%%
% conversions
%%%%%%%%%%%%%%%%%%%%%%%%%%%%%%%%%%%%%%%%%%%%%%%%%%

\ccMethod{double to_double(std::float_round_style r=get_default_rndmode());}
        {Returns a double precision approximation of \ccVar~using the
        rounding mode \ccc{r}.}

\ccMethod{std::pair<double,double> to_interval();}
        {Returns an interval of doubles which contains \ccVar. If a
        rounded endpoint does not fit in a double, the double is set to plus
        or minus infinity and the \ccc{overflow} or \ccc{underflow} flag.}

\ccMethod{std::pair<double,long> to_double_exp
                (std::float_round_style r=get_default_rndmode());}
        {Returns the pair \( (d,e) \) such that \( 0.5 \le |d| < 1 \) and
        \( d \times 2^e \) equals \ccVar~rounded to double precision,
        using the rounding mode \ccc{r}. If \ccVar~is \ccc{NaN} or
        infinity, then the corresponding double is returned, leaving
        the exponent undefined and setting the appropriate error flag.}

\ccMethod{std::pair<std::pair<double,double>,long> to_interval_exp();}
        {Returns \( ((m,M),e) \) such that \( m \times 2^e \le \ccVar
        \le M \times 2^e \). If \ccVar~is \ccc{NaN} or infinity, then
        the corresponding doubles are returned, leaving the exponent
        undefined and setting the appropriate error flag.}

\ccMethod{std::pair<Gmpz,long> to_integer_exp();}
        {Returns a pair of integers \( (m,e) \), such that
        \( \ccVar = m \times 2^e \). Note that the returned value of \(m\)
        is not necessarily the smallest possible value of \(m\) (that is,
        it might be that \(2|m\)).}


%%%%%%%%%%%%%%%%%%%%%%%%%%%%%%%%%%%%%%%%%%%%%%%%%%
% comparisons
%%%%%%%%%%%%%%%%%%%%%%%%%%%%%%%%%%%%%%%%%%%%%%%%%%
\ccHeading{Comparisons}

Comparison operators \verb-==-, \verb-!=-, \verb->-, \verb-<-,
\verb->=- \ and \verb-<=- \ are also overloaded. A \ccc{Gmpfr} can
be compared with other \ccc{Gmpfr}, as well as with a \ccc{Gmpz},
\verb=long=, \verb=unsigned long=, \verb=int=, \verb=double= \ or
\verb=long double=. It is worth noting that the numbers are never
converted nor rounded before comparison. In the case where one of
the compared numbers is \ccc{NaN}, the \ccc{erange} flag is set.


%%%%%%%%%%%%%%%%%%%%%%%%%%%%%%%%%%%%%%%%%%%%%%%%%%
% query functions
%%%%%%%%%%%%%%%%%%%%%%%%%%%%%%%%%%%%%%%%%%%%%%%%%%
\ccQueryFunctions

\ccMethod{Sign sign();}
        {Returns the sign of \ccVar.}

\ccMethod{bool is_zero();}
        {Returns \ccc{true} iff \ccVar~is zero.}

\ccMethod{bool is_one();}
        {Returns \ccc{true} iff \ccVar~is one.}

\ccMethod{bool is_nan();}
        {Returns \ccc{true} iff \ccVar~is NaN (not-a-number).}

\ccMethod{bool is_inf();}
        {Returns \ccc{true} iff \ccVar~is plus or minus infinity.}

\ccMethod{bool is_number();}
        {Returns \ccc{true} iff \ccVar~is a valid number.}

\ccMethod{bool is_square();}
        {Returns \ccc{true} iff \ccVar~is the square of a number
        representable by an object of this type.}

\ccMethod{bool is_square(const Gmpfr &y);}
        {Returns \ccc{true} iff \ccVar~is the square of a number
        representable by an object of this type, computing and storing it
        in \ccc{y}.}


%%%%%%%%%%%%%%%%%%%%%%%%%%%%%%%%%%%%%%%%%%%%%%%%%%
% input/output
%%%%%%%%%%%%%%%%%%%%%%%%%%%%%%%%%%%%%%%%%%%%%%%%%%
\ccHeading{Input/Output}

% TODO: add precision to i/o
\ccFunction{std::istream& operator>>(std::istream& in, Gmpfr& f);}
        {Reads a floating-point number from \ccc{in}. The number
        \(M \times 2^E\) must be in the form \(MeE\), where the mantissa
        \(M\) and the exponent \(E\) are integers in base 10.}

\ccFunction{std::ostream& operator<<(std::ostream& out, const Gmpfr& f);}
        {If the ostream \ccc{out} is in pretty-print mode, writes a decimal
        approximation of \ccc{f} to \ccc{out}. Otherwise, writes \ccc{f} to
        \ccc{out} in the form \(MeE\), where \(M\) is its mantissa and
        \(E\) is its exponent, both in base 10.}


\ccImplementation

Since the \mpfr\ library can be compiled to be thread-safe, this interface
is designed to keep the thread-safety.

\ccc{Gmpfr}s are reference counted. This behavior may be changed, by
setting the flag \verb+CGAL_GMPFR_NO_REFCOUNT+. A non-reference-counted
class is slightly more efficient in case the implementation does not need
to copy numbers (this is not usually the case). Nevertheless, setting this
flag may be useful for debugging purposes.

\ccSeeAlso
\ccRefIdfierPage{RealEmbeddable}\\
\ccRefIdfierPage{FieldWithKthRoot}

\end{ccRefClass}
\ccDefGlobalScope{CGAL::}
