\begin{ccRefConcept}{FieldNumberType}

The concept \ccRefName\ defines the syntactic requirements of a number type
to be used as a template parameter for the Cartesian kernels.  This number
type supports the operations $+$, $-$, $*$ and $/$.  This implies that
\ccc{CGAL::Number_type_traits<FieldNumberType>::Has_division} is
\ccc{CGAL::Tag_true}.

\ccRefines
RingNumberType

\ccCreationVariable{n1}

\ccOperations
\ccFunction{FieldNumberType operator/(const FieldNumberType &n1, 
                                      const FieldNumberType &n2);} {}
\ccGlue
\ccFunction{FieldNumberType operator/(int n1, 
                                      const FieldNumberType &n2);} {}
\ccGlue
\ccFunction{FieldNumberType operator/(const FieldNumberType &n1, 
                                      int n2);} {}
\ccGlue
\ccMethod{FieldNumberType operator/=(const FieldNumberType &n2) const;}{}
\ccGlue
\ccMethod{FieldNumberType operator/=(int n2) const;}{}


\ccHasModels
\ccc{float} \\
\ccc{double} \\
\ccc{CGAL::Filtered_exact<FieldNumberType, ET>} \\
\ccc{CGAL::Gmpq} \\
\ccc{CGAL::Interval_nt} \\
\ccc{CGAL::Interval_nt_advanced} \\
\ccc{CGAL::Lazy_exact_nt<FieldNumberType>} \\
\ccc{CGAL::MP_Float} \\
\ccc{CGAL::Quotient<RingNumberType>} \\
\ccc{leda_rational} \\
\ccc{leda_bigfloat} \\
\ccc{leda_real} \\

\ccSeeAlso
\ccRefConceptPage{EuclideanRingNumberType} \\
\ccRefConceptPage{Kernel} \\
\ccRefIdfierPage{CGAL::Field_tag} \\

\end{ccRefConcept}
