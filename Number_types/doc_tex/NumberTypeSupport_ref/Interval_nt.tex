% Interval Arithmetic documentation.
%   $Id$
%   $Date$

\begin{ccRefClass}{Interval_nt<Protected>}
\label{interval}

% \ccChapterAuthor{Sylvain Pion}

\ccDefinition

This section describes briefly what interval arithmetic is, its implementation
in {\cgal}, and its possible use by geometric programs.
The main reason for having interval arithmetic in {\cgal} is its integration
into the filtered robust and fast predicates scheme, but we also provide a
number type so that you can use it separately if you find any use for it,
such as interval analysis, or to represent data with tolerance...

The purpose of interval arithmetic is to provide an efficient way to bound
the roundoff errors made by floating point computations.
You can choose the behavior of your program depending on these errors. 
%This done for the filtered robust predicates (see Section~\ref{filter}).
You can find more theoretical information on this topic in
\cite{cgal:bbp-iayed-01}.

Interval arithmetic is a large concept and we will only consider here a 
simple arithmetic based on intervals whose bounds are {\it double}s.
So each variable is an interval representing any value inside the interval.
All arithmetic operations (+, -, $*$, $/$, $\sqrt{}$, \ccc{square()},
\ccc{min()}, \ccc{max()} and \ccc{abs()}) on intervals preserve the inclusion.
This property can be expressed by the following formula ($x$ and $y$ are
real, $X$ and $Y$ are intervals, $\mathcal{OP}$ is an arithmetic operation):

$$
\forall\ x \in X, \forall\ y \in Y, (x\ \mathcal{OP}\ y)
\in (X\ \mathcal{OP}\ Y)
$$

For example, if the final result of a sequence of arithmetic operations is
an interval that does not contain zero, then you can safely determine its sign.

\ccInclude{CGAL/Interval_nt.h}

\ccParameters

The template parameter \ccc{Protected} is a Boolean parameter, which defaults
to \ccc{true}.  It provides a way to select faster computations by avoiding
rounding mode switches, at the expense of more care to be taken by the user
(see below).  The default value, \ccc{true}, is the safe way, and takes care of
proper rounding mode changes.  When specifying \ccc{false}, the user has to
take care about setting the rounding mode towards plus infinity before
doing any computations with the interval class.  He can do so using the
\ccc{Protect_FPU_rounding} class for example.

\ccIsModel
\ccc{FieldWithSqrt}\\
\\
\ccc{RealEmbeddable}

\ccTypes
The class \ccc{Interval_nt} defines the following types:
\ccTypedef{typedef double value_type;}{The type of the bounds of the interval.}
\ccGlue
\ccTypedef{typedef Uncertain_conversion_exception unsafe_comparison;}{The type of the
exceptions raised when uncertain comparisons are performed.}
\ccGlue
\ccNestedType{Protector}{A type whose default constructor and destructor allow
to protect a block of code from FPU rounding modes necessary for the
computations with \ccc{Interval_nt<false>}.  It does nothing for
\ccc{Interval_nt<true>}.  It is implemented as \ccc{Protect_FPU_rounding<!Protected>}.}


\ccCreationVariable{I}
\ccCreation

\ccConstructor{Interval_nt(long long i);}
{introduces a small interval containing \ccc{i} (possibly a point).}
\ccGlue
\ccConstructor{Interval_nt(double d);}
{introduces the interval [\ccc{d};\ccc{d}].}
\ccGlue
\ccConstructor{Interval_nt(double i, double s);}
{introduces the interval [\ccc{i};\ccc{s}].}
\ccGlue
\ccConstructor{Interval_nt(std::pair<double, double> p);}
{introduces the interval [\ccc{p.first};\ccc{p.second}].}


\ccOperations
%SetThreeColumns{Interval_nt}{}{\hspace*{8cm}}

All functions required by a class to be considered as a {\cgal} number type
(see~\ref{Numbertype}) are present, as well as the utility functions,
sometimes with a particular semantic which is described below.  There are also
a few additional functions.


\ccMethod{Interval_nt operator/(Interval_nt J);} {returns
[$-\infty$;$+\infty$] when the denominator contains 0.}

\ccFunction{Interval_nt sqrt(Interval_nt I);} {returns
[0;$\sqrt{upper\_bound(I)}$] when only the lower bound is negative (expectable
case with roundoff errors), and is unspecified when the upper bound also is
negative (unexpected case).}

\ccFunction{double to_double(Interval_nt I);} {returns the
middle of the interval, as a double approximation of the interval.}

\ccMethod{double inf();} {returns the lower bound of the interval.}
\ccMethod{double sup();} {returns the upper bound of the interval.}
\ccMethod{bool is_point();} {returns whether both bounds are equal.}
\ccMethod{bool is_same(Interval_nt J);} {returns whether both intervals have
the same bounds.}
\ccMethod{bool do_overlap(Interval_nt J);} {returns whether both intervals
have a non empty intersection.}

The comparison operators ($<$, $>$, $<=$, $>=$, $==$, $!=$, \ccc{sign()}
and \ccc{compare()}) have the following semantic: it is the intuitive
one when for all couples of values in both intervals, the comparison
is identical (case of non-overlapping intervals).  This can be expressed
by the following formula ($x$ and $y$ are real, $X$ and $Y$ are
intervals, $\mathcal{OP}$ is a comparison operator):

$$
\left(\forall x \in X, \forall y \in Y, (x\ \mathcal{OP}\ y) = true\right)
\Rightarrow (X\ \mathcal{OP}\ Y) = true
$$
and
$$
\left(\forall x \in X, \forall y \in Y, (x\ \mathcal{OP}\ y) = false\right)
\Rightarrow (X\ \mathcal{OP}\ Y) =false
$$

Otherwise, the comparison is not safe, and we specify this by returning
a type encoding this uncertainty, namely using \ccc{Uncertain<bool>}
or \ccc{Uncertain<Sign>}, which
can be probed for uncertainty explicitly, and which has a conversion to
the normal type (e.g. \ccc{bool}) which throws an exception when the
conversion is not certain.  Note that each failed conversion increments
a profiling counter (see \ccc{CGAL_PROFILE}), and then throws the exception of
type \ccc{unsafe_comparison}.

%SetThreeColumns{xxxxxxxxxxxxxxxxxxxxxxxxxxxxxxxxxxx}{}{\hspace*{8cm}}
\ccFunction{Uncertain<bool> operator<(Interval_nt i, Interval_nt j);}{}
\ccGlue
\ccFunction{Uncertain<bool> operator>(Interval_nt i, Interval_nt j);}{}
\ccGlue
\ccFunction{Uncertain<bool> operator<=(Interval_nt i, Interval_nt j);}{}
\ccGlue
\ccFunction{Uncertain<bool> operator>=(Interval_nt i, Interval_nt j);}{}
\ccGlue
\ccFunction{Uncertain<bool> operator==(Interval_nt i, Interval_nt j);}{}
\ccGlue
\ccFunction{Uncertain<bool> operator!=(Interval_nt i, Interval_nt j);}{}
\ccGlue
\ccFunction{Uncertain<Comparison_result>
            compare(Interval_nt i, Interval_nt j);}{}
\ccGlue
\ccFunction{Uncertain<Sign> sign(Interval_nt i);}{}

\ccTypedef{typedef Interval_nt<false>  Interval_nt_advanced;}
{This typedef (at namespace \cgal\ scope) exists for backward compatibility,
 as well as removing the need to remember the Boolean value for the template
 parameter.}

\begin{ccAdvanced}

\ccImplementation

The operations on \ccc{Interval_nt} with the default parameter \ccc{true},
are automatically protected against rounding modes, and are thus slower than
those on \ccc{Interval_nt_advanced}, but easier to use.  Users that need
performance are encouraged to use \ccc{Interval_nt_advanced} instead.

Changing the rounding mode affects all floating point computations, and might
cause problems with parts of your code, or external libraries (even \cgal),
that expect the rounding mode to be the default (round to the nearest).

We provide two interfaces to change the rounding mode.  The first one is to
use a protector object whose default constructor and destructor will take care
of changing the rounding mode.  The protector is implemented using
\ccc{Protect_FPU_rounding}.

The second one is the following detailed set of functions :

\ccTypedef{typedef int FPU_CW_t;}
       {The type used by the following functions to deal with rounding modes.
       This is usually an \ccc{int}.}
\ccFunction{void FPU_set_cw (FPU_CW_t R);}
       {sets the rounding mode to \ccc{R}.}
\ccFunction{FPU_CW_t FPU_get_cw (void);}
       {returns the current rounding mode.}
\ccFunction{FPU_CW_t FPU_get_and_set_cw (FPU_CW_t R);}
       {sets the rounding mode to \ccc{R} and returns the old one. }

The macros \ccc{CGAL_FE_TONEAREST}, \ccc{CGAL_FE_TOWARDZERO},
\ccc{CGAL_FE_UPWARD} and \ccc{CGAL_FE_DOWNWARD} are the values corresponding
to the rounding modes.

\ccExample

Protecting an area of code that uses operations on the class
\ccc{Interval_nt_advanced} can be done in the following way:

\begin{verbatim}
{
  Interval_nt_advanced::Protector P;
  ... // The code to be protected.
}
\end{verbatim}

The basic idea is to use the directed rounding modes specified by the 
{\it IEEE 754} standard, which are implemented by almost all processors 
nowadays.
It states that you have the possibility, concerning the basic floating point
operations ($+,-,*,/,\sqrt{}$) to specify the rounding mode of each operation
instead of using the default, which is set to 'round to the nearest'.
This feature allows us to compute easily on intervals.  For example, to
add the two intervals [a.i;a.s] and [b.i;b.s], compute $c.i=a.i+b.i$ rounded
towards minus infinity, and $c.s=a.s+b.s$ rounded towards plus infinity, and
the result is the interval [c.i;c.s].  This method can be extended easily to
the other operations.

The problem is that we have to change the rounding mode very often, and the
functions of the C library doing this operation are slow and not portable.
That's why assembly versions are used as often as possible.
Another trick is to store the opposite of the lower bound, instead of the
lower bound itself, which allows us to never change the rounding mode inside
simple operations.  Therefore, all basic operations, which are in the class 
\ccc{Interval_nt_advanced} assume that the rounding mode is set to 
'round to infinity', and everything works with this correctly set.  

So, if the user needs the speed of \ccc{Interval_nt_advanced}, he must
take care of setting the rounding mode to 'round to infinity' before each
block of operations on this number type.  And if other operations might be
affected by this, he must take care to reset it to 'round to the nearest'
before they are executed.

% Note also that NaNs are not handled, so be careful with that
% (especially if you `divide by zero').

Notes:\\
\begin{itemize}
\item On Intel platforms (with any operating system and compiler), due to a
      misfeature of the floating point unit, which does not handle exactly
      IEEE compliant operations on doubles, we are forced to use a workaround
      which slows down the code, but is only useful when the intervals can
      overflow or underflow.  If you know that the intervals will never
      overflow nor underflow for your code, then you can disable this
      workaround with the flag \ccc{CGAL_IA_NO_X86_OVER_UNDER_FLOW_PROTECT}.
      Other platforms are not affected by this flag.
\item When optimizing, compilers usually propagate the value of variables when
      they know it's a constant.  This can break the interval routines because
      the compiler then does some floating point operations on these constants
      with the default rounding mode, which is wrong.  This kind of problem
      is avoided by stopping constant propagation in the interval routines.
      However, this solution slows down the code and is rarely useful, so you
      can disable it by setting the flag
      \ccc{CGAL_IA_DONT_STOP_CONSTANT_PROPAGATION}.
\end{itemize}

\end{ccAdvanced}

\end{ccRefClass}
