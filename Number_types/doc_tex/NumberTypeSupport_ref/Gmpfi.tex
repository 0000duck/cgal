\begin{ccRefClass} {Gmpfi}
\label{Gmpfi}

\ccDefinition

An object of the class \ccc{Gmpfi} is a closed interval, with endpoints
represented as \ccc{Gmpfr} floating-point numbers. An interval can have
finite or infinite endpoints and its meaning is straightforward. It can
also have one (or both) \ccc{NaN} endpoint(s): this indicates that an
invalid operation has been performed and that the resulting interval has no
mathematical meaning.

All the operations of \ccc{Gmpfi} were designed in such a way that the
mathematical correct result is always contained in the resulting interval.

This type is \ccc{ImplicitInteroperable} with \ccc{Gmpfr}, \ccc{Gmpz},
\ccc{Gmpq}, \verb-long-, \verb-unsigned long-, \verb-int-, \verb-double-
and \verb-long double-.

\ccInclude{CGAL/Gmpfi.h}

\ccIsModel
\ccc{FieldWithKthRoot}\\
\ccc{RealEmbeddable}

\ccTypes

\ccNestedType{Precision_type}
                {Type representing the precision (number of bits
                used to represent the mantissa) of a number.}

\ccCreation

All the constructors accept an optional last argument: a precision (a
\ccc{Precision_type}, which can be used to specify the precision of the
\ccc{Gmpfr} endpoints. If none is specified, the default precision will be
used. As the endpoints are represented with a fixed number of bits, they
may need to be rounded. In this case, the number from which the \ccc{Gmpfi}
was constructed is guaranteed to be included in the constructed interval.

\ccCreationVariable{i}

\ccConstructor{Gmpfi(Precision_type p=get_default_precision());}
        {creates an uninitialized \ccc{Gmpfi} interval \ccVar.}

\ccHidden \ccConstructor{Gmpfi(const Gmpfi& j);}
        {copy constructor.}

%--------------------------------------------------
% \ccConstructor{Gmpfi(mpfi_srcptr p);}
%         {creates a \ccc{Gmpfi} interval, initialized with the value of
%         \ccstyle{p}, a \mpfi\ number used in a C implementation. The data
%         structure containing \ccc{p} will not be cleared after object's
%         destruction.}
%--------------------------------------------------

\ccConstructor{template <class T>
                Gmpfi(const T& t,Precision_type p=get_default_precision());}
        {creates a \ccc{Gmpfi} initialized with the value of \ccc{t}.
        \ccc{T} is \ccc{Gmpfr}, \ccc{Gmpq}, or any type from which
        \ccc{Gmpfr} can be constructed from. The rounding of the
        endpoints will guarantee that \ccc{t} is included in \ccVar.}

\ccConstructor{Gmpfi(const Gmpfr &left,
                     const Gmpfr &right,
                     Precision_type p=get_default_precision());}
        {creates a \ccc{Gmpfi} initialized with endpoints \ccc{left}
        and \ccc{right}. The rounding of the endpoints will guarantee
        that \([\ccc{left},\ccc{right}]\) is included in \ccVar .}

\ccConstructor{template<class L, class R>
                Gmpfi(std::pair<const L&,const R&> endpoints,
                      Precision_type p=get_default_precision());}
        {creates a \ccc{Gmpfi} initialized with endpoints
        \ccc{endpoints.first} and \ccc{endpoints.second}. \ccc{L} and
        \ccc{R} are types from which \ccc{Gmpfr} can be constructed
        from. The rounding of the endpoints will guarantee that
        \([\ccc{endpoints.first},\ccc{endpoints.second}]\) is included in
        \ccVar .}

\ccOperations

\ccMethod{Gmpfr inf() const;}
        {Returns the smallest (or \emph{inferior}) \ccc{Gmpfr} endpoint of
        \ccc{i}.}

\ccMethod{Gmpfr sup()const;}
        {Returns the largest (or \emph{superior}) \ccc{Gmpfr} endpoint of
        \ccc{i}.}

%%%%%%%%%%%%%%%%%%%%%%%%%%%%%%%%%%%%%%%%%%%%%%%%%%
% precision handling
%%%%%%%%%%%%%%%%%%%%%%%%%%%%%%%%%%%%%%%%%%%%%%%%%%

\ccMethod{static Precision_type get_default_precision();}
        {Returns the default precision.}

\ccMethod{static Precision_type set_default_precision(Precision_type prec);}
        {Sets the default precision to \ccc{prec} and returns the
        old value.}

\ccMethod{Precision_type get_precision()const;}
        {Returns the precision of \ccVar.}

\ccMethod{Gmpfi round(Precision_type p)const;}
        {Returns the value of the number, rounded with precision \ccc{p}.}


%%%%%%%%%%%%%%%%%%%%%%%%%%%%%%%%%%%%%%%%%%%%%%%%%%
% arithmetic operators
%%%%%%%%%%%%%%%%%%%%%%%%%%%%%%%%%%%%%%%%%%%%%%%%%%

Arithmetic operators \verb=+=, \verb=-=, \verb=*=\ and \verb=/=\ are
overloaded, but special care must be taken when applying them. The
precision of an operation between two \ccc{Gmpfi}s is defined as the
maximum of the operands precision and the default precision.

The second operand of the former operations can be a \verb=Gmpfi=,
\verb=Gmpfr=, \verb=int=, \verb=long=, \verb=unsigned=,
\verb=unsigned long=, \verb=Gmpz= or \verb=Gmpq=. The precision of an
operation between a \ccc{Gmpfi} and a number of another type is defined
as the \ccc{Gmpfi}'s precision (even when operating with a \verb=Gmpfr=).

To specify the rounding mode and/or the precision to perform an operation,
this class provides the four static functions \verb=add=, \verb=sub=,
\verb=mul=~and \verb=div=. Only one of them is shown here, since their
interfaces are similar:

\ccFunction
        {static Gmpfi add (const Gmpfi &a,const Gmpfi &b,Precision_type p=0);}
        {}

When the precision is not specified in this family of functions, it is
defined as in the overloaded operators.


%%%%%%%%%%%%%%%%%%%%%%%%%%%%%%%%%%%%%%%%%%%%%%%%%%
% the rest
%%%%%%%%%%%%%%%%%%%%%%%%%%%%%%%%%%%%%%%%%%%%%%%%%%

Other arithmetic functions provided by the class are:

\ccMethod{Gmpfi abs(Precision_type p)const;}
        {Returns the absolute value of \ccVar, with precision \ccc{p}.
        If \ccc{p} is not specified, the precision used is the maximum
        between \ccVar 's precision and the default.}

\ccMethod{Gmpfi sqrt(Precision_type p)const;}
        {Returns the square root of \ccVar, with precision \ccc{p}.
        If \ccc{p} is not specified, the precision used is the maximum
        between \ccVar 's precision and the default.}

\ccMethod{Gmpfi kthroot(int k,Precision_type p)const;}
        {Returns the k-th root of \ccVar, with precision \ccc{p}.
        If \ccc{p} is not specified, the precision used is the maximum
        between \ccVar 's precision and the default.}

\ccMethod{Gmpfi square(Precision_type p)const;}
        {Returns the square of \ccVar, with precision \ccc{p}. If
        \ccc{p} is not specified, the precision used is the maximum
        between \ccVar 's precision and the default.}


%%%%%%%%%%%%%%%%%%%%%%%%%%%%%%%%%%%%%%%%%%%%%%%%%%
% conversion
%%%%%%%%%%%%%%%%%%%%%%%%%%%%%%%%%%%%%%%%%%%%%%%%%%

\ccMethod{std::pair<double,double> to_interval()const;}
        {Returns an interval of doubles which contains \ccVar. If a
        rounded endpoint does not fit in a double, sets its value to plus
        or minus infinity and the \ccc{overflow} or \ccc{underflow} flag.}

\ccMethod{std::pair<double,long> to_double_exp()const;}
        {Returns \((m,e)\) such that \(m \times 2^e\) is the center of
        \ccVar, rounded to nearest. If one of the endpoints of \ccVar~is
        \ccc{NaN} or infinity, then the corresponding double is returned,
        leaving the exponent undefined and setting the appropriate
        error flag.}

\ccMethod{std::pair<std::pair<double,double>,long> to_interval_exp()const;}
        {Returns \(((m_1,m_2),e)\), such that \([m_1 \times 2^e,m_2
        \times 2^e]\) contains \ccVar . If one of the endpoints of
        \ccVar~is \ccc{NaN} or infinity, then the corresponding doubles
        are returned, leaving the exponent undefined and setting the
        appropriate error flag.}


%%%%%%%%%%%%%%%%%%%%%%%%%%%%%%%%%%%%%%%%%%%%%%%%%%
% comparisons
%%%%%%%%%%%%%%%%%%%%%%%%%%%%%%%%%%%%%%%%%%%%%%%%%%
\ccHeading{Comparisons}

The semantics of the comparison operators is the same than on
\ccc{Interval_nt<Protected>}. The result of the comparison is always an
\ccc{Uncertain<bool>} (this type is convertible to \ccc{bool}, but may
throw an exception). If compared intervals have no common points, the
result is \ccc{true} or \ccc{false}; otherwise,
\ccc{Uncertain<bool>::indeterminate()} will be returned.

In the same way, we can explain the semantics of
\ccc{Uncertain<Comparison\_result>} and \ccc{Uncertain<Sign>}.

With the semantics described above, this class provides comparisons between
\ccc{Gmpfi} and \ccc{Gmpfi}, \ccc{Gmpfr}, \ccc{long}, \ccc{unsigned long},
\ccc{int}, \ccc{double}, \ccc{Gmpz} and \ccc{Gmpq}. Comparison operators
\verb-==-, \verb-!=-, \verb->-, \verb-<-, \verb->=- \ and \verb-<=- \ are
overloaded.

The class provides also functions to test efficiently some special kinds
of comparisons:

\ccMethod{bool is_same(const Gmpfi &j)const;}
        {Returns \ccc{true} iff left endpoints of \ccVar~and
        \ccc{j} are equal and right endpoints of them are also equal. Note
        that this does not mean equality between \ccVar~and \ccc{j}.}

\ccMethod{bool do_overlap(const Gmpfi &j)const;}
        {Returns \ccc{true} iff \ccVar~and \ccc{j} overlap,
        i.e., iff they have points in common.}

\ccMethod{Uncertain<Comparison_result> compare(const Gmpfi &j)const;}
        {If \ccVar~and \ccc{j} do not overlap, this function returns
        the result of the comparison. Otherwise, it returns
        \ccc{indeterminate}.}


%%%%%%%%%%%%%%%%%%%%%%%%%%%%%%%%%%%%%%%%%%%%%%%%%%
% query functions
%%%%%%%%%%%%%%%%%%%%%%%%%%%%%%%%%%%%%%%%%%%%%%%%%%
\ccQueryFunctions
\ccMethod{bool is_point()const;}
        {Returns \ccc{true} iff both endpoints are equal.}

\ccMethod{bool is_nan()const;}
        {Returns \ccc{true} iff at least one of the endpoints is
        \ccc{NaN}.}

\ccMethod{bool is_inf()const;}
        {Returns \ccc{true} iff at least one of the endpoints is plus or
        minus infinity.}

\ccMethod{bool is_number()const;}
        {Returns \ccc{true} iff \ccVar~is a bounded interval, i.e. its
        endpoints are neither invalid nor infinite.}

\ccMethod{Uncertain<bool> is_zero()const;}
        {Returns \ccc{true} if both endpoints are zero, \ccc{false} if
        the interval does not contain zero and \ccc{indeterminate}
        otherwise.}

\ccMethod{Uncertain<bool> is_one()const;}
        {Returns \ccc{true} if both endpoints are one, \ccc{false}
        if the interval does not contain one and \ccc{indeterminate}
        otherwise.}

\ccMethod{Uncertain<Sign> sign()const;}
        {If all numbers contained in the interval have the same sign, this
        function returns it. Otherwise it returns \ccc{indeterminate}.}

\ccMethod{Uncertain<bool> is_positive()const;}
        {Returns \ccc{true} if all numbers contained in the interval
        are positive, false if all of them are negative or zero and
        \ccc{indeterminate} otherwise.}

\ccMethod{Uncertain<bool> is_negative()const;}
        {Returns \ccc{true} if all numbers contained in the interval
        are negative, false if all of them are positive or zero and
        \ccc{indeterminate} otherwise.}

%--------------------------------------------------
% \ccMethod{Uncertain<bool> is_square(const Gmpfi &y)const;}
%         {Returns \ccc{true} if \ccVar~contains the squares of all
%         the numbers in interval \ccc{y}, \ccc{indeterminate} if
%         \ccVar~contains the squares of some numbers in \ccc{y} but not
%         of all of them, and \ccc{false} otherwise.}
%--------------------------------------------------

%--------------------------------------------------
% \ccMethod{Uncertain<bool> divides(const Gmpfi &n,Gmpfi &c,
%                 Precision_type p=Gmpfr::get_default_precision())const;}
%         {Returns \ccVar \ccc{.divides(n)}. If it is \ccc{true}, sets
%         \ccc{c} such that \( \ccc{n} = \ccVar \times \ccc{c} \), with
%         precision \ccc{p}.}
%--------------------------------------------------


%%%%%%%%%%%%%%%%%%%%%%%%%%%%%%%%%%%%%%%%%%%%%%%%%%
% input/output
%%%%%%%%%%%%%%%%%%%%%%%%%%%%%%%%%%%%%%%%%%%%%%%%%%
\ccHeading{Input/Output}

\ccFunction{std::istream& operator>>(std::istream &is,Gmpfi i);}
        {Reads \ccc{i} from \ccc{is}. \ccc{is} has the form
        \ccc{[inf,sup]}, where \ccc{inf} and \ccc{sup} have valid
        \ccc{Gmpfr} formats.}
        % TODO: add the possibility of reading a number in non-interval form

\ccFunction{std::ostream& operator<<(std::ostream &os,const Gmpfi &i);}
        {Writes \ccc{i} to \ccc{os}, in the form \ccc{[i.inf(),i.sup()]}.}


\ccImplementation
All interval operations are performed by the \mpfi~library. The class
\ccc{Gmpfi} is not reference counted, but its members are.

\ccSeeAlso
\ccRefIdfierPage{Gmpfr}\\
\ccRefIdfierPage{Interval_nt<Protected>}\\
\ccRefIdfierPage{Uncertain<T>}\\
\ccRefIdfierPage{RealEmbeddable}\\
\ccRefIdfierPage{FieldWithKthRoot}\\

\end{ccRefClass}
\ccDefGlobalScope{CGAL::}
