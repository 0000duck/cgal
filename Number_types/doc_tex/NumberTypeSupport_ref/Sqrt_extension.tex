%OPEN: new name for template argument ROOT ?
%But this will cause changes in the interface  i.e. root() should change too.

\begin{ccRefClass}{Sqrt_extension<NT,ROOT>}
\def\ccTagOperatorLayout{\ccFalse}

\ccDefinition

An instance of this class represents an extension of the type \ccc{NT} by
\textbf{one} square root of the type \ccc{ROOT}. \\ 
\ccc{NT} is required to be constructible from \ccc{ROOT}. \\
\ccc{NT} is required to be an \ccc{IntegralDomainWithoutDivision}. \\
\ccRefName\ is \ccc{RealEmbeddable} if NT is  \ccc{RealEmbeddable}.

For example: Let \ccc{Integer} be some type representing $\Z$, then 
\ccc{Sqrt_extension<Integer,Integer>} is able to represent $\Z[\sqrt{root}]$
for some arbitrary value $root$. The value of $root$ is set at 
construction time, or left open if $root$ is not specified.

Note that the given $root$ is stored as it is given to the constructor. In particular, an extension by a square root of a square is considered as an extension.  


\textbf {It is not possible to interoperate two values from two different 
extensions $\Z[\sqrt{a}]$ and $\Z[\sqrt{b}]$, since the result is in 
$\Z[\sqrt{a},\sqrt{b}]$, which is not representable by 
\ccc{Sqrt_extension<Integer,Integer>}. }

%However, both \ccc{NT} and \ccc{ROOT} can themselves be an instance of
%\ccc{Sqrt_extension}, yielding a nested extension, e.g.,
%\ccc{Sqrt_extension<Sqrt_extension<Integer,Integer>,Integer>} is able to represent 
%$\Z[\sqrt{a}][\sqrt{b}]$ for some integers $a$ and $b$. 


\begin{ccAdvanced}
 Since elements of \ccRefName\ that lie in different extensions 
are not interoperable with respect to any arithmetic operations, the full
 value range of \ccRefName\ does not represent an algebraic structure.
 However, each subset of the value range that represents the extension of
 NT by a particular square root is a valid algebraic structure, since 
this subset is closed under all provided arithmetic  operations.

From there, \ccRefName\ can be used as if it were a model of an algebraic structure concept, with the following correspondence:

\begin{tabular}{ll}
 NT & \ccRefName\  \\
\hline \\
\ccc{IntegralDomainWithoutDivision} & \ccc{IntegralDomainWithoutDivision}\\
\ccc{IntegralDomain}                & \ccc{IntegralDomain}\\
\ccc{UniqueFactorizationDomain}     & \ccc{IntegralDomain}\\
\ccc{EuclideanRing}                 & \ccc{IntegralDomain}\\
\ccc{Field}                         & \ccc{Field}\\
\hline 
\end{tabular}

The extension of an \ccc{UniqueFactorizationDomain} or 
\ccc{EuclideanRing} is just an \ccc{IntegralDomain}, since the extension in general destroys the unique factorization property, e.g., 
\[10 = \sqrt{10} \cdot \sqrt{10} = 2 \cdot 5 \mbox{ over} \Z[\sqrt{10}]\]

%For example take $\Z[\sqrt{10}]$:  \\
%as indicated by \cite{e-eacfcc-03} referring to \cite{hardy-wright,hasse}:
%For the fact that $2$, $5$, and $\sqrt{10}$ are irreducible elements 
%of $\Z[\sqrt{10}]$. It follows immediately that $\Z[\sqrt{10}]$ is not a 
%\ccc{UniqueFactorizationDomain}, 
%because \[10 = \sqrt{10} \cdot \sqrt{10} = 2 \cdot 5\]$ factors in two essentially
%different ways. \\

\textbf{ Note that the user is responsible for checking that the elements of 
\ccRefName\ that are manipulated are actually in a same extension field. 
If different extension fields are mixed together, 
the code may crash at runtime!} \\

%However, \ccRefName\ is useful for lengthy computations 
%over one quadratic extension, e.g. it can be used as coefficient type for 
%polynomials, vectors or matrices. In particular it may be preferable to use 
%\ccc{Sqrt_extension} in place of \ccc{leda_real} or \ccc{CORE::Expr}, 
%since it does not suffer from growing expression trees.

In the same way \ccc{Sqrt_extension} can be used as if it were a model of \ccc{RealEmbeddable} if \ccc{NT} is model of \ccc{RealEmbeddable}. \textbf{Comparisons of two values from different extension via operators is not allowed.} However, the 
member function compare provides this comparison. 
\end{ccAdvanced}

\ccIsModel 
\begin{tabular}{ll}
\ccc{Assignable}\\
\ccc{CopyConstructible}\\
\ccc{DefaultConstructible}\\
\ccc{EqualityComparable}\\
\ccc{ImplicitInteroperable} & with int\\
\ccc{ImplicitInteroperable} & with NT\\
\ccc{Fraction}          & if NT is a \ccc{Fraction}\\
\end{tabular}

%\begin{ccAdvanced}
%\ccc{Sqrt_extension} is \ccc{Modularizable} in case \ccc{NT} is 
%\ccc{Modularizable}.
%\end{ccAdvanced}

\ccCreation

\ccCreationVariable{ext}
\ccConstructor{Sqrt_extension ();} 
        {Introduces an variable initialized with 0.}
\ccConstructor{Sqrt_extension (const Sqrt_extension& x);}
 	{copy constructor.}
\ccConstructor{Sqrt_extension (const int &i);}
        {Constructor from int.}
\ccConstructor{Sqrt_extension (const NT &x);}
        {Constructor from type NT.}
\ccConstructor{Sqrt_extension (int a0, int a1, int r);}
 	{Constructor from int: \ccVar $= a0 +a1 \cdot sqrt(r)$.}
\ccConstructor{Sqrt_extension (NT a0, NT a1, ROOT r);}
 	{General constructor: \ccVar $= a0 + a1 \cdot sqrt(r)$.}

\ccOperations

An object of type \ccc{Sqrt_extension} represent an expression of the form: 
$a0 + a1 * sqrt(root)$. 

\ccMethod{bool is_extended () const;}
 	{Returns true in case root of \ccVar\ is not zero.  \\
         Note that $a1 == 0 $ does not imply $root == 0$. }   

\ccMethod{const NT & a0 () const ;    }{Const access operator for a0} \ccGlue
%\ccMethod{NT & 	a0 ()    ;    }{Access operator for a0.}\ccGlue
\ccMethod{const NT & a1 () const ;    }{Const access operator for a1}\ccGlue
%\ccMethod{NT & 	a1 ()    ;    }{Access operator for a1.}\ccGlue
\ccMethod{const ROOT & 	root () const;}{Const access operator for root}\ccGlue
%\ccMethod{ROOT & 	root ()  ;    }{Access operator for root.}

%Member functions:\\
\ccMethod{void 	simplify ();}
        { Propagates the simplify command to members of \ccVar. \\
          see also: \ccc{AlgebraicStructureTraits::Simplify}.}
\ccMethod{bool 	is_zero () const;}
        {returns true if \ccVar\ represents the value zero.}

\ccMethod{CGAL::Sign sign () const;}
        {Determines sign of \ccVar\ by (repeated) squaring.
        \ccPrecond \ccc{Sqrt_extension} is \ccc{RealEmbeddable}.}
\ccMethod{Sqrt_extension abs () const;}
 	{returns the absolute value of \ccVar. 
        \ccPrecond \ccc{Sqrt_extension} is \ccc{RealEmbeddable}.}

\ccMethod{
Sqrt_extension compare 
(const Sqrt_extension& y, bool in_same_extension = false) const;}{
  Compares \ccVar\ with y. \\
  The optional bool \ccc{in_same_extension} indicates whether \ccVar\ 
  and $y$ are in the same extension of NT.
}


\ccFunction{Sqrt_extension operator+(const Sqrt_extension&a,const Sqrt_extension& b);}
{\ccPrecond \ccc{(a.root()==0 || b.root()==0 || a.root() == b.root())} }\ccGlue
\ccFunction{Sqrt_extension operator-(const Sqrt_extension&a,const Sqrt_extension& b);}
{\ccPrecond \ccc{(a.root()==0 || b.root()==0 || a.root() == b.root())} }\ccGlue
\ccFunction{Sqrt_extension operator*(const Sqrt_extension&a,const Sqrt_extension& b);}
{\ccPrecond \ccc{(a.root()==0 || b.root()==0 || a.root() == b.root())} }\ccGlue


\ccMethod{Sqrt_extension & operator+=(const Sqrt_extension& a);}
{\ccPrecond \ccc{(this->root()==0 || a.root()==0 || this->root() == a.root())} }\ccGlue

\ccMethod{Sqrt_extension & operator-=(const Sqrt_extension& a);}
{\ccPrecond \ccc{(this->root()==0 || a.root()==0 || this->root() == a.root())} }\ccGlue

\ccMethod{Sqrt_extension & operator*=(const Sqrt_extension& a);}
{\ccPrecond \ccc{(this->root()==0 || a.root()==0 || this->root() == a.root())} }\ccGlue

In case \ccc{NT} is an \ccc{IntegralDomain} operator/ implements integral division. 
%In case \ccc{NT} is a \ccc{Field} operator/ implements the field division.    

\ccFunction{Sqrt_extension operator/(const Sqrt_extension&a,const Sqrt_extension& b);}
{\ccPrecond \ccc{(a.root()==0 || b.root()==0 || a.root() == b.root())} }\ccGlue
\ccMethod{Sqrt_extension & operator/=(const Sqrt_extension& a);}
{\ccPrecond \ccc{(this->root()==0 || a.root()==0 || this->root() == a.root())} }\ccGlue
\ccFunction{bool operator==(const Sqrt_extension&a,const Sqrt_extension& b);}
{\ccPrecond \ccc{(a.root()==0 || b.root()==0 || a.root() == b.root())} }\ccGlue
\ccFunction{bool operator!=(const Sqrt_extension&a,const Sqrt_extension& b);}
{\ccPrecond \ccc{(a.root()==0 || b.root()==0 || a.root() == b.root())} }\ccGlue

In case  \ccc{Sqrt_extension} is \ccc{RealEmbeddable}:

\ccFunction{bool operator< (const Sqrt_extension&a,const Sqrt_extension& b);}
{\ccPrecond \ccc{(a.root()==0 || b.root()==0 || a.root() == b.root())} }\ccGlue
\ccFunction{bool operator<=(const Sqrt_extension&a,const Sqrt_extension& b);}
{\ccPrecond \ccc{(a.root()==0 || b.root()==0 || a.root() == b.root())} }\ccGlue
\ccFunction{bool operator> (const Sqrt_extension&a,const Sqrt_extension& b);}
{\ccPrecond \ccc{(a.root()==0 || b.root()==0 || a.root() == b.root())} }\ccGlue
\ccFunction{bool operator>=(const Sqrt_extension&a,const Sqrt_extension& b);}
{\ccPrecond \ccc{(a.root()==0 || b.root()==0 || a.root() == b.root())} }\ccGlue



\ccIgnore{ 
% Mixed operations with NT (don't forget int in case you want to re add them)
\ccFunction{Sqrt_extension operator+(const Sqrt_extension&a, const NT &b);}{}\ccGlue
\ccFunction{Sqrt_extension operator-(const Sqrt_extension&a, const NT &b);}{}\ccGlue
\ccFunction{Sqrt_extension operator*(const Sqrt_extension&a, const NT &b);}{}\ccGlue
\ccFunction{Sqrt_extension operator+(const NT&a, const Sqrt_extension &b);}{}\ccGlue
\ccFunction{Sqrt_extension operator-(const NT&a, const Sqrt_extension &b);}{}\ccGlue
\ccFunction{Sqrt_extension operator*(const NT&a, const Sqrt_extension &b);}{}\ccGlue
\ccMethod{Sqrt_extension & 	operator += (const NT &a);}{}\ccGlue
\ccMethod{Sqrt_extension & 	operator -= (const NT &a);}{}\ccGlue
\ccMethod{Sqrt_extension & 	operator *= (const NT &a);}{}\ccGlue
\ccFunction{Sqrt_extension operator/(const Sqrt_extension&a, const NT &b);}{}\ccGlue
\ccFunction{Sqrt_extension operator/(const NT&a, const Sqrt_extension &b);}{}\ccGlue
\ccMethod{Sqrt_extension & 	operator /= (const NT &a);}{}
\ccFunction{bool operator==(const NT&a, const Sqrt_extension &b);}{}\ccGlue
\ccFunction{bool operator!=(const NT&a, const Sqrt_extension &b);}{}\ccGlue
\ccFunction{bool operator==(const Sqrt_extension&a, const NT &b);}{}\ccGlue
\ccFunction{bool operator!=(const Sqrt_extension&a, const NT &b);}{}
\ccFunction{bool operator< (const NT&a, const Sqrt_extension &b);}{}\ccGlue
\ccFunction{bool operator<=(const NT&a, const Sqrt_extension &b);}{}\ccGlue
\ccFunction{bool operator> (const NT&a, const Sqrt_extension &b);}{}\ccGlue
\ccFunction{bool operator>=(const NT&a, const Sqrt_extension &b);}{}
\ccFunction{bool operator< (const Sqrt_extension&a, const NT &b);}{}\ccGlue
\ccFunction{bool operator<=(const Sqrt_extension&a, const NT &b);}{}\ccGlue
\ccFunction{bool operator> (const Sqrt_extension&a, const NT &b);}{}\ccGlue
\ccFunction{bool operator>=(const Sqrt_extension&a, const NT &b);}{}
}

The stream operations are available as well. 
They assume that corresponding stream operators for type \ccc{NT} and \ccc{ROOT} exist.

\ccFunction{std::ostream& operator<<(std::ostream& os, const Sqrt_extension<NT,ROOT> &ext);}
{writes \ccc{ext} to ostream \ccc{os}. The format depends on the \ccc{CGAL::IO::MODE} of \ccc{os}.\\
        In case the mode is \ccc{CGAL::IO::ASCII} the format is {\tt EXT[a0,a1,root]}. \\
        In case the mode is \ccc{CGAL::IO::PRETTY} the format is human readable. \\
        }

\ccFunction{std::istream& operator>>(std::istream& is, const Sqrt_extension<NT,ROOT> &ext);}
{reads \ccc{ext} from istream \ccc{is} in format {\tt EXT[a0,a1,root]}, the output format in mode \ccc{CGAL::IO::ASCII} }



%void 	output_maple (std::ostream &os) const
% 	write Sqrt_extension to os in LiS::IO::PRETTY format
%void 	output_ascii (std::ostream &os) const
% 	write Sqrt_extension to os in a format readable by input_ascii()
\end{ccRefClass}
