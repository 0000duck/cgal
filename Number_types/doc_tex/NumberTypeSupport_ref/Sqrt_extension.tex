
\begin{ccRefClass}{Sqrt_extension<NT,ROOT>}

\ccDefinition

An instance of this class represents an extension of the type \ccc{NT} by a 
square root of the type \ccc{ROOT}. \\
In case \ccc{NT} and \ccc{ROOT} do not coincide, 
\ccc{NT} must be constructible from \ccc{ROOT}. \\
The type \ccc{NT} must be at least a model of the concept 
\ccc{IntegralDomainWithoutDivision}.\\

The table below shows the resulting algebraic type 
with respect to the algebraic type of \ccc{NT}.
Both \ccc{NT} and \ccc{ROOT} can themselves be an instance of
\ccc{Sqrt_extension}, yielding a nested extension.
        
  \label{sqrt_extension:tabular_1}       
  \begin{tabular}{|l|l|}
        \hline
        \ccc{NT} is model of:  &  \ccc{Sqrt_extension<NT,ROOT>} is model of:\\
        \hline
          \ccc{CGAL::IntegralDomainWithoutDivision} 
                & \ccc{IntegralDomainWithoutDivision}\\     
          \ccc{CGAL::IntegralDomain}    & \ccc{IntegralDomain}\\     
          \ccc{CGAL::UniqueFactorizationDomain} 
                & \ccc{IntegralDomain}\\     
          \ccc{CGAL::EuclideanRing}     & \ccc{IntegralDomain}\\     
          \ccc{CGAL::Field}             & \ccc{Field}\\     
          \ccc{CGAL::FieldWithSqrt}     & \ccc{Field} (not recommended)\\
          \ccc{CGAL::FieldWithKthRoot}  & \ccc{Field} (not recommended)\\
          \ccc{CGAL::FieldWithRootOf}   & \ccc{Field} (not recommended)\\
        \hline
        \end{tabular}

Note that the extension of an \ccc{UniqueFactorizationDomain} or 
\ccc{EuclideanRing} is just an 
\ccc{IntegralDomain}, since the extension in general destroys the unique 
factorization property. \\
For example take $\Z[\sqrt{10}]$:  
%as indicated by \cite{e-eacfcc-03} referring to \cite{hardy-wright,hasse}:
For the fact that $2$, $5$, and $\sqrt{10}$ are irreducible elements 
of $\Z[\sqrt{10}]$. It follows immediately that $\Z[\sqrt{10}]$ is not a 
\ccc{UniqueFactorizationDomain},
because \[10 = \sqrt{10} \cdot \sqrt{10} = 2 \cdot 5\] factors in two essentially
different ways. \\

%motivate the type , with respect to leda:real 
\ccc{Sqrt_extension} is motivated due to the fact that
 number types such as \ccc{leda_real}
or \ccc{CORE::Expr} 
%\footnote{ For a details discussion on these types see 
%\cite{mn-lpcgc-00, klpy-clp-99}} 
being a model of the superior \ccc{FieldWithSqrt} concept should 
%avoid to many arithmetic operations. As a consequence they should
not be use as the coefficient type of polynomials, since e.g. an polynomial gcd 
operation would cause large and thus slow expression trees.  
In contrast, an object of \ccc{Sqrt_extension} represents an extension 
by one square root only, which is defined at construction time. As a consequence
two objects of this type are interoperable if and only if they are defined 
within the same extension
%\footnote{It is also possible to leave the extension undefined, in particular 
%\ccc{Sqrt_extension} is constructible from \ccc{int}}.
The type \ccc{Sqrt_extension} is thus predestinated to be used as the 
coefficient type of polynomials defined over a particular algebraic extension 
of degree 2 or nested forms,  
%optional remark 
%in particular it is possible to use it as the coefficient type of an 
%\ccc{NiX::Algebraic\_real}.
\\


%\begin{ccAdvanced}
%\ccc{Sqrt_extension} is \ccc{Modularizable} in case \ccc{NT} is 
%\ccc{Modularizable}.
%\end{ccAdvanced}

\ccIsModel
\begin{tabular}{ll}
\ccc{IntegralDomainWithoutDivision}& if \ccc{NT} is an
        \ccc{IntegralDomainWithoutDivision}\\
\ccc{IntegralDomain} &if \ccc{NT} is an \ccc{IntegralDomain}\\
\ccc{Field} &if \ccc{NT} is a \ccc{Field}\\
\\ \ccc{RealEmbeddable} &if \ccc{NT} is \ccc{RealEmbeddable}\\
\\ \ccc{Fraction} &if \ccc{NT} is a \ccc{Fraction}\\
\end{tabular}

\ccCreation
\ccCreationVariable{ext}
\ccConstructor{Sqrt_extension ();}
        {Introduces an variable initialized with 0.}
\ccConstructor{Sqrt_extension (const int &i);}
        {Explicit constructor from int.}
\ccConstructor{Sqrt_extension (const NT &x);}
        {Explicit constructor from type NT.}
\ccConstructor{Sqrt_extension (int a0, int a1, int r);}
 	{Constructor from int: \ccVar $= a0 +a1 \cdot sqrt(r)$.}
\ccConstructor{Sqrt_extension (NT a0, NT a1, ROOT r);}
 	{General constructor: \ccVar $= a0 + a1 \cdot sqrt(r)$.}
\ccConstructor{Sqrt_extension (const Sqrt_extension& x);}
 	{copy constructor.}

\begin{ccAdvanced}
\ccConstructor{
        template<class NTX> 
        Sqrt_extension (const NTX &x);}
        {Explicit constructor of \ccc{Sqrt_extension}, from any type NTX.\\
         \ccc{NT} must be constructible from \ccc{NTX}.}
\ccConstructor{
        template<class NTX,class ROOTX> 
        Sqrt_extension (const Sqrt_extension< NTX, ROOTX > &x);}
        {Explicit constructor of \ccc{Sqrt_extension}, from \ccc{Sqrt_extension}<NTX,ROOTX>.\\
           \ccc{NT} must be constructible from \ccc{NTX}.\\
           \ccc{ROOT} must be constructible from \ccc{ROOTX}.}
\ccConstructor{
        template<class NTX,class ROOTX> 
        Sqrt_extension (NTX a0, NTX a1, ROOTX r);}
        {General constructor: \ccVar $= a0 + a1 \cdot sqrt(r)$.\\
         \ccc{NT} must be constructible from \ccc{NTX}.\\
         \ccc{ROOT} must be constructible from \ccc{ROOTX}.}
\end{ccAdvanced}

\ccOperations

An object of type \ccc{Sqrt_extension} is considered to be 
represented by the following expression: $a0 + a1 * sqrt(root)$. 
The following methods give access to these values:

\ccMethod{const NT & a0 () const ;    }{Access operator for a0, const.} \ccGlue
%\ccMethod{NT & 	a0 ()    ;    }{Access operator for a0.}\ccGlue
\ccMethod{const NT & a1 () const ;    }{Access operator for a1, const.}\ccGlue
%\ccMethod{NT & 	a1 ()    ;    }{Access operator for a1.}\ccGlue
\ccMethod{const ROOT & 	root () const;}{Access operator for root, const.}\ccGlue
%\ccMethod{ROOT & 	root ()  ;    }{Access operator for root.}

Member functions:\\
\ccMethod{bool is_extended () const;}
 	{Returns true in case root of \ccVar\ is not zero.  \\
         Note that $a1 == 0 $ does not imply $root == 0$. }   
\ccMethod{void 	simplify ();}
        { Propagates the simplify command to members of \ccVar. \\
          see also: \ccc{AlgebraicStructureTraits::Simplify}.}
\ccMethod{bool 	is_zero () const;}
        {returns true if \ccVar\ represents the value zero.}

\ccMethod{CGAL::Sign sign () const;}
        {Determines sign of \ccVar\ by repeated squaring.
        \ccPrecond\\ \ccc{Sqrt_extension} is \ccc{RealEmbeddable}.}
\ccMethod{Sqrt_extension abs () const;}
 	{returns the absolute value of \ccVar. 
        \ccPrecond\\ \ccc{Sqrt_extension} is \ccc{RealEmbeddable}.}

Mixed operations: 

\ccFunction{Sqrt_extension operator+(const Sqrt_extension&a, const NT &b);}{}\ccGlue
\ccFunction{Sqrt_extension operator-(const Sqrt_extension&a, const NT &b);}{}\ccGlue
\ccFunction{Sqrt_extension operator*(const Sqrt_extension&a, const NT &b);}{}\ccGlue


\ccFunction{Sqrt_extension operator+(const NT&a, const Sqrt_extension &b);}{}\ccGlue
\ccFunction{Sqrt_extension operator-(const NT&a, const Sqrt_extension &b);}{}\ccGlue
\ccFunction{Sqrt_extension operator*(const NT&a, const Sqrt_extension &b);}{}\ccGlue


\ccMethod{Sqrt_extension & 	operator += (const NT &a);}{}\ccGlue
\ccMethod{Sqrt_extension & 	operator -= (const NT &a);}{}\ccGlue
\ccMethod{Sqrt_extension & 	operator *= (const NT &a);}{}\ccGlue

In case  \ccc{Sqrt_extension} is a \ccc{Field}:
\ccFunction{Sqrt_extension operator/(const Sqrt_extension&a, const NT &b);}{}\ccGlue
\ccFunction{Sqrt_extension operator/(const NT&a, const Sqrt_extension &b);}{}\ccGlue
\ccMethod{Sqrt_extension & 	operator /= (const NT &a);}{}

\ccFunction{bool operator==(const NT&a, const Sqrt_extension &b);}{}\ccGlue
\ccFunction{bool operator!=(const NT&a, const Sqrt_extension &b);}{}\ccGlue
\ccFunction{bool operator==(const Sqrt_extension&a, const NT &b);}{}\ccGlue
\ccFunction{bool operator!=(const Sqrt_extension&a, const NT &b);}{}

In case  \ccc{Sqrt_extension} is \ccc{RealEmbeddable}:
\ccFunction{bool operator< (const NT&a, const Sqrt_extension &b);}{}\ccGlue
\ccFunction{bool operator<=(const NT&a, const Sqrt_extension &b);}{}\ccGlue
\ccFunction{bool operator> (const NT&a, const Sqrt_extension &b);}{}\ccGlue
\ccFunction{bool operator>=(const NT&a, const Sqrt_extension &b);}{}

\ccFunction{bool operator< (const Sqrt_extension&a, const NT &b);}{}\ccGlue
\ccFunction{bool operator<=(const Sqrt_extension&a, const NT &b);}{}\ccGlue
\ccFunction{bool operator> (const Sqrt_extension&a, const NT &b);}{}\ccGlue
\ccFunction{bool operator>=(const Sqrt_extension&a, const NT &b);}{}

The stream operations are available as well. 
They assume that corresponding stream operators for type \ccc{NT} and \ccc{ROOT} exist.

\ccFunction{std::ostream& operator<<(std::ostream& os, const Sqrt_extension<NT,ROOT> &ext);}
{writes \ccc{ext} to ostream \ccc{os}. The format depends on the \ccc{CGAL::IO::MODE} of \ccc{os}.\\
        In case the mode is \ccc{CGAL::IO::ASCII} the format is {\tt EXT[a0,a1,root]}. \\
        In case the mode is \ccc{CGAL::IO::PRETTY} the format is human readable. \\
        }

\ccFunction{std::istream& operator>>(std::istream& is, const Sqrt_extension<NT,ROOT> &ext);}
{reads \ccc{ext} from istream \ccc{is} in format {\tt EXT[a0,a1,root]}, the output format in mode \ccc{CGAL::IO::ASCII} }



%void 	output_maple (std::ostream &os) const
% 	write Sqrt_extension to os in LiS::IO::PRETTY format
%void 	output_ascii (std::ostream &os) const
% 	write Sqrt_extension to os in a format readable by input_ascii()
\end{ccRefClass}
