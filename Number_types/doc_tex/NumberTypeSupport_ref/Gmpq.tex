\begin{ccRefClass} {Gmpq}
\label{Gmpq}

\ccDefinition

An object of the class \ccStyle{Gmpq} is an arbitrary precision rational
number based on the {\sc Gnu} Multiple Precision Arithmetic Library. 

\ccInclude{CGAL/Gmpq.h}

\ccIsModel
\ccc{Field}\\
\ccc{RealEmbeddable}\\
\ccc{Fraction}

\ccTypes

\ccNestedType{NT}{the field type, which is \ccc{Gmpz}.  }

\ccCreation
\ccCreationVariable{q}

\ccConstructor{Gmpq();}
             {creates an uninitialized multiple precision rational number \ccVar.}

\ccHidden \ccConstructor{Gmpq(const Gmpq& q);}
            {copy constructor.}

\ccConstructor{Gmpq(int i)}
            {creates a multiple-precision rational number initialized with
             \ccStyle{i}.}

\ccConstructor{Gmpq(Gmpz n)}
            {creates a multiple-precision rational number initialized with
             \ccStyle{n}.}

\ccConstructor{Gmpq(int n, int d)}
            {creates a multiple-precision rational number initialized with
             \ccStyle{n/d}.}

\ccConstructor{Gmpq(signed long n, unsigned long d)}
            {creates a multiple-precision rational number initialized with
             \ccStyle{n/d}.}

\ccConstructor{Gmpq(unsigned long n, unsigned long d)}
            {creates a multiple-precision rational number initialized with
             \ccStyle{n/d}.}

\ccConstructor{Gmpq(Gmpz n, Gmpz d)}
            {creates a multiple-precision rational number initialized with
             \ccStyle{n/d}.}

\ccConstructor{Gmpq(double d)}
            {creates a multiple-precision rational number initialized with
             \ccStyle{d}.}

\ccConstructor{Gmpq(const std::string& str)}
            {creates a multiple-precision rational number initialized with
             \ccStyle{str}.}

\ccConstructor{Gmpq(const std::string& str, int base)}
            {creates a multiple-precision rational number initialized with
             \ccStyle{str}.}

%\ccSetTwoOfThreeColumns{5cm}{4cm}
%SetThreeColumns{std::ostream& }{}{\hspace*{8cm}}

\ccOperations

There are two access functions, namely to the
numerator and the denominator of a rational.
Note that these values are not uniquely defined. 
It is guaranteed that \ccStyle{q.numerator()} and 
\ccStyle{q.denominator()} return values \ccStyle{nt_num} and
\ccStyle{nt_den} such that \ccStyle{q = nt_num/nt_den}, only
if  \ccStyle{q.numerator()} and \ccStyle{q.denominator()} are called
consecutively wrt \ccStyle{q}, i.e.~\ccStyle{q} is not involved in 
any other operation between these calls.


\ccMethod{Gmpz numerator() const;}
       {returns the numerator of \ccStyle{q}.}
\ccGlue
\ccMethod{Gmpz denominator() const;}
       {returns the denominator of \ccStyle{q}.}

\ccImplementation
\ccc{Gmpq}s are reference counted.

\end{ccRefClass} 
\ccDefGlobalScope{CGAL::}
