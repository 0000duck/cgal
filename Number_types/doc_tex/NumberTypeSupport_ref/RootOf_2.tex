\begin{ccRefConcept}{RootOf_2}

\ccDefinition

Concept to represent algebraic numbers of
degree up to~2 over a \ccc{RingNumberType} denoted as \ccc{RT}.

%\ccc{FT} denotes the \ccc{FieldNumberType} constructed from \ccc{RT}.
%\footnote{see \ccc{make_rational}, we need something to get FT from RT in a clean way.}

\ccCreation

%\footnote{How to get RT...?}

\ccFunction{RootOf_2 make_root_of_2(RT a, RT b, RT c, bool s){Returns the 
smaller or larger root of the equation $aX^2+bX+c=0$, if \ccc{s}
is \ccc{true} or \ccc{false}, respectively. \ccPrecond{$b^2-4ac\geq 0$}}
%\footnote{numbering of roots from 0 or 1? to be checked}
\ccGlue
\ccFunction{RootOf_2 make_root_of_2(FT a, FT b, FT c, bool s);}{Returns the 
smaller or larger root of the equation $aX^2+bX+c=0$, if \ccc{s}
is \ccc{true} or \ccc{false}, respectively. \ccPrecond{$b^2-4ac\geq 0$}}
\ccGlue
\ccFunction{RootOf_2 make_root_of_2(FT alpha, FT beta, FT gamma);}{Constructs 
the number $\alpha + \beta \sqrt{\gamma}$. \ccPrecond{$\gamma\geq 0$}}


\ccOperations

The comparison operators \ccc{==, !=, <, >, <=, >=} as well as the \ccc{sign}
and \ccc{compare} functions need to be
provided to compare elements of types \ccc{RootOf_2, RT} and \ccc{FT}. 

In addition, the following operations must be provided:

\def\ccTagRmEigenClassName{\ccFalse}

\ccFunction{RootOf_2 operator+(const RT &a, const RootOf_2 &r);}{}
\ccGlue
\ccFunction{RootOf_2 operator+(const FT &a, const RootOf_2 &r);}{}
\ccGlue
\ccFunction{RootOf_2 operator+(const RootOf_2 &r, const RT &a);}{}
\ccGlue
\ccFunction{RootOf_2 operator+(const RootOf_2 &r, const FT &a);}{}

as well as
\ccFunction{RootOf_2 operator+(const RootOf_2 &r, const RootOf_2 &r);}
{Addition of two algebraic numbers, in the special case when they have 
been constructed from equations having the same discriminant.}

and similarly for operators -, * and /.

\ccFunction{RootOf_2 square(const RootOf_2 & r);}{}
\ccFunction{RootOf_2 inverse(const RootOf_2 & r);}{}
\ccFunction{RootOf_2 make_sqrt(const RT & r);}{}
\ccFunction{RootOf_2 make_sqrt(const FT & f);}{}

\def\ccTagRmEigenClassName{\ccTrue}

\ccHasModels

\ccc{double}, \ccc{CGAL::Root_of_2}, etc %\footnote{to be precised}

\ccSeeAlso

\ccRefIdfierPage{CGAL::make_root_of_2}\\
\ccRefIdfierPage{CGAL::Root_of_2<RT>}\\
\ccRefIdfierPage{CGAL::Root_of_traits_2<RT>}\\
\ccRefIdfierPage{AlgebraicKernelForCircles::PolynomialForCircles_2_2}\\
\ccRefIdfierPage{AlgebraicKernelForCircles}

\end{ccRefConcept}
