% $Id$
% $Date$
% author : Sylvain Pion

\begin{ccRefClass} {MP_Float}
%\subsection{Multi Precision Floats}
%\label{mpfloat}

\ccDefinition
An object of the class \ccc{MP_Float} is able to represent a floating point
value with arbitrary precision.  This number type has the property that
additions, subtractions and multiplications are computed exactly, as well as
the construction from \ccc{float}, \ccc{double} and \ccc{long double}.

Division and square root are not enabled by default since \cgal\ release 3.2,
since they are computed approximately.  We suggest that you use
rationals like \ccc{Quotient<MP_Float>} when you need exact divisions.
To enable division and square root, you have to define the preprocessor
macro \ccc{CGAL_MP_FLOAT_ALLOW_INEXACT}.

Note on the implementation : although the mantissa length is basically only
limited by the available memory, the exponent is currently represented by a
(integral valued) \ccc{double}, which can overflow in some circumstances.  We
plan to also have a multiprecision exponent to fix this issue.

\ccInclude{CGAL/MP_Float.h}

\ccIsModel
\ccc{RingNumberType}
or \ccc{SqrtFieldNumberType} if \ccc{CGAL_MP_FLOAT_ALLOW_INEXACT} is set.

\ccCreation
\ccCreationVariable{m}

\ccConstructor{MP_Float();}
{introduces an uninitialized variable \ccVar.}
\ccGlue
\ccConstructor{MP_Float(const MP_Float &);}
{copy constructor.}
\ccGlue
\ccConstructor{MP_Float(int i)}
{introduces the integral value i.}
\ccGlue
\ccConstructor{MP_Float(float d)}
{introduces the floating point value d (exact conversion).}
\ccGlue
\ccConstructor{MP_Float(double d)}
{introduces the floating point value d (exact conversion).}
\ccGlue
\ccConstructor{MP_Float(long double d)}
{introduces the floating point value d (exact conversion).}

\ccOperations

\ccFunction{std::ostream& operator<<(std::ostream& out, const MP_Float& m);}
{writes a double approximation of \ccc{m} to the ostream \ccc{out}.}

\ccFunction{std::istream& operator>>(std::istream& in, MP_Float& m);}
{reads a \ccc{double} from \ccc{in}, then converts it to an \ccc{MP_Float}.}

\ccFunction{MP_Float approximate_division(const MP_Float &a, const MP_Float &b);}
{computes an approximation of the division by converting the operands to
\ccc{double}, performing the division on \ccc{double}, and converting back to
\ccc{MP_Float}.}

\ccFunction{MP_Float approximate_sqrt(const MP_Float &a);}
{computes an approximation of the square root by converting the operand to
\ccc{double}, performing the square root on \ccc{double}, and converting back
to \ccc{MP_Float}.}


\ccImplementation 
The implementation of \ccc{MP_Float} is simple but provides a quadratic
complexity for multiplications.  This can be a problem for large operands.
For faster implementations of the same functionality with large integral
values, you may want to consider using \ccc{GMP} or \ccc{LEDA} instead.

\end{ccRefClass} 
