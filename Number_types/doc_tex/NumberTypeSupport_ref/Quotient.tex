
\begin{ccRefClass} {Quotient<NT>}
\label{Quotient}
%\subsection{Quotient}

\ccDefinition
An object of the class \ccStyle{Quotient<NT>} is an element of the 
field of quotients of the integral domain type \ccStyle{NT}.
If \ccStyle{NT} behaves like an integer, \ccStyle{Quotient<NT>}
behaves like a rational number. 
{\leda}'s class \ccStyle{rational} (see Section~\ref{leda-nt})
has been the basis for \ccStyle{Quotient<NT>}.
A \ccStyle{Quotient<NT>} \ccStyle{q} is represented as a pair of 
\ccStyle{NT}s, representing numerator and denominator.

\ccc{NT} must be at least model of concept \ccc{IntegralDomainWithoutDivision}.\\
\ccc{NT} must be a model of concept \ccc{RealEmbeddable}. \\


\ccInclude{CGAL/Quotient.h}

\ccIsModel

\ccc{Field}\\
\ccc{RealEmbeddable}\\
\ccc{Fraction}

\ccCreation
\ccCreationVariable{q}

\ccConstructor{Quotient();}
             {introduces an uninitialized variable \ccVar.}

\ccHidden \ccConstructor{Quotient(const Quotient<NT> &q);}
 	    {copy constructor.}
\ccGlue
\ccConstructor{template <class T> Quotient<NT>(const T& t);}
{introduces the quotient \ccStyle{t/1}. NT needs to have a constructor from T.}
\ccGlue
\ccConstructor{template <class T> Quotient<NT>(const Quotient<T>& t);}
{introduces the quotient \ccStyle{NT(t.numerator())/NT(t.denominator())}.
NT needs to have a constructor from T.}
\ccGlue
\ccConstructor{Quotient(const NT& n, const NT& d)}
            {introduces the quotient \ccStyle{n/d}.\\
            \ccPrecond $d \neq 0$.         }


\ccOperations

%\ccSetTwoOfThreeColumns{5cm}{4cm}
%SetThreeColumns{std::ostream& }{}{\hspace*{8cm}}

There are two access functions, namely to the
numerator and the denominator of a quotient.
Note that these values are not uniquely defined. 
It is guaranteed that \ccStyle{q.numerator()} and 
\ccStyle{q.denominator()} return values \ccStyle{nt_num} and
\ccStyle{nt_den} such that \ccStyle{q = nt_num/nt_den}, only
if  \ccStyle{q.numerator()} and \ccStyle{q.denominator()} are called
consecutively wrt \ccStyle{q}, i.e.~\ccStyle{q} is not involved in 
any other operation between these calls.

\ccMethod{NT numerator() const;}
       {returns a numerator of \ccStyle{q}.}
\ccGlue
\ccMethod{NT denominator() const;}
       {returns a denominator of \ccStyle{q}.}

\ccHidden \ccMethod{Quotient<NT>& normalize();}
{}

The stream operations are available as well. 
They assume that corresponding stream operators for type \ccc{NT} exist.

\ccFunction{std::ostream& operator<<(std::ostream& out, const Quotient<NT>& q);}
       {writes \ccc{q} to ostream \ccc{out} in format "{\tt n/d}", where
       {\tt n}$==$\ccc{q.numerator()} and {\tt d}$==$\ccc{q.denominator()}.}

\ccFunction{std::istream& operator>>(std::istream& in, Quotient<NT>& q);}
       {reads \ccc{q} from istream \ccc{in}. Expected format is
        "{\tt n/d}", where {\tt n} and {\tt d} are of type \ccc{NT}.
        A single {\tt n} which is not followed by a {\tt /}\  is also
        accepted and interpreted as {\tt n/1}.}

The following functions are added to fulfill the \cgal\ requirements
on number types.

\ccFunction{double to_double(const Quotient<NT>& q);}
       {returns some double approximation to \ccStyle{q}.}
\ccGlue
\ccFunction{bool  is_valid(const Quotient<NT>& q);}
       {returns true, if numerator and denominator are valid.}
\ccGlue
\ccFunction{bool  is_finite(const Quotient<NT>& q);}
       {returns true, if numerator and denominator are finite.}
\ccGlue
\ccFunction{Quotient<NT>  sqrt(const Quotient<NT>& q);}
       {returns the square root of \ccc{q}.  This is supported if and only if
        \ccc{NT} supports the square root as well.}

\end{ccRefClass} 
