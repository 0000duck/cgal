% +------------------------------------------------------------------------+
% | CGAL User Manual: 
% +------------------------------------------------------------------------+
% |
% | 28.05.2008   Peter Hachenberger
% | 
\RCSdef{\ConvexDecomposition3Rev}{$Id$}
\RCSdefDate{\ConvexDecomposition3Date}{$Date$}
% +------------------------------------------------------------------------+

\ccParDims

\ccUserChapter{Convex Decomposition of Polyhedra \label{chapterConvexDecomposition3}}
\ccChapterRelease{\ConvexDecomposition3Rev. \ \ConvexDecomposition3Date}
\ccChapterAuthor{Peter Hachenberger}

%
\begin{ccPkgDescription}{3D Convex Hulls\label{Pkg:ConvexHull3}}
\ccPkgHowToCiteCgal{cgal:hs-ch3-07}
\ccPkgSummary{This package provides functions 
for computing convex hulls in three dimensions as well as functions
for checking if sets of points are strongly convex are not. One can
compute the convex hull of a set of points in three dimensions in one
of three ways: using a static algorithm, using an incremental
construction algorithm, or using a triangulation to get a fully
dynamic computation.}

\ccPkgDependsOn{All algorithms produce as output a \ccRef[3D Polyhedron]{Pkg:Polyhedron}. 
                The dynamic algorithms depend on \ccRef[3D Triangulations]{Pkg:Triangulation3}}
\ccPkgIntroducedInCGAL{1.1}
\ccPkgLicense{\ccLicenseQPL}
\ccPkgIllustration{Convex_hull_3/bunny.png}{Convex_hull_3/bunny.png}
\end{ccPkgDescription}


% +------------------------------------------------------------------------+
\section{Introduction}

For many applications on non-convex polyhedra, there are efficient
solutions that first decompose the polyhedron into convex pieces. As
an example, the Minkowski sum of two polyhedra can be computed by
decomposing both polyhedra into convex pieces, compute pair-wise
Minkowski sums of the convex pieces, and unite the pair-wise sums.

While it is desirable to have a decomposition into a minimum number of
pieces, this problem is know to be NP-hard~\cite{cgal:c--}. Our
implementation decomposes a Nef polyhedron $N$ into $O(r^2)$ convex
pieces, where $r$ is the number of edges, which have two adjacent
facets that span an angle of more than 180 degrees with respect to the
interior of the polyhedron. Those edges are also called reflex edges.
The bound of $O(r^2)$ convex pieces is worst-case
optimal~\cite{cgal:c--}.

At the moment our implementation is restricted to the decomposition of
finite point sets. If the input polyhedron is infinite, i.e., the
outer volume is part of the polyhedron, then this volume is ignored
during the decomposition process. An extension to infinite point sets
is planned.

% +------------------------------------------------------------------------+
\section{Usage}

The following example illustrates the usage of the function
\ccc{convex_decomposition_3}. It takes a \ccc{Nef_polyhedron_3}
$N$ as input parameter. The result is the modified polyhedron $N$.
After the execution of the function, it contains additional facets,
such that each marked volume (except for the outer volume) is
subdivided into convex pieces. The convex pieces can then be used by
traversing $N$, or by converting them into separate Nef polyhedra, as
shown in the example code.

\ccIncludeExampleCode{Convex_decomposition_3/.cpp}
