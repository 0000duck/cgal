% +------------------------------------------------------------------------+
% | Reference manual page: convex_decomposition_3.tex
% +------------------------------------------------------------------------+
% | 28.05.2008   Peter Hachenberger
% | Package: Convex_decomposition_3
% | 
\RCSdef{\RCSconvex_decomposition_3Rev}{$Id$}
\RCSdefDate{\RCconvex_decomposition_3Date}{$Date$}
% |
%%RefPage: end of header, begin of main body
% +------------------------------------------------------------------------+

\ccHtmlNoClassLinks
\begin{ccRefFunction}{convex_decomposition_3}
\label{refconvex_decomposition_3}

\ccDefinition

The function \ccc{convex_decomposition_3} inserts additional facets
into the given \ccc{Nef_polyhedron_3} $N$, such that each bounded
marked volume (the outer volume is unbouned) is subdivided into convex
pieces. The modified polyhedron represents a decomposition into
$O(r^2)$ convex pieces, where $r$ is the number of edges that have two
adjacent facets that span an angle of more than 180 degrees with
respect to the interior of the polyhedron.

The worst-case running time of our implementation is
$O(n^2r^4\sqrt[3]{nr^2}\log{(nr)})$, where $n$ is the complexity of
the polyhedron (the complexity of a \ccc{Nef_polyhedron_3} is the sum
of its \ccc{Vertices}, \ccc{Halfedges} and \ccc{SHalfedges}) and $r$
is the number of reflex edges.

\ccGlobalFunction{void convex_decomposition_3(Nef_polyhedron_3& N);}

\ccPrecond

The polyhedron $N$ is bounded. Otherwise, the outer volume is ignored.

\ccPostcond

If the polyhedron $N$ is non-convex, it is modified to represent the
convex decomposition. If $N$ is convex, it is not modified.

\ccSeeAlso
\ccRefIdfierPage{CGAL::Nef_polyhedron_3<Traits>}\\ 

\end{ccRefFunction}
