\cgal, the {\em Computational Geometry Algorithms Library}, is written in 
\CC\ and consists of three major parts.
The first part is the kernel, which consists of constant-size non-modifiable 
geometric primitive objects and operations on these objects. 
The objects are represented both as stand-alone classes that are
parameterized by a representation class, which specifies
the underlying number types used for calculations and as members of the
kernel classes, which allows for more flexibility and adaptability of the 
kernel.
The second part is a collection of basic geometric data structures and
algorithms, which are parameterized by traits classes that define the 
interface between the data structure or algorithm and the primitives they use.
In many cases, the kernel classes provided in \cgal\ can be used as traits
classes for these data structures and algorithms.
The third part of the library consists of non-geometric support facilities, 
such as circulators, random sources, I/O support for debugging and for 
interfacing \cgal\ to various visualization tools.  
