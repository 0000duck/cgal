\begin{ccRefFunctionObjectConcept}{Kernel::ConstructPlane_3}
A model for this must provide:

\ccCreationVariable{fo}

\ccHidden \ccMemberFunction{Kernel::Plane_3 operator()();}
             {introduces an uninitialized variable .}

\ccHidden \ccMemberFunction{Kernel::Plane_3 operator()(const Kernel::Plane_3 &h);}
 	    {copy constructor.}

\ccMemberFunction{Kernel::Plane_3 operator()(const Kernel::RT &a, 
                       const Kernel::RT &b,
                       const Kernel::RT &c,
                       const Kernel::RT &d);}
{creates a plane  defined by the equation
 \ccTexHtml{$a\, px +b\, py +c\, pz + d = 0$}{a px + b py + c pz + d = 0}.
Notice that  is degenerate if \ccTexHtml{$a = b = c$}{a = b = c}.}

\ccMemberFunction{Kernel::Plane_3 operator()(const Kernel::Point_3 &p,
	               const Kernel::Point_3 &q,
	               const Kernel::Point_3 &r);}
{creates a plane  passing through the points \ccStyle{p},
 \ccStyle{q} and \ccStyle{r}. The plane is oriented such that \ccStyle{p}, 
 \ccStyle{q} and \ccStyle{r} are oriented in a positive sense 
 (that is counterclockwise) when seen from the positive side of the plane.
Notice that  is degenerate if the points are collinear.}


\ccMemberFunction{Kernel::Plane_3 operator()(const Kernel::Point_3 &p,
	                  const Kernel::Direction_3&d);}
{introduces a plane  that passes through point \ccStyle{p} and
 that has as an orthogonal direction equal to \ccStyle{d}.}

\ccMemberFunction{Kernel::Plane_3 operator()(const Kernel::Line_3 &l,
                          const Kernel::Point_3 &p);}
{introduces a plane  that is defined through the  three points 
 \ccStyle{l.point(0)}, \ccStyle{l.point(1)} and \ccStyle{p}.}

\ccMemberFunction{Kernel::Plane_3 operator()(const Kernel::Ray_3 &r,
                          const Kernel::Point_3 &p);}
{introduces a plane  that is defined through the  three points 
 \ccStyle{r.point(0)}, \ccStyle{r.point(1)} and \ccStyle{p}.}

\ccMemberFunction{Kernel::Plane_3 operator()(const Kernel::Segment_3 &s,
                          const Kernel::Point_3 &p);}
{introduces a plane  that is defined through the  three points 
 \ccStyle{s.source()}, \ccStyle{s.target()} and \ccStyle{p}.}

\ccSeeAlso
\ccRefIdfierPage{CGAL::Plane_3<R>} \\

\end{ccRefFunctionObjectConcept}
