\begin{ccRefFunctionObjectConcept}{Kernel::Oriented_side_2}
A model for this must provide:

\ccCreationVariable{fo}

\ccMemberFunction{Oriented_side operator()(const Kernel::Cricle_2&c, 
                                           const Kernel::Point_2&p);} 
{returns \ccStyle{ON_ORIENTED_BOUNDARY},
\ccStyle{ON_NEGATIVE_SIDE}, or the constant \ccStyle{ON_POSITIVE_SIDE},
depending on the position of $p$  relative to the oriented circle $c$.}

\ccMemberFunction{Oriented_side operator()(const Kernel::Line_2&l, 
                                           const Kernel::Point_2&p);} 
{returns \ccStyle{ON_ORIENTED_BOUNDARY},
\ccStyle{ON_NEGATIVE_SIDE}, or the constant \ccStyle{ON_POSITIVE_SIDE},
depending on the position of $p$  relative to the oriented line $l$.}

\ccMemberFunction{Oriented_side operator()(const Kernel::Triangle_2&t, 
                                           const Kernel::Point_2&p);} 
{returns \ccStyle{ON_ORIENTED_BOUNDARY},
\ccStyle{ON_NEGATIVE_SIDE}, or the constant \ccStyle{ON_POSITIVE_SIDE},
depending on the position of $p$  relative to the oriented triangle $t$.}

\end{ccRefFunctionObjectConcept}
