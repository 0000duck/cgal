\begin{ccRefClass} {Direction_2<R>}
\ccInclude{CGAL/Direction_2.h}

\ccDefinition
An object of the class \ccRefName\ is a vector in the two-dimensional 
vector space $\R^2$  where we forget about its length. They can be
viewed as unit vectors, although there is no normalization internally,
since this is error prone.  Directions are used whenever the length of
a vector does not matter. 
They also characterize a set of parallel oriented lines that have the same
orientations.  
For example, you can ask for the direction
orthogonal to an oriented plane, or the direction of an oriented line.
Further, they can be used to indicate angles. The slope of a direction
is \ccStyle{dy()/dx()}.


\ccCreation
\ccCreationVariable{d}


\ccHidden \ccConstructor{Direction_2();}
             {introduces an uninitialized direction \ccVar.}

\ccHidden \ccConstructor{Direction_2(const Direction_2<R> &d);}
 	    {copy constructor.}

\ccConstructor{Direction_2(const Vector_2<R> &v);}
	    {introduces the direction \ccVar\ of vector $v$.}

\ccConstructor{Direction_2(const R::RT &x, const R::RT &y);}
            {introduces a direction \ccVar\ passing through the origin
             and the point with Cartesian coordinates $(x, y)$.}


\ccOperations
%\ccSetTwoOfThreeColumns{5cm}{4cm}
\ccSetThreeColumns{Direction_2<R> & }{}{\hspace*{7.8cm}}

\ccHidden \ccMethod{Direction_2<R> & operator=(const Direction_2<R> &e);}
        {Assignment.}

\ccMethod{R::RT delta(int i) const;}
       {returns values, such that \ccVar \ccc{== Direction_2<R>(delta(0),delta(1))}.
        \ccPrecond: $0 \leq i \leq 1$.}

\ccMethod{R::RT dx() const;}
       {returns \ccc{delta(0)}.}

\ccMethod{R::RT dy() const;}
       {returns \ccc{delta(1)}.}

There is a total order on directions. We compare the angles between the
positive $x$-axis and the directions in counterclockwise order.

\ccMethod{bool operator==(const Direction_2<R> &e) const;}
       {}
\ccGlue
\ccMethod{bool operator!=(const Direction_2<R> &e) const;}
       {}
\ccGlue
\ccMethod{bool operator<(const Direction_2<R> &e) const;}
       {}
\ccGlue
\ccMethod{bool operator>(const Direction_2<R> &e) const;}
       {}
\ccGlue
\ccMethod{bool operator<=(const Direction_2<R> &e) const;}
       {}
\ccGlue
\ccMethod{bool operator>=(const Direction_2<R> &e) const;}
       {}

Furthermore, we have

\ccMethod{bool counterclockwise_in_between(const Direction_2<R> &d1,
                                   const Direction_2<R> &d2) const;}
       {returns true, iff \ccVar\ is not equal to \ccc{d1}, and 
        while rotating counterclockwise starting at \ccc{d1}, 
        \ccVar\ is reached strictly before \ccc{d2} is reached.
        Note that true is returned if \ccc{d1} == \ccc{d2}, unless
        also \ccVar\ == \ccc{d1}.
       }


\ccMethod{Direction_2<R>  operator-() const;}
       {The direction opposite to \ccVar.}

\ccHeading{Miscellaneous}

\ccMethod{Vector_2<R> vector() const;}
       {returns a vector that has the same direction as \ccVar.}

\ccMethod{Direction_2<R>  transform(const Aff_transformation_2<R> &t) const;}
       {returns the direction obtained by applying $t$ on \ccVar.}


\end{ccRefClass} 

