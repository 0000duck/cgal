\begin{ccRefFunctionObjectConcept}{Kernel::ConstructVector_2}
A model for this must provide:

\ccCreationVariable{fo}


\ccHidden\ccMemberFunction{Kernel::Vector_2 operator()();}
             {introduces an uninitialized variable .}

\ccHidden \ccMemberFunction{Kernel::Vector_2 operator()(const Kernel::Vector_2 &w);}
            {copy constructor.}

\ccMemberFunction{Kernel::Vector_2 operator()(const Kernel::Point_2 &a,
                                              const Kernel::Point_2 &b);}
            {introduces the vector $b-a$.}

\ccMemberFunction{Kernel::Vector_2 operator()(const Null_vector &NULL_VECTOR);}
            {introduces a null vector .}

\ccHidden\ccMemberFunction{Kernel::Vector_2 operator()(const Kernel::RT &hx, const Kernel::RT &hy, const Kernel::RT &hw = Kernel::RT(1));}
            {introduces a vector  initialized to $(hx/hw,hy/hw)$.
             If the third argument is not explicitly given, it defaults
             to \ccStyle{Kernel::RT(1)}.
             \ccPrecond \ccTexHtml{$hw\neq 0$}{hw not equal to 0} }

\ccRefines
AdaptableFunctor (with two arguments)

\ccSeeAlso
\ccRefIdfierPage{CGAL::Vector_2<R>}\\

\end{ccRefFunctionObjectConcept}
