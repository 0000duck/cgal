\begin{ccRefFunctionObjectConcept}{Kernel::Construct_aff_transformation_2}
A model for this must provide:

\ccCreationVariable{fo}

\ccMemberFunction{Kernel::Aff_transformation_2 operator()(
                                    const Translation,
                                    const Kernel::Vector_2&v);}
            {introduces a translation by a vector $v$.}

\ccMemberFunction{Kernel::Aff_transformation_2 operator()(
                             const Rotation,
                             const Kernel::Direction_2&d,
                             const Kernel::RT &num,
                             const Kernel::RT &den = Kernel::RT(1));}
            {approximates the rotation over the angle indicated by direction 
             $d$, such that the differences between the sines and cosines
             of the rotation given by d and the approximating rotation
             are at most $num/den$ each.
             \ccPrecond $num/den>0$. }

\ccMemberFunction{Kernel::Aff_transformation_2 operator()(
                                       const Rotation,
                                       const Kernel::RT &sine_rho, 
                                       const Kernel::RT &cosine_rho, 
                                       const Kernel::RT &hw = Kernel::RT(1));}
            {introduces a rotation by the angle \ccStyle{rho}.
             \ccPrecond 
             \ccTexHtml{$\mbox{\it sine\_rho}^2 +  \mbox{\it cosine\_rho}^2 == 
hw^2$}{<MATH><i>sine_rho</i><SUP>2</SUP> + <i>cosine_rho</i><SUP>2</SUP> == 
<i>hw</i><SUP>2</SUP></MATH>}.}

\ccMemberFunction{Kernel::Aff_transformation_2 operator()(
                                       const Scaling,
                                       const Kernel::RT &s,
                                       const Kernel::RT &hw = Kernel::RT(1));}
            {introduces a scaling by a scale factor $s/hw$.}

\ccMemberFunction{Kernel::Aff_transformation_2 operator()(
           const Kernel::RT &m00, const Kernel::RT &m01, const Kernel::RT &m02,
           const Kernel::RT &m10, const Kernel::RT &m11, const Kernel::RT &m12,
           const Kernel::RT &hw = Kernel::RT(1));}
            {introduces a general affine transformation in the
             \ccTexHtml{$3 \times 3$ matrix form \usebox{\arrtwo}.}%
             {3x3 matrix <IMG ALIGN=CENTER SRC=arrtwo.gif> .}
             The sub-matrix \ccTexHtml{$1\over hw$\usebox{\arrlintwo}}%
             {<MATH><i>hw</i><SUP>-1</SUP></MATH> <IMG ALIGN=CENTER 
             SRC=arrlintwo.gif>} contains the scaling and rotation 
             information, the vector \ccTexHtml{$1\over hw$
             \usebox{\transvectwo}}{<MATH><i>hw</i><SUP>-1</SUP></MATH>
             <IMG ALIGN=CENTER SRC=transvectwo.gif>}
             contains the translational part of the transformation.}

\savebox{\arrtwo}{\small $\left(\begin{array}{ccc}
                 m_{00} & m_{01} & 0\\
                 m_{10} & m_{11} & 0\\
                  0     &  0     & hw
              \end{array}\right)$}
                  
\ccMemberFunction{Kernel::Aff_transformation_2 operator()(
                  const Kernel::RT &m00, const Kernel::RT &m01,
                  const Kernel::RT &m10, const Kernel::RT &m11,
                  const Kernel::RT &hw = Kernel::RT(1));}
            {introduces a general linear transformation 
             \ccTexHtml{\usebox{\arrtwo},}{<IMG ALIGN=CENTER SRC=arrtwo2.gif> ,}
             i.e.\ there is no translational part.}

\end{ccRefFunctionObjectConcept}
