\begin{ccRefClass} {Point_3<R>}

\ccDefinition
An object of the class \ccRefName\ is a point in the three-dimensional
Euclidean space $\E^3$. 
%% 
%% \cgal\ defines a symbolic constant
%% \ccStyle{ORIGIN}  which denotes the point at the origin. It can be used
%% wherever a point can be used, with the only exception that you can not
%% access its dimension as it is dimensionless.
%% 

Remember that \ccStyle{R::RT} and \ccStyle{R::FT} denote a ring type 
and a field type.
For the representation class \ccStyle{Cartesian<T>}, the two types are 
equivalent.
However, for the representation class \ccStyle{Homogeneous<T>} the ring 
type is \ccStyle{R::RT} is equal to \ccStyle{T} and the field type is 
\ccStyle{R::FT} is equal to \ccStyle{Quotient<T>}.


\ccCreation
\ccCreationVariable{p}


\ccHidden \ccConstructor{Point_3();}
             {introduces an uninitialized variable \ccVar.}

\ccHidden \ccConstructor{Point_3(const Point_3<R> &q);}
 	    {copy constructor.}

\ccConstructor{Point_3(const Origin &ORIGIN);}
 	    {introduces a point with \ccHtmlNoLinks{Cartesian} coordinates$(0,0,0)$.}

\ccConstructor{Point_3(const R::RT &hx, const R::RT &hy, const R::RT &hz, const R::RT &hw = R::RT(1));}
            {introduces a point \ccVar\ initialized to $(hx/hw,hy/hw, hz/hw)$.
             If the third argument is not explicitly given it defaults
             to \ccStyle{R::RT(1)}.
	     \ccPrecond \ccc{hw} $\neq$ \ccc{Kernel::RT(0)}.}


\ccOperations
%\ccSetTwoOfThreeColumns{5cm}{4cm}

\ccHidden \ccMethod{Point_3<R> & operator=(const Point_3<R> &q);}
        {Assignment.}

\ccMethod{bool operator==(const Point_3<R> &q) const;}
       {Test for equality: Two points are equal, iff their $x$, $y$ and $z$
        coordinates are equal.}

\ccMethod{bool operator!=(const Point_3<R> &q) const;}
       {Test for inequality.}



There are two sets of coordinate access functions, namely to the
homogeneous and to the \ccHtmlNoLinks{Cartesian} coordinates. They can be used
independently from the chosen representation type \ccStyle{R}.

\ccMethod{R::RT hx() const;}
       {returns the homogeneous $x$ coordinate.}
\ccGlue
\ccMethod{R::RT hy() const;}
       {returns the homogeneous $y$ coordinate.}
\ccGlue
\ccMethod{R::RT hz() const;}
       {returns the homogeneous $z$ coordinate.}
\ccGlue
\ccMethod{R::RT hw() const;}
       {returns the homogenizing  coordinate.}

Here come the \ccHtmlNoLinks{Cartesian} access functions. Note that you do not loose
information with the homogeneous representation, because then the field
type is a quotient.

\ccMethod{R::FT x() const;}
       {returns the \ccHtmlNoLinks{Cartesian} $x$ coordinate, that is $hx/hw$.}
\ccGlue
\ccMethod{R::FT y() const;}
       {returns the \ccHtmlNoLinks{Cartesian} $y$ coordinate, that is $hy/hw$.}
\ccGlue
\ccMethod{R::FT z() const;}
       {returns the \ccHtmlNoLinks{Cartesian} $z$ coordinate, that is $hz/hw$.}


The following operations are for convenience and for making this
point class compatible with code for higher dimensional points.
Again they come in a \ccHtmlNoLinks{Cartesian} and homogeneous flavor.

\ccMethod{R::RT homogeneous(int i) const;}
       {returns the i'th homogeneous coordinate of \ccVar, starting with 0.
        \ccPrecond $0\leq i \leq 3$.}

\ccMethod{R::FT cartesian(int i) const;}
       {returns the i'th \ccHtmlNoLinks{Cartesian} coordinate of \ccVar, starting with 0.
        \ccPrecond $0\leq i \leq 2$.}

\ccMethod{R::FT operator[](int i) const;}
       {returns \ccStyle{cartesian(i)}.
        \ccPrecond $0\leq i \leq 2$.}

\ccMethod{int dimension() const;}
       {returns the dimension (the constant 3).}

\ccMethod{Bbox_3 bbox() const;}
       {returns a bounding box containing \ccVar.}

\ccMethod{Point_3<R>  transform(const Aff_transformation_3<R> &t) const;}
       {returns the point obtained by applying $t$ on \ccVar.}

\ccHeading{Operators}

The following operations can be applied on points:

\ccFunction{Vector_3<R> operator-(const Point_3<R> &p,
                                  const Point_3<R> &q);}
       {returns the difference vector between \ccStyle{q} and \ccStyle{p}. 
        You can substitute \ccc{ORIGIN} for either \ccc{p} or \ccc{q}
        ,but not for both.}

\ccFunction{Point_3<R> operator+(const Point_3<R> &p,
                                 const Vector_3<R> &v);}
       {returns the point obtained by translating \ccStyle{p} by the 
        vector \ccStyle{v}.}

\ccFunction{Point_3<R> operator-(const Point_3<R> &p,
                                 const Vector_3<R> &v);}
       {returns the point obtained by translating \ccStyle{p} by the 
        vector -\ccStyle{v}.}

\ccSeeAlso

\ccRefConceptPage{Kernel::Point_3}

\end{ccRefClass} 

