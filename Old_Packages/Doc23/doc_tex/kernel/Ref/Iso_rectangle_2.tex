\begin{ccRefClass} {Iso_rectangle_2<R>}
\ccInclude{CGAL/Iso_rectangle_2.h}

\ccDefinition  An object $s$ of the data type \ccRefName\ is a
rectangle in the Euclidean plane $\E^2$ with sides parallel to the $x$ and
$y$ axis of the coordinate system.
 
Although they are represented in a canonical form by only two
vertices, namely the lower left and the upper right vertex, we provide
functions for ``accessing'' the other vertices as well. The vertices
are returned in counterclockwise order.

Iso-oriented rectangles and bounding boxes are quite similar. The
difference however is that bounding boxes have always double coordinates, 
whereas the coordinate type of an iso-oriented rectangle is chosen by
the user.

\ccCreation
\ccCreationVariable{r}


\ccHidden \ccConstructor{Iso_rectangle_2();}
             {introduces an uninitialized variable \ccVar.}

\ccHidden \ccConstructor{Iso_rectangle_2(const Iso_rectangle_2<R> &u);}
 	    {copy constructor.}

\ccConstructor{Iso_rectangle_2(const Point_2<R> &p, 
	                       const Point_2<R> &q);}
            {introduces an iso-oriented rectangle \ccVar\ with diagonal
             opposite vertices $p$ and $q$. Note that the object is 
             brought in the canonical form.}

\ccConstructor{Iso_rectangle_2(const RT& min_hx, const RT& min_hy, 
                               const RT& max_hx, const RT& max_hy, 
                               const RT& hw = R::RT(1));}
            {introduces an iso-oriented rectangle \ccVar\ with diagonal
             opposite vertices (\ccc{min_hx/hw}, \ccc{min_hy/hw}) and 
             (\ccc{max_hx/hw}, \ccc{max_hy/hw}).  If the last argument 
             is not explicitly given, it defaults to \ccStyle{R::RT(1)}.
             \ccPrecond \ccc{hw} $\neq$ \ccc{R::RT(0)} }

\ccOperations
\ccHidden \ccMethod{Iso_rectangle_2<R> & operator=(const Iso_rectangle_2<R> &q);}
        {Assignment.}

\ccMethod{bool operator==(const Iso_rectangle_2<R> &r2) const;}
       {Test for equality: two iso-oriented rectangles are equal, iff their
        lower left and their upper right vertices are equal.}

\ccMethod{bool operator!=(const Iso_rectangle_2<R> &r2) const;}
       {Test for inequality.}

\ccMethod{Point_2<R> vertex(int i) const;}
       {returns the i'th vertex modulo 4  of \ccVar\ in counterclockwise order, 
        starting with the lower left vertex.}

\ccMethod{Point_2<R> operator[](int i) const;}
       {returns  \ccStyle{vertex(i)}.}

\ccMethod{Point_2<R> min() const;}
       {returns the lower left vertex of \ccVar\ (= \ccStyle{vertex(0)}).}


\ccMethod{Point_2<R> max() const;}
       {returns the upper right vertex of \ccVar\ (= \ccStyle{vertex(2)}).}

\ccMethod{Point_2<R> xmin() const;}
       {returns the $x$ coordinate of lower left vertex of \ccVar.}
\ccGlue
\ccMethod{Point_2<R> ymin() const;}
       {returns the $y$ coordinate of lower left vertex of \ccVar.}
\ccGlue
\ccMethod{Point_2<R> xmax() const;}
       {returns the $x$ coordinate of upper right vertex of \ccVar.}
\ccGlue
\ccMethod{Point_2<R> ymax() const;}
       {returns the $y$ coordinate of upper right vertex of \ccVar.}

\ccMethod{Point_2<R> min_coord(int i) const;}
       {returns the $i$'th \ccHtmlNoLinksFrom{Cartesian} coordinate of the
        lower left vertex of \ccVar. 
%        (\ccc{min_coord(0) == xmin()}; \ccc{min_coord(1) == ymin()})
        \ccPrecond $0 \leq i \leq 1$.}

\ccMethod{Point_2<R> max_coord(int i) const;}
       {returns the $i$'th \ccHtmlNoLinksFrom{Cartesian} coordinate of the
        upper right vertex of \ccVar. 
%        (\ccc{max_coord(0) == xmin()}; \ccc{max_coord(1) == ymin()})
        \ccPrecond $0 \leq i \leq 1$.}

\ccPredicates

\ccMethod{bool is_degenerate() const;}
       {%the iso-oriented rectangle 
        \ccVar\ is degenerate, if all vertices
        are collinear.}

\ccMethod{Bounded_side bounded_side(const Point_2<R> &p) const;}
       {returns either \ccStyle{ON_UNBOUNDED_SIDE},
        \ccStyle{ON_BOUNDED_SIDE}, or the constant
        \ccStyle{ON_BOUNDARY}, 
        depending on where point $p$ is.}

\ccMethod{bool has_on_boundary(const Point_2<R> &p) const;}
       {}
\ccGlue
\ccMethod{bool has_on_bounded_side(const Point_2<R> &p) const;}
       {}
\ccGlue
\ccMethod{bool has_on_unbounded_side(const Point_2<R> &p) const;}
       {}

\ccHeading{Miscellaneous}

\ccMethod{FT area() const;}
       {returns the area of \ccVar. }

\ccMethod{Bbox bbox() const;}
       {returns a bounding box containing \ccVar. }

\ccMethod{Iso_rectangle_2<R>  transform(const Aff_transformation_2<R> &t) const;}
       {returns the iso-oriented rectangle obtained by applying $t$ on 
        the lower left and the upper right corner of \ccVar.
        \ccPrecond The angle at a rotation must be a multiple of $\pi/2$,
        otherwise the resulting rectangle does not have the same side length.
        Note that rotating about an arbitrary angle can even result in
        a degenerate  iso-oriented rectangle.}





\end{ccRefClass} 
