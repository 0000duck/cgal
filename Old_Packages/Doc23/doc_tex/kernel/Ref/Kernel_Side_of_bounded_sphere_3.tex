\begin{ccRefFunctionObjectConcept}{Kernel::Side_of_bounded_sphere_3}
A model for this must provide:

\ccCreationVariable{fo}

\ccMemberFunction{Bounded_side operator()(const Kernel::Point_3&p, 
                                          const Kernel::Point_3&q, 
                                          const Kernel::Point_3&r, 
                                          const Kernel::Point_3&s, 
                                          const Kernel::Point_3&t);}
         {returns the relative position of point \ccStyle{t}
          to the sphere defined by $p$, $q$, $r$, and $s$. The order
          of the points $p$, $q$, $r$, and $s$ does not matter.
          \ccPrecond \ccStyle{p, q, r} and \ccStyle{s} are not coplanar.}

\ccMemberFunction{Bounded_side operator()(const Kernel::Point_3&p, 
                                          const Kernel::Point_3&q, 
                                          const Kernel::Point_3&r, 
                                          const Kernel::Point_3&t);}
   {returns the position of the point \ccStyle{t} relative to the sphere
    passing through $p$, $q$, and $r$ and whose center is in the plane defined
    by these three points.}

\ccMemberFunction{Bounded_side operator()(const Kernel::Point_3&p, 
                                          const Kernel::Point_3&q, 
                                          const Kernel::Point_3&t);}
         {returns the position of the point \ccStyle{t} relative to the sphere
          that has $pq$ as its diameter.}

\end{ccRefFunctionObjectConcept}
