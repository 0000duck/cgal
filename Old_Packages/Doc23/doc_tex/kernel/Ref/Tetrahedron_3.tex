\begin{ccRefClass} {Tetrahedron_3<R>}

\ccDefinition  An object $t$ of the class \ccRefName\ is an oriented
tetrahedron in the three-dimensional Euclidean space $\E^3$. 

It is defined by four vertices $p_0$, $p_1$, $p_2$ and $p_3$.
The orientation of a tetrahedron is the orientation of its four 
vertices. That means it is positive when $p_3$ is on the positive
side of the plane defined by $p_0$, $p_1$ and $p_2$.

The tetrahedron itself splits the space $\E_3$ in a {\em positive} and
a {\em negative} side.
 
The boundary of a tetrahedron splits the space in two open regions, a
bounded one and an unbounded one.

\ccCreation
\ccCreationVariable{t}


\ccHidden \ccConstructor{Tetrahedron_3();}
             {introduces an uninitialized variable \ccVar.}

\ccHidden \ccConstructor{Tetrahedron_3(const Tetrahedron_3<R> &u);}
 	    {copy constructor.}


\ccConstructor{Tetrahedron_3(const Point_3<R> &p0, 
	                     const Point_3<R> &p1, 
	                     const Point_3<R> &p2, 
	                     const Point_3<R> &p3);}
            {introduces a tetrahedron \ccVar\ with vertices $p_0$, $p_1$, $p_2$ and $p_3$.}


\ccOperations

\ccHidden \ccMethod{Tetrahedron_3<R> & operator=(const Tetrahedron_3<R> &t2);}
        {Assignment.}

\ccMethod{bool operator==(const Tetrahedron_3<R> &t2) const;}
%       {Test for equality: two tetrahedra are equal, iff there exists a 
%        cyclic permutation of the vertices of $t2$, such that they are 
%        equal to the vertices of~\ccVar.}
       {Test for equality: two tetrahedra \ccVar\ and \ccc{t2} are equal, 
        iff \ccVar\ and \ccc{t2} have the same orientation and 
        their sets (not sequences) of vertices are equal.}

\ccMethod{bool operator!=(const Tetrahedron_3<R> &t2) const;}
       {Test for inequality.}

\ccMethod{Point_3<R> vertex(int i) const;}
       {returns the i'th vertex modulo 4  of~\ccVar.}

\ccMethod{Point_3<R> operator[](int i) const;}
       {returns \ccStyle{vertex(int i)}.}

\ccPredicates

\ccMethod{bool is_degenerate() const;}
       {Tetrahedron \ccVar\ is degenerate, if the vertices are coplanar.}


\ccMethod{Orientation    orientation() const;}
       { }

\ccMethod{Oriented_side  oriented_side(const Point_3<R> &p) const;}
       {\ccPrecond: \ccVar\ is not degenerate.}

\ccMethod{Bounded_side  bounded_side(const Point_3<R> &p) const;}
       {\ccPrecond: \ccVar\ is not degenerate.}

For convenience we provide the following boolean functions:

\ccMethod{bool has_on_positive_side(const Point_3<R> &p) const;}
       {}
\ccGlue
\ccMethod{bool has_on_negative_side(const Point_3<R> &p) const;}
       {}
\ccGlue
\ccMethod{bool has_on_boundary(const Point_3<R> &p) const;}
       {}
\ccGlue
\ccMethod{bool has_on_bounded_side(const Point_3<R> &p) const;}
       {}
\ccGlue
\ccMethod{bool has_on_unbounded_side(const Point_3<R> &p) const;}
       {}

\ccHeading{Miscellaneous}

\ccMethod{FT volume() const;}
       {returns the signed volume of \ccVar.}

\ccMethod{Bbox_3 bbox() const;}
       {returns a bounding box containing \ccVar.}

\ccMethod{Tetrahedron_3<R>  transform(const Aff_transformation_3<R> &at) const;}
       {returns the tetrahedron obtained by applying $at$ on the three
        vertices of~\ccVar.}

\ccSeeAlso
\ccRefConceptPage{Kernel::Tetrahedron_3} \\

\end{ccRefClass} 
