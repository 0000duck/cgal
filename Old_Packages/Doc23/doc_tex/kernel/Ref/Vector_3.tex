\begin{ccRefClass} {Vector_3<Kernel>}

\ccDefinition

An object of the class \ccRefName\ is a vector in the three-dimensional 
vector space $\R^3$. Geometrically spoken a vector is the difference
of two points $p_2$, $p_1$ and denotes the direction and the distance
from   $p_1$ to $p_2$. 

\cgal\ defines a symbolic constant \ccStyle{NULL_VECTOR}. We 
will explicitly state where you can pass this constant as an argument
instead of a vector initialized with zeros.


\ccCreation
\ccCreationVariable{v}


\ccHidden\ccConstructor{Vector_3();}
             {introduces an uninitialized variable \ccVar.}

\ccHidden \ccConstructor{Vector_3(const Vector_3<Kernel> &w);}
 	    {copy constructor.}

\ccConstructor{Vector_3(const Point_3<Kernel> &a, const Point_3<Kernel> &b);}
 	    {introduces the vector $b-a$.}

\ccConstructor{Vector_3(const Null_vector &NULL_VECTOR);}
 	    {introduces a null vector \ccVar.}

\ccConstructor{Vector_3(const R::RT &hx, const R::RT &hy, const R::FT &hz, const R::RT &hw = R::RT(1));}
            {introduces a vector \ccVar\ initialized to $(hx/hw, hy/hw, hz/hw)$.
             If the third argument is not explicitly given it defaults
             to \ccStyle{R::RT(1)}.}


\ccOperations

\ccHidden \ccMethod{Vector_3<Kernel> & operator=(const Vector_3<Kernel> &w);}
        {Assignment.}

\ccMethod{bool operator==(const Vector_3<Kernel> &w) const;}
       {Test for equality: two vectors are equal, iff their $x$, $y$ 
        and $z$ coordinates are equal. You can compare a vector with the
        \ccStyle{NULL_VECTOR}.}

\ccMethod{bool operator!=(const Vector_3<Kernel> &w) const;}
       {Test for inequality. You can compare a vector with the
        \ccStyle{NULL_VECTOR}.}


There are two sets of coordinate access functions, namely to the
homogeneous and to the \ccHtmlNoLinksFrom{Cartesian} coordinates. They can be used
independently from the chosen representation type \ccStyle{R}.

\ccMethod{R::RT hx() const;}
       {returns the homogeneous $x$ coordinate.}
\ccGlue
\ccMethod{R::RT hy() const;}
       {returns the homogeneous $y$ coordinate.}
\ccGlue
\ccMethod{R::RT hz() const;}
       {returns the homogeneous $z$ coordinate.}
\ccGlue
\ccMethod{R::RT hw() const;}
       {returns the homogenizing  coordinate.}

Here come the \ccHtmlNoLinksFrom{Cartesian} access functions. Note that you do not loose
information with the homogeneous representation, because then the field
type is a quotient.


\ccMethod{R::FT x() const;}
       {returns the \ccStyle{x}-coordinate of \ccVar, that is $hx/hw$.}
\ccGlue
\ccMethod{R::FT y() const;}
       {returns the \ccStyle{y}-coordinate of \ccVar, that is $hy/hw$.}
\ccGlue
\ccMethod{R::FT z() const;}
       {returns the \ccStyle{z} coordinate of \ccVar, that is $hz/hw$.}

The following operations are for convenience and for making the
class \ccRefName\ compatible with code for higher dimensional vectors.
Again they come in a \ccHtmlNoLinksFrom{Cartesian} and homogeneous flavor.

\ccMethod{R::RT homogeneous(int i) const;}
       {returns the i'th homogeneous coordinate of \ccVar, starting with 0.
        \ccPrecond $0\leq i \leq 3$.}

\ccMethod{R::FT cartesian(int i) const;}
       {returns the i'th \ccHtmlNoLinksFrom{Cartesian} coordinate of \ccVar, starting at 0.
        \ccPrecond $0\leq i \leq 2$.}

\ccMethod{R::FT operator[](int i) const;}
       {returns  \ccStyle{cartesian(i)}.
        \ccPrecond $0\leq i \leq 2$.}

\ccMethod{int dimension() const;}
       {returns the dimension (the constant 3).}

\ccMethod{Vector_3<Kernel>  transform(const Aff_transformation_3<Kernel> &t) const;}
       {returns the vector obtained by applying $t$ on \ccVar.}

\ccMethod{Direction_3<Kernel> direction() const;}
       {returns the direction of \ccVar.}

\ccHeading{Operators}

The following operations can be applied on vectors:

\ccMethod{Vector_3<Kernel>        operator+(const Vector_3<Kernel> &w) const;}
       {Addition.}

\ccMethod{Vector_3<Kernel>        operator-(const Vector_3<Kernel> &w) const;}
       {Subtraction.}

\ccMethod{Vector_3<Kernel>        operator-() const;}
       {Negation.}

\ccMethod{R::FT              operator*(const Vector_3<Kernel> &w) const;}
       {returns the scalar product (= inner product) of the two vectors.}


\ccMethod{Vector_3<Kernel> operator*(const R::RT &s) const;}
       {Multiplication with a scalar from the right. Although it would
        be more natural, \cgal\ does not offer a multiplication with a 
        scalar from the left. (This is due to problems of some compilers.)}


\ccMethod{Vector_3<Kernel> operator*(const Quotient<RT> &s) const;}
       {Multiplication with a scalar from the right.}


%\ccFunction{Vector_3<Kernel> operator*(const R::RT &s, 
%				   const Vector_3<Kernel> &w);}
%       {Multiplication with a scalar from the left.}

\ccMethod{Vector_3<Kernel>        operator/(const R::RT &s) const;}
       {Division by a scalar.}


\ccSeeAlso
\ccRefConceptPage{Kernel::Vector_3} \\
\ccRefIdfierPage{CGAL::cross_product} \\

\end{ccRefClass} 
