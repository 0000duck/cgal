\begin{ccRefFunctionObjectConcept}{Kernel::ConstructDirection_3}
A model for this must provide:

\ccCreationVariable{fo}


\ccHidden \ccMemberFunction{Kernel::Direction_3 operator()();}
             {introduces an uninitialized direction .}

\ccHidden \ccMemberFunction{Kernel::Direction_3 operator()(const Kernel::Direction_3 &d);}
            {copy constructor.}

\ccMemberFunction{Kernel::Direction_3 operator()(const Kernel::Vector_3 &v);}
            {introduces a direction  initialised with the 
             direction of vector $v$.}

\ccMemberFunction{Kernel::Direction_3 operator()(const Kernel::Line_3 &l);}
            {introduces the direction of line $l$.}

\ccMemberFunction{Kernel::Direction_3 operator()(const Kernel::Ray_3 &r);}
            {introduces the direction of ray $r$.}

\ccMemberFunction{Kernel::Direction_3 operator()(const Kernel::Segment_3 &s);}
            {introduces the direction of segment $s$.}

\ccHidden\ccMemberFunction{Kernel::Direction_3 operator()(const Kernel::RT &x, const Kernel::RT &y, const Kernel::RT &z);}
            {introduces a direction  initialised with the direction 
             from the origin to the point with Cartesian coordinates $(x, y, z)$.}

\ccSeeAlso

\ccRefIdfierPage{CGAL::Direction_3<R>}  \\

\end{ccRefFunctionObjectConcept}
