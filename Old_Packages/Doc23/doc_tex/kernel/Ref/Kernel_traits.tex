\begin{ccRefClass}{Kernel_traits<T>}
\ccTexHtml{\ccSetThreeColumns{Point_2< Homogeneous<RT> > }{}{\hspace*{8.5cm}}}{}

\ccDefinition

The class \ccClassTemplateName\ provides access to the kernel model to
which the argument type \ccc{T} belongs. (Provided \ccc{T} belongs to
some kernel model.)  The default implementation assumes there is a
local type \ccc{T::Kernel} referring to the kernel model of \ccc{T}.
If this type does not exist, a specialization of \ccRefName\ can be
used to provide the desired information.

This class is, for example, useful in the following context. Assume
you want to write a generic function that accepts two points $p$ and
$q$ as argument and constructs the line segment between $p$ and $q$.
In order to specify the return type of this function, you need to know
what is the segment type corresponding to the Point type representing
$p$ and $q$. Using \ccRefName, this can be done as follows.
\begin{verbatim}
template < class Point >
typename Kernel_traits<Point>::Kernel::Segment
construct_segment(Point p, Point q)
{ ... } 
\end{verbatim}

% \ccInclude{CGAL/Kernel_traits.h}

\ccTypes

\ccTypedef {typedef T::Kernel Kernel;} { If \ccc{T} is a type
  \ccc{K::Point_2} of some kernel model \ccc{K}, then \ccc{Kernel} is
  equal to \ccc{K}. }

\ccTexHtml{\KernelRefLayout}{}
\end{ccRefClass}

