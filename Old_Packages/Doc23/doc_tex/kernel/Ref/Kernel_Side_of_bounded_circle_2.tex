\begin{ccRefFunctionObjectConcept}{Kernel::Side_of_bounded_circle_2}
A model for this must provide:

\ccCreationVariable{fo}

\ccMemberFunction{Bounded_side operator()(const Kernel::Point_2&p, 
                                  const Kernel::Point_2&q, 
                                  const Kernel::Point_2&r, 
                                  const Kernel::Point_2&t);}
         {returns the relative position of point \ccStyle{t}
          to the circle defined by $p$, $q$ and $r$. The order
          of the points $p$, $q$ and $r$ does not matter.
          \ccPrecond \ccStyle{p, q} and \ccStyle{r} are not collinear.}

\ccMemberFunction{Bounded_side operator()(const Kernel::Point_2&p, 
                                  const Kernel::Point_2&q, 
                                  const Kernel::Point_2&t);}
         {returns the position of the point \ccStyle{t} relative to the circle
          that has $pq$ as its diameter.}

\end{ccRefFunctionObjectConcept}
