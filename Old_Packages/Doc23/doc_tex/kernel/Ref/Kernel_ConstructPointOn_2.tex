\begin{ccRefFunctionObjectConcept}{Kernel::ConstructPointOn_2}
A model for this must provide:

\ccCreationVariable{fo}

\ccMemberFunction{Kernel::Point_2 operator()(const Kernel::Line_2& l,
                                             int i);}
       {returns an arbitrary point on \ccc{l}. It holds 
        \ccStyle{point(i) == point(j)}, iff \ccStyle{i==j}.
        Furthermore,  is directed from \ccStyle{point(i)}
        to \ccStyle{point(j)}, for all \ccStyle{i} $<$ \ccStyle{j}.}

\ccMemberFunction{Kernel::Point_2 operator()(const Kernel::Ray_2& r,
                                             int i);}
       {returns a point on \ccc{r}. \ccStyle{point(0)} is the source,
        \ccStyle{point(i)}, with $i>0$, is different from the 
        source. \ccPrecond $i \geq 0$.}

\ccMemberFunction{Kernel::Point_2 operator()(const Kernel::Segment_2& s,
                                             int i);}
       {returns source or target   of \ccc{s}:   \ccStyle{point(0)} returns
        the source of \ccc{s}, \ccStyle{point(1)} returns the target of \ccc{s}.
        The parameter \ccStyle{i} is taken modulo 2, which gives 
        easy access to the other end point. }

\ccSeeAlso
\ccRefIdfierPage{CGAL::Line_2<R>} \\
\ccRefIdfierPage{CGAL::Ray_2<R>} \\
\ccRefIdfierPage{CGAL::Segment_2<R>} \\

\end{ccRefFunctionObjectConcept}
