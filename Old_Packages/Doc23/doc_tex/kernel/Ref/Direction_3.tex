\begin{ccRefClass}{Direction_3<R>}

\ccDefinition
An object of the class \ccRefName\ is a vector in the three-dimensional 
vector space $\R^3$  where we forget about their length. They can be
viewed as unit vectors, although there is no normalization internally,
since this is error prone.  Directions are used whenever the length of
a vector does not matter. 
They also characterize a set of parallel lines that have the same orientation 
or the direction normal to parallel planes that have the same orientation.
For example, you can ask for the direction
orthogonal to an oriented plane, or the direction of an oriented line.


\ccCreation
\ccCreationVariable{d}


\ccHidden \ccConstructor{Direction_3();}
             {introduces an uninitialized direction \ccVar.}

\ccHidden \ccConstructor{Direction_3(const Direction_3<R> &d);}
 	    {copy constructor.}

\ccConstructor{Direction_3(const Vector_3<R> &v);}
	    {introduces a direction \ccVar\ initialised with the 
	     direction of vector $v$.}

\ccConstructor{Direction_3(const R::RT &x, const R::RT &y, const R::RT &z);}
            {introduces a direction \ccVar\ initialised with the direction 
	     from the origin to the point with Cartesian coordinates $(x, y, z)$.}


\ccOperations
%\ccSetTwoOfThreeColumns{5cm}{4cm}
\ccSetThreeColumns{Direction_3<R> & }{}{\hspace*{7.8cm}}

\ccHidden \ccMethod{Direction_3<R> & operator=(const Direction_3<R> &e);}
        {Assignment.}

\ccMethod{R::RT delta(int i) const;}
       {returns values, such that \ccVar \ccc{== Direction_2<R>(delta(0),delta(1),delta(2))}.
        \ccPrecond: $0 \leq i \leq 2$.}

\ccMethod{R::RT dx() const;}
       {returns \ccc{delta(0)}.}
\ccGlue
\ccMethod{R::RT dy() const;}
       {returns \ccc{delta(1)}.}
\ccGlue
\ccMethod{R::RT dz() const;}
       {returns \ccc{delta(2)}.}


\ccMethod{bool operator==(const Direction_3<R> &e) const;}
       {Test for equality.}
\ccGlue
\ccMethod{bool operator!=(const Direction_3<R> &e) const;}
       {Test for inequality.}


\ccMethod{Direction_3<R>  operator-() const;}
       {The direction opposite to \ccVar.}

\ccMethod{Vector_3<R> vector() const;}
       {returns a vector that has the same direction as \ccVar.}

\ccMethod{Direction_3<R>  transform(const Aff_transformation_3<R> &t) const;}
       {returns the direction obtained by applying $t$ on \ccVar.}



\end{ccRefClass} 

