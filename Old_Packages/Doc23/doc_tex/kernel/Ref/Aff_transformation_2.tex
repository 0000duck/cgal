\begin{ccRefClass}{Aff_transformation_2<R>}

\ccInclude{CGAL/Aff_transformation_2.h}

\ccDefinition
The class \ccRefName\ represents two-dimensioanl affine transformations. 
The general form of an affine transformation is based on homogeneous
representation of points. Thereby all transformations can be realized by
matrix multiplication. 

Since the general form is based on the homogeneous representation, a
transformation matrix multiplication by a scalar does not change
the represented transformation. Therefore, any transformation represented
by a matrix with rational entries can be represented by a 
transformation matrix with integer entries as well by multiplying
the matrix with the common denominator of the rational entries. 
Hence it is sufficient to have number type \ccStyle{R::RT} for the entries 
of an affine transformation.

{\cgal} offers several specialized affine transformations. 
Different constructors are provided to create them. 
They are parameterized with a symbolic name to
denote the transformation type, followed by additional parameters.
The symbolic name tags solve ambiguities in the function
overloading and they make the code more readable, i.e.\ what type
of transformation is created.

Since two-dimensional points have three 
homogeneous coordinates we have a $3\times 3$ matrix ($m_{ij}$).
Following C-style, the indices start at zero.

If the homogeneous representations are normalized such that the 
homogenizing coordinate is 1, then the upper left $2\times 2$ matrix realizes
linear transformations and in the matrix form of a translation,  the
translation vector $(v_0,$ $v_1,$ $1)$ appears in the last column of the 
matrix. In the normalized case, entry $hw$ is always 1.
Entries $m_{20}$ and $m_{21}$ are always zero and therefore do not appear in
the constructors.


\ccCreation
\ccCreationVariable{t}

\ccConstructor{Aff_transformation_2(const Identity_transformation& );}
            {introduces an identity transformation.}

\ccConstructor{Aff_transformation_2(const Translation,
                                    const Vector_2<R> &v);}
            {introduces a translation by a vector $v$.}

\ccConstructor{Aff_transformation_2(const Rotation,
                             const Direction_2<R> &d,
                             const R::RT &num,
                             const R::RT &den = RT(1));}
            {approximates the rotation over the angle indicated by direction 
             $d$, such that the differences between the sines and cosines
             of the rotation given by d and the approximating rotation
             are at most $num/den$ each.
             \ccPrecond $num/den>0$. }

\ccConstructor{Aff_transformation_2(const Rotation,
                                       const R::RT &sine_rho, 
                                       const R::RT &cosine_rho, 
                                       const R::RT &hw = RT(1));}
            {introduces a rotation by the angle \ccStyle{rho}.
             \ccPrecond 
             \ccTexHtml{$\mbox{\it sine\_rho}^2 +  \mbox{\it cosine\_rho}^2 == hw^2$}{<MATH><i>sine_rho</i><SUP>2</SUP> + <i>cosine_rho</i><SUP>2</SUP> == 
<i>hw</i><SUP>2</SUP></MATH>}.}

\ccConstructor{Aff_transformation_2(const Scaling,
                                       const R::RT &s,
                                       const R::RT &hw = RT(1));}
            {introduces a scaling by a scale factor $s/hw$.}

\newsavebox{\arrtwo}
\newsavebox{\arrlintwo}
\newsavebox{\transvectwo}

\savebox{\arrtwo}{\small $\left(\begin{array}{ccc}
                 m_{00} & m_{01} & m_{02}\\
                 m_{10} & m_{11} & m_{12}\\
                  0     &  0     & hw
              \end{array}\right)$}

\savebox{\arrlintwo}{\small $\left(\begin{array}{cc}
                 m_{00} & m_{01}\\
                 m_{10} & m_{11}
              \end{array}\right)$}

\savebox{\transvectwo}{\small $\left(\begin{array}{c}
                 m_{02}\\
                 m_{12}
              \end{array}\right)$}

\ccConstructor{Aff_transformation_2(
                const R::RT &m00, const R::RT &m01, const R::RT &m02,
                const R::RT &m10, const R::RT &m11, const R::RT &m12,
                const R::RT &hw = RT(1));}
            {introduces a general affine transformation in the
             \ccTexHtml{$3 \times 3$ matrix form \usebox{\arrtwo}.}%
             {3x3 matrix <IMG ALIGN=CENTER SRC=arrtwo.gif> .}
             The sub-matrix \ccTexHtml{$1\over hw$\usebox{\arrlintwo}}%
             {<MATH><i>hw</i><SUP>-1</SUP></MATH> <IMG ALIGN=CENTER 
             SRC=arrlintwo.gif>} contains the scaling and rotation 
             information, the vector \ccTexHtml{$1\over hw$
             \usebox{\transvectwo}}{<MATH><i>hw</i><SUP>-1</SUP></MATH>
             <IMG ALIGN=CENTER SRC=transvectwo.gif>}
             contains the translational part of the transformation.}

\savebox{\arrtwo}{\small $\left(\begin{array}{ccc}
                 m_{00} & m_{01} & 0\\
                 m_{10} & m_{11} & 0\\
                  0     &  0     & hw
              \end{array}\right)$}
                  
\ccConstructor{Aff_transformation_2(
                                       const R::RT &m00, const R::RT &m01,
                                       const R::RT &m10, const R::RT &m11,
                                       const R::RT &hw = RT(1));}
            {introduces a general linear transformation 
             \ccTexHtml{\usebox{\arrtwo},}{<IMG ALIGN=CENTER SRC=arrtwo2.gif> ,}
             i.e.\ there is no translational part.}


\ccOperations

The main thing to do with transformations is to apply them on
geometric objects. Each class \ccStyle{Class_2<R>} representing
a geometric object has a member function:

\ccStyle{Class_2<R>  transform(Aff_transformation_2<R> t)}.


The transformation classes provide a member function \ccStyle{transform()}
for points, vectors, directions, and lines:

\ccMethod{Point_2<R>  transform(const Point_2<R> &p) const;}
       {}
\ccGlue
\ccMethod{Vector_2<R>  transform(const Vector_2<R> &p) const;}
       {}
\ccGlue
\ccMethod{Direction_2<R>  transform(const Direction_2<R> &p) const;}
       {}
\ccGlue
\ccMethod{Line_2<R>  transform(const Line_2<R> &p) const;}
       {}

\cgal\ provides function operators for these member functions:

\ccMethod{Point_2<R>  operator()(const Point_2<R> &p) const;}
       {}
\ccGlue
\ccMethod{Vector_2<R>  operator()(const Vector_2<R> &p) const;}
       {}
\ccGlue
\ccMethod{Direction_2<R>  operator()(const Direction_2<R> &p) const;}
       {}
\ccGlue
\ccMethod{Line_2<R>  operator()(const Line_2<R> &p) const;}
       {}

\ccHeading{Miscellaneous}

\ccMethod{Aff_transformation_2<R> operator*(const Aff_transformation_2<R> &s) const;}{composes two affine transformations.}

\ccMethod{Aff_transformation_2<R>  inverse() const;}
       {gives the inverse transformation.}

%\ccMethod{Aff_transformation_2<R>  transpose() const;}
%       {returns the affine transformation defined by transposing
%       the linear transformation in \ccVar\ and setting the
%       translational part to zero.}

\ccMethod{bool                 is_even() const;}
       {returns \ccStyle{true}, if the transformation is not reflecting,
        i.e.\ the determinant of the involved linear transformation is
        non-negative.}

\ccMethod{bool                 is_odd() const;}
       {returns \ccStyle{true}, if the transformation is reflecting.}


%\ccMethod{Aff_transformation_2<R>  general_form() const;}
%       {returns the affine transformation in matrix form.}

The matrix entries of a matrix representation of a 
\ccStyle{Aff_transformation_2<R>}
can be accessed trough the following member functions:

\ccMethod{FT          cartesian(int i, int j) const;}
                      {}
\ccGlue
\ccMethod{FT          m(int i, int j) const;}
       {returns entry $m_{ij}$ in a matrix representation in which $m_{22}$ is 1.}

\ccMethod{RT          homogeneous(int i, int j) const;}
                      {}
\ccGlue
\ccMethod{RT          hm(int i, int j) const;}
       {returns entry $m_{ij}$ in some fixed matrix representation.} 


For affine transformations  no I/O operators are defined.

%\ccImplementation
%Depending on the constructor we have different internal representations.
%This approach uses less memory and the transformation can be applied
%faster.
%
%Affine transformations offer no \ccStyle{transform()} member function
%for complex objects because they are defined in terms of  points vectors and 
%directions.  As the internal representation of a complex object
%is private the transformation code should go there.

\ccSeeAlso
\ccc{Identity_transformation}, 
\ccc{Rotation}, 
\ccc{Scaling},
\ccc{Translation} \\
\ccc{rational_rotation_approximation}

\ccExample

\begin{cprog}
  typedef Cartesian<double> RepClass;
  typedef Aff_transformation_2<RepClass> Transformation;
  typedef Point_2<RepClass> Point;
  typedef Vector_2<RepClass> Vector;
  typedef Direction_2<RepClass> Direction;

  Transformation rotate(ROTATION, sin(pi), cos(pi));
  Transformation rational_rotate(ROTATION,Direction(1,1), 1, 100);
  Transformation translate(TRANSLATION, Vector(-2, 0));
  Transformation scale(SCALING, 3);

  Point q(0, 1);
  q = rational_rotate(q); 

  Point p(1, 1);

  p = rotate(p); 

  p = translate(p); 

  p = scale(p);
\end{cprog} 

The same would have been achieved with

\begin{cprog}

  Transformation transform = scale * (translate * rotate);
  p = transform(Point(1.0, 1.0));
\end{cprog} 
\end{ccRefClass} 

