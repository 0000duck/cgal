\begin{ccRefFunctionObjectConcept}{Kernel::BoundedSide_2}
A model for this must provide:

\ccCreationVariable{fo}

\ccMemberFunction{Bounded_side operator()(const Kernel::Circle_2&c, 
                                           const Kernel::Point_2&p);} 
{returns either \ccc{ON_UNBOUNDED_SIDE},
 \ccc{ON_BOUNDED_SIDE}, or the constant
 \ccc{ON_BOUNDARY}, depending on where point \ccc{p} is relative to
circle \ccc{c}.}

\ccMemberFunction{Bounded_side operator()(const Kernel::Triangle_2& t, 
                                           const Kernel::Point_2&p);} 
{returns either \ccc{ON_UNBOUNDED_SIDE},
 \ccc{ON_BOUNDED_SIDE}, or the constant
 \ccc{ON_BOUNDARY}, depending on where point \ccc{p} is relative to
triangle \ccc{t}.}

\ccMemberFunction{Bounded_side operator()(const Kernel::Iso_rectangle_2& r, 
                                          const Kernel::Point_2&p);} 
{returns either \ccc{ON_UNBOUNDED_SIDE},
 \ccc{ON_BOUNDED_SIDE}, or the constant
 \ccc{ON_BOUNDARY}, depending on where point \ccc{p} is relative to
rectangle \ccc{r}.}

\ccRefines
AdaptableFunctor (with two arguments)

\ccSeeAlso
\ccRefIdfierPage{CGAL::Circle_2<Kernel>} \\
\ccRefIdfierPage{CGAL::Triangle_2<Kernel>} \\
\ccRefIdfierPage{CGAL::Iso_rectangle_2<Kernel>}  \\


\end{ccRefFunctionObjectConcept}
