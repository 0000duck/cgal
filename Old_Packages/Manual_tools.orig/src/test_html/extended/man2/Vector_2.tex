% $Id$

\begin{ccClassTemplate} {CGAL_Vector_2<R>}
\ccSection{2D Vector A}

\ccDefinition

An object of the class \ccClassName\ is a vector in the two-dimensional 
vector space $\R^2$. Geometrically spoken a vector is the difference
of two points $p_2$, $p_1$ and denotes the direction and the distance
from   $p_1$ to $p_2$. 

\cgal\ defines a symbolic constant \ccStyle{CGAL_NULL_VECTOR}. We 
will explicitely state where you can pass this constant as an argument
instead of a vector initialized with zeros.


\ccCreation
\ccCreationVariable{v}

\ccInclude{CGAL/Vector_2.h}

\ccHidden\ccConstructor{CGAL_Vector_2();}
             {introduces an uninitialized variable \ccVar.}

\ccHidden \ccConstructor{CGAL_Vector_2(const CGAL_Vector_2<R> &w);}
            {copy constructor.}

\ccConstructor{CGAL_Vector_2(const R::RT &hx, const R::RT &hy, const R::RT &hw = R::RT(1));}
            {introduces a vector \ccVar\ initialized to $(hx/hw,hy/hw)$.
             If the third argument is not explicitely given it defaults
             to \ccStyle{R::RT(1)}.}


\ccOperations
\ccSetTwoOfThreeColumns{5cm}{4cm}

\ccHidden \ccMethod{CGAL_Vector_2<R> & operator=(const CGAL_Vector_2<R> &w);}
        {Assignment.}

\ccMethod{bool operator==(const CGAL_Vector_2<R> &w) const;}
       {Test for equality: two vectors are equal, iff their $x$ and $y$ 
        coodinates are equal. You can compare with the
        \ccStyle{CGAL_NULL_VECTOR}.}

\ccMethod{bool operator!=(const CGAL_Vector_2<R> &w) const;}
       {Test for inequality. You can compare with the
        \ccStyle{CGAL_NULL_VECTOR}.}


There are two sets of coordinate access functions, namely to the
homogeneous and to the Cartesian coordinates. They can be used
independently from the chosen representation type \ccStyle{R}.

\ccMethod{R::RT hx() const;}
       {returns the homogeneous $x$ coordinate.}

\ccMethod{R::RT hy() const;}
       {returns the homogeneous $y$ coordinate.}

\ccMethod{R::RT hw() const;}
       {returns the homogenizing  coordinate.}

Here come the Cartesian access functions.  Note that you do not loose
information with the homogeneous representation, because then the field
type is a quotient.


\ccMethod{R::FT x() const;}
       {returns the \ccStyle{x}-coordinate of \ccVar, that is $hx/hw$.}

\ccMethod{R::FT y() const;}
       {returns the \ccStyle{y}-coordinate of \ccVar, that is $hy/hw$.}

The following operations are for convenience and for making the
class \ccClassName\ compatible with code for higher dimensional vectors.
Again they come in a Cartesian and homogeneous flavor.

\ccMethod{R::RT homogeneous(int i) const;}
       {returns the i'th homogeneous coordinate of \ccVar, starting with 0.
        \ccPrecond $0\leq i \leq 2$.}

\ccMethod{R::FT cartesian(int i) const;}
       {returns the i'th Cartesian coordinate of \ccVar, starting at 0.
        \ccPrecond $0\leq i \leq 1$.}

\ccMethod{R::FT operator[](int i) const;}
       {returns  \ccStyle{coordinate(i)}.
        \ccPrecond $0\leq i \leq 1$.}

\ccMethod{int dimension() const;}
       {returns the dimension (the constant 2).}

\ccMethod{CGAL_Vector_2<R>  transform(const CGAL_Aff_transformation_2<R> &t) const;}
       {returns the vector obtained by applying $t$ on \ccVar.}

\ccMethod{CGAL_Vector_2<R>  perpendicular(const CGAL_Orientation &o) const;}
       {returns the vector perpendicular to \ccVar\ in clockwise or
        counterclockwise orientation.}

The following operations can be applied on vectors:

\ccMethod{CGAL_Vector_2<R>        operator+(const CGAL_Vector_2<R> &w) const;}
       {Addition.}

\ccMethod{CGAL_Vector_2<R>        operator-(const CGAL_Vector_2<R> &w) const;}
       {Subtraction.}

\ccMethod{CGAL_Vector_2<R>        operator-() const;}
       {returns the opposite vector.}

\ccMethod{R::FT                  operator*(const CGAL_Vector_2<R> &w) const;}
       {returns the scalar product (= inner product) of the two vectors.}


\ccMethod{CGAL_Vector_2<R> operator*(const R::RT &s) const;}
       {Multiplication with a scalar from the right. Although it would
        be more natural, \cgal does not offer a multiplication with a 
        scalar from the left.\ccTexHtml{\footnote{This is due to compiler problems
        which will hopefully be overcome sooner or later.}}{This is due to compiler problems
        which will hopefully be overcome sooner or later.}}


\ccMethod{CGAL_Vector_2<R> operator*(const R::FT &s) const;}
       {Multiplication with a scalar from the right.}


%\ccFunction{CGAL_Vector_2<R> operator*(const R::RT &s, 
%                                      const CGAL_Vector_2<R> &w);}
%       {Multiplication with a scalar from the left.}

\ccMethod{CGAL_Vector_2<R>        operator/(const R::RT &s) const;}
       {Division by a scalar.}

\ccMethod{CGAL_Direction_2<R> direction() const;}
       {returns the direction which passes through \ccVar.}



\end{ccClassTemplate} 

% $Log$
% Revision 1.1  2001/08/27 18:00:36  spion
% Initial revision
%
% Revision 1.2  1996/07/08 12:19:15  fabri
% *** empty log message ***
%
% Revision 1.1  1995/10/19 18:22:12  fabri
% Initial revision
%
