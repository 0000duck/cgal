% $Id$

\begin{ccClassTemplate} {CGAL_Direction_2<R>}
\ccSection{2D Direction A}

\ccDefinition
An object of the class \ccClassName\ is a vector in the two-dimensional 
vector space $\R^2$  where we forget about its length. They can be
viewed as unit vectors, although there is no normalization internally,
since this is error prone.  Directions are used whenever the length of
a vector does not matter, e.g.\ you can ask for the direction
orthogonal to an oriented plane, the direction of an oriented line,
and so on. They further can be used like angles. The slope of a direction
is \ccStyle{dy()/dx()}.

\ccCreation
\ccCreationVariable{d}

\ccInclude{CGAL/Direction_2.h}

\ccHidden \ccConstructor{CGAL_Direction_2();}
             {introduces an uninitialized direction \ccVar.}

\ccHidden \ccConstructor{CGAL_Direction_2(const CGAL_Direction_2<R> &d);}
            {copy constructor.}

\ccConstructor{CGAL_Direction_2(const CGAL_Vector_2<R> &v);}
            {introduces the direction \ccVar\ of vector $v$.}

\ccConstructor{CGAL_Direction_2(const R::RT &x, const R::RT &y);}
            {introduces a direction \ccVar\ passing through the point 
             at $(x, y)$.}


\ccOperations
\ccSetTwoOfThreeColumns{5cm}{4cm}

\ccHidden \ccMethod{CGAL_Direction_2<R> & operator=(const CGAL_Direction_2<R> &e);}
        {Assignment.}

\ccMethod{R::RT delta(int i) const;}
       {returns the i'th value of the slope of \ccVar.
        \ccPrecond: $0 \leq i \leq 1$.}

\ccMethod{R::RT dx() const;}
       {returns the $dx$ value of the slope of \ccVar.}

\ccMethod{R::RT dy() const;}
       {returns the $dy$ value of the slope of \ccVar.}

There is a total order on directions. We compare the angles between the
positive $x$-axis and the directions in counterclockwise direction.

\ccMethod{bool operator==(const CGAL_Direction_2<R> &e) const;}
       {Test for equality.}

\ccMethod{bool operator!=(const CGAL_Direction_2<R> &e) const;}
       {Test for inequality.}

\ccMethod{bool operator<(const CGAL_Direction_2<R> &e) const;}
       {}

\ccMethod{bool operator>(const CGAL_Direction_2<R> &e) const;}
       {}

\ccMethod{bool operator<=(const CGAL_Direction_2<R> &e) const;}
       {}

\ccMethod{bool operator>=(const CGAL_Direction_2<R> &e) const;}
       {}

\ccMethod{bool counterclockwise_in_between(const CGAL_Direction_2<R> &d1,
                                   const CGAL_Direction_2<R> &d2) const;}
       {}


\ccMethod{CGAL_Direction_2<R>  operator-() const;}
       {The direction opposite to \ccVar.}

\ccMethod{CGAL_Vector_2<R> vector() const;}
       {returns a vector that has the same direction as \ccVar.}

\ccMethod{CGAL_Direction_2<R>  transform(const CGAL_Aff_transformation_2<R> &t) const;}
       {returns the direction obtained by applying $t$ on \ccVar.}


\end{ccClassTemplate} 

% $Log$
% Revision 1.1  2001/08/27 18:00:36  spion
% Initial revision
%
% Revision 1.3  1996/07/08 12:19:15  fabri
% *** empty log message ***
%
% Revision 1.2  1996/03/13 15:42:07  fabri
% *** empty log message ***
%
% Revision 1.1  1995/10/19 18:22:12  fabri
% Initial revision
%
