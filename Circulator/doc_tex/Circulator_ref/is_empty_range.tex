% +------------------------------------------------------------------------+
% | CGAL Reference Manual: is_empty_range.tex
% +------------------------------------------------------------------------+
% | Requirements for circulators in analogy to STL iterators.
% | Adaptors between circulators and iterators.
% | Proposal for CGAL.
% |
% | 11.10.1996   Lutz Kettner
% | 
% | \RCSdef{\circRev}{$Id$}
% | \RCSdefDate{\circDate}{$Date$}
% +------------------------------------------------------------------------+

% +-----------------------------------------------------+
\begin{ccRefFunction}{is_empty_range}

\ccDefinition

In order to write algorithms that work with iterator ranges as well as
with circulator ranges we have to consider the difference of
representing an empty range. For iterators this is the range $[i,i)$,
while for circulators it would be \ccc{c == NULL},  the empty sequence test.
The function \ccc{is_empty_range} provides the necessary generic test
which accepts an iterator range or a circulator range and says whether
the range is empty or not.

\ccInclude{CGAL/circulator.h}

\ccSetThreeColumns{bool}{is_empty_range( IC i, IC j);}{}
\ccFunction{template< class IC> 
    bool is_empty_range( const IC& i, const IC& j);}{%
    is \ccc{true} if the range [\ccc{i, j}) is empty,
    \ccc{false} otherwise.
    \ccPrecond{\ccc{IC} is either a circulator or an iterator
    type. The range [\ccc{i, j}) is valid.}
}

\ccExample

The following function \ccc{process_all} accepts a range $\left[i,
  j\right)$ of an iterator or circulator \ccc{IC} and processes each
element in this range:

\begin{verbatim}
template <class IC>
void process_all( IC i, IC j) {
    if (! CGAL::is_empty_range( i, j)) { 
        do {
            process(*i);
        } while (++i != j);
    }
}
\end{verbatim}


\ccSeeAlso

\ccc{iterator_distance},
\ccc{CGAL_For_all},
\ccc{Circulator_tag},
\ccc{Circulator_traits},\\
\ccc{Assert_circulator_or_iterator},
\ccc{Circulator}.


\end{ccRefFunction}

% +-----------------------------------------------------+
% EOF
