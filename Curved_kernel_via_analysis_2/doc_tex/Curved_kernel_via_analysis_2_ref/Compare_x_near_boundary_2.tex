% +------------------------------------------------------------------------+
% | Reference manual page: Compare_x_near_boundary_2.tex
% +------------------------------------------------------------------------+
% | 27.03.2008   Author
% | Package: Curved_kernel_via_analysis_2
% |
\RCSdef{\RCSComparexnearboundaryRev}{$Id: header.tex 40270 2007-09-07 15:29:10Z lsaboret $}
\RCSdefDate{\RCSComparexnearboundaryDate}{$Date: 2007-09-07 17:29:10 +0200 (Fri, 07 Sep 2007) $}
% |
\ccRefPageBegin
%%RefPage: end of header, begin of main body
% +------------------------------------------------------------------------+


\begin{ccRefFunctionObjectClass}[CGALi::Curved_kernel_via_analysis_2_Functors::]{Compare_x_near_boundary_2}  %% add template arg's if necessary
\ccRefLabel{Compare_x_near_boundary_2}

%% \ccHtmlCrossLink{}     %% add further rules for cross referencing links
%% \ccHtmlIndexC[class]{} %% add further index entries

\ccDefinition

%The function object class \ccRefName\ does this and that.

% The section below is automatically generated. Do not edit!
%START-AUTO(\ccInclude)

\ccInclude{CGAL/Curved_kernel_via_analysis_2/Curved_kernel_via_analysis_2_functors.h}

%END-AUTO(\ccInclude)

% The section below is automatically generated. Do not edit!
%START-AUTO(\ccDefinition)

Functor that compares x-coordinates near the top or bottom boundary.

%END-AUTO(\ccDefinition)


\ccParameters

% The section below is automatically generated. Do not edit!
%START-AUTO(\ccParameters)

template$<$  \\
class \ccc{CurvedKernelViaAnalysis_2}$>$   \\
class \ccc{Compare_x_near_boundary_2};

%END-AUTO(\ccParameters)


\ccInheritsFrom

% The section below is automatically generated. Do not edit!
%START-AUTO(\ccInheritsFrom)

\ccc{Curved_kernel_via_analysis_2_functor_base<CurvedKernelViaAnalysis_2>}

%END-AUTO(\ccInheritsFrom)


\ccIsModel

\ccTypes

%\ccCreation
%\ccCreationVariable{a}  %% choose variable name

%% \ccIncludeExampleCode{Curved_kernel_via_analysis_2/Compare_x_near_boundary_2.C}

% The section below is automatically generated. Do not edit!
%START-AUTO(\ccTypes)

\ccNestedType{Curved_kernel_via_analysis_2}
{
this instance' first template parameter
}
\ccGlue
\ccNestedType{Base}
{
the base type
}
\ccGlue
\ccNestedType{Curve_2}
{
the curve type
}
\ccGlue
\ccNestedType{Point_2}
{
the point type
}
\ccGlue
\ccNestedType{Arc_2}
{
the arc type
}
\ccGlue
\ccNestedType{Curve_analysis_2}
{
type of curve analaysis
}
\ccGlue
\ccNestedType{X_coordinate_1}
{
the x-coordinate type
}
\ccGlue
\ccNestedType{result_type}
{
the result type
}
\ccGlue
\ccNestedType{Arity}
{
the arity of the functor
}
\ccGlue

%END-AUTO(\ccTypes)

\ccHeading{Variables}

% The section below is automatically generated. Do not edit!
%START-AUTO(\ccHeading{Variables})
%END-AUTO(\ccHeading{Variables})

\ccCreation
\ccCreationVariable{a}  %% choose variable name for \ccMethod below

% The section below is automatically generated. Do not edit!
%START-AUTO(\ccCreation)

\ccConstructor{Compare_x_near_boundary_2(Curved_kernel_via_analysis_2 * kernel);}
{
}
\ccGlue

%END-AUTO(\ccCreation)

\ccOperations

% The section below is automatically generated. Do not edit!
%START-AUTO(\ccOperations)

\ccMethod{result_type operator()(const Point_2& p, const Arc_2& cv, Arr_curve_end ce) const;}
{
Compare the x-coordinate of a point with the x-coordinate of an arcend near the boundary at bottom or top boundary.
}
\ccGlue
\begin{description}
\item[Parameters:]
\begin{description}
\item[p]the point direction. \item[cv]the arc, the endpoint of which is compared. \item[ce]the arc-end indicator - \ccc{ARR_MIN_END} - the minimal end of cv or \ccc{ARR_MAX_END} - the maximal end of cv. \end{description}
\end{description}
\begin{description}
\item[Returns:]the comparison result: SMALLER - x(p) $<$ x(cv, ce); EQUAL - x(p) = x(cv, ce); LARGER - x(p) $>$ x(cv, ce).\end{description}
\begin{description}
\item[Precondition:]p lies in the interior of the parameter space.
the ce end of the line cv lies on a boundary. \end{description}
\ccGlue
\ccMethod{result_type operator()(const Arc_2& cv1, Arr_curve_end ce1, const Arc_2& cv2, Arr_curve_end ce2) const;}
{
Compare the x-coordinates of 2 arcs ends near the top or bottom boundary of the parameter space
}
\ccGlue
\begin{description}
\item[Parameters:]
\begin{description}
\item[cv1]the first arc. \item[ce1]the first arc end indicator - \ccc{ARR_MIN_END} - the minimal end of cv1 or \ccc{ARR_MAX_END} - the maximal end of cv1. \item[cv2]the second arc. \item[ce2]the second arc end indicator - \ccc{ARR_MIN_END} - the minimal end of cv2 or \ccc{ARR_MAX_END} - the maximal end of cv2. \end{description}
\end{description}
\begin{description}
\item[Returns:]the second comparison result: SMALLER - x(cv1, ce1) $<$ x(cv2, ce2); EQUAL - x(cv1, ce1) = x(cv2, ce2); LARGER - x(cv1, ce1) $>$ x(cv2, ce2).\end{description}
\begin{description}
\item[Precondition:]the ce1 end of the arc cv1 lies on a boundary.
the ce2 end of the arc cv2 lies on a boundary. \end{description}
\ccGlue

%END-AUTO(\ccOperations)

\end{ccRefFunctionObjectClass}

% +------------------------------------------------------------------------+
%%RefPage: end of main body, begin of footer
\ccRefPageEnd
% EOF
% +------------------------------------------------------------------------+

