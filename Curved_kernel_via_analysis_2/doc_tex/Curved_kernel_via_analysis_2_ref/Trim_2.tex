% +------------------------------------------------------------------------+
% | Reference manual page: Trim_2.tex
% +------------------------------------------------------------------------+
% | 27.03.2008   Author
% | Package: Curved_kernel_via_analysis_2
% |
\RCSdef{\RCSTrimRev}{$Id: header.tex 40270 2007-09-07 15:29:10Z lsaboret $}
\RCSdefDate{\RCSTrimDate}{$Date: 2007-09-07 17:29:10 +0200 (Fri, 07 Sep 2007) $}
% |
\ccRefPageBegin
%%RefPage: end of header, begin of main body
% +------------------------------------------------------------------------+


\begin{ccRefFunctionObjectClass}[CGALi::Curved_kernel_via_analysis_2_Functors::]{Trim_2}  %% add template arg's if necessary
\ccRefLabel{Trim_2}

%% \ccHtmlCrossLink{}     %% add further rules for cross referencing links
%% \ccHtmlIndexC[class]{} %% add further index entries

\ccDefinition
  
%The function object class \ccRefName\ does this and that.

\ccInclude{Trim_2.h}

% The section below is automatically generated. Do not edit!
%START-AUTO(\ccDefinition)

Functors that trims an arc.

%END-AUTO(\ccDefinition)


\ccParameters

% The section below is automatically generated. Do not edit!
%START-AUTO(\ccParameters)

template$<$  \\
class \ccc{CurvedKernelViaAnalysis_2}$>$   \\
class \ccc{Trim_2};

%END-AUTO(\ccParameters)


\ccIsModel

\ccTypes

%\ccCreation
%\ccCreationVariable{a}  %% choose variable name

%% \ccIncludeExampleCode{Curved_kernel_via_analysis_2/Trim_2.C}

% The section below is automatically generated. Do not edit!
%START-AUTO(\ccTypes)

\ccNestedType{Curved_kernel_via_analysis_2}
{
this instance' first template parameter
}
\ccGlue
\ccNestedType{Base}
{
the base type
}
\ccGlue
\ccNestedType{Curve_2}
{
the curve type
}
\ccGlue
\ccNestedType{Point_2}
{
the point type
}
\ccGlue
\ccNestedType{Arc_2}
{
the arc type
}
\ccGlue
\ccNestedType{Curve_analysis_2}
{
type of curve analaysis
}
\ccGlue
\ccNestedType{X_coordinate_1}
{
the x-coordinate type
}
\ccGlue
\ccNestedType{result_type}
{
the result type
}
\ccGlue
\ccNestedType{Arity}
{
the arity of the functor
}
\ccGlue
\ccNestedType{Curve_kernel_2}
{
[inherited] \\
type of curve kernel
}
\ccGlue

%END-AUTO(\ccTypes)

\ccHeading{Variables}

% The section below is automatically generated. Do not edit!
%START-AUTO(\ccHeading{Variables})

\ccVariable{Curved_kernel_via_analysis_2* _m_curved_kernel;}
{
[protected, inherited] \\
stores pointer to \ccc{Curved_kernel_via_analysis_2}
}
\ccGlue

%END-AUTO(\ccHeading{Variables})

\ccCreation
\ccCreationVariable{a}  %% choose variable name for \ccMethod below

% The section below is automatically generated. Do not edit!
%START-AUTO(\ccCreation)

\ccConstructor{Trim_2(Curved_kernel_via_analysis_2 * kernel);}
{
Standard constructor.
}
\ccGlue
\begin{description}
\item[Parameters:]
\begin{description}
\item[kernel]The kernel \end{description}
\end{description}
\ccGlue

%END-AUTO(\ccCreation)

\ccOperations

% The section below is automatically generated. Do not edit!
%START-AUTO(\ccOperations)

\ccMethod{Arc_2 operator()(const Arc_2& cv, const Point_2& p, const Point_2& q);}
{
Returns a trimmed version of an arc.
}
\ccGlue
\begin{description}
\item[Parameters:]
\begin{description}
\item[cv]The arc \item[p]the new first endpoint \item[q]the new second endpoint \end{description}
\end{description}
\begin{description}
\item[Returns:]The trimmed arc\end{description}
\begin{description}
\item[Precondition:]p != q
both points must be interior and must lie on cv \end{description}
\ccGlue
\ccMethod{Curved_kernel_via_analysis_2* _ckva() const;}
{
[protected, inherited] \\
Return pointer to curved kernel.
}
\ccGlue
\begin{description}
\item[Returns:]Pointer to stored kernel \end{description}
\ccGlue

%END-AUTO(\ccOperations)

\end{ccRefFunctionObjectClass}

% +------------------------------------------------------------------------+
%%RefPage: end of main body, begin of footer
\ccRefPageEnd
% EOF
% +------------------------------------------------------------------------+

