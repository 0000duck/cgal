% +------------------------------------------------------------------------+
% | Reference manual page: Curve_interval_arcno_cache.tex
% +------------------------------------------------------------------------+
% | 26.03.2008   Author
% | Package: Curved_kernel_via_analysis_2
% |
\RCSdef{\RCSCurveintervalarcnocacheRev}{$Id: header.tex 40270 2007-09-07 15:29:10Z lsaboret $}
\RCSdefDate{\RCSCurveintervalarcnocacheDate}{$Date: 2007-09-07 17:29:10 +0200 (Fri, 07 Sep 2007) $}
% |
\ccRefPageBegin
%%RefPage: end of header, begin of main body
% +------------------------------------------------------------------------+


\begin{ccRefFunctionObjectClass}[CGALi::]{Curve_interval_arcno_cache}  %% add template arg's if necessary

%% \ccHtmlCrossLink{}     %% add further rules for cross referencing links
%% \ccHtmlIndexC[class]{} %% add further index entries

\ccDefinition
  
%The function object class \ccRefName\ does this and that.

\ccInclude{Curve_interval_arcno_cache.h}

% The section below is automatically generated. Do not edit!
%START-AUTO(\ccDefinition)

Given an arc number over an interval i this functor computes a corresponding event arcno (and asymptotic tendency) over a certain vertical line which lies on the given side w.r.t. the interval i.

For caching issues for each accessed \ccc{Curve_2} object we store \ccc{Interval_arcno_map} structure. In its turn, \ccc{Interval_arcno_map} for each \ccc{Status_line_1} object stores the precomputed mapping from interval arcnos (on left and right sides) to event arcnos

%END-AUTO(\ccDefinition)

\ccParameters

% The section below is automatically generated. Do not edit!
%START-AUTO(\ccParameters)

template$<$  \\
class \ccc{CurvedKernelViaAnalysis_2}$>$   \\
struct \ccc{Curve_interval_arcno_cache};

%END-AUTO(\ccParameters)

\ccIsModel

\ccTypes

%\ccCreation
%\ccCreationVariable{a}  %% choose variable name

%% \ccIncludeExampleCode{Curved_kernel_via_analysis_2/Curve_interval_arcno_cache.C}

% The section below is automatically generated. Do not edit!
%START-AUTO(\ccTypes)

\ccNestedType{Curved_kernel_via_analysis_2}
{
this instance's first template parameter
}
\ccGlue
\ccNestedType{Curve_kernel_2}
{
type of curve kernel
}
\ccGlue
\ccNestedType{Curve_analysis_2}
{
type of curve analysis
}
\ccGlue
\ccNestedType{X_coordinate_1}
{
type of x-coordinate
}
\ccGlue
\ccNestedType{Status_line_1}
{
type of status line
}
\ccGlue
\ccNestedType{Arcno_desc}
{
event arc number descriptor: stores an arc number along with curve's end type (+/-oo or \ccc{ARR_INTERIOR} )
}
\ccGlue
\ccNestedType{first_argument_type}
{
type of first argument
}
\ccGlue
\ccNestedType{second_argument_type}
{
type of second argument
}
\ccGlue
\ccNestedType{third_argument_type}
{
type of third arguent
}
\ccGlue
\ccNestedType{result_type}
{
the result type
}
\ccGlue
\ccNestedType{Arity}
{
the arity of the functor
}
\ccGlue

%END-AUTO(\ccTypes)

\ccCreation
\ccCreationVariable{a}  %% choose variable name for \ccMethod below

% The section below is automatically generated. Do not edit!
%START-AUTO(\ccCreation)

\ccConstructor{Curve_interval_arcno_cache(Curved_kernel_via_analysis_2 * kernel);}
{
Standard constructor.
}
\ccGlue
\begin{description}
\item[Parameters:]
\begin{description}
\item[kernel]The kernel that is used internally \end{description}
\end{description}
\ccGlue

%END-AUTO(\ccCreation)

\ccOperations

% The section below is automatically generated. Do not edit!
%START-AUTO(\ccOperations)

\ccMethod{result_type operator()(const Status_line_1& cv_line, bool side, int interval_arcno) const;}
{
Given arcno over an interval, this computes a corresponding event arcno on status line \ccc{cv_line} lying on the given side w.r.t. the interval.
}
\ccGlue
\begin{description}
\item[Parameters:]
\begin{description}
\item[\ccc{cv_line}]Status line \item[side]side = true: left side; side = false: right side \item[\ccc{interval_arcno}]Arcnumber of interval \end{description}
\end{description}
\begin{description}
\item[Returns:]Arcnumber information at event line \end{description}
\ccGlue

%END-AUTO(\ccOperations)

\end{ccRefFunctionObjectClass}

% +------------------------------------------------------------------------+
%%RefPage: end of main body, begin of footer
\ccRefPageEnd
% EOF
% +------------------------------------------------------------------------+

