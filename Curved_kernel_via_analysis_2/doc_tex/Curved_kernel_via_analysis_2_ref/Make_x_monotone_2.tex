% +------------------------------------------------------------------------+
% | Reference manual page: Make_x_monotone_2.tex
% +------------------------------------------------------------------------+
% | 26.03.2008   Author
% | Package: Curved_kernel_via_analysis_2
% |
\RCSdef{\RCSMakexmonotoneRev}{$Id: header.tex 40270 2007-09-07 15:29:10Z lsaboret $}
\RCSdefDate{\RCSMakexmonotoneDate}{$Date: 2007-09-07 17:29:10 +0200 (Fri, 07 Sep 2007) $}
% |
\ccRefPageBegin
%%RefPage: end of header, begin of main body
% +------------------------------------------------------------------------+


\begin{ccRefFunctionObjectClass}[CGALi::]{Make_x_monotone_2}  %% add template arg's if necessary

%% \ccHtmlCrossLink{}     %% add further rules for cross referencing links
%% \ccHtmlIndexC[class]{} %% add further index entries

\ccDefinition
  
%The function object class \ccRefName\ does this and that.

\ccInclude{Make_x_monotone_2.h}

% The section below is automatically generated. Do not edit!
%START-AUTO(\ccDefinition)

Splits a curve that can be analyzed into connected x-monotone sweepable arcs and isolated points.

Arcs are stored as \ccc{CurvedKernelViaAnalysis_2::Arc_2} objects, and each is either vertical or consists of an x-monotone piece of constant arc number wrt to the curve at every interior x-coordinate. Isolated points are stored as \ccc{CurvedKernelViaAnalysis_2::Point_2} objects.

The resulting arcs and points are written to the output iterator as polymorphic Object. Past-the-end value of the iterator is returned.

%END-AUTO(\ccDefinition)

\ccParameters

% The section below is automatically generated. Do not edit!
%START-AUTO(\ccParameters)

template$<$  \\
class \ccc{CurvedKernelViaAnalysis_2},   \\
class \ccc{ConstructArc_2} = typename \ccc{CurvedKernelViaAnalysis_2::Construct_arc_2}$>$   \\
struct \ccc{Make_x_monotone_2};

%END-AUTO(\ccParameters)

\ccIsModel

\ccTypes

%\ccCreation
%\ccCreationVariable{a}  %% choose variable name

%% \ccIncludeExampleCode{Curved_kernel_via_analysis_2/Make_x_monotone_2.C}

% The section below is automatically generated. Do not edit!
%START-AUTO(\ccTypes)

\ccNestedType{Curved_kernel_via_analysis_2}
{
this instance's first template parameter
}
\ccGlue
\ccNestedType{Construct_arc_2}
{
this instance's second template parameter
}
\ccGlue
\ccNestedType{Curve_kernel_2}
{
type of curve kernel
}
\ccGlue
\ccNestedType{X_coordinate_1}
{
type of x-coordinate
}
\ccGlue
\ccNestedType{Xy_coordinate_2}
{
type of xy-coordinate
}
\ccGlue
\ccNestedType{Curve_analysis_2}
{
type of curve analysis
}
\ccGlue
\ccNestedType{Status_line_1}
{
type of vertical line
}
\ccGlue
\ccNestedType{Point_2}
{
type of point on curve
}
\ccGlue
\ccNestedType{Arc_2}
{
type of curve arc
}
\ccGlue

%END-AUTO(\ccTypes)

\ccCreation
\ccCreationVariable{a}  %% choose variable name for \ccMethod below

% The section below is automatically generated. Do not edit!
%START-AUTO(\ccCreation)

\ccConstructor{Make_x_monotone_2(Curved_kernel_via_analysis_2 * kernel);}
{
Standard constructor.
}
\ccGlue
\begin{description}
\item[Parameters:]
\begin{description}
\item[kernel]The kernel instance to use \end{description}
\end{description}
\ccGlue

%END-AUTO(\ccCreation)

\ccOperations

% The section below is automatically generated. Do not edit!
%START-AUTO(\ccOperations)

\ccMethod{OutputIterator operator()(Curve_analysis_2 curve, OutputIterator oi);}
{
Splits a curve into x-monotone arcs and isolated points.
}
\ccGlue
\begin{description}
\item[Parameters:]
\begin{description}
\item[curve]The input curve \item[oi]Output iterator that stores Object, which either encapsulates \ccc{Point_2} or \ccc{Arc_2} \end{description}
\end{description}
\begin{description}
\item[Returns:]Past-the-end iterator of oi \end{description}
\ccGlue

%END-AUTO(\ccOperations)

\end{ccRefFunctionObjectClass}

% +------------------------------------------------------------------------+
%%RefPage: end of main body, begin of footer
\ccRefPageEnd
% EOF
% +------------------------------------------------------------------------+

