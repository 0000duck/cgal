% +------------------------------------------------------------------------+
% | Reference manual page: Compare_x_2.tex
% +------------------------------------------------------------------------+
% | 27.03.2008   Author
% | Package: Curved_kernel_via_analysis_2
% |
\RCSdef{\RCSComparexRev}{$Id: header.tex 40270 2007-09-07 15:29:10Z lsaboret $}
\RCSdefDate{\RCSComparexDate}{$Date: 2007-09-07 17:29:10 +0200 (Fri, 07 Sep 2007) $}
% |
\ccRefPageBegin
%%RefPage: end of header, begin of main body
% +------------------------------------------------------------------------+


\begin{ccRefFunctionObjectClass}[CGALi::Curved_kernel_via_analysis_2_Functors::]{Compare_x_2}  %% add template arg's if necessary
\ccRefLabel{Compare_x_2}

%% \ccHtmlCrossLink{}     %% add further rules for cross referencing links
%% \ccHtmlIndexC[class]{} %% add further index entries

\ccDefinition
  
%The function object class \ccRefName\ does this and that.

% The section below is automatically generated. Do not edit!
%START-AUTO(\ccInclude)

\ccInclude{CGAL/Curved_kernel_via_analysis_2/Curved_kernel_via_analysis_2_functors.h}

%END-AUTO(\ccInclude)

% The section below is automatically generated. Do not edit!
%START-AUTO(\ccDefinition)

Functor that compares x-coordinates of two interior points.

%END-AUTO(\ccDefinition)


\ccParameters

% The section below is automatically generated. Do not edit!
%START-AUTO(\ccParameters)

template$<$  \\
class \ccc{CurvedKernelViaAnalysis_2}$>$   \\
class \ccc{Compare_x_2};

%END-AUTO(\ccParameters)


\ccInheritsFrom

% The section below is automatically generated. Do not edit!
%START-AUTO(\ccInheritsFrom)

\ccc{Curved_kernel_via_analysis_2_functor_base<CurvedKernelViaAnalysis_2>}

%END-AUTO(\ccInheritsFrom)


\ccIsModel

\ccTypes

%\ccCreation
%\ccCreationVariable{a}  %% choose variable name

%% \ccIncludeExampleCode{Curved_kernel_via_analysis_2/Compare_x_2.C}

% The section below is automatically generated. Do not edit!
%START-AUTO(\ccTypes)

\ccNestedType{Curved_kernel_via_analysis_2}
{
this instance' first template parameter
}
\ccGlue
\ccNestedType{Base}
{
the base type
}
\ccGlue
\ccNestedType{Curve_2}
{
the curve type
}
\ccGlue
\ccNestedType{Point_2}
{
the point type
}
\ccGlue
\ccNestedType{Arc_2}
{
the arc type
}
\ccGlue
\ccNestedType{Curve_analysis_2}
{
type of curve analaysis
}
\ccGlue
\ccNestedType{X_coordinate_1}
{
the x-coordinate type
}
\ccGlue
\ccNestedType{result_type}
{
the result type
}
\ccGlue
\ccNestedType{Arity}
{
the arity of the functor
}
\ccGlue

%END-AUTO(\ccTypes)

\ccHeading{Variables}

% The section below is automatically generated. Do not edit!
%START-AUTO(\ccHeading{Variables})
%END-AUTO(\ccHeading{Variables})

\ccCreation
\ccCreationVariable{a}  %% choose variable name for \ccMethod below

% The section below is automatically generated. Do not edit!
%START-AUTO(\ccCreation)

\ccConstructor{Compare_x_2(Curved_kernel_via_analysis_2 * kernel);}
{
Standard constructor.
}
\ccGlue
\begin{description}
\item[Parameters:]
\begin{description}
\item[kernel]The kernel \end{description}
\end{description}
\ccGlue

%END-AUTO(\ccCreation)

\ccOperations

% The section below is automatically generated. Do not edit!
%START-AUTO(\ccOperations)

\ccMethod{result_type operator()(const Point_2& p1, const Point_2& p2) const;}
{
Compare the x-coordinates of two points.
}
\ccGlue
\begin{description}
\item[Parameters:]
\begin{description}
\item[p1]The first point. \item[p2]The second point. \end{description}
\end{description}
\begin{description}
\item[Returns:]LARGER if x(p1) $>$ x(p2); SMALLER if x(p1) $<$ x(p2); EQUAL if x(p1) = x(p2). \end{description}
\ccGlue

%END-AUTO(\ccOperations)

\end{ccRefFunctionObjectClass}

% +------------------------------------------------------------------------+
%%RefPage: end of main body, begin of footer
\ccRefPageEnd
% EOF
% +------------------------------------------------------------------------+

