% +------------------------------------------------------------------------+
% | Reference manual page: Compare_y_at_x_2.tex
% +------------------------------------------------------------------------+
% | 26.03.2008   Author
% | Package: Curved_kernel_via_analysis_2
% |
\RCSdef{\RCSCompareyatxRev}{$Id: header.tex 40270 2007-09-07 15:29:10Z lsaboret $}
\RCSdefDate{\RCSCompareyatxDate}{$Date: 2007-09-07 17:29:10 +0200 (Fri, 07 Sep 2007) $}
% |
\ccRefPageBegin
%%RefPage: end of header, begin of main body
% +------------------------------------------------------------------------+


\begin{ccRefFunctionObjectClass}[CGALi::Curved_kernel_via_analysis_2_Functors::]{Compare_y_at_x_2}  %% add template arg's if necessary

%% \ccHtmlCrossLink{}     %% add further rules for cross referencing links
%% \ccHtmlIndexC[class]{} %% add further index entries

\ccDefinition
  
%The function object class \ccRefName\ does this and that.

\ccInclude{Compare_y_at_x_2.h}

% The section below is automatically generated. Do not edit!
%START-AUTO(\ccDefinition)

Functor that computes relative vertical alignment of an interior point and an arc.

%END-AUTO(\ccDefinition)

\ccParameters

% The section below is automatically generated. Do not edit!
%START-AUTO(\ccParameters)

template$<$  \\
class \ccc{CurvedKernelViaAnalysis_2}$>$   \\
class \ccc{Compare_y_at_x_2};

%END-AUTO(\ccParameters)

\ccIsModel

\ccTypes

% The section below is automatically generated. Do not edit!
%START-AUTO(\ccTypes)

\ccNestedType{Curved_kernel_via_analysis_2}
{
this instance' first template parameter
}
\ccGlue
\ccNestedType{Base}
{
the base type
}
\ccGlue
\ccNestedType{Curve_2}
{
the curve type
}
\ccGlue
\ccNestedType{Point_2}
{
the point type
}
\ccGlue
\ccNestedType{Arc_2}
{
the arc type
}
\ccGlue
\ccNestedType{Curve_analysis_2}
{
type of curve analaysis
}
\ccGlue
\ccNestedType{X_coordinate_1}
{
the x-coordinate type
}
\ccGlue
\ccNestedType{result_type}
{
the result type
}
\ccGlue
\ccNestedType{Arity}
{
the arity of the functor
}
\ccGlue

%END-AUTO(\ccTypes)

\ccHeading{Variables}

% The section below is automatically generated. Do not edit!
%START-AUTO(\ccHeading{Variables})

\ccMethod{result_type operator()(const Point_2& p, const Arc_2& cv) const;}
{
Return the relative vertical alignment of a point with an arc.
}
\ccGlue
\begin{description}
\item[Parameters:]
\begin{description}
\item[p]The point \item[cv]The arc \end{description}
\end{description}
\begin{description}
\item[Returns:]\ccc{CGAL::SMALLER} if y(p) $<$ cv(x(p)), i.e., the point is below the arc; \ccc{CGAL::LARGER} if y(p) $>$ cv(x(p)), i.e., the point is above the arc; \ccc{CGAL::EQUAL} if p lies on the arc\end{description}
\begin{description}
\item[Precondition:]p is in the x-range of cv. \end{description}
\ccGlue

%END-AUTO(\ccHeading{Variables})

\ccCreation
\ccCreationVariable{foo}  %% choose variable name

\ccConstructor{Compare_y_at_x_2();}{default constructor.}

%% \ccIncludeExampleCode{Curved_kernel_via_analysis_2/Compare_y_at_x_2.C}

\ccCreation
\ccCreationVariable{a}  %% choose variable name for \ccMethod below

% The section below is automatically generated. Do not edit!
%START-AUTO(\ccCreation)

\ccConstructor{Compare_y_at_x_2(Curved_kernel_via_analysis_2 * kernel);}
{
Standard constructor.
}
\ccGlue
\begin{description}
\item[Parameters:]
\begin{description}
\item[kernel]The kernel \end{description}
\end{description}
\ccGlue

%END-AUTO(\ccCreation)

\end{ccRefFunctionObjectClass}

% +------------------------------------------------------------------------+
%%RefPage: end of main body, begin of footer
\ccRefPageEnd
% EOF
% +------------------------------------------------------------------------+

