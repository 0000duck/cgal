% +------------------------------------------------------------------------+
% | Reference manual page: Curved_kernel_via_analysis_2.tex
% +------------------------------------------------------------------------+
% | 26.03.2008   Author
% | Package: Curved_kernel_via_analysis_2
% |
\RCSdef{\RCSCurvedkernelviaanalysisRev}{$Id: header.tex 40270 2007-09-07 15:29:10Z lsaboret $}
\RCSdefDate{\RCSCurvedkernelviaanalysisDate}{$Date: 2007-09-07 17:29:10 +0200 (Fri, 07 Sep 2007) $}
% |
\ccRefPageBegin
%%RefPage: end of header, begin of main body
% +------------------------------------------------------------------------+


\begin{ccRefClass}{Curved_kernel_via_analysis_2}  %% add template arg's if necessary

%% \ccHtmlCrossLink{}     %% add further rules for cross referencing links
%% \ccHtmlIndexC[class]{} %% add further index entries

\ccDefinition
  
%The class \ccRefName\ does this and that.

\ccInclude{Curved_kernel_via_analysis_2.h}

% The section below is automatically generated. Do not edit!
%START-AUTO(\ccDefinition)

Kernel for curves in a two-dimensional space, and points and arcs of them.

It expects a \ccc{CurveKernel_2} type that fulfills the \ccc{CurveKernel_2} concept. The other template parameters can be left default. They are used during rebind to exchange certain point and arc type.

Is a model of CGAL's \ccc{ArrangementTraits_2} concept.

%END-AUTO(\ccDefinition)

\ccParameters

% The section below is automatically generated. Do not edit!
%START-AUTO(\ccParameters)

template$<$  \\
class \ccc{CurveKernel_2},   \\
class \ccc{CKvA_} = void,   \\
class \ccc{Point_} = void,   \\
class \ccc{Arc_} = void$>$   \\
class \ccc{Curved_kernel_via_analysis_2};

%END-AUTO(\ccParameters)

\ccIsModel

\ccTypes

% The section below is automatically generated. Do not edit!
%START-AUTO(\ccTypes)

\ccNestedType{Curve_kernel_2}
{
this instance's template argument
 \ccc{boost::mpl::if_<boost::mpl::bool_< boost::is_void< CKvA_>::value} $>$, \ccc{Curved_kernel_via_analysis_2<CurveKernel_2>}, \ccc{CKvA_} $>$\ccc{::type}, \ccc{CurveKernel_2} $>$ .
}
\ccGlue
\ccNestedType{CKvA}
{
[protected] \\
this instance's second template parameter
}
\ccGlue
\ccNestedType{Point}
{
[protected] \\
this instance's third template parameter
}
\ccGlue
\ccNestedType{Arc}
{
[protected] \\
this instance's fourth template parameter
}
\ccGlue
\ccNestedType{Self}
{
[protected] \\
this instance itself
 \ccc{boost::mpl::if_<boost::mpl::bool_< boost::is_void< CKvA_>::value} $>$, \ccc{Curved_kernel_via_analysis_2<CurveKernel_2>}, \ccc{CKvA_} $>$\ccc{::type}, \ccc{CurveKernel_2} $>$ .
}
\ccGlue
\ccNestedType{CKvA_2}
{
[protected] \\
type of \ccc{CKvA_2} used internally
}
\ccGlue
\ccNestedType{Curve_2}
{
type of curve that can be analyzed
 \ccc{boost::mpl::if_<boost::mpl::bool_< boost::is_void< CKvA_>::value} $>$, \ccc{Curved_kernel_via_analysis_2<CurveKernel_2>}, \ccc{CKvA_} $>$\ccc{::type}, \ccc{CurveKernel_2::Curve_analysis_2}, \ccc{boost::mpl::if_<boost::mpl::bool_< boost::is_void< Point_>::value} $>$, \ccc{CGALi::Point_2<boost::mpl::if_< boost::mpl::bool_< boost::is_void< CKvA_>::value} $>$, \ccc{Curved_kernel_via_analysis_2<CurveKernel_2>}, \ccc{CKvA_} $>$\ccc{::type} $>$, \ccc{Point_} $>$\ccc{::type}, \ccc{boost::mpl::if_<boost::mpl::bool_< boost::is_void< Arc_>::value} $>$, \ccc{CGALi::Arc_2<boost::mpl::if_< boost::mpl::bool_< boost::is_void< CKvA_>::value} $>$, \ccc{Curved_kernel_via_analysis_2<CurveKernel_2>}, \ccc{CKvA_} $>$\ccc{::type} $>$, \ccc{Arc_} $>$\ccc{::type} $>$ .
}
\ccGlue
\ccNestedType{Point_2}
{
type of a point on a curve that can be analyzed
 \ccc{boost::mpl::if_<boost::mpl::bool_< boost::is_void< CKvA_>::value} $>$, \ccc{Curved_kernel_via_analysis_2<CurveKernel_2>}, \ccc{CKvA_} $>$\ccc{::type}, \ccc{CurveKernel_2::Curve_analysis_2}, \ccc{boost::mpl::if_<boost::mpl::bool_< boost::is_void< Point_>::value} $>$, \ccc{CGALi::Point_2<boost::mpl::if_< boost::mpl::bool_< boost::is_void< CKvA_>::value} $>$, \ccc{Curved_kernel_via_analysis_2<CurveKernel_2>}, \ccc{CKvA_} $>$\ccc{::type} $>$, \ccc{Point_} $>$\ccc{::type}, \ccc{boost::mpl::if_<boost::mpl::bool_< boost::is_void< Arc_>::value} $>$, \ccc{CGALi::Arc_2<boost::mpl::if_< boost::mpl::bool_< boost::is_void< CKvA_>::value} $>$, \ccc{Curved_kernel_via_analysis_2<CurveKernel_2>}, \ccc{CKvA_} $>$\ccc{::type} $>$, \ccc{Arc_} $>$\ccc{::type} $>$ .
}
\ccGlue
\ccNestedType{Arc_2}
{
type of an arc on a curve that can be analyzed
 \ccc{boost::mpl::if_<boost::mpl::bool_< boost::is_void< CKvA_>::value} $>$, \ccc{Curved_kernel_via_analysis_2<CurveKernel_2>}, \ccc{CKvA_} $>$\ccc{::type}, \ccc{CurveKernel_2::Curve_analysis_2}, \ccc{boost::mpl::if_<boost::mpl::bool_< boost::is_void< Point_>::value} $>$, \ccc{CGALi::Point_2<boost::mpl::if_< boost::mpl::bool_< boost::is_void< CKvA_>::value} $>$, \ccc{Curved_kernel_via_analysis_2<CurveKernel_2>}, \ccc{CKvA_} $>$\ccc{::type} $>$, \ccc{Point_} $>$\ccc{::type}, \ccc{boost::mpl::if_<boost::mpl::bool_< boost::is_void< Arc_>::value} $>$, \ccc{CGALi::Arc_2<boost::mpl::if_< boost::mpl::bool_< boost::is_void< CKvA_>::value} $>$, \ccc{Curved_kernel_via_analysis_2<CurveKernel_2>}, \ccc{CKvA_} $>$\ccc{::type} $>$, \ccc{Arc_} $>$\ccc{::type} $>$ .
}
\ccGlue
\ccNestedType{X_monotone_curve_2}
{
type of weakly x-monotone arc for \ccc{ArrangementTraits_2}
}
\ccGlue
\ccNestedType{Non_x_monotone_arc_2}
{
type of non x-monotone arc on a curve that can be analyzed
}
\ccGlue
\ccNestedType{Base_kernel}
{
[protected] \\
class collecting basic types
}
\ccGlue
\ccNestedType{Base_functors}
{
[protected] \\
class collecting basic functors
}
\ccGlue
\ccNestedType{Construct_point_2}
{
functor
}
\ccGlue
\ccNestedType{Construct_point_on_arc_2}
{
functor
}
\ccGlue
\ccNestedType{Construct_arc_2}
{
functor
}
\ccGlue

%END-AUTO(\ccTypes)

\ccHeading{Variables}

% The section below is automatically generated. Do not edit!
%START-AUTO(\ccHeading{Variables})

\ccMethod{Construct_point_2 construct_point_2_object() const;}
{
returns instance of functor
}
\ccGlue
\ccMethod{Construct_point_on_arc_2 construct_point_on_arc_2_object() const;}
{
returns instance of functor
}
\ccGlue
\ccMethod{Construct_arc_2 construct_arc_2_object() const;}
{
returns instance of functor
}
\ccGlue

%END-AUTO(\ccHeading{Variables})

\ccCreation
\ccCreationVariable{foo}  %% choose variable name

\ccConstructor{Curved_kernel_via_analysis_2();}{default constructor.}


%% \ccIncludeExampleCode{Curved_kernel_via_analysis_2/Curved_kernel_via_analysis_2.C}

\ccCreation
\ccCreationVariable{a}  %% choose variable name for \ccMethod below

% The section below is automatically generated. Do not edit!
%START-AUTO(\ccCreation)

\ccConstructor{Curved_kernel_via_analysis_2();}
{
default constructor
}
\ccGlue
\ccConstructor{Curved_kernel_via_analysis_2(const Curve_kernel_2& kernel);}
{
construct from kernel
}
\ccGlue
\begin{description}
\item[Parameters:]
\begin{description}
\item[kernel]Kernel to use internally \end{description}
\end{description}
\ccGlue

%END-AUTO(\ccCreation)

\end{ccRefClass}

% +------------------------------------------------------------------------+
%%RefPage: end of main body, begin of footer
\ccRefPageEnd
% EOF
% +------------------------------------------------------------------------+

