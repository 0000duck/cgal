\section{Introduction}

The \ccc{Curved_kernel_via_analysis_2} is designed to support 
curved geometry in the two-dimensional space. It offers main types
and operations on them in form of functors. The design rationale is
to seperate the analysis of curves and pairs of them from the
combinatorial decision based on them. Current focus of the package
consists of \cgal's \ccc{Arrangement_on_surface_2}. Thus, 
properly instantiated, the \ccc{Curved_kernel_via_analysis_2}
models \cgal's \ccc{ArrangementTraits_2} concept.

\section{Software Design}

The main classes are designed to seperate from curves that can be analyzed
from the geometric types and functors that are provided. The main
template parameter is \ccc{CurveKernel_2} that has to fulfill the 
\ccc{CurveKernel_2} concept. In particular, a given model of
\cgal's \ccc{AlgebraicKernelWithAnalysis_2} concept can be easily
wrapped to be used in the \ccc{Curved_kernel_via_analysis_2}.

Such instantiated, the kernel provides four main types:
\begin{itemize}
\item \ccc{Curve_2} Represents a supporting curves
\item \ccc{Point_2} Represents a point on a curve
\item \ccc{Arc_2} Represents an $x$-monotone arc on a curve
\item \ccc{Non_x_monotone_arc_2} Represents a non-$x$-monotone arcs on a 
curve.
\end{itemize}

In addition, the kernel provides main geometric functors. Here are some
examples:
\begin{itemize}
\item \ccc{Make_x_monotone_2} To decompose a \ccc{Curve_2} or 
\ccc{Non_x_monotone_arc_2} into its $x$-monotone arcs (and maybe isolated
points)
\item \ccc{Compare_xy_2} To compare to points lexicographically
\item \ccc{Intersect_2} To compute the intersection of 
two $x$-monotone arcs
\end{itemize}
The full list can be found in the reference documentation, see 
Section~\ref{cgal:ckva2-ref}

\paragraph{Current Work}

The is also a filtered version, called 
\ccc{Filtered_curved_kernel_via_analysis_2}. It derives from the the
usual kernel, that is, it has the same interface. We applied
an initial set of geometric filters in the derived functors. 
In order to obtain best filtering, we encourage to instantiate the
\ccc{Filtered_curved_kernel_via_analysis_2} also with a
\ccc{Filtered_curve_kernel_2}. We currently work to improve the efficiency
of the filtered kernels.

Both kernels can also be used to compute arrangements on two-dimensional
parametric surfaces. We work on kernel-derivations that support 
points and arcs on curves embedded on parametric surfaces of 
general interest, like quadrics or torus.

\section{Design and Implementation History}

The basic ideas for this package has been proposed and implemented
by Arno Eigenwillig, when he created points and arcs on algebraic curves
of degree~3, so-called cubics~\cite{eksw-eecaccc-06}. 
He named the module {\sc Gaps} (generic algebraic points and segments), 
which have been part of {\sc Exacus}'\footnote{See project homepage at 
{\tt http://www.mpi-inf.mpg.de/EXACUS}}
{\sc SweepX} library. Eric Berberich completed the work with respect
to vertical arcs, covertical events and simplification of representations.
{\sc Gaps} has been used to represent points and arcs of conics, projected
intersection curves of quadrics in 3D, and also for algebraic curves
of arbitrary degree. See~\cite{beh+-eeeafcs-05} for details.

With the decision to merge {\sc Exacus} in \cgal\, the package has been 
redesigned and several improvements have been applied, e.g., curves are not
required to be algebraic, i.e., it suffices if they can be analyzed following
a new concept. The package also clearly distinguishes between 0-dimensional
points and 1-dimensional arcs. Responsible for the merge towards \cgal\ was
Pavel Emeliyanenko, while Eric Berberich enhanced the implementation with the
respect to possibilities offered by 
\cgal's \ccc{Arrangement_on_surface_2} package. Michael Kerber helped during
important design decisions.

This work was partially supported by the IST Programme of the EU as a
Shared-cost RTD (FET Open) Project under Contract No IST-2000-26473
(\ccAnchor{http://www-sop.inria.fr/prisme/ECG/}{ECG} - Effective
Computational Geometry for Curves and Surfaces).\\
This work was partially supported by the IST Programme of the 6th
Framework Programme of the EU as a STREP (FET Open Scheme) Project
under Contract No IST-006413 (\ccAnchor{http://acs.cs.rug.nl/}{ACS} -
Algorithms for Complex Shapes).
