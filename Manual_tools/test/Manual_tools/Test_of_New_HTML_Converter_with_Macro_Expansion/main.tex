
Accents and Umlaute: \"a\ \"o\ \"u\ \"A\ \"O\ \"{U}\ \'e\ \`e\ \^e\ 
  \~n\ \c{c}\ \ss{}\ \ae\ \AE\ \aa\ \AA\ \oe\ \OE\ \o\ \O\ \S\ 
  \P\ \copyright\ \pounds\ \dag\ \ddag\ \l\ \L.


Macros \TeX{} \LaTeX\ \LaTeXe\ to name the game.

Greek alphabet: $\alpha\ \beta\ \gamma\ \delta\ \epsilon\ 
    \varepsilon\ \zeta\ \eta\ \theta\ \vartheta\ \iota\ \kappa\ \lambda\
    \mu\ \nu\ \xi o\ \pi\ \varpi\ \rho\ \varrho\ \sigma\ \varsigma\ 
    \tau\ \upsilon\ \phi\ \varphi\ \chi\ \psi\ \omega$,  
    $\Gamma\ \Delta\ \Theta\ \Lambda\ \Xi\ \Pi\ \Sigma\ \Upsilon\ \Phi\ 
    \Psi\ \Omega$

Miscelleaneous math: $x \cdot y, x \times y$, $\sqrt{2}$, $\sqrt[4]{18}$,
    $\frac{1}{2}$.

Some dots: $M \ldots M \cdot M \cdots M \vdots M \ddots M$.

Calligraphic letters: $\mathcal{A}$ $\mathcal{B}$ $\mathcal{C}$ $\mathcal{D}$
 $\mathcal{E}$ $\mathcal{F}$ $\mathcal{G}$ $\mathcal{H}$ $\mathcal{I}$
 $\mathcal{J}$ $\mathcal{K}$ $\mathcal{L}$ $\mathcal{M}$ $\mathcal{N}$
 $\mathcal{O}$ $\mathcal{P}$ $\mathcal{Q}$ $\mathcal{R}$ $\mathcal{S}$
 $\mathcal{T}$ $\mathcal{U}$ $\mathcal{V}$ $\mathcal{W}$ $\mathcal{X}$
 $\mathcal{Y}$ $\mathcal{Z}$.

Mathbb letters: $\mathbb{A}$ $\mathbb{B}$ $\mathbb{C}$ $\mathbb{D}$
 $\mathbb{E}$ $\mathbb{F}$ $\mathbb{G}$ $\mathbb{H}$ $\mathbb{I}$
 $\mathbb{J}$ $\mathbb{K}$ $\mathbb{L}$ $\mathbb{M}$ $\mathbb{N}$
 $\mathbb{O}$ $\mathbb{P}$ $\mathbb{Q}$ $\mathbb{R}$ $\mathbb{S}$
 $\mathbb{T}$ $\mathbb{U}$ $\mathbb{V}$ $\mathbb{W}$ $\mathbb{X}$
 $\mathbb{Y}$ $\mathbb{Z}$.

Some not-relations: $ \not\in $.

Binary operation symbols: $ \pm \times \div \mp \ast \star \circ \bullet
 \cap \cup \uplus \sqcap \sqcup \vee \wedge \setminus \wr \diamond
 \bigtriangleup \bigtriangledown \triangleleft \triangleright \lhd
 \rhd \unlhd \unrhd \oplus \ominus \otimes \oslash \odot \bigcirc \dagger
 \ddagger \amalg$.

Relation symbols: $ \leq \prec \preceq \ll \subset \subseteq \sqsubset
 \sqsubseteq \in \vdash \geq \succ \succeq \gg \supset \supseteq \sqsupset
 \sqsupseteq \ni \dashv \equiv \sim \simeq \asymp \approx \cong \neq
 \doteq \notin \models \perp \mid \parallel \bowtie \Join \smile \frown
 \propto$.

Arrow symbols: $ \leftarrow \Leftarrow \rightarrow \Rightarrow \leftrightarrow
 \Leftrightarrow \mapsto \hookleftarrow \leftharpoonup \leftharpoondown
 \rightleftharpoons \longleftarrow \Longleftarrow \longrightarrow
 \Longrightarrow \longleftrightarrow \Longleftrightarrow \longmapsto
 \hookrightarrow \rightharpoonup \rightharpoondown \leadsto \uparrow \Uparrow
 \downarrow \Downarrow \updownarrow \Updownarrow \nearrow \searrow \swarrow
 \nwarrow$.

Miscellaneous symbols: $\prime \neg \| \backslash \aleph \hbar \imath \jmath
 \ell \wp \Re \Im \mho \emptyset \nabla \surd \top \bot \angle \forall \exists
 \flat \natural \sharp \partial \infty \Box \Diamond \triangle \clubsuit
 \diamondsuit \heartsuit \spadesuit$.

Variable sized symbols (small): $ \sum \prod \coprod \int \oint \bigcap
 \bigcup \bigsqcup \bigvee \bigwedge \bigodot \bigotimes \bigoplus \biguplus$.

Variable sized symbols (big): \[ \sum \prod \coprod \int \oint \bigcap
 \bigcup \bigsqcup \bigvee \bigwedge \bigodot \bigotimes \bigoplus \biguplus\].

Delimiters: $ \lfloor x \rfloor, \lceil y \rceil, \langle z \rangle$.

Math accents: $\hat{e}, \acute{e}, \grave{e}, \tilde{n}, \ddot{a}$.


\begin{alltt}
Ha
%
\LaTeX{}
%
Ha
\end{alltt}

\ccHtmlCrossLink{FunnyClass}This line should be the target for the 
hyperlinks emanating from the class \ccc{FunnyClass}.

Here another paragraph mentioning the class \ccc{FunnyClass}. %Place
%it in the index, first by default under class, and then additionally under
%nested types.\ccHtmlIndexC{FunnyClass}\ccHtmlIndexC[nested_type]{FunnyClass}
%Put some formated entries in the index. \ccHtmlIndex{PlainClass}
%\ccHtmlIndex[function]{This is {\em Stuff} related to \ldots.}


My \ccAnchor{http://www.inf.ethz.ch/personal/kettner/}{homepage} 
is at \path|http://www.inf.ethz.ch/personal/kettner/|.

\begin{ccHtmlClassFile}{qqq.html}{Declaration of \ccc{Qaa}}
\begin{ccClass}{Q}
\ccCreationVariable{y}
\ccConstructor{Q();}{default}
\end{ccClass}
\end{ccHtmlClassFile}

\newcommand{\xa}{hello}
\newcommand{\xb}[1]{\xa }
\newcommand{\xc}[1]{\xa}
\newcommand{\xd}{hello }
\newcommand{\xe}[1]{%
#1
}
\newcommand{\xf}[1]{ #1 }
\newcommand{\xg}[1]{#1}
\newcommand{\xh}[1]{\xg}

Test for spaces in macro expansion: all lines should look the same.

Say hello to me, hellome.\\
Say \xb{A} to me, \xb{A}me.\\
Say \xc{A} to me, \xb{A}me.\\
Say \xd    to me, \xb{A}me.\\
Say\xe{ hello}to me, hellome.\\
Say\xe{}hello to me, hellome.\\
Say\xf{hello}to me, hellome.\\
Say hello \xh{} t\xh{}o me, hellome.\\
Say hello\xh{}\ t\xh{}o me, hellome.

Another example, looking slightly different.

Say \ hello to me.\\
Say\xf{}hello to me.


%% ==========================================
\begin{lcHtmlBlock}
{\bf Macro expansion tests}

Should all look like

No no \textbf{Bla} bla no no.\\
%
\newcommand{\macro}{\textbf{Bla} bla}
No no \macro\ no no.\\
%
\def\macroa{\textbf{Bla} bla}
No no \macroa\ no no.\\
%
\newcommand{\macrob} %%

  {\textbf{Bla}  bla}
%
No no \macrob\ no no.\\
%
\newcommand{\macroc}{\textbf{Bla} % hahahha
  bla}
%
No no \macroc\ no no.\\
%
\newcommand{\a}[1]{\textbf{#1} bla}
%
No no \a{Bla} no no.\\
%
\def\aa #1{\textbf{#1} bla}
%
No no \aa{Bla} no no.\\
%
\def\aaa #1#2#3{\textbf{#3#2#1} bla}
%
No no \aaa alB no no.\\
%
\newcommand{\b}[1]{ #1}
\newcommand{\b@om}{\textbf{#1} #2}
%
No no \b[Bla]{bla}\b{no}\ no.\\
%
\newcommand{\bb}[1]{ #1}
\newcommand{\bb@mo}{\textbf{#1} #2}
%
No no \bb{Bla}[bla]\bb{no}\ no.\\
%
\newcommand{\bb@omo}{\textbf{#1} #2#3}
No no \bb[Bla]{bla}[ no]\ no.


%% ==========================================
Should all look like

No no \textbf{Bla} \% \{ \{ \} bla no no.\\
%
\newcommand{\macrod}{\textbf{Bla} \% \{ \{ \} bla}
No no \macrod\ no no.\\
%
\newcommand{\|}{\textbf{Bla} \% \{ \{ \} bla}
%
No no \| no no.

%% ==========================================
Should all look like

Hallo Dummy World.\\
%
\newenvironment{c}{Hallo}{World}
\begin{c} Dummy \end{c}.\\
%
\newenvironment{d}[1]{#1}{World}
\begin{d}{Hallo} Dummy \end{d}.\\
%
\newenvironment{e}[1]{#1}{World}
\newcommand{\begin@e@om}{#1#2}
\begin{e}{Hallo} Dummy \end{e}.\\
\begin{e}[Hal]{lo} Dummy \end{e}.\\
%
\newcommand{\begin@e@moo}{#1#2#3}
\begin{e}{Ha}[l][lo] Dummy \end{e}.\\
%
\newenvironment{ee}{Hallo }{World}
\newcommand{\begin@ee@o}{Ha#1o}
\begin{ee} Dummy \end{ee}.\\
\begin{ee}[ll] Dummy \end{ee}.\\
%
Hallo~Dummy World.

\newtheorem{mytheorem}{Theorem}

\begin{mytheorem}
    Hallo theorem.
\end{mytheorem}

\begin{mytheorem}[Optional]
    Hallo {\em emphasized\/} theorem.
\end{mytheorem}

\paragraph{Paragraph}

blablabla blablabla blablabla blablabla blablabla blablabla blablabla 
blablabla blablabla blablabla blablabla 

\paragraph*{Paragraph}

blablabla blablabla blablabla blablabla blablabla blablabla blablabla 
blablabla blablabla blablabla blablabla 

\paragraph[Bloek]{Paragraph}

blablabla blablabla blablabla blablabla blablabla blablabla blablabla 
blablabla blablabla blablabla blablabla 

\def\qtrue{\ccTrue}
\def\qfalse{\ccFalse}
\def\qauto{\ccAuto}
\newcommand{\qy}[1]{\lciIf{#1}{\lcAsciiToHtml{#1} is true.}
                              {\lcAsciiToHtml{#1} is false.}}

\newcommand{\qz}[1]{\lciIf{#1}[\lcAsciiToHtml{#1} is auto.]
                              {\lcAsciiToHtml{#1} is true.}
                              {\lcAsciiToHtml{#1} is false.}}

\qy\qtrue \\
\qy\qfalse

\qz\qtrue \\
\qz\qfalse\\
\qz\qauto

ccTrueAndFalse \lciIfDefined{\ccTrueAndFalse}{ is defined.}{ is not defined.}\\
ccTrue \lciIfDefined{\ccTrue}{ is defined.}{ is not defined.}

\lciIfEqual{Hallo}{Hallo}{correct}{wrong}\\
\lciIfEqual{<}{<}{correct}{wrong}\\
\lciIfEqual{Hallo}{<}{wrong}{correct}

\ccFunction{int foo( const Bla& ref) const;}{}

\ccGlobalFunction{int foo( const Bla& ref) const;}

\ccClassDeclaration{class Demo;}{huhu}

\ccInclude{CGAL/Vector_2.h}

Stack implemented using macros:

\newcommand{\mystack}{}

Empty stack: stack = \lciPrintStack{\mystack}\\
Test for emptyness: \lciIfEmpty{\mystack}{correct}{failed}.\\
\lciPushStack{\mystack}{first item}
Push an item: stack = \lciPrintStack{\mystack}\\
Test for non-emptyness: \lciIfEmpty{\mystack}{failed}{correct}.\\
Show top: \lciTopStack{\mystack}{\mytop}\mytop\\
\lciPushStack{\mystack}{second item}
Push an item: stack = \lciPrintStack{\mystack}\\
Test for non-emptyness: \lciIfEmpty{\mystack}{failed}{correct}.\\
Show top: \lciTopStack{\mystack}{\mytop}\mytop\\
\lciPushStack{\mystack}{third item}
Push an item: stack = \lciPrintStack{\mystack}\\
\lciPopStack{\mystack}
Pop an item: stack = \lciPrintStack{\mystack}\\
\lciPopStack{\mystack}
Pop an item: stack = \lciPrintStack{\mystack}\\
\lciPopStack{\mystack}
Pop an item: stack = \lciPrintStack{\mystack}\\
Test for emptyness: \lciIfEmpty{\mystack}{correct}{failed}.\\

\end{lcHtmlBlock}

\def\nn #1#2{A#1B#2C}

\begin{alltt}
\nn {D} {E} F
\nn {%} {
E} F

Ha
%
\LaTeX Hallo
%
Ha
\end{alltt}

{\it Hallo {\em Du} da}

{\tt Hallo {\em Du} da}

{\sc Hallo {\em Du} da}

$\mathcal{HALLO\ DU\ AUCH}$

\begin{abstract}
This is an abstract. Blabla Blabla Blabla Blabla
  Blabla Blabla Blabla Blabla Blabla Blabla Blabla Blabla Blabla
  Blabla Blabla.
\end{abstract}

\begin{quote}
  Two paragraphs in a quoted environment. Blabla Blabla Blabla Blabla
  Blabla Blabla Blabla Blabla Blabla Blabla Blabla Blabla Blabla
  Blabla Blabla.
  
  Blabla Blabla Blabla Blabla Blabla Blabla Blabla Blabla Blabla
  Blabla Blabla Blabla Blabla Blabla Blabla.  Blabla Blabla Blabla
  Blabla Blabla Blabla.
\end{quote}

\centerline{1st centerline.}
\centerline{2nd centerline.}

\begin{center}
  Two paragraphs in a quoted environment. Blabla Blabla Blabla Blabla
  Blabla Blabla Blabla Blabla Blabla Blabla Blabla Blabla Blabla
  Blabla Blabla.
  
  Blabla Blabla Blabla Blabla Blabla Blabla Blabla Blabla Blabla
  Blabla Blabla Blabla Blabla Blabla Blabla.  Blabla Blabla Blabla
  Blabla Blabla Blabla.
\end{center}

\paragraph{Paragraph}

  Blabla Blabla Blabla Blabla Blabla Blabla Blabla Blabla Blabla
  Blabla Blabla Blabla Blabla Blabla Blabla.  Blabla Blabla Blabla
  Blabla Blabla Blabla.

\subparagraph{Subparagraph}

  Blabla Blabla Blabla Blabla Blabla Blabla Blabla Blabla Blabla
  Blabla Blabla Blabla Blabla Blabla Blabla.  Blabla Blabla Blabla
  Blabla Blabla Blabla.

\begin{tabular}{llcr}
A & B & C & D
\end{tabular}

\begin{tabular}{llcr}\hline
\multicolumn{2}{l}{HalloLang} & C & D     \\
A & \multicolumn{2}{l}{HalloLang} & D     \\
A & B & \multicolumn{2}{l}{HalloLang}     \\
\multicolumn{4}{l}{HalloLang}             \\
%
\multicolumn{2}{l}{HalloLang} & C & D     \\\hline
A & \multicolumn{2}{l}{HalloLang} & D     \\\hline
A & B & \multicolumn{2}{l}{HalloLang}     \\\hline
\multicolumn{4}{l}{HalloLang}             \\\hline
%
\multicolumn{2}{l}{HalloLang} & C & D     \\\hline\hline
A & \multicolumn{2}{l}{HalloLang} & D     \\\hline\hline
A & B & \multicolumn{2}{l}{HalloLang}     \\\hline\hline
\multicolumn{4}{l}{HalloLang}             \\\hline\hline
%
A & B & C & D                             \\\hline\hline
\end{tabular}

\newcommand{\lcTabularBorder}{2}

\begin{tabular}{llcr}\hline
\multicolumn{2}{l}{HalloLang} & C & D     \\
A & \multicolumn{2}{l}{HalloLang} & D     \\
A & B & \multicolumn{2}{l}{HalloLang}     \\
\multicolumn{4}{l}{HalloLang}             \\
%
\multicolumn{2}{l}{HalloLang} & C & D     \\\hline
A & \multicolumn{2}{l}{HalloLang} & D     \\\hline
A & B & \multicolumn{2}{l}{HalloLang}     \\\hline
\multicolumn{4}{l}{HalloLang}             \\\hline
%
\multicolumn{2}{l}{HalloLang} & C & D     \\\hline\hline
A & \multicolumn{2}{l}{HalloLang} & D     \\\hline\hline
A & B & \multicolumn{2}{l}{HalloLang}     \\\hline\hline
\multicolumn{4}{l}{HalloLang}             \\\hline\hline
%
A & B & C & D                             \\\hline\hline
\end{tabular}

\begin{verbatim}
<Angle Brackets>
\end{verbatim}

\begin{alltt}
<Angle Brackets>
\end{alltt}

\label{section_here}

Please refer to Section~\ref{section_here}.

Hallo {\bf Boldface} abcdef.

Hallo {\it Italics} abcdef.

Hallo {\tt Typewriter} abcdef.

{\it Hallo {\em Du} da}

{\tt Hallo {\em Du} da}

{\sc Hallo {\em Du} da}

Hallo {\large large} abcdef.

Hallo {\huge huge} abcdef.

Hallo {\small small} abcdef.

\def\llca{Hallo}

\def\llcaa{NEIN}

\def\llcb{
\llca%
llca
}
\llcb

\def\llcc #1{\llca #1\llca }
\begin{alltt}
\llcc{\llca }B
\end{alltt}

% ----------------------------------------------------------------------
\begin{lcHtmlBlock}
\chapter{Test Chapter}

This is my chapter body.

A nice class herin.

\begin{ccClass}{NiceClass}
\ccNestedType{NType}{comment.}
\end{ccClass}
\end{lcHtmlBlock}

% ----------------------------------------------------------------------
