
This example tests the indentation and right margin setting
possibilities. Two long declarations with alternative formatting rules
for the function arguments are used. First the  declaration without
any indentation or margins.

\ccFunction{int a_really_long_function_name( double paramter1, double
  paramter2);}{the default formatting. A bit more text is necessary to
  demonstrate the right margin.}

Now with 10mm indentation and 10mm right margin. Note that the
description is also further indented since all dimensions are
calculated from left.

\ccTagDefaults
\def\ind{\hspace*{7mm}}
\ccTexHtml{\ccwIndent=10mm \ccwRightMargin=10mm}{}

\ccFunction{int a_really_long_function_name( double paramter1, double
  paramter2);}{the default formatting. A bit more text is necessary to
  demonstrate the right margin.}
\renewcommand{\ccLongParamLayout}{\ccTrue}

\ccFunction{int a_really_long_function_name( double paramter1, double
  paramter2);}{the alternative formatting. A bit more text is necessary to
  demonstrate the right margin.}

Show the indentation also for template functions.

\ccFunction{template<class A> int bar(A a);}{A bit more text to
  demonstrate the right margin.}

\renewcommand{\ccLongParamLayout}{\ccFalse}

Check the same alternate indentation with the compatibility mode 
\verb+\renewcommand{\ccAlternateThreeColumn}{\ccFalse}+

\renewcommand{\ccAlternateThreeColumn}{\ccFalse}

\ccFunction{int a_really_long_function_name( double paramter1, double
  paramter2);}{the alternative formatting. A bit more text is necessary to
  demonstrate the right margin.}

Show the indentation also for template functions.

\ccFunction{template<class A> int bar(A a);}{A bit more text to
  demonstrate the right margin.}

\renewcommand{\ccAlternateThreeColumn}{\ccTrue}
\ccTexHtml{\ccwIndent=0mm \ccwRightMargin=0mm}{}

% ----------------------------------------------------------------------
