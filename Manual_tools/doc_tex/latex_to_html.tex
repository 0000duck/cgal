% ___________________________________________________________________________
% |#########################################################################|
% |                                                                         |
% | The Manual for the LaTeX converter   latex_to_html.tex                  |
% | -------------------------------------------------------------           |
% |                                                                         |
% | 30.07.1998   Lutz Kettner   kettner@acm.org                             |
% | Zurich, Switzerland                                                     |
% | $Revision$                                                        |
% | $Date$                                            |
% |_________________________________________________________________________|
% |#########################################################################|

\documentclass[11pt]{article}
\usepackage{latexsym}
\usepackage{amssymb}
\usepackage{path}
\usepackage{epsfig}
\usepackage{cc_manual}
\usepackage{latex_to_html}
\usepackage{makeidx}
\makeindex

%\pagestyle{empty}
\textwidth 15.4cm 
\textheight 24 cm
\topmargin -14mm       
\evensidemargin 3mm 
\oddsidemargin 3mm

\parindent0em
\setlength{\parskip}{1.4ex}

\sloppy

{
  \begingroup
    \catcode`\|=0
    \catcode`\[=1
    \catcode`\]=2
    \catcode`\{=12
    \catcode`\}=12    
    \catcode`\\=12    
    |gdef|Open[[|tt {]]
    |gdef|Close[[|tt }]]
    |gdef|Backslash[[|tt \]]
  |endgroup
}

\newcommand{\Mindex}[1]{\index{#1@\protect\Backslash{\tt #1}}}
\newcommand{\ccIndexEntry}[1]{\index{#1@\protect\Backslash{\tt #1}}}
\newcommand{\lcIndex}[1]{\index{#1@\protect\Backslash{\tt #1}}}
\newcommand{\TTindex}[1]{\index{#1@{\tt #1}}}
\newcommand{\Dindex}[1]{#1\index{#1}}

\def\ind{\hspace*{7mm}}

% ----------------------------------------------------------------------
\title{\LaTeX\ to HTML Converter\\
       {\tt cc\_manual\_to\_html} and {\tt latex\_converter.sty}}
\author{Lutz Kettner}

\date{\lcRevision. \lcDate}

\begin{document}

\maketitle

\tableofcontents
\thispagestyle{empty}
\clearpage
\thispagestyle{empty}
~\vfill

% ___________________________________________________________________________
% |#########################################################################|
% |                                                                         |
% | The Manual for the C++ Reference Manual Style  disclaimer.tex           |
% | -------------------------------------------------------------           |
% |                                                                         |
% | 02.09.1996   Lutz Kettner   kettner@acm.org                             |
% | Zurich, Switzerland                                                     |
% | $Id$                                                       |
% | $Date$                                            |
% |_________________________________________________________________________|
% |#########################################################################|

% General disclaimer, copyright notice, and acknowledgements.

{\bf Note}

{\small These \LaTeX\ style and tools were created during the writing
  of the first versions of the CGAL Kernel Reference Manual and they
  are still evolving. New features are continuously added (even if
  this runs on low priority), so it might happen that the features
  implemented in the style file are non orthogonal, ugly named, and
  inconsistent. Please report errors, anomalies, and suggestions for
  improvements to {\tt <kettner@inf.ethz.ch>}. Please check the latest
  release, for example, available for members on the CGAL Home Page at
  {\tt <http://www.cs.uu.nl/CGAL/Members/>}.}\vspace{2ex}

{\bf Acknowledgements} 

{\small The CGAL Kernel is a group effort \cite{Fabri96,Fabri99} and
  so is the reference manual style, the \LaTeX\ to HTML converter and
  the related tools. I would like to thank Andreas Fabri, Geert-Jan
  Giezeman, Stefan Schirra, and Sven Sch\"onherr for all the
  suggestions and improvements, Susan Hert for taking over maintenance
  of this package, and the LEDA\TTindex{LEDA} group for their manual
  layout \cite{Naeher95} that we have mimicked here.}\vspace{2ex}

{\bf Copyright \copyright\ 1996,1997,1998,1999\\
Lutz Kettner}

{\small Permission to use, copy, and distribute this software and its
  documentation for non-comercial purpose is hereby granted without
  fee, provided that the above copyright notice appear in all copies
  and that both that copyright notice and this permission notice
  appear in supporting documentation.  I make no representations about
  the suitability of this software for any purpose.  It is provided
  "as is" without express or implied warranty.}



\cleardoublepage\setcounter{page}{1}

% ----------------------------------------------------------------------
\section{Introduction}

This manual describes the \LaTeX\ to HTML converter {\tt
  cc\_manual\_to\_html}. The converter is accompanied by a \LaTeX\ style
file, {\tt latex\_converter.sty}, that defines a couple of macros for
HTML output which are then usually ignored in the \LaTeX\ formatting,
for example, hyper-links.

The converter is a full fledged \LaTeX\ interpreter and understands
most of \LaTeX2e's language repertoire including user macro
definitions and several additional style files.  Examples for the
additionally supported style files are {\tt alltt}, {\tt path}, {\tt
  cprog}, and {\tt cc\_manual}~\cite{k-clswr-99}. In fact, the
converter is, similar to \TeX, realized as a macro-expansion engine
and a set of style files.  Currently, restrictions of the converter
are mostly in the area of complex formulas, complex tables and figure
environments. The converter does not repeat the layout of \LaTeX\ in
HTML. Instead, it tries to find a logical translation into HTML which
may result in a different layout.

The following sections stretches from the documentation of the \LaTeX\ 
style file to the use, configuration, and some internal implementation
details of the converter. We start with the new macros and
environments to support HTML generation in the new style file. We
continue with an introduction to the assumed structure of the manual
and the induced output files using the converter. The use of the
converter is subdivided into two section. The first section documents
the usual converter call to translate a single document. The second
section addresses how to convert a larger manual spread over several
subdirectories. Such an organization of the manual helps in the case
of name clashes, e.g., two concepts of the same name.  Thereafter we
discuss the part of \LaTeX\ covered with our converter and give a list
of unsupported macros. In the remaining sections we go into some
implementation details of the converter and its internal macros that
can help to implement own style files for the converter.

% -------------------------------------------------------
\section{\LaTeX\ Style File {\tt latex\_converter.sty}}
\label{sectionHTMLsupport}

The converter distinguishes different text coding standards: {\em
  text} for the usual \LaTeX\ source text, {\em ascii-text} for ASCII
encoded text, {\em html-text} for HTML encoded text, and sometimes
others.

\def\lcIx  #1\lcIxEnd{\lcIndex{#1}}
\def\lcIxE #1\lcIxEnd{\index{#1 environment@{\tt #1} environment}}

\newcommand{\plainitem}[1]{\item[{\normalfont {\tt #1}}]
    \lcIx #1\lcIxEnd~\par}

\newcommand{\macroitem}[1]{\item[{\normalfont {\tt \Backslash #1}}]
    \lcIx #1\lcIxEnd~\par}
\newcommand{\macroitemX}[2]{\item[{\normalfont {\tt \Backslash #1\Open}{\em
    #2\/}{\tt\Close}}]\lcIx #1\lcIxEnd~\par}
\newcommand{\macroitemXX}[3]{\item[{\normalfont {\tt \Backslash #1\Open}{\em
    #2\/}{\tt\Close\Open}{\em #3\/}{\tt\Close}}]\lcIx #1\lcIxEnd~\par}
\newcommand{\macroitemXXX}[4]{\item[{\normalfont {\tt \Backslash #1\Open}{\em
    #2\/}{\tt\Close\Open}{\em #3\/}{\tt\Close\Open}{\em #4\/}{\tt\Close}}]
    \lcIx #1\lcIxEnd~\par}
\newcommand{\macroitemXOX}[4]{\item[{\normalfont {\tt \Backslash #1\Open}{\em
    #2\/}{\tt\Close[}{\em #3\/}{\tt]\Open}{\em #4\/}{\tt\Close}}]
    ~\par}
\newcommand{\macroitemXXXX}[5]{\item[{\normalfont {\tt \Backslash #1\Open}{\em
    #2\/}{\tt\Close\Open}{\em #3\/}{\tt\Close\Open}{\em #4\/}{\tt\Close
    \Open}{\em #5\/}{\tt\Close}}]\lcIx #1\lcIxEnd~\par}
\newcommand{\envitem}[2]{\item[{\normalfont {\tt \Backslash begin\Open 
    #1\Close}{\em\ #2 }{\tt \Backslash end\Open #1\Close}}]
    \lcIxE #1\lcIxEnd~\par}
\newcommand{\envitemX}[3]{\item[{\normalfont {\tt \Backslash begin\Open 
    #1\Close\Open}{\em #2}{\tt\Close}{\em\ #3 }{\tt \Backslash 
    end\Open #1\Close}}]\lcIxE #1\lcIxEnd~\par}

\begin{description}
  \macroitemX{lcRawHtml}{html-text}
    Outputs {\em html-text\/} without any formatting, i.e., {\em html-text\/} 
    is supposed to be in HTML format.
  \macroitemX{lcRawHtmlExpanded}{html-text}
    Outputs {\em html-text\/} without any formatting, i.e., {\em html-text\/} 
    is supposed to be in HTML format, except that if {\em html-text\/} 
    starts with a macro, this macro is expanded before the output.    
  \macroitemX{lcAsciiToHtml}{ascii-text}
    Outputs {\em ascii-text\/} which is supposed to be in ASCII format.
    Creates the necessary conversions for the special characters of HTML.
  \macroitemX{lcTex}{text}
    Encapsulates {\em text\/} that is only meant to be formatted in \LaTeX.
    It is ignored during the conversion.
  \macroitemX{lcHtml}{text}
    Encapsulates {\em text\/} that is only meant to be formatted with the 
    converter to HTML.
  \envitem{lcTexBlock}{text}
    Encapsulates {\em text\/} that is only meant to be formatted in
    \LaTeX. It is ignored during the conversion.
  \envitem{lcHtmlBlock}{text}
    Encapsulates {\em text\/} that is only meant to be formatted with the 
    converter to HTML.
  \envitem{lcRawHtmlBlock}{html-text}
    Encapsulates {\em html-text\/} that is only meant to be formatted with the 
    converter to HTML. The {\em html-text\/} is supposed to be in HTML and
    is written literally to the output.
  \macroitemXX{lcAnchor}{URL}{text}
    Formats text and surrounds it with an anchor pointing to {\em URL}.
\end{description}

The remainder of this section addresses possible customizations of the
converter with respect to the {\tt cc\_manual.sty} style file
(see~\cite{k-clswr-99}).  The conversion usually creates a new file
for each class with the class name as file name. In addition it adds
an entry to the table of contents, to the index, and it links all
occurences of this class name in all other places of the manual to
point to its place of declaration. For a manual documenting classes in
a single namespace this behavior is reasonable. For other purposes,
more flexibility is provided here. One example are class requirements
that can be documented like a class but they are nowhere implemented.
The flexibility introduced here separates apart the creation of the
file, the table of contents entry, the index entry, the automatic
cross linking and the layout management of the class. The
\verb+ccClass+ and \verb+ccClassTemplate+ environments are responsible
for the layout and class name variable managment (e.g.~the
\verb+\ccClassName+ variable).  The other default mechanisms can be
deactivated for a single environment with the following macros by
placing them right before the environment. These macros also influence
local declarations like \verb+\ccStruct+ within the class environment.
Since \verb+\ccFunction+ denotes global functions, they are not
involved.  For this purpose and for global declarations the
\verb+\ccHtmlNoLinks+ deactivates the automatic cross linking and
\verb+\ccHtmlNoIndex+ deactivates the automatic index generation of
the declaration following these macros.

\index{table of contents}
\ccIndexEntry{ccHtmlNoClassToc}
\ccIndexEntry{ccHtmlNoClassFile}
\ccIndexEntry{ccHtmlNoClassLinks}
\ccIndexEntry{ccHtmlNoClassIndex}
\ccIndexEntry{ccHtmlNoLinks}
\ccIndexEntry{ccHtmlNoIndex}
%
\begin{tabbing}
  M \= CCimplementationNNNMM \= ImplementationMMMMM \= \kill
  \> \verb+\ccHtmlNoClassFile+ \> deactivates the creation of an own
                                   file\footnotemark.\\
  \> \verb+\ccHtmlNoClassLinks+ \> deactivates the cross linking for
                                     this class name.\\
  \> \verb+\ccHtmlNoClassToc+   \> no entry into the table of contents
                                     for this class. \\
  \> \verb+\ccHtmlNoClassIndex+ \> no index entries for this class. \\
  \> \verb+\ccHtmlNoLinks+      \> no cross linking for the following
                                   declaration. \\
  \> \verb+\ccHtmlNoIndex+      \> no index entry for the following
                                   declaration.
\end{tabbing}

\footnotetext{A class without its own file will be formatted in the
  enclosing chapter file. The embedded layout within this chapter file
  may be not be optimal and might be customized with the {\tt HTML} 
  specific macros.}

The \verb+ccHtmlClassFile{+{\em filename\/}\verb+}{+{\em
    desc\/}\verb+}+ environment enclose parts of the manual that are
written to their own file with name {\em filename}. The parameter {\em
  desc\/} contains a descriptive text that will be placed in the
anchor in the chapter file and in the table of contents to refer to
this new file. Note that class files cannot be nested and neither
 can this file. Within this environment new class environments
will automatically be stopped to create own files.  \index{htmlclassfile
  environment@{\tt ccHtmlClassFile} environment} \index{class files}

\begin{tabbing}
  M \= CCimplementationNNNMM \= ImplementationMMMMM \= \kill
  \> 
  \verb+\begin{ccHtmlClassFile}{My_point.html}{Declaration of \ccc{My_point}}+
     \\ 
  \> \ldots\\
  \> \verb+\end{ccHtmlClassFile}+
\end{tabbing}

\ccIndexEntry{ccHtmlCrossLink}\index{crosslinking} The macro
\verb+\ccHtmlCrossLink{+{\em C-idfier\/}\verb+}+ activates the
automatic cross linking for the given C identifier {\em C-idfier}. The
generated links will point to the place where this macro is placed.
The following example demonstrates the explicit coding to achieve the
default cross linking for template classes including the option that
the template argument {\tt R} is captured with the anchor's text if
possible.

\begin{tabbing}
  M \= CCimplementationNNNMM \= ImplementationMMMMM \= \kill
  \> \verb+\ccHtmlCrossLink{My_point}+ \\
  \> \verb+\ccHtmlCrossLink{My_point<R>}+
\end{tabbing}

The index is organized in categories. The macros
\verb+\ccHtmlIndex[+{\em category\/}\verb+]{+{\em key\/}\verb+}+ and
\verb+\ccHtmlIndexC[+{\em category\/}\verb+]{+{\em C-idfier\/}\verb+}+
have an optional parameter to state the category for the {\em key\/} or
the {\em C-idfier}. If the optional argument is missing the entry will
be made for a class name. Possible categories are: {\tt
  class, nested\_type, struct, enum, enum\_tags, typedef, variable,
  function,} and {\tt member\_function}.  An index entry will point to
the place where its generating macro is placed.  The difference between
both macros is that \verb+\ccHtmlIndexC+ parses C code in its
argument {\em C-idfier}.
\ccIndexEntry{ccHtmlIndex}\ccIndexEntry{ccHtmlIndexC}\index{index}

\begin{tabbing}
  M \= CCimplementationNNNMM \= ImplementationMMMMM \= \kill
  \> \verb+\ccHtmlIndex[function]{Style guides for functions}+ \\
  \> \verb+\ccHtmlIndexC{My_point<R>}+
\end{tabbing}

% -------------------------------------------------------
\section{Manual Structure and Induced Output Files}

\index{document structure}\index{structure of document}\index{online
  manual} The tools for the conversion to an online {\tt HTML} manual
impose certain restrictions on the \LaTeX\ specification. Each chapter
and each reference page or class defining environment ({\tt ccRef}{\em
  Category}, {\tt ccClass}, {\tt ccClassTemplate}, and {\tt
  ccHtmlClassFile} from {\tt cc\_manual.sty}, see~\cite{k-clswr-99})
opens a new file. For a chapter, the new file gets the filename of the
current \LaTeX\ file with the prefix {\tt Chapter\_} and the file
suffix {\tt .html}. For a reference page, the new file gets the
filename of the item documented in this page with a prefix according
to its {\em Category} (avoiding name clashes) and the file suffix {\tt
  .html}.  For a class defining environment, the new file gets the
name of the class as filename with the postfix {\tt .html} added.
These conventions make hyper-linking easier.  A set of requirements and
recommendations follow from these conventions.

\begin{itemize}
  \item
    All files in one directory.
  \item
    No two chapters in one file. A chapter macro should be located at
    the beginning of a file.
  \item
    No nesting, i.e., no chapter or class definition within a class
    defining environment. (The conventions impose a two-level file
    hierarchy.) Class defining environments within a reference page
    are allowed, but they will not create a new file in this case.
  \item
    No two classes, concepts, etc., of the same name in one manual.
\end{itemize}

Section~\ref{sectionHtmlExtended} describes an extended conversion
model using multiple directories to overcome the filename problem.
Section~\ref{sectionHTMLsupport} explains how the file generation of
reference pages and class environments can be customized with
\verb+\ccHtmlNoClassFile+ and the environment {\tt ccHtmlClassFile} to
gain more flexibility.

\index{title page}\index{show\_main@{\tt -show\_main}}
By default, a \LaTeX\ file is assumed to start with a chapter macro or
a class defining environment. Any pieces of \LaTeX\ code in front of
it are most probably definitions and not meant for conversion. For
example a complete manual starts with a title page and the table of
contents. These are provided by other means for the online {\tt HTML}
manual, i.e., the title page, {\tt title.html}, has to be written
manually, and the table of contents, {\tt contents.html}, is
automatically generated.  In the case of a file without a chapter
macro or a class defining environment the conversion result for this
part can be directed to the stdout using the command line option {\tt
  -show\_main}, see below.


% -------------------------------------------------------
\section{Conversion Tools}

The conversion of a specification to an online {\tt HTML} manual is a
two step process. In the first step, all \LaTeX\ files are converted
to {\tt HTML} files. Meanwhile auxiliary information about
\Dindex{hyper-links} are collected.  In the second step, the hyper-links
are added to all {\tt HTML} files.  The {\tt csh}-script {\tt
  cc\_manual\_to\_html} integrates the first and the second step. It
uses the program {\tt cc\_extract\_html}, which does the first part,
the program {\tt flex}, which generates the filter program to add the
hyper-links, the standard C compiler {\tt cc}, and the program {\tt
  cc\_index\_sort}, which sorts the index case-insensitive. The
synopsis for the script {\tt cc\_manual\_to\_html} is:

\ind{\tt cc\_manual\_to\_html [<options>] <tex-files\ldots>}

This command translates all LaTeX files in its command-line to {\tt
  HTML} files. All files included with the \verb+\input+ or
\verb+\include+ macros of \LaTeX\ are converted as well. The converter
{\em does not\/} understand the \verb+\includeonly+ macro. In summary,
the converter can translate a complete manual at once if it is called
only with the main \LaTeX\ file.
\Mindex{input}\Mindex{include}\Mindex{includeonly}

\index{bibliography}\TTindex{BiBTeX}\index{auxiliary file}
\Mindex{bibcite}\Mindex{bibitem}
The converter supports the generation of an {\tt HTML} bibliography
called {\tt biblio.html}. If the bibliography is generated using
BiB\TeX\ the {\tt *.bbl} file should be given to the converter.  If
the bibliography is written by hand using the \verb+thebibliography+
environment the file containing this environment also should be in the
list of conversion files.  In both cases, the related {\tt *.aux} will
contain \verb+\bibcite+ entries that are needed to generate proper
labels instead of the internally used citation keys. However, if the
bibliography is written by hand with optional arguments in the bibitem
entries the {\tt *.aux} file is not necessary (these optional
arguments also will be used by \LaTeX\ to label the citations).  The
\verb+\+\verb+ref+ macros in \LaTeX\ will be instrumented with
\Dindex{hyper-links} to the bibliography.  A table of contents {\tt
  contents.html} and an index {\tt manual\_index.html} are always
automatically generated. A handwritten title page {\tt title.html} is
assumed, but not supplied.  A common page layout on all pages
guarantees easy navigation through the different files.
\index{title page}\index{table of contents}\index{index}

Available command line options for {\tt cc\_manual\_to\_html} are:

\begin{tabular}{ll}
    {\tt -defaults}        & show the settings of the internal variables.\\
    {\tt -extended}        & extended conversion model, see
                                Section~\ref{sectionHtmlExtended} \\
    {\tt -ref\_manual}     & for manuals split into user and reference parts, 
                               see Section~\ref{sectionHtmlExtended} \\
    {\tt -show\_main}      & print the translation result for the main
                               file to stdout.\\
    {\tt -date <text>}     & set a date for the manual. Default: system date.\\
    {\tt -release <text>}  & set a release number for the
                             manual. Default: empty\\
    {\tt -title <text>}    & set a title text for the manual.\\
                           & Default: ``Reference Manual''.\\
    {\tt -author <text>}   & set an author address (email) for the manual. \\
                           & Default: ``The \cgal\ Project''. \\
    {\tt -config <dir>} & path to the configuration files.\\
    {\tt -tmp <dir>}    & temporary directory to keep all intermediate files.\\
    {\tt -header <dir>} & set the path where the C headers are.\\
    {\tt -o <dir>}    & output directory for the generated {\tt HTML} manual.\\
    {\tt -aux <file>} & \LaTeX\ auxiliary file with the
                                \verb+\bibcite+ entries.\\
    {\tt -bbl <file>} & bibliography file produced by BiB\TeX.\\
    {\tt -sty <style>} & use style file.\\
    {\tt -quiet}      & no output, no warnings for unknown macros.\\
    {\tt -macrodef}   & Print trace messages for macro definitions.\\
    {\tt -macroexp}   & Print trace messages for macro expansions.\\
    {\tt -stymacro}   & Print trace messages for style macros as well.\\
    {\tt -stacktrace} & Print stack trace for each error.\\
    {\tt -h, -help}   & A short usage message and a summary of the options.
\end{tabular}

\TTindex{TMP}\TTindex{TEMP}\index{environment variable} 

The default settings of the internal variables can be viewed with {\tt
  -defaults}. The result of other options is also shown when applied
to the left of {\tt -defaults}. The option {\tt -show\_main} has
already been explained above.

A moderate kind of customization is possible with the four options
{\tt -date, -release, -title,} and {\tt -author}. The manual creation
date, its release number, the manual title, and the authors address or
email might decorate each page, depending on the page layout.
Internal variables are set to the value of the second parameter of
these command line options.  The header and footer files (see below)
make use of these variables.

The default output directory is the current directory.  The default
temporary directory is taken from the environment variables {\tt
  TMP} or {\tt TEMP}. If these are empty, {\tt /usr/tmp} is tried. If
not available, the local directory is chosen. In all cases, a
subdirectory {\tt extract\_html\_tmp\_\$}\{{\tt USER}\} is created within
the temporary directory.  It keeps all intermediate files. It will be
removed after the conversion.  The standard layout of the {\tt HTML}
pages can be customized with own header and footer files. The default
files are provided in a the directory that can be changed with the
option {\tt -config}.

The \verb+\ccInclude+ macro from {\tt cc\_manual.sty} is converted to a
hyper-link referencing the original header files if the path to the
header files is properly predefined in the conversion script or given
with the {\tt -header} option.\index{include directory}

\TTindex{DEBUG}\index{environment variable} 
In addition to the command line options another debug technique is
implemented in the {\tt csh}-script. If the environment variable DEBUG
is set the script echos each command executed. The {\tt rm} commands
to remove files are echoed with a \# in front. The temporary files are
not removed in DEBUG mode in order to examine them later.

% -------------------------------------------------------
\section{Getting input from other places using {\tt LATEX\_CONV\_INPUTS}} 
\label{sectionLatexConvInputs}

Beginning with release 3.6, the converter supports the use of an 
environment variable, \verb|LATEX_CONV_INPUTS|, which is analogous
to the \verb|TEXINPUTS| environment variable used with \LaTeX.  This
variable is used by the \verb|cc_extract_html| program to find input
files (included, for example, using the \verb|\input| or \verb|\include| 
commands, or their derivatives).  It should contain a colon-separated
list of directories in which the input files can be found.  If this variable
is not set, the default is that files are looked for in the current
directory (meaning the directory where \verb|cc_extract_html| is being
run).  If it is set, directories 
are searched in the order provided in this variable (which means you should
include \verb|"."| somewhere in the list if you want it to find files in the
current directory).
\index{LATEX_CONV_INPUTS@{\tt LATEX\_CONV\_INPUTS}}
\Mindex{input}\Mindex{include}

% -------------------------------------------------------
\section{Converting a Manual Using Multiple Directories}
\label{sectionHtmlExtended}

\index{subdirectories}\index{multiple directories}

The restriction to keep all files in one directory is weakened with
the capabilities of the {\tt cc\_manual\_to\_html} script. Use the
{\tt -extended} option to activate it. However, the actual converter
{\tt cc\_extract\_html} has not changed and does not support
subdirectories. Similar to \TeX, subdirectories in \verb+\input+ and
\verb+\include+ macros do not cause the converter to change its
local working directory while processing the included file.  Thus,
also similar to \TeX, if relative paths are provided, they should be
relative to the place where the {\tt cc\_extract\_html} command is
run, not relative to the place where the file containing the command
is.  

With the introduction of the {\tt LATEX\_CONV\_INPUTS} variable 
(Section~\ref{sectionLatexConvInputs}), it is now also possible 
for {\tt cc\_extract\_html} to find input files in other directories 
without the directory being specified in the body of the source file.  
However, if the {\tt -extended} option is used, it is NOT 
necessary (and probably undesirable) to include the list of all the
subdirectories containing the various files in the variable 
{\tt LATEX\_CONV\_INPUTS}; {\tt cc\_manual\_to\_html} attaches each
separate subdirectory to the existing {\tt LATEX\_CONV\_INPUTS} before 
processing the subdirectory.  This allows different subdirectories 
to have files with the same name. 
\index{LATEX_CONV_INPUTS@{\tt LATEX\_CONV\_INPUTS}}

The file structure proposed here is as follows. The manual is split
into different parts, each part in its own subdirectory. Each part
could be converted on its own using the non-extended conversion mode, e.g., no
additional subdirectories. Each part has its own main \LaTeX\ file
that includes all other files of this part. (It might be necessary to
have a separate main file to create the manual using \LaTeX.) The {\tt
  cc\_manual\_to\_html} script is called in the common parent
directory with all main files from all parts at once.  The main files
are given with their relative paths. The bibliography files must be
supplied to the script using the {\tt -bbl} and {\tt -aux} options
(instead of giving the {\tt *.bbl} file directly as a \LaTeX\ file).

For manuals with their source files split into subdirectories in this
way and for which one set of subdirectories corresponds to a user manual
and the other set corresonds to a reference manual, the {\tt -ref\_manual}
option should be used in conjunction with {\tt -extended}.  This will 
cause the two manuals to have a common table of contents and to be
easily hyperlinked together.  All input files following the {\tt -ref\_manual}
option will be considered part of the reference manual.  Thus, the first 
input file following this option should contain the 
{\texttt \\part\{<part title>\}} command giving the title of the reference 
manual.\index{ref_manual@{\tt ref\_manual}}\index{part@{\tt \protect$\backslash$part}}. 

The script converts each subdirectory independently and generates the
automatic hyper-links locally. It then collects the table of contents,
index files, and hyper-linking rules globally and applies the
hyper-linking rules once again on all files. This way, local classes
gets hyper-linked within their part without naming conflicts with other
parts and more global classes will nevertheless be referenced from
other parts. The conversion script will issue lots of warnings for
duplicated hyper-linking rules for classes that exists in multiple
parts. These warnings can be ignored since the locally created links
will not be changed any more, they are stable during the global
processing.

The result of the conversion will be a duplicated subdirectory
structure in the output directory, where each part is again in its own
subdirectory. The table of contents, the index, and the bibliography
are located directly in the output directory (with some other
auxiliary files).

\index{example} 
In the source code distribution is an example manual with three parts
in the directory {\tt Tools/src/test\_html/extended} which can be converted
with the script {\tt convert} located in the same directory. This
script calls the {\tt cc\_manual\_to\_html} converter with the
appropriate parameters for the example.  There is a second example in
{\tt Tools/example/test\_html/extended\_with\_ref} that shows the conversion
of a user manual and reference manual using the {\tt -ref\_manual} 
option.  Again, the script {\tt convert} in this directory does the
conversion.  Further examples can be found in the \texttt{Tools/examples}
directory.

% -------------------------------------------------------
\section{Conversion Capabilities and Unsupported Features}

A note on the formatting quality. The converter repeats the formatting
task that the {\tt cc\_manual.sty} does in \LaTeX~\cite{k-clswr-99}.
The converter cannot be as accurate nor flexible than the \LaTeX\ 
style. A couple of customizations will not work that easily with HTML,
for example the column widths are not parameterized and may depend on
the browser.  However, these converter converts a lot, including
itemized lists, enumerations, tables, arrays, fractions, \ldots.  If
this is not enough, Section~\ref{sectionHTMLsupport} explains macros
to write certain passages directly in {\tt HTML}.

\index{three-column layout} 
In the case of formatting own pieces in the three- or two-column
layout it is recommended to copy an example from the output of the
{\tt cc\_extract\_html} program rather than from the {\tt
  cc\_manual\_to\_html} script, since the script does certain
optimizations in the table layout. Most notable, each declaration is
formatted as its own table, and the script glues adjacent tables with
equal column number together. To accomplish this, each declaration
table starts with {\tt <!3><TABLE\ldots} or {\tt <!2><TABLE\ldots},
and ends with {\tt\ldots </TABLE><!3>} or {\tt\ldots </TABLE><!2>}.
The gluing works only for matching {\tt <!3>} or {\tt <!2>} pairs and
if nothing than whitespace separates them.

\Mindex{kill}
The {\tt tabular} and {\tt tabbing} environments are converted to
standard {\tt HTML} tables. Most likely, the conversion fails to
produce an appealing layout for non-regular {\tt tabbing}
environments. If it is quite regular but only the first table defining
line uses the \verb+\kill+ possibility of \LaTeX, this line can
removed from the conversion with the \verb+\ccTexHtml{}{}+ macro from
Section~\ref{sectionHTMLsupport}. The {\tt ccTexOnly} environment will
not work in this case.
\index{tabbing environment}\index{tabular environment}

Footnotes are managed. They are printed at the end of each file and
referenced with hyper-links.\index{footnotes}

The latest and now most fundamental technique in the converter is the
capability of expanding macros and defining new macros including
macros with parameters.  This works for user definitions with
\verb+\newcommand+, \verb+\newevironment+, and simple cases of
\verb+\def+ definitions. The \verb+\RCSdef+ and \verb+\RCSdefDate+
macros are also recognized.
\index{macro expansion}\Mindex{newcommand}\Mindex{newenvironment}\Mindex{def}
\Mindex{RCSdef}\Mindex{RCSdefDate}

The {\tt flushrigth} environment and the arguments to the {\tt
  tabular} environment are not fully supported. Graphics and color are
not supported. The following environments of \LaTeX\ are not
supported.

\ind \verb+list+\hfill
\verb+trivlist+\hfill
\verb+verbatim*+\hfill
\verb+theindex+\hfill{}

The converter does not support the following macros and commands of \LaTeX.

\begin{tabbing}
  M \= \hspace*{0.24\textwidth} \= \hspace*{0.24\textwidth} 
    \= \hspace*{0.24\textwidth} \=  \kill
  \>\verb+\filecontents+ \>    \verb+\listfiles+ \>
    \verb+\addcontentsline+ \>    \verb+\addtocontents+\\
  \>\verb+\verb*+ \>    \verb+\underline+ \>
    \verb+\overline+ \>    \verb+\widehat+\\
  \>\verb+\widetilde+ \>    \verb+\imath+ \>
    \verb+\jmath+ \>    \verb+\stackrel+\\
  \>\verb+\ifthenelse+ \>    \verb+\whiledo+ \>
    \verb+\kill+ \>    \verb+\pushtabs+\\
  \>\verb+\poptabs+ \>    \verb+\printindex+ \>
    \verb+\subitem+ \>    \verb+\subsubitem+\\
  \>\verb+\newlength+ \>    \verb+\setlength+ \>
    \verb+\addtolength+ \>    \verb+\settowidth+\\
  \>\verb+\settoheight+ \>    \verb+\settodepth+ \>
    \verb+\rule+ \>    \verb+\raisebox+\\
  \>\verb+\centering+ \>    \verb+\raggedright+ \>
    \verb+\raggedleft+
\end{tabbing}

% -------------------------------------------------------
\section{Configuration Files and Variable Substitution}

\index{variables}\index{customization, HTML} Configurations files are
used to create uniform headers and footers for all HTML files.  The
default files in the distribution are the recommended starting point
for further customizations. These are {\tt cc\_toc\_header}, {\tt
  cc\_toc\_footer}, {\tt cc\_index\_header}, {\tt cc\_index\_footer},
{\tt cc\_manual\_header}, {\tt cc\_manual\_footer}, {\tt
  cc\_biblio\_header}, and {\tt cc\_biblio\_footer}.  A variable
substitution scheme makes some internal variables available in the
header and footer files. Variables start with a `\%' character and are
identified by a single letter. A special option are blocks between
curly braces right behind a variable. If a particular variable is
empty the whole block is omitted. If a particular variable is not
empty the variable notation \% with its single letter and the two
curly braces are removed, leaving the block's body in the file. The
blocks might be nested. The list of variables:

\begin{tabular}{ll}
    {\tt\%0}  &  the program name itself (cc\_manual\_to\_html) \\
    {\tt\%p}  &  the release number of the program. \\
    {\tt\%f}  &  current {\tt HTML} output file name. \\
    {\tt\%c}  &  full class name, including template parameters (if any). \\
    {\tt\%u}  &  file name of the {\tt HTML} file of the chapter wherein a
    class defining envi- \\ & ronment is used (and is responsible for the
    {\tt HTML} file we are in at the\\ &  moment). This filename is empty
    in all other situations. It is used \\ & to build an up-link from a
    class definition back to its chapter file. \\
    {\tt\%d}  &  a creation date for the manual.\\
    {\tt\%r}  &  a release number for the manual.\\
    {\tt\%t}  &  a manual title. \\
    {\tt\%a}  &  an author address or email.\\
    {\tt\%\%}  &  escape sequence for {\tt\%}.\\
    {\tt\%}\{{\tt , \%}\}  &  escape sequences for \{ or \}.
\end{tabular}

The variables which might be empty are {\tt\%c, \%u, \%d, \%r, \%t},
and {\tt\%a}.  An example for a block between parentheses is {\tt
  \%r}\{{\tt \%r<BR>}\}. It formats the release number in its own line.
Whenever the release number is empty the superfluous line break
is omitted.

% -------------------------------------------------------
\section{Style File and Internal Macros of the Converter}

The converter is basically a macro expansion engine and its major data
structure is a hash table for macro definitions. In consequence, the
major implementation of the various \LaTeX, \TeX, and other style file
macros is put into style files of the converter, usually located at
{\tt \$\Open LATEX\_CONV\_CONFIG\Close/html/*.sty}. Only a small core
part is realized in the \CC\ implementation together with a {\tt flex}
scanner and a {\tt bison/yacc} parser. An important note: All macros
presented in this section work only for the converter and not for
\LaTeX\ itself. So they can only be used to extend the converter with own
definitions and style files.
\TTindex{flex}\TTindex{bison}\TTindex{yacc}
\index{parser}\index{scanner}
\index{style files}

For macro expansions, two key characteristics have to be defined: A
macro body is expanded when the macro is used, not when it is defined.
Arguments to the macro are pasted literally into the macro body
without expanding the arguments. However, this is not sufficient to
implement \TeX's functionality and we have built in another argument
expansion possibility.

Since we tried to keep the converter reasonably small, we have not
fully implemented the \TeX\ scanner with its catcodes\index{catcode}.
In particular, the \TeX\ scanner causes some strange behaviors with
\Dindex{parsing} spaces\index{parsing space}\index{space, parsing}.
For example, after macros, spaces are silently ignored. We cope with
spaces by introducing artificial delimiters, {\tt Ctrl-A}, at block
boundaries. These delimiters will never appear in user output. In some
places, macro implementations explicitly have to get rid of
surrounding spaces of arguments. We have built in another argument
expansion possibility that does this.

\LaTeX\ has added optional arguments in angle brackets to \TeX\ 
macros. The converter has a simple notion of implementing macros with
optional arguments. A name mangling scheme gives different names to
all possible macros with optional arguments. The original macro name
is extended with the letter `@' and for each optional argument a
letter `o' and for each mandatory argument a letter `m'. Each
combination of optional and mandatory argument is therefore a new
macro with its own body. Optional and mandatory arguments are numbered
as usual from left to right, but now including the optional arguments.
The macro without any optional argument is defined without the mangled
name. It must always be defined and it must be the first definition.
The additional macros with optional arguments are defined with the
mangled name. The number of arguments is omitted in the definition,
but the number of mandatory arguments must be equal to the number of
arguments specified earlier. An example:
\index{optional arguments}\index{mandatory arguments}\index{macro arguments}
\index{arguments, macro}

\begin{tabbing}
  M \= implementationNNNMMMMMMMMMMMMMMMMMMM \= \kill
  \> \verb+\newcommand{\a}[1]{{\bf #1}}+ \\
  \> \verb+\newcommand{\a@om}{#1{\bf #2}}+ \\
  \> \verb+\newcommand{\a@mo}{{\bf #1}#2#2}+ \\
  \> \verb+\a{text}+         \> {\bf text}\\
  \> \verb+\a[pre]{text}+    \> pre{\bf text}\\
  \> \verb+\a{text}[post]+   \> {\bf text}postpost
\end{tabbing}

The name mangling scheme is also used to encode environments {\em
  env\/} in the same macro dictionary using the names
\verb+\begin@+{\em env\/} and \verb+\end@+{\em env}. Thus,
environments with optional arguments are just the straightforward
combination with the name mangling described for macros. An example:

\begin{tabbing}
  M \= implementationNNNMMMMMMMMMMMMMMMMMM \= \kill
  \> \verb+\newenvironment{b}{Open\\}{Close}+ \\
  \> \verb+\newcommand{\begin@b@o}{Open #1\\}{Close}+ \\
  \> \verb+\begin{b}[Opt]+        \> Open Opt\\
  \> \verb+\end{b}+               \> Close
\end{tabbing}

Single characters can be defined to act as macros, an example is
`\verb+~+'.  They are defined using the usual \verb+\newcommand+
definitions. Internally, a table for single character commands instead
of the hash table speeds up the access.\index{single character commands}

Two extensions of the plain macro expansion mechanism have been
explained above: expanding a macro argument right at the time of
replacement, and removing whitespace around an argument. Together
with two more extensions they are encoded with the argument token of
\TeX, {\tt \#}. Usually, the argument number, 1 to 9, follows
immediately. Here, we allow one or several of the following key
capitals to follow {\tt \#} before the digit comes. To extend the
limited numbers of arguments (0 to 9), we allow also the small letters
`a' to `z' to denote the arguments of number 10 to 25.\index{extensions}
\index{whitespace}\index{crop}\index{skip}\index{expand}\index{length}

\begin{tabbing}
  M \= impM \= expandM:: \= \kill
  \> \verb+X+ \> expand: \> if the argument starts with a parameterless macro 
                            it is\\ \> \> \> immediately expanded.\\
  \> \verb+S+ \> skip:   \> skip the fist character of the argument.\\
  \> \verb+C+ \> crop:   \> removes whitespace on both sides of the 
                            argument.\\
  \> \verb+L+ \> length: \> gives the length of the argument.
\end{tabbing}

The extensions have to be given in the order shown in this table. An
example of a useful combination is {\tt \#XC1} that skips whitespaces
before expanding a possible macro. The length is useful for some
internal macros that accept as the first parameter the actual length
of the second parameter.  These internal macros just do not parse the
second argument and ignore therefore block braces therein, which is
what one needs for verbatim output. The skip option is useful to get
rid of unwanted prefixes, for example a `{\tt \Backslash}'.

We continue with a short description of the internal macros recognized
by the converter without reading any style file. After some initial
setup the converter reads the style file {\tt default.sty} which in
turn includes all style files currently supported. Before \LaTeX\ is
addressed with the style file {\tt latex.sty}, the style file {\tt
  latex\_converter.sty} is included. It extends the base functionality
of the converter and it is always recommended to include this file
first.

% -------------------------------------------------------
\subsection{Internal Macros of the Converter}
\index{internal macros}

The following list of internal macros is initialized and implemented
in the module {\tt internal\_macros.C}. It is not the complete list of
internal macros, many more (yet not documented) are recognized in the
lexical scanner phase in {\tt html\_lex.yy}.

\begin{description}
  %\plainitem{\Open}
  %  Maps to \verb+\begingroup+.
  %\plainitem{\Close}
  %  Maps to \verb+\endgroup+.
  \macroitemXX{newcommand}{macro}{body}
    Defines {\em macro\/} to expand to {\em body}.
  \macroitemXOX{newcommand@mom}{macro}{number}{body}
    Defines {\em macro\/} with {\em number\/} many arguments to expand to
    {\em body}.
  \macroitemXX{def}{macro}{body}
    Defines {\em macro\/} to expand to {\em body}.
  \macroitemXX{gdef}{macro}{body}
    Globally defines {\em macro\/} to expand to {\em body}.
  \macroitemX{lciUndef}{macro}
    Removes definition of {\em macro\/} from the dictionary.
  \macroitem{lciBeginScope}
    Opens a new scope for macros.
  \macroitem{lciEndScope}
    Closes a scope for macros removing all macro definitions that 
    have been (not globally) defined in the current scope.
  \macroitemXXX{lciIfDefined}{macro}{if-text}{else-text}
    Expands to {\em if-text\/} if {\em macro\/} is defined, and to
    {\em else-text\/} otherwise.
  \macroitemXXXX{lciIfEqual}{arg1}{arg2}{if-text}{else-text}
    Expands to {\em if-text\/} if {\em arg1\/} is equal to {\em arg2}, and to
    {\em else-text\/} otherwise.
  \macroitemXXXX{lciIfLess}{num1}{num2}{if-text}{else-text}
    Expands to {\em if-text\/} if {\em num1\/} is less than {\em num2}, and to
    {\em else-text\/} otherwise.
  \macroitemXXXX{lciIfLessOrEqual}{num1}{num2}{if-text}{else-text}
    Expands to {\em if-text\/} if {\em num1\/} is less than or equal 
    to {\em num2}, and to {\em else-text\/} otherwise.
  \macroitemXX{lciAddTo}{macro}{num}
    Redefines {\em macro\/} to be the sum of its value with {\em num}.
  \macroitemXX{lciMultTo}{macro}{num}
    Redefines {\em macro\/} to be the product of its value with {\em num}.
  \macroitemXX{lciGlobalAddTo}{macro}{num}
    Globally redefines {\em macro\/} to be the sum of its value with {\em num}.
  \macroitemXX{lciGlobalMultTo}{macro}{num}
    Globally redefines {\em macro\/} to be the product of its value 
    with {\em num}.
  \macroitemX{lciToUpper}{text}
    Expands {\em text\/} to all upper-case letters.
  \macroitemX{lciError}{ascii-text}
    Prints {\em ascii-text\/} as error message and terminates conversion.
  \macroitemX{lciMessage}{ascii-text}
    Prints {\em ascii-text\/} as a message to the user on {\tt cerr}
    and continues with the conversion.
  \macroitemX{lciStyle}{C++-text}
    Formats {\em C++-text} for HTML.
  \macroitem{lciCCParameter}
    \CC\ parameters are parsed separately and stored in an internal state.
    The \CC\ text in this internal state is returned accessed with this macro.
  \macroitemX{lciTwoColumnLayout}{decl-category}
    Formats two-column layout using the \CC\ test in the internal \CC\ text
    (see \verb+\lciCCParameter+) and {\em decl-category} as the category 
    to format.
  \macroitemXX{lciThreeColumnLayout}{decl-category}{bool}
    Formats three-column layout using the \CC\ test in the internal \CC\ text
    (see \verb+\lciCCParameter+) and {\em decl-category} as the category 
    to format. The {\em bool\/} parameter tells whether the comment that 
    is supposed to follow will be empty or not.
  \macroitemXX{lciHtmlIndex}{category}{text}
    Inserts {\em text\/} in the index under {\em category}.
  \macroitemX{lciHtmlIndexC}{category}
    Inserts the internal \CC\ text (see \verb+\lciCCParameter+) in the 
    index under {\em category}.
  \macroitemX{lciHtmlCrossLink}{dummy}
    Activates the creation of hyper-links for the identifier kept in the 
    internal \CC\ text (see \verb+\lciCCParameter+). The {\em dummy\/}
    argument is not used.
  \macroitem{lciHtmlClassFileEnd}
    Triggers file postprocessing.
  \macroitem{lciPopOutput}
    Pops current output stream.
  \macroitemX{lciPushOutput}{key}
    Pushes current output stream and makes a new stream current.
    The new stream is identified by the {\em key\/} from a fixed set.
  \macroitem{lciOpenBibliography}
    Opens bibliography file and writes header.
  \macroitem{lciCloseBibliography}
    Closes bibliography file and writes footer.
  \macroitem{lciLineNumber}
    Expands to the current line number of the current input file.
\end{description}



% -------------------------------------------------------
%\subsection{Style File {\tt latex\_converter.sty}}
%\index{style file}


% =====================================================
\newpage
\bibliographystyle{plain}
%\bibliography{kettner}
\bibliography{Manual_tools/manual}

\small
\printindex

\end{document}

