% +------------------------------------------------------------------------+
% | main.tex
% +------------------------------------------------------------------------+
% | chapter title, some introduction and a kind of table-of-contents
% +------------------------------------------------------------------------+

\chapter{Example User Manual}
\ccChapterRelease{\ccRevision. \ \ccDate}\\
\ccChapterAuthor{Lutz Kettner and Susan Hert}

This small example illustrates the creation of a single manual with different
chapters for the user manual and reference manual parts.  Creation of the
manual with \LaTeX\ is done via
\begin{verbatim}
   latex manual.tex
\end{verbatim}
The HTML version is created with
\begin{verbatim}
   cc_manual_to_html -o html manual.tex
\end{verbatim}
Notice that this simple conversion is possible with HTML because the two
chapter commands appear in files with differnet names (\texttt{main.tex} 
in subdirectory \texttt{Example} and \texttt{intro.tex} in subdirectory
\texttt{Example\_ref}).  If the chapter command for the reference pages
were in the file \texttt{Example\_ref/main.tex}, one would have to use
the \texttt{-extended} option of \texttt{cc\_manual\_to\_html} to avoid
name collisions.  See the directory \texttt{../one\_manual\_ext\_html} for an
example using this option.  The directory \texttt{../two\_manuals} contains
an example of the creation of a single document containing separate
user and reference manuals.
