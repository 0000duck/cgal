% +------------------------------------------------------------------------+
% | Reference manual page: is_empty_range.tex
% +------------------------------------------------------------------------+
% | 03.09.1999   Lutz Kettner
% | Package: Example
% | 
\RCSdef{\RCSisemptyrangeRev}{$Id$}
\RCSdefDate{\RCSisemptyrangeDate}{$Date$}
% |
%%RefPage: end of header, begin of main body
% +------------------------------------------------------------------------+

\gdef\ccRefPageBegin{\ccParDims\cgalColumnLayout\begin{ccAdvanced}}
\gdef\ccRefPageEnd{\ccParDims\cgalColumnLayout\end{ccAdvanced}}
\begin{ccRefFunction}{is_empty_range}  %% add template arg's if necessary

%% \ccHtmlCrossLink{}     %% add further rules for cross referencing links
%% \ccHtmlIndexC[function]{} %% add further index entries

\ccDefinition
  
The function \ccRefName\ does this and that.

\ccInclude{DIR/is_empty_range.h}

\ccGlobalFunction{void is_empty_range();}

\ccIsModel

\ccc{Concept}

\ccSeeAlso

\ccc{Some_other_class},
\ccc{some_other_function}.

\ccExample

A short example program.
Instead of a short program fragment, a full running program can be
included using the 
\verb|\ccIncludeExampleCode{examples/Example/is_empty_range_prog.C}| 
macro. The program example would be part of the source code distribution and
also part of the automatic test suite.

\begin{ccExampleCode}
void your_example_code() {
}
\end{ccExampleCode}

%% \ccIncludeExampleCode{examples/Example/is_empty_range_prog.C}

\end{ccRefFunction}
\gdef\ccRefPageBegin{\ccParDims\cgalColumnLayout}
\gdef\ccRefPageEnd{\ccParDims\cgalColumnLayout}


% +------------------------------------------------------------------------+
%%RefPage: end of main body, begin of footer
% EOF
% +------------------------------------------------------------------------+

