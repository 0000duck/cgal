	
\ccUserChapter{2D Complex embedded in a 3D Triangulation\label{chapter_Complex2inTriangulation3}}
\label{chapter_C2inT3}


\vspace*{-15mm}
\minitoc
\vspace*{30mm}

\section{Introduction\label{section_C2inT3_Intro}}
This package provides a data structure to  represent a
2D simplicial complex as a subcomplex of a 3D triangulation.
By 2D simplicial complex we mean a simplicial complex of dimension at
most 2. The faces of the 2D complex are called
facets, edges and vertices according to their dimension.
The represented 2D complex is required has to be simplicial 
in order to appear 
as a subcomplex of a 3D triangulation, however 
it does not need to be connected, nor pure.
This   means that the complex may have antenna edges, that is 
edges which do not belong to a facet  or 
isolated vertices that is vertices which do not belong an edge.
%Antenna edges are also called isolated edges in the following.

The data structure enables to mark certain edges of the 2D complex.
This feature can be used in particular
to mark constrained edges and enables 
to represent as well non simplicial 2D complexes 
in a conforming 3D triangulation.
If the 3D triangulation is conforming any non triangular facet
of the 2D complex appear as a union of facets of the 3D triangulation.
The 2D embedded complex represents this triangulated version of the
2D complex and marked edges are used to distinguish  the original
edges  of the complex from edges used to triangulate
the non simplicial facets of the complex.

The data structure enables also to attach some information to the
facets of the 2D complex and to antenna or marked edges.

The package can for example be used to represent 3D constrained
triangulations where the constraints facets form a piecewise linear
complex (PLC) as described in \cite{s-cgehd-98}.

Another typical application of the data structure is surface
reconstruction. Many reconstruction algorithms
build a 3D triangulation and extract from there
a 2D simplicial complex approximating the surface to be reconstructed.
Embedding the approximated surface in a 3D
triangulation allows to deal with any kind of topology and to handle
topology changes occurring in the process of reconstruction.
The surface to be reconstructed can be 
a polyhedral surface, a smooth surface or a general curved surface
with singularities and sharp features. See e.g. \cite{prisme-bo-03}.

\section{Representation  and main functionalities\label{section_C2inT3_Functionalities}}

\subsection{Representation}
The 2D complex is represented as a 3D triangulations in which
a subset of facets, edges and vertices are marked as belonging to the 
2D complex.  
A 2D complex is fully described by the set of its maximal faces
where a maximal face of a complex is a face which is not
a subface of a higher dimensional face of the complex.
The data structure maintains explicitly the set of maximal faces 
of the 2D complex, that is 
the set of facets, antenna edges and isolated vertices.

In addition, the data structure also allows to maintain a subset
of the complex edges which are of particular interest to the user.
Those edges are referred to in the documentation as marked
edges.

\subsection{Creation, modification}
The data structure can be created from any 3D triangulation
by specifying the subset of maximal faces of the 2D complex
and eventually the subset of marked edges.

The data structure offers member functions to add or remove faces from the
2D complex and member functions  to add or removed edges
from the subset of marked edges. It also provide member functions
to set or get the information attached to facets in the 2D complex
and  to antenna or  marked edges.

The underlying 3D triangulation remains  available to the user
and can be modified,  however it is the responsibility
of the user, to update the set of faces of the 2D complex
when  it is affected by  the modification of the 3D triangulation.


\subsection{Queries on the complex}
The data structure enables to answer  a certain number of queries
regarding the complex.

Any facet, edge or vertex  of the  3D triangulation 
can be classified as belonging to the 2D complex or not.

Edges of the 2D complex  are furthermore  classified 
as isolated, boundary, regular and singular edges. An edge is said
to be  isolated if it not incident to a facet in the complex
boundary if it is incident to a single facet of the complex,
regular if it is incident to exactly two facets in the complex  and 
singular if it is incident to three or more facets in the complex.

Likewise, the vertices of the complex are classified as  isolated, boundary
regular or singular.  The classification of a vertex is best
described by looking  as the adjacency graph of the set 
of facets of the 2D complex
incidents to that vertex. 
Two facets are connected by an arc in this graph if
they share an edge of the 2D complex. The vertex is isolated
there is no incident facet, regular  if the graph is a cycle,
boundary if the graph is a simple path (which can be reduced to a
single node) and singular in any other case.

In the following we consider that the 2D complex is 
actually two dimensional and pure or we focus
on its pure 2D component 
that is on  the subcomplex formed by the facets and
their subfaces.
The data structure allows to know if this subcomplex is manifold or not
and if it has a boundary or not.  If the pure component is manifold
the number of its connected components as well
as the number of connected components of its boundary.
can be obtained. 
Counting the number of vertices, edges and
facets of each  pure connected component  and the number
of its boundary edges allows to get
 (through Euler formula)
the  topology, that is the number of handles of this component.

 

\subsection{Traversal of the complex}
The data structure provides iterators to enumerate 
the facets of the complex, the isolated vertices and edges.

The basic tool to traverse the 2D  complex is the \ccc{neighbor_facet}
function which provide the facets of the complex adjacent to  a given
facet if any.

The data structure provides circulators to explore the facet of the
complex incidents to a regular or boundary vertex. The circulators 
chooses an orientation around the vertex. In case of a boundary vertex,
after the last incident facet, the circulator will
come back to the first one.

\subsection{Input-output}
CGAL provides an  output operator of the
2D complex into  various streams:
files in format  OFF or VRML, \ccc{CGAL::Geomview_stream}.
The 2D complex can also be output together with its underlying 3D triangulation
in a special internal file format from where it can be reread for later use.



\section{Examples\label{section_C2inT3_Examples}}






