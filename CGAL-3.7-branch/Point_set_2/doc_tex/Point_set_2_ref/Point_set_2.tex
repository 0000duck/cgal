% +----------------------------------------------------------+
%---------------------------------------------
% Point_set_2 data type
%---------------------------------------------
\begin{ccRefClass}{Point_set_2<Gt,Tds>}

\ccDefinition

\ccInclude{CGAL/Point_set_2.h}

An instance $PS$ of the data type \ccRefName\  is a {\em Delaunay Triangulation}
of its vertex set. The class \ccRefName\ is inherited from the \cgal\ Delaunay triangulation,
and provides additional nearest neighbor query operations and range searching operations.

The \ccRefName\ class of \cgal\ depends on template parameters standing for the
geometric traits classes used by the point set and by the Delaunay triangulation (Gt)
and for the triangulation data structure (Tds). 


\ccCreationVariable{PS}


\ccHtmlLinksOff


\ccTypes
\ccThree{typedef Gt::Point_2 Point;}{the point type}{}
\ccThreeToTwo
\ccTypedef{typedef Gt::Point_2 Point;}{the point type}{}
\ccTypedef{typedef Gt::Segment_2 Segment;}{the segment type}
\ccTypedef{typedef Gt::Circle_2 Circle;}{the circle type}
\ccTypedef{typedef Gt::FT Numb_type;}{the representation field number type.}
\ccNestedType{Triangulation::size_type}{the size type of the underlying triangulation.}
\ccNestedType{Triangulation::Vertex}{the vertex type of the underlying triangulation.}
\ccNestedType{Triangulation::Edge}{the edge type of the underlying triangulation.}
\ccNestedType{Triangulation::Vertex_handle }{handles to vertices.}

\ccHtmlLinksOn


\ccCreation

\ccConstructor{Point_set_2();}
{creates an empty \ccRefName\ .}


\ccConstructor{template<class InputIterator>
Point_set_2(InputIterator first, InputIterator last);}
{creates a \ccRefName\ \ccVar\ of the points in the range
[$first$,$last$).}


\ccOperations


\ccHtmlLinksOff


\ccMethod{Vertex\_handle  lookup(Point p);}
{ if \ccVar\ contains a Vertex $v$ with $|pos(v)| = p$ 
the result is a handle to $v$ otherwise the result is $NULL$. }


\ccMethod{Vertex\_handle  nearest_neighbor(Point p);}
{ computes a handle to a vertex $v$ of \ccVar\ that is closest to $p$.
If \ccVar\ is empty, $NULL$ is returned.
}

\ccMethod{Vertex\_handle  nearest_neighbor(Vertex_handle v);}
{ computes a handle to a vertex $w$ of \ccVar\ that is closest to $v$.
If $v$ is the only vertex in \ccVar\ , $NULL$ is returned.
}

\ccMethod{template<class OutputIterator>
OutputIterator   nearest_neighbors(Point p, size_type k, OutputIterator res);}
{ computes the $k$ nearest neighbors of $p$ in \ccVar, and places the
handles to the corresponding vertices as a sequence of objects of type
Vertex\_handle in a container of value type of $res$
which points to the first object in the sequence. The function
returns an output iterator pointing to the position beyond the end
of the sequence. }

\ccMethod{template<class OutputIterator>  
OutputIterator  nearest_neighbors(Vertex_handle v, size_type k,OutputIterator res);}
{ computes the $k$ nearest neighbors of $v$, and places them as a sequence of objects of type
Vertex\_handle in a container of value type of $res$
which points to the first object in the sequence. The function
returns an output iterator pointing to the position beyond the end
of the sequence. }

\ccMethod{template<class OutputIterator>
OutputIterator range_search(const Circle& C, OutputIterator res);}
{ computes handles to all vertices contained in the closure of disk $C$.
The computed vertex handles will be placed as a sequence of objects in a container of value type
of $res$
which points to the first object in the sequence. The function
returns an output iterator pointing to the position beyond the end
of the sequence.
}

\ccMethod{template<class OutputIterator>
OutputIterator range_search(const Point& a, const Point& b, const Point& c,OutputIterator res);}
{ computes handles to all vertices contained in the closure of the triangle $(a,b,c)$.\\
\ccPrecond $a$, $b$, and $c$ must not be collinear. 
The computed vertex handles will be placed as a sequence of objects in a container of value type
of $res$
which points to the first object in the sequence. The function
returns an output iterator pointing to the position beyond the end
of the sequence.
}

\ccMethod{template<class OutputIterator>
OutputIterator range_search(const Point& a1, const Point& b1, const Point& c1,const Point&
d1,OutputIterator res);}
{ computes handles to all vertices contained in the closure of the iso-rectangle $(a1,b1,c1,d1)$.\\
\ccPrecond $a1$ is the upper left point, $b1$ the lower left, $c1$ the lower
right and $d1$ the upper right point of the iso rectangle.
The computed vertex handles will be placed as a sequence of objects in a container of value type
of $res$
which points to the first object in the sequence. The function
returns an output iterator pointing to the position beyond the end
of the sequence.
}


\ccHtmlLinksOn

\end{ccRefClass} 
