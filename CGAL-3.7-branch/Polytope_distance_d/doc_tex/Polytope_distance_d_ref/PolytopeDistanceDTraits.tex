
\begin{ccRefConcept}{PolytopeDistanceDTraits}

% -----------------------------------------------------------------------------
\ccDefinition

This concept defines the requirements for traits classes of $d$-dimensional
optimisation algorithms.

% -----------------------------------------------------------------------------
\ccTypes

\ccSaveThreeColumns
\ccSetTwoColumns{PolytopeDistanceDTraits:: Rep_tag}{}

\ccNestedType{Point_d}{point type used to represent the input points.}

\ccNestedType{Rep_tag}{%
  compile time tag to distinguish between Cartesian and homogeneous
  representation of the input points. \ccc{Rep_tag} has to be either
  \ccGlobalScope\ccc{Cartesian_tag} or
  \ccGlobalScope\ccc{Homogeneous_tag}.}

\ccNestedType{RT}{%
  number type used to represent the coordinates of the input points.
  It has to be a model for \ccc{RingNumberType}.}

\ccNestedType{FT}{%
  number type used to return either the squared radius of the smallest
  enclosing sphere or annulus, or the squared distance of the polytopes.
  \ccc{FT} has to be either \ccc{RT} or \ccGlobalScope\ccc{Quotient<RT>} if
  the input points have Cartesian or homogeneous representation,
  respectively (cf.~\ccc{Rep_tag}).}

\ccNestedType{Access_dimension_d}{%
  data accessor object used to access the dimension of the input points.}

\ccNestedType{Access_coordinates_begin_d}{%
  data accessor object used to access the coordinates of the input points.}

\ccNestedType{Construct_point_d}{%
  constructor object used to construct either the center of the smallest
  enclosing sphere or annulus, or the points realizing the distance between
  the two polytopes.}

The following two number types are only needed for
\ccGlobalScope\ccc{Min_annulus_d<Traits>} and
\ccGlobalScope\ccc{Polytope_distance_d<Traits>}.

\ccNestedType{ET}{%
  exact number type used to do the exact computations in the
  underlying solver for linear programs. It has to to be a model for
  \ccc{RingNumberType}. There must be an implicit conversion from
  \ccc{RT} to \ccc{ET} available.}

\ccNestedType{NT}{%
  fast (possibly inexact) number type used to speed up the pricing step in
  the underlying solver for linear programs. It has to be a model for
  \ccc{RingNumberType}. There must be implicit conversions from \ccc{RT} to
  \ccc{NT} and from \ccc{NT} to \ccc{ET} available.}

\ccRestoreThreeColumns

% -----------------------------------------------------------------------------
\ccCreation
\ccCreationVariable{traits}

Only default and copy constructor are required.

\ccConstructor{ PolytopeDistanceDTraits( );}{}

\ccConstructor{ PolytopeDistanceDTraits( const PolytopeDistanceDTraits&);}{}

% -----------------------------------------------------------------------------
\ccOperations

\ccSetThreeColumns{Access_coordinates_begin_d}{}{\hspace*{8.5cm}}

The following functions just return the corresponding function class
object.

\ccMemberFunction{ Access_dimension_d
                   access_dimension_d_object() const;}{}

\ccMemberFunction{ Access_coordinates_begin_d
                   access_coordinates_begin_d_object() const;}{}

\ccMemberFunction{ Construct_point_d
                   construct_point_d_object() const;}{}

\ccRestoreThreeColumns

% -----------------------------------------------------------------------------
\ccHasModels

\ccRefIdfierPage{CGAL::Polytope_distance_d_traits_2<K,ET,NT>}\\
\ccRefIdfierPage{CGAL::Polytope_distance_d_traits_3<K,ET,NT>}\\
\ccRefIdfierPage{CGAL::Polytope_distance_d_traits_d<K,ET,NT>}

% -----------------------------------------------------------------------------
\ccSeeAlso

\ccIndexTraitsClassRequirements[C]{Polytope_distance_d}
\ccRefIdfierPage{CGAL::Polytope_distance_d<Traits>}

% -----------------------------------------------------------------------------

\end{ccRefConcept}


