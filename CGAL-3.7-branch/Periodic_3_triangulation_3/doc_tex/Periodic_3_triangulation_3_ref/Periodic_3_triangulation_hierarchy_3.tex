% +------------------------------------------------------------------------+
% | Reference manual page: Periodic_3_triangulation_hierarchy_3.tex
% +------------------------------------------------------------------------+
% | 4.3.2009   Manuel Caroli
% | Package: Periodic_3_triangulation_3
% | 
\RCSdef{\RCSPeriodictriangulationhierarchyRev}{$Id$}
\RCSdefDate{\RCSPeriodictriangulationhierarchyDate}{$Date$}
% |
%%RefPage: end of header, begin of main body
% +------------------------------------------------------------------------+


\begin{ccRefClass}{Periodic_3_triangulation_hierarchy_3<PTr>}

\ccDefinition

The class \ccc{Periodic_3_triangulation_hierarchy_3} implements a
triangulation augmented with a data structure which allows fast point
location queries. 

\ccInclude{CGAL/Periodic_3_triangulation_hierarchy_3.h}

\ccParameters

It is templated by a parameter which must be instantiated by one of the \cgal\
periodic triangulation classes.  \textit{In the current
  implementation, only \ccc{Periodic_3_Delaunay_triangulation_3} is
  supported for \ccc{PTr}.} 

\ccc{PTr::Vertex} has to be a model of the concept
\ccc{Periodic_3TriangulationHierarchyVertexBase_3}.\\
\ccc{PTr::Geom_traits} has to be a model of the concept
\ccc{Periodic_3DelaunayTriangulationTraits_3}.

\ccInheritsFrom

\ccc{PTr}

\ccRefName\ offers exactly the same functionalities as \ccc{PTr}.
Most of them (point location, insertion, removal\ldots) are overloaded to
improve their efficiency by using the hierarchic structure. 

Note that, since the algorithms that are provided are randomized, the
running time of constructing a triangulation with a hierarchy may be
improved when shuffling the data points.

However, the I/O operations are not overloaded. So, writing a
hierarchy into a file will lose the hierarchic structure and reading
it from the file will result in an ordinary triangulation whose
efficiency will be the same as \ccc{PTr}.

\ccImplementation
The data structure is a hierarchy of triangulations. The triangulation
at the lowest level is the original triangulation where operations and
point location are to be performed.
Then at each succeeding level, the data structure
stores a triangulation of a small random sample of the vertices
of the triangulation at the preceding level. Point location
is done through a top-down nearest neighbor query.
The nearest neighbor query is first
performed naively in the top level triangulation.
Then, at each following level, the nearest neighbor at that level
is found through a linear walk performed from
the nearest neighbor found at the preceding level.
Because the number of vertices in each triangulation is only a small
fraction of the number of vertices of the preceding triangulation 
the data structure remains small and achieves fast point location 
queries on real data.

\ccSeeAlso
\ccc{CGAL::Periodic_3_triangulation_hierarchy_vertex_base_3}\\
\ccc{CGAL::Periodic_3_Delaunay_triangulation_3}

\end{ccRefClass}
