% +------------------------------------------------------------------------+
% | Reference manual page: MeshWSETraits_3.tex
% +------------------------------------------------------------------------+
% | 19.05.2008   Laurent Rineau
% | Package: Mesh_3
% |
\RCSdef{\RCSMeshWSETraitsRev}{$Id: MeshWSETraits_3.tex 43417 2008-06-03 13:33:01Z lrineau $}
\RCSdefDate{\RCSMeshWSETraitsDate}{$Date: 2008-06-03 15:33:01 +0200 (Tue, 03 Jun 2008) $}
% |
\ccRefPageBegin
%%RefPage: end of header, begin of main body
% +------------------------------------------------------------------------+


\begin{ccRefConcept}{MeshWSETraits_3}

%% \ccHtmlCrossLink{}     %% add further rules for cross referencing links
%% \ccHtmlIndexC[concept]{} %% add further index entries

\ccDefinition
  
The concept \ccRefName\ refines the concept
\ccc{SurfaceMeshWSETraits_3} and adds a predicate that tells if
a given point lies in the volume inside the object to be meshed, or not.

\ccRefines

\ccc{SurfaceWSEMeshTraits_3} 

\ccTypes

\ccNestedType{Point_3}{The type of points.}
\ccNestedType{Input_complex_cell_label}{The type of labels for cells of the
  input complex (the object to be meshed). The default value is supposed to
be the label of the exterior of the object (which contains at least the
unbounded component).}

\ccCreationVariable{traits}  %% choose variable name

\ccOperations

\ccMethod{Input_complex_cell_label get_cell_label(const Point_3&p
  p);}{Returns the label of the cell that contains the point~\ccc{p}. If
  the point~\ccc{p} lies in the exterior of the object to be meshed, the
  default value should be returned.}

\ccSeeAlso

\ccc{make_mesh_3_with_sharp_edges}


\end{ccRefConcept}

% +------------------------------------------------------------------------+
%%RefPage: end of main body, begin of footer
\ccRefPageEnd
% EOF
% +------------------------------------------------------------------------+

