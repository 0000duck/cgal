% +------------------------------------------------------------------------+
% | Reference manual page: make_mesh_3_with_sharp_edges.tex
% +------------------------------------------------------------------------+
% | 13.05.2008   Laurent Rineau
% | Package: Mesh_3
% |
\RCSdef{\RCSmakemeshfrompiecewisesmoothsurfacesRev}{$Id: make_mesh_3_with_sharp_edges.tex 43417 2008-06-03 13:33:01Z lrineau $}
\RCSdefDate{\RCSmakemeshfrompiecewisesmoothsurfacesDate}{$Date: 2008-06-03 15:33:01 +0200 (Tue, 03 Jun 2008) $}
% |
\ccRefPageBegin
%%RefPage: end of header, begin of main body
% +------------------------------------------------------------------------+


\begin{ccRefFunction}{make_mesh_3_with_sharp_edges}  %% add template arg's if necessary

%% \ccHtmlCrossLink{}     %% add further rules for cross referencing links
%% \ccHtmlIndexC[function]{} %% add further index entries

\ccDefinition
  
The function \ccRefName\ is a volume mesh generator. It produces a surface
mesh at the same time. The object to be meshed is supposed to be bounded by
a piecewise-smooth surface. That object is known through a meshing traits
that provides several predicates on the object and its boundary.

\ccInclude{CGAL/make_mesh_3_with_sharp_edges.h}

\ccGlobalFunction{
  template <class C2T3,
  class MeshTraits,
  class MeshCriteria>
  void make_mesh_3_with_sharp_edges(C2T3& c2t3,
                                    MeshTraits meshtraits,
                                    MeshCriteria mesh_criteria);}{}

\ccParameters

The template parameter \ccc{C2T3} is required to be a model of
\ccc{SurfaceMeshComplex_2InTriangulation_3}, a data structure able to
represent a two dimensional complex embedded in a three dimensional
triangulation. The argument \ccc{c2t3} of type \ccc{C2T3}, passed by
reference to the mesh generator, is used to maintain the current
approximating surface mesh and it stores the final surface mesh at the end
of the procedure.

The type \ccc{C2T3} is in particular required to provide a nested type
\ccc{C2T3::Triangulation_3} for the three dimensional triangulation
embedding the surface mesh.  The vertex and cell base classes of the
triangulation \ccc{C2T3::Triangulation_3} are required to be models of the
concepts \ccc{MeshWSEVertexBase_3} and \ccc{MeshCellBase_3}
respectively.

The template parameter \ccc{MeshTraits} is required to be a model of
\ccc{MeshWSETraits_3}.The argument \ccc{meshtraits} of type
\ccc{MeshTraits} is the link between object to be meshed and
the meshing algorithm.

The template argument \ccc{MeshCriteria} has to be a model of the concepts
\ccc{SurfaceMeshWSEEdgesCriteria_3}, \ccc{SurfaceMeshFacetsCriteria_3} and
\ccc{MeshCriteria_3}. The argument of type \ccc{MeshCriteria}
passed to the mesh generator specifies:
\begin{itemize}
\item the requirements on the marked edges of the output mesh,
\item the size and shape requirements on the facets of the output surface
  mesh,
\item the size and shape requirements on the tetrahedra of the output mesh.
\end{itemize}

\ccSeeAlso
\ccc{make_mesh_3}

\end{ccRefFunction}

% +------------------------------------------------------------------------+
%%RefPage: end of main body, begin of footer
\ccRefPageEnd
% EOF
% +------------------------------------------------------------------------+

