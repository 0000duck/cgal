% +------------------------------------------------------------------------+
% | Reference manual page: odt_optimize_mesh_3.tex
% +------------------------------------------------------------------------+
% | 28.07.2009   Stephane Tayeb
% | Package: Mesh_3
% |
\RCSdef{\RCSodtoptimizemeshRev}{$Id$}
\RCSdefDate{\RCSodtoptimizemeshDate}{$Date$}
% |
\ccRefPageBegin
%%RefPage: end of header, begin of main body
% +------------------------------------------------------------------------+


\begin{ccRefFunction}{odt_optimize_mesh_3}  %% add template arg's if necessary

%% \ccHtmlCrossLink{}     %% add further rules for cross referencing links
%% \ccHtmlIndexC[function]{} %% add further index entries

\ccDefinition
  
The function \ccRefName\ is a mesh optimization process
based on the minimization of a global energy function.

In \ccRefName, the minimized global energy may be interpreted
as the $L^1$-norm of the error achieved 
when  the function $x^2$ is interpolated on the  mesh domain
using a piecewise linear function which is linear  in each mesh cell.

The optimizer \ccRefName\  works in iterative steps.
At each iteration,  mesh vertices are moved into 
positions that bring to zero the energy gradient
and the Delaunay triangulation is updated.
Vertices on the mesh boundaries are handled
in a special way so as to preserve an accurate
representation of the domain boundaries.


\ccInclude{CGAL/odt_optimize_mesh_3.h}

\ccGlobalFunction{
	template<typename C3T3, typename MeshDomain>
	Mesh_optimization_return_code
        odt_optimize_mesh_3(C3T3& c3t3,
          MeshDomain domain,
          double time_limit=0,
          std::size_t max_iteration_number=0,
          double convergence=0.02,
          double freeze_bound = 0.01);}{
  \ccPrecond{\ccc{time_limit} $\geq$ 0 and 0 $\leq$ \ccc{convergence} $\leq$ 1 and 0 $\leq$ \ccc{freeze_bound} $\leq$ 1}}

\ccParameters

Parameter \ccc{C3T3} is required to be a model of the concept
\ccc{MeshComplex_3InTriangulation_3}.
The argument \ccc{c3t3}, passed by
reference, provides the initial mesh  
and is modified  by the algorithm 
to represent the final optimized mesh.

Parameter \ccc{MeshDomain} is required to be a model of the concept
\ccc{MeshDomain_3}. The argument \ccc{domain} must be the \ccc{MeshDomain}
object used to create the \ccc{c3t3} parameter.

The function has four optional parameters which are named parameters
 (we use the Boost.Parameters library). 
Therefore, when calling the function,  the parameters can be provided in any order 
provided that the names of the parameters are used
 (see example at the bottom of this page).
\begin{itemize}
\item
Parameter \ccc{time_limit}, whose name is \ccc{parameters::time_limit},
 is used to set up, in seconds,
 a CPU time limit after which the optimization process is stopped. This time is
 measured using \ccc{CGAL::Timer}.
The default value is \ccc{0} and means that there is no time limit.
\item  Parameter \ccc{max_iteration_number}, whose name is 
 \ccc{parameters::max_iteration_number}, sets a limit on the
number of performed  iterations. The default value of \ccc{0} means that there is 
no limit on the number of performed iterations.
\item  Parameter \ccc{convergence}, whose name is 
 \ccc{parameters::convergence}, is a stopping  criterion based on convergence: 
the optimization process is stopped, when at the last iteration,
the displacement of any vertex is less than a given percentage of the length
the shortest edge incident that vertex.
The parameter \ccc{convergence} gives the threshold ratio.
\item 
Parameter \ccc{freeze_bound}, whose name is \ccc{parameters::freeze_bound},
is designed to reduce running time of each optimization iteration.
 Any vertex that has
a displacement less than a given percentage of the  length of its shortest incident edge,  is frozen (i.e. is
not relocated). The parameter \ccc{freeze_bound} gives the threshold ratio. At each iteration, any vertex that
moves, unfreezes the neighboring vertices.
\end{itemize}

\ccHeading{Return Values}
The function \ccRefName\ returns a value of type \ccc{Mesh_optimization_return_code}
which is:
\begin{itemize}
\item \ccc{TIME_LIMIT_REACHED} when the time limit is reached.
\item \ccc{MAX_ITERATION_NUMBER_REACHED} when \ccRefName\ stops because it has performed \ccc{max_iteration_number} iterations.
\item \ccc{CONVERGENCE_REACHED} when \ccRefName\ stops because the convergence criterion
is achieved.
\end{itemize}



\ccSeeAlso

\ccc{Mesh_optimization_return_code} \\
\ccc{make_mesh_3} \\
\ccc{refine_mesh_3} \\
\ccc{exude_mesh_3} \\
\ccc{perturb_mesh_3} \\
\ccc{lloyd_optimize_mesh_3} \\

\ccExample 

\begin{ccExampleCode}
// 100 iterations of Odt-smoothing
odt_optimize_mesh_3(c3t3, domain, parameters::max_iteration_number=100, 
                    parameters::convergence=0);
\end{ccExampleCode}

\end{ccRefFunction}

% +------------------------------------------------------------------------+
%%RefPage: end of main body, begin of footer
\ccRefPageEnd
% EOF
% +------------------------------------------------------------------------+

