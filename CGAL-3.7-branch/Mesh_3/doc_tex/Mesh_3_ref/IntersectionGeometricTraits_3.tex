% +------------------------------------------------------------------------+
% | Reference manual page: IntersectionGeometricTraits_3.tex
% +------------------------------------------------------------------------+
% | 23.02.2009   Author
% | Package: Package
% |
\RCSdef{\RCSIntersectionGeometricTraitsRev}{$Id: header.tex 40270 2007-09-07 15:29:10Z lsaboret $}
\RCSdefDate{\RCSIntersectionGeometricTraitsDate}{$Date: 2007-09-07 17:29:10 +0200 (Ven, 07 sep 2007) $}
% |
\ccRefPageBegin
%%RefPage: end of header, begin of main body
% +------------------------------------------------------------------------+


\begin{ccRefConcept}{IntersectionGeometricTraits_3}

%% \ccHtmlCrossLink{}     %% add further rules for cross referencing links
%% \ccHtmlIndexC[concept]{} %% add further index entries

\ccDefinition
  
The concept \ccRefName\ provides  types and functors
required to implement a model of \ccc{MeshDomain_3},
when the domain is described  by a simplicial surface mesh
forming its boundary.
The concept \ccRefName\  mainly provides  the detection 
and construction of intersections between segments and triangles.



% \ccRefines

% ThisConcept \\
% ThatConcept

\ccTypes
\ccNestedType{Point_3} {Point type.}
\ccGlue
\ccNestedType{Segment_3}{Segment type.}
\ccGlue
\ccNestedType{Triangle_3}{Triangle type.}

\ccNestedType{Do_intersect_3} 
{Function object that detects an intersection between a 3D segment and a 3D triangle.
Provides the operators: \\
\ccc{bool operator()(Segment_3 seg, Triangle_3 tr)} \\
\ccc{bool operator()(Triangle_3 tr, Segment_3 seg)} \\
which return \ccc{true}, iff the triangle and the segment
have a non empty intersection.}

\ccNestedType{Intersect_3} 
{Function object that constructs the intersection
between a 3D segment and a 3D triangle.
Provides the operators: \\
\ccc{CGAL::Object operator()(Segment_3 seg, Triangle_3 tr)} \\
\ccc{CGAL::Object operator()(Triangle_3 tr, Segment_3 seg)} \\
which computes as a \ccc{CGAL::Object}
the intersection between  the triangle and the segment.
\ccc{CGAL::Object} is  either a point, a segment or 
an empty object.}


\ccCreationVariable{traits}  %% choose variable name

\ccOperations

%\ccMethod{void foo();}{some member functions}
\ccMethod{Do_intersect_3       
           do_intersect_3_object();}
{Returns the intersection detection functor.}


\ccMethod{Intersect_3
           intersect_3_object();}
{Returns the intersection constructor.}

\ccHasModels
Any instantiation of \ccc{CGAL::Kernel} is a model of this traits concept.

\ccSeeAlso
\ccc{BisectionGeometricTraits_3}\\
\ccc{Polyhedral_mesh_domain_3<Polyhedron,IGT,TriangleAccessor>}


% \ccExample

% A short example program.
% Instead of a short program fragment, a full running program can be
% included using the 
% \verb|\ccIncludeExampleCode{Package/IntersectionGeometricTraits_3.C}| 
% macro. The program example would be part of the source code distribution and
% also part of the automatic test suite.

% \begin{ccExampleCode}
% void your_example_code() {
% }
% \end{ccExampleCode}

%% \ccIncludeExampleCode{Package/IntersectionGeometricTraits_3.C}

\end{ccRefConcept}

% +------------------------------------------------------------------------+
%%RefPage: end of main body, begin of footer
\ccRefPageEnd
% EOF
% +------------------------------------------------------------------------+

