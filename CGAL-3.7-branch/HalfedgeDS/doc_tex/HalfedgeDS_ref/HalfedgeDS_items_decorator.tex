% +------------------------------------------------------------------------+
% | Reference manual page: HalfedgeDS_items_decorator.tex
% +------------------------------------------------------------------------+
% | 22.03.1999   Lutz Kettner
% | Package: HalfedgeDS
% | 
\RCSdef{\RCSHalfedgeDSitemsdecoratorRev}{$Id$}
\RCSdefDate{\RCSHalfedgeDSitemsdecoratorDate}{$Date$}
% +------------------------------------------------------------------------+

\ccRefPageBegin

%%RefPage: end of header, begin of main body
% +------------------------------------------------------------------------+


\begin{ccRefClass}{HalfedgeDS_items_decorator<HDS>}

\ccDefinition

The classes \ccc{CGAL::HalfedgeDS_items_decorator<HDS>},
\ccc{CGAL::HalfedgeDS_decorator<HDS>}, and
\ccc{CGAL::HalfedgeDS_const_decorator<HDS>} provide additional functions
to examine and to modify a halfedge data structure \ccc{HDS}. The class
\ccc{CGAL::HalfedgeDS_items_decorator<HDS>} provides additional functions
for vertices, halfedges, and faces of a halfedge data structure
without knowing the containing halfedge data structure. The class
\ccc{CGAL::HalfedgeDS_decorator<HDS>} stores a reference to the halfedge
data structure and provides functions that modify the halfedge data
structure, for example Euler-operators. The class
\ccc{CGAL::HalfedgeDS_const_decorator<HDS>} stores a const reference to
the halfedge data structure. It contains non-modifying functions, for
example the test for validness of the data structure.

All these additional functions take care of the different capabilities
a halfedge data structure may have or may not have.  The functions
evaluate the type tags of the halfedge data structure to decide on the
actions. If a particular feature is not supported nothing is done.
Note that for example the creation of new halfedges is mandatory for
all halfedge data structures and will not appear here again.

\ccInclude{CGAL/HalfedgeDS_items_decorator.h}

\ccTypes

\ccThree{Halfedge_handle}{D.get_vertex_halfedge( Vertex_handle v);}{}
\ccThreeToTwo

\ccNestedType{HalfedgeDS}{halfedge data structure.}
\ccGlue
\ccNestedType{Traits}{traits class.}
\ccGlue
\ccNestedType{Vertex}{vertex type of \ccc{HalfedgeDS}.}
\ccGlue
\ccNestedType{Halfedge}{halfedge type of \ccc{HalfedgeDS}.}
\ccGlue
\ccNestedType{Face}{face type of \ccc{HalfedgeDS}.}

\ccNestedType{Vertex_handle}{}
\ccGlue
\ccNestedType{Halfedge_handle}{}
\ccGlue
\ccNestedType{Face_handle}{}
\ccGlue
\ccNestedType{Vertex_iterator}{}
\ccGlue
\ccNestedType{Halfedge_iterator}{}
\ccGlue
\ccNestedType{Face_iterator}{}

The respective \ccc{const_handle}'s and \ccc{const_iterator}'s are
available as well.

\ccNestedType{size_type}{}
\ccGlue
\ccNestedType{difference_type}{}
\ccGlue
\ccNestedType{iterator_category}{}

\ccNestedType{Supports_vertex_halfedge}{}
\ccGlue
\ccNestedType{Supports_halfedge_prev}{}
\ccGlue
\ccNestedType{Supports_halfedge_vertex}{}
\ccGlue
\ccNestedType{Supports_halfedge_face}{}
\ccGlue
\ccNestedType{Supports_face_halfedge}{}
\ccGlue
\ccNestedType{Supports_removal}{}


\ccCreation
\ccCreationVariable{D}

\ccConstructor{HalfedgeDS_items_decorator();}{default constructor.}

% -----------------------------------------
\ccHeading{Access Functions}

\ccMethod{Halfedge_handle get_vertex_halfedge( Vertex_handle v);}
    {returns the incident halfedge of $v$ if supported, 
    \ccc{Halfedge_handle()} otherwise.}
\ccGlue
\ccMethod{Vertex_handle get_vertex(Halfedge_handle h);}
    {returns the incident vertex of $h$ if supported, \ccc{Vertex_handle()} 
     otherwise.}
\ccGlue
\ccMethod{Halfedge_handle get_prev(Halfedge_handle h);}
    {returns the previous halfedge of $h$ if supported, 
     \ccc{Halfedge_handle()} otherwise.}
\ccGlue
\ccMethod{Halfedge_handle find_prev(Halfedge_handle h);}
    {returns the previous halfedge of $h$. Uses the \ccc{prev()} method 
     if supported or performs a search around the face using \ccc{next()}.}
\ccGlue
\ccMethod{Halfedge_handle find_prev_around_vertex(Halfedge_handle h);}
    {returns the previous halfedge of $h$. Uses the \ccc{prev()} method 
     if supported or performs a search around the vertex using \ccc{next()}.}
\ccGlue
\ccMethod{Face_handle get_face(Halfedge_handle h);}
    {returns the incident face of $h$ if supported, 
    \ccc{Face_handle()} otherwise.}
\ccGlue
\ccMethod{Halfedge_handle get_face_halfedge( Face_handle f);}
    {returns the incident halfedge of $f$ if supported, 
    \ccc{Halfedge_handle()} otherwise.}

Corresponding member functions for \ccc{const_handle}'s are provided as well.

% -----------------------------------------
\ccHeading{Modifying Functions (Composed)}

\ccMethod{void close_tip( Halfedge_handle h) const;}
    {makes \ccc{h->opposite()} the successor of $h$.}

\ccMethod{void close_tip( Halfedge_handle h, Vertex_handle v) const;}
    {makes \ccc{h->opposite()} the successor of $h$ and sets the
    incident vertex of $h$ to $v$.}

\ccMethod{void insert_tip( Halfedge_handle h, Halfedge_handle v) const;}
    {inserts the tip of the edge $h$ into the halfedges around the vertex 
    pointed to by $v$. Halfedge \ccc{h->opposite()} is the new successor of 
    $v$ and \ccc{h->next()} will be set to \ccc{v->next()}. The vertex of $h$
    will be set to the vertex $v$ refers to if vertices are supported.}

\ccMethod{void remove_tip( Halfedge_handle h) const;}
   {removes the edge \ccc{h->next()->opposite()} from the halfedge
   circle around the vertex referred to by $h$. The new successor
   halfedge of $h$ will be  \ccc{h->next()->opposite()->next()}.}

\ccMethod{void insert_halfedge( Halfedge_handle h, Halfedge_handle f) const;}
    {inserts the halfedge $h$ between $f$ and \ccc{f->next()}.
    The face of $h$ will be the one $f$ refers to if faces
    are supported.}

\ccMethod{void remove_halfedge( Halfedge_handle h) const;}
   {removes edge \ccc{h->next()} from the halfedge circle around 
    the face referred to by $h$. The new successor of $h$ will be 
    \ccc{h->next()->next()}.}

\vspace*{-1mm}
\ccMethod{void set_vertex_in_vertex_loop( Halfedge_handle h, 
                                          Vertex_handle v) const;}
   {loops around the vertex incident to $h$ and sets all vertex
    pointers to $v$. \ccPrecond \ccc{h != Halfedge_handle()}.}

\vspace*{-1mm}
\ccMethod{void set_face_in_face_loop( Halfedge_handle h, Face_handle f) const;}
   {loops around the face incident to $h$ and sets all face 
    pointers to $f$. \ccPrecond \ccc{h != Halfedge_handle()}.}

\ccMethod{Halfedge_handle flip_edge( Halfedge_handle h) const;}
   {performs an edge flip. It returns $h$ after rotating the edge $h$ one
    vertex in the direction of the face orientation. 
    \ccPrecond \ccc{h != Halfedge_handle()} and both incident faces 
    of $h$ are triangles.}

% -----------------------------------------
\vspace*{-2mm}
\ccHeading{Modifying Functions (Primitives)}

\ccThree{void}{D.set_face_halfedge( Face_handle f, Halfedge_handle g)}{}

\ccMethod{void set_vertex_halfedge( Vertex_handle v, Halfedge_handle g) const;}
    {sets the incident halfedge of $v$ to $g$.}
\ccGlue
\ccMethod{void set_vertex_halfedge( Halfedge_handle h) const;}
    {sets the incident halfedge of the vertex incident to $h$ to $h$.}
\ccGlue
\ccMethod{void set_vertex( Halfedge_handle h, Vertex_handle v) const;}
    {sets the incident vertex of $h$ to $v$.}
\ccGlue
\ccMethod{void set_prev( Halfedge_handle h, Halfedge_handle g) const;}
    {sets the previous link of $h$ to $g$.}
\ccGlue
\ccMethod{void set_face( Halfedge_handle h, Face_handle f) const;}
    {sets the incident face of $h$ to $f$.}
\ccGlue
\ccMethod{void set_face_halfedge( Face_handle f, Halfedge_handle g) const;}
    {sets the incident halfedge of $f$ to $g$.}
\ccGlue
\ccMethod{void set_face_halfedge( Halfedge_handle h) const;}
    {sets the incident halfedge of the face incident to $h$ to $h$.}


\ccSeeAlso

\ccRefIdfierPage{CGAL::HalfedgeDS_decorator<HDS>}\\
\ccRefIdfierPage{CGAL::HalfedgeDS_const_decorator<HDS>}

\ccExample

The following program fragment illustrates how a refined halfedge
class for a polyhedron can make use of the \ccc{find_prev()} member
function to implement a \ccc{prev()} member function that works
regardless of whether the halfedge data structure \ccc{HDS} provides a
\ccc{prev()} member function for its halfedges or not. In the case that not,
the implementation given here runs in time proportional to the size of the 
incident face. For const-correctness a second implementation with signature
\ccc{Halfedge_const_handle  prev() const;} is needed.

Note also the use of the static member function \ccc{halfedge_handle()}
of the halfedge data structure. It converts a pointer to the halfedge
into a halfedge handle. This conversion encapsulates possible
adjustments for hidden data members in the true halfedge type, such as
linked-list pointers.

\begin{ccExampleCode}
struct Polyhedron_halfedge {
    // ...
    Halfedge_handle  prev() {
        CGAL::HalfedgeDS_items_decorator<HDS> decorator;
        return decorator.find_prev( HDS::halfedge_handle(this));
    }
};
\end{ccExampleCode}


\end{ccRefClass}

% +------------------------------------------------------------------------+
%%RefPage: end of main body, begin of footer
\ccRefPageEnd
% EOF
% +------------------------------------------------------------------------+

