% +------------------------------------------------------------------------+
% | Reference manual page: HalfedgeDS_default.tex
% +------------------------------------------------------------------------+
% | 22.03.1999   Lutz Kettner
% | Package: HalfedgeDS
% | 
\RCSdef{\RCSHalfedgeDSdefaultRev}{$Id$}
\RCSdefDate{\RCSHalfedgeDSdefaultDate}{$Date$}
% +------------------------------------------------------------------------+

\ccRefPageBegin

%%RefPage: end of header, begin of main body
% +------------------------------------------------------------------------+


\begin{ccRefClass}{HalfedgeDS_default<Traits,HalfedgeDSItems,Alloc>}
\ccRefLabel{CGAL::HalfedgeDS_default}
\ccRefLabel{CGAL_HALFEDGEDS_DEFAULT}

\ccHtmlCrossLink{CGAL_HALFEDGEDS_DEFAULT}
\ccIndexMainItem[C]{CGAL_HALFEDGEDS_DEFAULT}

\ccDefinition

\begin{tabbing}
\ccc{template <} \=\ccc{class Traits,}\\
                 \>\ccc{class HalfedgeDSItems = CGAL::HalfedgeDS_items_2,}\\
                 \>\ccc{class Alloc           = CGAL_ALLOCATOR(int)>}\\
     \ccc{class HalfedgeDS_default;}
\end{tabbing}

The class \ccRefName\ is a model for the \ccc{HalfedgeDS} concept. The
second template parameter \ccc{HalfedgeDSItems} has a default argument
\ccc{CGAL::HalfedgeDS_items_2}. The third template parameter \ccc{Alloc}
uses the \cgal\ default allocator as default setting. \ccRefName\ is a
list-based representation with bidirectional iterators that supports
removal.

\ccInclude{CGAL/HalfedgeDS_default.h}

\ccIsModel

\ccRefConceptPage{HalfedgeDS<Traits,Items,Alloc>}

\ccTypes
\ccThree{typedef bidirectional_iterator_tag;}{Supports_removal;;}{}

\ccTypedef{typedef bidirectional_iterator_tag iterator_category;}{}
\ccGlue
\ccTypedef{typedef CGAL::Tag_true Supports_removal;}{}

\ccSeeAlso

\ccRefIdfierPage{CGAL::HalfedgeDS_list}\\
\ccRefIdfierPage{CGAL::HalfedgeDS_vector}\\
\ccRefConceptPage{HalfedgeDSItems}\\
\ccRefIdfierPage{CGAL::HalfedgeDS_items_2}\\
\ccRefIdfierPage{CGAL::Polyhedron_3<Traits>}\\
\ccRefIdfierPage{CGAL::HalfedgeDS_items_decorator<HDS>}\\
\ccRefIdfierPage{CGAL::HalfedgeDS_decorator<HDS>}\\
\ccRefIdfierPage{CGAL::HalfedgeDS_const_decorator<HDS>}


\ccImplementation

Currently, \ccRefName\ is derived from \ccc{CGAL::HalfedgeDS_list<Traits>}.
The copy constructor and the assignment operator need $O(n)$ time with
$n$ the total number of vertices, halfedges, and faces. The former 
suboptimal implementation with an $O(n \log n)$ runtime has been replaced
with a faster implementation based on hashing for the pointer lookup.

Due to a workaround for the flag \ccc{CGAL_CFG_NO_TMPL_IN_TMPL_PARAM},
a halfedge data structure cannot be instantiated directly. For the
\ccRefName\ a macro simplifies its direct use. However, when using a
halfedge data structure as an argument for another class template,
the class template name \ccc{HalfedgeDS_default} must be used, not 
the macro.

\begin{ccExampleCode}
// The macro definition.
#ifndef CGAL_CFG_NO_TMPL_IN_TMPL_PARAM
    #define CGAL_HALFEDGEDS_DEFAULT  ::CGAL::HalfedgeDS_default
#else
    #define CGAL_HALFEDGEDS_DEFAULT  ::CGAL::HalfedgeDS_default::HDS
#endif

// The direct instantiation of the default HalfedgeDS given a Traits class.
typedef CGAL_HALFEDGEDS_DEFAULT<Traits> HDS;
\end{ccExampleCode}

\end{ccRefClass}

% +------------------------------------------------------------------------+
%%RefPage: end of main body, begin of footer
\ccRefPageEnd
% EOF
% +------------------------------------------------------------------------+

