\ccDefGlobalScope{CGAL::}
\begin{ccRefClass}[Qt::]{Converter<K>}

\ccDefinition
An object of type \ccRefName\ converts 2D \cgal\ kernel objects to their equivalent 
objects in \qt, and the other way round. Note that some objects have no equivalent.
For example the \ccc{CGAL::Circle_2<K>} cannot be converted to something in Qt,
and the unbounded objects \ccc{CGAL::Ray_2<K>} and \ccc{CGAL::Line_2<K>} are clipped.
Note also that \cgal\ and \qt\ sometimes also use the same word for different things.
For example {\em line} denotes an unbounded line in \cgal, wheras it denotes a bounded
segment in \qt.

\ccInclude{CGAL/Qt/Converter.h}

\ccCreationVariable{c}

\ccCreation
\ccConstructor{Converter<K>(QRectF clippingRect);}{The clipping rect is used for unbounded \cgal\ objects.}

\ccHeading{From CGAL to Qt}

\ccMethod{QRectF operator()(Bbox_2);}{Converts a bounding box.}

\ccMethod{QPointF operator()(K::Point_2);}{Converts a point.}

\ccMethod{QLineF operator()(K::Segment_2);}{Converts a segment.}

\ccMethod{QLineF operator()(K::Ray_2);}{Clips the ray and converts the resulting segment.}

\ccMethod{QLineF operator()(K::Line_2);}{Clips the line and converts the resulting segment.}

\ccMethod{QPolygonF operator()(K::Triangle_2);}{Converts a triangle.}

\ccMethod{QRectF operator()(K::Iso_rectangle_2);}{Converts an iso rectangle.}

\ccMethod{QPolygonF operator()(std::list<K::Point_2>);}{Converts a list of points to a polygon.}


\ccHeading{From Qt to CGAL}

\ccMethod{K::Point_2 operator()(QPointF);}{Converts a point.}

\ccMethod{K::Segment_2 operator()(QLineF);}{Converts a segment.}

\ccMethod{K::Iso_rectangle_2 operator()(QRectF);}{Converts an iso rectangle.}

\ccMethod{std::list<K::Point_2> operator()(QPolygonF);}{Converts a polygon to a list of points.}

\end{ccRefClass}







