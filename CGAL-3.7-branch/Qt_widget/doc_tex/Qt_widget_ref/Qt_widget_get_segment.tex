% +------------------------------------------------------------------------+
% | CGAL Reference Manual: Reference manual for Qt_widget.tex
% +------------------------------------------------------------------------+
% |
% | 03.01.2001  Radu Ursu
% | 
% | \RCSdef{\qtwidgetRev}{$Id$}
% | \RCSdefDate{\qtwidgetDate}{$Date$}
% +------------------------------------------------------------------------+

% +-----------------------------------------------------+
\begin{ccRefClass}{Qt_widget_get_segment<T>}

\ccDefinition
An object of type \ccRefName\ creates a \cgal\ segment in this 
way: one left click on the mouse will be the first point and the second point 
of the segment will be considered at the coordinates where the left
mouse button is pressed for the second time.
You can always cancel the creation process by pressing the ESC key.
The use of \ccRefName\ requires that the mouse tracking is
enabled for widgets attaching it.

\ccInclude{CGAL/IO/Qt_widget_get_segment.h}

\ccParameters

The full template declaration of \ccc{Qt_widget_get_segment} states one parameter:

\begin{tabbing}
\ccc{template <} \=\ccc{class T >}\\
        \ccc{class Qt_widget_get_segment;}
\end{tabbing}

If T is one of the \cgal\ kernels you don't need additional types. If
not, the parameter T has to provide this types:

\ccTypes
\ccTypedef{typedef T::Point_2 Point_2;}{This should be a Point type}
\ccTypedef{typedef T::Segment_2 Segment_2;}{This should be a Segment type}
\ccTypedef{typedef T::FT FT;}{This should be a Field type}

\ccInheritsFrom
\ccc{Qt_widget_layer}

\ccGlue

\ccCreation
\ccCreationVariable{getsegment}
\ccSetTwoColumns{Qt_widget_get_segment}{}

\ccConstructor{Qt_widget_get_segment<T>(const QCursor
c=QCursor(Qt::crossCursor), QObject* parent = 0, const char* name =
0);}{\ccc{c} is the cursor that this layer will use when is
active. \ccc{parent} is the parent widget and \ccc{name} is the name you 
give to this layer.}

\end{ccRefClass}
% +-----------------------------------------------------+
% EOF








