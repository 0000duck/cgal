\begin{ccRefFunction}{side_of_oriented_circle}

\ccFunction{Oriented_side side_of_oriented_circle(
	                           const Point_2<Kernel> &p, 
	                           const Point_2<Kernel> &q,
                                   const Point_2<Kernel> &r, 
                                   const Point_2<Kernel> &test);}
         {returns the relative position of point \ccStyle{test}
          to the oriented circle defined by $p$, $q$ and $r$.
	  The order of the points $p$, $q$ and $r$ is important,
	  since it determines the orientation of the implicitly
          constructed circle.

If $p$, $q$ and $r$ are collinear, the circle degenerates in a line.
\ccStyle{ON_ORIENTED_BOUNDARY} is returned if $test$ is also collinear or if two
points are identical, 
otherwise, \ccc{side_of_oriented_circle(r, q, test, p)} is returned.
}

\ccSeeAlso

\ccRefIdfierPage{CGAL::side_of_bounded_circle} \\

\end{ccRefFunction}

