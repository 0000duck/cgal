\begin{ccRefFunctionObjectConcept}{Kernel::SideOfOrientedCircle_2}
A model for this must provide:

\ccCreationVariable{fo}

\ccMemberFunction{Oriented_side operator()(const Kernel::Point_2&p, 
                                  const Kernel::Point_2&q, 
                                  const Kernel::Point_2&r, 
                                  const Kernel::Point_2&t);}
         {returns the relative position of point \ccStyle{t}
          to the oriented circle defined by $p$, $q$ and $r$.
          The order of the points $p$, $q$ and $r$ is important,
          since it determines the orientation of the implicitly
          constructed circle. 

If $p$, $q$ and $r$ are collinear, the circle degenerates in a line.
\ccStyle{ON_ORIENTED_BOUNDARY} is returned if $t$ is also collinear or if two
points are identical, 
otherwise, \ccc{side_of_oriented_circle(r, q, t, p)} is returned.
}

\ccRefines
\ccc{AdaptableFunctor} (with four arguments)

\ccSeeAlso
\ccRefIdfierPage{CGAL::side_of_oriented_circle} \\


\end{ccRefFunctionObjectConcept}
