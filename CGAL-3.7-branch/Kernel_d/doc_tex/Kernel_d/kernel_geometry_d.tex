\section{Kernel Geometry}

\subsection{Points and Vectors}
In \cgal, we strictly distinguish between points, vectors and
directions.  A {\em point} is a point in the Euclidean space $\E^d$, a
{\em vector} is the difference of two points $p_2$, $p_1$ and denotes
the direction and the distance from $p_1$ to $p_2$ in the vector space
$\R^d$, and a {\em direction} is a vector where we forget about its
length.  They are different mathematical concepts. For example, they
behave different under affine transformations and an addition of two
points is meaningless in affine geometry.  By putting them in
different classes we not only get cleaner code, but also type checking
by the compiler which avoids ambiguous expressions. Hence, it pays
twice to make this distinction.

\cgal\ defines a symbolic constant \ccStyle{ORIGIN} of type
\ccc{Origin} which denotes the point at the origin. This constant is
used in the conversion between points and vectors. Subtracting it from
a point $p$ results in the locus vector of $p$.

\begin{cprog}
  double coord[] = {1.0, 1.0, 1.0, 1.0};
  Point_d< Cartesian_d<double> >  p(4,coord,coord+4), q(4);
  Vector_d< Cartesian_d<double> >  v(4);
  v = p - ORIGIN;
  q = ORIGIN + v;  
  assert( p == q );
\end{cprog} 

In order to obtain the point corresponding to a vector $v$ you simply
have to add $v$ to \ccStyle{ORIGIN}. If you want to determine
the point $q$ in the middle between two points $p_1$ and $p_2$, you can 
write%
\footnote{you might call \ccc{midpoint(p_1,p_2)} instead}

\begin{cprog}
  q = p_1 + (p_2 - p_1) / 2.0;
\end{cprog}  

Note that these constructions do not involve any performance overhead
for the conversion with the currently available representation
classes.

\subsection{Kernel Objects}

Besides points (\ccc{Point_d<R>}), vectors (\ccc{Vector_d<R>}), and
directions (\ccc{Direction_d<R>}), \cgal\ provides lines, rays,
segments, hyperplanes, and spheres.

Lines (\ccc{Line_d<R>}) in {\cgal} are oriented.  A ray
(\ccc{Ray_d<R>}) is a semi-infinite interval on a line, and this line
is oriented from the finite endpoint of this interval towards any
other point in this interval. A segment (\ccc{Segment_d<R>}) is a
bounded interval on a directed line, and the endpoints are ordered so
that they induce the same direction as that of the line.

Hyperplanes are affine subspaces of dimension $d-1$ in $\E^d$, passing
through $d$ points.  Hyperplanes are oriented and partition space into
a positive side and a negative side.  In \cgal, there are no special
classes for halfspaces.  Halfspaces are supposed to be represented by
oriented hyperplanes. All kernel objects are equality comparable via
\ccc{operator==} and \ccc{operator!=}. For those oriented objects
whose orientation can be reversed (segments, lines, hyperplanes,
spheres) there is also a global function \ccc{weak_equality} that
allows to test for point set equality disregarding the orientation.


\subsection{Orientation and Relative Position}

Geometric objects in \cgal\ have member functions that test the
position of a point relative to the object.  Full dimensional objects
and their boundaries are represented by the same type, e.g.\ 
halfspaces and hyperplanes are not distinguished, neither are balls
and spheres. Such objects split the ambient space into two
full-dimensional parts, a bounded part and an unbounded part (e.g.\ 
spheres), or two unbounded parts (e.g.\ hyperplanes). By default these
objects are oriented, i.e., one of the resulting parts is called the
positive side, the other one is called the negative side. Both of
these may be unbounded.

For these objects there is a function \ccStyle{oriented_side()} that
determines whether a test point is on the positive side, the negative
side, or on the oriented boundary. These function returns a value of
type \ccc{Oriented_side}.

Those objects that split the space in a bounded and an unbounded part,
have a member function \ccStyle{bounded_side()} with return type
\ccc{Bounded_side}.

If an object is lower dimensional, e.g.\ a segment in $d$-dimensional
space, there is only a test whether a point belongs to the object or
not. This member function, which takes a point as an argument and
returns a Boolean value, is called \ccStyle{has_on()}

