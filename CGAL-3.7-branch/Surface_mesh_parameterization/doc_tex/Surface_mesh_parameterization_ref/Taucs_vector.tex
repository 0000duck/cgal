% +------------------------------------------------------------------------+
% | Reference manual page: Taucs_vector.tex
% +------------------------------------------------------------------------+
% | 21.09.2005   Laurent Saboret, Pierre Alliez, Bruno Levy
% | Package: Surface_mesh_parameterization
% |
\RCSdef{\RCSTaucsvectorRev}{$Id$}
\RCSdefDate{\RCSTaucsvectorDate}{$Date$}
% |
\ccRefPageBegin
%%RefPage: end of header, begin of main body
% +------------------------------------------------------------------------+


\begin{ccRefClass}{Taucs_vector<T>}

%% \ccHtmlCrossLink{}     %% add further rules for cross referencing links
%% \ccHtmlIndexC[class]{} %% add further index entries


\ccDefinition

% The section below is automatically generated. Do not edit!
%START-AUTO(\ccDefinition)

The class \ccc{Taucs_vector} is a C++ wrapper around TAUCS' vector type, which is a simple array.

%END-AUTO(\ccDefinition)

% The section below is automatically generated. Do not edit!
%START-AUTO(\ccInclude)

\ccInclude{CGAL/Taucs_vector.h}

%END-AUTO(\ccInclude)


\ccIsModel

% The section below is automatically generated. Do not edit!
%START-AUTO(\ccIsModel)

Model of the \ccc{SparseLinearAlgebraTraits_d::Vector} concept.

%END-AUTO(\ccIsModel)


\ccParameters

The full template declaration is:

% The section below is automatically generated. Do not edit!
%START-AUTO(\ccParameters)

template$<$class T$>$   \\
class \ccc{Taucs_vector};

%END-AUTO(\ccParameters)


\ccTypes

% The section below is automatically generated. Do not edit!
%START-AUTO(\ccTypes)

\ccNestedType{NT}
{
}
\ccGlue

%END-AUTO(\ccTypes)


\ccCreation
\ccCreationVariable{v}  %% variable name for \ccMethod

% The section below is automatically generated. Do not edit!
%START-AUTO(\ccCreation)

\ccConstructor{Taucs_vector(int dimension);}
{
Create a vector initialized with zeros.
}
\ccGlue
\ccConstructor{Taucs_vector(const Taucs_vector<T>& toCopy);}
{
Copy constructor.
}
\ccGlue

%END-AUTO(\ccCreation)


\ccOperations

% The section below is automatically generated. Do not edit!
%START-AUTO(\ccOperations)

\ccMethod{int dimension() const;}
{
Return the vector's number of coefficients.
}
\ccGlue
\ccMethod{T operator[](int i) const;}
{
Read/write access to a vector coefficient.
\ccCommentHeading{Preconditions} 0 $<$= i $<$ dimension().
}
\ccGlue
\ccMethod{T& operator[](int i);}
{
}
\ccGlue
\ccMethod{const T* get_taucs_vector() const;}
{
Get TAUCS vector wrapped by this object.
}
\ccGlue
\ccMethod{T* get_taucs_vector();}
{
}
\ccGlue

%END-AUTO(\ccOperations)


\ccSeeAlso

\ccRefIdfierPage{CGAL::Taucs_solver_traits<T>}  \\
\ccRefIdfierPage{CGAL::Taucs_symmetric_solver_traits<T>}  \\
\ccRefIdfierPage{CGAL::Taucs_matrix<T>}  \\
\ccRefIdfierPage{CGAL::Taucs_symmetric_matrix<T>}  \\
\ccc{OpenNL::DefaultLinearSolverTraits<COEFFTYPE, MATRIX, VECTOR, SOLVER>} in OpenNL package  \\
\ccc{OpenNL::SymmetricLinearSolverTraits<COEFFTYPE, MATRIX, VECTOR, SOLVER>} in OpenNL package  \\


\end{ccRefClass}

% +------------------------------------------------------------------------+
%%RefPage: end of main body, begin of footer
\ccRefPageEnd
% EOF
% +------------------------------------------------------------------------+

