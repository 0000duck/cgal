\begin{ccRefFunction}{complement}
\label{ref_bso_complement}

\ccThree{OutputIterator}{complement}{}
\ccThreeToTwo

\ccDefinition

\ccInclude{CGAL/Boolean_set_operations_2.h}

\ccGlobalFunction{void complement(const Type & pgn, Type & res);}
{Each one of these functions computes the complement of a given
polygon \ccc{pgn}, and stores the resulting polygon with holes in
\ccc{res}.}

\begin{ccTexOnly}
\begin{longtable}[c]{|l|}
\multicolumn{1}{l}{\sl \ \ }
\endfirsthead
\multicolumn{1}{l}{\sl continued}
\endhead
\hline
\textbf{Arg Type}\\
\hline
\hline
\ccc{Polygon_2}\\
\hline
\ccc{General_polygon_2}\\
\hline
\end{longtable}
\end{ccTexOnly}

\begin{ccHtmlOnly}
<div align="center">
<table cellpadding=3 border="1">
<tr><th> Arg type</th></tr>
<tr><td valign="center">Polygon_2</td></tr>
<tr><td valign="center">General_polygon_2</td></tr>
</table>
</div>
\end{ccHtmlOnly}

\ccGlobalFunction{
OutputIterator complement(const Type & pgn, OutputIterator oi);}
{Each one of these functions computes the complement of a given
polygon \ccc{pgn}, inserts the resulting polygons with
holes into an output container through a given output iterator
\ccc{oi}, and returns the output iterator. The value type of the
\ccc{OutputIterator} is either \ccc{Polygon_with_holes_2} or 
\ccc{General_polygon_with_holes_2}.}

\begin{ccTexOnly}
\begin{longtable}[c]{|l|}
\multicolumn{1}{l}{\sl \ \ }
\endfirsthead
\multicolumn{1}{l}{\sl continued}
\endhead
\hline
\textbf{Arg Type}\\
\hline
\hline
\ccc{Polygon_with_holes_2}\\
\hline
\ccc{General_polygon_with_holes_2}\\
\hline
\end{longtable}
\end{ccTexOnly}

\begin{ccHtmlOnly}
<div align="center">
<table cellpadding=3 border="1">
<tr><th> Arg type</th></tr>
<tr><td valign="center">Polygon_with_holes_2</td></tr>
<tr><td valign="center">General_polygon_with_holes_2</td></tr>
</table>
</div>
\end{ccHtmlOnly}

\ccGlobalFunction{template <class Kernel, class Container>
void complement(const Polygon_2<Kernel, Container> & pgn,
                Polygon_with_holes_2<Kernel, Container> & res);}
\ccGlue
\ccGlobalFunction{template <class Traits>
void complement(const General_polygon_2<Traits> & pgn,
                General_polygon_with_holes_2<Traits> & res);}

\ccGlobalFunction{template <class Traits, class OutputIterator>
OutputIterator complement(const Polygon_with_holes_2<Kernel, Container> & pgn,
                          OutputIterator oi);}
\ccGlue
\ccGlobalFunction{template <class Traits, class OutputIterator>
OutputIterator complement(const General_polygon_with_holes_2<General_polygon_2<Traits> > & pgn,
                          OutputIterator oi);}

\ccSeeAlso
\ccRefIdfierPage{CGAL::do_intersect}\\
\ccRefIdfierPage{CGAL::intersection}\\
\ccRefIdfierPage{CGAL::join}\\
\ccRefIdfierPage{CGAL::difference}\\
\ccRefIdfierPage{CGAL::symmetric_difference}

\end{ccRefFunction}

\ccRefPageEnd
