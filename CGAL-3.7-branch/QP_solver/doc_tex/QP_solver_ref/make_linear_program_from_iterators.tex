\begin{ccRefFunction}{make_linear_program_from_iterators}

\ccInclude{CGAL/QP_models.h}

This template function creates an instance of 
\ccc{Linear_program_from_iterators<A_it, B_it, R_it, FL_it, L_it, FU_it, U_it, C_it>} from given iterators. This function can be useful if the types of these
iterators are too complicated (or of too little interest for you) 
to write them down explicitly.

\ccFunction{template <
   A_it,   
   B_it,   
   R_it,   
   FL_it,  
   L_it,   
   FU_it, 
   U_it,   
   C_it >  
Linear_program_from_iterators<A_it, B_it, R_it, FL_it, L_it, FU_it, U_it, C_it>
make_linear_program_from_iterators (
   int n, int m, 
   const A_it& a, 
   const B_it& b, 
   const R_it& r, 
   const FL_it& fl, 
   const L_it& l,
   const FU_it& fu, 
   const U_it& u, 
   const C_it& c, 
   std::iterator_traits<C_it>::value_type c0 = 
   std::iterator_traits<C_it>::value_type(0));}
{returns an instance of \ccc{Linear_program_from_iterators<A_it, B_it, R_it, FL_it, L_it, FU_it, U_it, C_it>}, constructed from the given iterators.}

\ccExample
The following example demonstrates the typical usage of makers
with the simpler function \ccc{make_nonnegative_linear_program_from_iterators}.

\ccReferToExampleCode{QP_solver/solve_convex_hull_containment_lp2.h}\\
\ccReferToExampleCode{QP_solver/convex_hull_containment2.cpp}

\ccSeeAlso
\ccc{Linear_program_from_iterators<A_it, B_it, R_it, FL_it, L_it, FU_it, U_it, C_it>}
\end{ccRefFunction}
