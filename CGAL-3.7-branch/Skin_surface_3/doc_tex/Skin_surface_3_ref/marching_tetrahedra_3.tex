% +------------------------------------------------------------------------+
% | Reference manual page: marching_tetrahedra_3.tex
% +------------------------------------------------------------------------+
% | 27.09.2005   Nico Kruithof
% | Package: Skin_surface_3
% | 
\RCSdef{\RCSmarchingtetrahedraRev}{$Id$}
\RCSdefDate{\RCSmarchingtetrahedraDate}{$Date$}
% |
%%RefPage: end of header, begin of main body
% +------------------------------------------------------------------------+


\begin{ccRefFunction}{marching_tetrahedra_3}

\ccDefinition

The function \ccRefName\ implements the marching tetrahedra algorithm
\cite{cgal:tpg-rmtiise-99}. To allow maximal flexibility, it uses the
traits class \ccc{MarchingTetrahedraTraits_3} for labeling the
vertices as inside and outside and the construction of vertices on
edges of the triangulation. The observer class
\ccc{MarchingTetrahedraObserver_3} is used to add callbacks after the
insertion of an vertex or facet of the polyhedron.

The input is a triangulation, a class implementing the
\ccc{MarchingTetrahedraTraits_3} and an implementation of the
\ccc{MarchingTetrahedraObserver_3}, which is empty by default. The
triangulation has to define the types \ccc{Finite_cells_iterator} and
\ccc{Vertex_handle}. The vertices are accessed by the function
\ccc{vertex(i)}, with $i=0,1,2,3$ and each tetrahedron has a positive
orientation. A Strict Weak Ordering and equality has to be defined for
\ccc{Vertex_handle}. These requirements are satisfied by
\ccc{Triangulation_3}.

The output is a \ccc{Polyhedron_3} and is constructed using the
\ccc{Polyhedral_incremental_builder_3}.

\ccInclude{CGAL/marching_tetrahedra_3.h}

\ccParameters The first template argument defines the type of the
triangulation that forms the input for the marching tetrahedra
algorithm. The second parameter is the type of the polyhedral mesh in
which the extracted surface is stored. The third template argument is
a model of the \ccc{MarchingTetrahedraTraits_3} concept and defines
which vertices lie inside and which outside the surface. It also
provides a function to obtain a vertex from the intersection of an
edge with the surface. The last template argument is a model of the
\ccc{MarchingTetrahedraObserver_3} concept.

\ccGlobalFunction{
template <class Triangulation_3,
	  class Polyhedron_3,
	  class MarchingTetrahedraTraits_3,
          class MarchingTetrahedraObserver_3 >
void marching_tetrahedra_3(
  const Triangulation_3 &triangulation,
  Polyhedron_3 &polyhedron,
  const MarchingTetrahedraTraits_3 &marching_traits,
  const MarchingTetrahedraObserver_3 &marching_observer = 
    CGAL::Marching_tetrahedra_observer_default_3());}

\ccSeeAlso
\ccRefIdfierPage{MarchingTetrahedraTraits_3}\\
\ccRefIdfierPage{MarchingTetrahedraObserver_3}\\
\ccRefIdfierPage{Polyhedron_3}\\
\ccRefIdfierPage{Triangulation_3}

\end{ccRefFunction}

% +------------------------------------------------------------------------+
%%RefPage: end of main body, begin of footer
% EOF
% +------------------------------------------------------------------------+

