% +------------------------------------------------------------------------+
% | Reference manual page: Skin_surface_quadratic_surface_3.tex
% +------------------------------------------------------------------------+
% | 27.09.2005   Nico Kruithof
% | Package: Skin_surface
% | 
\RCSdef{\RCSSkinsurfacequadraticsurfaceRev}{$Id$}
\RCSdefDate{\RCSSkinsurfacequadraticsurfaceDate}{$Date$}
% +------------------------------------------------------------------------+

\ccRefPageBegin

%%RefPage: end of header, begin of main body
% +------------------------------------------------------------------------+


\begin{ccRefClass}{Skin_surface_quadratic_surface_3<SkinSurfaceQuadraticSurfaceTraits_3>}
\label{refSkinsurfacequadraticsurface}
\ccDefinition
  
The class \ccRefName\ defines a quadratic surface designed for the
skin surface package. The class stores the defining matrix $Q$, the
apex $a$, and the function value $c$ at the apex. The surface is then
defined as the zero-set of the function $(x-a)^T\cdot Q\cdot (x-a)+c$.

The template parameter should be a model of the concept
\ccc{SkinSurfaceQuadraticSurfaceTraits_3}.

\ccInclude{CGAL/Skin_surface_quadratic_surface_3.h}

\ccTypes
The geometric types used by the quadratic surface:
\ccThree{typedef Weighted_point<Point, RT> x}{Triangulated}{}
\ccThreeToTwo

\ccTypedef{typedef  Kernel::RT                  RT;}{}
\ccGlue
\ccTypedef{typedef  Kernel::Point_3             Point;}{}
\ccGlue
\ccTypedef{typedef  Kernel::Vector_3            Vector;}{}
\ccGlue
\ccTypedef{typedef  Kernel::Segment_3           Segment;}{}

\ccCreation
\ccCreationVariable{surf}  %% choose variable name

\ccConstructor{Skin_surface_quadratic_surface_3(RT Q[], Point a, RT
  c);}{Constructs a quadratic surface with defining matrix $Q$, apex
  $a$ and function value $c$ at the apex.}

%\ccOperations
%\ccAccessFunctions
\ccQueryFunctions

\ccThree{Input_point}{to_surface(Point p0, Point p1)asdfasdf}{}
\ccThreeToTwo

\ccMethod{ template <class Input_point> RT value( const Input_point &
  x) const;} {Returns the function value at the point $x$.
  \ccPrecond{\ccc{Input_point} is convertible to \ccc{Point} using the
    \ccc{Cartesian_converter}.}}
%
\ccGlue \ccMethod{template <class Input_point> Sign sign(const
  Input_point & x) const;}{Returns the sign of the function value at
  the point $x$.  \ccPrecond{\ccc{Input_point} is convertible to
    \ccc{Point} using the \ccc{Cartesian_converter}.}}

\ccMethod{template <class Input_point> typename
  Input_point::R::Vector_3 gradient(const Input_point & x);}{Returns
  the gradient at the point $x$.  \ccPrecond{\ccc{Input_point} is
    convertible to \ccc{Point} and \ccc{Input_point::R::Vector_3} to
    \ccc{Vector} using the \ccc{Cartesian_converter}.}}

\ccMethod{template <class Input_point> Input_point to_surface(const
  Input_point &p0, const Input_point &p1);}{ Returns an intersection
  point of the quadratic surface with the line segment p0p1.
  \ccPrecond{\ccc{Input_point} is convertible to \ccc{Point} using the
    \ccc{Cartesian_converter}.}}

\ccSeeAlso
\ccRefIdfierPage{SkinSurfaceQuadraticSurfaceTraits_3}
\end{ccRefClass}

