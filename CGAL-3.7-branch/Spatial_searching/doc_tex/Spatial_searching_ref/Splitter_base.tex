% +------------------------------------------------------------------------+
% | Reference manual page: Splitter.tex
% +------------------------------------------------------------------------+
% | 1.07.2001   Johan W.H. Tangelder
% | Package: ASPAS
% | 
\RCSdef{\RCSSplitterRev}{$Id$}
\RCSdefDate{\RCSSplitterDate}{$Date$}
% |
%%RefPage: end of header, begin of main body
% +------------------------------------------------------------------------+


\begin{ccRefClass}{Splitter_base<FT>}

%% \ccHtmlCrossLink{}     %% add further rules for cross referencing links
%% \ccHtmlIndexC[concept]{} %% add further index entries

\begin{ccAdvanced}

\ccDefinition
  
The class \ccRefName\ serves as a base class  for implementing splitting rules.

\ccTypes

\ccNestedType{FT}{Number type.}
\ccNestedType{Container}{Point container.}
\ccNestedType{Separator}{Separator.} 

\ccCreationVariable{s}  %% choose variable name

The parameters \ccc{aspect_ratio} and  \ccc{bucket_size}
define the way in which $k$-$d$ tree is constructed.

\ccOperations

\ccMethod{
void operator()(Separator& s, Container& c0, Container& c1);} 
{Modifies the separator $s$, 
and splits \ccc{c0} into \ccc{c0} and \ccc{c1},
using the splitting dimension and the splitting value of the modified separator.
}

\ccMethod{FT aspect_ratio();}{Returns the maximal ratio between the largest and smallest side
of a cell allowed for fair splitting.}

\ccMethod{int bucket_size();} {Returns the bucket size of the leaf nodes.}


 

\ccSeeAlso
\ccc{CGAL::Fair<Traits, PointContainer, SpatialSeparator>}, \\
\ccc{CGAL::Median_of_rectangle<Traits, PointContainer, SpatialSeparator>}, \\
\ccc{CGAL::Median_of_max_spread<Traits, PointContainer, SpatialSeparator>}, \\
\ccc{CGAL::Midpoint_of_rectangle<Traits, PointContainer, SpatialSeparator>}, \\
\ccc{CGAL::Midpoint_of_max_spread<Traits, PointContainer, SpatialSeparator>}, \\
\ccc{CGAL::Sliding_fair<Traits, PointContainer, SpatialSeparator>}, \\
\ccc{CGAL::Sliding_midpoint<Traits, PointContainer, SpatialSeparator>.}

\end{ccAdvanced}

\end{ccRefClass}

% +------------------------------------------------------------------------+
%%RefPage: end of main body, begin of footer
% EOF
% +------------------------------------------------------------------------+

