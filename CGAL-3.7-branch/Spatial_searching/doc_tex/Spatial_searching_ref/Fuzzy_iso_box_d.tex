% +------------------------------------------------------------------------+
% | Reference manual page: Fuzzy_iso_box.tex
% +------------------------------------------------------------------------+
% | 1.07.2001   Johan W.H. Tangelder
% | Package: ASPAS
% | 
\RCSdef{\RCSFuzzyisoboxdRev}{$Id$}
\RCSdefDate{\RCSFuzzyisoboxdDate}{$Date$}
% |
%%RefPage: end of header, begin of main body
% +------------------------------------------------------------------------+


\begin{ccRefClass}{Fuzzy_iso_box<Traits>}  %% add template arg's if necessary

%% \ccHtmlCrossLink{}     %% add further rules for cross referencing links
%% \ccHtmlIndexC[class]{} %% add further index entries

\ccDefinition
  
The class \ccRefName\ implements fuzzy $d$-dimensional iso boxes.  A
fuzzy iso box with fuzziness value $\epsilon$ has as outer
approximation a box dilated, and as inner approximation a box eroded
by a $d$-dim square with side length $\epsilon$.

\ccInclude{CGAL/Fuzzy_iso_box.h}

\ccParameters
Expects for the template argument a model of the concept
\ccc{SearchTraits}, for example \ccc{CGAL::Search_traits_2<CGAL::Simple_cartesian<double> >}.


 
\ccIsModel

\ccc{FuzzyQueryItem}

\ccTypes

\ccTypedef{Traits::Point_d Point_d;}{Point type.}

\ccTypedef{Traits::FT FT;}{Number type.}

\ccCreation
\ccCreationVariable{b}  %% choose variable name

\ccConstructor{Fuzzy_iso_box(Point_d p, Point_d q, FT epsilon=FT(0));}{Constructs a fuzzy iso box
specified by the minimal iso box containing \ccc{p} and \ccc{q} and fuzziness value \ccc{epsilon}.
\ccPrecond{ \ccc{p} must be lexicographically smaller than \ccc{q}.}}

\ccOperations

\ccMethod{bool contains(Point_d p);}{test whether \ccVar\ contains \ccc{p}.}

\ccMethod{bool inner_range_intersects(const Kd_tree_rectangle<FT>& rectangle);}
{test whether the inner box intersects the rectangle
associated with a node of a tree.}

\ccMethod{bool outer_range_contains(const Kd_tree_rectangle<FT>& rectangle);}
{test whether the outer box encloses the rectangle associated with a node of a tree.}

\ccSeeAlso

\ccc{FuzzyQueryItem}

\end{ccRefClass}

% +------------------------------------------------------------------------+
%%RefPage: end of main body, begin of footer
% EOF
% +------------------------------------------------------------------------+

