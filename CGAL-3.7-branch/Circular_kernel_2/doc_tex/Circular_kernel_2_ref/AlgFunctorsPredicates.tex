\begin{ccRefConcept}{AlgebraicKernelForCircles::CompareX}

\ccCreationVariable{fo}

A model \ccVar\ of this type must provide:

\ccMethod{template < class OutputIterator >
    CGAL::Comparison_result
    operator()(const AlgebraicKernelForCircles::Root_for_circles_2_2 & r1,
	     const AlgebraicKernelForCircles::Root_for_circles_2_2 & r2);}
{Compares the \ccc{x} (first) variables of two \ccc{Root_for_circles_2_2}.}

\ccSeeAlso

\ccRefIdfierPage{AlgebraicKernelForCircles::CompareY}\\
\ccRefIdfierPage{AlgebraicKernelForCircles::CompareXY}\\
\ccRefIdfierPage{CircularKernel::CompareX_2}

\end{ccRefConcept}
\begin{ccRefConcept}{AlgebraicKernelForCircles::CompareY}

\ccDefinition

\ccCreationVariable{fo}

A model \ccVar\ of this type must provide:

\ccMethod{template < class OutputIterator >
    CGAL::Comparison_result
    operator()(const AlgebraicKernelForCircles::Root_for_circles_2_2 & r1,
	     const AlgebraicKernelForCircles::Root_for_circles_2_2 & r2);}
{Compares the \ccc{y} (second) variables of two \ccc{Root_for_circles_2_2}.}

\ccSeeAlso

\ccRefIdfierPage{AlgebraicKernelForCircles::CompareX}\\
\ccRefIdfierPage{AlgebraicKernelForCircles::CompareXY}\\
\ccRefIdfierPage{CircularKernel::CompareY_2}

\end{ccRefConcept}
\begin{ccRefConcept}{AlgebraicKernelForCircles::CompareXY}

\ccDefinition

\ccCreationVariable{fo}

A model \ccVar\ of this type must provide:

\ccMethod{template < class OutputIterator >
    CGAL::Comparison_result
    operator()(const AlgebraicKernelForCircles::Root_for_circles_2_2 & r1,
	     const AlgebraicKernelForCircles::Root_for_circles_2_2 & r2);}
{Compares two \ccc{Root_for_circles_2_2} lexicographically.}

\ccSeeAlso

\ccRefIdfierPage{AlgebraicKernelForCircles::CompareX}\\
\ccRefIdfierPage{AlgebraicKernelForCircles::CompareY}\\
\ccRefIdfierPage{CircularKernel::CompareXY_2}

\end{ccRefConcept}

\begin{ccRefConcept}{AlgebraicKernelForCircles::SignAt}

\ccDefinition

\ccCreationVariable{fo}

A model \ccVar\ of this type must provide:

\ccMethod{template < class OutputIterator >
    CGAL::Sign
    operator()(const AlgebraicKernelForCircles::Polynomial_1_2 & p,
	     const AlgebraicKernelForCircles::Root_for_circles_2_2 & r);}
{Computes the sign of polynomial \ccc{p} evaluated at a root \ccc{r}.}

\ccMethod{template < class OutputIterator >
    CGAL::Sign
    operator()(const AlgebraicKernelForCircles::Polynomial_for_circles_2_2 & p,
	     const AlgebraicKernelForCircles::Root_for_circles_2_2 & r);}
{Same as previous.}

\end{ccRefConcept}

