\begin{ccRefFunctionObjectConcept}{SphericalKernel::HasOn_3}

\ccRefines

\ccc{Kernel::HasOn_3}

\ccCreationVariable{fo}

An object \ccVar\ of this type must provide:

\ccMemberFunction{bool 
	operator()(const Type1 &obj1, const Type2 &obj2);}
{Returns \ccc{true} when \ccc{obj1} contains \ccc{obj2},}

where \ccc{Type1} and \ccc{Type2} can be respectively:

\begin{tabular}{|l|l|}
\hline
\ccc{Type1} & \ccc{Type2} \\
\hline
\ccc{SphericalKernel::Sphere_3}, & \ccc{SphericalKernel::Circular_arc_point_3}\\
\ccc{SphericalKernel::Plane_3}, & \\
\ccc{SphericalKernel::Line_3}, & \\
\ccc{SphericalKernel::Circle_3}, & \\
\ccc{SphericalKernel::Line_arc_3}, or & \\
\ccc{SphericalKernel::Circular_arc_3} & \\
\hline
\ccc{SphericalKernel::Plane_3}, & \ccc{SphericalKernel::Line_arc_3}\\
\ccc{SphericalKernel::Line_3} & \\
\hline
\ccc{SphericalKernel::Sphere_3}, & \ccc{SphericalKernel::Circular_arc_3} \\
\ccc{SphericalKernel::Circle_3}, or & \\
\ccc{SphericalKernel::Plane_3} & \\
\hline
\end{tabular}

\end{ccRefFunctionObjectConcept}
\begin{ccRefFunctionObjectConcept}{SphericalKernel::DoOverlap_3}

\ccRefines

\ccc{Kernel::DoOverlap_3}

\ccCreationVariable{fo}

An object \ccVar\ of this type must provide:

\ccMemberFunction{bool operator()
	(const SphericalKernel::Line_arc_3 & l0,
	const SphericalKernel::Line_arc_3 & l1,
	const bool known_equal_supporting_line = false);}
{For two line arcs.}

\ccMemberFunction{bool operator()
	(const SphericalKernel::Circular_arc_3 & a0,
	const SphericalKernel::Circular_arc_3 & a1,
	const bool known_equal_supporting_circle = false);}
{For two circular arcs.}

The computation may be faster when the boolean is set to \ccc{true}.

\end{ccRefFunctionObjectConcept}
\begin{ccRefFunctionObjectConcept}{SphericalKernel::DoIntersect_3}

\ccDefinition

Testing whether two curves intersect.

\ccRefines

\ccc{Kernel::DoIntersect_3}

\ccCreationVariable{fo}

An object \ccVar\ of this type must provide:

\ccMemberFunction{bool operator()
	(const Type1 & obj1, const Type2 & obj2);}
{determines if two geometric objects of type Type1 and Type2 intersect or not.}

for all pairs \ccc{Type1} and \ccc{Type2}, where the types 
\ccc{Type1} and \ccc{Type2} can be  either, any of the following:
\begin{itemize}
\item {} \ccc{SphericalKernel::Plane_3}
\item {} \ccc{SphericalKernel::Line_3}
\item {} \ccc{SphericalKernel::Line_arc_3}
\item {} \ccc{SphericalKernel::Sphere_3}
\item {} \ccc{SphericalKernel::Circle_3}
\end{itemize} 

\ccSeeAlso

\ccRefIdfierPage[Kernel::do_intersect]{CGAL::do_intersect}

\end{ccRefFunctionObjectConcept}
\begin{ccRefFunctionObjectConcept}{SphericalKernel::BoundedSide_3}

\ccRefines

\ccc{Kernel::BoundedSide_3}

\ccCreationVariable{fo}

An object \ccVar\ of this type must provide:

\ccMemberFunction{Bounded_side 
	operator()
	(const SphericalKernel::Sphere_3& s, 
	const SphericalKernel::Circular_arc_point_3& p);}
{For a sphere.}

\ccMemberFunction{Bounded_side
	operator()
	(const SphericalKernel::Circle_3& s, 
	const SphericalKernel::Circular_arc_point_3& p);}
{For a circle.}

\ccSeeAlso

\ccRefIdfierPage{SphericalKernel::HasOnBoundedSide_3}\\
\ccRefIdfierPage{SphericalKernel::HasOnUnboundedSide_3}

\end{ccRefFunctionObjectConcept}
\begin{ccRefFunctionObjectConcept}{SphericalKernel::HasOnBoundedSide_3}

\ccRefines

\ccc{Kernel::HasOnBoundedSide_3}

\ccCreationVariable{fo}

An object \ccVar\ of this type must provide:

\ccMemberFunction{bool
	operator()
	(const SphericalKernel::Sphere_3& s, 
	const SphericalKernel::Circular_arc_point_3& p);}
{For a sphere.}

\ccMemberFunction{bool
	operator()
	(const SphericalKernel::Circle_3& s, 
	const SphericalKernel::Circular_arc_point_3& p);}
{For a circle.}

\ccSeeAlso

\ccRefIdfierPage{SphericalKernel::BoundedSide_3}\\
\ccRefIdfierPage{SphericalKernel::HasOnUnboundedSide_3}

\end{ccRefFunctionObjectConcept}
\begin{ccRefFunctionObjectConcept}{SphericalKernel::HasOnUnboundedSide_3}

\ccRefines

\ccc{Kernel::HasOnUnboundedSide_3}

\ccCreationVariable{fo}

An object \ccVar\ of this type must provide:

\ccMemberFunction{bool
	operator()
	(const SphericalKernel::Sphere_3& s, 
	const SphericalKernel::Circular_arc_point_3& p);}
{For a sphere.}

\ccMemberFunction{bool
	operator()
	(const SphericalKernel::Circle_3& s, 
	const SphericalKernel::Circular_arc_point_3& p);}
{For a circle.}

\ccSeeAlso

\ccRefIdfierPage{SphericalKernel::BoundedSide_3}\\
\ccRefIdfierPage{SphericalKernel::HasOnBoundedSide_3}

\end{ccRefFunctionObjectConcept}


\begin{ccRefFunctionObjectConcept}{SphericalKernel::IsThetaMonotone_3} 

\ccCreationVariable{fo}

An object \ccVar\ of this type must provide: 

% \ccConstructor{IsThetaMonotone_3(const Sphere_3 &sphere)}
% {Constructs a functor \ccVar\ to test if an arc is $\theta$-monotone relatively to \ccc{sphere}.}

\ccMemberFunction{bool operator()
	(const SphericalKernel::Circular_arc_3 &a);}
{Tests whether the arc $a$ is $\theta$-monotone, i.e. the intersection of
any meridian anchored at the poles of the context sphere used by the function \ccc{SphericalKernel::is_theta_monotone_3_object}
and the arc $a$ is reduced to at most one point in general, and two points if a pole of that sphere is
an endpoint of \ccc{a}. Note that a bipolar circle has no such arcs.
\ccPrecond{\ccc{a} lies on the context sphere used by the function \ccc{SphericalKernel::is_theta_monotone_3_object},
and the supporting circle of \ccc{a} is not bipolar.}
}


\ccSeeAlso

\ccRefIdfierPage{SphericalKernel::MakeThetaMonotone_3}


\end{ccRefFunctionObjectConcept}
