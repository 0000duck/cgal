% +------------------------------------------------------------------------+
% | Reference manual page: approx_offset.tex
% +------------------------------------------------------------------------+
% | 
% | Package: Minkowski_sum_2
% | 
% +------------------------------------------------------------------------+

\ccRefPageBegin

\begin{ccRefFunction}{approximated_offset_2}

\ccInclude{CGAL/approximated_offset_2.h}

\ccFunction{template<class Kernel, class Container>
            typename Gps_circle_segment_traits_2<Kernel>::Polygon_with_holes_2
            approximated_offset_2 (const Polygon_2<Kernel, Container>& P,
                                   const typename Kernel::FT& r,
                                   const double& eps);}
   {Provides a guaranteed approximation of the offset of the given polygon
    \ccc{P} by a given radius \ccc{r} --- namely, the function computes the
    Minkowski sum $P \oplus B_r$, where $B_r$ is a disc of radius
    \ccc{r} centered at the origin.
    The function actually outputs a set $S$ that contains the Minkowski sum,
    such that the approximation error is bounded by \ccc{eps}.
    Note that as the input polygon may not be convex, its offset may not be a
    simple polygon. The result is therefore represented as a polygon with
    holes, whose edges are either line segments or circular arcs.
    \ccPrecond{\ccc{P} is a simple polygon.}}

\ccFunction{template<class Kernel, class Container>
            typename Gps_circle_segment_traits_2<Kernel>::Polygon_with_holes_2
            approximated_offset_2 (const Polygon_with_holes_2<Kernel, Container>& wh,
                                   const typename Kernel::FT& r,
                                   const double& eps);}
   {Provides a guaranteed approximation of offset the given polygon with holes
    \ccc{pwh} by a given radius \ccc{r}, such that the approximation error is bounded
    by \ccc{eps}. It does so by offsetting outer boundary of \ccc{pwh} and insetting
    its holes.
    The result is represented as a generalized polygon with holes, such that the edges
    of the polygon correspond to line segment and circular arcs.
    \ccPrecond{\ccc{pwh} is {\em not} unbounded (it has a valid outer boundary).}}


\ccFunction{template<class Kernel, class Container,
                     class DecompositionStrategy>
            typename Gps_circle_segment_traits_2<Kernel>::Polygon_with_holes_2
            approximated_offset_2 (const Polygon_2<Kernel, Container>& P,
                                   const typename Kernel::FT& r,
                                   const double& eps,
                                   const DecompositionStrategy& decomp);}
   {Provides a guaranteed approximation of the offset of the given polygon
    \ccc{P} by a radius \ccc{r}, as described above.
    If the input polygon \ccc{P} is not convex, the function
    decomposes it into convex sub-polygons $P_1, \ldots, P_k$ and computes
    the union of the sub-offsets (namely $\bigcup_{i}{(P_i \oplus B_r)}$).
    The decomposition is performed using the given decomposition strategy
    \ccc{decomp}, which must be an instance of a class that models the
    concept \ccc{PolygonConvexDecomposition}.
    \ccPrecond{\ccc{P} is a simple polygon.}}

\end{ccRefFunction}

\ccRefPageEnd
