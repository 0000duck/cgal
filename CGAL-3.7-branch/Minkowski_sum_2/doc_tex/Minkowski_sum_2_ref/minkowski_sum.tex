% +------------------------------------------------------------------------+
% | Reference manual page: minkowski_sum.tex
% +------------------------------------------------------------------------+
% | 
% | Package: Minkowski_sum_2
% | 
% +------------------------------------------------------------------------+

\ccRefPageBegin

\begin{ccRefFunction}{minkowski_sum_2}

\ccInclude{CGAL/minkowski_sum_2.h}

\ccFunction{template<class Kernel, class Container>
            Polygon_with_holes_2<Kernel,Container>
            minkowski_sum_2 (const Polygon_2<Kernel,Container>& P,
                             const Polygon_2<Kernel,Container>& Q);}
   {Computes the Minkowski sum $P \oplus Q$ of the two given polygons.
    The function computes the convolution cycles of the two polygons and
    extract the regions having positive winding number with respect to these
    cycles. This method work very efficiently, regardless of whether \ccc{P}
    and \ccc{Q} are convex or non-convex.
    Note that as the input polygons may not be convex, their Minkowski
    sum may not be a simple polygon. The result is therefore represented
    as a polygon with holes.
    \ccPrecond{Both \ccc{P} and \ccc{Q} are simple polygons.}}

\ccFunction{template<class Kernel, class Container,
            class DecompositionStrategy>
            Polygon_with_holes_2<Kernel,Container>
            minkowski_sum_2 (const Polygon_2<Kernel,Container>& P,
                             const Polygon_2<Kernel,Container>& Q,
                             const DecompositionStrategy& decomp);}
   {Computes the Minkowski sum $P \oplus Q$ of the two given polygons.
    If the input polygons \ccc{P} and \ccc{Q} are not convex, the function
    decomposes them into convex sub-polygons $P_1, \ldots, P_k$ and
    $Q_1, \ldots, Q_{\ell}$ and computes the union of pairwise sub-sums
    (namely $\bigcup_{i,j}{(P_i \oplus Q_j)}$).
    The decomposition is performed using the given decomposition strategy
    \ccc{decomp}, which must be an instance of a class that models the
    concept \ccc{PolygonConvexDecomposition}.
    Note that as the input polygons may not be convex, their Minkowski
    sum may not be a simple polygon. The result is therefore represented
    as a polygon with holes.
    \ccPrecond{Both \ccc{P} and \ccc{Q} are simple polygons.}}

\end{ccRefFunction}

\ccRefPageEnd
