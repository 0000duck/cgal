% +------------------------------------------------------------------------+
% | Reference manual page: Ssab_decomp.tex
% +------------------------------------------------------------------------+
% | 
% | Package: Minkowski_sum_2
% | 
% +------------------------------------------------------------------------+

\ccRefPageBegin

\begin{ccRefClass}{Small_side_angle_bisector_decomposition_2<Kernel,Container>}
\label{mink_ref:ssab_decomp}

\ccDefinition
%============

The \ccRefName\ class implements a simple yet efficient heuristic for
decomposing an input polygon into convex sub-polygons. It is based
on the algorithm suggested by Flato and Halperin~\cite{fh-recpm-00},
but without introducing Steiner points. The algorithm operates in two
major steps. In the first step, it tries to subdivide the polygon by
connect two reflex vertices with an edge. When this is not possible any
more, it eliminates the reflex vertices one by one by connecting them
to other convex vertices, such that the new edge best approximates
the angle bisector of the reflex vertex. The algorithm operates in
$O(n^2)$ time an takes $O(n)$ space at the worst case, where $n$ is the
size of the input polygon.

The \ccc{Polygon_2} type defined by the class is simply
\ccc{Polygon_2<Kernel,Container>}. The \ccc{Container} parameter
is by default \ccc{std::vector<typename Kernel::Point_2>}.

\ccInclude{CGAL/Small_side_angle_bisector_decomposition_2.h}

\ccIsModel
  \ccc{PolygonConvexDecomposition_2}

\end{ccRefClass}

\ccRefPageEnd
