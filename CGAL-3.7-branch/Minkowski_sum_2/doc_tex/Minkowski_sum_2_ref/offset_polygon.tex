% +------------------------------------------------------------------------+
% | Reference manual page: offset_polygon.tex
% +------------------------------------------------------------------------+
% | 
% | Package: Minkowski_sum_2
% | 
% +------------------------------------------------------------------------+

\ccRefPageBegin

\begin{ccRefFunction}{offset_polygon_2}

\ccInclude{CGAL/offset_polygon_2.h}

\ccFunction{template<class ConicTraits, class Container>
            typename Gps_traits_2<ConicTraits>::Polygon_with_holes_2
            offset_polygon_2
                 (const Polygon_2<typename ConicTraits::Rat_kernel,
                                  Container>& P,
                  const typename ConicTraits::Rat_kernel::FT& r,
                  const ConicTraits& traits);}
   {Computes the offset of the given polygon \ccc{P} by a given radius
    \ccc{r} --- namely, the function computes the Minkowski sum
    $P \oplus B_r$, where $B_r$ is a disc of radius \ccc{r} centered at the
    origin.
    Note that as the input polygon may not be convex, its offset may not be a
    simple polygon. The result is therefore represented as a generalized
    polygon with holes, such that the edges of the polygon correspond to
    line segments and circular arcs, both are special types of conic arcs,
    as represented by the \ccc{traits} class.
    \ccPrecond{\ccc{P} is a simple polygon.}}

\ccFunction{template<class ConicTraits, class Container>
            typename Gps_traits_2<ConicTraits>::Polygon_with_holes_2
            offset_polygon_2
                 (const Polygon_with_holes_2<typename ConicTraits::Rat_kernel,
                                             Container>& pwh,
                  const typename ConicTraits::Rat_kernel::FT& r,
                  const ConicTraits& traits);}
   {Computes the offset of the given polygon with holes \ccc{pwh} by a given
    radius \ccc{r}. It does so by offsetting outer boundary of \ccc{pwh} and
    insetting its holes.
    The result is represented as a generalized polygon with holes, such that the
    edges of the polygon correspond to line segments and circular arcs, both are
    special types of conic arcs, as represented by the \ccc{traits} class.
    \ccPrecond{\ccc{pwh} is {\em not} unbounded (it has a valid outer boundary).}}

\ccFunction{template<class ConicTraits, class Container,
                     class DecompositionStrategy>
            typename Gps_traits_2<ConicTraits>::Polygon_with_holes_2
            offset_polygon_2
                 (const Polygon_2<typename ConicTraits::Rat_kernel,
                                  Container>& P,
                  const typename ConicTraits::Rat_kernel::FT& r,
                  const DecompositionStrategy& decomp,
                  const ConicTraits& traits);}
   {Computes the exact representation of the offset of the given polygon
    \ccc{P} by a radius \ccc{r}, as described above.
    If \ccc{P} is not convex, the function decomposes it into convex
    sub-polygons $P_1, \ldots, P_k$ and computes the union of sub-offsets
    (namely $\bigcup_{i}{(P_i \oplus B_r)}$).
    The decomposition is performed using the given decomposition strategy
    \ccc{decomp}, which must be an instance of a class that models the
    concept \ccc{PolygonConvexDecomposition}.
    \ccPrecond{\ccc{P} is a simple polygon.}}

\end{ccRefFunction}

\ccRefPageEnd
