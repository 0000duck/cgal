\section{Introduction}

A subset $S \subseteq \R^2$ is convex if for any two points $p$ and $q$
in the set the line segment with endpoints $p$ and $q$ is contained
in $S$. The convex hull\ccIndexMainItemDef{convex hull} of a set $S$ 
is the smallest convex set containing
$S$. The convex hull of a set of points $P$ is a convex 
polygon with vertices in $P$. A point in $P$ is an extreme point 
(with respect to $P$)\ccIndexMainItemDef{extreme point} if it is a vertex of 
the convex hull of $P$.  A set of points is said to be strongly convex %
\ccIndexMainItemDef{strongly convex} if it consists of only extreme points.

This chapter describes the functions provided in
\cgal\ for producing convex hulls in two dimensions as well as
functions for checking if sets of points are strongly convex are not.  
There are also a number of functions described for computing particular 
extreme points and subsequences of hull points, such as the lower and 
upper hull of a set of points.
