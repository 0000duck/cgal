% +------------------------------------------------------------------------+
% | Reference manual page: box_intersection_d.tex
% +------------------------------------------------------------------------+
% | 19.05.2004   Lutz Kettner, Andreas Meyer
% | Package: Package
% | 
\RCSdef{\RCSboxintersectiondRev}{$Id$}
\RCSdefDate{\RCSboxintersectiondDate}{$Date$}
% |
%%RefPage: end of header, begin of main body
% +------------------------------------------------------------------------+


\begin{ccRefFunction}{box_intersection_d}
%\ccHtmlCrossLink{CGAL::box_intersection_custom_predicates_d}
%\ccRefLabel{CGAL::box_intersection_custom_predicates_d}
\ccIndexSubitemDef{intersection}{iso-oriented boxes}
\ccIndexSubitemDef{iso-oriented boxes}{intersection}

\ccDefinition

The function \ccRefName\ computes the pairwise intersecting boxes
between two sequences of iso-oriented boxes in arbitrary dimension.
The sequences of boxes are given with two random-access iterator
ranges and will be reordered in the course of the algorithm. For each
intersecting pair of boxes a \ccc{callback} function object is called
with the two intersecting boxes as argument; the first argument is a
box from the first sequence, the second argument a box from the second
sequence.  The performance of the algorithm can be tuned with a
\ccc{cutoff} parameter, see the implementation section below for more
details.

The algorithm reorders the boxes in the course of the algorithm. Now,
depending on the size of a box it can be faster to copy the boxes, or
to work with pointers to boxes and copy only pointers. We offer
automatic support for both options. To simplify the description, let
us call the \ccc{value_type} of the iterators \emph{box handle}.  The
\emph{box handle\/} can either be our box type itself or a pointer (or
const pointer) to the box type.

A $d$-dimensional iso-oriented box is defined as the
Cartesian product of $d$ intervals. We call the
box \emph{half-open} if the $d$ intervals $\{ [lo_i,hi_i) \,|\, 0 \leq
i < d\}$ are half-open intervals, and we call the box \emph{closed} if
the $d$ intervals $\{ [lo_i,hi_i] \,|\, 0 \leq i < d\}$ are closed
intervals. Note that closed boxes support zero-width boxes and they
can intersect at their boundaries, while non-empty half-open boxes
always have a positive volume and they only intersect iff their
interiors overlap.  The distinction between closed or half-open boxes
does not require a different representation of boxes, just a different
interpretation when comparing boxes, which is selected with the
\ccc{topology} parameter and its two values,
\ccc{CGAL::Box_intersection_d::HALF_OPEN} and
\ccc{CGAL::Box_intersection_d::CLOSED}.

In addition, a box has an unique \ccc{id}-number. It is used to order
boxes consistently in each dimension even if boxes have identical
coordinates. In consequence, the algorithm guarantees that a pair of
intersecting boxes is reported only once. Boxes with equal
\ccc{id}-number are not reported since they obviously intersect trivially.

The algorithm uses a traits class of the \ccc{BoxIntersectionTraits_d}
concept to access the boxes. A default traits class is provided that
assumes that the box type is a model of the \ccc{BoxIntersectionBox_d}
concept and that the box handle, i.e., the iterators value type, is
identical to the box type or a pointer to the box type.

An important special application of this algorithm is the test for
self-intersections where the second box sequence is an identical copy
of the first sequence including the preserved \ccc{id}-number. Note
that this implies that the address of the box is not sufficient for
the \ccc{id}-number if boxes are copied by value. To ease the use of
this special case we offer a simplified version of this function with
one iterator range only, which then creates internally the second copy
of the boxes, under the name \ccc{CGAL::box_self_intersection_d}.

In the general case, we distinguish between the bipartite case (the
boxes are from different sequences) and the complete case (the boxes
are from the same sequence, i.e., the self intersection case). The
default is the bipartite case, since the complete case is typically
handled with the simplified function call mentioned above. However,
the general function call offers the \ccc{setting} parameter with the
values \ccc{CGAL::Box_intersection_d::COMPLETE} and
\ccc{CGAL::Box_intersection_d::BIPARTITE}.

\ccInclude{CGAL/box_intersection_d.h}

\ccThree{void}{box_inter}{}
%\def\ccLongParamLayout{\ccTrue}
\ccFunction{template< class RandomAccessIterator1, 
            class RandomAccessIterator2, 
            class Callback >
void box_intersection_d(
    RandomAccessIterator1 begin1, RandomAccessIterator1 end1,
    RandomAccessIterator2 begin2, RandomAccessIterator2 end2,
    Callback callback,
    std::ptrdiff_t cutoff = 10,
    Box_intersection_d::Topology topology = Box_intersection_d::CLOSED,
    Box_intersection_d::Setting  setting  = Box_intersection_d::BIPARTITE);
}{%
    Invocation of box intersection with default box traits
    \ccc{CGAL::Box_intersection_d::Box_traits_d<Box_handle>}, where
    \ccc{Box_handle} corresponds to the iterator value type  of
    \ccc{RandomAccessIterator1}.}
 
\ccFunction{template< class RandomAccessIterator1,
            class RandomAccessIterator2,
            class Callback, class BoxTraits >
void box_intersection_d(
    RandomAccessIterator1 begin1, RandomAccessIterator1 end1,
    RandomAccessIterator2 begin2, RandomAccessIterator2 end2,
    Callback callback,
    BoxTraits box_traits,
    std::ptrdiff_t cutoff = 10,
    Box_intersection_d::Topology topology = Box_intersection_d::CLOSED,
    Box_intersection_d::Setting  setting  = Box_intersection_d::BIPARTITE);
}{Invocation with custom box traits.}

% \begin{ccAdvanced}
% \def\ccLongParamLayout{\ccTrue}
% \ccFunction{template< class RandomAccessIterator1,
%                   class RandomAccessIterator2,
%                   class Callback, class BoxPredicateTraits >
% void box_intersection_custom_predicates_d(
%     RandomAccessIterator1 begin1, RandomAccessIterator1 end1,
%     RandomAccessIterator2 begin2, RandomAccessIterator2 end2,
%     Callback callback,
%     BoxPredicateTraits traits,
%     std::ptrdiff_t cutoff = 10,
%     Box_intersection_d::Setting  setting  = Box_intersection_d::BIPARTITE);
% }{%
%     This version of the box-intersection algorithms uses a very
%     fine-grained predicate traits class that allows the hand-crafted
%     implementation of the individual interval comparison tests. This is
%     only for experts and might only be useful in the context of special
%     number types or exploitation of special assembler instructions!
%     The details of half-open or closed boxes and the \ccc{id}-number of a
%     box are encoded in this predicate traits.
% }
% \def\ccLongParamLayout{\ccFalse}
% \end{ccAdvanced}


\ccRequirements

\begin{itemize}
  \item
    \ccc{RandomAccessIterator1}, and \ldots \ccc{2}, must be
    mutable random-access iterators and both value types must be
    the same. We call this value type \ccc{Box_handle} in the following.
  \item
    \ccc{Callback} must be of the \ccc{BinaryFunction} concept. 
    The \ccc{Box_handle} must be convertible to both argument types. The
    return type is not used and can be \ccc{void}.
  \item
    The \ccc{Box_handle} must be a model of the \ccc{Assignable} concept.
  \item
    In addition, if the default box traits is used the \ccc{Box_handle} must 
    be a class type \ccc{T} or a pointer to a class type \ccc{T}, where
    \ccc{T} must be a model of the \ccc{BoxIntersectionBox_d} concept.
    In both cases, the default box traits specializes to a suitable
    implementation.
  \item
    \ccc{BoxTraits} must be of the \ccc{BoxIntersectionTraits_d} concept.
%  \item
%    \ccc{BoxPredicateTraits} must be of the
%    \ccc{BoxIntersectionPredicateTraits_d} concept.
\end{itemize}

\ccSeeAlso

\ccRefIdfierPage{CGAL::box_self_intersection_d}\\
\ccRefIdfierPage{CGAL::box_intersection_all_pairs_d}

\ccRefIdfierPage{CGAL::Box_intersection_d::Box_traits_d<BoxHandle>}\\
\ccRefConceptPage{BoxIntersectionBox_d}\\
\ccRefConceptPage{BoxIntersectionTraits_d}

%\begin{ccAdvanced}
%
%\ccRefIdfierPage{CGAL::Box_intersection_d::Predicate_traits_d<BoxTraits,bool closed>}\\
%\ccRefConceptPage{BoxIntersectionPredicateTraits_d}
%
%\end{ccAdvanced}

\ccImplementation

The implemented algorithm is described in~\cite{cgal:ze-fsbi-02} as
version two.  Its performance depends on a \ccc{cutoff} parameter.
When the size of both iterator ranges drops below the \ccc{cutoff}
parameter the function switches from the streamed segment-tree
algorithm to the two-way-scan algorithm, see~\cite{cgal:ze-fsbi-02}
for the details.

The streamed segment-tree algorithm needs $O(n \log^d (n) + k)$
worst-case running time and $O(n)$ space, where $n$ is the number of
boxes in both input sequences, $d$ the (constant) dimension of the
boxes, and $k$ the output complexity, i.e., the number of pairwise
intersections of the boxes. The two-way-scan algorithm needs $O(n \log
(n) + l)$ worst-case running time and $O(n)$ space, where $l$ is the
number of pairwise overlapping intervals in one dimensions (the
dimension where the algorithm is used instead of the segment tree).
Note that $l$ is not necessarily related to $k$ and using the
two-way-scan algorithm is a heuristic.

Unfortunately, we have no general method to automatically determine an
optimal cutoff parameter, since it depends on the used hardware, the
runtime ratio between callback runtime and segment-tree runtime, and
of course the number of boxes to be checked and their distribution. In
cases where the callback runtime is dominant, it may be best to make
the threshold parameter small. Otherwise a \ccc{cutoff}$=\sqrt{n}$ can
lead to acceptable results. For well distributed boxes the original
paper~\cite{cgal:ze-fsbi-02} gives optimal cutoffs in the thousands.
Anyway, for optimal runtime some experiments to compare different
cutoff parameters are recommended. See also
Section~\ref{sec:box-inters-performance}.

\ccExample

The box implementation provided with
\ccc{CGAL::Box_intersection_d::Box_d<double,2>} has a special
constructor for the \cgal\ bounding box type \ccc{CGAL::Bbox_2} (and
similar for dimension 3). We use this in the example to create $3
\times 3$ \ccc{boxes} in a grid layout. Additionally we pick the
center box and the box in the upper-right corner as our second box
sequence \ccc{query}.

The default policy of the box type implements the \ccc{id}-number with
an explicit counter in the boxes, which is the default choice since it
always works. We use the \ccc{id}-number in our callback function to
report the result of the intersection algorithm call. The result will
be that the first \ccc{query} box intersects all nine \ccc{boxes} and
the second \ccc{query} box intersects the four boxes in the
upper-right quadrant.

\ccIncludeExampleCode{Box_intersection_d/minimal.cpp}

\end{ccRefFunction}

% +------------------------------------------------------------------------+
%%RefPage: end of main body, begin of footer
% EOF
% +------------------------------------------------------------------------+

