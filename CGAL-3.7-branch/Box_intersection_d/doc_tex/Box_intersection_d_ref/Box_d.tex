% +------------------------------------------------------------------------+
% | Reference manual page: Box_d.tex
% +------------------------------------------------------------------------+


\begin{ccRefClass}[Box_intersection_d::]{Box_d<NT,int D,IdPolicy>}
\ccDefinition

\ccRefName\ is a generic iso-oriented bounding box in dimension $D$.
It provides in each dimension an interval with lower and upper
endpoints represented with the number type \ccc{NT}.  This class is
designed to work smoothly with the algorithms for intersecting
sequences of iso-oriented boxes. For degeneracy handling, the boxes
need to provide a unique \ccc{id}-number. The policy parameter
\ccc{IdPolicy} offers several choices. The template parameters have to
comply with the following requirements:

\begin{itemize}
  \item
    \ccc{NT}: number type for the box boundaries, needs to be a model
    of the \ccc{Assignable} and the \ccc{LessThanComparable} concept.
  \item
    \ccc{int D}: the dimension of the box.
  \item
    \ccc{IdPolicy}: specifies how the \ccc{id}-number will be
    provided. Can be one of the following types, where
    \ccc{ID_EXPLICIT} is the default for this parameter:
    \begin{itemize}
      \item
        \ccc{ID_NONE}: no \ccc{id}-number is provided. Can be useful
        if \ccc{Box_d} is used as a base class for a different
        implementation of \ccc{id}-numbers than the ones provided
        here.
      \item
        \ccc{ID_EXPLICIT}: the \ccc{id}-number is stored explicitly in
        the box and automatically created and assigned at construction
        time of the box. Note that copying a box (copy-constructor and
        assignment) does not create a new \ccc{id}-number but keeps
        the old one, which is the behavior needed by the
        \ccc{CGAL::box_self_intersection} algorithm. This is therefore
        the safe default implementation. 
      \item
        \ccc{ID_FROM_BOX_ADDRESS}: casts the address of the box into a
        \ccc{std::ptrdiff_t} to create the \ccc{id}-number. Works fine 
        if the intersection algorithms work effectively with pointers
        to boxes, but not in the case where the algorithms work with
        box values, because the algorithms modify the order of the
        boxes, and the \ccc{CGAL::box_self_intersection} algorithm
        creates copies of the boxes that would not have identical
        \ccc{id}-numbers.  
    \end{itemize}
\end{itemize}

\ccInclude{CGAL/Box_intersection_d/Box_d.h}\\
and also automatically with\\
\ccInclude{CGAL/box_intersection_d.h}

\ccIsModel

\ccRefConceptPage{BoxIntersectionBox_d}

\ccTypes
\ccThree{std::size_tMMMMM}{Mbox.min_coord( int dim)}{}\ccThreeToTwo

\ccNestedType{NT}{number type to represent the box
  boundaries. Allowed  are the built-in types \ccc{int}, \ccc{unsigned
  int}, \ccc{float}, and \ccc{double}.}

%\footnote{The support is limited
%  since we need the largest and the smallest number representable for
%  this number type, which are currently only provided for the types
%  \ccc{int}, \ccc{unsigned int}, \ccc{float}, and \ccc{double} with the
%  class \ccc{CGAL::Box_intersection_d::box_limits<NT>}.}

\ccTypedef{typedef std::size_t ID;}{type for the box \ccc{id}-number.}


\ccCreation
\ccCreationVariable{box}
\ccThree{std::size_t}{MMMMMMbox.min_coord( int dim)}{}\ccThreeToTwo

\ccConstructor{Box_d();}{Default constructor. No particular initialization.}

\ccConstructor{Box_d(bool complete);}{initializes to the complete or 
  the empty space. If empty, all interval starting(end) points will be
  set to positive(negative) infinity.} 

\ccConstructor{Box_d(NT lo[D], NT hi[D]);}{initializes the box
  intervals to [\ccc{lo[i]},\ccc{hi[i]}], $0 \leq i < D$. 
  \ccPrecond \ccc{lo[i]} $<$ \ccc{hi[i]} for $0 \leq i < D$.}

\ccConstructor{Box_d( const Bbox_2& bbox);}{constructs from bbox
  \ccRequire $D=2$ and \ccc{NT}$ \equiv $\ccc{double}.}
\ccGlue
\ccConstructor{Box_d( const Bbox_3& bbox);}{constructs from bbox
  \ccRequire $D=3$ and \ccc{NT}$ \equiv $\ccc{double}.} 


\ccModifiers

\ccMethod{void init( bool complete = false);}{initializes to the complete or 
  the empty space. If empty, all interval starting(end) points will be
  set to positive(negative) infinity.}

\ccMethod{void extend(NT point[D]);}{extend \ccc{box} to contain the
  old \ccc{box} and \ccc{point}.}

\ccAccessFunctions

\ccFunction{int Box_d::dimension();}{%
  returns $D$, the dimension of the box.}

\ccMethod{std::size_t id();} {returns a unique box id, see the
  \ccc{IdPolicy} template parameter above for the different
  choices.
  \ccRequire \ccc{IdPolicy}$ \neq $\ccc{ID_NONE}}

\ccMethod{NT min_coord(int d) const;}{%
  returns the lower boundary in dimension \ccc{d}
  \ccPrecond $0 \leq $\ccc{d}$ < D$}

\ccMethod{NT max_coord(int d) const;}{%
  returns the upper boundary in dimension \ccc{d}
  \ccPrecond $0 \leq $\ccc{d}$ < D$}

\ccMethod{const Bbox_2& bbox() const;}{returns the bounding box
  \ccRequire $D=2$ and \ccc{NT}$ \equiv $\ccc{double}}
\ccGlue
\ccMethod{const Bbox_3& bbox() const;}{returns the bounding box
  \ccRequire $D=3$ and \ccc{NT}$ \equiv $\ccc{double}}

\ccMethod{void extend(NT p[N]);}{extends \ccVar\ to the smallest
  box that additionally contains the point represented by coordinates in \ccc{p}.}
\ccMethod{void extend(std::pair<NT,NT> p[N]);}{extends \ccVar\ to the smallest
  box that additionally contains the point represented by coordinate intervals in \ccc{p}.}

\ccSeeAlso

\ccRefIdfierPage{CGAL::box_intersection_d}\\
\ccRefIdfierPage{CGAL::box_self_intersection_d}\\
\ccRefIdfierPage{CGAL::box_intersection_all_pairs_d}\\
\ccRefIdfierPage{CGAL::box_self_intersection_all_pairs_d}

\ccRefIdfierPage{CGAL::Box_intersection_d::Box_with_handle_d<NT, int D, Handle, IdPolicy>}

\ccRefIdfierPage{CGAL::Box_intersection_d::Box_traits_d<BoxHandle>}\\
\ccRefConceptPage{BoxIntersectionTraits_d}



\end{ccRefClass}
