% +------------------------------------------------------------------------+
% | Reference manual page: Event.tex
% +------------------------------------------------------------------------+
% | 20.03.2005   Author
% | Package: Kinetic_data_structures
% | 
\RCSdef{\RCSEventRev}{$Id$}
\RCSdefDate{\RCSEventDate}{$Date$}
% |
%%RefPage: end of header, begin of main body
% +------------------------------------------------------------------------+


\begin{ccRefConcept}{Key}

%% \ccHtmlCrossLink{}     %% add further rules for cross referencing links
%% \ccHtmlIndexC[concept]{} %% add further index entries

\ccDefinition
  
The concept \ccRefName\ is a unique identifier for something in some
sort of table. In general, they can be only created by the table and
are returned when a appropriate \ccc{new_foo()} method is called on
the table. There are two classes of values for a \ccRefName, valid and
invalid. The latter cannot refer to something in a table. Use the
method \ccc{is_valid()} to differentiate.

\ccCreationVariable{a}  %% choose variable name

\ccConstructor{Key()}{The default constructor is guaranteed to construct an invalid key (i.e.\ one which is false when cast to a bool.}

\ccOperations

\ccMethod{bool is_valid() const;}{This method returns false if the key
  was created using the default constructor or was otherwise created
  to be invalid.}

\ccGlobalFunction{std::ostream& operator<<(std::ostream&, Event);}{Write a text description of the key to a standard stream.}

\ccHasModels

\ccc{Kinetic::Simulator::Event_key}, \ccc{Kinetic::Active_objects_vector<Object>::Key}.


\end{ccRefConcept}


% +------------------------------------------------------------------------+
%%RefPage: end of main body, begin of footer
% EOF
% +------------------------------------------------------------------------+

