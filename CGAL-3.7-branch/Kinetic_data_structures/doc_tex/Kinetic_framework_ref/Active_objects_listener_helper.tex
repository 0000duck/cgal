% +------------------------------------------------------------------------+
% | Reference manual page: Moving_object_table_listener_helper.tex
% +------------------------------------------------------------------------+
% | 20.03.2005   Author
% | Package: Kinetic_data_structures
% | 
\RCSdef{\RCSMovingobjecttablelistenerhelperRev}{$Id$}
\RCSdefDate{\RCSMovingobjecttablelistenerhelperDate}{$Date$}
% |
%%RefPage: end of header, begin of main body
% +------------------------------------------------------------------------+


\begin{ccRefClass}{Kinetic::Active_objects_listener<ActiveObjectsTable, KDS>}  %% add template arg's if necessary

%% \ccHtmlCrossLink{}     %% add further rules for cross referencing links
%% \ccHtmlIndexC[class]{} %% add further index entries

\ccDefinition
  
The class \ccRefName\ acts as an intermediate between a moving object
table and a KDS. It translates the
\ccc{ActiveObjectsTable::Listener::IS_EDITING} notification events into
appropriate calls to \ccc{KDS::insert(Key)}, \ccc{KDS::set(Key)},
\ccc{KDS::erase(Key)}. 

Kinetic data structures can still take advantage of the batch editing
if they are careful. The methods (such as \ccc{KDS::set(Key)} are
called in lexicographical order in the \ccc{Key}s. So, when a KDS is
preparing to update some certificate involving a recently set object,
it can first check if the certificate involves another changed object
which is lexicographically prior. If so, then the certificate has
already been updated and can be skipped.

\ccInclude{CGAL/Kinetic/listeners.h}


\ccSeeAlso

\ccc{Kinetic::Active_objects_vector<MovingObject>},
\ccc{Kinetic::ActiveObjectsTable}.


\end{ccRefClass}

% +------------------------------------------------------------------------+
%%RefPage: end of main body, begin of footer
% EOF
% +------------------------------------------------------------------------+

