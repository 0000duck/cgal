%\documentclass[]{article}
\documentclass[]{book}

\usepackage{amssymb}
\usepackage{latexsym}
\usepackage{alltt}
\usepackage{makeidx}
\usepackage{cc_manual}
\usepackage{latex_to_html}
\usepackage{graphicx}
\usepackage{cprog}
\usepackage{nonlinkedpath}
\usepackage{path}

% page dimensions, for example as used for the CGAL manuals
% ---------------------------------------------------------
\textwidth 15.6cm
\textheight 23cm
\topmargin -14mm
\evensidemargin 3mm
\oddsidemargin 3mm
\sloppy

% let HTML converter know it is dealing with a manual in multiple parts
\ccMultiplePartsToc
% make chapters start over at 1 for each part
\ccNumberChaptersByPart

\makeindex
%\marginparsep7mm
%\marginparwidth15mm
\gdef\ccNewRefManualStyle{\ccTrue}
\ccDefGlobalScope{CGAL::}

% The tab marker are aligned with the top of the main text. To align
% them with the page header, the following redefinition of the actual
% formatting command can be used.
\setlength{\ccRefTabLift}{12.5mm}


\newcommand{\cgalColumnLayout}{% example setting of the CGAL manuals
    \ccSetThreeColumns{CGAL_Oriented_side}{}{\hspace*{8.5cm}}%
    \ccPropagateThreeToTwoColumns
}
\cgalColumnLayout

\def\ccRefPageBegin{\ccParDims\cgalColumnLayout}
\def\ccRefPageEnd{\ccParDims\cgalColumnLayout}

\begin{document}

\part{User Manual}
\chapter{First chapter}

The first chapter of the user manual. Blah blah blah.


\chapter{Second chapter}

The second chapter of the user manaul.  Blah blah blah.


\part{Reference Manual}
\ccRefChapter{First chapter}{first_user_chap_label}
\label{first_ref_chap_label}

The first chapter of the reference manual defines \ccc{operator+}
and \ccc{HDS} concept

\gdef\ccTagOperatorLayout{\ccFalse}
\begin{ccRefFunction}{operator+}
\ccTexHtml{\ccSetThreeColumns{Vector_2<R>X }{}{\hspace*{8.0cm}}}{}

\ccInclude{CGAL/Vector_2.h}

\ccFunction{Vector_2<R> operator+(const R::RT &s, const Vector_2<R> &w);}
       {Multiplication with a scalar from the left.}

\ccInclude{CGAL/Vector_3.h}

\ccFunction{Vector_3<R> operator+(const R::RT &s, const Vector_3<R> &w);}
       {Multiplication with a scalar from the left.}
\end{ccRefFunction}

\begin{ccRefConcept}{HDS}

\ccDefinition
\ccRefName\ is a hungry data structure

Blah blah blah.
\end{ccRefConcept}

\chapter{Second chapter}

The second chapter of the reference manual defines \ccc{operator*}.

\gdef\ccTagOperatorLayout{\ccTrue}
\begin{ccRefFunction}{operator*}
\ccTexHtml{\ccSetThreeColumns{Vector_2<R>X }{}{\hspace*{8.0cm}}}{}

\ccInclude{CGAL/Vector_2.h}

\ccFunction{Vector_2<R> operator*(const R::RT &s, const Vector_2<R> &w);}
       {Multiplication with a scalar from the left.}

\ccInclude{CGAL/Vector_3.h}

\ccFunction{Vector_3<R> operator*(const R::RT &s, const Vector_3<R> &w);}
       {Multiplication with a scalar from the left.}
\end{ccRefFunction}


\part{Tutorials}
\chapter{Getting Started}

This is a tutorial. Blah blah blah.


\printindex

\end{document}
