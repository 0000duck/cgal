% +------------------------------------------------------------------------+
% | main.tex
% +------------------------------------------------------------------------+

\chapter{Example User Manual}
\ccChapterRelease{\ccRevision. \ \ccDate}\\
\ccChapterAuthor{Lutz Kettner and Susan Hert}

As of release 3.11 of the manual tools, it is possible to create more than    
one manual in a single document that share a table of contents, bibliography
and index.  This allows for easy hyperlinking between the different manuals
while keeping them logically separate.  This example illustrates the creation
of a user and reference manual from a set of files whose names are distinct
and thus do not require the use of the \texttt{-extended} option for the
HTML conversion.

Creation of the manual with \LaTeX\ is done via
\begin{verbatim}
   latex manual.tex
\end{verbatim}
The HTML version is created with
\begin{verbatim}
   cc_manual_to_html -o html manual.tex
\end{verbatim}
Notice that this simple conversion is possible with HTML because the two
chapter commands appear in files with differnet names (\texttt{main.tex} 
in subdirectory \texttt{Example} and \texttt{intro.tex} in subdirectory
\texttt{Example\_ref}).  If the chapter command for the reference pages
were in the file \texttt{Example\_ref/main.tex}, one would have to use
the \texttt{-extended} option of \texttt{cc\_manual\_to\_html} to avoid
name collisions and the \texttt{-ref\_manual} option to generate the two
separate manuals.  See the directory \texttt{../two\_manuals\_ext\_html} for an
example using these options.  The directory \texttt{../one\_manual} contains
an example of the creation of a single manual containing both the user
and reference manuals with the same directory structure as this one.
