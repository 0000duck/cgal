\ccRefPageBegin

\begin{ccRefConcept}{ArrangementDcelWithRebind}

\ccDefinition
% ===========

The concept \ccRefName{} refines the \ccc{ArrangementDcel} concept by adding
a policy clone idiom in form of a rebind struct-template.

Instantiate a dcel{} class with many different possible types without ad-hoc limitations on type of the dcel classes.

\ccRefines
\ccc{ArrangementDcel}

\ccTypes
% ======
\ccNestedType{template <class T> rebind}
   {allows the instantiation of a model of the base concept
   \ccc{ArrangementDcel} with a different possible geometry-traits
   classe without ad-hoc limitations on it.

   Followng the standard clone policy, the rebind struct-template must
   have a nested type named \ccc{other} that defines the type of the
   model replica.}

\ccCreation
\ccCreationVariable{dcel}
% =======================  
\ccConstructor{Arr_dcel();}
  {constructs an empty \dcel{} with one unbouned face.}

\ccHasModels
% ==========
\ccc{Arr_default_dcel<Traits>}
  \lcTex{(\ccRefPage{CGAL::Arr_default_dcel<Traits>})}\\
\ccc{Arr_face_extended_dcel<Traits,FData,V,H,F>}
  \lcTex{(\ccRefPage{CGAL::Arr_face_extended_dcel<Traits,FData,V,H,F>})}\\
\ccc{Arr_extended_dcel<Traits,VData,HData,FData,V,H,F>}
  \lcTex{(\ccRefPage{CGAL::Arr_extended_dcel<Traits,VData,HData,FData,V,H,F>})}

\end{ccRefConcept}
\ccRefPageEnd
