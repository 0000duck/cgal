% +------------------------------------------------------------------------+
% | Reference manual page: Arr_polyline_traits.tex
% +------------------------------------------------------------------------+
% | 
% | Package: Arrangement_2
% | 
% +------------------------------------------------------------------------+

\ccRefPageBegin
\begin{ccRefClass}{Arr_polyline_traits_2<SegmentTraits>}

\ccDefinition
%============

The traits class \ccRefName\ is a model of the \ccc{ArrangementTraits_2}
concept. It handles piecewise linear curves, commonly referred to as
polylines. Each polyline is a chain of segments, where each two neighboring
segments in the chain share a common endpoint. The traits class exploits the
functionality of the \ccc{SegmentTraits} template-parameter to handle the
segments that comprise the polyline curves.

The class instantiated for the template parameter \ccc{SegmentTraits} must
be a model of the \ccc{ArrangementTraits_2} concept that handles line
segments (e.g., \ccc{Arr_segment_traits_2<Kernel>} or 
\ccc{Arr_non_caching_segment_traits_2<Kernel>}, where the first
alternative is recommended).

The number type used by the injected segment traits should support exact
rational arithmetic (that is, the number type should support
the arithmetic operations $+$, $-$, $\times$ and $\div$ that should be
carried out without loss of precision), in order to avoid robustness
problems, although other inexact number types could be used at the user's
own risk.

\ccInclude{CGAL/Arr_polyline_traits_2.h}

\ccIsModel
  \ccc{ArrangementTraits_2}\\
  \ccc{ArrangementLandmarkTraits_2}\\

\subsection*{Class Arr\_polyline\_traits\_2$<$SegmentTraits$>$::Curve\_2}
%========================================================================

The \ccc{Curve_2} class nested within the polyline traits is used to
represent general continuous piecewise-linear curves (a polyline can be
self-intersecting) and support their construction from any range of points.

The copy and default constructor as well as 
the assignment operator are provided for polyline curves. In addition, 
an \ccc{operator<<} for the curves is defined for standard output streams,
and an \ccc{operator>>} for the curves is defined for standard input streams.

\begin{ccClass}{Arr_polyline_traits_2<SegmentTraits>::Curve_2}
%=============================================================

\ccTypes
%-------

\ccNestedType{const_iterator}{A bidirectional iterator that allows
traversing the points that comprise a polyline curve.}

\ccNestedType{const_reverse_iterator}{A bidirectional iterator that
allows traversing the points that comprise a polyline curve.}

\ccCreation
\ccCreationVariable{pi}
%----------------------

\ccConstructor{Curve_2 ();}
  {default constructor that constructs an empty polyline.}

\ccConstructor{template <class InputIterator>
               Curve_2 (Iterator first, Iterator last);}
  {constructs a polyline defined by the given range of points
   \ccc{[first, last)} (the value-type of \ccc{InputIterator} must be
   \ccc{SegmentTraits::Point_2}.
   If the range contains $(n + 1)$ points labeled $(p_{0},p_{1},\ldots,p_{n})$,
   the generated polyline consists of $n$ segments, where the $k$th segment 
   is defined by the endpoints $[p_{k-1},p_{k}]$. The first point in the 
   range is considered as the source point of the polyline while the last 
   point is considered as its target.
   \ccPrecond{There are at least two points in the range}.}

\ccAccessFunctions
%-----------------

\ccMethod{size_t points() const;}
  {returns the number of points that comprise the polyline.
   Note that if there are $n$ points in the polyline, it is comprised
   of $(n - 1)$ segments.}

\ccMethod{const_iterator begin() const;}
  {returns an iterator pointing at the source point of the polyline.}

\ccMethod{const_iterator end() const;}
  {returns an iterator pointing after the end of the polyline.}

\ccMethod{const_iterator rbegin() const;}
  {returns an iterator pointing at the target point of the polyline.}

\ccMethod{const_iterator rend() const;}
  {returns an iterator pointing before the beginning of the polyline.}

\ccMethod{size_t size() const;}
  {returns the number of line segments comprising the polyline
   (equivalent to \ccc{pi.points() - 1}).}

\ccMethod{typename SegmentTraits::X_monotone_curve_2
          operator[] (size_t k) const;}
  {returns the $k$th segment of the polyline.
   \ccPrecond{\ccc{k} is not greater or equal to \ccc{pi.size() - 1}.}}

\ccMethod{Bbox_2 bbox() const;}
  {return a bounding box of the polyline \ccVar{}.}

\ccOperations
%------------

\ccMethod{void push_back (const Point_2 & p);}
  {adds a new point to the polyline, which becomes the new target point
   of \ccVar{}.}

\ccMethod{void clear();}
  {resets the polyline.}

\end{ccClass}

\subsection*{Class 
  Arr\_polyline\_traits\_2$<$SegmentTraits$>$::X\_monotone\_curve\_2}
%====================================================================

The \ccc{X_monotone_curve_2} class nested within the polyline traits is used
to represent $x$-monotone piecewise linear curves. It inherits from the
\ccc{Curve_2} type. It has a default constructor and a constructor from a
range of points, just like the \ccc{Curve_2} class. However, there is
precondition that the point range define an $x$-monotone polyline.

The points that define the $x$-monotone polyline are
always stored in an ascending lexicographical $xy$-order, so their order may
be reversed with respect to the input sequence. Also note that the
$x$-monotonicity ensures that an $x$-monotone polyline is never
self-intersecting (thus, a self-intersecting polyline will be subdivided
to several interior-disjoint $x$-monotone subcurves).

\begin{ccClass}{Arr_polyline_traits_2<SegmentTraits>::X_monotone_curve_2}
\end{ccClass}

\ccSeeAlso
    \ccc{Arr_segment_traits_2<Kernel>}\\
    \ccc{Arr_non_caching_segment_traits_2<Kernel>}\\

\end{ccRefClass}
\ccRefPageEnd
