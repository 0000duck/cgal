% +------------------------------------------------------------------------+
% | Reference manual page: Arr_algebraic_segment_traits.tex
% +------------------------------------------------------------------------+
% | 
% | Package: Arrangement_2
% | 
% +------------------------------------------------------------------------+

\ccRefPageBegin
\begin{ccRefClass}{Arr_algebraic_segment_traits_2<Coefficient>}

\ccDefinition
%============

The traits class \ccRefName\ is a model of the \ccc{ArrangementTraits_2}
concept that handles planar algebraic curves of arbitrary degree, 
and $x$-monotone of such curves.
A planar (real) \emph{algebraic curve} 
is the vanishing set of a polynomial in two variables, that is, the
curve is defined by the defining equation
$$f(x):=\sum_{i+j\leq n} a_{ij} x^i y^j =0,$$
where $n$ is the degree of the curve.

The traits class allows the construction of algebraic curves,
by specifying their implicit equation. $x$-monotone and vertical segments 
of a curve can also be defined; unbounded curves and segments are supported. 
The template parameter \ccc{Coefficient} defines
the innermost coefficient type of the polynomials. Currently, 
the types \ccc{leda::integer} and \ccc{CORE::BigInt} are supported as well
as any instance of \ccc{CGAL::Sqrt_extension} that is instantiated with
one of the integral types above.

\ccInclude{CGAL/Arr_algebraic_segment_traits_2.h}

\ccIsModel
  \ccc{ArrangementTraits_2}

\ccTypes
%-------

\ccNestedType{enum Site_of_point {
        POINT_IN_INTERIOR = 0,
        MIN_ENDPOINT = -1,
        MAX_ENDPOINT = 1 };} 
        {Value to specify whether a point should be in the interior
         of a segment, or its minimal point, 
         or its maximal point in lexicographic order.}

\ccNestedType{Polynomial\_2}{the type for bivariate polynomials,
        with innermost coefficient type \ccc{Coefficient}.
        Constitutes a model of the concept \ccc{Polynomial_d} 
        with two variables.
        (see page~\ccRefPage{Polynomial_d})}

\ccNestedType{Algebraic\_kernel\_1}{model for the concept 
      \ccc{AlgebraicKernel_1}}

\ccNestedType{Algebraic\_real\_1}{represents coordinates of points.
      Typedef from \ccc{Algebraic_kernel_1::Algebraic_real_1}}

\ccNestedType{Bound}{Typedef from \ccc{Algebraic_kernel_1::Bound}}

\subsection*{Class
         Arr\_algebraic\_segment\_traits\_2$<$Coefficient$>$::Curve\_2}
%========================================================================

Models the \ccc{ArrangementTraits_2::Curve_2} concept. 
Represents algebraic curves. Internally, the type stores
topological-geometric information about the particular curve.
In order to use internal caching, instances should only be created
using the \ccc{Construct_curve_2} functor of the traits class.

\begin{ccClass}{Arr_algebraic_segment_traits_2<Coefficient>::Curve_2}
%======================================================================

\ccCreationVariable{C}

\ccMethod{Polynomial_2 polynomial () const;}
  {returns the defining polynomial of the curve.}

\end{ccClass}


\subsection*{Class
         Arr\_algebraic\_segment\_traits\_2$<$Coefficient$>$::Point\_2}
%========================================================================

Models the \ccc{ArrangementBasicTraits_2::Point_2} concept.
Represents points in $\mathbb{R}^2$. Intersection points of algebraic curves
are in general non-rational, so we need a data structure that is capable
of representing arbitrary points with algebraic coordinates.

The traits class represents algebraic coordinates by the type
\ccc{Algebraic_real_1}, which is a model of the \ccc{AlgebraicReal_1}
concept.
A point $p$ is stored by a triple $(x,cv,arcno)$, 
where $x$ is the $x$-coordinate of a point, $cv$ is an instance 
of \ccc{Curve_2} that contains the point, (and has no vertical line at $x$),
and $arcno$ is an \ccc{int}, denoting that $p$ is met as the $arcno$-th
point when shooting a vertical ray at $x$, starting from $-\infty$
(where counting starts with $0$). 

In addition to the methods listed below, the copy constructor and assignment
operator for \ccc{Point_2} objects are also supported.

The functor \ccc{Construct_point_2} constructs \ccc{Point_2} instances.

\begin{ccClass}{Arr_algebraic_segment_traits_2<Coefficient>::Point_2}
%======================================================================

\ccCreationVariable{p}

\ccMethod{Algebraic_real_1 x () const;}
  {returns the $x$-coordinate of \ccVar.}

\ccMethod{Algebraic_real_1 y () const;}
  {returns the $y$-coordinates of \ccVar. \\
   \textbf{Attention:} As described above, points are not stored
   by their $y$-coordinate in \ccc{Algebraic_real_1} representation. In fact,
   this representation must be computed on demand, and might become quite
   costly for points defined by high-degree polynomials. Therefore, it is
   recommended to avoid to call this function as much as possible.}

\ccMethod{Curve_2 curve () const;}
  {returns a \ccc{Curve_2} instance that \ccVar is part of.}

\ccMethod{int arcno () const;}
  {returns the arc number of \ccVar.}

\ccMethod{std::pair<double,double> to_double () const;}
  {returns double-approximations of the $x$- and $y$-coordinates.}

\end{ccClass}


\subsection*{Class 
Arr\_algebraic\_segment\_traits\_2$<$Coefficient$>$::X\_monotone\_curve\_2}
%==============================================================================

Models the \ccc{ArrangementBasicTraits_2::X\_monotone\_curve\_2} concept.
Represents terminal segments of an algebraic curves,
that means vertical segments or $x$-monotone segments with no critical
$x$-coordinate in the interior of their $x$-range.
Terminal segments might either be bounded or unbounded.
By definition, each interior point of 
a non-vertical segment has the same arc number (see the documentation of 
type \ccc{Point_2} above, which is called the \emph{arc number} of the segment
(note the arc number at the endpoints might differ).
Such segments are represented internally by a 4-tuple $(p,q,cv,arcno)$,
where $p$ and $q$ are the endpoints, $cv$ is the \emph{supporting curve}
that the segment belongs to, and arcno is the arc number of the segment.

Arbitrary (weakly) $x$-monotone segments are presented by a range 
of \ccc{X_monotone_curve_2} instances, whose union equals the segment.
The functor \ccc{Construct_x_monotone_segment_2} allows their construction.
To construct all (maximal) terminal segments of a curve,
use the \ccc{Make_x_monotone_2} functor supplied by the traits class.

\begin{ccClass}{Arr_algebraic_segment_traits_2<Coefficient>::X_monotone_curve_2}
%===========================================================================

\ccCreationVariable{s}

\ccMethod{Curve_2 curve () const;}
  {returns the supporting algebraic curve of \ccVar.}

\ccMethod{bool is\_vertical () const;}
  {returns whether \ccVar\ is a vertical segment.}

\ccMethod{bool is\_finite (CGAL::Arr_curve_end ce) const;}
  {returns whether \ccVar\ has a finite endpoint on the left 
    (if \ccc{ce==CGAL::ARR_MIN_END}) or on the right 
    (if \ccc{ce==CGAL::ARR_MAX_END}).}

\ccMethod{Point_2 curve_end (CGAL::Arr_curve_end ce) const;}
  {returns the left or right endpoint of \ccVar for \ccc{ce==CGAL::ARR_MIN_END}
   and \ccc{ce==CGAL::ARR_MAX_END}.
   \ccPrecond(The corresponding curve end is finite)}

\ccMethod{int arcno () const;}
  {returns the arc number of the segment.
   \ccPrecond(The segment is non-vertical)}

\ccMethod{Algebraic_real_1 x () const;}
  {returns the $x$-coordinate of a vertical segment.
   \ccPrecond(The segment is vertical)}

\end{ccClass}

\ccHeading{Object Creation Functors}

Curves, points, and $x$-monotone segments are created by special functors. 
The functors are not default constructible; the only possibility to
obtain them is by the corresponding accessing functions.

\subsection*{Class 
Arr\_algebraic\_segment\_traits\_2$<$Coefficient$>$::Construct\_curve\_2}

\begin{ccClass}{Arr_algebraic_segment_traits_2<Coefficient>::Construct_curve_2}
%===========================================================================

\ccCreationVariable{fo}

\ccMethod{Curve_2 operator() (Polynomial_2 p);}
        {Returns a \ccc{Curve_2} object that represents the curve defined by
         the polynomial \ccc{p}} 

\ccMethod{Curve_2 operator() (std::string s);}
        {Returns a \ccc{Curve_2} object specified by \ccc{s}. 
	The passed string represents the defining polynomial of the curve 
	and must be given in a MAPLE-readable format using "x" as first 
	and "y" as second variable, e.g., 
        "(x\^{}3*y-2*x)*(-6*x-y\^{}3*x\^{}6)" 
	for integer coefficients, and "3/2*x*y\^{}4-5/7*x\^{}2+3/1" 
	for rational coefficients.} 


\end{ccClass}

\subsection*{Class 
Arr\_algebraic\_segment\_traits\_2$<$Coefficient$>$::Construct\_point\_2}

\begin{ccClass}{Arr_algebraic_segment_traits_2<Coefficient>::Construct_point_2}
%===========================================================================

\ccCreationVariable{fo}

\ccMethod{Point_2 operator() (Algebraic_real_1 x, Curve_2 cv, int arcno);}
        {Returns a \ccc{Point_2} object that represents the \ccc{arcno}-th
         point in the fiber of \ccc{cv} at $x$-coordinate \ccc{x}, 
         counted from the bottom, starting with zero.
         \ccPrecond(\ccc{cv} must not have a vertical line at \ccc{x}, 
                    and $0\leq arcno < c$, where $c$ is the number of points 
                     in the fiber of \ccc{cv} at \ccc{x}.)} 

\ccMethod{Point_2 operator() (Algebraic_real_1 x, X_monotone_curve_2 xcv);}
        {Returns a \ccc{Point_2} object that represents the 
         point on \ccc{xcv} at $x$-coordinate \ccc{x} 
         \ccPrecond(\ccc{x} is in the $x$-range of \ccc{xcv}.)}

\ccMethod{Point_2 operator() (Algebraic_real_1 x, Algebraic_real_1 y);}
        {Returns a \ccc{Point_2} object that represents (x,y)} 

\ccMethod{Point_2 operator() (Coefficient x, Coefficient y);}
        {Returns a \ccc{Point_2} object that represents (x,y)} 

\ccMethod{Point_2 operator() (Bound x, Bound y);}
        {Returns a \ccc{Point_2} object that represents (x,y)} 

\ccMethod{Point_2 operator() (int x, int y);}
        {Returns a \ccc{Point_2} object that represents (x,y)} 

\end{ccClass}

\subsection*{Class 
Arr\_algebraic\_segment\_traits\_2$<$Coefficient$>$::Construct\_x\_monotone\_segment\_2}

\begin{ccClass}{Arr_algebraic_segment_traits_2<Coefficient>::Construct_x_monotone_segment_2}
%===========================================================================

\ccCreationVariable{fo}

\ccMethod{template<class OutputIterator> OutputIterator 
          operator() (Curve_2 cv, Point_2 end_min, Point_2 end_max,
                      OutputIterator out);}
        {Writes a sequence of \ccc{X_monotone_curve_2} objects into \ccc{out}.
         These segments form an $x$-monotone (or vertical) 
         segment of the curve \ccc{cv} 
         that starts in \ccc{end_max}, and end in \ccc{end_max}.
         \ccPrecond(\ccc{end_min} must have a unique $x$-monotone segment
                    to its right, or \ccc{end_max} must have a unique
                    $x$-monotone segment to its left. Furthermore, 
                    \ccc{end_min} and \ccc{end_max} must be connected
                    by a $x$-monotone segment of \ccc{cv})}

\ccMethod{template<class OutputIterator> OutputIterator 
          operator() (Curve_2 cv, Point_2 p, 
                      Site_of_point site_of_p,
                      OutputIterator out);}
        {Writes a sequence of \ccc{X_monotone_curve_2} objects into \ccc{out}.
         These segments form an $x$-monotone (or vertical) 
         segment of the curve \ccc{cv}.

         If \ccc{site_of_p==POINT_IN_INTERIOR}, the maximal segment is
         returned that contains \ccc{p} in its interior.

         If \ccc{site_of_p==MIN_ENDPOINT}, the maximal segment is
         returned that contains \ccc{p} as its left endpoint.

         If \ccc{site_of_p==MAX_ENDPOINT}, the maximal segment is
         returned that contains \ccc{p} as its left endpoint.
         \ccPrecond(If \ccc{site_of_p==POINT_IN_INTERIOR}, \ccc{p}
                    must be an interior point of an $x$-monotone or a vertical
                    segment.
                    If \ccc{site_of_p==MIN_ENDPOINT}, \ccc{p}
                    must either have a unique $x$-monotone segment to the right,
                    or a vertical segment from \ccc{p} upwards.
                    If \ccc{site_of_p==MAX_ENDPOINT}, \ccc{p}
                    must either have a unique $x$-monotone segment to the left,
                    or a vertical segment from \ccc{p} downwards.)}

\ccMethod{template<class OutputIterator> OutputIterator 
          operator() (Point_2 p, Point_2 q,
                      OutputIterator out);}
        {Writes a sequence of \ccc{X_monotone_curve_2} objects into \ccc{out}.
         These segments form a straight-line segment connecting
	 the points \ccc{p} and \ccc{q}. If \ccc{p} and \ccc{q} share the
	 same $x$-coordinate, the constructed vertical segment consists of
	 only one \ccc{X_monotone_curve_2} object and can be computed
	 efficiently. In the non-vertical case, 
	 the construction is only possible if \ccc{p} and \ccc{q} 
         have both rational x- and y-coordinates.
         \ccPrecond(\ccc{p} must not be equal to \ccc{q}.)
	}


\end{ccClass}

\ccHeading{Accessing functor objects}

%\begin{ccClass}{Arr_algebraic_segment_traits_2<Coefficient>}

\ccCreationVariable{traits}

\ccMethod{Construct_curve_2 construct_curve_2_object() const;}{}
\ccGlue
\ccMethod{Construct_point_2 construct_point_2_object() const;}{}
\ccGlue
\ccMethod{Construct_x_monotone_segment_2 construct_x_monotone_segment_2_object() const;}{}
%\ccGlue

%\end{ccClass}

\end{ccRefClass}

\ccRefPageEnd

