% +------------------------------------------------------------------------+
% | Reference manual page: Arr_face_ex_dcel.tex
% +------------------------------------------------------------------------+
% | 
% | Package: Arrangement_2
% | 
% +------------------------------------------------------------------------+

\ccRefPageBegin

\begin{ccRefClass}{Arr_face_extended_dcel<Traits,FData,V,H,F>}

\ccDefinition
%============

The \ccRefName\ class-template extends the \dcel\ face-records, making it
possible to store extra (non-geometric) data with the arrangement faces.
The class should be instantiated by an \ccc{FData} type which represents the
extra data stored with each face. 

Note that all types of \dcel\ features (namely vertex, halfedge and face)
are provided as template parameters. However, by default they are defined
as follows:

\begin{tabbing}
\ccc{V} = \=\ccc{Arr_vertex_base<typename Traits::Point_2>}\\
\ccc{H} = \>\ccc{Arr_halfedge_base<typename Traits::X_monotone_curve_2>}\\
\ccc{F} = \>\ccc{Arr_face_base}
\end{tabbing}

\ccInclude{CGAL/Arr_extended_dcel.h}

\ccIsModel
  \ccc{ArrangementDcelWithRebind}

\ccInheritsFrom
  \ccc{Arr_dcel_base<V, H, Arr_extended_face<F, FData> >}

\ccSeeAlso

\ccc{Arr_dcel_base<V,H,F>}\lcTex{
  (\ccRefPage{CGAL::Arr_dcel_base<V,H,F>})}

\end{ccRefClass}

\ccRefPageEnd
