% +------------------------------------------------------------------------+
% | Reference manual page: PolylineSimplificationStopPredicate.tex
% +------------------------------------------------------------------------+
% | April 2009   Fernando Cacciola
% | Package: Package
% |
\RCSdef{\RCSPolylineSimplificationStopPredicateRev}{$Id: header.tex 40270 2007-09-07 15:29:10Z lsaboret $}
\RCSdefDate{\RCSPolylineSimplificationStopPredicateDate}{$Date: 2007-09-07 12:29:10 -0300 (Fri, 07 Sep 2007) $}
% |
\ccRefPageBegin
%%RefPage: end of header, begin of main body
% +------------------------------------------------------------------------+


\begin{ccRefConcept}{PolylineSimplificationStopPredicate}

%% \ccHtmlCrossLink{}     %% add further rules for cross referencing links
%% \ccHtmlIndexC[concept]{} %% add further index entries

\ccDefinition

% The section below is automatically generated. Do not edit!
%START-AUTO(\ccDefinition)

Models of this concept are passed to the polyline simplification algorithm to indicate when to stop.

%END-AUTO(\ccDefinition)
  
\ccCreationVariable{stop}  %% choose variable name

\ccOperations

% Reduce left margin
\ccThree{12}{12}{}

% The section below is automatically generated. Do not edit!
%START-AUTO(\ccOperations)

\ccMethod{template<class Tr> 
          bool operator()(Polyline_constrained_triangulation_2<Tr> const& pct, 
          Polyline_constrained_triangulation_2<Tr>::Vertices_in_constraint_iterator viq, 
          Polyline_constrained_triangulation_2<Tr>::FT cost, 
          std::size_t initial_count, 
          std::size_t current_count) const;}
{
Indicates if the simplification must be stopped. This method is called right before each vertex is about to be removed
\ccCommentHeading{Parameters}  \\
\ccc{cdt}: The underlying polyline constrained Delaunay triangulation which embeds the polyline set.\\
\ccc{viq}: The current vertex about to be removed.\\
\ccc{cost}: The associated cost for removing the current vertex (as given by \ccc{PolylineSimplificationCostFunction}).\\
\ccc{initial_count}: The initial number of vertices in the entire polyline set (including intersection vertices not in any source polyline).\\
\ccc{current_count}: The current number of vertices.
\ccCommentHeading{Returns} \ccc{true} if the algorithm shall stop, \ccc{false} if it shall continue.
}
\ccGlue

%END-AUTO(\ccOperations)

\ccHasModels

% The section below is automatically generated. Do not edit!
%START-AUTO(\ccHasModels)

\ccc{Polyline_simplification_2::Stop_below_count_ratio_threshold}  \\
\ccc{Polyline_simplification_2::Stop_below_count_threshold}  \\
\ccc{Polyline_simplification_2::Stop_below_cost_threshold}

%END-AUTO(\ccHasModels)

\end{ccRefConcept}

% +------------------------------------------------------------------------+
%%RefPage: end of main body, begin of footer
\ccRefPageEnd
% EOF
% +------------------------------------------------------------------------+

