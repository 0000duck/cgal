% +------------------------------------------------------------------------+
% | Reference manual page: Periodic_3_triangulation_iterator_type_3.tex
% +------------------------------------------------------------------------+
% | 19.2.2009   Manuel Caroli
% | Package: Periodic_3_triangulation_3
% | 
\RCSdef{\RCSPeriodictriangulationiteratortypeRev}{$Id$}
\RCSdefDate{\RCSPeriodictriangulationiteratortypeDate}{$Date$}
% |
%%RefPage: end of header, begin of main body
% +------------------------------------------------------------------------+

\ccModifierCrossRefOff
\begin{ccRefEnum}[Periodic_3_triangulation_3::]{Iterator_type}

\ccDefinition
  
The enum \ccRefName\ is defined by \ccc{Periodic_3_triangulation_3} to
specify the behavior of geometric iterators.

\ccEnum{enum Iterator_type {STORED=0, UNIQUE, STORED_COVER_DOMAIN,
    UNIQUE_COVER_DOMAIN};}
{}

The elements of the enum have the following meaning:
\begin{itemize}
\item{STORED:} Return all geometric primitives as they are stored internally
  in \ccc{Triangulation_data_structure_3}.
\item{UNIQUE:} Return only one representative of each geometric
  primitive even if the triangulation is computed in a multiply
  sheeted covering space. Choose the representative whose maximum
  offset is minimal but non-negative in each direction of space. 
\item{STORED\_COVER\_DOMAIN:} Same as STORED but return additionally
  all primitives whose intersection with the original domain of the
  current covering space is non-empty.
\item{UNIQUE\_COVER\_DOMAIN:} Same as UNIQUE but return additionally
  all primitives whose intersection with the original domain is
  non-empty.
\end{itemize}

\ccSeeAlso

\ccc{CGAL::Periodic_3_triangulation_3}

\end{ccRefEnum}
\ccModifierCrossRefOn
