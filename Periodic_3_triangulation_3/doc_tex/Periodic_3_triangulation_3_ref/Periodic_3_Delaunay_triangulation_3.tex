% +------------------------------------------------------------------------+
% | Reference manual page: Periodic_3_Delaunay_triangulation_3.tex
% +------------------------------------------------------------------------+
% | 19.2.2009   Monique Teillaud, Manuel Caroli
% | Package: Periodic_3_triangulation_3
% | 
\RCSdef{\RCSPeriodicDelaunaytriangulationRev}{$Id$}
\RCSdefDate{\RCSPeriodicDelaunaytriangulationDate}{$Date$}
% |
%%RefPage: end of header, begin of main body
% +------------------------------------------------------------------------+


\begin{ccRefClass}{Periodic_3_Delaunay_triangulation_3<PT,TDS>}

\ccDefinition
  
The class \ccc{Periodic_3_Delaunay_triangulation_3} represents a
Delaunay triangulation in three-dimensional periodic space.

\ccInclude{CGAL/Periodic_3_Delaunay_triangulation_3.h}

\ccParameters

The first template argument \ccc{PT} must be a model of the
\ccc{Periodic_3DelaunayTriangulationTraits_3} concept.

The second template argument \ccc{TDS} must be a model of the
\ccc{TriangulationDataStructure_3} concept with some additional
functionality in cells and vertices.
Its default value is
\ccc{Triangulation_data_structure_3<Triangulation_vertex_base_3<PT,Periodic_3_triangulation_ds_vertex_base_3<>>,Triangulation_cell_base_3<PT,Periodic_3_triangulation_ds_cell_base_3<>>>}.

\ccInheritsFrom{\ccc{Periodic_3_triangulation_3<Periodic_3DelaunayTriangulationTraits_3,TriangulationDataStructure_3>}}


\ccCreation
\ccCreationVariable{dt}
\ccTwo{Vertex_handle}{}

\ccConstructor{Periodic_3_Delaunay_triangulation_3(
  const Iso_cuboid & domain = Iso_cuboid(0,0,0,1,1,1),
  const Geom_traits & traits = Geom_traits());} 
{Creates an empty periodic Delaunay triangulation \ccVar, with
  \ccc{domain} as original domain and possibly specifying
  a traits class \ccc{traits}.
\ccPrecond{\ccc{domain} is a cube.}}

\ccConstructor{Periodic_3_Delaunay_triangulation_3 (const
  Periodic_3_Delaunay_triangulation_3 & dt1);} 
{Copy constructor.}

\ccConstructor{template < class InputIterator >
       Periodic_3_Delaunay_triangulation_3 (
       InputIterator first,
       InputIterator last,
       const Iso_cuboid & domain = Iso_cuboid(0,0,0,1,1,1),
       const Geom_traits & traits = Geom_traits());} 
{Equivalent to constructing an empty triangulation with the optional 
domain and traits class arguments and calling \ccc{insert(first,last)}.
\ccPrecond{The \ccc{value_type} of \ccc{first} and \ccc{last} are
  \ccc{Point}s lying inside \ccc{domain} and \ccc{domain} is a cube.}} 

\ccOperations

\ccHeading{Insertion}

The following methods insert points in the triangulation ensuring the
empty sphere property of Delaunay triangulations. The inserted points
need to lie in the original domain (see
Section~\ref{P3Triangulation3-sec-space} of the user manual).

In the degenerate case when there are co-spherical points, the Delaunay
triangulation is known not to be uniquely defined. In this case, \cgal\
chooses a particular Delaunay triangulation using a symbolic perturbation
scheme~\cite{cgal:dt-pvr3d-03}.

Note that insertion of a new point can cause a switch from computing
in 27-sheeted covering space to computing in 1-sheeted covering space,
which invalidates some \ccc{Vertex_handle}s and \ccc{Cell_handle}s.

\ccThree{Vertex_handle}{dt.remove()}{}

\ccMethod{Vertex_handle insert(const Point & p,
                               Cell_handle start = Cell_handle() );}
{Inserts point \ccc{p} in the triangulation and returns the corresponding
 vertex. The optional argument \ccc{start} is used as a starting place
 for the point location.
\ccPrecond{\ccc{p} lies in the original domain \ccc{domain}.}}

\ccMethod{Vertex_handle insert(const Point & p, Locate_type lt,
                               Cell_handle loc, int li, int lj);}
{Inserts point \ccc{p} in the triangulation and returns the corresponding
 vertex. Similar to the above \ccc{insert()} function, but takes as additional
 parameter the return values of a previous location query.  See description of
 \ccc{Periodic_3_triangulation_3::locate()}.
\ccPrecond{\ccc{p} lies in the original domain \ccc{domain}.}}

The following method allows one to insert several points. It returns the
number of inserted points. 

\ccMethod{template < class InputIterator >
          int
          insert(InputIterator first, InputIterator last,
          bool is_large_point_set = false);}
{Inserts the points in the iterator range $\left[\right.$\ccc{first},
\ccc{last}$\left.\right)$.  Returns the number of inserted points.
This function uses spatial sorting (cf.\ chapter~\ref{sec:spatial_sorting})
and therefore is not guaranteed to insert the points following the
order of \ccc{InputIterator}. If the third argument
\ccc{is_large_point_set} is set to \ccc{true} a heuristic for
optimizing the insertion of large point sets is applied.
\ccPrecond{The \ccc{value_type} of \ccc{first} and \ccc{last} are
\ccc{Point}s lying inside \ccc{domain}.}}


\ccHeading{Point moving}

\ccMethod{Vertex_handle move_point(Vertex_handle v, const Point & p);}
{Moves the point stored in \ccc{v} to \ccc{p}, while preserving the Delaunay
property.  This performs an action semantically equivalent to \ccc{remove(v)}
followed by \ccc{insert(p)}, but is supposedly faster when the point has
not moved much.  Returns the handle to the new vertex.
\ccPrecond{\ccc{p} lies in the original domain \ccc{domain}.}}

\ccHeading{Removal}

When a vertex \ccc{v} is removed from a triangulation, all the cells
incident to \ccc{v} must be removed, and the polyhedral region
consisting of all the tetrahedra that are incident to \ccc{v} must be
re-triangulated. 
So, the problem reduces to triangulating a polyhedral
region, while preserving its boundary, or to compute a
\textit{constrained} triangulation. This is known to be sometimes
impossible: the Sch\"onhardt polyhedron cannot be triangulated
\cite{cgal:s-cgehd-98}. 

However, when dealing with Delaunay triangulations, the case of such
polyhedra that cannot be re-triangulated cannot happen, so \cgal\
proposes a vertex removal.

\ccMethod{void remove(Vertex_handle v);}
{Removes the vertex \ccc{v} from the triangulation.}

\ccHeading{Queries}

\ccMethod{Bounded_side
          side_of_sphere(Cell_handle c, const Point & p,
          const Offset & off = Offset(0,0,0)) const;}
{Returns a value indicating on which side of the circumscribed sphere
of \ccc{c} the point-offset pair (\ccc{p},\ccc{off}) lies. More
precisely, it returns:\\ 
- \ccc{ON_BOUNDED_SIDE} if (\ccc{p},\ccc{off}) is inside the sphere.\\
- \ccc{ON_BOUNDARY} if (\ccc{p},\ccc{off}) on the boundary of the sphere.\\
- \ccc{ON_UNBOUNDED_SIDE} if (\ccc{p},\ccc{off}) lies outside the
sphere.
\ccPrecond{\ccc{p} lies in the original domain \ccc{domain}.}}

\ccMethod{Vertex_handle nearest_vertex(Point p,
                                       Cell_handle c = Cell_handle());}
{Returns any nearest vertex to the point \ccc{p}, or the default constructed
handle if the triangulation is empty.  The optional argument \ccc{c} is a hint
specifying where to start the search. It always returns a vertex
corresponding to a point inside $\ccc{domain}$ even if computing in a
multiply sheeted covering space.
\ccPrecond{\ccc{c} is a cell of \ccVar\ and \ccc{p} lies in the
  original domain \ccc{domain}.}
}

\ccMethod{Vertex_handle nearest_vertex_in_cell(Cell_handle c,
  Point p, Offset off = Offset(0,0,0)) const;}
{Returns the  vertex  of the cell \ccc{c} that is nearest to the
  point-offset pair (\ccc{p},\ccc{off}).
\ccPrecond{\ccc{p} lies in the original domain \ccc{domain}.}}

A point-offset pair (\ccc{p},\ccc{off}) is said to be in conflict with
a cell \ccc{c} iff \ccVar.\ccc{side_of_sphere(c, p, off)} returns
\ccc{ON_BOUNDED_SIDE}. 
The set of cells that are in conflict with (\ccc{p},\ccc{off}) is
star-shaped.

\ccMethod{template <class OutputIteratorBoundaryFacets,
                    class OutputIteratorCells,
                    class OutputIteratorInternalFacets>
  Triple<OutputIteratorBoundaryFacets,
         OutputIteratorCells,
         OutputIteratorInternalFacets>
  find_conflicts(Point p, Cell_handle c,
                 OutputIteratorBoundaryFacets bfit,
                 OutputIteratorCells cit,
                 OutputIteratorInternalFacets ifit);}
{Computes the  conflict hole induced by \ccc{p}.  The starting cell
\ccc{c} must be in conflict.  Then this function returns
respectively in the output iterators:\\
-- \ccc{cit}: the cells in conflict.\\
-- \ccc{bfit}: the facets on the boundary, that is, the facets
\ccc{(t, i)} where the cell \ccc{t} is in
conflict, but \ccc{t->neighbor(i)} is not.\\
-- \ccc{ifit}: the facets inside the hole, that is, delimiting two
cells in conflict.\\
Returns the pair composed of the resulting output iterators.
\ccPrecond{\ccc{c} is in conflict with \ccc{p} and \ccc{p} lies in the
  original domain \ccc{domain}.}}

\ccMethod{template <class OutputIterator>
  OutputIterator
  vertices_in_conflict(Point p, Cell_handle c,
                 OutputIterator res);}
{Similar to \ccc{find_conflicts()}, but reports the vertices which are on the
boundary of the conflict hole of \ccc{p}, in the output iterator \ccc{res}.
Returns the resulting output iterator.
\ccPrecond{\ccc{c} is in conflict with \ccc{p} and \ccc{p} lies in the
  original domain \ccc{domain}.}}

A face (cell, facet or edge) is said to be a Gabriel face iff
its smallest circumscribing sphere do not enclose
any vertex of the triangulation.  Any Gabriel face belongs to the
Delaunay triangulation, but the reciprocal is not true.
The following member functions test the Gabriel property of
Delaunay faces.
\ccMethod{bool is_Gabriel(Cell_handle c, int i);}{}
\ccGlue 
\ccMethod{bool is_Gabriel(Cell_handle c, int i, int j);}{}
\ccGlue 
\ccMethod{bool is_Gabriel(const Facet& f);}{}
\ccGlue
\ccMethod{bool is_Gabriel(const Edge& e);} {}


\ccHeading{Voronoi diagram}
\ccIndexMainItem{Voronoi diagram}
\cgal\ offers several functions to display the Voronoi diagram of 
a set of points in 3D.


Note that a traits class providing exact constructions should be used
in order to guarantee the computation of the Voronoi diagram (as
opposed to computing the triangulation only, which requires only exact
predicates). 

\ccThree{Periodic_segment}{dt.dual(Cell_handle c, int i)X}{}

\ccMethod{Point dual(Cell_handle c) const;}
{Returns the representative of the circumcenter of the four vertices
  of c that lies in the original domain \ccc{domain}.}

\ccMethod{Periodic_segment dual(Facet f) const;}
{Returns the dual of facet \ccc{f}, which is a periodic segment.}

\ccMethod{Periodic_segment dual(Cell_handle c, int i) const;}
{same as the previous method for facet \ccc{(c,i)}.
\ccPrecond{$i\in\{0,1,2,3\}$}}

\ccMethod{template <class OutputIterator>
OutputIterator
dual(Edge e, OutputIterator pts) const;}
{Returns in the output iterator the points of the dual polygon of
  edge \ccc{e} in the same order as the \ccc{Facet_circulator} returns
  facets incident to the edge \ccc{e}. The points form the dual polygon
  in $\mathbb R^3$, so they do not necessarily all lie inside the
  original domain.}

\ccMethod{template <class OutputIterator>
OutputIterator
dual(Cell_handle c, int i, int j, OutputIterator pts) const;}
{same as the previous method for edge \ccc{(c,i,j)}.
\ccPrecond{$i,j\in\{0,1,2,3\}, i\neq j$}}

\ccMethod{template <class OutputIterator>
OutputIterator
dual(Vertex_handle v, OutputIterator pts) const;}
{Returns in the output iterator the points of the dual polyhedron of
  vertex \ccc{v} in no particular order. The points form the dual
  polyhedron in $\mathbb R^3$, so they do not necessarily lie all
  inside the original domain.}

\ccMethod{template <class Stream> Stream & draw_dual(Stream & os);}
{Sends the set of duals to all the facets of \ccVar\ into \ccc{os}.}

\ccMethod{Geom_traits::FT dual_volume(Vertex_handle v) const;}
{Returns the volume of the Voronoi cell dual to \ccc{v}.}

\ccMethod{Point dual_centroid(Vertex_handle v) const;}
{Returns the centroid of the Voronoi cell dual to \ccc{v}.}

\begin{ccAdvanced}
\ccHeading{Checking}
\ccMethod{bool
          is_valid(bool verbose = false) const;}
{Checks the combinatorial validity of the triangulation and the
validity of its geometric embedding (see
Section~\ref{P3Triangulation3-sec-intro}). Also checks that all the
circumscribing spheres of cells are empty.\\
When \ccc{verbose} is set to true,  messages describing the first
invalidity encountered are printed.}

\ccMethod{bool
          is_valid(Cell_handle c, bool verbose = false) const;}
{Checks the combinatorial and geometric validity of the cell (see
Section~\ref{P3Triangulation3-sec-intro}). Also checks that the
circumscribing sphere of cells is empty.\\
When \ccc{verbose} is set to true, messages are printed to give
a precise indication of the kind of invalidity encountered.}

These methods are  mainly a debugging help for the users of advanced features.
\end{ccAdvanced}

\ccSeeAlso

\ccc{Periodic_3_triangulation_3}

%% \ccExample

%% \ccIncludeExampleCode{Triangulation3/Delaunay_triangulation_3_prog.cpp}

\end{ccRefClass}
