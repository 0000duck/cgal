% +------------------------------------------------------------------------+
% | Reference manual page: Constrained_Delaunay_triangulation_2.tex
% +------------------------------------------------------------------------+
% | 12.04.2000   Author
% | Package: Package
% | 
\RCSdef{\RCSConstrainedDelaunaytriangulationRev}{$Id$}
\RCSdefDate{\RCSConstrainedDelaunaytriangulationDate}{$Date$}
% |
%%RefPage: end of header, begin of main body
% +------------------------------------------------------------------------+


\begin{ccRefClass}{Constrained_Delaunay_triangulation_2<Traits,Tds,Itag>}  %% add template arg's if necessary

%% \ccHtmlCrossLink{}     %% add further rules for cross referencing links
%% \ccHtmlIndexC[class]{} %% add further index entries

\ccDefinition
A constrained Delaunay triangulation is a triangulation with
constrained edges which tries to be as much Delaunay as possible.
Constrained edges are not necessarily Delaunay edges,
therefore a constrained Delaunay triangulation is not a Delaunay
triangulation. A constrained Delaunay is a triangulation
whose faces do not
necessarily fulfill the empty circle property
but fulfill a weaker property called the
{\em constrained empty circle}.
 To state this property,
it is convenient to think of  constrained
edges as blocking the view. Then, a triangulation is 
constrained Delaunay if
 the circumscribing circle
of any of its triangular faces includes in its interior 
no vertex  that is visible
from the interior of the triangle.  
The class \ccRefName\ is designed to represent
constrained Delaunay triangulations.

As in the case of constrained triangulations, three different versions
of Delaunay constrained triangulations are offered
depending on whether the user wishes to handle
intersecting input constraints or not.
The desired version can  be selected through the instantiation of the
third template parameter \ccc{Itag} which can be one of the
following~: \\
\ccc{CGAL::No_intersection_tag} if  intersections of
 input constraints are disallowed,\\
\ccc{CGAL::Exact_predicates_tag} allows intersections between input
constraints
 and is to be used when the traits
class
provides exact predicates but approximate constructions of the
intersection points,\\
\ccc{CGAL::Exact_intersections_tag} allows intersections between input
constraints
 and is to be used in conjunction
with an exact arithmetic type.


The template parameters \ccc{Tds} 
has to be instantiate with a model of \ccc{TriangulationDataStructure_2}.
The geometric traits 
of a constrained Delaunay triangulation is required
to provide the \ccc{side_of_oriented_circle} test as the geometric traits
of a Delaunay triangulation and the \ccc{Traits}
parameter has 
to be instantiated with a model
\ccc{DelaunayTriangulationTraits_2}.
When intersection of input constraints are supported, 
the geometric traits class 
is required to provide additional function object  types
to compute the intersection of two segments.
and  has then to be also a model of the concept
\ccc{ConstrainedTriangulationTraits_2}.

A constrained Delaunay triangulation is not a Delaunay
triangulation but it is a constrained triangulation.
Therefore the class \ccRefName\ derives from
the class \ccc{Constrained_triangulation_2<Traits,Tds>}.
Also, information about the status (constrained or not)
of the edges of the triangulation is stored
in the faces. 
Thus the nested \ccc{Face}
type of a constrained triangulation offers
additional functionalities to deal with this information.
These additional functionalities 
induce additional requirements on the base face class
plugged into the triangulation data structure of 
 a constrained Delaunay triangulation.
The base face of a constrained Delaunay triangulation
has to be a model of the concept
\ccc{ConstrainedTriangulationFaceBase_2}.

\cgal\ provides a default for the template parameters.
If \ccc{Gt} is the geometric traits
parameter,
the default  for
\ccc{ConstrainedTriangulationFaceBase_2}  is the class
\ccc{CGAL::Constrained_triangulation_face_base_2<Gt>}
and the default for the
triangulation data structure parameter is the class
\ccc{CGAL::Triangulation_data_structure_2<
                       CGAL::Triangulation_vertex_base_2<Gt>,
		       CGAL::Constrained_triangulation_face_base_2<Gt>
>}.
The default intersection tag is \ccc{CGAL::No_intersection_tag}.

\ccInclude{CGAL/Constrained_Delaunay_triangulation_2.h}

\ccInheritsFrom \ccc{Constrained_triangulation_2<Traits,Tds,Itag>}

\ccTypes
All types used in this class are inherited from the base class
\ccc{Constrained_triangulation_2<Traits,Tds,Itag>}.



\ccCreation
\ccCreationVariable{cdt}  %% choose variable name

\ccConstructor{Constrained_Delaunay_triangulation_2(const Traits &t = Traits())}
{Introduces an empty constrained Delaunay triangulation \ccVar.}

\ccConstructor{Constrained_Delaunay_triangulation_2(const
Constrained_Delaunay_triangulation_2& cdt1)}
 {Copy constructor, all faces and vertices
are duplicated and  the constrained status of edges
is copied.}


\ccConstructor{ template<class InputIterator> Constrained_triangulation_2(
        InputIterator first,
                               InputIterator last,
                               const Traits& t=Traits());}
{A templated constructor which introduces and builds
 a constrained triangulation with constrained edges in the range 
$\left[\right.$\ccc{first}, \ccc{last}$\left.\right)$.
\ccPrecond The \ccc{value_type} of \ccc{first} and \ccc{last}
 is \ccc{Constraint}.}



\ccHeading{Insertion and Removal}
The following member functions overwrite the corresponding
members of the base class to include a step restoring
 the Delaunay constrained
property after modification of the triangulation.

\ccMethod{ Vertex_handle insert(Point p, Face_handle f = Face_handle());}
{ Inserts point \ccc{p} in the triangulation. 
If present \ccc{f} is used as an hint
for the location of \ccc{p}.}

\ccMethod{Vertex_handle 
          insert(const Point& p,
                 Locate_type& lt,
                 Face_handle loc, int li );}
{Same as above except that the location of the point
\ccc{p}  to be inserted is assumed to be given by
\ccc{(lt,loc,i)}.}

\ccMethod{Vertex_handle push_back(const Point& p);}
{Equivalent to \ccc{insert(p)}.}

\ccMethod{template < class InputIterator >
          std::ptrdiff_t      
          insert(InputIterator first, InputIterator last);}
{Inserts the points in the range
 $\left[\right.$\ccc{first}, \ccc{last}$\left.\right)$.
 Returns the number of inserted points.
 \ccPrecond The \ccc{value_type} of \ccc{first} and \ccc{last}
 is \ccc{Point}.}

\ccMethod{ void insert_constraint(Point a, Point b);}
{ Inserts segment ab as a constrained edge in the triangulation. }

\ccMethod{ void push_back(const Constraint& c);}
{Inserts constraints \ccc{c} as above.}


\ccMethod{ void insert_constraint(Vertex_handle va, Vertex_handle vb);}
{ Inserts the line segment whose endpoints are the vertices \ccc{va} and
\ccc{vb} as a constrained  edge e in the triangulation. }


\ccMethod{void remove(Vertex_handle & v);}
{ Removes vertex v. 
\ccPrecond Vertex \ccc{v}  is not incident to a constrained edge.}

\ccMethod{ void remove_incident_constraints(Vertex_handle  v);}
{Make the edges incident to vertex \ccc{v} unconstrained edges.}

\ccMethod{void remove_constraint(const Face_handle & f, int i);}
{ Edge \ccc{(f,i)}  is no longer constrained.}


\ccHeading{Queries}
The following template member functions query the
set of faces in conflict with a point \ccc{p}.
The notion of conflict refers here to a 
constrained Delaunay setting which means the following.
Constrained edges are considered as visibility obstacles
and a point \ccc{p} is considered to be 
in conflict with a face \ccc{f} iff it is 
 visible from the interior of \ccc{f} and 
included in the circumcircle of \ccc{f}.

\ccMethod{template <class OutputItFaces, class OutputItBoundaryEdges> 
  std::pair<OutputItFaces,OutputItBoundaryEdges>
  get_conflicts_and_boundary(const Point  &p, 
			     OutputItFaces fit, 
			     OutputItBoundaryEdges eit,
                             Face_handle start) const;}
{ \ccc{OutItFaces} is an output iterator with \ccc{Face_handle} as value type.
\ccc{OutItBoundaryEdges} stands for an output iterator with \ccc{Edge} as value type.
This members function outputs in the container pointed to by \ccc{fit}
the faces which are in conflict with point \ccc{p}.
It outputs in the container pointed to by \ccc{eit} 
the boundary of the zone in conflict with \ccc{p}.
The boundary edges
of the conflict  zone are output in counterclockwise order
and each edge is described  through its incident face
which is not in conflict with \ccc{p}.
The function returns in a std::pair the resulting output iterators.
\ccPrecond dimension()==2}

\ccMethod{template <class OutputItFaces> 
	  OutputItFaces
          get_conflicts (const Point  &p, 
   		         OutputItFaces fit, 
  		         Face_handle start) const;}
{Same as above except that only the faces in conflict with \ccc{p}
are output. The function returns the resulting output iterator.
\ccPrecond dimension()==2}

\ccMethod{template <class OutputItBoundaryEdges>
          OutputItBoundaryEdges
          get_boundary_of_conflicts(const Point  &p, 
   		   		    OutputItBoundaryEdges eit,
  		       		    Face_handle start) const;}
{ \ccc{OutputItBoundaryEdges} stands for an output iterator with 
\ccc{Edge} as value
type.
 This functions  outputs in the container pointed to by \ccc{eit},
the boundary of the zone in conflict with \ccc{p}. The boundary edges
of the conflict  zone are output in counter-clockwise order
and each edge is described  through the incident face
which is not in conflict with \ccc{p}.
The function returns the resulting output iterator.}


\ccHeading{Checking}

\ccMethod{bool is_valid();}
{ Checks if the triangulation is valid and if each constrained edge is
 consistently marked constrained in its two incident faces.}

\begin{ccAdvanced}
\ccHeading{Flips}

\ccMethod{bool is_flipable(Face_handle f, int i);}
{ Determines if edge \ccc{(f,i)} can be flipped. Returns true if
edge \ccc{(f,i)} is not constrained and  the circle circumscribing f
contains  the vertex of \ccc{f->neighbor(i)}
opposite to edge \ccc{(f,i)}. }

\ccMethod{void flip(Face_handle& f, int i);}
{ Flip \ccc{f} and \ccc{f->neighbor(i)}.}


\ccMethod{void propagating_flip(List_edges & edges);}
{ Makes the triangulation constrained Delaunay by flipping edges. 
List edges contains an
initial list of edges to be flipped. The returned
triangulation is constrained Delaunay 
if the initial list contains at least all the edges of the
input triangulation that failed to be 
constrained Delaunay. (An edge is said to be 
constrained Delaunay if it is  either constrained or
locally Delaunay.) }
\end{ccAdvanced}


\ccSeeAlso
\ccc{CGAL::Constrained_triangulation_2<Traits,Tds,Itag>} \\
\ccc{TriangulationDataStructure_2} \\
\ccc{DelaunayTriangulationTraits_2} \\
\ccc{ConstrainedTriangulationTraits_2} \\
\ccc{ConstrainedDelaunayTriangulationTraits_2} \\
\ccc{ConstrainedTriangulationFaceBase_2} 

%\ccExample



%% \ccIncludeExampleCode{Package/Constrained_Delaunay_triangulation_2_prog.cpp}

\end{ccRefClass}

% +------------------------------------------------------------------------+
%%RefPage: end of main body, begin of footer
% EOF
% +------------------------------------------------------------------------+

