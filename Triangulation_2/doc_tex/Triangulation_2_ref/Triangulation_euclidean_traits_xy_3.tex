% +------------------------------------------------------------------------+
% | Reference manual page: Triangulation_euclidean_traits_xy_3.tex
% +------------------------------------------------------------------------+
% | 06.04.2000   Author
% | Package: Package
% | 
\RCSdef{\RCSTriangulationeuclideantraitsxyRev}{$Id$}
\RCSdefDate{\RCSTriangulationeuclideantraitsxyDate}{$Date$}
% |
%%RefPage: end of header, begin of main body
% +------------------------------------------------------------------------+


\begin{ccRefClass}{Triangulation_euclidean_traits_xy_3<K>}  %% add template arg's if necessary

\begin{ccDeprecated}
The functionality of this class has been generalized to other packages than 2D triangulations.
The more general class \ccc{Projection_traits_xy_3} can be found in the 2D and 3D Linear Geometric Kernel.
\end{ccDeprecated}

%% \ccHtmlCrossLink{}     %% add further rules for cross referencing links
%% \ccHtmlIndexC[class]{} %% add further index entries
\ccCreationVariable{traits}
\ccDefinition
  
The class \ccRefName\ is a geometric traits class which allows to
triangulate
a terrain. This traits class is designed to build 
a two dimensional triangulation  embedded in 3D space,
i.e. a triangulated surface, such that 
its
on the $xy$ plane  is a Delaunay triangulation.
This is a usual construction for GIS terrains.
Instead of really projecting the 3D points and
maintaining a mapping between each point and its projection
 (which costs space and is error prone)
the class \ccRefName\  supplies geometric predicates that ignore the
\ccc{z}-coordinate of the points.

The class is a model of the concept \ccc{DelaunayTriangulationTraits_2} 
except that it does not provide the type and constructors
required to build the dual Voronoi diagram. The class is also a model
of the concept \ccc{ConstrainedTriangulationTraits_2}.

\ccParameters
The template parameter \ccc{K}  has to
be instantiated by a model of the \ccc{Kernel} concept.
\ccRefName\ uses types 
and predicates defined in \ccc{K}.

\ccInclude{CGAL/Triangulation_euclidean_traits_xy_3.h}

\ccTypes
\ccTwo{typedef Triangle_3<K>}{}
\ccTypedef{typedef Point_3<K>   Point_2;}{}
\ccGlue
\ccTypedef{typedef Segment_3<K>   Segment_2;}{}
\ccGlue
\ccTypedef{typedef Triangle_3<K>   Triangle_2;}{}
\ccGlue
\ccTypedef{typedef Line_3<K>   Line_2;}{}

The following predicates and constructor types are provided

\ccNestedType{Construct_segment_2}
{A constructor object for
\ccc{Segment_2}. Provides~: \\
\ccc{Segment_2 operator()(Point_2 p,Point_2 q)}, \\ 
which constructs a  segment from two points.}
\ccGlue
\ccNestedType{Construct_triangle_2}{A constructor object for
\ccc{Triangle_2}. Provides~: \\
\ccc{Triangle_2 operator()(Point_2 p,Point_2 q,Point_2 r )}, \\
which constructs a triangle from three points.}
\ccGlue
\ccNestedType{Construct_line_2}{A constructor object for
\ccc{Line_2}. Provides~: \\
\ccc{Line_2 operator()(Point_2 p,Point_2 q)}, \\
which constructs a line from two points.}
\ccGlue
\ccNestedType{Compute_squared_distance_2}{A construction object.
Provides the operator~:\\
\ccc{RT operator()(Line_2 l, Point_2 p);}
which returns the squared distance between the projection of \ccc{p} 
and the projection of \ccc{l}.}
\ccGlue
\ccNestedType{Intersect_2}{A construction object.
Provides the operator~:\\
\ccc{Object_2 operator()(Segment_2 s1, Segment_2 s2);}
which returns the intersection of the projection of \ccc{s1}
and the projection of \ccc{s2} embedded in \ccc{3D}. If the intersection
is a segment, the \ccc{z}-coordinates of its extremities is \ccc{0}.
If the intersection is a point \ccc{p}, let \ccc{p1} and \ccc{p2} be the points on \ccc{s1}
and \ccc{s2} respectively, such that their projections are \ccc{p}. The point returned is the
middle of the segment \ccc{p1}\ccc{p2}.
\ccPrecond{The projection of \ccc{s1} and the projection of \ccc{s2} are non-degenerate 
\ccc{2D} segments.}
}
\ccGlue
\ccNestedType{Compare_x_2}{Predicate object. Provides
the operator~: \\
\ccc{Comparison_result operator()(Point_2 p, Point_2 q)} \\
which returns 
\ccc{SMALLER, EQUAL} or \ccc{ LARGER}
 according to the
$x$-ordering of points \ccc{p} and \ccc{q}.}
\ccGlue
\ccNestedType{Compare_y_2}{Predicate object. Provides
the operator~: \\ 
\ccc{Comparison_result operator()(Point_2 p, Point_2 q)}\\
which returns 
(\ccc{SMALLER, EQUAL} or \ccc{ LARGER})
according to the
$y$-ordering of points \ccc{p} and \ccc{q}.}
\ccGlue
\ccNestedType{Orientation_2}{Predicate object. Provides
the  operator~: \\
\ccc{Orientation operator()(Point_2 p, Point_2 q, Point_2 r)}\\
which returns
\ccc{LEFT_TURN}, \ccc{RIGHT_TURN} or \ccc{COLLINEAR}
 according to the position of the projection of $r$
with respect to the projection of the 
oriented line \ccc{pq}.}
\ccGlue
\ccNestedType{Side_of_oriented_circle_2}{Predicate object. 
Provides the operator~:
\ccc{Oriented_side operator()(Point_2 p, Point_2 q, Point_2 r, Point_2 s)}
which takes four points $p, q, r, s$ as arguments and returns
\ccc{ON_POSITIVE_SIDE}, \ccc{ON_NEGATIVE_SIDE} or, 
\ccc{ON_ORIENTED_BOUNDARY} according to the position of 
the projection of point\ccc{s}
with respect to the oriented circle through the projections of $p,q$
and $r$.}
\ccGlue
\ccNestedType{Compare_distance_2} {Predicate object. Provides
 the operator : \\
\ccc{Comparison_result operator()(Point_2 p, Point_2 q, Point_2 r)}
which returns \ccc{SMALLER}, \ccc{EQUAL} or \ccc{LARGER}
according to the distance between the projection of p and the projection of q being smaller, equal or larger
than the distance between the projection of p and the projection of r.}

\ccCreation
\ccCreationVariable{traits}  %% choose variable name

\ccThree{Triangulation_euclidean_traits_xy_3()}{traits=tr  }{}

\ccConstructor{Triangulation_euclidean_traits_xy_3();}{
default constructor.}
\ccGlue
\ccConstructor{Triangulation_euclidean_traits_xy_3(
                      Triangulation_euclidean_traits_xy_3 tr);}
{Copy constructor.}
\ccMethod{Triangulation_euclidean_traits_xy_3 operator=
              (Triangulation_euclidean_traits_xy_3  tr);}
{Assignment operator.}


\ccHeading{Access to predicate objects}
The following access functions are provided

\ccThree{Side_of_oriented_circle_2}{side_of_oriented_circle_2_object();}{}
\ccMethod{Construct_segment_2 construct_segment_2_object();}{}
\ccGlue
\ccMethod{Construct_triangle_2 construct_triangle_2_object();}{}
\ccGlue
\ccMethod{Construct_line_2 construct_line_2_object();}{}
\ccGlue
\ccMethod{Comparison_x_2 compare_x_2_object();}{}
\ccGlue
\ccMethod{Comparison_y_2 compare_y_2_object();}{}
\ccGlue
\ccMethod{Orientation_2  orientation_2_object();}{}
\ccGlue
\ccMethod{Side_of_oriented_circle_2
side_of_oriented_circle_2_object();}{}
\ccGlue
\ccMethod{Compare_distance_2
compare_distance_2_object();} {}
\ccGlue
\ccMethod{Intersect_2  intersect_2_object();}{}
\ccGlue
\ccMethod{Compute_squared_distance_2 compute_squared_distance_2_object();}

\ccSeeAlso
\ccc{TriangulationTraits_2} \\
\ccc{DelaunayTriangulationTraits_2} \\
\ccc{CGAL::Triangulation_2<Traits,Tds>}\\
\ccc{CGAL::Delaunay_triangulation_2<Traits,Tds>} 


\cgal\ provides also predefined geometric traits class
\ccc{Triangulation_euclidean_traits_yz_3<K>} and
\ccc{Triangulation_euclidean_traits_zx_3<K>} to
deal with projections on the
 \ccc{xz}- or  the \ccc{yz}-plane,
respectively.

\ccInclude{CGAL/Triangulation_euclidean_traits_xz_3.h}\\
\ccInclude{CGAL/Triangulation_euclidean_traits_yz_3.h}


\end{ccRefClass}

% +------------------------------------------------------------------------+
%%RefPage: end of main body, begin of footer
% EOF
% +------------------------------------------------------------------------+

