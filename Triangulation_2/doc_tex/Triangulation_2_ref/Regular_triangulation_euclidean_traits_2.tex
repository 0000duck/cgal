% +------------------------------------------------------------------------+
% | Reference manual page: Regular_triangulation_traits_2.tex
% +------------------------------------------------------------------------+
% | 12.04.2000   Author
% | Package: Package
% | 
\RCSdef{\RCSRegulartriangulationtraitsRev}{$Revision$}
\RCSdefDate{\RCSRegulartriangulationtraitsDate}{$Date$}
% |
%%RefPage: end of header, begin of main body
% +------------------------------------------------------------------------+


\begin{ccRefClass}{Regular_triangulation_euclidean_traits_2<K,Weight>}  %% add template arg's if necessary

%% \ccHtmlCrossLink{}     %% add further rules for cross referencing links
%% \ccHtmlIndexC[class]{} %% add further index entries
\ccCreationVariable{traits}
\ccDefinition
  

 \ccRefName\  is a  model for the concept \ccc{RegularTriangulationTraits_2}
This traits class is templated by a kernel class \ccc{K}
and a weight type \ccc{Weight}.
This class inherits from \ccc{K}
and uses a \ccc{Weighted_point} type
derived from the type \ccc{K::Point_2}.

Note that this template class is specialized for
\ccc{CGAL::Exact_predicates_inexact_constructions_kernel}, so that it is as if
\ccc{Regular_triangulation_filtered_traits_2} was used, i.e. you get
filtered predicates automatically.

\ccInclude{CGAL/Regular_triangulation_euclidean_traits_2.h}

\ccIsModel
\ccc{RegularTriangulationTraits_2}

\ccInheritsFrom
\ccc{K}


\ccSeeAlso
\ccc{RegularTriangulationTraits_2}\\
\ccc{CGAL::Regular_triangulation_filtered_traits_2}\\
\ccc{CGAL::Regular_triangulation_2}


\end{ccRefClass}

% +------------------------------------------------------------------------+
%%RefPage: end of main body, begin of footer
% EOF
% +------------------------------------------------------------------------+

