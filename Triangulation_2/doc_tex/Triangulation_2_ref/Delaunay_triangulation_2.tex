% +------------------------------------------------------------------------+
% | Reference manual page: Delaunay_triangulation_2.tex
% +------------------------------------------------------------------------+
% | 12.04.2000   Author
% | Package: Package
% | 
\RCSdef{\RCSDelaunaytriangulationRev}{$Id$}
\RCSdefDate{\RCSDelaunaytriangulationDate}{$Date$}
% |
%%RefPage: end of header, begin of main body
% +------------------------------------------------------------------------+


\begin{ccRefClass}{Delaunay_triangulation_2<Traits,Tds>}  %% add template arg's if necessary

%% \ccHtmlCrossLink{}     %% add further rules for cross referencing links
%% \ccHtmlIndexC[class]{} %% add further index entries

\ccDefinition
  
The class \ccRefName\ is designed to represent
the Delaunay triangulation of a set of points in a plane.
A  Delaunay triangulation of a set of points
is a triangulation of the sets of points that fulfills
the following {\em empty circle property} 
(also called {\em Delaunay property}): the circumscribing
circle of any facet
of the triangulation contains no point of the set in its interior.
For a point set with no case of co-circularity 
of more than three points,
the Delaunay triangulation is unique, it is  the dual
of the Voronoi diagram of the points.

\ccInclude{CGAL/Delaunay_triangulation_2.h}

\ccParameters
The template parameter \ccc{Tds}
is to be instantiated with a model of
\ccc{TriangulationDataStructure_2}.
\cgal\ provides a default instantiation for this parameter,
which is the class 
\ccc{CGAL::Triangulation_data_structure_2 <
                       CGAL::Triangulation_vertex_base_2<Traits>,
                       CGAL::Triangulation_face_base_2<Traits> >}.

The geometric traits \ccc{Traits} 
is to be instantiated with a model of
\ccc{DelaunayTriangulationTraits_2}.
The concept \ccc{DelaunayTriangulationTraits_2} refines the
concept \ccc{TriangulationTraits_2}, providing
a predicate type 
to check the empty circle property.

Changing this predicate type
allows to build Delaunay triangulations for different metrics
such that $L_1$ or $L_{\infty}$ or any metric defined by a
convex object. However, the user of an exotic metric
must be careful that the constructed triangulation 
has to be a triangulation of the convex hull
which means that convex hull edges have to be Delaunay edges.
This is granted for any smooth convex metric (like $L_2$)
and can be ensured for other metrics (like  $L_{\infty}$)
by the addition to the point set of well chosen sentinel points
The concept of \ccc{DelaunayTriangulationTraits_2} is  described
\ccRefPage{DelaunayTriangulationTraits_2}.

When dealing
with a large triangulations, the user is advised to 
encapsulate  the Delaunay triangulation class into a
triangulation hierarchy, which means to use  the class
\ccc{Triangulation_hierarchy_2<Tr>}
with the template parameter instantiated with \ccRefName\ .
The triangulation hierarchy will then  offer the same 
functionalities  but be much more for efficient 
for locations and insertions.



\ccInheritsFrom

\ccc{Triangulation_2<Traits,Tds>}

\ccTypes
Inherits all the types defined in  \ccc{Triangulation_2<Traits,Tds>}.

\ccCreation
\ccCreationVariable{dt}  %% choose variable name

\ccConstructor{Delaunay_triangulation_2(const Traits& gt =
Traits());}{default constructor.}

\ccConstructor{Delaunay_triangulation_2(
               const Delaunay_triangulation_2<Traits,Tds> &tr);}
{copy constructor. All the vertices and faces are duplicated.}

\ccHeading{Insertion and Removal}

The following insertion and removal  functions overwrite
the functions inherited from the class
\ccc{Triangulation_2<Traits,Tds>} to maintain the Delaunay property.

In the degenerate case when there are co-circular points, the Delaunay
triangulation is known not to be uniquely defined. In this case, \cgal\ 
chooses a particular Delaunay triangulation using a symbolic perturbation
scheme~\cite{cgal:dt-pvr3d-03}.


\ccMethod{Vertex_handle insert(const Point& p, Face_handle f=Face_handle());}
{inserts  point \ccc{p}.
If point \ccc{p} coincides with an already existing vertex, this 
vertex is returned and the triangulation is not updated.
Optional parameter \ccc{f} is used to initialize the location of \ccc{p}.
}

\ccMethod{Vertex_handle insert(const Point& p, Locate_type& lt,
Face_handle loc, int li );}
{ inserts a point \ccc{p}, the location of which is supposed to be
given by \ccc{(lt,loc,li)}, see the description of member function
\ccc{locate} in class \ccc{Triangulation_2<Traits,Tds>}.}

\ccMethod{Vertex_handle push_back(const Point& p);}
{equivalent to \ccc{insert(p)}.}

\ccMethod{template < class PointInputIterator >
          std::ptrdiff_t
          insert(PointInputIterator first, PointInputIterator last);}
{inserts the points in the range
 $\left[\right.$\ccc{first}, \ccc{last}$\left.\right)$.
 Returns the number of inserted points.
 Note that this function is not guaranteed to insert the points
 following the order of \ccc{PointInputIterator}, as \ccc{spatial_sort}
 is used to improve efficiency.
 \ccPrecond The \ccc{value_type} of \ccc{first} and \ccc{last}
 is \ccc{Point}.}

\ccMethod{template < class PointWithInfoInputIterator >
          std::ptrdiff_t
          insert(PointWithInfoInputIterator first, PointWithInfoInputIterator last);}
{
inserts the points in the iterator range $\left[\right.$\ccc{first},
\ccc{last}$\left.\right)$.  Returns the number of inserted points.
Note that this function is not guaranteed to insert the points
following the order of \ccc{PointWithInfoInputIterator}, as \ccc{spatial_sort}
is used to improve efficiency.
Given a pair \ccc{(p,i)},  the vertex \ccc{v} storing \ccc{p} also stores \ccc{i}, that is 
\ccc{v.point() == p} and \ccc{v.info() == i}. If several pairs have the same point, 
only one vertex is created, and one of the objects of type \ccc{Vertex::Info} will be stored in the vertex.
\ccPrecond{\ccc{Vertex} must be model of the concept \ccc{TriangulationVertexBaseWithInfo_2}.
The \ccc{value_type} of \ccc{first} and \ccc{last} is \ccc{std::pair<Point,Vertex::Info>}.}
}

\ccMethod{void    remove(Vertex_handle v);}
{removes the vertex from the triangulation.}

Note that the other modifier functions  of
\ccc{Triangulation_2<Traits,Tds>} are not overwritten.
Thus a  call to \ccc{insert_in_face}
\ccc{insert_in_edge}, \ccc{insert_outside_convex_hull},
\ccc{insert_outside_affine_hull}  or \ccc{flip}
on a valid Delaunay triangulation might lead to a triangulation
which is no longer a Delaunay triangulation.

\ccHeading{Displacement}

\ccMethod{Vertex_handle    move_if_no_collision(Vertex_handle v, const Point & p);}
{if there is not already another vertex placed on \ccc{p}, 
the triangulation is modified such that the new position of vertex \ccc{v}
is \ccc{p}, and \ccc{v} is returned. Otherwise, the triangulation is not
modified and the vertex at point \ccc{p} is returned.
\ccPrecond Vertex \ccc{v} must be finite.}

\ccMethod{Vertex_handle    move(Vertex_handle v, const Point & p);}
{same as above if there is no collision. Otherwise, \ccc{v}
is deleted and the vertex placed on \ccc{p} is returned.
 \ccPrecond Vertex \ccc{v} must be finite.}

\ccHeading{Queries}

\ccMethod{Vertex_handle
        nearest_vertex(const Point& p, Face_handle f=Face_handle());}
{returns any nearest vertex of \ccc{p}. The implemented function
begins with a location step and
\ccc{f} may be used to initialize the location.}

\ccMethod{template <class OutputItFaces, class OutputItBoundaryEdges> 
  std::pair<OutputItFaces,OutputItBoundaryEdges>
  get_conflicts_and_boundary(const Point  &p, 
                             OutputItFaces fit, 
                             OutputItBoundaryEdges eit,
                             Face_handle start) const;}
{ \ccc{OutputItFaces} is an output iterator with \ccc{Face_handle} as value type.
\ccc{OutputItBoundaryEdges} stands for an output iterator with \ccc{Edge} as value type.
This members function outputs in the container pointed to by \ccc{fit}
the faces which are in conflict with point \ccc{p}
i. e. the faces whose circumcircle contains \ccc{p}.
It outputs in the container pointed to by \ccc{eit} the 
the boundary of the zone in conflict with \ccc{p}.
The boundary edges
of the conflict  zone are output in counter-clockwise order
and each edge is described  through its incident face
which is not in conflict with \ccc{p}.
The function returns in a std::pair the resulting output iterators.
\ccPrecond dimension()==2}

\ccMethod{template <class OutputItFaces> 
          OutputItFaces
          get_conflicts (const Point  &p, 
                         OutputItFaces fit, 
                         Face_handle start) const;}
{same as above except that only the faces in conflict with \ccc{p}
are output. The function returns the resulting output iterator.
\ccPrecond dimension()==2}

\ccMethod{template <class OutputItBoundaryEdges>
          OutputItBoundaryEdges
          get_boundary_of_conflicts(const Point  &p, 
                                    OutputItBoundaryEdges eit,
                                    Face_handle start) const;}
{ \ccc{OutputItBoundaryEdges} stands for an output iterator with 
\ccc{Edge} as value
type.
 This function  outputs in the container pointed to by \ccc{eit},
the boundary of the zone in conflict with \ccc{p}. The boundary edges
of the conflict  zone are output in counterclockwise order
and each edge is described  through the incident face
which is not in conflict with \ccc{p}.
The function returns the resulting output iterator.}


\ccHeading{Voronoi diagram}
\ccThree{Object}{dt.dual(Edge_circulator ec)}{}
\ccIndexMainItem{Voronoi diagram}
The following member functions provide the elements of the
dual Voronoi diagram.

\ccMethod{Point dual(const Face_handle &f) const;}
{Returns the center of the circle circumscribed to face \ccc{f}.
\ccPrecond \ccc{f} is not infinite}

\ccMethod{Object dual(const Edge &e) const;}
{returns a segment, a ray or a line supported by the bisector of the
endpoints of \ccc{e}.
If faces incident to \ccc{e} are both finite, a segment whose endpoints are the
duals of each incident face is returned. If only one incident face is
finite, a
ray whose endpoint is the dual of the finite incident face is returned.
Otherwise both incident faces
are infinite and the bisector line is returned.}

\ccMethod{Object dual(const Edge_circulator& ec) const;}
{Idem}

\ccMethod{Object dual(const Edge_iterator& ei) const;}
{Idem}

\ccMethod{
template < class Stream>
  Stream& draw_dual(Stream & ps);}
{output the dual Voronoi diagram to stream \ccc{ps}.}


\ccPredicates
\ccThree{Oriented_side}{side_of_}{}
\ccThreeToTwo

\ccMethod{Oriented_side
  side_of_oriented_circle(Face_handle f,
                          const Point& p) const;}
{Returns the side of \ccc{p} with respect to the circle circumscribing
 the triangle associated with \ccc{f}}

\begin{ccAdvanced}
\ccHeading{Miscellaneous}
\ccThree{Vertex_handle}{T.push_back(const Point &p);}{}

The checking function \ccc{is_valid()} is also overwritten
to additionally test the empty circle property.

\ccMethod{bool is_valid(bool verbose = false, int level = 0) const;}
{ tests the validity of the triangulation as a \ccc{Triangulation_2}
and additionally tests the Delaunay property. This method is
 mainly  useful for debugging Delaunay triangulation algorithms designed by
 the user.}
\end{ccAdvanced}




\ccSeeAlso

\ccc{CGAL::Triangulation_2<Traits,Tds>}, \\
\ccc{TriangulationDataStructure_2}, \\
\ccc{DelaunayTriangulationTraits_2}, \\
\ccc{Triangulation_hierarchy_2<Tr>}.




\ccHeading{Implementation}

Insertion is implemented by inserting in the triangulation, then
performing a sequence of Delaunay flips. The number of flips is $O(d)$
if the new vertex is of degree $d$ in the new triangulation. For
points distributed uniformly at random, insertion takes time $O(1)$ on
average.

Removal calls the removal in the triangulation and then re-triangulates
the hole in such a way that  the Delaunay criterion is satisfied. Removal of a
vertex of degree $d$ takes time $O(d^2)$.
The degree $d$ is $O(1)$ for a random
vertex in the triangulation.

After a point location step, the nearest neighbor 
is found in time $O(n)$ in the
worst case, but in time $O(1)$
for vertices distributed uniformly at random  and any query point. 


\end{ccRefClass}

% +------------------------------------------------------------------------+
%%RefPage: end of main body, begin of footer
% EOF
% +------------------------------------------------------------------------+

