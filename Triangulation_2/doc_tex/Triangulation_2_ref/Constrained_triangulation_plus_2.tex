% +------------------------------------------------------------------------+
% | Reference manual page: Constrained_triangulation_plus_2.tex
% +------------------------------------------------------------------------+
% | 05.02.2002   Mariette Yvinec
% | Package: Triangulation_2
% | 
\RCSdef{\RCSConstrainedtriangulationplusRev}{$Id$}
\RCSdefDate{\RCSConstrainedtriangulationplusDate}{$Date$}
% |
%%RefPage: end of header, begin of main body
% +------------------------------------------------------------------------+


\begin{ccRefClass}{Constrained_triangulation_plus_2<Tr>}  %% add template arg's if necessary

%% \ccHtmlCrossLink{}     %% add further rules for cross referencing links
%% \ccHtmlIndexC[class]{} %% add further index entries

The class \ccRefName\ 
implements a constrained triangulation 
with an additional data
structure called the  constraint hierarchy 
that keeps track of  the input constraints and of their refinement
in the triangulation.

The class \ccc{Constrained_triangulation_plus_2<Tr>}
inherits from its template parameter Tr, which has to be instantiated
by a constrained or constrained Delaunay  triangulation.

According to its intersection tag, the base class
will support intersecting input constraints or not.
When intersections of input constraints are supported,
the base class constructs a triangulation  of the  arrangement
of the constraints,
introducing new vertices at each proper intersection
point and  refining the input constraints into sub-constraints
which are edges (more precisely constrained edges) of the
triangulation. 
In this context, the constraint hierarchy 
keeps track of  the input constraints and of their refinement
in the triangulation. This data structure 
maintains for each input constraints 
the sequence of intersection vertices added on this constraint.
The constraint hierarchy also allows the user to retrieve the set
of constrained edges of the triangulation, and for each
constrained edge,  the set of input constraints that overlap it.

\ccInclude{CGAL/Constrained_triangulation_plus_2.h}

\ccInheritsFrom
\ccc{Tr} which is either a \ccc{CGAL::Constrained_triangulation_2<Traits,Tds>} or a \ccc{CGAL::Constrained_Delaunay_triangulation_2<Traits,Tds>}.

\ccTypes 
\ccTypedef{typedef Tr Triangulation;}{the triangulation base class.}
\ccGlue
\ccTypedef{typedef Itag Intersection_tag;}{the intersection tag.s}

\ccNestedType{Constraint_iterator}{An iterator to visit
all the input  constraints. The order of visit is arbitrary.
The value type of this iterator  is a pair 
\ccc{std::pair<Vertex_handle, Vertex_handle>} corresponding to the
endpoints of the constraint.}

\ccNestedType{Subconstraint_iterator}{An iterator 
to visit all the sub-constraints  of the triangulation.
The order of visit is arbitrary.
The value type of this iterator is a pair 
\ccc{std::pair<Vertex_handle, Vertex_handle>}
corresponding to the vertices of  the
sub-constraint.}

\ccNestedType{Vertices_in_constraint_iterator}{ An iterator on the
vertices of the chain of triangulation edges representing a
constraint. The value type of this iterator is \ccc{Vertex_handle}.}

\ccTypedef{typedef Context;}{This type is intended to describe
a constraint enclosing  a sub-constraint  and the position of the
sub-constraint in this constraint.
It provides three member functions \ccc{vertices_begin()}, \ccc{vertices_end()}
and \ccc{current()} returning 
iterators of the type \ccc{Vertices_in_constraint_iterator}
on the sequence of vertices of the enclosing constraint. 
These iterators point
respectively on the first vertex of the enclosing constraint,
past the last vertex  
and on the first vertex of the sub-constraint.}



\ccTypedef{typedef Context_iterator;}{An iterator on 
 constraints enclosing a given sub-constraint. The value type of this
iterator
is \ccc{Context}.}


\ccCreation
\ccCreationVariable{ctp}

\ccConstructor{Constrained_triangulation_plus_2(const Geom_traits& gt=Geom_traits())}
{Introduces an empty triangulation.}

\ccConstructor{Constrained_triangulation_plus_2(const
Constrained_triangulation_plus_2& ct);}{Copy constructor.}

\ccConstructor{Constrained_triangulation_plus_2(std::list<Constraint>& lc,
const Geom_traits& t = Geom_traits())}
{Introduces and builds a constrained triangulation 
from the list of constraints \ccc{lc}.}

\ccConstructor{ template<class InputIterator> 
Constrained_triangulation_plus_2(
        			InputIterator first,
                               	InputIterator last,
                               const Geom_traits& gt= Geom_traits());}
{Introduces and builds a constrained triangulation 
from the constraints in the range
$\left[\right.$\ccc{first}, \ccc{last}$\left.\right)$.
\ccPrecond The \ccc{value_type} of \ccc{first} and \ccc{last}
 is \ccc{Constraint}.
}

%\ccFunction{void ~ctp();}
%{Destructor. All vertices and faces are deleted.
%The constraint hierarchy is deleted.}



\ccHeading{Assignment}

\ccMethod{Constrained_triangulation_plus_2 operator=(const
Constrained_triangulation_plus_2& tr);}
{Assignment. All the vertices and faces are duplicated.
 The constraint hierarchy is also duplicated.}

\ccMethod{void swap(Constrained_triangulation_plus_2 tr);}
{The triangulations \ccc{tr} and \ccVar\ are swapped.
This operation should be preferred to \ccVar\ = \ccc{tr} or to
\ccc{t(tr)} if \ccc{tr} is deleted after that.}

\ccHeading{Insertion and Removal}
The class \ccRefName\ overwrites the following insertion and removal
member functions for points and constraints.

\ccMethod{Vertex_handle insert(const Point& p, 
Face_handle start = Face_handle() );}
{Inserts point \ccc{p} as a vertex of the triangulation.}

\ccMethod{Vertex_handle insert(const Point& p,
		       Locate_type lt,
		       Face_handle loc, int li );}
{inserts a point \ccc{p} whose location is assumed 
to be given by \ccc{(lt,loc,li)}.}

\ccMethod{Vertex_handle push_back(const Point& p);}
{Equivalent to \ccc{insert(p)}.}

\ccMethod{template < class InputIterator >
          size_type
          insert(InputIterator first, InputIterator last);}
{Inserts the points in the range
 $\left[\right.$\ccc{first}, \ccc{last}$\left.\right)$.
 Returns the number of inserted points.
 \ccPrecond The \ccc{value_type} of \ccc{first} and \ccc{last}
 is \ccc{Point}.}

\ccMethod{ void insert_constraint(Point a, Point b);}
{Inserts the constraint segment ab in the triangulation. }

\ccMethod{ void push_back(const Constraint& c);}
{Inserts the constraint \ccc{c}.}

\ccMethod{ void insert_constraint(Vertex_handle va, Vertex_handle vb);}
{ Inserts a constraint whose endpoints are the vertices 
pointed by \ccc{va} and \ccc{vb} in the triangulation. }

\ccMethod{void remove_constraint(Vertex_handle va, Vertex_handle vb);}
{Removes the constraint  joining the vertices pointed by \ccc{va} and \ccc{vb}.
\ccPrecond \ccc{va} and \ccc{vb} have to refer to  the endpoint vertices
of an  input constraint.}


\ccHeading{Queries}
\ccMethod{ Constraint_iterator constraints_begin() const;}
{Returns a \ccc{Constraint_iterator} pointing on the first
constraint.}
\ccGlue
\ccMethod{Constraint_iterator constraints_end()   const;}
{Returns a \ccc{Constraint_iterator} pointing past the last
constraint.}
\ccGlue
\ccMethod{Subconstraint_iterator subconstraints_begin() const;}
{Returns a \ccc{Subconstraint_iterator} pointing on the first
sub-constraint.}
\ccGlue
\ccMethod{Subconstraint_iterator subconstraints_end() const;}
{Returns a \ccc{Subconstraint_iterator} pointing past the last
sub-constraint.}
\ccGlue
\ccMethod{int number_of_enclosing_constraints(Vertex_handle va, 
				      Vertex_handle vb);}
{Returns the number of constraints enclosing the sub-constraint
\ccc{(va,vb).}
\ccPrecond \ccc{va} and \ccc{vb} refer to  the vertices
of a constrained edge of the triangulation.}
\ccGlue
\ccMethod{Context  context(Vertex_handle va, Vertex_handle vb);}
{Returns the \ccc{Context}  relative to  one of the constraint 
enclosing  the sub-constraint  \ccc{(va,vb)}.
\ccPrecond \ccc{va} and \ccc{vb} refer to  the vertices
of a constrained edge of the triangulation.}
\ccGlue
\ccMethod{Context_iterator   contexts_begin(Vertex_handle va, 
				    Vertex_handle vb);}
{Returns an iterator pointing on the first \ccc{Context}
of the sequence of \ccc{Contexts}
corresponding to the constraints enclosing the sub-constraint\ccc{(va,vb).}
\ccPrecond \ccc{va} and \ccc{vb} refer to  the vertices
of a constrained edge of the triangulation.}
\ccGlue
\ccMethod{Context_iterator   contexts_end(Vertex_handle va, 
				  Vertex_handle vb);}
{Returns an iterator past the last  \ccc{Context}
of the sequence of \ccc{Contexts}
corresponding to the constraints enclosing the  \ccc{(va,vb).}
\ccPrecond \ccc{va} and \ccc{vb} refer to  the vertices
of a constrained edge of the triangulation.}
\ccGlue
\ccMethod{Vertices_in_constraint_iterator 
vertices_in_constraint_begin(Vertex_handle va, Vertex_handle vb);}
{Returns an iterator on the first vertex on the constraint
\ccc{(va,vb)}
\ccPrecond \ccc{va} and \ccc{vb} refer to  the vertices
of an input constraint.}
\ccGlue
\ccMethod{Vertices_in_constraint_iterator 
vertices_in_constraint_end(Vertex_handle va, Vertex_handle vb);}
{Returns an iterator past  the last  vertex on  the constraint
\ccc{(va,vb)}
\ccPrecond \ccc{va} and \ccc{vb} refer to  the vertices
of an input constraints.}


\ccSeeAlso
\ccc{CGAL::Constrained_triangulation_2<Traits,Tds>} \\
\ccc{CGAL::Constrained_Delaunay_triangulation_2<Traits,Tds>} \\
\ccc{ConstrainedTriangulationTraits_2} \\
\ccc{ConstrainedDelaunayTriangulationTraits_2}
%\ccExample

%A short example program.
%Instead of a short program fragment, a full running program can be
%included using the 
%\verb|\ccIncludeExampleCode{Package/Constrained_triangulation_plus_2_prog.cpp}| 
%macro. The program example would be part of the source code distribution and
%also part of the automatic test suite.

%\begin{ccExampleCode}
%void your_example_code() {
%}
%\end{ccExampleCode}

%% \ccIncludeExampleCode{Package/Constrained_triangulation_plus_2_prog.cpp}

\end{ccRefClass}

% +------------------------------------------------------------------------+
%%RefPage: end of main body, begin of footer
% EOF
% +------------------------------------------------------------------------+

