%%
%%  *** CGAL Installation Guide ***
%%
%% file:     appendix.tex
%%
%% authors:  Michael Hoffmann, Dima Pasechnik and Wieger Wesselink
%%
%% $Revision$ $Date$
%%

\section{Using \cgal\ and \leda}\label{sec:leda}
\index{LEDA@\leda!support in \cgal}
\cgal\ supports \leda\ in the following ways.

\begin{enumerate}
\item There are support functions defined for the \leda\ number types
  \texttt{big\_float}, \texttt{integer}, \texttt{rational} and
  \texttt{real} (see the files \texttt{<CGAL/leda\_*>}).
\item For all two-dimensional geometric objects there are input/output
  operators for a \texttt{leda\_window}.
\item For all two-dimensional geometric objects there are output
  operators to a \texttt{leda\_ps\_file}.
\item The registration functions needed to interact with a
  \texttt{leda\_geowin} are defined for all geometric objects from the
  \cgal\ kernel.
\item \cgal\ defines the following \leda-related compiler flags when
      \leda\ is used: \texttt{CGAL\_USE\_LEDA} and \texttt{LEDA\_PREFIX}.
\end{enumerate}

The include makefiles in the \texttt{\cgaldir/make} directory
corresponding to \leda\ can be recognized by the suffix
``\texttt{\_LEDA}''.

\section{Compiler workarounds}
\index{compilers!workarounds}\index{workaround flags}

In \cgal, a number of compiler flags is defined. All of them start
with the prefix \texttt{CGAL\_CFG}. These flags are used to work
around compiler bugs and limitations. For example, the flag
\texttt{CGAL\_CFG\_NO\_LONG\_LONG} denotes that the compiler does not
know the type \texttt{long long}.

For each compiler a file \texttt{<CGAL/compiler\_config.h>}
\index{files!\texttt{compiler\_config.h}} is defined, with the correct
settings of all flags. This file is generated automatically by the
\texttt{install\_cgal} script, and it is located in the compiler
specific include directory. This directory can be found below
\texttt{include/CGAL/config/};
\index{directories!include/CGAL/config@\texttt{include/CGAL/config}}
\index{directories!compiler specific}\index{compiler specific include
  directory} it is named according to the compiler's \cgal-OS
description (cf. Section~\ref{sec:os-compiler-id}).

The test programs used to generate the \texttt{compiler\_config.h}
file can be found in \texttt{config/testfiles}.
\index{directories!config/testfiles@\texttt{config/testfiles}} Both
\texttt{compiler\_config.h} and the test programs contain a short
description of the problem. In case of trouble with one of the
\texttt{CGAL\_CFG} flags, it is a good idea to take a look at it.

Within \cgal, the file \texttt{<CGAL/basic.h>}
\index{files!\texttt{basic.h}} manages all configuration problems. In
particular, it includes the file \texttt{CGAL/compiler\_config.h}. It
is therefore \textbf{important that \texttt{<CGAL/basic.h>} is always
  included before any other file}. In most cases you do not have to do
anything special for this, because many \cgal\ files (in particular,
\texttt{<CGAL/Cartesian.h>} and \texttt{<CGAL/Homogeneous.h>}) already
take care of including \texttt{<CGAL/basic.h>} first. Nevertheless it
is a good idea to always start your \cgal\ programs with including
\texttt{<CGAL/basic.h>}.

\subsection{Standard Header Replacements}
\index{compilers!missing standard header files} %
\index{standard header replacements} %
\index{missing standard header files}

Some compilers do still not provide a complete standard library. In
particular they fail to provide the \texttt{C++} wrappers for files
from the standard \ccc{C} library, like \texttt{cstddef} for
\texttt{stddef.h}. The \cgal\ install scripts checks for all standard
header files and generates a simple wrapper file in the \cgal\ include
directory for those that are missing. These wrapper files include the
corresponding \texttt{C} header files and add all symbols required by
the \texttt{C++} standard into namespace \ccc{std}. You can turn off
the additions to namespace \ccc{std} by defining the macro
\ccc{CGAL_NO_STDC_NAMESPACE}.
\index{CGAL\_NO\_STDC\_NAMESPACE@\texttt{CGAL\_NO\_STDC\_NAMESPACE}}

\section{Compiler Optimizations}\label{sec:compiler-optimisations}
\index{compilers!optimization}
\index{optimization compiler flags}

You may have noticed that we do not set optimizer flags as \texttt{-O}
by default in the include makefiles(see section~\ref{sec:makefiles} for
a description of the makefile structure in \cgal).  The main reason
for not doing this is that compilers run much more stable without. On
the other hand, most if not all \cgal\ programs will run considerably
faster when compiled with optimizations! So if you are going for
performance, you should/have to add \texttt{-O}, \texttt{-O3} or maybe
more specific optimizer flags (please refer to the compiler
documentation for that) to the \texttt{CXXFLAGS} variable in your
application makefile:
\begin{alltt}
#---------------------------------------------------------------------#
#                    compiler flags
#---------------------------------------------------------------------#
# The flag CGAL_CXXFLAGS contains the path to the compiler and is defined
# in the file CGAL_MAKEFILE. You may add your own compiler flags to CXXFLAGS.

CXXFLAGS = $(CGAL_CXXFLAGS) -O
\end{alltt}
%$

\section{Troubleshooting}\label{sec:troubleshooting}
\index{troubleshooting}

This section contains some remarks about known problems and the
solutions we propose. If your problem is not listed here, please have
a look at the \cgal\ homepage:
\begin{alltt}
\cgalhomepage
\end{alltt}
or send an email to \texttt{info at cgal dot org}.

\subsection{Compiler version test execution failed}
\label{sec:compvertestfails}
\index{Compiler version test}\index{compilers!version test}

Possibly already during the startup of the install script, the
execution of the compiler version test might fail with the following
(or similar) error message.
\begin{verbatim}
  ld.so.1: ./tmp_test: fatal: libstdc++.so.5: 
    open failed: No such file or directory
\end{verbatim}
This means that the standard \CC\ library for your compiler is
installed in a directory that is not on your current runtime linker
path. You can solve this problem by adding the directory containing
\texttt{libstdc++.so} to your runtime linker path, usually
represented by the environment variable \texttt{LD\_LIBRARY\_PATH}.

For example, if you have a standard \texttt{gcc} installation below
\texttt{/software/gcc-3.3.2/}, you would type 
\begin{alltt}
      export LD_LIBRARY_PATH=/software/gcc-3.3.2/lib:$LD_LIBRARY_PATH
\end{alltt}
for bourne shell alikes, while for \texttt{csh} descendants the syntax
is
\begin{alltt}
      setenv LD_LIBRARY_PATH /software/gcc-3.3.2/lib:$LD_LIBRARY_PATH
\end{alltt}
You might want to add this command to your shell startup file.

Alternatively, you can build the runtime linker path into the
executables by setting corresponding custom linker flags (cf.
Section~\ref{sec:compiler-menu}).

\subsection{Defect in the G++ 3.2 ABI}
\label{sec:gcc32abi}
\index{gcc@\texttt{g++} 3.2 ABI}\index{compilers!gcc@\texttt{g++} 3.2 ABI}

Some versions of \texttt{gcc}, for example \Gcc{3.3.0}, have problems
in their \CC-ABI, that surface in error messages similar to the
following.
%
\begin{verbatim}
  error: due to a defect in the G++ 3.2 ABI, G++ has assigned 
    the same mangled name to two different types.
\end{verbatim}
%
If this occurs to you, please seriously consider upgrading your
compiler. This issue is fixed starting from \Gcc{3.3.1}.
Alternatively, you can add \texttt{-fabi-version=0} to your custom
compiler flags.
In interactive mode, this is done via the Compiler Menu,
as described in Section~\ref{sec:compiler-menu}. Afterwards rebuild
the libraries. But note that changing the \textsc{Abi} might have side
effects. Hence, a compiler upgrade is the recommended fix here.


\subsection{The ``Long-Name-Problem'' on Solaris}\label{sec:longnamesolaris}
\index{problems with long names}\index{long name problem}

The system assembler and linker on Solaris 2.5 and 2.6 cannot handle
symbols with more than 1024 characters. But this number is quickly
exceeded where one starts nesting templates into each other. So if you
encounter strange assembler or linker errors like 
\begin{alltt}
/usr/ccs/bin/as: "/var/tmp/cc0B5iGc.s", line 24: 
error: can't compute value of an expression involving an external symbol
\end{alltt}
there is a good chance that you suffer from this ``long-name''
problem.

A solution is to install the \gnu -binutils\footnote{see
  \path'http://www.gnu.org/software/binutils/'} and to tell the
compiler that it shall use the \gnu-- instead of the native tools.
From the compiler-menu (described in section \ref{sec:compiler-menu})
you can set the corresponding option through the custom compiler
flags, i.e. for \texttt{gcc} you would add
\begin{alltt}
  -B/my/path/to/gnu/binutils/bin
\end{alltt}
assuming you installed the \gnu -binutils executables in
\texttt{/my/path/to/gnu/binutils/bin}.

If you cannot (or do not want to) install \gnu -binutils, there is a
workaround that lets you compile, link and run your programs, but it
prevents debugging, since the executables have to be stripped. In
short the workaround is to compile with \texttt{-g} and to link with
\texttt{-z nodefs -s} on Solaris, \texttt{-U -s} on IRIX,
respectively.

In order to still have portable makefiles (see
section\ref{sec:makefiles}), we define flags
\texttt{LONG\_NAME\_PROBLEM\_CXXFLAGS} and
\texttt{LONG\_NAME\_PROBLEM\_LDFLAGS} in the include makefiles which
are empty except for the Solaris platform where they are set as stated
above. In order to use these flags, edit your application makefile and
add the flags to \texttt{CXXFLAGS} resp. \texttt{LDFLAGS} as indicated
below.
\begin{alltt}
#---------------------------------------------------------------------#
#                    compiler flags
#---------------------------------------------------------------------#
# The flag CGAL_CXXFLAGS contains the path to the compiler and is defined
# in the file CGAL_MAKEFILE. You may add your own compiler flags to CXXFLAGS.

CXXFLAGS = $(LONG_NAME_PROBLEM_CXXFLAGS) $(CGAL_CXXFLAGS) 

#---------------------------------------------------------------------#
#                    linker flags
#---------------------------------------------------------------------#
# The flag CGAL_LDFLAGS contains common linker flags and is defined
# in the file CGAL_MAKEFILE. You may add your own linker flags to CXXFLAGS.

LDFLAGS = $(LONG_NAME_PROBLEM_LDFLAGS) $(CGAL_LDFLAGS) 
\end{alltt}

\subsection{\leda\ and \stl\ conflicts}\label{subs:ledastlconfl}

If you are using an old version of \leda, the combination of \leda\ 
and \stl\ may give some problems. In order to avoid them, it is highly
recommended to use the latest \leda\ release, since this is what we
test \cgal\ with.

\index{LEDA@\leda!on \bcc}\index{LEDA@\leda!on \msvc}
With \msvc\ or \bcc, \leda\ has to be compiled and used with the
\texttt{LEDA\_STD\_HEADERS} flag set. \cgal\ uses \CC\ standard
conformant headers\footnote{the ones that do not have \texttt{.h}
  suffix}, while \leda\ can also work with the old-style header files;
but mixing the styles is strictly forbidden. Before compiling \leda\,
edit the file \texttt{\$(LEDAROOT)/incl/LEDA/system.h} and uncomment
the \texttt{\#define} in the following fragment there.
\begin{alltt}
// use c++ std headers
//#define LEDA_STD_HEADERS
\end{alltt}

\paragraph{\msvc-specific problems.}

Also, the \leda\ and \cgal\ libraries have to be compiled with the
same options controlling the use of debugging and multithreading.
\footnote{\msvc\ compilation/linking options \texttt{-ML, -MT, -MD,
    -MLD, -MTD, -MDD}}

If a binary release of \leda{} is used, make sure that it is one of
them that uses new-style headers. Namely, among the self-extracting
executables, choose one of these that have the name ending with
\texttt{-std.exe}.

\subsection{\msvc{}-specific \CC\ problems}
\index{msvc@\msvc{}!specific \CC\ problems}

\subsubsection{\msvc{6.0}}

This compiler version is no longer supported starting with \cgal\ 3.0.

\subsubsection{\msvc{7.0} (\textsc{.NET})}

This compiler version is no longer supported starting with \cgal\ 3.1.

\medskip

\textbf{Other problems.}

Here goes an incomplete list of problems encountered, and
\cgal-specific workarounds, if available.  Compiler error messages are
meant to be hints only, and do not pretend to be complete, as well.
\begin{enumerate}
%
\item Internal compiler errors can sometimes be avoided by increasing
  the amount of memory available to the compiler. Use \texttt{-Zm<number>}
option. In \cgal\ makefiles it is set to \texttt{-Zm900}, meaning
``using 900\% out of the usual memory limit''. \label{msvc::Zm}
%
\item \texttt{[...]/VC98/INCLUDE/xlocnum(268) :
    error C2587: '\_U' :\\
    illegal use of local variable as default parameter} can
  occur\footnote{For instance, in \cgal\ \texttt{Min\_circle}
    package}. The only workaround we know is to redefine the macro
  \texttt{\_VIRTUAL} in \texttt{<xlocnum>}\footnote{Yes, in the \msvc\ 
    header! You need not edit the actual file though. Copy it to
    a directory that is searched ahead of the other directories.
    DISCLAIMER:
    {\em We do not know if the actions described in this footnote are
      legal in your country. You are on your own here.}}
  to be empty. Search for\\
  \texttt{\#define~\_VIRTUAL~virtual} there and replace it by
  \texttt{\#define~\_VIRTUAL~}. \label{msvc::VIRTUAL}
%
\end{enumerate}

\section{Scripts}

\subsection{\texttt{create\_makefile}}\label{sec:create_cgal_makefile}
\TTindex{create\_makefile}\index{scripts!\texttt{create\_makefile}}

The bourne-shell script \texttt{create\_makefile} is contained in the
\cgaldir\texttt{/scripts} directory. It can be used to create
makefiles for compiling \cgal\ applications. Executing
\texttt{create\_makefile} in an application directory creates a
\texttt{makefile} containing rules for every \texttt{*.C} file there.

In order to use this makefile, you have to specify the \cgal\ include
makefile (see section~\ref{sec:makefiles}) to be used. This can be
done be either setting the environment variable
\texttt{CGAL\_MAKEFILE} or by editing the line
\begin{alltt}
  # CGAL_MAKEFILE = ENTER_YOUR_INCLUDE_MAKEFILE_HERE
\end{alltt}
of the created makefile. First remove the ``\texttt{\#}'' at the
beginning of the line and then replace the text after ``\texttt{=}''
by the location of the include makefile. 

Finally type \texttt{make} to compile the application programs.

%%
%% EOF
%%


