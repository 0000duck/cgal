\section{Configuring and Building Programs Using \cgal}

All examples and demos of \cgal contain \cmake\ scripts
(i.e.~\texttt{CMakeLists.txt}) to configure and build them. 
\cmake\ can also be used to configure and build user programs via such
scripts. 

During configuration of the \cgal\ libraries a file named {\tt
CGALConfig.cmake} is generated in the binary directory. This file
contains the definitions of several \cmake\ variable that summarize the
configuration of \cgal. In order to configure a user program, you need
to indicate the location of that config file in the \cmake\ variable
\texttt{CGAL\_DIR}:

{\ccTexHtml{\scriptsize}{}
\begin{alltt}

cd \cgalrel/examples/Straight_skeleton_2
cmake -DCGAL_DIR=$HOME/\cgalrel . 
make

\end{alltt}
}

%$ % <- added to close non-correct "math" enviroment ("HOME"), DO NOT DELETE

\texttt{CGAL\_DIR} can also be an environment variable.

If you have installed CGAL, \texttt{CGAL\_DIR} must afterward be set to
\texttt{\$CMAKE\_INSTALLED\_PREFIX/lib/CGAL}. Note that \cgal is
recommended to be installed in release mode when using it to build
new user programs.

\subsection{Custom flags in the programs using \cgal}

Normally, programs linked with \cgal\ must be compiled with the same flags
used by the compilation of \cgal\
libraries. For this reason, the {\em very first} time
a program is configured, all the flags given by the \cmake\ variables \texttt{CMAKE\_*\_FLAGS}
are {\em locked} in the sense that the values recorded in \texttt{CGALConfig.cmake} 
are used to override any values given by \cmake\ itself or yourself.

This does not apply to the additional flags that can be given via \texttt{CGAL\_*\_FLAGS}.

Such {\em inherited} values are then recorded in the current \cmake\ cache for the program.
The flags are then {\em unlocked} in the sense that at any subsequent configuration you can
provide your own flags and this time they will not be overridden.

When using the interactive \texttt{cmake-gui} the first press on \texttt{Configure} unlocks
the flags, so that you can edit them as needed. 

\begin{ccAdvanced}
The locking of flags is controlled by the variable {\tt CGAL\_DONT\_OVERRIDE\_CMAKE\_FLAGS}
which starts out FALSE and is toggled right after the flags have been loaded from
\texttt{CGALConfig.cmake}.

If you use the command line tool you can specify flags {\em directly} by setting the
controlling variable right up front:

{\ccTexHtml{\scriptsize}{}
\begin{alltt}

cd \cgalrel

cmake -DCMAKE_BUILD_TYPE=Release -DCMAKE_CXX_FLAGS=-g .

cd \cgalrel/examples/Straight_skeleton_2

cmake -DCGAL_DIR=\cgalrel -DCMAKE_BUILD_TYPE=Debug -DCMAKE_CXX_FLAGS=-O2 -DCGAL_DONT_OVERRIDE_CMAKE_FLAGS=TRUE . 

\end{alltt}
}
\end{ccAdvanced}


\subsection{Creating a cmake script for a program using \cgal}

For relatively simple programs two Bourne-shell scripts can be
used. They are introduced in the subsections below.

More generally, within a \cmake\ script, once \cgal\ has been found using \texttt{find\_package}
the variable \texttt{CGAL\_USE\_FILE} is set to a compilation environment CMake file. Including
this file within a cmake script sets up include paths and libraries to
link with of \cgal\ and essential third party libraries. Beyond,
\texttt{find\_package} can demand for \texttt{COMPONENTS} of \cgal,
that is, all \cgal\ libaries libCGALCore (``Core''),
libCGALimageIO (``ImageIO'') , libCGALQt3 (``Qt3'') and libCGALQt4
(``Qt4'') or optional 3rd party software such as ``MPFI'', ``RS3'' or
``LAPACK''. 

\subsection{\texttt{cgal\_create\_CMakeLists.txt\label{sec:create_cgal_CMakeLists.txt}}}
\TTindex{cgal\_create\_CMakeLists.txt}\index{scripts!\texttt{cgal\_create\_CMakeLists.txt}}

The Bourne-shell script \texttt{cgal\_create\_CMakeLists.txt} is contained in the
\texttt{\cgalrel/scripts} directory. It can be used to create
\texttt{CmakeLists.txt} files for compiling \cgal\ applications. Executing
\texttt{cgal\_create\_CMakeLists.txt} in an application directory creates a
\texttt{CMakeLists.txt} containing rules to build the contained
application(s). Three command line options determine details of the
configuration.

\begin{description}
\item [\texttt{-s source}] If this parameter is given the script will
  create \textbf{a single executable} for 'source' linked with
  compilations of all other source files (\texttt{*.cpp}). This
  behaviour is usually needed for (graphical) demos. \\
  If the parameter is not given, the script creates \textbf{one executable for each given
  source file}.
\item [\texttt{-c com1:com2:...}] Lists components (``com1'',
  ``com2'') of \cgal\ to which the executable(s) should be linked. Valid components are \cgal's
  libraries (i.e.~``Core'', ``ImageIO'', ``Qt3'' and ``Qt4''; note
  that it only make sense to either pick ``Qt3'' or ``Qt4'') and all
  preconfigured 3rd party software, such as ``MPFI'', ``RS3'',
  or ``LAPACK''). An example is \texttt{-c Core:GMP:RS3:MPFI}

\item [\texttt{-b boost1:boost2:...}] Lists components (``boost1'',
  ``boost2'') of \boost\ to which the executable(s) should be
  linked. Valid options are, for instance, ``filesystem'' or ``program\_options''.

\end{description}

This options should suffice to create \texttt{CMakeLists.txt} script
for most directories containing programs. However, in some special
cases, it might still be required to create the script manually, for
instance, if some source files/executables need a different linking than
other source files. 

\subsection{\texttt{cgal\_create\_cmake\_script\label{sec:create_cgal_cmake_script}}}
\TTindex{cgal\_create\_cmake\_script}\index{scripts!\texttt{cgal\_create\_cmake\_script}}

For backward-compatibility we still provide the 
Bourne-shell script \texttt{cgal\_create\_cmake\_script} that 
is contained in the
\texttt{\cgalrel/scripts} directory. It can be used to create
\texttt{CmakeLists.txt} files for compiling \cgal\ applications. Executing
\texttt{cgal\_create\_cmake\_script} in an application directory creates a
\texttt{CMakeLists.txt} containing rules for every \texttt{*.cpp} file
there. The script is deprecated, as it only works for applications 
with a single course file that only need the \cgal\ and CGALCore
libraries.


%%
%% EOF
%%


