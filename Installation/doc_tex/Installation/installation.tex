%%
%%  *** CGAL Installation Guide ***
%%
%% file:     installation.tex
%%
%% authors:  Michael Hoffmann, Dima Pasechnik and Wieger Wesselink
%%
%% $Id$
%%


%% macro for Windows-specific installation
\newcommand{\MSInst}{Windows-specific Installation}

%% macro for Windows-specific installation
\newcommand{\cmake}{CMake}
\newcommand{\cmakepage}{\path'http://www.cmake.org/'}

%% macro for g++
\newcommand{\Gcc}[1]{\gnu~\texttt{g++}~{\rm #1}}
\newcommand{\gccurl}{\path'http://gcc.gnu.org/'}

%% macro for SUNPRO CC
\newcommand{\sunprocc}[1]{\textsc{Sunpro}~\texttt{CC}~{\rm #1}}
\newcommand{\sunproccurl}{\path'http://developers.sun.com/sunstudio/'}

%% macro for Borland C++
\newcommand{\bcc}[1]{\textsc{Borland}~\texttt{C++}~{\rm #1}}
\newcommand{\bccurl}{\path'http://www.inprise.com/bcppbuilder/'}

%% macro for Microsoft Visual C++
\newcommand{\msvc}[1]{\textsc{MS}~Visual~\texttt{C++}~{\rm #1}}
\newcommand{\msvcurl}{\path'http://msdn.microsoft.com/visualc/'}

%% macro for Intel C++ Compiler
\newcommand{\icl}[1]{\textsc{Intel}~\texttt{C++}~{\rm #1}}
\newcommand{\iclurl}{\path'http://developer.intel.com/software/products/compilers/'}

%% macro for Microsoft Windows
\newcommand{\mswin}{\textsc{MS}~Windows}

%% macro for CGAL release number
\newcommand{\cgalrelease}{3.4}

%% macro for CGAL directory
\newcommand{\cgaldir}{\texttt{CGAL-\cgalrelease}}

%% macro for YOUR CGAL directory :)
\newcommand{\yourcgaldir}{$<$\textit{insert your \cgaldir\ dir}$>$}

%% macro for CGAL install config directory
\newcommand{\cgalinstconfdir}{\cgaldir\texttt{/config/install}\index{directories!config/install@\texttt{config/install}}}

%% macros for WWW pages
\newcommand{\cgalhomepage}{\path'http://www.cgal.org'}
\newcommand{\gmppage}{\path'http://gmplib.org/'}
\newcommand{\mpfrpage}{\path'http://www.mpfr.org/'}
\newcommand{\boostpage}{\path'http://www.boost.org/'}
\newcommand{\corepage}{\path'http://www.cs.nyu.edu/exact/core\_pages/'}
\newcommand{\ledapage}{\path'http://www.algorithmic-solutions.com/'}
\newcommand{\trolltechpage}{\path'http://www.trolltech.com'}
\newcommand{\qtpage}{\path'http://doc.trolltech.com/'}
\newcommand{\qtmocpage}{\path'http://doc.trolltech.com/moc.html'}
\newcommand{\blaspage}{\path'http://www.netlib.org/blas/'}
\newcommand{\atlaspage}{\path'http://math-atlas.sourceforge.net/'}
\newcommand{\lapackpage}{\path'http://www.netlib.org/lapack/'}
\newcommand{\taucspage}{\path'http://www.tau.ac.il/~stoledo/taucs/'}
\newcommand{\xpage}{\path'http://www.x.org/'}
\newcommand{\zlibpage}{\path'http://www.zlib.net/'}

%% write index in typewriter type. (c) Susan :-)
\newcommand{\TTindex}[1]{\index{#1@{\tt #1}}}
\newcommand{\TTsubindex}[2]{\index{#1@{\tt #1}!{#2}}}
\newcommand{\TTsubindextwo}[2]{\index{#1!#2@{\tt #2} }}

\section{Introduction}

\cgal\ stands for \textit{Computational Geometry Algorithms Library}.
It is a software library written in \CC, whose development started in
an {\sc Esprit Ltr} project. The goal of \cgal\ is to make the large
body of geometric algorithms developed in the field of computational
geometry available for industrial application.

This document describes how to install \cgal\ on Windows, Unix-like systems and Mac.

Besides that, you will find some information about the build system
structure of \cgal\ and the support for using \cgal\ together with
other software libraries, such as the \boost\
libraries\footnote{\boostpage}, the \gnu\ Multiple Precision library
\gmp\footnote{\gmppage}, the \mpfr\ library\footnote{\mpfrpage} for
multiple-precision floating-point computations with exact rounding,
the \core\ library\footnote{\corepage} for robust numerical and
geometric computation, \leda, the Library of Efficient Datatypes and
Algorithms\footnote{\ledapage}, \blas, the Basic Linear Algebra Subprograms
library\footnote{\blaspage}, \lapack, the Linear Algebra PACKage library
\footnote{\lapackpage}, \taucs, a library of sparse linear
solvers\footnote{\taucspage}, or Trolltech's\footnote{\trolltechpage}
\qt\ toolkit.

\section{Prerequisites\label{sec:prerequisites}}

In order to build the \cgal\ libraries you need a \CC\ compiler.  Most
recent compilers on Unix platforms and \mswin\ are supported, provided
that they reasonably conform to the ISO 14882 standard for \CC\ (C++98
and C++03).

\cgaldir\ supports the following compilers/operating systems:

\begin{center}\index{compilers!supported}\index{supported compilers}
  \renewcommand{\arraystretch}{1.3}
  \gdef\lcTabularBorder{2}
  \begin{tabular}{|l|l|} \hline
    \textbf{compiler}        & \textbf{operating system}\\\hline\hline
    \Gcc{3.4, 4.0, 4.1, 4.2, 4.3} \footnotemark[10]
    & Solaris 2.6+ / Linux 2.x / MacOS X /
      \\ & \mswin\ 95/98/2000/XP/NT4\footnotemark[11]\\\hline
    \msvc{8.0, 9.0} (\textsc{Visual Studio 2005 and 2008}) \footnotemark[12]
    & \mswin\ 95/98/2000/XP/NT4/Vista\footnotemark[11]\\\hline
    \icl{8.1} \footnotemark[13]
    & \mswin\ 95/98/2000/XP/NT4/Vista\footnotemark[11]\\\hline
  \end{tabular}
\end{center}
\footnotetext[10]{\gccurl}\addtocounter{footnote}{1}
\footnotetext[11]{\msvcurl}\addtocounter{footnote}{1}
\footnotetext[12]{\iclurl}\addtocounter{footnote}{1}

Note that neither prerelease versions nor repository snapshots of
\texttt{GCC} are supported.

In order to configure, build and install the \cgal\ libraries, as well
as any program using \cgal, you need \cmake, a Cross Platform Make system.
If \cmake\ is not installed already you can obtain it from \cmakepage,
version of 2.4-patch-7 or later.

Moreover, \cgal\ requires a working installation of the \boost\
libraries (the header files and the threading library binaries), 
Version 1.33.1 or later. In case
the \boost\ libraries are not installed on your system already, you
can obtain them from \boostpage. For Windows you can download an
installer from \path'http://www.boost-consulting.com/download.html'.
Since \ccc{Boost.Thread} is required, make sure to either install the precompiled 
{\em static} binaries for your compiler or to build \ccc{libboost-thread} appropiately.

Having \gmp\ version 4.1.4 or later and \mpfr\ version 2.0.1 or later
installed is highly recommended. These libraries can be obtained from
\gmppage\ and \mpfrpage, respectively, if they should not be present
on your system. Note that \mpfr\ used to be included with \gmp\ up to
\gmp\ version 4.1.X, but it is not included anymore starting from
\gmp~4.2. Within \cgal, support for \gmp\ requires also \mpfr\ and
vice versa.

If you are going to install \cgal\ using
Cygwin\footnote{\path~http://www.cygwin.com~}, please read
Section~\ref{sec:cygwin} first.

\section{Getting CGAL\label{sec:gettingcgal}}
\index{CGAL@\cgal!getting}\index{getting \cgal}

The \cgal\ library can be downloaded from the \cgal\
homepage:\index{CGAL@\cgal!homepage}
\begin{quote}
      \cgalhomepage
\end{quote}
and go to the `Download' section. Just follow the instructions on this
page to obtain your copy of the library.

After you have downloaded the file containing the \cgal\ library, you
have to decompress it. Under a Unix-like shell, use the commands

\begin{verbatim}
      gunzip <filename>.tar.gz
      tar xvf <filename>.tar
\end{verbatim}

Alternatively, your browser might be able to invoke the right
decompression program by itself.

In both cases the directory \cgaldir\ will be created. This directory
contains the following subdirectories:\index{directories!structure}

\begin{center}
  \renewcommand{\arraystretch}{1.3}
  \gdef\lcTabularBorder{2}
  \begin{tabular}{|l|l|} \hline
    \textbf{directory} & \textbf{contents}\\\hline\hline
    \texttt{auxiliary} & packages that can optionally be used with \cgal\\\hline
    \texttt{config}    & configuration files for install script\\\hline
    \texttt{demo}      & demo programs (most of them need \qt, geomview
                         or other third-party products)\\\hline
    \texttt{doc\_html} & documentation (HTML)\\\hline
    \texttt{doc\_pdf}  & documentation (PDF)\\\hline
    \texttt{doc\_ps}   & documentation (Postscript)\\\hline
    \texttt{examples}  & example programs\\\hline
    \texttt{include}   & header files\\\hline
    \texttt{lib}       & (shared) object libraries\\\hline
    \texttt{make}      & files with platform dependent makefile settings\\\hline
    \texttt{scripts}   & some useful scripts (e.g. for creating makefiles)\\\hline
    \texttt{src}       & source files\\\hline
  \end{tabular}
\end{center}

The directories \texttt{include/CGAL/CORE} and \texttt{src/CGALCore} contain a
distribution of the \core\ library\footnote{\corepage} version~1.7 for
robust numerical and geometric computation. \core\ is not part of
\cgal\ and has its own license.

\subsection{Visualization\label{sec:vis}}

The programs below the \texttt{demo} directory provide visual output.
Most of these use \ccc{CGAL::Qt_widget}, a widget and some helper
classes that allow to interact with two dimensional \cgal\ objects in
\qt~3 based applications.\index{visualization!Qt} There is no support
for \qt~4 in \cgal\ yet.

Some demo programs for 3D structures require the geomview program for
visualization.\index{visualization!geomview} This is available from
\path'http://www.geomview.org' (note that it does not run on \mswin).

\section{Configuring, Building and Installing CGAL}

These tasks are handled by \cmake, as explained in the the rest of this chapter.
\footnote{Earlier versions of \cgal\ used a custom build system based on a shell-script,
but starting at version 3.4, this is replaced by \cmake.}
\footnote{\cmake\ is a sophisticated tool with its own scripting language, modules 
and details. This manual provides only a simple overview of \cmake\ sufficient to work
with \cgal. Please refer to the \cmake\ documentation at \cmakepage\ for further details.}

\subsection{CMake}

\cmake\ is cross-platform make system which can be used to configure, build and install
libraries and executables. It works uniformly on many platforms (such as Windows, 
UNIX/Linux and the Mac) and it supports a varierty of development tools 
(compilers, linkers etc) on each platform.

\subsection{Configuring \cgal}

As with any library/program, building \cgal\ involves setting up {\em configuration parameters}:
the choice of compiler/linker, the corresponding flags, the details of the required third-party libraries,
some user choices such as which cgal component to buid, etc... Gathering all these parameters is called
{\em configuration}, it is the very first step in the process and \cmake\ is the tool responsible for that. 
When \cmake\ finishes the configuration, it produces a makefile or project file than you can use to build cgal.

The simplest way to start the configuration is to run a command-line tool called \ccc{cmake} passing 
as argument the root directory of \cgal. For example:

{\ccTexHtml{\scriptsize}{}
\begin{verbatim}
  cd CGAL 
  cmake . # Notice the dot to indicate the current directory.
\end{verbatim}
}

This executes a special-purpose \cmake\ script
\footnote{Called \ccc{CMakeLists.txt} and located in the \cgal\ root} 
which attempts to determine the value of all the configuration parameters needed to build the \cgal\
libraries and any programs using \cgal. 

Here are the first lines of a typical output in a Windows machine:

{\ccTexHtml{\scriptsize}{}
\begin{verbatim}
-- Check for working C compiler: cl
-- Check for working C compiler: cl -- works
-- Detecting C compiler ABI info
-- Detecting C compiler ABI info - done
-- Check for working CXX compiler: cl
-- Check for working CXX compiler: cl -- works
-- Detecting CXX compiler ABI info
-- Detecting CXX compiler ABI info - done
\end{verbatim}
}

As you can see, the very first thing \cmake\ does is to detect the compiler to use.
This detection is performed by a special cmake module called a {\em generator}. A
\cmake\ generator understands the build requirements for a particular compiler/linker
 and generates the necessary files for that. For example, the 
{\em UNIX Makefiles} generator understands the GNU toolchain (g++,gcc,ld etc) and produces makefiles
which can be used to build a target by a simple call to \ccc{make}.
Likewise, the {\em Visual Studio 2005} generator produces solution and project files and can be manually
launched in the VS IDE to build the target.

Each platform has a default generator so you only need to select one when the default is
not what you want. 
\footnote{Running \ccc{cmake} with no parameters in a command-line prints the list of available generators supported 
by your platform and \cmake\ version. If the generator you need is not listed there, you can
try a newer \cmake\ version since generators are hardcoded into \cmake\ and more and
more generators are added with each release.}
For example, under Windows, it is possible to generate {\em NMakefiles}
instead of Visual Studio project files in order to build the library via \ccc{nmake}.

Since the choice of generator determines the type of build files to generate, in some cases
you choose a particular generator as a mean to choose a specific compiler (because they use different 
build files. For example, the following generates solution files for use in Visual C++ 9.0:

{\ccTexHtml{\scriptsize}{}
\begin{verbatim}
  cd CGAL 
  cmake -G"Visual Studio 9 2008" . 
\end{verbatim}
}

In other cases, however, the generator doesn't directly identify a specific compiler but a tool chain.
For example, the \ccc{UNIX Makefiles} generator produces \ccc{makefiles} wich call some auto-detected
command-line compiler, like \ccc{gcc}. If you need the makefiles to use a different compiler you need to
say so in the call to cmake, as in this example:

{\ccTexHtml{\scriptsize}{}
\begin{verbatim}
  cd CGAL 
  cmake -DCMAKE_CXX_COMPILER:FILEPATH=some_non_default_g++ . 
\end{verbatim}
}


\cmake\ mantains configuration parameters in so-called {\em cmake variables}, like the \ccc{CMAKE_CXX_COMPILER}
in the example above. These variables {\em are not environment variables} and they must be set and modified in
certain specific ways as explained below. Some of the cmake variables represent user choices, such as
\ccc{WITH_examples} or \ccc{CMAKE_BUILD_TYPE=Release}, while others indicate the details of a third-party library, 
such as \ccc{Boost_INCLUDE_DIR} or the compiler flags to use, such as \ccc{CMAKE_CXX_FLAGS}. 

The command line tool \ccc{cmake} accepts cmake variables as arguments of the form \ccc{-D<VAR>:<TYPE>=<VALUE>}, as
in the example above, but this is only useful if you already know which variables needs to be explicitely defined.
\cmake\ provides an interactive tool which lets you configure \cgal\ by suppling the missing pieces 
which \cmake\ indicates. In Windows, this is the \ccc{CMakeSetup.exe} program, while in Unix/Linux/Mac it is \ccc{ccmake}
(or \ccc{cmake-gui} if you have a recent cmake version). 

These interactive configuration tools execute the cmake script just like the non-interactive \ccc{cmake} does. It assign
values to all the cmake variables used in the configuration and presents the variables to you. You can then change 
any of the values (such as those representing user choices: \ccc{WITH_<choice>}) and once they are all OK you can select 
\ccc{Configure} to finish. It will produce the build files just like the non-interactive cmake.

Some variables however, like those corresponding to the path of required third-party dependencies, are critical: they 
must be defined with proper values. If the script fails to locate a required dependency, for example, the corresponding variable,
say \ccc{Boost_INCLUDE_DIR}, will have a \ccc{NOTFOUND} value and will show up distinctively (i.e. in red in the GUI cmake).
\cmake\ will refuse to generate the build files until all these variables are given proper values, so you need
to define them and press \ccc{Configure} iteratively until there are no more critical variables left undefined. Only then
pressing \ccc{Configure} will finish the configuration, build the files and return.

\begin{ccAdvanced}
\cmake\ keeps the variables that a user can manipulate in a so-called {\em cmake cache}, a simple text file specifically 
named \ccc{CMakeCache.txt} whose entries are of the form VARIABLE:TYPE=VALUE. Advanced users can just manually edit this 
file instead of going through the interactive configuration session explained before.
\end{ccAdvanced}


The configuration process not only determines the location of the required dependencies. It also dynamically generates a
\ccc{compiler_config.h} file which encodes the properties of your system. It also generates a special file named 
\ccc{CGALConfig.cmake} which is used to build programs using \cgal\ (the usage of this file will be explained later).

\subsection{Building \cgal libraries}


\begin{ccAdvanced}
The makefiles/project-files produced by cmake are autoconfigured. That is, if you change any 
of the dependencies, the build step automatically goes all the way back to the configure step
during the building. This way, once the target has been configured the very first time by
invoking cmake, you don't neccesarily need to invoke \ccc{cmake} again and invoking just \ccc{make}
(or building from within an IDE) will call itself \ccc{cmake} and regenerate the makefile/project-file
if needed. Keep this in mind if you configure \cgal\ for the Visual Studio IDE since a build
can change the project file in place and VS will prompt you to reload it.
\end{ccAdvanced}

%%
%% EOF
%%


