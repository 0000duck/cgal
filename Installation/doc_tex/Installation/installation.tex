%%
%%  *** CGAL Installation Guide ***
%%
%% file:     installation.tex
%%
%% authors:  Michael Hoffmann, Dima Pasechnik and Wieger Wesselink
%%
%% $Id$
%%

\newcommand{\cmake}{CMake}
\newcommand{\cpack}{CPack}
\newcommand{\nsis}{NullSoft Scriptable Installer}


%% macro for Windows-specific installation
\newcommand{\MSInst}{Windows-specific Installation}

%% macro for g++
\newcommand{\Gcc}[1]{\gnu~\texttt{g++}~{\rm #1}}
\newcommand{\gccurl}{\path'http://gcc.gnu.org/'}

%% macro for SUNPRO CC
\newcommand{\sunprocc}[1]{\textsc{Sunpro}~\texttt{CC}~{\rm #1}}
\newcommand{\sunproccurl}{\path'http://developers.sun.com/sunstudio/'}

%% macro for Borland C++
\newcommand{\bcc}[1]{\textsc{Borland}~\texttt{C++}~{\rm #1}}
\newcommand{\bccurl}{\path'http://www.inprise.com/bcppbuilder/'}

%% macro for Microsoft Visual C++
\newcommand{\msvc}[1]{\textsc{MS}~Visual~\texttt{C++}~{\rm #1}}
\newcommand{\msvcurl}{\path'http://msdn.microsoft.com/visualc/'}

%% macro for Intel C++ Compiler
\newcommand{\icl}[1]{\textsc{Intel}~\texttt{C++}~{\rm #1}}
\newcommand{\iclurl}{\path'http://developer.intel.com/software/products/compilers/'}

%% macro for Microsoft Windows
\newcommand{\mswin}{\textsc{MS}~Windows}

%% macro for CGAL release number
\newcommand{\cgalrelease}{3.4}

%% macro for CGAL directory
\newcommand{\cgaldir}{\texttt{CGAL-\cgalrelease}}

%% macro for YOUR CGAL directory :)
\newcommand{\yourcgaldir}{$<$\textit{insert your \cgaldir\ dir}$>$}

%% macro for CGAL install config directory
\newcommand{\cgalinstconfdir}{\cgaldir\texttt{/config/install}\index{directories!config/install@\texttt{config/install}}}

%% macros for WWW pages
\newcommand{\cgalhomepage}{\path'http://www.cgal.org'}
\newcommand{\gmppage}{\path'http://gmplib.org/'}
\newcommand{\mpfrpage}{\path'http://www.mpfr.org/'}
\newcommand{\boostpage}{\path'http://www.boost.org/'}
\newcommand{\corepage}{\path'http://www.cs.nyu.edu/exact/core\_pages/'}
\newcommand{\ledapage}{\path'http://www.algorithmic-solutions.com/'}
\newcommand{\trolltechpage}{\path'http://www.trolltech.com'}
\newcommand{\qtpage}{\path'http://doc.trolltech.com/'}
\newcommand{\qtmocpage}{\path'http://doc.trolltech.com/moc.html'}
\newcommand{\blaspage}{\path'http://www.netlib.org/blas/'}
\newcommand{\atlaspage}{\path'http://math-atlas.sourceforge.net/'}
\newcommand{\lapackpage}{\path'http://www.netlib.org/lapack/'}
\newcommand{\taucspage}{\path'http://www.tau.ac.il/~stoledo/taucs/'}
\newcommand{\xpage}{\path'http://www.x.org/'}
\newcommand{\zlibpage}{\path'http://www.zlib.net/'}
\newcommand{\cmakepage}{\path'http://www.cmake.org/'}
\newcommand{\cpackpage}{\path'http://www.cmake.org/cmake/help/cpack2.6docs.html'}
\newcommand{\nsispage}{\path'http://nsis.sourceforge.net/Main\_Page'}
\newcommand{\fhs}{\path'http://www.pathname.com/fhs/pub/fhs-2.3.html'}

%% write index in typewriter type. (c) Susan :-)
\newcommand{\TTindex}[1]{\index{#1@{\tt #1}}}
\newcommand{\TTsubindex}[2]{\index{#1@{\tt #1}!{#2}}}
\newcommand{\TTsubindextwo}[2]{\index{#1!#2@{\tt #2} }}

\section{Introduction}

\cgal\ stands for \textit{Computational Geometry Algorithms Library}.
It is a software library written in \CC, whose development started in
an {\sc Esprit Ltr} project. The goal of \cgal\ is to make the large
body of geometric algorithms developed in the field of computational
geometry available for industrial application.

This document describes how to install \cgal\ on Windows, Unix-like systems and Mac.

Besides that, you will find some information about the build system
structure of \cgal\ and the support for using \cgal\ together with
other software libraries, such as the \boost\
libraries\footnote{\boostpage}, the \gnu\ Multiple Precision library
\gmp\footnote{\gmppage}, the \mpfr\ library\footnote{\mpfrpage} for
multiple-precision floating-point computations with exact rounding,
the \core\ library\footnote{\corepage} for robust numerical and
geometric computation, \leda, the Library of Efficient Datatypes and
Algorithms\footnote{\ledapage}, \blas, the Basic Linear Algebra Subprograms
library\footnote{\blaspage}, \lapack, the Linear Algebra PACKage library
\footnote{\lapackpage}, \taucs, a library of sparse linear
solvers\footnote{\taucspage}, or Trolltech's\footnote{\trolltechpage}
\qt\ toolkit.

\section{Prerequisites\label{sec:prerequisites}}

In order to build the \cgal\ libraries you need a \CC\ compiler.  Most
recent compilers on Unix platforms and \mswin\ are supported, provided
that they reasonably conform to the ISO 14882 standard for \CC\ (C++98
and C++03).

\cgaldir\ supports the following compilers/operating systems:

\begin{center}\index{compilers!supported}\index{supported compilers}
  \renewcommand{\arraystretch}{1.3}
  \gdef\lcTabularBorder{2}
  \begin{tabular}{|l|l|} \hline
    \textbf{compiler}        & \textbf{operating system}\\\hline\hline
    \Gcc{3.4, 4.0, 4.1, 4.2, 4.3} \footnotemark[10]
    & Solaris 2.6+ / Linux 2.x / MacOS X /
      \\ & \mswin\ 95/98/2000/XP/NT4\footnotemark[11]\\\hline
    \msvc{8.0, 9.0} (\textsc{Visual Studio 2005 and 2008}) \footnotemark[12]
    & \mswin\ 95/98/2000/XP/NT4/Vista\footnotemark[11]\\\hline
    \icl{8.1} \footnotemark[13]
    & \mswin\ 95/98/2000/XP/NT4/Vista\footnotemark[11]\\\hline
  \end{tabular}
\end{center}
\footnotetext[10]{\gccurl}\addtocounter{footnote}{1}
\footnotetext[11]{\msvcurl}\addtocounter{footnote}{1}
\footnotetext[12]{\iclurl}\addtocounter{footnote}{1}

Note that neither prerelease versions nor repository snapshots of
\texttt{GCC} are supported.

In order to configure, build and install the \cgal\ libraries, as well
as any program using \cgal, you need \cmake, a Cross Platform Make system.
If \cmake\ is not installed already you can obtain it from \cmakepage,
version of 2.4-patch-7 or later.

Moreover, \cgal\ requires a working installation of the \boost\
libraries (the header files and the threading library binaries), 
Version 1.33.1 or later. In case
the \boost\ libraries are not installed on your system already, you
can obtain them from \boostpage. For Windows you can download an
installer from \path'http://www.boost-consulting.com/download.html'.
Since \texttt{Boost.Thread} is required, make sure to either install the precompiled 
{\em static} binaries for your compiler or to build \texttt{libboost-thread} appropiately.

Having \gmp\ version 4.1.4 or later and \mpfr\ version 2.0.1 or later
installed is highly recommended. These libraries can be obtained from
\gmppage\ and \mpfrpage, respectively, if they should not be present
on your system. Note that \mpfr\ used to be included with \gmp\ up to
\gmp\ version 4.1.X, but it is not included anymore starting from
\gmp~4.2. Within \cgal, support for \gmp\ requires also \mpfr\ and
vice versa.

If you are going to install \cgal\ using
Cygwin\footnote{\path~http://www.cygwin.com~}, please read
Section~\ref{sec:cygwin} first.

\section{Getting CGAL\label{sec:gettingcgal}}
\index{CGAL@\cgal!getting}\index{getting \cgal}

The \cgal\ library can be downloaded from the \cgal\
homepage:\index{CGAL@\cgal!homepage}
\begin{quote}
      \cgalhomepage
\end{quote}
and go to the `Download' section. Just follow the instructions on this
page to obtain your copy of the library.

After you have downloaded the file containing the \cgal\ library, you
have to decompress it. Under a Unix-like shell, use the commands

\begin{verbatim}
      gunzip <filename>.tar.gz
      tar xvf <filename>.tar
\end{verbatim}

Alternatively, your browser might be able to invoke the right
decompression program by itself.

In both cases the directory \cgaldir\ will be created. This directory
contains the following subdirectories:\index{directories!structure}

\begin{center}
  \renewcommand{\arraystretch}{1.3}
  \gdef\lcTabularBorder{2}
  \begin{tabular}{|l|l|} \hline
    \textbf{directory} & \textbf{contents}\\\hline\hline
    \texttt{auxiliary} & packages that can optionally be used with \cgal\\\hline
    \texttt{config}    & configuration files for install script\\\hline
    \texttt{demo}      & demo programs (most of them need \qt, geomview
                         or other third-party products)\\\hline
    \texttt{doc\_html} & documentation (HTML)\\\hline
    \texttt{doc\_pdf}  & documentation (PDF)\\\hline
    \texttt{doc\_ps}   & documentation (Postscript)\\\hline
    \texttt{examples}  & example programs\\\hline
    \texttt{include}   & header files\\\hline
    \texttt{lib}       & (shared) object libraries\\\hline
    \texttt{make}      & files with platform dependent makefile settings\\\hline
    \texttt{scripts}   & some useful scripts (e.g. for creating makefiles)\\\hline
    \texttt{src}       & source files\\\hline
  \end{tabular}
\end{center}

The directories \texttt{include/CGAL/CORE} and \texttt{src/CGALCore} contain a
distribution of the \core\ library\footnote{\corepage} version~1.7 for
robust numerical and geometric computation. \core\ is not part of
\cgal\ and has its own license.

\subsection{Visualization\label{sec:vis}}

The programs below the \texttt{demo} directory provide visual output.
Most of these use \ccc{CGAL::Qt_widget}, a widget and some helper
classes that allow to interact with two dimensional \cgal\ objects in
\qt~3 based applications.\index{visualization!Qt} There is no support
for \qt~4 in \cgal\ yet.

Some demo programs for 3D structures require the geomview program for
visualization.\index{visualization!geomview} This is available from
\path'http://www.geomview.org' (note that it does not run on \mswin).

\section{Configuring, Building and Installing CGAL}

These tasks are handled by \cmake, as explained in the the rest of this chapter.
\footnote{Earlier versions of \cgal\ used a custom build system based on a shell-script,
but starting at version 3.4, this is replaced by \cmake.}
\footnote{\cmake\ is a sophisticated tool with its own scripting language, modules 
and details. This manual provides only a simple overview of \cmake\ sufficient to work
with \cgal. Please refer to the \cmake\ documentation at \cmakepage\ for further details.}

\subsection{CMake}

\cmake\ is cross-platform make system which can be used to configure, build and install
libraries and executables. It works uniformly on many platforms (such as Windows, 
UNIX/Linux and the Mac) and it supports a varierty of development tools 
(compilers, linkers etc) on each platform.

\subsection{Configuring \cgal}

As with any library/program, building \cgal\ involves setting up {\em configuration parameters}:
the choice of compiler/linker, the corresponding flags, the details of the required third-party libraries,
some user choices such as which cgal component to buid, etc... Gathering all these parameters is called
{\em configuration}, it is the very first step in the process and \cmake\ is the tool responsible for that. 
When \cmake\ finishes the configuration, it produces a makefile or project file than you can use to build cgal.

The simplest way to start the configuration is to run a command-line tool called \texttt{cmake} passing 
as argument the root directory of \cgal. For example:

{\ccTexHtml{\scriptsize}{}
\begin{verbatim}
  cd <CGALROOT>
  cmake . # Notice the dot to indicate the current directory.
\end{verbatim}
}

This executes a special-purpose \cmake\ script
\footnote{Called \texttt{CMakeLists.txt} and located in the \cgal\ root} 
which attempts to determine the value of all the configuration parameters needed to build the \cgal\
libraries and any programs using \cgal. 

The very first thing \cmake\ does is to detect the compiler to use.
This detection is performed by a special cmake module called a {\em generator}. A
\cmake\ generator understands the build requirements for a particular compiler/linker
 and generates the necessary files for that. For example, the 
{\em UNIX Makefiles} generator understands the GNU toolchain (g++,gcc,ld etc) and produces makefiles
which can be used to build a target by a simple call to \texttt{make}.
Likewise, the {\em Visual Studio 2005} generator produces solution and project files and can be manually
launched in the VS IDE to build the target.

Each platform has a default generator so you only need to select one when the default is
not what you want. 
\footnote{Running \texttt{cmake} with no parameters in a command-line prints the list of available generators supported 
by your platform and \cmake\ version. If the generator you need is not listed there, you can
try a newer \cmake\ version since generators are hardcoded into \cmake\ and more and
more generators are added with each release.}
For example, under Windows, it is possible to generate {\em NMakefiles}
instead of Visual Studio project files in order to build the library via \texttt{nmake}.

Since the choice of generator determines the type of build files to generate, in some cases
you choose a particular generator as a mean to choose a specific compiler (because they use different 
build files. For example, the following generates solution files for use in Visual C++ 9.0:

{\ccTexHtml{\scriptsize}{}
\begin{verbatim}
  cd ~/CGAL-3.4
  cmake -G"Visual Studio 9 2008" . 
\end{verbatim}
}

In other cases, however, the generator doesn't directly identify a specific compiler but a tool chain.
For example, the \texttt{UNIX Makefiles} generator produces \texttt{makefiles} wich call some auto-detected
command-line compiler, like \texttt{gcc}. If you need the makefiles to use a different compiler you need to
say so in the call to cmake, as in this example:

{\ccTexHtml{\scriptsize}{}
\begin{verbatim}
  cd ~/CGAL-3.4
  cmake -DCMAKE_CXX_COMPILER:FILEPATH=g++-3.4 . 
\end{verbatim}
}


\cmake\ mantains configuration parameters in so-called {\em cmake variables}, like the \texttt{CMAKE\_CXX\_COMPILER}
in the example above. These variables {\em are not environment variables} and they must be set and modified in
certain specific ways as explained below. Some of the cmake variables represent user choices, such as
\texttt{WITH\_examples} or \texttt{CMAKE\_BUILD\_TYPE=Release}, while others indicate the details of a third-party library, 
such as \texttt{Boost\_INCLUDE\_DIR} or the compiler flags to use, such as \texttt{CMAKE\_CXX\_FLAGS}. 

The command line tool \texttt{cmake} accepts cmake variables as arguments of the form \texttt{-D<VAR>:<TYPE>=<VALUE>}, as
in the example above, but this is only useful if you already know which variables needs to be explicitely defined.
\cmake\ provides an interactive tool which lets you configure \cgal\ by suppling the missing pieces 
which \cmake\ indicates. In Windows, this is the \texttt{CMakeSetup.exe} program, while in Unix/Linux/Mac it is \texttt{ccmake}
(or \texttt{cmake-gui} if you have a recent cmake version). 

These interactive configuration tools execute the cmake script just like the non-interactive \texttt{cmake} does. It assign
values to all the cmake variables used in the configuration and presents the variables to you. You can then change 
any of the values (such as those representing user choices: \texttt{WITH\_<choice>}) and once they are all OK you can select 
\texttt{Configure} to finish. It will produce the build files just like the non-interactive cmake.

Some variables however, like those corresponding to the path of required third-party dependencies, are critical: they 
must be defined with proper values. If the script fails to locate a required dependency, for example, the corresponding variable,
say \texttt{Boost\_INCLUDE\_DIR}, will have a \texttt{NOTFOUND} value and will show up distinctively (i.e. in red in the GUI cmake).
\cmake\ will refuse to generate the build files until all these variables are given proper values, so you need
to define them and press \texttt{Configure} iteratively until there are no more critical variables left undefined. Only then
pressing \texttt{Configure} will finish the configuration, build the files and return.

\begin{ccAdvanced}
\cmake\ keeps the variables that a user can manipulate in a so-called {\em cmake cache}, a simple text file specifically 
named \texttt{CMakeCache.txt} whose entries are of the form VARIABLE:TYPE=VALUE. Advanced users can just manually edit this 
file instead of going through the interactive configuration session explained before.
\end{ccAdvanced}


The configuration process not only determines the location of the required dependencies, it also dynamically generates a
\texttt{compiler\_config.h} file which encodes the properties of your system and a special file named 
\texttt{CGALConfig.cmake} which is used to build programs using \cgal\ (this file will be explained later).

\subsubsection{\cgal\ components}

For convenience, \cgal\ is physically separated into {\em library components} in order to push some auxiliary 
functionality outside of the main library. During configuration, each component (except the main library) is selected 
via a cmake variable of the form {\tt WITH\_<component>}. All components are {\tt ON} by default.

\begin{center}\index{cgal!components}\index{cgal components}
  \renewcommand{\arraystretch}{1.3}
  \gdef\lcTabularBorder{2}
  \begin{tabular}{|l|l|l|l|} \hline
    \textbf{library} & \textbf{cmake variable} & \textbf{functionality} & \textbf{dependencies}
    \\\hline\hline
    {\tt CGAL} & {\em none} & Main library (all kernels and packages) & GMP, MPFR, Boost (headers) and Boost.Thread (library)
    \\\hline
    {\tt CGAL\_Core} & {\tt WITH\_CGAL\_Core} & The CORE library used for robust numerical and geometric computation.
    \footnote{CGAL-core++ is not part of \cgal, it is a custom version the CORE library ditributed by \cgal\
              for the user convenience and it has it's own license. }
    & GMP and MPFR
    \\\hline
    {\tt CGAL\_Qt3} & {\tt WITH\_CGAL\_Qt3} & \ccc{CGAL::Qt_widget} used by some old Qt3-based demos & Qt3 and OpenGL
    \\\hline
    {\tt CGAL\_Qt4} & {\tt WITH\_CGAL\_Qt4} & \ccc{CGAL::GraphicsView} used by the Qt4-based demos & Qt4 and OpenGL
    \\\hline
    {\tt CGAL\_ImageIO} & {\tt WITH\_CGAL\_ImageIO} & Utilities to read and write image files in various formats 
    & OpenGL and ZLib and VTK\footnote{Only if VTK is available}
    \\\hline
    {\tt CGAL\_PDB} & {\tt WITH\_CGAL\_PDB} & \ccc{CGAL::PDB} used to manipulate {\tt PDB} files & {\em none}
    \\\hline
  \end{tabular}
\end{center}

If you turn off the configuration of a component you can still configure it manually from the component source directory:

{\ccTexHtml{\scriptsize}{}
\begin{verbatim}
  cd ~/CGAL-3.4/src/CGALQt4 
  cmake . # configures only the CGAL_Qt4 component
\end{verbatim}
}

Notice that all kernels and packages are bundled toghether in the main library; the components are only to separate
auxiliary functionality.

\cgal\ is ditributed with a large collection of examples and demos. By default, these are all configured along with
the \cgal\ libraries. When configured, they are not automatically built along with the libraries, you do that
by selecting the {\tt examples} or {\tt demo} make targets as explained in the following section, but you can set the 
variables {\tt WITH\_examples=OFF} and {\tt WITH\_demo=OFF} to prevent configuration of examples and demos alltoghether.

\subsubsection{Library variants}

The cmake variable {\tt CMAKE\_BUILD\_TYPE} can take the values {\tt Release} or {\tt Debug} to indicate how to build
the libraries. The default depends on the platform so it is recommended that you always indicate the build type explicitely.
\footnote{\cmake\ suports additional build types but those are not supported by \cgal\, thus you should choose only
{\tt Release} or {\tt Debug}}

In windows, the \cgal\ libraries are neccesarily static\footnote{There is currently no support for DLLs under Windows}
({\tt .lib}), but in the other platforms, including {\tt cygwin}, they are by default shared libraries ({\tt .so}).
You can choose to produce static libs instead by setting the cmake variable {\tt CMAKE\_SHARED\_LIBS=FALSE} 

These setting affect the variants of third-party libraries selected whenever the choice is available.

\subsubsection{Multiple variants}\label{sec:cmake-out-of-source}

While you can choose between release/debug builds, shared/static libraries\footnote{Except on windows}, etc...
it is not possible to have both variants on a single configuration. To do that you need to run cmake in a 
different directory for each variant, each with its own selection of configuration parameters.

\cmake\ stores the resulting makefiles/projectfiles, along with several temporary and auxiliary files such
as the variables cache, in the directory where it is executed, called {\tt CMAKE\_BINARY\_DIR}, but it
takes the source files and configuration scripts from whatever directory is indicated, called
{\tt CMAKE\_SOURCE\_DIR} 

The binary and source directory need not be the same, thus, you can configure multiple variants by creating a
distinct directory for each one and running cmake from there. This is known in \cmake\ as 
{\em out-of-sources configuration}, as opposite to an {\em in-source configuration} as the one shown in the 
previous short examples.

The \cgal\ directory tree contains directories \cgaldir/cmake/platforms and you can create your 
variant subdirectory in there:

{\ccTexHtml{\scriptsize}{}
\begin{verbatim}
mkdir ~/CGAL-3.4/cmake/platforms/debug
cd ~/CGAL-3.4/cmake/platforms/debug
cmake -DCMAKE_BUILD_TYPE=Debug ../../..

mkdir ~/CGAL-3.4/cmake/platforms/release
cd ~/CGAL-3.4/cmake/platforms/release
cmake -DCMAKE_BUILD_TYPE=Release ../../..
\end{verbatim}
}

\subsubsection{Compiler and Linker flags}

\cmake\ determines the set of compiler and linker flags to use based on the target compiler and automatically defines
the following cmake variables:

\begin{enumerate}
\item {\tt CMAKE\_CXX\_FLAGS}
\item {\tt CMAKE\_CXX\_FLAGS\_RELEASE}
\item {\tt CMAKE\_CXX\_FLAGS\_DEBUG}
\item {\tt CMAKE\_MODULE\_LINKER\_FLAGS}
\item {\tt CMAKE\_MODULE\_LINKER\_FLAGS\_RELEASE}
\item {\tt CMAKE\_MODULE\_LINKER\_FLAGS\_DEBUG}
\item {\tt CMAKE\_SHARED\_LINKER\_FLAGS}
\item {\tt CMAKE\_SHARED\_LINKER\_FLAGS\_RELEASE}
\item {\tt CMAKE\_SHARED\_LINKER\_FLAGS\_DEBUG}
\item {\tt CMAKE\_EXE\_LINKER\_FLAGS}
\item {\tt CMAKE\_EXE\_LINKER\_FLAGS\_RELEASE}
\item {\tt CMAKE\_EXE\_LINKER\_FLAGS\_DEBUG}
\end{enumerate}

The {\tt MODULE} family of flags correspond to static libraries, the {\tt SHARED} flags to shared libraries
and {\tt EXE} flags to executables.

The effective flags are always a concatenation of the general and build-type specific flags, such as
{\tt CMAKE\_CXX\_FLAGS} {\em then} {\tt CMAKE\_CXX\_FLAGS\_RELEASE}.

You can {\em completely override} any of these variables if you pass them on the command-line {\tt cmake}
Alternatively, you can change the values given by \cmake\ using the interactive cmake tools
\footnote{{\tt CMakeSetup.exe, ccmake or cmake-guid}}  This is usefull for example to produce a fast debug
variant without diagnostics (-DNDEBUG added). 

If you do not want to override the automatically configured flags but only add more, you can 
define any of the following cmake variables, which are counterparts of the default
variables but starting with {\tt CGAL} instead of {\tt CMAKE}:

\begin{enumerate}
\item {\tt CGAL\_CXX\_FLAGS}
\item {\tt CGAL\_CXX\_FLAGS\_RELEASE}
\item {\tt CGAL\_CXX\_FLAGS\_DEBUG}
\item {\tt CGAL\_MODULE\_LINKER\_FLAGS}
\item {\tt CGAL\_MODULE\_LINKER\_FLAGS\_RELEASE}
\item {\tt CGAL\_MODULE\_LINKER\_FLAGS\_DEBUG}
\item {\tt CGAL\_SHARED\_LINKER\_FLAGS}
\item {\tt CGAL\_SHARED\_LINKER\_FLAGS\_RELEASE}
\item {\tt CGAL\_SHARED\_LINKER\_FLAGS\_DEBUG}
\item {\tt CGAL\_EXE\_LINKER\_FLAGS}
\item {\tt CGAL\_EXE\_LINKER\_FLAGS\_RELEASE}
\item {\tt CGAL\_EXE\_LINKER\_FLAGS\_DEBUG}
\end{enumerate}


\subsection{Building \cgal\ libraries}

If configuration succeeded there will be certain {\em build files} ready to build the libraries.
The nature of the build files depend on the generator used during configuration, but in all cases they 
will contain several {\em targets}, one per component,  and a default global target corresponding 
to all the library components at once.

For example, in a UNIX-like environment the default generator produces makefiles and you would
use the {\tt make} command-line tool as follow:

{\ccTexHtml{\scriptsize}{}
\begin{verbatim}

cd ~/CGAL-3.4
cmake .

# build all the selected components at once
make 

# build each selected component independently,
# continuing with the next component in case of error
# due to the '-k' switch.

make -k CGAL CGAL_Core CGAL_PDB CGAL_ImageIO CGAL_Qt3 CGAL_Qt4

\end{verbatim}
}

Notice that according to the rules of 'make', an error in the default target will abort compilation/linking of
all the \cgal\ components (which are part of the same single default target) even if {\tt -k} 
is specified, so you need to list the components explicitely to prevent that.

If you have not turned off configuration of examples and/or demos, there will be additional
targets named {\tt examples} and {\tt demo}, plus one target for each example and each demo.
None of these targets are included in the default so you need to build them explicitely
{\em after} the \cgal\ libraries have been successfully compiled and linked
\footnote{The targets {\tt examples} and {\tt demo} include themselves all the targets
for examples and demos respectively.}.

{\ccTexHtml{\scriptsize}{}
\begin{verbatim}

# build all examples at once
make examples 

# build only the Striaght Skeleton demo
make Straight_skeleton_2_demo

\end{verbatim}
}

The libraries will all be located in a subdirectory {\tt /lib} under {\tt <CMAKE\_BINARY\_DIR>}
\footnote{Which need not be the cgal root directory as explained in Section ~\ref{sec:cmake-outofsource}}

With generators other than {\tt UNIX Makefiles} the resulting build files are solution/project files which
should be launched in an IDE, such as Visual Studio or KDevelop3. They will contain the targets described
above which you can manually build as any other solution/project within your IDE.

\begin{ccAdvanced}
The build files produced by cmake are autoconfigured. That is, if you change any 
of the dependencies, the build step automatically goes all the way back to the configure step
during the building. This way, once the target has been configured the very first time by
invoking cmake, you don't necessarily need to invoke \texttt{cmake} again. Rebuilding will call
itself \texttt{cmake} and re-generate the build file whenever needed. Keep this in mind if you
configure \cgal\ for the Visual Studio IDE since a build could then change the solution/project 
file in-place and VS will prompt you to reload it.
\end{ccAdvanced}


\subsection{Installing \cgal\ libraries}

I many platforms, library pieces such as headers, docs and binaries are expected to be placed in specific locations. The
typical example being the Filesystem Hierarchy Standard (see \fhs) of UNIX-like operating systems. The process of placing or copying
the library elements into its standard location is sometimes referred to as {\em Installation} and it is a post step after
configuration and compilation.

\cmake\ supports installation by producing a build target named {\em install}. The following example shows a typical session
from configuration to installation in a UNIX-like environment:

{\ccTexHtml{\scriptsize}{}
\begin{verbatim}

cd ~/CGAL-3.4

cmake . # configure
make    # make

sudo make install # install

\end{verbatim}
}

The installation step copies files to locations wich are typically accessible only to the root user so
it is typically neccessary to do so as a super-user, hence the call to {\tt 'sudo'} first.

If you used a generator which produced IDE files (for Visual Studio for instance) there will be an optional
{\tt INSTALL} project which you can {\em ``build''} to execute the installation step. 

The files are copied into a directory tree relative to the {\em installation directory} determined by the 
cmake variable {\tt CMAKE\_INSTALL\_PREFIX}. This variable defaults to {\tt /usr/local} under UNIX-like operating systems
and {\tt PROGRAMFILES } under Windows. If you want to install to a different location you must override that cmake
variable explicitely {\em at configuration time} (not when executing the install step).

If you wish the be able to specify the install location when executing the install step rather than fixing it during
configuration you can use the environment variable {\tt DESTDIR}. However, since {\tt CMAKE\_INSTALL\_PREFIX} is prepended
to the value of {\tt DESTDIR} (hence its name), in practice you need to define it as an empty string in order to gain complete
control via {\tt DESTDIR}.
This is illustrated in the following exampple:

{\ccTexHtml{\scriptsize}{}
\begin{verbatim}

cd ~/CGAL-3.4

cmake -DCMAKE_INSTALL_PREFIX="" . 

make

DESTDIR=/usr/share sudo make install

\end{verbatim}
}


If for whatever reason you wish the configuration to avoid creating the install target you can define the cmake variable
 {\tt WITH\_INSTALL=FALSE}.

\subsection{Packaging \cgal}

\cmake\ version 2.4.2 is distributed with a packaging tool called \cpack. You can use it to generate an archive containing
\cgal\ for simple distribution along multiple similar systems. To do that you must define the cmake variable
{\tt WITH\_CPACK=ON} when configuring \cgal.
Under Windows this will generate a \nsis\ installer (see \nsispage).
Under UNIX-like environments, the configuration step will create targets named {\tt package} and {\tt package\_source}. When built, they would generate 
{\tt .tar.gz} archives containing the \cgal\ libraries in binary and source form, respectively.

Further details about the packaging tools can be found at \cpackpage.

\section{Configuring and Building Programs Using \cgal}

\cmake can be used to configure and build user programs as well via a \cmake script ({\tt CMakeLists.txt}) for the program.
All the examples and demos contain such \cmake scripts.

During configuration of the \cgal\ libraries a file named {\tt CGALConfig.cmake} will be generated in the binary directory. This file contains
the definitions of several cmake variable which summarize the configuration of \cgal. In order to configure a user program you need to indicate
the location of that config file in the cmake variable {\tt CGAL\_DIR}:.

{\ccTexHtml{\scriptsize}{}
\begin{verbatim}

# sources are in /CGAL-3.4 but the compiled cgal is in /my_cgal_bin
cd ~/my_cgal_bin
cmake ~/CGAL-3.4
make

cd ~/my_cgal_program
cmake -DCGAL_DIR=~/my_cgal_bin . 
make

\end{verbatim}
}

By default, \cgal\ programs inherit the flags and buil-type used
for \cgal\ to avoid runtime problems, unless you specifically indicate otherwise by
defining the cmake variable {\tt CGAL\_DONT\_OVERRIDE\_DEFAULT\_SETTINGS=TRUE}.
The following example illustrates this:

{\ccTexHtml{\scriptsize}{}
\begin{verbatim}

cd ~/CGAL-3.4

# CGAL configured in release mode but with debug information 
cmake -DCMAKE_BUILD_TYPE=Release -DCMAKE_CXX_FLAGS=-g .

cd ~/my_cgal_program

# In this case, the settings of build type and flags HAVE NO EFFECT
# because they are overriden by the values used above
cmake -DCGAL_DIR=~/CGAL-3.4 -DCMAKE_BUILD_TYPE=Debug -DCMAKE_CXX_FLAGS=-O2 . 

# But in this case they are respected
cmake -DCGAL_DIR=~/CGAL-3.4 -DCMAKE_BUILD_TYPE=Debug -DCMAKE_CXX_FLAGS=-O2 -DCGAL_DONT_OVERRIDE_DEFAULT_SETTINGS=TRUE . 

\end{verbatim}
}



\section{Example} 

Here is an example output on a Windows machine with VC8 installed, using \cmake\ 2.6,
and the following command-line call to {\tt cmake}:

{\ccTexHtml{\scriptsize}{}
\begin{verbatim}
cmake -DWITH_examples=false 
      -DWITH_demo=false 
      -DCMAKE_BUILD_TYPE=Release 
      ../../..
\end{verbatim}
}


{\ccTexHtml{\scriptsize}{}
\begin{verbatim}
-- Check for working C compiler: cl
-- Check for working C compiler: cl -- works
-- Detecting C compiler ABI info
-- Detecting C compiler ABI info - done
-- Check for working CXX compiler: cl
-- Check for working CXX compiler: cl -- works
-- Detecting CXX compiler ABI info
-- Detecting CXX compiler ABI info - done
-- CGAL_REFERENCE_CACHE_DIR=
-- Building static libraries
-- Targetting Visual Studio 8 2005
-- Target build enviroment supports auto-linking
-- Using VC80 compiler.
-- Build type: Release
-- Generator uses intermediate configuration directory: $(OutDir)
-- CMake version: 2.6.1
-- System: Windows
-- CGAL_MAJOR_VERSION=3
-- CGAL_MINOR_VERSION=4
-- CGAL_BUILD_VERSION=442
-- CGAL_SONAME_VERSION=3
-- CGAL_SOVERSION     =3.0.0
-- Boost version: 1.35.0
-- Found the following Boost libraries:
--   thread
-- Boost include:      C:/Program Files/boost/boost_1_35_0
-- Boost libraries:    
-- Boost definitions:  
-- USING BOOST_VERSION = '1.35.0'
-- Could NOT find GMP
-- Could NOT find MPFR
-- Performing Test CGAL_CFG_DENORMALS_COMPILE_BUG - Success
-- Performing Test CGAL_CFG_IEEE_754_BUG - Success
-- Performing Test CGAL_CFG_ISTREAM_INT_BUG - Success
-- Performing Test CGAL_CFG_LONGNAME_BUG - Failed
-- Performing Test CGAL_CFG_MATCHING_BUG_5 - Success
-- Performing Test CGAL_CFG_MATCHING_BUG_6 - Failed
-- Performing Test CGAL_CFG_NESTED_CLASS_FRIEND_DECLARATION_BUG - Failed
-- Performing Test CGAL_CFG_NET2003_MATCHING_BUG - Failed
-- Performing Test CGAL_CFG_NO_CPP0X_ARRAY - Failed
-- Performing Test CGAL_CFG_NO_CPP0X_DECLTYPE - Failed
-- Performing Test CGAL_CFG_NO_CPP0X_DEFAULT_TEMPLATE_ARGUMENTS_FOR_FUNCTION_TEMPLATES - Failed
-- Performing Test CGAL_CFG_NO_CPP0X_DELEGATING_CONSTRUCTORS - Failed
-- Performing Test CGAL_CFG_NO_CPP0X_ISFINITE - Failed
-- Performing Test CGAL_CFG_NO_CPP0X_LONG_LONG - Success
-- Performing Test CGAL_CFG_NO_CPP0X_RVALUE_REFERENCE - Failed
-- Performing Test CGAL_CFG_NO_CPP0X_VARIADIC_TEMPLATES - Failed
-- Performing Test CGAL_CFG_NO_LIMITS - Success
-- Performing Test CGAL_CFG_NO_NEXTAFTER - Failed
-- Performing Test CGAL_CFG_NO_STATEMENT_EXPRESSIONS - Failed
-- Performing Test CGAL_CFG_NO_STL - Success
-- Performing Test CGAL_CFG_NO_TMPL_IN_TMPL_PARAM - Success
-- Performing Test CGAL_CFG_NO_TR1_ARRAY - Failed
-- Performing Test CGAL_CFG_NUMERIC_LIMITS_BUG - Success
-- Performing Test CGAL_CFG_OUTOFLINE_MEMBER_DEFINITION_BUG - Success
-- Performing Test CGAL_CFG_TYPENAME_BEFORE_DEFAULT_ARGUMENT_BUG - Failed
-- Performing Test CGAL_CFG_USING_BASE_MEMBER_BUG_2 - Success
-- CMAKE_INSTALL_PREFIX=C:/Program Files/CGAL
-- USING CUSTOM_CXXFLAGS = ' -D_CRT_SECURE_NO_DEPRECATE -D_SCL_SECURE_NO_DEPRECATE -D_CRT_SECURE_NO_WARNINGS -D_SCL_SECURE_NO_WARNINGS   -D_SECURE_SCL=0'
-- USING        CXXFLAGS = ' /DWIN32 /D_WINDOWS /W3 /Zm1000 /EHsc /GR /MD /O2 /Ob2 /D NDEBUG'
-- USING CUSTOM_LDFLAGS = '  '
-- USING        LDFLAGS = ' '
-- USING CUSTOM_EXEFLAGS = '  '
-- USING        EXEFLAGS = ' /MANIFEST /STACK:10000000 /machine:I386 /INCREMENTAL:NO'
-- Configuring CGALCore. Set WITH_CGALCore to FALSE to unselect it.
-- CGAL-core++ needs GMP and MPFR, cannot be configured.
-- Configuring CGALimageIO. Set WITH_CGALimageIO to FALSE to unselect it.
-- Could NOT find ZLIB
-- CGAL-ImageIO needs ZLib and OpenGL, cannot be configured.
-- Configuring CGALPDB. Set WITH_CGALPDB to FALSE to unselect it.
-- Configuring CGALQt. Set WITH_CGALQt to FALSE to unselect it.
-- CGAL-Qt3 needs Qt3, cannot be configured.
-- Configuring CGALQt4. Set WITH_CGALQt4 to FALSE to unselect it.
-- Looking for Q_WS_X11
-- Looking for Q_WS_X11 - not found.
-- Looking for Q_WS_WIN
-- Looking for Q_WS_WIN - found
-- Looking for Q_WS_QWS
-- Looking for Q_WS_QWS - not found.
-- Looking for Q_WS_MAC
-- Looking for Q_WS_MAC - not found.
-- OpenGL include:     
-- OpenGL libraries:   glu32;opengl32
-- OpenGL definitions: 
-- Qt4 include:        C:/Work/Downloaded/Libraries/qt-win-opensource-src-4.4.1/include
-- Qt4 libraries:      optimized;C:/Work/Downloaded/Libraries/qt-win-opensource-src-4.4.1/lib/qtmain.lib;debug;C:/Work/Downloaded/Libraries/qt-win-opensource-src-4.4.1/lib/qtmaind.lib;optimized;C:/Work/Downloaded/Libraries/qt-win-opensource-src-4.4.1/lib/QtOpenGL4.lib;debug;C:/Work/Downloaded/Libraries/qt-win-opensource-src-4.4.1/lib/QtOpenGLd4.lib;opengl32.lib glu32.lib gdi32.lib user32.lib;optimized;C:/Work/Downloaded/Libraries/qt-win-opensource-src-4.4.1/lib/QtGui4.lib;debug;C:/Work/Downloaded/Libraries/qt-win-opensource-src-4.4.1/lib/QtGuid4.lib;imm32;winmm;optimized;C:/Work/Downloaded/Libraries/qt-win-opensource-src-4.4.1/lib/QtCore4.lib;debug;C:/Work/Downloaded/Libraries/qt-win-opensource-src-4.4.1/lib/QtCored4.lib;ws2_32
-- Qt4 definitions:    -DQT_DLL
-- moc executable:     C:/Work/Downloaded/Libraries/qt-win-opensource-src-4.4.1/bin/moc.exe
-- uic executable:     C:/Work/Downloaded/Libraries/qt-win-opensource-src-4.4.1/bin/uic.exe
-- Configuring done
-- Generating done
-- Build files have been written to: C:/Work/Active/GeometryFactory/CGAL/Autotest/CGAL-3.4-I-442/cmake/platforms/a
\end{verbatim}
}


%%
%% EOF
%%


