%%
%%  *** CGAL Installation Guide ***
%%
%% file:     installation.tex
%%
%% authors:  Michael Hoffmann, Dima Pasechnik and Wieger Wesselink
%%
%% $Id$
%%

%% macro for GNU
\newcommand{\gnu}{\textsc{Gnu}}

%% macro for QT
\newcommand{\qt}{\textsc{Qt}}

%% macro for GMP
\newcommand{\gmp}{\textsc{Gmp}}

%% macro for Boost
\newcommand{\boost}{\textsc{Boost}}

%% macro for Core
\newcommand{\core}{\textsc{Core}}

%% macro for Taucs
\newcommand{\taucs}{\textsc{Taucs}}

%% macro for Windows-specific installation
\newcommand{\MSInst}{Windows-specific Installation}

%% macro for g++
\newcommand{\Gcc}[1]{\gnu~\texttt{g++}~{\rm #1}}
\newcommand{\gccurl}{\path'http://gcc.gnu.org/'}

%% macro for SUNPRO CC
\newcommand{\sunprocc}[1]{\textsc{Sunpro}~\texttt{CC}~{\rm #1}}
\newcommand{\sunproccurl}{\path'http://www.sun.com/software/Developer-products/cplusplus/'}

%% macro for SGI CC
\newcommand{\mipsprocc}{\textsc{SGI}~Mips(Pro)~\texttt{CC}}
\newcommand{\mipsprourl}{\path'http://www.sgi.com/software/irix/tools/c++.html'}

%% macro for Borland C++
\newcommand{\bcc}[1]{\textsc{Borland}~\texttt{C++}~{\rm #1}}
\newcommand{\bccurl}{\path'http://www.inprise.com/bcppbuilder/'}

%% macro for Microsoft Visual C++
\newcommand{\msvc}[1]{\textsc{MS}~Visual~\texttt{C++}~{\rm #1}}
\newcommand{\msvcurl}{\path'http://msdn.microsoft.com/visualc/'}

%% macro for Intel C++ Compiler
\newcommand{\icl}[1]{\textsc{Intel}~\texttt{C++}~{\rm #1}}
\newcommand{\iclurl}{\path'http://developer.intel.com/software/products/compilers/'}

%% macro for Microsoft Windows
\newcommand{\mswin}{\textsc{MS}~Windows}

%% macro for CGAL release number
\newcommand{\cgalrelease}{3.2}

%% macro for CGAL directory
\newcommand{\cgaldir}{\texttt{CGAL-\cgalrelease}}

%% macro for YOUR CGAL directory :)
\newcommand{\yourcgaldir}{$<$\textit{insert your \cgaldir\ dir}$>$}

%% macro for CGAL install config directory
\newcommand{\cgalinstconfdir}{\cgaldir\texttt{/config/install}\index{directories!config/install@\texttt{config/install}}}

%% macros for WWW pages
\newcommand{\cgalhomepage}{\path'http://www.cgal.org'}
\newcommand{\gmppage}{\path'http://www.swox.com/gmp/'}
\newcommand{\boostpage}{\path'http://www.boost.org/'}
\newcommand{\corepage}{\path'http://www.cs.nyu.edu/exact/core\_pages/'}
\newcommand{\ledapage}{\path'http://www.mpi-sb.mpg.de/LEDA'}
\newcommand{\trolltechpage}{\path'http://www.trolltech.com'}
\newcommand{\qtpage}{\path'http://doc.trolltech.com/'}
\newcommand{\qtmocpage}{\path'http://doc.trolltech.com/moc.html'}
\newcommand{\taucspage}{\path'http://www.tau.ac.il/~stoledo/taucs/'}

%% write index in typewriter type. (c) Susan :-)
\newcommand{\TTindex}[1]{\index{#1@{\tt #1}}}
\newcommand{\TTsubindex}[2]{\index{#1@{\tt #1}!{#2}}}
\newcommand{\TTsubindextwo}[2]{\index{#1!#2@{\tt #2} }}

\section{Introduction}

\cgal\ stands for \textit{Computational Geometry Algorithms Library}.
It is a software library written in \CC, whose development started in
an {\sc Esprit Ltr} project. The goal of \cgal\ is to make the large
body of geometric algorithms developed in the field of computational
geometry available for industrial application.

This document describes how to install \cgal\ on Unix-like systems.
Besides that, you will find some information about the makefile
structure of \cgal\ and the support for using \cgal\ together with
other software libraries, such as the \boost\
libraries\footnote{\boostpage}, the \gnu\ Multiple Precision library
\gmp\footnote{\gmppage}, the \core\ library\footnote{\corepage} for
robust numerical and geometric computation, \leda, the Library of
Efficient Datatypes and Algorithms\footnote{\ledapage}, \taucs, a
library of sparse linear solvers\footnote{\taucspage}, or
Trolltech's\footnote{\trolltechpage} \qt\ toolkit.

\section{Prerequisites}\label{sec:prerequisites}

In order to build the \cgal\ libraries you need a \CC\ compiler.  Most
recent compilers on Unix platforms and \mswin\ are supported, provided
that they reasonably conform to the ISO 14882 standard for \CC.

\cgaldir\ supports the following compilers/operating systems:

\begin{center}\index{compilers!supported}\index{supported compilers}
  \renewcommand{\arraystretch}{1.3}
  \gdef\lcTabularBorder{2}
  \begin{tabular}{|l|l|} \hline
    \textbf{compiler}        & \textbf{operating system}\\\hline\hline
    %%\mipsprocc\ 7.4 (n32 and 64) \footnotemark[7] & IRIX 6.5\\\hline
    \Gcc{3.3.3, 3.4, 4.0, 4.1} \footnotemark[9]            
    & IRIX 6.5 / Solaris 2.6+ / Linux 2.x / MacOS X /
      \\ & \mswin\ 95/98/2000/XP/NT4\footnotemark[10]\\\hline
    %%\sunprocc{5.5} \footnotemark[11] & Solaris 2.6+\\\hline
    \msvc{7.1} (\textsc{.NET}) \footnotemark[12]           
    & \mswin\ 95/98/2000/XP/NT4\footnotemark[10]\\\hline
    \icl{8.1} \footnotemark[13]                          
    & \mswin\ 95/98/2000/XP/NT4\footnotemark[10]\\\hline
%%    \bcc{5.5.1} \footnotemark[13]                          
%%    & \mswin\ 95/98/2000/XP/NT4\footnotemark[10]\\\hline
  \end{tabular}
\end{center}
%%\footnotetext[7]{\mipsprourl}\addtocounter{footnote}{1}
\footnotetext[9]{\gccurl}\addtocounter{footnote}{1}
\footnotetext[10]{with Cygwin
  (\path~http://www.cygwin.com~)}\addtocounter{footnote}{1}
%%\footnotetext[11]{\sunproccurl}\addtocounter{footnote}{1}
\footnotetext[12]{\msvcurl}\addtocounter{footnote}{1}
\footnotetext[13]{\iclurl}\addtocounter{footnote}{1}
%%\footnotetext[13]{\bccurl}\addtocounter{footnote}{1}
\addtocounter{footnote}{1}

Moreover, \cgal\ requires a working installation of the \boost\
libraries (at least, the header files) Version 1.32 or later. In case
the \boost\ libraries are not installed on your system already, you
can obtain them from \boostpage.

If you are going to install \cgal\ using
Cygwin\footnote{\path~http://www.cygwin.com~}, please read
Section~\ref{sec:cygwin} first.

Note that neither prerelease versions nor repository snapshots of
\texttt{GCC} are supported.

\section{Getting \cgal} \label{sec:gettingcgal}
\index{CGAL@\cgal!getting}\index{getting \cgal} 

The \cgal\ library can be downloaded from the \cgal\ 
homepage:\index{CGAL@\cgal!homepage}
\begin{quote}
      \cgalhomepage
\end{quote}
and go to the `Download' section. Just follow the instructions on this
page to obtain your copy of the library.

After you have downloaded the file containing the \cgal\ library, you
have to decompress it. Use the commands

\begin{verbatim}
      gunzip <filename>.tar.gz
      tar xvf <filename>.tar
\end{verbatim}

Alternatively, your browser might be able to invoke the right
decompression program by itself.

In both cases the directory \cgaldir\ will be created. This directory
contains the following subdirectories:\index{directories!structure}

\begin{center}
  \renewcommand{\arraystretch}{1.3}
  \gdef\lcTabularBorder{2}
  \begin{tabular}{|l|l|} \hline
    \textbf{directory} & \textbf{contents}\\\hline\hline
    \texttt{auxiliary} & packages that can optionally be used with \cgal\\\hline
    \texttt{config}    & configuration files for install script\\\hline
    \texttt{demo}      & demo programs (some of them need \leda, geomview
                         or other third-party products)\\\hline
    \texttt{doc\_html} & documentation (HTML)\\\hline
    \texttt{doc\_pdf}  & documentation (PDF)\\\hline
    \texttt{doc\_ps}   & documentation (Postscript)\\\hline
    \texttt{examples}  & example programs\\\hline
    \texttt{include}   & header files\\\hline
    \texttt{lib}       & (shared) object libraries\\\hline
    \texttt{make}      & files with platform dependent makefile settings\\\hline
    \texttt{scripts}   & some useful scripts (e.g. for creating makefiles)\\\hline
    \texttt{src}       & source files\\\hline
  \end{tabular}
\end{center}

The directory \texttt{src/Core} contains a distribution of the \core\
library\footnote{\corepage} for robust numerical and geometric
computation. \core\ is not part of \cgal\ and has its own license.

\subsection{Visualization}\label{sec:vis}

The programs in the \texttt{demo} directory provide visual output.
Most of these use \ccc{CGAL::Qt_widget}, a widget and some helper
classes that allow to interact with two dimensional \cgal\ objects in
\qt~3 based applications.\index{visualization!Qt} There is no support
for \qt~4 in \cgal\ yet.

If you have \leda\ installed, you might want to use
\ccc{CGAL::Window_stream} as an interface between two dimensional
\cgal\ objects and a \ccc{leda_window}. To be able to use the
\ccc{Window_stream}, you need do nothing more than compile \cgal\ with
\leda\ support.\index{visualization!LEDA}

Some demo programs for 3D structures require the geomview program for
visualization.\index{visualization!geomview} This is available from
\path'http://www.geomview.org' (note that it does not run on \mswin).

\section{Installing \cgal}

The directory \cgaldir\ contains a Bourne shell script called
\texttt{install\_cgal}\TTindex{install\_cgal}
\index{scripts!\texttt{install\_cgal}}. The script can be run in two
modes: a menu-driven interactive mode and a non-interactive mode.
Normally you should use the interactive mode, but in case you run into
problems with it or do not like it for some reason, you can still use
the non-interactive mode.

We first describe a sample installation in section
\ref{sec:sample-inst}. This provides you with an overview on how the
interactive installation works. If you want more detailed information
about specific menus and their options, take a look at section
\ref{sec:interactive-mode}. Finally, for the non-interactive mode
refer to section \ref{sec:non-interactive}.

\section{A Sample Installation}\label{sec:sample-inst}

In this section we sketch an example installation on a Linux machine
with the \Gcc{3.4.5} compiler. For a complete description of the
different menus and their options refer to section
\ref{sec:interactive-mode}.

\subsection{Starting the script}

Go to the \cgaldir\ directory and enter the command
\begin{verbatim}
      ./install_cgal -i
\end{verbatim}

The script first parses support options for various third party
libraries that can be use together with \cgal. Then you get a message
indicating the \cgal\ version you are going to install and that you
are running the interactive mode. It takes some time while the script
locates a number of utility programs. You will not get informed about
this\footnote{If you are that curious what happens exactly, have a
  look at the file
  \texttt{\cgaldir/install.log}.\TTindex{install.log}}, but see some
dots written to the screen indicating progress.

{\ccTexHtml{\scriptsize}{}
\begin{alltt}
Parsing support specfiles for 3rd party libraries:
BOOST, BOOSTPROGRAMOPTIONS, EXTERNTAUCSNIX, EXTERNTAUCSMS, TAUCSNIX,
TAUCSMS, X11, GMP, GMPXX, MPFR, CORE, CGALCORE, ZLIB, ZLIBMS, LEDA, 
LEDAWIN, LEDAMS, LEDAWINMS, QT3MT, QT3ST, QT3MSMT, QT3MSST, done.

--------------------------------------------------------
  This is the install script for CGAL \cgalrelease
--------------------------------------------------------

starting interactive mode - one moment, please
.
--------------------------------------------------------
Settings from Command Line/Environment:
--------------------------------------------------------

  Choosing compiler GNU 3.4.5.
\end{alltt}}
  
The script searches for \CC\ compilers that are installed on your
system and accessible through your \texttt{PATH} environment variable.
If more than one compiler (that is supported by \cgal) is found on
your system, you are presented a list of compilers to choose from. In
this example, only one compiler, \Gcc{3.4.5}, was found on the system
and the script selects this compiler without any user interaction. In
general, the selection of which compiler to use can be made from the
compiler menu (cf. Section~\ref{sec:compiler-menu}). 

The script interprets the environment variables \texttt{CXX},
\texttt{CXXFLAGS}, and \texttt{LDFLAGS} and accordingly sets the
compiler, compiler flags, and linker flags, respectively.  Therefore,
another way to select the compiler is to set the environment variable
\texttt{CXX} to the compiler to be selected.

A menu similar to the following will appear on your screen.\index{main
  menu}\index{menus!main}

{\ccTexHtml{\scriptsize}{} \label{pic:main-menu}
\begin{alltt}
  ****************************************************************
  **             CGAL 3.2 Installation Main Menu                **
  **             ===============================                **
  **                                                            **
  **   OS:          i686_Linux-2.6                              **
  **   Compiler:    GNU 3.4.5                                   **
  **   Support for: no other library.                           **
  **                                                            **
  **   Compiler is supported by CGAL.                           **
  **   The setup has not been tested.                           **
  **                                                            **
  **   There are no libs for this os/compiler.                  **
  **                                                            **
  **   <C>  Compiler Menu                                       **
  **   <S>  Support Menu                                        **
  **   <T>  Test (and save) setup                               **
  **   <A>  Run all setup tests (no cache)                      **
  **                                                            **
  **   <B>  Build CGAL Libraries                                **
  **   <I>  Install CGAL Libraries                              **
  **                                                            **
  **   <Q>  Back to OS                                          **
  **                                                            **
  **   Your Choice:                                             **
  **                                                            **
  ****************************************************************
\end{alltt}}

The first lines below the headline contain some kind of status report:
current OS and compiler, and which third-party software libraries are
supported (such as \gmp, \core, \leda, or \qt). Moreover you can see
that the current setup has not yet been tested, and that there do not
exist \cgal\ libraries for this OS/compiler combination in the \cgal\ 
lib directory by now.

\subsection{Testing the setup}\label{sec:test-the-setup}

As a first step, you should test the current setup by typing ``{\tt
  t}''. Then a number of tests are done to check whether your compiler
supports certain language constructs or has specific bugs. The script
also tries to locate other libraries that are installed on your system
and can be used together with \cgal. There is quite a number of these
tests, so this step may take a while.  For each test you should get a
message what particularly is tested at the moment and what the result
is. More details about what failed

{\ccTexHtml{\scriptsize}{}
\begin{verbatim}
  ****************************************************************
  **   The following lines show results of configuration tests. **
  **   Some of the tests might fail, since many compilers are   **
  **   still not completely ISO standard compliant.             **
  **   Since we worked around the arising problems,             **
  **                *** CGAL will work fine ***                 **
  **   regardless of the outcome of these tests.                **
  ****************************************************************
  Checking for standard header files
  All headers at once : ok.
  Testing for CCTYPE_MACRO_BUG ... ok.

  <many lines omitted>

  Saving current setup ... done.
\end{verbatim}
}

After all these tests are completed, the current settings are saved
into a file that resides in the directory \cgalinstconfdir.  Thus, if
you run the install script a second time for this OS/compiler, you
will not have to go through the whole config-/test cycle again, but
the configuration will be retrieved from the corresponding config file
instead. In case you want to re-run all tests, pick the option
\texttt{<A> Run all setup tests (no cache)} from the main menu.

Looking at the menu header field ``\texttt{Support for}'', you can see
whether the setup has been tested successfully and which other
libraries have been found on your system. The install script
automatically enables \cgal\ support for any library that has been
found.

{\ccTexHtml{\scriptsize}{}
\begin{verbatim}
  ****************************************************************
  **           CGAL 3.2-I-432 Installation Main Menu            **
  **           =====================================            **
  **                                                            **
  **   OS:          i686_Linux-2.6                              **
  **   Compiler:    GNU 3.4.5                                   **
  **   Support for: BOOST, BOOST_PROGRAM_OPTIONS, X11, GMP,     **
  **                MPFR, GMPXX, CORE, ZLIB, and QT.            **
  **                                                            **
  **   Compiler is supported by CGAL.                           **
  **   The setup has been tested ok.                            **

  <some lines omitted>

  ****************************************************************
\end{verbatim}
}

In case you are not happy with the set of libraries for which support
is configured, you can change the settings for each library
individually from the support menu (cf.
Section~\ref{sec:support-menu}). In case you wonder why a certain
library was not automatically configured correctly, you will find more
details in the installation logfile \texttt{\cgaldir/install.log}.

\subsection{Building the \cgal\ libraries}\label{sec:build-the-libs}

We are now ready to build the \cgal\ libraries.  Just type ``{\tt b}''
to start compilation. Building consists of several steps:
%%
\begin{enumerate}
\item writing the include makefile,
\item compiling the static libraries, 
\item \textit{(on some systems)} compiling the shared libraries,
\item \textit{(if \qt\ support is configured)} compiling the CGALQt
  library, \textit{and}
\item \textit{(if \textsc{CgalCore} support is configured)} compiling
  the \core\ library.
\end{enumerate}
%%
The include makefile\index{include makefile} encapsulates the OS-- and
compiler-specific settings and should be included (hence the name) in
all makefiles that compile \cgal\ applications. If everything went ok,
the output should look as follows. (Otherwise, you should have a look
at the error messages from compiler or linker.)

{\ccTexHtml{\scriptsize}{}
\begin{verbatim}
  ****************************************************************
  **                                                            **
  **                     Compiling CGAL 3.2                     **
  **                     ==================                     **
  **                                                            **
  ****************************************************************

  OS:                           i686_Linux-2.6
  COMPILER:                     GNU 3.4.5
  BOOST:                        supported
  BOOST_PROGRAM_OPTIONS:        supported
  EXTERNTAUCS:                  not supported
  TAUCS:                        not supported
  X11:                          supported
  GMP:                          supported
  MPFR:                         supported
  GMPXX:                        supported
  CORE:                         supported
  ZLIB:                         supported
  LEDA:                         not supported
  LEDAWIN:                      not supported
  QT:                           supported

  Generating Makefiles ... done.
  Building CGAL_lib ... done.
  Building CGAL_sharedlib ... done.
  Building CGAL_Qt ... done.
  Building CGAL_Core ... done.

  ****************************************************************
  **             Please press <ENTER> to continue.              **
  ****************************************************************
\end{verbatim}}
{
}  

That's all, it's done. Press ``\texttt{<ENTER>}'' to return to the
main menu and proceed by installing for a different compiler (go to
the compiler menu and choose ``\texttt{c}'' to get a list of supported
compilers detected on your system). 

Another option is to simply quit the install script by typing
``\texttt{q}''. When leaving the script, you get a list of successful
builds during the session.  Furthermore, the script prints the setting
of \texttt{CGAL\_MAKEFILE} for the last active configuration. Remember
to set this environment variable before compiling \cgal\ applications.
On bourne shell derivatives, you would type in our example
\begin{alltt}
      export CGAL_MAKEFILE=\cgaldir/make/makefile_sparc_SunOS-5.6_g++-3.4.3
\end{alltt}
while for \texttt{csh} descendants the syntax is
\begin{alltt}
      setenv CGAL_MAKEFILE \cgaldir/make/makefile_sparc_SunOS-5.6_g++-3.4.3
\end{alltt}
In Section~\ref{sec:makefiles} you can find more information on the
\cgal\ makefile structure, and how to set \texttt{CGAL\_MAKEFILE} when
using \cgal\ on several platforms.

%% TODO: some remark regarding visualization
%% If you want to use the visualization provided by
%% \ccc{CGAL::Window_stream} (Section~\ref{sec:vis}) and you do not have
%% the \leda\ libraries installed, or do not want to use them with \cgal,
%% you should now proceed to Section~\ref{sec:cgalwin} and install the
%% \texttt{CGALWin} library.

%% Now it would be a good idea to print and read the document ``Getting
%% Started with \cgal'' that can be found in various formats in the
%% \texttt{doc\_html}, \texttt{doc\_pdf} and \texttt{doc\_ps}
%% directories.

\section{The interactive mode}\label{sec:interactive-mode}
\index{interactive installation}\index{installation!interactive}
\TTsubindex{install\_cgal}{interactive mode}

To run the install script in the interactive mode, go to the \cgaldir\ 
directory and enter the command
\begin{verbatim}
      ./install_cgal -i
\end{verbatim}

After initialization during which certain utility programs are located
and your system is searched for compilers supported by \cgal, you get
into the \cgal\ installation \textit{main menu} (see page
\pageref{pic:main-menu} for a picture).

From the main menu you can reach a number of different sub-menus, of
which the most important maybe is the \textit{compiler menu}. This is
where you can choose the compiler you want to work with and set custom
compiler or linker options. The compiler menu is described in
Section~\ref{sec:compiler-menu}.

If you want to use other libraries such as \boost, \taucs, \gmp,
\core, \leda, or \qt\ with \cgal, you can setup support for these
libraries from the support menu that is described in
Section~\ref{sec:support-menu}.

Finally, you can build the \cgal\ libraries by typing \texttt{b}.
However, it is recommended to run the \textit{setup test} -- which is
available in all menus as option \texttt{t} -- before. The setup test
includes tests for certain compiler features or bugs as well as tests
for the presence of other libraries that can be used in conjunction
with \cgal. The install script tries to find and enable support for as
many other libraries as possible.

The install script keeps track of the tests passed and only tests
again, if you change the setup in a way that may affect the test
result. If you want to redo \textit{all} tests, you have to choose
option ``\texttt{a}'' from the main menu.

\subsection{Files created during installation}\label{sec:filescreated}

The install script stores all relevant settings for an OS/compiler
combination in the directory
\begin{center}
  \texttt{\cgalinstconfdir/$<$\textit{CGAL-OS-description}$>$}
\end{center}
where $<$\textit{CGAL-OS-description}$>$ identifies your OS/compiler
combination in a way specified in section \ref{sec:os-compiler-id}.
This saves you the work of configuring everything again, if you
upgrade \index{upgrading \cgal}\index{CGAL@\cgal!upgrade} \cgal\ or
another package that makes recompiling the \cgal\ libraries necessary.

Besides the config files, the install script uses several temporary
files during interactive installation. Most of them are removed after
use, but some are not, since it might be helpful to keep some
information about the last run.  You can keep or delete them as you
like, as they are not needed anymore once the script terminated.  A
list of these files (all are plain ASCII and reside in \cgaldir)
follows.
\begin{center}\index{files!temporary}\index{logfiles}
  \renewcommand{\arraystretch}{1.3}
  \gdef\lcTabularBorder{2}
  \begin{tabular}{|l|l|} \hline
    \textbf{filename} & \textbf{content}\\\hline\hline
    \texttt{install.log}\TTindex{install.log} & 
    detailed overall protocol\\\hline
    \texttt{install.completed}\TTindex{install.completed} &
    list of systems for which \cgal\ libraries 
    have been built\\\hline
    \texttt{compile.log}\TTindex{compile.log} & 
    output of the last compiler call\\\hline
  \end{tabular}
\end{center}

\subsection{The Compiler Menu}\label{sec:compiler-menu}
\index{compiler menu}\index{menus!compiler}

Here is the place to set the compiler specific options, such as the
compiler to use (if more than one has been detected), custom
compiler or linker flags, or deciding whether to build shared libraries
or not.\bigskip

\fbox{
\begin{minipage}{.95\linewidth}
\vspace{2pt}
\centerline{{\large{\bf Compiler Menu}}}
\vspace{2pt}
\begin{description}
\item[$<$C$>$] Choose the compiler to be used from the list of
  detected compilers.  You can also register other compilers, if they
  have not been detected automatically.\index{compilers!choosing}
\item[$<$F$>$] Set custom compiler flags. These are the first flags
  given to the compiler in every call. Under normal circumstances
  there should be no need to set any such
  flag.\index{compilers!setting custom flags}
\item[$<$L$>$] Set custom linker flags. These are the first flags
  given to the linker in every call. Under normal circumstances
  there should be no need to set any such flag.
\item[$<$S$>$] Toggle shared libraries building.
  By default, shared libraries are built (e.g. \texttt{libCGAL.so}),
  but it is possible to only build static libraries using this option.
\item[$<$O$>$] Set custom ostype tag.  Set a tag that is mangled into
  the CGAL-OS-description. This is useful to distinguish \cgal\
  installations with different settings for the same OS-compiler
  combination.
\end{description}
\vspace{2pt}
\end{minipage}
\hfill}

\subsection{The Support Menu}\label{sec:support-menu}
\index{support menu}\index{menus!support}

This menu provides the starting point to setup the support for
third-party software libraries such as \boost, \taucs, \gmp, \core,
\leda, or \qt.\bigskip

First, it provides an option ``\texttt{Auto-find all libraries}" to
find and enable support for as many of those third-party software
libraries as possible. Note that this is also done automatically
during the first setup test for a platform.

Then there is a sub-menu for each library which can be used in
conjunction with \cgal. Each of these sub-menus contains an autofind
option that is similar to the global autofind in the support menu but
restricted to the particular library under consideration.

In case a library is installed in a non-standard location, you may
have to provide a path to its header files and/or library/ies.
Therefore, the menu for each library \texttt{L} offers an option to
set an include directory \texttt{L\_INCL\_DIR} and a lib directory
\texttt{L\_LIB\_DIR} for \texttt{L}. For instance, if on your system
\gmp\ is installed in \texttt{/opt/sw/gmp/}, you would set
\texttt{GMP\_INCL\_DIR} to \texttt{/opt/sw/gmp/include} and
\texttt{GMP\_LIB\_DIR} to \texttt{/opt/sw/gmp/lib}. Note that on
Cygwin you have to provide Posix-style paths, e.g.,
\texttt{/cygdrive/c/gmp} instead of \verb|C:\gmp|.

\section{The non-interactive mode}\label{sec:non-interactive}
\index{non-interactive
  installation}\index{installation!non-interactive}
\TTsubindex{install\_cgal}{non-interactive mode}

To run the install script in the non-interactive mode, go to the
\cgaldir\ directory and enter the command
\begin{verbatim}
      ./install_cgal -ni <compiler>
\end{verbatim}
where \texttt{<compiler>} is the C++ compiler
executable.\index{compilers!choosing}\\ You can either specify a full
path, e.g. \texttt{/usr/local/bin/g++}, or just the basename, e.g.
\texttt{g++}, which means the script searches your \texttt{PATH} for
the compiler location. If your compiler call contains whitespaces it
has to be quoted, e.g.  \texttt{./install\_cgal -ni "CC -n32"}.  The
options given this way become part of your \cgal-OS
description\index{OS description} (see section
\ref{sec:os-compiler-id}) which is useful e.g. to distinguish between
different compilers using the same frontend (\mipsprocc\ on
\texttt{IRIX6}).

There are a number of additional command line options to customize
your \cgal\ setup which are discussed below. You should read the
corresponding paragraphs before you continue, especially if one or
more of the following conditions apply to you:
\begin{itemize}
\item you want to use \gmp\ together with \cgal\ 
  (Section~\ref{sec:gmp-setup}),
\item you want to use \gmp's built-in C++ interface together with
  \cgal\ (Section~\ref{sec:gmpxx-setup}),
\item you want to use \core\ together with \cgal\ 
  (Section~\ref{sec:core-setup}),
\item you want to use \leda\ together with \cgal\ 
  (Section~\ref{sec:leda-setup}),
\item you want to use \qt\ together with \cgal\ 
  (Section~\ref{sec:qt-setup}).
\end{itemize}

Once you started the script, it should give you a message indicating
the \cgal\ version you are going to install and that you are running
the non-interactive mode. Then it proceeds by locating some utility
programs, determining your OS and compiler version and displaying the
settings you gave via command line. Your compiler is also checked for
a number of bugs resp. support of certain language features; a message
\texttt{ok} always indicates that your compiler works as it should,
that is, a feature is supported or a bug is \textit{not} present. On
the other hand, \texttt{no} or \texttt{unfortunately} indicate a lack
of support or the presence of a bug.

Finally the current setup is summarized, system specific directories
for makefiles and libraries are created (if they did not exist before)
and a new include makefile is written into the makefile directory. If
there already exists a makefile for the current OS/compiler
combination, it is backed up and you should get a corresponding
message.

To compile the \cgal\ libraries go now to the \texttt{src} directory.
Then type \texttt{make -f makefile\_lib} to compile the \cgal\ object
library and \texttt{make -f makefile\_sharedlib} to compile the \cgal\ 
shared object library. If you want to make changes to the makefiles
first, see section \ref{sec:makefiles} for an explanation of the
makefile structure of \cgal.

If you enabled \core\ support and want to use the \core\ distribution
shipped with \cgal, go to the \texttt{src/Core} directory and type
\texttt{make} to compile the \core\ library.

If you enabled \qt\ support, go to the \texttt{src/CGALQt} directory
and type \texttt{make} to compile the \cgal\ \qt\ support library.

%%When this is finished it would be a good idea to print and read the
%%`Getting Started with \cgal'\index{CGAL@\cgal!getting started} document
%%\texttt{getting\_started.ps} that can be found in the \texttt{doc\_ps}
%%directory.

\subsection{Setting up support for GMP}\label{sec:gmp-setup}
\index{GMP@\gmp!enable support}

By default there is no support for \gmp, but you can change this
easily by use of the command line option ``\texttt{-gmp}''. If \gmp\ is
installed in system directories on your system, you are already done
now. If this is not the case, you have to supply the directories
containing the \gmp\ header files (``\texttt{--GMP\_INCL\_DIR}
\textit{$<$dir$>$}'')\TTindex{GMP\_INCL\_DIR} and the \gmp\ library
(``\texttt{--GMP\_LIB\_DIR}
\textit{$<$dir$>$}'')\TTindex{GMP\_LIB\_DIR}.

\subsection{Setting up support for GMPXX}\label{sec:gmpxx-setup}
\index{GMPXX@\texttt{GMPXX}!enable support}

By default there is no support for GMPXX, \gmp's built-in C++
interface, but you can change this easily by use of the command line
option ``\texttt{-gmpxx}''.  The only requirement for this to work is
that \gmp\ support is enabled correctly.

\subsection{Setting up support for CORE}\label{sec:core-setup}
\index{CORE@\core!enable support}

By default there is no support for \core, but you can change this
easily by use of the command line option ``\texttt{-core}''. If you
want to use the \core\ distribution shipped with \cgal, this is all
you have to do. Otherwise, you also have to supply the directories
containing the \core\ header files (``\texttt{--CORE\_INCL\_DIR}
\textit{$<$dir$>$}'')\TTindex{CORE\_INCL\_DIR} and the \core\ library
(``\texttt{--CORE\_LIB\_DIR}
\textit{$<$dir$>$}'')\TTindex{CORE\_LIB\_DIR}.

\subsection{Setting up \leda\ support}\label{sec:leda-setup}
\index{LEDA@\leda!enable support}

See also section \ref{sec:leda}. By default there is no support for
\leda, but you can change this easily by use of the command line
option ``\texttt{-leda}''.  Unless \leda\ is installed in system
directories, you have to supply the directories containing
the \leda\ header files (``\texttt{--LEDA\_INCL\_DIR}
\textit{$<$dir$>$}'')\TTindex{LEDA\_INCL\_DIR} resp. the \leda\ 
libraries for your compiler (``\texttt{--LEDA\_LIB\_DIR} {\it
  $<$dir$>$}''\TTindex{LEDA\_LIB\_DIR}).

\subsection{Setting up support for Qt}\label{sec:qt-setup}
\index{Qt!enable support}

By default there is no support for \qt, but you can change this easily
by use of the command line option ``\texttt{-qt}''. If \qt\ is
installed in system directories on your system or the \texttt{\$QTDIR}
environment variable points to the \qt\ directory, you are already
done now. If this is not the case, you have to supply either the \qt\ 
directory (``\texttt{--QT\_DIR}~\textit{$<$dir$>$}'')\TTindex{QT\_DIR}
if your \qt\ installation is standard, either directories containing
the \qt\ header files
(``\texttt{--QT\_INCL\_DIR}~\textit{$<$dir$>$}'')\TTindex{QT\_INCL\_DIR}
and the \qt\ library
(``\texttt{--QT\_LIB\_DIR}~\textit{$<$dir$>$}'')\TTindex{QT\_LIB\_DIR},
and the path to the MOC\footnote{\qtmocpage} executable
(``\texttt{--QT\_MOC} \textit{$<$exe$>$}'')\TTindex{QT\_MOC} if your
\qt\ installation is not standard.

\subsection{Setting custom compiler/linker flags}\label{sec:custom-setup}
\index{compilers!setting custom flags}

You can supply custom compiler and linker flags using the options
\mbox{(``\texttt{--CUSTOM\_CXXFLAGS}
  \textit{$<$flags$>$}'')\TTindex{CUSTOM\_CXXFLAGS}} and
\mbox{(``\texttt{--CUSTOM\_LDFLAGS}
  \textit{$<$flags$>$}'')\TTindex{CUSTOM\_LDFLAGS}}.  These are the
first flags given to the compiler/linker in every call.

\textit{Note:} Do not forget to quote your options in case they
contain spaces. Example:
\begin{verbatim}
  ./install_cgal -ni g++ --CUSTOM_CXXFLAGS "-I/my/include -O2"
\end{verbatim}


\subsection{Disabling shared libraries building}\label{sec:disable-shared-setup}
\index{compilers!disabling shared libraries building}

You can disable the building of shared libraries (e.g. \texttt{libCGAL.so})
using the option \mbox{``\texttt{--disable-shared}''\TTindex{disable-shared}}.
This way, only static libraries (object file archives) are built.

Example:
\begin{verbatim}
  ./install_cgal --disable-shared -ni g++
\end{verbatim}


\subsection{Other Options}\label{sec:other-options}

There are some less important features of the install script we will
summarize here.

First of all, you can get the version number of \texttt{install\_cgal}
with option ``\texttt{--version}''. Note that all other options are
ignored in this case.\TTsubindex{install\_cgal}{version number}

Second there is an option ``\texttt{-os \textit{$<$compiler$>$}}''
where \textit{$<$compiler$>$} is your \CC\ compiler. This allows you
to determine your \cgal-OS description\index{OS description} (see section
\ref{sec:os-compiler-id}). The compiler can either be given by an
absolute path like
\begin{verbatim}
  ./install_cgal -os /usr/local/gcc-3.4.3/sun/bin/g++
\end{verbatim}
or just by denoting its basename, as long as it is on your path:
\begin{verbatim}
  ./install_cgal -os CC
\end{verbatim}
The option is intended for testing purposes and automatic detection of
the correct include makefile (see also section \ref{sec:makefiles}).

Finally, there exists an option
``\texttt{--verbose}''\TTsubindex{install\_cgal}{verbose mode} that
can be set in interactive mode as well as in non-interactive mode.
When set you get a detailed summary of error messages occurring during
\textit{any} compiler test (determining \stl\ version etc.). Normally
you only get these messages, if a required test (such as the general
\stl\ test) fails, otherwise you are just informed, \textit{if} it
succeeded or not.  This option is not recommended for general use, but
it can be useful to check why a certain test fails that was expected
to be passed.

\section{Upgrading a previous \cgal\ installation}
\label{sec:upgrade}\index{upgrading \cgal}\index{CGAL@\cgal!upgrade}

In case you already have a previous release of \cgal\ installed on
your system, you might like to reuse your configuration files and
\gmp\ installations. Simply use the following command to copy them
into the right place:\TTsubindex{install\_cgal}{upgrade option}
\begin{verbatim}
./install_cgal --upgrade <OLD_CGAL_DIR>
\end{verbatim}
where \texttt{<OLD\_CGAL\_DIR>} is the root directory of your existing
\cgal\ installation\\ (e.g. \texttt{/pub/local/CGAL-3.0}).  You can
then build all libraries for the actual operating system that existed
in your previous \cgal\ installation with
\TTsubindex{install\_cgal}{rebuild-all option}
\begin{verbatim}
./install_cgal --rebuild-all
\end{verbatim}

If you want to install \cgal\ for more than one operating system in
the same directory structure, you have to run the latter command
(\texttt{rebuild-all}) once on each operating system.

Using \texttt{--build-all} instead of \texttt{--rebuild-all} will save
you the time of the configuration tests, and will only rebuild the libraries.

If you want to install only one configuration on a given operating system,
you can specify its name (the base name of a file in \cgalinstconfdir) with
the option \texttt{--rebuild <config>} or \texttt{--build <config>}.

\textbf{Note} that some compilers that have been supported in previous
\cgal\ releases might not be supported in \cgal-\cgalrelease\ anymore,
see section \ref{sec:prerequisites}. Trying to build
\cgal-\cgalrelease\ with these compilers will most probably fail. You
can solve this problem by deleting the obsolete config files (see
section \ref{sec:filescreated}) from \cgalinstconfdir\ before issuing
the \texttt{rebuild-all} command.

Similarly, you might want to use compilers with \cgal-\cgalrelease\ 
that have not been supported in previous releases. For these compilers
please follow the usual procedure as described in section
\ref{sec:interactive-mode} or \ref{sec:non-interactive}.

\section{Identifying OS and Compiler}\label{sec:os-compiler-id}
\index{OS description}\index{identifying OS and compiler}

Since \cgal\ supports several different operating systems and
compilers, this is also reflected in the structure of the \cgal\ 
directory tree. Each OS/compiler combination has its own lib directory
under \texttt{\cgaldir/lib}) (and analogously its own include
makefile\index{include makefile} in \texttt{\cgaldir/make}) named as
determined by the following scheme.
\begin{center}
  \textit{$<$arch$>$\texttt{\_}$<$os$>$\texttt{-}$<$os-version$>$\texttt{\_}$<$comp$>${\tt
    -}$<$comp-version$>$}[\texttt{\_}LEDA]
\end{center}

\begin{description}
\item[$<$arch$>$] is the system architecture as defined by ``{\tt
    uname -p}'' or ``\texttt{uname -m}'',
\item[$<$os$>$] is the operating system as defined by ``\texttt{uname
    -s}'',
\item[$<$os-version$>$] is the operating system version as defined by
  ``\texttt{uname -r}'',
\item[$<$comp$>$] is the basename of the compiler executable (if it
  contains spaces, these are replaced by "-") \textit{and}
\item[$<$comp-version$>$] is the compiler's version number (which
  unfortunately can not be derived in a uniform manner, since it is
  quite compiler specific).
\end{description}

The suffix \texttt{\_}LEDA is appended to indicate \leda\ support. 

We call the resulting string \cgal-OS description.\\ Examples are
\texttt{mips\_IRIX-6.2\_CC-7.2} or {\tt
  sparc\_SunOS-5.5\_g++-3.4.3\_LEDA}.\\ You can use the install script
to get your \cgal-OS description, see section \ref{sec:other-options}.

\section{The \cgal\ makefile structure}\label{sec:makefiles}
\index{makefile structure}

The \cgal\ distribution contains the following makefiles:
\begin{itemize}
\item \texttt{\cgaldir/src/makefile\_lib} for compiling the \cgal\ 
  object library \texttt{libCGAL.a},
  
\item \texttt{\cgaldir/src/makefile\_sharedlib} for compiling the
  \cgal\ shared object library \texttt{libCGAL.so}
  \textit{and}
  
\item \texttt{\cgaldir/examples/*/makefile} for compiling the \cgal\ 
  example programs.
\end{itemize}

All these makefiles are generic: they can be used for more than one
compiler.  To achieve this, the first section of each makefile
contains an include statement that looks as follows:

\begin{verbatim}
      CGAL_MAKEFILE = /users/jannes/CGAL-3.1/make/makefile_<CGAL-OS description>
      include $(CGAL_MAKEFILE)
\end{verbatim}
%$

The file \texttt{CGAL\_MAKEFILE}\TTindex{CGAL\_MAKEFILE} is an include
file\index{include makefile} with platform dependent makefile
settings. The abbreviation \texttt{<CGAL-OS description>} (see section
\ref{sec:os-compiler-id} for details) is used to identify the
operating system and compiler for which the settings hold. For
example, the file \texttt{makefile\_mips\_IRIX64-6.5\_CC-n32-7.30}
contains makefile settings for the IRIX 6.5 operating system and the
\mipsprocc\ 7.3 compiler.  These include files are automatically
generated by the \texttt{install\_cgal} script and they are all
located in the \texttt{\cgaldir/make} directory.  For convenience, the
\texttt{install\_cgal} script will substitute the include makefile
that was generated most recently.

If you want to compile an application or an object library with a
different compiler, the only thing you need to do is to substitute
another include makefile for the \texttt{CGAL\_MAKEFILE} variable. An
alternative way to do this is to create an environment variable
\texttt{CGAL\_MAKEFILE}. To pass the value of the environment variable
to the makefile you can either comment out the \texttt{CGAL\_MAKEFILE}
line in the makefile or use an appropriate command line option for the
make utility.  A comfortable way to set \texttt{CGAL\_MAKEFILE} is by
using \texttt{install\_cgal~-os} (see section
\ref{sec:other-options}).  E.g. if your compiler is \texttt{g++}, you
would type
\begin{alltt}
CGAL_MAKEFILE=`\yourcgaldir/install_cgal -os g++`
\end{alltt}
in bourne shell resp.
\begin{alltt}
setenv CGAL_MAKEFILE `\yourcgaldir/install_cgal -os g++`
\end{alltt}
in csh derivatives. 

\textit{Tip:} Include the setting of \texttt{CGAL\_MAKEFILE} into your
shell startup script (e.g. \texttt{.}(\texttt{t})\texttt{cshrc} for
(\texttt{t})\texttt{csh} or \texttt{.bashrc} for \texttt{bash}).

All makefiles contain sections with compiler and linker flags.  You
can add your own flags here. For example, you might want to add the
flag \texttt{-DCGAL\_NO\_PRECONDITIONS} to turn off precondition
checking. The flags \texttt{\$(CGAL\_CXXFLAGS)} and
\texttt{\$(CGAL\_LDFLAGS)} should never be removed.

The default extension for \cgal\ source files is \texttt{.C}.  The
last section of the makefiles contains a suffix rule that tells the
compiler how to create a \texttt{.o}-file from a \texttt{.C}-file.  If
you want to use the default rule that is defined by the make utility,
you may want to remove this suffix rule.  However, note that this may
have consequences for the makefile variables \texttt{CGAL\_CXX} and
\texttt{CXXFLAGS}.

\section{Compiling a \cgal\ application}
\index{building applications} 
\index{compiling applications}

There is a script for conveniently creating makefiles for \cgal\ applications,
see section~\ref{sec:create_cgal_makefile}.

The directories \texttt{\cgaldir/examples} and \texttt{\cgaldir/demo}
contain many subdirectories with non-graphical and graphical example
programs.  In all these directories you will find a makefile that is
ready for use.

You either need to substitute the \texttt{CGAL\_MAKEFILE} variable in
these makefiles (see section \ref{sec:makefiles}), or set the
environment variable, to point to the makefile in the \texttt{\cgaldir/make}
directory.  Then just type \texttt{make}.

\section{Installation on Cygwin}\label{sec:cygwin}
\index{Cygwin!installation on}\index{installation!on Cygwin}

Cygwin is a free Unix-like environment for MS-Windows, distributed by
Cygnus Solutions. For our tests we have used version 1.3.2 and
$B$-20.1.

It consists of a port of a large number of GNU tools, such as bash,
make, gcc, gas, file utilities, etc, as well as tools ensuring an
ability to emulate Unix-like access to resources, for instance mount.
For a comprehensive introduction and details, see
\path~http://www.cygwin.com/~ .

Make sure that the link \texttt{/bin/sh.exe} exists. If not, create
it:
\begin{verbatim}
cd /bin
ln -s bash.exe sh.exe
\end{verbatim}

\subsection{Pathnames}
\index{Cygwin!pathnames}

Cygwin has a UNIX-like way of navigating hard drives, NFS shares, etc.
This is also the way in which directories and pathnames have to given
to the installation script. They are automatically converted to
Win32-style pathnames when given to the compiler or linker.

The main difference is that directories are seperated by slash (``/'')
rather than by backslash (``$\backslash$'').  The other difference is
concerned with specifying drives. One way is to use POSIX-style
pathnames that map Win32-style drives (\texttt{A:}, \texttt{B:}) to
\texttt{//a/\ldots}, \texttt{//b/\ldots} respectively. For instance,
the path
\texttt{D:$\backslash$Mystuff$\backslash$Mydir$\backslash$LEDA}
translates to \texttt{//d/Mystuff/Mydir/LEDA}.

Alternatively, it can be done using the mount utility, that can be
used to establish a map between Win32-style drives and the Unix-like
style. More precisely, it maps the forest of the directories/files on
Win32-drives to a tree with the root that is usually located at the top
level of the boot drive, say \texttt{C:}.  The root location can be
seen by typing \texttt{mount} command without parameters.  For
instance, if \texttt{D:} is mounted on
\texttt{C:$\backslash$ddrive}\footnote{by typing \texttt{mount D:
    /ddrive}} then the path
\texttt{D:$\backslash$Mystuff$\backslash$Mydir$\backslash$LEDA}
translates to \texttt{/ddrive/Mystuff/Mydir/LEDA}.

\paragraph{Upper/lower case and spaces in file names}
Behavior of Cygwin in this regard might be different from the \mswin\ 
behavior. In particular, using spaces in filenames should better be
avoided.

\paragraph{Links, shortcuts, etc} should be avoided as well.

\subsection{\msvc{}-setup}
\index{Cygwin!setup for \msvc{}}
\index{msvc@\msvc{}!setup on cygwin}

A number of environment variables has to be set (or updated)
in order to use the installation.

\texttt{PATH} should contain \msvc{} command line tools locations.
The environment variables \texttt{INCLUDE} and \texttt{LIB} should
point to the location of \msvc{} header files and to the location of
the \msvc{} libraries, respectively.  The interface for doing this is
different for NT and for Win9*.

\paragraph{\mswin-NT4.0.}
\index{Cygwin!setup on NT4}

One can set the corresponding environment variables using the
usual NT interface\footnote{open MyComputer, press right mouse button,
  select Properties, select Environment, set the relevant variables}.
Alternatively, they can be set in the \texttt{.bashrc} file for the
particular user, or in the system-wide \texttt{bash} customization
file (usually \texttt{/etc/bashrc}).

The result should look roughly as follows, assuming that
\texttt{C:$\backslash$PROGRA$\sim$1$\backslash$MICROS$\sim$2$\backslash$}
is the location of the \msvc{} installation.
\begin{verbatim}
LIB=C:\PROGRA~1\MICROS~2\VC98\LIB
INCLUDE=C:\PROGRA~1\MICROS~2\VC98\INCLUDE
\end{verbatim}
and \texttt{PATH} should contain
\begin{verbatim}
/PROGRA~1/MICROS~2/Common/msdev98/BIN:
/PROGRA~1/MICROS~2/VC98/BIN:/PROGRA~1/MICROS~2/Common/TOOLS:
/PROGRA~1/MICROS~2/Common/TOOLS/WINNT
\end{verbatim}

\paragraph{\mswin-9*.} 
\index{Cygwin!setup on \mswin-9*}

First, the memory for environment variables has to be increased.
Select the Cygwin icon from the Start-menu, press the right mouse
button and choose \textit{Properties}. Go to \textit{Memory}, select
\textit{Initial Environment}, set it to at least 2048 and
\textit{apply} the changes.

Second, edit the file \texttt{cygwin.bat} (or \texttt{cygnus.bat} in
Cygwin~0.9), located in the cygwin main directory and add the line
\begin{verbatim}
call C:\PROGRA~1\MICROS~2\VC98\Bin\MSCVARS32.BAT
\end{verbatim}
%\nonlinkedpath'C:\PROGRA~1\MICROS~2\'
%\texttt{C:$\backslash$PROGRA$\sim$1$\backslash$MICROS$\sim$2$\backslash$}
customized according to where \msvc{} is installed on your
system. Depending on the version of \msvc{} you might have to replace
\texttt{MSCVARS32.BAT} by \texttt{VCVARS32.BAT}.


%%
%% EOF
%%


