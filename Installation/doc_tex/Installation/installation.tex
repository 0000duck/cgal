%%
%%  *** CGAL Installation Guide ***
%%
%% file:     installation.tex
%%
%% authors:  Michael Hoffmann, Dima Pasechnik and Wieger Wesselink
%%
%% $Id$
%%

\newcommand{\cmake}{CMake}
\newcommand{\cpack}{CPack}
\newcommand{\nsis}{NullSoft Scriptable Installer}


%% macro for Windows-specific installation
\newcommand{\MSInst}{Windows-specific Installation}

%% macro for g++
\newcommand{\Gcc}[1]{\gnu~\texttt{g++}~{\rm #1}}
\newcommand{\gccurl}{\path'http://gcc.gnu.org/'}

%% macro for SUNPRO CC
\newcommand{\sunprocc}[1]{\textsc{Sunpro}~\texttt{CC}~{\rm #1}}
\newcommand{\sunproccurl}{\path'http://developers.sun.com/sunstudio/'}

%% macro for Borland C++
\newcommand{\bcc}[1]{\textsc{Borland}~\texttt{C++}~{\rm #1}}
\newcommand{\bccurl}{\path'http://www.inprise.com/bcppbuilder/'}

%% macro for Microsoft Visual C++
\newcommand{\msvc}[1]{\textsc{MS}~Visual~\texttt{C++}~{\rm #1}}
\newcommand{\msvcurl}{\path'http://msdn.microsoft.com/visualc/'}

%% macro for Intel C++ Compiler
\newcommand{\icl}[1]{\textsc{Intel}~\texttt{C++}~{\rm #1}}
\newcommand{\iclurl}{\path'http://developer.intel.com/software/products/compilers/'}

%% macro for Microsoft Windows
\newcommand{\mswin}{\textsc{MS}~Windows}

%% macro for CGAL release number
\newcommand{\cgalrelease}{3.4}

%% macro for CGAL directory
\newcommand{\cgaldir}{\texttt{CGAL-\cgalrelease}}

%% macro for YOUR CGAL directory :)
\newcommand{\yourcgaldir}{$<$\textit{insert your \cgaldir\ dir}$>$}

%% macro for CGAL install config directory
\newcommand{\cgalinstconfdir}{\cgaldir\texttt{/config/install}\index{directories!config/install@\texttt{config/install}}}

%% macros for WWW pages
\newcommand{\cgalhomepage}{\path'http://www.cgal.org'}
\newcommand{\gmppage}{\path'http://gmplib.org/'}
\newcommand{\mpfrpage}{\path'http://www.mpfr.org/'}
\newcommand{\boostpage}{\path'http://www.boost.org/'}
\newcommand{\corepage}{\path'http://www.cs.nyu.edu/exact/core\_pages/'}
\newcommand{\ledapage}{\path'http://www.algorithmic-solutions.com/'}
\newcommand{\trolltechpage}{\path'http://www.trolltech.com'}
\newcommand{\qtpage}{\path'http://doc.trolltech.com/'}
\newcommand{\qtmocpage}{\path'http://doc.trolltech.com/moc.html'}
\newcommand{\blaspage}{\path'http://www.netlib.org/blas/'}
\newcommand{\atlaspage}{\path'http://math-atlas.sourceforge.net/'}
\newcommand{\lapackpage}{\path'http://www.netlib.org/lapack/'}
\newcommand{\taucspage}{\path'http://www.tau.ac.il/~stoledo/taucs/'}
\newcommand{\xpage}{\path'http://www.x.org/'}
\newcommand{\zlibpage}{\path'http://www.zlib.net/'}
\newcommand{\cmakepage}{\path'http://www.cmake.org/'}
\newcommand{\cpackpage}{\path'http://www.cmake.org/cmake/help/cpack2.6docs.html'}
\newcommand{\nsispage}{\path'http://nsis.sourceforge.net/Main\_Page'}
\newcommand{\fhs}{\path'http://www.pathname.com/fhs/pub/fhs-2.3.html'}

%% write index in typewriter type. (c) Susan :-)
\newcommand{\TTindex}[1]{\index{#1@{\tt #1}}}
\newcommand{\TTsubindex}[2]{\index{#1@{\tt #1}!{#2}}}
\newcommand{\TTsubindextwo}[2]{\index{#1!#2@{\tt #2} }}

\section{Introduction}

\cgal\ stands for \textit{Computational Geometry Algorithms Library}.
It is a software library written in \CC, whose development started in
an {\sc Esprit Ltr} project. The goal of \cgal\ is to make the large
body of geometric algorithms developed in the field of computational
geometry available for industrial application.

This document describes how to install \cgal\ on Windows, Unix-like systems and Mac.

Besides that, you will find some information about the build system
structure of \cgal\ and the support for using \cgal\ together with
other software libraries, such as the \boost\
libraries\footnote{\boostpage}, the \gnu\ Multiple Precision library
\gmp\footnote{\gmppage}, the \mpfr\ library\footnote{\mpfrpage} for
multiple-precision floating-point computations with exact rounding,
the \core\ library\footnote{\corepage} for robust numerical and
geometric computation, \leda, the Library of Efficient Datatypes and
Algorithms\footnote{\ledapage}, \blas, the Basic Linear Algebra Subprograms
library\footnote{\blaspage}, \lapack, the Linear Algebra PACKage library
\footnote{\lapackpage}, \taucs, a library of sparse linear
solvers\footnote{\taucspage}, or Trolltech's\footnote{\trolltechpage}
\qt\ toolkit.

\section{Prerequisites\label{sec:prerequisites}}

In order to build the \cgal\ libraries you need a \CC\ compiler.  
\cgal 3.4 is supported for the following compilers/operating systems:

\begin{center}\index{compilers!supported}\index{supported compilers}
  \renewcommand{\arraystretch}{1.3}
  \gdef\lcTabularBorder{2}
  \begin{tabular}{|l|l|} \hline
    \textbf{compiler}        & \textbf{operating system}\\\hline\hline
    \Gcc{3.4, 4.0, 4.1, 4.2, 4.3} \footnotemark[10]
    & Solaris 2.6+ / Linux 2.x / MacOS X /
      \\ & \mswin\ 95/98/2000/XP/NT4\footnotemark[11]\\\hline
    \msvc{8.0, 9.0} (\textsc{Visual Studio 2005 and 2008}) \footnotemark[12]
    & \mswin\ 95/98/2000/XP/NT4/Vista\footnotemark[11]\\\hline
  \end{tabular}
\end{center}
\footnotetext[10]{\gccurl}\addtocounter{footnote}{1}
\footnotetext[11]{\msvcurl}\addtocounter{footnote}{1}
\footnotetext[12]{\iclurl}\addtocounter{footnote}{1}

Note that neither prerelease versions nor repository snapshots of
\texttt{GCC} are supported.



In order to configure, build and install the \cgal\ libraries, as well
as any program using \cgal, you need \cmake, a Cross Platform Make system.
If \cmake\ is not installed already you can obtain it from \cmakepage.
We recommend the usage of cmake Rel. 2.6, and you need at least
cmake Rel. 2.4-patch-7.

Moreover, \cgal\ requires a working installation of the \boost\
libraries (the header files and the threading library binaries), 
Version 1.33.1 or later. In case
the \boost\ libraries are not installed on your system already, you
can obtain them from \boostpage. For Windows you can download an
installer from \path'http://www.boostpro.com/products/free'.
Since \texttt{Boost.Thread} is required, make sure to either install the precompiled 
{\em static} binaries for your compiler or to build \texttt{libboost-thread}.

Having \gmp\ version 4.1.4 or later and \mpfr\ version 2.0.1 or later
installed is highly recommended. These libraries can be obtained from
\gmppage\ and \mpfrpage, respectively, if they should not be present
on your system. Within \cgal, support for \gmp\ requires also \mpfr\ and
vice versa. As the \gmp\ project doesn't properly support Visual \CC\ we
provide precompiled versions of \gmp\ and \mpfr.  


\section{Downloading CGAL\label{sec:gettingcgal}}
\index{CGAL@\cgal!getting}\index{getting \cgal}

The \cgal\ library can be downloaded from  \path'http://www.cgal.org/download.html'.

After you have downloaded the file \texttt{CGAL-3.4.tar.gz} containing the \cgal\ library, you
have to unpack it. Under a Unix-like shell, use the command

\begin{verbatim}
      tar xzf CGAL-3.4.tar.gz
\end{verbatim}


When you are on Windows you may download and run the \texttt{CGAL-3.4-Installer.exe}. It is a
self extracting executable that installs the \cgal\ source and header
files, and that allows you to select and download some precompiled
third party libraries.

In both cases the directory \cgaldir\ will be created. This directory
contains the following subdirectories:\index{directories!structure}

\begin{center}
  \renewcommand{\arraystretch}{1.3}
  \gdef\lcTabularBorder{2}
  \begin{tabular}{|l|l|} \hline
    \textbf{directory} & \textbf{contents}\\\hline\hline
    \texttt{auxiliary} & precompiled GMP, MPFR and TAUCS for Windows\\\hline
    \texttt{config}    & configuration files for install script\\\hline
    \texttt{cmake/modules}    & modules for finding and using libraries\\\hline
    \texttt{demo}      & demo programs (most of them need \qt, geomview
                         or other third-party products)\\\hline
    \texttt{doc\_html} & documentation (HTML)\\\hline
    \texttt{doc\_pdf}  & documentation (PDF)\\\hline
    \texttt{examples}  & example programs\\\hline
    \texttt{include}   & header files\\\hline
    \texttt{lib}       & (shared) object libraries\\\hline
    \texttt{scripts}       & some useful scripts (e.g. for creating CMakeLists.txt files)\\\hline
    \texttt{src}       & source files\\\hline
  \end{tabular}
\end{center}

The directories \texttt{include/CGAL/CORE} and \texttt{src/CGALCore} contain a
distribution of the \core\ library\footnote{\corepage} version~1.7 for
robust numerical and geometric computation. \core\ is not part of
\cgal\ and has its own license.



\section{Configuring CGAL}

Before building \cgal\ you have to choose the compiler/linker, 
set compiler and linker  flags, specify which
third-party libraries you want to use and where they can be found, 
which \cgal\ libraries you want to build. Gathering
all this information is called {\em configuration}. 

For CGAL-3.4, the configuration is generated with \cmake, a
cross-platform build system.  This manual explains only those features of
\cmake\ which are needed in order to build \cgal. Please refer to the \cmake\ 
documentation at \cmakepage\ for further details.

When \cmake\ has finished the configuration, it has generated
a makefile or a Visual \CC\  solution and project file that you 
can use to build \cgal.

\subsection{Configuring \cgal{} with the cmake {\sc Gui} }

The simplest way to start the configuration is to run the graphical
user interface of \cmake. We recommend to use \texttt{cmake-gui}, which
is available on all platforms since \cmake\ Rel. 2.6.   You must pass as 
argument the root directory of \cgal. For example:

{\ccTexHtml{\scriptsize}{}
\begin{verbatim}
  cd  CGAL-3.4
  cmake-gui . # Notice the dot to indicate the current directory.
\end{verbatim}
}

Once \texttt{cmake-gui} has started up, you must press 'Configure'. 
A dialog will pop up and you have to choose what shall get generated.
After you have made your choice and pressed 'ok', you will see
the output of configuration tests in the lower window of the application. 
When these tests are done, you will see many
red entries in the upper window. Just ignore them and  press once again  'Configure'. 
By now cmake should have found many libraries and have initialized variables. 
If you still find red entries you have to provide the necessary information. 
This typically happens if you have installed software at non-standard locations.
Providing information and pressing 'Configure' goes on until 
all entries are grayed. You are now ready to press 'Configure' and you
can quit \texttt{cmake-gui}.

 

\subsection{Configuring \cgal{} with the cmake command-line tool }

Alternatively, you can run the command-line tool called
\texttt{cmake}. You pass as argument the root directory of
\cgal. For example:

{\ccTexHtml{\scriptsize}{}
\begin{verbatim}
  cd  CGAL-3.4
  cmake . # Notice the dot to indicate the current directory.
\end{verbatim}
}


The very first thing \cmake\ does is to detect the compiler to use.
This detection is performed by a special cmake module called a {\em generator}. A
\cmake\ generator understands the build requirements for a particular compiler/linker
 and generates the necessary files for that. For example, the 
{\em UNIX Makefiles} generator understands the GNU toolchain (g++,gcc,ld etc) and produces makefiles
which can be used to build a target by a simple call to \texttt{make}.
Likewise, the {\em Visual Studio 2005} generator produces solution and project files and can be manually
launched in the VS IDE to build the target.

Each platform has a default generator so you only need to select one when the default is
not what you want. 
\footnote{Running \texttt{cmake} with no parameters in a command-line prints the list of available generators supported 
by your platform and \cmake\ version. If the generator you need is not listed there, you can
try a newer \cmake\ version since generators are hardcoded into \cmake\ and more and
more generators are added with each release.}
For example, under Windows, it is possible to generate {\em NMakefiles}
instead of Visual Studio project files in order to build the library with \texttt{nmake}.

Since the choice of the generator determines the type of build files to generate, in some cases
you choose a particular generator as a mean to choose a specific compiler (because they use different 
build files. For example, the following generates solution files for use in Visual \CC\ 9.0:

{\ccTexHtml{\scriptsize}{}
\begin{verbatim}
  cd ~/CGAL-3.4
  cmake -G"Visual Studio 9 2008" . 
\end{verbatim}
}

In other cases, however, the generator doesn't directly identify a specific compiler but a tool chain.
For example, the \texttt{UNIX Makefiles} generator produces \texttt{makefiles} that call some auto-detected
command-line compiler, like \texttt{gcc}. If you need the makefiles to use a different compiler you need to
say so in the call to cmake, as in this example:

{\ccTexHtml{\scriptsize}{}
\begin{verbatim}
  cd ~/CGAL-3.4
  cmake -DCMAKE_CXX_COMPILER:FILEPATH=g++-3.4 . 
\end{verbatim}
}


\cmake\ maintains configuration parameters in so-called {\em cmake variables}, like the \texttt{CMAKE\_CXX\_COMPILER}
in the example above. These variables {\em are not environment variables} and they must be set and modified in
certain specific ways as explained below. Some of the cmake variables represent user choices, such as
\texttt{WITH\_examples} or \texttt{CMAKE\_BUILD\_TYPE=Release}, while others indicate the details of a third-party library, 
such as \texttt{Boost\_INCLUDE\_DIR} or the compiler flags to use, such as \texttt{CMAKE\_CXX\_FLAGS}. 

The command line tool \texttt{cmake} accepts cmake variables as arguments of the form \texttt{-D<VAR>:<TYPE>=<VALUE>}, as
in the example above, but this is only useful if you already know which variables needs to be explicitly defined.

\begin{ccAdvanced}
\cmake\ keeps the variables that a user can manipulate in a so-called {\em cmake cache}, a simple text file specifically 
named \texttt{CMakeCache.txt} whose entries are of the form VARIABLE:TYPE=VALUE. Advanced users can just manually edit this 
file instead of going through the interactive configuration session explained before.
\end{ccAdvanced}


The configuration process not only determines the location of the required dependencies, it also dynamically generates a
\texttt{compiler\_config.h} file which encodes the properties of your system and a special file named 
\texttt{CGALConfig.cmake} which is used to build programs using \cgal\ (this file will be explained later).

\subsection{\cgal\ Libraries}

\cgal\ is split in four libraries, in order to push some auxiliary 
functionality out of the main library. During configuration, you can selected the libraries that 
shall be built by setting the cmake variable of the form {\tt WITH\_<library>}. By default all 
are switched \texttt{ON}.

\begin{center}\index{cgal!libraries}\index{cgal libraries}
  \renewcommand{\arraystretch}{1.3}
  \gdef\lcTabularBorder{2}
  \begin{tabular}{|l|l|l|l|} \hline
    \textbf{library} & \textbf{cmake variable} & \textbf{functionality} & \textbf{dependencies}
    \\\hline\hline
    \texttt{CGAL} & {\em none} & Main library  & GMP, MPFR, Boost (headers) and Boost.Thread (library)
    \\\hline
    \texttt{CGAL\_Core} & \texttt{WITH\_CGAL\_Core} & The CORE library for dealing with algebraic numbers.
    \footnote{CGAL-core++ is not part of \cgal, it is a custom version the CORE library distributed by \cgal\
              for the user convenience and it has it's own license. }
    & GMP and MPFR
    \\\hline
    \texttt{CGAL\_Qt3} & \texttt{WITH\_CGAL\_Qt3} & \ccc{CGAL::Qt_widget} used by some old Qt3-based demos & Qt3 and OpenGL
    \\\hline
    \texttt{CGAL\_Qt4} & \texttt{WITH\_CGAL\_Qt4} & \ccc{CGAL::GraphicsView} used by the Qt4-based demos & Qt4 and OpenGL
    \\\hline
    \texttt{CGAL\_ImageIO} & \texttt{WITH\_CGAL\_ImageIO} & Utilities to read and write image files in various formats 
    & OpenGL, ZLib, and VTK (optional)
    \\\hline
    \texttt{CGAL\_PDB} & \texttt{WITH\_CGAL\_PDB} & \ccc{CGAL::PDB} used to manipulate \texttt{PDB} files & {\em none}
    \\\hline
  \end{tabular}
\end{center}

If you turn off the configuration of a library you can still configure it manually from the  source directory:

{\ccTexHtml{\scriptsize}{}
\begin{verbatim}
  cd ~/CGAL-3.4/src/CGALQt4 
  cmake . # configures only the CGAL_Qt4 library
\end{verbatim}
}

\subsection{Examples and Demos}

\cgal\ is distributed with a large collection of examples and demos. By default, these are all configured along with
the \cgal\ libraries. When configured, they are not automatically built along with the libraries, you do that
by selecting the \texttt{examples} or \texttt{demo} make targets as explained in the following section, but you can set the 
variables {\tt WITH\_examples=OFF} and {\tt WITH\_demo=OFF} to prevent configuration of examples and demos altogether.


\subsection{Debug vs.\ Release}

The cmake variable \texttt{CMAKE\_BUILD\_TYPE} can take the values \texttt{Release} or \texttt{Debug} to indicate how to build
the libraries. The default depends on the platform so it is recommended that you always indicate the build type explicitly.

This is not an issue for solution/project files, as there the user selects the build type from within the IDE.

\subsection{Static vs.\ Dynamic Libraries }

Under Windows we only support  the static versions of the \cgal\ libraries, but on the other platforms
they are by default shared libraries ({\tt .so}).
You can choose to produce static libraries instead by setting the cmake variable {\tt CMAKE\_SHARED\_LIBS=FALSE} 

These setting affect the variants of third-party libraries selected whenever the choice is available.

\subsection{Multiple Variants of makefiles}\label{sec:cmake-out-of-source}

While you can choose between release/debug builds, shared/static libraries,
it is not possible to generate different variants during a single configuration. To do that you need to run cmake in a 
different directory for each variant, each with its own selection of configuration parameters.

\cmake\ stores the resulting makefiles/projectfiles, along with several temporary and auxiliary files such
as the variables cache, in the directory where it is executed, called \texttt{CMAKE\_BINARY\_DIR}, but it
takes the source files and configuration scripts from whatever directory is indicated, called
\texttt{CMAKE\_SOURCE\_DIR} 

The binary and source directory need not be the same, thus, you can configure multiple variants by creating a
distinct directory for each configuration and by running cmake from there. This is known in \cmake\ terminology
as  {\em out-of-source configuration}, as opposite to an {\em in-source configuration} as showed in the 
previous sections.

You can for example generate subdirectories \cgaldir{}\texttt{/cmake/platforms/debug} and 
\cgaldir{}\texttt{/cmake/platforms/release} for two configurations:

{\ccTexHtml{\scriptsize}{}
\begin{verbatim}
mkdir ~/CGAL-3.4/cmake/platforms/debug
cd ~/CGAL-3.4/cmake/platforms/debug
cmake -DCMAKE_BUILD_TYPE=Debug ../../..

mkdir ~/CGAL-3.4/cmake/platforms/release
cd ~/CGAL-3.4/cmake/platforms/release
cmake -DCMAKE_BUILD_TYPE=Release ../../..
\end{verbatim}
}



\subsection{Compiler and Linker Flags}

\cmake\ determines the set of compiler and linker flags to use based on the target compiler and automatically defines
the following cmake variables:

\begin{enumerate}
\item \texttt{CMAKE\_CXX\_FLAGS}
\item \texttt{CMAKE\_CXX\_FLAGS\_RELEASE}
\item \texttt{CMAKE\_CXX\_FLAGS\_DEBUG}
\item \texttt{CMAKE\_MODULE\_LINKER\_FLAGS}
\item \texttt{CMAKE\_MODULE\_LINKER\_FLAGS\_RELEASE}
\item \texttt{CMAKE\_MODULE\_LINKER\_FLAGS\_DEBUG}
\item \texttt{CMAKE\_SHARED\_LINKER\_FLAGS}
\item \texttt{CMAKE\_SHARED\_LINKER\_FLAGS\_RELEASE}
\item \texttt{CMAKE\_SHARED\_LINKER\_FLAGS\_DEBUG}
\item \texttt{CMAKE\_EXE\_LINKER\_FLAGS}
\item \texttt{CMAKE\_EXE\_LINKER\_FLAGS\_RELEASE}
\item \texttt{CMAKE\_EXE\_LINKER\_FLAGS\_DEBUG}
 \end{enumerate}

The \texttt{MODULE} family of flags correspond to static libraries, the \texttt{SHARED} flags to shared libraries
and \texttt{EXE} flags to executables.

The effective flags are always a concatenation of the general and build-type specific flags, such as
\texttt{CMAKE\_CXX\_FLAGS} {\em then} \texttt{CMAKE\_CXX\_FLAGS\_RELEASE}.

You can {\em completely override} any of these variables if you pass them on the command-line \texttt{cmake}
Alternatively, you can change the values given by \cmake\ using the
interactive cmake tools (\texttt{CMakeSetup.exe}, \texttt{ccmake} or \texttt{cmake-gui}).  This is useful for example to produce a fast debug
variant without diagnostics (-DNDEBUG added). 

If you do not want to override the automatically configured flags but only add more, you can 
define any of the following cmake variables, which are counterparts of the default
variables but starting with \texttt{CGAL} instead of \texttt{CMAKE}:

\begin{enumerate}
\item \texttt{CGAL\_CXX\_FLAGS}
\item \texttt{CGAL\_CXX\_FLAGS\_RELEASE}
\item \texttt{CGAL\_CXX\_FLAGS\_DEBUG}
\item \texttt{CGAL\_MODULE\_LINKER\_FLAGS}
\item \texttt{CGAL\_MODULE\_LINKER\_FLAGS\_RELEASE}
\item \texttt{CGAL\_MODULE\_LINKER\_FLAGS\_DEBUG}
\item \texttt{CGAL\_SHARED\_LINKER\_FLAGS}
\item \texttt{CGAL\_SHARED\_LINKER\_FLAGS\_RELEASE}
\item \texttt{CGAL\_SHARED\_LINKER\_FLAGS\_DEBUG}
\item \texttt{CGAL\_EXE\_LINKER\_FLAGS}
\item \texttt{CGAL\_EXE\_LINKER\_FLAGS\_RELEASE}
\item \texttt{CGAL\_EXE\_LINKER\_FLAGS\_DEBUG}
\end{enumerate}


\section{Building \cgal\ Libraries}

If configuration succeeded there will be certain {\em build files} ready to build the libraries.
The nature of the build files depend on the generator used during configuration, but in all cases they 
will contain several {\em targets}, one per library,  and a default global target corresponding 
to all the libraries at once.

For example, in a UNIX-like environment the default generator produces makefiles and you would
use the \texttt{make} command-line tool as follow:

{\ccTexHtml{\scriptsize}{}
\begin{verbatim}

cd ~/CGAL-3.4

# build all the selected library at once

make 


# build just the libraries needed for the Surface mesher demo

make  CGAL  CGAL_ImageIO CGAL_Qt4

\end{verbatim}
}

The libraries will all be located in a subdirectory {\tt /lib} under {\tt <CMAKE\_BINARY\_DIR>}
which is \cgaldir\ in case you run an in-source-configuration.

With generators other than {\tt UNIX Makefiles} the resulting build files are solution/project files which
should be launched in an {\sc Ide}, such as Visual Studio or KDevelop3. They will contain the targets described
above which you can manually build as any other solution/project within your {\sc Ide}.

Alternatively, you can build it with the command line version of the {\sc Ide}:

{\ccTexHtml{\scriptsize}{}
\begin{verbatim}

    devenv CGAL.sln /Build Debug

\end{verbatim}
}

The "Debug" argument is needed because CMake creates solution files with
all four configurations and you need to explicitly choose one when building
(the other choices are Release, RelWithDebInfo and MinSizeRel).


\begin{ccAdvanced}
The build files produced by cmake are autoconfigured. That is, if you change any 
of the dependencies, the build step automatically goes all the way back to the configure step
during the building. This way, once the target has been configured the very first time by
invoking cmake, you don't necessarily need to invoke \texttt{cmake} again. Rebuilding will call
itself \texttt{cmake} and re-generate the build file whenever needed. Keep this in mind if you
configure \cgal\ for the Visual Studio IDE since a build could then change the solution/project 
file in-place and VS will prompt you to reload it.
\end{ccAdvanced}


If you have not turned off configuration of examples and/or demos, there will be additional
targets named \texttt{examples} and \texttt{demo}, plus one target for each example and each demo.
None of these targets are included in the default so you need to build them explicitly
{\em after} the \cgal\ libraries have been successfully compiled and linked
The targets \texttt{examples} and \texttt{demo} include themselves all the targets
for examples and demos respectively.

{\ccTexHtml{\scriptsize}{}
\begin{verbatim}

# build all examples at once
make examples 

# build only the Straight Skeleton demo
make Straight_skeleton_2_demo

\end{verbatim}
}

\section{Installing \cgal\ Libraries}

% prevent links to /usr/include and /usr/lib (\path commands)
\ccHtmlLinksOff%
On many platforms, library pieces such as headers, docs and binaries
are expected to be placed in specific locations. The typical example
being \path'/usr/include' or \path'/usr/lib' on {\sc Unix}-like
operating systems or \path'C:/Program Files/' on Windows. The process
of placing or copying the library elements into its standard location
is sometimes referred to as {\em Installation} and it is a
postprocessing step after configuration and compilation.
\ccHtmlLinksOn%

\cmake\ supports installation by producing a build target named {\em install}. 
The following example shows a typical session from configuration to
installation in a {\sc Unix}-like environment:

{\ccTexHtml{\scriptsize}{}
\begin{verbatim}

cd ~/CGAL-3.4

cmake . # configure
make    # make

sudo make install # install

\end{verbatim}
}

The installation step copies files to locations which are typically accessible only to the root user so
it is typically necessary to do so as a super-user, hence the call to {\tt 'sudo'} first.

If you used a generator which produced IDE files (for Visual Studio for instance) there will be an optional
\texttt{INSTALL} project which you can {\em ``build''} to execute the installation step. 

The files are copied into a directory tree relative to the {\em installation directory} determined by the 
cmake variable \texttt{CMAKE\_INSTALL\_PREFIX}. This variable defaults to {\tt /usr/local} under UNIX-like operating systems
and {\tt PROGRAMFILES } under Windows. If you want to install to a different location you must override that cmake
variable explicitly {\em at configuration time} (not when executing the install step).

If you wish the be able to specify the install location when executing the install step rather than hard-coding it during
configuration you can use the environment variable \texttt{DESTDIR}. However, since \texttt{CMAKE\_INSTALL\_PREFIX} is prepended
to the value of \texttt{DESTDIR} (hence its name), in practice you need to define it as an empty string in order to gain complete
control via \texttt{DESTDIR}.
This is illustrated in the following example:

{\ccTexHtml{\scriptsize}{}
\begin{verbatim}

cd ~/CGAL-3.4

cmake -DCMAKE_INSTALL_PREFIX="" . 

make

DESTDIR=/usr/share sudo make install

\end{verbatim}
}



\section{Configuring and Building Programs Using \cgal}

\cmake can be used to configure and build user programs as well via a \cmake\ 
script ({\tt CMakeLists.txt}) for the program.  All the examples and
demos contain such \cmake\ scripts.

During configuration of the \cgal\ libraries a file named {\tt
CGALConfig.cmake} will be generated in the binary directory. This file
contains the definitions of several cmake variable which summarize the
configuration of \cgal. In order to configure a user program you need
to indicate the location of that config file in the cmake variable
\texttt{CGAL\_DIR}:.

{\ccTexHtml{\scriptsize}{}
\begin{verbatim}

cd ~/CGAL-3.4/examples/Straight_skeleton_2
cmake -DCGAL_DIR=~/CGAL-3.4 .
make

\end{verbatim}
}

By default, \cgal\ programs inherit the flags and build-type used
for \cgal\ to avoid runtime problems unless you explicitely indicate otherwise by
defining the \cmake\ variable {\tt CGAL\_DONT\_OVERRIDE\_DEFAULT\_SETTINGS=TRUE}.
The following example illustrates this:

{\ccTexHtml{\scriptsize}{}
\begin{verbatim}

cd ~/CGAL-3.4

# CGAL configured in release mode but with debug information 
cmake -DCMAKE_BUILD_TYPE=Release -DCMAKE_CXX_FLAGS=-g .

cd ~/CGAL-3.4/examples/Straight_skeleton_2

# In this case, the settings of build type and flags HAVE NO EFFECT
# because they are overriden by the values used above
cmake -DCGAL_DIR=~/CGAL-3.4 -DCMAKE_BUILD_TYPE=Debug -DCMAKE_CXX_FLAGS=-O2 . 

# But in this case they are respected
cmake -DCGAL_DIR=~/CGAL-3.4 -DCMAKE_BUILD_TYPE=Debug -DCMAKE_CXX_FLAGS=-O2 -DCGAL_DONT_OVERRIDE_DEFAULT_SETTINGS=TRUE . 

\end{verbatim}
}

\section{Summary of Configuration Variables}

\subsection{CMake variables used to configure both \cgal\ libraries and user programs}

\begin{center}\index{cgal!general-config-cmake-vars}\index{cgal configuration cmake variables - general}
  \renewcommand{\arraystretch}{1.3}
  \gdef\lcTabularBorder{2}
  \begin{tabular}{|l|l|l|} \hline
    \textbf{variable}  & \textbf{description} & \textbf{default value}
    \\\hline\hline
    \texttt{CMAKE\_BUILD\_TYPE}  & Indicates type of build. Possible values are Debug,Release & Debug
    \\\hline
    \texttt{CMAKE\_CXX\_FLAGS} & Space-separated list of C++ flags for both Debug and Release modes & platform-dependent, determined by cmake
    \\\hline
    \texttt{CMAKE\_CXX\_FLAGS\_RELEASE} & Space-separated list of C++ flags for Release mode only  & platform-dependent, determined by cmake
    \\\hline
    \texttt{CMAKE\_CXX\_FLAGS\_DEBUG} & Space-separated list of C++ flags for Debug mode only & platform-dependent, determined by cmake
    \\\hline
    \texttt{CMAKE\_SHARED\_LINKER\_FLAGS} & Space-separated list of linker flags for shared libraries on both Debug and Release modes & platform-dependent, determined by cmake
    \\\hline
    \texttt{CMAKE\_SHARED\_LINKER\_FLAGS\_RELEASE} & Space-separated list of linker flags for shared libraries on Release mode only & platform-dependent, determined by cmake
    \\\hline
    \texttt{CMAKE\_SHARED\_LINKER\_FLAGS\_DEBUG} & Space-separated list of linker flags for shared libraries on Debug mode only & platform-dependent, determined by cmake
    \\\hline
    \texttt{CMAKE\_MODULE\_LINKER\_FLAGS} & Space-separated list of linker flags for static libraries on both Debug and Release modes & platform-dependent, determined by cmake
    \\\hline
    \texttt{CMAKE\_MODULE\_LINKER\_FLAGS\_RELEASE} & Space-separated list of linker flags for static libraries on Release mode only & platform-dependent, determined by cmake
    \\\hline
    \texttt{CMAKE\_MODULE\_LINKER\_FLAGS\_DEBUG} & Space-separated list of linker flags for static libraries on Debug mode only & platform-dependent, determined by cmake
    \\\hline
    \texttt{CMAKE\_EXE\_LINKER\_FLAGS} & Space-separated list of linker flags for executables on both Debug and Release modes & platform-dependent, determined by cmake
    \\\hline
    \texttt{CMAKE\_EXE\_LINKER\_FLAGS\_RELEASE} & Space-separated list of linker flags for executables on Release mode only & platform-dependent, determined by cmake
    \\\hline
    \texttt{CMAKE\_EXE\_LINKER\_FLAGS\_DEBUG} & Space-separated list of linker flags for executables on Debug mode only & platform-dependent, determined by cmake
    \\\hline
    \texttt{CGAL\_CXX\_FLAGS} & Additional space-separated list of C++ flags for both Debug and Release modes & (empty)
    \\\hline
    \texttt{CGAL\_CXX\_FLAGS\_RELEASE} & Additional space-separated list of C++ flags for Release mode only & (empty)
    \\\hline
    \texttt{CGAL\_CXX\_FLAGS\_DEBUG} & Additional space-separated list of C++ flags for Debug mode only & (empty)
    \\\hline
    \texttt{CGAL\_SHARED\_LINKER\_FLAGS} & Additional space-separated list of linker flags for shared libraries on both Debug and Release modes & (empty)
    \\\hline
    \texttt{CGAL\_SHARED\_LINKER\_FLAGS\_RELEASE} & Additional space-separated list of linker flags for shared libraries on Release mode only & (empty)
    \\\hline
    \texttt{CGAL\_SHARED\_LINKER\_FLAGS\_DEBUG} & Additional space-separated list of linker flags for shared libraries on Debug mode only & (empty)
    \\\hline
    \texttt{CGAL\_MODULE\_LINKER\_FLAGS} & Additional space-separated list of linker flags for static libraries on both Debug and Release modes & (empty)
    \\\hline
    \texttt{CGAL\_MODULE\_LINKER\_FLAGS\_RELEASE} & Additional space-separated list of linker flags for static libraries on Release mode only & (empty)
    \\\hline
    \texttt{CGAL\_MODULE\_LINKER\_FLAGS\_DEBUG} & Additional space-separated list of linker flags for static libraries on Debug mode only & (empty)
    \\\hline
    \texttt{CGAL\_EXE\_LINKER\_FLAGS} & Additional space-separated list of linker flags for executables on both Debug and Release modes & (empty)
    \\\hline
    \texttt{CGAL\_EXE\_LINKER\_FLAGS\_RELEASE} & Additional space-separated list of linker flags for executables on Release mode only & (empty)
    \\\hline
    \texttt{CGAL\_EXE\_LINKER\_FLAGS\_DEBUG} & Additional space-separated list of linker flags for executables on Debug mode only & (empty)
    \\\hline
    \texttt{CMAKE\_CXX\_COMPILER} & Full-path to the executable corresponding to the C++ compiler to use. This is used only if Makefiles are generated & platform-dependent, determined by cmake
    \\\hline
  \end{tabular}
\end{center}

\subsection{Environment variables used to configure both \cgal\ libraries and user programs}

\begin{center}\index{cgal!general-config-env-vars}\index{cgal configuration environment variables - general }
  \renewcommand{\arraystretch}{1.3}
  \gdef\lcTabularBorder{2}
  \begin{tabular}{|l|l|} \hline
    \textbf{variable}  & \textbf{description} 
    \\\hline\hline
    {\tt CXX } & Full-path to the executable corresponding to the C++ compiler to use. This is used only if Makefiles are generated
    \\\hline
  \end{tabular}
\end{center}

\subsection{CMake variables used to configure the \cgal\ libraries only}

\begin{center}\index{cgal!lib-config-cmake-vars}\index{cgal configuration cmake variables - libs }
  \renewcommand{\arraystretch}{1.3}
  \gdef\lcTabularBorder{2}
  \begin{tabular}{|l|l|l|l|} \hline
    \textbf{variable} & \textbf{description} & \textbf{default value}
    \\\hline\hline
    \texttt{WITH\_examples}  & Whether to configure examples along with the library (TRUE/FALSE) & TRUE
    \\\hline
    \texttt{WITH\_demos} & Whether to configure demos along with the library (TRUE/FALSE) & TRUE
    \\\hline
    \texttt{WITH\_CGAL\_Core} & Whether to configure CGAL\_Core component along with the main library (TRUE/FALSE) & TRUE
    \\\hline
    \texttt{WITH\_CGAL\_Qt3} & Whether to configure CGAL\_Qt3 component along with the main library (TRUE/FALSE) & TRUE
    \\\hline
    \texttt{WITH\_CGAL\_Qt4} & Whether to configure CGAL\_Qt4 component along with the main library (TRUE/FALSE) & TRUE
    \\\hline
    \texttt{WITH\_CGAL\_ImageIO} & Whether to configure CGAL\_ImageIO component along with the main library (TRUE/FALSE) & TRUE
    \\\hline
    \texttt{WITH\_CGAL\_PDB} & Whether to configure CGAL\_PDB component along with the main library (TRUE/FALSE) & TRUE
    \\\hline
    \texttt{WITH\_CPACK} & Whether to create a package for distribution of the library (TRUE/FALSE) & FALSE
    \\\hline
  \end{tabular}
\end{center}


\subsection{CMake variables used to configure programs linking to \cgal}

\begin{center}\index{cgal!exe-config-cmake-vars}\index{cgal configuration cmake variables - user programs }
  \renewcommand{\arraystretch}{1.3}
  \gdef\lcTabularBorder{2}
  \begin{tabular}{|l|l|} \hline
    \textbf{variable} & \textbf{description} 
    \\\hline\hline
    \texttt{CGAL\_DONT\_OVERRIDE\_CMAKE\_FLAGS} & Set this to TRUE to prevent the flags to be overriden with the values used to build the \cgal\ libraries
    \\\hline
  \end{tabular}
\end{center}


\subsection{Environment variables used to configure programs linking to \cgal}

\begin{center}\index{cgal!exe-config-env-vars}\index{cgal configuration environment variables - user programs }
  \renewcommand{\arraystretch}{1.3}
  \gdef\lcTabularBorder{2}
  \begin{tabular}{|l|l|} \hline
    \textbf{variable} & \textbf{description} 
    \\\hline\hline
    \texttt{CGAL\_DIR} & Directory where { \tt CGALConfig.cmake} is located. That is, the binary directory where you configured \cgal
    \\\hline
  \end{tabular}
\end{center}

\section{Example} 

Here is an example output on a Windows machine with VC8 installed, using \cmake\ 2.6,
and the following command-line call to \texttt{cmake}:

{\ccTexHtml{\scriptsize}{}
\begin{verbatim}
cmake -DWITH_examples=false 
      -DWITH_demo=false 
      -DCMAKE_BUILD_TYPE=Release 
      ../../..
\end{verbatim}
}


{\ccTexHtml{\scriptsize}{}
\begin{verbatim}
-- Check for working C compiler: cl
-- Check for working C compiler: cl -- works
-- Detecting C compiler ABI info
-- Detecting C compiler ABI info - done
-- Check for working CXX compiler: cl
-- Check for working CXX compiler: cl -- works
-- Detecting CXX compiler ABI info
-- Detecting CXX compiler ABI info - done
-- CGAL_REFERENCE_CACHE_DIR=
-- Building static libraries
-- Targetting Visual Studio 8 2005
-- Target build enviroment supports auto-linking
-- Using VC80 compiler.
-- Build type: Release
-- Generator uses intermediate configuration directory: $(OutDir)
-- CMake version: 2.6.1
-- System: Windows
-- CGAL_MAJOR_VERSION=3
-- CGAL_MINOR_VERSION=4
-- CGAL_BUILD_VERSION=442
-- CGAL_SONAME_VERSION=3
-- CGAL_SOVERSION     =3.0.0
-- Boost version: 1.35.0
-- Found the following Boost libraries:
--   thread
-- Boost include:      C:/Program Files/boost/boost_1_35_0
-- Boost libraries:    
-- Boost definitions:  
-- USING BOOST_VERSION = '1.35.0'
-- Could NOT find GMP
-- Could NOT find MPFR
-- Performing Test CGAL_CFG_DENORMALS_COMPILE_BUG - Success
-- Performing Test CGAL_CFG_IEEE_754_BUG - Success
-- Performing Test CGAL_CFG_ISTREAM_INT_BUG - Success
-- Performing Test CGAL_CFG_LONGNAME_BUG - Failed
-- Performing Test CGAL_CFG_MATCHING_BUG_5 - Success
-- Performing Test CGAL_CFG_MATCHING_BUG_6 - Failed
-- Performing Test CGAL_CFG_NESTED_CLASS_FRIEND_DECLARATION_BUG - Failed
-- Performing Test CGAL_CFG_NET2003_MATCHING_BUG - Failed
-- Performing Test CGAL_CFG_NO_CPP0X_ARRAY - Failed
-- Performing Test CGAL_CFG_NO_CPP0X_DECLTYPE - Failed
-- Performing Test CGAL_CFG_NO_CPP0X_DEFAULT_TEMPLATE_ARGUMENTS_FOR_FUNCTION_TEMPLATES - Failed
-- Performing Test CGAL_CFG_NO_CPP0X_DELEGATING_CONSTRUCTORS - Failed
-- Performing Test CGAL_CFG_NO_CPP0X_ISFINITE - Failed
-- Performing Test CGAL_CFG_NO_CPP0X_LONG_LONG - Success
-- Performing Test CGAL_CFG_NO_CPP0X_RVALUE_REFERENCE - Failed
-- Performing Test CGAL_CFG_NO_CPP0X_VARIADIC_TEMPLATES - Failed
-- Performing Test CGAL_CFG_NO_LIMITS - Success
-- Performing Test CGAL_CFG_NO_NEXTAFTER - Failed
-- Performing Test CGAL_CFG_NO_STATEMENT_EXPRESSIONS - Failed
-- Performing Test CGAL_CFG_NO_STL - Success
-- Performing Test CGAL_CFG_NO_TMPL_IN_TMPL_PARAM - Success
-- Performing Test CGAL_CFG_NO_TR1_ARRAY - Failed
-- Performing Test CGAL_CFG_NUMERIC_LIMITS_BUG - Success
-- Performing Test CGAL_CFG_OUTOFLINE_MEMBER_DEFINITION_BUG - Success
-- Performing Test CGAL_CFG_TYPENAME_BEFORE_DEFAULT_ARGUMENT_BUG - Failed
-- Performing Test CGAL_CFG_USING_BASE_MEMBER_BUG_2 - Success
-- CMAKE_INSTALL_PREFIX=C:/Program Files/CGAL
-- USING CUSTOM_CXXFLAGS = ' -D_CRT_SECURE_NO_DEPRECATE -D_SCL_SECURE_NO_DEPRECATE -D_CRT_SECURE_NO_WARNINGS -D_SCL_SECURE_NO_WARNINGS   -D_SECURE_SCL=0'
-- USING        CXXFLAGS = ' /DWIN32 /D_WINDOWS /W3 /Zm1000 /EHsc /GR /MD /O2 /Ob2 /D NDEBUG'
-- USING CUSTOM_LDFLAGS = '  '
-- USING        LDFLAGS = ' '
-- USING CUSTOM_EXEFLAGS = '  '
-- USING        EXEFLAGS = ' /MANIFEST /STACK:10000000 /machine:I386 /INCREMENTAL:NO'
-- Configuring CGALCore. Set WITH_CGALCore to FALSE to unselect it.
-- CGAL-core++ needs GMP and MPFR, cannot be configured.
-- Configuring CGALimageIO. Set WITH_CGALimageIO to FALSE to unselect it.
-- Could NOT find ZLIB
-- CGAL-ImageIO needs ZLib and OpenGL, cannot be configured.
-- Configuring CGALPDB. Set WITH_CGALPDB to FALSE to unselect it.
-- Configuring CGALQt. Set WITH_CGALQt to FALSE to unselect it.
-- CGAL-Qt3 needs Qt3, cannot be configured.
-- Configuring CGALQt4. Set WITH_CGALQt4 to FALSE to unselect it.
-- Looking for Q_WS_X11
-- Looking for Q_WS_X11 - not found.
-- Looking for Q_WS_WIN
-- Looking for Q_WS_WIN - found
-- Looking for Q_WS_QWS
-- Looking for Q_WS_QWS - not found.
-- Looking for Q_WS_MAC
-- Looking for Q_WS_MAC - not found.
-- OpenGL include:     
-- OpenGL libraries:   glu32;opengl32
-- OpenGL definitions: 
-- Qt4 include:        C:/Work/Downloaded/Libraries/qt-win-opensource-src-4.4.1/include
-- Qt4 libraries:      optimized;C:/Work/Downloaded/Libraries/qt-win-opensource-src-4.4.1/lib/qtmain.lib;debug;C:/Work/Downloaded/Libraries/qt-win-opensource-src-4.4.1/lib/qtmaind.lib;optimized;C:/Work/Downloaded/Libraries/qt-win-opensource-src-4.4.1/lib/QtOpenGL4.lib;debug;C:/Work/Downloaded/Libraries/qt-win-opensource-src-4.4.1/lib/QtOpenGLd4.lib;opengl32.lib glu32.lib gdi32.lib user32.lib;optimized;C:/Work/Downloaded/Libraries/qt-win-opensource-src-4.4.1/lib/QtGui4.lib;debug;C:/Work/Downloaded/Libraries/qt-win-opensource-src-4.4.1/lib/QtGuid4.lib;imm32;winmm;optimized;C:/Work/Downloaded/Libraries/qt-win-opensource-src-4.4.1/lib/QtCore4.lib;debug;C:/Work/Downloaded/Libraries/qt-win-opensource-src-4.4.1/lib/QtCored4.lib;ws2_32
-- Qt4 definitions:    -DQT_DLL
-- moc executable:     C:/Work/Downloaded/Libraries/qt-win-opensource-src-4.4.1/bin/moc.exe
-- uic executable:     C:/Work/Downloaded/Libraries/qt-win-opensource-src-4.4.1/bin/uic.exe
-- Configuring done
-- Generating done
-- Build files have been written to: C:/Work/Active/GeometryFactory/CGAL/Autotest/CGAL-3.4-I-442/cmake/platforms/a
\end{verbatim}
}


%%
%% EOF
%%


