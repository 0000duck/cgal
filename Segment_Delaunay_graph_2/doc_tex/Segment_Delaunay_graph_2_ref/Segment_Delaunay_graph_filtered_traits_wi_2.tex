%% Copyright (c) 2003,2004,2005  INRIA Sophia-Antipolis (France) and
%% Notre Dame University (U.S.A.).  All rights reserved.
%%
%% This file is part of CGAL (www.cgal.org); you may redistribute it under
%% the terms of the Q Public License version 1.0.
%% See the file LICENSE.QPL distributed with CGAL.
%%
%% Licensees holding a valid commercial license may use this file in
%% accordance with the commercial license agreement provided with the software.
%%
%% This file is provided AS IS with NO WARRANTY OF ANY KIND, INCLUDING THE
%% WARRANTY OF DESIGN, MERCHANTABILITY AND FITNESS FOR A PARTICULAR PURPOSE.
%%
%% $Source$
%% $Revision$ $Date$
%% $Name$
%%
%% Author(s)     : Menelaos Karavelas <mkaravel@cse.nd.edu>



\begin{ccRefClass}
{Segment_Delaunay_graph_filtered_traits_without_intersections_2<CK,CM,EK,EM,FK,FM>}
%% add template arg's if necessary

%% \ccHtmlCrossLink{}     %% add further rules for cross referencing links
%% \ccHtmlIndexC[class]{} %% add further index entries

\ccDefinition
  
The class \ccRefName\ provides a model for the
\ccc{SegmentDelaunayGraphTraits_2} concept.

The class \ccRefName\ uses the filtering technique \cite{cgal:bbp-iayed-01}
to achieve traits for the \ccc{Segment_Delaunay_graph_2<Gt,DS>}
class with efficient and exact predicates given an exact
kernel \ccc{EK} and a filtering kernel \ccc{FK}. The geometric
constructions associated provided by this class are equivalent
to those provided by the traits class
\ccc{Segment_Delaunay_graph_traits_without_intersections_2<CK,CM>},
which means that they may be inexact, depending on the choice of the
\ccc{CK} kernel.

This class has six template parameters. The first, third and fifth
template parameters must be a models of the \ccc{Kernel} concept. The
parameter \ccc{CK} is the construction kernel and it is the kernel
that will be used for constructions. The parameter \ccc{FK} is the
filtering kernel; this kernel will be used for performing the
arithmetic filtering for the predicates involved in the computation of
the segment Delaunay graph. Finally, the parameter \ccc{EK} is the
exact kernel; this kernel will be used for computing the predicates if
the filtering kernel fails to produce an answer.

The second, fourth and sixth template parameters
correspond to how predicates are evaluated. There are two predefined
possible values for these parameters, namely \ccc{CGAL::Sqrt_field_tag}
and \ccc{CGAL::Ring_tag}. The first one must be used when the number
type used in the representation supports the exact evaluation of signs
of expressions involving all four basic operations and square roots,
whereas the second requires the exact evaluation of signs of ring-type
expressions, i.e., expressions involving only additions, subtractions
and multiplications. Finally, in order to get exact constructions
\ccc{CM} must be set to \ccc{CGAL::Sqrt_field_tag} and the number type
in \ccc{CK} must support operations involing divisions and square
roots (as well as the other three basic operations of course).
%
The way the predicates are evaluated is discussed in
\cite{b-ecvdl-96} and \cite{cgal:k-reisv-04} (the geometric filtering
part).

The default values for the template parameters are as follows:
\ccc{CM = CGAL::Sqrt_field_tag} (it is assumed that
\ccc{CGAL::Cartesian<double>} or \ccc{CGAL::Simple_cartesian<double>}
will be the entry for the template parameter \ccc{CK}), 
\ccc{EM = CGAL::Ring_tag},
\ccc{FK = CGAL::Simple_cartesian<CGAL::Interval_nt<false> >},
\ccc{FM = CGAL::Sqrt_field_tag}. If the \textsc{Gmp} package is
installed with \cgal, the template parameter \ccc{EK} has the default
value: \ccc{EK = CGAL::Simple_cartesian<CGAL::Gmpq>}, otherwise its
default value is
\ccc{EK = CGAL::Simple_cartesian<CGAL::MP_Float>}.


\ccInclude{CGAL/Segment_Delaunay_graph_filtered_traits_2.h}

\ccIsModel
\ccc{SegmentDelaunayGraphTraits_2}\\
\ccc{DefaultConstructible}\\
\ccc{CopyConstructible}\\
\ccc{Assignable}

\ccTypes

\ccThree{typedef CGAL::Tag_false}{Intersections_tag;}{}
\ccTypedef{typedef CGAL::Tag_false Intersections_tag;}{}

In addition to the types required by the
\ccc{SegmentDelaunayGraphTraits_2} concept the class \ccRefName\
defines the following types:

\ccTypedef{typedef CK  Kernel;}{}
\ccGlue
\ccTypedef{typedef CK  Construction_kernel;}{}
\ccGlue
\ccTypedef{typedef FK  Filtering_kernel;}{}
\ccGlue
\ccTypedef{typedef EK  Exact_kernel;}{}
\ccGlue
\ccTypedef{typedef CM  Method_tag;}{}
\ccGlue
\ccTypedef{typedef CM  Construction_traits_method_tag;}{}
\ccGlue
\ccTypedef{typedef FM  Filtering_traits_method_tag;}{}
\ccGlue
\ccTypedef{typedef EM  Exact_traits_method_tag;}{}
\ccGlue
\ccNestedType{Construction_traits;}{A type for the segment Delaunay
  graph traits, where the kernel is \ccc{CK}.}
\ccGlue
\ccNestedType{Filtering_traits;}{A type for the segment Delaunay
  graph traits, where the kernel is \ccc{FK}.}
\ccGlue
\ccNestedType{Exact_traits;}{A type for the segment Delaunay
  graph traits, where the kernel is \ccc{EK}.}



\ccSeeAlso
\ccc{Kernel}\\
\ccc{SegmentDelaunayGraphTraits_2} \\
\ccc{CGAL::Ring_tag}\\
\ccc{CGAL::Sqrt_field_tag}\\
\ccc{CGAL::Segment_Delaunay_graph_2<Gt,DS>}\\
\ccc{CGAL::Segment_Delaunay_graph_hierarchy_2<Gt,STag,DS>}\\
\ccc{CGAL::Segment_Delaunay_graph_traits_2<K,MTag>}\\
\ccc{CGAL::Segment_Delaunay_graph_traits_without_intersections_2<K,MTag>}\\
\ccc{CGAL::Segment_Delaunay_graph_filtered_traits_2<CK,CM,EK,EM,FK,FM>}

\end{ccRefClass}

% +------------------------------------------------------------------------+
%%RefPage: end of main body, begin of footer
% EOF
% +------------------------------------------------------------------------+

