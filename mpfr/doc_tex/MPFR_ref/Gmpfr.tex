\begin{ccRefClass} {Gmpfr}
\label{Gmpfr}

\ccDefinition

An object of the class \ccStyle{Gmpfr} is a fixed precision
floating-point number, based on the MPFR library.

\ccInclude{CGAL/Gmpfr.h}

\ccIsModel
\ccc{FieldWithKthRoot}\\
\ccc{RealEmbeddable}

\ccTypes

\ccNestedType{NT}{the field type, which is \ccc{Gmpfr}.}

\ccCreation
\ccCreationVariable{f}

\ccConstructor{Gmpfr();}
	{creates an uninitialized MPFR fixed precision floating point number
	\ccVar.}

\ccHidden \ccConstructor{Gmpfr(const Gmpfr& n);}
	{copy constructor.}

\ccConstructor{Gmpfr(mpfr_srcptr p);}
	{creates a MPFR number, initialized with the value of \ccStyle{p}.}

\ccConstructor{Gmpfr(long si);}
	{creates a MPFR number, initialized with the value of \ccStyle{si}.}

\ccConstructor{Gmpfr(unsigned long ui);}
	{creates a MPFR number, initialized with the value of \ccStyle{ui}.}

\ccConstructor{Gmpfr(int i);}
	{creates a MPFR number, initialized with the value of \ccStyle{i}.}

\ccConstructor{Gmpfr(double d);}
	{creates a MPFR number, initialized with the value of \ccStyle{d}.}

\ccConstructor{Gmpfr(long double ld);}
	{creates a MPFR number, initialized with the value of \ccStyle{ld}.}

\ccConstructor{Gmpfr(const Gmpz &z);}
	{creates a MPFR number, initialized with the value of \ccStyle{z}.}

\ccConstructor{Gmpfr(const Gmpq &q);}
	{creates a MPFR number, initialized with the value of \ccStyle{q}.}

Note that all constructors, except the first one, have two other ways
of be called.
One can optionally specify as second parameter the rounding mode desired
for the conversion from the source number and as a third parameter the
precision with which this MPFR number will be created.
The last calling possibility consists in only specifying the desired
precision.

These optional parameters, along with other functions which will be
explained below, allow users to gain control over rounding and precision,
two of the main characteristics of MPFR.


\ccOperations


%%%%%%%%%%%%%%%%%%%%%%%%%%%%%%%%%%%%%%%%%%%%%%%%%%
% precision and rounding modes
%%%%%%%%%%%%%%%%%%%%%%%%%%%%%%%%%%%%%%%%%%%%%%%%%%

\ccFunction{static mp_rnd_t get_defrnd();}
	{This function returns the current rounding mode used by MPFR.}

\ccFunction{static void set_defrnd(mp_rnd_t r);}
	{This function sets the MPFR rounding mode to r.}

\ccFunction{static mp_prec_t get_defprec();}
	{This returns the current precision used in MPFR creation
	by default.}

\ccFunction{static void set_defprec(mp_prec_t p);}
	{This function sets the default MPFR precision to p.}

\ccFunction{mp_prec_t get_prec()const;}
	{}

\ccMethod{mp_prec_t get_prec()const;}
	{Returns the precision of \ccVar.}

\ccMethod{void set_prec(mp_prec_t);}
	{Sets the precision of \ccVar.
	Note that with this function, the previous value of \ccVar is lost.}

\ccMethod{void prec_round(mp_prec_t p, mp_rnd_t r);}
	{Sets the precision of \ccVar to p, using the rounding mode r
	for rounding.}


%%%%%%%%%%%%%%%%%%%%%%%%%%%%%%%%%%%%%%%%%%%%%%%%%%
%% TODO: add here an advanced section for inexact flags handling functions
%%%%%%%%%%%%%%%%%%%%%%%%%%%%%%%%%%%%%%%%%%%%%%%%%%


%%%%%%%%%%%%%%%%%%%%%%%%%%%%%%%%%%%%%%%%%%%%%%%%%%
% arithmetic operators
%%%%%%%%%%%%%%%%%%%%%%%%%%%%%%%%%%%%%%%%%%%%%%%%%%

Arithmetic operators +, -, * and / are overloaded, but special care must
be taken when applying them, because the user can't directly handle the
precision with them.
The only way to specify the precision of the result is to change the
default MPFR precision.


%%%%%%%%%%%%%%%%%%%%%%%%%%%%%%%%%%%%%%%%%%%%%%%%%%
% TODO: comparisons
%%%%%%%%%%%%%%%%%%%%%%%%%%%%%%%%%%%%%%%%%%%%%%%%%%



%%%%%%%%%%%%%%%%%%%%%%%%%%%%%%%%%%%%%%%%%%%%%%%%%%
% input/output
%%%%%%%%%%%%%%%%%%%%%%%%%%%%%%%%%%%%%%%%%%%%%%%%%%

\ccFunction{std::istream& operator>>(std::istream& in, Gmpfr& f);}
	{reads a number from \ccc{in}, then converts it to a \ccc{Gmpfr}.}

\ccFunction{std::ostream& operator<<(std::ostream& out, const Gmpfr& f);}
	{writes \ccc{f} to the ostream \ccc{out}, in the form
	\(.MeE\), where \(M\) is the mantissa and \(E\) is the exponent.}


%--------------------------------------------------
% \ccImplementation
% \ccc{Gmpfr}s are reference counted.
%-------------------------------------------------- 

\end{ccRefClass} 
\ccDefGlobalScope{CGAL::}
