% +------------------------------------------------------------------------+
% | Reference manual page: PointContainer.tex
% +------------------------------------------------------------------------+
% | 1.07.2001   Johan W.H. Tangelder
% | Package: ASPAS
% | 
\RCSdef{\RCSPointContainerRev}{$Id$}
\RCSdefDate{\RCSPointContainerDate}{$Date$}
% |
%%RefPage: end of header, begin of main body
% +------------------------------------------------------------------------+


\begin{ccRefConcept}{PointContainer}

%% \ccHtmlCrossLink{}     %% add further rules for cross referencing links
%% \ccHtmlIndexC[concept]{} %% add further index entries

\begin{ccAdvanced}

\ccDefinition
  
The concept \ccRefName\ defines the requirements for
a point container. 


\ccTypes

\ccNestedType{Point_d} { A $d$-dimensional point.}
\ccCreation
\ccCreationVariable{c}  %% choose variable name

\ccConstructor{PointContainer(int d);}
{
Construct an empty container for storing \ccc{d}-dimensional points.
}

\ccConstructor{
template <class InputIterator>
PointContainer(int d, InputIterator begin, InputIterator end);}
{
Construct the container of $d$-dimensional points of type \ccc{Point_d}
given by the iterator sequence \ccc{[begin,end)}.
}

\ccOperations

\ccMethod{template <class SpatialSeparator>
void split(PointContainer& c2, SpatialSeparator sep, bool sliding=false);}
{Given an empty container \ccc{c2} with the same dimension as \ccc{c}, splits \ccVar\ into
\ccVar\ and \ccc{c2} using the separator \ccc{sep}. If sliding is \ccc{true}, after splitting 
each container contains at least one point. Container \ccVar\ should contain at least two points.}

\ccHasModels

\ccc{CGAL::Point_container<SpatialPoint>}.

\ccSeeAlso

\ccc{SearchTraits}\\
\ccc{SpatialSeparator}\\
\ccc{CGAL::Point_container<SearchTraits>}

\end{ccAdvanced}
\end{ccRefConcept}

% +------------------------------------------------------------------------+
%%RefPage: end of main body, begin of footer
% EOF
% +------------------------------------------------------------------------+

