% +------------------------------------------------------------------------+
% | Reference manual page: Kd_tree_node.tex
% +------------------------------------------------------------------------+
% | 1.07.2001   Johan W.H. Tangelder
% | Package: ASPAS
% | 
\RCSdef{\RCSKdtreenodeRev}{$Revision$}
\RCSdefDate{\RCSKdtreenodeDate}{$Date$}
% |
%%RefPage: end of header, begin of main body
% +------------------------------------------------------------------------+


\begin{ccRefClass}{Kd_tree_node<Traits, Splitter, UseExtendedNode>}  %% add template arg's if necessary

%% \ccHtmlCrossLink{}     %% add further rules for cross referencing links
%% \ccHtmlIndexC[class]{} %% add further index entries

\begin{ccAdvanced}

\ccDefinition
  
The class \ccRefName\ implements a node class for a $k$-$d$ tree. 
A node is either a leaf node, an internal node or an extended internal node.
A leaf node contains one or more points. An internal node contains a pointer
to its lower child, a pointer to its upper child, and a pointer to its separator.
An extended internal node is an internal node containing the lower and 
upper limit of an extended node's rectangle
along the node's cutting dimension.

\ccInclude{CGAL/Kd_tree_node.h}

\ccHeading{Parameters}

Expects for the template argument a model of the concept \ccc{SearchTraits},
for example \ccc{CGAL::Search_traits_2<CGAL::Cartesian<double> >}, or \ccc{CGAL::Cartesian_d<double>}.

\ccTypes

\ccEnum{ enum Node_type { LEAF, INTERNAL, EXTENDED_INTERNAL};}{Denotes type of node.}

\ccTypedef{Traits::FT FT;}{Number type.}
\ccTypedef{Traits::Point_d Point_d;}{Point type.}
\ccTypedef{Splitter::Separator Separator;}{Separator type.}
\ccTypedef{Kd_tree<Traits,Splitter,UseExtendedNode>::Point_d_iterator Point_d_iterator;}{Iterator over points.} 
\ccTypedef{Kd_tree<Traits,Splitter,UseExtendedNode>::Node_handle Node_handle;}{Node handle.}


\ccCreation
\ccCreationVariable{n}  %% choose variable name



\ccOperations

\ccMethod{
template <class OutputIterator, class FuzzyQueryItem>
OutputIterator search(OutputIterator it, FuzzyQueryItem q);}
{Reports the points from the subtree of the node, that are approximately contained by q.} 

\ccMethod{
template <class OutputIterator>
OutputIterator tree_points(OutputIterator it);}
{Reports all the points contained by the subtree of the node.} 

\ccMethod{bool is_leaf();}{Indicates whether a node is a leaf node.}
\ccMethod{int  size();}{Returns the number of items stored in a leaf node.}
\ccMethod{Point_d_iterator begin();}{Returns the iterator to the first item in a leaf node.}
\ccMethod{Point_d_iterator end();}{Returns the past-the-end iterator in a leaf node.}
\ccMethod{Node_handle lower();}{Returns a handle to the lower child of an internal node.}
\ccMethod{Node_handle upper();}{Returns a handle to the upper child of an internal node.}
\ccMethod{Separator& separator();}{Returns a reference to the separator.}
\ccMethod{FT low_val();}{Returns the lower limit of an exteneded node's rectangle
along the node's cutting dimension.}
\ccMethod{FT high_val();}{Returns the upper limit of an extended node's rectangle
along the node's cutting dimension.}


\end{ccAdvanced}
\end{ccRefClass}

% +------------------------------------------------------------------------+
%%RefPage: end of main body, begin of footer
% EOF
% +------------------------------------------------------------------------+

