% +------------------------------------------------------------------------+
% | Reference manual page: IsoBox_d.tex
% +------------------------------------------------------------------------+
% | 1.07.2001   Johan W.H. Tangelder
% | Package: ASPAS
% | 
\RCSdef{\RCSIsoBoxdRev}{$Id$}
\RCSdefDate{\RCSIsoBoxdDate}{$Date$}
% |
%%RefPage: end of header, begin of main body
% +------------------------------------------------------------------------+


\begin{ccRefConcept}{IsoBox}

%% \ccHtmlCrossLink{}     %% add further rules for cross referencing links
%% \ccHtmlIndexC[concept]{} %% add further index entries

\ccDefinition
  
The concept \ccRefName\ describes the requirements for iso-boxes.


\ccTypes

\ccNestedType{Point_d;}{Point type.} 

\ccCreation
\ccCreationVariable{b}  %% choose variable name

\ccConstructor{IsoBox_d(Point_d p, Point_d q);}{introduces an iso-oriented box with diagonal opposite points 
\ccc{p} and \ccc{q}.}

\ccOperations

\ccMethod{Point_d min();}
{Returns the lexicographically smallest vertex of the iso-oriented box.}


\ccMethod{Point_d max();}
{Returns the  lexicographically largest vertex of the iso-oriented box.}


\ccHasModels

\ccc{Kernel::Iso_rectangle_2},\\
\ccc{Kernel::Iso_cuboid_3},\\
\ccc{Kernel_d::Iso_box_d}.

\end{ccRefConcept}

% +------------------------------------------------------------------------+
%%RefPage: end of main body, begin of footer
% EOF
% +------------------------------------------------------------------------+

