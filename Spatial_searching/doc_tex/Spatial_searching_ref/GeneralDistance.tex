% +------------------------------------------------------------------------+
% | Reference manual page: GeneralDistance.tex
% +------------------------------------------------------------------------+
% | 1.07.2001   Johan W.H. Tangelder
% | Package: ASPAS
% |
\RCSdef{\RCSGeneralDistanceRev}{$Id$}
\RCSdefDate{\RCSGeneralDistanceDate}{$Date$}
% |
%%RefPage: end of header, begin of main body
% +------------------------------------------------------------------------+


\begin{ccRefConcept}{GeneralDistance}

%% \ccHtmlCrossLink{}     %% add further rules for cross referencing links
%% \ccHtmlIndexC[concept]{} %% add further index entries

\ccDefinition

Requirements of a distance class defining a distance between a query item
denoting a spatial object and a point.
To optimize distance computations transformed distances are used,
e.g., for a Euclidean distance the transformed distance is the squared
Euclidean distance.

\ccHasModels

\ccc{CGAL::Manhattan_distance_iso_box_point<Traits>}\\
\ccc{CGAL::Euclidean_distance_sphere_point<Traits>}.


\ccTypes

\ccNestedType{FT}{Number type.}
\ccNestedType{Point_d}{Point type.}
\ccNestedType{Query_item}{Query item type.}

\ccCreationVariable{gd}  %% choose variable name

\ccOperations

\ccMethod{FT transformed_distance(Query_item q, Point_d r);}{Returns the transformed distance between \ccc{q} and~\ccc{r}.}

\ccMethod{FT min_distance_to_rectangle(Query_item q, Kd_tree_rectangle<FT> r) const;}
{Returns the transformed distance between \ccc{q} and
the point on the boundary of \ccc{r} closest to \ccc{q}.}

\ccMethod{FT max_distance_to_rectangle(Query_item q, Kd_tree_rectangle<FT> r) const;}
{Returns the transformed distance between \ccc{q} and
the point on the boundary of \ccc{r} furthest to \ccc{q}.}

\ccMethod{FT transformed_distance(FT d) const;} {Returns the transformed distance.}

\ccMethod{FT inverse_of_transformed_distance(FT d) const;} {Returns the inverse of the transformed distance.}




\end{ccRefConcept}

% +------------------------------------------------------------------------+
%%RefPage: end of main body, begin of footer
% EOF
% +------------------------------------------------------------------------+

