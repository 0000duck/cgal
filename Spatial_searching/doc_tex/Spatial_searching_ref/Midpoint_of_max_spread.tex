% +------------------------------------------------------------------------+
% | Reference manual page: Midpoint_of_max_spread.tex
% +------------------------------------------------------------------------+
% | 1.07.2001   Johan W.H. Tangelder
% | Package: ASPAS
% | 
\RCSdef{\RCSMidpointofmaxspreadRev}{$Id$}
\RCSdefDate{\RCSMidpointofmaxspreadDate}{$Date$}
% |
%%RefPage: end of header, begin of main body
% +------------------------------------------------------------------------+


\begin{ccRefFunctionObjectClass}{Midpoint_of_max_spread<Traits, SpatialSeparator>}

%% \ccHtmlCrossLink{}     %% add further rules for cross referencing links
%% \ccHtmlIndexC[class]{} %% add further index entries

\ccDefinition
Implements the {\em midpoint of max spread} splitting rule.
A rectangle is cut through $(Mind+Maxd)/2$ orthogonal
to the dimension with the maximum point spread $[Mind,Maxd]$.

\ccInclude{CGAL/Splitters.h}

\ccParameters

Expects for the first template argument a model of
the concept \ccc{SearchTraits}, for example
the type \ccc{CGAL::Search_traits_3< Cartesian<double> >}. 

Expects for the second template argument a model of the concept \ccc{Separator}. It has as default value
the type, \ccc{CGAL::Plane_separator<Traits::FT>}


\ccIsModel

\ccc{Splitter}


\ccCreation
\ccCreationVariable{s}  %% choose variable name

\ccConstructor{Midpoint_of_max_spread();}{Default constructor.}
\ccConstructor{Midpoint_of_max_spread(unsigned int bucket_size);}{Constructor.}

\ccSeeAlso

\ccc{Splitter},\\
\ccc{SpatialSeparator}

\end{ccRefFunctionObjectClass}

% +------------------------------------------------------------------------+
%%RefPage: end of main body, begin of footer
% EOF
% +------------------------------------------------------------------------+

