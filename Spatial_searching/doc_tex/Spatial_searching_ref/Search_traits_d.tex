% +------------------------------------------------------------------------+
% | Reference manual page: Search_traits_d.tex
% +------------------------------------------------------------------------+
% | 1.07.2001   Johan W.H. Tangelder
% | Package: ASPAS
% | 
\RCSdef{\RCSKdtreetraitspoint2Rev}{$Id$}
\RCSdefDate{\RCSKdtreetraitspoint2Date}{$Date$}
% |
%%RefPage: end of header, begin of main body
% +------------------------------------------------------------------------+


\begin{ccRefClass}{Search_traits_d<Kernel>}

%% \ccHtmlCrossLink{}     %% add further rules for cross referencing links
%% \ccHtmlIndexC[class]{} %% add further index entries

\ccDefinition
  
The class \ccRefName\ can be used as a template parameter of the kd tree
and the search classes.  \ccc{Kernel} must be a \cgal\ kernel.


\ccc{Kernel} must be a d-dimensional \cgal\ kernel.


\ccInclude{CGAL/Search_traits_d.h}

\ccParameters
Expects for the template argument a model of the concept \ccc{Kernel_d},
for example \ccc{CGAL::Cartesian_d<double>} or \ccc{CGAL::Homogeneous_d<CGAL::Gmp_z>}.


\ccIsModel

\ccc{SearchTraits} and \ccc{RangeSearchTraits}.

\ccTypes

\ccTypedef{Kernel::FT NT;}{Number type.}

\ccTypedef {Kernel::Point_d Point_d;}{Point type.}
\ccTypedef {Kernel::Iso_box_d Iso_box_d;}{Iso box type.}
\ccTypedef {Kernel::Sphere_d Sphere_d;}{Sphere type.}

\ccTypedef {Kernel::Cartesian_const_iterator_d Cartesian_const_iterator;}{An iterator over the Cartesian 
	coordinates.}

\ccTypedef {Kernel::Construct_cartesian_const_iterator_d Construct_cartesian_const_iterator;}{A functor with
two function operators, which return the begin and past the end iterator for the Cartesian coordinates. 
The functor for begin has as argument a \ccc{Point_d}. The functor for the past the end iterator, 
has as argument a \ccc{Point_d} and an \ccc{int}.}


\ccTypedef {Kernel::Construct_min_vertex_d Construct_min_vertex_d;}{Functor with operator to construct 
the vertex with lexicographically smallest coordinates of an object of type \ccc{Iso_box_d}.}

\ccTypedef {Kernel::Construct_max_vertex_d Construct_max_vertex_d;}{Functor with operator to construct 
the vertex with lexicographically largest coordinates of an object of type \ccc{Iso_box_d}.}

\ccSeeAlso

\ccc{Search_traits_2<Kernel>}\\
\ccc{Search_traits_3<Kernel>}\\
\ccc{Search_traits<Point,CartesianConstIterator,ConstructCartesianConstIterator>}

\end{ccRefClass}


% +------------------------------------------------------------------------+
%%RefPage: end of main body, begin of footer
% EOF
% +------------------------------------------------------------------------+

