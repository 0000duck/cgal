% +------------------------------------------------------------------------+
% | Reference manual page: SpatialSeparator.tex
% +------------------------------------------------------------------------+
% | 1.07.2001   Johan W.H. Tangelder
% | Package: ASPAS
% | 
\RCSdef{\RCSSpatialSeparatorRev}{$Id$}
\RCSdefDate{\RCSSpatialSeparatorDate}{$Date$}
% |
%%RefPage: end of header, begin of main body
% +------------------------------------------------------------------------+


\begin{ccRefConcept}{SpatialSeparator}

\begin{ccAdvanced}

%% \ccHtmlCrossLink{}     %% add further rules for cross referencing links
%% \ccHtmlIndexC[concept]{} %% add further index entries

\ccDefinition
  
The concept \ccRefName\ defines the requirements for a separator.
A separator is a ($d$-1)-dimensional subspace that separates a $d$-dimensional space into two parts.
One part of space is said to be on the negative side of the separator and the other part of space is 
said to be on the positive side of the separator.

\ccHasModels

\ccc{CGAL::Plane_separator<FT>}.

\ccTypes

\ccNestedType{FT;}{Number type.}

\ccCreation
\ccCreationVariable{s}  %% choose variable name

\ccConstructor{Separator();}{Default constructor.}

\ccOperations

\ccMethod{
void set_cutting_dimension(int d);}
{Sets the cutting dimension to \ccc{d}. }

\ccMethod{
void set_cutting_value(FT v);}
{Sets the cutting value to \ccc{v}. }

\ccMethod{int cutting_dimension();}{Returns the number of the cutting dimension.}

\ccMethod{FT cutting_value();}{Returns the cutting value.}



\ccMethod{template <class Point_d>
bool has_on_negative_side(Point_d p);}
{Returns true if and only if the point \ccc{p} is on the negative side of the separator.}



\end{ccAdvanced}

\end{ccRefConcept}

% +------------------------------------------------------------------------+
%%RefPage: end of main body, begin of footer
% EOF
% +------------------------------------------------------------------------+

