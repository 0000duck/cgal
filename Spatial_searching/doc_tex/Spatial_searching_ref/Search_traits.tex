% +------------------------------------------------------------------------+
% | Reference manual page: Search_traits.tex
% +------------------------------------------------------------------------+
% | 1.07.2001   Johan W.H. Tangelder
% | Package: ASPAS
% | 
\RCSdef{\RCSKdtreetraitspointRev}{$Id$}
\RCSdefDate{\RCSKdtreetraitspointDate}{$Date$}
% |
%%RefPage: end of header, begin of main body
% +------------------------------------------------------------------------+


\begin{ccRefClass}{Search_traits<NT,Point,CartesianIterator,ConstructCartesianIterator,ConstructMinVertex,ConstructMaxVertex>}

%% \ccHtmlCrossLink{}     %% add further rules for cross referencing links
%% \ccHtmlIndexC[class]{} %% add further index entries

\ccDefinition
  
The class \ccRefName\ can be used as a template parameter of the kd tree
and the search classes. It is a mere wrapper for the geometric types needed
by these classes.


\ccInclude{CGAL/Search_traits.h}

\ccIsModel

\ccc{SearchTraits}.


\ccTypes
\ccTypedef{NT FT;}{The number type of the coordinates.}
\ccTypedef{Point Point_d;}{Point type.}
\ccTypedef{CartesianIterator Cartesian_const_iterator_d;}{An iterator over the coordinates.}
\ccTypedef{ConstructCartesianIterator Construct_Cartesian_const_iterator_d;}{A functor with
two function operators, which return the begin and past the end iterator for the Cartesian coordinates. 
The functor for begin has as argument a \ccc{Point_d}. The functor for the past the end iterator, 
has as argument a \ccc{Point_d} and an \ccc{int}.}

\ccTypedef {ConstructMinVertex Construct_min_vertex_d;}{Functor with operator to construct 
the vertex with lexicographically smallest coordinates of an object of type \ccc{Iso_box_d}.}

\ccTypedef {Kernel::ConstructMaxVertex Construct_max_vertex_d;}{Functor with operator to construct 
the vertex with lexicographically largest coordinates of an object of type \ccc{Iso_box_d}.}

\ccSeeAlso

\ccc{Search_traits_2<Kernel>}\\
\ccc{Search_traits_3<Kernel>}\\
\ccc{Search_traits_d<Kernel>}

\end{ccRefClass}


% +------------------------------------------------------------------------+
%%RefPage: end of main body, begin of footer
% EOF
% +------------------------------------------------------------------------+

