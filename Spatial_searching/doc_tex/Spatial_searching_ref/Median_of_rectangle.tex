% +------------------------------------------------------------------------+
% | Reference manual page: Median_of_rectangle.tex
% +------------------------------------------------------------------------+
% | 1.07.2001   Johan W.H. Tangelder
% | Package: ASPAS
% | 
\RCSdef{\RCSMedianofrectangleRev}{$Id$}
\RCSdefDate{\RCSMedianofrectangleDate}{$Date$}
% |
%%RefPage: end of header, begin of main body
% +------------------------------------------------------------------------+

\begin{ccRefFunctionObjectClass}{Median_of_rectangle<Traits, SpatialSeparator>}

\ccDefinition
Implements the {\em median of rectangle} splitting rule.
The splitting dimension is the dimension of the longest side of the rectangle.
The splitting value is defined by the median of the coordinates of the data points
along this dimension.

\ccInclude{CGAL/Splitters.h}

\ccParameters

Expects for the first template argument a model of
the concept \ccc{SearchTraits}, for example
the type \ccc{CGAL::Search_traits_3< Cartesian<double> >}.


Expects for the second template argument a model of the concept \ccc{Separator}. It has as default value
the type, \ccc{CGAL::Plane_separator<Traits::FT>}.


\ccIsModel

\ccc{Splitter}

\ccCreation
\ccCreationVariable{s}  %% choose variable name

\ccConstructor{Median_of_rectangle();}{Default constructor.}
\ccConstructor{Median_of_rectangle(unsigned int bucket_size);}{Constructor.}

\ccSeeAlso

\ccc{Splitter},\\
\ccc{SpatialSeparator}

\end{ccRefFunctionObjectClass}



% +------------------------------------------------------------------------+
%%RefPage: end of main body, begin of footer
% EOF
% +------------------------------------------------------------------------+

