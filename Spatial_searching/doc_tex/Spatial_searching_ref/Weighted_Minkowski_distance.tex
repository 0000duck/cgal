% +------------------------------------------------------------------------+
% | Reference manual page: Weighted_Minkowski_distance.tex
% +------------------------------------------------------------------------+
% | 1.07.2001   Johan W.H. Tangelder
% | Package: ASPAS
% | 
\RCSdef{\RCSWeightedMinkowskidistanceRev}{$Id$}
\RCSdefDate{\RCSWeightedMinkowskidistanceDate}{$Date$}
% |
%%RefPage: end of header, begin of main body
% +------------------------------------------------------------------------+


\begin{ccRefClass}{Weighted_Minkowski_distance<Traits>}  %% add template arg's if necessary

%% \ccHtmlCrossLink{}     %% add further rules for cross referencing links
%% \ccHtmlIndexC[class]{} %% add further index entries

\ccDefinition
   
The class \ccRefName\ provides an implementation of the concept \ccc{OrthogonalDistance}, with a weighted 
Minkowski metric on $d$-dimensional points
defined by $l_p(w)(r,q)= ({\Sigma_{i=1}^{i=d} \, w_i(r_i-q_i)^p})^{1/p}$ for $0 < p <\infty$ and
defined by $l_{\infty}(w)(r,q)=max \{w_i |r_i-q_i| \mid 1 \leq i \leq d\}$.
For the purpose of the distance computations it is more efficient to compute
the transformed distance ${\sigma_{i=1}^{i=d} \, w_i(r_i-q_i)^p}$ instead of the actual distance.

\ccInclude{CGAL/Weighted_Minkowski_distance.h}

\ccParameters

Expects for the template argument a model of the concept
\ccc{SearchTraits}, for example \ccc{CGAL::Search_traits_2<Kernel>}.

\ccIsModel

\ccc{OrthogonalDistance}

\ccTypes

\ccTypedef{Traits::FT FT;}{Number type.}
\ccTypedef{Traits::Point_d Point_d;}{Point type.}

 

\ccCreation
\ccCreationVariable{wd}  %% choose variable name

\ccConstructor{Weighted_Minkowski_distance(int d,Traits t=Traits());}{Constructor implementing $l_2$ metric for $d$-dimensional points.}

\ccConstructor{template <class InputIterator>
               Weighted_Minkowski_distance(FT power, int dim, InputIterator wb, InputIterator we,Traits t=Traits());}
{Constructor implementing the $l_{power}(weights)$ metric. $power \leq 0$ denotes the $l_{\infty}(weights)$ metric.
The values in the iterator range \ccc{[wb,we)} are the weight. }

\ccOperations


\ccMethod{FT transformed_distance(Point_d q, Point_d r) const;}{Returns $d^{power}$, 
where $d$ denotes the distance between \ccc{q} and~\ccc{r}.}

\ccMethod{FT min_distance_to_rectangle(Point_d q, Kd_tree_rectangle<FT> r;) const;}
{Returns $d^{power}$, where $d$ denotes the distance between the query item \ccc{q} and
the point on the boundary of \ccc{r} closest to \ccc{q}.}

\ccMethod{FT max_distance_to_rectangle(Point_d q, Kd_tree_rectangle<FT> r;) const;}
{Returns $d^{power}$, where $d$ denotes the distance between the query item \ccc{q} and
the point on the boundary of \ccc{r} farthest to \ccc{q}.}

\ccMethod{FT new_distance(FT dist, FT old_off, FT new_off, int cutting_dimension) const;} {Updates \ccc{dist} incrementally
and returns the updated distance.}

\ccMethod{FT transformed_distance(FT d) const;} {Returns $d^p$ for $0 < p <\infty$ . Returns $d$ for $p=\infty$ .}

\ccMethod{FT inverse_of_transformed_distance(FT d) const;} {Returns $d^{1/p}$ for $0 < p <\infty$.
Returns $d$ for $p=\infty$.}
 

\ccSeeAlso

\ccc{OrthogonalDistance}\\
\ccc{CGAL::Euclidean_distance<Traits>}

\end{ccRefClass}

% +------------------------------------------------------------------------+
%%RefPage: end of main body, begin of footer
% EOF
% +------------------------------------------------------------------------+

