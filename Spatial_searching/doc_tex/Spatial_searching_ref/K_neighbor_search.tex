% +------------------------------------------------------------------------+
% | Reference manual page: K_neighbor_search.tex
% +------------------------------------------------------------------------+
% | 1.07.2001   Johan W.H. Tangelder
% | Package: ASPAS
% | 
\RCSdef{\RCSKneighborsearchRev}{$Id$}
\RCSdefDate{\RCSKneighborsearchDate}{$Date$}
% |
%%RefPage: end of header, begin of main body
% +------------------------------------------------------------------------+


\begin{ccRefClass}{K_neighbor_search<Traits, GeneralDistance, Splitter, SpatialTree>} 

%% \ccHtmlCrossLink{}     %% add further rules for cross referencing links
%% \ccHtmlIndexC[class]{} %% add further index entries

\ccDefinition
  
The class \ccRefName\ implements approximate $k$-nearest and $k$-furthest neighbor searching
using standard search on a tree using a general distance class. The tree may be built with 
extended or unextended nodes.

\ccInclude{CGAL/K_neighbor_search.h}

\ccParameters

Expects for the first template argument an implementation of the concept \ccc{SearchTraits},
for example \ccc{CGAL::Cartesian_d<double>}.

Expects for the second template argument a model of the
concept \ccc{GeneralDistance}. If \ccc{Traits} is 
\ccc{CGAL::Search_traits_adapter<Key,PointPropertyMap,BaseTraits>}
the default type is \ccc{CGAL::Distance_adapter<Key,PointPropertyMap,CGAL::Euclidean_distance<Traits> >},
and \ccc{CGAL::Euclidean_distance<Traits>} otherwise.


Expects for fourth template argument an implementation of the concept \ccc{SpatialTree}.
The default type is \ccc{CGAL::Kd_tree<Traits, Splitter, CGAL::Tag_false>}. The 
template argument \ccc{CGAL::Tag_false} makes that the tree is built with unextended nodes.

\ccTypes

\ccTypedef{Traits::Point_d Point_d;}{Point type.}
\ccTypedef{Traits::FT FT;}{Number type.}
\ccTypedef{GeneralDistance Distance;}{Distance type.}
\ccTypedef{std::pair<Point_d,FT> Point_with_transformed_distance;}{Pair of point and transformed distance.} 

\ccNestedType{iterator}{Bidirectional const iterator with value type \ccc{Point_with_distance}
                        for enumerating approximate neighbors.}

\ccTypedef{GeneralDistance::Query_item Query_item;}{Query item type.}

\ccTypedef{SpatialTree Tree;}{The tree type.}


\ccCreation
\ccCreationVariable{s}  %% choose variable name


\def\ccLongParamLayout{\ccTrue}
\ccConstructor{K_neighbor_search(const Tree& tree, Query_item q, unsigned int k=1, FT eps=FT(0.0),
	                         bool search_nearest=true, GeneralDistance d=GeneralDistance(),bool sorted=true);}
{Constructor for searching approximately $k$ neighbors of the query item \ccc{q} 
in the points stored in \ccc{tree} using 
distance class $d$ and approximation factor \ccc{eps}. \ccc{sorted} indicates
if the computed sequence of $k$-nearest neighbors needs to be sorted.}

\ccOperations

%\ccMethod{template <class OutputIterator>
%OutputIterator  the_k_neighbors(OutputIterator res);}
%{Writes \ccc{k} elements of type \ccc{Point_with_distance} to \ccc{res}.} 

\ccMethod{iterator begin() const;}{Returns a const iterator to the approximate nearest or furthest neighbor.}

\ccMethod{iterator end() const;}{Returns the appropriate past-the-end const iterator.}


\begin{ccAdvanced}
\ccMethod{std::ostream& statistics(std::ostream& s);}
{
Inserts statistics of the search process into the output stream \ccc{s}.
}
\end{ccAdvanced}

\ccSeeAlso

\ccc{CGAL::Orthogonal_k_neighbor_search<Traits, OrthogonalDistance, Splitter, SpatialTree>}

\end{ccRefClass}

% +------------------------------------------------------------------------+
%%RefPage: end of main body, begin of footer
% EOF
% +------------------------------------------------------------------------+

