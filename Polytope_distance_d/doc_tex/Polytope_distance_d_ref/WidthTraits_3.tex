
\begin{ccRefConcept}{WidthTraits_3}

% -----------------------------------------------------------------------------
\ccDefinition

This concept defines the requirements for traits classes of
\ccc{Width_3<Traits>}.

% -----------------------------------------------------------------------------
\ccTypes

\ccSaveThreeColumns
\ccSetTwoColumns{WidthTraits_3:: ChullTraits;}{}

\ccGlueBegin
\ccNestedType{Point_3}{The point type. The (in)equality tests must be
  available. Access to the point coordinates is done via the \ccc{get_.()}
  functions. Constructing a point is done with the \ccc{make_point( )}
  operation.}
\ccNestedType{Plane_3}{The plane type. Access to the coefficients of the
  plane is made via the \ccc{get_.()} functions. Constructing a plane is
  done with the \ccc{make_plane()} operation.}
\ccNestedType{Vector_3}{The vector type. There is no need to access the
  coefficients of a vector; only constructing is required and is done with
  the \ccc{make_vector} operation.}
\ccNestedType{ChullTraits}{The traits class for using the convex hull
  algorithm. It must be a model of the concept ConvexHullTraits\_3.
  This class is used only if the width is computed from a set
  of points.}
\ccNestedType{RT}{Ring type numbers. Internally all numbers are treated as
  ring type numbers, i.e., neither $/$-operator nor $\sqrt{.}$ nor other
  inexact operations are used. But because the algorithm does not use any
  divisions, but multiplication instead, the numbers can get really big.
  Therefore it is recommended to use a ring type number, that provides
  values of arbitrary length. Furthermore it is assumed that the underlying
  number type of \ccc{Point_3}, \ccc{Plane_3} and \ccc{Vector_3} equals
  \ccc{RT}.}
\ccGlueEnd

\emph{Notes:} If you want to compute the width of a \emph{polyhedron}
then you have to make sure that the point type in the traits class and
the point type in the polyhedron class are the same! The same holds
for \ccc{Traits::Plane_3} and \ccc{Polyhedron::Plane_3}.


% -----------------------------------------------------------------------------
\ccCreation
\ccCreationVariable{traits}

Only a default constructor is required.

\ccConstructor{ WidthTraits_3( );}{}

% -----------------------------------------------------------------------------
\ccOperations
Whatever the coordinates of the points are, it is required for the
width-algorithm to have access to the homogeneous representation of
points. 

\ccSetThreeColumns{Vector_3}{get_plane_coordinates( );}{}
\ccGlueBegin
\ccFunction{ RT get_hx(const Point_3& p) const;}{returns the
  homogeneous $x$-coordinate of point $p$.}  
\ccFunction{ RT get_hy(const Point_3& p) const;}{returns the
  homogeneous $y$-coordinate of point $p$.}  
\ccFunction{ RT get_hz(const Point_3& p) const;}{returns the
  homogeneous $z$-coordinate of point $p$.}  
\ccFunction{ RT get_hw(const Point_3& p) const;}{returns the
  homogenizing coordinate of point $p$.}  
\ccFunction{ void get_point_coordinates(const Point_3& p, RT& px,
  RT& py, RT& pz, RT& ph) const;}{returns all homogeneous coordinates
  of point $p$ at once.}
\ccGlueEnd

\ccGlueBegin
\ccFunction{ RT get_a(const Plane_3& f) const;}{returns the first
  coefficient of plane $f$.}
\ccFunction{ RT get_b(const Plane_3& f) const;}{returns the second
  coefficient of plane $f$.}
\ccFunction{ RT get_c(const Plane_3& f) const;}{returns the third
  coefficient of plane $f$.}
\ccFunction{ RT get_d(const Plane_3& f) const;}{returns the fourth
  coefficient of plane $f$.}
\ccFunction{ void get_plane_coefficients(const Plane_3& f, 
  RT& a, RT& b, RT& c, RT& d)
  const;}{returns all four plane coefficients of $f$ at once.}
\ccGlueEnd

\ccGlueBegin
\ccFunction{ Point_3 make_point(const RT& hx, const RT& hy, const RT& hz, 
  const RT& hw) const;}{returns a point of type
  \ccc{Point_3} with homogeneous coordinates $hx$, $hy$, $hz$ and $hw$.}
\ccFunction{ Plane_3 make_plane(const RT& a, const RT& b, const
  RT& c, const RT& d) const;}{returns a plane of type \ccc{Plane_3}
  with coefficients $a$, $b$, $c$ and $d$.}
\ccFunction{ Vector_3 make_vector(const RT& a, const RT& b, const
  RT& c) const;}{returns a vector of type \ccc{Vector_3} with the four
  homogeneous coefficients $a$, $b$, $c$ and 1.}
\ccGlueEnd

\ccRestoreThreeColumns

% -----------------------------------------------------------------------------
\ccHasModels

\ccRefIdfierPage{CGAL::Width_default_traits_3<K>}

% -----------------------------------------------------------------------------
\ccSeeAlso

\ccIndexTraitsClassRequirements[C]{Width_3}
\ccRefIdfierPage{CGAL::Width_3<Traits>}

% -----------------------------------------------------------------------------

\end{ccRefConcept}


