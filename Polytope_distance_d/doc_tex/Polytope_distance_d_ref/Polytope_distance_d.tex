
\begin{ccRefClass}{Polytope_distance_d<Traits>}

\ccIndexSubitem       [t]{distance}{of polytopes}
\ccIndexSubitem       [t]{polytope}{distance of polytopes}
\ccIndexSubitem       [t]{optimal distances}{distance of polytopes}

\ccSaveThreeColumns
\cgalPolytopeDistanceLayout

\newcommand{\PDempty}{%
  $pd(\emptyset, \emptyset)$}

% -----------------------------------------------------------------------------
\ccDefinition

An object of the class \ccRefName\ represents the (squared) distance
between two convex polytopes, given as the convex hulls of two finite point
sets in $d$-dimensional Euclidean space $\E_d$. For point sets $P$ and $Q$
we denote by $pd(P,Q)$ the distance between the convex hulls of $P$ and
$Q$. Note that $pd(P,Q)$ can be
degenerate\lcTex{\ccIndexSubitem[t]{degeneracies}{\ccFont Polytope_distance_d}},
i.e.~$pd(P,Q)=\infty$ if $P$ or $Q$ is empty.

Two inclusion-minimal subsets $S_P$ of $P$ and $S_Q$ of $Q$ with
$pd(S_P,S_Q)=pd(P,Q)$ are called \emph{pair of support
  sets}\lcTex{\ccIndexSubitem[t]{support set}{\ccFont Polytope_distance_d}}, the
points in $S_P$ and $S_Q$ are the \emph{support points}. A pair of support
sets has size at most $d+2$ (by size we mean $|S_P|+|S_Q|$). The distance
between the two polytopes is \emph{realized} by a pair of points $p$ and
$q$ lying in the convex hull of $S_P$ and $S_Q$, repectively,
i.e.~$\sqrt{||p-q||}=pd(P,Q)$. In general, neither the support sets nor the
realizing points are necessarily unique.

The underlying algorithm can cope with all kinds of input, e.g.~$P$ and~$Q$
may be in non-convex position or points may occur more than once. The
algorithm computes a pair of support sets $S_P$ and $S_Q$ with realizing
points $p$ and $q$ which remain fixed until the next set, insert, or clear
operation.

\ccInclude{CGAL/Polytope_distance_d.h}

% -----------------------------------------------------------------------------
\ccRequirements
\ccIndexRequirements

The template parameter \ccc{Traits} is a model for \ccc{OptimisationDTraits}.

We provide the models \ccc{Optimisation_d_traits_2},
\ccc{Optimisation_d_traits_3}, and \ccc{Optimisation_d_traits_d} using the
two-, three-, and $d$-dimensional \cgal~kernel, respectively.

% -----------------------------------------------------------------------------
\ccTypes
\ccIndexClassTypes

\ccSetTwoColumns{\ccRefName:: Point}{}

\ccNestedType{ Point}{ typedef to \ccc{Traits::Point_d}.
        Point type used to represent the input points.}

\ccNestedType{ FT}{ typedef to \ccc{Traits::FT}.
        Number type used to return the squared distance
        between the two polytopes.}

\ccNestedType{ ET}{ typedef to \ccc{Traits::ET}.
        Number type used to do the exact computations in the underlying
        solver for quadratic programs (cf.~\textbf{Implementation}).}

\ccNestedType{ Point_iterator}{
        non-mutable model of the \stl\ concept \emph{RandomAccessIterator}
        with value type \ccc{Point}. Used to access the points
        of the two polytopes.}

\ccNestedType{ Support_point_iterator}{
        non-mutable model of the \stl\ concept \emph{RandomAccessIterator}
        with value type \ccc{Point}. Used to access the support points.}

\ccNestedType{ Support_point_index_iterator}{
        non-mutable model of the \stl\ concept \emph{RandomAccessIterator}
        with value type \ccc{int}. Used to access the indices of the
        support points in the provided input order (starting from 0 
        in both point sets).}

\ccNestedType{ Coordinate_iterator}{
        non-mutable model of the \stl\ concept \emph{RandomAccessIterator}
        with value type \ccc{ET}. Used to access the coordinates of
        the realizing points.}

\ccPropagateThreeToTwoColumns

% -----------------------------------------------------------------------------
\ccCreation
\ccIndexClassCreation
\ccCreationVariable{poly_dist}

\ccConstructor{ Polytope_distance_d( const Traits&  traits  = Traits(),
                                     int            verbose = 0,
                                     std::ostream&  stream  = std::cout);}{
        initializes \ccVar\ to \PDempty.}

\ccConstructor{ template < class InputIterator1, class InputIterator2 >
                Polytope_distance_d( InputIterator1  p_first,
                                     InputIterator1  p_last,
                                     InputIterator2  q_first,
                                     InputIterator2  q_last,
                                     const Traits&   traits = Traits(),
                                     int             verbose = 0,
                                     std::ostream&   stream  = std::cout);}{
        initializes \ccVar\ to $pd(P,Q)$ with $P$ and $Q$ being the
        sets of points in the range [\ccc{p_first},\ccc{p_last}) and
        [\ccc{q_first},\ccc{q_last}), respectively.
        \ccRequire  The value type of \ccc{InputIterator1} and
        \ccc{InputIterator2} is \ccc{Point}.
        \ccPrecond  All points have the same dimension.}

\begin{ccAdvanced}
  If \ccc{verbose} is set to $1$, $2$, or $3$ then some, more, or full
  verbose output of the underlying solver for quadratic programs is
  written to \ccc{stream}, resp.
\end{ccAdvanced}

% -----------------------------------------------------------------------------
\ccAccessFunctions
\begin{ccIndexMemberFunctions}
\ccIndexMemberFunctionGroup{access}

\ccMemberFunction{ int  ambient_dimension( ) const;}{
        returns the dimension of the points in $P$ and $Q$.
        If \ccVar\ is \PDempty, the ambient dimension is $-1$.}

\medskip
\ccGlueBegin
\ccMemberFunction{ int  number_of_points( ) const;}{
        returns the number of all points of \ccVar, i.e.~$|P|+|Q|$.}
%
\ccMemberFunction{ int  number_of_points_p( ) const;}{
        returns the number of points in $P$.}
%
\ccMemberFunction{ int  number_of_points_q( ) const;}{
        returns the number of points in $Q$.}
\ccGlueEnd

\medskip
\ccGlueBegin
\ccMemberFunction{ int  number_of_support_points( ) const;}{
        returns the number of support points of \ccVar, i.e.~$|S_P|+|S_Q|$.}
\lcTex{\ccIndexSubitem[t]{support set}{\ccFont Polytope_distance_d}}
%
\ccMemberFunction{ int  number_of_support_points_p( ) const;}{
        returns the number of support points in $S_P$.}
%
\ccMemberFunction{ int  number_of_support_points_q( ) const;}{
        returns the number of support points in $S_Q$.}
\ccGlueEnd

\medskip
\ccGlueBegin
\ccMemberFunction{ Point_iterator  points_p_begin( ) const;}{
        returns an iterator referring to the first point in $P$.}
%
\ccMemberFunction{ Point_iterator  points_p_end( ) const;}{
        returns the corresponding past-the-end iterator.}
\ccGlueEnd

\medskip
\ccGlueBegin
\ccMemberFunction{ Point_iterator  points_q_begin( ) const;}{
        returns an iterator referring to the first point in $Q$.}
%
\ccMemberFunction{ Point_iterator  points_q_end( ) const;}{
        returns the corresponding past-the-end iterator.}
\ccGlueEnd

\medskip
\ccGlueBegin
\ccMemberFunction{ Support_point_iterator  support_points_p_begin( ) const;}{
        returns an iterator referring to the first support point in $S_P$.}
%
\ccMemberFunction{ Support_point_iterator  support_points_p_end( ) const;}{
        returns the corresponding past-the-end iterator.}
\ccGlueEnd

\medskip
\ccGlueBegin
\ccMemberFunction{ Support_point_iterator  support_points_q_begin( ) const;}{
        returns an iterator referring to the first support point in $S_Q$.}
%
\ccMemberFunction{ Support_point_iterator  support_points_q_end( ) const;}{
        returns the corresponding past-the-end iterator.}
\ccGlueEnd

\medskip
\ccGlueBegin
\ccMemberFunction{ Support_point_index_iterator  support_points_p_indices_begin( ) const;}{
        returns an iterator referring to the index of the first support point in $P$.}
%
\ccMemberFunction{ Support_point_index_iterator  support_points_p_indices_end( ) const;}{
        returns the corresponding past-the-end iterator.}
\ccGlueEnd

\medskip
\ccGlueBegin
\ccMemberFunction{ Support_point_index_iterator  support_points_q_indices_begin( ) const;}{
        returns an iterator referring to the index of the first support point in $Q$.}
%
\ccMemberFunction{ Support_point_index_iterator  support_points_q_indices_end( ) const;}{
        returns the corresponding past-the-end iterator.}
\ccGlueEnd

\ccMemberFunction{ Point  realizing_point_p( ) const;}{
        returns the realizing point of $P$.
        \ccRequire  An implicit conversion from \ccc{ET} to \ccc{RT} is
        available.
        \ccPrecond $pd(P,Q)$ is finite.}

\ccMemberFunction{ Point  realizing_point_q( ) const;}{
        returns the realizing point of $Q$.
        \ccRequire  An implicit conversion from \ccc{ET} to \ccc{RT} is
        available.
        \ccPrecond $pd(P,Q)$ is finite.}

\ccMemberFunction{ FT  squared_distance( ) const;}{
        returns the squared distance of \ccVar, i.e.~$(pd(P,Q))^2$.
        \ccRequire  An implicit conversion from \ccc{ET} to \ccc{RT} is
        available.
        \ccPrecond $pd(P,Q)$ is finite.}

\medskip
\ccGlueBegin
\ccMemberFunction{ Coordinate_iterator
                   realizing_point_p_coordinates_begin() const;}{
        returns an iterator referring to the first coordinate of the
        realizing point of $P$.
        \ccCommentHeading{Note} The coordinates have a rational
        representation, i.e.~the first $d$ elements of the iterator
        range are the numerators and the $(d\!+\!1)$-st element is the
        common denominator.}
%
\ccMemberFunction{ Coordinate_iterator
                   realizing_point_p_coordinates_end() const;}{
        returns the corresponding past-the-end iterator.}
\ccGlueEnd

\medskip
\ccGlueBegin
\ccMemberFunction{ Coordinate_iterator
                   realizing_point_q_coordinates_begin() const;}{
        returns an iterator referring to the first coordinate of the
        realizing point of $Q$.
        \ccCommentHeading{Note} The coordinates have a rational
        representation, i.e.~the first $d$ elements of the iterator
        range are the numerators and the $(d\!+\!1)$-st element is the
        common denominator.}
%
\ccMemberFunction{ Coordinate_iterator
                   realizing_point_q_coordinates_end() const;}{
        returns the corresponding past-the-end iterator.}
\ccGlueEnd

\ccMemberFunction{ ET  squared_distance_numerator( ) const;}{
        returns the numerator of the squared distance of \ccVar.}

\ccMemberFunction{ ET  squared_distance_denominator( ) const;}{
        returns the denominator of the squared distance of \ccVar.}

% -----------------------------------------------------------------------------
\ccPredicates
\ccIndexMemberFunctionGroup{predicates}

\ccMemberFunction{ bool  is_finite( ) const;}{
        returns \ccc{true}, if $pd(P,Q)$ is finite,
        i.e.~none of the two polytopes is empty.}

\ccMemberFunction{ bool  is_zero( ) const;}{
        returns \ccc{true}, if $pd(P,Q)$ is zero,
        i.e.~the two polytopes intersect (this implies degeneracy).}

\ccMemberFunction{ bool  is_degenerate( ) const;}{
        returns \ccc{true}, iff $pd(P,Q)$ is degenerate,
        i.e.~$pd(P,Q)$ is not finite.}
\lcTex{\ccIndexSubitem[t]{degeneracies}{\ccFont Polytope_distance_d}}

% -----------------------------------------------------------------------------
\ccModifiers
\ccIndexMemberFunctionGroup{modifiers}

\ccMemberFunction{ void  clear( );}{
        resets \ccVar\ to \PDempty.}

\ccMemberFunction{ template < class InputIterator1, class InputIterator2 >
                   void  set( InputIterator1 p_first,
                              InputIterator1 p_last,
                              InputIterator2 q_first,
                              InputIterator2 q_last );}{
        sets \ccVar\ to $pd(P,Q)$ with $P$ and $Q$ being the sets of
        points in the ranges [\ccc{p_first},\ccc{p_last}) and
        [\ccc{q_first},\ccc{q_last}), respectively.
        \ccRequire  The value type of \ccc{InputIterator1} and
        \ccc{InputIterator2} is \ccc{Point}.
        \ccPrecond  All points have the same dimension.}

\ccMemberFunction{ template < class InputIterator >
                   void  set_p( InputIterator p_first,
                                InputIterator p_last );}{
        sets \ccVar\ to $pd(P,Q)$ with $P$ being the set of points
        in the range [\ccc{p_first},\ccc{p_last}) ($Q$ remains unchanged).
        \ccRequire  The value type of \ccc{InputIterator} is \ccc{Point}.
        \ccPrecond  All points in $P$ have dimension
        \ccVar\ccc{.ambient_dimension()} if $Q$ is not empty.}

\ccMemberFunction{ template < class InputIterator >
                   void  set_q( InputIterator q_first,
                                InputIterator q_last );}{
        sets \ccVar\ to $pd(P,Q)$ with $Q$ being the set of points
        in the range [\ccc{q_first},\ccc{q_last}) ($P$ remains unchanged).
        \ccRequire  The value type of \ccc{InputIterator} is \ccc{Point}.
        \ccPrecond  All points in $Q$ have dimension
        \ccVar\ccc{.ambient_dimension()} if $P$ is not empty.}

\ccMemberFunction{ void  insert_p( const Point& p);}{
        inserts \ccc{p} into $P$.
        \ccPrecond  The dimension of \ccc{p} equals
        \ccVar\ccc{.ambient_dimension()} if \ccVar\ is not \PDempty.}

\ccMemberFunction{ void  insert_q( const Point& q);}{
        inserts \ccc{q} into $Q$.
        \ccPrecond  The dimension of \ccc{q} equals
        \ccVar\ccc{.ambient_dimension()} if \ccVar\ is not \PDempty.}

\ccMemberFunction{ template < class InputIterator1, class InputIterator2 >
                   void  insert( InputIterator1  p_first,
                                 InputIterator1  p_last,
                                 InputIterator2  q_first,
                                 InputIterator2  q_last );}{
        inserts the points in the range [\ccc{p_first},\ccc{p_last})
        and [\ccc{q_first},\ccc{q_last}) into $P$ and $Q$, respectively,
        and recomputes the (squared) distance.
        \ccRequire  The value type of \ccc{InputIterator1} and
        \ccc{InputIterator2} is \ccc{Point}.
        \ccPrecond  All points have the same dimension.
        If \ccVar\ is not \PDempty, this dimension must be equal to
        \ccVar\ccc{.ambient_dimension()}.}

\ccMemberFunction{ template < class InputIterator >
                   void  insert_p( InputIterator  p_first,
                                   InputIterator  p_last );}{
        inserts the points in the range [\ccc{p_first},\ccc{p_last}) into
        $P$ and recomputes the (squared) distance ($Q$ remains unchanged).
        \ccRequire  The value type of \ccc{InputIterator} is \ccc{Point}.
        \ccPrecond  All points have the same dimension.
        If \ccVar\ is not empty, this dimension must be equal to
        \ccVar\ccc{.ambient_dimension()}.}

\ccMemberFunction{ template < class InputIterator >
                   void  insert_q( InputIterator  q_first,
                                   InputIterator  q_last );}{
        inserts the points in the range [\ccc{q_first},\ccc{q_last}) into
        $Q$ and recomputes the (squared) distance ($P$ remains unchanged).
        \ccRequire  The value type of \ccc{InputIterator} is \ccc{Point}.
        \ccPrecond  All points have the same dimension.
        If \ccVar\ is not empty, this dimension must be equal to
        \ccVar\ccc{.ambient_dimension()}.}

% -----------------------------------------------------------------------------
\ccHeading{Validity Check}
\ccIndexMemberFunctionGroup{validity check}
\lcTex{\ccIndexSubitem[t]{validity check}{\ccFont Polytope_distance_d}}

An object \ccVar\ is valid, iff
\ldots
\begin{itemize}
  \item \ccVar\ contains all points of its defining set $P$,
  \item \ccVar\ is the smallest sphere containing its support set $S$, and
  \item $S$ is minimal, i.e.\ no support point is redundant.
\end{itemize}

\ccMemberFunction{ bool is_valid( bool verbose = false,
                                  int  level   = 0    ) const;}{
      returns \ccc{true}, iff \ccVar\ is valid. If \ccc{verbose} is
      \ccc{true}, some messages concerning the performed checks are
      written to standard error stream. The second parameter
      \ccc{level} is not used, we provide it only for consistency
      with interfaces of other classes.}

% -----------------------------------------------------------------------------
\ccHeading{Miscellaneous}
\ccIndexMemberFunctionGroup{miscellaneous}

\def\ccTagRmConstRefPair{\ccFalse}

\ccMemberFunction{ const Traits&  traits( ) const;}{
        returns a const reference to the traits class object.}

\def\ccTagRmConstRefPair{\ccTrue}

\end{ccIndexMemberFunctions}

% -----------------------------------------------------------------------------
\ccHeading{I/O}
\begin{ccIndexGlobalFunctions}

\ccIndexGlobalFunctionGroup{output}
\ccFunction{ std::ostream&
             operator << ( std::ostream& os,
                           const Polytope_distance_d<Traits>& poly_dist);}{
        writes \ccVar\ to output stream \ccc{os}.
        \ccRequire  The output operator is defined for \ccc{Point_d}.}

\ccFunction{ std::istream&
             operator >> ( std::istream& is,
                           Polytope_distance_d<Traits> poly_dist&);}{
        reads \ccVar\ from input stream \ccc{is}.
        \ccRequire  The input operator is defined for \ccc{Point_d}.}

\end{ccIndexGlobalFunctions}

% -----------------------------------------------------------------------------
\ccSeeAlso

\ccRefIdfierPage{CGAL::Optimisation_d_traits_2<K,ET,NT>}\\
\ccRefIdfierPage{CGAL::Optimisation_d_traits_3<K,ET,NT>}\\
\ccRefIdfierPage{CGAL::Optimisation_d_traits_d<K,ET,NT>}\\[1ex]
\ccRefConceptPage{OptimisationDTraits}

% -----------------------------------------------------------------------------
\ccImplementation
\ccIndexImplementation

The problem of finding the distance between two convex polytopes given as
the convex hulls of two finite point sets can be formulated as an
optimization problem with linear constraints and a convex quadratic
objective function\lcTex{\ccIndexSubitem[t]{quadratic program}
{\ccFont Polytope_distance_d}}. The solution is obtained using our exact solver
for quadratic programs~\cite{gs-eegqp-00}.

The creation time is almost always linear in the number of points.  Access
functions and predicates take constant time, inserting a point might take
up to linear time.  The clear operation and the check for validity each
take linear time.

% -----------------------------------------------------------------------------

\ccRestoreThreeColumns

\end{ccRefClass}

