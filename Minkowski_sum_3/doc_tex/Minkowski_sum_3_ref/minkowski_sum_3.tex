% +------------------------------------------------------------------------+
% | Reference manual page: minkowski_sum_3.tex
% +------------------------------------------------------------------------+
% | 11.06.2008   Peter Hachenberger
% | Package: Minkowski_sum_3
% | 
\RCSdef{\RCSminkowski_sum_3Rev}{$Id$}
\RCSdefDate{\RCminkowski_sum_3Date}{$Date$}
% |
%%RefPage: end of header, begin of main body
% +------------------------------------------------------------------------+

\ccHtmlNoClassLinks
\begin{ccRefFunction}{minkowski_sum_3}
\label{refminkowski_sum_3}

\ccDefinition

The function \ccc{minkowski_sum_3} computes the Minkowski sum of two
given 3D Nef polyhedra $N0$ and $N1$.

\ccGlobalFunction{Nef_polyhedron_3 convex_decomposition_3(Nef_polyhedron_3 N0, Nef_polyhedron_3 N1);}

\ccPrecond

An input polyhedron may consist of:
\begin{enumerate}
\item singular vertices
\item singular edges
\item singular convex facets without holes
\item surfaces with convex facets that have no holes.
\item three-dimensional features, whose coplanar facets have
common selection marks (this includes open and closed solids)
\end{enumerate}

Taking a different viewpoint, the implementation is restricted as
follows:
\begin{enumerate}
\item The input polyhedra must be bounded (selected outer volume is ignored).
\item All sets of coplanar facets of a full-dimensional
feature must have the same selection mark (in case of different
selection marks, unselected is assumed).
\item All facets of lower-dimensional features need to be convex and 
must not have holes (non-convex facets and holes are ignored).
\end{enumerate}

\ccPostcond

If either of the input polyhedra is non-convex, it is modified during
the computation, i.e., it is decomposed into convex pieces.

\ccSeeAlso
\ccRefIdfierPage{CGAL::Nef_polyhedron_3<Traits>}\\ 
\ccRefIdfierPage{CGAL::convex_decomposition_3}\\ 

\end{ccRefFunction}
