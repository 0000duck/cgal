% +------------------------------------------------------------------------+
% | Reference manual page: convex_hull_incremental_3.tex
% +------------------------------------------------------------------------+
% | 09.05.2001   Susan Hert, Stefan Schirra, Michael Seel
% | Package: Convex_hull_3
% | 
% +------------------------------------------------------------------------+


\begin{ccRefFunction}{convex_hull_incremental_3}  
\ccIndexMainItemBegin{convex hull, 3D}
\ccIndexSubitemBegin{convex hull, 3D}{incremental}

\ccDefinition
  
The function \ccRefName\ computes the convex hull polyhedron from a set
of given three-dimensional points.

This function is provided for completeness and educational
purposes. When an efficient incremental implementation is needed,
using \ccc{CGAL::Delaunay_triangulation_3} together with
\ccc{CGAL::convex_hull_3_to_polyhedron_3} is highly recommended.

\ccInclude{CGAL/convex_hull_incremental_3.h}

\ccFunction{template <class InputIterator, class Polyhedron>
            void
            convex_hull_incremental_3(InputIterator first, 
                                                 InputIterator beyond, 
                                                 Polyhedron& P, 
                                                 bool test_correctness = false);
            }
            {computes the convex hull polyhedron 
            of the points in the range [\ccc{first},\ccc{beyond})
            and assigns it to \ccc{P}.  If \ccc{test_correctness} is set to
            \ccc{true}, the tests described in \cite{mnssssu-cgpvg-96} are
            used to determine the correctness of the resulting polyhedron.
            }

\ccHeading{Requirements}
\begin{enumerate}
   \item    \ccc{Polyhedron} must provide a type \ccc{Polyhedron::Traits}
            that defines the following types
             \begin{itemize}
               \item \ccc{Polyhedron::Traits::R}, which is a model of 
                     the representation class \ccc{R} required by
                     \ccc{CGAL::Convex_hull_d_traits_3<R>}
               \item \ccc{Polyhedron::Traits::Point}
             \end{itemize}
   \item    \ccc{InputIterator::value_type} must be the same as
            \ccc{Polyhedron::Traits::Point}
\end{enumerate}

\ccSeeAlso

\ccRefIdfierPage{CGAL::convex_hull_3} \\
\ccRefIdfierPage{CGAL::convex_hull_2} 

\ccImplementation

This function uses the $d$-dimensional convex hull incremental construction 
algorithm \cite{cms-frric-93}
with $d$ fixed to 3.  The algorithm requires $O(n^2)$ time in the
worst case and $O(n \log n)$ expected time.

\ccSeeAlso

\ccRefIdfierPage{CGAL::Convex_hull_d<R>}

\ccExample

The following example computes the convex hull of a set of 250 random 
points chosen uniformly in a sphere of radius 100.

\ccIncludeExampleCode{Convex_hull_3/incremental_hull_3.cpp}

\ccIndexSubitemEnd{convex hull, 3D}{incremental}
\ccIndexMainItemEnd{convex hull, 3D}
\end{ccRefFunction}

% +------------------------------------------------------------------------+
%%RefPage: end of main body, begin of footer
% EOF
% +------------------------------------------------------------------------+

