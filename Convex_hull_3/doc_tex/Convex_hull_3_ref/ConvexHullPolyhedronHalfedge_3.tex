% +------------------------------------------------------------------------+
% | Reference manual page: ConvexHullPolyhedronHalfedge_3.tex
% +------------------------------------------------------------------------+
% | 17.03.1999   Lutz Kettner
% | Package: Polyhedron
% | 
\RCSdef{\RCSHalfedgeRev}{$Id$}
\RCSdefDate{\RCSHalfedgeDate}{$Date$}
% +------------------------------------------------------------------------+

\ccRefPageBegin

%%RefPage: end of header, begin of main body
% +------------------------------------------------------------------------+


\begin{ccRefConcept}{ConvexHullPolyhedronHalfedge_3}

\ccDefinition

The requirements of the halfedge type required for the polyhedron built
by the function \ccc{convex_hull_3}.
  
\ccHasModels
\ccRefIdfierPage{CGAL::Polyhedron_3<Traits>::Halfedge}

\ccCreation
\ccCreationVariable{h}

\ccConstructor{Halfedge();}{default constructor.}

\ccTagFullDeclarations
\ccOperations

\ccMethod{Halfedge_handle opposite();}{the opposite halfedge.}

\ccMethod{Halfedge_handle next();}
    {the next halfedge around the facet.}

\ccMethod{Halfedge_handle prev();}
    {the previous halfedge around the facet.}

\ccMethod{bool             is_border() const;}
    {is true if \ccVar\ is a border halfedge.}


\ccMethod{Halfedge_around_facet_circulator       facet_begin();}
    {circulator of halfedges around the facet (counterclockwise).}


\ccMethod{Vertex_handle       vertex();}{the incident vertex of \ccVar.}

\ccMethod{Facet_handle       facet();}
    {the incident facet of \ccVar.  If \ccVar\ is a border halfedge 
      the result is default construction of the handle.}


\ccSeeAlso

\ccRefIdfierPage{CGAL::Polyhedron_3<Traits>} \\
\ccRefConceptPage{ConvexHullPolyhedronVertex_3} \\
\ccRefConceptPage{ConvexHullPolyhedronFacet_3} 

\ccTagDefaults
\end{ccRefConcept}

% +------------------------------------------------------------------------+
%%RefPage: end of main body, begin of footer
\ccRefPageEnd
% EOF
% +------------------------------------------------------------------------+
