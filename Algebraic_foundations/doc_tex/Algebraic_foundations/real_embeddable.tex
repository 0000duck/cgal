\section{Real Embeddable}

Most number types represent some subset of the real numbers. From those types 
we expect functionality to compute the sign, absolute value or double 
approximations. In particular we can expect an order on such a type that 
reflects the order along the real axis. 
All these properties are gathered in the concept \ccc{RealComparable}. 
The concept is orthogonal to the algebraic structure concepts, 
i.e., it is possible 
that a type is a model of \ccc{RealEmbeddable} only, 
since the type may just represent values on the real axis
but does not provide any arithmetic operations.


As for algebraic structures this concept is also traits class oriented.
The main functionality related to \ccc{RealEmbeddable} is gathered in 
the class \ccc{Real_embeddable_traits}. In particular, it porivdes the boolean 
tag  \ccc{Is_real_embeddable} indicating whether a type is a model of 
\ccc{RealEmbeddable}. The comparison operators are required to be realized via 
\CC\ operator overloading. 
All unary functions (e.g. {\em sign}, {\em to\_double}) and 
binary functions (e.g. {\em compare} ) are models of the \stl-concepts 
\ccc{AdaptableUnaryFunction} and \ccc{AdaptableBinaryFunction} and are local 
to \ccc{Real_embeddable_traits}. 


In case a type is a model of \ccc{IntegralDomainWithoutDivision} and  
\ccc{RealEmbeddable} the number represented by an object of this type is 
the same for arithmetic and comparison.
It follows that the ring represented  by this type is a superset of the integers
and a subset of the real numbers and hence has characteristic zero.
\ccIgnore{see http://mathworld.wolfram.com/CharacteristicField.html )}
In case the type is a model of \ccc{Field} and \ccc{RealEmbeddable} it is a 
superset of the rational numbers. 


\ccIgnore{
All algebraic structure concepts do not require any order on a type, 
in particular they refine \ccc{EqualityComparable} but not \ccc{LessThanComparable}. 
From that it is possible for a type representing $\Z/p \Z$ to be a model of \ccc{Field}. 
However, from those types that are embedded on the real axis we can expect an order
that correlates to the one of the real numbers, a sign, an absolute value and 

However, most number types are representing some subset of the real numbers and from that
we can expect an order on these types that correlates to the one of the real numbers. 
}              