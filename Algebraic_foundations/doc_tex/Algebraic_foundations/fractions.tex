\section{Fractions}

Beyond the need for performing algebraic operations on objects as a 
whole, there are also number types which one would like to decompose into 
numerator and denominator. This does not only hold for rational numbers 
as \ccc{Quotient}, \ccc{Gmpq}, \ccc{mpq_class} or \ccc{leda_rational}, but 
also for compound objects as \ccc{Sqrt_extension} or \ccc{Polynomial} 
which may decompose into a (scalar) 
denominator and a compound numerator with a simpler coefficient type 
(e.g. integer instead of rational). Often operations can be performed faster on 
these denominator-free multiples. In case a type is a \ccc{Fraction} 
the relevant functionality as well as the numerator and denominator 
type are provided by \ccc{CGAL::Fraction_traits}. In particular  
\ccc{CGAL::Fraction_traits} provides a tag \ccc{Is_fraction} that can be
used for dispatching.

A related class is \ccc{CGAL::Rational_traits} which has been kept for backward 
compatibility reasons. However, we recommend to use \ccc{Fraction_traits} since
it is more general and offers dispatching functionality. 

\subsection{Examples}

The following example show a simple use of \ccc{Fraction_traits}:
\ccIncludeExampleCode{Algebraic_foundations/fraction_traits.cpp}

The following example illustrates the integralization of a vector, 
i.e., the coefficient vector of a polynomial. Note that for minimizing 
coefficient growth \ccc{Fraction_traits<Type>::Common_factor} is used to 
compute the 'least' common multiple of the denominators.

\ccIncludeExampleCode{Algebraic_foundations/integralize.cpp}

