\begin{ccRefConcept}{AlgebraicStructureTraits}


\ccDefinition

A model of \ccc{AlgebraicStructureTraits} refelcts the algebraic structure
of an assosiated type AS. 


Depending on the concepts that AS fulfills, it contains various functors and 
descriptive tags. See the list of members above.
Moreover it gives access to the several possible 
algebraic operations within that structure.


\ccTypes

A model of \ccc{AlgebraicStructureTraits} is supposed to provide:\\

\ccNestedType{AS}
        {The assosiated algebraic structure.}

\ccNestedType{Algebraic_structure_tag}
        { Tag indicating the algebraic structure of the assoziated type. \\
          This is one of: 
        \begin{tabular}{ll}
          - \ccc{Null_type}& AS fullfills no algebraic concept\\
          - \ccc{Integral_domain_without_div_tag}& AS is model of \ccc{IntegralDomainWithoutDiv}\\     
          - \ccc{Integral_domain_tag}& AS is model of \ccc{IntegralDomain}\\     
          - \ccc{UFDomain_tag}& AS is model of \ccc{UFDomain}\\     
          - \ccc{Euclidean_ring_tag}& AS is model of \ccc{EuclideanRing}\\     
          - \ccc{Field_tag}& AS is model of \ccc{Field}\\     
          - \ccc{Field_with_sqrt_tag}& AS is model of \ccc{FieldWithSqrt}\\
        \end{tabular}        
        }

\ccNestedType{Is_exact_tag}
        { Tag indicating whether AS is an exact type. \\
          This is either \ccc{CGAL::Tag_true} or \ccc{CGAL::Tag_false}. }

\ccTypes

\ccNestedType{Simplify}{ 
Required in case AS is at least an \ccc{IntegralDomainWithoutDiv}
,\\this is an  \ccAdaptableUnaryFunction
{void}
{AS\&}
{may simplify the representation of the object.}
}


\ccNestedType{Unit_part}{ 
Required in case AS is at least an \ccc{IntegralDomainWithoutDiv},
\\this is an  \ccAdaptableUnaryFunction
{AS}
{const AS\&}
{ extracts the unit part the unit part of x 

The mathematical definition of unit part is as follows: Two ring elements $a$ 
and $b$ are said to be associate if there exists an invertible ring element 
(i.e. a unit) $u$ such that $a = ub$. This defines an equivalence relation. 
We can distinguish exactly one element of every equivalence class as being 
unit normal. Then each element of a ring possesses a factorization into a unit 
(called its unit part) and a unit-normal ring element 
(called its unit normal associate).

For the integers, the non-negative numbers are by convention unit normal, 
hence the unit-part of a non-zero integer is its sign. For a \ccc{Field}, every 
non-zero element is a unit and is its own unit part, its unit normal 
associate being one. The unit part of zero is, by convention, one.

%In a polynomial ring $R[x]$, the unit part of a polynomial is the unit 
%part of its leading coefficient, whose definition depends on $R$. 
}}

\ccNestedType{Integral_div}{
Required in case AS is at least an \ccc{IntegralDomain},
\\this is an  \ccAdaptableBinaryFunction
{AS}
{const AS\&}
{const AS\&}
{returns $z = x/y$ , ,\\this is an  integral division. 

Integral division (a.k.a. exact division or division without remainder) maps 
ring elements $(x,y)$ to ring element $z$ such that $x = yz$ if such a $z$ 
exists (i.e. if $x$ is divisible by $y$). Otherwise the effect of invoking 
this operation is undefined. Since the ring represented is an integral domain, 
$z$ is uniquely defined if it exists. 
}
}

\ccNestedType{Gcd}{
Required in case AS is at least an \ccc{UFDomain},
\\this is an  \ccAdaptableBinaryFunction
{AS}
{const AS\&}
{const AS\&}
{returns $z = gcd(x,y)$. 

The greatest common divisor ($gcd$) of ring elements $x$ and $y$ is the unique 
ring element $d$ (up to a unit) with the property that any common divisor of 
$x$ and $y$ also divides $d$. (In other words: $d$ is the greatest lower bound 
of $x$ and $y$ in the partial order of divisibility.) We demand the $gcd$ to be 
unit-normal (i.e. have unit part 1). $gcd(0,0)$ is defined as $1$.
}
}

\ccNestedType{Mod}{
Required in case AS is at least an \ccc{EuclideanRing},
\\this is an  \ccAdaptableBinaryFunction
{AS}
{const AS\&}
{const AS\&}
{returns remainder of x modulo y.}       
}       

\ccNestedType{Div}{     
Required in case AS is at least an \ccc{EuclideanRing},
\\this is an  \ccAdaptableBinaryFunction
{AS}
{const AS\&}
{const AS\&}
{returns quotient from division with remainder. 
}
}
\ccNestedType{Div_mod}{     
Required in case AS is at least an \ccc{EuclideanRing},
\\this is an  \ccAdaptableQuadFunction
{void}
{const AS\&}
{const AS\&}
{AS\&}
{AS\&}
{
divide $a$ by $b$ with quotient $c$ and remainder $d$. 
}
}

\ccNestedType{Sqrt}{ 
Required in case AS is at least an \ccc{FieldWithSqrt},
\\this is an  \ccAdaptableUnaryFunction
{AS}
{const AS\&}
{functor to compute the square root, required for FieldWithSqrt. 

{ \tt OLD EXACUS: \\ 
A number type NT may provide this functor even if the set of numbers it models
 does not contain real square roots in general. The most important example are 
number types modeling the integers. For them, Sqrt()(x) has to return the 
largest integer not exceeding the square root of x. \\
COMMENT: I suggest to put this as an extra functor into ComparableTraits, e.g. Sqrt\_approx.  
}
}
}




{\small TODO: ?\\
\ccNestedType{KthRoot}{ defined in EXACUS, but not needed }
\ccNestedType{Root_of}{ 
is this an functor for algebraic structure ? 
since it is defined as the kth real root of an polynomial.. } 
}




\ccHasModels


\end{ccRefConcept}
