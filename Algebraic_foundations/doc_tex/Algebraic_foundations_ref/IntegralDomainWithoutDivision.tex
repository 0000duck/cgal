\begin{ccRefConcept}{IntegralDomainWithoutDivision}


\ccDefinition

This is the most basic concept for algebraic structures considered within CGAL. 

A model \ccc{IntegralDomainWithoutDivision}  represents an integral domain,
i.e. commutative ring with 0, 1, +, * and unity free of zero divisors.\\ 
{\bf Note:} A model is not required to offer the always well defined integral division.

It refines \ccc{Assignable}, \ccc{CopyConstructible}, \ccc{DefaultConstructible}
and \ccc{FromSmallIntConstructible}. \\ 
It refines \ccc{EqualityComparable}, where equality is defined w.r.t. 
the ring element being represented. \\
The operators unary and binary plus +, unary and binary minus -, 
multiplication * and their compound forms +=, -=, *= are required and 
implement the respective ring operations.

Moreover, \ccc{CGAL::Algebraic_structure_traits< IntegralDomainWithoutDivision >} is a model of 
\ccc{AlgebraicStructureTraits} providing:\\ 
- \ccc{CGAL::Algebraic_structure_traits< IntegralDomainWithoutDivision >::Algebraic_type} derived from \ccc{Integral_domain_without_division_tag} \\
- \ccc{CGAL::Algebraic_structure_traits< IntegralDomainWithoutDivision >::Is_zero}  \\
- \ccc{CGAL::Algebraic_structure_traits< IntegralDomainWithoutDivision >::Is_one}  \\
- \ccc{CGAL::Algebraic_structure_traits< IntegralDomainWithoutDivision >::Square}  \\
- \ccc{CGAL::Algebraic_structure_traits< IntegralDomainWithoutDivision >::Simplify} \\
- \ccc{CGAL::Algebraic_structure_traits< IntegralDomainWithoutDivision >::Unit_part} \\


{ \em \small FROM EXACUS: \\
It is permissible for the constructor 
to convert from another built-in type than int, provided ints are
automatically converted to this type. It is not permissible to 
rely in a similar fashion on a chain of user-defined conversions.
}



\ccRefines
\ccc{Assignable}\\
\ccc{CopyConstructible}\\
\ccc{DefaultConstructible}\\
\ccc{FromSmallIntConstructible}\\ 
\ccc{EqualityComparable}\\

\ccOperations

\ccSetThreeColumns{xxxxxxxxxxxxxxxxxxxxxxxxxxxxxxxxxx}{xxx}{}
\ccFunction{IntegralDomainWithoutDivision 
            operator+(const IntegralDomainWithoutDivision &a, 
                      const IntegralDomainWithoutDivision &b);}{}
\ccGlue

\ccFunction{IntegralDomainWithoutDivision 
            operator-(const IntegralDomainWithoutDivision &a, 
                      const IntegralDomainWithoutDivision &b);}{}
\ccGlue

\ccFunction{IntegralDomainWithoutDivision 
            operator*(const IntegralDomainWithoutDivision &a, 
                      const IntegralDomainWithoutDivision &b);}{}
\ccGlue
\ccFunction{IntegralDomainWithoutDivision 
            operator+(const IntegralDomainWithoutDivision &a);}{}
\ccGlue
\ccFunction{IntegralDomainWithoutDivision 
            operator-(const IntegralDomainWithoutDivision &a);}{}
\ccGlue
\ccMethod{IntegralDomainWithoutDivision 
            operator+=(const IntegralDomainWithoutDivision &a);}{}
\ccGlue
\ccMethod{IntegralDomainWithoutDivision 
            operator-=(const IntegralDomainWithoutDivision &a);}{}
\ccGlue
\ccMethod{IntegralDomainWithoutDivision 
            operator*=(const IntegralDomainWithoutDivision &a);}{}

Equality comparable: 

\ccFunction{bool 
            operator==(const IntegralDomainWithoutDivision &a, 
                      const IntegralDomainWithoutDivision &b);}{}
\ccGlue
\ccFunction{bool 
            operator!=(const IntegralDomainWithoutDivision &a, 
                      const IntegralDomainWithoutDivision &b);}{}
\ccGlue
 
\ccRefines
 \ccc{DefaultConstructible} \\
 \ccc{EqualityComparable}\\
 \ccc{Assignable}\\
 \ccc{FromSmallIntConstructible}

\ccSeeAlso

\ccRefIdfierPage{IntegralDomainWithoutDivision}\\
\ccRefIdfierPage{IntegralDomain}\\
\ccRefIdfierPage{UniqueFactorizationDomain}\\
\ccRefIdfierPage{EuclideanRing}\\
\ccRefIdfierPage{Field}\\
\ccRefIdfierPage{FieldWithSqrt}\\
\ccRefIdfierPage{FieldWithKthRoot}\\
\ccRefIdfierPage{FieldWithRootOf}\\
\ccRefIdfierPage{AlgebraicStructureTraits}\\

\ccHasModels
\CC\ built-in number types \\
\ccc{CGAL::Gmpq} \\
\ccc{CGAL::Gmpz} \\
\ccc{CGAL::Gmpzf} \\

\ccc{mpz_class} \\
\ccc{mpq_class} \\
%\ccc{mpf_class} \\

\ccc{leda_integer} \\
\ccc{leda_rational} \\
\ccc{leda_bigfloat} \\
\ccc{leda_real} \\

\ccc{CORE::BigInt} \\
\ccc{CORE::BigRat} \\
%\ccc{CORE::BigFloat} \\
\ccc{CORE::Expr} \\

%\ccc{CGAL::Interval_nt} \\
%\ccc{CGAL::Interval_nt_advanced} \\

\ccc{CGAL::MP_Float} \\
\ccc{CGAL::Lazy_exact_nt< NT >} (depends on NT) \\
\ccc{CGAL::Quotient< NT >} \\

\ccc{CGAL::Sqrt_extension< NT, Root >} \\
%\ccc{CGAL::Polynomial< T >} \\

\end{ccRefConcept}