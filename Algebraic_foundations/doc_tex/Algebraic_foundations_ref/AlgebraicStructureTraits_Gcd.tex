\begin{ccRefConcept}{AlgebraicStructureTraits::Gcd}

\ccDefinition

\ccc{AdaptableBinaryFunction} providing the gcd. 

The greatest common divisor ($gcd$) of ring elements $x$ and $y$ is the unique 
ring element $d$ (up to a unit) with the property that any common divisor of 
$x$ and $y$ also divides $d$. (In other words: $d$ is the greatest lower bound 
of $x$ and $y$ in the partial order of divisibility.) We demand the $gcd$ to be 
unit-normal (i.e. have unit part 1). $gcd(0,0)$ is defined as $1$.

(TODO: This is the old CGAL def: 
function gcd computes the greatest common divisor of two numbers.
If \ccc{x} is 0, the function returns \ccc{y}.
\ccPrecond: \ccc{y} is not zero.
 )

 
\ccRefines 

\ccc{AdaptableBinaryFunction} 

\ccTypes

\ccSetThreeColumns{xxxxxxxxxxxxxxxxxxxxxxxxxxxxxxxxxxxxxxxxxxxxxxxxxxxxxx}{xxx}{}
\ccTypedef{typedef AlgebraicStructureTraits::Algebraic_structure result_type;}{}\ccGlue
\ccTypedef{typedef AlgebraicStructureTraits::Algebraic_structure first_argument_type;}{}\ccGlue
\ccTypedef{typedef AlgebraicStructureTraits::Algebraic_structure second_argument_type;}{}

\ccOperations
\ccSetThreeColumns{xxxxxxxxxx}{xxxxx}{}
\ccMethod{result_type operator()(
        const first_argument_type&  x, 
        const second_argument_type& y);}{ returns  $gcd(x,y)$. }

%\ccHasModels

\ccSeeAlso

\ccRefIdfierPage{AlgebraicStructureTraits}

\end{ccRefConcept} 
