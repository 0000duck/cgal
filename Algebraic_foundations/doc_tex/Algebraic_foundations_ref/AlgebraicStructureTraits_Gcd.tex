\begin{ccRefFunctionObjectConcept}{AlgebraicStructureTraits::Gcd}

\ccDefinition

\ccc{AdaptableBinaryFunction} providing the gcd. 

The greatest common divisor ($gcd$) of ring elements $x$ and $y$ is the unique 
ring element $d$ (up to a unit) with the property that any common divisor of 
$x$ and $y$ also divides $d$. (In other words: $d$ is the greatest lower bound 
of $x$ and $y$ in the partial order of divisibility.) We demand the $gcd$ to be 
unit-normal (i.e. have unit part 1). 

$gcd(0,0)$ is defined as $0$, since $0$ is the greatest element with respect 
to the partial order of divisibility. This is because an element $a \in R$ is said to divide $b \in R$, iff $\exists r \in R$ such that $a \cdot r = b$. 
Thus, $0$ is divided by every element of the Ring, in particular by itself. 

\ccRefines 

\ccc{AdaptableBinaryFunction} 

\ccTypes
\ccNestedType{result_type} 
        { Is \ccc{AlgebraicStructureTraits::Type}.}
\ccGlue
\ccNestedType{first_argument} 
        { Is \ccc{AlgebraicStructureTraits::Type}.}
\ccGlue
\ccNestedType{second_argument} 
        { Is \ccc{AlgebraicStructureTraits::Type}.}

\ccOperations
\ccCreationVariable{gcd}
\ccThree{xxxxxxxxxxx}{xxxxxxxxxxx}{}

\ccMethod{result_type operator()(first_argument_type  x, 
                                 second_argument_type y);}
        { returns  $gcd(x,y)$. }

\ccMethod{template <class NT1, class NT2> result_type operator()(NT1  x, NT2  y);}
         {This operator is defined if \ccc{NT1} and \ccc{NT2} are \ccc{ExplicitInteroperable} 
          with coercion type \ccc{AlgebraicStructureTraits::Type}. }

%\ccHasModels

\ccSeeAlso

\ccRefIdfierPage{AlgebraicStructureTraits}

\end{ccRefFunctionObjectConcept} 
