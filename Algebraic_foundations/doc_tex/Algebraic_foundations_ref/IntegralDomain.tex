\begin{ccRefConcept}{IntegralDomain}


\ccDefinition

\ccc{IntegralDomain} refines \ccc{IntegralDomainWithoutDivision} by 
providing an integral division.\\
{\bf Note:} The concept does not require the operator / for this operation. 
We intend to reserve the operator syntax for use with a \ccc{Field}.
%Nevertheless, some non-field models of IntegralDomain have one. 



Moreover, \ccc{CGAL::Algebraic_structure_traits< IntegralDomain >} is a model of 
\ccc{AlgebraicStructureTraits} providing:\\
 
- \ccc{CGAL::Algebraic_structure_traits< IntegralDomain >::Algebraic_type} derived from \ccc{Integral_domain_tag} \\
- \ccc{CGAL::Algebraic_structure_traits< IntegralDomain >::Integral_div} an \ccc{AdaptableBinaryFunction}\\
  
\ccRefines
 \ccc{IntegralDomainWithoutDivision}

\ccSeeAlso 

\ccRefIdfierPage{IntegralDomainWithoutDivision}\\
\ccRefIdfierPage{IntegralDomain}\\
\ccRefIdfierPage{UniqueFactorizationDomain}\\
\ccRefIdfierPage{EuclideanRing}\\
\ccRefIdfierPage{Field}\\
\ccRefIdfierPage{FieldWithSqrt}\\
\ccRefIdfierPage{AlgebraicStructureTraits}\\

\ccHasModels

\end{ccRefConcept}