\begin{ccRefConcept}{EuclideanRing}


\ccDefinition

A model of \ccc{EuclideanRing} represents an euclidean ring (or Euclidean domain). 
It is an \ccc{UFDomain} that affords a suitable notion of minimality of remainders 
such that given $x$ and $y != 0$ we obtain an (almost) unique solution to 
$x = qy + r by$ demanding that a solution $(q,r)$ is chosen to minimize $r$. 
In particular, $r$ is chosen to be $0$ if possible.

Moreover, \ccc{CGAL::Algebraic_structure_traits< EuclideanRing >} is a model of 
\ccc{AlgebraicStructureTraits} providing:\\
 
- \ccc{CGAL::Algebraic_structure_traits< EuclideanRing >::Algebraic_type} derived from \ccc{UFDomain_tag} \\
- \ccc{CGAL::Algebraic_structure_traits< EuclideanRing >::Mod} an \ccc{AdaptableBinaryFunction}\\
- \ccc{CGAL::Algebraic_structure_traits< EuclideanRing >::Div} an \ccc{AdaptableBinaryFunction}\\
- \ccc{CGAL::Algebraic_structure_traits< EuclideanRing >::Div_mod}\\




\ccHeading{Remarks}

The most prominent example of a Euclidean ring are the integers. 
Whenever both $x$ and $y$ are positive, then it is conventional to choose 
the smallest positive remainder $r$. 


{\em \small
In other cases, there seems to be no universally observed convention on how to choose 
the sign. (In particular, the ISO C++ Standard fixes none for the modulo operation 
on the built-in integral types.)

Todo: Decide what to do about the sign ambiguity for the Mod operation of Euclidean rings.
Proposal: choose the unit normal remainder. 
}

%Another important example of a Euclidean ring are univariate polynomials over a field. 
%Here the degree of $r$ is to be minimized. There is no remaining ambiguity.

%Any Euclidean ring is necessarily a \ccc{UFDomain}, and the ability to take remainders allows 
%to use the Euclidean Algorithm to compute gcds.   

\ccRefines
 \ccc{UFDomain}

\ccSeeAlso

\ccRefIdfierPage{IntegralDomainWithoutDiv}\\
\ccRefIdfierPage{IntegralDomain}\\
\ccRefIdfierPage{UFDomain}\\
\ccRefIdfierPage{EuclideanRing}\\
\ccRefIdfierPage{Field}\\
\ccRefIdfierPage{FieldWithSqrt}\\
\ccRefIdfierPage{AlgebraicStructureTraits}\\

\ccHasModels


\end{ccRefConcept}