\begin{ccRefClass}{Advancing_front_surface_reconstruction<Kernel,Delaunay_3>}

\ccDefinition
  
The class \ccRefName\ extracts a surface from a 3D Delaunay triangulation.

\ccInclude{CGAL/Advancing_front_surface_reconstruction.h}

\ccParameters

The parameter for \ccc{Delaunay_3} must be a 3D Delaunay triangulation where \ccc{CGAL::AFSR_vertex_base_3}
or \ccc{CGAL::AFSR_vertex_base_with_id_3} and \ccc{CGAL::AFSR_cell_base_3} must be blended in the vertex and face class. 



\ccTypes

\ccTypedef{typedef Delaunay_3 Triangulation_3;}{}
\ccTypedef{typedef CGAL::Triple<Cell_handle, int,int> Edge;}{For \ccc{(ch,i,j)}, this is the edge between vertices \ccc{i} and \ccc{j} in cell \ccc{*ch}.}

\ccTypedef{typedef std::pair<Edge,int> Edge_incident_facet;}{For \ccc{((ch,i,j),k)}, this is the facet adjacent to the edge \ccc{(,i,j)}, and
                                                             opposite to vertex \ccc{k}, in the cell \ccc{*ch}.}
\ccGlue
\ccNestedType{TDS_2}{The type of the 2D triangulation data structure describing the reconstructed surface.}
\ccNestedType{TDS_2::Vertex}{It is model of the concept \ccc{TriangulationDataStructure_2::Vertex} and has additionally the 
                             method \ccc{vertex_3()} that returns a \ccc{Triangulation_3::Vertex_handle} to the associated 3D vertex.}
\ccNestedType{TDS_2::Face}{It is model of the concept \ccc{TriangulationDataStructure_2::Face} and  has additionally the
                            method \ccc{facet()} that returns the associated \ccc{Triangulation_3::Facet}.}

\ccNestedType{Boundary_iterator}{This forward iterator allows to visit all contours. It 
                                  visits the entry point of the contour twice. This allows to
                                  detect that the traversal of the border is finished. One more increment
                                  brings us to the vertex on the next boundary. The value type of the iterator is \ccc{Triangulation_3::Vertex_handle}.} 

\ccNestedType{Outlier_iterator}{This bidirectional iterator allows to enumerate all points that were removed
                                from the 3D Delaunay triangulation during the surface reconstruction. The value type
                                of the iterator is \ccc{Kernel::Point_3}.}
  
\ccCreation
\ccCreationVariable{es}

\ccConstructor{Advancing_front_surface_reconstruction(Delaunay_3& del, AFSR_options opt);}{}

\ccOperations

\ccMemberFunction{const TDS_2& tds_2() const;}{Returns a const reference to the reconstructed surface. }
\ccMemberFunction{bool is_on_surface(TDS_2::Vertex_handle vh;}{Returns true, iff the vertex is on the reconstructed surface.}


\ccMemberFunction{const Triangulation_3& triangulation() const;}{Returns a const reference to the 3D Delaunay triangulation.}

\ccMemberFunction{Boundary_iterator boundaries_begin() const;}{}
\ccMemberFunction{Boundary_iterator boundaries_end() const;}{}

\ccMemberFunction{Outlier_iterator outliers_begin() const;}{}
\ccMemberFunction{Outlier_iterator outliers_end() const;}{}

\ccMemberFunction{bool has_on_surface(Triangulation_3::Facet f) const;}
                 {Returns \ccc{true}, iff the facet is on the reconstructed surface.}

\ccMemberFunction{Edge_incident_facet next(const Edge_incident_facet& f) const;}{returns the next facet around the edge.}
\ccMemberFunction{Edge_incident_facet previous(const Edge_incident_facet& f) const;}{returns the previous facet around the edge.}

\ccMemberFunction{Facet next_surface_facet(const Edge_incident_facet& f) const}{returns the next facet ariund the edge,
                                           which is on the reconstructed surface.}

%\ccSeeAlso

%\ccRefIdfierPage{}  

\end{ccRefClass}
