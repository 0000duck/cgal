% to be written by pierre

% pierre: I think the section is currently either way too long,
% or too short if one wants enough details to implement.
% Andy, you might want to put just the two first blocks
% without giving more details since one wants to emphasis on
% the generic one


The quad-triangle subdivision scheme was introduced by
Levin~\cite{l-pg-03}, then Stam and Loop~\cite{qts}. It applies to
polygon meshes and basically features Loop subdivision on triangles
and Catmull-Clark subdivision on polygons of the control mesh (see
Fig.\ref{fig:quad-triangle}). After one iteration of subdivision the
subdivided model is only composed of triangles and quads.

\begin{figure}
    \centering{\includegraphics[width=7.0cm]{figs/quad-triangle}}
    \caption{Quad-triangle subdivision scheme.}
    \label{fig:quad-triangle}
\end{figure}

One solution for implementing quad-triangle scheme is to use an
auxiliary modifier class that supports the incremental construction of
polyhedral surfaces. The incremental construction starts with a list
of all point coordinates and concludes with a list of all facet
polygons given by a sequence of vertex indices.  Beside being of
interest for constructing polyhedral surfaces from indexed file
formats, the incremental builder can be used from any other data
structures, namely from an input mesh.\\

The vertex, face and halfedge class primitives of the Polyhedron are
initially enriched with an integer tag. The incremental builder first
iterates over all primitives of the input mesh and circulates around
their incident primitives to deduce the positions of the new vertices
from the geometry rules. The integer tags keep track of the index of
the new vertices in the input mesh, allowing a straightforward assembly
of all indexed triangles, resp. quads as the Loop, resp. barycentric
subdivision refinement rules are applied for all facets of the input
model.

