% +------------------------------------------------------------------------+
% | Reference manual page: box_self_intersection_d.tex
% +------------------------------------------------------------------------+
% | 19.05.2004   Lutz Kettner, Andreas Meyer
% | Package: Package
% | 
\RCSdef{\RCSboxintersectiondRev}{$Id$}
\RCSdefDate{\RCSboxintersectiondDate}{$Date$}
% |
%%RefPage: end of header, begin of main body
% +------------------------------------------------------------------------+


\begin{ccRefFunction}{box_self_intersection_d}
\ccIndexSubitemDef{self-intersection}{iso-oriented boxes}
\ccIndexSubitem{intersection}{self-intersection}

\ccDefinition
  
The function \ccRefName\ computes the pairwise intersecting boxes
in a sequence of iso-oriented boxes in arbitrary dimension.
The sequence of boxes is given with as a random-access iterator
range and will be reordered in the course of the algorithm. For each
intersecting pair of boxes a \ccc{callback} function object is called
with the two intersecting boxes as argument; the first argument is a
box from the sequence, the second argument is a copy of a box from the
sequence.  The performance of the algorithm can be tuned with a
\ccc{cutoff} parameter, see the implementation section of the
\ccc{CGAL::box_intersection_d} function on page
\pageref{ccRef_CGAL::box_intersection_d}.  

The algorithm creates a second copy of the boxes and reorders the
boxes in the course of the algorithm. Now, depending on the size of a
box it can be faster to copy the boxes, or to work with pointers to
boxes and copy only pointers. We offer automatic support for both
options. To simplify the description, let us call the \ccc{value_type}
of the iterators \emph{box handle}.  The \emph{box handle\/} can
either be our box type itself or a pointer (or const pointer) to the
box type.

A $d$-dimensional iso-oriented box is defined as the
\ccHtmlNoLinksFrom{Cartesian} product of $d$ intervals. We call the
box \emph{half-open} if the $d$ intervals $\{ [lo_i,hi_i) \,|\, 0 \leq
i < d\}$ are half-open intervals, and we call the box \emph{closed} if
the $d$ intervals $\{ [lo_i,hi_i] \,|\, 0 \leq i < d\}$ are closed
intervals. Note that closed boxes support zero-width boxes and they
can intersect at their boundaries, while non-empty half-open boxes
always have a positive \ccHtmlNoLinksFrom{volume} and they only intersect iff their
interiors overlap.  The distinction between closed or half-open boxes
does not require a different representation of boxes, just a different
interpretation when comparing boxes, which is selected with the
\ccc{topology} parameter and its two values,
\ccc{CGAL::Box_intersection_d::HALF_OPEN} and
\ccc{CGAL::Box_intersection_d::CLOSED}.

In addition, a box has an unique \ccc{id}-number. It is used to order
boxes consistently in each dimension even if boxes have identical
coordinates. In consequence, the algorithm guarantees that a pair of
intersecting boxes is reported only once.  This self-intersection
function creates internally a second copy of the box sequence. The
copying has to preserve the \ccc{id}-number of boxes.  Note that this
implies that the address of the box is not sufficient for the
\ccc{id}-number if boxes are copied by value. Boxes of equal
\ccc{id}-number are not reported as intersecting pairs since they are
always intersecting trivially.

The algorithm uses a traits class of the \ccc{BoxIntersectionTraits_d}
concept to access the boxes. A default traits class is provided that
assumes that the box type is a model of the \ccc{BoxIntersectionBox_d}
concept and that the box handle, i.e., the iterators value type, is
identical to the box type or a pointer to the box type.

\ccInclude{CGAL/box_intersection_d.h}

\ccThree{void}{box_intersection_custom_}{}
%\def\ccLongParamLayout{\ccTrue}
\ccFunction{template< class RandomAccessIterator, class Callback >
void box_self_intersection_d(
    RandomAccessIterator begin, RandomAccessIterator end,
    Callback callback,
    std::ptrdiff_t cutoff = 10,
    Box_intersection_d::Topology topology = Box_intersection_d::CLOSED);
}{%
    Invocation of box intersection with default box traits
    \ccc{CGAL::Box_intersection_d::Box_traits_d<Box_handle>}, where
    \ccc{Box_handle} corresponds to the iterator value type  of
    \ccc{RandomAccessIterator}.}
 
\ccFunction{template< class RandomAccessIterator,
            class Callback, class BoxTraits >
void box_self_intersection_d(
    RandomAccessIterator begin, RandomAccessIterator end,
    Callback callback,
    BoxTraits box_traits,
    std::ptrdiff_t cutoff = 10,
    Box_intersection_d::Topology topology = Box_intersection_d::CLOSED);
}{Invocation with custom box traits.}


\ccRequirements

\begin{itemize}
  \item
    \ccc{RandomAccessIterator} must be a mutable random-access
    iterator. We call its value type \ccc{Box_handle} in the following.
  \item
    \ccc{Callback} must be of the \ccc{BinaryFunction} concept. 
    The \ccc{Box_handle} must be convertible to both argument types. The
    return type is not used and can be \ccc{void}.
  \item
    The \ccc{Box_handle} must be a model of the \ccc{Assignable} concept.
  \item
    In addition, if the default box traits is used the \ccc{Box_handle} must
    be a class type \ccc{T} or a pointer to a class type \ccc{T}, where
    \ccc{T} must be a model of the \ccc{BoxIntersectionBox_d} concept.
    In both cases, the default box traits specializes to a suitable
    implementation.
  \item
    \ccc{BoxTraits} must be of the \ccc{BoxIntersectionTraits_d} concept.
\end{itemize}

\ccSeeAlso

\ccRefIdfierPage{CGAL::box_intersection_d}\\
\ccRefIdfierPage{CGAL::box_self_intersection_all_pairs_d}

\ccRefIdfierPage{CGAL::Box_intersection_d::Box_traits_d<BoxHandle>}\\
\ccRefConceptPage{BoxIntersectionBox_d}\\
\ccRefConceptPage{BoxIntersectionTraits_d}

\ccImplementation

See the implementation section of the \ccc{CGAL::box_intersection_d}
function on page~\pageref{ccRef_CGAL::box_intersection_d}.

\ccExample

The box implementation provided with
\ccc{CGAL::Box_intersection_d::Box_d<double,2>} has a special
constructor for the \cgal\ bounding box type \ccc{CGAL::Bbox_2} (and
similar for dimension 3). We use this in the example to create $3
\times 3$ \ccc{boxes} in a grid layout.

The default policy of the box type implements the \ccc{id}-number with
an explicit counter in the boxes, which is the default choice since it
always works. We use the \ccc{id}-number in our callback function to
report the result of the \ccHtmlNoLinksFrom{intersection} algorithm call. The result will
be 20 pairwise intersections, but the order in which they are reported
is non-intuitive.

\ccIncludeExampleCode{Box_intersection_d/minimal_self.cpp}

\end{ccRefFunction}

% +------------------------------------------------------------------------+
%%RefPage: end of main body, begin of footer
% EOF
% +------------------------------------------------------------------------+

