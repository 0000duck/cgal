% +------------------------------------------------------------------------+
% | Reference manual page: Atom.tex
% +------------------------------------------------------------------------+
% | 09.04.2009   Pierre Alliez, Laurent Saboret, Gael Guennebaud
% | Package: PDB
% |
\RCSdef{\RCSAtomRev}{$Id$}
\RCSdefDate{\RCSAtomDate}{$Date$}
% |
\ccRefPageBegin
%%RefPage: end of header, begin of main body
% +------------------------------------------------------------------------+


\begin{ccRefClass}[PDB::]{Atom}  %% add template arg's if necessary

%% \ccHtmlCrossLink{}     %% add further rules for cross referencing links
%% \ccHtmlIndexC[class]{} %% add further index entries

\ccDefinition

% The section below is automatically generated. Do not edit!
%START-AUTO(\ccDefinition)

A class repesenting an atom.

%END-AUTO(\ccDefinition)
  
% The section below is automatically generated. Do not edit!
%START-AUTO(\ccInclude)

\ccInclude{CGAL/PDB/Atom.h}

%END-AUTO(\ccInclude)

\ccIsModel

\ccTypes

% The section below is automatically generated. Do not edit!
%START-AUTO(\ccTypes)

\ccNestedType{Index}
{
}
\ccGlue

%END-AUTO(\ccTypes)

\ccConstants

% The section below is automatically generated. Do not edit!
%START-AUTO(\ccConstants)

\ccEnum{enum Type { INVALID = 0, C, N, H, O, S, P, FE, PT };}
{
The type (element) of an atom. The currently supported types are C,N,H,O,S, INVALID.
\ccCommentHeading{Values}
\begin{description}
\item[INVALID
]\item[C
]\item[N
]\item[H
]\item[O
]\item[S
]\item[P
]\item[FE
]\item[PT
]\end{description}
}
\ccGlue

%END-AUTO(\ccConstants)

\ccCreation
\ccCreationVariable{a}  %% choose variable name

% The section below is automatically generated. Do not edit!
%START-AUTO(\ccCreation)

\ccConstructor{Atom();}
{
Construct and invalid atom.
}
\ccGlue

%END-AUTO(\ccCreation)

\ccOperations

% The section below is automatically generated. Do not edit!
%START-AUTO(\ccOperations)

\ccMethod{const Point& point() const;}
{
Cartesian coordinates (x,y,z) for the atom.
}
\ccGlue
\ccMethod{void set_point(const Point& pt);}
{
}
\ccGlue
\ccMethod{bool operator ==(const Atom& al) const;}
{
}
\ccGlue
\ccMethod{bool operator!=(const Atom& al) const;}
{
}
\ccGlue
\ccMethod{const float& occupancy() const;}
{
The PDB occupancy field.
}
\ccGlue
\ccMethod{void set_occupancy(const float& k);}
{
}
\ccGlue
\ccMethod{const float& temperature_factor() const;}
{
The PDB temperature factor field.
}
\ccGlue
\ccMethod{void set_temperature_factor(const float& k);}
{
}
\ccGlue
\ccMethod{const std::string& segment_id() const;}
{
The PDB segment ID char.
}
\ccGlue
\ccMethod{void set_segment_id(const std::string& k);}
{
}
\ccGlue
\ccMethod{const std::string& element() const;}
{
The PDB element field.
}
\ccGlue
\ccMethod{void set_element(const std::string& k);}
{
}
\ccGlue
\ccMethod{const std::string& charge() const;}
{
The PDB charge field.
}
\ccGlue
\ccMethod{void set_charge(const std::string& k);}
{
}
\ccGlue
\ccMethod{const Type& type() const;}
{
The type of the atoms (basically what element).
}
\ccGlue
\ccMethod{void set_type(const Type& k);}
{
}
\ccGlue
\ccMethod{const double& radius() const;}
{
Returns the van der Waals radius of the atom.
Values take from the wikipedia so beware.
}
\ccGlue
\ccMethod{const Index& index() const;}
{
This is a label which identifies an Atom uniquely within some scale.
The uniqueness is only valid if working within the object which assigned the indices, and if nothing has changed since the corresponding \ccc{index_atoms}() function was called.
}
\ccGlue
\ccMethod{void set_index(Index i) const;}
{
note const
}
\ccGlue
\ccMethod{static Type string_to_type(const char * c);}
{
[static] \\
}
\ccGlue
\ccMethod{void swap_with(Atom& o);}
{
}
\ccGlue

%END-AUTO(\ccOperations)

\ccSeeAlso

\ccExample

A short example program.
Instead of a short program fragment, a full running program can be
included using the 
\verb|\ccIncludeExampleCode{PDB/Atom.C}| 
macro. The program example would be part of the source code distribution and
also part of the automatic test suite.

\begin{ccExampleCode}
void your_example_code() {
}
\end{ccExampleCode}

%% \ccIncludeExampleCode{PDB/Atom.C}

\end{ccRefClass}

% +------------------------------------------------------------------------+
%%RefPage: end of main body, begin of footer
\ccRefPageEnd
% EOF
% +------------------------------------------------------------------------+

