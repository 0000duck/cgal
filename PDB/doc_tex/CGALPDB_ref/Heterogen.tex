% +------------------------------------------------------------------------+
% | Reference manual page: Heterogen.tex
% +------------------------------------------------------------------------+
% | 09.04.2009   Pierre Alliez, Laurent Saboret, Gael Guennebaud
% | Package: PDB
% |
\RCSdef{\RCSHeterogenRev}{$Id$}
\RCSdefDate{\RCSHeterogenDate}{$Date$}
% |
\ccRefPageBegin
%%RefPage: end of header, begin of main body
% +------------------------------------------------------------------------+


\begin{ccRefClass}[PDB::]{Heterogen}  %% add template arg's if necessary

%% \ccHtmlCrossLink{}     %% add further rules for cross referencing links
%% \ccHtmlIndexC[class]{} %% add further index entries

\ccDefinition

% The section below is automatically generated. Do not edit!
%START-AUTO(\ccDefinition)

The class representing a residue or nucleotide.

All the information concerning atoms and bonds for each monomer is stored here. To add atoms to monomers, new monomers, or bonds to monomers, look in \ccc{Monomer_data}.cpp. There is documentation there of what you need to do.

%END-AUTO(\ccDefinition)
  
% The section below is automatically generated. Do not edit!
%START-AUTO(\ccInclude)

\ccInclude{CGAL/PDB/Heterogen.h}

%END-AUTO(\ccInclude)

\ccIsModel

\ccTypes

% The section below is automatically generated. Do not edit!
%START-AUTO(\ccTypes)

\ccNestedType{Atom_key}
{
}
\ccGlue
\ccNestedType{AtomsMap}
{
}
\ccGlue
\ccNestedType{Atoms}
{
}
\ccGlue
\ccNestedType{Atom_consts}
{
}
\ccGlue
\ccNestedType{Bond}
{
A bond between two atoms in a monomer.
The ints refer the the atom index.
}
\ccGlue
\ccNestedType{Bonds}
{
Return a list of all the bonds in the monomer.
}
\ccGlue

%END-AUTO(\ccTypes)

\ccCreation
\ccCreationVariable{a}  %% choose variable name

% The section below is automatically generated. Do not edit!
%START-AUTO(\ccCreation)

\ccConstructor{Heterogen();}
{
Default constructor. Makes and invalid monomer.
}
\ccGlue
\ccConstructor{Heterogen(std::string name);}
{
Make a monomer of a given type.
}
\ccGlue
\ccConstructor{Heterogen(const Heterogen& o);}
{
}
\ccGlue

%END-AUTO(\ccCreation)

\ccOperations

% The section below is automatically generated. Do not edit!
%START-AUTO(\ccOperations)

\ccMethod{Heterogen& operator =(const Heterogen& o);}
{
}
\ccGlue
\ccMethod{void copy_from(const Heterogen& o);}
{
}
\ccGlue
\ccMethod{void swap_with(Heterogen& o);}
{
}
\ccGlue
\ccMethod{std::string type() const;}
{
The label for the monomer.
}
\ccGlue
\ccMethod{char chain() const;}
{
}
\ccGlue
\ccMethod{void set_chain(const char& k);}
{
}
\ccGlue
\ccMethod{Atoms atoms();}
{
}
\ccGlue
\ccMethod{Atom_consts atoms() const;}
{
}
\ccGlue
\ccMethod{bool contains(Atom_key k) const;}
{
}
\ccGlue
\ccMethod{Atoms::iterator::reference get(Atom_key k);}
{
}
\ccGlue
\ccMethod{Atoms::iterator find(Atom_key k);}
{
}
\ccGlue
\ccMethod{Atom_consts::const_iterator::value_type get(Atom_key k) const;}
{
}
\ccGlue
\ccMethod{Atom_consts::const_iterator find(Atom_key k) const;}
{
}
\ccGlue
\ccMethod{void insert(Atom_key k, const Atom& m);}
{
}
\ccGlue
\ccMethod{void dump(std::ostream& out) const;}
{
Write it for debugging.
}
\ccGlue
\ccMethod{std::ostream& write(std::ostream& out) const;}
{
Write it for debugging.
}
\ccGlue
\ccMethod{int write(std::string name, int num, int start_index, std::ostream& out) const;}
{
Write the lines for a pdb file.
Indices start at the \ccc{start_index} and the new \ccc{start_index} is returned.
}
\ccGlue
\ccMethod{Bonds bonds() const;}
{
}
\ccGlue
\ccMethod{bool connect(Atom::Index a, Atom::Index b);}
{
}
\ccGlue

%END-AUTO(\ccOperations)

\ccSeeAlso

\ccExample

A short example program.
Instead of a short program fragment, a full running program can be
included using the 
\verb|\ccIncludeExampleCode{PDB/Heterogen.C}| 
macro. The program example would be part of the source code distribution and
also part of the automatic test suite.

\begin{ccExampleCode}
void your_example_code() {
}
\end{ccExampleCode}

%% \ccIncludeExampleCode{PDB/Heterogen.C}

\end{ccRefClass}

% +------------------------------------------------------------------------+
%%RefPage: end of main body, begin of footer
\ccRefPageEnd
% EOF
% +------------------------------------------------------------------------+

