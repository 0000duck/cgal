% +------------------------------------------------------------------------+
% | Reference manual page: Chain.tex
% +------------------------------------------------------------------------+
% | 10.04.2009   Author
% | Package: PDB
% |
\RCSdef{\RCSChainRev}{$Id: header.tex 40270 2007-09-07 15:29:10Z lsaboret $}
\RCSdefDate{\RCSChainDate}{$Date: 2007-09-07 08:29:10 -0700 (Fri, 07 Sep 2007) $}
% |
\ccRefPageBegin
%%RefPage: end of header, begin of main body
% +------------------------------------------------------------------------+


\begin{ccRefClass}[PDB::]{Chain}  %% add template arg's if necessary

%% \ccHtmlCrossLink{}     %% add further rules for cross referencing links
%% \ccHtmlIndexC[class]{} %% add further index entries

\ccDefinition

% The section below is automatically generated. Do not edit!
%START-AUTO(\ccDefinition)

A class representing a single chain of a protein.

%END-AUTO(\ccDefinition)
  
% The section below is automatically generated. Do not edit!
%START-AUTO(\ccInclude)

\ccInclude{CGAL/PDB/Chain.h}

%END-AUTO(\ccInclude)

\ccIsModel

\ccTypes

% The section below is automatically generated. Do not edit!
%START-AUTO(\ccTypes)

\paragraph[{\ccc{Monomer_key}}]{typedef Label$<$Chain$>$ \ccc{Monomer_key}}
The type for storing residue indices in the PDB.
\paragraph[{\ccc{IR_key}}]{typedef Label$<$\ccc{IR_tag}$>$ \ccc{IR_key}}
\paragraph[{Container}]{typedef \ccc{small_map<Monomer_pair>} Container}
\paragraph[{Monomers}]{typedef \ccc{boost::iterator_range<Container::iterator>} Monomers}
\paragraph[{\ccc{Monomer_consts}}]{typedef \ccc{boost::iterator_range<Container::const_iterator>} \ccc{Monomer_consts}}
\paragraph[{Bond}]{typedef \ccc{std::pair<Bond_endpoint, Bond_endpoint>} Bond}
A chemical bond within the protein.
\paragraph[{Atoms}]{typedef \ccc{boost::iterator_range<internal::Nested_iterator<Iterator_traits>} $>$ Atoms}
An iterator to iterate through all the atoms of the protein
\paragraph[{\ccc{Atom_consts}}]{typedef \ccc{boost::iterator_range<internal::Nested_iterator<Iterator_const_traits>} $>$ \ccc{Atom_consts}}
\paragraph[{Bonds}]{typedef \ccc{boost::iterator_range<Bond_it>} Bonds}

%END-AUTO(\ccTypes)

\ccCreation
\ccCreationVariable{a}  %% choose variable name

% The section below is automatically generated. Do not edit!
%START-AUTO(\ccCreation)

\paragraph[{Chain}]{Chain ()}
Default.

%END-AUTO(\ccCreation)

\ccOperations

% The section below is automatically generated. Do not edit!
%START-AUTO(\ccOperations)

\paragraph[{monomers}]{Monomers monomers ()}
\paragraph[{monomers}]{\ccc{Monomer_consts} monomers () const}
\paragraph[{insert}]{void insert (\ccc{Monomer_key} k, const Monomer\& m)}
\paragraph[{atoms}]{Atoms atoms ()}
\paragraph[{atoms}]{\ccc{Atom_consts} atoms () const}
\paragraph[{\ccc{number_of_atoms}}]{unsigned int \ccc{number_of_atoms} () const}
This is non-const time.
\paragraph[{bonds}]{Bonds bonds () const}
\paragraph[{\ccc{number_of_bonds}}]{unsigned int \ccc{number_of_bonds} () const}
This is non-const time.
\paragraph[{sequence}]{\ccc{std::vector<Monomer::Type>} sequence () const}
The sequence of residue types.
\paragraph[{write}]{int write (char chain, int \ccc{start_index}, \ccc{std::ostream}\& out) const}
Write as part of pdb file.
\paragraph[{\ccc{write_pdb}}]{void \ccc{write_pdb} (\ccc{std::ostream}\& out) const}
Write a pdb file.
See \ccc{check_protein}.cpp for an example of using this to write a pdb file.
\paragraph[{dump}]{void dump (\ccc{std::ostream}\& out) const}
Dump as human readable.
\paragraph[{write}]{\ccc{std::ostream}\& write (\ccc{std::ostream}\& out) const}
Dump as human readable.
\paragraph[{contains}]{bool contains (\ccc{Monomer_key} k) const}
\paragraph[{get}]{\ccc{Monomers::iterator::reference} get (\ccc{Monomer_key} k)}
\paragraph[{find}]{\ccc{Monomers::iterator} find (\ccc{Monomer_key} k)}
\paragraph[{get}]{\ccc{Monomer_consts::const_iterator::value_type} get (\ccc{Monomer_key} k) const}
\paragraph[{find}]{\ccc{Monomer_consts::const_iterator} find (\ccc{Monomer_key} k) const}
\paragraph[{\ccc{has_bonds}}]{bool \ccc{has_bonds} () const}
Return whether bonds have been computed for this protein.
\paragraph[{\ccc{set_has_bonds}}]{void \ccc{set_has_bonds} (bool tf)}
Set whether the protein has bonds or not.
\paragraph[{name}]{const \ccc{std::string}\& name () const}
\paragraph[{\ccc{set_name}}]{void \ccc{set_name} (const \ccc{std::string}\& k)}

%END-AUTO(\ccOperations)

\ccSeeAlso

\ccExample

A short example program.
Instead of a short program fragment, a full running program can be
included using the 
\verb|\ccIncludeExampleCode{PDB/Chain.C}| 
macro. The program example would be part of the source code distribution and
also part of the automatic test suite.

\begin{ccExampleCode}
void your_example_code() {
}
\end{ccExampleCode}

%% \ccIncludeExampleCode{PDB/Chain.C}

\end{ccRefClass}

% +------------------------------------------------------------------------+
%%RefPage: end of main body, begin of footer
\ccRefPageEnd
% EOF
% +------------------------------------------------------------------------+

