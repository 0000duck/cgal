\begin{ccRefFunctionObjectConcept}{Kernel::ConstructIsoCuboid_3}
A model for this must provide:

\ccCreationVariable{fo}

\ccMemberFunction{Kernel::Iso_cuboid_3 operator()(const Kernel::Point_3 &p, 
                                                  const Kernel::Point_3 &q);}
            {introduces an iso-oriented cuboid  with diagonal
             opposite vertices $p$ and $q$ such that $p$ is the
             lexicographically smallest point in the cuboid.}

\ccMemberFunction{Kernel::Iso_cuboid_3 operator()(const Kernel::Point_3 &p,
                                                  const Kernel::Point_3 &q,
                                                  int);}
            {introduces an iso-oriented cuboid with diagonal
             opposite vertices $p$ and $q$.  The \ccc{int} argument value is
             only used to distinguish the two overloaded functions.
             \ccPrecond{$p.x()<=q.x()$, $p.y()<=q.y()$ and $p.z()<=q.z()$.}}

\ccMemberFunction{Kernel::Iso_cuboid_3
                  operator()(const Kernel::Point_3 &left, 
                             const Kernel::Point_3 &right,
                             const Kernel::Point_3 &bottom,
                             const Kernel::Point_3 &top,
                             const Kernel::Point_3 &far,
                             const Kernel::Point_3 &close);}
            {introduces an iso-oriented cuboid \ccVar\ whose
             minimal $x$ coordinate is the one of \ccc{left}, the
             maximal $x$ coordinate is the one of \ccc{right}, the
             minimal $y$ coordinate is the one of \ccc{bottom}, the
             maximal $y$ coordinate is the one of \ccc{top}, the
             minimal $z$ coordinate is the one of \ccc{far}, the
             maximal $z$ coordinate is the one of \ccc{close}.}

\ccRefines
\ccc{AdaptableFunctor} (with two arguments)

\ccSeeAlso
\ccRefIdfierPage{CGAL::Iso_cuboid_3<Kernel>}  \\

\end{ccRefFunctionObjectConcept}
