\begin{ccRefFunctionObjectConcept}{Kernel::ConstructCircle_2}
A model for this must provide:

\ccCreationVariable{fo}


\ccHidden
\ccMemberFunction{Kernel::Circle_2 operator()( );}{
        introduces an uninitialized variable  of type
        \ccClassTemplateName.}

\ccMemberFunction{Kernel::Circle_2 operator()( Kernel::Point_2  const& center, 
                               Kernel::FT  const& squared_radius,
                               Orientation const& orientation
                                                    = COUNTERCLOCKWISE);}{
        introduces a variable  of type \ccc{Kernel::Circle_2}.
        It is initialized to the circle with center \ccc{center},
        squared radius \ccc{squared_radius} and orientation
        \ccc{orientation}.
        \ccPrecond \ccc{orientation} $\neq$ \ccc{COLLINEAR}, and further,
                   \ccc{squared_radius} $\geq$ 0.}

\ccMemberFunction{Kernel::Circle_2 operator()( Kernel::Point_2 const& p,
                               Kernel::Point_2 const& q,
                               Kernel::Point_2 const& r);}{
        introduces a variable  of type \ccc{Kernel::Circle_2}.
        It is initialized to the unique circle which passes through
        the points \ccc{p}, \ccc{q} and \ccc{r}. The orientation of
        the circle is the orientation of the point triple \ccc{p},
        \ccc{q}, \ccc{r}.
        \ccPrecond \ccc{p}, \ccc{q}, and \ccc{r} are not collinear.}

\ccMemberFunction{Kernel::Circle_2 operator()( Kernel::Point_2  const& p, 
                               Kernel::Point_2  const& q,
                               Orientation const& orientation 
                                                    = COUNTERCLOCKWISE);}{
        introduces a variable  of type \ccc{Kernel::Circle_2}.
        It is initialized to the circle with diameter 
        $\ccTexHtml{\overline{pq}}{pq}$
        and orientation \ccc{orientation}.
        \ccPrecond \ccc{orientation} $\neq$ \ccc{COLLINEAR}.}

\ccMemberFunction{Kernel::Circle_2 operator()( Kernel::Point_2  const& center,
                               Orientation const& orientation
                                                    = COUNTERCLOCKWISE);}{
        introduces a variable  of type \ccc{Kernel::Circle_2}.
        It is initialized to the circle with center \ccc{center}, squared
        radius zero and orientation \ccc{orientation}.
        \ccPrecond \ccc{orientation} $\neq$ \ccc{COLLINEAR}.
        \ccPostcond .\ccc{is_degenerate()} = \ccc{true}.}

\ccHidden
\ccMemberFunction{Kernel::Circle_2 operator()( Kernel::Circle_2 const&);}{copy constructor.}

\ccRefines
\ccc{AdaptableFunctor} (with three arguments)

\ccSeeAlso
\ccRefIdfierPage{CGAL::Circle_2<Kernel>}  \\

\end{ccRefFunctionObjectConcept}
