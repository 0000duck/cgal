\begin{ccRefClass} {Vector_2<Kernel>}

\ccDefinition

An object of the class \ccRefName\ is a vector in the two-dimensional 
vector space $\R^2$. Geometrically spoken, a vector is the difference
of two points $p_2$, $p_1$ and denotes the direction and the distance
from   $p_1$ to $p_2$. 

\cgal\ defines a symbolic constant \ccStyle{NULL_VECTOR}. We 
will explicitly state where you can pass this constant as an argument
instead of a vector initialized with zeros.


\ccCreation
\ccCreationVariable{v}


\ccHidden\ccConstructor{Vector_2();}
             {introduces an uninitialized variable \ccVar.}

\ccHidden \ccConstructor{Vector_2(const Vector_2<Kernel> &w);}
            {copy constructor.}

\ccConstructor{Vector_2(const Point_2<Kernel> &a, const Point_2<Kernel> &b);}
            {introduces the vector $b-a$.}

\ccConstructor{Vector_2(const Segment_2<Kernel> &s);}
            {introduces the vector $s.target()-s.source()$.}

\ccConstructor{Vector_2(const Ray_2<Kernel> &r);}
            {introduces the vector having the same direction as $r$.}

\ccConstructor{Vector_2(const Line_2<Kernel> &l);}
            {introduces the vector having the same direction as $l$.}

\ccConstructor{Vector_2(const Null_vector &NULL_VECTOR);}
            {introduces a null vector \ccVar.}

\ccConstructor{Vector_2(const Kernel::RT &hx, const Kernel::RT &hy, const Kernel::RT &hw = RT(1));}
            {introduces a vector \ccVar\ initialized to $(hx/hw,hy/hw)$.
             \ccPrecond \ccTexHtml{$hw\neq 0$}{hw not equal to 0} }


\ccOperations

\ccHidden \ccMethod{Vector_2<Kernel> & operator=(const Vector_2<Kernel> &w);}
        {Assignment.}

\ccMethod{bool operator==(const Vector_2<Kernel> &w) const;}
       {Test for equality: two vectors are equal, iff their $x$ and $y$ 
        coordinates are equal. You can compare a vector with the
        \ccStyle{NULL_VECTOR}.}

\ccMethod{bool operator!=(const Vector_2<Kernel> &w) const;}
       {Test for inequality. You can compare a vector with the
        \ccStyle{NULL_VECTOR}.}

There are two sets of coordinate access functions, namely to the
homogeneous and to the \ccHtmlNoLinksFrom{Cartesian} coordinates. They can be used
independently from the chosen kernel model.

\ccMethod{Kernel::RT hx() const;}
       {returns the homogeneous $x$ coordinate.}
\ccGlue
\ccMethod{Kernel::RT hy() const;}
       {returns the homogeneous $y$ coordinate.}
\ccGlue
\ccMethod{Kernel::RT hw() const;}
       {returns the homogenizing  coordinate.}

Note that you do not loose information with the homogeneous
representation, because the FieldNumberType is a quotient.

\ccMethod{Kernel::FT x() const;}
       {returns the \ccStyle{x}-coordinate of \ccVar, that is $hx/hw$.}
\ccGlue
\ccMethod{Kernel::FT y() const;}
       {returns the \ccStyle{y}-coordinate of \ccVar, that is $hy/hw$.}
       
The following operations are for convenience and for compatibility
with higher dimensional vectors.  Again they come in a
\ccHtmlNoLinksFrom{Cartesian} and homogeneous flavor.

\ccMethod{Kernel::RT homogeneous(int i) const;}
       {returns the i'th homogeneous coordinate of \ccVar, starting with 0.
        \ccPrecond $0\leq i \leq 2$.}
\ccGlue
\ccMethod{Kernel::FT cartesian(int i) const;}
       {returns the i'th \ccHtmlNoLinksFrom{Cartesian} coordinate of \ccVar, starting at 0.
        \ccPrecond $0\leq i \leq 1$.}
\ccGlue
\ccMethod{Kernel::FT operator[](int i) const;}
       {returns  \ccStyle{cartesian(i)}.
        \ccPrecond $0\leq i \leq 1$.}

\ccMethod{int dimension() const;}
       {returns the dimension (the constant 2).}

\ccMethod{Direction_2<Kernel> direction() const;}
       {returns the direction which passes through \ccVar.}

\ccMethod{Vector_2<Kernel>  transform(const Aff_transformation_2<Kernel> &t) const;}
       {returns the vector obtained by applying $t$ on \ccVar.}

\ccMethod{Vector_2<Kernel>  perpendicular(const Orientation &o) const;}
       {returns the vector perpendicular to \ccVar\ in clockwise or
        counterclockwise orientation.}

\ccHeading{Operators}

The following operations can be applied to vectors:

\ccMethod{Vector_2<Kernel>        operator+(const Vector_2<Kernel> &w) const;}
       {Addition.}

\ccMethod{Vector_2<Kernel>        operator-(const Vector_2<Kernel> &w) const;}
       {Subtraction.}

\ccMethod{Vector_2<Kernel>        operator-() const;}
       {returns the opposite vector.}

\ccMethod{Kernel::FT                  operator*(const Vector_2<Kernel> &w) const;}
       {returns the scalar product (= inner product) of the two vectors.}


\ccFunction{Vector_2<Kernel>
            operator*(const Vector_2<Kernel> &v, const Kernel::RT &s);}
       {Multiplication with a scalar from the right.}

\ccFunction{Vector_2<Kernel>
            operator*(const Vector_2<Kernel> &v, const Kernel::FT &s);}
       {Multiplication with a scalar from the right.}

\ccFunction{Vector_2<Kernel>
            operator*(const Kernel::RT &s, const Vector_2<Kernel> &v);}
       {Multiplication with a scalar from the left.}

\ccFunction{Vector_2<Kernel>
            operator*(const Kernel::FT &s, const Vector_2<Kernel> &v);}
       {Multiplication with a scalar from the left.}

\ccMethod{Vector_2<Kernel>        operator/(const Kernel::RT &s) const;}
       {Division by a scalar.}

\ccSeeAlso
\ccRefConceptPage{Kernel::Vector_2}

\end{ccRefClass} 
