\begin{ccRefClass} {Plane_3<Kernel>}

\ccDefinition 
An object \ccStyle{h} of the data type \ccRefName\ is an oriented
plane in the three-dimensional Euclidean space $\E^3$. It is defined
by the set of points with Cartesian coordinates $(x,y,z)$ that satisfy
the plane equation

\begin{ccTexOnly}
\[h :\;  a\, x +b\, y +c\, z + d = 0.\]
\end{ccTexOnly}
\begin{ccHtmlOnly}
h : a x + b y + c z + d = 0
\end{ccHtmlOnly}

The plane splits $\E^3$ in a {\em positive} and a {\em negative side}.
A point $p$ with Cartesian coordinates $(px, py, pz)$ is on the
positive side of \ccStyle{h}, iff \ccTexHtml{$a\, px +b\, py +c\, pz +
  d > 0$}{a px + b py +c pz + d > 0}. It is on the negative side, iff
\ccTexHtml{$a\, px +b\, py\, +c\, pz + d < 0$}{a px + b py +c pz + d <
  0}.

\ccCreation
\ccCreationVariable{h}

\ccHidden \ccConstructor{Plane_3();}
             {introduces an uninitialized variable \ccVar.}

\ccHidden \ccConstructor{Plane_3(const Plane_3<Kernel> &h);}
            {copy constructor.}

\ccConstructor{Plane_3(const Kernel::RT &a, 
                       const Kernel::RT &b,
                       const Kernel::RT &c,
                       const Kernel::RT &d)}
{creates a plane \ccVar\ defined by the equation
 \ccTexHtml{$a\, px +b\, py +c\, pz + d = 0$}{a px + b py + c pz + d = 0}.
Notice that \ccVar\ is degenerate if 
 \ccTexHtml{$a = b = c = 0$}{a = b = c = 0}.}

\ccConstructor{Plane_3(const Point_3<Kernel> &p,
                       const Point_3<Kernel> &q,
                       const Point_3<Kernel> &r);}
{creates a plane \ccVar\ passing through the points \ccStyle{p},
 \ccStyle{q} and \ccStyle{r}. The plane is oriented such that \ccStyle{p}, 
 \ccStyle{q} and \ccStyle{r} are oriented in a positive sense 
 (that is counterclockwise) when seen from the positive side of \ccVar.
Notice that \ccVar\ is degenerate if the points are collinear.}


\ccConstructor{Plane_3(const Point_3<Kernel> &p,
                       const Vector_3<Kernel> &v)}
{introduces a plane \ccVar\ that passes through point \ccStyle{p} and
 that is orthogonal to \ccStyle{v}.}

\ccConstructor{Plane_3(const Point_3<Kernel> &p,
                          const Direction_3<Kernel>&d)}
{introduces a plane \ccVar\ that passes through point \ccStyle{p} and
 that has as an orthogonal direction equal to \ccStyle{d}.}

\ccConstructor{Plane_3(const Line_3<Kernel> &l,
                          const Point_3<Kernel> &p)}
{introduces a plane \ccVar\ that is defined through the  three points 
 \ccStyle{l.point(0)}, \ccStyle{l.point(1)} and \ccStyle{p}.}

\ccConstructor{Plane_3(const Ray_3<Kernel> &r,
                          const Point_3<Kernel> &p)}
{introduces a plane \ccVar\ that is defined through the  three points 
 \ccStyle{r.point(0)}, \ccStyle{r.point(1)} and \ccStyle{p}.}

\ccConstructor{Plane_3(const Segment_3<Kernel> &s,
                          const Point_3<Kernel> &p)}
{introduces a plane \ccVar\ that is defined through the  three points 
 \ccStyle{s.source()}, \ccStyle{s.target()} and \ccStyle{p}.}

\ccOperations
\ccHidden \ccMethod{Plane_3<Kernel> & operator=(const Plane_3<Kernel> &h);}
        {Assignment.}

\ccMethod{bool operator==(const Plane_3<Kernel> &h2) const;}
       {Test for equality: two planes are equal, iff they have a non 
        empty intersection and the same orientation.}

\ccMethod{bool operator!=(const Plane_3<Kernel> &h2) const;}
       {Test for inequality.}

\ccMethod{Kernel::RT a() const;}
       {returns the first coefficient of \ccVar.}
\ccGlue
\ccMethod{Kernel::RT b() const;}
       {returns the second coefficient of \ccVar.}
\ccGlue
\ccMethod{Kernel::RT c() const;}
       {returns the third coefficient of \ccVar.}
\ccGlue
\ccMethod{Kernel::RT d() const;}
       {returns the fourth coefficient of \ccVar.}

\ccMethod{Line_3<Kernel> perpendicular_line(const Point_3<Kernel> &p) const;}
       {returns the line that is perpendicular to \ccVar\ and that
        passes through point \ccStyle{p}. The line is oriented from
        the negative to the positive side of \ccVar.}

\ccMethod{Point_3<Kernel>    projection(const Point_3<Kernel> &p) const;}
       {returns the orthogonal projection of $p$ on \ccVar.}

\ccMethod{Plane_3<Kernel>      opposite() const;}
       {returns the plane with opposite orientation.}

\ccMethod{Point_3<Kernel> point() const;}
       {returns an arbitrary point on \ccVar.}

\ccMethod{Vector_3<Kernel> orthogonal_vector() const;}
       {returns a vector that is orthogonal to \ccVar\ and that
        is directed to the positive side of \ccVar.}

\ccMethod{Direction_3<Kernel> orthogonal_direction() const;}
       {returns the direction that is orthogonal to \ccVar\ and that
        is directed to the positive side of \ccVar.}

\ccMethod{Vector_3<Kernel>      base1() const;}
       {returns a vector orthogonal to 
        \ccStyle{orthogonal_vector()}.}

\ccMethod{Vector_3<Kernel>      base2() const;}
       {returns a vector that is both orthogonal to \ccStyle{base1()},
        and to \ccStyle{orthogonal_vector()}, and such that the result of 
        \ccStyle{orientation( point(), point() + base1(), 
        point()+base2(), point() + orthogonal_vector() )} is positive.}

\ccHeading{2D Conversion}

The following functions provide conversion between a plane and 
\cgal's two-dimensional space. The transformation is affine, but
not necessarily an isometry. This means, the transformation preserves
combinatorics, but not distances.

\ccMethod{Point_2<Kernel>       to_2d(const Point_3<Kernel> &p) const;}
       {returns the image point of the projection of \ccStyle{p} 
       under an affine transformation, which maps \ccVar\ onto the 
       $xy$-plane, with the $z$-coordinate removed.}

\ccMethod{Point_3<Kernel>       to_3d(const Point_2<Kernel> &p) const;}
       {returns a point $q$, such that \ccStyle{to_2d( to_3d( p ))}
        is equal to \ccStyle{p}.}

\ccPredicates

\ccMethod{Oriented_side oriented_side(const Point_3<Kernel> &p) const;}
       {returns either \ccStyle{ON_ORIENTED_BOUNDARY}, or
        the constant \ccStyle{ON_POSITIVE_SIDE}, or the con\-stant
        \ccStyle{ON_NEGATIVE_SIDE}, 
        determined by the position of $p$ relative to the oriented plane \ccVar.
        }

For convenience we provide the following boolean functions:

\ccMethod{bool has_on(const  Point_3<Kernel> &p) const;}
       {}
%%\ccGlue
%%\ccMethod{bool has_on_boundary(const  Point_3<Kernel> &p) const;}
%%       {}
\ccGlue
\ccMethod{bool has_on_positive_side(const  Point_3<Kernel> &p) const;}
       {}
\ccGlue
\ccMethod{bool has_on_negative_side(const  Point_3<Kernel> &p) const;}
       {}

\ccMethod{bool has_on(const  Line_3<Kernel> &l) const;}
       {}
%%\ccGlue
%%\ccMethod{bool has_on_boundary(const  Line_3<Kernel> &l) const;}
%%       {}

\ccMethod{bool is_degenerate() const;}
       {Plane \ccVar\ is degenerate, if the coefficients \ccStyle{a},  
        \ccStyle{b}, and \ccc{c} of the plane equation are zero.}

\ccHeading{Miscellaneous}

\ccMethod{Plane_3<Kernel>  transform(const Aff_transformation_3<Kernel> &t) const;}
       {returns the plane obtained by applying $t$ on a point of \ccVar\ 
        and the orthogonal direction of \ccVar.}

\ccSeeAlso
\ccRefConceptPage{Kernel::Plane_3}

\end{ccRefClass} 
