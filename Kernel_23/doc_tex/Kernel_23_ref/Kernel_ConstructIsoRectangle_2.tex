\begin{ccRefFunctionObjectConcept}{Kernel::ConstructIsoRectangle_2}
A model for this must provide:

\ccCreationVariable{fo}

\ccMemberFunction{Kernel::Iso_rectangle_2 operator()(const Kernel::Point_2 &p, 
                                                     const Kernel::Point_2 &q);}
            {introduces an iso-oriented rectangle  with diagonal
             opposite vertices $p$ and $q$ such that $p$ is the
             lexicographically smallest point in the rectangle.}

\ccMemberFunction{Kernel::Iso_rectangle_2 operator()(const Kernel::Point_2 &p, 
                                                     const Kernel::Point_2 &q,
                                                     int);}
            {introduces an iso-oriented rectangle with diagonal
             opposite vertices $p$ and $q$.  The \ccc{int} argument value is
             only used to distinguish the two overloaded functions.
             \ccPrecond{$p.x()<=q.x()$ and $p.y()<=q.y()$.}}

\ccMemberFunction{Kernel::Iso_rectangle_2
                  operator()(const Kernel::Point_2 &left, 
                             const Kernel::Point_2 &right,
                             const Kernel::Point_2 &bottom,
                             const Kernel::Point_2 &top);}
            {introduces an iso-oriented rectangle \ccVar\ whose
             minimal $x$ coordinate is the one of \ccc{left}, the
             maximal $x$ coordinate is the one of \ccc{right}, the
             minimal $y$ coordinate is the one of \ccc{bottom}, the
             maximal $y$ coordinate is the one of \ccc{top}.}

\ccRefines
AdaptableFunctor (with two arguments)

\ccSeeAlso
\ccRefIdfierPage{CGAL::Iso_rectangle_2<Kernel>} \\

\end{ccRefFunctionObjectConcept}
