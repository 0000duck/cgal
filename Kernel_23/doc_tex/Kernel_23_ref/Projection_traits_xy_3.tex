
\begin{ccRefClass}{Projection_traits_xy_3<K>}  %% add template arg's if necessary

%% \ccHtmlCrossLink{}     %% add further rules for cross referencing links
%% \ccHtmlIndexC[class]{} %% add further index entries
\ccCreationVariable{traits}
\ccDefinition
  
The class \ccRefName\ is an adapter to apply 2D algorithms to the projections of 3D data on the \ccc{xy}-plane.


\cgal\ provides also predefined geometric traits classes
\ccc{Projection_traits_yz_3<K>} and
\ccc{Projection_traits_xz_3<K>} to
deal with projections on the
 \ccc{zx}- and  the \ccc{zy}-plane,
respectively.  

\ccParameters
The template parameter \ccc{K}  has to
be instantiated by a model of the \ccc{Kernel} concept.
\ccRefName\ uses types 
and predicates defined in \ccc{K}.

\ccInclude{CGAL/Projection_traits_xy_3.h}

\ccIsModel

The class is a model of several 2D triangulation traits class concepts,
except that it does not provide the type and constructors
required to build the dual Voronoi diagram. The class is also a model
of the concepts \ccc{PolygonTraits_2} and \ccc{ConvexHullTraits_2}.

\ccRefConceptPage{TriangulationTraits_2} \\
\ccRefConceptPage{DelaunayTriangulationTraits_2} \\
\ccRefConceptPage{ConstrainedTriangulationTraits_2} \\
\ccRefConceptPage{PolygonTraits_2} \\
\ccRefConceptPage{ConvexHullTraits_2}

\ccTypes

\ccTwo{typedef Triangle_3<K>}{}
\ccTypedef{typedef Point_3<K>   Point_2;}{}
\ccGlue
\ccTypedef{typedef Segment_3<K>   Segment_2;}{}
\ccGlue
\ccTypedef{typedef Triangle_3<K>   Triangle_2;}{}
\ccGlue
\ccTypedef{typedef Line_3<K>   Line_2;}{}

\ccTypedef{typedef K::Object_3   Object_2;}{}


The functors provided by this class are those listed in the concepts,
except that it does not provide the type and constructors
required to build the dual Voronoi diagram.
The functors operate on the 2D projection of their arguments. They come with preconditions that projections of the arguments are non-degenerate, eg. a line segment does not project on a single point, two points do not project on the same point, etc.

In the following, we specify the choice of the \ccc{z}-coordinate
in case a new point is constructed. 


\ccNestedType{Intersect_2}{A construction object.
Provides the operator~:\\
\ccHtmlNoLinksFrom{\ccc{Object_2 operator()(Segment_2 s1, Segment_2 s2);}}
which returns a 3D object whose projection on the xy-plane 
                               is the intersection of the projections of \ccc{s1} and \ccc{s2}.
If non empty, the returned object is either a segment or a point.
                               Its embedding in 3D is computed as the interpolation 
                               between \ccc{s1} and \ccc{s2},
                               meaning that any point \ccc{p} of the returned object
                               is the midpoint of  segment \ccc{p1p2} where \ccc{p1} and \ccc{p2} are the two points of \ccc{s1} and \ccc{s2} respectively, both projecting on \ccc{p}.
\ccPrecond{The projection of \ccc{s1} and the projection of \ccc{s2} are non-degenerate 
\ccc{2D} segments.}
}


\ccCreation
\ccCreationVariable{traits}
 
\ccThree{Projection_traits_xy_3()}{traits=tr  }{}

\ccConstructor{Projection_traits_xy_3();}{
default constructor.}
\ccGlue
\ccConstructor{Projection_traits_xy_3(
                      Projection_traits_xy_3 tr);}
{Copy constructor.}
\ccMethod{Projection_traits_xy_3 operator=
              (Projection_traits_xy_3  tr);}
{Assignment operator.}

\end{ccRefClass}

% +------------------------------------------------------------------------+
%%RefPage: end of main body, begin of footer
% EOF
% +------------------------------------------------------------------------+

