\begin{ccRefClass}{Homogeneous_converter<K1, K2, RTConverter, FTConverter>}

\KernelRefLayout\gdef\ccTagOperatorLayout{\ccFalse}

\ccDefinition

\ccClassTemplateName converts objects from the kernel traits \ccc{K1} to
the kernel traits \ccc{K2}.  Those traits must be of the form
\ccc{Homogeneous<RT1>} and \ccc{Homogeneous<RT2>} (or the equivalent with
\ccc{Simple_homogeneous}).  It then provides the following operators to
convert objects from \ccc{K1} to \ccc{K2}.

The third template parameter \ccc{RT_Converter} is a function object that must
provide \ccc{K2::RT operator()(const K1::RT &n);} that 
converts \ccc{n} to an \ccc{K2::RT} that has the same value.

The default value of this parameter is \ccc{CGAL::NT_converter<K1::RT, K2::RT>},
which uses the conversion operator from
\ccc{K1::RT} to \ccc{K2::RT}.

Similarly, the fourth template parameter must provide
\ccc{K2::FT operator()(const K1::FT &n);} that
converts \ccc{n} to an \ccc{K2::FT} that has the same value.  Its
default value is  \ccc{CGAL::NT_converter<K1::FT, K2::FT>}.

\ccInclude{CGAL/Homogeneous_converter.h}

\ccCreation
\ccCreationVariable{conv}

\ccConstructor{Homogeneous_converter<>();}{Default constructor.}

\ccOperations

\ccMemberFunction{K2::Point_2 operator()(const K1::Point_2&p);}
{ returns a \ccc{K2::Point_2} which coordinates are those of \ccc{p},
converted by \ccc{RTConverter}.}

Similar operators are defined for the other kernel traits geometric types
\ccc{Point_3}, \ccc{Vector_2}...

\ccSeeAlso
\ccRefIdfierPage{CGAL::Homogeneous<RingNumberType>} \\
\ccRefIdfierPage{CGAL::Simple_homogeneous<RingNumberType>} \\

\end{ccRefClass}
