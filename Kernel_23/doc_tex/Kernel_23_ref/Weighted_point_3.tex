\begin{ccRefClass} {Weighted_point_3<Kernel, Weight>}

\ccDefinition
An object of the class \ccRefName\ is a point in the three-dimensional
Euclidean space $\E^3$ with an associated weight (and thus can also be 
viewed as a sphere centered at the point with radius equal to the weight). 
The default type of \ccc{Weight} is \ccc{Kernel::FT}, the
field type of the defining kernel.

\ccInheritsFrom
\ccc{Kernel::Point_3}

\ccTypes

\ccNestedType{Kernel}{the kernel in which the point lies}
\ccNestedType{Weight}{the type of the weight associated with this point}

\ccCreation
\ccCreationVariable{p}


\ccHidden \ccConstructor{Weighted_point_3();}
             {introduces an uninitialized variable \ccVar.}

\ccHidden \ccConstructor{Weighted_point_3(const Weighted_point_3<Kernel> &q);}
            {copy constructor.}

\ccConstructor{Weighted_point_3(const Origin &ORIGIN, const Weight& w = 0);}
            {introduces a point with \ccHtmlNoLinks{Cartesian} 
             coordinates $(0,0,0)$ and weight \ccc{w}.}

\ccConstructor{Weighted_point_3(const Kernel::RT &hx, const Kernel::RT &hy, 
                                const Kernel::RT &hz, const Kernel::RT &hw = RT(1),
                                const Weight& w = 0);}
            {introduces a point \ccVar\ initialized to $(hx/hw,hy/hw, hz/hw)$
             with weight \ccc{w}.
             \ccPrecond \ccc{hw} $\neq$ 0.}

\ccConstructor{Weighted_point_3(const Point_3<Kernel> &p, const Weight& w = 0);}
            {introduces a variable \ccVar\ with \ccHtmlNoLinksFrom{Cartesian}
             coordinates \ccc{(p.x(),p.y(),p.z())} and weight \ccc{w}.}

\ccOperations
%\ccSetTwoOfThreeColumns{5cm}{4cm}

\ccHidden \ccMethod{Weighted_point_3<Kernel> & operator=(const Weighted_point_3<Kernel> &q);}
        {Assignment.}

\ccMethod{bool operator==(const Weighted_point_3<Kernel> &q) const;}
       {Test for equality: Two points are equal, iff their $x$, $y$ and $z$
        coordinates are equal.}

\ccMethod{bool operator!=(const Weighted_point_3<Kernel> &q) const;}
       {Test for inequality.}

There are two sets of coordinate access functions, namely to the
homogeneous and to the \ccHtmlNoLinksFrom{Cartesian} coordinates. They can be used
independently from the chosen kernel model.

\ccMethod{Kernel::RT hx() const;}
       {returns the homogeneous $x$ coordinate.}
\ccGlue
\ccMethod{Kernel::RT hy() const;}
       {returns the homogeneous $y$ coordinate.}
\ccGlue
\ccMethod{Kernel::RT hz() const;}
       {returns the homogeneous $z$ coordinate.}
\ccGlue
\ccMethod{Kernel::RT hw() const;}
       {returns the homogenizing  coordinate.}

Note that you do not loose information with the homogeneous
representation, because the \ccc{FieldNumberType} is a quotient.

\ccMethod{Kernel::FT x() const;}
       {returns the \ccHtmlNoLinks{Cartesian} $x$ coordinate, that is $hx/hw$.}
\ccGlue
\ccMethod{Kernel::FT y() const;}
       {returns the \ccHtmlNoLinks{Cartesian} $y$ coordinate, that is $hy/hw$.}
\ccGlue
\ccMethod{Kernel::FT z() const;}
       {returns the \ccHtmlNoLinks{Cartesian} $z$ coordinate, that is $hz/hw$.}
       
The following operations are for convenience and for compatibility
with higher dimensional points. Again they come in a
\ccHtmlNoLinksFrom{Cartesian} and homogeneous flavor.

\ccMethod{Kernel::RT homogeneous(int i) const;}
       {returns the i'th homogeneous coordinate of \ccVar, starting with 0.
        \ccPrecond $0\leq i \leq 3$.}

\ccMethod{Kernel::FT cartesian(int i) const;}
       {returns the i'th \ccHtmlNoLinks{Cartesian} coordinate of \ccVar, starting with 0.
        \ccPrecond $0\leq i \leq 2$.}

\ccMethod{Kernel::FT operator[](int i) const;}
       {returns \ccStyle{cartesian(i)}.
        \ccPrecond $0\leq i \leq 2$.}

\ccMethod{int dimension() const;}
       {returns the dimension (the constant 3).}

\ccMethod{Bbox_3 bbox() const;}
       {returns a bounding box containing \ccVar.}

\ccMethod{Weighted_point_3<Kernel>  transform(const Aff_transformation_3<Kernel> &t) const;}
       {returns the point obtained by applying $t$ on \ccVar.}

Here are the access functions for the components of the weighted point.

\ccMethod{Weight weight() const;}
       {returns the weight associated with the point.}
\ccGlue
\ccMethod{Point_3<Kernel> point() const;}
       {returns the unweighted point.}

\ccHeading{Operators}

The following operations can be applied on weighted points:

\ccFunction{Weighted_point_3<Kernel> operator+(const Weighted_point_3<Kernel> &p,
                                          const Vector_3<Kernel> &v);}
       {returns the point obtained by translating \ccStyle{p} by the 
        vector \ccStyle{v}.}

\ccFunction{Weighted_point_3<Kernel> operator-(const Weighted_point_3<Kernel> &p,
                                          const Vector_3<Kernel> &v);}
       {returns the point obtained by translating \ccStyle{p} by the 
        vector -\ccStyle{v}.}

\ccSeeAlso

\ccRefIdfierPage{CGAL::Point_3<Kernel>} \\
\ccRefIdfierPage{CGAL::Weighted_point_2<Kernel>}

\end{ccRefClass} 

