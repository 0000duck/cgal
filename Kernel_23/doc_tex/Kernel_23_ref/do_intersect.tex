\begin{ccRefFunction}{do_intersect}

\ccUnchecked{
\ccRefLabel{Kernel::do_intersect}
\ccFunction{bool do_intersect(Type1 obj1, Type2 obj2);}
{checks whether \ccc{obj1} and \ccc{obj2} intersect.
Two objects \ccStyle{obj1} and \ccStyle{obj2} intersect if there is a point
\ccStyle{p} that is part of both \ccStyle{obj1} and \ccStyle{obj2}.
The \ccHtmlNoLinksFrom{intersection} region of those two objects is defined as the set of all
points \ccStyle{p} that are part of both \ccStyle{obj1} and \ccStyle{obj2}.
Note that for objects like triangles and polygons that enclose a
bounded region, this region is part of the object.
}}

Depending on which \cgal\ \ccHtmlNoLinksFrom{kernel} is used,
\ccStyle{Type1} and \ccStyle{Type2} can be of different types. 

%%%%%%%%%%%%%%%%%%%%%%%%%%%%%%%%%%%%%%%%%%%%%%
\paragraph{With the basic 2D and 3D Kernel} (see Chapter~\ref{chapter-kernel-23})

\ccInclude{CGAL/intersections.h}

The types \ccStyle{Type1} and \ccStyle{Type2} can be any of the following:
\begin{itemize}\ccTexHtml{\itemsep0pt\topsep0pt\partopsep0pt\parskip0pt\parsep0pt}{}
\item \ccStyle{Point_2<Kernel>}
\item \ccStyle{Line_2<Kernel>}
\item \ccStyle{Ray_2<Kernel>}
\item \ccStyle{Segment_2<Kernel>}
\item \ccStyle{Triangle_2<Kernel>}
\item \ccStyle{Iso_rectangle_2<Kernel>}
\end{itemize}

Also, \ccStyle{Type1} and \ccStyle{Type2} can be both of type
\begin{itemize}\ccTexHtml{\itemsep0pt\topsep0pt\partopsep0pt\parskip0pt\parsep0pt}{}
\item \ccStyle{Line_2<Kernel>}
\item \ccStyle{Circle_2<Kernel>}
\end{itemize}

In three-dimensional space, \ccc{Type1} can be \ccc{Plane_3<Kernel>} or
\ccc{Triangle_3<Kernel>} and \ccc{Type2} any of 
\begin{itemize}\ccTexHtml{\itemsep0pt\topsep0pt\partopsep0pt\parskip0pt\parsep0pt}{}
\item \ccStyle{Plane_3<Kernel>}
\item \ccStyle{Line_3<Kernel>}
\item \ccStyle{Ray_3<Kernel>}
\item \ccStyle{Segment_3<Kernel>}
\item \ccStyle{Triangle_3<Kernel>}.
\end{itemize}

\ccc{Type1} can also be \ccc{Bbox_3} and \ccc{Type2} any of 
\begin{itemize}\ccTexHtml{\itemsep0pt\topsep0pt\partopsep0pt\parskip0pt\parsep0pt}{}
\item \ccStyle{Bbox_3}
\item \ccStyle{Line_3<Kernel>}
\item \ccStyle{Ray_3<Kernel>}
\item \ccStyle{Segment_3<Kernel>}.
\end{itemize}

Also, \ccc{Type1} and \ccc{Type2} can be respectively of types
\begin{itemize}\ccTexHtml{\itemsep0pt\topsep0pt\partopsep0pt\parskip0pt\parsep0pt}{}
\item \ccStyle{Triangle_3<Kernel>} and \ccStyle{Tetrahedron_3<Kernel>}
\item \ccStyle{Line_3<Kernel>} and \ccStyle{Line_3<Kernel>}
\item \ccStyle{Plane_3<Kernel>} and \ccStyle{Sphere_3<Kernel>} (or the contrary)
\item \ccStyle{Sphere_3<Kernel>} and \ccStyle{Sphere_3<Kernel>}.
\end{itemize} 

%%%%%%%%%%%%%%%%%%%%%%%%%%%%%%%%%%%%%%%%%%%%%%
\paragraph{With the 2D Circular Kernel} (see Chapter~\ref{chapter-circular-kernel}) 

\ccInclude{CGAL/Circular_kernel_intersections.h}

If this kernel is used, in addition to the combinations of 2D types
previously listed, \ccStyle{Type1} and \ccStyle{Type2} can be any of
the following:
\begin{itemize}\ccTexHtml{\itemsep0pt\topsep0pt\partopsep0pt\parskip0pt\parsep0pt}{}
\item \ccStyle{Line_2<CircularKernel>}
\item \ccStyle{Circle_2<CircularKernel>}
\item \ccStyle{Line_arc_2<CircularKernel>}
\item \ccStyle{Circular_arc_2<CircularKernel>}
\end{itemize}

An example illustrating this is presented in
Chapter~\ref{chapter-circular-kernel}.

%%%%%%%%%%%%%%%%%%%%%%%%%%%%%%%%%%%%%%%%%%%%%%
\paragraph{With the 3D Spherical Kernel} (see Chapter~\ref{chapter-spherical-kernel}) 

\ccInclude{CGAL/Spherical_kernel_intersections.h}

If this kernel is used, in addition to the combinations of 3D types
previously listed, \ccStyle{Type1} and \ccStyle{Type2} can be any of
the following:
\begin{itemize}\ccTexHtml{\itemsep0pt\topsep0pt\partopsep0pt\parskip0pt\parsep0pt}{}
\item \ccStyle{Line_3<SphericalKernel>}
\item \ccStyle{Circle_3<SphericalKernel>}
\item \ccStyle{Plane_3<SphericalKernel>}
\item \ccStyle{Sphere_3<SphericalKernel>}
\item \ccStyle{Line_arc_3<SphericalKernel>}
\item \ccStyle{Circular_arc_3<SphericalKernel>}
\end{itemize}

An example illustrating this is presented in
Chapter~\ref{chapter-spherical-kernel}.

Also, another predicate is provided with this kernel:

\ccFunction{bool do_intersect(Type1 obj1, Type2 obj2, Type3 obj3);}
{checks whether \ccc{obj1}, \ccc{obj2} and \ccc{obj3} intersect.}

where \ccStyle{Type1}, \ccStyle{Type2} and \ccStyle{Type3} can be:
\begin{itemize}\ccTexHtml{\itemsep0pt\topsep0pt\partopsep0pt\parskip0pt\parsep0pt}{}
\item \ccStyle{Sphere_3<SphericalKernel>}
\item \ccStyle{Plane_3<SphericalKernel>}
\end{itemize}

\ccSeeAlso
\ccRefIdfierPage{CGAL::intersection}

\end{ccRefFunction}
