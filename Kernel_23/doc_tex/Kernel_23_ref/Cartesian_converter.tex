\begin{ccRefClass}{Cartesian_converter<K1, K2, NTConverter>}
%\ccTexHtml{\ccSetThreeColumns{Point_2< us<RT> > }{}{\hspace*{8.5cm}}}{}

\KernelRefLayout\gdef\ccTagOperatorLayout{\ccFalse}

\ccDefinition

\ccClassTemplateName converts objects from the kernel traits \ccc{K1} to
the kernel traits \ccc{K2} using \ccc{Converter} to do the conversion.  
Those traits must be of the form
\ccc{Cartesian<FT1>} and \ccc{Cartesian<FT2>} (or the equivalent with
\ccc{Simple_cartesian}).  It then provides the following operators to convert
objects from \ccc{K1} to \ccc{K2}.

The third template parameter \ccc{NTConverter} is a function object that must
provide \ccc{K2::FT operator()(K1::FT n)} that converts \ccc{n} to an
\ccc{K2::FT} which has the same value.

The default value of this parameter is \ccc{CGAL::NT_converter<K1::FT, K2::FT>}.

\ccInclude{CGAL/Cartesian_converter.h}

\ccCreation
\ccCreationVariable{conv}

\ccConstructor{Cartesian_converter<>();}{Default constructor.}

\ccOperations

\ccMemberFunction{K2::Point_2 operator()(const K1::Point_2&p);}
{ returns a \ccc{K2::Point_2} which coordinates are those of \ccc{p},
converted by \ccc{NTConverter}.}

Similar operators are defined for the other kernel traits types \ccc{Point_3},
\ccc{Vector_2}...

\ccSeeAlso
\ccRefIdfierPage{CGAL::Cartesian<FieldNumberType>} \\
\ccRefIdfierPage{CGAL::Simple_cartesian<FieldNumberType>} \\

\end{ccRefClass}
