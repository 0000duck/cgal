\begin{ccRefFunction}{intersection}

Depending on which \cgal\ \ccHtmlNoLinksFrom{kernel} is used,
different versions of this global function are available. This is
described below.

\paragraph{Notes on backward compatibility}

The \ccStyle{intersection} function used to return
\ccStyle{CGAL::Object}, but starting with {\cgal} VERSION HERE the
return type is determined by a metafunction defined by the kernel. To
preserve backward compatibility \ccStyle{CGAL::Object} can be
constructed from the new return types implicitly, but switching to the
new style is recommended. To enable the old style without any overhead,
the macro \ccc{CGAL_INTERSECTION_VERSION} must be defined to
\ccStyle{1} before any {\cgal} header is included.

%%%%%%%%%%%%%%%%%%%%%%%%%%%%%%%%%%%%%%%%%%%%%%
\paragraph{With the basic 2D and 3D Kernel} (see Chapter~\ref{chapter-kernel-23})

\ccInclude{CGAL/intersections.h}

\ccUnchecked{
\ccRefLabel{Kernel::intersection}
\ccFunction{boost::result_of<Intersect(Type1, Type2)>::type intersection(Type1 obj1, Type2 obj2);}
{Two objects \ccStyle{obj1} and \ccStyle{obj2} intersect if there is a point
\ccStyle{p} that is part of both \ccStyle{obj1} and \ccStyle{obj2}.
The \ccHtmlNoLinksFrom{intersection} region of those two objects is defined as the set of all
points \ccStyle{p} that are part of both \ccStyle{obj1} and \ccStyle{obj2}.
Note that for objects like triangles and polygons that enclose a
bounded region, this region is considered part of the object.
If a segment lies completely inside a triangle, then those two objects
intersect and the \ccHtmlNoLinksFrom{intersection} region is the complete segment. 
Here, \ccStyle{Intersect_23} means either \ccStyle{Intersect_2} or \ccStyle{Intersect_3}, respectively.
}}

The same functionality is also available through the functors \ccc{Kernel::Intersect_2} and \ccc{Kernel::Intersect_3}.

The possible values for types \ccStyle{Type1} and \ccStyle{Type2} and
the value for T\ldots in \ccStyle{boost::optional< boost::variant<
  T\ldots >} are the following and can be obtained through
\ccStyle{Intersect_2::Result<A, B>::Type} or
\ccStyle{Intersect_3::Result<A, B>::Type}.

\begin{ccTexOnly}
\begin{longtable}[c]{|l|l|l|}
%\caption{All available intersection computations}\\
\multicolumn{3}{l}{\sl \ \ }
\endfirsthead
\multicolumn{3}{l}{\sl continued}
\endhead
\hline
type A & type B & \parbox{4 cm}{\vspace{1 mm}{return type}} \\
\hline
\ccStyle{Line_2} & \ccStyle{Line_2} & \parbox{4 cm}{\vspace{1 mm}
    \ccStyle{Point_2}
    \\ \ccStyle{Line_2}
  \vspace{1 mm}} \\
\hline
\ccStyle{Segment_2} & \ccStyle{Line_2} & \parbox{4 cm}{\vspace{1 mm}
    \ccStyle{Point_2}
    \\ \ccStyle{Segment_2}
  \vspace{1 mm}} \\
\hline
\ccStyle{Segment_2} & \ccStyle{Segment_2} & \parbox{4 cm}{\vspace{1 mm}
    \ccStyle{Point_2}
    \\ \ccStyle{Segment_2}
  \vspace{1 mm}} \\
\hline
\ccStyle{Ray_2} & \ccStyle{Line_2} & \parbox{4 cm}{\vspace{1 mm}
    \ccStyle{Point_2}
    \\ \ccStyle{Ray_2}
  \vspace{1 mm}} \\
\hline
\ccStyle{Ray_2} & \ccStyle{Segment_2} & \parbox{4 cm}{\vspace{1 mm}
    \ccStyle{Point_2}
    \\ \ccStyle{Segment_2}
  \vspace{1 mm}} \\
\hline
\ccStyle{Ray_2} & \ccStyle{Ray_2} & \parbox{4 cm}{\vspace{1 mm}
    \ccStyle{Point_2}
    \\ \ccStyle{Segment_2}
    \\ \ccStyle{Ray_2}
  \vspace{1 mm}} \\
\hline
\ccStyle{Triangle_2} & \ccStyle{Line_2} & \parbox{4 cm}{\vspace{1 mm}
    \ccStyle{Point_2}
    \\ \ccStyle{Segment_2}
  \vspace{1 mm}} \\
\hline
\ccStyle{Triangle_2} & \ccStyle{Segment_2} & \parbox{4 cm}{\vspace{1 mm}
    \ccStyle{Point_2}
    \\ \ccStyle{Segment_2}
  \vspace{1 mm}} \\
\hline
\ccStyle{Triangle_2} & \ccStyle{Ray_2} & \parbox{4 cm}{\vspace{1 mm}
    \ccStyle{Point_2}
    \\ \ccStyle{Segment_2}
  \vspace{1 mm}} \\
\hline
\ccStyle{Triangle_2} & \ccStyle{Triangle_2} & \parbox{4 cm}{\vspace{1 mm}
    \ccStyle{Point_2}
    \\ \ccStyle{Segment_2}
    \\ \ccStyle{Triangle_2}
    \\ \ccStyle{std::vector<Point_2>}
  \vspace{1 mm}} \\
\hline
\ccStyle{Iso_rectangle_2} & \ccStyle{Line_2} & \parbox{4 cm}{\vspace{1 mm}
    \ccStyle{Point_2}
    \\ \ccStyle{Segment_2}
  \vspace{1 mm}} \\
\hline
\ccStyle{Iso_rectangle_2} & \ccStyle{Segment_2} & \parbox{4 cm}{\vspace{1 mm}
    \ccStyle{Point_2}
    \\ \ccStyle{Segment_2}
  \vspace{1 mm}} \\
\hline
\ccStyle{Iso_rectangle_2} & \ccStyle{Ray_2} & \parbox{4 cm}{\vspace{1 mm}
    \ccStyle{Point_2}
    \\ \ccStyle{Segment_2}
  \vspace{1 mm}} \\
\hline
\ccStyle{Iso_rectangle_2} & \ccStyle{Triangle_2} & \parbox{4 cm}{\vspace{1 mm}
    \ccStyle{Point_2}
    \\ \ccStyle{Segment_2}
    \\ \ccStyle{Triangle_2}
  \vspace{1 mm}} \\
\hline
\ccStyle{Iso_rectangle_2} & \ccStyle{Iso_rectangle_2} & \parbox{4 cm}{\vspace{1 mm}
    \ccStyle{Iso_rectangle_2}
  \vspace{1 mm}} \\
\hline
{\ccStyle{Line_3}} & {\ccStyle{Line_3}} & \parbox{4 cm}{\vspace{1 mm}
\ccStyle{Point_3} \\
 \ccStyle{Line_3} 
\vspace{1 mm}} \\
\hline
{\ccStyle{Line_3}} & {\ccStyle{Plane_3}} & \parbox{4 cm}{\vspace{1 mm}
\ccStyle{Point_3} \\
 \ccStyle{Line_3} 
\vspace{1 mm}} \\
\hline
{\ccStyle{Line_3}} & {\ccStyle{Ray_3}} & \parbox{4 cm}{\vspace{1 mm}
\ccStyle{Point_3} \\
 \ccStyle{Ray_3} 
\vspace{1 mm}} \\
\hline
{\ccStyle{Line_3}} & {\ccStyle{Segment_3}} & \parbox{4 cm}{\vspace{1 mm}
\ccStyle{Point_3} \\
 \ccStyle{Segment_3}
 \vspace{1 mm}}  \\
\hline
{\ccStyle{Line_3}} & {\ccStyle{Triangle_3}} & \parbox{4 cm}{\vspace{1 mm}
\ccStyle{Point_3} \\
 \ccStyle{Segment_3}
\vspace{1 mm}}  \\
\hline
{\ccStyle{Plane_3}} & {\ccStyle{Plane_3}} & \parbox{4 cm}{\vspace{1 mm}
\ccStyle{Line_3} \\
 \ccStyle{Plane_3}
\vspace{1 mm}}  \\
\hline
{\ccStyle{Plane_3}} & {\ccStyle{Ray_3}} & \parbox{4 cm}{\vspace{1 mm}
\ccStyle{Point_3} \\
 \ccStyle{Ray_3}
\vspace{1 mm}}  \\
\hline
{\ccStyle{Plane_3}} & {\ccStyle{Segment_3}} & \parbox{4 cm}{\vspace{1 mm}
\ccStyle{Point_3} \\
 \ccStyle{Segment_3}
\vspace{1 mm}}  \\
\hline
{\ccStyle{Plane_3}} & {\ccStyle{Sphere_3}} & \parbox{4 cm}{\vspace{1 mm}
\ccStyle{Point_3} \\
 \ccStyle{Circle_3}
\vspace{1 mm}}  \\
\hline
{\ccStyle{Plane_3}} & {\ccStyle{Triangle_3}} & \parbox{4 cm}{\vspace{1 mm}
\ccStyle{Triangle_3} \\
 \ccStyle{Segment_3} \\
 \ccStyle{Point_3}
\vspace{1 mm}}  \\
\hline
{\ccStyle{Ray_3}} & {\ccStyle{Ray_3}} & \parbox{4 cm}{\vspace{1 mm}
\ccStyle{Point_3} \\
\ccStyle{Ray_3} \\
\ccStyle{Segment_3}
\vspace{1 mm}}  \\
\hline
{\ccStyle{Ray_3}} & {\ccStyle{Segment_3}} & \parbox{4 cm}{\vspace{1 mm}
\ccStyle{Point_3} \\
 \ccStyle{Segment_3} 
\vspace{1 mm}} \\
\hline
{\ccStyle{Ray_3}} & {\ccStyle{Triangle_3}} & \parbox{4 cm}{\vspace{1 mm}
\ccStyle{Point_3} \\
 \ccStyle{Segment_3}
\vspace{1 mm}}  \\
\hline
{\ccStyle{Segment_3}} & {\ccStyle{Segment_3}} & \parbox{4 cm}{\vspace{1 mm}
\ccStyle{Point_3} \\
 \ccStyle{Segment_3}
 \vspace{1 mm}}  \\
\hline
{\ccStyle{Segment_3}} & {\ccStyle{Triangle_3}} & \parbox{4 cm}{\vspace{1 mm}
\ccStyle{Point_3} \\
 \ccStyle{Segment_3}
\vspace{1 mm}}  \\
\hline
{\ccStyle{Sphere_3}} & {\ccStyle{Sphere_3}} & \parbox{4 cm}{\vspace{1 mm}
\ccStyle{Point_3} \\
 \ccStyle{Circle_3} \\
 \ccStyle{Sphere_3}
\vspace{1 mm}}  \\
\hline
{\ccStyle{Triangle_3}} & {\ccStyle{Triangle_3}} & \parbox{4 cm}{\vspace{1 mm}
\ccStyle{Point_3} \\
 \ccStyle{Segment_3}\\
 \ccStyle{Triangle_3}\\
 \ccStyle{std::vector<Point_3>}
\vspace{1 mm}}  \\
\hline
\end{longtable}
\end{ccTexOnly}

\begin{ccHtmlOnly}
<DIV ALIGN="CENTER">
<TABLE CELLPADDING=3 BORDER="1">
<TR> <TH> type A </TH>
 <TH> type B </TH>
 <TH> return type </TH>
</TR>
<TR>
    <TD VALIGN="CENTER" > Line_2 </TD>
    <TD VALIGN="CENTER" > Line_2 </TD>
    <TD><TABLE>
	<TR><TD>Point_2</TD></TR>
	<TR><TD>Line_2</TD></TR>
        </TABLE></TD>
</TR>
<TR>
    <TD VALIGN="CENTER" > Segment_2 </TD>
    <TD VALIGN="CENTER" > Line_2 </TD>
    <TD><TABLE>
	<TR><TD>Point_2</TD></TR>
	<TR><TD>Segment_2</TD></TR>
      </TABLE></TD>
</TR>
<TR>
    <TD VALIGN="CENTER" > Segment_2 </TD>
    <TD VALIGN="CENTER" > Segment_2 </TD>
    <TD><TABLE>
	<TR><TD>Point_2</TD></TR>
	<TR><TD>Segment_2</TD></TR>
      </TABLE></TD>
</TR>
<TR>
    <TD VALIGN="CENTER" > Ray_2 </TD>
    <TD VALIGN="CENTER" > Line_2 </TD>
    <TD><TABLE>
	<TR><TD>Point_2</TD></TR>
	<TR><TD>Ray_2</TD></TR>
      </TABLE></TD>
</TR>
<TR>
    <TD VALIGN="CENTER" > Ray_2 </TD>
    <TD VALIGN="CENTER" > Segment_2 </TD>
    <TD><TABLE>
	<TR><TD>Point_2</TD></TR>
	<TR><TD>Segment_2</TD></TR>
      </TABLE></TD>
</TR>
<TR>
    <TD VALIGN="CENTER" > Ray_2 </TD>
    <TD VALIGN="CENTER" > Ray_2 </TD>
    <TD><TABLE>
	<TR><TD>Point_2</TD></TR>
	<TR><TD>Segment_2</TD></TR>
	<TR><TD>Ray_2</TD></TR>
      </TABLE></TD>
</TR>
<TR>
    <TD VALIGN="CENTER" > Triangle_2 </TD>
    <TD VALIGN="CENTER" > Line_2 </TD>
    <TD><TABLE>
	<TR><TD>Point_2</TD></TR>
	<TR><TD>Segment_2</TD></TR>
      </TABLE></TD>
</TR>
<TR>
    <TD VALIGN="CENTER" > Triangle_2 </TD>
    <TD VALIGN="CENTER" > Segment_2 </TD>
    <TD><TABLE>
	<TR><TD>Point_2</TD></TR>
	<TR><TD>Segment_2</TD></TR>
      </TABLE></TD>
</TR>
<TR>
    <TD VALIGN="CENTER" > Triangle_2 </TD>
    <TD VALIGN="CENTER" > Ray_2 </TD>
    <TD><TABLE>
	<TR><TD>Point_2</TD></TR>
	<TR><TD>Segment_2</TD></TR>
      </TABLE></TD>
</TR>
<TR>
    <TD VALIGN="CENTER" > Triangle_2 </TD>
    <TD VALIGN="CENTER" > Triangle_2 </TD>
    <TD><TABLE>
	<TR><TD>Point_2</TD></TR>
	<TR><TD>Segment_2</TD></TR>
	<TR><TD>Triangle_2</TD></TR>
	<TR><TD>std::vector&lt;Point_2&gt;</TD></TR>
      </TABLE></TD>
</TR>
<TR>
    <TD VALIGN="CENTER" > Iso_rectangle_2 </TD>
    <TD VALIGN="CENTER" > Line_2 </TD>
    <TD><TABLE>
	<TR><TD>Point_2</TD></TR>
	<TR><TD>Segment_2</TD></TR>
      </TABLE></TD>
</TR>
<TR>
    <TD VALIGN="CENTER" > Iso_rectangle_2 </TD>
    <TD VALIGN="CENTER" > Segment_2 </TD>
    <TD><TABLE>
	<TR><TD>Point_2</TD></TR>
	<TR><TD>Segment_2</TD></TR>
      </TABLE></TD>
</TR>
<TR>
    <TD VALIGN="CENTER" > Iso_rectangle_2 </TD>
    <TD VALIGN="CENTER" > Ray_2 </TD>
    <TD><TABLE>
	<TR><TD>Point_2</TD></TR>
	<TR><TD>Segment_2</TD></TR>
      </TABLE></TD>
</TR>
<TR>
    <TD VALIGN="CENTER" > Iso_rectangle_2 </TD>
    <TD VALIGN="CENTER" > Iso_rectangle_2 </TD>
    <TD><TABLE>
	<TR><TD>Iso_rectangle_2</TD></TR>
      </TABLE></TD>
</TR>
<TR>
    <TD VALIGN="CENTER" > Line_3 </TD>
    <TD VALIGN="CENTER" > Line_3 </TD>
    <TD><TABLE>
	<TR><TD>Point_3</TD></TR>
	<TR><TD>Line_3</TD></TR>
        </TABLE></TD>
</TR>
<TR>
    <TD VALIGN="CENTER" > Line_3 </TD>
    <TD VALIGN="CENTER" > Plane_3 </TD>
    <TD><TABLE>
	<TR><TD>Point_3</TD></TR>
	<TR><TD>Line_3</TD></TR>
        </TABLE></TD>
</TR>
<TR>
    <TD VALIGN="CENTER" > Line_3 </TD>
    <TD VALIGN="CENTER" > Ray_3 </TD>
    <TD><TABLE>
	<TR><TD>Point_3</TD></TR>
	<TR><TD>Ray_3</TD></TR>
        </TABLE></TD>
</TR>
<TR>
    <TD VALIGN="CENTER" > Line_3 </TD>
    <TD VALIGN="CENTER" > Segment_3 </TD>
    <TD><TABLE>
	<TR><TD>Point_3</TD></TR>
	<TR><TD>Segment_3</TD></TR>
        </TABLE></TD>
</TR>
<TR>
    <TD VALIGN="CENTER" > Line_3 </TD>
    <TD VALIGN="CENTER" > Triangle_3 </TD>
    <TD><TABLE>
	<TR><TD>Point_3</TD></TR>
	<TR><TD>Segment_3</TD></TR>
        </TABLE></TD>
</TR>
<TR>
    <TD VALIGN="CENTER" > Plane_3 </TD>
    <TD VALIGN="CENTER" > Plane_3 </TD>
    <TD><TABLE>
	<TR><TD>Line_3</TD></TR>
	<TR><TD>Plane_3</TD></TR>
        </TABLE></TD>
</TR>
<TR>
    <TD VALIGN="CENTER" > Plane_3 </TD>
    <TD VALIGN="CENTER" > Ray_3 </TD>
    <TD><TABLE>
	<TR><TD>Point_3</TD></TR>
	<TR><TD>Ray_3</TD></TR>
        </TABLE></TD>
</TR>
<TR>
    <TD VALIGN="CENTER" > Plane_3 </TD>
    <TD VALIGN="CENTER" > Segment_3 </TD>
    <TD><TABLE>
	<TR><TD>Point_3</TD></TR>
	<TR><TD>Segment_3</TD></TR>
        </TABLE></TD>
</TR>
<TR>
    <TD VALIGN="CENTER" > Plane_3 </TD>
    <TD VALIGN="CENTER" > Sphere_3 </TD>
    <TD><TABLE>
	<TR><TD>Point_3</TD></TR>
	<TR><TD>Circle_3</TD></TR>
        </TABLE></TD>
</TR>
<TR>
    <TD VALIGN="CENTER" > Plane_3 </TD>
    <TD VALIGN="CENTER" > Triangle_3 </TD>
    <TD><TABLE>
	<TR><TD>Point_3</TD></TR>
	<TR><TD>Segment_3</TD></TR>
	<TR><TD>Triangle_3</TD></TR>
        </TABLE></TD>
</TR>
<TR>
    <TD VALIGN="CENTER" > Ray_3 </TD>
    <TD VALIGN="CENTER" > Ray_3 </TD>
    <TD><TABLE>
	<TR><TD>Point_3</TD></TR>
	<TR><TD>Ray_3</TD></TR>
        <TR><TD>Segment_3</TD></TR>
        </TABLE></TD>
</TR>
<TR>
    <TD VALIGN="CENTER" > Ray_3 </TD>
    <TD VALIGN="CENTER" > Segment_3 </TD>
    <TD><TABLE>
	<TR><TD>Point_3</TD></TR>
	<TR><TD>Segment_3</TD></TR>
        </TABLE></TD>
</TR>
<TR>
    <TD VALIGN="CENTER" > Ray_3 </TD>
    <TD VALIGN="CENTER" > Triangle_3 </TD>
    <TD><TABLE>
	<TR><TD>Point_3</TD></TR>
	<TR><TD>Segment_3</TD></TR>
        </TABLE></TD>
</TR>
<TR>
    <TD VALIGN="CENTER" > Segment_3 </TD>
    <TD VALIGN="CENTER" > Segment_3 </TD>
    <TD><TABLE>
	<TR><TD>Point_3</TD></TR>
	<TR><TD>Segment_3</TD></TR>
        </TABLE></TD>
</TR>
<TR>
    <TD VALIGN="CENTER" > Segment_3 </TD>
    <TD VALIGN="CENTER" > Triangle_3 </TD>
    <TD><TABLE>
	<TR><TD>Point_3</TD></TR>
	<TR><TD>Segment_3</TD></TR>
        </TABLE></TD>
</TR>
<TR>
    <TD VALIGN="CENTER" > Sphere_3 </TD>
    <TD VALIGN="CENTER" > Sphere_3 </TD>
    <TD><TABLE>
	<TR><TD>Point_3</TD></TR>
	<TR><TD>Circle_3</TD></TR>
	<TR><TD>Sphere_3</TD></TR>
        </TABLE></TD>
</TR>
<TR>
    <TD VALIGN="CENTER" > Triangle_3 </TD>
    <TD VALIGN="CENTER" > Triangle_3 </TD>
    <TD><TABLE>
	<TR><TD>Point_3</TD></TR>
	<TR><TD>Segment_3</TD></TR>
	<TR><TD>Triangle_3</TD></TR>
	<TR><TD>std::vector &lt; Point_3  &gt; </TD></TR>
        </TABLE></TD>
</TR>
</TABLE>
</DIV>
\end{ccHtmlOnly}

There is also an intersection function between 3 planes.

\ccFunction{boost::optional< boost::variant< Point_3, Line_3, Plane_3 > >   
  intersection(const Plane_3<Kernel>& pl1,
               const Plane_3<Kernel>& pl2,
               const Plane_3<Kernel>& pl3);}
{returns the intersection of 3 planes, which can be either a point, a line,
a plane, or empty.}

%%%%%%%%%%%%%%%%%%%%%%%%%%%%%%%%%%%%%%%%%%%%%%
\paragraph{With the 2D Circular Kernel} (see Chapter~\ref{chapter-circular-kernel}) 

\ccInclude{CGAL/Circular_kernel_intersections.h}

If this kernel is used, in addition to the function and the
combination of 2D types described above, another version of the function
is provided.

The iterator versions of those functions can be used in conjunction
with \ccc{CGAL::Dispatch_output_iterator<V,O>}.

Since both the number of intersections, if any, and their type,
depend on the arguments, the function expects an output
iterator on \ccStyle{K::Intersect_2::Result<Type1, Type2>::type}, as presented below. 

\ccFunction{template < class OutputIterator >
OutputIterator
intersection(const  Type1 &obj1, const Type2 &obj2,
		OutputIterator intersections);}
{Copies the intersection elements between the
two objects to the OutputIterator. \ccc{intersections} iterates on
elements of type \ccStyle{IT< Type1, Type2 >}, in lexicographic order,}

where \ccStyle{Type1} and \ccStyle{Type2} can both be either
\begin{itemize}
\item {} \ccStyle{Line_2<CircularKernel>} or
\item {} \ccStyle{Line_arc_2<CircularKernel>} or
\item {} \ccStyle{Circle_2<CircularKernel>} or
\item {} \ccStyle{Circular_arc_2<CircularKernel>}.
\end{itemize} 

Depending on the types \ccStyle{Type1} and \ccStyle{Type2}, these
elements can be assigned to
\begin{itemize}
\item {} \ccStyle{std::pair<Circular_arc_point_2<CircularKernel>, unsigned>},
where the unsigned integer is the multiplicity of the corresponding
intersection point between \ccc{obj1} and \ccc{obj2},
\item {} \ccStyle{Circular_arc_2<CircularKernel>} in case of an overlap of 
two circular arcs,
\item {} \ccStyle{Line_arc_2<CircularKernel>} in case of an overlap of two 
line segments or
\item {} \ccStyle{Line_2<CircularKernel>} or 
\ccStyle{Circle_2<CircularKernel>} in case of two equal input lines or circles.
\end{itemize} 


%%%%%%%%%%%%%%%%%%%%%%%%%%%%%%%%%%%%%%%%%%%%%%
\paragraph{With the 3D Spherical Kernel} (see Chapter~\ref{chapter-spherical-kernel}) 

\ccInclude{CGAL/Spherical_kernel_intersections.h}

If this kernel is used, in addition to the function and the
combination of 3D types described above, two other versions of the function
are provided.

The iterator versions of those functions can be used in conjunction
with \ccc{CGAL::Dispatch_output_iterator<V,O>}.

Since both the number of intersections, if any, and their type,
depend on the arguments, the functions expects an output
iterator on \ccStyle{Kernel::Intersect_3::Result<Type1, Type2>::Type}, as presented below.

The \textbf{first function} is:

\ccFunction{template < class OutputIterator >
OutputIterator
intersection(const  Type1 &obj1, const Type2 &obj2,
		OutputIterator intersections);}
{Copies in the output iterator the intersection elements between the
two objects. \ccc{intersections} iterates over
elements of type \ccStyle{IT< A, B >}, in lexicographic order,
when this ordering is defined on the computed objects,}

where \ccStyle{Type1} and \ccStyle{Type2} can both be either:
\begin{itemize}
\item {} \ccStyle{Sphere_3<SphericalKernel>},
\item {} \ccStyle{Plane_3<SphericalKernel>},
\item {} \ccStyle{Line_3<SphericalKernel>},
\item {} \ccStyle{Circle_3<SphericalKernel>},
\item {} \ccStyle{Line_arc_3<SphericalKernel>} or
\item {} \ccStyle{Circular_arc_3<SphericalKernel>},
\end{itemize} 

and depending on the types \ccStyle{Type1} and \ccStyle{Type2}, the computed 
type can be:
\begin{itemize}
\item {} \ccStyle{std::pair<Circular_arc_point_3<SphericalKernel>, unsigned>},
where the unsigned integer is the multiplicity of the corresponding
intersection point between \ccc{obj1} and \ccc{obj2},
\item {} \ccStyle{Type1}, when \ccStyle{Type1} and \ccStyle{Type2} are equal, 
and if the two objets \ccc{obj1} and \ccc{obj2} are equal,
\item {} \ccStyle{Line_3<SphericalKernel>} or 
\ccStyle{Circle_3<SphericalKernel>} when \ccStyle{Type1} and \ccStyle{Type2} 
are two-dimensional objets intersecting along a curve (2 planes, or 2
spheres, or one plane and one sphere),
\item {} \ccStyle{Circular_arc_3<SphericalKernel>} in case of an overlap of 
two circular arcs or
\item {} \ccStyle{Line_arc_3<SphericalKernel>} in case of an overlap of two 
line segments. 
\end{itemize} 

The \textbf{second function} is:

\ccFunction{template < class OutputIterator >
OutputIterator
intersection(const Type1 &obj1, const Type2 &obj2, const Type3 &obj3,
	OutputIterator intersections);}
{Copies in the output iterator the intersection elements between the
three objects. \ccc{intersections} iterates on
elements of type \ccStyle{boost::variant< Circle_3, Plane_3, Sphere_3, std::pair< Circular_arc_point_3, unsigned > >}
, in lexicographic order when this ordering is defined on the computed objects.}

where \ccStyle{Type1}, \ccStyle{Type2} and \ccStyle{Type3}
can be either
\begin{itemize}
\item {} \ccStyle{Sphere_3<SphericalKernel>} or
\item {} \ccStyle{Plane_3<SphericalKernel>}
\end{itemize}

and depending of these types, the computed return value
\begin{itemize}
\item {} \ccStyle{std::pair<Circular_arc_point_3<SphericalKernel>, unsigned>},
where the unsigned integer is the multiplicity of the corresponding
intersection point,
\item {} \ccStyle{Circle_3<SphericalKernel>} or
\item {} \ccStyle{Type1}, when \ccStyle{Type1}, \ccStyle{Type2} and 
\ccc{Type3} are equal, and if the three objets \ccc{obj1} and \ccc{obj2} 
and \ccc{obj3} are equal.
\end{itemize} 

%%%%%%%%%%%%%%%%%%%%%%%%%%%%%%%%%%%%%%%%%%%%%%
\ccExample

The following example demonstrates the most common use of 
\ccc{intersection} routines with the basic 2D and 3D Kernels.
\ccHtmlLinksOff%
\begin{verbatim}
#include <CGAL/intersections.h>

template<typename R>
struct Intersection_visitor {
  typedef result_type void;
  void operator()(const Point_2<R>& p) const { /* handle point */ }
  void operator()(const Segment_2<R>& s) const { /* handle segment */ }
};

template <class R>
void foo(Segment_2<R> seg, Line_2<R> lin)
{
  R::Intersect_2::template Result<Segment_2<R>, Line_2<R> >::Type 
    result = intersection(seg, lin);

  // alternatively, with C++11 support
  auto result = intersection(seg, lin);

  if(result) { boost::apply_visitor(Intersection_visitor(), *result); } 
  else { /* no intersection */ }

  // alternatively:
  if(result) {
    if(const Segment_2<R>* s = boost::get<Segment_2>(&*result)) {
      // handle segment
    } else if(const Point_2<R>* p = boost::get<Point_2>(&*result)) {
      // handle point
    }
  } else {
    // no intersection
  }
}
\end{verbatim}%
\ccHtmlLinksOn%

Examples illustrating the use of this function in the case of the 2D
Circular Kernel and the 3D Spherical Kernel are presented respectively
in Chapters~\ref{chapter-circular-kernel}
and~\ref{chapter-spherical-kernel}.

\paragraph{Upgrading code from CGAL::Object to boost::variant}

Code can be upgraded by using either \ccStyle{boost::get} or the
\ccStyle{boost::static_visitor<T>}.

\ccHtmlLinksOff%
\begin{verbatim}
#include <CGAL/intersections.h>

template<typename R>
struct Intersection_visitor : public boost::static_visitor<> {
  void operator()(const Point_2& p) const { /* Point_2 */ }
  void operator()(const Segment_2& s) const { /* Segment_2 */ }
};

template <class R>
void foo(Segment_2<R> seg, Line_2<R> lin)
{
  CGAL::Object obj = intersection(seg1, seg2);
  if(const Point_2* foo = object_cast<Point_2>(&obj)) {
    /* Point_2 */
  } else if(const Segment_2* foo = object_cast<Segment_2>(&obj)) {
    /* Segment_2 */
  } else {
    /* empty */
  }

  // becomes
  auto result = intersection(seg, lin);
  if(result) {
    if(const Point_2* foo = boost::get<Point_2>(&*obj)) {
      /* Point_2 */
    } else if(const Segment_2* foo = boost::get<Segment_2>(&*obj)) {
      /* Segment_2 */
    }
  } else {
    /* empty */
  }

  // or with boost::static_visitor<T>
  if(result) { boost::apply_visitor(Intersection_visitor(), *result); } 
  else { /* empty */ }
}
\end{verbatim}%
\ccHtmlLinksOn%

\ccExample

An example showing the use of the intersection function a plain function call and with \ccc{CGAL::Dispatch_output_iterator} combined with a standard library algorithm.

\ccIncludeExampleCode{Kernel_23/intersections.cpp}

\ccSeeAlso
\ccRefIdfierPage{CGAL::do_intersect}, \\
\ccRefIdfierPage{CGAL_INTERSECTION_VERSION}, \\
\ccAnchor{http://www.boost.org/doc/libs/release/libs/optional/index.html}{boost::optional}, \\
\ccAnchor{http://www.boost.org/doc/html/variant.html}{boost::variant}

\end{ccRefFunction}
