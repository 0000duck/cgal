\section{Introduction}
\cgal, the {\em Computational Geometry Algorithms Library}, is written in 
\CC\ and consists of three major parts.

The first part is the kernel, which consists of constant-size non-modifiable 
geometric primitive objects and operations on these objects. 
The objects are represented both as stand-alone classes that are
parameterized by a representation class, which specifies
the underlying number types used for calculations and as members of the
kernel classes, which allows for more flexibility and adaptability of the 
kernel.

The second part is a collection of basic geometric data structures and
algorithms, which are parameterized by traits classes that define the 
interface between the data structure or algorithm and the primitives they use.
In many cases, the kernel classes provided in \cgal\ can be used as traits
classes for these data structures and algorithms.
The collection of basic geometric algorithms and data structures
currently includes polygons, half-edge data structures, polyhedral surfaces, 
topological maps, planar maps, arrangements of curves, triangulations, 
convex hulls, alpha shapes, optimisation algorithms, dynamic
point sets for geometric queries, and multidimensional search trees.

The third part of the library consists of non-geometric support
facilities, such as support for number types, STL extensions for
\cgal, handles, circulators, protected access to internal
representations, geometric object generators, timers, I/O stream
operators and other stream support including PostScript, colors,
windows, and visualization tools GeoWin, Geomview and a Qt widget for
2D \cgal\ objects.

Additional documents accompanying the \cgal\ distribution are the
`Installation Guide' and `The Use of \stl\ and \stl\ Extensions in
\cgal', which gives a manual style introduction to \stl\ constructs
such as iterators and containers, as well an extension, called
circulator, used in many places in \cgal. We also recommend the
standard text book by Austern~\cite{cgal:a-gps-98} for the \stl\ and
its notion of \emph{concepts} and \emph{models}.

% and `Getting Started with \cgal'
% The third document gives an introduction in
% \cgal programming for the novice user.

Other resources for \cgal\ are the tutorials at
\path|http://www.cgal.org/Tutorials/| and the user support page at
\path|www.cgal.org|.



{This} part of the reference manual covers the kernel.
The kernel contains objects of constant size, such as point, vector, 
direction, line, ray, segment, triangle, iso-oriented rectangle and 
tetrahedron. 
With each type comes a set of functions which can be applied to an object 
of this type.  
You will typically find access functions (e.g.\ to the coordinates of a point), 
tests of the position of a point relative to the object, a function returning 
the bounding box, the length, or the area of an object, and so on.
The \cgal\ kernel further contains basic operations such as affine
transformations, detection and computation of intersections, and distance 
computations.

\subsection{Robustness}
The correctness proof of nearly all geometric algorithms presented
in theory papers assumes exact computation with real numbers.
This leads to a fundamental problem with the implementation of geometric
algorithms.
Naively, often the exact real arithmetic is replaced by inexact floating-point 
arithmetic in the implementation.
This often leads to acceptable results for many input data.
However, even for the implementation of the simplest geometric 
algorithms this simplification occasionally does not work.
Rounding errors introduced by an inaccurate arithmetic may lead to 
inconsistent decisions, causing unexpected failures for some correct 
input data.
There are many approaches to this problem, one of them is to compute
exactly (compute so accurate that all decisions made by the algorithm
are exact) which is possible in many cases but more expensive
than standard floating-point arithmetic.
C.~M.~Hoffmann~\cite{h-gsm-89}, \cite{h-pargc-89} illustrates some 
of the problems arising in the implementation of geometric algorithms
and discusses some approaches to solve them. 
A more recent overview is given in \cite{s-rpigc-00}.
The exact computation paradigm is discussed by Yap and Dub\'e 
\cite{yd-ecp-95} and Yap \cite{y-tegc-97}.

In \cgal\ you can choose the underlying number types and arithmetic.  
You can use different types of arithmetic simultaneously and the choice can 
be easily changed, e.g.\ for testing.  
So you can choose between implementations with fast but occasionally inexact 
arithmetic and implementations guaranteeing exact computation and exact 
results.  
Of course you have to pay for the exactness in terms of execution time 
and storage space.  
See the dedicated chapter
for more details on number types and their capabilities and performance.
