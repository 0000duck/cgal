\section{Kernel Representations}
Our object of study is the $d$-dimensional affine Euclidean space. 
Here we are mainly concerned with cases $d=2$ and $d=3$.
Objects in that space are sets of points. A common way to represent 
the points is the use of \ccHtmlNoLinksFrom{Cartesian} coordinates,
which assumes a reference frame (an origin and $d$ orthogonal axes).
In that framework, a point is represented by a $d$-tuple
\ccTexHtml{$(c_0,c_1,\ldots,c_{d-1})$}{<MATH>(c<SUB>0</SUB>,c<SUB>1</SUB>,...,c<SUB>d-1</SUB>)</MATH>},
and so are vectors in the underlying linear space. Each point is 
represented uniquely by such \ccHtmlNoLinksFrom{Cartesian} coordinates.
%An object can then be represented by an equation, that is, it is the 
%set of
%points whose coordinates satisfy the equation. For example, the 
%equation of a
%line is usually given as $ax + by +c = 0$.
%The line can then be represented by three numbers $a$, $b$, $c$.
Another way to represent points is by homogeneous coordinates. In that
framework, a point is represented by a $(d+1)$-tuple
\ccTexHtml{$(h_0,h_1,\ldots,h_d)$}{<MATH>(h<SUB>0</SUB>,h<SUB>1</SUB>,...,h<SUB>d</SUB>)</MATH>}.
Via the formulae
\ccTexHtml{$c_i = h_i/h_d$}{<MATH>c<SUB>i</SUB>=h<SUB>i</SUB>/h<SUB>d</SUB></MATH>},
the corresponding point with \ccHtmlNoLinksFrom{Cartesian} coordinates
\ccTexHtml{$(c_0,c_1,\ldots,c_{d-1})$}{<MATH>(c<SUB>0</SUB>,c<SUB>1</SUB>,...,c<SUB>d-1</SUB>)</MATH>}
can be computed.  Note that homogeneous coordinates are not unique. 
For \ccTexHtml{$\lambda\ne 0$}{<MATH>lambda != 0</MATH>},
the tuples 
\ccTexHtml{$(h_0,h_1,\ldots,h_d)$}{<MATH>(h<SUB>0</SUB>,h<SUB>1</SUB>
,...,h<SUB>d</SUB>)</MATH>}
 and 
\ccTexHtml{$(\lambda\cdot h_0,\lambda\cdot h_1,\ldots,\lambda\cdot 
h_d)$}{<MATH>(lambda  h<SUB>0</SUB>,lambda  h<SUB>1</SUB>,...,lambda  
h<SUB>d</SUB>)</MATH>} represent the
same point.
For a point with \ccHtmlNoLinksFrom{Cartesian} coordinates \ccTexHtml{$(c_0,c_1,\ldots,c_{
d-1})$}{<MATH>(c<SUB>0</SUB>,c<SUB>1</SUB>,...,c<SUB>d-1</SUB>)</MATH>} a
possible homogeneous representation is
\ccTexHtml{$(c_0,c_1,\ldots,c_{d-1},1)$}{<MATH>(c<SUB>0</SUB>,c<SUB>1</SUB>,...,c<SUB>d-1</SUB>,1)</MATH>}.
\ccHtmlNoLinksFrom{Homogeneous} coordinates in fact allow to represent 
objects in a more general space, the projective space
\ccTexHtml{$\P_d$}{<MATH>P<SUB>d</SUB></MATH>}.
In \cgal, we do not compute in projective geometry. Rather, we use 
homogeneous coordinates to avoid division operations,
since the additional coordinate can serve as a common denominator.

\subsection{Genericity through Parameterization}

Almost all the kernel objects (and the corresponding functions) are
templates with a parameter that allows the user to choose the
representation of the kernel objects. A type that is used as an
argument for this parameter must fulfill certain requirements on
syntax and semantics. The list of requirements defines an abstract
kernel concept. For all kernel objects types, the types
\ccHtmlNoLinksFrom{\ccc{CGAL::Type<Kernel>}} and 
\ccHtmlNoLinksFrom{\ccc{Kernel::Type}} are identical.

CGAL offers four families of concrete models for the concept Kernel,
two based on the \ccHtmlNoLinksFrom{Cartesian} representation of
points and two based on the homogeneous representation of points. The
interface of the kernel objects is designed such that it works well
with both \ccHtmlNoLinksFrom{Cartesian} and homogeneous
representation. For example, points in 2D have a constructor with
three arguments as well (the three homogeneous coordinates of the
point).  The common interfaces parameterized with a kernel class allow
one to develop code independent of the chosen representation.  We said
``families'' of models, because both families are parameterized too.
A user can choose the number type used to represent the coordinates.

For reasons that will become evident later, a kernel class provides
two typenames for number types,
namely \ccc{Kernel::FT} and \ccc{Kernel::RT}.%
\ccTexHtml{\footnote{The double colon :: is the \CC\ scope
    operator.}}{} The type \ccStyle{Kernel::FT} must fulfill the
requirements on what is called a {\em FieldNumberType} in \cgal. This
roughly means that \ccStyle{Kernel::FT} is a type for which operations
$+$, $-$, $*$ and $/$ are defined with semantics (approximately)
corresponding to those of a field in a mathematical sense. Note that,
strictly speaking, the built-in type \ccc{int} does not fulfill the
requirements on a field type, since \ccc{int}s correspond to elements
of a ring rather than a field, especially operation $/$ is not the
inverse of $*$.  The requirements on the type \ccStyle{Kernel::RT} are
weaker.  This type must fulfill the requirements on what is called a
{\em RingNumberType} in \cgal. This roughly means that
\ccStyle{Kernel::RT} is a type for which operations $+$, $-$, $*$ are
defined with semantics (approximately) corresponding to those of a
ring in a mathematical sense. 
%%Furthermore, both number types should fulfill \cgal's requirements on
%%a number type.  Note that a ring type is always a field type but not
%%the other way round.

\subsection{Cartesian Kernels}
With \ccc{Cartesian<FieldNumberType>} you can choose a
\ccHtmlNoLinksFrom{Cartesian} representation of coordinates. When you
choose \ccHtmlNoLinksFrom{Cartesian} representation you have to
declare at the same time the type of the coordinates.  A number type
used with the \ccc{Cartesian} representation class should be a
FieldNumberType as described above. As mentioned above, the built-in
type \ccc{int} is not a FieldNumberType. However, for some
computations with \ccHtmlNoLinksFrom{Cartesian} representation, no
division operation is needed, i.e., a \ccc{RingNumberType} is sufficient in
this case. With \ccc{Cartesian<FieldNumberType>}, both
\ccc{Cartesian<FieldNumberType>::FT} and
\ccc{Cartesian<FieldNumberType>::RT} are mapped to
\ccc{FieldNumberType}.

\ccc{Cartesian<FieldNumberType>} uses reference counting internally to
save copying costs. CGAL also provides
\ccc{Simple_cartesian<FieldNumberType>}, a kernel that uses
\ccHtmlNoLinksFrom{Cartesian} representation but no reference
counting.  Debugging is easier with
\ccc{Simple_cartesian<FieldNumberType>}, since the coordinates are
stored within the class and hence direct access to the coordinates is
possible.  Depending on the algorithm, it can also be slightly more or
less efficient than \ccc{Cartesian<FieldNumberType>}. Again, in
\ccc{Simple_cartesian<FieldNumberType>} both
\ccc{Simple_cartesian<FieldNumberType>::FT} and
\ccc{Simple_cartesian<FieldNumberType>::RT} are mapped to
\ccc{FieldNumberType}.

\subsection{Homogeneous Kernels}
Homogeneous coordinates permit to avoid division operations in
numerical computations, since the additional coordinate can serve as a
common denominator.  Avoiding divisions can be useful for exact
geometric computation.  With \ccc{Homogeneous<RingNumberType>} you can
choose a homogeneous representation for the coordinates of the kernel
objects. As for the \ccHtmlNoLinksFrom{Cartesian} representation, one
has to declare the type used to store the coordinates. Since the
homogeneous representation does not use divisions, the number type
associated with a homogeneous representation class must be a model for
the weaker concept \ccc{RingNumberType} only. However, some operations
provided by this kernel involve divisions, for example computing
squared distances or \ccHtmlNoLinksFrom{Cartesian} coordinates. To
keep the requirements on the number type parameter of
\ccc{Homogeneous} low, the number type
\ccStyle{Quotient<RingNumberType>} is used for operations that require
divisions. This number type can be viewed as an adaptor which turns a
\ccc{RingNumberType} into a \ccc{FieldNumberType}. It maintains numbers as
quotients, i.e., a numerator and a denominator. With
\ccc{Homogeneous<RingNumberType>},
\ccc{Homogeneous<RingNumberType>::FT} is equal to
\ccStyle{Quotient<RingNumberType>}, while
\ccc{Homogeneous<RingNumberType>::RT} is equal to
\ccc{RingNumberType}.

\ccc{Homogeneous<RingNumberType>} uses reference counting internally
to save copying costs. CGAL also provides
\ccc{Simple_homogeneous<RingNumberType>}, a kernel that uses
\ccHtmlNoLinksFrom{homogeneous} representation but no reference
counting. Debugging is easier with
\ccc{Simple_homogeneous<RingNumberType>}, since the coordinates are
stored within the class and hence direct access to the coordinates is
possible. Depending on the algorithm, it can also be slightly more or
less efficient than \ccc{Homogeneous<RingNumberType>}. Again, in
\ccc{Simple_homogeneous<RingNumberType>} the type
\ccc{Simple_homogeneous<RingNumberType>::FT} is equal to
\ccStyle{Quotient<RingNumberType>} while
\ccc{Simple_homogeneous<RingNumberType>::RT} is equal to
\ccc{RingNumberType}.

\subsection{Naming conventions}

The use of kernel classes not only avoids problems, it also makes all
\cgal\ classes very uniform. They {\bf always} consist of:
\begin{enumerate}
\begin{ccTexOnly}
\itemsep0pt\topsep0pt\partopsep0pt\parskip0pt\parsep0pt
\end{ccTexOnly}

\item The {\em capitalized base name} of the geometric object, such as 
      \ccStyle{Point}, \ccStyle{Segment}, or \ccStyle{Triangle}.

\item An {\em underscore} followed by the {\em dimension} of the object, 
      for example $\_2$, $\_3$, or $\_d$.
      
    \item A {\em kernel class} as parameter, which itself is
      parameterized with a number type, such as
      \ccStyle{Cartesian<double>} or
      \ccStyle{Homogeneous<leda_integer>}.
\end{enumerate}

\subsection{Kernel as a Traits Class}
Algorithms and data structures in the basic library of \cgal\ are
parameterized by a traits class that subsumes the objects on which the
algorithm or data structure operates as well as the operations to do
so. For most of the algorithms and data structures in the basic
library you can use a kernel as a traits class. For some algorithms
you even do not have to specify the kernel; it is detected
automatically using the types of the geometric objects passed to the
algorithm. In some other cases, the algorithms or data structures
needs more than is provided by the kernel concept. In these cases, a
kernel can not be used as a traits class.

\subsection{Choosing a Kernel and Predefined Kernels}
If you start with integral \ccHtmlNoLinksFrom{Cartesian} coordinates,
many geometric computations will involve integral numerical values
only. Especially, this is true for geometric computations that
evaluate only predicates, which are tantamount to determinant
computations. Examples are triangulation of point sets and convex hull
computation.  In this case, the \ccHtmlNoLinksFrom{Cartesian}
representation is probably the first choice, even with a ring type.
You might use limited precision integer types like \ccc{int} or
\ccc{long}, use \ccc{double} to present your integers (they have more
bits in their mantissa than an \ccc{int} and overflow nicely), or an
arbitrary precision integer type like the wrapper \ccc{Gmpz} for the
GMP integers, \ccc{leda_integer}, or \ccc{MP_Float}. Note, that unless
you use an arbitrary precision ring type, incorrect results might
arise due to overflow.

If new points are to be constructed, for example the
\ccHtmlNoLinksFrom{intersection} point of two lines, computation of
\ccHtmlNoLinksFrom{Cartesian} coordinates usually involves divisions.
Hence, one needs to use a \ccc{FieldNumberType} with
\ccHtmlNoLinksFrom{Cartesian} representation, or alternatively, switch
to homogeneous representation. The type \ccc{double} is a -- though
imprecise -- model for \ccc{FieldNumberType}. You can also put any
\ccc{RingNumberType} into the \ccc{Quotient} adaptor to get a field type
which then can be put into \ccc{Cartesian}. But using homogeneous
representation on the \ccc{RingNumberType} is usually the better option.
Other valid \ccc{FieldNumberType}s are \ccc{leda_rational} and
\ccc{leda_real}.

If it is crucial for you that the computation is reliable, the right
choice is probably a number type that guarantees exact computation.
The \ccc{Filtered_kernel} provides a way to apply filtering techniques
\cite{cgal:bbp-iayed-01} to achieve a kernel with exact and efficient
predicates. % given an exact kernel and a filtering kernel.
%The number type \ccc{leda_real} guarantees
%that all decisions and hence all branchings in a computation
%are correct. They also allow you to compute approximations to whatever
%precision you need. Furthermore computation with 
%\ccc{leda_real} is faster than computation with arbitrary precision
%arithmetic. So if you would like to avoid surprises caused by imprecise
%computation, this is a good choice. In fact, it is a good choice with
%both representations, since divisions slow down the computation of
%the reals and hence it might pay-off to avoid them.
Still other people will prefer the built-in
type {\tt double}, because they need speed and can live with
approximate results, or even algorithms that, from time to time,
crash or compute incorrect results due to accumulated rounding errors.

\paragraph{Predefined kernels.}
For the user's convenience, \cgal\ provides 3 typedefs to generally useful
kernels.

\begin{itemize}
\item They are all \ccHtmlNoLinksFrom{Cartesian} kernels.
\item They all support constructions of points from \texttt{double} \ccHtmlNoLinksFrom{Cartesian}
      coordinates.
\item All these 3 kernels provide exact geometric predicates.
\item They handle geometric constructions differently:
  \begin{itemize}
  \item \ccc{Exact_predicates_exact_constructions_kernel}: provides exact
        geometric constructions, in addition to exact geometric predicates.
  \item \ccc{Exact_predicates_exact_constructions_kernel_with_sqrt}:
        same as \ccc{Exact_predicates_exact_constructions_kernel}, but the
        number type it provides
        (\ccc{Exact_predicates_exact_constructions_kernel_with_sqrt::FT})
        supports the square root operation exactly
        \footnote{Currently it requires having either LEDA or CORE installed}.
  \item \ccc{Exact_predicates_inexact_constructions_kernel}: provides exact
        geometric predicates, but geometric constructions may be inexact due to
        round-off errors.  It is however enough for most \cgal\ algorithms, and
        faster than both \ccc{Exact_predicates_exact_constructions_kernel} and
        \ccc{Exact_predicates_exact_constructions_kernel_with_sqrt}.
  \end{itemize}
\end{itemize}

