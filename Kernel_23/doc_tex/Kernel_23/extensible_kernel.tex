\section{Extensible Kernel}
\label{section-extensible-kernel}

This manual section describe how users can plug user defined
geometric classes in existing \cgal\ kernels.  This is best
illustrated by an example.


\subsection{Introduction}

\cgal\ defines the concept of a geometry kernel. Such a kernel provides types,
construction objects and generalized predicates. Most implementations
of Computational Geometry algorithms and data structures in the basic
library of \cgal\ were done in a way that classes or functions can be
parametrized with a geometric traits class.

In most cases this geometric traits class must be a model of the \cgal\ geometry
kernel concept (but there are some exceptions).




\subsection{An Extensive Example}

Assume you have the following point class, where the coordinates are
stored in an array of \ccc{doubles}, where we have another data member
\ccc{color}, which shows up in the constructor.

\ccHtmlLinksOff
\begin{ccExampleCode}
class MyPointC2 {

private:
  double vec[2];
  int col;

public:

  MyPointC2()
    : col(0)
  {
    *vec = 0;
    *(vec+1) = 0;
  }

  
  MyPointC2(const double x, const double y, int c)
    : col(c)
  {
    *vec = x;
    *(vec+1) = y;
  }

  const double& x() const  { return *vec; }

  const double& y() const { return *(vec+1); }

  double & x() { return *vec; }

  double& y() { return *(vec+1); }

  int color() const { return col; }

  int& color() { return col; }
  
  
  bool operator==(const MyPointC2 &p) const
  {
    return ( *vec == *(p.vec) )  && ( *(vec+1) == *(p.vec + 1) && ( col == p.col) );
  }

  bool operator!=(const MyPointC2 &p) const
  {
      return !(*this == p);
  }

};
\end{ccExampleCode}
\ccHtmlLinksOn


As said earlier the class is pretty minimalistic, for
example it has no \ccc{bbox()} method.  One
might assume that a basic library algorithm which computes 
a bounding box (e.g, to compute the bounding box of a polygon),
will not compile. Luckily it will, because it does not
use of member functions of geometric objects, but it makes
use of the functor \ccc{Kernel::Construct_bbox_2}.

To make the right thing happen with \ccc{MyPointC2} we
have to provide the following functor.

\ccHtmlLinksOff
\begin{ccExampleCode}
template <class ConstructBbox_2>
class MyConstruct_bbox_2 : public ConstructBbox_2 {
public:
  CGAL::Bbox_2 operator()(const typename MyPointC2& p) const {
    return CGAL::Bbox_2(p.x(), p.y(), p.x(), p.y());
  }
};
\end{ccExampleCode}
\ccHtmlLinksOn


Things are similar for random access to the \ccHtmlNoLinksFrom{Cartesian}
coordinates of a point. As the coordinates are stored
in an array of \ccc{doubles} we can use \ccc{double*} as
random access iterator.

\ccHtmlLinksOff
\begin{ccExampleCode}
class MyConstruct_coord_iterator {
public:
  const double* operator()(const MyPointC2& p)
  {
    return &p.x();
  }

  const double* operator()(const MyPointC2& p, int)
  {
    const double* pyptr = &p.y();
    pyptr++;
    return pyptr;
  }
};
\end{ccExampleCode}
\ccHtmlLinksOn

The last functor we have to provide is the one which constructs
points. That is you are not forced to add the constructor 
with the \ccc{Origin} as parameter to your class, nor the constructor with 
homogeneous coordinates, and at the same time you can 
pass the additional color argument to your point constructor.
The functor is a kind of glue layer between the \cgal\ algorithms
and your class.

\ccHtmlLinksOff
\begin{ccExampleCode}
 template <typename K>
  class MyConstruct_point_2
  {
    typedef typename K::RT         RT;
    typedef typename K::Point_2    Point_2;
  public:
    typedef Point_2          result_type;
    typedef CGAL::Arity_tag< 1 >   Arity;

    Point_2
    operator()(CGAL::Origin o) const
    { return Point_2(0,0, 0); }

    Point_2
    operator()(const RT& x, const RT& y) const
    { return Point_2(x, y, 0); }

    
    // We need this one, as such a functor is in the Filtered_kernel
    Point_2
    operator()(const RT& x, const RT& y, const RT& w) const
    { 
      if(w != 1){
	return Point_2(x/w, y/w, 0); 
      } else {
	return Point_2(x,y, 0);
      }
    }
  };

\end{ccExampleCode}
\ccHtmlLinksOn


Now we are ready to put the puzzle together. We won't explain it in
detail, but you see that there are \ccc{typedefs} to the new point
class and the functors. All the other types are inherited.



\ccHtmlLinksOff

\ccIncludeExampleCode{Kernel_23/MyKernel.h}

\ccHtmlLinksOn



Finally, we give an example how this new kernel can be used.
Predicates and constructions work with the new point, they
can be a used to construct segments and triangles with, and
data structures from the Basic Library, as the Delaunay
triangulation work with them.  

The kernel itself can be
made robust by plugging it in the \ccc{Filtered_kernel_adaptor}.
This class has the same functionality as the class \ccc{Filtered_kernel}
with the only difference that it does not enforce type equality
between \ccc{Kernel::Point_2} and \ccc{Point_2<Kernel>}, which
does not make sense in our case as our point does not offer
the functionality of the latter.


\ccHtmlLinksOff
\ccIncludeExampleCode{Kernel_23/MyKernel.C}
\ccHtmlLinksOn


\subsection{Limitations}

The point class must have member functions \ccc{x()} and \ccc{y()}
(and \ccc{z()} for the 3d point). We will probably
introduce function objects that take care of coordinate
access.

Global functions operating on, for example
\ccc{CGAL::orientation(CGAL::MyKernel<double>::Point_2,CGAL::MyKernel<double>::Point_2,CGAL::MyKernel<double>::Point_2} will not work. 
Instead you have to use the functor \ccc{MyKernel<double>::Orientation_2}.

Rewriting the code is however straightforward: you can give
the function object the same name as the global function.

\ccHtmlLinksOff
\begin{ccExampleCode}

K::Orientation_2 orientation_2;
K::Point_2 p(0,0,RED), q(1,1,BLACK), r(2,1, RED);

orientation(p, q, r);

\end{ccExampleCode}
\ccHtmlLinksOn

We did not go through the entire kernel, to also make
it possible to plug in user defined lines, rays or triangles.







