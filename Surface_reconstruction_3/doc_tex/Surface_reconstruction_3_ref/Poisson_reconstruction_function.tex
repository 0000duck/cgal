% +------------------------------------------------------------------------+
% | Reference manual page: Poisson_reconstruction_function.tex
% +------------------------------------------------------------------------+
% | 07.09.2007   Pierre Alliez, Laurent Saboret, Gael Guennebaud
% | Package: Surface_reconstruction_3
% |
\RCSdef{\RCSPoissonreconstructionfunctionRev}{$Id$}
\RCSdefDate{\RCSPoissonreconstructionfunctionDate}{$Date$}
% |
\ccRefPageBegin
%%RefPage: end of header, begin of main body
% +------------------------------------------------------------------------+


\begin{ccRefClass}{Poisson_reconstruction_function<GeomTraits, ReconstructionTriangulation_3>}

%% \ccHtmlCrossLink{}     %% add further rules for cross referencing links
%% \ccHtmlIndexC[class]{} %% add further index entries

\ccDefinition

% Insert image poisson.jpg/eps
\begin{center}
    \label{Surface_reconstruction_3-fig-poisson}
    % Image
    \begin{ccTexOnly}
        \includegraphics[width=0.7\textwidth]{Surface_reconstruction_3/poisson} % omit .eps suffix
    \end{ccTexOnly}
    \begin{ccHtmlOnly}
        <img width="70%" border=0 src="../Surface_reconstruction_3/poisson.jpg"><P>
    \end{ccHtmlOnly}
    % Title
    \begin{figure}[h]
        \caption{Poisson reconstruction}
    \end{figure}
\end{center}

Kazhdan, Bolitho and Hoppe described in \cite{Kazhdan06}
the Poisson Surface Reconstruction algorithm.
\cgal\ implements a variant of this algorithm which computes an implicit function
piecewise-linear in a 3D Delaunay triangulation instead of an octree.

% The section below is automatically generated. Do not edit!
%START-AUTO(\ccDefinition)

\ccc{Poisson_reconstruction_function} computes an indicator function f() piecewise-linear over the tetrahedra. We solve the Poisson equation Laplacian(f) = divergent(normals field) at each vertex of the triangulation via the TAUCS sparse linear solver. One vertex must be constrained.

%END-AUTO(\ccDefinition)

% The section below is automatically generated. Do not edit!
%START-AUTO(\ccInclude)

\ccInclude{CGAL/Poisson_reconstruction_function.h}

%END-AUTO(\ccInclude)

\ccParameters

The full template declaration is:

% The section below is automatically generated. Do not edit!
%START-AUTO(\ccParameters)

template$<$  \\
class Gt,   \\
class \ccc{ReconstructionTriangulation_3}$>$   \\
class \ccc{Poisson_reconstruction_function};

\begin{description}
\item[Parameters:]
\begin{description}
\item[Gt]Geometric traits class \item[\ccc{ReconstructionTriangulation_3}]3D Delaunay triangulation, model of \ccc{ReconstructionTriangulation_3} concept. \end{description}
\end{description}

%END-AUTO(\ccParameters)

\ccIsModel

% The section below is automatically generated. Do not edit!
%START-AUTO(\ccIsModel)

Model of the ImplicitFunction concept.

%END-AUTO(\ccIsModel)

\ccHeading{Design Pattern}

% The section below is automatically generated. Do not edit!
%START-AUTO(\ccHeading{Design Pattern})

A model of ImplicitFunction is a Strategy \cite{cgal:ghjv-dpero-95}: it implements a strategy of surface mesh reconstruction.

%END-AUTO(\ccHeading{Design Pattern})

\ccTypes

% The section below is automatically generated. Do not edit!
%START-AUTO(\ccTypes)

\ccNestedType{Triangulation}
{
}
\ccGlue
\ccNestedType{Geom_traits}
{
Kernel's geometric traits.
}
\ccGlue
\ccNestedType{FT}
{
}
\ccGlue
\ccNestedType{Point}
{
}
\ccGlue
\ccNestedType{Vector}
{
}
\ccGlue
\ccNestedType{Iso_cuboid}
{
}
\ccGlue
\ccNestedType{Sphere}
{
}
\ccGlue
\ccNestedType{Point_with_normal}
{
Model of \ccc{PointWithNormal_3}.
}
\ccGlue
\ccNestedType{Normal}
{
Model of \ccc{Kernel::Vector_3} concept.
}
\ccGlue

%END-AUTO(\ccTypes)

\ccCreation
\ccCreationVariable{fct}  %% variable name for \ccMethod below

% The section below is automatically generated. Do not edit!
%START-AUTO(\ccCreation)

\ccConstructor{Poisson_reconstruction_function(Triangulation& pdt);}
{
Create a Poisson indicator function f() piecewise-linear over the tetrahedra of pdt. If pdt is empty, create an empty implicit function.
}
\ccGlue
\begin{description}
\item[Parameters:]
\begin{description}
\item[pdt]\ccc{ReconstructionTriangulation_3} base of the Poisson indicator function. \end{description}
\end{description}
\ccGlue
\ccConstructor{template<class InputIterator>Poisson_reconstruction_function(Triangulation& pdt, InputIterator first, InputIterator beyond);}
{
Create an implicit function from a point set. Insert the first...beyond point set into pdt and create a Poisson indicator function f() piecewise-linear over the tetrahedra of pdt.
Precondition: the value type of InputIterator must be convertible to \ccc{Point_with_normal}.
}
\ccGlue
\begin{description}
\item[Parameters:]
\begin{description}
\item[pdt]\ccc{ReconstructionTriangulation_3} base of the Poisson indicator function. \item[first]First point to add. \item[beyond]Past-the-end point to add. \end{description}
\end{description}
\ccGlue

%END-AUTO(\ccCreation)

\ccOperations

% The section below is automatically generated. Do not edit!
%START-AUTO(\ccOperations)

\ccMethod{template<class InputIterator>int insert(InputIterator first, InputIterator beyond);}
{
Insert points.
Precondition: the value type of InputIterator must be convertible to \ccc{Point_with_normal}.
}
\ccGlue
\begin{description}
\item[Parameters:]
\begin{description}
\item[first]First point to add. \item[beyond]Past-the-end point to add. \end{description}
\end{description}
\begin{description}
\item[Returns:]the number of inserted points. \end{description}
\ccGlue
\ccMethod{void clear();}
{
Remove all points.
}
\ccGlue
\ccMethod{Triangulation& triangulation();}
{
Get embedded triangulation.
}
\ccGlue
\ccMethod{const Triangulation& triangulation() const;}
{
}
\ccGlue
\ccMethod{Iso_cuboid bounding_box() const;}
{
Get the surface's bounding box.
}
\ccGlue
\ccMethod{Sphere bounding_sphere() const;}
{
Get the surface's bounding sphere.
}
\ccGlue
\ccMethod{Sphere region_of_interest() const;}
{
Get the region of interest, ignoring the outliers. This method is used to define the OpenGL arcball sphere.
}
\ccGlue
\ccMethod{bool compute_implicit_function();}
{
You should call \ccc{compute_implicit_function}() once when points insertion is over. It computes the Poisson indicator function f() at each vertex of the triangulation by:\begin{itemize}
\item applying a Delaunay refinement to define the function inside and outside the surface.\item solving the Poisson equation Laplacian(f) = divergent(normals field) at each vertex of the triangulation via the TAUCS sparse linear solver. One vertex must be constrained.\item shifting and orienting f() such that f() = 0 on the input points, and f() $<$ 0 inside the surface.\end{itemize}
Return false on error. TODO: add parameters to \ccc{compute_implicit_function}()?
Compute the Poisson indicator function f() at each vertex of the triangulation.
}
\ccGlue
\ccMethod{void average_spacing_avg_knn_sq_distance_3();}
{
}
\ccGlue
\ccMethod{unsigned int delaunay_refinement(FT radius_edge_ratio_bound, FT cell_radius_bound, unsigned int max_vertices, FT enlarge_ratio);}
{
Delaunay refinement (break bad tetrahedra, where bad means badly shaped or too big). The normal of Steiner points is set to zero. Return the number of vertices inserted.
}
\ccGlue
\begin{description}
\item[Parameters: ]
\begin{description}
\item[\ccc{radius_edge_ratio_bound}]radius edge ratio bound (ignored if zero) \item[\ccc{cell_radius_bound}]cell radius bound (ignored if zero) \item[\ccc{max_vertices}]number of vertices bound \item[\ccc{enlarge_ratio}]bounding box enlarge ratio \end{description}
\end{description}
\ccGlue
\ccMethod{unsigned int delaunay_refinement_shell(FT size_shell, FT sizing, unsigned int max_vertices);}
{
}
\ccGlue
\ccMethod{void extrapolate_normals();}
{
Extrapolate the normals field: compute null normals by averaging neighbor normals.
}
\ccGlue
\ccMethod{FT gaussian_function(FT sigma, FT distance);}
{
}
\ccGlue
\ccMethod{int extrapolate_normals_using_gaussian_kernel();}
{
Extrapolate the normals field. Return the number of normals computed.
}
\ccGlue
\ccMethod{bool solve_poisson(double lambda, double * duration_assembly, double * duration_factorization, double * duration_solve, bool is_normalized = false);}
{
Poisson reconstruction. Return false on error.
}
\ccGlue
\ccMethod{void SaveAsMeshFile();}
{
}
\ccGlue
\ccMethod{FT set_contouring_value(FT contouring_value);}
{
Shift and orient the implicit function such that:\begin{itemize}
\item the implicit function = 0 for points / f() = \ccc{contouring_value},\item the implicit function $<$ 0 inside the surface.\end{itemize}
Return the minimum value of the implicit function.
}
\ccGlue
\ccMethod{FT f(const Point& p) const;}
{
Evaluate implicit function for any 3D point.
}
\ccGlue
\ccMethod{FT operator()(const Point& p) const;}
{
[ImplicitFunction interface]
Evaluate implicit function for any 3D point.
}
\ccGlue
\ccMethod{Point get_inner_point() const;}
{
Get point inside the surface.
}
\ccGlue
\ccMethod{FT average_value_at_input_vertices() const;}
{
Get average value of the implicit function over input vertices.
}
\ccGlue
\ccMethod{FT median_value_at_input_vertices() const;}
{
Get median value of the implicit function over input vertices.
}
\ccGlue
\ccMethod{FT min_value_at_input_vertices() const;}
{
Get min value of the implicit function over input vertices.
}
\ccGlue
\ccMethod{FT max_value_at_input_vertices() const;}
{
Get max value of the implicit function over input vertices.
}
\ccGlue
\ccMethod{FT median_value_at_convex_hull() const;}
{
Get median value of the implicit function over convex hull vertices.
}
\ccGlue
\ccMethod{FT average_value_at_convex_hull() const;}
{
Get average value of the implicit function over convex hull vertices.
}
\ccGlue

%END-AUTO(\ccOperations)

\ccSeeAlso

\ccRefIdfierPage{CGAL::Reconstruction_triangulation_3<GeomTraits, TriangulationDataStructure_3>}  \\
\ccRefIdfierPage{CGAL::Reconstruction_cell_base_3<GeomTraits, TriangulationCellBase_3>}  \\
\ccRefIdfierPage{CGAL::Reconstruction_vertex_base_3<GeomTraits, TriangulationVertexBase_3>}  \\
\ccRefIdfierPage{CGAL::APSS_reconstruction_function<GeomTraits>}  \\

\ccExample

\ccIncludeExampleCode{Surface_reconstruction_3/poisson_reconstruction_example.cpp}

\end{ccRefClass}

% +------------------------------------------------------------------------+
%%RefPage: end of main body, begin of footer
\ccRefPageEnd
% EOF
% +------------------------------------------------------------------------+

