% +------------------------------------------------------------------------+
% | Reference manual page: surface_reconstruction_read_xyz.tex
% +------------------------------------------------------------------------+
% | 07.01.2009   Pierre Alliez, Laurent Saboret, Gael Guennebaud
% | Package: Surface_reconstruction_3
% |
\RCSdef{\RCSsurfacereconstructionreadxyzRev}{$Id$}
\RCSdefDate{\RCSsurfacereconstructionreadxyzDate}{$Date$}
% |
\ccRefPageBegin
%%RefPage: end of header, begin of main body
% +------------------------------------------------------------------------+


\begin{ccRefFunction}{surface_reconstruction_read_xyz}  %% add template arg's if necessary

%% \ccHtmlCrossLink{}     %% add further rules for cross referencing links
%% \ccHtmlIndexC[function]{} %% add further index entries

\ccDefinition

\ccc{CGAL::surface_reconstruction_read_xyz()} reads points (positions + normals, if available) from a .xyz file (ASCII).

The \ccc{CGAL::surface_reconstruction_read_xyz()} function exists in two flavors: 
the output can be either a container of \ccc{Point_3} or of \ccc{PointWithNormal_3} points.

\ccInclude{CGAL/surface_reconstruction_read_xyz.h}

% The section below is automatically generated. Do not edit!
%START-AUTO(\ccDefinition)

\ccFunction{bool surface_reconstruction_read_xyz(const char * pFilename, OutputIterator output);}
{
Read points (positions + normals, if available) from a .xyz file (ASCII).
}
\ccGlue
\begin{description}
\item[Template Parameters:]
\begin{description}
\item[OutputIterator]\ccc{value_type} must be a model of the \ccc{PointWithNormal_3} concept.\end{description}
\end{description}
\begin{description}
\item[Returns:]true on success. \end{description}
\ccGlue
\ccFunction{bool surface_reconstruction_read_xyz(const char * pFilename, OutputIterator output, bool read_normals);}
{
Read points (positions only) from a .xyz file (ASCII). Normals are ignored.
}
\ccGlue
\begin{description}
\item[Template Parameters:]
\begin{description}
\item[OutputIterator]\ccc{value_type} must be \ccc{Point_3}.\end{description}
\end{description}
\begin{description}
\item[Returns:]true on success. \end{description}
\ccGlue

%END-AUTO(\ccDefinition)

\ccParameters

The full template declarations are:

% The section below is automatically generated. Do not edit!
%START-AUTO(\ccParameters)

template$<$  \\
typename OutputIterator$>$  \\
bool  \\
\ccc{surface_reconstruction_read_xyz} (const char $\ast$pFilename, OutputIterator output);  \\
  \\
template$<$  \\
typename OutputIterator$>$  \\
bool  \\
\ccc{surface_reconstruction_read_xyz} (const char $\ast$pFilename, OutputIterator output, bool \ccc{read_normals});  \\

%END-AUTO(\ccParameters)

\ccSeeAlso

\ccRefIdfierPage{surface_reconstruction_write_xyz}  \\
\ccRefIdfierPage{surface_reconstruction_read_off_point_cloud}  \\
\ccRefIdfierPage{surface_reconstruction_write_off_point_cloud}  \\

\ccExample

See \ccc{normal_estimation.cpp} example.

\end{ccRefFunction}

% +------------------------------------------------------------------------+
%%RefPage: end of main body, begin of footer
\ccRefPageEnd
% EOF
% +------------------------------------------------------------------------+

