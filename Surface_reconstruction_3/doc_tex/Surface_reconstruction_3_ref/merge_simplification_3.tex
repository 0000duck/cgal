% +------------------------------------------------------------------------+
% | Reference manual page: merge_simplification_3.tex
% +------------------------------------------------------------------------+
% | 02.06.2008   Pierre Alliez, Laurent Saboret, Gael Guennebaud
% | Package: Surface_reconstruction_3
% |
\RCSdef{\RCSmergesimplificationRev}{$Id$}
\RCSdefDate{\RCSmergesimplificationDate}{$Date$}
% |
\ccRefPageBegin
%%RefPage: end of header, begin of main body
% +------------------------------------------------------------------------+


\begin{ccRefFunction}{merge_simplification_3}  %% add template arg's if necessary

%% \ccHtmlCrossLink{}     %% add further rules for cross referencing links
%% \ccHtmlIndexC[function]{} %% add further index entries

\ccDefinition

\ccc{CGAL::merge_simplification_3()} iteratively merges pairs of closest points which are epsilon closed. The function exists in four flavors: it may either mutate the input points or create a copy, and may either require the kernel or deduce it from the input data.

\ccInclude{CGAL/merge_simplification_3.h}

% The section below is automatically generated. Do not edit!
%START-AUTO(\ccDefinition)

\ccFunction{template<typename InputIterator, typename OutputIterator, typename Kernel> OutputIterator merge_simplification_3(InputIterator first, InputIterator beyond, OutputIterator output, double epsilon, const Kernel& );}
{
Merge points which belong to the same cell of a grid of cell size = epsilon. This variant requires the kernel.
\ccPrecond epsilon $>$ 0.
\ccCommentHeading{Template Parameters}
\begin{description}
\item \ccc{InputIterator}: \ccc{value_type} must be convertible to OutputIterator's \ccc{value_type}. \item \ccc{OutputIterator}: \ccc{value_type} must be convertible to \ccc{Point_3}. \item \ccc{Kernel}: Geometric traits class.\end{description}
\ccCommentHeading{Returns} past-the-end output iterator.
\ccCommentHeading{Parameters}
\begin{description}
\item \ccc{first}: iterator over the first input point \item \ccc{beyond}: past-the-end iterator over input points \item \ccc{output}: iterator over the first output point \item \ccc{epsilon}: tolerance value when comparing 3D points \end{description}
}
\ccGlue
\ccFunction{template<typename ForwardIterator, typename Kernel> ForwardIterator merge_simplification_3(ForwardIterator first, ForwardIterator beyond, double epsilon, const Kernel& );}
{
Merge points which belong to the same cell of a grid of cell size = epsilon. This function is mutating the input point set. This variant requires the kernel.
Warning: This method modifies the order of points, thus should not be called on sorted containers.
\ccPrecond epsilon $>$ 0.
\ccCommentHeading{Template Parameters}
\begin{description}
\item \ccc{ForwardIterator}: \ccc{value_type} must be convertible to \ccc{Point_3}. \item \ccc{Kernel}: Geometric traits class.\end{description}
\ccCommentHeading{Returns} First iterator to remove (see erase-remove idiom).
\ccCommentHeading{Parameters}
\begin{description}
\item \ccc{first}: iterator over the first input/output point \item \ccc{beyond}: past-the-end iterator \item \ccc{epsilon}: tolerance value when comparing 3D points \end{description}
}
\ccGlue
\ccFunction{template<typename InputIterator, typename OutputIterator> OutputIterator merge_simplification_3(InputIterator first, InputIterator beyond, OutputIterator output, double epsilon);}
{
Merge points which belong to the same cell of a grid of cell size = epsilon. This variant deduces the kernel from iterator types.
\ccPrecond epsilon $>$ 0.
\ccCommentHeading{Template Parameters}
\begin{description}
\item \ccc{InputIterator}: \ccc{value_type} must be convertible to OutputIterator's \ccc{value_type}. \item \ccc{OutputIterator}: \ccc{value_type} must be convertible to \ccc{Point_3}.\end{description}
\ccCommentHeading{Returns} past-the-end output iterator.
\ccCommentHeading{Parameters}
\begin{description}
\item \ccc{first}: iterator over the first input point \item \ccc{beyond}: past-the-end iterator over input points \item \ccc{output}: iterator over the first output point \item \ccc{epsilon}: tolerance value when comparing 3D points \end{description}
}
\ccGlue
\ccFunction{template<typename ForwardIterator> ForwardIterator merge_simplification_3(ForwardIterator first, ForwardIterator beyond, double epsilon);}
{
Merge points which belong to the same cell of a grid of cell size = epsilon. This function is mutating the input point set. This variant deduces the kernel from iterator types.
Warning: This method modifies the order of points, thus should not be called on sorted containers.
\ccPrecond epsilon $>$ 0.
\ccCommentHeading{Template Parameters}
\ccc{ForwardIterator}: \ccc{value_type} must be convertible to \ccc{Point_3}.
\ccCommentHeading{Returns} First iterator to remove (see erase-remove idiom).
\ccCommentHeading{Parameters}
\begin{description}
\item \ccc{first}: iterator over the first input/output point \item \ccc{beyond}: past-the-end iterator \item \ccc{epsilon}: tolerance value when comparing 3D points \end{description}
}
\ccGlue

%END-AUTO(\ccDefinition)

\ccSeeAlso

\ccRefIdfierPage{CGAL::random_simplification_3}  \\

\ccExample

\begin{ccExampleCode}
typedef CGAL::Exact_predicates_inexact_constructions_kernel Kernel;
typedef Kernel::Point_3 Point;
std::deque<Point> points = ...;
double epsilon = 0.001;

// put result in output iterator...
std::deque<Point> output;
CGAL::merge_simplification_3(points.begin(), points.end(),
                                     std::back_inserter(output),
                                     epsilon);

// ...or use mutating version of the same function
std::deque<Point>::iterator first_iterator_to_remove =
  CGAL::merge_simplification_3(points.begin(), points.end(),
                                       epsilon);
std::erase(std::remove(first_iterator_to_remove, points.end()), // erase-remove idiom
           points.end());
\end{ccExampleCode}

\end{ccRefFunction}

% +------------------------------------------------------------------------+
%%RefPage: end of main body, begin of footer
\ccRefPageEnd
% EOF
% +------------------------------------------------------------------------+

