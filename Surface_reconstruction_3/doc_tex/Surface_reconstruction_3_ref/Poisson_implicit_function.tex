% +------------------------------------------------------------------------+
% | Reference manual page: Poisson_implicit_function.tex
% +------------------------------------------------------------------------+
% | 07.09.2007   Laurent Saboret, Pierre Alliez
% | Package: Surface_reconstruction_3
% |
\RCSdef{\RCSPoissonimplicitfunctionRev}{$Id$}
\RCSdefDate{\RCSPoissonimplicitfunctionDate}{$Date$}
% |
\ccRefPageBegin
%%RefPage: end of header, begin of main body
% +------------------------------------------------------------------------+


\begin{ccRefClass}{Poisson_implicit_function<ImplicitFctDelaunayTriangulation_3>}

%% \ccHtmlCrossLink{}     %% add further rules for cross referencing links
%% \ccHtmlIndexC[class]{} %% add further index entries

\ccDefinition

% The section below is automatically generated. Do not edit!
%START-AUTO(\ccDefinition)

\ccc{Poisson_implicit_function} computes an indicator function f() piecewise-linear over the tetrahedra. We solve the Poisson equation Laplacian(f) = divergent(normals field) at each vertex of the triangulation via the TAUCS sparse linear solver. One vertex must be constrained.

%END-AUTO(\ccDefinition)

\ccInclude{CGAL/Poisson_implicit_function.h}

\ccParameters

The full template declaration is:

% The section below is automatically generated. Do not edit!
%START-AUTO(\ccParameters)

template$<$  \\
class Gt,   \\
class \ccc{ImplicitFctDelaunayTriangulation_3}$>$   \\
class \ccc{Poisson_implicit_function};

\begin{description}
\item[Parameters:]
\begin{description}
\item[\ccc{ImplicitFctDelaunayTriangulation_3}]3D Delaunay triangulation, model of \ccc{ImplicitFctDelaunayTriangulation_3} concept. \end{description}
\end{description}

%END-AUTO(\ccParameters)

\ccIsModel

% The section below is automatically generated. Do not edit!
%START-AUTO(\ccIsModel)

Model of the \ccc{Reconstruction_implicit_function} concept.

%END-AUTO(\ccIsModel)

\ccHeading{Design Pattern}

% The section below is automatically generated. Do not edit!
%START-AUTO(\ccHeading{Design Pattern})

\ccc{Poisson_implicit_function} is a Strategy \cite{cgal:ghjv-dpero-95}: it implements a strategy of surface mesh reconstruction.

%END-AUTO(\ccHeading{Design Pattern})

\ccTypes

% The section below is automatically generated. Do not edit!
%START-AUTO(\ccTypes)

\ccNestedType{Triangulation}
{
}
\ccGlue
\ccNestedType{Geom_traits}
{
Kernel's geometric traits.
}
\ccGlue
\ccNestedType{FT}
{
}
\ccGlue
\ccNestedType{Point}
{
}
\ccGlue
\ccNestedType{Iso_cuboid_3}
{
}
\ccGlue
\ccNestedType{Sphere}
{
}
\ccGlue
\ccNestedType{Point_with_normal}
{
Model of \ccc{PointWithNormal_3}.
}
\ccGlue
\ccNestedType{Normal}
{
Model of \ccc{OrientedNormal_3} concept.
}
\ccGlue

%END-AUTO(\ccTypes)

\ccCreation
\ccCreationVariable{pif}  %% choose variable name for \ccMethod

% The section below is automatically generated. Do not edit!
%START-AUTO(\ccCreation)

\ccConstructor{Poisson_implicit_function(Triangulation& pdt);}
{
Create a Poisson indicator function f() piecewise-linear over the tetrahedra of pdt.
}
\ccGlue
\begin{description}
\item[Parameters:]
\begin{description}
\item[pdt]\ccc{ImplicitFctDelaunayTriangulation_3} base of the Poisson indicator function. \end{description}
\end{description}
\ccGlue

%END-AUTO(\ccCreation)

\ccOperations

% The section below is automatically generated. Do not edit!
%START-AUTO(\ccOperations)

\ccMethod{Triangulation& triangulation();}
{
Get embedded triangulation.
}
\ccGlue
\ccMethod{const Triangulation& triangulation() const;}
{
}
\ccGlue
\ccMethod{Iso_cuboid_3 bounding_box() const;}
{
Get the bounding box.
}
\ccGlue
\ccMethod{Sphere bounding_sphere() const;}
{
Get bounding sphere.
}
\ccGlue
\ccMethod{Sphere region_of_interest() const;}
{
Get the region of interest, ignoring the outliers. This method is used to define the OpenGL arcball sphere.
}
\ccGlue
\ccMethod{bool compute_implicit_function();}
{
You should call \ccc{compute_implicit_function}() once when points insertion is over. It computes the Poisson indicator function f() at each vertex of the triangulation by:\begin{itemize}
\item applying a Delaunay refinement to define the function inside and outside the surface.\item solving the Poisson equation Laplacian(f) = divergent(normals field) at each vertex of the triangulation via the TAUCS sparse linear solver. One vertex must be constrained.\item shifting and orienting f() such that f() = 0 on the input points, and f() $<$ 0 inside the surface.\end{itemize}
Return false on error. TODO: add parameters to \ccc{compute_implicit_function}()?
}
\ccGlue
\ccMethod{void estimate_normals_pca(unsigned int k);}
{
Estimate normal directions using linear least squares fitting of a plane on the k nearest neighbors.
}
\ccGlue
\ccMethod{void estimate_normals_jet_fitting(unsigned int k);}
{
Estimate normal directions using jet fitting on the k nearest neighbors.
}
\ccGlue
\ccMethod{unsigned int delaunay_refinement(const FT threshold, const unsigned int maximum, const FT enlarge_ratio, const unsigned int restart_each);}
{
Delaunay refinement (break bad tetrahedra, where bad means badly shaped or too big). The normal of Steiner points is set to zero. Return the number of vertices inserted.
}
\ccGlue
\ccMethod{unsigned int delaunay_refinement_shell(FT size_shell, FT sizing, const unsigned int maximum);}
{
}
\ccGlue
\ccMethod{void extrapolate_normals();}
{
Extrapolate the normals field: compute null normals by averaging neighbour normals.
}
\ccGlue
\ccMethod{bool solve_poisson(double * duration_assembly, double * duration_factorization, double * duration_solve);}
{
Poisson reconstruction. Return false on error.
}
\ccGlue
\ccMethod{FT set_contouring_value(FT contouring_value);}
{
Shift and orient the implicit function such that:\begin{itemize}
\item the implicit function = 0 for points / f() = \ccc{contouring_value},\item the implicit function $<$ 0 inside the surface.\end{itemize}
Return the minimum value of the implicit function.
}
\ccGlue
\ccMethod{FT f(const Point& p);}
{
Evaluate implicit function for any 3D point.
}
\ccGlue
\ccMethod{FT operator()(Point p);}
{
[ImplicitFunction interface]
Evaluate implicit function for any 3D point.
}
\ccGlue
\ccMethod{const Point& sink() const;}
{
Get point / the implicit function is minimum.
}
\ccGlue
\ccMethod{FT average_value_at_input_vertices() const;}
{
Get average value of the implicit function over input vertices.
}
\ccGlue
\ccMethod{FT median_value_at_input_vertices() const;}
{
Get median value of the implicit function over input vertices.
}
\ccGlue
\ccMethod{FT min_value_at_input_vertices() const;}
{
Get min value of the implicit function over input vertices.
}
\ccGlue
\ccMethod{FT max_value_at_input_vertices() const;}
{
Get max value of the implicit function over input vertices.
}
\ccGlue
\ccMethod{FT median_value_at_convex_hull() const;}
{
Get median value of the implicit function over convex hull vertices.
}
\ccGlue
\ccMethod{FT average_value_at_convex_hull() const;}
{
Get average value of the implicit function over convex hull vertices.
}
\ccGlue

%END-AUTO(\ccOperations)

\end{ccRefClass}

% +------------------------------------------------------------------------+
%%RefPage: end of main body, begin of footer
\ccRefPageEnd
% EOF
% +------------------------------------------------------------------------+

