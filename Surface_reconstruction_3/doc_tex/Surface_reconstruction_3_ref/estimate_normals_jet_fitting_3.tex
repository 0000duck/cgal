% +------------------------------------------------------------------------+
% | Reference manual page: estimate_normals_jet_fitting_3.tex
% +------------------------------------------------------------------------+
% | 19.02.2008   Pierre Alliez, Laurent Saboret, Gael Guennebaud
% | Package: Surface_reconstruction_3
% |
\RCSdef{\RCSestimatenormalsjetfittingRev}{$Id$}
\RCSdefDate{\RCSestimatenormalsjetfittingDate}{$Date$}
% |
\ccRefPageBegin
%%RefPage: end of header, begin of main body
% +------------------------------------------------------------------------+


\begin{ccRefFunction}{estimate_normals_jet_fitting_3}  %% add template arg's if necessary

%% \ccHtmlCrossLink{}     %% add further rules for cross referencing links
%% \ccHtmlIndexC[function]{} %% add further index entries

\ccDefinition

\ccc{CGAL::estimate_normals_jet_fitting_3()} estimates normals direction of a point set using jet fitting on the K nearest neighbors.
The default jet is a quadric.
The result is an unoriented normal vector for each input point.

The \ccc{CGAL::estimate_normals_jet_fitting_3()} function exists in three flavors. 
The complete version applies to a point set and requires the kernel to use for computations. 
The main version applies to a point set and deduces the kernel from input parameters.
The low level call applies to a single point.

\ccInclude{CGAL/estimate_normals_jet_fitting_3.h}

% The section below is automatically generated. Do not edit!
%START-AUTO(\ccDefinition)

\ccFunction{OrientableNormal_3 estimate_normals_jet_fitting_3(const typename Kernel::Point_3& query, Tree& tree, unsigned int KNN, unsigned int degre_fitting);}
{
Estimate normal direction using jet fitting on the K nearest neighbors.
Precondition: KNN $>$= 2.
}
\ccGlue
\begin{description}
\item[Template Parameters:]
\begin{description}
\item[Kernel]Geometric traits class. \item[Tree]KD-tree. \item[\ccc{OrientableNormal_3}]Type of return value.\end{description}
\end{description}
\begin{description}
\item[Returns:]Computed normal, model of \ccc{OrientableNormal_3}. \end{description}
\begin{description}
\item[Parameters: ]
\begin{description}
\item[query]3D point whose normal we want to compute \item[tree]KD-tree \end{description}
\end{description}
\ccGlue
\ccFunction{OutputIterator estimate_normals_jet_fitting_3(InputIterator first, InputIterator beyond, OutputIterator normals, unsigned int KNN, const Kernel& , unsigned int degre_fitting);}
{
Estimate normal directions using jet fitting on the KNN nearest neighbors. This variant requires the kernel.
Precondition: KNN $>$= 2.
}
\ccGlue
\begin{description}
\item[Template Parameters:]
\begin{description}
\item[InputIterator]\ccc{value_type} is \ccc{Point_3}. \item[OutputIterator]\ccc{value_type} is \ccc{Point_3}. \item[Kernel]Geometric traits class.\end{description}
\end{description}
\begin{description}
\item[Returns:]past-the-end output iterator. \end{description}
\begin{description}
\item[Parameters: ]
\begin{description}
\item[first]input points \item[normals]output normals \item[KNN]number of neighbors \end{description}
\end{description}
\ccGlue
\ccFunction{OutputIterator estimate_normals_jet_fitting_3(InputIterator first, InputIterator beyond, OutputIterator normals, unsigned int KNN, unsigned int degre_fitting = 2);}
{
Estimate normal directions using jet fitting on the KNN nearest neighbors. This variant deduces the kernel from iterator types.
Precondition: KNN $>$= 2.
}
\ccGlue
\begin{description}
\item[Template Parameters:]
\begin{description}
\item[InputIterator]\ccc{value_type} is \ccc{Point_3}. \item[OutputIterator]\ccc{value_type} is \ccc{Point_3}.\end{description}
\end{description}
\begin{description}
\item[Returns:]past-the-end output iterator. \end{description}
\begin{description}
\item[Parameters: ]
\begin{description}
\item[first]input points \item[normals]output normals \item[KNN]number of neighbors \end{description}
\end{description}
\ccGlue

%END-AUTO(\ccDefinition)

\ccParameters

The full template declarations are:

% The section below is automatically generated. Do not edit!
%START-AUTO(\ccParameters)

template$<$  \\
typename Kernel,   \\
typename Tree,   \\
typename \ccc{OrientableNormal_3}$>$  \\
\ccc{OrientableNormal_3}  \\
\ccc{estimate_normals_jet_fitting_3} (const typename \ccc{Kernel::Point_3}\& query, Tree\& tree, unsigned int KNN, unsigned int \ccc{degre_fitting});  \\
  \\
template$<$  \\
typename InputIterator,   \\
typename OutputIterator,   \\
typename Kernel$>$  \\
OutputIterator  \\
\ccc{estimate_normals_jet_fitting_3} (InputIterator first, InputIterator beyond, OutputIterator normals, unsigned int KNN, const Kernel\& , unsigned int \ccc{degre_fitting});  \\
  \\
template$<$  \\
typename InputIterator,   \\
typename OutputIterator$>$  \\
OutputIterator  \\
\ccc{estimate_normals_jet_fitting_3} (InputIterator first, InputIterator beyond, OutputIterator normals, unsigned int KNN, unsigned int \ccc{degre_fitting}=2);  \\

%END-AUTO(\ccParameters)

\ccSeeAlso

\ccRefIdfierPage{CGAL::estimate_normals_pca_3}  \\
\ccRefIdfierPage{CGAL::orient_normals_minimum_spanning_tree_3}  \\

\ccExample

See \ccc{normal_estimation.cpp} example.

\end{ccRefFunction}

% +------------------------------------------------------------------------+
%%RefPage: end of main body, begin of footer
\ccRefPageEnd
% EOF
% +------------------------------------------------------------------------+

