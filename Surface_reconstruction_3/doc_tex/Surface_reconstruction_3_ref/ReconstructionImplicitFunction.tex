% +------------------------------------------------------------------------+
% | Reference manual page: ReconstructionImplicitFunction.tex
% +------------------------------------------------------------------------+
% | 07.09.2007   Laurent Saboret, Pierre Alliez
% | Package: Surface_reconstruction_3
% |
\RCSdef{\RCSReconstructionImplicitFunctionRev}{$Id$}
\RCSdefDate{\RCSReconstructionImplicitFunctionDate}{$Date$}
% |
\ccRefPageBegin
%%RefPage: end of header, begin of main body
% +------------------------------------------------------------------------+


\begin{ccRefConcept}{ReconstructionImplicitFunction}

%% \ccHtmlCrossLink{}     %% add further rules for cross referencing links
%% \ccHtmlIndexC[concept]{} %% add further index entries

\ccDefinition

% The section below is automatically generated. Do not edit!
%START-AUTO(\ccDefinition)

ReconstructionImplicitFunction is the concept of implicit function representing a 3D surface used by the \ccc{Surface_reconstruction_3} package. The surface is the zero-level of the implicit function.

%END-AUTO(\ccDefinition)

\ccRefines

% The section below is automatically generated. Do not edit!
%START-AUTO(\ccRefines)

\ccc{Surface_mesher::ImplicitFunction}

%END-AUTO(\ccRefines)

\ccHeading{Design Pattern}

% The section below is automatically generated. Do not edit!
%START-AUTO(\ccHeading{Design Pattern})

A model of ReconstructionImplicitFunction is a Strategy \cite{cgal:ghjv-dpero-95}: it implements a strategy of surface mesh reconstruction.

%END-AUTO(\ccHeading{Design Pattern})

\ccTypes

% The section below is automatically generated. Do not edit!
%START-AUTO(\ccTypes)

\ccNestedType{Geom_traits}
{
Kernel's geometric traits.
}
\ccGlue
\ccNestedType{FT}
{
}
\ccGlue
\ccNestedType{Point}
{
}
\ccGlue
\ccNestedType{Iso_cuboid}
{
}
\ccGlue
\ccNestedType{Sphere}
{
}
\ccGlue
\ccNestedType{Point_with_normal}
{
Model of \ccc{PointWithNormal_3}.
}
\ccGlue
\ccNestedType{Normal}
{
Model of \ccc{Kernel::Vector_3} concept.
}
\ccGlue
\ccNestedType{Vector}
{
}
\ccGlue

%END-AUTO(\ccTypes)

\ccCreation
\ccCreationVariable{fct}  %% variable name for \ccMethod below

% The section below is automatically generated. Do not edit!
%START-AUTO(\ccCreation)

\ccConstructor{ReconstructionImplicitFunction(InputIterator first, InputIterator beyond);}
{
Create an implicit function from a point set.
Precondition: the value type of InputIterator must be convertible to \ccc{Point_with_normal}.
}
\ccGlue
\begin{description}
\item[Parameters:]
\begin{description}
\item[first]First point to add. \item[beyond]Past-the-end point to add. \end{description}
\end{description}
\ccGlue

%END-AUTO(\ccCreation)

\ccOperations

% The section below is automatically generated. Do not edit!
%START-AUTO(\ccOperations)

\ccMethod{Sphere bounding_sphere() const;}
{
Get the surface's bounding sphere.
}
\ccGlue
\ccMethod{Sphere region_of_interest() const;}
{
Get the region of interest, ignoring the outliers. This method is used to define the OpenGL arcball sphere.
}
\ccGlue
\ccMethod{bool compute_implicit_function();}
{
You should call \ccc{compute_implicit_function}() once when points insertion is over. Return false on error.
}
\ccGlue
\ccMethod{FT operator()(const Point& p) const;}
{
[ImplicitFunction interface]
Evaluate implicit function for any 3D point.
}
\ccGlue
\ccMethod{Point get_inner_point() const;}
{
Get point inside the surface.
}
\ccGlue

%END-AUTO(\ccOperations)

\ccHasModels

% The section below is automatically generated. Do not edit!
%START-AUTO(\ccHasModels)

\ccc{Poisson_implicit_function<PoissonDelaunayTriangulation_3>}

%END-AUTO(\ccHasModels)

\end{ccRefConcept}

% +------------------------------------------------------------------------+
%%RefPage: end of main body, begin of footer
\ccRefPageEnd
% EOF
% +------------------------------------------------------------------------+

