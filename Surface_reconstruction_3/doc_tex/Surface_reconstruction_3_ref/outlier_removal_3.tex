% +------------------------------------------------------------------------+
% | Reference manual page: outlier_removal_3.tex
% +------------------------------------------------------------------------+
% | 02.06.2008   Pierre Alliez, Laurent Saboret, Gael Guennebaud
% | Package: Surface_reconstruction_3
% |
\RCSdef{\RCSoutlierremovalRev}{$Id$}
\RCSdefDate{\RCSoutlierremovalDate}{$Date$}
% |
\ccRefPageBegin
%%RefPage: end of header, begin of main body
% +------------------------------------------------------------------------+


\begin{ccRefFunction}{outlier_removal_3}  %% add template arg's if necessary

%% \ccHtmlCrossLink{}     %% add further rules for cross referencing links
%% \ccHtmlIndexC[function]{} %% add further index entries

\ccDefinition

\ccc{CGAL::outlier_removal_3()} deletes outliers in a point set. It sorts points
wrt average squared distance to the K nearest neighbors, then delete the worst ones.

The \ccc{CGAL::outlier_removal_3()} function exists in four flavors.
First, the function may modify the input point set or create a copy.
Second, the function may require the kernel to use for computations, or deduce it from input parameters.

\ccInclude{CGAL/outlier_removal_3.h}

% The section below is automatically generated. Do not edit!
%START-AUTO(\ccDefinition)

\ccFunction{template<typename InputIterator, typename OutputIterator, typename Kernel> OutputIterator outlier_removal_3(InputIterator first, InputIterator beyond, OutputIterator output, unsigned int KNN, const Kernel& , double threshold_percent);}
{
Remove outliers:\begin{itemize}
\item compute average squared distance to the K nearest neighbors,\item output (100-\ccc{threshold_percent}) \% best points wrt this distance. This variant requires the kernel.\end{itemize}
\ccPrecond KNN $>$= 2.
\ccCommentHeading{Template Parameters}
\begin{description}
\item \ccc{InputIterator}: \ccc{value_type} must be convertible to OutputIterator's \ccc{value_type}. \item \ccc{OutputIterator}: \ccc{value_type} must be convertible to \ccc{Point_3}. \item \ccc{Kernel}: Geometric traits class.\end{description}
\ccCommentHeading{Returns} past-the-end output iterator.
\ccCommentHeading{Parameters}
\begin{description}
\item \ccc{first}: input points \item \ccc{output}: output points \item \ccc{KNN}: number of neighbors \item \ccc{threshold_percent}: percentage of points to remove \end{description}
}
\ccGlue
\ccFunction{template<typename ForwardIterator, typename Kernel> ForwardIterator outlier_removal_3(ForwardIterator first, ForwardIterator beyond, unsigned int KNN, const Kernel& , double threshold_percent);}
{
Remove outliers:\begin{itemize}
\item compute average squared distance to the K nearest neighbors,\item sort the points wrt this distance. This function is mutating the input point set. This variant requires the kernel.\end{itemize}
Warning: This method modifies the order of points, thus should not be called on sorted containers.
\ccPrecond KNN $>$= 2.
\ccCommentHeading{Template Parameters}
\begin{description}
\item \ccc{ForwardIterator}: \ccc{value_type} must be convertible to \ccc{Point_3}. \item \ccc{Kernel}: Geometric traits class.\end{description}
\ccCommentHeading{Returns} First iterator to remove (see erase-remove idiom).
\ccCommentHeading{Parameters}
\begin{description}
\item \ccc{first}: input/output points \item \ccc{KNN}: number of neighbors \item \ccc{threshold_percent}: percentage of points to remove \end{description}
}
\ccGlue
\ccFunction{template<typename InputIterator, typename OutputIterator> OutputIterator outlier_removal_3(InputIterator first, InputIterator beyond, OutputIterator output, unsigned int KNN, double threshold_percent);}
{
Remove outliers:\begin{itemize}
\item compute average squared distance to the K nearest neighbors,\item output (100-\ccc{threshold_percent}) \% best points wrt this distance. This variant deduces the kernel from iterator types.\end{itemize}
\ccPrecond KNN $>$= 2.
\ccCommentHeading{Template Parameters}
\begin{description}
\item \ccc{InputIterator}: \ccc{value_type} must be convertible to OutputIterator's \ccc{value_type}. \item \ccc{OutputIterator}: \ccc{value_type} must be convertible to \ccc{Point_3}.\end{description}
\ccCommentHeading{Returns} past-the-end output iterator.
\ccCommentHeading{Parameters}
\begin{description}
\item \ccc{first}: input points \item \ccc{output}: output points \item \ccc{KNN}: number of neighbors \item \ccc{threshold_percent}: percentage of points to remove \end{description}
}
\ccGlue
\ccFunction{template<typename ForwardIterator> ForwardIterator outlier_removal_3(ForwardIterator first, ForwardIterator beyond, unsigned int KNN, double threshold_percent);}
{
Remove outliers:\begin{itemize}
\item compute average squared distance to the K nearest neighbors,\item sort the points wrt this distance. This function is mutating the input point set. This variant deduces the kernel from iterator types.\end{itemize}
Warning: This method modifies the order of points, thus should not be called on sorted containers.
\ccPrecond KNN $>$= 2.
\ccCommentHeading{Template Parameters}
\ccc{ForwardIterator}: \ccc{value_type} must be convertible to \ccc{Point_3}.
\ccCommentHeading{Returns} First iterator to remove (see erase-remove idiom).
\ccCommentHeading{Parameters}
\begin{description}
\item \ccc{first}: input/output points \item \ccc{KNN}: number of neighbors \item \ccc{threshold_percent}: percentage of points to remove \end{description}
}
\ccGlue

%END-AUTO(\ccDefinition)

\ccExample

\ccIncludeExampleCode{Surface_reconstruction_3/outlier_removal_example.cpp}

\end{ccRefFunction}

% +------------------------------------------------------------------------+
%%RefPage: end of main body, begin of footer
\ccRefPageEnd
% EOF
% +------------------------------------------------------------------------+

