% +------------------------------------------------------------------------+
% | Reference manual page: ImplicitFctDelaunayTriangulation_3.tex
% +------------------------------------------------------------------------+
% | 22.02.2008   Laurent Saboret, Pierre Alliez
% | Package: Package
% |
\RCSdef{\RCSImplicitFctDelaunayTriangulationRev}{$Id$}
\RCSdefDate{\RCSImplicitFctDelaunayTriangulationDate}{$Date$}
% |
\ccRefPageBegin
%%RefPage: end of header, begin of main body
% +------------------------------------------------------------------------+


\begin{ccRefConcept}{ImplicitFctDelaunayTriangulation_3}

%% \ccHtmlCrossLink{}     %% add further rules for cross referencing links
%% \ccHtmlIndexC[concept]{} %% add further index entries

\ccDefinition

% The section below is automatically generated. Do not edit!
%START-AUTO(\ccDefinition)

The \ccc{ImplicitFctDelaunayTriangulation_3} concept defines the interface of a 3D Delaunay triangulation requested by the \ccc{Poisson_implicit_function} class. The cell base class must be a model of \ccc{ImplicitFctDelaunayTriangulationCellBase_3} and the vertex base class must be a model of \ccc{ImplicitFctDelaunayTriangulationVertexBase_3}.

%END-AUTO(\ccDefinition)
  
\ccRefines

% The section below is automatically generated. Do not edit!
%START-AUTO(\ccRefines)

\ccc{DelaunayTriangulation_3}, DefaultConstructible, CopyConstructible, and Assignable

%END-AUTO(\ccRefines)

\ccTypes

% The section below is automatically generated. Do not edit!
%START-AUTO(\ccTypes)

\ccNestedType{FT}
{
}
\ccGlue
\ccNestedType{Vector}
{
}
\ccGlue
\ccNestedType{Iso_cuboid}
{
}
\ccGlue
\ccNestedType{Sphere}
{
}
\ccGlue
\ccNestedType{Point}
{
The geometric traits class's \ccc{Point_3} type is a model of \ccc{PointWithNormal_3}.
Model of \ccc{PointWithNormal_3}
}
\ccGlue
\ccNestedType{Point_with_normal}
{
Model of \ccc{PointWithNormal_3}.
}
\ccGlue
\ccNestedType{Normal}
{
Model of \ccc{OrientedNormal_3} concept.
}
\ccGlue
\ccNestedType{Normal_iterator}
{
Iterator over normals.
}
\ccGlue

%END-AUTO(\ccTypes)

\ccHeading{Variables}

% The section below is automatically generated. Do not edit!
%START-AUTO(\ccHeading{Variables})

\ccVariable{const unsigned char INPUT = 0;}
{
[static] \\
Point type.
}
\ccGlue
\ccVariable{const unsigned char STEINER = 1;}
{
[static] \\
}
\ccGlue

%END-AUTO(\ccHeading{Variables})

\ccCreation
\ccCreationVariable{pdt}  %% variable name for \ccMethod calls below

\ccOperations

% The section below is automatically generated. Do not edit!
%START-AUTO(\ccOperations)

\ccMethod{Normal_iterator normals_begin();}
{
Get first iterator over finite vertices normals.
}
\ccGlue
\ccMethod{Normal_iterator normals_end();}
{
Get past-the-end iterator over finite vertices normals.
}
\ccGlue
\ccMethod{Iso_cuboid bounding_box() const;}
{
Get the bounding box.
}
\ccGlue
\ccMethod{Sphere bounding_sphere() const;}
{
Get bounding sphere.
}
\ccGlue
\ccMethod{Point barycenter() const;}
{
Get points barycenter.
}
\ccGlue
\ccMethod{FT diameter_standard_deviation() const;}
{
Get the standard deviation of the distance to barycenter.
}
\ccGlue
\ccMethod{void invalidate_bounding_box();}
{
Update barycenter, bounding box, bounding sphere and standard deviation. Owner is responsible to call this function after modifying the triangulation.
}
\ccGlue
\ccMethod{Vertex_handle insert(const Point& p, unsigned char type = INPUT, Cell_handle start = Cell_handle());}
{
Insert point to the triangulation.
}
\ccGlue
\ccMethod{int insert(InputIterator first, InputIterator beyond, unsigned char type = INPUT);}
{
Insert points to the triangulation using a spatial sort.
Precondition: the value type of InputIterator must \ccc{Point}.
}
\ccGlue
\begin{description}
\item[Parameters:]
\begin{description}
\item[first]First point to add to pdt. \item[beyond]Past-the-end point to add to pdt. \end{description}
\end{description}
\begin{description}
\item[Returns:]the number of inserted points. \end{description}
\ccGlue
\ccMethod{unsigned int index_vertices();}
{
Index all (finite) vertices following the order of \ccc{Finite_vertices_iterator}.
}
\ccGlue
\begin{description}
\item[Returns:]the number of (finite) vertices. \end{description}
\ccGlue
\ccMethod{unsigned int index_unconstrained_vertices();}
{
Index unconstraint vertices following the order of \ccc{Finite_vertices_iterator}.
}
\ccGlue
\begin{description}
\item[Returns:]the number of unconstraint vertices. \end{description}
\ccGlue

%END-AUTO(\ccOperations)

\ccHasModels

% The section below is automatically generated. Do not edit!
%START-AUTO(\ccHasModels)

\ccc{Implicit_fct_delaunay_triangulation_3<GeomTraits, TriangulationDataStructure_3>}

%END-AUTO(\ccHasModels)


\end{ccRefConcept}

% +------------------------------------------------------------------------+
%%RefPage: end of main body, begin of footer
\ccRefPageEnd
% EOF
% +------------------------------------------------------------------------+

