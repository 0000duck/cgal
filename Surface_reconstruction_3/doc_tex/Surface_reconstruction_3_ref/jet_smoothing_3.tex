% +------------------------------------------------------------------------+
% | Reference manual page: jet_smoothing_3.tex
% +------------------------------------------------------------------------+
% | 02.06.2008   Pierre Alliez, Laurent Saboret, Gael Guennebaud
% | Package: Surface_reconstruction_3
% |
\RCSdef{\RCSjetsmoothingRev}{$Id$}
\RCSdefDate{\RCSjetsmoothingDate}{$Date$}
% |
\ccRefPageBegin
%%RefPage: end of header, begin of main body
% +------------------------------------------------------------------------+


\begin{ccRefFunction}{jet_smoothing_3}  %% add template arg's if necessary

%% \ccHtmlCrossLink{}     %% add further rules for cross referencing links
%% \ccHtmlIndexC[function]{} %% add further index entries

\ccDefinition

\ccc{CGAL::jet_smoothing_3()} smooths a point set using jet fitting on the K nearest neighbors and reprojection onto the jet.
The default jet is a quadric.

The \ccc{CGAL::jet_smoothing_3()} function exists in four flavors: first, the function may modify the input point set or
create a copy; second, the function may require the kernel to use for computations, or deduce it from input parameters.

\ccInclude{CGAL/jet_smoothing_3.h}

% The section below is automatically generated. Do not edit!
%START-AUTO(\ccDefinition)

\ccFunction{OutputIterator jet_smoothing_3(InputIterator first, InputIterator beyond, OutputIterator output, const unsigned int KNN, const Kernel& , const unsigned int degre_fitting = 2, const unsigned int degree_monge = 2);}
{
Smooth a point set using jet fitting on the KNN nearest neighbors and reprojection onto the jet. This variant requires the kernel.
Precondition: KNN $>$= 2.
}
\ccGlue
\begin{description}
\item[Template Parameters:]
\begin{description}
\item[InputIterator]\ccc{value_type} convertible to \ccc{Point_3}. \item[OutputIterator]\ccc{value_type} convertible to \ccc{Point_3}. \item[Kernel]Geometric traits class.\end{description}
\end{description}
\begin{description}
\item[Returns:]past-the-end output iterator. \end{description}
\begin{description}
\item[Parameters: ]
\begin{description}
\item[first]input points \item[output]output points \item[KNN]number of neighbors \end{description}
\end{description}
\ccGlue
\ccFunction{void jet_smoothing_3(ForwardIterator first, ForwardIterator beyond, unsigned int KNN, const Kernel& , const unsigned int degre_fitting = 2, const unsigned int degree_monge = 2);}
{
Smooth a point set using jet fitting on the KNN nearest neighbors and reprojection onto the jet. This function is mutating the input point set. This variant requires the kernel.
Warning: This method moves the points, thus should not be called on containers sorted wrt points position.
Precondition: KNN $>$= 2.
}
\ccGlue
\begin{description}
\item[Template Parameters:]
\begin{description}
\item[ForwardIterator]\ccc{value_type} convertible to \ccc{Point_3}. \item[Kernel]Geometric traits class. \end{description}
\end{description}
\begin{description}
\item[Parameters: ]
\begin{description}
\item[first]input/output points \item[KNN]number of neighbors \end{description}
\end{description}
\ccGlue
\ccFunction{OutputIterator jet_smoothing_3(InputIterator first, InputIterator beyond, OutputIterator output, unsigned int KNN, const unsigned int degre_fitting = 2, const unsigned int degree_monge = 2);}
{
Smooth a point set using jet fitting on the KNN nearest neighbors and reprojection onto the jet. This variant deduces the kernel from iterator types.
Precondition: KNN $>$= 2.
}
\ccGlue
\begin{description}
\item[Returns:]past-the-end output iterator. \end{description}
\begin{description}
\item[Parameters: ]
\begin{description}
\item[first]input points \item[output]output points \item[KNN]number of neighbors \end{description}
\end{description}
\ccGlue
\ccFunction{void jet_smoothing_3(ForwardIterator first, ForwardIterator beyond, unsigned int KNN, const unsigned int degre_fitting = 2, const unsigned int degree_monge = 2);}
{
Smooth a point set using jet fitting on the KNN nearest neighbors and reprojection onto the jet. This function is mutating the input point set. This variant deduces the kernel from iterator types.
Warning: This method moves the points, thus should not be called on containers sorted wrt points position.
Precondition: KNN $>$= 2.
}
\ccGlue
\begin{description}
\item[Template Parameters:]
\begin{description}
\item[ForwardIterator]\ccc{value_type} convertible to \ccc{Point_3}. \end{description}
\end{description}
\begin{description}
\item[Parameters: ]
\begin{description}
\item[first]input/output points \item[KNN]number of neighbors \end{description}
\end{description}
\ccGlue

%END-AUTO(\ccDefinition)

\ccParameters

The full template declarations are:

% The section below is automatically generated. Do not edit!
%START-AUTO(\ccParameters)

template$<$  \\
typename InputIterator,   \\
typename OutputIterator,   \\
typename Kernel$>$  \\
OutputIterator  \\
\ccc{jet_smoothing_3} (InputIterator first, InputIterator beyond, OutputIterator output, const unsigned int KNN, const Kernel\& , const unsigned int \ccc{degre_fitting}=2, const unsigned int \ccc{degree_monge}=2);  \\
  \\
template$<$  \\
typename ForwardIterator,   \\
typename Kernel$>$  \\
void  \\
\ccc{jet_smoothing_3} (ForwardIterator first, ForwardIterator beyond, unsigned int KNN, const Kernel\& , const unsigned int \ccc{degre_fitting}=2, const unsigned int \ccc{degree_monge}=2);  \\
  \\
template$<$  \\
typename InputIterator,   \\
typename OutputIterator$>$  \\
OutputIterator  \\
\ccc{jet_smoothing_3} (InputIterator first, InputIterator beyond, OutputIterator output, unsigned int KNN, const unsigned int \ccc{degre_fitting}=2, const unsigned int \ccc{degree_monge}=2);  \\
  \\
template$<$  \\
typename ForwardIterator$>$  \\
void  \\
\ccc{jet_smoothing_3} (ForwardIterator first, ForwardIterator beyond, unsigned int KNN, const unsigned int \ccc{degre_fitting}=2, const unsigned int \ccc{degree_monge}=2);  \\

%END-AUTO(\ccParameters)

\ccExample

\ccIncludeExampleCode{Surface_reconstruction_3/jet_smoothing_example.cpp}

\end{ccRefFunction}

% +------------------------------------------------------------------------+
%%RefPage: end of main body, begin of footer
\ccRefPageEnd
% EOF
% +------------------------------------------------------------------------+

