% +------------------------------------------------------------------------+
% | Reference manual page: orient_normals_minimum_spanning_tree_3.tex
% +------------------------------------------------------------------------+
% | 02.06.2008   Pierre Alliez, Laurent Saboret, Gael Guennebaud
% | Package: Surface_reconstruction_3
% |
\RCSdef{\RCSorientnormalsminimumspanningtreeRev}{$Id$}
\RCSdefDate{\RCSorientnormalsminimumspanningtreeDate}{$Date$}
% |
\ccRefPageBegin
%%RefPage: end of header, begin of main body
% +------------------------------------------------------------------------+


\begin{ccRefFunction}{orient_normals_minimum_spanning_tree_3}  %% add template arg's if necessary

%% \ccHtmlCrossLink{}     %% add further rules for cross referencing links
%% \ccHtmlIndexC[function]{} %% add further index entries

\ccDefinition

\ccc{CGAL::orient_normals_minimum_spanning_tree_3()} orients the normals of a point set using the method described by Hoppe, DeRose, Duchamp, McDonald and Stuetzle in {\em Surface reconstruction from unorganized points} \cite{cgal:hddms-srup-92}.
The result is an oriented normal vector for each input point/normal.

\ccInclude{CGAL/orient_normals_minimum_spanning_tree_3.h}

% The section below is automatically generated. Do not edit!
%START-AUTO(\ccDefinition)

\ccFunction{unsigned int orient_normals_minimum_spanning_tree_3(VertexIterator first, VertexIterator beyond, VertexIndexMap vertex_index_map, VertexPointMap vertex_point_map, VertexNormalMap vertex_normal_map, unsigned int KNN);}
{
Orient the normals of a point set using the method described by Hoppe, DeRose, Duchamp, McDonald and Stuetzle in {\em Surface reconstruction from unorganized points} \cite{cgal:hddms-srup-92}.
This variant implements the original algorithm. Note that it does not orient normals that are already oriented.
Preconditions:\begin{itemize}
\item VertexIterator is a model of ForwardIterator.\item VertexIndexMap is a model of \ccc{boost::readable_property_map}.\item VertexPointMap is a model of \ccc{boost::readable_property_map}.\item VertexNormalMap is a model of \ccc{boost::lvalue_property_map}.\item Normals must be unit vectors.\item KNN $>$= 2.\end{itemize}
}
\ccGlue
\begin{description}
\item[Returns:]the number of un-oriented normals. \end{description}
\begin{description}
\item[Parameters: ]
\begin{description}
\item[first]first input vertex \item[beyond]past-the-end input vertex \item[\ccc{vertex_index_map}]property map VertexIterator -$>$ index \item[\ccc{vertex_point_map}]property map VertexIterator -$>$ \ccc{Point_3} \item[\ccc{vertex_normal_map}]property map VertexIterator -$>$ Normal (in and out) \item[KNN]number of neighbors \end{description}
\end{description}
\ccGlue
\ccFunction{unsigned int orient_normals_minimum_spanning_tree_3(VertexIterator first, VertexIterator beyond, VertexIndexMap vertex_index_map, VertexPointMap vertex_point_map, VertexNormalMap vertex_normal_map, unsigned int KNN, double angle_max);}
{
Orient the normals of a point set using the method described by Hoppe, DeRose, Duchamp, McDonald and Stuetzle in {\em Surface reconstruction from unorganized points} \cite{cgal:hddms-srup-92}.
This is a variant of the original algorithm. It:\begin{itemize}
\item orients the top point towards +Z axis.\item does not orient normals that are already oriented.\item does not propagate the orientation if the angle between 2 normals $>$ \ccc{angle_max}.\end{itemize}
Preconditions:\begin{itemize}
\item VertexIterator is a model of ForwardIterator.\item VertexIndexMap is a model of \ccc{boost::readable_property_map}.\item VertexPointMap is a model of \ccc{boost::readable_property_map}.\item VertexNormalMap is a model of \ccc{boost::lvalue_property_map}.\item Normals must be unit vectors.\item KNN $>$= 2.\item 0 $<$ \ccc{angle_max} $<$= PI/2.\end{itemize}
}
\ccGlue
\begin{description}
\item[Returns:]the number of un-oriented normals. \end{description}
\begin{description}
\item[Parameters: ]
\begin{description}
\item[first]first input vertex \item[beyond]past-the-end input vertex \item[\ccc{vertex_index_map}]property map VertexIterator -$>$ index \item[\ccc{vertex_point_map}]property map VertexIterator -$>$ \ccc{Point_3} \item[\ccc{vertex_normal_map}]property map VertexIterator -$>$ Normal (in and out) \item[KNN]number of neighbors \item[\ccc{angle_max}]max angle to propagate the normals orientation (radians) \end{description}
\end{description}
\ccGlue

%END-AUTO(\ccDefinition)

\ccParameters

The full template declarations are:

% The section below is automatically generated. Do not edit!
%START-AUTO(\ccParameters)

template$<$  \\
class VertexIterator,   \\
class VertexPointMap,   \\
class VertexIndexMap,   \\
class VertexNormalMap$>$  \\
unsigned int  \\
\ccc{orient_normals_minimum_spanning_tree_3} (VertexIterator first, VertexIterator beyond, VertexIndexMap \ccc{vertex_index_map}, VertexPointMap \ccc{vertex_point_map}, VertexNormalMap \ccc{vertex_normal_map}, unsigned int KNN);  \\
  \\
template$<$  \\
class VertexIterator,   \\
class VertexPointMap,   \\
class VertexIndexMap,   \\
class VertexNormalMap$>$  \\
unsigned int  \\
\ccc{orient_normals_minimum_spanning_tree_3} (VertexIterator first, VertexIterator beyond, VertexIndexMap \ccc{vertex_index_map}, VertexPointMap \ccc{vertex_point_map}, VertexNormalMap \ccc{vertex_normal_map}, unsigned int KNN, double \ccc{angle_max});  \\

%END-AUTO(\ccParameters)

\ccSeeAlso

\ccRefIdfierPage{CGAL::estimate_normals_pca_3}  \\
\ccRefIdfierPage{CGAL::estimate_normals_jet_fitting_3}  \\

\ccExample

See \ccc{normal_estimation.cpp} example.

\end{ccRefFunction}

% +------------------------------------------------------------------------+
%%RefPage: end of main body, begin of footer
\ccRefPageEnd
% EOF
% +------------------------------------------------------------------------+

