% +------------------------------------------------------------------------+
% | Reference manual page: Oriented_normal_3.tex
% +------------------------------------------------------------------------+
% | 05.03.2008   Laurent Saboret, Pierre Alliez
% | Package: Package
% |
\RCSdef{\RCSOrientednormalRev}{$Id$}
\RCSdefDate{\RCSOrientednormalDate}{$Date$}
% |
\ccRefPageBegin
%%RefPage: end of header, begin of main body
% +------------------------------------------------------------------------+


\begin{ccRefClass}{Oriented_normal_3}  %% add template arg's if necessary

%% \ccHtmlCrossLink{}     %% add further rules for cross referencing links
%% \ccHtmlIndexC[class]{} %% add further index entries

\ccDefinition

\ccInclude{CGAL/Oriented_normal_3.h}

% The section below is automatically generated. Do not edit!
%START-AUTO(\ccDefinition)

The \ccc{Oriented_normal_3} class represents a normal vector (oriented or not). The normal vector is allocated only when needed.

%END-AUTO(\ccDefinition)

\ccParameters

% The section below is automatically generated. Do not edit!
%START-AUTO(\ccParameters)

template$<$  \\
class Gt$>$   \\
class \ccc{Oriented_normal_3};

\begin{description}
\item[Parameters:]
\begin{description}
\item[Gt]Kernel's geometric traits. \end{description}
\end{description}

%END-AUTO(\ccParameters)

\ccIsModel

% The section below is automatically generated. Do not edit!
%START-AUTO(\ccIsModel)

Model of the \ccc{OrientedNormal_3} concept.

%END-AUTO(\ccIsModel)

\ccTypes

% The section below is automatically generated. Do not edit!
%START-AUTO(\ccTypes)

\ccNestedType{Geom_traits}
{
Kernel's geometric traits.
}
\ccGlue
\ccNestedType{Vector}
{
}
\ccGlue

%END-AUTO(\ccTypes)

\ccCreation
\ccCreationVariable{n}  %% choose variable name

% The section below is automatically generated. Do not edit!
%START-AUTO(\ccCreation)

\ccConstructor{Oriented_normal_3(Null_vector = NULL_VECTOR);}
{
Normal vector is (0,0,0) by default. Normal is oriented by default.
}
\ccGlue
\ccConstructor{Oriented_normal_3(const Vector& vector, bool oriented = true);}
{
}
\ccGlue
\ccConstructor{Oriented_normal_3(const Oriented_normal_3<Gt>& that);}
{
Copy constructor.
}
\ccGlue

%END-AUTO(\ccCreation)

\ccOperations

% The section below is automatically generated. Do not edit!
%START-AUTO(\ccOperations)

\ccMethod{Oriented_normal_3& operator=(const Oriented_normal_3<Gt>& that);}
{
Operator =().
}
\ccGlue
\ccMethod{Vector get_vector() const;}
{
Get normal vector.
}
\ccGlue
\ccMethod{bool is_normal_oriented() const;}
{
Get normal orientation.
}
\ccGlue
\ccMethod{void set(const Vector& vector, bool oriented = true);}
{
Set normal (vector + orientation).
}
\ccGlue

%END-AUTO(\ccOperations)

\ccSeeAlso

\ccc{Point_with_normal_3<Geom_traits>}

\ccExample

A short example program.
Instead of a short program fragment, a full running program can be
included using the
\verb|\ccIncludeExampleCode{Package/Oriented_normal_3.C}|
macro. The program example would be part of the source code distribution and
also part of the automatic test suite.

\begin{ccExampleCode}
void your_example_code() {
}
\end{ccExampleCode}

%% \ccIncludeExampleCode{Package/Oriented_normal_3.C}

\end{ccRefClass}

% +------------------------------------------------------------------------+
%%RefPage: end of main body, begin of footer
\ccRefPageEnd
% EOF
% +------------------------------------------------------------------------+

