% +------------------------------------------------------------------------+
% | Reference manual page: jet_normal_estimation.tex
% +------------------------------------------------------------------------+
% | 19.02.2008   Pierre Alliez, Laurent Saboret, Gael Guennebaud
% | Package: Surface_reconstruction_3
% |
\RCSdef{\RCSjetnormalestimationRev}{$Id$}
\RCSdefDate{\RCSjetnormalestimationDate}{$Date$}
% |
\ccRefPageBegin
%%RefPage: end of header, begin of main body
% +------------------------------------------------------------------------+


\begin{ccRefFunction}{jet_normal_estimation}  %% add template arg's if necessary

%% \ccHtmlCrossLink{}     %% add further rules for cross referencing links
%% \ccHtmlIndexC[function]{} %% add further index entries

\ccDefinition

\ccc{CGAL::jet_normal_estimation()} estimates normals direction of a point set using jet fitting on the K nearest neighbors.
The default jet is a quadric.
The result is an unoriented normal vector for each input point.

The \ccc{CGAL::jet_normal_estimation()} function exists in two flavors.
The complete version requires the kernel to use for computations.
The main version deduces the kernel from input parameters.

\ccInclude{CGAL/jet_normal_estimation.h}

% The section below is automatically generated. Do not edit!
%START-AUTO(\ccDefinition)

\ccFunction{OutputIterator jet_normal_estimation(InputIterator first, InputIterator beyond, OutputIterator normals, unsigned int KNN, const Kernel& , unsigned int degre_fitting);}
{
Estimate normal directions using jet fitting on the KNN nearest neighbors. This variant requires the kernel.
\ccPrecond KNN $>$= 2.
\ccCommentHeading{Template Parameters}
\begin{description}
\item \ccc{InputIterator}: \ccc{value_type} is \ccc{Point_3}. \item \ccc{OutputIterator}: \ccc{value_type} is \ccc{Point_3}. \item \ccc{Kernel}: Geometric traits class.\end{description}
\ccCommentHeading{Returns} past-the-end output iterator.
\ccCommentHeading{Parameters}
\begin{description}
\item \ccc{first}: input points \item \ccc{normals}: output normals \item \ccc{KNN}: number of neighbors \end{description}
}
\ccGlue
\ccFunction{OutputIterator jet_normal_estimation(InputIterator first, InputIterator beyond, OutputIterator normals, unsigned int KNN, unsigned int degre_fitting = 2);}
{
Estimate normal directions using jet fitting on the KNN nearest neighbors. This variant deduces the kernel from iterator types.
\ccPrecond KNN $>$= 2.
\ccCommentHeading{Template Parameters}
\begin{description}
\item \ccc{InputIterator}: \ccc{value_type} is \ccc{Point_3}. \item \ccc{OutputIterator}: \ccc{value_type} is \ccc{Point_3}.\end{description}
\ccCommentHeading{Returns} past-the-end output iterator.
\ccCommentHeading{Parameters}
\begin{description}
\item \ccc{first}: input points \item \ccc{normals}: output normals \item \ccc{KNN}: number of neighbors \end{description}
}
\ccGlue

%END-AUTO(\ccDefinition)

\ccSeeAlso

\ccRefIdfierPage{CGAL::pca_normal_estimation}  \\
\ccRefIdfierPage{CGAL::mst_normal_orientation}  \\

\ccExample

See \ccc{normal_estimation.cpp} example.

\end{ccRefFunction}

% +------------------------------------------------------------------------+
%%RefPage: end of main body, begin of footer
\ccRefPageEnd
% EOF
% +------------------------------------------------------------------------+

