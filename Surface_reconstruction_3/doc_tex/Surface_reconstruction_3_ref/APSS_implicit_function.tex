% +------------------------------------------------------------------------+
% | Reference manual page: APSS_implicit_function.tex
% +------------------------------------------------------------------------+
% | 02.06.2008   Laurent Saboret, Pierre Alliez
% | Package: Surface_reconstruction_3
% |
\RCSdef{\RCSAPSSimplicitfunctionRev}{$Id$}
\RCSdefDate{\RCSAPSSimplicitfunctionDate}{$Date$}
% |
\ccRefPageBegin
%%RefPage: end of header, begin of main body
% +------------------------------------------------------------------------+


\begin{ccRefClass}{APSS_implicit_function<GeomTraits, PointWithNormal_3>}

%% \ccHtmlCrossLink{}     %% add further rules for cross referencing links
%% \ccHtmlIndexC[class]{} %% add further index entries

\ccDefinition

% The section below is automatically generated. Do not edit!
%START-AUTO(\ccDefinition)

\ccc{APSS_implicit_function} computes an implicit function that defines a Point Set Surface (PSS) based on moving least squares (MLS) fitting of algebraic spheres. See {\em Algebraic Point Set Surfaces} by Guennebaud and Gross (2007).

%END-AUTO(\ccDefinition)

% The section below is automatically generated. Do not edit!
%START-AUTO(\ccInclude)

\ccInclude{CGAL/APSS_implicit_function.h}

%END-AUTO(\ccInclude)

\ccParameters

The full template declaration is:

% The section below is automatically generated. Do not edit!
%START-AUTO(\ccParameters)

template$<$  \\
class Gt,   \\
class \ccc{PointWithNormal_3}$>$   \\
class \ccc{APSS_implicit_function};

\begin{description}
\item[Parameters:]
\begin{description}
\item[Gt]Geometric traits class. \item[\ccc{PointWithNormal_3}]Model of \ccc{PointWithNormal_3} concept. \end{description}
\end{description}

%END-AUTO(\ccParameters)

\ccIsModel

% The section below is automatically generated. Do not edit!
%START-AUTO(\ccIsModel)

Model of the \ccc{Reconstruction_implicit_function} concept.

%END-AUTO(\ccIsModel)

\ccHeading{Design Pattern}

% The section below is automatically generated. Do not edit!
%START-AUTO(\ccHeading{Design Pattern})

A model of ReconstructionImplicitFunction is a Strategy \cite{cgal:ghjv-dpero-95}: it implements a strategy of surface mesh reconstruction.

%END-AUTO(\ccHeading{Design Pattern})

\ccTypes

% The section below is automatically generated. Do not edit!
%START-AUTO(\ccTypes)

\ccNestedType{Geom_traits}
{
Kernel's geometric traits.
}
\ccGlue
\ccNestedType{FT}
{
}
\ccGlue
\ccNestedType{Point}
{
}
\ccGlue
\ccNestedType{Iso_cuboid}
{
}
\ccGlue
\ccNestedType{Sphere}
{
}
\ccGlue
\ccNestedType{Point_with_normal}
{
Model of \ccc{PointWithNormal_3} concept.
}
\ccGlue
\ccNestedType{Normal}
{
Model of \ccc{Kernel::Vector_3} concept.
}
\ccGlue
\ccNestedType{Vector}
{
}
\ccGlue

%END-AUTO(\ccTypes)

\ccCreation
\ccCreationVariable{fct}  %% variable name for \ccMethod below

% The section below is automatically generated. Do not edit!
%START-AUTO(\ccCreation)

\ccConstructor{APSS_implicit_function(InputIterator first, InputIterator beyond, unsigned int k, FT projection_error = 3.16e-4);}
{
Create an APSS implicit function from a point set.
Precondition: the value type of InputIterator must be convertible to \ccc{Point_with_normal}.
}
\ccGlue
\begin{description}
\item[Parameters:]
\begin{description}
\item[first]First point of point set. \item[beyond]Past-the-end point of point set. \item[k]Number of nearest neighbours. \item[\ccc{projection_error}]Dichotomy error when projecting point. \end{description}
\end{description}
\ccGlue
\ccConstructor{APSS_implicit_function(const APSS_implicit_function<Gt, PointWithNormal_3>& other);}
{
}
\ccGlue

%END-AUTO(\ccCreation)

\ccOperations

% The section below is automatically generated. Do not edit!
%START-AUTO(\ccOperations)

\ccMethod{APSS_implicit_function& operator=(const APSS_implicit_function<Gt, PointWithNormal_3>& other);}
{
}
\ccGlue
\ccMethod{void setNofNeighbors(unsigned int k);}
{
}
\ccGlue
\ccMethod{Iso_cuboid bounding_box() const;}
{
Get the bounding box.
}
\ccGlue
\ccMethod{const Sphere& bounding_sphere() const;}
{
Get the surface's bounding sphere.
}
\ccGlue
\ccMethod{Sphere region_of_interest() const;}
{
Get the region of interest, ignoring the outliers. This method is used to define the OpenGL arcball sphere.
}
\ccGlue
\ccMethod{bool compute_implicit_function();}
{
You should call \ccc{compute_implicit_function}() once when points insertion is over. Return false on error.
}
\ccGlue
\ccMethod{FT operator()(const Point& p) const;}
{
[ImplicitFunction interface]
Evaluate implicit function for any 3D point.
}
\ccGlue
\ccMethod{Point get_inner_point() const;}
{
Get point inside the surface.
}
\ccGlue

%END-AUTO(\ccOperations)

\ccSeeAlso

\ccExample

A short example program.
Instead of a short program fragment, a full running program can be
included using the
\verb|\ccIncludeExampleCode{Surface_reconstruction_3/APSS_implicit_function.C}|
macro. The program example would be part of the source code distribution and
also part of the automatic test suite.

\begin{ccExampleCode}
void your_example_code() {
}
\end{ccExampleCode}

%% \ccIncludeExampleCode{Surface_reconstruction_3/APSS_implicit_function.C}

\end{ccRefClass}

% +------------------------------------------------------------------------+
%%RefPage: end of main body, begin of footer
\ccRefPageEnd
% EOF
% +------------------------------------------------------------------------+

