% +------------------------------------------------------------------------+
% | Reference manual page: Surface_reconstruction_3/intro.tex
% +------------------------------------------------------------------------+
% | 06.01.2009   Pierre Alliez, Laurent Saboret, Gael Guennebaud
% | Package: Surface_reconstruction_3
% |
% |
% +------------------------------------------------------------------------+

%\clearpage
%\section{Reference Pages for Surface Reconstruction}
\ccRefChapter{Surface Reconstruction
\label{chap:surface_reconstruction_3_ref}}

\ccChapterAuthor{Pierre Alliez, Laurent Saboret, Gael Guennebaud}


This CGAL package implements some of the state-of-the-art surface
reconstruction methods. Priority is given to methods
which take as input unorganized point sets and
which compute an implicit function.

The package proposes an interface to the CGAL Surface Mesh Generator.

The input is an unorganized point set, possibly with attributes
such as unoriented normals, oriented normals.

Since reconstruction methods often require to preprocess a point set
(e.g. estimate and orient the normals), we provide components devoted
to this task.

The output can be either an implicit function (ready for evaluation
by any contouring algorithm), or a surface mesh.


\section{Classified Reference Pages}


\subsection{Concepts}

\subsubsection{Points and Normals}

Functions and classes of this package expect as input and output parameters range iterators over:

\begin{itemize}
\item 3D points
\item Normals (orientable 3D vectors)
\item 3D points with unoriented normals
\item 3D points with oriented normals
\end{itemize}

This leads to the next concepts to represent these types:

\begin{itemize}
\item \ccc{Kernel::Point_3}: 3D position.
\item \ccc{Kernel::Vector_3}: 3D vector.
\item \ccc{OrientableNormal_3}: normal vector (oriented or not). Inherits from \ccc{Kernel::Vector_3}.
\item \ccc{PointWithNormal_3}: inherits from \ccc{Kernel::Point_3} and contains a \ccc{Kernel::Vector_3} normal.
\item \ccc{PointWithOrientableNormal_3}: inherits from \ccc{Kernel::Point_3} and contains a \ccc{OrientableNormal_3} normal.
\end{itemize}

\subsubsection{Reconstruction}

We define a ReconstructionImplicitFunction concept that will provide
a common interface to the Marching Cubes and the Surface Mesh Generator:

\ccRefIdfierPage{ReconstructionImplicitFunction}  \\

Eventually, the \ccc{ImplicitFctDelaunayTriangulation_3} concept defines the interface of a 3D Delaunay triangulation requested by the \ccc{Poisson_implicit_function} class:

\ccRefIdfierPage{ImplicitFctDelaunayTriangulation_3}  \\


\subsection{Input}

We provide several models of points and normals concepts, with different speed/space trade-offs:

\begin{itemize}
\item \ccc{Point_3<Geom_traits>}: CGAL regular 3D position.
\item \ccc{Vector_3<Geom_traits>}: CGAL regular 3D vector.
\item \ccc{Lightweight_vector_3<Geom_traits>}: model of the \ccc{Kernel::Vector_3} concept. 3D vector allocated only if not (0,0,0).
\item \ccc{Orientable_normal_3<Geom_traits>}: model of the \ccc{OrientableNormal_3} concept.
Inherits from \ccc{Vector_3<Geom_traits>} and contains an "is normal oriented?" flag.
\item \ccc{Point_with_normal_3<Geom_traits, Normal_3>}: model of the \ccc{PointWithNormal_3} concept.
Also model of \ccc{PointWithOrientableNormal_3} if \ccc{Normal_3} is a model of \ccc{OrientableNormal_3}.
\end{itemize}

For convenience, we provide also functions to read point sets from standard file formats:

\begin{itemize}
\item XYZ
\item OFF
\end{itemize}

\ccRefIdfierPage{surface_reconstruction_read_xyz()}  \\
\ccRefIdfierPage{surface_reconstruction_read_off_point_cloud()}  \\


\subsection{Analysis}

The purpose of the analysis stage is to compute parameters for the next stages algorithms.

\begin{itemize}
\item Point set barycenter, bounding box, bounding sphere (provided by CGAL Principal Components Analysis package)
\item Average spacing to the K nearest neighbors
\end{itemize}

\ccRefIdfierPage{centroid()}  \\
\ccRefIdfierPage{bounding_box()}  \\
\ccRefIdfierPage{average_spacing_3()}  \\


\subsection{Processing}

\subsubsection{Outliers Removal Methods}

\begin{itemize}
\item Outliers removal wrt average squared distance to the K nearest neighbors
\end{itemize}

\ccRefIdfierPage{remove_outliers_wrt_avg_knn_sq_distance_3()}  \\


\subsubsection{Point Set Simplification Methods}

\begin{itemize}
\item Point set simplification by clustering
\item Random point set simplification
\end{itemize}

\ccRefIdfierPage{merge_epsilon_nearest_points_3()}  \\
\ccRefIdfierPage{random_simplification_points_3()}  \\


\subsubsection{Smoothing Methods}

\begin{itemize}
\item Smoothing via Jet fitting over the K nearest neighbors + reprojection
\end{itemize}

\ccRefIdfierPage{smooth_jet_fitting_3()}  \\


\subsubsection{Normals Estimation Methods}

\begin{itemize}
\item Normals estimation by Principal Components Analysis over the K nearest neighbors
\item Normals estimation by Jet fitting over the K nearest neighbors
\end{itemize}

\ccRefIdfierPage{estimate_normals_jet_fitting_3()}  \\
\ccRefIdfierPage{estimate_normals_pca_3()}  \\


\subsubsection{Normals Orientation Methods}

\begin{itemize}
\item Normals orientation using a Minimal Spanning Tree \cite{cgal:hddms-srup-92}
\end{itemize}

\ccRefIdfierPage{orient_normals_minimum_spanning_tree_3()}  \\


\subsection{Surface Reconstruction}

\subsubsection{Implicit Functions}

This package implements:

\begin{itemize}
\item Delaunay-based Poisson reconstruction \cite{Kazhdan06}
\item Algebraic Point Set Surfaces \cite{Guennebaud07}
\end{itemize}

\ccRefIdfierPage{Poisson_implicit_function}  \\
\ccRefIdfierPage{APSS_implicit_function}  \\


\subsubsection{Implicit Functions Contouring}

Implicit functions can be contoured to reconstruct a surface by:

\begin{itemize}
\item CGAL Surface Mesh Generator~\cite{cgal:ry-gsddrm-06,cgal:bo-pgsms-05}
\end{itemize}

\ccRefIdfierPage{make_surface_mesh()}  \\

The parameter \ccc{Tag} affects the behavior of \ccc{make_surface_mesh()}: \\
- \ccc{Manifold_tag}: the output mesh is guaranteed to be a manifold
surface without boundary.\\
- \ccc{Manifold_with_boundary_tag}: the output mesh is guaranteed to be
manifold but may have boundaries.\\
- \ccc{Non_manifold_tag}: the output mesh will be a polygon soup.


\subsection{Output}

\subsubsection{Point Set Output}

The output of the processing stage is a point set with normals.
For convenience, we provide functions to write point sets to standard file formats:

\begin{itemize}
\item XYZ
\item OFF
\end{itemize}

\ccRefIdfierPage{surface_reconstruction_write_off_point_cloud()}  \\
\ccRefIdfierPage{surface_reconstruction_write_xyz()}  \\


\subsubsection{Surface Output}

The surface reconstructed by \ccc{make_surface_mesh()}
is required to be a model of the concept
\ccc{SurfaceMeshComplex_2InTriangulation_3},
a data structure able to represent a two dimensional
complex embedded in a three dimensional triangulation. \\
\ccc{SurfaceMeshComplex_2InTriangulation_3} defines the methods to traverse the reconstructed surface.

As examples, we provide functions to:

\begin{itemize}
\item write the reconstructed surface to standard OFF file format
\item convert the reconstructed surface to a polygon soup
\end{itemize}

\ccRefIdfierPage{output_surface_facets_to_off()}  \\
\ccRefIdfierPage{surface_reconstruction_output_surface_facets()}  \\


\subsection{Checks and Assertions}

The package performs the next checks:

\begin{itemize}
\item Preconditions:
    \begin{itemize}
    \item check that the input point set is not empty.
    \item check that the input point set contains normals.
    \end{itemize}
\item Postconditions:
    \begin{itemize}
    \item check normals estimation and orientation.
    \item check Poisson reconstruction success.
    \end{itemize}
\end{itemize}

Assertions are optional checks. The assertion flags for the package
use \ccc{SURFACE_RECONSTRUCTION} in their names (\textit{e.g.},
\ccc{CGAL_SURFACE_RECONSTRUCTION_NO_ASSERTIONS}).


