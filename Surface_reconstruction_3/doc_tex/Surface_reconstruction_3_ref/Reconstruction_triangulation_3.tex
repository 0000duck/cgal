% +------------------------------------------------------------------------+
% | Reference manual page: Reconstruction_triangulation_3.tex
% +------------------------------------------------------------------------+
% | 06.09.2007   Pierre Alliez, Laurent Saboret, Gael Guennebaud
% | Package: Surface_reconstruction_3
% |
\RCSdef{\RCSImplicitfctdelaunaytriangulationRev}{$Id$}
\RCSdefDate{\RCSImplicitfctdelaunaytriangulationDate}{$Date$}
% |
\ccRefPageBegin
%%RefPage: end of header, begin of main body
% +------------------------------------------------------------------------+


\begin{ccRefClass}{Reconstruction_triangulation_3<GeomTraits, TriangulationDataStructure_3>}

%% \ccHtmlCrossLink{}     %% add further rules for cross referencing links
%% \ccHtmlIndexC[class]{} %% add further index entries

\ccDefinition

% The section below is automatically generated. Do not edit!
%START-AUTO(\ccDefinition)

The \ccc{Reconstruction_triangulation_3} class is the default implementation of the \ccc{ReconstructionTriangulation_3} concept. The cell class must be a model of \ccc{ReconstructionCellBase_3} and the vertex class must be a model of \ccc{ReconstructionVertexBase_3}.

It provides the interface requested by the \ccc{Poisson_reconstruction_function} class:\begin{itemize}
\item Each vertex owns a normal vector.\item A vertex is either an input point or a Steiner point added by Delaunay refinement.\item In order to solve a linear system over the triangulation, a vertex may be constrained or not, and has a unique index.\end{itemize}


CAUTION: \ccc{invalidate_bounds}() must be called after modifying the points.

%END-AUTO(\ccDefinition)

% The section below is automatically generated. Do not edit!
%START-AUTO(\ccInclude)

\ccInclude{CGAL/Reconstruction_triangulation_3.h}

%END-AUTO(\ccInclude)

\ccParameters

The full template declaration is:

% The section below is automatically generated. Do not edit!
%START-AUTO(\ccParameters)

template$<$  \\
class BaseGt,   \\
class Gt = \ccc{Reconstruction_triangulation_default_geom_traits_3<BaseGt>},   \\
class Tds = \ccc{Triangulation_data_structure_3<Reconstruction_vertex_base_3<Gt>}, \ccc{Reconstruction_cell_base_3<Gt>} $>$$>$   \\
class \ccc{Reconstruction_triangulation_3};

\begin{description}
\item[Parameters:]
\begin{description}
\item[BaseGt]Kernel's geometric traits. \item[Gt]Geometric traits class / \ccc{Point_3} is a model of \ccc{PointWithNormal_3}. \item[Tds]Model of \ccc{TriangulationDataStructure_3}. The cell class must be a model of \ccc{ReconstructionCellBase_3} and the vertex class must be a model of \ccc{ReconstructionVertexBase_3}. \end{description}
\end{description}

%END-AUTO(\ccParameters)

\ccInheritsFrom

% The section below is automatically generated. Do not edit!
%START-AUTO(\ccInheritsFrom)

\ccc{Delaunay_triangulation_3<Gt, Tds>}

%END-AUTO(\ccInheritsFrom)

\ccIsModel

% The section below is automatically generated. Do not edit!
%START-AUTO(\ccIsModel)

Model of the \ccc{ReconstructionTriangulation_3} concept.

%END-AUTO(\ccIsModel)

\ccTypes

% The section below is automatically generated. Do not edit!
%START-AUTO(\ccTypes)

\ccNestedType{FT}
{
}
\ccGlue
\ccNestedType{Vector}
{
}
\ccGlue
\ccNestedType{Iso_cuboid}
{
}
\ccGlue
\ccNestedType{Sphere}
{
}
\ccGlue
\ccNestedType{Point}
{
The geometric traits class's \ccc{Point_3} type is a model of \ccc{PointWithNormal_3}.
Model of \ccc{PointWithNormal_3}
}
\ccGlue
\ccNestedType{Point_with_normal}
{
Model of \ccc{PointWithNormal_3}.
}
\ccGlue
\ccNestedType{Normal}
{
Model of \ccc{Kernel::Vector_3} concept.
}
\ccGlue
\ccNestedType{Normal_iterator}
{
Iterator over all normals.
}
\ccGlue
\ccNestedType{Input_vertices_iterator}
{
Iterator over input vertices.
}
\ccGlue
\ccNestedType{Input_point_iterator}
{
Iterator over input points.
}
\ccGlue

%END-AUTO(\ccTypes)

\ccHeading{Variables}

% The section below is automatically generated. Do not edit!
%START-AUTO(\ccHeading{Variables})

\ccVariable{const unsigned char INPUT = 0;}
{
[static] \\
Point type.
}
\ccGlue
\ccVariable{const unsigned char STEINER = 1;}
{
[static] \\
}
\ccGlue
\ccVariable{Iso_cuboid m_bounding_box;}
{
[mutable] \\
}
\ccGlue
\ccVariable{Iso_cuboid m_input_points_bounding_box;}
{
[mutable] \\
}
\ccGlue
\ccVariable{Sphere m_bounding_sphere;}
{
[mutable] \\
}
\ccGlue
\ccVariable{Sphere m_input_points_bounding_sphere;}
{
[mutable] \\
}
\ccGlue
\ccVariable{Point m_barycenter;}
{
[mutable] \\
}
\ccGlue
\ccVariable{FT m_diameter_standard_deviation;}
{
[mutable] \\
}
\ccGlue

%END-AUTO(\ccHeading{Variables})

\ccCreation
\ccCreationVariable{pdt}  %% variable name for \ccMethod calls below

% The section below is automatically generated. Do not edit!
%START-AUTO(\ccCreation)

\ccConstructor{Reconstruction_triangulation_3();}
{
Default constructor.
}
\ccGlue

%END-AUTO(\ccCreation)

\ccOperations

% The section below is automatically generated. Do not edit!
%START-AUTO(\ccOperations)

\ccMethod{Normal_iterator normals_begin();}
{
Get first iterator over finite vertices normals.
}
\ccGlue
\ccMethod{Normal_iterator normals_end();}
{
Get past-the-end iterator over finite vertices normals.
}
\ccGlue
\ccMethod{Input_vertices_iterator input_vertices_begin() const;}
{
Get first iterator over input vertices.
}
\ccGlue
\ccMethod{Input_vertices_iterator input_vertices_end() const;}
{
Get past-the-end iterator over input vertices.
}
\ccGlue
\ccMethod{Input_point_iterator input_points_begin() const;}
{
Get first iterator over input points.
}
\ccGlue
\ccMethod{Input_point_iterator input_points_end() const;}
{
Get past-the-end iterator over input points.
}
\ccGlue
\ccMethod{Iso_cuboid bounding_box() const;}
{
Get the bounding box of all points.
}
\ccGlue
\ccMethod{Iso_cuboid input_points_bounding_box() const;}
{
Get the bounding box of input points.
}
\ccGlue
\ccMethod{Sphere bounding_sphere() const;}
{
Get the bounding sphere of all points.
}
\ccGlue
\ccMethod{Sphere input_points_bounding_sphere() const;}
{
Get the bounding sphere of input points.
}
\ccGlue
\ccMethod{Point barycenter() const;}
{
Get the barycenter of all points.
}
\ccGlue
\ccMethod{FT diameter_standard_deviation() const;}
{
Get the standard deviation of the distance to barycenter (for all points).
}
\ccGlue
\ccMethod{void invalidate_bounds();}
{
Update barycenter, bounding box, bounding sphere and standard deviation. Owner is responsible to call this function after modifying the triangulation.
}
\ccGlue
\ccMethod{Vertex_handle insert(const Point& p, unsigned char type = INPUT, Cell_handle start = Cell_handle());}
{
Insert point in the triangulation. Default type is INPUT.
}
\ccGlue
\ccMethod{int insert(InputIterator first, InputIterator beyond, unsigned char type = INPUT);}
{
Insert points in the triangulation using a spatial sort. Default type is INPUT.
Precondition: the value type of InputIterator must be \ccc{Point}.
}
\ccGlue
\begin{description}
\item[Parameters:]
\begin{description}
\item[first]First point to add to pdt. \item[beyond]Past-the-end point to add to pdt. \end{description}
\end{description}
\begin{description}
\item[Returns:]the number of inserted points. \end{description}
\ccGlue
\ccMethod{Vertex_handle insert_in_hole(const Point& p, CellIt cell_begin, CellIt cell_end, Cell_handle begin, int i, unsigned char type = STEINER);}
{
Delaunay refinement callback: insert STEINER point in the triangulation.
}
\ccGlue

%END-AUTO(\ccOperations)

\ccSeeAlso

\ccRefIdfierPage{CGAL::Reconstruction_cell_base_3<GeomTraits, TriangulationCellBase_3>}  \\
\ccRefIdfierPage{CGAL::Reconstruction_vertex_base_3<GeomTraits, TriangulationVertexBase_3>}  \\
\ccRefIdfierPage{CGAL::Poisson_reconstruction_function<GeomTraits, ReconstructionTriangulation_3>}  \\

\end{ccRefClass}

% +------------------------------------------------------------------------+
%%RefPage: end of main body, begin of footer
\ccRefPageEnd
% EOF
% +------------------------------------------------------------------------+

