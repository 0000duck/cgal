% +------------------------------------------------------------------------+
% | Reference manual page: Orientable_normal_3.tex
% +------------------------------------------------------------------------+
% | 05.03.2008   Pierre Alliez, Laurent Saboret, Gael Guennebaud
% | Package: Surface_reconstruction_3
% |
\RCSdef{\RCSOrientable_normalRev}{$Id$}
\RCSdefDate{\RCSOrientable_normalDate}{$Date$}
% |
\ccRefPageBegin
%%RefPage: end of header, begin of main body
% +------------------------------------------------------------------------+


\begin{ccRefClass}{Orientable_normal_3<Geom_traits>}

%% \ccHtmlCrossLink{}     %% add further rules for cross referencing links
%% \ccHtmlIndexC[class]{} %% add further index entries

\ccDefinition

% The section below is automatically generated. Do not edit!
%START-AUTO(\ccInclude)

\ccInclude{CGAL/Orientable_normal_3.h}

%END-AUTO(\ccInclude)

% The section below is automatically generated. Do not edit!
%START-AUTO(\ccDefinition)

The \ccc{Orientable_normal_3} class represents a normal vector (oriented or not).

%END-AUTO(\ccDefinition)

\ccParameters

The full template declaration is:

% The section below is automatically generated. Do not edit!
%START-AUTO(\ccParameters)

template$<$  \\
class Gt$>$   \\
class \ccc{Orientable_normal_3};

\begin{description}
\item[Parameters:]
\begin{description}
\item[Gt]Kernel's geometric traits. \end{description}
\end{description}

%END-AUTO(\ccParameters)

\ccInheritsFrom

% The section below is automatically generated. Do not edit!
%START-AUTO(\ccInheritsFrom)

\ccc{Gt::Vector_3}

%END-AUTO(\ccInheritsFrom)

\ccIsModel

% The section below is automatically generated. Do not edit!
%START-AUTO(\ccIsModel)

Model of the \ccc{OrientableNormal_3} concept.

%END-AUTO(\ccIsModel)

\ccTypes

% The section below is automatically generated. Do not edit!
%START-AUTO(\ccTypes)

\ccNestedType{Geom_traits}
{
Kernel's geometric traits.
}
\ccGlue
\ccNestedType{FT}
{
}
\ccGlue
\ccNestedType{RT}
{
}
\ccGlue
\ccNestedType{Vector}
{
Kernel's \ccc{Vector_3} class.
}
\ccGlue

%END-AUTO(\ccTypes)

\ccCreation
\ccCreationVariable{n}  %% variable name for \ccMethod calls below

% The section below is automatically generated. Do not edit!
%START-AUTO(\ccCreation)

\ccConstructor{Orientable_normal_3(Null_vector = NULL_VECTOR, bool oriented = true);}
{
Normal vector is (0,0,0) by default. Normal is oriented by default.
}
\ccGlue
\ccConstructor{Orientable_normal_3(const Vector& vector, bool oriented = true);}
{
}
\ccGlue
\ccConstructor{Orientable_normal_3(FT x, FT y, FT z, bool oriented = true);}
{
}
\ccGlue
\ccConstructor{Orientable_normal_3(RT hx, RT hy, RT hz, RT hw, bool oriented = true);}
{
}
\ccGlue
\ccConstructor{Orientable_normal_3(const Orientable_normal_3<Gt>& that);}
{
Copy constructor.
}
\ccGlue
\ccConstructor{template<class K>Orientable_normal_3(const Orientable_normal_3<K>& that);}
{
}
\ccGlue

%END-AUTO(\ccCreation)

\ccOperations

% The section below is automatically generated. Do not edit!
%START-AUTO(\ccOperations)

\ccMethod{Orientable_normal_3& operator=(const Orientable_normal_3<Gt>& that);}
{
Operator =().
}
\ccGlue
\ccMethod{const Vector& get_vector() const;}
{
Get (a copy of) the actual vector.
}
\ccGlue
\ccMethod{bool is_oriented() const;}
{
Get/set normal orientation.
}
\ccGlue
\ccMethod{void set_orientation(bool oriented);}
{
}
\ccGlue

%END-AUTO(\ccOperations)

\ccSeeAlso

\ccc{Vector_3<Geom_traits>} \\
\ccc{Lightweight_vector_3<Geom_traits>} \\
\ccc{Point_with_normal_3<Geom_traits, Normal_3>}

\end{ccRefClass}

% +------------------------------------------------------------------------+
%%RefPage: end of main body, begin of footer
\ccRefPageEnd
% EOF
% +------------------------------------------------------------------------+

