% +------------------------------------------------------------------------+
% | Reference manual page: merge_epsilon_nearest_points_3.tex
% +------------------------------------------------------------------------+
% | 02.06.2008   Laurent Saboret, Pierre Alliez
% | Package: Surface_reconstruction_3
% |
\RCSdef{\RCSmergeepsilonnearestpointsRev}{$Id$}
\RCSdefDate{\RCSmergeepsilonnearestpointsDate}{$Date$}
% |
\ccRefPageBegin
%%RefPage: end of header, begin of main body
% +------------------------------------------------------------------------+


\begin{ccRefFunction}{merge_epsilon_nearest_points_3}  %% add template arg's if necessary

%% \ccHtmlCrossLink{}     %% add further rules for cross referencing links
%% \ccHtmlIndexC[function]{} %% add further index entries

\ccDefinition

% The section below is automatically generated. Do not edit!
%START-AUTO(\ccDefinition)

\ccFunction{OutputIterator merge_epsilon_nearest_points_3(InputIterator first, InputIterator beyond, OutputIterator output, double epsilon, const Kernel &);}
{
Merge points which belong to the same cell of a grid of cell size = epsilon. This variant requires the kernel.
Precondition: epsilon $>$ 0.
}
\ccGlue
\begin{description}
\item[Template Parameters:]
\begin{description}
\item[InputIterator]\ccc{value_type} must be convertible to OutputIterator's \ccc{value_type}. \item[OutputIterator]\ccc{value_type} must be convertible to \ccc{Point_3}. \item[Kernel]Geometric traits class.\end{description}
\end{description}
\begin{description}
\item[Returns:]past-the-end output iterator. \end{description}
\begin{description}
\item[Parameters: ]
\begin{description}
\item[first]input points \item[output]output points \item[epsilon]tolerance value when comparing 3D points \end{description}
\end{description}
\ccGlue
\ccFunction{ForwardIterator merge_epsilon_nearest_points_3(ForwardIterator first, ForwardIterator beyond, double epsilon, const Kernel &);}
{
Merge points which belong to the same cell of a grid of cell size = epsilon. This function is mutating the input point set. This variant requires the kernel.
Warning: This method modifies the order of points, thus Precondition: epsilon $>$ 0.
}
\ccGlue
\begin{description}
\item[Template Parameters:]
\begin{description}
\item[ForwardIterator]\ccc{value_type} must be convertible to \ccc{Point_3}. \item[Kernel]Geometric traits class.\end{description}
\end{description}
\begin{description}
\item[Returns:]First iterator to remove (see erase-remove idiom). \end{description}
\begin{description}
\item[Parameters: ]
\begin{description}
\item[first]input/output points \item[epsilon]tolerance value when comparing 3D points \end{description}
\end{description}
\ccGlue
\ccFunction{OutputIterator merge_epsilon_nearest_points_3(InputIterator first, InputIterator beyond, OutputIterator output, double epsilon);}
{
Merge points which belong to the same cell of a grid of cell size = epsilon. This variant deduces the kernel from iterator types.
Precondition: epsilon $>$ 0.
}
\ccGlue
\begin{description}
\item[Template Parameters:]
\begin{description}
\item[InputIterator]\ccc{value_type} must be convertible to OutputIterator's \ccc{value_type}. \item[OutputIterator]\ccc{value_type} must be convertible to \ccc{Point_3}.\end{description}
\end{description}
\begin{description}
\item[Returns:]past-the-end output iterator. \end{description}
\begin{description}
\item[Parameters: ]
\begin{description}
\item[first]input points \item[output]output points \item[epsilon]tolerance value when comparing 3D points \end{description}
\end{description}
\ccGlue
\ccFunction{ForwardIterator merge_epsilon_nearest_points_3(ForwardIterator first, ForwardIterator beyond, double epsilon);}
{
Merge points which belong to the same cell of a grid of cell size = epsilon. This function is mutating the input point set. This variant deduces the kernel from iterator types.
Warning: This method modifies the order of points, thus Precondition: epsilon $>$ 0.
}
\ccGlue
\begin{description}
\item[Template Parameters:]
\begin{description}
\item[ForwardIterator]\ccc{value_type} must be convertible to \ccc{Point_3}.\end{description}
\end{description}
\begin{description}
\item[Returns:]First iterator to remove (see erase-remove idiom). \end{description}
\begin{description}
\item[Parameters: ]
\begin{description}
\item[first]input/output points \item[epsilon]tolerance value when comparing 3D points \end{description}
\end{description}
\ccGlue

%END-AUTO(\ccDefinition)
  
\ccInclude{Surface_reconstruction_3/merge_epsilon_nearest_points_3.h}

\ccParameters

The full template declarations are:

% The section below is automatically generated. Do not edit!
%START-AUTO(\ccParameters)

template$<$  \\
typename InputIterator,   \\
typename OutputIterator,   \\
typename Kernel$>$  \\
OutputIterator  \\
\ccc{merge_epsilon_nearest_points_3} (InputIterator first, InputIterator beyond, OutputIterator output, double epsilon, const Kernel \&);  \\
  \\
template$<$  \\
typename ForwardIterator,   \\
typename Kernel$>$  \\
ForwardIterator  \\
\ccc{merge_epsilon_nearest_points_3} (ForwardIterator first, ForwardIterator beyond, double epsilon, const Kernel \&);  \\
  \\
template$<$  \\
typename InputIterator,   \\
typename OutputIterator$>$  \\
OutputIterator  \\
\ccc{merge_epsilon_nearest_points_3} (InputIterator first, InputIterator beyond, OutputIterator output, double epsilon);  \\
  \\
template$<$  \\
typename ForwardIterator$>$  \\
ForwardIterator  \\
\ccc{merge_epsilon_nearest_points_3} (ForwardIterator first, ForwardIterator beyond, double epsilon);  \\

%END-AUTO(\ccParameters)

\ccExample

A short example program.
Instead of a short program fragment, a full running program can be
included using the 
\verb|\ccIncludeExampleCode{Surface_reconstruction_3/merge_epsilon_nearest_points_3.C}| 
macro. The program example would be part of the source code distribution and
also part of the automatic test suite.

\begin{ccExampleCode}
void your_example_code() {
}
\end{ccExampleCode}

%% \ccIncludeExampleCode{Surface_reconstruction_3/merge_epsilon_nearest_points_3.C}

\end{ccRefFunction}

% +------------------------------------------------------------------------+
%%RefPage: end of main body, begin of footer
\ccRefPageEnd
% EOF
% +------------------------------------------------------------------------+

