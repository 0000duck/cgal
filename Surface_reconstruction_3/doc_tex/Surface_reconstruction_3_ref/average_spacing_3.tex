% +------------------------------------------------------------------------+
% | Reference manual page: average_spacing_3.tex
% +------------------------------------------------------------------------+
% | 02.06.2008   Pierre Alliez, Laurent Saboret, Gael Guennebaud
% | Package: Surface_reconstruction_3
% |
\RCSdef{\RCSaveragespacingRev}{$Id$}
\RCSdefDate{\RCSaveragespacingDate}{$Date$}
% |
\ccRefPageBegin
%%RefPage: end of header, begin of main body
% +------------------------------------------------------------------------+


\begin{ccRefFunction}{average_spacing_3}  %% add template arg's if necessary

%% \ccHtmlCrossLink{}     %% add further rules for cross referencing links
%% \ccHtmlIndexC[function]{} %% add further index entries

\ccDefinition

\ccc{CGAL::average_spacing_3()} computes the average spacing in a point set from the K nearest neighbors.

The \ccc{CGAL::average_spacing_3()} function exists in two flavors.
The complete version requires the kernel to use for computations.
The main version deduces the kernel from input parameters.

\ccInclude{CGAL/average_spacing_3.h}

% The section below is automatically generated. Do not edit!
%START-AUTO(\ccDefinition)

\ccFunction{Kernel::FT average_spacing_3(InputIterator first, InputIterator beyond, unsigned int KNN, const Kernel& );}
{
Compute average spacing from K nearest neighbors. This variant requires the kernel.
\ccPrecond KNN $>$= 2.
\ccCommentHeading{Template Parameters}
\begin{description}
\item \ccc{InputIterator}: \ccc{value_type} is \ccc{Point_3}. \item \ccc{Kernel}: Geometric traits class.\end{description}
\ccCommentHeading{Returns} average spacing (scalar).
\ccCommentHeading{Parameters}
\begin{description}
\item \ccc{first}: input points \item \ccc{KNN}: number of neighbors \end{description}
}
\ccGlue
\ccFunction{FT average_spacing_3(InputIterator first, InputIterator beyond, unsigned int KNN);}
{
Compute average spacing from K nearest neighbors. This variant deduces the kernel from iterator types.
\ccPrecond KNN $>$= 2.
\ccCommentHeading{Template Parameters}
\begin{description}
\item \ccc{InputIterator}: \ccc{value_type} is \ccc{Point_3}. \item \ccc{FT}: number type.\end{description}
\ccCommentHeading{Returns} average spacing (scalar).
\ccCommentHeading{Parameters}
\begin{description}
\item \ccc{first}: input points \item \ccc{KNN}: number of neighbors \end{description}
}
\ccGlue

%END-AUTO(\ccDefinition)

\ccExample

\ccIncludeExampleCode{Surface_reconstruction_3/average_spacing_example.cpp}

\end{ccRefFunction}

% +------------------------------------------------------------------------+
%%RefPage: end of main body, begin of footer
\ccRefPageEnd
% EOF
% +------------------------------------------------------------------------+

