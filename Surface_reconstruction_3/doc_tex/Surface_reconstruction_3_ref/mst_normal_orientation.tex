% +------------------------------------------------------------------------+
% | Reference manual page: mst_normal_orientation.tex
% +------------------------------------------------------------------------+
% | 02.06.2008   Pierre Alliez, Laurent Saboret, Gael Guennebaud
% | Package: Surface_reconstruction_3
% |
\RCSdef{\RCSmstnormalorientationRev}{$Id$}
\RCSdefDate{\RCSmstnormalorientationDate}{$Date$}
% |
\ccRefPageBegin
%%RefPage: end of header, begin of main body
% +------------------------------------------------------------------------+


\begin{ccRefFunction}{mst_normal_orientation}  %% add template arg's if necessary

%% \ccHtmlCrossLink{}     %% add further rules for cross referencing links
%% \ccHtmlIndexC[function]{} %% add further index entries

\ccDefinition

\ccc{CGAL::mst_normal_orientation()} orients the normals of a point set using the method described by Hoppe, DeRose, Duchamp, McDonald and Stuetzle in {\em Surface reconstruction from unorganized points} \cite{cgal:hddms-srup-92}.
The result is an oriented normal vector for each input point/normal.

\ccInclude{CGAL/mst_normal_orientation.h}

% The section below is automatically generated. Do not edit!
%START-AUTO(\ccDefinition)

\ccFunction{unsigned int mst_normal_orientation(VertexIterator first, VertexIterator beyond, VertexIndexMap vertex_index_map, VertexPointMap vertex_point_map, VertexNormalMap vertex_normal_map, unsigned int KNN);}
{
Orient the normals of a point set using the method described by Hoppe, DeRose, Duchamp, McDonald and Stuetzle in {\em Surface reconstruction from unorganized points} \cite{cgal:hddms-srup-92}.
This variant implements the original algorithm. Note that it does not orient normals that are already oriented.
\ccPrecond\begin{itemize}
\item VertexIterator is a model of ForwardIterator.\item VertexIndexMap is a model of \ccc{boost::readable_property_map}.\item VertexPointMap is a model of \ccc{boost::readable_property_map}.\item VertexNormalMap is a model of \ccc{boost::lvalue_property_map}.\item Normals must be unit vectors.\item KNN $>$= 2.\end{itemize}
\ccCommentHeading{Returns} the number of un-oriented normals.
\ccCommentHeading{Parameters}
\begin{description}
\item \ccc{first}: first input vertex \item \ccc{beyond}: past-the-end input vertex \item \ccc{vertex_index_map}: property map VertexIterator -$>$ index \item \ccc{vertex_point_map}: property map VertexIterator -$>$ \ccc{Point_3} \item \ccc{vertex_normal_map}: property map VertexIterator -$>$ Normal (in and out) \item \ccc{KNN}: number of neighbors \end{description}
}
\ccGlue
\ccFunction{unsigned int mst_normal_orientation(VertexIterator first, VertexIterator beyond, VertexIndexMap vertex_index_map, VertexPointMap vertex_point_map, VertexNormalMap vertex_normal_map, unsigned int KNN, double angle_max);}
{
Orient the normals of a point set using the method described by Hoppe, DeRose, Duchamp, McDonald and Stuetzle in {\em Surface reconstruction from unorganized points} \cite{cgal:hddms-srup-92}.
This is a variant of the original algorithm. It:\begin{itemize}
\item orients the top point towards +Z axis.\item does not orient normals that are already oriented.\item does not propagate the orientation if the angle between 2 normals $>$ \ccc{angle_max}.\end{itemize}
\ccPrecond\begin{itemize}
\item VertexIterator is a model of ForwardIterator.\item VertexIndexMap is a model of \ccc{boost::readable_property_map}.\item VertexPointMap is a model of \ccc{boost::readable_property_map}.\item VertexNormalMap is a model of \ccc{boost::lvalue_property_map}.\item Normals must be unit vectors.\item KNN $>$= 2.\item 0 $<$ \ccc{angle_max} $<$= PI/2.\end{itemize}
\ccCommentHeading{Returns} the number of un-oriented normals.
\ccCommentHeading{Parameters}
\begin{description}
\item \ccc{first}: first input vertex \item \ccc{beyond}: past-the-end input vertex \item \ccc{vertex_index_map}: property map VertexIterator -$>$ index \item \ccc{vertex_point_map}: property map VertexIterator -$>$ \ccc{Point_3} \item \ccc{vertex_normal_map}: property map VertexIterator -$>$ Normal (in and out) \item \ccc{KNN}: number of neighbors \item \ccc{angle_max}: max angle to propagate the normal orientation (radians) \end{description}
}
\ccGlue

%END-AUTO(\ccDefinition)

\ccSeeAlso

\ccRefIdfierPage{CGAL::pca_normal_estimation}  \\
\ccRefIdfierPage{CGAL::jet_normal_estimation}  \\

\ccExample

\ccIncludeExampleCode{Surface_reconstruction_3/pca_normal_estimation_example.cpp}

\end{ccRefFunction}

% +------------------------------------------------------------------------+
%%RefPage: end of main body, begin of footer
\ccRefPageEnd
% EOF
% +------------------------------------------------------------------------+

