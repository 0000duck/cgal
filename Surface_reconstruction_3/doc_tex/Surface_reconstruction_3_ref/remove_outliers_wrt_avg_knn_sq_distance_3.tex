% +------------------------------------------------------------------------+
% | Reference manual page: remove_outliers_wrt_avg_knn_sq_distance_3.tex
% +------------------------------------------------------------------------+
% | 02.06.2008   Laurent Saboret, Pierre Alliez
% | Package: Surface_reconstruction_3
% |
\RCSdef{\RCSremoveoutlierswrtavgknnsqdistanceRev}{$Id$}
\RCSdefDate{\RCSremoveoutlierswrtavgknnsqdistanceDate}{$Date$}
% |
\ccRefPageBegin
%%RefPage: end of header, begin of main body
% +------------------------------------------------------------------------+


\begin{ccRefFunction}{remove_outliers_wrt_avg_knn_sq_distance_3}  %% add template arg's if necessary

%% \ccHtmlCrossLink{}     %% add further rules for cross referencing links
%% \ccHtmlIndexC[function]{} %% add further index entries

\ccDefinition

% The section below is automatically generated. Do not edit!
%START-AUTO(\ccDefinition)

\ccFunction{OutputIterator remove_outliers_wrt_avg_knn_sq_distance_3(InputIterator first, InputIterator beyond, OutputIterator output, unsigned int KNN, const Kernel &, double threshold_percent);}
{
Remove outliers:\begin{itemize}
\item compute average squared distance to the K nearest neighbors,\item percentage of points to remove. This variant requires the kernel.\end{itemize}
Precondition: KNN $>$= 2.
}
\ccGlue
\begin{description}
\item[Template Parameters:]
\begin{description}
\item[InputIterator]\ccc{value_type} is \ccc{Point_3}. \item[OutputIterator]\ccc{value_type} is \ccc{Point_3}. \item[Kernel]Geometric traits class.\end{description}
\end{description}
\begin{description}
\item[Returns:]past-the-end output iterator. \end{description}
\begin{description}
\item[Parameters: ]
\begin{description}
\item[first]input points \item[output]output points \item[KNN]number of neighbors \item[\ccc{threshold_percent}]percentage of points to remove \end{description}
\end{description}
\ccGlue
\ccFunction{void remove_outliers_wrt_avg_knn_sq_distance_3(ForwardIterator first, ForwardIterator beyond, unsigned int KNN, const Kernel &, double threshold_percent);}
{
Remove outliers:\begin{itemize}
\item compute average squared distance to the K nearest neighbors,\item percentage of points to remove. This function is mutating the input point set. This variant requires the kernel.\end{itemize}
Precondition: KNN $>$= 2.
}
\ccGlue
\begin{description}
\item[Template Parameters:]
\begin{description}
\item[ForwardIterator]\ccc{value_type} is \ccc{Point_3}. \item[Kernel]Geometric traits class. \end{description}
\end{description}
\begin{description}
\item[Parameters: ]
\begin{description}
\item[first]input/output points \item[KNN]number of neighbors \item[\ccc{threshold_percent}]percentage of points to remove \end{description}
\end{description}
\ccGlue
\ccFunction{OutputIterator remove_outliers_wrt_avg_knn_sq_distance_3(InputIterator first, InputIterator beyond, OutputIterator output, unsigned int KNN, double threshold_percent);}
{
Remove outliers:\begin{itemize}
\item compute average squared distance to the K nearest neighbors,\item percentage of points to remove. This variant deduces the kernel from iterator types.\end{itemize}
Precondition: KNN $>$= 2.
}
\ccGlue
\begin{description}
\item[Template Parameters:]
\begin{description}
\item[InputIterator]\ccc{value_type} is \ccc{Point_3}. \item[OutputIterator]\ccc{value_type} is \ccc{Point_3}.\end{description}
\end{description}
\begin{description}
\item[Returns:]past-the-end output iterator. \end{description}
\begin{description}
\item[Parameters: ]
\begin{description}
\item[first]input points \item[output]output points \item[KNN]number of neighbors \item[\ccc{threshold_percent}]percentage of points to remove \end{description}
\end{description}
\ccGlue
\ccFunction{void remove_outliers_wrt_avg_knn_sq_distance_3(ForwardIterator first, ForwardIterator beyond, unsigned int KNN, double threshold_percent);}
{
Remove outliers:\begin{itemize}
\item compute average squared distance to the K nearest neighbors,\item percentage of points to remove. This function is mutating the input point set. This variant deduces the kernel from iterator types.\end{itemize}
Precondition: KNN $>$= 2.
}
\ccGlue
\begin{description}
\item[Template Parameters:]
\begin{description}
\item[ForwardIterator]\ccc{value_type} is \ccc{Point_3}. \end{description}
\end{description}
\begin{description}
\item[Parameters: ]
\begin{description}
\item[first]input/output points \item[KNN]number of neighbors \item[\ccc{threshold_percent}]percentage of points to remove \end{description}
\end{description}
\ccGlue

%END-AUTO(\ccDefinition)
  
\ccInclude{Surface_reconstruction_3/remove_outliers_wrt_avg_knn_sq_distance_3.h}

\ccParameters

The full template declarations are:

% The section below is automatically generated. Do not edit!
%START-AUTO(\ccParameters)

template$<$  \\
typename InputIterator,   \\
typename OutputIterator,   \\
typename Kernel$>$  \\
OutputIterator  \\
\ccc{remove_outliers_wrt_avg_knn_sq_distance_3} (InputIterator first, InputIterator beyond, OutputIterator output, unsigned int KNN, const Kernel \&, double \ccc{threshold_percent});  \\
  \\
template$<$  \\
typename ForwardIterator,   \\
typename Kernel$>$  \\
void  \\
\ccc{remove_outliers_wrt_avg_knn_sq_distance_3} (ForwardIterator first, ForwardIterator beyond, unsigned int KNN, const Kernel \&, double \ccc{threshold_percent});  \\
  \\
template$<$  \\
typename InputIterator,   \\
typename OutputIterator$>$  \\
OutputIterator  \\
\ccc{remove_outliers_wrt_avg_knn_sq_distance_3} (InputIterator first, InputIterator beyond, OutputIterator output, unsigned int KNN, double \ccc{threshold_percent});  \\
  \\
template$<$  \\
typename ForwardIterator$>$  \\
void  \\
\ccc{remove_outliers_wrt_avg_knn_sq_distance_3} (ForwardIterator first, ForwardIterator beyond, unsigned int KNN, double \ccc{threshold_percent});  \\

%END-AUTO(\ccParameters)

\ccExample

A short example program.
Instead of a short program fragment, a full running program can be
included using the 
\verb|\ccIncludeExampleCode{Surface_reconstruction_3/remove_outliers_wrt_avg_knn_sq_distance_3.C}| 
macro. The program example would be part of the source code distribution and
also part of the automatic test suite.

\begin{ccExampleCode}
void your_example_code() {
}
\end{ccExampleCode}

%% \ccIncludeExampleCode{Surface_reconstruction_3/remove_outliers_wrt_avg_knn_sq_distance_3.C}

\end{ccRefFunction}

% +------------------------------------------------------------------------+
%%RefPage: end of main body, begin of footer
\ccRefPageEnd
% EOF
% +------------------------------------------------------------------------+

