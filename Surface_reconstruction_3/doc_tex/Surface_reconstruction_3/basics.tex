\section{Basics}


\subsection{Poisson Reconstruction Overview}

Kazhdan, Bolitho and Hoppe introduced the Poisson Surface Reconstruction algorithm \cite{Kazhdan06}.
Given a set of 3D points with oriented normals (denoted oriented points in the sequel)
sampled on the boundary of a 3D solid, this method solves for an approximate indicator function of the inferred solid,
whose gradient best matches the input normals. The output scalar function, represented in an adaptive octree,
is then iso-contoured using an adaptive marching cubes.\\

\cgal\ implements a variant of this algorithm which solves for a piecewise linear function
on a 3D Delaunay triangulation instead of an adaptive octree.
The algorithm takes as input a set of 3D oriented points.
It builds a 3D Delaunay triangulation from these points and refines it by Delaunay refinement
so as to remove all badly shaped (non isotropic) tetrahedra and to tessellate a loose bounding box
of the input oriented points. The normal of each Steiner point added during refinement is set to zero.
It then solves for a scalar indicator function $f$ represented as a piecewise linear function over the refined triangulation.
More specifically, it solves for the Poisson equation  $\Delta f = div(\mathbf{n})$ at each vertex of the triangulation
using a sparse linear solver.
Eventually, the \cgal\ surface mesh generator extracts an isosurface with function value set by default
to be the median value of $f$ at all input points.

\subsection{Poisson Reconstruction API}

The class template declaration is:

template$<$  \\
class Gt,   \\
class \ccc{ReconstructionTriangulation_3}$>$   \\
class \ccc{Poisson_reconstruction_function};
\ccGlue
\ccCommentHeading{Parameters}
\begin{description}
\item \ccc{Gt}: Geometric traits class \item \ccc{ReconstructionTriangulation_3}: 3D Delaunay triangulation, model of \ccc{ReconstructionTriangulation_3} concept. \end{description}

The main constructor is:

\ccFunction{template<class InputIterator> Poisson_reconstruction_function(ReconstructionTriangulation_3& pdt, InputIterator first, InputIterator beyond);}
{
Creates a scalar function from a set of oriented points. Inserts the iterator range first...beyond into the triangulation pdt, refines it and solves for a piecewise linear scalar function which gradient best matches the input normals.
\ccPrecond the value type of InputIterator must be convertible to \ccc{Point_with_normal}.
\ccCommentHeading{Parameters}
\begin{description}
\item \ccc{pdt}: \ccc{ReconstructionTriangulation_3} base of the Poisson indicator function. \item \ccc{first}: First point to add. \item \ccc{beyond}: Past-the-end point to add. \end{description}
}

The main methods are:

\ccFunction{bool compute_implicit_function();}
{
The function \ccc{compute_implicit_function}() must be called after each insertion of oriented points. It computes the piecewise linear scalar function $f$ by:
\begin{itemize}
\item applying Delaunay refinement.
\item solving for $f$ at each vertex of the triangulation with a sparse linear solver.
\item shifting and orienting $f$ such that $f=0$ at all input points and $f<0$ inside the inferred surface.
\end{itemize}
Returns false on error.
}
\ccGlue
\ccFunction{FT value(const Point& p) const;}
{
Evaluates the implicit function at a given 3D query point.
}
\ccGlue
\ccFunction{Point get_inner_point() const;}
{
Returns a point located inside the inferred surface.
}
\ccFunction{Sphere bounding_sphere() const;}
{
Returns a sphere bounding the inferred surface.
}

See details in \\
\ccRefIdfierPage{CGAL::Poisson_reconstruction_function<GeomTraits, ReconstructionTriangulation_3>}


\subsection{Poisson Reconstruction Example}

\ccc{poisson_reconstruction_example.cpp} reads a point set, creates a Poisson implicit function and reconstructs a surface.

% I feel like we should further simplify it by providing
% just a list of oriented points.
\ccIncludeExampleCode{Surface_reconstruction_3/poisson_reconstruction_example.cpp}


