\section{Software Design}

\subsection{Points and Normals}

We decided to implement the algorithms of the \\
\begin{itemize}
\item Input
\item Analysis
\item Processing
\item Output
\end{itemize}
steps as global functions that expect as input and output parameters range iterators over:

\begin{itemize}
\item 3D points
\item Normals (orientable 3D vectors)
\item 3D points with unoriented normals
\item 3D points with oriented normals
\end{itemize}

This leads to the next concepts to represent these types:

\begin{itemize}
\item \ccRefIdfierPage{Kernel::Point_3} \\
3D position.
\item \ccRefIdfierPage{Kernel::Vector_3} \\
3D vector.
\item \ccRefIdfierPage{OrientableNormal_3} \\
Normal vector (oriented or not). Inherits from \ccc{Kernel::Vector_3}.
\item \ccRefIdfierPage{PointWithNormal_3} \\
Inherits from \ccc{Kernel::Point_3} and contains a \ccc{Kernel::Vector_3} normal.
\item \ccRefIdfierPage{PointWithOrientableNormal_3} \\
Inherits from \ccc{Kernel::Point_3} and contains a \ccc{OrientableNormal_3} normal.
\end{itemize}

We provide several models of points and normals concepts, with different speed/space trade-offs:

\begin{itemize}
\item \ccRefIdfierPage{CGAL::Point_3<Geom_traits>} \\
CGAL regular 3D position.
\item \ccRefIdfierPage{CGAL::Vector_3<Geom_traits>} \\
CGAL regular 3D vector.
\item \ccRefIdfierPage{CGAL::Lightweight_vector_3<Geom_traits>} \\
3D vector allocated only if not (0,0,0).
Model of the \ccc{Kernel::Vector_3} concept.
\item \ccRefIdfierPage{CGAL::Orientable_normal_3<Geom_traits>} \\
Normal (oriented or not).
Model of the \ccc{OrientableNormal_3} concept.
Inherits from \ccc{Vector_3<Geom_traits>} and contains an "is normal oriented?" flag.
\item \ccRefIdfierPage{CGAL::Point_with_normal_3<Geom_traits, Normal_3>} \\
3D position + normal.
Model of the \ccc{PointWithNormal_3} concept.
Also model of \ccc{PointWithOrientableNormal_3} if \ccc{Normal_3} is a model of \ccc{OrientableNormal_3}.
\end{itemize}


\subsection{Reconstruction}

The package entry point is the contouring algorithm, CGAL Surface Mesh Generator,
templated by an implicit function.

We plan to add to this package a Marching Cubes implementation and
several Delaunay-based reconstruction algorithms.
As these algorithms would be the entry points of the package and are quite
different, we plan to make them a set of independent classes or functions
with no common concept.

We define a ReconstructionImplicitFunction concept that will provide
a common interface to the Marching Cubes and the Surface Mesh Generator:

\ccRefIdfierPage{ReconstructionImplicitFunction}  \\

Eventually, the \ccc{ImplicitFctDelaunayTriangulation_3} concept defines the interface of a 3D Delaunay triangulation requested by the \ccc{Poisson_implicit_function} class:

\ccRefIdfierPage{ImplicitFctDelaunayTriangulation_3}  \\


\subsection{Output}

Ideally, we should define a concept for the output of all reconstruction algorithms.
This is not obvious as the current implementations have incompatible
outputs. In the worst case, we will end up with outputs specific to
each contouring algorithm.

Candidates are:

\begin{itemize}
\item a polygon soup
\item an organized surface like a BGL Graph or CGAL Polyhedron
\item \ccc{SurfaceMeshComplex_2InTriangulation_3} (Surface Mesh Generator's
native output format)
\end{itemize}

