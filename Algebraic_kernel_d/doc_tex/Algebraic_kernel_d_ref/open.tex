
\section{Questions and Remarks}

\section{Notes}
\begin{itemize}
 \item Events at infinity are only important for a single curve. Since
   inf doesn't play a role in CPA\_2 we discuss it for the sake of
   simplicity as a special case in CA\_2 only. 
\item AlgebraicReal\_1 - refine: with relative precision is enough. \\
      Moreover, you can use 'refine() until' for absolute prec.
\item SquareFreeFactorization is not returning an additional constant 
      factor. This prevents us from providing an extra interface for 
      algebraic coefficients.
\item Is Derivative and XCriticalPoints/YCriticalPoints are not
   redundant. We decided to keep them both because it is possible that
   XCriticalPoints/YCriticalPoints use a different methodology, other
   than residing to the Derivative functor.\\ As Michael pointed out:
   \emph{``I think we should keep it, since it is more abstract than
     $solve(derivative(p),p)$.  This would again move the {\em
       AlgebraicKernel} towards a more abstract layer.''}\\
\texttt{(Monique: I am not convinced any more about this now... the 
definition of x-crit point is when the derivative vanishes or is not
defined. for polynomials it is identical. so, we could remove
derivatives and keep only the (maybe) more abstract crit points)}
\item MakeCoprime: changed return type to bool\\
      because IsCoprime is almost as costly as computing the coprime
      parts iff the polynomials are not coprime. This avoids caching.
\end{itemize}

\section{Questions}
\begin{itemize}
\item AlgebraicReal\_2 - Replace .x() and .y() methods by GetX() and 
	GetY() functors? (or just add the functors?)
\item problem is due to \ccc{AK_1}$<double>$.\\
    The Boundary type is documented as dense in R, which is not true
    for double.\\ (Monique: But well, double is also documented as a
    field, while it is obviously not (no operation is associative
    due to rounding errors). So, the density problem is not worse
    than usual problems for double.)
\item Use "int" at all places, or stick to distinction of \texttt{unsigned
   int} + \texttt{int}?\\
   \texttt{(Menelaos: if the semantics of \texttt{unsigned int} is to enumerate
   objects in a sequence, then a new nested type \texttt{size\_type}
   can be introduced and used as the \texttt{return\_type} or
   \texttt{argument\_type}).}
\end{itemize}

\begin{itemize}
\item Move methods from AlgebraicReal\_2 to a traits class (?)
\end{itemize}

\section{done} 
\begin{itemize}
\item all functors for \ccc{PolynomialTraits\_d} \ccc{PolynomialToolBox\_d}
      are documented. \\ \texttt{(Monique asks: where...?)}
\end{itemize}

\section{todo} 
\begin{itemize}
\item {} answer questions above...
\item {} finish check and improve consistency between AK1 and AK2 
(list of available functionality, order of functors, etc)
\item {} order manual pages in main.tex and list of functors in intro.tex in a more systematic way
\end{itemize} 

