
\section{Questions and Remarks}

\subsection{Design issues}

\begin{itemize}
\item 
Removed \ccc{AlgebraicRealTraits_?} from Kernel:\\
This flat design is consistent with the flat design of the linear kernel. 
(Michael)
\item 
Removed C'tor for Polynomials from Algebraic Kernels. 
This is provided by \ccc{Polynomial_traits_d}.
(Michael)
\item 
Functors DeriveX* removed from the Kernel since covered by 
\ccc{Polynomial_traits_d}.
(Michael)\\
It is recommended to use XCriticalPoints instead of \ccc{solve(Derive_x_2(p),p)} 
anyway, since this is more abstract. (all)\\
removed \ccc{Has_critical_points,} that is, 
\ccc{X_critical_points} and \ccc{Y_critical_points} are not optional.
(Michael)
\item
Other Functors such as SquareFree* are part of the Kernel since they are not 
identical to those provided in the \ccc{Polynomial_tool_box_d}, that is, 
the functors in the 
kernel don't care about constant factors, which is intended. In this way we 
don't have to care about the algebraic category of the coefficient type. 
(Michael)
\item 
MakeCoprime: changed return type to bool\\
because IsCoprime is almost as costly as computing the coprime
parts iff the polynomials are not coprime. This avoids caching. 
%\item 
%\ccc{AlgebraicReal\_1} - refine: with relative precision is enough. \\
%Moreover, you can use 'refine() until' for absolute prec.
%(Michael)
\item
We keep \ccc{Compare_1}, for consistency with \ccc{AlgebraicKernel_2}.
(Michael)
\item 
\ccc{AlgebraicReal_2}: removed .x() .y(), replaced by GetX() and 
GetY() functors.
(Michael)
\item \ccc{AlgebraicReal_2}: Added \ccc{LessThanComparable} with respect 
to lex-order. This improves consistency with \ccc{AlgebraicReal\_1}. 
(Michael)
\item
The Boundary type is documented as dense in R, which is not true
for \ccc{AK_1<double>}.\\ 
This is ok, since double is also documented as a
field, while it is obviously not (no operation is associative
due to rounding errors). Hence, the density problem is not worse
than usual problems for double. (Monique)
\item 
Events at infinity are only important for a single curve. Since
inf doesn't play a role in CPA\_2 we discuss it for the sake of
simplicity as a special case in CA\_2 only. 
\end{itemize}

\subsection{Questions}
\begin{itemize}
\item Use "int" at all places, or stick to distinction of \texttt{unsigned
   int} + \texttt{int}?\\
   \texttt{(Menelaos: if the semantics of \texttt{unsigned int} is to enumerate
   objects in a sequence, then a new nested type \texttt{size\_type}
   can be introduced and used as the \texttt{return\_type} or
   \texttt{argument\_type}).}
\end{itemize}

\subsection{todo} 
\begin{itemize}
\item 
Rename Functors in Polynomial package according to Ron's 
suggestions. (Michael)
\item {} answer questions above...
\item {} finish check 
\end{itemize} 

\subsection{done} 
\begin{itemize}
\item 
improved consistency between AK1 and AK2 (Michael)
\item 
order manual pages in main.tex and list of functors in intro.tex 
in a more systematic way (Michael)

\end{itemize}
\footnote{to be added: sparse polynomials}
