\begin{ccRefFunctionObjectConcept}{AlgebraicKernel_d_1::BoundaryBetween_1}

\ccDefinition
Computes a number of type 
\ccc{AlgebraicKernel_d_1::Boundary} in-between two 
\ccc{AlgebraicKernel_d_1::AlgebraicReal_1}.

\ccRefines 
\ccc{AdaptableBinaryFunction} 

\ccTypes
\ccThree{typedef AlgebraicKernel_d_1::Algebraic_real_1}{second_argument_type+}{}
\ccTypedef{typedef AlgebraicKernel_d_1::Boundary         result_type;}{}
\ccGlue
\ccTypedef{typedef AlgebraicKernel_d_1::Algebraic_real_1  first_argument_type;}{}
\ccGlue
\ccTypedef{typedef AlgebraicKernel_d_1::Algebraic_real_1  second_argument_type;}{}

\ccOperations
\ccCreationVariable{fo}
A model \ccVar\ of this type must provide:

\ccThree{result_type}{fo(first_argument_type,++}{}
\ccMethod{
result_type
operator()(const first_argument_type & x,
           const second_argument_type & y);}{
Computes a number of type \ccc{AlgebraicKernel_d_1::Boundary}
in-between $x$ and $y$.} 
 
\ccSeeAlso
\ccRefIdfierPage{AlgebraicKernel_d_1::AlgebraicReal_1}\\
\ccRefIdfierPage{AlgebraicKernel_d_1::Boundary}\\
\ccRefIdfierPage{AlgebraicKernel_d_1::LowerBoundary_1}\\
\ccRefIdfierPage{AlgebraicKernel_d_1::UpperBoundary_1}\\

\end{ccRefFunctionObjectConcept}
