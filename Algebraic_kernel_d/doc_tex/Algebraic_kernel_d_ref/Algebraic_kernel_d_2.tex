\begin{ccRefClass}{Algebraic_kernel_d_2<Coeff>}
%\label{Algebraic_kernel_d_2}

\ccInclude{CGAL/Algebraic_kernel_d_2.h}

\ccDefinition

This class is based on an algorithm computing a 
geometric-topological analysis of a single curve~\cite{ekw-fast-07} and of a
pair of curves~\cite{ek-exact-08}.
The main idea behind both analyses is to compute the critical
x-coordinates of curves and curve pairs by projection (resultants), and compute
additional information about the critical fibers using subresultants
and Sturm-Habicht sequences~\cite{grlr-sturm-habicht-98}. 
With that information, the fiber at    
critical x-coordinates is computed by a variant of the Bitstream
Descartes method.
See also \cite{kerber-phd-09} for a comprehensive description of
these techniques. 

A point $p$ of type \ccc{Algebraic_real_2} is represented
by its $x$-coordinate $x_0$ (as described in the \ccc{Algebraic_kernel_d_1}
paragraph above), an algebraic curve where $p$ lies on, and an
integer $i$, denoting that $p$ is the $i$th point in the fiber at $x_0$,
counted from the bottom (ignoring a possible vertical line at $x_0$).
This determines the point uniquely, but the $y$-coordinate 
is not stored internally in terms of an \ccc{Algebraic_real_1} object.
Querying such a representation by calling \ccc{Compute_y_2} is a
time-consuming step, and should be avoided for efficiency reasons if possible.
Note that this representation is not exposed in the interface. 

The template argument \ccc{Coeff} determines the coefficient type of the 
kernel, which is also the innermost coefficient type of the supported polynomials.  

Currently, the following coefficient types are supported:\\
-- \ccc{Gmpz}, \ccc{Gmpq}, (requires configuration with external libraries GMP, MPFR and MPFI)\\
-- \ccc{CORE::BigInt}, \ccc{CORE::BigRat}, (requires configuration with external library GMP) \\ 
-- \ccc{leda_integer}, \ccc{leda_rational}. (requires configuration with external library LEDA)\\

\begin{ccAdvanced}
The template argument type can also be set to \ccc{Sqrt_extension<NT,ROOT>}, where \ccc{NT} 
is one of the types listed above. \ccc{ROOT} should be one of the integer types. 
See also the documentation of \ccc{Sqrt_extension<NT,ROOT>}. 
\end{ccAdvanced}

\ccIsModel
\ccc{AlgebraicKernel_d_2}.

\ccTypes \ccThree{}{+++++++++++++}{++++++++}

\ccNestedType{Coefficient}{Same type as the template argument \ccc{Coeff}. }

\ccNestedType{Polynomial_2}{A model of \ccc{AlgebraicKernel_d_2::Polynomial_2}}. 

\ccNestedType{Algebraic_real_2}{A model of \ccc{AlgebraicKernel_d_2::AlgebraicReal_2}}

\ccNestedType{Bound}{The choice of \ccc{Coeff} also determines the provided bound, type. 
In case of \ccc{Coeff} is 
- \ccc{Gmpz} or \ccc{Gmpq} this is \ccc{Gmpq}  \\
- \ccc{CORE::BigInt} or \ccc{CORE::BigInt} this is \ccc{CORE::BigRat}  \\
- \ccc{leda_integer} or \ccc{leda_integer} this is \ccc{leda_rational} \\
}

\ccNestedType{Multiplicity_type}{The multiplicity type is \ccc{int}.}

\ccSeeAlso
\ccRefConceptPage{AlgebraicKernel_d_1}\\
\ccRefConceptPage{AlgebraicKernel_d_2}\\
\ccRefConceptPage{Polynomial_d}\\
\ccRefIdfierPage{CGAL::Algebraic_kernel_d_2<Coeff>}

\end{ccRefClass}
