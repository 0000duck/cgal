\begin{ccRefFunctionObjectConcept}{AlgebraicKernel_d_2::Isolate_2}

\ccDefinition
Computes an isolating box for a given \ccc{AlgebraicKernel_d_2::Algebraic_real_2}.
 
\ccRefines 
\ccc{AdaptableFunctor}

\ccTypes \ccThree{}{+++++++++++++}{++++++++} 
\ccTypedef{typedef CGAL::cpp0x::array<AlgebraicKernel_d_1::Bound, 4> result_type;}{}

\ccOperations \ccThree{+++++++++}{+++++++++}{}
\ccCreationVariable{fo}

\ccMethod{
result_type
operator()( AlgebraicKernel_d_2::Algebraic_real_2 a,  AlgebraicKernel_d_2::Polynomial_2 f);}
{The returned \ccc{CGAL::cpp0x::array} $[xl,xu,yl,yu]$ represents an open isolating box $B=(xl,xu)\times(yl,yu)$ 
for $a$ with respect to $f$.
\ccPrecond{$f(a)\neq 0$ }
\ccPostcond{$a \in B$. } 
\ccPostcond{$\{ r | f(r)=0 \} \cap \overline{B} = \emptyset$.}
}

\ccMethod{
result_type   
operator()( 
        AlgebraicKernel_d_2::Algebraic_real_2 a,  
        AlgebraicKernel_d_2::Polynomial_2     f,
        AlgebraicKernel_d_2::Polynomial_2     g);}
{The returned \ccc{CGAL::cpp0x::array} $[xl,xu,yl,yu]$ represents an open isolating box $B=(xl,xu)\times(yl,yu)$ 
for $a$ with respect to the common solutions of $f$ and $g$. 
It is not necessary that $a$ is a common solution of $f$ and $g$. 
\ccPostcond{ $a \in B$. } 
\ccPostcond{ $\{ r | f(r)=g(r)=0 \} \cap \overline{B} \in \{\{a\},\emptyset\}$.}
}  

%\ccHasModels
\ccSeeAlso
\ccRefIdfierPage{AlgebraicKernel_d_2::IsolateX_2}\\
\ccRefIdfierPage{AlgebraicKernel_d_2::IsolateY_2}\\
\ccRefIdfierPage{AlgebraicKernel_d_2::ComputePolynomialX_2}\\
\ccRefIdfierPage{AlgebraicKernel_d_2::ComputePolynomialY_2}\\

\end{ccRefFunctionObjectConcept}



\begin{ccRefFunctionObjectConcept}{AlgebraicKernel_d_2::IsolateX_2}

\ccDefinition
Computes an isolating interval for the first coordinate of an \ccc{AlgebraicKernel_d_2::Algebraic_real_2}
 with respect to the real roots of a univariate polynomial.
 
\ccRefines 
\ccc{AdaptableBinaryFunction}  

\ccTypes \ccThree{}{+++++++++++++}{++++++++}
\ccTypedef{typedef std::pair<AlgebraicKernel_d_2::Bound,AlgebraicKernel_d_2::Bound> result_type;}{}
\ccTypedef{typedef AlgebraicKernel_d_2::Algebraic_real_2 first_argument_type;}{}
\ccTypedef{typedef AlgebraicKernel_d_2::Polynomial_1 second_argument_type;}{}

\ccOperations \ccThree{+++++++++}{+++++++++}{}
\ccCreationVariable{fo}

\ccMethod{result_type operator()(first_argument_type a, second_argument_type p);}
{Computes an open isolating interval $I=(l,u)$ for the first coordinate $x$ of $a$ with respect to the real roots of $p$.
It is not required that $x$ is a root of $p$. 
\ccPostcond{ $x \in I$. } 
\ccPostcond{ $p(\alpha) \neq 0 | \forall \alpha \in \overline{I}\backslash x$.}
}

%\ccHasModels
\ccSeeAlso
\ccRefIdfierPage{AlgebraicKernel_d_2::IsolateY_2}\\
\ccRefIdfierPage{AlgebraicKernel_d_2::ComputePolynomialX_2}\\
\ccRefIdfierPage{AlgebraicKernel_d_2::ComputePolynomialY_2}\\

\end{ccRefFunctionObjectConcept}


\begin{ccRefFunctionObjectConcept}{AlgebraicKernel_d_2::IsolateY_2}

\ccDefinition
 Computes an isolating interval for the second coordinate of an \ccc{AlgebraicKernel_d_2::Algebraic_real_2}
 with respect to the real roots of a univariate polynomial.

\ccRefines 
\ccc{AdaptableBinaryFunction}  

\ccTypes \ccThree{}{+++++++++++++}{++++++++}
\ccTypedef{typedef std::pair<AlgebraicKernel_d_2::Bound,AlgebraicKernel_d_2::Bound> result_type;}{}
\ccTypedef{typedef AlgebraicKernel_d_2::Algebraic_real_2 first_argument_type;}{}
\ccTypedef{typedef AlgebraicKernel_d_2::Polynomial_1 second_argument_type;}{}

\ccOperations \ccThree{+++++++++}{+++++++++}{}
\ccCreationVariable{fo}

\ccMethod{result_type operator()(first_argument_type a, second_argument_type p);}
{Computes an open isolating interval $I=(l,u)$ for the second coordinate $y$ of $a$ with respect to the real roots of $p$.
It is not required that $x$ is a root of $p$. 
\ccPostcond{ $y \in I$. } 
\ccPostcond{ $p(\alpha) \neq 0 | \forall \alpha \in \overline{I}\backslash y$.}
}
%\ccHasModels
\ccSeeAlso
\ccRefIdfierPage{AlgebraicKernel_d_2::IsolateX_2}\\
\ccRefIdfierPage{AlgebraicKernel_d_2::ComputePolynomialX_2}\\
\ccRefIdfierPage{AlgebraicKernel_d_2::ComputePolynomialY_2}\\

\end{ccRefFunctionObjectConcept}
