\begin{ccRefConcept}{AlgebraicKernel_d_2} 

\ccDefinition

The \ccc{AlgebraicKernel_d_2} concept is meant to provide the curved
kernel with all the algebraic functionalities on bivariate polynomials
required for the manipulation of arcs of algebraic curves of general degree
$d$ in $\R^2$.

\ccRefines
\ccc{AlgebraicKernel_d_1}

\ccTypes

A model of \ccc{AlgebraicKernel_d_2} is supposed to provide

\ccNestedType{Coefficient}{A model of \ccc{IntegralDomain}. }

\ccNestedType{Polynomial_2}{A model of \ccc{Polynomial_2}, for
bivariate polynomials on an \ccc{IntegralDomain} coefficient type.} 

\ccNestedType{Algebraic_real_2}{A model of
\ccc{AlgebraicKernel_d_2::AlgebraicReal_2}, for solutions of systems
of two bivariate polynomials of type \ccc{Polynomial_2}.}

\ccNestedType{Construct_polynomial_2}{
Must provide:\\
\ccc{template <class InputIterator>}\\ 
\ccc{Polynomial_2 operator()(InputIterator first, InputIterator last)}\\
that constructs a polynomial from a range of coefficients of type
\ccc{Coefficient}, given in the lexicographic order of monomial degrees.
}
\footnote{to be added: sparse polynomials}

\ccNestedType{Solve_2}{A model of the concept
\ccc{AlgebraicKernel_d_2::Solve_2}.} 

\ccNestedType{Sign_at_2}{A model of the concept
\ccc{AlgebraicKernel_d_2::SignAt_2}.}

\ccNestedType{Derivative_x_2}{A model of the concept
\ccc{AlgebraicKernel_d_2::DerivativeX_2}.} 

\ccNestedType{Derivative_y_2}{A model of the concept
\ccc{AlgebraicKernel_d_2::DerivativeY_2}.} 

\ccNestedType{X_critical_points_2}{A model of the concept 
\ccc{AlgebraicKernel_d_2::XCriticalPoints_2}.}
\ccGlue
\ccNestedType{Y_critical_points_2}{A model of the concept 
\ccc{AlgebraicKernel_d_2::YCriticalPoints_2}.}

\ccNestedType{Compare_x_2}{A model of the concept 
\ccc{AlgebraicKernel_d_2::CompareX_2}.}
\ccGlue
\ccNestedType{Compare_y_2}{A model of the concept 
\ccc{AlgebraicKernel_d_2::CompareY_2}.}
\ccGlue
\ccNestedType{Compare_xy_2}{A model of the concept 
\ccc{AlgebraicKernel_d_2::CompareXY_2}.}

\ccNestedType{IsCoprime_2}{A model of the concept 
\ccc{AlgebraicKernel_d_2::IsCoprime_2}.}
\ccGlue
\ccNestedType{IsSquareFree_2}{A model of the concept 
\ccc{AlgebraicKernel_d_2::IsSquareFree_2}.}

\ccNestedType{MakeSquareFree_2}{A model of the concept 
\ccc{AlgebraicKernel_d_2::MakeSquareFree_2}.}
\ccGlue
\ccNestedType{MakeCoprime_2}{A model of the concept 
\ccc{AlgebraicKernel_d_2::MakeCoprime_2}.}
\ccGlue
\ccNestedType{SquareFreeFactorization_2}{A model of the concept 
\ccc{AlgebraicKernel_d_2::SquareFreeFactorization_2}.}

\begin{ccAdvanced}

A bivariate polynomial can be seen an algebraic curve. 
Additional functors might be provided to implement the analysis of a single
curve or a pair of curves using an $y$-per-$x$-view.
In case a functor is not provided, it is set to \ccc{CGAL::Null_functor}.

\ccNestedType{CurveAnalysis_2}{A model of concept
\ccc{CurvePairAnalyis_2::CurveAnalysis_2}.}
\ccGlue
\ccNestedType{CurvePairAnalysis_2}{A model of concept
\ccc{AlgebraicKernel_d_2::CurvePairAnalysis_2}.}

\end{ccAdvanced}

\ccHasModels

\ccSeeAlso

\end{ccRefConcept}
