\begin{ccRefConcept}{Solution_1}

\ccDefinition

The concept \ccc{Solution_1} is meant to store a finite 
coordinate ($x$, or $y$) of a point on curve. 

\ccRefines
 \ccc{DefaultConstructible, Assignable, LessThanComparable}


\begin{ccAdvanced}
  
\ccTypes

\ccTypedef{typedef typename Solution_1::Boundary Boundary;}{A NT being able
to represent values between two Solution\_1}

\ccCreationVariable{sol1}

\ccAccessFunctions

Solution\_1 is required to have some functionality. The final question
where to put such methods is undecided. Possibilities are
member functions of AK, or traits like Algebraic\_structure\_traits that 
either exists in the AK, or outside of it. 

\ccMethod{Boundary between(Solution_1 s);}{
  returns a rational between \ccc{sol1} and \ccc{s}
  \ccPrecond{sol1 != s}
}


\ccMethod{Boundary lower_bound();}{
  Gives the lower bound of the number.
} 

\ccMethod{Boundary upper_bound();}{
  Gives the upper bound of the number.
} 

\ccMethod{void refine();}{
  Refines the representation.
} 

\ccMethod{void refine(int prec);}{
  Refines the representation to the given precision (binary digits 
  after point). Internally the precision can already be higher.
} 

\end{ccAdvanced}

Remark: Note that this concept only deals with the interface to upper
layers. There might be additional requirements for number types to 
implement a model of this concept.

\ccHasModels
\ccc{double, NiX::Algebraic_real}

\end{ccRefConcept}
