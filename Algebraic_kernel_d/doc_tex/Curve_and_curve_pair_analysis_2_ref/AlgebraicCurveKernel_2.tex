\begin{ccRefConcept}{AlgebraicKernelBase_2}

\ccDefinition
The \ccc{AlgebraicKernelBase_2} concept is meant to provide the curved
kernel with very fundamental algebraic functionalities on uni- and bivariate
polynomials.

\ccTypes

A model of \ccc{AlgebraicKernelBase_2} is supposed to provide

\ccNestedType{Coefficient}{A model of \ccc{IntegralDomain}. }

\ccNestedType{Polynomial_1}{A model of \ccc{Polynomial_1}, for
univariate polynomials when the \ccc{Coefficient} type is an 
\ccc{IntegralDomain}. It must also be possible to perform \ccc{Canonicalize, 
    GcdUpToConstantFactor, 
    IntegralDivUpToConstantFactor, 
    MakeSquareFreeUpToConstantFactor, and 
    SquareFreeFactorizationUpToConstantFactor. 
    Also needed some PolynomialConstruction concept.}
} 

\ccNestedType{Polynomial_2}{A model of \ccc{Polynomial_2}, for
  bivariate polynomials on an \ccc{IntegralDomain} coefficient type. 
  it must also be possible to perform \ccc{Canonicalize, 
    GcdUpToConstantFactor, 
    IntegralDivUpToConstantFactor, 
    MakeSquareFreeUpToConstantFactor, and 
    SquareFreeFactorizationUpToConstantFactor} on this type - maybe using a
  traits class. Also needed some PolynomialConstruction concept.
} 

\ccNestedType{Solution_1}{A model of
\ccc{AlgebraicKernelBase_2::Solution_1}, for real roots of univariate 
polynomials of type \ccc{Polynomial_1}.}

\ccNestedType{Solution_2}{A model of
\ccc{AlgebraicKernelBase_2::Solution_2}, for solutions of systems
of two bivariate polynomials of type \ccc{Polynomial_2}.}

\ccNestedType{Compare_x_2}{A model of the concept 
\ccc{AlgebraicKernel_d_2::CompareX_2}.}
\ccGlue
\ccNestedType{Compare_y_2}{A model of the concept 
\ccc{AlgebraicKernel_d_2::CompareY_2}.}
\ccGlue
\ccNestedType{Compare_xy_2}{A model of the concept 
\ccc{AlgebraicKernel_d_2::CompareXY_2}.}


\end{ccRefConcept}

\begin{ccRefConcept}{AlgebraicKernelBasic_2}
\footnote{Adapted the concepts \ccc{AlgebraicKernel_d_1} and 
\ccc{AlgebraicKernel_d_2} and re-assigned types/functors 
to Base and Basic part.}

\ccDefinition

The \ccc{AlgebraicKernelBasic_2} concept is meant to provide the curved
kernel with all the algebraic functionalities on uni- and bivariate polynomials
required for the manipulation of arcs of algebraic curves of general degree
$d$ in $\R^2$.

\ccNestedType{Solve_1}{A model of the concept
\ccc{AlgebraicKernelBasic_1::Solve_1}.} 

\ccNestedType{Sign_at_1}{A model of the concept
\ccc{AlgebraicKernelBasic_1::SignAt_1}.}

\ccNestedType{Derivative_1}{A model of the concept
\ccc{AlgebraicKernelBasic_1::Derivative}.} 

\ccNestedType{Compare_1}{A model of the concept 
\ccc{AlgebraicKernelBasic_1::Compare_1}.}

\textbf{Question: what is the righ interface with AK 
for \ccc{compare_y_at_x_right}?}

\ccNestedType{Solve_2}{A model of the concept
\ccc{AlgebraicKernelBasic_2::Solve_2}.} 

\ccNestedType{Sign_at_2}{A model of the concept
\ccc{AlgebraicKernelBasic_2::SignAt_2}.}

\ccNestedType{Derivative_x_2}{A model of the concept
\ccc{AlgebraicKernelBasic_2::DerivativeX_2}.} 

\ccNestedType{Derivative_y_2}{A model of the concept
\ccc{AlgebraicKernelBasic_2::DerivativeY_2}.} 

\ccNestedType{X_critical_points_2}{A model of the concept 
\ccc{AlgebraicKernelBasic_2::XCriticalPoints_2}.}
\ccGlue
\ccNestedType{Y_critical_points_2}{A model of the concept 
\ccc{AlgebraicKernelBasic_2::YCriticalPoints_2}.}

\end{ccRefConcept}

\begin{ccRefConcept}{AlgebraicKernelCCPA_2}

\ccDefinition

The \ccc{AlgebraicKernelCCPA_2} concept refines the \ccc{AlgebraicKernelBase_2}
concept with functionality on bivariate polynomials
required for the manipulation of arcs of algebraic curves of general degree
$\R^2$ using an $y$-per-$x$-view on the curves.

\ccRefines
\ccc{AlgebraicKernelBase_2}

\ccTypes

\ccNestedType{CurvePairAnalysis_2}{A model of
\ccc{AlgebraicKernelCCPA_2::CurvePairAnalysis_2}, 
for analysing a pair of curves
defined as two analyses of type \ccc{CurveAnalysis_2}.}

\ccNestedType{CurveAnalysis_2}{A model of
\ccc{CurvePairAnalyis_2::CurveAnalysis_2}, for analysing single
curves defined as bivariate polynomials of type \ccc{Polynomial_2}.}

{\small TODO: Boundary: RealEmbeddable that is 'dense R'??}

\end{ccRefConcept}
