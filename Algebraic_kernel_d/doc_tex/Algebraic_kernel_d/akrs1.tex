% TODO: remove references to Gmpfr and Gmpfi, since they will be part of CGAL.

\subsection{Models}

\subsubsection{Algebraic kernels based on RS}

The package offers two univariate algebraic kernels that are based on 
the library \rs{} \cite{cgal:r-rs}, namely \ccc{CGAL::Algebraic_kernel_d_1_RS_Gmpz}
and \ccc{CGAL::Algebraic_kernel_d_1_RS_Gmpq}. As the names indicate, 
the kernels are based on the library \rs{} \cite{cgal:r-rs} and support univariate 
polynomials over \ccc{CGAL::Gmpz} or \ccc{CGAL::Gmpq}, respectively. 

In general we encourage to use \ccc{CGAL::Algebraic_kernel_d_1_RS_Gmpz}
instead of \ccc{CGAL::Algebraic_kernel_d_1_RS_Gmpq}. This is caused by
the fact that the most efficient way to compute operations (such as gcd)
on polynomials with rational coefficients is to use the corresponding
implementation for polynomials with integer coefficients.  That is,
the \ccc{CGAL::Algebraic_kernel_d_1_RS_Gmpq} is slightly slower due to
overhead caused by the necessary conversions.  However, since this may
not always be a major issue the \ccc{CGAL::Algebraic_kernel_d_1_RS_Gmpq}
is provided for convenience.


The core of both kernels is the implementation of the interval Descartes 
algorithm~\cite{cgal:rz-jcam-04} of the library RS \cite{cgal:r-rs}, 
which is used to isolate the roots of the polynomial. 
The RS library restricts its attention to univariate integer
polynomials and some substantial gain of efficiency can be made by using a kernel
that does not follow the generic programming paradigm, by avoiding
interfaces between layers.  Specifically, the fact of working with
only a number type allows to optimize some polynomial operations
as well as memory handling.  The implementation of these kernels
make heavy use of the MPFR \cite{cgal:mt-mpfr} and MPFI \cite{cgal:r-mpfi}
libraries, and of their CGAL interfaces, \ccc{Gmpfr} and \ccc{Gmpfi}.
The algebraic numbers (roots of the polynomials) are represented
in the two RS kernels by a \ccc{Gmpfi} interval and a pointer to
the polynomial of which they are roots.  See \cite{cgal:lpt-wea-09} 
for more details on the implementation, tests of these kernels,
comparisons with other algebraic kernels and discussions about the
efficiency.  


%--------------------------------------------------
% \subsubsection{Installation of the kernel}
% 
% As said before, this kernel depends on several libraries. First of
% all, it requires CGAL to be compiled with GMP support. Secondly,
% this needs the libraries MPFI and RS.
% 
% MPFI can be obtained from \ccc{http://gforge.inria.fr/projects/mpfi/}.
% As for RS, it can be downloaded from
% \ccc{http://www.loria.fr/equipes/vegas/rs}, you should get the
% right package for your architecture and operating system. Once both
% libraries are installed, you may want to set the environment variables
% {\tt MPFI\_INC\_DIR} and {\tt RS\_INC\_DIR} pointing to the include
% directories and {\tt MPFI\_LIB\_DIR} and {\tt RS\_LIB\_DIR} pointing to
% the library directories.
% 
% In order to create the \ccc{cmake} script to compile a program using these
% kernels, you should use the example {\tt CMakeLists.txt} from the
% {\tt examples/Algebraic\_kernel\_d} directory.
%-------------------------------------------------- 

