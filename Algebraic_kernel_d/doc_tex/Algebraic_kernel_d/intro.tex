
%\section{General Layout of an Algebraic Kernel}
\section{Introduction}
Real solving of polynomials is a fundamental problem with a wide application range. 
This package is targeted to provide black-box implementations of state-of-the-art 
algorithms to determine, compare and approximate real roots of univariate polynomials
and bivariate polynomial systems. Such a black-box is called an {\bf Algebraic Kernel}. 
Since this package is aimed at providing more than one implementation, the interface of the algebraic kernels 
is expressed in concepts. The main concepts provided by this package are the 
\ccc{AlgebraicKernel_d_1} for univariate polynomials and \ccc{AlgebraicKernel_d_2} 
for bivariate polynomials systems, the latter being a refinement of the first. 

So far the package only provides models for the univariate kernel. Nevertheless, 
we also provide the concepts for the bivariate kernel, since this settles the interface
for upcoming implementations.

%The package introduces a concept for a univariate 
%\ccc{AlgebraicKernel_d_1} and a concept for bivariate \ccc{AlgebraicKernel_d_2}.
%\ccc{AlgebraicKernel_d_1} provides univariate root solving as well as comparison 
%and refinement of the resulting algebraic numbers. 
%\ccc{AlgebraicKernel_d_2} provides about the same for bivariate polynomial. 
%Beside the functionalities related to one bivariate polynomial, it is in 
%particular possible to solve a $2\times2$ system and to compare and approximate 
%the resulting intersection points. The computation of the sign of a 
%polynomial at given point is also provide.

\section{Univariate Algebraic Kernel}
%\subsection{Layout}
\subsubsection{Major types}
First of all, the univariate algebraic kernel provides construction, comparison and 
approximation of real roots of univariate polynomials. 
Thus, the major public types the \ccc{AlgebraicKernel_d_1} provides are: \\
\ccc{AlgebraicKernel_d_1::Polynomial_1} - the type representing univariate polynomials;\\
\ccc{AlgebraicKernel_d_1::Coefficient} - the coefficient type of these polynomials; \\
\ccc{AlgebraicKernel_d_1::Algebraic_real_1} - the type representing real roots;\\
\ccc{AlgebraicKernel_d_1::Bound} - the type which is used to approximate these algebraic reals, 
in particular, it is used to represent the boundaries of isolating intervals. \\

\subsubsection{Construction of Algebraic Real Numbers }

The kernel provides two different function objects to construct an
\ccc{AlgebraicKernel_d_1::Algebraic_real_1}. The most general way
is to use \ccc{AlgebraicKernel_d_1::Isolate_1}; the function object
takes a univariate polynomial and writes all real roots into a given
output iterator. It is also possible to retrieve the multiplicity of
each root. The second option is to construct one particular algebraic
real using \ccc{AlgebraicKernel_d_1::Construct_algebraic_real_1}.
This function object provides construction from the native int type, the
coefficient type as well as the bound type.  Moreover, it is possible
to construct an algebraic real by giving a polynomial and either an
isolating interval or the index of the root. A related function object
is \ccc{AlgebraicKernel_d_1::Number_of_solutions_1} computing the number
of real roots of a polynomial.

%Examples can be found in Section~\ref{CGAL::AK1::EG::Construct_algebraic_real_1}.


\subsubsection{ Comparison and Approximation of Algebraic Real Numbers}

An \ccc{AlgebraicKernel_d_1::Algebraic_real_1} is model of
\ccc{RealEmbeddable}, for instance, it is possible to compare two
algebraic reals, to determine the sign of an algebraic real or to ask
for its double approximation, see also section \ref{sec:RealEmbeddable}.
Moreover, there is \ccc{AlgebraicKernel_d_1::Compare_1} which provides
comparison with int, the coefficient type and the bound type.

There are several ways to approximate an \ccc{AlgebraicKernel_d_1::Algebraic_real_1}:\\
\ccc{AlgebraicKernel_d_1::Approximate_absolute_1} - provides an approximation that is 
better than the passed absolute error bound. \\
\ccc{AlgebraicKernel_d_1::Approximate_relative_1} - provides an approximation that is 
better than the passed relative error bound. \\
\ccc{AlgebraicKernel_d_1::Isolate_1} - returns an isolating interval with respect to 
a given univariate polynomial. \\
A related function object is \ccc{AlgebraicKernel_d_1::Bound_between_1}, which computes 
a number that isolates two algebraic real numbers. 

%An example can be found in Section~\ref{CGAL::AK1::EG::Compare_1}.


\subsubsection{Interplay with Polynomials}
It is also possible to retrieve a representing polynomial from an
algebraic real using \ccc{AlgebraicKernel_d_1::Compute_polynomial_1}, that
is, it is guaranteed that the algebraic real is a root of the returned
polynomial. As the name already indicates, this operation may be very
costly since the polynomial may not be computed yet. Moreover, it is
not guaranteed that the returned polynomial is the minimal polynomial
of the number.  Together with \ccc{AlgebraicKernel_d_1::Isolate_1}, it
is possible to retrieve the traditional representation of an algebraic
real, that is, as a square free polynomial and an isolating interval.

Though the \ccc{AlgebraicKernel_d_1} does not provide arithmetic
operations on \ccc{AlgebraicKernel_d_1::Algebraic_real_1}, it is
possible to compute the sign of a polynomial at a given algebraic
real using \ccc{AlgebraicKernel_d_1::Sign_at_1}.  Or alternatively,
just compute whether the polynomial is zero at an algebraic real number
using \ccc{AlgebraicKernel_d_1::Is_zero_at_1}.  Note that this operation
can be significantly less expensive, in particular, if the polynomial
is not zero at the given algebraic real.

%An example can be found in Section~\ref{CGAL::AK1::EG::Sign_at_1}.

\subsubsection{Auxiliary Functionality for Polynomials}

First of all the type \ccc{AlgebraicKernel_d_1::Polynomial_1} is required
to be a model of the concept \ccc{Polynomial_1}, which is defined
in the Polynomial package (see chapter \ref{ChapterPolynomial}).
This implies that all essential functionality is provided via
\ccc{CGAL::Polynomial_traits_d}. However, the algebraic kernel also
provides several function objects to handle polynomials:\\
\ccc{AlgebraicKernel_d_1::Is_square_free_1} 
- determines whether a polynomial is square free \\
\ccc{AlgebraicKernel_d_1::Make_square_free_1} 
- computes the square free part of a polynomial \\
\ccc{AlgebraicKernel_d_1::Square_free_factorize_1} 
- computes a square free factorization of a polynomial \\
\ccc{AlgebraicKernel_d_1::Is_coprime_1} 
- Computes whether a pair of polynomials is square free\\
\ccc{AlgebraicKernel_d_1::Make_coprime_1} 
- decompose two polynomials into the coprime factors and their common factor. 

Though the polynomial package provides similar functionality we suggest to use 
the function objects provided by the kernel, since the design of the algebraic kernel 
allows for instance internal caching by the kernel. 

%An example can be found in Section~\ref{CGAL::AK1::EG::Polynomial}.

%\begin{ccAdvanced}
Also note that  \ccc{AlgebraicKernel_d_1::Square_free_factorize_1} only computes the square free 
factorization up to a constant factor. This is a slight modification with respect to its 
counter part in \ccc{CGAL::Polynomial_traits_d}. In this way it was possible that the concepts just require 
the coefficient type to be a model of \ccc{IntegralDomain}, instead of \ccc{Field} or \ccc{UniqueFactorizationDomain}. 
For more details see also:\\
\ccRefIdfierPage{PolynomialTraits_d::SquareFreeFactorize} \\
\ccRefIdfierPage{PolynomialTraits_d::SquareFreeFactorizeUpToConstantFactor}\\
%\end{ccAdvanced}


\subsubsection{Design Rationale}

Most implementations of an \ccc{AlgebraicKernel_d_1} will represent
an algebraic real number by the root of a square free polynomial and
an isolating interval, that is, the number is defined as the only
root of the polynomial within the interval. Usually, one will refrain
from computing the minimal polynomial since the computation of the
minimal polynomial is much more expensive and does not pay of. However,
besides the representation by a polynomial and an isolating interval
one can also imagine the representation by a polynomial and the index
of the root, e.g., as the $i$th real root when enumerated from minus
to plus infinity. Moreover, it may very well be that the kernel just
computes an approximation of the number, whereas the representing
polynomial is not computed yet. 
This is in particular relevant in relation to the \ccc{AlgebraicKernel_d_2},
where \ccc{AlgebraicKernel_d_1::Algebraic_real_1} is used to represent coordinates of solutions of bivariate systems.
Hence, the design does
not allow a direct access to any, seemingly obvious, members of an
\ccc{AlgebraicKernel_d_1::Algebraic_real_1}. Instead there is, e.g.,
\ccc{AlgebraicKernel_d_1::Compute_polynomial_1} which emphasizes
that the requested polynomial may not be computed yet. Similarly,
there is no way to directly ask for the refinement of the current
isolating interval since this would impose a state to every object of
an \ccc{AlgebraicKernel_d_1::Algebraic_real_1}.

