% +------------------------------------------------------------------------+
% | Reference manual page: Vertex.tex
% +------------------------------------------------------------------------+
% | 17.03.1999   Lutz Kettner
% | Package: Polyhedron
% | 
\RCSdef{\RCSVertexRev}{$Id$}
\RCSdefDate{\RCSVertexDate}{$Date$}
% +------------------------------------------------------------------------+

\ccRefPageBegin

%%RefPage: end of header, begin of main body
% +------------------------------------------------------------------------+


\begin{ccRefClass}[Polyhedron_3<Traits>::]{Vertex}

\ccDefinition
  
A vertex optionally stores a point and a reference to an incident
halfedge that points to the vertex.  Type tags indicate whether these
member functions are supported.  
Figure~\ccTexHtml{\ref{figurePolyOptionalMethods} on page 
\pageref{figurePolyOptionalMethods}}{}\begin{ccHtmlOnly}
  <A HREF="Class_Polyhedron_3-Traits---Halfedge.html#figurePolyOptionalMethods">&#x261E;</A>
\end{ccHtmlOnly}
depicts the relationship between a halfedge and its incident
halfedges, vertices, and facets. The circulator is assignable to the
\ccc{Halfedge_handle}. The circulator is bidirectional if the
halfedge provided to the polyhedron with the \ccc{Items} template
argument provides a member function \ccc{prev()}, otherwise it is
of the forward category.


\ccInclude{CGAL/Polyhedron_3.h}

\ccTypes
\ccThree{Halfedge_const_handle}{h.halfedge() const;;}{}
\ccThreeToTwo

\ccNestedType{Halfedge}{type of incident halfedges.}
\ccGlue
\ccNestedType{Facet}{type of incident facets.}
\ccGlue
\ccNestedType{Point_3}{point type stored in vertices.}

\ccNestedType{Vertex_handle}{handle to vertex.}
\ccGlue
\ccNestedType{Halfedge_handle}{handle to halfedge.}
\ccGlue
\ccNestedType{Facet_handle}{handle to facet.}
\ccGlue
\ccNestedType{Halfedge_around_vertex_circulator}{circulator of
  halfedges around a vertex.}

\ccNestedType{Vertex_const_handle}{}
\ccGlue
\ccNestedType{Halfedge_const_handle}{}
\ccGlue
\ccNestedType{Facet_const_handle}{}
\ccGlue
\ccNestedType{Halfedge_around_vertex_const_circulator}{}

\ccNestedType{Supports_vertex_halfedge}{$\equiv$ \ccc{CGAL::Tag_true} or 
  \ccc{CGAL::Tag_false}.}
\ccGlue
\ccNestedType{Supports_vertex_point}{$\equiv$ \ccc{CGAL::Tag_true} or 
  \ccc{CGAL::Tag_false}.}


\ccCreation
\ccCreationVariable{v}

\ccConstructor{Vertex();}{default constructor.}
\ccGlue
\ccConstructor{Vertex( const Point& p);}{vertex initialized with a point.}

\ccTagFullDeclarations
\ccHeading{Operations available if \ccc{Supports_vertex_point} $\equiv$ 
           \ccc{CGAL::Tag_true}}

\ccMethod{Point_3&       point();}{}
\ccGlue
\ccMethod{const Point_3& point() const;}{the point.}

\newpage
\ccHeading{Operations available if \ccc{Supports_vertex_halfedge} $\equiv$ 
           \ccc{CGAL::Tag_true}}

\ccMethod{Halfedge_handle       halfedge();}{}
\ccGlue
\ccMethod{Halfedge_const_handle halfedge() const;}{
    an incident halfedge that points to \ccVar.}

%%\ccThree{Halfedge_const_handleh.halfedge() const;}{;}{}
\ccMethod{Halfedge_around_vertex_circulator       vertex_begin();}{}

\ccMethod{Halfedge_around_vertex_const_circulator vertex_begin() const;}
    {circulator of halfedges around the vertex (clockwise).}

\ccMethod{void set_halfedge( Halfedge_handle h);}
    {sets incident halfedge to \ccc{h}.
     \ccPrecond \ccc{h} is incident, i.e., \ccc{h->vertex() ==} \ccVar.}

\ccMethod{std::size_t vertex_degree() const;}{the degree of the 
  vertex, i.e., number of edges emanating from this vertex.}

\ccMethod{bool is_bivalent() const;}{returns \ccc{true} if the vertex
  has exactly two incident edges.}

\ccMethod{bool is_trivalent() const;}{returns \ccc{true} if the vertex
  has exactly three incident edges.}


\ccSeeAlso

\ccRefIdfierPage{CGAL::Polyhedron_3<Traits>::Halfedge}\\
\ccRefIdfierPage{CGAL::Polyhedron_3<Traits>::Facet}\\
\ccRefIdfierPage{CGAL::Polyhedron_3<Traits>}

\ccTagDefaults
\end{ccRefClass}

% +------------------------------------------------------------------------+
%%RefPage: end of main body, begin of footer
\ccRefPageEnd
% EOF
% +------------------------------------------------------------------------+

