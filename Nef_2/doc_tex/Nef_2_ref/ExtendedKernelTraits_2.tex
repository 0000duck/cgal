% begin cgal manual page

\begin{ccRefConcept}{ExtendedKernelTraits_2}
\ccIndexTraitsClassRequirements[p]{extended kernel, 2D}     
\ccCreationVariable{K}

\ccDefinition

\ccc{ExtendedKernelTraits_2} is a kernel concept providing extended
geometry\footnote{It is called extended geometry for simplicity,
though it is not a real geometry in the classical sense.}. Let \ccc{K}
be an instance of the data type \ccc{ExtendedKernelTraits_2}.  The
central notion of extended geometry are extended points. An extended
point represents either a standard affine point of the Cartesian plane
or a non-standard point representing the equivalence class of rays
where two rays are equivalent if one is contained in the other.

Let $R$ be an infinimaximal number\footnote{A finite but very large
number.}, $F$ be the square box with corners $NW(-R,R)$, $NE(R,R)$,
$SE(R,-R)$, and $SW(-R,-R)$. Let $p$ be a non-standard point and let
$r$ be a ray defining it. If the frame $F$ contains the source point
of $r$ then let $p(R)$ be the intersection of $r$ with the frame $F$,
if $F$ does not contain the source of $r$ then $p(R)$ is undefined.
For a standard point let $p(R)$ be equal to $p$ if $p$ is contained in
the frame $F$ and let $p(R)$ be undefined otherwise. Clearly, for any
standard or non-standard point $p$, $p(R)$ is defined for any
sufficiently large $R$. Let $f$ be any function on standard points,
say with $k$ arguments. We call $f$ {\em extensible} if for any $k$
points $p_1$, \ldots, $p_k$ the function value
$f(p_1(R),\ldots,p_k(R))$ is constant for all sufficiently large
$R$. We define this value as $f(p_1,\ldots,p_k)$.  Predicates like
lexicographic order of points, orientation, and incircle tests are
extensible.

An extended segment is defined by two extended points such that it is
either an affine segment, an affine ray, an affine line, or a segment
that is part of the square box. Extended directions extend the affine
notion of direction to extended objects.

This extended geometry concept serves two purposes. It offers
functionality for changing between standard affine and extended
geometry. At the same time it provides extensible geometric primitives
on the extended geometric objects.

\ccSetOneOfTwoColumns{7cm}

\ccTypes

\ccHeading{Affine kernel types} 



\ccNestedType{Standard_kernel}{the standard affine kernel. 
}

\ccNestedType{Standard_RT}{the standard ring type. 
}

\ccNestedType{Standard_point_2}{standard points. 
}

\ccNestedType{Standard_segment_2}{standard segments. 
}

\ccNestedType{Standard_line_2}{standard oriented lines. 
}

\ccNestedType{Standard_direction_2}{standard directions. 
}

\ccNestedType{Standard_ray_2}{standard rays. 
}

\ccNestedType{Standard_aff_transformation_2}{standard affine transformations. 
}

\ccHeading{Extended kernel types} 



\ccNestedType{RT}{the ring type of our extended kernel. 
}

\ccNestedType{Point_2}{extended points. 
}

\ccNestedType{Segment_2}{extended segments. 
}

\ccNestedType{Direction_2}{extended directions. 
}

\ccEnum{enum Point_type { SWCORNER,  LEFTFRAME,  NWCORNER,  BOTTOMFRAME,  STANDARD,  TOPFRAME,  SECORNER,  RIGHTFRAME,  NECORNER }}{a type descriptor for extended points. 
}

\ccSetOneOfTwoColumns{2cm}

\ccOperations

\ccHeading{Interfacing the affine kernel types} 



\ccMethod{Point_2 construct_point(const Standard_point_2& p) ;}{
creates an extended point and initializes it to the standard point
\ccc{p}.  }

\ccMethod{Point_2 construct_point(const Standard_line_2& l) ;}{
creates an extended point and initializes it to the equivalence class
of all the rays underlying the oriented line \ccc{l}.  }

\ccMethod{Point_2 construct_point(const Standard_point_2& p1, const
Standard_point_2& p2) ;}{ creates an extended point and initializes it
to the equivalence class of all the rays underlying the oriented line
\ccc{l(p1,p2)}.  }

\ccMethod{Point_2 construct_point(const Standard_point_2& p, const
Standard_direction_2& d) ;}{creates an extended point and initializes
it to the equivalence class of all the rays underlying the ray
starting in \ccc{p} in direction \ccc{d}.  }

\ccMethod{Point_2 construct_opposite_point(const Standard_line_2& l)
;}{creates an extended point and initializes it to the equivalence
class of all the rays underlying the oriented line opposite to
\ccc{l}.  }

\ccMethod{Point_type type(const Point_2& p) ;}{determines the type of
\ccc{p} and returns it.  }

\ccMethod{bool is_standard(const Point_2& p) ;}{returns \ccc{true} iff
\ccc{p} is a standard point.  }

\ccMethod{Standard_point_2 standard_point(const Point_2& p) ;}{returns
the standard point represented by \ccc{p}.  \ccPrecond
\ccc{K.is_standard(p)}.  }

\ccMethod{Standard_line_2 standard_line(const Point_2& p) ;}{returns
the oriented line representing the bundle of rays defining \ccc{p}.
\ccPrecond \ccc{!K.is_standard(p)}.  }

\ccMethod{Standard_ray_2 standard_ray(const Point_2& p) ;}{a ray
defining \ccc{p}. \ccPrecond \ccc{!K.is_standard(p)}.  }

\ccMethod{Point_2 NE() ;}{returns the point on the northeast frame
corner.  }

\ccMethod{Point_2 SE() ;}{returns the point on the southeast frame
corner.  }

\ccMethod{Point_2 NW() ;}{returns the point on the northwest frame
corner.  }

\ccMethod{Point_2 SW() ;}{returns the point on the southwest frame
corner.  }

\ccHeading{Geometric kernel calls} 

\ccMethod{Point_2 source(const Segment_2& s) ;}{returns the source
point of \ccc{s}.  }

\ccMethod{Point_2 target(const Segment_2& s) ;}{returns the target
point of \ccc{s}.  }

\ccMethod{Segment_2 construct_segment(const Point_2& p, const Point_2&
q) ;}{constructs a segment \ccc{pq}.  }

\ccMethod{int orientation(const Segment_2& s, const Point_2& p)
;}{returns the orientation of \ccc{p} with respect to the line through
\ccc{s}.  }

\ccMethod{int orientation(const Point_2& p1, const Point_2& p2, const
Point_2& p3) ;}{returns the orientation of \ccc{p3} with respect to
the line through \ccc{p1p2}.  }

\ccMethod{bool left_turn(const Point_2& p1, const Point_2& p2, const
Point_2& p3) ;}{return true iff the \ccc{p3} is left of the line
through \ccc{p1p2}.  }

\ccMethod{bool is_degenerate(const Segment_2& s) ;}{return true iff
\ccc{s} is degenerate.  }

\ccMethod{int compare_xy(const Point_2& p1, const Point_2& p2)
;}{returns the lexicographic order of \ccc{p1} and \ccc{p2}.  }

\ccMethod{int compare_x(const Point_2& p1, const Point_2& p2)
;}{returns the order on the $x$-coordinates of \ccc{p1} and \ccc{p2}.
}

\ccMethod{int compare_y(const Point_2& p1, const Point_2& p2)
;}{returns the order on the $y$-coordinates of \ccc{p1} and \ccc{p2}.
}

\ccMethod{Point_2 intersection( const Segment_2& s1, const Segment_2&
s2) ;}{returns the point of intersection of the lines supported by
\ccc{s1} and \ccc{s2}. \ccPrecond the intersection point exists.  }

\ccMethod{Direction_2 construct_direction( const Point_2& p1, const
Point_2& p2) ;}{returns the direction of the vector \ccc{p2} -
\ccc{p1}.  }

\ccMethod{bool strictly_ordered_ccw(const Direction_2& d1, const
Direction_2& d2, const Direction_2& d3) ;}{returns \ccc{true} iff
\ccc{d2} is in the interior of the counterclockwise angular sector
between \ccc{d1} and \ccc{d3}.  }

\ccMethod{bool strictly_ordered_along_line( const Point_2& p1, const
Point_2& p2, const Point_2& p3) ;}{returns \ccc{true} iff \ccc{p2} is
in the relative interior of the segment \ccc{p1p3}.  }

\ccMethod{bool contains(const Segment_2& s, const Point_2& p)
;}{returns true iff \ccc{s} contains \ccc{p}.  }

\ccMethod{bool first_pair_closer_than_second( const Point_2& p1, const
Point_2& p2, const Point_2& p3, const Point_2& p4) ;}{returns true iff
$\|p1-p2\| < \|p3-p4\|$.  }

\ccMethod{const char* output_identifier() ;}{returns a unique
identifier for kernel object Input/Output. Usually this should be the
name of the model.}

\ccHasModels        

\ccRefIdfierPage{CGAL::Extended_cartesian<FT>} \\ 
\ccRefIdfierPage{CGAL::Extended_homogeneous<RT>} \\ 
\ccRefIdfierPage{CGAL::Filtered_extended_homogeneous<RT>} 

\end{ccRefConcept}


