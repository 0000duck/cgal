%% =============================================================================
%% The CGAL Reference Manual
%% Chapter: STL Extensions - The Reference Part
%% -----------------------------------------------------------------------------
%% author: Sylvain Pion
%% -----------------------------------------------------------------------------
%% $Id$
%% $URL$
%% =============================================================================

%% +=========================================================================+

\begin{ccRefClass}{cpp11::tuple<...>}

\ccDefinition

An object of the class \ccClassTemplateName\ represents a heterogeneous tuple of elements
of the types specified in parameters, which are in variadic number.

There is actually no class in namespace \ccc{CGAL::cpp11} with this name, but a using declaration which
imports a class from another namespace.  By order of priority: the one in namespace
\ccc{std} is used (provided by C++0x), if not found, then the one in namespace
\ccc{std::tr1} is used (provided by TR1), and finally, the fallback solution 
is taken from Boost.

  \ccInclude{CGAL/tuple.h}

  \ccParameters

  The parameters \ccStyle{...} are the value types.

  \ccHeading{Free functions and helper classes}

Some free functions part of the standard interface of \ccc{tuple} are also
brought in namespace \ccc{CGAL::cpp11} with using declarations, these are \ccc{make_tuple},
\ccc{get}, \ccc{tie}. Like in C++0x, the \ccc{get} function template is
specialized so that it can take \ccc{std::pair} as argument.
Two standard helper classes are also provided for convenience (\ccc{tuple_size} and \ccc{tuple_element}).

\end{ccRefClass}

\ccParDims
