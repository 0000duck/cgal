%% =============================================================================
%% The CGAL Reference Manual
%% Chapter: STL Extensions - The Reference Part
%% -----------------------------------------------------------------------------
%% file  : doc_tex/support/STL_Extension/STL_Extension_ref/stl_extension.tex
%% author: Michael Hoffmann, Lutz Kettner 
%% -----------------------------------------------------------------------------
%% $CGAL_Chapter: STL_Extension $
%% $Id$
%% $Date$
%% =============================================================================

%% +=========================================================================+

%% +---------------------------------------------+
\begin{ccRefFunction}{cpp11::next}
  \label{sectionNext}
  \label{sectionGenericFunctions}
  There is actually no function in namespace \ccc{CGAL::cpp11} with
  this name, but a using declaration which imports a function from
  another namespace. By order of priority: the one in namespace
  \ccc{std} is used (provided by C++0x), if not found, then the one in
  namespace \ccc{boost} is used.

  \ccDefinition The function returns the result of \ccc{operator++} on
  a ForwardIterator. The exact behaviour is described in $\mathsection
  24.4.4$ of the C++ standard draft
  \ccAnchor{http://www.open-std.org/jtc1/sc22/wg21/docs/papers/2011/n3242.pdf}{N3242}.

  \ccInclude{CGAL/algorithm.h}

  \ccSeeAlso
  \ccAnchor{http://www.boost.org/doc/libs/1_46_1/libs/utility/utility.htm#functions_next_prior}{boost::next} \\
  \ccRefIdfierPage{CGAL::cpp11::prev}

\end{ccRefFunction}

%% +---------------------------------------------+
\begin{ccRefFunction}{cpp11::prev}
  \label{sectionPrev}
  If C++0x is available the function \ccc{std::prev} is imported into
  the namespace \ccc{CGAL::cpp11}, otherwise \ccc{CGAL::cpp11::prev} is declared with the
  signature as given in $\mathsection 24.4.4$ of the ISO C++ Standard
  and forwarded to \ccc{boost::prior}.

  \ccDefinition The function returns the result of \ccc{operator--} on
  a BidirectionalIterator. The exact behaviour is described in
  $\mathsection 24.4.4$ of the C++ standard draft
  \ccAnchor{http://www.open-std.org/jtc1/sc22/wg21/docs/papers/2011/n3242.pdf}{N3242}.

  \ccInclude{CGAL/algorithm.h}

  \ccSeeAlso
  \ccAnchor{http://www.boost.org/doc/libs/1_46_1/libs/utility/utility.htm#functions_next_prior}{boost::prior} \\
  \ccRefIdfierPage{CGAL::cpp11::next}

\end{ccRefFunction}

%% +---------------------------------------------+
\begin{ccRefFunction}{predecessor}
  \label{sectionPredecessor}
  This function is deprecated. \ccc{CGAL::cpp11::prev} should be used
  instead.
  

  \begin{ccDeprecated}
    \ccDefinition The function \ccRefName\ returns the previous iterator,
    i.e. the result of \ccc{operator--} on a bidirectional iterator.
    
    \ccInclude{CGAL/algorithm.h}
  
    \ccThree{BidirectionalIterator}{predecessor}{}
    \ccFunction{template <class BidirectionalIterator>
      BidirectionalIterator predecessor(BidirectionalIterator it);}
    {returns \ccc{--it}.}
    
    \ccSeeAlso
    \ccRefIdfierPage{CGAL::successor}
  \end{ccDeprecated}
\end{ccRefFunction}

%% +---------------------------------------------+
\begin{ccRefFunction}{successor}
  \label{sectionSuccessor}

  This function is deprecated. \ccc{CGAL::cpp11::next} should be used
  instead.

  \begin{ccDeprecated}
    \ccDefinition The function \ccRefName\ returns the next iterator, i.e.
    the result of \ccc{operator++} on a forward iterator.
  
    \ccInclude{CGAL/algorithm.h}
  
    \ccThree{ForwardIterator}{successor}{}
    \ccFunction{template <class ForwardIterator>
      ForwardIterator successor(ForwardIterator it);}
    {returns \ccc{++it}.}
  
    \ccSeeAlso
    \ccRefIdfierPage{CGAL::predecessor}
  \end{ccDeprecated}
\end{ccRefFunction}

%% +---------------------------------------------+

\begin{ccRefFunction}{cpp11::copy_n}
  This provides an implementation of the standard function
  \ccc{copy_n} from the C++0x standard. If \ccc{copy_n} is available
  in the \ccc{std::} namespace a using declaration is used, otherwise
  an alternative implementation from \cgal\ is used.

  \ccDefinition The function \ccRefName\ copies $n$ items from an
  input iterator to an output iterator. Its exact behaviour is defined
  in $\mathsection 25.3.1$ of the C++ standard draft
  \ccAnchor{http://www.open-std.org/jtc1/sc22/wg21/docs/papers/2011/n3242.pdf}{N3242}.

  \ccInclude{CGAL/algorithm.h}
\end{ccRefFunction}

\begin{ccRefFunction}{copy_n}
  This function is deprecated, \ccc{CGAL::cpp11::copy_n} should be used instead.
  \begin{ccDeprecated}
    \label{sectionCopyN}

    \ccDefinition The function \ccRefName\ copies $n$ items from an
    input iterator to an output iterator which is useful for possibly
    infinite
    sequences of random geometric objects.\footnote{%
      The \stl\ release June 13, 1997, from SGI contains an equivalent
      function, but it is not part of the ISO standard.}
  
    \ccInclude{CGAL/algorithm.h}
  
    \ccThree{OutputIterator}{copy_n}{}
    \ccFunction{template <class InputIterator, class Size, class
      OutputIterator> OutputIterator copy_n(InputIterator first, Size n,
      OutputIterator result);}{copies the first $n$ items from
      \ccc{first} to \ccc{result}. Returns the value of \ccc{result}
      after inserting the $n$ items.}
  
    \ccSeeAlso
    \ccRefIdfierPage{CGAL::Counting_iterator<Iterator, Value>}
  \end{ccDeprecated}
\end{ccRefFunction}

%% +---------------------------------------------+
\begin{ccRefFunction}{min_max_element}
  \label{sectionMinmaxelement}
  
  \ccDefinition The function \ccRefName\ computes the minimal and the
  maximal element of a range. It is modeled after the STL functions
  \ccc{min_element} and \ccc{max_element}. The advantage of
  \ccc{min_max_element} compared to calling both STL functions is that
  one only iterates once over the sequence. This is more efficient
  especially for large and/or complex sequences.

  \ccInclude{CGAL/algorithm.h}
  
  \ccFunction{template < class ForwardIterator > std::pair<
    ForwardIterator, ForwardIterator > min_max_element(ForwardIterator
    first, ForwardIterator last);}{returns a pair of iterators where
    the first component refers to the minimal and the second component
    refers to the maximal element in the range [\ccc{first},
    \ccc{last}). The ordering is defined by \ccc{operator<} on the
    value type of \ccc{ForwardIterator}.}
  
  \ccFunction{template < class ForwardIterator, class CompareMin,
    class CompareMax > std::pair< ForwardIterator, ForwardIterator >
    min_max_element(ForwardIterator first, ForwardIterator last,
    CompareMin comp_min, CompareMax comp_max);}{returns a pair of
    iterators where the first component refers to the minimal and the
    second component refers to the maximal element in the range
    [\ccc{first}, \ccc{last}). \ccRequire
    \ccc{CompareMin} and \ccc{CompareMax} are adaptable binary
    function objects:
    \ccc{VT}~$\times$~\ccc{VT}~$\rightarrow$~\ccc{bool} where \ccc{VT}
    is the value type of \ccc{ForwardIterator}.}
  
  \ccExample The following example program computes the minimal and
  maximal element of the sequence $(3,\,6,\,5)$. Hence the output is
  \ccc{min = 3, max = 6}.
  
  \ccIncludeExampleCode{STL_Extension/min_max_element_example.cpp}

\end{ccRefFunction}

%% Michael: I commented these, as I think they should be replaced by combining 
%% Filter_iterator with std::min/max_element().
%%
%% %% +---------------------------------------------+
%% \begin{ccRefFunction}{min_element_if}
%%   \label{sectionMinElementIf}
  
%%   \ccDefinition The function \ccRefName\ computes the minimum among
%%   the elements of a range which satisfy a certain predicate. It is
%%   modeled after the STL function \ccc{min_element}.

%%   \ccInclude{CGAL/algorithm.h}
  
%%   \ccFunction{template < class ForwardIterator, class Predicate >
%%     ForwardIterator min_element_if(ForwardIterator first,
%%     ForwardIterator last, Predicate pred);}{returns an iterator
%%     referring to the minimal element among those satifying the
%%     predicate \ccc{pred} in the range [\ccc{first}, \ccc{last}). The
%%     ordering is defined by the \ccc{operator<} on \ccc{VT} where
%%     \ccc{VT} is the value type of \ccc{ForwardIterator}.
%%     \ccRequire \ccc{pred} is an unary function
%%     object: \ccc{VT}~$\rightarrow$~\ccc{bool}.}
  
%%   \ccFunction{template < class ForwardIterator, class Compare, class
%%     Predicate > ForwardIterator min_element_if(ForwardIterator first,
%%     ForwardIterator last, Compare comp, Predicate pred);} {return an
%%     iterator referring to the minimal element among those satifying
%%     the predicate \ccc{pred} in the range [\ccc{first}, \ccc{last}).
%%     The ordering is defined by \ccc{comp}.
%%     \ccRequire \ccc{comp} is a binary function
%%     object: \ccc{VT}~$\times$~\ccc{VT}~$\rightarrow$~\ccc{bool} where
%%     \ccc{VT} is the value type of \ccc{ForwardIterator}. \ccc{pred} is
%%     an unary function object: \ccc{VT}~$\rightarrow$~\ccc{bool}.}
  
%%   \ccSeeAlso
%%   \ccRefIdfierPage{CGAL::max_element_if}\\
%%   \ccRefIdfierPage{CGAL::min_max_element}

%%   \ccExample The following example program computes the minimal odd
%%   element of the sequence $(3,\,5,\,2)$. Hence the output is
%%   \ccc{min_odd = 3}.
  
%%   \ccIncludeExampleCode{STL_Extension_ref/min_element_if_example.cpp}

%% \end{ccRefFunction}

%% %% +---------------------------------------------+
%% \begin{ccRefFunction}{max_element_if}
  
%%   \ccDefinition The function \ccRefName\ computes the maximum among
%%   the elements of a range which satisfy a certain predicate. It is
%%   modeled after the STL function \ccc{max_element}.

%%   \ccInclude{CGAL/algorithm.h}
  
%%   \ccFunction{template < class ForwardIterator, class Predicate >
%%     ForwardIterator max_element_if(ForwardIterator first,
%%     ForwardIterator last, Predicate pred);}{returns an iterator
%%     referring to the maximal element among those satifying the
%%     predicate \ccc{pred} in the range [\ccc{first}, \ccc{last}). The
%%     ordering is defined by the \ccc{operator<} on \ccc{VT} where
%%     \ccc{VT} is the value type of \ccc{ForwardIterator}.
%%     \ccRequire \ccc{pred} is an unary function
%%     object: \ccc{VT}~$\rightarrow$~\ccc{bool}.}
  
%%   \ccFunction{template < class ForwardIterator, class Compare, class
%%     Predicate > ForwardIterator max_element_if(ForwardIterator first,
%%     ForwardIterator last, Compare comp, Predicate pred);} {return an
%%     iterator referring to the maximal element among those satifying
%%     the predicate \ccc{pred} in the range [\ccc{first}, \ccc{last}).
%%     The ordering is defined by \ccc{comp}.
%%     \ccRequire \ccc{comp} is a binary function
%%     object: \ccc{VT}~$\times$~\ccc{VT}~$\rightarrow$~\ccc{bool} where
%%     \ccc{VT} is the value type of \ccc{ForwardIterator}. \ccc{pred} is
%%     an unary function object: \ccc{VT}~$\rightarrow$~\ccc{bool}.}

%%   \ccSeeAlso
%%   \ccRefIdfierPage{CGAL::min_element_if}\\
%%   \ccRefIdfierPage{CGAL::min_max_element}

%% \end{ccRefFunction}

%% +=========================================================================+

\begin{ccRefClass}{Dispatch_output_iterator<V,O>}

  \ccCreationVariable{i}

  \ccDefinition The class \ccClassTemplateName\ defines an
  \ccc{OutputIterator} that contains a tuple of output iterators, and
  dispatches among those based on the type of the value type which is
  put in it. Besides defining assignment for all parameters of V it is
  also defined for the types \ccStyle{boost::variant<T\ldots>} and
  \ccStyle{boost::optional<boost::variant<T\ldots>>}, where T\ldots\
  must be a subset of the parameters of \ccc{V}. Should the
  \ccStyle{boost::optional} be empty, it will be discarded.

  It also inherits from \ccc{O}, which makes it easy to treat like a tuple.

  \ccParameters
  \ccc{V} must be a \ccc{CGAL::cpp11::tuple<...>} of the types of values to be accepted and dispatched.
  \ccc{O} must be a \ccc{CGAL::cpp11::tuple<...>} of the types of corresponding output iterators.

  \ccInclude{CGAL/iterator.h}

  \ccIsModel
  \ccc{OutputIterator}

  \ccInheritsFrom \ccc{O}

  \ccTypes

  \ccTypedef{typedef V Value_type_tuple;}{}
  \ccGlue
  \ccTypedef{typedef O Iterator_tuple;}{}

  \ccCreation

  \ccTwo{Exptyset_iterator()}{}

  \ccConstructor{Dispatch_output_iterator(I...o);}{Constructor taking all the output iterators.}

  \ccMemberFunction{const Iterator_tuple& get_iterator_tuple() const;}
  {returns a reference to the tuple of output iterators.}

  \ccFunction{template < typename... V, typename... O>
              Dispatch_output_iterator<tuple<V...>, tuple<O...> >
              dispatch_output(O... o);}
  {returns a \ccc{Dispatch_output_iterator} constructed from the arguments.}

  \ccIncludeExampleCode{STL_Extension/Dispatch_output_iterator.cpp}

  \ccSeeAlso
  \ccRefIdfierPage{CGAL::Dispatch_or_drop_output_iterator<V,O>}

\end{ccRefClass}


\begin{ccRefClass}{Dispatch_or_drop_output_iterator<V,O>}

  \ccCreationVariable{i}

  \ccDefinition The class \ccClassTemplateName\ defines an
  \ccc{OutputIterator} that contains a tuple of output iterators, and
  dispatches among those based on the type of the value type which is
  put in it. Other types are also accepted, and the object is simply
  discarded in this case. Besides defining assignment for all
  parameters of V it is also defined for the types
  \ccStyle{boost::variant<T\ldots>} and
  \ccStyle{boost::optional<boost::variant<T\ldots>>}, where T\ldots\
  can be a list of arbitrary types.

  It also inherits from \ccc{O}, which makes it easy to treat like a
  tuple.

  \ccParameters
  \ccc{V} must be a \ccc{CGAL::cpp11::tuple<...>} of the types of values to be accepted and dispatched.
  \ccc{O} must be a \ccc{CGAL::cpp11::tuple<...>} of the types of corresponding output iterators.

  \ccInclude{CGAL/iterator.h}

  \ccIsModel
  \ccc{OutputIterator}

  \ccInheritsFrom \ccc{O}

  \ccTypes

  \ccTypedef{typedef V Value_type_tuple;}{}
  \ccGlue
  \ccTypedef{typedef O Iterator_tuple;}{}

  \ccCreation

  \ccTwo{Exptyset_iterator()}{}

  \ccConstructor{Dispatch_or_drop_output_iterator(I...o);}{Constructor taking all the output iterators.}

  \ccMemberFunction{const Iterator_tuple& get_iterator_tuple() const;}
  {returns a reference to the tuple of output iterators.}

  \ccFunction{template < typename... V, typename... O>
              Dispatch_or_drop_output_iterator<tuple<V...>, tuple<O...> >
              dispatch_or_drop_output(O... o);}
  {returns a \ccc{Dispatch_or_drop_output_iterator} constructed from the arguments.}

  \ccSeeAlso
  \ccRefIdfierPage{CGAL::Dispatch_output_iterator<V,O>}

\end{ccRefClass}


\begin{ccRefClass}{Emptyset_iterator}
  \label{sectionEmptysetIterator}

  \ccCreationVariable{i}
  
  \ccDefinition The class \ccClassName\ defines an
  \ccc{OutputIterator} that ignores everything written to it. One can
  think of it as being connected to \texttt{/dev/null}.

  \ccInclude{CGAL/iterator.h}

  \ccIsModel
  \ccc{OutputIterator}
  
  \ccCreation

  \ccTwo{Emptyset_iterator()}{}
  
  \ccConstructor{Emptyset_iterator();}{default constructor.}
  
  \ccSeeAlso
  \ccRefIdfierPage{CGAL::Oneset_iterator}\\
  \ccRefIdfierPage{CGAL::Const_oneset_iterator}

\end{ccRefClass}

\begin{ccRefClass}{Oneset_iterator<T>}
  \label{sectionOnesetIterator}

  \ccCreationVariable{i}
  
  \ccDefinition The class \ccClassTemplateName\ defines an
  \ccc{BidirectionalIterator} that always refers to one specific
  object of type \ccc{T}.  Internally, \ccClassTemplateName\ stores a
  pointer to the referred object.
  
  \ccInclude{CGAL/iterator.h}

  \ccIsModel
  \ccc{BidirectionalIterator}
  
  \ccCreation

  \ccTwo{Oneset_iterator(T& t);}{}
  
  \ccTagFullDeclarations\ccConstructor{Oneset_iterator(T& t);}{creates
    an iterator referring to \ccc{t}.}\ccTagDefaults
  
  \ccSeeAlso
  \ccRefIdfierPage{CGAL::Emptyset_iterator}\\
  \ccRefIdfierPage{CGAL::Const_oneset_iterator}

\end{ccRefClass}

\begin{ccRefClass}{Const_oneset_iterator<T>}
  \label{sectionConst_onesetIterator}

  \ccCreationVariable{i}
  
  \ccDefinition The class \ccClassTemplateName\ defines an
  \ccc{RandomAccessIterator} that always refers to a copy of a
  specific object of type \ccc{T}.
  
  \ccInclude{CGAL/iterator.h}

  \ccIsModel
  \ccc{RandomAccessIterator}
  
  \ccCreation

  \ccTwo{Const_oneset_iterator(const T& t);}{}
  
  \ccTagFullDeclarations\ccConstructor{Const_oneset_iterator(T&
    t);}{creates an iterator that always refers to some copy of
    \ccc{t}.  The copy is constructed by invoking \ccc{T}'s copy
    constructor and remains constant during $i$'s
    lifetime.}\ccTagDefaults
  
  \ccSeeAlso
  \ccRefIdfierPage{CGAL::Emptyset_iterator}\\
  \ccRefIdfierPage{CGAL::Oneset_iterator}

\end{ccRefClass}

%% +--------------------------------------------------------+
\begin{ccRefClass}{Counting_iterator<Iterator, Value>}
  \label{sectionCountingIterator}

  \ccCreationVariable{i}
  
  \ccDefinition The iterator adaptor \ccClassTemplateName\ adds a
  counter to the internal iterator of type \ccc{Iterator} and defines
  equality of two instances in terms of this counter. It can be used
  to create finite sequences of possibly infinite sequences of values
  from input iterators.

  \ccInclude{CGAL/iterator.h}

  \ccIsModel
  \ccc{InputIterator}
  
  \ccRequirements \ccc{Iterator} is a model for
  \ccc{InputIterator}.

  \ccCreation

  \ccTwo{Identity<Value>MMMMMM}{}
  
  \ccConstructor{Counting_iterator( std::size_t n = 0);}{initializes
    the internal counter to $n$ and \ccVar\ has a singular value.}
  
  \ccConstructor{Counting_iterator( Iterator j, std::size_t n = 0);}{
    initializes the internal counter to $n$ and \ccVar\ to $j$.}

  \ccSeeAlso
  \ccRefIdfierPage{CGAL::copy_n}

\end{ccRefClass}

%% +--------------------------------------------------------+
\begin{ccRefClass}{Insert_iterator<Container>}
  \label{sectionInsertIterator}

  \ccCreationVariable{i}
  
  \ccDefinition The output iterator \ccClassTemplateName\ is similar
  to \ccc{std::insert_iterator}, but differs in that it calls the
  \ccc{insert()} function of the container without the iterator
  additional argument.

  \ccInclude{CGAL/iterator.h}

  \ccIsModel
  \ccc{OutputIterator}
  
  \ccRequirements \ccc{Container} provides a member function
  \ccc{insert(const Container::const_reference&)}.

  \ccCreation

  \ccTwo{Identity<Value>MMMMMM}{}
  
  \ccConstructor{Insert_iterator( Container &c );}{initializes
    the internal container reference to $c$.}

  There is also a global function similar to \ccc{std::inserter}:

  \ccFunction{template < class Container >
              Insert_iterator<Container>
              inserter(Container &x);}
  { Constructs \ccc{Insert_iterator<Container>(x)}. }
  
\end{ccRefClass}

%% +--------------------------------------------------------+
\begin{ccRefClass}{N_step_adaptor<I,int N>}
  \ccCreationVariable{i}
  
  \ccDefinition The adaptor \ccRefName\ changes the step width of the
  iterator or circulator class \ccStyle{I} to $N$. It is itself an
  iterator or circulator respectively. The behavior is undefined if
  the adaptor is used on a range [$i,j$) where $j-i$ is not a multiple
  of $n$.
  
  \ccInclude{CGAL/iterator.h}
  
  \ccCreation
  \ccTwo{N_step_adaptor<I,int N> i( I j);;M}{}
  
  \ccConstructor{N_step_adaptor(const I& j);}{down cast.}
  
\end{ccRefClass}

\begin{ccRefClass}{Filter_iterator<Iterator, Predicate>}
  \label{sectionFilterIterator}

  \ccCreationVariable{i}
  
  \ccDefinition The iterator adaptor \ccClassTemplateName\ acts as a
  filter on a given range. Whenever the iterator is in-- or
  decremented, it ignores all iterators for which the given
  \ccc{Predicate} is true. The iterator category is the same as for
  \ccc{Iterator}.

  Note: Boost also provides the same functionality via the
  \ccc{boost::filter_iterator} class.  Unfortunately, the semantics
  chosen for accepting or rejecting elements based on the predicate's
  result are opposite as the semantic chosen here. What is more, the
  argument of the predicate is different: the predicate used with
  \ccc{boost::filter_iterator} must take the value type of the iterator, as
  argument, and not the iterator itself.

  \ccInclude{CGAL/iterator.h}
  
  \ccRequirements
  \begin{itemize}
  \item \ccc{Iterator} is a model for \ccc{ForwardIterator}.
  \item \ccc{Predicate} is a functor: \ccc{Iterator} $\rightarrow$
    \ccc{bool}.
  \end{itemize}

  \ccCreation

  %%\ccTwo{Identity<Value>MMMMMM}{}
  
  \ccConstructor{Filter_iterator();}{}
  
  \ccConstructor{Filter_iterator(Iterator e, Predicate p, Iterator c = e);}
   {creates an iterator which filters values according to \ccc{p}.
    Initializes by taking the first valid iterator (according to \ccc{p}),
    starting at \ccc{c}, and stopping at \ccc{e} if none is found.}

  There is also a global function to help the use of \ccClassTemplateName{}:

  \ccFunction{template < class Iterator, class Predicate >
    inline Filter_iterator< Iterator, Predicate >
    filter_iterator(Iterator e, const Predicate& p, Iterator c = e);}
  { Constructs \ccc{Filter_iterator<Iterator, Predicate>(e, p, c)}. }
\end{ccRefClass}

%% +---------------------------------------------+

\begin{ccRefClass}{Join_input_iterator_1<Iterator, Creator>}
  \label{sectionJoinInputIterator}
  
  \ccDefinition The class \ccRefName\ joins an iterator and a creator
  function object. The result is again an iterator (of the same
  iterator category type as the original iterator) that reads an object
  from the stream and applies a creator function object to that
  object.

  \ccInclude{CGAL/iterator.h}
  
  \ccIsModel \ccc{InputIterator}

  \ccTypes
  \ccNestedType{value_type}{typedef to \ccc{Creator::result_type}.}
  
  \ccCreation\ccCreationVariable{join}
  
  \ccConstructor{Join_input_iterator_1( Iterator i, const Creator&
    creator);} {creates a join iterator from the given iterator $i$
    and the functor \ccc{creator}. Applies \ccc{creator} to each item
    read from $i$.}

  \ccConstructor{Join_input_iterator_1( Iterator i);} {creates a join
    iterator from the given iterator $i$ and a default constructed
    instance of \ccc{Creator}. The latter instance is applied to each
    item read from $i$.}

  \ccSeeAlso
  \ccRefIdfierPage{CGAL::Creator_1<Arg, Result>}
\end{ccRefClass}


%%  +--------------------------------------------------------+
\begin{ccRefClass}{Inverse_index<IC>}
  
  \ccDefinition The class \ccClassTemplateName\ constructs an inverse
  index for a given range [$i,j$) of two iterators or circulators of
  type \ccc{IC}.  The first element $I$ in the range [$i,j$) has the
  index 0.  Consecutive elements are numbered incrementally. The
  inverse index provides a query for a given iterator or circulator
  $k$ to retrieve its index number. {\em Precondition:}\/ The iterator
  or circulator must be either of the random access category or the
  dereference operator must return stable and distinguishable
  addresses for the values, e.g.~proxies or non-modifiable iterator
  with opaque values will not work.

  \ccInclude{CGAL/iterator.h}

  \ccCreation\ccCreationVariable{inverse}
  
  \ccTwo{Inverse_index< IC,> inverse( IC i, IC j);;}{}
  \ccConstructor{Inverse_index();}{invalid index.}
  
  \ccGlue\ccConstructor{Inverse_index( const IC& i);}{empty inverse
    index initialized to start at $i$.}
  
  \ccGlue\ccConstructor{Inverse_index( const IC& i, const IC& j);}
  {inverse index initialized with range [$i,j$).}

  \ccOperations
  \ccThree{std::size_t}{inverse.find( const T* p);}{}
  
  \ccMethod{std::size_t operator[]( const IC& k);}{returns inverse
    index of $k$. \ccPrecond $k$ has been stored in the inverse
    index.}
  
  \ccMethod{void push_back( const IC& k);}{adds $k$ at the end of the
    indices.}

  \ccImplementation
  
  For random access iterators or circulators, it is done in constant
  time by subtracting $i$. For other iterator categories, an \stl\ 
  \ccc{map} is used, which results in a $\log j-i$ query time. The
  comparisons are done using the operator \ccc{operator<} on pointers.

  \ccSeeAlso
  \ccRefIdfierPage{CGAL::Random_access_adaptor<IC>}\\
  \ccRefIdfierPage{CGAL::Random_access_value_adaptor<IC,T>}

\end{ccRefClass}

%% +--------------------------------------------------------+
\begin{ccRefClass}{Random_access_adaptor<IC>}
  
  \ccDefinition The class \ccClassTemplateName\ provides a random
  access for data structures. Either the data structure supports
  random access iterators or circulators where this class maps
  function calls to the iterator or circulator, or a \stl\ 
  \ccc{std::vector} is used to provide the random access. The iterator
  or circulator of the data structure are of type \ccc{IC}.

  \ccInclude{CGAL/iterator.h}

  \ccTypes

  \ccNestedType{size_type}{size type of the \stl\ \ccc{std::vector}.}

  \ccCreation\ccCreationVariable{random_access}

  \ccTwo{Random_access_adaptor< IC> random_access;}{}
  \ccConstructor{Random_access_adaptor();}{invalid index.}
  
  \ccConstructor{Random_access_adaptor( const IC& i);} {empty random
    access index initialized to start at $i$.}
  
  \ccConstructor{Random_access_adaptor( const IC& i, const IC& j);}
  {random access index initialized to the range [$i,j$).}
  
  \ccThree{Dist}{random_access.push_back( IC k);}{} \ccMethod{void
    reserve( size_type r);}{reserve $r$ entries, if a
    \ccc{std::vector} is used internally.}
  
  \ccOperations
  
  \ccMethod{IC operator[]( size_type n);}{returns iterator or
    circulator to the $n$-th item.  \ccPrecond $n <$ number of items
    in \ccVar.}
  
  \ccMethod{void push_back( const IC& k);}{adds $k$ at the end of the
    indices.}
  
  \ccSeeAlso
  \ccRefIdfierPage{CGAL::Inverse_index<IC>}\\
  \ccRefIdfierPage{CGAL::Random_access_value_adaptor<IC,T>}

\end{ccRefClass}


%% +--------------------------------------------------------+
\begin{ccRefClass}{Random_access_value_adaptor<IC,T>}
  
  \ccDefinition The class \ccClassTemplateName\ provides a random
  access for data structures. It is derived from
  \ccc{Random_access_adaptor<IC>}. Instead of returning iterators from
  the \ccc{operator[]} methods, it returns the dereferenced value of
  the iterator.  The iterator or circulator of the data structure are
  of type \ccc{IC}. Their value type is $T$.

  \ccInclude{CGAL/iterator.h}

  \ccOperations
  
  Creation and operations see \ccc{Random_access_adaptor<IC>}, with
  the exception of:
  
  \ccCreationVariable{random_access}

  \ccThree{Dist}{random_access.push_back( IC k);}{}
  
  \ccMethod{T& operator[]( size_type n);}{returns a reference to the
    $n$-th item.  \ccPrecond $n <$ number of items in \ccVar.}
  
  \ccSeeAlso
  \ccRefIdfierPage{CGAL::Inverse_index<IC>}\\
  \ccRefIdfierPage{CGAL::Random_access_adaptor<IC>}

\end{ccRefClass}

%% +=========================================================================+

\begin{ccRefFunction}{compare_to_less}
  \ccDefinition The function \ccRefName\ is used to change a functor
  returning a \ccc{Comparison_result} to one which returns a bool.
  The returned functor will return \ccc{true} iff the original one
  returns \ccc{SMALLER}.
  
  \ccInclude{CGAL/function_objects.h}
  
  \ccThree{typename Compare_to_less< F >::Type}{compare_to_less}{}
  
  \ccFunction{template < class F > Compare_to_less< F >
    compare_to_less(const F& f);}{returns a functor equivalent to
    \ccc{f}, but which returns a bool instead of a
    \ccc{Comparison_result}.}
  
  \ccSeeAlso
  \ccRefIdfierPage{CGAL::Compare_to_less<F>}
  
\end{ccRefFunction}

\begin{ccRefClass}{Compare_to_less<F>}
  \ccDefinition The class \ccRefName\ is used to convert a functor which
  returns a \ccc{Comparison_result} to a predicate (returning bool) : it
  will return true iff the return value of \ccc{F} is \ccc{SMALLER}.
  The class is used in conjunction with the \ccc{compare_to_less}
  function; see there for an explanation on how exactly the functors
  are combined.

  \ccInclude{CGAL/function_objects.h}
  
  \ccTypes
  \ccNestedType{Type}{type of the composed functor.}
  
  \ccSeeAlso
  \ccRefIdfierPage{CGAL::compare_to_less}
  
\end{ccRefClass}

%% +=========================================================================+

\begin{ccRefFunctionObjectConcept}{Projection_object}
  \label{sectionProjectionFunctionObjects}
  
  \ccDefinition The concept \ccRefName\ is modeled after the STL
  concept \ccc{UnaryFunction}, but takes also care of (const)
  references.
  
  \ccTagFullDeclarations
  \ccNestedType{argument_type}{argument type.}
  \ccNestedType{result_type}{result type.}
  \ccCreationVariable{o}
  \ccCreation
  \ccConstructor{Projection_object();}{default constructor.}
  \ccOperations
  \ccThree{const result_type&;;}{A}{}
  \ccMethod{result_type& operator()(argument_type &) const;}{}
  \ccGlue
  \ccMethod{const result_type& operator()(const argument_type &) const;}{}
  \ccTagDefaults

  \ccHasModels
  \ccRefIdfierPage{CGAL::Identity<Value>}\\
  \ccRefIdfierPage{CGAL::Dereference<Value>}\\
  \ccRefIdfierPage{CGAL::Get_address<Value>}\\
  \ccRefIdfierPage{CGAL::Cast_function_object<Arg, Result>}\\
  \ccRefIdfierPage{CGAL::Project_vertex<Node>}\\
  \ccRefIdfierPage{CGAL::Project_facet<Node>}\\
  \ccRefIdfierPage{CGAL::Project_point<Node>}\\
  \ccRefIdfierPage{CGAL::Project_normal<Node>}\\
  \ccRefIdfierPage{CGAL::Project_plane<Node>}\\
  \ccRefIdfierPage{CGAL::Project_next<Node>}\\
  \ccRefIdfierPage{CGAL::Project_prev<Node>}\\
  \ccRefIdfierPage{CGAL::Project_next_opposite<Node>}\\
  \ccRefIdfierPage{CGAL::Project_opposite_prev<Node>}
  
\end{ccRefFunctionObjectConcept}

\begin{ccRefFunctionObjectClass}{Identity<Value>}
  \ccDefinition The class \ccRefName\ represents the identity function
  on \ccc{Value}.
  
  \ccInclude{CGAL/function_objects.h}

  \ccIsModel
  \ccRefConceptPage{Projection_object}

  \ccTagFullDeclarations
  \ccNestedType{argument_type}{typedef to \ccc{Value}.}
  \ccNestedType{result_type}{typedef to \ccc{Value}.}
  \ccCreationVariable{o}
  \ccCreation
  \ccConstructor{Identity();}{default constructor.}
  \ccOperations
  \ccThree{const result_type&;;}{A}{}
  
  \ccMethod{result_type& operator()(argument_type& x) const;}{returns
    \ccc{x}.}
  
  \ccGlue\ccMethod{const result_type& operator()(const argument_type&
    x) const;}{returns \ccc{x}.}  \ccTagDefaults

\end{ccRefFunctionObjectClass}

\begin{ccRefFunctionObjectClass}{Dereference<Value>}
  \ccDefinition The class \ccRefName\ dereferences a pointer
  (\ccc{operator*}).
  
  \ccInclude{CGAL/function_objects.h}

  \ccIsModel
  \ccRefConceptPage{Projection_object}

  \ccTagFullDeclarations
  \ccNestedType{argument_type}{typedef to \ccc{Value*}.}
  \ccNestedType{result_type}{typedef to \ccc{Value}.}
  \ccCreationVariable{o}
  \ccCreation
  \ccConstructor{Dereference();}{default constructor.}
  \ccOperations
  \ccThree{const result_type&;;}{A}{}
  
  \ccMethod{result_type& operator()(argument_type& x) const;}{returns
    \ccc{*x}.}
  
  \ccGlue\ccMethod{const result_type& operator()(const argument_type&
    x) const;}{returns \ccc{*x}.} \ccTagDefaults

\end{ccRefFunctionObjectClass}

\begin{ccRefFunctionObjectClass}{Get_address<Value>}
  \ccDefinition The class \ccRefName\ gets the address of an lvalue
  (\ccc{operator&}).
  
  \ccInclude{CGAL/function_objects.h}

  \ccIsModel
  \ccRefConceptPage{Projection_object}

  \ccTagFullDeclarations
  \ccNestedType{argument_type}{typedef to \ccc{Value}.}
  \ccNestedType{result_type}{typedef to \ccc{Value*}.}
  \ccCreationVariable{o}
  \ccCreation
  \ccConstructor{Get_address();}{default constructor.}
  \ccOperations
  \ccThree{const result_type&;;}{A}{}
  
  \ccMethod{result_type& operator()(argument_type& x) const;}{returns
    \ccc{&x}.}
  
  \ccGlue\ccMethod{const result_type& operator()(const argument_type&
    x) const;}{returns \ccc{&x}.} \ccTagDefaults

\end{ccRefFunctionObjectClass}

\begin{ccRefFunctionObjectClass}{Cast_function_object<Arg, Result>}
  \ccDefinition The class \ccRefName\ applies a C-style type cast to
  its argument.
  
  \ccInclude{CGAL/function_objects.h}

  \ccIsModel
  \ccRefConceptPage{Projection_object}

  \ccTagFullDeclarations
  \ccNestedType{argument_type}{typedef to \ccc{Arg}.}
  \ccNestedType{result_type}{typedef to \ccc{Result}.}
  \ccCreationVariable{o}
  \ccCreation
  \ccConstructor{Cast_function_object();}{default constructor.}
  \ccOperations
  \ccThree{const result_type&;;}{A}{}
  
  \ccMethod{result_type& operator()(argument_type& x) const;}{returns
    \ccc{(Result)x}.}
  
  \ccGlue\ccMethod{const result_type& operator()(const argument_type&
    x) const;}{returns \ccc{(Result)x}.} \ccTagDefaults

\end{ccRefFunctionObjectClass}

\begin{ccRefFunctionObjectClass}{Project_vertex<Node>}
  \ccDefinition The class \ccRefName\ calls the member function
  \ccc{vertex()} on an instance of type \ccc{Node}.
  
  \ccInclude{CGAL/function_objects.h}

  \ccIsModel
  \ccRefConceptPage{Projection_object}

  \ccTagFullDeclarations
  \ccNestedType{argument_type}{typedef to \ccc{Node}.}
  \ccNestedType{result_type}{typedef to \ccc{Node::Vertex}.}
  \ccCreationVariable{o}
  \ccCreation
  \ccConstructor{Project_vertex();}{default constructor.}
  \ccOperations
  \ccThree{const result_type&;;}{A}{}
  
  \ccMethod{result_type& operator()(argument_type& n) const;}{returns
    \ccc{n.vertex()}.}
  
  \ccGlue \ccMethod{const result_type& operator()(const argument_type&
    n) const;}{returns \ccc{n.vertex()}.}  \ccTagDefaults

\end{ccRefFunctionObjectClass}

\begin{ccRefFunctionObjectClass}{Project_facet<Node>}
  \ccDefinition The class \ccRefName\ calls the member function
  \ccc{facet()} on an instance of type \ccc{Node}.
  
  \ccInclude{CGAL/function_objects.h}

  \ccIsModel
  \ccRefConceptPage{Projection_object}

  \ccTagFullDeclarations
  \ccNestedType{argument_type}{typedef to \ccc{Node}.}
  \ccNestedType{result_type}{typedef to \ccc{Node::Facet}.}
  \ccCreationVariable{o}
  \ccCreation
  \ccConstructor{Project_facet();}{default constructor.}
  \ccOperations
  \ccThree{const result_type&;;}{A}{}
  
  \ccMethod{result_type& operator()(argument_type& n) const;}{returns
    \ccc{n.facet()}.}
  
  \ccGlue\ccMethod{const result_type& operator()(const argument_type&
    n) const;}{returns \ccc{n.facet()}.} \ccTagDefaults

\end{ccRefFunctionObjectClass}

\begin{ccRefFunctionObjectClass}{Project_point<Node>}
  \ccDefinition The class \ccRefName\ calls the member function
  \ccc{point()} on an instance of type \ccc{Node}.
  
  \ccInclude{CGAL/function_objects.h}

  \ccIsModel
  \ccRefConceptPage{Projection_object}

  \ccTagFullDeclarations
  \ccNestedType{argument_type}{typedef to \ccc{Node}.}
  \ccNestedType{result_type}{typedef to \ccc{Node::Point}.}
  \ccCreationVariable{o}
  \ccCreation
  \ccConstructor{Project_point();}{default constructor.}
  \ccOperations
  \ccThree{const result_type&;;}{A}{}
  
  \ccMethod{result_type& operator()(argument_type& n) const;}{returns
    \ccc{n.point()}.}
  
  \ccGlue \ccMethod{const result_type& operator()(const argument_type&
    n) const;}{returns \ccc{n.point()}.} \ccTagDefaults

\end{ccRefFunctionObjectClass}

\begin{ccRefFunctionObjectClass}{Project_normal<Node>}
  \ccDefinition The class \ccRefName\ calls the member function
  \ccc{normal()} on an instance of type \ccc{Node}.
  
  \ccInclude{CGAL/function_objects.h}

  \ccIsModel
  \ccRefConceptPage{Projection_object}

  \ccTagFullDeclarations
  \ccNestedType{argument_type}{typedef to \ccc{Node}.}
  \ccNestedType{result_type}{typedef to \ccc{Node::Normal}.}
  \ccCreationVariable{o}
  \ccCreation
  \ccConstructor{Project_normal();}{default constructor.}
  \ccOperations
  \ccThree{const result_type&;;}{A}{}
  
  \ccMethod{result_type& operator()(argument_type& n) const;}{returns
    \ccc{n.normal()}.}
  
  \ccGlue \ccMethod{const result_type& operator()(const argument_type&
    n) const;}{returns \ccc{n.normal()}.}  \ccTagDefaults

\end{ccRefFunctionObjectClass}

\begin{ccRefFunctionObjectClass}{Project_plane<Node>}
  \ccDefinition The class \ccRefName\ calls the member function
  \ccc{plane()} on an instance of type \ccc{Node}.
  
  \ccInclude{CGAL/function_objects.h}

  \ccIsModel
  \ccRefConceptPage{Projection_object}

  \ccTagFullDeclarations
  \ccNestedType{argument_type}{typedef to \ccc{Node}.}
  \ccNestedType{result_type}{typedef to \ccc{Node::Plane}.}
  \ccCreationVariable{o}
  \ccCreation
  \ccConstructor{Project_plane();}{default constructor.}
  \ccOperations
  \ccThree{const result_type&;;}{A}{}
  
  \ccMethod{result_type& operator()(argument_type& n) const;}{returns
    \ccc{n.plane()}.}
  
  \ccGlue\ccMethod{const result_type& operator()(const argument_type&
    n) const;}{returns \ccc{n.plane()}.}  \ccTagDefaults

\end{ccRefFunctionObjectClass}

\begin{ccRefFunctionObjectClass}{Project_next<Node>}
  \ccDefinition The class \ccRefName\ calls the member function
  \ccc{next()} on an instance of type \ccc{Node}.
  
  \ccInclude{CGAL/function_objects.h}

  \ccIsModel
  \ccRefConceptPage{Projection_object}

  \ccTagFullDeclarations
  \ccNestedType{argument_type}{typedef to \ccc{Node*}.}
  \ccNestedType{result_type}{typedef to \ccc{Node*}.}
  \ccCreationVariable{o}
  \ccCreation
  \ccConstructor{Project_next();}{default constructor.}
  \ccOperations
  \ccThree{const result_type&;;}{A}{}
  
  \ccMethod{result_type& operator()(argument_type& n) const;}{returns
    \ccc{n->next()}.}
  
  \ccGlue\ccMethod{const result_type& operator()(const argument_type&
    n) const;}{returns \ccc{n->next()}.}  \ccTagDefaults

\end{ccRefFunctionObjectClass}

\begin{ccRefFunctionObjectClass}{Project_prev<Node>}
  \ccDefinition The class \ccRefName\ calls the member function
  \ccc{prev()} on an instance of type \ccc{Node}.
  
  \ccInclude{CGAL/function_objects.h}

  \ccIsModel
  \ccRefConceptPage{Projection_object}

  \ccTagFullDeclarations
  \ccNestedType{argument_type}{typedef to \ccc{Node*}.}
  \ccNestedType{result_type}{typedef to \ccc{Node*}.}
  \ccCreationVariable{o}
  \ccCreation
  \ccConstructor{Project_prev();}{default constructor.}
  \ccOperations
  \ccThree{const result_type&;;}{A}{}
  
  \ccMethod{result_type& operator()(argument_type& n) const;}{returns
    \ccc{n->prev()}.}
  
  \ccGlue\ccMethod{const result_type& operator()(const argument_type&
    n) const;}{returns \ccc{n->prev()}.}  \ccTagDefaults

\end{ccRefFunctionObjectClass}

\begin{ccRefFunctionObjectClass}{Project_next_opposite<Node>}
  \ccDefinition The class \ccRefName\ calls the member functions
  \ccc{next()->opposite()} on an instance of type \ccc{Node}.
  
  \ccInclude{CGAL/function_objects.h}

  \ccIsModel
  \ccRefConceptPage{Projection_object}

  \ccTagFullDeclarations
  \ccNestedType{argument_type}{typedef to \ccc{Node*}.}
  \ccNestedType{result_type}{typedef to \ccc{Node*}.}
  \ccCreationVariable{o}
  \ccCreation
  \ccConstructor{Project_next_opposite();}{default constructor.}
  \ccOperations
  \ccThree{const result_type&;;}{A}{}
  
  \ccMethod{result_type& operator()(argument_type& n) const;}{returns
    \ccc{n->next()->opposite()}.}
  
  \ccGlue\ccMethod{const result_type& operator()(const argument_type&
    n) const;}{returns \ccc{n->next()->opposite()}.}  \ccTagDefaults

\end{ccRefFunctionObjectClass}

\begin{ccRefFunctionObjectClass}{Project_opposite_prev<Node>}
  \ccDefinition The class \ccRefName\ calls the member functions
  \ccc{opposite()->prev()} on an instance of type \ccc{Node}.
  
  \ccInclude{CGAL/function_objects.h}

  \ccIsModel
  \ccRefConceptPage{Projection_object}

  \ccTagFullDeclarations
  \ccNestedType{argument_type}{typedef to \ccc{Node*}.}
  \ccNestedType{result_type}{typedef to \ccc{Node*}.}
  \ccCreationVariable{o}
  \ccCreation
  \ccConstructor{Project_opposite_prev();}{default constructor.}
  \ccOperations
  \ccThree{const result_type&;;}{A}{}
  
  \ccMethod{result_type& operator()(argument_type& n) const;}{returns
    \ccc{n->opposite()->prev()}.}
  
  \ccGlue\ccMethod{const result_type& operator()(const argument_type&
    n) const;}{returns \ccc{n->opposite()->prev()}.}
  \ccTagDefaults

\end{ccRefFunctionObjectClass}

%% +--------------------------------------------------------+

\begin{ccRefFunctionObjectClass}{Creator_1<Arg, Result>}
  \label{sectionCreatorFunctionObjects}
  
  \ccDefinition The concept \ccRefName\ defines types and operations
  for creating objects from one argument.
  
  \ccInclude{CGAL/function_objects.h}

  \ccHeading{Requirements} \ccc{Arg} is convertible to \ccc{Result}.
  
  \ccTagFullDeclarations\ccCreationVariable{c}
  \ccNestedType{argument_type}{type of argument.}
  \ccNestedType{result_type}{type of object to create.}
  \ccThree{result_type;;}{operator()(argument_type a) const;;}{}
  
  \ccMethod{result_type operator()(argument_type a) const;}{returns
    \ccc{result_type(a)}.}

  \ccTagDefaults
\end{ccRefFunctionObjectClass}

\begin{ccRefFunctionObjectClass}{Creator_2<Arg1, Arg2, Result>}
  \ccDefinition The concept \ccRefName\ defines types and operations
  for creating objects from two arguments.
  
  \ccInclude{CGAL/function_objects.h}
  
  \ccRequirements \ccc{Result} defines a corresponding
  constructor.
  
  \def\ccLongParamLayout{\ccTrue}
  \ccTagFullDeclarations\ccCreationVariable{c}
  \ccNestedType{argument1_type}{type of first argument.}
  \ccNestedType{argument2_type}{type of second argument.}
  \ccNestedType{result_type}{type of object to create.}
  \ccThree{result_type;;}{operator()(argument_type a) const;;}{}
  
  \ccMethod{result_type operator()(argument_type1 a1, argument_type2
    a2) const;}{returns \ccc{result_type(a1, a2)}.}

  \ccTagDefaults\def\ccLongParamLayout{\ccFalse}
\end{ccRefFunctionObjectClass}

\begin{ccRefFunctionObjectClass}{Creator_3<Arg1, Arg2, Arg3, Result>}
  \ccDefinition The concept \ccRefName\ defines types and operations
  for creating objects from three arguments.
  
  \ccInclude{CGAL/function_objects.h}
  
  \ccRequirements \ccc{Result} defines a corresponding
  constructor.
  
  \def\ccLongParamLayout{\ccTrue}
  \ccTagFullDeclarations\ccCreationVariable{c}
  \ccNestedType{argument1_type}{type of first argument.}
  \ccNestedType{argument2_type}{type of second argument.}
  \ccNestedType{argument3_type}{type of third argument.}
  \ccNestedType{result_type}{type of object to create.}
  \ccThree{result_type;;}{operator()(argument_type a) const;;}{}
  
  \ccMethod{result_type operator()(argument_type1 a1, argument_type2
    a2, argument_type3 a3) const;}{returns \ccc{result_type(a1, a2,
      a3)}.}

  \ccTagDefaults\def\ccLongParamLayout{\ccFalse}
\end{ccRefFunctionObjectClass}

\begin{ccRefFunctionObjectClass}{Creator_4<Arg1, Arg2, Arg3, Arg4, Result>}
  \ccDefinition The concept \ccRefName\ defines types and operations
  for creating objects from four arguments.
  
  \ccInclude{CGAL/function_objects.h}
  
  \ccRequirements \ccc{Result} defines a corresponding
  constructor.
  
  \def\ccLongParamLayout{\ccTrue}
  \ccTagFullDeclarations\ccCreationVariable{c}
  \ccNestedType{argument1_type}{type of first argument.}
  \ccNestedType{argument2_type}{type of second argument.}
  \ccNestedType{argument3_type}{type of third argument.}
  \ccNestedType{argument4_type}{type of 4th argument.}
  \ccNestedType{result_type}{type of object to create.}
  \ccThree{result_type;;}{operator()(argument_type a) const;;}{}
  
  \ccMethod{result_type operator()(argument_type1 a1, argument_type2
    a2, argument_type3 a3, argument_type4 a4) const;}{returns
    \ccc{result_type(a1, a2, a3, a4)}.}

  \ccTagDefaults\def\ccLongParamLayout{\ccFalse}
\end{ccRefFunctionObjectClass}

\begin{ccRefFunctionObjectClass}{Creator_5<Arg1, Arg2, Arg3, Arg4, Arg5, Result>}
  \ccDefinition The concept \ccRefName\ defines types and operations
  for creating objects from five arguments.
  
  \ccInclude{CGAL/function_objects.h}
  
  \ccRequirements \ccc{Result} defines a corresponding
  constructor.
  
  \def\ccLongParamLayout{\ccTrue}
  \ccTagFullDeclarations\ccCreationVariable{c}
  \ccNestedType{argument1_type}{type of first argument.}
  \ccNestedType{argument2_type}{type of second argument.}
  \ccNestedType{argument3_type}{type of third argument.}
  \ccNestedType{argument4_type}{type of 4th argument.}
  \ccNestedType{argument5_type}{type of 5th argument.}
  \ccNestedType{result_type}{type of object to create.}
  \ccThree{result_type;;}{operator()(argument_type a) const;;}{}
  
  \ccMethod{result_type operator()(argument_type1 a1, argument_type2
    a2, argument_type3 a3, argument_type4 a4, argument_type5 a5)
    const;}{returns \ccc{result_type(a1, a2, a3, a4, a5)}.}

  \ccTagDefaults\def\ccLongParamLayout{\ccFalse}
\end{ccRefFunctionObjectClass}

\begin{ccRefFunctionObjectClass}{Creator_uniform_2<Arg, Result>}
  \ccDefinition The concept \ccRefName\ defines types and operations
  for creating objects from two arguments of the same type.
  
  \ccInclude{CGAL/function_objects.h}
  
  \ccRequirements \ccc{Result} defines a constructor from two
  \ccc{Arg} arguments.
  
  \def\ccLongParamLayout{\ccTrue}
  \ccTagFullDeclarations\ccCreationVariable{c}
  \ccNestedType{argument_type}{type of arguments; typedef to
    \ccc{Arg}.}
  \ccNestedType{result_type}{type of object to create; typedef to
    \ccc{Result}.}
  \ccThree{result_type;;}{operator()(argument_type a) const;;}{}
  
  \ccMethod{result_type operator()(argument_type a1, argument_type a2)
    const;}{returns \ccc{result_type(a1, a2)}.}

  \ccTagDefaults\def\ccLongParamLayout{\ccFalse}
\end{ccRefFunctionObjectClass}

\begin{ccRefFunctionObjectClass}{Creator_uniform_3<Arg, Result>}
  \ccDefinition The concept \ccRefName\ defines types and operations
  for creating objects from three arguments of the same type.
  
  \ccInclude{CGAL/function_objects.h}
  
  \ccRequirements \ccc{Result} defines a constructor from
  three \ccc{Arg} arguments.
  
  \def\ccLongParamLayout{\ccTrue}
  \ccTagFullDeclarations\ccCreationVariable{c}
  \ccNestedType{argument_type}{type of arguments; typedef to
    \ccc{Arg}.}
  \ccNestedType{result_type}{type of object to create; typedef to
    \ccc{Result}.}
  \ccThree{result_type;;}{operator()(argument_type a) const;;}{}
  
  \ccMethod{result_type operator()(argument_type a1, argument_type a2,
    argument_type a3) const;}{returns \ccc{result_type(a1, a2, a3)}.}

  \ccTagDefaults\def\ccLongParamLayout{\ccFalse}
\end{ccRefFunctionObjectClass}

\begin{ccRefFunctionObjectClass}{Creator_uniform_4<Arg, Result>}
  \ccDefinition The concept \ccRefName\ defines types and operations
  for creating objects from four arguments of the same type.
  
  \ccInclude{CGAL/function_objects.h}
  
  \ccRequirements \ccc{Result} defines a constructor from
  four \ccc{Arg} arguments.
  
  \def\ccLongParamLayout{\ccTrue}
  \ccTagFullDeclarations\ccCreationVariable{c}
  \ccNestedType{argument_type}{type of arguments; typedef to
    \ccc{Arg}.}
  \ccNestedType{result_type}{type of object to create; typedef to
    \ccc{Result}.}
  \ccThree{result_type;;}{operator()(argument_type a) const;;}{}
  
  \ccMethod{result_type operator()(argument_type a1, argument_type a2,
    argument_type a3, argument_type a4) const;}{returns
    \ccc{result_type(a1, a2, a3, a4)}.}

  \ccTagDefaults\def\ccLongParamLayout{\ccFalse}
\end{ccRefFunctionObjectClass}

\begin{ccRefFunctionObjectClass}{Creator_uniform_5<Arg, Result>}
  \ccDefinition The concept \ccRefName\ defines types and operations
  for creating objects from five arguments of the same type.
  
  \ccInclude{CGAL/function_objects.h}
  
  \ccRequirements \ccc{Result} defines a constructor from
  five \ccc{Arg} arguments.
  
  \def\ccLongParamLayout{\ccTrue}
  \ccTagFullDeclarations\ccCreationVariable{c}
  \ccNestedType{argument_type}{type of arguments; typedef to
    \ccc{Arg}.}
  \ccNestedType{result_type}{type of object to create; typedef to
    \ccc{Result}.}
  \ccThree{result_type;;}{operator()(argument_type a) const;;}{}
  
  \ccMethod{result_type operator()(argument_type a1, argument_type a2,
    argument_type a3, argument_type a4, argument_type a5)
    const;}{returns \ccc{result_type(a1, a2, a3, a4, a5)}.}

  \ccTagDefaults\def\ccLongParamLayout{\ccFalse}
\end{ccRefFunctionObjectClass}

\begin{ccRefFunctionObjectClass}{Creator_uniform_6<Arg, Result>}
  \ccDefinition The concept \ccRefName\ defines types and operations
  for creating objects from six arguments of the same type.
  
  \ccInclude{CGAL/function_objects.h}
  
  \ccRequirements \ccc{Result} defines a constructor from six
  \ccc{Arg} arguments.
  
  \def\ccLongParamLayout{\ccTrue}
  \ccTagFullDeclarations\ccCreationVariable{c}
  \ccNestedType{argument_type}{type of arguments; typedef to
    \ccc{Arg}.}
  \ccNestedType{result_type}{type of object to create; typedef to
    \ccc{Result}.}
  \ccThree{result_type;;}{operator()(argument_type a) const;;}{}
  
  \ccMethod{result_type operator()(argument_type a1, argument_type a2,
    argument_type a3, argument_type a4, argument_type a5,
    argument_type a6) const;}{returns \ccc{result_type(a1, a2, a3, a4,
      a5, a6)}.}

  \ccTagDefaults\def\ccLongParamLayout{\ccFalse}
\end{ccRefFunctionObjectClass}

\begin{ccRefFunctionObjectClass}{Creator_uniform_7<Arg, Result>}
  \ccDefinition The concept \ccRefName\ defines types and operations
  for creating objects from seven arguments of the same type.
  
  \ccInclude{CGAL/function_objects.h}
  
  \ccRequirements \ccc{Result} defines a constructor from
  seven \ccc{Arg} arguments.
  
  \def\ccLongParamLayout{\ccTrue}
  \ccTagFullDeclarations\ccCreationVariable{c}
  \ccNestedType{argument_type}{type of arguments; typedef to
    \ccc{Arg}.}
  \ccNestedType{result_type}{type of object to create; typedef to
    \ccc{Result}.}
  \ccThree{result_type;;}{operator()(argument_type a) const;;}{}
  
  \ccMethod{result_type operator()(argument_type a1, argument_type a2,
    argument_type a3, argument_type a4, argument_type a5,
    argument_type a6, argument_type a7) const;}{returns
    \ccc{result_type(a1, a2, a3, a4, a5, a6, a7)}.}

  \ccTagDefaults\def\ccLongParamLayout{\ccFalse}
\end{ccRefFunctionObjectClass}

\begin{ccRefFunctionObjectClass}{Creator_uniform_8<Arg, Result>}
  \ccDefinition The concept \ccRefName\ defines types and operations
  for creating objects from eight arguments of the same type.
  
  \ccInclude{CGAL/function_objects.h}
  
  \ccRequirements \ccc{Result} defines a constructor from
  eight \ccc{Arg} arguments.
  
  \def\ccLongParamLayout{\ccTrue}
  \ccTagFullDeclarations\ccCreationVariable{c}
  \ccNestedType{argument_type}{type of arguments; typedef to
    \ccc{Arg}.}
  \ccNestedType{result_type}{type of object to create; typedef to
    \ccc{Result}.}
  \ccThree{result_type;;}{operator()(argument_type a) const;;}{}
  
  \ccMethod{result_type operator()(argument_type a1, argument_type a2,
    argument_type a3, argument_type a4, argument_type a5,
    argument_type a6, argument_type a7, argument_type a8)
    const;}{returns \ccc{result_type(a1, a2, a3, a4, a5, a6, a7,
      a8)}.}

  \ccTagDefaults\def\ccLongParamLayout{\ccFalse}
\end{ccRefFunctionObjectClass}

\begin{ccRefFunctionObjectClass}{Creator_uniform_9<Arg, Result>}
  \ccDefinition The concept \ccRefName\ defines types and operations
  for creating objects from nine arguments of the same type.
  
  \ccInclude{CGAL/function_objects.h}
  
  \ccRequirements \ccc{Result} defines a constructor from
  nine \ccc{Arg} arguments.
  
  \def\ccLongParamLayout{\ccTrue}
  \ccTagFullDeclarations\ccCreationVariable{c}
  \ccNestedType{argument_type}{type of arguments; typedef to
    \ccc{Arg}.}
  \ccNestedType{result_type}{type of object to create; typedef to
    \ccc{Result}.}
  \ccThree{result_type;;}{operator()(argument_type a) const;;}{}
  
  \ccMethod{result_type operator()(argument_type a1, argument_type a2,
    argument_type a3, argument_type a4, argument_type a5,
    argument_type a6, argument_type a7, argument_type a8,
    argument_type a9) const;}{returns \ccc{result_type(a1, a2, a3, a4,
      a5, a6, a7, a8, a9)}.}

  \ccTagDefaults\def\ccLongParamLayout{\ccFalse}
\end{ccRefFunctionObjectClass}

\begin{ccRefFunctionObjectClass}{Creator_uniform_d<Arg, Result>}
  \ccDefinition The concept \ccRefName\ defines types and operations
  for creating objects from two arguments of the same type.

  \ccInclude{CGAL/function_objects.h}

  \ccRequirements \ccc{Result} defines a constructor from three arguments: 
  one \ccc{d} dimension and two \ccc{Arg} arguments.

  \def\ccLongParamLayout{\ccTrue}
  \ccTagFullDeclarations\ccCreationVariable{c}
  \ccNestedType{argument_type}{type of arguments; typedef to
    \ccc{Arg}.}
  \ccNestedType{result_type}{type of object to create; typedef to
    \ccc{Result}.}
  \ccThree{result_type;;}{operator()(argument_type a) const;;}{}

  \ccMethod{result_type operator()(argument_type a1, argument_type a2)
    const;}{returns \ccc{result_type(d, a1, a2)}.}

  \ccTagDefaults\def\ccLongParamLayout{\ccFalse}
\end{ccRefFunctionObjectClass}


%% +--------------------------------------------------------+

\begin{ccRefClass}{Twotuple<T>}
  This class is deprecated, and will be removed in some future \cgal\ release.
  Please use \ccc{CGAL::cpp11::array} instead.

  \begin{ccDeprecated}
  \ccDefinition The \ccRefName\ class stores a homogeneous (same type) pair
  of objects of type \ccc{T}.  A \ccRefName\ is much like a container, in that
  it "owns" its elements. It is not actually a model of container, though,
  because it does not support the standard methods (such as iterators) for
  accessing the elements of a container.

  \ccInclude{CGAL/Twotuple.h}
  
  \ccRequirements \ccc{T} must be \ccc{Assignable}.

  %\ccSetThreeColumns{T*}{next_link ;}{}
  \ccThree{result_type;;}{operator()(argument a) ;;}{}

  \ccTypes
  \ccTypedef{typedef T value_type;}{}

  \ccHeading{Variables}
  \ccVariable{T e0;}{first element}
  \ccGlue
  \ccVariable{T e1;}{second element}

  \ccCreation
  \ccCreationVariable{t}
  
  \ccConstructor{Twotuple();}{introduces a \ccRefName\ using the default
    constructor of the elements.}
  
  \ccConstructor{Twotuple(T x, T y);}{constructs a \ccRefName\ such
    that \ccc{e0} is constructed from \ccc{x} and \ccc{e1} is
    constructed from \ccc{y}.}
  \end{ccDeprecated}
\end{ccRefClass}

%% +--------------------------------------------------------+

\begin{ccRefClass}{Threetuple<T>}
  This class is deprecated, and will be removed in some future \cgal\ release.
  Please use \ccc{CGAL::cpp11::array} instead.
  
  \begin{ccDeprecated}
  \ccDefinition The \ccRefName\ class stores a homogeneous (same type) triple
  of objects of type \ccc{T}.  A \ccRefName\ is much like a container, in that
  it "owns" its elements. It is not actually a model of container, though,
  because it does not support the standard methods (such as iterators) for
  accessing the elements of a container.

  \ccInclude{CGAL/Threetuple.h}
  
  \ccRequirements \ccc{T} must be \ccc{Assignable}.

  %\ccSetThreeColumns{T*}{next_link ;}{}
  \ccThree{result_type;;}{operator()(argument a) ;;}{}

  \ccTypes
  \ccTypedef{typedef T value_type;}{}

  \ccHeading{Variables}
  \ccVariable{T e0;}{first element}
  \ccGlue
  \ccVariable{T e1;}{second element}
  \ccGlue
  \ccVariable{T e2;}{third element}

  \ccCreation
  \ccCreationVariable{t}
  
  \ccConstructor{Threetuple();}{introduces a \ccRefName\ using the default
    constructor of the elements.}
  
  \ccConstructor{Threetuple(T x, T y, T z);}{constructs a \ccRefName\ such
    that \ccc{e0} is constructed from \ccc{x}, \ccc{e1} is
    constructed from \ccc{y} and \ccc{e2} is constructed from \ccc{z}.}
  \end{ccDeprecated}
\end{ccRefClass}

%% +--------------------------------------------------------+

\begin{ccRefClass}{Fourtuple<T>}
  This class is deprecated, and will be removed in some future \cgal\ release.
  Please use \ccc{CGAL::cpp11::array} instead.
  
  \begin{ccDeprecated}
  \ccDefinition The \ccRefName\ class stores a homogeneous (same type)
  fourtuple of objects of type \ccc{T}.  A \ccRefName\ is much like a
  container, in that it "owns" its elements. It is not actually a model of
  container, though, because it does not support the standard methods (such as
  iterators) for accessing the elements of a container.

  \ccInclude{CGAL/Fourtuple.h}
  
  \ccRequirements \ccc{T} must be \ccc{Assignable}.

  %\ccSetThreeColumns{T*}{next_link ;}{}
  \ccThree{result_type;;}{operator()(argument a) ;;}{}

  \ccTypes
  \ccTypedef{typedef T value_type;}{}

  \ccHeading{Variables}
  \ccVariable{T e0;}{first element}
  \ccGlue
  \ccVariable{T e1;}{second element}
  \ccGlue
  \ccVariable{T e2;}{third element}
  \ccGlue
  \ccVariable{T e3;}{fourth element}

  \ccCreation
  \ccCreationVariable{t}
  
  \ccConstructor{Fourtuple();}{introduces a \ccRefName\ using the default
    constructor of the elements.}
  
  \ccConstructor{Fourtuple(T x, T y, T z, T t);}{constructs a \ccRefName\ such
    that \ccc{e0} is constructed from \ccc{x}, \ccc{e1} from \ccc{y},
    \ccc{e2} from \ccc{z} and \ccc{e3} from \ccc{t}.}
  \end{ccDeprecated}
\end{ccRefClass}

%% +--------------------------------------------------------+

\begin{ccRefClass}{Sixtuple<T>}
  This class is deprecated, and will be removed in some future \cgal\ release.
  Please use \ccc{CGAL::cpp11::array} instead.

  \begin{ccDeprecated}
  \ccDefinition The \ccRefName\ class stores a homogeneous (same type)
  sixtuple of objects of type \ccc{T}.  A \ccRefName\ is much like a
  container, in that it "owns" its elements. It is not actually a model of
  container, though, because it does not support the standard methods (such as
  iterators) for accessing the elements of a container.

  \ccInclude{CGAL/Sixtuple.h}
  
  \ccRequirements \ccc{T} must be \ccc{Assignable}.

  %\ccSetThreeColumns{T*}{next_link ;}{}
  \ccThree{result_type;;}{operator()(argument a) ;;}{}

  \ccTypes
  \ccTypedef{typedef T value_type;}{}

  \ccHeading{Variables}
  \ccVariable{T e0;}{first element}
  \ccGlue
  \ccVariable{T e1;}{second element}
  \ccGlue
  \ccVariable{T e2;}{third element}
  \ccGlue
  \ccVariable{T e3;}{fourth element}
  \ccGlue
  \ccVariable{T e4;}{fifth element}
  \ccGlue
  \ccVariable{T e5;}{sixth element}

  \ccCreation
  \ccCreationVariable{t}
  
  \ccConstructor{Sixtuple();}{introduces a \ccRefName\ using the default
    constructor of the elements.}
  
  \ccConstructor{Sixtuple(T x, T y, T z, T t, T u, T v);}{constructs a
    \ccRefName\ such that \ccc{e0} is constructed from \ccc{x}, \ccc{e1} from
    \ccc{y}, \ccc{e2} from \ccc{z}, \ccc{e3} from \ccc{t}, \ccc{e4} from
    \ccc{u} and \ccc{e5} from \ccc{v}.}
  \end{ccDeprecated}
  
\end{ccRefClass}

%% +--------------------------------------------------------+

\begin{ccRefClass}{Triple<T1, T2, T3>}
  
  \ccDefinition The Triple class is an extension of \ccc{std::pair}.
  \ccRefName\ is a heterogeneous triple: it holds one object of type
  \ccc{T1}, one of type \ccc{T2}, and one of type \ccc{T3}.  A
  \ccRefName\ is much like a container, in that it "owns" its
  elements. It is not actually a model of container, though, because
  it does not support the standard methods (such as iterators) for
  accessing the elements of a container.

  This class is NOT RECOMMENDED anymore.  We recommend that you use
  \ccc{CGAL::cpp11::tuple} or \ccc{CGAL::cpp11::array} instead for new uses.

  \ccInclude{CGAL/utility.h}
  
  \ccRequirements \ccc{T1}, \ccc{T2} and \ccc{T3} must be \ccc{Assignable}.
  Additional operations have additional requirements.

  %\ccSetThreeColumns{T*}{next_link ;}{}
  \ccThree{result_type;;}{operator()(argument a) ;;}{}

  \ccTypes
  \ccTypedef{typedef T1 first_type;}{}
  \ccGlue
  \ccTypedef{typedef T2 second_type;}{}
  \ccGlue
  \ccTypedef{typedef T3 third_type;}{}

  \ccHeading{Variables}
  \ccVariable{T1 first;}{first element.  Please access it using \ccc{get<0>()}.}
  \ccGlue
  \ccVariable{T2 second;}{second element.  Please access it using \ccc{get<1>()}.}
  \ccGlue
  \ccVariable{T3 third;}{third element.  Please access it using \ccc{get<2>()}.}

  \ccCreation
  \ccCreationVariable{t}
  
  \ccConstructor{Triple();}{introduces a triple using the default
    constructor of the three elements.}
  
  \ccConstructor{Triple(T1 x, T2 y, T3 z);}{constructs a triple such
    that \ccc{first} is constructed from \ccc{x}, \ccc{second} is
    constructed from \ccc{y}, and \ccc{third} is constructed from
    \ccc{z}.}
  
  \ccConstructor{template <class U, class V, class W> Triple(U u, V v,
    W w);} {constructs a triple such that \ccc{first} is constructed
    from \ccc{u}, \ccc{second} is constructed from \ccc{v}, and
    \ccc{third} is constructed from \ccc{w}. \ccRequire Proper
    conversion operators exist from \ccc{U} to \ccc{T1}, \ccc{V} to
    \ccc{T2}, and \ccc{W} to \ccc{T3}.}

  \ccMemberFunction{template <int i> T get();}
    {Gives access to \ccc{first}, \ccc{second} or \ccc{third} whenever
     \ccc{i} is 0, 1 or 2, via a, potentially const, reference.
     Note: \ccc{T} stands for \ccc{T1}, \ccc{T2} or \ccc{T3} depending
     on \ccc{i}.}

  \ccFunction{template <class T1, class T2, class T3> bool
    operator<(Triple<T1, T2, T3> x, Triple<T1, T2, T3> y);} {The
    comparison operator. It uses lexicographic comparison: the return
    value is true if the first element of \ccc{x} is less than the
    first element of \ccc{y}, and false if the first element of
    \ccc{y} is less than the first element of \ccc{x}. If neither of
    these is the case, then it returns true if the second element of
    \ccc{x} is less than the second element of \ccc{y}, and false if
    the second element of \ccc{y} is less than the second element of
    \ccc{x}.  If neither of these is the case, then it returns the
    result of comparing the third elements of \ccc{x} and \ccc{y}.
    This operator may only be used if \ccc{T1}, \ccc{T2} and \ccc{T3}
    define the comparison operator.}

  \ccFunction{template <class T1, class T2, class T3> bool
    operator==(Triple<T1, T2, T3> x, Triple<T1, T2, T3> y);} {The
    equality operator. The return value is true if and only the first
    elements of \ccc{x} and \ccc{y} are equal, the second elements of
    \ccc{x} and \ccc{y} are equal, and the third elements of \ccc{x}
    and \ccc{y} are equal.  This operator may only be used if
    \ccc{T1}, \ccc{T2} and \ccc{T3} define the equality operator.}
  
  \ccFunction{template <class T1, class T2, class T3> Triple<T1, T2,
    T3> make_triple(T1 x, T2 y, T3 z);} {Equivalent to
    \ccStyle{Triple<T1, T2, T3>(x, y, z)}.}
  
  \ccFunction{template <class T1, class T2, class T3> Triple<T1, T2,
    T3> make_tuple(T1 x, T2 y, T3 z);} {Equivalent to
    \ccStyle{Triple<T1, T2, T3>(x, y, z)}.}

\end{ccRefClass}

%% +--------------------------------------------------------+

\begin{ccRefClass}{Quadruple<T1, T2, T3, T4>}
  
  \ccDefinition The Quadruple class is an extension of
  \ccc{std::pair}.  \ccRefName\ is a heterogeneous quadruple: it holds
  one object of type \ccc{T1}, one of type \ccc{T2}, one of type
  \ccc{T3}, and one of type \ccc{T4}.  A \ccRefName\ is much like a
  container, in that it ``owns'' its elements. It is not actually a
  model of container, though, because it does not support the standard
  methods (such as iterators) for accessing the elements of a
  container.

  This class is NOT RECOMMENDED anymore.  We recommend that you use
  \ccc{CGAL::cpp11::tuple} or \ccc{CGAL::cpp11::array} instead for new uses.

  \ccInclude{CGAL/utility.h}
  
  \ccRequirements \ccc{T1}, \ccc{T2}, \ccc{T3} and \ccc{T4} must be
  \ccc{Assignable}. Additional operations have additional requirements.

  %\ccSetThreeColumns{T*}{next_link ;}{}
  \ccThree{result_type;;}{operator()(argument a) ;;}{}

  \ccTypes
  \ccTypedef{typedef T1 first_type;}{}
  \ccGlue
  \ccTypedef{typedef T2 second_type;}{}
  \ccGlue
  \ccTypedef{typedef T3 third_type;}{}
  \ccGlue
  \ccTypedef{typedef T4 fourth_type;}{}

  \ccHeading{Variables}
  \ccVariable{T1 first;}{first element.  Please access it using \ccc{get<0>()}.}
  \ccGlue
  \ccVariable{T2 second;}{second element.  Please access it using \ccc{get<1>()}.}
  \ccGlue
  \ccVariable{T3 third;}{third element.  Please access it using \ccc{get<2>()}.}
  \ccGlue
  \ccVariable{T4 fourth;}{fourth element.  Please access it using \ccc{get<3>()}.}

  \ccCreation
  \ccCreationVariable{t}
  
  \ccConstructor{Quadruple();} {introduces a quadruple using the
    default constructor of the four elements.}
  
  \ccConstructor{Quadruple(T1 x, T2 y, T3 z, T4 w);} {constructs a
    quadruple such that \ccc{first} is constructed from \ccc{x},
    \ccc{second} is constructed from \ccc{y}, \ccc{third} is
    constructed from \ccc{z}, and \ccc{fourth} is constructed from
    \ccc{w}.}
  
  \ccConstructor{template <class U, class V, class W, class X>
    Quadruple(U u, V v, W w, X x);} {constructs a quadruple such that
    \ccc{first} is constructed from \ccc{u}, \ccc{second} is
    constructed from \ccc{v}, \ccc{third} is constructed from \ccc{w},
    and \ccc{fourth} is constructed from \ccc{x}. \ccRequire Proper
    conversion operators exist from \ccc{U} to \ccc{T1}, \ccc{V} to
    \ccc{T2}, \ccc{W} to \ccc{T3}, and \ccc{X} to \ccc{T4}.  }

  \ccMemberFunction{template <int i> T get();}
    {Gives access to \ccc{first}, \ccc{second}, \ccc{third} or \ccc{fourth}
     whenever \ccc{i} is 0, 1, 2 or 3, via a, potentially const, reference.
     Note: \ccc{T} stands for \ccc{T1}, \ccc{T2}, \ccc{T3} or \ccc{T4}
     depending on \ccc{i}.}
   
  \ccFunction{template <class T1, class T2, class T3, class T4> bool
    operator<(Quadruple<T1, T2, T3, T4> x, Quadruple<T1, T2, T3, T4>
    y);} {The comparison operator. It uses lexicographic comparison:
    the return value is true if the first element of \ccc{x} is less
    than the first element of \ccc{y}, and false if the first element
    of \ccc{y} is less than the first element of \ccc{x}. If neither
    of these is the case, then it returns true if the second element
    of \ccc{x} is less than the second element of \ccc{y}, and false
    if the second element of \ccc{y} is less than the second element
    of \ccc{x}. If neither of these is the case, then it returns true
    if the third element of \ccc{x} is less than the third element of
    \ccc{y}, and false if the third element of \ccc{y} is less than
    the third element of \ccc{x}.  If neither of these is the case,
    then it returns the result of comparing the fourth elements of
    \ccc{x} and \ccc{y}. This operator may only be used if \ccc{T1},
    \ccc{T2}, \ccc{T3}, and \ccc{T4} define the comparison operator.}
  
  \ccFunction{template <class T1, class T2, class T3, class T4> bool
    operator==(Quadruple<T1, T2, T3, T4> x, Quadruple<T1, T2, T3, T4>
    y);} {The equality operator. The return value is true if and only
    the first elements of \ccc{x} and \ccc{y} are equal, the second
    elements of \ccc{x} and \ccc{y} are equal, the third elements of
    \ccc{x} and \ccc{y} are equal, and the fourth elements of \ccc{x}
    and \ccc{y} are equal.  This operator may only be used if
    \ccc{T1}, \ccc{T2}, \ccc{T3}, and \ccc{T4} define the equality
    operator.}
  
  \ccFunction{template <class T1, class T2, class T3, class T4>
    Quadruple<T1, T2, T3, T4> make_quadruple(T1 x, T2 y, T3 z, T4 w);}
  {Equivalent to \ccStyle{Quadruple<T1, T2, T3, T4>(x, y, z, w)}.}
  
  \ccFunction{template <class T1, class T2, class T3, class T4>
    Quadruple<T1, T2, T3, T4> make_tuple(T1 x, T2 y, T3 z, T4 w);}
  {Equivalent to \ccStyle{Quadruple<T1, T2, T3, T4>(x, y, z, w)}.}

\end{ccRefClass}

%% +=========================================================================+

\begin{ccRefClass}{Boolean_tag<bool value>}

\ccDefinition

Depending on \ccc{bool value} the class \ccRefName\ indicates that
something is \ccc{true} or \ccc{false} respectively.

\ccInclude{CGAL/tags.h}

\ccConstants

\ccVariable{static const bool value;}{}

\ccSeeAlso
\ccRefIdfierPage{CGAL::Tag_true} \\
\ccRefIdfierPage{CGAL::Tag_false}

\end{ccRefClass}


\begin{ccRefClass}{Tag_true}

\ccDefinition

The typedef \ccRefName\ is \ccc{Boolean_tag<true>}.
It is used to indicate, for example,
that a certain feature is available in a class.

\ccInclude{CGAL/tags.h}

\ccVariable{static const bool value;}{ is \ccc{true} }

\ccSeeAlso
\ccRefIdfierPage{CGAL::Boolean_tag<bool value>} \\
\ccRefIdfierPage{CGAL::Tag_false}

\end{ccRefClass}


\begin{ccRefClass}{Tag_false}

\ccDefinition

The typedef \ccRefName\ is \ccc{Boolean_tag<false>}.
It is used to indicate, for example,
that a certain feature is not available in a class.

\ccInclude{CGAL/tags.h}

\ccVariable{static const bool value;}{ is \ccc{false} }

\ccSeeAlso
\ccRefIdfierPage{CGAL::Boolean_tag<bool value>} \\
\ccRefIdfierPage{CGAL::Tag_true}

\end{ccRefClass}

\begin{ccRefClass}{Null_functor}
\label{Null_functor}

\ccDefinition
Class indicating the absence of a functor. 

\ccIsModel
\ccc{DefaultConstructible}

\ccSeeAlso
\ccRefConceptPage{AlgebraicStructureTraits}\\
\ccRefConceptPage{RealEmbeddableTraits}\\
\end{ccRefClass} 
\begin{ccRefClass}{Null_tag}
\label{Null_tag}

\ccDefinition
General tag indicating that non of any other possible tags is valid.  

\ccIsModel
\ccc{DefaultConstructible}

\ccSeeAlso
\ccRefConceptPage{AlgebraicStructureTraits}\\
\end{ccRefClass} 

%% +--------------------------------------------------------+
\ccParDims

%% EOF


