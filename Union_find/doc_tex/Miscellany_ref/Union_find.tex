\begin{ccRefClass}{Union_find<T,A>}\ccCreationVariable{P}

\ccDefinition

An instance \ccc{P} of the data type \ccc{Union_find<T,A>} is a
partition of values of type \ccc{T} into disjoint sets.  The template 
parameter \ccc{A} has to be a model of the allocator concept as defined 
in the C++ standard. It has a default argument \ccc{CGAL_ALLOCATOR(T)}.


\ccInclude{CGAL/Union_find.h}

\ccSetOneOfTwoColumns{6.5cm}

\ccTypes

\ccNestedType{value_type}{values stored in items (equal to \ccc{T}).}

\ccNestedType{handle}{handle to values.}

\ccNestedType{iterator}{iterator over values.}

There are also constant versions \ccc{const_handle} and \ccc{const_iterator}.


\ccNestedType{allocator}{allocator.}

\ccSetOneOfTwoColumns{6.5cm}

\ccCreation

\ccConstructor{Union_find<T,A>()}{creates an instance \ccc{P} of type
  \ccc{Union_find<T,A>} and initializes it to the empty partition.}

\ccSetTwoOfThreeColumns{2cm}{4cm}

\ccOperations

\ccMethod{allocator get_allocator() ;}{the allocator of \ccc{P}.}

\ccMethod{std::size_t number_of_sets() ;}{returns the number of disjoint
  sets of \ccc{P}.  }

\ccMethod{std::size_t size() ;}{returns the number of values of \ccc{P}.}

\ccMethod{std::size_t bytes() ;}{returns the memory consumed by \ccc{P}.}

\ccMethod{std::size_t size( const_handle p) ;}{returns the size of the set
  containing $p$.}

\ccMethod{void clear();}{reinitializes \ccc{P} to an empty partition.}

\ccMethod{handle make_set(const T& x);}{creates a new singleton set
  containing \ccc{x} and returns a handle to it.}

\ccMethod{handle push_back(const T& x) ;}{same as \ccc{make_set(x)}.}

\pagebreak
\ccMethod{template <class Forward_iterator> void
  insert(Forward_iterator first, Forward_iterator beyond) ;}{insert
  the range of values referenced by \ccc{[first,beyond)}.  \ccRequire
  value type of \ccc{Forward_iterator} is \ccc{T}.}

\ccMethod{handle find(handle p) ;}{}
\ccGlue
\ccMethod{const_handle find( const_handle p) ;}{returns a 
  canonical handle of the set that contains \ccc{p}, i.e., 
  \ccc{P.same_set(p,q)} iff \ccc{P.find(p)} and \ccc{P.find(q)} 
  return the same handle. 
  \ccPrecond \ccc{p} is a handle in \ccc{P}.}

\ccMethod{void unify_sets( handle p, handle q);}{unites the sets of
  partition \ccc{P} containing $p$ and $q$. \ccPrecond $p$ and $q$ are
  in \ccc{P}.}

\ccMethod{bool same_set( const_handle p, const_handle q) ;}{returns 
  true iff $p$ and $q$ belong to the same set of \ccc{P}. 
  \ccPrecond $p$ and $q$ are in \ccc{P}.}

\ccMethod{iterator begin() ;}{returns an iterator pointing to the
  first value of \ccc{P}.}

\ccMethod{iterator end() ;}{returns an iterator pointing beyond the
  last value of \ccc{P}.  }

\ccImplementation

\ccc{Union_find<T,A>} is implemented with union by rank and path
compression.  The running time for $m$ set operations on $n$ elements
is $O(n \alpha(m,n))$ where $\alpha(m,n)$ is the extremely slow growing
inverse of Ackermann's function.

\end{ccRefClass}


