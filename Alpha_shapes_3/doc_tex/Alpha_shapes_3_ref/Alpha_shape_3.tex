% ======================================================================
%
% Copyright (c) 1999 The GALIA Consortium
%
% This software and related documentation is part of the
% Computational Geometry Algorithms Library (CGAL).
%
% Every use of CGAL requires a license. Licenses come in three kinds:
%
% - For academic research and teaching purposes, permission to use and
%   copy the software and its documentation is hereby granted free of  
%   charge, provided that
%   (1) it is not a component of a commercial product, and
%   (2) this notice appears in all copies of the software and
%       related documentation.
% - Development licenses grant access to the source code of the library 
%   to develop programs. These programs may be sold to other parties as 
%   executable code. To obtain a development license, please contact
%   the GALIA Consortium (at cgal@cs.uu.nl).
% - Commercialization licenses grant access to the source code and the
%   right to sell development licenses. To obtain a commercialization 
%   license, please contact the GALIA Consortium (at cgal@cs.uu.nl).
%
% This software and documentation is provided "as-is" and without
% warranty of any kind. In no event shall the CGAL Consortium be
% liable for any damage of any kind.
%
% The GALIA Consortium consists of Utrecht University (The Netherlands),
% ETH Zurich (Switzerland), Free University of Berlin (Germany),
% INRIA Sophia-Antipolis (France), Martin-Luther-University Halle-Wittenberg
% (Germany), Max-Planck-Institute Saarbruecken (Germany),
% and Tel-Aviv University (Israel).
%
% ----------------------------------------------------------------------
%
% package       : Alpha_shapes_3
% author(s)     : Tran Kai Frank DA <Frank.Da@sophia.inria.fr>
%
% coordinator   : INRIA Sophia-Antipolis (<Mariette.Yvinec@sophia.inria.fr>)
%
% ======================================================================

\RCSdef{\alphashapeRevision}{$Revision$}
\RCSdefDate{\alphashapeDate}{$Date$}

%----------------------------------------------------------------------

\begin{ccRefClass} {Alpha_shape_3<Dt>}

\ccDefinition

The class \ccClassTemplateName\ represents the family of
alpha shapes of points in the 3D space for {\em all} positive
$\alpha$. It maintains an  underlying triangulation 
of the class \ccc{Dt} which
represents connectivity and order among its faces. Each
$k$-dimensional face of the \ccc{Dt} is associated with
an interval that specifies for which values of $\alpha$ the face
belongs to the alpha shape. 

Note that this class is at the same time used for {\em basic} and
for {\em weighted} Alpha Shapes\ccIndexMainItem[C]{Weighted_alpha_shapes_3}.

\ccInheritsFrom

\ccc{Dt}

This class is the underlying triangulation class.

The modifying functions \ccc{insert} and \ccc{remove} will overwrite
the inherited functions. At the moment, only the static version is implemented.

\ccTypes
\ccSetThreeColumns{Oriented_side}{}{\hspace*{10cm}}
\ccThreeToTwo

\ccNestedType{Gt}{the alpha shape traits type.}
it has to derive from a triangulation traits class.  
For example \ccc{Dt::Point} is a Point class. 

\ccTypedef{typedef Gt::FT NT;}{the number type of alpha values.}

\ccNestedType{Alpha_iterator}{A bidirectional and non-mutable iterator that allow to traverse 
the increasing sequence of different alpha values.
\ccPrecond Its \ccc{value_type} is \ccc{NT}}

%\ccNestedType{Alpha_shape_vertices_iterator}{A bidirectional and non-mutable iterator that allow to traverse 
%the vertices which belongs to the alpha shape for a fixed $\alpha$. 
%\ccPrecond Its \ccc{value_type} is \ccc{Dt::Vertex_handle}}

%\ccNestedType{Alpha_shape_facets_iterator}{A bidirectional and non-mutable iterator that allow to traverse 
%the facets which belongs to the alpha shape for a fixed $\alpha$. 
%\ccPrecond Its \ccc{value_type} is \ccc{Dt::Facet}}


\ccEnum{enum Mode {GENERAL, REGULARIZED};}
{ In GENERAL mode, the alpha complex  can have singular faces,
 i. e. faces of dimension $k$, for $k=(0,1,2)$
that are not subfaces of a $k+1$ face of the complex.
In  REGULARIZED mode, the complex is regularized, that is
singular faces are dropped and the alpha complex 
includes only a subset of the tetrahedral cells 
of the triangulation and the subfaces of those cells.}


\ccEnum{enum Classification_type {EXTERIOR, SINGULAR, REGULAR, INTERIOR};}
{Enum to classify the faces of the underlying 
triangulation with respect to the alpha shape. \\ 
In GENERAL mode, for $k=(0,1,2)$,
 each k-dimensional simplex of the triangulation
can be classified as EXTERIOR, SINGULAR, REGULAR
or INTERIOR.
In GENERAL mode a $k$ simplex is REGULAR if it is on the boundary
f the alpha complex and belongs to a $k+1$ simplex in this complex
and it is SINGULAR  if it is  a boundary simplex that is not included in a $k+1$ simplex of the complex. \\  
In REGULARIZED mode, for $k=(0,1,2)$
each k-dimensional simplex of the triangulation
can be classified as EXTERIOR, REGULAR or INTERIOR, i.e.
there is no singular faces.
A $k$ simplex is REGULAR if it is on the boundary of alpha complex
and belongs to a tetrahedral cell of the complex.}




\ccCreation
\ccCreationVariable{A}

\ccConstructor{Alpha_shape_3(FT alpha = 0,
		        Mode m = REGULARIZED);}
{Introduces an empty alpha shape data structure
\ccVar\ for and set the
current alpha value to \ccc{alpha} and the mode to \ccc{m}.}

\ccConstructor{ Alpha_shape_3(Dt& dt, 
	NT alpha = 0, 
	Mode m = REGULARIZED);}
{Build an alpha shape of mode \ccc{m} 
from the triangulation dt. 
Be carefull that this operation destroy the triangulation.}

\ccConstructor{template < class InputIterator >
		Alpha_shape_3(
			InputIterator first,
			InputIterator last,
                const FT& alpha = 0,
	        Mode m = REGULARIZED);}
{Buil an alpha shape data structure in mode \ccc{m} 
for the points in the range
$\left[\right.$\ccc{first}, \ccc{last}$\left.\right)$ and 
set the current alpha value to \ccc{alpha}.
\ccPrecond The \ccc{value_type} of \ccc{first} and
\ccc{last} is \ccc{Point} (the type point of the underlying 
triangulation.)}


\ccModifiers

\ccMethod{template < class InputIterator >
		int make_alpha_shape(
			InputIterator first,
			InputIterator last);}
{Initialize the alpha shape data structure
for  points in the range
$\left[\right.$\ccc{first}, \ccc{last}$\left.\right)$. 
Returns the number of data points inserted in the underlying
triangulation. \\ 
If the function is applied to an non-empty alpha shape data structure, it is cleared
before initialization.
\ccPrecond The \ccc{value_type} of \ccc{first} and
\ccc{last} is \ccc{Point}.}

\ccMethod{void
	clear();}
{Clears the structure.}

\ccMethod{FT
	set_alpha(const Coord_type& alpha);}
{Sets the $\alpha$-value to \ccc{alpha}.
 Returns the previous $\alpha$-value.
\ccPrecond \ccc{alpha} $\geq 0$.}

\ccMethod{Mode
	set_mode(Mode m = REGULARIZED );}
{Sets \ccVar\ in GENERAL  or REGULARIZED. 
Returns the previous mode. 
Changing the mode of an alpha shape data structure
entails a partial recomputation of the data structure.}

% dynamic version	
% 
% \ccMethod{Vertex_handle insert(const Point& p);}
% {Inserts point \ccc{p} in the alpha shape and returns the
% corresponding vertex of the underlying Delaunay triangulation.\\ If
% point \ccc{p} coincides with an already existing vertex, this
% vertex is returned and the alpha shape remains unchanged.\\ Otherwise,
% the vertex is inserted in the underlying Delaunay triangulation and
% the associated intervals are updated. }
% 
% \ccMethod{void remove(Vertex *v);}
% {Removes the vertex from the underlying Delaunay triangulation. The
% created hole is retriangulated and the associated intervals are
% updated.}
% 

% deprecated iterators on factes and vertices
%\ccMethod{Alpha_shape_vertices_iterator alpha_shape_vertices_begin();}{Starts at
%an arbitrary finite vertex which belongs to the alpha shape for a fixed $\alpha$.}
%\ccMethod{Alpha_shape_vertices_iterator alpha_shape_vertices_end();}{Past-the-end iterator.}
%\ccMethod{Alpha_shape_facets_iterator alpha_shape_facets_begin();}{Starts at
%an arbitrary finite facet which belongs to the alpha shape for a fixed
%$\alpha$. In regularised mode, facets are represented as a pair (f,i), where f is 
%an interior cell of the alpha shape.}
%\ccMethod{Alpha_shape_facets_iterator alpha_shape_facets_end();}{Past-the-end iterator.}

%\ccMethod{template  < class OutputIterator >
%	  OutputIterator get_alpha_shape_vertices(
%					 OutputIterator result);}
%{Writes the vertices of the alpha shape \ccVar\ for the current $\alpha$-value
%to the container where \ccc{result} refers to. 
%The \ccc{value_type} of \ccc{result} is \ccc{Vertex_handle}. 
%Returns an output iterator which is the end of the constructed range.}


\ccHeading{Query Functions}


\ccMethod{Mode
	get_mode(void) const;}
{Returns whether \ccVar\ is general or regularized.}

\ccMethod{const FT&
	get_alpha(void) const;}
{Returns the current $\alpha$-value.}

\ccMethod{const FT&  get_nth_alpha(int n) const;}
{Returns the $n$-th alpha-value, sorted in an increasing order.
 \ccPrecond \ccc{n} < number of alphas.}

\ccMethod{int number_of_alphas() const;}
{Returns the number of different alpha-values.}

\ccMethod{Classification_type
           classify(const Point& p, 
	const FT& alpha = get_alpha()) const;}
{Locates a point  \ccc{p} in the underlying triangulation and Classifies the 
associated k-face with respect to \ccc{alpha}.}

\ccMethod{Classification_type
           classify(Cell_handle f, const FT& alpha = get_alpha()) const;}
{Classifies the cell \ccc{f} of the underlying triangulation with
respect 
to \ccc{alpha}.}

\ccMethod{Classification_type
           classify(Facet f, const FT& alpha = get_alpha()) const;}
{Classifies the facet \ccc{e} of the underlying triangulation with
respect to \ccc{alpha}.}

\ccMethod{Classification_type
           classify(Cell_handle f, int i, const FT& alpha = get_alpha()) const;}
{Classifies the facet of the cell \ccc{f} opposite to the vertex with index
\ccc{i} 
of the underlying triangulation with respect to \ccc{alpha}.}

\ccMethod{Classification_type
           classify(const Edge& e, const FT& alpha = get_alpha()) const;}
{Classifies the edge \ccc{e} with respect to \ccc{alpha} . }

\ccMethod{Classification_type
           classify(Vertex_handle v, const FT& alpha = get_alpha()) const;}
{Classifies the vertex \ccc{v} of the underlying triangulation with respect to \ccc{alpha}.}

\ccMethod {template<class OutputIterator>
  OutputIterator get_alpha_shape_cells(OutputIterator it, 
				       Classification_type type,
				 const NT& alpha =  get_alpha());}
{Write the cells which are of type \ccc{type} for 
the alpha value \ccc{alpha} to the sequence
pointed to by the output iterator \ccc{it}. Return past the end
of the output sequence.}


\ccMethod {
template<class OutputIterator>
  OutputIterator get_alpha_shape_facets(OutputIterator it, 
					Classification_type type,
				const NT& alpha=  get_alpha());}
{Write the facets which are of type \ccc{type} for 
the alpha value \ccc{alpha} to the sequence
pointed to by the output iterator \ccc{it}. Return past the end
of the output sequence.}

\ccMethod {
template<class OutputIterator>
  OutputIterator get_alpha_shape_edges(OutputIterator it, 
				       Classification_type type,
			const NT& alpha =  get_alpha());}
{Write the edges which are of type \ccc{type} for 
the alpha value \ccc{alpha} to the sequence
pointed to by the output iterator \ccc{it}. Return past the end
of the output sequence.}

\ccMethod {
 template<class OutputIterator>
   OutputIterator get_alpha_shape_vertices(OutputIterator it, 
					Classification_type type,
			const NT& alphaget_alpha());} 
{Write the vertices which are of type \ccc{type} for 
the alpha value \ccc{alpha} to the sequence
pointed to by the output iterator \ccc{it}. Return past the end
of the output sequence.}

\ccMethod {
template<class OutputIterator> 
   OutputIterator filtration(OutputIterator it);}
{Output all the faces  of the triangulation
in increasing order of  the alpha value for which they appear
in the alpha complex. In case of equal alpha value
lower dimensional faces are output first.
The value type of the OutputIterator has to be a  CGAL::Object}


\ccHeading{Traversal of the $\alpha$-Values}

\ccMethod{Alpha_iterator alpha_begin() const;}
{Returns an iterator that allows to traverse the
sorted sequence of $\alpha$-values of the family of alpha shapes.}

\ccMethod{Alpha_iterator alpha_end() const;}
{Returns the corresponding past-the-end iterator.}

\ccMethod{Alpha_iterator alpha_find(const FT& alpha) const;}
{Returns an iterator pointing to an element with $\alpha$-value
\ccc{alpha}, or the corresponding past-the-end iterator if such 
an element is not found.}

\ccMethod{Alpha_iterator alpha_lower_bound(const FT& alpha) const;}
{Returns an iterator pointing to the first element with
$\alpha$-value not less than \ccc{alpha}.}

\ccMethod{Alpha_iterator alpha_upper_bound(const FT& alpha) const;}
{Returns an iterator pointing to the first element with $\alpha$-value
greater than \ccc{alpha}.}

\ccHeading{Operations}

\ccMethod{int number_of_solid_components(const FT& alpha = get_alpha()) const;}
{Returns the number of solid components of \ccVar, that is, the number of
components of its 
regularized version.}

\ccMethod{Alpha_iterator find_optimal_alpha(int nb_components) const;}
{Returns an iterator pointing to smallest $\alpha$ value
such that \ccVar\ satisfies the following two properties:\\
all data points are either on the boundary or in the interior 
of the regularized version of \ccVar. \\
The number of solid component of \ccVar\  is  equal to or
smaller than \ccc{nb_components}.}

\ccHeading{I/O}

The I/O operators are defined for \ccc{iostream}, and for
the window stream provided by \cgal. The format for the iostream
is an internal format. 

\ccInclude{CGAL/IO/io.h}

\ccFunction{ostream& operator<<(ostream& os,
                  const Alpha_shape_3<Dt>& A);}
{Inserts the alpha shape \ccVar\ for the current alpha value into the stream \ccc{os}.
\ccPrecond The insert operator must be defined for \ccc{Point}.}

\ccInclude{CGAL/IO/Geomview_stream.h}

\ccInclude{CGAL/IO/alpha_shape_geomview_ostream_3.h}

\ccFunction{Geomview_stream& operator<<(Geomview_stream& W,
                         	const Alpha_shape_3<Dt>& A);}
{Inserts the alpha shape \ccVar\ for the current alpha value into the Geomview stream \ccc{W}.
\ccPrecond The insert operator must be defined for \ccc{Point} and \ccc{Triangle}.}

\ccImplementation

In GENERAL mode, the alpha  intervals of each triangulation
face is computed and stored at initialization time.
In REGULARIZED mode, the alpha shape intervals of edges
are not stored nor computed at initialization.
Edges are simply classified on the fly upon request.
This allows to have much faster building of alpha shapes in
REGULARIZED mode.


\ccc{A.alpha find} uses linear search, while 
\ccc{A.alpha lower bound} and \ccc{A.alpha upper bound} 
use binary search.
\ccc{A.number of solid components} performs a graph traversal and takes time
linear in the number of cells of the underlying triangulation.
\ccc{A.find of optimal alpha} uses binary search and takes time
$O(\mbox{ \em n } \log{\mbox{ \em n } })$, where  $n$ is the number of points.

\end{ccRefClass}

