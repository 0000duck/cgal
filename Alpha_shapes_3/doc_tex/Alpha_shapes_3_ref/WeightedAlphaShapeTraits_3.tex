% +------------------------------------------------------------------------+
% | Reference manual page: WeightedAlphaShapeTraits_3.tex
% +------------------------------------------------------------------------+
% | 13.08.2004   Author Mariette Yvinec
% | Package: Alpha_shapes_3
% | 
\RCSdef{\RCSWeightedAlphaShapeTraitsRev}{$Revision$}
\RCSdefDate{\RCSWeightedAlphaShapeTraitsDate}{$Date$}
% |
%%RefPage: end of header, begin of main body
% +------------------------------------------------------------------------+


\begin{ccRefConcept}{WeightedAlphaShapeTraits_3}

%% \ccHtmlCrossLink{}     %% add further rules for cross referencing links
%% \ccHtmlIndexC[concept]{} %% add further index entries

\ccDefinition
  
The concept \ccRefName\ describes the requirements 
for the geometric traits class
of  the underlying  regular triangulation of a weighted alpha shape.

\ccGeneralizes
\ccc{RegularTriangulationTraits_3}

In addition to the requirements described in the concept 
\ccc{RegularTriangulationTraits_3},
the geometric traits class of a
Regular triangulation plugged in a basic alpha shapes
provides the following.

\ccTypes


\ccNestedType{NT} {A number type compatible with the type used for
  the points coordinates.}
\ccNestedType{Compute_squared_radius_3}
{An object constructor able to compute the squared radius of the
 smallest sphere  orthogonal to  four weighted points \ccc{p0, p1, p2, p3},
and the squared radius of the
smallest sphere  orthogonal to  three  weighted points \ccc{p0, p1, p2},
and the squared radius of smallest  sphere orthogonal to 
two weighted points \ccc{p0, p1}.} 

\ccCreation
\ccCreationVariable{wast}  %% choose variable name

\ccConstructor{WeightedAlphaShapeTraits_3();}{default constructor.}

\ccAccessFunctions
\ccMethod{Compute_squared_radius_3 compute_squared_radius_3_object();} {}


\ccHasModels

\ccc{CGAL::Weighted_alpha_shape_euclidean_traits_3<K>},

\end{ccRefConcept}

% +------------------------------------------------------------------------+
%%RefPage: end of main body, begin of footer
% EOF
% +------------------------------------------------------------------------+

