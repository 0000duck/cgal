% ======================================================================
%
% Copyright (c) 1999 The GALIA Consortium
%
% This software and related documentation is part of the
% Computational Geometry Algorithms Library (CGAL).
%
% Every use of CGAL requires a license. Licenses come in three kinds:
%
% - For academic research and teaching purposes, permission to use and
%   copy the software and its documentation is hereby granted free of  
%   charge, provided that
%   (1) it is not a component of a commercial product, and
%   (2) this notice appears in all copies of the software and
%       related documentation.
% - Development licenses grant access to the source code of the library 
%   to develop programs. These programs may be sold to other parties as 
%   executable code. To obtain a development license, please contact
%   the GALIA Consortium (at cgal@cs.uu.nl).
% - Commercialization licenses grant access to the source code and the
%   right to sell development licenses. To obtain a commercialization 
%   license, please contact the GALIA Consortium (at cgal@cs.uu.nl).
%
% This software and documentation is provided "as-is" and without
% warranty of any kind. In no event shall the CGAL Consortium be
% liable for any damage of any kind.
%
% The GALIA Consortium consists of Utrecht University (The Netherlands),
% ETH Zurich (Switzerland), Free University of Berlin (Germany),
% INRIA Sophia-Antipolis (France), Martin-Luther-University Halle-Wittenberg
% (Germany), Max-Planck-Institute Saarbruecken (Germany),
% and Tel-Aviv University (Israel).
%
% ----------------------------------------------------------------------
%
% package       : Alpha_shapes_3
% author(s)     : Tran Kai Frank DA <Frank.Da@sophia.inria.fr>
%
% coordinator   : INRIA Sophia-Antipolis (<Mariette.Yvinec@sophia.inria.fr>)
%
% ======================================================================

\RCSdef{\alphashapeRevision}{$Revision$}
\RCSdefDate{\alphashapeDate}{$Date$}

%----------------------------------------------------------------------

\begin{ccRefConcept}{AlphaShapeCell_3}

\ccDefinition
This concept describes the requirements for the base cell of an alpha shape.


\ccRefines
\ccc{TriangulationCellBase_3}.


\ccCreationVariable{f}

\ccNestedType{NT}{A number type. Must be the same as the number type used
in the traits class of the triangulation underlying the alpha shape.}

\ccNestedType {Alpha_status_iterator}{ An iterator with value type \ccc{CGAL::Alpha_status<NT>}.}


\ccCreation


\ccConstructor{Alpha_shape_cell_base_3();}{default constructor.}
\ccGlue
\ccConstructor{Alpha_shape_cell_base_3(void* v0, void* v1, void* v2, void* v3);}{constructor setting the incident vertices.}
\ccGlue
\ccConstructor{Alpha_shape_cell_base_3(void* v0, void* v1, void*  v2, void* v3,
void* n0, void* n1, void* n2, void* n3);} 
{constructor setting the incident vertices and the neighboring cells.}



\ccAccessFunctions
\ccMethod{NT get_alpha();}{Returns the alpha value of the cell.}
\ccMethod{ Alpha_status_iterator get_facet_status(int i);}
{Returns an iterator on the \ccc{CGAL::Alpha_status<NT>} of the facet
\ccc{i} of the cell.}


\ccModifiers
\ccMethod{void set_alpha(const NT &  alpha);}
 { Sets the critical value of the cell.}
\ccMethod{void set_facet_status(int i, Alpha_status_iterator as);}
{Sets the iterator pointing to the \ccc{CGAL::Alpha_status<NT>}
of the facet \ccc{i} of the cell.}

\ccSeeAlso
\ccc{CGAL::Alpha_status} 

\end{ccRefConcept}
