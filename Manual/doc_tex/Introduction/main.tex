
\chapter{General Introduction}

\cgal, the {\em Computational Geometry Algorithms Library}, is written in 
\CC\ and consists of several parts.

The first part is about the kernels, which consist of constant-size non-modifiable 
geometric primitive objects and operations on these objects. 
The objects are represented both as stand-alone classes that are
parameterized by a representation class, which specifies
the underlying number types used for calculations and as members of the
kernel classes, which allows for more flexibility and adaptability of the 
kernel.  \cgal\ has several kernels, for 2D and 3D, for arbitrary dimensional
objects, and for 2D curved objects.

The following parts present a collection of basic geometric data structures and
algorithms, which are parameterized by traits classes that define the 
interface between the data structure or algorithm and the primitives they use.
In many cases, the kernel classes provided in \cgal\ can be used as traits
classes for these data structures and algorithms.
The collection of basic geometric algorithms and data structures
currently includes polygons, half-edge data structures, polyhedral surfaces, 
arrangements of curves, triangulations in 2D and 3D,  surface mesh
genrators, subdivision and parametrisation of suerface meshes,
Voronoi diagrams of points, disks and segments, Booelean operations on polygons and polyhedra,
convex hulls, alpha shapes, optimisation algorithms, dynamic
point sets for geometric queries,  range and segment trees, and 
kinetic data atructures.

The last part of the library consists of non-geometric support
facilities, such as support for number types, {\sc Stl} extensions for
\cgal, handles, circulators, protected access to internal
representations, geometric object generators, timers, I/O stream
operators and other stream support including PostScript, colors,
windows, and visualization tools Geomview and a Qt widget for
2D \cgal\ objects.

Additional documents accompanying the \cgal\ distribution are the
`Installation Guide' and `The Use of \stl\ and \stl\ Extensions in
\cgal', which gives a manual style introduction to \stl\ constructs
such as iterators and containers, as well an extension, called
circulator, used in many places in \cgal. We also recommend the
standard text book by Austern~\cite{cgal:a-gps-98} for the \stl\ and
its notion of \emph{concepts} and \emph{models}.


Other resources for \cgal\ are the tutorials at
\path|http://www.cgal.org/Tutorials/| and the user support page at
\path|www.cgal.org|.


\section{License Issues}

\cgal\ is Open Source software, and consists of different parts covered by
different licenses.  In this section we explain the essence of the different
licenses, as well as the rationale why we have chosen them. 

The fact that \cgal\ is Open Source software does not mean that users are free
to do whatever they want with the software. Using the software means to accept
the license, which has the status of a contract between the user and the owner
of the \cgal\ software.  A more detailed description of the license terms is
available in the \cgal\ software tarball.


\subsection{QPL \label{licenses:QPL}}

The {\sc Qpl} is an Open Source license that obliges you to distribute your
software based on {\sc Qpl}ed \cgal\ data structures.  The rationale behind
this is that we can claim access to your software.  The license further obliges
you to put your software under an Open Source license as well. The rationale
behind is that we can distribute your software, even if this is not your
intention.  Finally, the {\sc Qpl} requires that, if you modify \cgal, you
distribute the modifications in the form of patches and you distribute the
sources of your changes as well.


The exact license terms as well as an annotated version of the license can be
found at the Trolltech web site: \path'http://www.trolltech.com/products/qt/licenses/licensing/qpl'
and \path'http://www.trolltech.com/products/qt/licenses/licensing/qpl-annotated'

\subsection{LGPL \label{licenses:LGPL}}

The {\sc Lgpl} is an Open Source license that obliges you to distribute
modifications you make on \cgal\ software accessible to the users. There is no
obligation to make the source code of software you build on top of {\sc Lgpl}ed
\cgal\ data structures available.

Currently the linear kernel, the support library, the halfedge data structure,
the kinetic data structures, and the mesh subdivision framework are distributed
under the {\sc Lgpl}. The rationale behind is that we want to promote them as
de facto standards.

The exact license terms can be found at the Free Software Foundation web site:
\path'http://www.gnu.org/copyleft/lesser.html'.

\subsection{Commercial Licenses \label{licenses:Commercial}}

Users who cannot comply to the Open Source license terms can buy individual
data structures under various commercial licenses from GeometryFactory:
\path'http://www.geometryfactory.com'.


\subsection{License Compatibility \label{licenses:Compatibility}}

The General Public License ({\sc Gpl}) has a viral effect which makes it
incompatible with the {\sc Qpl}. For more information, please refer to the
paragraph about the {\sc Qpl} on the licenses web page of the Free Software
Foundation ({\sc Fsf}): \path'http://www.fsf.org/licensing/licenses/index_html'. 
It is therefore not possible to build a program including {\sc Gpl}
code and some {\sc Qpl} parts of {\sc Cgal}. In this case, if you are
the copyright owner of the {\sc Gpl} code, you can amend the license
by adding an exception allowing the use of {\sc Cgal} with it (see
again the {\sc Fsf} web page).




\section{Third Party Software}

In this section we list the software that is used by the various
\cgal\ packages.


\subsection{Standard Template Library \label{thirdparty:stl}}

\cgal\ heavily uses the {\sc Stl}, and in particular adopted
many of its design ideas.   The {\sc Stl} comes with the compiler,
but it is possible to use the compiler together with an
alternative {\sc Stl} implementation. You can find online
documentation for the {\sc Stl} at various web sites, e.g., 
\path+http://www.sgi.com/tech/stl/+, \path+http://www.cplusplus.com/reference/stl/+,
or \path+http://msdn.microsoft.com/en-us/library/1fe2x6kt(VS.71).aspx+.


\subsection{Boost \label{thirdparty:Boost}}

Boost is a collection of libraries. \cgal\ needs some of them, that is
it is mandatory.  If Boost is not already on your system, e.g., on
Windows, you can download it from \path'http://www.boost.org'.


\subsection{Blas \label{thirdparty:Blas}}

The \blas\ (Basic Linear Algebra Subprograms) are routines that provide
standard building blocks for performing basic vector and matrix operations.
In \cgal, \blas\ is required by the packages
\ccRef[Estimation of Local Differential Properties]{Pkg:Jet_fitting_3}
and \ccRef[Approximation of Ridges and Umbilics]{Pkg:Ridges_3} only.

You can download the official release from \path'http://www.netlib.org/blas/'
or download optimized implementations from \path'http://www.netlib.org/blas/faq.html#5'.
Alternatively, installing \taucs\ provides \blas.


\subsection{Lapack \label{thirdparty:Lapack}}

\lapack\ provides routines for solving systems of simultaneous linear equations,
least-squares solutions of linear systems of equations, eigenvalue problems,
and singular value problems.
In \cgal, \lapack\ is required by the packages
\ccRef[Estimation of Local Differential Properties]{Pkg:Jet_fitting_3}
and \ccRef[Approximation of Ridges and Umbilics]{Pkg:Ridges_3} only.

You can download the official release from \path'http://www.netlib.org/lapack/'.
Alternatively, installing \taucs\ customized for \cgal\ provides \lapack.


\subsection{GMP \label{thirdparty:GMP}}

A library for multi precision integers and rational numbers.
\cgal\ offers adapters for these number types. The usage
of the {\sc Gmp} library is optional.  If it is not already on your system,
e.g., on Windows, you can download it from \path'http://gmplib.org/'
or from the download section of \path'http://www.cgal.org'.


\subsection{MPFR \label{thirdparty:MPFR}}

A library for multi precision floating point numbers.  The usage of
the {\sc Mpfr} library is optional, and you must install it when you
use {\sc Gmp}.  You can download {\sc Mpfr} from \path'http://www.mpfr.org'
or from the download section of \path'http://www.cgal.org'.

\subsection{RS \label{thirdparty:RS}}

\rs{} stands for Real Solutions and is devoted to the study of the real
roots of polynomial systems with a finite number of complex roots
(including univariate polynomials). \rs{} is used only by one model of the
\ccRef[Algebraic Kernel]{Pkg:AlgebraicKerneld}.

\rs{} is freely distributable for non-commercial use. You can download it
from \rspage{}.

\subsection{Leda \label{thirdparty:Leda}}

A library of efficient data structures and algorithms. \cgal\ offers
adapters to the {\sc Leda} number types. The usage is optional.
It is available commercially from \path'http://www.algorithmic-solutions.com',
and there exists a binary ``free edition''.


\subsection{Taucs \label{thirdparty:Taucs}}

\taucs\ is a library of sparse linear solvers.
In \cgal, it is used to improve the computations within the
\ccRef[Planar Parameterization of Triangulated Surface Meshes]{Pkg:SurfaceParameterization}
package only.

The \taucs\ web site is \path'http://www.tau.ac.il/~stoledo/taucs/'.\\
The latest official version is \taucs\ version 2.2, September 4, 2003.
Copyright (c) 2001, 2002, 2003 by Sivan Toledo, Tel-Aviv University,
stoledo@tau.ac.il. All Rights Reserved.\\
See \path'http://www.tau.ac.il/~stoledo/taucs/' for the license and the availability note.\\
Used by permission of Sivan Toledo.

The \cgal\ project provides a modified version of \taucs\ in the download
section of \path'http://www.cgal.org'. This version fixes some bugs,
supports 64-bit platforms and allows a simplified installation process.
It also contains a complete \lapack\ implementation.\\
{\em CAUTION:} Since version 3.3.1, \cgal\ is no longer compatible with the official
release of \taucs\ (currently 2.2). Make sure to use the modified
version provided in the download section.


\subsection{OpenNL \label{thirdparty:OpenNL}}

OpenNL (Open Numerical Library) is a library to easily construct and solve
sparse linear systems. It is the default solver of the
\ccRef[Surface Mesh Parameterization]{Pkg:SurfaceParameterization} package.

OpenNL's main page is \path'http://www.loria.fr/~levy/software/'.

\cgal\ includes a version of OpenNL in C++, made especially for \cgal\ by Bruno L\'evy.


\subsection{zlib \label{thirdparty:zlib}}

A data compression library.
It is used in the examples of the \ccRef[Surface Mesh Generation]{Pkg:SurfaceMesher3} package.
If it is not already on your system,
e.g., on Windows, you can download it from  \path'http://www.gzip.org/zlib'.

\subsection{Qt \label{thirdparty:Qt}}

Qt is a cross-platform application framework. The usage of Qt is optional, but note that
it is used for many \cgal\ 2D as well as 3D demos.

As Qt is the layer underneath {\sc Kde}, Qt is installed
on many Linux systems. Otherwise you can download it from
 \path'http://qt.nokia.com/'.

\subsection{libQGLViewer \label{thirdparty:libQGLViewer}}

A 3D widget based on \qt~4's \ccc{QGLWidget}. It can be downloaded from
\path'http://www.libqglviewer.com/'.

\subsection{Coin \label{thirdparty:Coin}}

An implementation of Open Inventor.  It is used in the demo
of the \ccRef[Kinetic Data Structures]{Pkg:Kds} package. You can download
it from \path'http://www.coin3d.org/'.



\section{Advanced}

In this manual you will encounter sections marked as follows.

\begin{ccAdvanced}
Some functionality is considered more advanced.
Such functionality is described in sections such as this one that are bounded 
by horizontal brackets.
\end{ccAdvanced}


\section{Namespace CGAL}

All names introduced by \cgal, especially those documented in these
manuals, are in a namespace called \ccc{CGAL}, which is in global
scope. A user can either qualify names from \cgal\ by adding
\ccc{CGAL::}, e.g., \ccc{CGAL::Point_2< CGAL::Homogeneous< int> >},
make a single name from \cgal\ visible in a scope via a \ccc{using}
statement, e.g., \ccc{using CGAL::Cartesian;}, and then use this name
unqualified in this scope, or even make all names from namespace
\ccc{CGAL} visible in a scope with \ccc{using namespace CGAL;}. The
latter, however, is likely to give raise to name conflicts and is
therefore not recommended.


\section{Inclusion Order of Header Files}

Not all compilers fully support standard header names. \cgal\ provides 
workarounds for these problems in \ccc{CGAL/basic.h}. Consequently, as a 
golden rule, you should always include \ccc{CGAL/basic.h} first in your 
programs (or \ccc{CGAL/Cartesian.h}, or \ccc{CGAL/Homogeneous.h}, since they 
include \ccc{CGAL/basic.h} first).






\section{Compile-time Flags to Control Inlining}
\ccIndexMainItem{code optimization}
\ccIndexMainItem{inlining}
\ccIndexMainItem{\tt inline}

Making funcitons inlined can, at times, improve the efficiency of your code.
However this is not always the case and it can differ for a single function
depending on the application in which it is used. Thus \cgal\ defines a set 
of compile-time macros that can be used to control whether certain functions 
are designated as inlined functions or not.  The following table lists the 
macros and their default values, which are set in one of the \cgal\ include
files.  

\begin{tabular}{l|l}
               macro name        & default \\ \hline
\ccc{CGAL_KERNEL_INLINE}         & inline \\
\ccc{CGAL_KERNEL_MEDIUM_INLINE}  &  \\
\ccc{CGAL_KERNEL_LARGE_INLINE}   &  \\
\ccc{CGAL_MEDIUM_INLINE}         & inline \\
\ccc{CGAL_LARGE_INLINE}          &  \\
\ccc{CGAL_HUGE_INLINE}           & 
\end{tabular}

If you wish to change the value of one or more of these macros,
you can simply give it a new value when compiling.  For example, to make
functions that use the macro \ccc{CGAL_KERNEL_MEDIUM_INLINE} inline functions,
you should set the value of this macro to \texttt{inline} instead of the
default blank. 






% =============================================================================
% The CGAL Developers' Manual
% Chapter: Checks: Pre- and Postconditions, Assertions, and Warnings
% -----------------------------------------------------------------------------
% file   : checks.tex
% authors: Sven Sch�nherr <sven@inf.ethz.ch>
% -----------------------------------------------------------------------------
% $Id$
% $Date$
% =============================================================================

\index{preconditions!\seealso{checks}|none}
\index{postconditions!\seealso{checks}|none}
\index{assertions!\seealso{checks}|none}
\index{warnings!\seealso{checks}|none}

\chapter{Checks: Pre- and Postconditions, Assertions, and Warnings\label{chap:checks}}
\ccIndexMainItemBegin{checks}

\ccChapterRelease{Chapter Version: 1.1}
\ccChapterAuthor{Sven Sch{\"o}nherr ({\tt sven@inf.ethz.ch})}

%
Much of the {\cgal} code contains checks.  Some are there to check if
the code behaves correctly, others check if the user calls routines in
an acceptable manner. We describe the different categories of checks
(Section~\ref{sec:checks_categories}), the usage of checks
(Section~\ref{sec:checks_using}), and a more selective means of controlling
checks (Section~\ref{sec:checks_controlling}). Finally, a statement
about exception handling is given
(Section~\ref{sec:exception_handling}).

It is forbidden to call \ccc{std::abort}, \ccc{std::exit}
or \ccc{assert} directly from \cgal, as these do not allow the user code to
react after the error (application processes are killed).
Thus, the default behavior of all checks is to throw exceptions for reporting
failures.

% -----------------------------------------------------------------------------
\section{Categories of checks\label{sec:checks_categories}}
\ccIndexSubitemBegin{checks}{types of}

There are five types of checks.
%
\begin{itemize}
\item \textbf{Preconditions}\ccIndexMainItemDef{preconditions}
      check if a routine has been called in a proper fashion and the input
      adheres to the specifications given by the author of the function.
      If a precondition fails, it is the responsibility of the caller
      (usually the user of the library) to fix the problem.
\item \textbf{Postconditions}\ccIndexMainItemDef{postconditions}
      check if a routine does what it promises
      to do.  If a postcondition fails it is the fault of this routine, so
      the author of the code is responsible.
\item \textbf{Assertions}\ccIndexMainItemDef{assertions}
      are other checks that do not fit in the above
      two categories, \eg~they can be used to check invariants.
\item \textbf{Warnings}\ccIndexMainItemDef{warnings}
      are checks for which it is not so severe if they fail.
\item \textbf{Static assertions}\ccIndexMainItemDef{static assertions}
      are compile-time assertions, used \eg~to verify the values of compile-time
      constants or compare types for (in)equality.
\end{itemize}

The according macro names all have the format \ccc{CGAL_<check_type>} where
\ccc{<check_type>} can be one of
\begin{itemize}
 \item \ccc{precondition}
 \item \ccc{postcondition}
 \item \ccc{assertion}
 \item \ccc{warning}
 \item \ccc{static\_assertion}
\end{itemize}

%
Failures of the first three types are errors and lead to a halt of the
program, failures of the last one only lead to a warning.  Checks of
four categories can be marked with one or both of the following
attributes:
%
\begin{itemize}
\item \textbf{Expensive}\ccIndexSubitemDef{checks}{expensive}
      checks take considerable time to compute.
      ``Considerable'' is an imprecise phrase.  Checks that add less than 10
      percent to the execution time of their routine are not expensive.
      Checks that can double the execution time are. Somewhere in between
      lies the border line.
\item \textbf{Exactness}\ccIndexSubitemDef{checks}{exactness}%
      \ccIndexSubitem{exactness}{checking}
      checks rely on exact arithmetic. For example,
      if the intersection of two lines is computed, the postcondition of
      this routine may state that the intersection point lies on both
      lines.  However, if the computation is done with \ccc{double}s as
      the number type, this may not be the case, due to roundoff errors.
\end{itemize}
By definition, static assertions are both inexpensive and unaffected by precision
management. Thus, the categories do not apply for static assertions.

The format is one of
\begin{itemize}
 \item \ccc{CGAL_<check_type>}
 \item \ccc{CGAL_expensive_<check_type>}
 \item \ccc{CGAL_exactness_<check_type>}
 \item \ccc{CGAL_expensive_exactness_<check_type>}
\end{itemize}

\ccIndexSubitem{checks}{default}
By default, all standard checks (without any attribute) are enabled,
while expensive and exactness checks are disabled. How this can be
changed and how checks are actually used in the code are described in
the next section.
\ccIndexSubitemEnd{checks}{types of}

Additionally, we provide macros \ccc{CGAL_error()} and \ccc{CGAL_error_msg(MSG_TEXT)}
which are equivalent to always-failing assertions. However,
they cannot be disabled.


% -----------------------------------------------------------------------------
\section{Using checks\label{sec:checks_using}}
\ccIndexSubitemBegin{checks}{using}

The checks are implemented as preprocessor macros;
\ccIndexSubitem{macros}{for checks}
\ie,~\ccc{CGAL_<check_type>(<Cond>)} realizes a check of type \ccc{<check_type>}
that asserts the condition \ccc{<Cond>}. For example,
%
\begin{verbatim}
  CGAL_precondition( first != last);
\end{verbatim}
%
checks the precondition that a given iterator range is not empty. If
the check fails, an error message similar to
%
\begin{verbatim}
  CGAL error: precondition violation!
  Expr: first != last
  File: <file name>
  Line: <source code line>
\end{verbatim}
%
is written to the standard error stream and the program is aborted. If
an additional explanation should be given to the user,%
\ccIndexSubitem{checks}{adding failure message to}
macros \ccc{CGAL_<check_type>_msg(<Cond>,<Msg>)} can be used. The text in
\ccc{<Msg>} is just appended to the failure message given above.

\ccIndexSubitem{checks}{multiple-statement}
In case a check is more complicated and the computation does not fit
into a single statement, the additional code can be encapsulated using
\ccc{CGAL_<check_type>_code(<Code>)}.  This has the advantage that the
computation is not done if the corresponding category is disabled.  For
an example, suppose an algorithm computes a convex polygon. Thus we
want to check the postcondition that the polygon is indeed convex,
which we consider an expensive check. The code would look like this.
%
\begin{verbatim}
  CGAL_expensive_postcondition_code( bool is_convex; )
  CGAL_expensive_postcondition_code( /* compute convexity */ )
  CGAL_expensive_postcondition_code( /* ... */ )
  CGAL_expensive_postcondition_msg ( is_convex, \
                                     "The computed polygon is NOT convex!" );
\end{verbatim}

\ccIndexSubitemBegin{checks}{disabling}
As already mentioned above, the standard checks are enabled by
default.  This can be changed through the use of compile-time flags.
\ccIndexSubitem{compile-time flags}{for checks}
By setting the flag \ccc{CGAL_NO_<CHECK_TYPE>} all checks of type
\ccc{<CHECK_TYPE>} are disabled, \eg~adding \ccc{-DCGAL_NO_ASSERTIONS}%
\index{CGAL_NO_<CHECK_TYPE> flag@{\tt CGAL\_NO\_$<$CHECK\_TYPE$>$} flag}
to the compiler call switches off all checks for static and dynamic assertions.
To disable all checks in the library, the flag \ccc{CGAL_NDEBUG} can be set.
Note that the standard flag \ccc{NDEBUG} sets \ccc{CGAL_NDEBUG},
but it also affects the \ccc{assert} macro.
\ccIndexSubitemEnd{checks}{disabling}

To enable expensive and exactness checks, respectively, the compile-time
flags \ccc{CGAL_CHECK_EXPENSIVE}%
\index{CGAL_CHECK_EXPENSIVE flag@{\tt CGAL\_CHECK\_EXPENSIVE} flag}
\ccIndexSubsubitem{checks}{expensive}{enabling}
and \ccc{CGAL_CHECK_EXACTNESS}
\index{CGAL_CHECK_EXACTNESS flag@{\tt CGAL\_CHECK\_EXACTNESS} flag}
\ccIndexSubsubitem{checks}{exactness}{enabling}
have to be supplied. However, exactness checks should only be turned on if
the computation is done with some exact number type.%
\ccIndexSubitem{exactness}{checking}
\ccIndexSubitemEnd{checks}{using}


% -----------------------------------------------------------------------------
\section{Controlling checks at a finer granularity\label{sec:checks_controlling}}
\ccIndexSubitemBegin{checks}{package-level}

The macros and related compile-time flags described so far all operate
on the whole library. Sometimes the user may want to have a more
selective control.  \cgal\ offers the possibility to turn checks on
and off on a per-package basis. Therefore a package-specific term is
inserted in the macro names directly after the \cgal\ prefix,
\eg,~\ccc{CGAL_kernel_assertion(<Cond>)}. Similarly, the uppercase
term is used for the compile-time flags;
\eg,~\ccc{CGAL_KERNEL_NO_WARNINGS} switches off the warnings
in \emph{only} the kernel. Other packages have their own specific
terms as documented in the corresponding chapters of the
reference manual.

%
For a new package you will first have to create a suitable header file
with all macro definitions.  This is done with the shell script
\texttt{create\_assertions.sh} (Section~\ref{sec:create_assertions}).
\index{create_assertions.sh script@{\tt create\_assertions.sh} script}
The following command will create a file {\tt optimisation\_assertions.h}.

{\centerline{\tt  sh create\_assertions.sh optimisation}}

You should place the generated file in the proper directory (and possibly
rename it).  Then you can use the checks in the following fashion.

\lcTex{
\begin{tabbing}
xxx\=xx\=\kill
\>  \texttt{\#include $<$CGAL/optimisation\_assertions.h$>$} \\ \\

\>  \texttt{void optimisation\_foo( int i)} \\
\>  \texttt{\{} \\
\>\>\texttt{   CGAL\_optimisation\_precondition\_msg( i == 42, "Only 42 allowed!");} \\
\>\>\texttt{    // ...} \\
\>  \texttt{\}}
\end{tabbing}
}
\lcHtml{
\begin{verbatim}
 #include <CGAL/optimisation\_assertions.h>

 void optimisation\_foo( int i)
 {
   CGAL_optimisation_precondition_msg( i == 42, "Only 42 allowed!");
   // ...
 }
\end{verbatim}
}

\ccIndexSubsubitem{checks}{package-level}{documenting}
\ccIndexSubitem{documentation}{of checks}
The documentation of your new package has to name the term chosen to be
part of the package-specific macros in
order to enable the user to selectively turn off and on the checks of
your package. For example, in the documentation of the optimisation
package you can find a sentence similar to the following.
\begin{quote}
  The optimisation code uses the term OPTIMISATION for the checks;
  \eg,~setting the compile time flag
  \ccc{CGAL_OPTIMISATION_NO_PRECONDITIONS} switches off precondition
  checking in the optimisation code.
\end{quote}
\ccIndexSubitemEnd{checks}{package-level}


% -----------------------------------------------------------------------------
\section{Exception handling\label{sec:exception_handling}}
\ccIndexMainItem{exception handling}

Some parts of the library use exceptions, but there is no general specific
policy concerning exception handling in \cgal.  It is nevertheless good to
target exception safety, as much as possible.  Good references on exception
safety are: Appendix E of~\cite{cgal:s-cpl-97} (also available at
\path|http://www.research.att.com/~bs/3rd_safe0.html|),
and~\cite{cgal:a-esgc-98} (also available at
\path|http://www.boost.org/more/generic_exception_safety.html|).

\section{Requirements and recommendations\label{sec:checks_req_and_rec}}

\noindent
Requirements:
\begin{itemize}
   \item Write pre- and postcondition checkers for your functions wherever
         possible.
   \item Use the \cgal\ preprocessor macros (Sections~\ref{sec:checks_using}
         and~\ref{sec:checks_controlling})
         exclusively throughout your code (instead of,
         for example, the \ccc{assert} macro or the \ccc{std::abort} or
         \ccc{std::exit} functions) for all checks
         to assure that all \cgal\ invariant tests can be handled in a uniform
         way.
\end{itemize}

\ccIndexMainItemEnd{checks}
% ===== EOF ===================================================================
 % extra chapter
