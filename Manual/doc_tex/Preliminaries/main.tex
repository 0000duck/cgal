
\chapter{Preliminaries}
\ccChapterAuthor{CGAL Editorial Board}

\begin{ccPkgDescription}{3D Convex Hulls\label{Pkg:ConvexHull3}}
\ccPkgHowToCiteCgal{cgal:hs-ch3-07}
\ccPkgSummary{This package provides functions 
for computing convex hulls in three dimensions as well as functions
for checking if sets of points are strongly convex are not. One can
compute the convex hull of a set of points in three dimensions in one
of three ways: using a static algorithm, using an incremental
construction algorithm, or using a triangulation to get a fully
dynamic computation.}

\ccPkgDependsOn{All algorithms produce as output a \ccRef[3D Polyhedron]{Pkg:Polyhedron}. 
                The dynamic algorithms depend on \ccRef[3D Triangulations]{Pkg:Triangulation3}}
\ccPkgIntroducedInCGAL{1.1}
\ccPkgLicense{\ccLicenseQPL}
\ccPkgIllustration{Convex_hull_3/bunny.png}{Convex_hull_3/bunny.png}
\end{ccPkgDescription}


This chapter lists the licenses
under which the \cgal\ datastructures and algorithms are distributed.
The chapter further explains how to control inlining, thread safety, 
code deprecation, checking of pre- and postconditions,
and how to alter the failure behavior. 


\section{License Issues}

\cgal\ is Open Source software, and consists of different parts covered by
different licenses.  In this section we explain the essence of the different
licenses, as well as the rationale why we have chosen them. 

The fact that \cgal\ is Open Source software does not mean that users are free
to do whatever they want with the software. Using the software means to accept
the license, which has the status of a contract between the user and the owner
of the \cgal\ software.  A more detailed description of the license terms is
available in the \cgal\ software tarball.


\subsection{QPL \label{licenses:QPL}}

The {\sc Qpl} is an Open Source license that obliges you to distribute your
software based on {\sc Qpl}ed \cgal\ data structures.  The rationale behind
this is that we can claim access to your software.  The license further obliges
you to put your software under an Open Source license as well. The rationale
behind is that we can distribute your software, even if this is not your
intention.  Finally, the {\sc Qpl} requires that, if you modify \cgal, you
distribute the modifications in the form of patches and you distribute the
sources of your changes as well.


The exact license terms as well as an annotated version of the license can be
found at the Trolltech web site: \path'http://www.trolltech.com/products/qt/licenses/licensing/qpl'
and \path'http://www.trolltech.com/products/qt/licenses/licensing/qpl-annotated'

\subsection{LGPL \label{licenses:LGPL}}

The {\sc Lgpl} is an Open Source license that obliges you to distribute
modifications you make on \cgal\ software accessible to the users. There is no
obligation to make the source code of software you build on top of {\sc Lgpl}ed
\cgal\ data structures available.

Currently the linear kernel, the support library, the halfedge data structure,
the kinetic data structures, and the mesh subdivision framework are distributed
under the {\sc Lgpl}. The rationale behind is that we want to promote them as
de facto standards.

The exact license terms can be found at the Free Software Foundation web site:
\path'http://www.gnu.org/copyleft/lesser.html'.

\subsection{Commercial Licenses \label{licenses:Commercial}}

Users who cannot comply to the Open Source license terms can buy individual
data structures under various commercial licenses from GeometryFactory:
\path'http://www.geometryfactory.com'.


\subsection{License Compatibility \label{licenses:Compatibility}}

The General Public License ({\sc Gpl}) has a viral effect which makes it
incompatible with the {\sc Qpl}. For more information, please refer to the
paragraph about the {\sc Qpl} on the licenses web page of the Free Software
Foundation ({\sc Fsf}): \path'http://www.fsf.org/licensing/licenses/index_html'. 
It is therefore not possible to build a program including {\sc Gpl}
code and some {\sc Qpl} parts of {\sc Cgal}. In this case, if you are
the copyright owner of the {\sc Gpl} code, you can amend the license
by adding an exception allowing the use of {\sc Cgal} with it (see
again the {\sc Fsf} web page).






\section{Marking of Special Functionality}

In this manual you will encounter sections marked as follows.

\subsection{Advanced Features}

Some functionality is considered more advanced, for example because it is
relatively low-level, or requires special care to be properly used.

\begin{ccAdvanced}
Such functionality is identified this way in the manual.
\end{ccAdvanced}

\subsection{Debugging Support Features}

Usually related to advanced features that for example may not guarantee
class invariants, some functionality is provided that helps debugging,
for example by performing invariants checks on demand.

\begin{ccDebug}
Such functionality is identified this way in the manual.
\end{ccDebug}

\subsection{Deprecated Code}

Sometimes, the \cgal\ project decides that a feature is deprecated.  This means
that it still works in the current release, but it will be removed in the next,
or a subsequent release.  This can happen when we have found a better way to do
something, and we would like to reduce the maintenance cost of \cgal\ at some
point in the future.  There is a trade-off between maintaining backward
compatibility and implementing new features more easily.

In order to help users manage the changes to apply to their code, we attempt
to make \cgal\ code emit warnings when deprecated code is used.  This can be
done using some compiler specific features.  Those warnings can be disabled
by defining the macro \ccc{CGAL_NO_DEPRECATION_WARNINGS}.  On top of this, we
also provide a macro, \ccc{CGAL_NO_DEPRECATED_CODE}, which, when defined,
disables all deprecated features.  This allows users to easily test if their
code relies on deprecated features.

\begin{ccDeprecated}
Such functionality is identified this way in the manual.
\end{ccDeprecated}


\section{Namespace CGAL}

All names introduced by \cgal, especially those documented in these
manuals, are in a namespace called \ccc{CGAL}, which is in global
scope. A user can either qualify names from \cgal\ by adding
\ccc{CGAL::}, e.g., \ccc{CGAL::Point_2< CGAL::Exact_predicates_inexact_constructions_kernel >},
make a single name from \cgal\ visible in a scope via a \ccc{using}
statement, e.g., \ccc{using CGAL::Point_2;}, and then use this name
unqualified in this scope, or even make all names from namespace
\ccc{CGAL} visible in a scope with \ccc{using namespace CGAL;}. The
latter, however, is likely to give raise to name conflicts and is
therefore not recommended.


\section{Inclusion Order of Header Files}

Not all compilers fully support standard header names. \cgal\ provides 
workarounds for these problems in \ccc{CGAL/basic.h}. Consequently, as a 
golden rule, you should always include \ccc{CGAL/basic.h} first in your 
programs (or \ccc{CGAL/Cartesian.h}, or \ccc{CGAL/Homogeneous.h}, since they 
include \ccc{CGAL/basic.h} first).


\section{Thread Safety}

\cgal\ is progressively being made thread-safe.  The guidelines which are followed
are:
\begin{itemize}
\item it should be possible to use different objects in different threads at
the same time (of the same type or not),
\item it is not safe to access the same object from different threads
at the same time, unless otherwise specified in the class documentation.
\end{itemize}

If the macro \ccc{CGAL_HAS_THREADS} is not defined, then \cgal\ assumes it can use
any thread-unsafe code (such as static variables).  By default, this macro is not
defined, unless \ccc{BOOST_HAS_THREADS} or \ccc{_OPENMP} is defined.  It is possible
to force its definition on the command line, and it is possible to prevent its default
definition by setting \ccc{CGAL_HAS_NO_THREADS} from the command line.


\section{C++11 Support}

\cgal\ is based on the \CC\ standard released in 1998 (and later refined in 2003).
A new major version of this standard has been released, and is refered to as C++11.
Some compilers and standard library implementations already provide some of the
functionality of this new standard.  For example, \gcc\ provides
a command-line switch (\ccc{-std=c++0x}  or \ccc{-std=c++11} depending on the compiler version)
which enables some of those features.

\cgal\ attempts to support this mode progressively, and already makes use of
some of these features if they are available, although no extensive support has
been implemented yet.

\section{Functor Return Types}

\cgal\ functors support the
\ccAnchor{http://www.boost.org/doc/libs/release/libs/utility/utility.htm#result_of}{result\_of}
protocol. If a functor \ccStyle{F} has the same return type across all
overloads of \ccStyle{operator()}, the nested type
\ccStyle{F::result_type} is defined to be that type. Otherwise the
return type of calling the functor with an argument of type
\ccStyle{Arg} can be accessed through
\ccStyle{CGAL::cpp11::result_of<F(Arg)>::type}.

% =============================================================================
% The CGAL Developers' Manual
% Chapter: Checks: Pre- and Postconditions, Assertions, and Warnings
% -----------------------------------------------------------------------------
% file   : checks.tex
% authors: Sven Sch�nherr <sven@inf.ethz.ch>
% -----------------------------------------------------------------------------
% $Id$
% $Date$
% =============================================================================

\index{preconditions!\seealso{checks}|none}
\index{postconditions!\seealso{checks}|none}
\index{assertions!\seealso{checks}|none}
\index{warnings!\seealso{checks}|none}

\chapter{Checks: Pre- and Postconditions, Assertions, and Warnings\label{chap:checks}}
\ccIndexMainItemBegin{checks}

\ccChapterRelease{Chapter Version: 1.1}
\ccChapterAuthor{Sven Sch{\"o}nherr ({\tt sven@inf.ethz.ch})}

%
Much of the {\cgal} code contains checks.  Some are there to check if
the code behaves correctly, others check if the user calls routines in
an acceptable manner. We describe the different categories of checks
(Section~\ref{sec:checks_categories}), the usage of checks
(Section~\ref{sec:checks_using}), and a more selective means of controlling
checks (Section~\ref{sec:checks_controlling}). Finally, a statement
about exception handling is given
(Section~\ref{sec:exception_handling}).

It is forbidden to call \ccc{std::abort}, \ccc{std::exit}
or \ccc{assert} directly from \cgal, as these do not allow the user code to
react after the error (application processes are killed).
Thus, the default behavior of all checks is to throw exceptions for reporting
failures.

% -----------------------------------------------------------------------------
\section{Categories of checks\label{sec:checks_categories}}
\ccIndexSubitemBegin{checks}{types of}

There are five types of checks.
%
\begin{itemize}
\item \textbf{Preconditions}\ccIndexMainItemDef{preconditions}
      check if a routine has been called in a proper fashion and the input
      adheres to the specifications given by the author of the function.
      If a precondition fails, it is the responsibility of the caller
      (usually the user of the library) to fix the problem.
\item \textbf{Postconditions}\ccIndexMainItemDef{postconditions}
      check if a routine does what it promises
      to do.  If a postcondition fails it is the fault of this routine, so
      the author of the code is responsible.
\item \textbf{Assertions}\ccIndexMainItemDef{assertions}
      are other checks that do not fit in the above
      two categories, \eg~they can be used to check invariants.
\item \textbf{Warnings}\ccIndexMainItemDef{warnings}
      are checks for which it is not so severe if they fail.
\item \textbf{Static assertions}\ccIndexMainItemDef{static assertions}
      are compile-time assertions, used \eg~to verify the values of compile-time
      constants or compare types for (in)equality.
\end{itemize}

The according macro names all have the format \ccc{CGAL_<check_type>} where
\ccc{<check_type>} can be one of
\begin{itemize}
 \item \ccc{precondition}
 \item \ccc{postcondition}
 \item \ccc{assertion}
 \item \ccc{warning}
 \item \ccc{static\_assertion}
\end{itemize}

%
Failures of the first three types are errors and lead to a halt of the
program, failures of the last one only lead to a warning.  Checks of
four categories can be marked with one or both of the following
attributes:
%
\begin{itemize}
\item \textbf{Expensive}\ccIndexSubitemDef{checks}{expensive}
      checks take considerable time to compute.
      ``Considerable'' is an imprecise phrase.  Checks that add less than 10
      percent to the execution time of their routine are not expensive.
      Checks that can double the execution time are. Somewhere in between
      lies the border line.
\item \textbf{Exactness}\ccIndexSubitemDef{checks}{exactness}%
      \ccIndexSubitem{exactness}{checking}
      checks rely on exact arithmetic. For example,
      if the intersection of two lines is computed, the postcondition of
      this routine may state that the intersection point lies on both
      lines.  However, if the computation is done with \ccc{double}s as
      the number type, this may not be the case, due to roundoff errors.
\end{itemize}
By definition, static assertions are both inexpensive and unaffected by precision
management. Thus, the categories do not apply for static assertions.

The format is one of
\begin{itemize}
 \item \ccc{CGAL_<check_type>}
 \item \ccc{CGAL_expensive_<check_type>}
 \item \ccc{CGAL_exactness_<check_type>}
 \item \ccc{CGAL_expensive_exactness_<check_type>}
\end{itemize}

\ccIndexSubitem{checks}{default}
By default, all standard checks (without any attribute) are enabled,
while expensive and exactness checks are disabled. How this can be
changed and how checks are actually used in the code are described in
the next section.
\ccIndexSubitemEnd{checks}{types of}

Additionally, we provide macros \ccc{CGAL_error()} and \ccc{CGAL_error_msg(MSG_TEXT)}
which are equivalent to always-failing assertions. However,
they cannot be disabled.


% -----------------------------------------------------------------------------
\section{Using checks\label{sec:checks_using}}
\ccIndexSubitemBegin{checks}{using}

The checks are implemented as preprocessor macros;
\ccIndexSubitem{macros}{for checks}
\ie,~\ccc{CGAL_<check_type>(<Cond>)} realizes a check of type \ccc{<check_type>}
that asserts the condition \ccc{<Cond>}. For example,
%
\begin{verbatim}
  CGAL_precondition( first != last);
\end{verbatim}
%
checks the precondition that a given iterator range is not empty. If
the check fails, an error message similar to
%
\begin{verbatim}
  CGAL error: precondition violation!
  Expr: first != last
  File: <file name>
  Line: <source code line>
\end{verbatim}
%
is written to the standard error stream and the program is aborted. If
an additional explanation should be given to the user,%
\ccIndexSubitem{checks}{adding failure message to}
macros \ccc{CGAL_<check_type>_msg(<Cond>,<Msg>)} can be used. The text in
\ccc{<Msg>} is just appended to the failure message given above.

\ccIndexSubitem{checks}{multiple-statement}
In case a check is more complicated and the computation does not fit
into a single statement, the additional code can be encapsulated using
\ccc{CGAL_<check_type>_code(<Code>)}.  This has the advantage that the
computation is not done if the corresponding category is disabled.  For
an example, suppose an algorithm computes a convex polygon. Thus we
want to check the postcondition that the polygon is indeed convex,
which we consider an expensive check. The code would look like this.
%
\begin{verbatim}
  CGAL_expensive_postcondition_code( bool is_convex; )
  CGAL_expensive_postcondition_code( /* compute convexity */ )
  CGAL_expensive_postcondition_code( /* ... */ )
  CGAL_expensive_postcondition_msg ( is_convex, \
                                     "The computed polygon is NOT convex!" );
\end{verbatim}

\ccIndexSubitemBegin{checks}{disabling}
As already mentioned above, the standard checks are enabled by
default.  This can be changed through the use of compile-time flags.
\ccIndexSubitem{compile-time flags}{for checks}
By setting the flag \ccc{CGAL_NO_<CHECK_TYPE>} all checks of type
\ccc{<CHECK_TYPE>} are disabled, \eg~adding \ccc{-DCGAL_NO_ASSERTIONS}%
\index{CGAL_NO_<CHECK_TYPE> flag@{\tt CGAL\_NO\_$<$CHECK\_TYPE$>$} flag}
to the compiler call switches off all checks for static and dynamic assertions.
To disable all checks in the library, the flag \ccc{CGAL_NDEBUG} can be set.
Note that the standard flag \ccc{NDEBUG} sets \ccc{CGAL_NDEBUG},
but it also affects the \ccc{assert} macro.
\ccIndexSubitemEnd{checks}{disabling}

To enable expensive and exactness checks, respectively, the compile-time
flags \ccc{CGAL_CHECK_EXPENSIVE}%
\index{CGAL_CHECK_EXPENSIVE flag@{\tt CGAL\_CHECK\_EXPENSIVE} flag}
\ccIndexSubsubitem{checks}{expensive}{enabling}
and \ccc{CGAL_CHECK_EXACTNESS}
\index{CGAL_CHECK_EXACTNESS flag@{\tt CGAL\_CHECK\_EXACTNESS} flag}
\ccIndexSubsubitem{checks}{exactness}{enabling}
have to be supplied. However, exactness checks should only be turned on if
the computation is done with some exact number type.%
\ccIndexSubitem{exactness}{checking}
\ccIndexSubitemEnd{checks}{using}


% -----------------------------------------------------------------------------
\section{Controlling checks at a finer granularity\label{sec:checks_controlling}}
\ccIndexSubitemBegin{checks}{package-level}

The macros and related compile-time flags described so far all operate
on the whole library. Sometimes the user may want to have a more
selective control.  \cgal\ offers the possibility to turn checks on
and off on a per-package basis. Therefore a package-specific term is
inserted in the macro names directly after the \cgal\ prefix,
\eg,~\ccc{CGAL_kernel_assertion(<Cond>)}. Similarly, the uppercase
term is used for the compile-time flags;
\eg,~\ccc{CGAL_KERNEL_NO_WARNINGS} switches off the warnings
in \emph{only} the kernel. Other packages have their own specific
terms as documented in the corresponding chapters of the
reference manual.

%
For a new package you will first have to create a suitable header file
with all macro definitions.  This is done with the shell script
\texttt{create\_assertions.sh} (Section~\ref{sec:create_assertions}).
\index{create_assertions.sh script@{\tt create\_assertions.sh} script}
The following command will create a file {\tt optimisation\_assertions.h}.

{\centerline{\tt  sh create\_assertions.sh optimisation}}

You should place the generated file in the proper directory (and possibly
rename it).  Then you can use the checks in the following fashion.

\lcTex{
\begin{tabbing}
xxx\=xx\=\kill
\>  \texttt{\#include $<$CGAL/optimisation\_assertions.h$>$} \\ \\

\>  \texttt{void optimisation\_foo( int i)} \\
\>  \texttt{\{} \\
\>\>\texttt{   CGAL\_optimisation\_precondition\_msg( i == 42, "Only 42 allowed!");} \\
\>\>\texttt{    // ...} \\
\>  \texttt{\}}
\end{tabbing}
}
\lcHtml{
\begin{verbatim}
 #include <CGAL/optimisation\_assertions.h>

 void optimisation\_foo( int i)
 {
   CGAL_optimisation_precondition_msg( i == 42, "Only 42 allowed!");
   // ...
 }
\end{verbatim}
}

\ccIndexSubsubitem{checks}{package-level}{documenting}
\ccIndexSubitem{documentation}{of checks}
The documentation of your new package has to name the term chosen to be
part of the package-specific macros in
order to enable the user to selectively turn off and on the checks of
your package. For example, in the documentation of the optimisation
package you can find a sentence similar to the following.
\begin{quote}
  The optimisation code uses the term OPTIMISATION for the checks;
  \eg,~setting the compile time flag
  \ccc{CGAL_OPTIMISATION_NO_PRECONDITIONS} switches off precondition
  checking in the optimisation code.
\end{quote}
\ccIndexSubitemEnd{checks}{package-level}


% -----------------------------------------------------------------------------
\section{Exception handling\label{sec:exception_handling}}
\ccIndexMainItem{exception handling}

Some parts of the library use exceptions, but there is no general specific
policy concerning exception handling in \cgal.  It is nevertheless good to
target exception safety, as much as possible.  Good references on exception
safety are: Appendix E of~\cite{cgal:s-cpl-97} (also available at
\path|http://www.research.att.com/~bs/3rd_safe0.html|),
and~\cite{cgal:a-esgc-98} (also available at
\path|http://www.boost.org/more/generic_exception_safety.html|).

\section{Requirements and recommendations\label{sec:checks_req_and_rec}}

\noindent
Requirements:
\begin{itemize}
   \item Write pre- and postcondition checkers for your functions wherever
         possible.
   \item Use the \cgal\ preprocessor macros (Sections~\ref{sec:checks_using}
         and~\ref{sec:checks_controlling})
         exclusively throughout your code (instead of,
         for example, the \ccc{assert} macro or the \ccc{std::abort} or
         \ccc{std::exit} functions) for all checks
         to assure that all \cgal\ invariant tests can be handled in a uniform
         way.
\end{itemize}

\ccIndexMainItemEnd{checks}
% ===== EOF ===================================================================
 % extra chapter

\section{Identifying the Version of CGAL\label{sec:cgal_version}}

\ccInclude{CGAL/config.h}

Every release of \cgal\ defines the following preprocessor macros:
\begin{description}
\item[\texttt{CGAL\_VERSION}]
     \index{CGAL_VERSION macro@{\tt CGAL\_VERSION} macro}
     -- a textual description of the current release
        (e.g., or 3.3 or 3.2.1 or 3.2.1-I-15), and 
\item[\texttt{CGAL\_VERSION\_STR}]
     \index{CGAL_VERSION_STR macro@{\tt CGAL\_VERSION\_STR} macro}
     -- same as \texttt{CGAL\_VERSION} but as a string constant token, and
\item[\texttt{CGAL\_VERSION\_NR}]
     \index{CGAL_VERSION_NR macro@{\tt CGAL\_VERSION\_NR} macro}
     -- a numerical description of the current release such that
        more recent releases have higher number.

     More precisely, it is defined as \texttt{1MMmmbiiii},
     where \texttt{MM} is the major release number (e.g. 03),
     \texttt{mm} is the minor release number (e.g. 02),
     \texttt{b} is the bug-fix release number (e.g. 0), and
     \texttt{iiii} is the internal release number (e.g. 0001). For
     public releases, the latter is defined as 1000.
     Examples: for the public release 3.2.4 this number is 
     1030241000; for internal release 3.2-I-1, it is 1030200001.
     Note that this scheme was modified around 3.2-I-30.
\item[\texttt{CGAL\_VERSION\_NUMBER(M,m,b)}]
     \index{CGAL_VERSION_NUMBER macro@{\tt CGAL\_VERSION\_NUMBER} macro}
     -- a function macro computing the version number macro
     from the M.m.b release version.  Note that the internal release
     number is dropped here.  Example: \texttt{CGAL\_VERSION\_NUMBER(3,2,4)}
     is equal to 1030241000.
\end{description}
 

\begin{ccAdvanced}
\section{Compile-time Flags to Control Inlining}
\ccIndexMainItem{code optimization}
\ccIndexMainItem{inlining}
\ccIndexMainItem{\tt inline}

Making functions inlined can, at times, improve the efficiency of your code.
However this is not always the case and it can differ for a single function
depending on the application in which it is used. Thus \cgal\ defines a set 
of compile-time macros that can be used to control whether certain functions 
are designated as inlined functions or not.  The following table lists the 
macros and their default values, which are set in one of the \cgal\ include
files.  

\begin{tabular}{l|l}
               macro name        & default \\ \hline
\ccc{CGAL_KERNEL_INLINE}         & inline \\
\ccc{CGAL_KERNEL_MEDIUM_INLINE}  &  \\
\ccc{CGAL_KERNEL_LARGE_INLINE}   &  \\
\ccc{CGAL_MEDIUM_INLINE}         & inline \\
\ccc{CGAL_LARGE_INLINE}          &  \\
\ccc{CGAL_HUGE_INLINE}           & 
\end{tabular}

If you wish to change the value of one or more of these macros,
you can simply give it a new value when compiling.  For example, to make
functions that use the macro \ccc{CGAL_KERNEL_MEDIUM_INLINE} inline functions,
you should set the value of this macro to \texttt{inline} instead of the
default blank. 

Note that setting inline manually is very fragile, especially in a template
context.  It is usually better to let the compiler select by himself which
functions should be inlined or not.
\end{ccAdvanced}
