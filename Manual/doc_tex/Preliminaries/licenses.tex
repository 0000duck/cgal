
\section{License Issues}

\cgal\ is distributed under a dual license scheme, that is under the 
{\sc Gpl/\sc Lgpl} open source license, as well as under commercial licenses.

\cgal\ consists of different parts covered by different open source licenses.  
In this section we explain the essence of the different licenses, as well as 
the rationale why we have chosen them. 

The fact that \cgal\ is Open Source software does not mean that users are free
to do whatever they want with the software. Using the software means to accept
the license, which has the status of a contract between the user and the owner
of the \cgal\ software. 

\subsection{GPL \label{licenses:GPL}}

The {\sc Gpl} is an Open Source license that, if you distribute your software
based on {\sc Gpl}ed \cgal\ data structures, obliges you to distribute the 
source code of your software under the {\sc Gpl}. 

The exact license terms can be found at the  Free Software Foundation 
web site: \path'http://www.gnu.org/licenses/gpl-3.0.html'.

\subsection{LGPL \label{licenses:LGPL}}

The {\sc Lgpl} is an Open Source license that obliges you to distribute
modifications you make on \cgal\ software accessible to the users. 
In contrast to the {\sc Gpl} , there is no obligation to make the source 
code of software you build on top of {\sc Lgpl}ed \cgal\ data structures 

The exact license terms can be found at the Free Software Foundation web site:
\path'http://www.gnu.org/copyleft/lesser.html'.

\subsection{Rationale of the License Choice}

We have chosen the {\sc Gpl} and the {\sc Lgpl} as they are well known
and well understood open source licenses. The former restricts
commercial use, and the latter allows to promote software as de facto standard 
so that people can build new higher level data structures on top.

Therefore, the packages forming a foundation layer are distributed under
the {\sc Lgpl}, and the higher level packages under the {\sc Gpl}.
The package overview states for each package under which license
it is distributed.

\subsection{Commercial Licenses \label{licenses:Commercial}}

Users who cannot comply to the Open Source license terms can buy individual
data structures under various commercial licenses from GeometryFactory:
\path'http://www.geometryfactory.com'.  License fees paid by commercial
customers are reinvested in R&D performed by the \cgal\ project partners, 
as well as in evolutive maintenance.


