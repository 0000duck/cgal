
\section{License Issues}

\cgal\ is Open Source software, and consists of different parts covered by
different licenses.  In this section we explain the essence of the different
licenses, as well as the rationale why we have chosen them. 

The fact that \cgal\ is Open Source software does not mean that users are free
to do whatever they want with the software. Using the software means to accept
the license, which has the status of a contract between the user and the owner
of the \cgal\ software.  A more detailed description of the license terms is
available in the \cgal\ software tarball.


\subsection{QPL \label{licenses:QPL}}

The {\sc Qpl} is an Open Source license that obliges you to distribute your
software based on {\sc Qpl}ed \cgal\ data structures.  The rationale behind
this is that we can claim access to your software.  The license further obliges
you to put your software under an Open Source license as well. The rationale
behind is that we can distribute your software, even if this is not your
intention.  Finally, the {\sc Qpl} requires that, if you modify \cgal, you
distribute the modifications in the form of patches and you distribute the
sources of your changes as well.


The exact license terms can be
found at the Trolltech web site: \path'http://doc.qt.nokia.com/4.3/qpl.html'.
\subsection{LGPL \label{licenses:LGPL}}

The {\sc Lgpl} is an Open Source license that obliges you to distribute
modifications you make on \cgal\ software accessible to the users. There is no
obligation to make the source code of software you build on top of {\sc Lgpl}ed
\cgal\ data structures available.

Currently the linear kernel, the support library, the halfedge data structure,
the kinetic data structures, and the mesh subdivision framework are distributed
under the {\sc Lgpl}. The rationale behind is that we want to promote them as
de facto standards.

The exact license terms can be found at the Free Software Foundation web site:
\path'http://www.gnu.org/copyleft/lesser.html'.

\subsection{Commercial Licenses \label{licenses:Commercial}}

Users who cannot comply to the Open Source license terms can buy individual
data structures under various commercial licenses from GeometryFactory:
\path'http://www.geometryfactory.com'.


\subsection{License Compatibility \label{licenses:Compatibility}}

The General Public License ({\sc Gpl}) has a viral effect which makes it
incompatible with the {\sc Qpl}. For more information, please refer to the
paragraph about the {\sc Qpl} on the licenses web page of the Free Software
Foundation ({\sc Fsf}): \path'http://www.fsf.org/licensing/licenses/index_html'. 
It is therefore not possible to build a program including {\sc Gpl}
code and some {\sc Qpl} parts of {\sc Cgal}. In this case, if you are
the copyright owner of the {\sc Gpl} code, you can amend the license
by adding an exception allowing the use of {\sc Cgal} with it (see
again the {\sc Fsf} web page).


