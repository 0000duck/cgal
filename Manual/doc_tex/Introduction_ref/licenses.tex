
\section{License Issues}

\cgal\ is Open Source software, and consists of different parts covered by different licenses.
In this section we explain the essence of the different licenses, as well as the rationale
why we have chosen them. 

The fact that \cgal is Open Source software does not mean that users are free to do whatever they
want with the software. Using the software means to accept the license,
which has the status of a contract between the user and the owner of \cgal\ software.


\subsection{Qpl \label{licenses:QPL}}

The {\sc Qpl} is an Open Source license that obliges you to distribute your software 
based on {\sc Qpl}ed \cgal\ data structures.  The rationale behind this is that we
can claim access to your software.  The license further obliges you to put your
software under an Open Source license as well. The rationale behind is that
we can distribute your software, even if this is not your intention.  
Finally, the {\sc Qpl} requires that, if you modify \cgal, you distribute the
modification in the form of patches and you distribute the source of your changes
as well.


The exact license terms as well as an annotated version of the license can be found 
at the Trolltech web site.

\subsection{LGPL \label{licenses:LGPL}}

The {\sc Lgpl} is an Open Source license that obliges you to distribute modifications
you make on \cgal\ software accessible to the users. There is no obligation to make
the source cod of software you build on top of {\sc Lgpl}ed \cgal\ data structures available.

Currently the linear kernel, the support library, the halfedge data structure,
the Kinetic data structures, and the mesh subdivision framework are distrubuted
under the {\sc Lgpl}. The rationale behind is that we want to promote them as a defacto
standard.

The exact license terms can be found at the Free Software Foundation web site.

\subsection{Commercial Licenses \label{licenses:Commercial}}

Users who cannot comply to the Open Source license terms can buy individual
data structures under various commercial licenses from GeometryFactory (www.geometryfactory.com).


\subsection{License Compatibility \label{licenses:Compatibility}}

Note that the {\sc Gpl} is not compatible with the {\sc Qpl}. If you write
{\sc Gpl}ed software and you want to combine it with {\sc Qpl}ed \cgal\ data
structures, you can make it compatible by adding....
