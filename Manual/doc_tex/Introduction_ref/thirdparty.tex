\section{Third Party Software}

In this section we list the software


\subsection{Standard Template Library \label{thirdparty:stl}}

\cgal\ heavily uses the {\sc Stl}, and in particular adopted
many of its design ideas.   The {\sc Stl} comes with the compiler,
but it is possible to use the compiler together with an
alternative {\sc Stl} implementation.


\subsection{Boost \label{thirdparty:Boost}}

Boost is a collection of libraries. \cgal\ needs some of them, that is
it is mandatory.  If Boost is not already on your system, e.g., on
Windows, you can download it from \path'www.boost.org'.


\subsection{GMP \label{thirdparty:GMP}}

A library for multi precision integers and rational numbers.
\cgal\ offers adapters for these number types. The usage
of the {\sc Gmp} library is optional.  If it is not already on your system,
e.g., on Windows, you can download it from \path'www.swox.com/gmp'.

\subsection{MPFR \label{thirdparty:MPFR}}

A library for multi precision floating point numbers.  The usage of
the {\sc Mpfr} library is optional, and you must install it when you
use {\sc Gmp}.  You can download {\sc Mpfr} from \path'www.mpfr.org'.


\subsection{Leda \label{thirdparty:Leda}}

A library of efficient data structures and algorithms. \cgal\ offers
adapters to the {\sc Leda} number types. The usage is optional.
It is only available commercially from \path'www.algorithmic-solutions.com'.


\subsection{Taucs \label{thirdparty:Taucs}}

A library of sparse linear solvers.
It can be used by the \ccc{Surface_mesh_parametrization} package
in order to speed up the algorithm. you can download it from
 \path'www.tau.ac.il/\~stoledo/taucs/'.

\subsection{OpenNL \label{thirdparty:OpenNL}}

A solver.
OpenNL is distributed together with \cgal.


\subsection{zlib \label{thirdparty:zlib}}

A data compression library.
It is used in the examples of the \ccc{Surface_mesher} package.
If it is not already on your system,
e.g., on Windows, you can download it from  \path'www.gzip.org/zlib'.

\subsection{Qt \label{thirdparty:Qt}}

A {\sc Gui} library. The usage of Qt is optional, but note that
it is used for all 2D demos.

As Qt is the layer underneath {\sc Kde}, Qt is installed
on many Linux systems. Otherwise you can download it from
 \path'www.trolltech.com'.


\subsection{Coin \label{thirdparty:Coin}}

An implementation of Open Inventor.  It is used in the demo
of the \ccc{Kinetic_data_structures} package. You can download
it from \path'www.sim.no'.

