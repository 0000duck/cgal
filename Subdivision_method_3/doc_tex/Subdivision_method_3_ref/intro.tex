% +------------------------------------------------------------------------+
% | Cbp Reference Manual:  intro.tex
% +------------------------------------------------------------------------+
% | Polyhedron reference manual pages
% |
% | 03/01/2005   Le-Jeng Andy Shiue
\RCSdef{\polyhedronRefRev}{$Id$}
\RCSdefDate{\polyhedronRefDate}{$Date$}
% +------------------------------------------------------------------------+

\chapter{Subdivision Methods}
\label{chapterSubdivisionRef}
\ccChapterRelease{\subdivisionRefRev. \ \subdivisionRefDate}
\ccChapterAuthor{Le-Jeng Andy Shiue}


% +------------------------------------------------------------------------+
\section*{Summary}

Subdivision methods recursively refine the control mesh 
(i.e.~the input mesh) and generate points approximating 
the limit surface. 
%\ccc{Subdivision_method_3} 
%realizes a subdivision method as a refinement host parameterized 
%with a geometry policy. The refinement host refines the 
%control mesh, maintains the stencils, and
%computes the geometry of the refined mesh. The computation
%of the geometry is done through the geometry policy.
Designed to work on the class \ccc{Polyhedron_3},
\ccc{Subdivision_method_3} aims to be easy to use and to extend.
%(see Chapter~\ref{chapterPolyhedron})
\ccc{Subdivision_method_3} is not a class, but a namespace 
which consists of four popular subdivision methods and their refinement
hosts. Supported subdivision methods include Catmull-Clark, Loop, 
Doo-Sabin and $\sqrt{3}$ subdivisions. Their respective refinement 
hosts are PQQ, PTQ, DQQ and $\sqrt{3}$ refinements.
Variations of those methods can be easily 
extended by substituting the geometry computation of the refinement
host.

%\subsection*{Concepts}
%\ccRefIdfierPage{PolyhedronTraits_3}\\ 
%\ccRefIdfierPage{PolyhedronItems_3}

\subsection*{Classes}

\ccRefIdfierPage{CGAL::Subdivision_method_3}\\ 
%\ccRefIdfierPage{CGAL::PQQ_stencil_3<Polyhedron_3>}\\ 
%\ccRefIdfierPage{CGAL::DQQ_stencil_3<Polyhedron_3>}\\ 
%\ccRefIdfierPage{CGAL::Linear_mask_3<Polyhedron_3>}\\ 
\ccRefIdfierPage{CGAL::CatmullClark_mask_3<Polyhedron_3>}\\ 
\ccRefIdfierPage{CGAL::Loop_mask_3<Polyhedron_3>}\\ 
\ccRefIdfierPage{CGAL::DooSabin_mask_3<Polyhedron_3>}\\ 
\ccRefIdfierPage{CGAL::Sqrt3_mask_3<Polyhedron_3>}\\ 

%\subsection*{Functions}
%\ccThree{ostream&}{}{page 999.}

%\ccHtmlNoLinks\ccHtmlNoIndex
%\ccFunction{template <class Traits>
%            ostream& operator<<( ostream& out, 
%                                 const CGAL::Polyhedron_3<Traits>& P);}
%    {\lcRawHtml{<A HREF="Function_operator.html">(go there)</A>}
%     \lcTex{\hfill page~\pageref{refPolyhedron_operator_leftshift}}}

%\ccHtmlNoLinks\ccHtmlNoIndex
%\ccFunction{template <class Traits>
%            istream& operator>>( istream& in, 
%                                 CGAL::Polyhedron_3<Traits>& P);}
%    {\lcRawHtml{<A HREF="Function_operator--.html">(go there)</A>}
%     \lcTex{\hfill page~\pageref{refPolyhedron_operator_rightshift}}}

%\lcHtml{\subsection*{Links to the Reference Sections}}

%% EOF %%
