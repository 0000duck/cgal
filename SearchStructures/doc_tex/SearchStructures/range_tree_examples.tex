\subsection{Example ofRange Tree on Map-like Data\label{sec:range_tree_ex}}

The following example program uses the predefined \ccc{
  Range_tree_2} data structure together with the predefined traits
  class \ccc{Range_tree_map_traits_2} which has two template
  arguments specifying the
  type of the point data in each dimension
  (\ccc{CGAL::Cartesian<double>}) and the value type of the
  2-dimensional point data (\ccc{char}). Therefore the \ccc{
  Range_tree_2} is defined on 2-dimensional point data each of which is
  associated with a character.
Then, a few data items are created and put into a list. After
  that the tree is constructed according to that list, a window
  query is performed, and the query elements are given out.

\begin{verbatim}

#include <CGAL/Cartesian.h>
#include <CGAL/Range_segment_tree_traits.h>
#include <CGAL/Range_tree_k.h>

typedef CGAL::Cartesian<double> K;
typedef CGAL::Range_tree_map_traits_2<K, char> Traits;
typedef CGAL::Range_tree_2<Traits> Range_tree_2_type;

int main()
{
  typedef Traits::Key Key;                
  typedef Traits::Interval Interval;    

  std::vector<Key> InputList, OutputList;
  InputList.push_back(Key(K::Point_2(8,5.1), 'a'));
  InputList.push_back(Key(K::Point_2(1,1.1), 'b'));
  InputList.push_back(Key(K::Point_2(3,2.1), 'c'));

  Range_tree_2_type Range_tree_2(InputList.begin(),InputList.end());
  Interval win(Interval(K::Point_2(4,8.1),K::Point_2(5,8.2)));
  std::cout << "\n Window Query:\n ";
  Range_tree_2.window_query(win, std::back_inserter(OutputList));
  std::vector<Key>::iterator current=OutputList.begin();
  while(current!=OutputList.end()){
    std::cout << (*current).first.x() << "," << (*current).first.y()
         << ":" << (*current++).second << std::endl;
  }
}
\end{verbatim}



\subsection{Example of Range Tree on Set-like Data}

This example illustrates the use of the range tree on
2-dimensional point data (no value is associated to a data item).
After the definition of the tree, some input data items are
created and the tree is constructed according to the input data
items.
After that, a window query is performed and the query elements
are given to standard out.

\begin{verbatim}
#include <CGAL/Cartesian.h>
#include <CGAL/Range_segment_tree_traits.h>
#include <CGAL/Range_tree_k.h>

typedef CGAL::Cartesian<double> K;
typedef CGAL::Range_segment_tree_set_traits_2<K> Traits;
typedef CGAL::Range_tree_2<Traits> Range_tree_2_type;

int main()
{
  typedef Traits::Key Key;
  typedef Traits::Interval Interval;
  std::vector<Key> InputList, OutputList;
  std::vector<Key>::iterator first, last, current;

  InputList.push_back(Key(8,5.1));
  InputList.push_back(Key(1,1.1));
  InputList.push_back(Key(3,2.1));

  Range_tree_2_type Range_tree_2(InputList.begin(),InputList.end());

  Interval win=Interval(Key(4,8.1),Key(5,8.2));
  std::cout << std::endl << "Window Query: lower left point: (4.0,5.0),";
  std::cout << "upper right point: (8.1,8.2)" << std::endl;
  Range_tree_2.window_query(win, std::back_inserter(OutputList));
  current=OutputList.begin();
  while(current!=OutputList.end()){
    std::cout << (*current).x()<< "-" << (*current).y() << std::endl;
    current++;
  }
}
\end{verbatim}

