\ccRefChapter{dD Range and Segment Trees}
\ccChapterAuthor{Gabriele Neyer}

This chapter presents the {\cgal} range tree and  segment tree data structures.


The range tree is theoretically superior to the $Kd$-tree, but the
latter often seems to perform better.
However, the range tree as implemented in {\cgal} is more flexible than the
$Kd$-tree implementation, in that it enables to layer together range trees
and segment trees in the same data structure.



\section{Classified Reference Pages}

\ccHeading{Concepts}

\ccRefConceptPage{RangeSegmentTreeTraits_k} \\
\ccRefConceptPage{Sublayer} \\

\ccHeading{Traits Classes}

\ccRefIdfierPage{CGAL::Range_segment_tree_traits_set_2<R>} \\
\ccRefIdfierPage{CGAL::Range_segment_tree_traits_set_3<R>} \\
\ccRefIdfierPage{CGAL::Range_tree_traits_map_2<R,T>} \\
\ccRefIdfierPage{CGAL::Range_tree_traits_map_3<R,T>} \\
\ccRefIdfierPage{CGAL::Segment_tree_traits_map_2<R,T>} \\
\ccRefIdfierPage{CGAL::Segment_tree_traits_map_3<R,T>} \\
\ccRefIdfierPage{CGAL::tree_interval_traits} \\
\ccRefIdfierPage{CGAL::tree_point_traits} \\


\ccHeading{Search Structure Classes}

\ccRefIdfierPage{CGAL::Range_tree_d<Data, Window, Traits>}\\
\ccRefIdfierPage{CGAL::Range_tree_k<Traits>}\\
\ccRefIdfierPage{CGAL::Segment_tree_d<Data, Window, Traits>}\\
\ccRefIdfierPage{CGAL::Segment_tree_k<Traits>}\\
\ccRefIdfierPage{CGAL::Tree_anchor<Data, Window>}\\


