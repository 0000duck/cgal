% +------------------------------------------------------------------------+
% | Reference manual page: Dart.tex
% +------------------------------------------------------------------------+
% | 04.02.2010   Guillaume Damiand
% | Package: Combinatorial_map
% +------------------------------------------------------------------------+
\ccRefPageBegin
%%RefPage: end of header, begin of main body
% +------------------------------------------------------------------------+
\begin{ccRefClass}{Dart<d,CMap>}

\ccInclude{CGAL/Dart.h}

\ccDefinition
  
The class \ccRefName\ represents a \emph{d}D dart.

$\beta_i$ pointers are coded in a array of \emph{d+1} \ccc{Dart_handle}
(because we describe also the $\beta_0$ link).  Attributes are
associated to each dart by \ccc{Attribute_handle<i>}, one for each
non void \emph{i}-attribute.

\ccIsModel
\ccRefConceptPage{Dart}

\ccParameters
\ccc{d} an integer for the dimension of the dart.\\
\ccc{CMap} must be a model of the \ccc{CombinatorialMap} concept.

% A dart \ccc{a} is never constructed directly, but always created withing a
% combinatorial map \ccc{cm} by using the method \ccc{cm.create_dart();}

\ccTypes
\ccThree{typedef CMap::Attribute_const_handle<i>::type}{}{}
\ccTypedef{typedef CMap::Dart_handle Dart_handle;}{}
\ccGlue
\ccTypedef{typedef CMap::Dart_const_handle Dart_const_handle;}{}

\ccTypedef{typedef CMap::Attribute_handle<i>::type Attribute_handle<i>::type;}{}
\ccGlue
\ccTypedef{typedef CMap::Attribute_const_handle<i>::type Attribute_const_handle<i>::type;}{}

\ccHeading{Complexity}

Each $\beta_i$ link is initialized to \ccc{CMap::null_dart_handle}, and each 
attribute handle of non void \emph{i}-attribute is initialized to NULL
at the creation of the dart, thus the complexity of the creation is in
\emph{O}(\emph{d+1}).

The complexity of \ccc{opposite} and \ccc{other_extremity} methods is in
\emph{O}(\emph{d+1}).

Other methods have all a constant time complexity.

\ccSeeAlso
\ccRefConceptPage{CombinatorialMap}

\end{ccRefClass}
% +------------------------------------------------------------------------+
%%RefPage: end of main body, begin of footer
\ccRefPageEnd
% EOF
% +------------------------------------------------------------------------+
