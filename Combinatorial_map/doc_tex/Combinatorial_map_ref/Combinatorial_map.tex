% +------------------------------------------------------------------------+
% | Reference manual page: Combinatorial_map.tex
% +------------------------------------------------------------------------+
% | 04.02.2010   Guillaume Damiand
% | Package: Combinatorial_map
% +------------------------------------------------------------------------+
\ccRefPageBegin
%%RefPage: end of header, begin of main body
% +------------------------------------------------------------------------+

\begin{ccRefClass}{Combinatorial_map<d,CMItems,Alloc>}
\ccRefLabel{CGAL::Combinatorial_map}

\ccInclude{CGAL/Combinatorial_map.h}

\ccDefinition
  
The class \ccRefName\ represents a \emph{d}D combinatorial map.

Darts and non void attributes are stored in memory using
\ccc{CGAL::Compact_container}, using \ccc{Alloc} as allocator.

\ccIsModel
\ccRefConceptPage{CombinatorialMap}

\ccParameters
\ccc{d} an integer for the dimension of the map.\\
\ccc{CMItems} must be a model of the \ccc{CombinatorialMapItems} concept. \\
\ccc{Alloc} has to match the standard allocator requirements. The  \ccc{rebind} mechanism from 
\ccc{Alloc} will be used to create appropriate allocators internally with value type \ccc{Dart}.

There are two default template arguments:
\ccc{Combinatorial_map_min_items<d>} for \ccc{CMItems} and
\ccc{CGAL_ALLOCATOR(int)} from the \ccc{<CGAL/memory.h>} header file for \ccc{Alloc}.

\ccTypes
\ccThree{typedef Combinatorial_map<d,CMItems,Alloc>}{}{}
\ccTypedef{typedef Combinatorial_map<d,CMItems,Alloc> Self;}{}
\ccGlue
\ccTypedef{typedef CMItems::Dart_wrapper<Self>::Dart Dart;}{}

\ccHeading{Complexity}

The complexity of \ccc{sew} and \ccc{unsew} is in \emph{O}($|$\emph{S}$|$\mytimes{}$|$\emph{c}$|$), \emph{S}
being the set of darts of the orbit
\orbit{\betaun{},\myldots{},\betaimdeux{},\betaipdeux{},\myldots{},\betad{}} for the
considered dart, and \emph{c} the biggest \emph{j}-cell merged or
split during the sew such that \emph{j}-attributes are non void. 
The complexity of \ccc{is_sewable} is in \emph{O}($|$\emph{S}$|$).

% The complexity of \ccc{topo_sew}, and \ccc{topo_unsew} are in
% $O(|S|\times c)$, $S$ being the set of darts of the orbit
% $\orb{\mb{1},\ldots,\mb{i-2},\mb{i+2},\ldots,\mb{d}}$ for the
% considered dart.

The complexity of \ccc{set_attribute} is in \emph{O}($|$\emph{c}$|$), \emph{c} being the
\emph{i}-cell containing the considered dart.

The complexity of \ccc{is_without_boundary(unsigned int i)} is in \emph{O}($|$\emph{D}$|$), 
\emph{D} being the set of darts of the combinatorial map, and the complexity of 
\ccc{is_without_boundary()} is in \emph{O}($|$\emph{D}$|$\mytimes{}\emph{d}).

The complexity of \ccc{unmark_all} and \ccc{free_mark} is in \emph{O}(1) if
all the darts of the combinatorial map have the same mark, and in
\emph{O}($|$\emph{D}$|$) otherwise.

The complexity of \ccc{is_valid} is in
\emph{O}($|$\emph{D}$|$\mytimes{}\emph{d}\ccTexHtml{$^2$}{<sup>2</sup>}).

The complexity of \ccc{clear} is in \emph{O}($|$\emph{D}$|$\mytimes{}\emph{d}).

Other methods have all a constant time complexity.

\ccSeeAlso
\ccRefConceptPage{CombinatorialMapItems}\\
\ccRefConceptPage{Dart}\\

\end{ccRefClass}
% +------------------------------------------------------------------------+
%%RefPage: end of main body, begin of footer
\ccRefPageEnd
% EOF
% +------------------------------------------------------------------------+
