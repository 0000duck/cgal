% +------------------------------------------------------------------------+
% | Reference manual page: CombinatorialMapItems.tex
% +------------------------------------------------------------------------+
% | 04.02.2010   Guillaume Damiand
% | Package: Combinatorial_map
% +------------------------------------------------------------------------+
\ccRefPageBegin
%%RefPage: end of header, begin of main body
% +------------------------------------------------------------------------+

\begin{ccRefConcept}{CombinatorialMapItems}
%\ccRefLabel{CGAL::CombinatorialMapItems}

\ccDefinition
  
The concept \ccRefName\ allows to customize a \emph{d}D combinatorial map
by choosing the type of darts, and by enabling and disabling some
attributes.  For that, it defines an inner class template named
\ccc{Dart_wrapper}, with one template parameter, \ccc{CMap}, a model
of the \ccc{CombinatorialMap} concept. This inner class must define
two types: \ccc{Dart} and \ccc{Attributes}.

%   A \ccRefName\ contains one static const unsigned
% int \ccc{NB_MARKS} for the number of Boolean marks.  It contains also
% one member class template named \ccc{Dart_wrapper}, with one template
% parameter, \ccc{Refs}. \ccc{Refs} must be a model of
% \ccc{CombinatorialMap} concept.  The \ccc{Dart_wrapper} member class
% template provides a local type named \ccc{Dart} which must be a model
% of \ccc{Dart} concept, and a local type named \ccc{Attributes_enabled}
% which must be a tuple of types, each one being a model of
% \ccc{CellAttribute} concept, or equal to \ccc{Disabled}.

% The tuple \ccc{Attributes_enabled} must contain at most $d+1$ types
% (one for each possible cell of the $d$-map).  The first type
% corresponds to 0-cell attributes, the second to 1-cell attributes and
% so on. If a type is \ccc{CGAL::Disabled}, the corresponding cell
% attribute is disabled.  If the size of the tuple is $k < d+1$, each
% $i$-cell attributes, $\forall i: k \leq d+1$, is disabled (see
% examples below).

% \ccConstants
% \ccVariable{static unsigned int NB_MARKS;}{The number of Boolean marks.}

\begin{ccClass}{CombinatorialMapItems::Dart_wrapper<CMap>}
\ccCreationVariable{dw}

\ccHeading{%
  Types in \ccc{CombinatorialMapItems::Dart_wrapper<CMap>}}

\ccNestedType{Dart}
    {The type of dart, a model of the \ccc{Dart} concept.}

\ccNestedType{Attributes}
{The tuple of attributes, containing at most
  \emph{dimension+1} types (one for each possible cell of the combinatorial
  map).  Each type of the tuple must be either a model of the
  \ccc{CellAttribute} concept or \ccc{void}.
  The first type corresponds to 0-attributes, 
  the second to 1-attributes and so on. 
  If the \ccTexHtml{$i^{\mbox{th}}$}{i<sup>th</sup>} type in the tuple is \ccc{void}, 
  (\emph{i}-1)-attributes are disabled. Otherwise, (\emph{i}-1)-attributes are enabled and
  have the given type.  If the size of the tuple is \emph{k},
  with \emph{k}<\emph{dimension+1}, 
  \myforall{}\emph{i}: \emph{k}\myleq{}\emph{i}\myleq{}\emph{dimension},
  \emph{i}-attributes are disabled.}
\end{ccClass}

\ccExample
The following examples show two possible models of the
\ccc{CombinatorialMapItems} concept: the first one for a 4D
combinatorial map without enabled attributes, the second one for a 3D
combinatorial map with edge attributes enabled, and associated with a
\ccc{Cell_attribute} containing an \ccc{int}.

\begin{ccExampleCode}
struct Exemple_Item_4
{
   template < class CMap >
   struct Dart_wrapper
   {
     typedef CGAL::Dart<4, CMap> Dart;
     typedef CGAL::cpp0x::tuple<> Attributes;
   };
};
\end{ccExampleCode}

\begin{ccExampleCode}
struct Exemple_Item_3
{
   template < class CMap >
   struct Dart_wrapper
   {
     typedef CGAL::Dart<3, CMap> Dart;
     typedef Cell_attribute<CMap, int> Edge_attrib;
     typedef CGAL::cpp0x::tuple<void,Edge_attrib> Attributes;
   };
};
\end{ccExampleCode}    

\ccHasModels
\ccRefIdfierPage{CGAL::Combinatorial_map_min_items<d>}

\ccSeeAlso
% \ccRefConceptPage{HalfedgeDS<Traits,Items,Alloc>}\\
% \ccRefConceptPage{HalfedgeDSVertex}\\
% \ccRefConceptPage{HalfedgeDSHalfedge}\\
% \ccRefConceptPage{HalfedgeDSFace}\\
% \ccRefConceptPage{PolyhedronItems_3}\\
% \ccRefIdfierPage{CGAL::HalfedgeDS_vertex_base<Refs>}\\
% \ccRefIdfierPage{CGAL::HalfedgeDS_halfedge_base<Refs>}\\
\ccRefConceptPage{CombinatorialMap}\\
\ccRefConceptPage{Dart}

\end{ccRefConcept}
% +------------------------------------------------------------------------+
%%RefPage: end of main body, begin of footer
\ccRefPageEnd
% EOF
% +------------------------------------------------------------------------+

