\section{Introduction}

The Ipe extensible drawing editor (\ccAnchor{http://tclab.kaist.ac.kr/ipe/}{http://tclab.kaist.ac.kr/ipe/}) \cite{schwarzkopf1995ede,ipe:man-09}
is a tool used by computational geometry researchers to produce 2D figures for inclusion in articles or presentations.
The extensible adjective sheds a light on an important feature:
the possibility for users to write small extensions (called \textit{ipelets})
to interface any algorithm with Ipe. This feature is especially interesting 
to produce illustrations in a paper describing a new algorithm.
This package provides one class
that eases the writing of ipelets based on \cgal{}, reducing the needed knowledge of the API of Ipe.
This class is designed to fulfill most of the needs to implement an ipelet for a 2D \cgal{} algorithm.
In addition, this package comes with more than ten complete examples
of ipelets: alpha-shape, arrangements, Voronoi diagrams, convex hulls, Hilbert curve, k-order Delaunay,
2D mesh, Minkowski sum, polygon partition, random generators, triangulations \dots



% @MANUAL{ipe:man-09,
%    TITLE        = {{IPE} manual and library documentation},
%    AUTHOR       = {Otfried Cheong},
%    EDITION      = {6.0pre32},
%    YEAR         = {2009},
%    URL          = {http://tclab.kaist.ac.kr/ipe/}
% }
