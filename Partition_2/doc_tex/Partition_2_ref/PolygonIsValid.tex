% +------------------------------------------------------------------------+
% | Reference manual page: PolygonIsValid.tex
% +------------------------------------------------------------------------+
% | 10.05.2000   Susan Hert
% | Package: Partition_2
% | 
% |
%%RefPage: end of header, begin of main body
% +------------------------------------------------------------------------+

\renewcommand\ccRefPageBegin{\ccParDims\cgalColumnLayout\begin{ccAdvanced}}
\renewcommand\ccRefPageEnd{\ccParDims\cgalColumnLayout\end{ccAdvanced}}
\begin{ccRefFunctionObjectConcept}{PolygonIsValid}

\ccDefinition
  
Function object that determines if a sequence of points represents a
valid partition polygon or not, where "valid" can assume any of several
meanings ({\it e.g.}, convex or $y$-monotone).
\ccIndexSubitem{polygon partitioning}{valid}


\ccCreation
\ccCreationVariable{f}  %% choose variable name

\ccConstructor{PolygonIsValid(const Traits& t);}{
\ccc{Traits} is a model of the concept required by the function that checks
for validity of the polygon.
}

\ccOperations

\ccMethod{
template<class InputIterator>
bool operator()(InputIterator first, InputIterator beyond);}
{ 
  returns \ccc{true} iff the points of type \ccc{Traits::Point_2}
  in the range [\ccc{first},\ccc{beyond}) define a valid polygon.
}

\ccHasModels

\ccRefIdfierPage{CGAL::Is_convex_2<Traits>}  \\
\ccRefIdfierPage{CGAL::Is_y_monotone_2<Traits>}

\ccSeeAlso

\ccRefIdfierPage{CGAL::approx_convex_partition_2} \\
\ccRefIdfierPage{CGAL::convex_partition_is_valid_2} \\
\ccRefIdfierPage{CGAL::greene_approx_convex_partition_2} \\
\ccRefIdfierPage{CGAL::optimal_convex_partition_2} \\
\ccRefIdfierPage{CGAL::partition_is_valid_2} \\
\ccRefIdfierPage{CGAL::y_monotone_partition_2} \\
\ccRefIdfierPage{CGAL::y_monotone_partition_is_valid_2}

\end{ccRefFunctionObjectConcept}
\renewcommand\ccRefPageBegin{\ccParDims\cgalColumnLayout}
\renewcommand\ccRefPageEnd{\ccParDims\cgalColumnLayout}


% +------------------------------------------------------------------------+
%%RefPage: end of main body, begin of footer
% EOF
% +------------------------------------------------------------------------+

