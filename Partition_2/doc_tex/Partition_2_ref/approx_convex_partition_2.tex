\begin{ccRefFunction}{approx_convex_partition_2}

\ccDefinition
Function that produces a set of 
convex polygons that represent a partitioning of a polygon defined
on a sequence of points.  The number of convex polygons produced is 
no more than four times the minimal number.%
\ccIndexSubsubitem{polygon partitioning}{convex}{approximately optimal}


\ccInclude{CGAL/partition_2.h}

\ccFunction{
template <class InputIterator, class OutputIterator, class Traits>
OutputIterator approx_convex_partition_2(InputIterator first,
                                         InputIterator beyond,
                                         OutputIterator result,
                                         const Traits& traits = Default_traits);
}
{
computes a partition of the polygon defined 
by the points in the range [\ccc{first}, \ccc{beyond}) into convex 
polygons. The counterclockwise-oriented partition polygons are written to
the sequence starting at position \ccc{result}.  The past-the-end iterator for 
the resulting sequence of polygons is returned.
\ccPrecond The points in the range [\ccc{first}, \ccc{beyond}) define a simple 
counterclockwise-oriented polygon.
%\ccIndexSubitem[C]{approx_convex_partition_2}{preconditions}
}

\ccHeading{Requirements}
\begin{enumerate}
    \item \ccc{Traits} is a model of the concept 
          \ccc{PartitionTraits_2}\ccIndexMainItem[c]{PartitionTraits_2}
          and, for the purposes of checking the postcondition that the partition
          produced is valid, it should also be a model of
          the concept \ccc{ConvexPartitionIsValidTraits_2}%
          \ccIndexMainItem[c]{ConvexPartitionIsValidTraits_2}.
    \item \ccc{OutputIterator::value_type} should be \ccc{Traits::Polygon_2}.
    \item \ccc{InputIterator::value_type} should be \ccc{Traits::Point_2},
          which should also be the type of the points stored in an object
          of type \ccc{Traits::Polygon_2}.
    \item Points in the range $[first, beyond)$ must define a simple polygon
          whose vertices are oriented counterclockwise.
\end{enumerate}

The default traits class \ccc{Default_traits} is \ccc{Partition_traits_2},
%\ccIndexTraitsClassDefault{approx_convex_partition_2} 
with the representation type determined by \ccc{InputIterator1::value_type}.

\ccSeeAlso

\ccRefIdfierPage{CGAL::convex_partition_is_valid_2} \\
\ccRefIdfierPage{CGAL::greene_approx_convex_partition_2} \\
\ccRefIdfierPage{CGAL::optimal_convex_partition_2} \\
\ccRefIdfierPage{CGAL::partition_is_valid_2} \\
\ccRefIdfierPage{CGAL::Partition_is_valid_traits_2<Traits, PolygonIsValid>} \\
\ccRefIdfierPage{CGAL::y_monotone_partition_2}

\ccImplementation
This function implements the algorithm of Hertel and Mehlhorn
\cite{hm-ftsp-83} and is based on the class 
\ccc{CGAL::Constrained_triangulation_2}.  Given a triangulation of
the polygon, the function requires $O(n)$ time and
space for a polygon with $n$ vertices.

\ccExample

The following program computes an approximately optimal
convex partitioning of a polygon using the default
traits class and stores the partition polygons in the list 
\ccc{partition_polys}.

\ccIncludeExampleCode{Partition_2/approx_convex_partition_2.cpp}

\end{ccRefFunction}
