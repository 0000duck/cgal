% +------------------------------------------------------------------------+
% | Reference manual page: HalfedgeDSFace.tex
% +------------------------------------------------------------------------+
% | 22.03.1999   Lutz Kettner
% | Package: HalfedgeDS
% | 
\RCSdef{\RCSHalfedgeCycleRev}{$Id: HalfedgeDSFace.tex 29618 2006-03-19 22:54:59Z ameyer $}
\RCSdefDate{\RCSHalfedgeCycleDate}{$Date: 2006-03-19 19:54:59 -0300 (Sun, 19 Mar 2006) $}
% +------------------------------------------------------------------------+

\ccRefPageBegin

%%RefPage: end of header, begin of main body
% +------------------------------------------------------------------------+


\begin{ccRefConcept}{HalfedgeDSHalfedgeCycle}
\label{pageHalfedgeDSItemsHalfedgeCycleRef}


\ccDefinition
  
The concept \ccRefName\ defines the requirements for the local \ccc{Halfedge_cycle} 
type in the \ccc{HalfedgeDS} concept. It is also required in 
the \ccc{Halfedge_cycle_wrapper<Refs,Traits>} member class template of an
items class, see the \ccc{HalfedgeDSItems} concept.

All the halfedges incident to the same face are collectively represented
by this \ccRefName\ \ccc{HDS} item which centralizes all their common properties.

A \ccRefName\ stores a reference to one of the halfedges in the cycle (which one is unspecified).
It optionally stores a reference to the face it belongs to. A type tag indicates whether 
the related member functions are supported. 

All the halfedges connected to the one referenced in a \ccRefName\ must form a cycle, not just a path.

Two halfedges are allowed to be in the same cycle even if they are opposite to each other. 
This allows models of \ccRefName\ to represent open curves, antennas and empty alleys by simply arranging 
the cycle of halfedges to stitch over itself. 

Furthermore, a \ccRefName\ can also represent an isolated vertex composed of just one halfedge
connected to itself and to a single vertex (forming a self-loop).

For the protection of the integrity of the data structure classes such as
\ccc{CGAL::Polyhedron_3} are allowed to redefine the modifying member 
functions to be private. In order to make them accessible for the 
halfedge data structure they must be derived from a base class \ccc{Base}
where the modifying member functions are still public. (The protection
can be bypassed by the user, but not by accident.)

\ccTypes

\ccThree{Halfedge_const_handle}{v.set_halfedge( Halfedge_handle h);}{}
\ccThreeToTwo
%\ccTwo{HalfedgeDSFace:: Halfedge_const_handleXX}{}

\ccNestedType{HalfedgeDS}
    {instantiated \ccc{HalfedgeDS} ( $\equiv$ \ccc{Refs}).}
\ccGlue
\ccNestedType{Base}{base class that allows modifications.}
\ccGlue
\ccNestedType{Vertex}{model of \ccc{HalfedgeDSVertex}.}
\ccGlue
\ccNestedType{Halfedge}{model of \ccc{HalfedgeDSHalfedge}.}
\ccGlue
\ccNestedType{Face}{model of \ccc{HalfedgeDSFace}.}
\ccGlue
\ccNestedType{Vertex_handle}{handle to vertex.}
\ccGlue
\ccNestedType{Halfedge_handle}{handle to halfedge.}
\ccGlue
\ccNestedType{Face_handle}{handle to face.}
\ccGlue
\ccNestedType{Vertex_const_handle}{}
\ccGlue
\ccNestedType{Halfedge_const_handle}{}
\ccGlue
\ccNestedType{Face_const_handle}{}


\ccCreation
\ccCreationVariable{b}

\ccConstructor{Halfedge_cycle();}{default constructor.}

\ccTagFullDeclarations

\ccMethod{Halfedge_handle       halfedge();}{}
\ccGlue
\ccMethod{Halfedge_const_handle halfedge() const;}{
    halfedge of \ccVar.}
\ccGlue
\ccMethod{void set_halfedge( Halfedge_handle h);}{
    sets the halfedge to $h$.}

\ccMethod{Face_handle       face();}{}
\ccGlue
\ccMethod{Face_const_handle face() const;}{
    the face bounded by \ccVar.}
\ccGlue
\ccMethod{void set_face( Face_handle f);}{
    sets bounded face to $h$.}


\ccMethod{bool is_border() const;}
    {is true if \ccVar\ is a border \ccc{halfedge_cycle}.
     \ccPostcond: \ccc{Halfedge_cycle::is_border()} returns the same as \ccc{Halfedge::is_border()}
     for all halfedges in the cycle.
    }

\ccMethod{bool is_isolated_vertex() const;}
    {is true if \ccVar\ is a self-loop of one single halfedge incident on one single vertex}

\ccHasModels

\ccRefIdfierPage{CGAL::HalfedgeDS_halfedge_cycle_base<Refs>}

\ccSeeAlso

\ccRefConceptPage{HalfedgeDS<Traits,Items,Alloc>}\\
\ccRefConceptPage{HalfedgeDSItems}\\
\ccRefConceptPage{HalfedgeDSVertex}\\
\ccRefConceptPage{HalfedgeDSHalfedge}\\
\ccRefConceptPage{HalfedgeDSFace}

\ccTagDefaults
\end{ccRefConcept}

% +------------------------------------------------------------------------+
%%RefPage: end of main body, begin of footer
\ccRefPageEnd
% EOF
% +------------------------------------------------------------------------+

