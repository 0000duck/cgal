% +------------------------------------------------------------------------+
% | Reference manual page: HalfedgeDSFace.tex
% +------------------------------------------------------------------------+
% | 22.03.1999   Lutz Kettner
% | Package: HalfedgeDS
% | 
\RCSdef{\RCSFaceRev}{$Id$}
\RCSdefDate{\RCSFaceDate}{$Date$}
% +------------------------------------------------------------------------+

\ccRefPageBegin

%%RefPage: end of header, begin of main body
% +------------------------------------------------------------------------+


\begin{ccRefConcept}{HalfedgeDSFace}
\label{pageHalfedgeDSItemsFaceRef}

\ccDefinition
  
The concept \ccRefName\ defines the requirements for the local \ccc{Face} 
type in the \ccc{HalfedgeDS} concept. It is also required in 
the \ccc{Face_wrapper<Refs,Traits>} member class template of an
items class, see the \ccc{HalfedgeDSItems} concept.

See page~\ref{pageHalfedgeDSHalfedgeCycleIntro} for an introduction to Faces with Several Boundaries.

{\XHDS
If \ccc{halfedge_cycles} are used, a face optionally stores begin and end iterators for the outer and inner \ccc{halfedge_cycle} sequences. It also optionally stores iterators for the sequence of \ccc{halfedge_cycles} corresponding to isolated vertices.
If, alternatively, direct mutual reference is used, a face optionally stores a reference 
to an incident halfedge that points to the face.
A type tag indicates whether the related member functions are supported. 
}

{\XHDS

The choice between using \ccc{halfedge_cycles} or direct mutual reference
(to define the incidence relation between faces and halfedges) is selected 
via a combination of type tags:

For direct mutual reference use

\ccc{Supports_face_halfedge} $\equiv$  \ccc{CGAL::Tag_true}
and
\ccc{Supports_face_halfedge_cycle} $\equiv$  \ccc{CGAL::Tag_false}

For halfedge\_cycle items use:

\ccc{Supports_face_halfedge} $\equiv$  \ccc{CGAL::Tag_false}
and
\ccc{Supports_face_halfedge_cycle} $\equiv$  \ccc{CGAL::Tag_true}
 
}

\begin{ccAdvanced}
{\XHDS
The storage for \ccc{halfedge_cycle} items is controlled by the \ccc{HDS} itself. Therefore, 
there is a \ccc{Halfedge_cycle_iterator} type local to \ccc{HalfedgeDS} corresponding to a 
container whose elements are models of \ccc{HalfedgeDSHalfedgeCycle}.

Faces maintain their own local sequences of \ccc{halfedge_cycles}. These sequences are
containers of \ccc{Halfedge_cycle_handle} elements, not of \ccc{Halfedge_cycle} elements (like the
container in the HDS). Thus, a Face defines a local \ccc{Halfedge_cycle_handle_iterator} type.

While a \ccc{Halfedge_cycle_iterator} is implicitly convertible to a \ccc{Halfedge_cycle_handle},
a \ccc{Halfedge_cycle_handle_iterator} is {\em dereferenceable} as a \ccc{Halfedge_cycle_handle} 
(i.e. you need to apply the \ccc{operator*} to access the handle)
}
\end{ccAdvanced}

\begin{ccAdvanced}
{\XHDS
As faces can have more than one outer boundary, an upper level data structure could use a single
face to represent disjoint regions each bounded by a distinct outer \ccc{halfedge_cycle}. However,
it is highly recommended to restrict faces to correspond to singly connected regions, and
to use multiple outer cycles to represent disjoint boundary components around the same 
connected region. For example, the surface of a cross section of a cylinder is a single
connected region with two disjoint outer boundaries. This would be given by 1 face with
2 outer cycles.
}
\end{ccAdvanced}

Figure~\ccTexHtml{\ref{figureHalfedgeDSOptionalMethods} 
on page \pageref{figureHalfedgeDSOptionalMethods}}{}\begin{ccHtmlOnly}
  <A HREF="Concept_HalfedgeDS.html#figureHalfedgeDSOptionalMethods"><IMG 
  SRC="cc_ref_up_arrow.gif" ALT="reference arrow" WIDTH="10" HEIGHT="10"></A>
\end{ccHtmlOnly}
depicts the relationship between a halfedge and its incident
halfedges, vertices, and faces.

For the protection of the integrity of the data structure classes such as
\ccc{CGAL::Polyhedron_3} are allowed to redefine the modifying member 
functions to be private. In order to make them accessible for the 
halfedge data structure they must be derived from a base class \ccc{Base}
where the modifying member functions are still public. (The protection
can be bypassed by the user, but not by accident.)

\ccHeading{Release Notes}

% +------------------------------------------------------------------------+
% | Reference manual page: HalfedgeDS.tex
% +------------------------------------------------------------------------+
% | 22.03.1999   Lutz Kettner
% | Package: HalfedgeDS
% | 
\RCSdef{\RCSHalfedgeDSRev}{$Id: HalfedgeDS.tex 38221 2007-04-17 16:31:42Z spion $}
\RCSdefDate{\RCSHalfedgeDSDate}{$Date: 2007-04-17 13:31:42 -0300 (Tue, 17 Apr 2007) $}
% +------------------------------------------------------------------------+

{\bf\ttfamily
Beginning with \cgal\ version 3.4 this package has been extended
to efficiently support the orientable-2-manifolds that correspond to the topology 
of the arrangement of curves on surfaces.
The added features are purely an extension so the extended structure is completely 
backward compatible with version 3.3.1.
}


\ccTypes

\ccThree{Halfedge_cycle_const_handle_const_iterator}{v.set_halfedge( Halfedge_handle h);}{}
\ccThreeToTwo
%\ccTwo{HalfedgeDSFace:: Halfedge_const_handleXX}{}

\ccNestedType{HalfedgeDS}
    {instantiated \ccc{HalfedgeDS} ( $\equiv$ \ccc{Refs}).}
\ccGlue
\ccNestedType{size_type}{size type.}
\ccGlue
\ccNestedType{Base}{base class that allows modifications.}
\ccGlue
\ccNestedType{Vertex}{model of \ccc{HalfedgeDSVertex}.}
\ccGlue
\ccNestedType{Halfedge}{model of \ccc{HalfedgeDSHalfedge}.}
\ccGlue
\ccNestedType{\XHDS Halfedge_cycle}{\XHDS model of \ccc{HalfedgeDSHalfedgeCycle}.}
\ccGlue
\ccNestedType{Vertex_handle}{handle to vertex.}
\ccGlue
\ccNestedType{Halfedge_handle}{handle to halfedge.}
\ccGlue
\ccNestedType{\XHDS Halfedge_cycle_handle}{\XHDS handle to \ccc{halfedge_cycle}.}
\ccGlue
\ccNestedType{Face_handle}{handle to face.}
\ccGlue
\ccNestedType{Vertex_const_handle}{}
\ccGlue
\ccNestedType{Halfedge_const_handle}{}
\ccGlue
\ccNestedType{Face_const_handle}{}
\ccGlue
\ccNestedType{\XHDS Halfedge_cycle_const_handle}{}


\ccNestedType{\XHDS Halfedge_cycle_const_handle_const_iterator}
\ccGlue
\ccNestedType{\XHDS Halfedge_cycle_handle_iterator}{\XHDS iterator to the local sequence of \ccc{Halfedge_cycle_handles}.}

\ccNestedType{Supports_face_halfedge}{\ccc{CGAL::Tag_true} or \ccc{CGAL::Tag_false}.}
\ccGlue
\ccNestedType{\XHDS Supports_face_halfedge_cycle}{\XHDS \ccc{CGAL::Tag_true} or \ccc{CGAL::Tag_false}.}


\ccCreation
\ccCreationVariable{f}

\ccConstructor{Face();}{default constructor.}

\ccTagFullDeclarations

\ccHeading{Operations available if \ccc{Supports_face_halfedge_cycle} $\equiv$ \ccc{CGAL::Tag_true}}

\ccMethod{\XHDS size_type size_of_isolated_vertices() const;}{\XHDS number of isolated  vertices of \ccVar}

\ccMethod{\XHDS size_type size_of_outer_halfedge_cycles() const;}
\ccGlue
\ccMethod{\XHDS size_type size_of_inner_halfedge_cycles() const;}{\XHDS number of outer/inner cycles of \ccVar}



\ccMethod{\XHDS Halfedge_cycle_handle_iterator isolated_vertices_begin();}{}
\ccGlue
\ccMethod{\XHDS Halfedge_cycle_const_handle_const_iterator  isolated_vertices_begin() const;}
         {\XHDS the begin iterator for the sequence of isolated vertices of \ccVar.}

\ccMethod{\XHDS Halfedge_cycle_handle_iterator outer_halfedge_cycles_begin();}{}
\ccGlue
\ccMethod{\XHDS Halfedge_cycle_const_handle_const_iterator  outer_halfedge_cycles_begin() const;}
\ccGlue
\ccMethod{\XHDS Halfedge_cycle_handle_iterator inner_halfedge_cycles_begin();}{}
\ccGlue
\ccMethod{\XHDS Halfedge_cycle_const_handle_const_iterator  inner_halfedge_cycles_begin() const;}
         {\XHDS the begin iterator for the sequence of outer/inner cycles of \ccVar.}


\ccMethod{\XHDS Halfedge_cycle_handle_iterator isolated_vertices_end();}{}
\ccGlue
\ccMethod{\XHDS Halfedge_cycle_const_handle_const_iterator isolated_vertices_end() const;}
         {\XHDS the past-the-end iterator for the sequence of isolated vertices of \ccVar.}


\ccMethod{\XHDS Halfedge_cycle_handle_iterator outer_halfedge_cycles_end();}{}
\ccGlue
\ccMethod{\XHDS Halfedge_cycle_const_handle_const_iterator outer_halfedge_cycles_end() const;}
\ccGlue
\ccMethod{\XHDS Halfedge_cycle_handle_iterator inner_halfedge_cycles_end();}{}
\ccGlue
\ccMethod{\XHDS Halfedge_cycle_const_handle_const_iterator inner_halfedge_cycles_end() const;}
         {\XHDS the past-the-end iterator for the sequence of outer/inner cycles of \ccVar.}


\ccMethod{\XHDS void isolated_vertex_push_back( Halfedge_cycle_handle v );}
         {\XHDS appends $v$ to the isolated vertices sequence}

\ccMethod{\XHDS void outer_halfedge_cycle_push_back( Halfedge_cycle_handle c );}
\ccGlue
\ccMethod{\XHDS void inner_halfedge_cycle_push_back( Halfedge_cycle_handle c );}
         {\XHDS appends $c$ to the outer/inner cycles sequence}

\ccMethod{\XHDS Halfedge_cycle_handle_iterator isolated_vertex_insert( Halfedge_cycle_handle_iterator pos, Halfedge_cycle_handle v );}
         {\XHDS inserts $v$ before $pos$ in the isolated vertices sequence}

\ccMethod{\XHDS Halfedge_cycle_handle_iterator outer_halfedge_cycle_insert( Halfedge_cycle_handle_iterator pos, Halfedge_cycle_handle c );}
\ccGlue
\ccMethod{\XHDS Halfedge_cycle_handle_iterator inner_halfedge_cycle_insert( Halfedge_cycle_handle_iterator pos, Halfedge_cycle_handle c );}
         {\XHDS inserts $c$ before $pos$ in the outer/inner cycles sequence}

\ccMethod{\XHDS void isolated_vertex_erase( Halfedge_cycle_handle_iterator it );}
         {\XHDS removes the isolated vertex pointed to be $it$ in the coresponding sequence}

\ccMethod{\XHDS void outer_halfedge_cycle_erase( Halfedge_cycle_handle_iterator it );}
\ccGlue
\ccMethod{\XHDS void inner_halfedge_cycle_erase( Halfedge_cycle_handle_iterator it );}
         {\XHDS removes the cycle pointed to be $it$ in the corresponding sequence}

\ccMethod{\XHDS void isolated_vertices_clear();}{\XHDS removes all isolated vertices. }

\ccMethod{\XHDS void outer_halfedge_cycles_clear();}{}
\ccGlue
\ccMethod{\XHDS void inner_halfedge_cycles_clear();}{\XHDS removes all outer/inner cycles. }


\ccHeading{Operations available if \ccc{Supports_face_halfedge} $\equiv$ \ccc{CGAL::Tag_true}}

\ccMethod{Halfedge_handle       halfedge();}{}
\ccGlue
\ccMethod{Halfedge_const_handle halfedge() const;}{
    incident halfedge that points to \ccVar.}

\ccMethod{void set_halfedge( Halfedge_handle h);}{
    sets incident halfedge to $h$.}


\ccHasModels

\ccRefIdfierPage{CGAL::HalfedgeDS_face_base<Refs>}\\
\ccRefIdfierPage{CGAL::HalfedgeDS_face_min_base<Refs>}

\ccSeeAlso

\ccRefConceptPage{HalfedgeDS<Traits,Items,Alloc>}\\
\ccRefConceptPage{HalfedgeDSItems}\\
\ccRefConceptPage{HalfedgeDSVertex}\\
\ccRefConceptPage{HalfedgeDSHalfedge}\\
\ccRefConceptPage{HalfedgeDSHalfedgeCycle}

\ccTagDefaults
\end{ccRefConcept}

% +------------------------------------------------------------------------+
%%RefPage: end of main body, begin of footer
\ccRefPageEnd
% EOF
% +------------------------------------------------------------------------+

