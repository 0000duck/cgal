% +------------------------------------------------------------------------+
% | Reference manual page: HalfedgeDS_items_2.tex
% +------------------------------------------------------------------------+
% | 22.03.1999   Lutz Kettner
% | Package: HalfedgeDS
% | 
\RCSdef{\RCSHalfedgeDSitemsRev}{$Id$}
\RCSdefDate{\RCSHalfedgeDSitemsDate}{$Date$}
% +------------------------------------------------------------------------+

\ccRefPageBegin

%%RefPage: end of header, begin of main body
% +------------------------------------------------------------------------+


\begin{ccRefClass}{HalfedgeDS_items_2}
\label{pageHalfedgeDSitemsRef}

\ccDefinition
  
The class \ccRefName\ is a model of the \ccc{HalfedgeDSItems} concept.
It uses the default types for vertices, halfedges, and faces that
declare all incidences supported by a \ccc{HalfedgeDS}.
{\XHDS Halfedge cycles are not supported. 
The vertex also contains a point of type \ccc{Traits::Point_2}, 
where \ccc{Traits} is the template argument of the corresponding \ccc{HalfedgeDS}}.

\ccInclude{CGAL/HalfedgeDS_items_2.h}

\ccIsModel

\ccRefConceptPage{HalfedgeDSItems}

\ccSeeAlso

\ccRefIdfierPage{CGAL::HalfedgeDS_items_with_halfedge_cycle_2}\\
\ccRefIdfierPage{CGAL::HalfedgeDS_min_items}\\
\ccRefIdfierPage{CGAL::Polyhedron_items_3}\\
\ccRefConceptPage{HalfedgeDS<Traits,Items,Alloc>}\\
\ccRefConceptPage{PolyhedronItems_3}\\
\ccRefIdfierPage{CGAL::HalfedgeDS_vertex_base<Refs>}\\
\ccRefIdfierPage{CGAL::HalfedgeDS_halfedge_base<Refs>}\\
\ccRefIdfierPage{CGAL::HalfedgeDS_face_base<Refs>}\\
\ccRefIdfierPage{CGAL::HalfedgeDS_halfedge_cycle_base<Refs>}

\ccExample

The following example shows the canonical implementation of the
\ccRefName\ class. It uses the base classes for the item types that
are provided in the library.

\begin{ccExampleCode}
struct HalfedgeDS_items_2 {
    template < class Refs, class Traits>
    struct Vertex_wrapper {
        typedef typename Traits::Point_2 Point;
        typedef CGAL::HalfedgeDS_vertex_base< Refs, Tag_true, Point> Vertex;
    };
    template < class Refs, class Traits>
    struct Halfedge_wrapper {
        typedef CGAL::HalfedgeDS_halfedge_base< Refs> Halfedge;
    };
    template < class Refs, class Traits>
    struct Face_wrapper {
        typedef CGAL::HalfedgeDS_face_base< Refs> Face;
    };

    // Halfedge_cycles are not supported by default, hence a dummy type is used.
    template < class Refs, class Traits>
    struct Halfedge_cycle_wrapper {
        typedef int Halfedge_cycle;
    };
};
\end{ccExampleCode}

The following example shows a class definition for a new items class
derived from the \ccRefName\ class. It replaces the \ccc{Face_wrapper}
with a new definition of a face that contains a member variable for
color. The new face makes use of the face base class provided in the
library.

\begin{ccExampleCode}
// A face type with a color member variable.
template <class Refs>
struct My_face : public CGAL::HalfedgeDS_face_base<Refs> {
    CGAL::Color color;
    My_face() {}
    My_face( CGAL::Color c) : color(c) {}
};

// An items type using my face.
struct My_items : public CGAL::HalfedgeDS_items_2 {
    template <class Refs, class Traits>
    struct Face_wrapper {
        typedef My_face<Refs> Face;
    };
};
\end{ccExampleCode}

\end{ccRefClass}

% +------------------------------------------------------------------------+
%%RefPage: end of main body, begin of footer
\ccRefPageEnd
% EOF
% +------------------------------------------------------------------------+

