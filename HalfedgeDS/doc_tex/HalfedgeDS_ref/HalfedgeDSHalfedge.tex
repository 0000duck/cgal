% +------------------------------------------------------------------------+
% | Reference manual page: HalfedgeDSHalfedge.tex
% +------------------------------------------------------------------------+
% | 22.03.1999   Lutz Kettner
% | Package: HalfedgeDS
% | 
\RCSdef{\RCSHalfedgeRev}{$Id$}
\RCSdefDate{\RCSHalfedgeDate}{$Date$}
% +------------------------------------------------------------------------+

\ccRefPageBegin

%%RefPage: end of header, begin of main body
% +------------------------------------------------------------------------+


\begin{ccRefConcept}{HalfedgeDSHalfedge}
\label{pageHalfedgeDSItemsHalfedgeRef}

\ccDefinition
  
The concept \ccRefName\ defines the requirements for the local \ccc{Halfedge} 
type in the \ccc{HalfedgeDS} concept. It is also required in 
the \ccc{Halfedge_wrapper<Refs,Traits>} member class template of an
items class, see the \ccc{HalfedgeDSItems} concept.

A halfedge is an oriented edge between two vertices. It is always
paired with a halfedge pointing in the opposite direction. The
\ccc{opposite()} member function returns this halfedge of opposite
orientation. The \ccc{next()} member function points to the successor
halfedge along the face -- or if the halfedge is a border halfedge --
along the open region. 

A halfedge optionally stores a reference to the
previous halfedge along the face, a reference to an incident vertex,
{\XHDS a reference to the \ccc{halfedge_cycle} containing it},
and a reference to the incident face,
Type tags indicate whether the related member functions are supported.

{\XHDS

The current design supports two {\em mutually exclusive} methods for 
storing the incidence relationship between faces and halfedge cycles:
{\em direct mutual reference}, that is, having faces and halfedges reference 
each other directly, or {\em through a common ccb (Connected Components of the Boundary) item}.
 


The choice is selected via a combination of type tags:

For direct mutual reference use

\ccc{Supports_halfedge_face} $\equiv$  \ccc{CGAL::Tag_true}
and
\ccc{Supports_halfedge_halfedge_cycle} $\equiv$  \ccc{CGAL::Tag_false}

For \ccc{halfedge_cycle} items use:

\ccc{Supports_halfedge_face} $\equiv$  \ccc{CGAL::Tag_false}
and
\ccc{Supports_halfedge_halfedge_cycle} $\equiv$  \ccc{CGAL::Tag_true}
 
}


Figure~\ccTexHtml{\ref{figureHalfedgeDSOptionalMethods}}{}\begin{ccHtmlOnly}
  <A HREF="#figureHalfedgeDSOptionalMethods"><IMG 
  SRC="cc_ref_up_arrow.gif" ALT="reference arrow" WIDTH="10" HEIGHT="10"></A>
\end{ccHtmlOnly}
depicts the relationship between a halfedge and its incident
halfedges, vertices, and faces.

\begin{ccTexOnly}
    \begin{figure}[bht]
        \begin{center}
          \parbox{\textwidth}{%
              \includegraphics[width=\textwidth]%
                  {HalfedgeDS_ref/fig/hds_optional}%
          }
        \end{center}
        \caption{The three classes \protect\ccc{Vertex}, 
          \protect\ccc{Halfedge}, and 
          \protect\ccc{Face} of the halfedge data structure. Member
          functions with shaded background are mandatory. The others
          are optionally supported.}
        \label{figureHalfedgeDSOptionalMethods}
    \end{figure}
\end{ccTexOnly}

\begin{ccHtmlOnly}
    <CENTER>
    <A NAME="figureHalfedgeDSOptionalMethods">
    <A HREF="fig/hds_optional.gif">
        <img src="fig/hds_optional_small.gif" 
             alt="Class Diagram"></A><BR>
    <A HREF="fig/hds_optional.gif">Figure:</A>
    The three classes <I>Vertex</I>, <I>Halfedge</I>, and 
          <I>Face</I> of the halfedge data structure. Member
          functions with shaded background are mandatory. The others
          are optionally supported.
    </CENTER>
\end{ccHtmlOnly}

For the protection of the integrity of the data structure classes such
as \ccc{CGAL::Polyhedron_3} are allowed to redefine the modifying member
functions to be private. In order to make them accessible for the
halfedge data structure they must be derived from a base class
\ccc{Base} where the modifying member functions are still public. Even
more protection is provided for the \ccc{set_opposite()} member
function. The base class \ccc{Base_base} provides access to it.  (The
protection could be bypassed also by an user, but not by accident.)

\ccHeading{Release Notes}

% +------------------------------------------------------------------------+
% | Reference manual page: HalfedgeDS.tex
% +------------------------------------------------------------------------+
% | 22.03.1999   Lutz Kettner
% | Package: HalfedgeDS
% | 
\RCSdef{\RCSHalfedgeDSRev}{$Id: HalfedgeDS.tex 38221 2007-04-17 16:31:42Z spion $}
\RCSdefDate{\RCSHalfedgeDSDate}{$Date: 2007-04-17 13:31:42 -0300 (Tue, 17 Apr 2007) $}
% +------------------------------------------------------------------------+

{\bf\ttfamily
Beginning with \cgal\ version 3.4 this package has been extended
to efficiently support the orientable-2-manifolds that correspond to the topology 
of the arrangement of curves on surfaces.
The added features are purely an extension so the extended structure is completely 
backward compatible with version 3.3.1.
}


\ccTypes

\ccThree{Halfedge_const_handle}{v.set_halfedge( Halfedge_handle h);}{}
\ccThreeToTwo

\ccNestedType{HalfedgeDS}
    {instantiated \ccc{HalfedgeDS} ( $\equiv$ \ccc{Refs}).}
\ccGlue
\ccNestedType{Base}{base class that allows modifications.}
\ccGlue
\ccNestedType{Base_base}{base class to access \ccc{set_opposite()}.}
\ccGlue
\ccNestedType{Vertex}{model of \ccc{HalfedgeDSVertex}.}
\ccGlue
\ccNestedType{Face}{model of \ccc{HalfedgeDSFace}.}
\ccGlue
\ccNestedType{\XHDS Halfedge_cycle}{\XHDS model of \ccc{HalfedgeDSHalfedgeCycle}.}

\ccNestedType{Vertex_handle}{handle to vertex.}
\ccGlue
\ccNestedType{Halfedge_handle}{handle to halfedge.}
\ccGlue
\ccNestedType{Face_handle}{handle to face.}
\ccGlue
\ccNestedType{\XHDS Halfedge_cycle_handle}{\XHDS handle to \ccc{halfedge_cycle}.}
\ccGlue
\ccNestedType{Vertex_const_handle}{}
\ccGlue
\ccNestedType{Halfedge_const_handle}{}
\ccGlue
\ccNestedType{Face_const_handle}{}
\ccGlue
\ccNestedType{\XHDS Halfedge_cycle_const_handle}{}

\ccNestedType{Supports_halfedge_prev}{\ccc{CGAL::Tag_true} or 
  \ccc{CGAL::Tag_false}.}
\ccGlue
\ccNestedType{Supports_halfedge_vertex}{\~{}}
\ccGlue
\ccNestedType{Supports_halfedge_face}{\~{}}
\ccGlue
\ccNestedType{\XHDS Supports_halfedge_halfedge_cycle}{\~{}}


\ccCreation
\ccCreationVariable{h}

\ccConstructor{Halfedge();}{default constructor.}

\ccOperations
\ccTagFullDeclarations

\ccMethod{Halfedge_handle opposite();}{}
\ccGlue
\ccMethod{Halfedge_const_handle opposite() const;}{the opposite halfedge.}
\ccGlue
\ccMethod{void set_opposite( Halfedge_handle h);}{
    sets opposite halfedge to $h$.}

\ccMethod{Halfedge_handle next();}{}
\ccGlue
\ccMethod{Halfedge_const_handle next() const;}
    {the next halfedge along the \ccc{halfedge_cycle}.}
\ccGlue
\ccMethod{void set_next( Halfedge_handle h);}{
    sets next halfedge to $h$.}

\ccMethod{bool             is_border() const;}
    {is true if \ccVar\ is a border halfedge.}

\ccHeading{Operations available if \ccc{Supports_halfedge_prev} $\equiv$ 
           \ccc{CGAL::Tag_true}}

\ccMethod{Halfedge_handle prev();}{}
\ccGlue
\ccMethod{Halfedge_const_handle prev() const;}
    {the previous halfedge along the \ccc{halfedge_cycle}.}
\ccGlue
\ccMethod{void set_prev( Halfedge_handle h);}{
    sets prev halfedge to $h$.}

\ccHeading{Operations available if \ccc{Supports_halfedge_vertex} $\equiv$ 
           \ccc{CGAL::Tag_true}}

\ccMethod{Vertex_handle       vertex();}{}
\ccGlue
\ccMethod{Vertex_const_handle vertex() const;}{the incident vertex of \ccVar.}
\ccGlue
\ccMethod{void set_vertex( Vertex_handle v);}{
    sets incident vertex to $v$.}

\ccHeading{Operations available if \ccc{Supports_halfedge_halfedge_cycle} $\equiv$ 
           \ccc{CGAL::Tag_true}}

\ccMethod{\XHDS Halfedge_cycle_handle       halfedge_cycle();}{}
\ccGlue
\ccMethod{\XHDS Halfedge_cycle_const_handle halfedge_cycle() const;}
    {\XHDS the \ccc{halfedge_cycle} of \ccVar.}
\ccGlue
\ccMethod{\XHDS void set_halfedge_cycle( Halfedge_cycle_handle c);}{\XHDS 
    sets the \ccc{halfedge_cycle} to $c$.}

\ccHeading{Operations available if \ccc{Supports_halfedge_face} $\equiv$ \ccc{CGAL::Tag_true} 
          $or$ 
          ( \ccc{Supports_halfedge_halfedge_cycle} $\equiv$ \ccc{CGAL::Tag_true}
          $and$
          \ccc{Supports_halfedge_cycle_face} $\equiv$ \ccc{CGAL::Tag_true}
          ) }

\ccMethod{Face_handle       face();}{}
\ccGlue
\ccMethod{Face_const_handle face() const;}
    {the incident face of \ccVar.  If \ccVar\ is a border halfedge 
      the result is default construction of the handle.}

\ccHeading{Operations available if \ccc{Supports_halfedge_face} $\equiv$ \ccc{CGAL::Tag_true} }

\ccMethod{void set_face( Face_handle f);}{set face to $f$.}

\ccHasModels

\ccRefIdfierPage{CGAL::HalfedgeDS_halfedge_base<Refs>}\\
\ccRefIdfierPage{CGAL::HalfedgeDS_halfedge_min_base<Refs>}

\ccSeeAlso

\ccRefConceptPage{HalfedgeDS<Traits,Items,Alloc>}\\
\ccRefConceptPage{HalfedgeDSItems}\\
\ccRefConceptPage{HalfedgeDSVertex}\\
\ccRefConceptPage{HalfedgeDSFace}
\ccRefConceptPage{HalfedgeDSHalfedgeCycle}

\end{ccRefConcept}

% +------------------------------------------------------------------------+
%%RefPage: end of main body, begin of footer
\ccRefPageEnd
% EOF
% +------------------------------------------------------------------------+

