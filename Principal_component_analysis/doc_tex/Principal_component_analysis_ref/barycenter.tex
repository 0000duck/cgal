% +------------------------------------------------------------------------+
% | Reference manual page: barycenter.tex
% +------------------------------------------------------------------------+
% | 
% | March 2005  Pierre Alliez and Sylvain Pion
% | Package:   Principal Component Analysis
% |
% +------------------------------------------------------------------------+

\begin{ccRefFunction}{barycenter}  
%% add template arg's if necessary

\ccIndexMainItemBegin{barycenter}

\ccDefinition
  
The function \ccRefName\ computes the barycenter (weighted center of mass) of a
set of weighted 2D or 3D objects.  The weight associated to each object is
specified using a \ccc{std::pair} storing the object and its weight.

\ccInclude{CGAL/barycenter.h}

There is a set of overloaded \ccc{barycenter} functions for 2D and 3D
weighted objects.
The user can also optionally pass an explicit kernel, in case the default,
based on \ccc{Kernel_traits} is not sufficient.
The dimension is also deduced automatically.

\ccFunction{template < typename InputIterator >
            K::Point_2
            barycenter(InputIterator first, InputIterator beyond);}
{ computes the barycenter of a non-empty set of 2D weighted points.
  \ccc{K} is \ccc{Kernel_traits<std::iterator_traits<InputIterator>::value_type::first_type>::Kernel}.
  The value type must be \ccc{std::pair<K::Point_2, K::FT>}.
\ccPrecond{first != beyond, and the sum of the weights is non-zero.} }

\ccFunction{template < typename InputIterator, typename K >
            K::Point_2
            barycenter(InputIterator first, InputIterator beyond, const K & k);}
{ computes the barycenter of a non-empty set of 2D weighted points.
  The value type must be \ccc{std::pair<K::Point_2, K::FT>}.
\ccPrecond{first != beyond, and the sum of the weights is non-zero.} }

\ccFunction{template < typename InputIterator >
            K::Point_3
            barycenter(InputIterator first, InputIterator beyond);}
{ computes the barycenter of a non-empty set of 3D weighted points.
  \ccc{K} is \ccc{Kernel_traits<std::iterator_traits<InputIterator>::value_type::first_type>::Kernel}.
  The value type must be \ccc{std::pair<K::Point_3, K::FT>}.
\ccPrecond{first != beyond, and the sum of the weights is non-zero.} }

\ccFunction{template < typename InputIterator, typename K >
            K::Point_3
            barycenter(InputIterator first, InputIterator beyond, const K & k);}
{ computes the barycenter of a non-empty set of 3D weighted points.
  The value type must be \ccc{std::pair<K::Point_3, K::FT>}.
\ccPrecond{first != beyond, and the sum of the weights is non-zero.} }


%\ccExample
%
%In the following example we use \stl\ containers of 2D and 3D weighted points,
%and compute their barycenters. The kernel from which the input points come is
%automatically deduced by the function.
%
%\ccIncludeExampleCode{Linear_least_squares_fitting/barycenter.C}

\ccSeeAlso
\ccRefIdfierPage{CGAL::centroid}

\ccIndexMainItemEnd{barycenter}

\end{ccRefFunction}
