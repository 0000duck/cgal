% +------------------------------------------------------------------------+
% | Reference manual page
% +------------------------------------------------------------------------+
% | March 2005  Pierre Alliez and Sylvain Pion
% | Package: Principal Component Analysis
% | 
% |
% +------------------------------------------------------------------------+

\chapter{Principal Component Analysis}
\label{ref_chapter_pca}

\ccChapterAuthor{Pierre Alliez and Sylvain Pion}

This \cgal\ package provides functions to compute global informations
on the shape of a set of 2D or 3D objects such as points. It provides the computation of bounding boxes, centroids of point sets, barycenters of weighted point sets, as well as linear least squares fitting. It assumes the set of kernel primitive elements to be stored into an iterator range of a container.


\subsection*{Definition}
Given a point set in $R^d$, linear least squares fitting amounts to
find the linear sub-space of $R^d$ which minimizes the sum of squared
distances from the points to their projection onto this linear
sub-space. This problem is equivalent to search for the linear
sub-space which maximizes the variance of projected points, the latter
being obtained by eigen decomposition of the covariance
matrix. Eigenvectors corresponding to large eigenvalues are the
directions in which the data has strong component, or equivalently
large variance. If eigenvalues are the same there is no preferable
sub-space.

\ccHeading{Functions}

\ccRefIdfierPage{CGAL::barycenter}\\
\ccRefIdfierPage{CGAL::bounding_box}\\
\ccRefIdfierPage{CGAL::centroid}\\
\ccRefIdfierPage{CGAL::linear_least_squares_fitting_2}\\
\ccRefIdfierPage{CGAL::linear_least_squares_fitting_3}\\
\clearpage

\lcHtml{\ccHeading{Alphabetical Listing of Reference Pages}}
