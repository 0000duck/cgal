% +------------------------------------------------------------------------+
% | Reference manual page: linear_least_squares_fitting_3.tex
% +------------------------------------------------------------------------+
% | 
% | March 2005  Pierre Alliez and Sylvain Pion and Ankit Gupta
% | Package:   Principal Component Analysis
% | 
% |
% +------------------------------------------------------------------------+

\begin{ccRefFunction}{linear_least_squares_fitting_3}  
%% add template arg's if necessary

\ccDefinition
  
The function \ccRefName\ computes the best fitting 3D line or plane of 3D
sets of Points, Segments, Triangles, Rectangles, Circles, Spheres, Cuboids and Tetrahedrons. The best fit linear sub-space minimizes
the sum of squared distances from the points to their projections onto the
sub-space.

\ccInclude{CGAL/linear_least_squares_fitting_3.h}

\ccFunction{template < typename InputIterator, typename K, typename Tag > 
            typename K::FT 
             linear_least_squares_fitting_3(InputIterator first,
                                            InputIterator beyond,
                                            typename K::Line_3 & line,
                                            typename K::Point_3 & centroid,     
                                            const K & k,
					    const Tag& t);}
{ computes the best fitting 3D line of a 3D object set in the range
[\ccc{first},\ccc{beyond}). The value returned is a fitting quality
between $0$ and $1$, where $0$ means that the variance is the same
along any line (a horizontal line going through the centroid is output
by default), and $1$ means that the variance is null orthogonally
to the best fitting line. }

The class \ccc{K} is the kernel in which the type
\ccc{InputIterator::value_type} is defined. It can be omitted and
deduced automatically from the value type.

The tag \ccc{t} identifies the type of manifold of the objects in the object set.

\ccHeading{Requirements}

\begin{enumerate}
   \item  \ccc{InputIterator::value_type} is equivalent to \ccc{K::Point_3} or \ccc{K::Triangle_3}.
   \item  \ccc{line} is the best fitting line computed.
   \item  \ccc{centroid} is the centroid computed. This parameter can be
          omitted.
   \item  \ccc{t} is the tag identifying the type of manifold of the objects in the object set. It should be one of \ccc{PCA_dimension_0_tag}, \ccc{PCA_dimension_1_tag}, \ccc{PCA_dimension_2_tag} or \ccc{PCA_dimension_3_tag}. Also, it shouldn't be of a greater dimension than the geometry of the object. For example, a \ccc{Circle} can't have a \ccc{PCA_dimension_3_tag} tag.
\end{enumerate}


\ccFunction{template < typename InputIterator, typename K, typename Tag > 
            typename K::FT 
             linear_least_squares_fitting_3(InputIterator first,
                                            InputIterator beyond,
                                            typename K::Plane_3 & plane,
                                            typename K::Point_3 & centroid,     
                                            const K & k,
					    const Tag& t);}
{ computes the best fitting 3D plane of a 3D object set in the range
[\ccc{first},\ccc{beyond}). The value returned is a fitting quality
between $0$ and $1$, where $0$ means that the variance is the same
along any plane (a horizontal plane going through the centroid is output
by default), and $1$ means that the variance is null orthogonally
to the best fitting plane. }

The class \ccc{K} is the kernel in which the type
\ccc{InputIterator::value_type} is defined. It can be omitted and
deduced automatically from the value type.

The tag \ccc{t} identifies the type of manifold of the objects in the object set.

\ccHeading{Requirements}

\begin{enumerate}
   \item  \ccc{InputIterator::value_type} is equivalent to \ccc{K::Point_3} or \ccc{K::Triangle_3}.
   \item  \ccc{plane} is the best fitting plane computed.
   \item  \ccc{centroid} is the centroid computed. This parameter can be
          omitted.
   \item  \ccc{t} is the tag identifying the type of manifold of the objects in the object set. It should be one of \ccc{PCA_dimension_0_tag}, \ccc{PCA_dimension_1_tag}, \ccc{PCA_dimension_2_tag} or \ccc{PCA_dimension_3_tag}. Also, it shouldn't be of a greater dimension than the geometry of the object. For example, a \ccc{Circle} can't have a \ccc{PCA_dimension_3_tag} tag.
\end{enumerate}

\end{ccRefFunction}
