% +------------------------------------------------------------------------+
% | Reference manual page: linear_least_squares_fitting_3.tex
% +------------------------------------------------------------------------+
% | 
% | March 2005  Pierre Alliez and Sylvain Pion and Ankit Gupta
% | Package:   Principal Component Analysis
% | 
% |
% +------------------------------------------------------------------------+

\begin{ccRefFunction}{linear_least_squares_fitting_3}  
%% add template arg's if necessary

\ccDefinition
  
The function \ccRefName\ computes the best fitting 3D line or plane of 3D sets of points, segments, triangles, spheres, cuboids and tetrahedrons. The best fit linear sub-space minimizes the sum of squared distances from the points to their projections onto the linear sub-space.

\ccInclude{CGAL/linear_least_squares_fitting_3.h}

\ccFunction{template < typename InputIterator, typename K, typename Tag > 
            typename K::FT 
             linear_least_squares_fitting_3(InputIterator first,
                                            InputIterator beyond,
                                            typename K::Line_3& line,
                                            typename K::Point_3& centroid,     
                                            const K& k,
					                                  const Tag& tag);}
{ computes the best fitting 3D line of a 3D object set in the range
[\ccc{first},\ccc{beyond}). The value returned is a fitting quality
between $0$ and $1$, where $0$ means that the variance is the same
along any line (a horizontal line going through the centroid is output
by default), and $1$ means that the variance is null orthogonally
to the best fitting line. }

The class \ccc{K} is the kernel in which the type
\ccc{InputIterator::value_type} is defined. It can be omitted and
deduced automatically from the value type.

The tag \ccc{tag} identifies the dimension of the objects in the object set. For point sets it should be 0. For segment sets it could be 1 or 0 according to whether one wants to fit the entire segment or just the end points. For triangle sets it can range from 0 to 2 according to whether one wants to fit the end points, the segments or the entire triangle. For cuboid sets it can range from 0 to 3 according to whether one wants to fit the corners, the segments, the faces or the entire solid cuboid. For sphere sets it can be 2 or 3 according to whether one wants to fit the surface of the sphere or the entire solid ball. For tetrahedron sets it ranges from 0 to 3 according to whether one wants to fit the  points, the segments, the surface triangles or the entire solid tetrahedron.

\ccHeading{Requirements}

\begin{enumerate}
   \item  \ccc{InputIterator::value_type} is equivalent to \ccc{K::Point_3} or
          \ccc{K::Segment_3} or \ccc{Triangle_3} or \ccc{Cuboid_3} or
          \ccc{Sphere_3} or \ccc{Tetrahedron_3}.
   \item  \ccc{line} is the best fitting line computed.
   \item  \ccc{centroid} is the centroid computed. This parameter can be
          omitted.
   \item  \ccc{tag} is the tag identifying the dimension of the objects in the object set. It should be one of \ccc{PCA_dimension_0_tag}, \ccc{PCA_dimension_1_tag}, \ccc{PCA_dimension_2_tag} or \ccc{PCA_dimension_3_tag}. Also, it shouldn't be of a greater dimension than the geometry of the object. For example, a \ccc{Triangle} can't have a \ccc{PCA_dimension_3_tag} tag.
\end{enumerate}


\ccFunction{template < typename InputIterator, typename K, typename Tag > 
            typename K::FT 
             linear_least_squares_fitting_3(InputIterator first,
                                            InputIterator beyond,
                                            typename K::Plane_3& plane,
                                            typename K::Point_3& centroid,     
                                            const K& k,
					                                  const Tag& tag);}
{ computes the best fitting 3D plane of a 3D object set in the range
[\ccc{first},\ccc{beyond}). The value returned is a fitting quality
between $0$ and $1$, where $0$ means that the variance is the same
along any plane (a horizontal plane going through the centroid is output
by default), and $1$ means that the variance is null orthogonally
to the best fitting plane. }

The class \ccc{K} is the kernel in which the type
\ccc{InputIterator::value_type} is defined. It can be omitted and
deduced automatically from the value type.

The tag \ccc{tag} identifies the dimension of the objects in the object set (see above). 

\ccHeading{Requirements}

\begin{enumerate}
   \item  \ccc{InputIterator::value_type} is equivalent to \ccc{K::Point_3} or
          \ccc{K::Segment_3} or \ccc{Triangle_3} or \ccc{Cuboid_3} or
          \ccc{Sphere_3} or \ccc{Tetrahedron_3}.
   \item  \ccc{plane} is the best fitting plane computed.
   \item  \ccc{centroid} is the centroid computed. This parameter can be
          omitted.
   \item  \ccc{tag} is the tag identifying the dimension of the objects in the object set. It should be one of \ccc{PCA_dimension_0_tag}, \ccc{PCA_dimension_1_tag}, \ccc{PCA_dimension_2_tag} or \ccc{PCA_dimension_3_tag}. Also, it shouldn't be of a greater dimension than the geometry of the object. For example, a \ccc{Triangle} cannot have a \ccc{PCA_dimension_3_tag} tag.
\end{enumerate}

\end{ccRefFunction}
