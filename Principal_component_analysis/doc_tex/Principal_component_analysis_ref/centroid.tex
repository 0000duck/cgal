% +------------------------------------------------------------------------+
% | Reference manual page: centroid.tex
% +------------------------------------------------------------------------+
% | 
% | March 2005  Pierre Alliez and Sylvain Pion and Ankit Gupta
% | Package:   Principal Component Analysis
% |
% +------------------------------------------------------------------------+

\begin{ccRefFunction}{centroid}  
%% add template arg's if necessary

\ccDefinition
  
The function \ccRefName\ computes the centroid (center of mass) of a set
of 2D or 3D objects. These include Points, Segments, Triangles, Rectangles and Circles in 2D and in addition, Cuboids, Spheres and Tetrahedrons in 3D.

\ccInclude{CGAL/centroid.h}

There is a set of overloaded \ccc{centroid} functions for 2D and 3D objects.
The user can also optionally pass an explicit kernel, in case the default,
based on \ccc{Kernel_traits} is not sufficient.
The dimension is also deduced automatically. The user also has to pass a \ccc{tag}
specifying the type of manifold of the objects.

\ccFunction{template < typename InputIterator, typename Tag >
            K::Point_2
            centroid(InputIterator first, InputIterator beyond, const Tag& t);}
{ computes the centroid of a non-empty set of 2D objects having a manifold described by t.
  \ccc{K} is \ccc{Kernel_traits<std::iterator_traits<InputIterator>::value_type>::Kernel}.  The value type must be \ccc{K::Point_2}.
\ccPrecond{first != beyond.} }

\ccFunction{template < typename InputIterator, typename K, typename Tag >
            K::Point_2
            centroid(InputIterator first, InputIterator beyond, const K & k, const Tag& t);}
{ computes the centroid of a non-empty set of 2D objects having a manifold described by t.
  The value type must be \ccc{K::Point_2}.
\ccPrecond{first != beyond.} }

\ccFunction{template < typename InputIterator, typename Tag >
            K::Point_3
            centroid(InputIterator first, InputIterator beyond, const Tag& t);}
{ computes the centroid of a non-empty set of 3D points having a manifold described by t.
  \ccc{K} is \ccc{Kernel_traits<std::iterator_traits<InputIterator>::value_type>::Kernel}.
  The value type must be \ccc{K::Point_3}.
\ccPrecond{first != beyond.} }

\ccFunction{template < typename InputIterator, typename K, typename Tag >
            K::Point_3
            centroid(InputIterator first, InputIterator beyond, const K & k, const Tag& t);}
{ computes the centroid of a non-empty set of 3D points having a manifold described by t.
  The value type must be \ccc{K::Point_3}.
\ccPrecond{first != beyond.} }


\ccSeeAlso
\ccRefIdfierPage{CGAL::barycenter}

\end{ccRefFunction}
