\ccUserChapter{Principal Component Analysis\label{chap:pca}}

\ccChapterAuthor{Pierre Alliez, Sylvain Pion and Ankit Gupta}

\minitoc

% +------------------------------------------------------------------------+
% | Reference manual page: Convex_hull_d_ref/intro.tex
% +------------------------------------------------------------------------+

%\clearpage
%\section{Reference Pages for dD Convex Hulls and Delaunay Triangulations}
\ccRefChapter{dD Convex Hulls and Delaunay Triangulations\label{chap:convex_hull_d_ref}}
\ccChapterAuthor{Susan Hert \and Michael Seel}

A subset $S \subseteq \R^3$ is convex if for any two points $p$ and $q$
in the set the line segment with endpoints $p$ and $q$ is contained
in $S$. The convex hull\ccIndexMainItemDef{convex hull} of a set $S$ is 
the smallest convex set containing
$S$. The convex hull of a set of points $P$ is a convex 
polytope with vertices in $P$.  A point in $P$ is an extreme point 
(with respect to $P$)\ccIndexMainItemDef{extreme point} if it is a vertex 
of the convex hull of $P$.

\cgal\ provides functions for computing convex hulls in two, three 
and arbitrary dimensions as well as functions for testing if a given set of 
points in is strongly convex or not.  This chapter describes the class
available for arbitrary dimensions and its companion class for 
computing the nearest and furthest side Delaunay triangulation. 

\section{Classified Reference Pages}

\ccHeading{Concepts}

\ccRefConceptPage{ConvexHullTraits_d} \\
\ccRefConceptPage{DelaunayLiftedTraits_d} \\
\ccRefConceptPage{DelaunayTraits_d} \\

\ccHeading{Classes}

\ccRefIdfierPage{CGAL::Convex_hull_d_traits_3<R>} \\
\ccRefIdfierPage{CGAL::Convex_hull_d<R>}  \\
\ccRefIdfierPage{CGAL::Delaunay_d< R, Lifted_R >} 

\clearpage




\begin{ccPkgDescription}{3D Convex Hulls\label{Pkg:ConvexHull3}}
\ccPkgHowToCiteCgal{cgal:hs-ch3-07}
\ccPkgSummary{This package provides functions 
for computing convex hulls in three dimensions as well as functions
for checking if sets of points are strongly convex are not. One can
compute the convex hull of a set of points in three dimensions in one
of three ways: using a static algorithm, using an incremental
construction algorithm, or using a triangulation to get a fully
dynamic computation.}

\ccPkgDependsOn{All algorithms produce as output a \ccRef[3D Polyhedron]{Pkg:Polyhedron}. 
                The dynamic algorithms depend on \ccRef[3D Triangulations]{Pkg:Triangulation3}}
\ccPkgIntroducedInCGAL{1.1}
\ccPkgLicense{\ccLicenseQPL}
\ccPkgIllustration{Convex_hull_3/bunny.png}{Convex_hull_3/bunny.png}
\end{ccPkgDescription}


\section{Examples\label{subsec:pca_examples}}

\subsection{Bounding Box}
In the following example we use \stl\ containers of 2D points, 3D points and 3D segments, and compute their axis-aligned bounding box. The kernel from which the input points originate is automatically deduced by the function.
\ccIncludeExampleCode{Principal_component_analysis/bounding_box.cpp}

\subsection{Centroid}
In the following example we use \stl\ containers of 2D points, 3D points and 3D triangles, and compute their centroid. The kernel from which the input points originate is automatically deduced by the function. The dimension tag indicates the dimension of the objects being fitted (0 for points, and 2 for triangles). Note that it is also possible to compute the centroid of the triangle edges (3D segments) by specifying 1 as dimension tag, and of the triangle vertices (3D points) by specifying 0 as dimension tag.
\ccIncludeExampleCode{Principal_component_analysis/centroid.cpp}

\subsection{Barycenter of a Set of Weighted Points}
In the following example we use \stl\ containers of 2D and 3D weighted points, and compute their barycenter. The kernel from which the input points originate is automatically deduced by the function.
\ccIncludeExampleCode{Principal_component_analysis/barycenter.cpp}

\subsection{Least Squares Fitting of a Line to a 2D Point Set}
In the following example we use an \stl\ container of 2D points, and compute the best fitting line in the least squares sense. The kernel from which the input points originate is automatically deduced by the function.
\ccIncludeExampleCode{Principal_component_analysis/linear_least_squares_fitting_points_2.cpp}

\subsection{Least Squares Fitting of a Line and a Plane to a 3D Triangle Set}
In the following example we use an \stl\ container of 3D triangles, and compute the best fitting line and plane in the least squares sense. The kernel from which the input points originate is automatically deduced by the function. We fit either the whole triangles, the triangle edges or the triangle vertices.
\ccIncludeExampleCode{Principal_component_analysis/linear_least_squares_fitting_triangles_3.cpp}

