% +------------------------------------------------------------------------+
% | Reference manual page: Arr_vertex.tex
% +------------------------------------------------------------------------+
% | 
% | Package: Arrangement_2
% | 
% +------------------------------------------------------------------------+

\ccRefPageBegin

\begin{ccRefClass}[Arrangement_2<Traits,Dcel>::]{Vertex}
\ccRefLabel{arr_ref:vertex}

\ccDefinition
An object $v$ of the class \ccRefName\ represents an arrangement vertex,
that is --- a $0$-dimensional cell, associated with a point on the plane. 

\ccInheritsFrom
    \ccHtmlNoLinksFrom{\ccc{typename Dcel::Vertex}}

\ccThree{typename Traits::Point_2}{v.is_isolated()~}{}
\ccThreeToTwo

\ccCreation
\ccCreationVariable{v}
%=====================

\ccConstructor{Vertex();}
    {default constructor.}    

\ccAccessFunctions
%=================

All non-const methods listed below also have \ccc{const} counterparts
that return constant handles, iterators or circulators:

\ccMethod{bool is_at_infinity() const;}
    {checks whether the vertex lies at infinity and not associated with
     a point with bounded coordinates.}

\ccMethod{bool is_isolated() const;}
    {checks whether the vertex is isolated (i.e., has no incident edges).}

\ccMethod{typename Dcel::Size degree() const;}
    {returns the number of edges incident to \ccVar{}.}

\ccThree{Halfedge_around_vertex_circulator}{}{\hspace*{\ccwComment}}
\ccThreeToTwo

\ccMethod{Halfedge_around_vertex_circulator incident_halfedges();}
    {returns a circulator circulator that allows going over the halfedges
     incident to \ccVar{} (that have \ccVar{} as their target).
     The edges are traversed in a clockwise direction around \ccVar{}.
     \ccPrecond{\ccVar{} is {\em not} an isolated vertex.}}

\ccThree{typename Traits::Point_2}{v.is_isolated()~}{}
\ccThreeToTwo

\ccMethod{Face_handle face();}
    {returns a handle to the face that contains \ccVar{} in its interior.
     \ccPrecond{\ccVar{} is an isolated vertex.}}

\ccMethod{Boundary_type boundary_in_x () const;}
    {returns whether the $x$-coordinate of the vertex is finite or whether
     it lies at $\pm\infty$.}

\ccMethod{Boundary_type boundary_in_y () const;}
    {returns whether the $y$-coordinate of the vertex is finite or whether
     it lies at $\pm\infty$.}

\ccMethod{const typename Traits::Point_2& point() const;}
    {returns the point associated with the vertex.
     \ccPrecond{\ccVar{} is not a vertex at infinity.}}

\end{ccRefClass}

\ccRefPageEnd
