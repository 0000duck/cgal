% +------------------------------------------------------------------------+
% | Reference manual page: Arr_extended_vertex.tex
% +------------------------------------------------------------------------+
% | 
% | Package: Arrangement_2
% | 
% +------------------------------------------------------------------------+

\ccRefPageBegin

\begin{ccRefClass}{Arr_extended_vertex<VertexBase,VData>}

\ccThree{VData &}{v.set_data(const VData & data)~}{}
\ccThreeToTwo

\ccDefinition
%============
  The \ccRefName\ class-template extends the vertex
  topological-features of the \dcel. It is parameterized by a
  vertex base-type \ccc{VertexBase} and a data type \ccc{VData} used to extend
  the vertex base-type.

\ccInclude{CGAL/Arr_extended_dcel.h}

\ccIsModel
%============
  \ccc{ArrangementDcelVertex}

\ccInheritsFrom
%============
  \ccc{VertexBase}

\ccCreation
\ccCreationVariable{v}
%==========
  \ccMethod{void assign (const Self & other);}
  {assigns \ccVar{} with the contents of the \ccc{other} vertex.}

\ccAccessFunctions
%=================

\def\ccLongParamLayout{\ccTrue}
  \ccMethod{const VData & data () const;}
  {obtains the auxiliary data (a non-const version, returning a reference
   to a mutable data object is also available).}
\def\ccLongParamLayout{\ccFalse}

\ccModifiers
%============
  \ccMethod{void set_data (const VData & data);}
  {sets the auxiliary data.}

\ccSeeAlso
%============
  \ccc{Arr_dcel_base<V,H,F>}\lcTex{
  (\ccRefPage{CGAL::Arr_dcel_base<V,H,F>})}

\end{ccRefClass}

\ccRefPageEnd
