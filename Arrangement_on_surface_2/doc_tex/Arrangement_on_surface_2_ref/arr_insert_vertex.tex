\ccRefPageBegin

\begin{ccRefFunction}{insert_point}

\ccDefinition

The function \ccRefName{} inserts a given point into a given arrangement.
It uses a given point-location object to locate the given
point in the given arrangement. If the point conincides with an existing
vertex, there is nothing left to do; if it lies on an edge, the edge is
split at the point. Otherwise, the point is contained inside a face, and is
inserted as an isolated vertex inside this face.
By default, the function uses the ``walk along line'' point-location
strategy --- namely, an instance of the class
\ccc{Arr_walk_along_line_point_location<Arrangement_2<Traits,Dcel> >}.
In either case, the function returns a handle for the vertex associated
with the point.

\ccInclude{CGAL/Arrangement_2.h}

\ccGlobalFunction{template<typename Traits, typename Dcel,
                           typename PointLocation>
                       typename Arrangement_2<Traits,Dcel>::Vertex_handle
                       insert_point (Arrangement_2<Traits,Dcel>& arr,
                                         const typename Traits::Point_2& p,
                                         const PointLocation& pl = walk_pl);}
\ccPrecond{If provided, \ccc{pl} must be attached to the given arrangement
\ccc{arr}.}

\ccRequirements
\begin{itemize}
\item The instantiated \ccc{Traits} class must model the
  \ccc{ArrangementXMonotoneTraits_2} concept. Not all expressions listed 
  by this concept are required. In fact the traits class must model the 
  \ccc{ArrangementBasicTraits_2} concept, and support the splitting
  functionality.
\item The point-location object \ccc{pl}, must model the
  \ccc{ArrangementPointLocation_2} concept.
\end{itemize}

\end{ccRefFunction}

\ccRefPageEnd
