\RCSdef{\ArrangementOnSurfaceRefRev}{$Id$}
\RCSdefDate{\ArrangementOnSurfaceRefDate}{$Date$}
\ccRefChapter{2D Arrangements\label{chapterArrangement_on_surface_2_ref}}
\ccChapterRelease{\ArrangementOnSurfaceRefRev. \ \ArrangementOnSurfaceRefDate}
\ccChapterAuthor{Ron Wein, Efi Fogel, Baruch Zukerman, Dan Halperin, Eric Berberich, and Oren Zalzman}

% ===============================================================
%\section*{Introduction}
%\label{arr_ref_sec:intro}
% =======================
Given a set $\mathcal{C}$ of planar curves, the {\em arrangement}
${\mathcal A}({\mathcal C})$ is the subdivision of the plane induced
by the curves in $\mathcal{C}$ into maximally connected cells. The cells
can be $0$-dimensional ({\em vertices}), $1$-dimensional ({\em edges})
or $2$-dimensional ({\em faces}).

The class \ccc{Arrangement_2<Traits,Dcel>} encapsulates a data structure
that maintains arrangements of arbitrary bounded planar curves. It comes
with a variety of algorithms that operate on planar arrangement,
such as point-location queries and overlay computations, which are
implemented as peripheral classes or as free (global) functions.

\section{Classified Reference Pages}

\subsection*{Enumerations}

\ccRefIdfierPage{CGAL::Arr_parameter_space}\\
\ccRefIdfierPage{CGAL::Arr_curve_end}\\
\ccRefIdfierPage{CGAL::Arr_halfedge_direction}

\subsection*{Tags}

\ccRefIdfierPage{CGAL::Arr_oblivious_side_tag}\\
\ccRefIdfierPage{CGAL::Arr_open_side_tag}

\subsection*{Concepts}

\ccRefConceptPage{ArrangementDcel}\\
\ccRefConceptPage{ArrangementDcelWithRebind}\\
\ccRefConceptPage{ArrangementDcelVertex}\\
\ccRefConceptPage{ArrangementDcelHalfedge}\\
\ccRefConceptPage{ArrangementDcelFace}\\
\ccRefConceptPage{ArrangementDcelHole}\\
\ccRefConceptPage{ArrangementDcelIsolatedVertex}\\
~\\
\ccRefConceptPage{ArrangementBasicTraits_2}\\
\ccRefConceptPage{ArrangementLandmarkTraits_2}\\
\ccRefConceptPage{ArrangementXMonotoneTraits_2}\\
\ccRefConceptPage{ArrangementTraits_2}\\
\ccRefConceptPage{ArrangementOpenBoundaryTraits_2}\\
~\\
\ccRefConceptPage{ArrangementInputFormatter}\\
\ccRefConceptPage{ArrangementOutputFormatter} \\
\ccRefConceptPage{ArrangementWithHistoryInputFormatter}\\
\ccRefConceptPage{ArrangementWithHistoryOutputFormatter}
~\\
\ccRefConceptPage{ArrangementPointLocation_2}\\
\ccRefConceptPage{ArrangementVerticalRayShoot_2}

\subsection*{Geometric Object Concepts}

\ccRefConceptPage{ArrTraits::Point_2}\\
\ccRefConceptPage{ArrTraits::XMonotoneCurve_2}\\
\ccRefConceptPage{ArrTraits::Curve_2}

\subsection*{Function Object Concepts}

\ccRefConceptPage{ArrTraits::CompareX_2}\\
\ccRefConceptPage{ArrTraits::CompareXy_2}\\
\ccRefConceptPage{ArrTraits::ConstructMinVertex_2}\\
\ccRefConceptPage{ArrTraits::ConstructMaxVertex_2}\\
\ccRefConceptPage{ArrTraits::IsVertical_2}\\
\ccRefConceptPage{ArrTraits::CompareYAtX_2}\\
\ccRefConceptPage{ArrTraits::CompareYAtXLeft_2}\\
\ccRefConceptPage{ArrTraits::CompareYAtXRight_2}\\
\ccRefConceptPage{ArrTraits::Equal_2}\\
\ccRefConceptPage{ArrTraits::ParameterSpaceInX_2}\\
\ccRefConceptPage{ArrTraits::ParameterSpaceInY_2}\\
\ccRefConceptPage{ArrTraits::CompareXAtLimit_2}\\
\ccRefConceptPage{ArrTraits::CompareXNearLimit_2}\\
\ccRefConceptPage{ArrTraits::CompareYNearBoundary_2}\\
% \ccRefConceptPage{ArrTraits::IsOnXIdentification_2}\\
% \ccRefConceptPage{ArrTraits::IsOnYIdentification_2}\\
% \ccRefConceptPage{ArrTraits::CompareXOnBoundary_2}\\
% \ccRefConceptPage{ArrTraits::CompareYOnBoundary_2}\\
\ccRefConceptPage{ArrTraits::Intersect_2}\\
\ccRefConceptPage{ArrTraits::Split_2}\\
\ccRefConceptPage{ArrTraits::AreMergeable_2}\\
\ccRefConceptPage{ArrTraits::Merge_2}\\
\ccRefConceptPage{ArrTraits::MakeXMonotone_2}\\
\ccRefConceptPage{ArrTraits::Approximate_2}\\
\ccRefConceptPage{ArrTraits::ConstructXMonotoneCurve_2}


\subsection*{Classes}

\ccRefIdfierPage{CGAL::Arrangement_2<Traits,Dcel>}\\
\ccRefIdfierPage{CGAL::Arr_accessor<Arrangement>}\\
\ccRefIdfierPage{CGAL::Arr_observer<Arrangement>}\\
\ccRefIdfierPage{CGAL::Arrangement_with_history_2<Traits,Dcel>} \\
~\\
\ccRefIdfierPage{CGAL::Arrangement_2<Traits,Dcel>::Vertex}\\
\ccRefIdfierPage{CGAL::Arrangement_2<Traits,Dcel>::Halfedge}\\
\ccRefIdfierPage{CGAL::Arrangement_2<Traits,Dcel>::Face}\\
\ccRefIdfierPage{CGAL::Arr_dcel_base<V,H,F>}\\
\ccRefIdfierPage{CGAL::Arr_default_dcel<Traits>}\\
\ccRefIdfierPage{CGAL::Arr_face_extended_dcel<Traits,FData,V,H,F>}\\
\ccRefIdfierPage{CGAL::Arr_extended_dcel<Traits,VData,HData,FData,V,H,F>}\\
~\\
\ccRefIdfierPage{CGAL::Arr_segment_traits_2<Kernel>}\\
\ccRefIdfierPage{CGAL::Arr_non_caching_segment_traits_2<Kernel>}\\
\ccRefIdfierPage{CGAL::Arr_linear_traits_2<Kernel>}\\
\ccRefIdfierPage{CGAL::Arr_polyline_traits_2<SegmentTraits>}\\
\ccRefIdfierPage{CGAL::Arr_circle_segment_traits_2<Kernel>}\\
\ccRefIdfierPage{CGAL::Arr_line_arc_traits_2<CircularKernel>}\\
\ccRefIdfierPage{CGAL::Arr_circular_arc_traits_2<CircularKernel>}\\
\ccRefIdfierPage{CGAL::Arr_circular_line_arc_traits_2<CircularKernel>}\\
\ccRefIdfierPage{CGAL::Arr_conic_traits_2<RatKernel,AlgKernel,NtTraits>}\\
\ccRefIdfierPage{CGAL::Arr_rational_function_traits_2<AlgebraicKernel_d_1>}\\
\ccRefIdfierPage{CGAL::Arr_Bezier_curve_traits_2<RatKernel,AlgKernel,NtTraits>}\\
\ccRefIdfierPage{CGAL::Arr_algebraic_segment_traits_2<Coefficient>}\\
\ccRefIdfierPage{CGAL::Arr_curve_data_traits_2<Tr,XData,Mrg,CData,Cnv>}\\
\ccRefIdfierPage{CGAL::Arr_consolidated_curve_data_traits_2<Traits,Data>}\\
~\\
\ccRefIdfierPage{CGAL::Arr_text_formatter<Arrangement>}\\
\ccRefIdfierPage{CGAL::Arr_face_extended_text_formatter<Arrangement>}\\
\ccRefIdfierPage{CGAL::Arr_extended_dcel_text_formatter<Arrangement>}\\
\ccRefIdfierPage{CGAL::Arr_with_history_text_formatter<ArrFormatter>}\\
~\\
\ccRefIdfierPage{CGAL::Arr_naive_point_location<Arrangement>}\\
\ccRefIdfierPage{CGAL::Arr_walk_along_line_point_location<Arrangement>}\\
\ccRefIdfierPage{CGAL::Arr_trapezoid_ric_point_location<Arrangement>}\\
\ccRefIdfierPage{CGAL::Arr_landmarks_point_location<Arrangement,Generator>}\\
~\\
\ccRefIdfierPage{CGAL::Arr_vertex_index_map<Arrangement>}\\
\ccRefIdfierPage{CGAL::Arr_face_index_map<Arrangement>}
\ccRefIdfierPage{CGAL::Arr_point_location_result<Arrangement>}\\

\subsection*{Tags}

\ccRefIdfierPage{CGAL::Arr_oblivious_side_tag}\\
\ccRefIdfierPage{CGAL::Arr_open_side_tag}
% \ccRefIdfierPage{CGAL::Arr_closed_side_tag}\\
% \ccRefIdfierPage{CGAL::Arr_contracted_side_tag}\\
% \ccRefIdfierPage{CGAL::Arr_identified_side_tag}

\subsection*{Functions}

\ccRefIdfierPage{CGAL::is_valid}\\
\ccRefIdfierPage{CGAL::insert}\\
\ccRefIdfierPage{CGAL::insert_non_intersecting_curve}\\
\ccRefIdfierPage{CGAL::insert_non_intersecting_curves}\\
\ccRefIdfierPage{CGAL::insert_point}\\
\ccRefIdfierPage{CGAL::remove_edge}\\
\ccRefIdfierPage{CGAL::remove_vertex}\\
\ccRefIdfierPage{CGAL::locate}\\
\ccRefIdfierPage{CGAL::decompose}\\
\ccRefIdfierPage{CGAL::overlay} \\
\ccRefIdfierPage{CGAL::read} \\
\ccRefIdfierPage{CGAL::write} \\
\ccRefIdfierPage{CGAL::remove_curve} \\
\ccc{CGAL::operator<<}
  {\lcRawHtml{<A HREF="Function_operator.html">(go there)</A>}
   \lcTex{\dotfill page~\pageref{ref_arr_operator_leftshift}}}\\
\ccc{CGAL::operator>>}
  {\lcRawHtml{<A HREF="Function_operator--.html">(go there)</A>}
   \lcTex{\dotfill page~\pageref{ref_arr_operator_rightshift}}}

\subsection*{Macros}

\ccRefIdfierPage{CGAL_ARR_POINT_LOCATION_VERSION}
