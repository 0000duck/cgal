\ccRefPageBegin

\begin{ccRefFunction}{insert}

\ccDefinition

The function \ccRefName{} inserts one or more curves or $x$-monotone curves
into a given arrangement, where no restrictions are imposed on the inserted
curves. If an inserted curve is not $x$-monotone curve, it is subdivided
into $x$-monotone subcurves (and perhaps isolated points), which are
inserted into the arrangement.

%%%%

\ccInclude{CGAL/Arrangement_2.h}

\ccGlobalFunction{template<class Traits, class Dcel,
                  class Curve, class PointLocation>
            void insert (Arrangement_2<Traits,Dcel>& arr,
                               const Curve& c,
			       const PointLocation& pl = walk_pl);}
Inserts the given curve \ccc{c} into the arrangement \ccc{arr}.
\ccc{c} is subdivided into $x$-monotone subcurves (and perhaps isolated
points). Each subcurve is in turn inserted into the arrangement by locating
its left endpoint and computing its zone until reaching the right endpoint.

The given point-location object \ccc{pl} is used to locate the left
endpoints of the $x$-monotone curves. By default, the function uses the
``walk along line'' point-location strategy --- namely an instance of
the class \ccc{Arr_walk_along_line_point_location<Arrangement_2<Traits,Dcel> >}.

\ccPrecond{If provided, \ccc{pl} must be attached to the given arrangement
\ccc{arr}.}

\ccGlobalFunction{template<typename Traits, typename Dcel>
            void insert (Arrangement_2<Traits,Dcel>& arr,
                         const typename Traits::X_monotone_curve_2& xc,
                         const Object& obj);}
Inserts the{\em $x$-monotone (only)} curve \ccc{xc} into the
arrangement \ccc{arr}. The object \ccc{obj}, which either
wraps a \ccc{Vertex_const_handle}, a \ccc{Halfedge_const_handle}, or a
\ccc{Face_const_handle}, represents the location of \ccc{xc}'s left
endpoint in the arrangement. The zone of \ccc{xc} is computed starting
from the feature represented by \ccc{obj}. As in the case above, the
zone computation terminates, when the right endpoint is reached.
Thus, point-location is not required. 

%%%%

\ccGlobalFunction{template<class Traits, class Dcel, class InputIterator>
            void insert (Arrangement_2<Traits,Dcel>& arr,
                                InputIterator first, InputIterator last);}
Aggregately inserts the curves or $x$-monotone curves in the range
\ccc{[first,last)} into the arrangement \ccc{arr} using the sweep-line
framework.

%%%%

\ccInclude{CGAL/Arrangement_with_history_2.h}

\ccThree{Orientation~~}{}{\hspace*{8.5cm}}

\ccGlobalFunction{template<typename Traits, typename Dcel,
                           typename PointLocation>
            typename Arrangement_with_history_2<Traits,Dcel>::Curve_handle
            insert (Arrangement_with_history_2<Traits,Dcel>& arr,
                    const typename Traits::Curve_2& c,
                    const PointLocation& pl = walk_pl);}
Inserts the given curve \ccc{c} into the arrangement with history \ccc{arr},
and returns a handle to the inserted curve. \ccc{c} is subdivided into
$x$-monotone subcurves (and perhaps isolated points). Each subcurve is in
turn inserted into the arrangement by locating its left endpoint and
computing its zone until reaching the right endpoint.

The given point-location object \ccc{pl} is used to locate the left
endpoints of the $x$-monotone curves. By default, the function uses the
``walk along line'' point-location strategy --- namely an instance of
the class \ccc{Arr_walk_along_line_point_location<Arrangement_2<Traits,Dcel> >}.

\ccPrecond{If provided, \ccc{pl} is attached to the given arrangement
\ccc{arr}.}

%%%%

\ccGlobalFunction{template<class Traits, class Dcel, InputIterator>
            void insert (Arrangement_with_history_2<Traits,Dcel>& arr,
                         InputIterator first, InputIterator last);}
Aggregately inserts the curves in the range \ccc{[first,last)} into the
arrangement with history \ccc{arr} using the sweep-line framework.

%%%%

\ccRequirements
\begin{itemize}
\item If the curve is $x$-monotone curve then The instantiated
  \ccc{Traits} class must model the \ccc{ArrangementXMonotoneTraits_2}
  concept. In case that the curve is not $x$-monotone then the
  instantiated \ccc{Traits} class must model the
  \ccc{ArrangementTraits_2} concept. That is, it should define the
  \ccc{Curve_2} type, and support its subdivision into $x$-monotone
  subcurves (and perhaps isolated points). 
\item The point-location object \ccc{pl}, must model the
  \ccc{ArrangementPointLocation_2} concept.
\end{itemize}

\end{ccRefFunction}

\ccRefPageEnd
