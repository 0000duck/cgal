\ccRefPageBegin

\begin{ccRefFunction}{insert}

\ccDefinition

The function \ccRefName{} inserts a given curve or $x$-monotone curve
into a given arrangement, where no restrictions are made on the nature
of the inserted curve. If the curve is not $x$-monotone curve then it
subdivides the given curve into $x$-monotone subcurves. Each subcurve
is in turn inserted into the arrangement by locating its left endpoint
and computing its zone until reaching the right endpoint.

A given point-location object is used for answering point-location
queries during the insertion process (locating the left endpoints of
the $x$-monotone curves). By default, the function uses the ``walk
along line'' point-location strategy --- namely an instance of the
class
\ccc{Arr_walk_along_line_point_location<Arrangement_2<Traits,Dcel>
  >}.

%%%%

\ccInclude{CGAL/Arrangement_2.h}

\ccGlobalFunction{template<class Traits, class Dcel,
                  class Curve, class PointLocation>
            void insert (Arrangement_2<Traits,Dcel>& arr,
                               const Curve& c,
			       const PointLocation& pl = walk_pl);}
Inserts the given curve \ccc{c} into the arrangement \ccc{arr}.
\ccPrecond{If provided, \ccc{pl} must be attached to the given arrangement
\ccc{arr}.}

%%%%

\ccInclude{CGAL/Arrangement_with_history_2.h}

\ccThree{Orientation~~}{}{\hspace*{8.5cm}}

\ccGlobalFunction{template<typename Traits, typename Dcel,
                           typename PointLocation>
            typename Arrangement_with_history_2<Traits,Dcel>::Curve_handle
            insert (Arrangement_with_history_2<Traits,Dcel>& arr,
                    const typename Traits::Curve_2& c,
                    const PointLocation& pl = walk_pl);}
Inserts the given curve \ccc{c} into the arrangement with history
\ccc{arr}, and returns a handle to the inserted curve.
\ccPrecond{If provided, \ccc{pl} is attached to the given arrangement
\ccc{arr}.}

\ccGlobalFunction{template<typename Traits, typename Dcel>
            void insert (Arrangement_2<Traits,Dcel>& arr,
                         const typename Traits::X_monotone_curve_2& xc,
                         const Object& obj);}
Inserts the{\em $x$-monotone (only)} curve \ccc{xc} into the
arrangement \ccc{arr}. The object \ccc{obj}, which either
wraps a \ccc{Vertex_const_handle}, a \ccc{Halfedge_const_handle}, or a
\ccc{Face_const_handle}, represents the location of \ccc{xc}'s left
endpoint in the arrangement. The zone of \ccc{xc} is computed strating
from the feature represented by \ccc{obj}. As in the case above, the
zone computation terminates, when the right endpoint is reached.
Thus, point-location is not required. 

\ccRequirements
\begin{itemize}
\item The instantiated \ccc{Traits} class must model the
  \ccc{ArrangementXMonotoneTraits_2} concept.
\item The point-location object \ccc{pl}, must model the
  \ccc{ArrangementPointLocation_2} concept.
\end{itemize}
			

\ccRequirements
\begin{itemize}
\item If the curve is $x$-monotone curve then The instantiated
  \ccc{Traits} class must model the \ccc{ArrangementXMonotoneTraits_2}
  concept. In case that the curve is not $x$-monotone then the
  instantiated \ccc{Traits} class must model the
  \ccc{ArrangementTraits_2} concept. That is, it should define the
  \ccc{Curve_2} type, and support its subdivision into $x$-monotone
  subcurves (and perhaps isolated points). 
\item The point-location object \ccc{pl}, must model the
  \ccc{ArrangementPointLocation_2} concept.
\end{itemize}

\end{ccRefFunction}

\ccRefPageEnd
