% Reference manual page: ArrangementTraits.tex
% Package: Arrangement_2

\ccRefPageBegin
\begin{ccRefConcept}{ArrangementTraits_2}

\ccDefinition
% ===========
The concept \ccRefName{} allows the construction of arrangement
of {\sl general} planar curves. Models of this concept are used
by the free \ccc{insert()} functions of the arrangement package and
by the \ccc{Arrangement_with_history_2} class.

A model of this concept must define the nested \ccc{Curve_2} type, which
represents a general planar curve that is not necessarily $x$-monotone
and is not necessarily connected. Such curves are eventually subdivided into
$x$-monotone subcurves and isolated points (represented by the \ccc{Point_2}
and \ccc{X_monotone_curve_2} types, defined in the basic traits concept).

A model of the concept \ccRefName{} that handles arbitrary curves,
which are always $x$-monotone, such as a traits class that handles
linear curves may define the nested types \ccc{Curve\_2} and
\ccc{X_monotone_curve\_2} to be of equivalent types. Moreover, defining
them as of equivalent types is advantageous, as it enables a generic
simple implementation of the nested Functor
\ccc{Make_x_monotone_2}.

On the other hand, a model of the \ccRefName{} concept that handles
arbitrary curves, which may be not $x$-monotone must define the
\ccc{Curve\_2} and \ccc{X_monotone_curve\_2} nested types to be of
different types to allow proper dispatching of the free functions
that accept such curves, such as \ccc{intsert()}.

\ccRefines
\ccc{ArrangementXMonotoneTraits_2}

\ccTypes
% ======
\ccNestedType{Curve_2}{models the concept \ccc{ArrTraits::Curve_2}.}

\ccHeading{Functor Types}
% =======================
\ccThree{Make_x_monotone_2}{}{\hspace*{14cm}}
\ccThreeToTwo
\ccNestedType{Make_x_monotone_2}%
  {models the concept \ccc{ArrTraits::MakeXMonotone_2}.}

\ccCreationVariable{traits}
% \ccCreation
% ===========

\ccHeading{Accessing Functor Objects}
% ===================================
\ccMethod{Make_x_monotone_2 make_x_monotone_2_object() const;} {}

\ccHasModels
% ==========
\ccc{CGAL::Arr_segment_traits_2<Kernel>}\\
\ccc{CGAL::Arr_non_caching_segment_traits_2<Kernel>}\\
\ccc{CGAL::Arr_linear_traits_2<Kernel>}\\
\ccc{CGAL::Arr_polyline_traits_2<SegmentTraits>}\\
\ccc{CGAL::Arr_circle_segment_traits_2<Kernel>}\\
\ccc{CGAL::Arr_line_arc_traits_2<CircularKernel>}\\
\ccc{CGAL::Arr_circular_arc_traits_2<CircularKernel>}\\
\ccc{CGAL::Arr_circular_line_arc_traits_2<CircularKernel>}\\
\ccc{CGAL::Arr_conic_traits_2<RatKernel,AlgKernel,NtTraits>}\\
\ccc{CGAL::Arr_rational_function_traits_2<AlgebraicKernel_d_1>}\\
\ccc{CGAL::Arr_Bezier_curve_traits_2<RatKernel,AlgKernel,NtTraits>}\\
\ccc{CGAL::Arr_algebraic_segment_traits_2<Coefficient>}\\
\ccc{CGAL::Arr_curve_data_traits_2<Tr,XData,Mrg,CData,Cnv>}\\
\ccc{CGAL::Arr_consolidated_curve_data_traits_2<Traits,Data>}

\ccSeeAlso
% ========
\ccc{ArrangementBasicTraits_2}\lcTex{
  (\ccRefPage{ArrangementBasicTraits_2})}\\
\ccc{ArrangementXMonotoneTraits_2}\lcTex{
  (\ccRefPage{ArrangementXMonotoneTraits_2})}\\
\ccc{ArrangementLandmarkTraits_2}\lcTex{
  (\ccRefPage{ArrangementLandmarkTraits_2})}

\end{ccRefConcept}
\ccRefPageEnd

%%%%%%%% Geometric Objects %%%%%%%%
% =================================

%%%%%%%% Curve_2
% ==============
\ccRefPageBegin
\begin{ccRefConcept}{ArrTraits::Curve_2}
\ccDefinition
represents a general planar curve.

\ccRefines{DefaultConstructible, CopyConstructible, Assignable}

\ccHasModels\ccc{ArrangementTraits_2::Curve_2}
\end{ccRefConcept}
\ccRefPageEnd

%%%%%%%% Functors %%%%%%%%
% ========================

%%%%%%%% MakeXMonotone_2
\ccRefPageBegin
\begin{ccRefConcept}{ArrTraits::MakeXMonotone_2}
\ccRefines{Functor}

\ccHasModels\ccc{ArrangementTraits_2::Make_x_monotone_2}

\ccCreationVariable{fo}

\ccMethod{template <typename OutputIterator>
          OutputIterator operator()( ArrTraits::Curve_2 c,
	                             OutputIterator oi);}{%
  subdivides the input curve \ccc{c} into $x$-monotone subcurves and
  isolated points, and inserts the results into a container through the
  given output iterator. The value type of \ccc{OutputIterator} is
  \ccc{CGAL::Object}, where each \ccc{Object} wraps either an
  \ccc{ArrTraits::X_monotone_curve_2} object or a \ccc{ArrTraits::Point_2}
  object. The operator returns a past-the-end iterator for the output
  sequence.}
\end{ccRefConcept}
\ccRefPageEnd
