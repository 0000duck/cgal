% +------------------------------------------------------------------------+
% | Reference manual page: Arr_extended_halfedge.tex
% +------------------------------------------------------------------------+
% | 
% | Package: Arrangement_2
% | 
% +------------------------------------------------------------------------+

\ccRefPageBegin

\begin{ccRefClass}{Arr_extended_halfedge<HalfedgeBase,HData>}

\ccThree{HData &}{he.set_data(const HData & data)~}{}
\ccThreeToTwo

\ccDefinition
%============
  The \ccRefName\ class-template extends the halfedge topological-features of
  the \dcel. It is parameterized by a halfedge base-type \ccc{HalfedgeBase}
  and a data type \ccc{HData} used to extend the halfedge base-type.

\ccInclude{CGAL/Arr_extended_dcel.h}

\ccIsModel
%============
  \ccc{ArrangementDcelHalfedge}

\ccInheritsFrom
%============
  \ccc{HalfedgeBase}

\ccCreation
\ccCreationVariable{he}
%==========
  \ccMethod{void assign (const Self & other);}
  {assigns \ccVar{} with the contents of the \ccc{other} vertex.}

\ccAccessFunctions
%=================

\def\ccLongParamLayout{\ccTrue}
  \ccMethod{const HData & data () const;}
  {obtains the auxiliary data (a non-const version, returning a reference
   to a mutable data object is also available).}
\def\ccLongParamLayout{\ccFalse}

\ccModifiers
%============
  \ccMethod{void set_data (const HData & data);}
  {sets the auxiliary data.}

\ccSeeAlso
%============
  \ccc{Arr_dcel_base<V,H,F>}\lcTex{
  (\ccRefPage{CGAL::Arr_dcel_base<V,H,F>})}

\end{ccRefClass}

\ccRefPageEnd
