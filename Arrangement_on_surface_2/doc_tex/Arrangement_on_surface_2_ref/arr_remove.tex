\ccRefPageBegin

\begin{ccRefFunction}{remove_edge}

\ccDefinition

The function \ccRefName{} removes an edge given by one of the twin halfedges
that forms it, from a given arrangement. Once the edge is removed, if the
vertices associated with its endpoints become isolated, they are removed as
well. The call \ccc{remove_edge(arr, e)} is equivalent to the call
\ccc{arr.remove_edge (e, true, true)}. However, this free function requires
that \ccc{Traits} be a model of the refined concept
\ccc{ArrangementXMonotoneTraits_2}, which requires merge operations
on $x$-monotone curves. If one of the end-vertices of the given edge
becomes redundant after the edge is removed (see \ccc{remove_vertex()}
for the definition of a redundant vertex), it is removed, and its
incident edges are merged.
If the edge-removal operation causes two faces to merge, the merged face
is returned. Otherwise, the face to which the edge was incident before the
removal is returned.

\ccInclude{CGAL/Arrangement_2.h}

\ccGlobalFunction{template<typename Traits, typename Dcel>
        typename Arrangement_2<Traits,Dcel>::Face_handle
        remove_edge (Arrangement_2<Traits,Dcel>& arr,
                     typename Arrangement_2<Traits,Dcel>::Halfedge_handle e);}

\ccRequirements
\begin{itemize}
\item The instantiated traits class must model the concept
  \ccc{ArrangementXMonotoneTraits_2}. 
\end{itemize}

\end{ccRefFunction}

\ccRefPageEnd
