% +------------------------------------------------------------------------+
% | Reference manual page: Arr_vertex_index_map.tex
% +------------------------------------------------------------------------+
% | 
% | Package: Arrangement_2
% | 
% +------------------------------------------------------------------------+

\ccRefPageBegin

\begin{ccRefClass}{Arr_vertex_index_map<Arrangement>}
\label{arr_ref:arr_vertex_index_map}

\ccDefinition
%============

\ccClassTemplateName{} maintains a mapping of vertex handles of an
attached arrangement object to indices (of type \ccc{unsigned int}).
This class template is a model of the \boost{} concept
\ccc{ReadablePropertyMap}. A mapping between vertex handles and indices
enables convenient usage of property-map classes supplied by \boost{}.
For example, the property-map class templates
\ccc{boost::vector_property_map}, which is based on \ccc{std::vector},
and \ccc{boost::iterator_property_map}, which can be used to implement
a property map based on a native C\hbox{\tt ++} array, require the
user to supply a mapping such as \ccClassTemplateName{}.

As new vertices might be inserted into the attached arrangement, and
existing vertices might be removed, the notification mechanism is used
to dynamically maintain the mapping of vertex handles to indices.

\ccRefines{DefaultConstructible, CopyConstructible, Assignable}

\ccIsModel
  \ccc{boost::ReadablePropertyMap}
  
\ccInheritsFrom\ccc{Arr_observer<Arrangement>}

\ccInclude{CGAL/Arr_vertex_index_map.h}

\ccTypes
%=======

\ccNestedType{Arrangement_2}{the type of the attached arrangement.}

\ccNestedType{category}{\ccc{boost::readable_property_map_tag}}
\ccGlue
\ccNestedType{value_type}{\ccc{unsigned int}}
\ccGlue
\ccNestedType{reference}{\ccc{unsigned int}}
\ccGlue
\ccNestedType{key_type}{V\ccc{ertex_handle}}

\ccTypedef{typedef typename Arrangement_2::Vertex_handle Vertex_handle;}
    {The vertex handle type.}

\ccTypedef{typedef Unique_hash_map<Vertex_handle, value_type> Index_map;}
    {The type of mapping of vertices to indices.}

\ccCreation
\ccCreationVariable{vertex_index_map}
%===================================

\ccConstructor{Arr_vertex_index_map();} 
    {constructs a map that is unattached to any arrangement instance.}
 
\ccConstructor{Arr_vertex_index_map(Arrangement_2& arr);} 
    {constructs a map and attaches it to the given arrangement \ccc{arr}.}

\ccSeeAlso
%=========
\ccc{Arr_observer<Arrangement>}\lcTex{ 
     (\ccRefPage{Arr_observer<Arrangement>})}\\
\ccc{Arr_face_index_map<Arrangement>}\lcTex{ 
     (\ccRefPage{Arr_face_index_map<Arrangement>})}
  
\end{ccRefClass}

\ccRefPageEnd
