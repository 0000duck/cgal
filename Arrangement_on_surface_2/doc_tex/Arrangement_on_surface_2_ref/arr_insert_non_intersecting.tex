\ccRefPageBegin

\begin{ccRefFunction}{insert_non_intersecting_curve}

\ccDefinition

The function \ccRefName{} inserts a given $x$-monotone curve into a given
arrangement, where the interior of the given curve is disjoint from all
existing arrangement vertices and edges. Under this assumption, it is
possible to locate the endpoints of the given curve in the arrangement,
and use one of the specialized insertion member-functions of the
arrangement according to the results. The insertion operations creates a
single new edge, that is, two twin halfedges, and the function returns a
handle for the one directed lexicographically in increasing order (from
left to right).

A given point-location object is used for answering the two point-location
queries on the given curve endpoints. By default, the function uses the
``walk along line'' point-location strategy --- namely, an instance of the
class \ccc{Arr_walk_along_line_point_location<Arrangement_2<Traits,Dcel> >}.

\ccInclude{CGAL/Arrangement_2.h}

\ccGlobalFunction{template<typename Traits, typename Dcel,
                           typename PointLocation>
            typename Arrangement_2<Traits,Dcel>::Halfedge_handle 
            insert_non_intersecting_curve (Arrangement_2<Traits,Dcel>& arr,
                         const typename Traits::X_monotone_curve_2& xc,
                         const PointLocation& pl = walk_pl);}
\ccPrecond{If provided, \ccc{pl} must be attached to the given arrangement
\ccc{arr}.}

\ccRequirements
\begin{itemize}
\item The instantiated \ccc{Traits} class must model the restricted
  \ccc{ArrangementBasicTraits_2} concept, as no intersections are computed.
\item The point-location object \ccc{pl} must model the
  \ccc{ArrangementPointLocation_2} concept. 
\end{itemize}

\end{ccRefFunction}

\ccRefPageEnd
