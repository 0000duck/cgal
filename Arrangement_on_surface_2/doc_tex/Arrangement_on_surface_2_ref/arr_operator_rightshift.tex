\ccHtmlNoClassLinks
\begin{ccRefFunction}{operator>>}
\label{ref_arr_operator_rightshift}

\ccDefinition

%%%%

Extracts an arrangement from a given input stream using the
input format defined by the \ccc{Arr_text_formatter} class --- that is,
only the basic geometric and topological features of the arrangement are
read and no auxiliary data is attached to the \dcel\ features.

\ccInclude{CGAL/IO/Arr_iostream.h}

\ccGlobalFunction{template<class Traits, class Dcel>
            std::istream& operator>> (std::istream& is,
                                      Arrangement_2<Traits,Dcel>& arr);}
%%%%

Extracts an arrangement-with-history from a given
input stream using the default input format.

\ccInclude{CGAL/IO/Arr_with_history_iostream.h}

\ccGlobalFunction{template<class Traits, class Dcel>
            std::istream& operator>> (std::istream& is,
                                Arrangement_with_history_2<Traits,Dcel>& arr);}

\ccSeeAlso
  \ccc{read}\lcTex{(\ccRefPage{CGAL::read})}\\
  \ccc{write}\lcTex{(\ccRefPage{CGAL::write})}\\
  \lcTex{\ccc{operator>>} page~\pageref{ref_arr_operator_leftshift}}
  \lcRawHtml{<I><A HREF="Function_operator.html">operator&lt;&lt;</A></I>}

\end{ccRefFunction}
\ccRefPageEnd
