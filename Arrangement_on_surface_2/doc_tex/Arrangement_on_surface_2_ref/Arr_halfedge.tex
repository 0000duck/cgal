% +------------------------------------------------------------------------+
% | Reference manual page: Arr_halfedge.tex
% +------------------------------------------------------------------------+
% | 
% | Package: Arrangement_2
% | 
% +------------------------------------------------------------------------+

\ccRefPageBegin

\begin{ccRefClass}[Arrangement_2<Traits,Dcel>::]{Halfedge}
\ccRefLabel{arr_ref:halfedge}

\ccThree{Ccb_halfedge_circulator}{e.direction()~}{}
\ccThreeToTwo

\ccDefinition
An object $e$ of the class \ccRefName\ represents a halfedge in the
arrangement. A halfedge is directed from its \emph{source} vertex
to its \emph{target} vertex, and has an \emph{incident face} lying to
its right. Each halfedge has a \emph{twin} halfedge directed in the
opposite direction, where the pair of twin halfedges form together
an arrangement edge, that is --- a $1$-dimensional cell, associated
with planar $x$-monotone curve.

Halfedges are sotred in doubly-connected lists and form chains. These
chains define the inner and outer boundaries of connceted components.

\ccInheritsFrom
    \ccHtmlNoLinksFrom{\ccc{typename Dcel::Halfedge}}

\ccCreation
\ccCreationVariable{e}
%=====================

\ccConstructor{Halfedge();}
   {default constructor.}    	    
    
\ccAccessFunctions
%=================

All non-const methods listed below also have \ccc{const} counterparts
that return constant handles, iterators or circulators:

\ccMethod{bool is_fictitious () const;}
    {returns whether the halfedge is fictitious (i.e., connects two vertices at
     infinity and is not associated with a valid curve).}
    
\ccMethod{Vertex_handle source();}
    {returns a handle for the source vertex of \ccVar{}.}
\ccGlue
\ccMethod{Vertex_handle target();}
    {returns a handle for the target vertex of \ccVar{}.}

\ccMethod{bool source_at_infinity();}
    {returns whether the source vertex of \ccVar{} lies at infinity.}
\ccGlue
\ccMethod{bool target_at_infinity();}
    {returns whether the target vertex of \ccVar{} lies at infinity.}

\ccMethod{Arr_halfedge_direction direction() const;}
    {returns the direction of the halfedge: \ccc{ARR_LEFT_TO_RIGHT} if
      \ccVar{}'s source vertex is lexicographically smaller than it
      target (so the halfedge is directed from left to right), and
      \ccc{ARR_RIGHT_TO_LEFT} if it is lexicographically larger than
      the target (so the halfedge is directed from right to left).}

\ccMethod{Face_handle face();}
    {returns the face that \ccVar{} is incident to (The face lies to
    the left of \ccc{e}).}

\ccMethod{Halfedge_handle twin();}
    {returns the twin halfedge.}

\ccMethod{Halfedge_handle prev();}
    {returns \ccVar{}'s predecessor in the connected component it belongs to.}
\ccGlue
\ccMethod{Halfedge_handle next();}
    {returns \ccVar{}'s successor in the connected component it belongs to.}

\ccMethod{Ccb_halfedge_circulator ccb();}
    {returns a circulator that allows traversing the halfedges of the
     connected component boundary (CCB) that contains \ccVar{}.
     The circulator is initialized to point to \ccVar{}.}

\ccMethod{const typename Traits::X_monotone_curve_2& curve() const;}
    {returns the $x$-monotone curve associated with \ccVar{}.
     \ccPrecond{\ccVar{} is not a fictitious halfedge.}}

\end{ccRefClass}

\ccRefPageEnd
