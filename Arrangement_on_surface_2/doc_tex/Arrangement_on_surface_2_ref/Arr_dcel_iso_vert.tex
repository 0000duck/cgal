% +------------------------------------------------------------------------+
% | Reference manual page: Arr_dcel_iso_vert.tex
% +------------------------------------------------------------------------+
% | 
% | Package: Arrangement_2
% | 
% +------------------------------------------------------------------------+

\ccRefPageBegin

\begin{ccRefConcept}{ArrangementDcelIsolatedVertex}

\ccDefinition

An isolated vertex-information record in a \dcel\ data structure, which stores
the face that contains the isolated vertex in its interior, along with an
iterator for the isolated vertex in the isolated vertices' container of this
face.

\ccCreationVariable{iv}

\ccTypes
%=======

\ccNestedType{Face}{the corresponding \dcel\ face type.}

\ccTypedef{typedef Face::Isolated_vertex_iterator Isolated_vertex_iterator;}{}

\ccCreation
%==========

\ccConstructor{Arr_dcel_isolated_vertex();}
   {default constructor.}

\ccHtmlNoLinksFrom{   % to avoid linkage of Vertex, etc to HDS::Vertex

\ccAccessFunctions
%=================

All functions below also have \ccc{const} counterparts, returning non-mutable
pointers or iterators:

\ccMethod{Face* face ();}
    {returns the incident face, which contains \ccVar{} in its interior.}

\ccMethod{Isolated_vertex_iterator iterator();}
    {returns an iterator for the isolated vertex.}

\ccModifiers
%===========

  \ccMethod{void set_face (Face* f);}
    {sets the incident face.}

  \ccMethod{void set_iterator (Isolated_vertex_iterator it);}
    {sets the isolated vertex iterator.}

} % ccHtmlNoLinksFrom

\ccSeeAlso

\ccc{ArrangementDcel}\lcTex{ 
     (\ccRefPage{ArrangementDcel})}\\
\ccc{ArrangementDcelFace}\lcTex{
     (\ccRefPage{ArrangementDcelFace})}

\end{ccRefConcept}  

\ccRefPageEnd
