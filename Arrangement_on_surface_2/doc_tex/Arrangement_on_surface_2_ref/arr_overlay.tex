\ccRefPageBegin

\begin{ccRefFunction}{overlay}

\ccDefinition

The function \ccRefName\ computes the overlay of two input arrangement
objects, and returns the overlaid arrangement. 
All three arrangements can be instantiated with different geometric
traits classes and different \dcel{} (encapsulated in the various
topology-traits classes)  classes.
The geometry traits of the result arrangement is 
used to construct the result arrangement. This means that all
the types (e.g., \ccc{Traits::Point_2}, \ccc{Traits::Curve_2} and
\ccc{Traits::Point_2}) of both input arrangements have to convertible
to the types in the result arrangement.
A given overlay-traits object is used to properly construct the
overlaid \dcel{} that represents the resulting arrangement.

%%%%

\ccInclude{CGAL/Arr_overlay_2.h}

\ccGlobalFunction{template <class GeomTraitsA, class GeomTraitsB,
  class GeomTraitsRes, class TopTraitsA, class TopTraitsB, class
  TopTraitsRes, class OverlayTraits>
  void overlay (const Arrangement_on_surface_2<GeomTraitsA, TopTraitsA>& arr1,
  const Arrangement_on_surface_2<GeomTraitsB, TopTraitsB>& arr2,
  Arrangement_on_surface_2<GeomTraitsRes, TopTraitsRes>& arr_res,
  OverlayTraits& ovl_tr);}
Computes the overlay of two arrangements \ccc{arr1} and \ccc{arr2}, and sets
the output arrangement \ccc{res} to represent the overlaid arrangement.
\ccPrecond{\ccc{res} does not refer to either \ccc{arr1} or \ccc{arr2}
(that is, ``self overlay'' is not supported).}

%%%%

\ccThree{Orientation~~}{}{\hspace*{8.5cm}}

\ccInclude{CGAL/Arrangement_with_history_2.h}

\ccGlobalFunction{template<typename Traits, typename Dcel1, typename Dcel2,
                           typename ResDcel, typename OverlayTraits>
           void overlay (const Arrangement_with_history_2<Traits,Dcel1>& arr1,
                         const Arrangement_with_history_2<Traits,Dcel2>& arr2,
                         Arrangement_with_history_2<Traits,ResDcel>& res,
                         OverlayTraits& ovl_tr);}
Computes the overlay of two arrangements with history \ccc{arr1} and
\ccc{arr2}, and sets the output arrangement with history \ccc{res} to
represent the overlaid arrangement. The function also constructs a
consolidated set of curves that induce \ccc{res}.
\ccPrecond{\ccc{res} does not refer to either \ccc{arr1} or \ccc{arr2}
(that is, ``self overlay'' is not supported).}

\ccRequirements
\begin{itemize}
\item The overlay-traits object \ccc{ovl_tr} must model the \ccc{OverlayTraits}
  concept, which is able to construct records of the \ccc{ResDcel} class on
  the basis of the \ccc{Dcel1} and \ccc{Dcel2} records that induce them.
\end{itemize}

\ccSeeAlso
  \ccc{OverlayTraits}\lcTex{(\ccRefPage{OverlayTraits})}

\end{ccRefFunction}

\ccRefPageEnd
