\ccRefPageBegin

\begin{ccRefFunction}{overlay}

\ccDefinition

The function \ccRefName{} computes the overlay of two input arrangement
objects, and returns the overlaid arrangement. All three arrangements
can be instantiated with different geometric traits classes and different
\dcel{} (encapsulated in the various topology-traits classes) classes.
The geometry traits of the resulting arrangement is used to construct the
resulting arrangement. This means that all the types (e.g.,
\ccc{Traits::Point_2}, \ccc{Traits::Curve_2}, and \ccc{Traits::Point_2})
of both input arrangements have to be convertible to the types in the
resulting arrangement. A given overlay-traits object is used to properly
construct the overlaid \dcel{} that represents the resulting arrangement.

%%%%

\ccInclude{CGAL/Arr_overlay_2.h}

\ccGlobalFunction{template <class GeomTraitsA, class GeomTraitsB,
                            class GeomTraitsRes, class TopTraitsA,
                            class TopTraitsB, class TopTraitsRes,
                            class OverlayTraits>
  void overlay (const Arrangement_2<GeomTraitsA, TopTraitsA>& arr1,
                const Arrangement_2<GeomTraitsB, TopTraitsB>& arr2,
                Arrangement_2<GeomTraitsRes, TopTraitsRes>& arr_res,
                OverlayTraits& ovl_tr);}
Computes the overlay of two arrangements \ccc{arr1} and \ccc{arr2}, and sets
the output arrangement \ccc{res} to represent the overlaid arrangement.
\ccPrecond{\ccc{res} does not refer to either \ccc{arr1} or \ccc{arr2}
(that is, ``self overlay'' is not supported).}

\ccGlobalFunction{template <class GeomTraitsA, class GeomTraitsB,
                            class GeomTraitsRes, class TopTraitsA,
                            class TopTraitsB, class TopTraitsRes>
  void overlay (const Arrangement_2<GeomTraitsA, TopTraitsA>& arr1,
                const Arrangement_2<GeomTraitsB, TopTraitsB>& arr2,
                Arrangement_2<GeomTraitsRes, TopTraitsRes>& arr_res);}
Computes the (simple) overlay of two arrangements \ccc{arr1} and \ccc{arr2},
and sets the output arrangement \ccc{res} to represent the overlaid
arrangement. It employs the default overlay-traits, which practically does
nothing.
\ccPrecond{\ccc{res} does not refer to either \ccc{arr1} or \ccc{arr2}
(that is, ``self overlay'' is not supported).}

%%%%

\ccThree{Orientation~~}{}{\hspace*{8.5cm}}

\ccInclude{CGAL/Arrangement_with_history_2.h}

\ccGlobalFunction{template<typename Traits, typename Dcel1, typename Dcel2,
                           typename ResDcel, typename OverlayTraits>
           void overlay (const Arrangement_with_history_2<Traits,Dcel1>& arr1,
                         const Arrangement_with_history_2<Traits,Dcel2>& arr2,
                         Arrangement_with_history_2<Traits,ResDcel>& res,
                         OverlayTraits& ovl_tr);}
Computes the overlay of two arrangements with history \ccc{arr1} and
\ccc{arr2}, and sets the output arrangement with history \ccc{res} to
represent the overlaid arrangement. The function also constructs a
consolidated set of curves that induce \ccc{res}.
\ccPrecond{\ccc{res} does not refer to either \ccc{arr1} or \ccc{arr2}
(that is, ``self overlay'' is not supported).}

\ccGlobalFunction{template<typename Traits, typename Dcel1, typename Dcel2,
                           typename ResDcel>
           void overlay (const Arrangement_with_history_2<Traits,Dcel1>& arr1,
                         const Arrangement_with_history_2<Traits,Dcel2>& arr2,
                         Arrangement_with_history_2<Traits,ResDcel>& res);}
Computes the (simple) overlay of two arrangements with history \ccc{arr1}
and \ccc{arr2}, and sets the output arrangement with history \ccc{res} to
represent the overlaid arrangement. The function also constructs a
consolidated set of curves that induce \ccc{res}. It employs the default
overlay-traits, which practically does nothing.
\ccPrecond{\ccc{res} does not refer to either \ccc{arr1} or \ccc{arr2}
(that is, ``self overlay'' is not supported).}

\ccRequirements
\begin{itemize}
\item The overlay-traits object \ccc{ovl_tr} must model the \ccc{OverlayTraits}
  concept, which is able to construct records of the \ccc{ResDcel} class on
  the basis of the \ccc{Dcel1} and \ccc{Dcel2} records that induce them.
\end{itemize}

\ccSeeAlso
  \ccc{OverlayTraits}\lcTex{(\ccRefPage{OverlayTraits})}

\end{ccRefFunction}

\ccRefPageEnd
