\ccRefPageBegin

\begin{ccRefFunction}{do_intersect}

\ccDefinition

The function \ccRefName{} checks if a given curve or $x$-monotone
curve intersects an existing arrangement's edges or vertices. 

If the give curve is not an $x$-monotone curve then the function
subdivides the given curve into $x$-monotone subcurves and isolated
vertices . Each subcurve is in turn checked for intersection.
The function uses the zone algorithm to check if the curve intersects
the arrangement. First, the curve's left endpoint is located. Then, 
its zone is computed starting from its left endpoint location. The
zone computation terminates when an intersection with an arrangement's
edge/vertex is found or when the right endpoint is reached. 

A given point-location object is used for locating the left endpoint 
of the given curve in the existing arrangement. By default, the function 
uses the "walk along line" point-location strategy --- namely an 
instance of the class 
\ccc{Arr_walk_along_line_point_location<Arrangement_2<Traits,Dcel> >}.

%%%%

\ccInclude{CGAL/Arrangement_2.h}

\ccGlobalFunction{template <class GeomTraits, class TopTraits, 
  class Curve, class PointLocation>
  bool do_intersect (
  Arrangement_on_surface_2<GeomTraits, TopTraits>& arr, 
  const Curve& c, const PointLocation& pl);}
Checks if the given curve or $x$-monotone curve \ccc{c} intersects
  edges or vertices of the existing arrangement \ccc{arr}.
\ccPrecond{If provided, \ccc{pl} must be attached to the given arrangement
\ccc{arr}.}

%%%%

\ccRequirements
\begin{itemize}
\item If \ccc{c} is $x$-monotone then the instantiated \ccc{GeomTraits}
  class must model the  \ccc{ArrangementXMonotoneTraits_2} concept. If
  \ccc{c} is a curve then the instantiated \ccc{GeomTraits} class must
  model the \ccc{ArrangementTraits_2} concept. That is, it should
  define the \ccc{Curve_2} type, and support its subdivision into
  $x$-monotone subcurves (and perhaps isolated points).
\item The point-location object \ccc{pl}, must model the
  \ccc{ArrangementPointLocation_2} concept.
\end{itemize}

\end{ccRefFunction}

\ccRefPageEnd
