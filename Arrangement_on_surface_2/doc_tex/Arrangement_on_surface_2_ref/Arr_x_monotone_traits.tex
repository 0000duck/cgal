% Reference manual page: ArrangementXMonotoneTraits.tex
% Package: Arrangement_2

\ccRefPageBegin
\begin{ccRefConcept}{ArrangementXMonotoneTraits_2}

\ccThree{ArrangementXMonotoneTraits::Has_merge_category}{}{}
\ccThreeToTwo

\ccDefinition
% ===========
The concept \ccRefName{} refines the basic arrangement-traits concept.
A model of this concept is able to handle $x$-monotone curves that
intersect in their interior (and points that coincide with curve
interiors). This is necessary for constructing arrangements of sets of
intersecting $x$-monotone curves.

As the resulting structure, represented by the \ccc{Arrangement_2} class,
stores pairwise interior-disjoint curves, the input curves are split at
the intersection points before being inserted into the arrangement. 
A model of this refined concept therefore needs to compute the intersections
(and possibly overlaps) between two $x$-monotone curves and to support
curve splitting.

\ccRefines
\ccc{ArrangementBasicTraits_2}

\ccTypes
% ======
\ccNestedType{Multiplicity}{the multiplicity type.}

\ccHeading{Tags}
% ==============
\ccNestedType{Has_merge_category}
  {indicates whether the nested functors \ccc{Are_mergeable_2} and
  \ccc{Merge_2} are provided.} 

\ccHeading{Functor Types}
% =======================
\ccThree{Compare_y_at_x_2}{}{\hspace*{14cm}}
\ccThreeToTwo
\ccNestedType{Intersect_2}{models the concept \ccc{ArrTraits::Intersect_2}.}
\ccGlue
\ccNestedType{Split_2}{models the concept \ccc{ArrTraits::Split_2}.}

The two following function-object types are optional. If they are
supported, the \ccc{Has_merge_category} tag should be defined as
\ccc{Tag_true} (and \ccc{Tag_false} otherwise.

\ccNestedType{Are_mergeable_2}%
  {models the concept \ccc{ArrTraits::AreMergeable_2}.}
\ccGlue
\ccNestedType{Merge_2}{models the concept \ccc{ArrTraits::Merge_2}.}

\ccThree{Construct_x_monotone_curve_2~~~}{}{\hspace*{7cm}}
\ccThreeToTwo

% \ccCreation
\ccCreationVariable{traits}
%==========================

\ccHeading{Accessing Functor Objects}
% ===================================
\ccMethod{Intersect_2 intersect_2_object() const;} {}
\ccGlue
\ccMethod{Split_2 split_2_object() const;} {}

The two following methods are optional. If they are supported, 
the \ccc{Has_merge_category} tag should be defined as \ccc{Tag_true} (and
\ccc{Tag_false} otherwise.

\ccMethod{Are_mergeable_2 are_mergeable_2_object() const;} {}
\ccGlue
\ccMethod{Merge_2 merge_2_object() const;} {}

\ccHasModels
% ==========
\ccc{CGAL::Arr_segment_traits_2<Kernel>}\\
\ccc{CGAL::Arr_non_caching_segment_traits_2<Kernel>}\\
\ccc{CGAL::Arr_linear_traits_2<Kernel>}\\
\ccc{CGAL::Arr_polyline_traits_2<SegmentTraits>}\\
\ccc{CGAL::Arr_circle_segment_traits_2<Kernel>}\\
\ccc{CGAL::Arr_line_arc_traits_2<CircularKernel>}\\
\ccc{CGAL::Arr_circular_arc_traits_2<CircularKernel>}\\
\ccc{CGAL::Arr_circular_line_arc_traits_2<CircularKernel>}\\
\ccc{CGAL::Arr_conic_traits_2<RatKernel,AlgKernel,NtTraits>}\\
\ccc{CGAL::Arr_rational_function_traits_2<AlgebraicKernel_d_1>}\\
\ccc{CGAL::Arr_Bezier_curve_traits_2<RatKernel,AlgKernel,NtTraits>}\\
\ccc{CGAL::Arr_algebraic_segment_traits_2<Coefficient>}\\
\ccc{CGAL::Arr_curve_data_traits_2<Tr,XData,Mrg,CData,Cnv>}\\
\ccc{CGAL::Arr_consolidated_curve_data_traits_2<Traits,Data>}

\ccSeeAlso
% ========
\ccc{ArrangementBasicTraits_2}\lcTex{(\ccRefPage{ArrangementBasicTraits_2})}

\end{ccRefConcept}
\ccRefPageEnd

%%%%%%%% Functors  %%%%%%%%

%%%%%%%% Intersect_2
% ==================
\ccRefPageBegin
\begin{ccRefConcept}{ArrTraits::Intersect_2}
\ccRefines{Functor}

\ccHasModels\ccc{ArrangementXMonotoneTraits_2::Intersect_2}

\ccCreationVariable{fo}

\ccMethod{Output_iterator operator()(ArrTraits::X_monotone_curve_2 xc1,
                                     ArrTraits::X_monotone_curve_2 xc2,
                                     Output_iterator& oi);}{%
  computes the intersections of \ccc{xc1} and \ccc{xc2} and 
  inserts them {\sl in an ascending lexicographic $xy$-order} into the 
  output iterator \ccc{oi}. The value-type of \ccc{Output_iterator} is 
  \ccc{CGAL::Object}, where each \ccc{Object} wraps either a 
  \ccc{pair<ArrTraits::Point_2,ArrTraits::Multiplicity>} object, which
  represents an intersection point with its multiplicity (in case the
  multiplicity is undefined or unknown, it should be set to $0$) or an 
  \ccc{ArrTraits::X_monotone_curve_2} object, representing an
  overlapping subcurve of \ccc{xc1} and \ccc{xc2}. The operator
  returns a past-the-end iterator for the output sequence.}
\end{ccRefConcept}
\ccRefPageEnd

%%%%%%%% Split_2
% ==============
\ccRefPageBegin
\begin{ccRefConcept}{ArrTraits::Split_2}
\ccRefines{Functor}

\ccHasModels\ccc{ArrangementXMonotoneTraits_2::Split_2}

\ccCreationVariable{fo}

\def\ccTagRmConstRefPair{\ccFalse}%
\ccMethod{void operator()(ArrTraits::X_monotone_curve_2 xc,
                          ArrTraits::Point_2 p,
                          ArrTraits::X_monotone_curve_2& xc1,
                          ArrTraits::X_monotone_curve_2& xc2);}{%
  accepts an input curve \ccc{xc} and a split point \ccc{p} in its
  interior. It splits \ccc{xc} at the split point into two subcurves \ccc{xc1}
  and \ccc{xc2}, such that \ccc{p} is \ccc{xc1}'s {\sl right} endpoint and
  \ccc{xc2}'s {\sl left} endpoint.}%
\ccTagDefaults
\end{ccRefConcept}
\ccRefPageEnd

%%%%%%%% AreMergeable_2
% =====================
\ccRefPageBegin
\begin{ccRefConcept}{ArrTraits::AreMergeable_2}
\ccRefines{Functor}

\ccHasModels\ccc{ArrangementXMonotoneTraits_2::Are_mergeable_2}

\ccCreationVariable{fo}

\ccMethod{bool operator()(ArrTraits::X_monotone_curve_2 xc1,
                          ArrTraits::X_monotone_curve_2 xc2);}{%
  accepts two $x$-monotone curves \ccc{xc1} and \ccc{xc2} and determines
  whether they can be merged to form a single $x$-monotone curve.
  \ccc{xc1} and \ccc{xc2} are mergeable if their underlying curves are
  identical, they share a common endpoint, and they do not bend to form
  a non-$x$-monotone curve.}
  % \ccPrecond{\ccc{xc1} and \ccc{xc2} share a common endpoint.}}
\end{ccRefConcept}
\ccRefPageEnd

%%%%%%%% Merge_2
% ==============
\ccRefPageBegin
\begin{ccRefConcept}{ArrTraits::Merge_2}
\ccRefines{Functor}

\ccHasModels\ccc{ArrangementXMonotoneTraits_2::Merge_2}

\ccCreationVariable{fo}

\def\ccTagRmConstRefPair{\ccFalse}%
\ccMethod{void merge(ArrTraits::X_monotone_curve_2 xc1,
                     ArrTraits::X_monotone_curve_2 xc2,
                     ArrTraits::X_monotone_curve_2& xc);}{%
  accepts two \emph{mergeable} $x$-monotone curves \ccc{xc1} and \ccc{xc2}
  and asigns \ccc{xc} with the merged curve.
  \ccPrecond{\ccc{are_mergeable_2}(\ccc{xc1}, \ccc{xc2}) is true.}}
\ccTagDefaults
\end{ccRefConcept}
\ccRefPageEnd
