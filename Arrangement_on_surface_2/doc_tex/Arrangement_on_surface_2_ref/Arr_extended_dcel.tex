% +------------------------------------------------------------------------+
% | Reference manual page: Arr_extended_dcel.tex
% +------------------------------------------------------------------------+
% | 
% | Package: Arrangement_2
% | 
% +------------------------------------------------------------------------+

\ccRefPageBegin

\begin{ccRefClass}{Arr_extended_dcel<Traits,VData,HData,FData,V,H,F>}

\ccDefinition
%============
The \ccRefName\ class-template extends the topological-features of the \dcel\,
namely the vertex, halfedge, and face types. While it is possible to maintain
extra (non-geometric) data with the curves or points of the arrangement by
extending their types respectively, it is also possible to extend the vertex,
halfedge, or face types of the \dcel\ through inheritance. As the technique to
extend these types is somewhat cumbersome and difficult for inexperienced
users, the \ccRefName\ class-template provides a convenient way to do that.
Each one of the three features is extended with a corresponding data type
provided as parameters. This class template is also parameterized with a
traits class used to extract default values for the vertex, halfedge, and face
base classes, which are the remaining three template parameters respectively.
The default values follow:

\begin{tabbing}
\ccc{V} = \=\ccc{Arr_vertex_base<typename Traits::Point_2>}\\
\ccc{H} = \>\ccc{Arr_halfedge_base<typename Traits::X_monotone_curve_2>}\\
\ccc{F} = \>\ccc{Arr_face_base}
\end{tabbing}

\ccInclude{CGAL/Arr_extended_dcel.h}

\ccIsModel
  \ccc{ArrangementDcelWithRebind}

\ccInheritsFrom
  \begin{tabbing}
  \ccc{Arr_dcel_base<}\=\ccc{Arr_extended_vertex<V, VData>,}\\
                      \>\ccc{Arr_extended_halfedge<H, HData>,}\\
                      \>\ccc{Arr_extended_face<F, FData> >}
  \end{tabbing}

\ccSeeAlso

\ccc{Arr_dcel_base<V,H,F>}\lcTex{
  (\ccRefPage{CGAL::Arr_dcel_base<V,H,F>})}

\end{ccRefClass}

\ccRefPageEnd
