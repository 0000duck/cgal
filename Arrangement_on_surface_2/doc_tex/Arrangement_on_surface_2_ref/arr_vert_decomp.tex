\ccRefPageBegin

\begin{ccRefFunction}{decompose}

\ccDefinition

The function \ccRefName\ produces the symbolic vertical decomposition of a
given arrangement, performing a batched vertical ray-shooting query from
all arrangement vertices, such that every vertex is associated with a pair
of objects, one corresponds to the arrangement feature that lies below it,
and the other corresponds to the feature that lies above it.
The output of this function can be readily used for inserting vertical walls
and physically decomposing the arrangement into pseudo-trapezoids. To do
this, it is convenient to process the vertices in an ascending
$xy$-lexicographic order. The visible objects are therefore returned through
an output iterator, which pairs each finite arrangement vertex with the two
features it ``sees'', such that the vertices are given in ascending
$xy$-lexicographic order. 

\ccInclude{CGAL/Arr_vertical_decomposition_2.h}

\ccThree{OutputIterator  }{}{\hspace*{12cm}}

\ccGlobalFunction{template<typename Traits, typename Dcel,
                           typename OutputIterator>
            OutputIterator decompose (const Arrangement_2<Traits,Dcel>& arr,
                                      OutputIterator oi);}
Produces the symbolic vertical decomposition of the \ccc{arr} arrangement.
More precisely, it performs a batched vertical ray-shooting query from all
arrangement vertices, such that every vertex is associated with a pair of
objects, one corresponding to the arrangement feature that lies below it,
while the other corresponds to the feature that lies above it. 
The query results are returned through the output iterator, which pairs
each finite arrangement vertex with a pair of \ccc{Object}s, the first
represents the feature below the vertex, and the second represents the
feature that lies above it. Each \ccc{Object} may be one of the following:
\begin{itemize}
\item \ccc{Halfedge_const_handle}, if the vertex is located above (or
  below) an edge. The given halfedge is always directed from right to left.
  In case there is no concrete edge below (or above) the vertex, and
  the arrangement is unbounded, then the object returned is a
  \emph{fictitious} halfedge.
\item \ccc{Face_const_handle}, in case there is no edge below (or above)
  the vertex, and the arrangement is bounded.
\item \ccc{Vertex_const_handle}, in case the vertex is located vertically
  above (or below) another arrangement vertex.
\item An empty object, in case the vertex is the top end-vertex of
  a vertical edge, we define there is no feature below it. Similarly, if
  it is the bottom end-vertex of a vertical edge, we define that there
  is no feature above it.
\end{itemize}
The function returns a past-the-end iterator for its output sequence.

\ccHeading{Requirements}
\ccc{OutputIterator::value_type} must be
  \ccc{pair<Arrangement_2::Vertex_const_handle, pair<Object, Object> >}.

\end{ccRefFunction}

\ccRefPageEnd
