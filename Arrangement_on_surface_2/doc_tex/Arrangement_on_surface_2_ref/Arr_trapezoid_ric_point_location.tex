% +------------------------------------------------------------------------+
% | Reference manual page: Arr_trapezoid_ric_point_location.tex
% +------------------------------------------------------------------------+
% | 
% | Package: Arrangement_2
% | 
% +------------------------------------------------------------------------+

\ccRefPageBegin

\begin{ccRefClass}{Arr_trapezoid_ric_point_location<Arrangement>}
\label{arr_ref:trap_pl}

\ccDefinition
%============

The \ccRefName\ class implements the incremental randomized algorithm
introduced by Mulmuley~\cite{m-fppa-90} as presented by
Seidel~\cite{s-sfira-91} (see also~\cite[Chapter~6]{bkos-cgaa-00}).
It subdivides each arrangement face to pseuso-trapezoidal cells, each
of constant complexity, and constructs and maintains a search
structure on top of these cells, such that each query can be answered
in $O(\log n)$ time, where $n$ is the complexity of the arrangement.

Constructing the search structures takes $O(n \log n)$ time, such that
attaching a trapezoidal point-location object to an existing arrangement
may incur some overhead in running times. In addition, the point-location
object needs to keep its auxiliary data structures up-to-date as the
arrangement goes thorugh structural changes. It is therefore recommended
to use this point-location strategy for static arrangements (or arrangement
that do not alter frequently), and when the number of issued queries
is relatively large.

\ccInclude{CGAL/Arr_trapezoid_ric_point_location.h}

\ccIsModel
  \ccc{ArrangementPointLocation_2} \\
  \ccc{ArrangementVerticalRayShoot_2}

\end{ccRefClass}

\ccRefPageEnd
