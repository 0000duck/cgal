% +------------------------------------------------------------------------+
% | Reference manual page: Arr_face_overlay_traits.tex
% +------------------------------------------------------------------------+
% | 
% | Package: Arrangement_2
% | 
% +------------------------------------------------------------------------+

\ccRefPageBegin

\begin{ccRefClass}{Arr_face_overlay_traits<Arr_A,Arr_B,Arr_R,OvlFaceData>}

\ccDefinition
%============

An instance of \ccRefName\ should be used for overlaying two arrangements
of types \ccc{Arr_A} and \ccc{Arr_B}, which are instantiated using the same
geometric traits-class and with the \dcel\ classes \ccc{Dcel_A} and
\ccc{Dcel_B} respetively, in order to store their overlay in an arrangement
of type \ccc{Arr_R}, which is instantiated using a third \dcel\ class
\ccc{Dcel_R}. All three \dcel\ classes are assumed to be instantiations of the
\ccc{Arr_face_extended_dcel} template with types \ccc{FaceData_A},
\ccc{FaceData_B} and \ccc{FaceData_R}, respectively.

This class gives empty implementation for all overlay traits-class functions,
except the function that computes the overlay of two faces. In this case,
it uses the functor \ccc{OvlFaceData}, which accepts a \ccc{FaceData_A} object
and a \ccc{FaceData_B} object and computes a corresponding \ccc{FaceData_R}
object, in order to set the auxiliary data of the overlay face.

\ccInclude{CGAL/Arr_default_overlay_traits.h}

\ccIsModel
    \ccc{OverlayTraits}

\ccSeeAlso
    \ccc{overlay}\lcTex{
     (\ccRefPage{CGAL::overlay})} \\
    \ccc{CGAL::Arr_face_extended_dcel<Traits,FData,V,H,F>}\lcTex{
     (\ccRefPage{CGAL::Arr_face_extended_dcel<Traits,FData,V,H,F>})} \\

\end{ccRefClass}

\ccRefPageEnd
