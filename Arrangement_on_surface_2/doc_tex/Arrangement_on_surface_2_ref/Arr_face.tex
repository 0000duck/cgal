% +------------------------------------------------------------------------+
% | Reference manual page: Arr_face.tex
% +------------------------------------------------------------------------+
% | 
% | Package: Arrangement_2
% | 
% +------------------------------------------------------------------------+

\ccRefPageBegin

\begin{ccRefClass}[Arrangement_2<Traits,Dcel>::]{Face}
\ccRefLabel{arr_ref:face}

\ccDefinition
An object of the class \ccRefName{} represents an arrangement face,
namely, a $2$-dimensional arrangement cell. An arrangement that supports
only bounded curves contains exactly one \emph{unbounded} face, and a
number of bounded faces. An arrangement that supports also unbounded
curves has one or more unbounded faces. Such an arrangement has also
exactly one fictitious face, which does not correspond to a real
two-dimensional cell of the arrangement (and thus it is ignored in
counting the number of faces of the arrangement.)
Each bounded face has an outer boundary comprising a halfedge chain
winding in counterclockwise orientation around it. Each unbounded face of
an arrangement that has a fictitious face also has a boundary comprising
a counterclockwise halfedge-chain. The edges on the boundary of a face
incident to the fictitious face are fictitious, as they do not correspond
to real curves. A face may also contain holes, which are defined by
clockwise-oriented halfedge chains, and isolated vertices.

\ccInheritsFrom
    \ccHtmlNoLinksFrom{\ccc{typename Dcel::Face}}

\ccCreation
\ccCreationVariable{f}
%=====================

\ccConstructor{Face();}
   {default constructor.}    	    
    
\ccAccessFunctions
%=================

All non-const methods listed below also have \ccc{const} counterparts
that return constant handles, iterators or circulators:

\ccMethod{bool is_fictitious () const;}
  {returns a Boolean indicating whether this is the fictitious face,
  which contain the entire arrangement (and does not have an outer CCB).
  An arrangement that supports only bounded curves does not have a
  fictitious face at all.}

\ccMethod{bool is_unbounded() const;} 
  {returns a Boolean indicating whether the face is unbounded.}    

\ccMethod{bool has_outer_ccb() const;} 
  {returns a Boolean indicating whether the face has an outer CCB.
   (The fictitious face and the unbounded face of an arrangement that
   does not have a fictitious face do not have outer CCBs.)}    

\ccMethod{Ccb_halfedge_circulator outer_ccb();}
    {returns a circulator that enables traversing the outer boundary of
     \ccVar{}. The edges along the CCB are traversed in a counterclockwise
     direction.
     \ccPrecond{The face \ccVar{} has an outer CCB.}}
    
\ccMethod{Hole_iterator holes_begin();}
    {returns an iterator for traversing all the holes (inner CCBs) of
     \ccVar{}.}
\ccGlue
\ccMethod{Hole_iterator holes_end();}
    {returns a past-the-end iterator for the holes of \ccVar{}.}

\ccMethod{Isolated_vertex_iterator isolated_vertices_begin();}
    {returns an iterator for traversing all the isolated vertices
     contained in the interior of \ccVar{}.}
\ccGlue
\ccMethod{Isolated_vertex_iterator isolated_vertices_end();}
    {returns a past-the-end iterator for the isolated vertices 
     contained inside \ccVar{}.}

\end{ccRefClass}

\ccRefPageEnd
