% +------------------------------------------------------------------------+
% | Reference manual page: Arr_non_caching_segment_basic_traits.tex
% +------------------------------------------------------------------------+
% | 
% | Package: Arrangement_2
% | 
% +------------------------------------------------------------------------+

\ccRefPageBegin

\begin{ccRefClass}{Arr_non_caching_segment_basic_traits_2<Kernel>}
    
\ccDefinition 

The traits class \ccRefName{} is a model of the \ccc{ArrangementTraits_2}
concept that allow the construction and maintenance of arrangements of
sets of pairwise interior-disjoint line segments. It is templated with a
\cgal-Kernel model, and it is derived from it. This traits class is a
thin layer above the parameterized kernel. It inherits the \ccc{Point_2}
from the kernel and its \ccc{X_monotone_curve_2} type is defined as
\ccc{Kernel::Segment_2}. Most traits-class functor are inherited from the
kernel functor, and the traits class only supplies the necessary functors
that are not provided by the kernel. The kernel is parameterized with a
number type, which should support the arithmetic operations $+$, $-$ and
$\times$ in an exact manner in order to avoid robustness problems.
Using \ccc{Cartesian<MP_Float>} or \ccc{Cartesian<Gmpz>} are possible
instantiations for the kernel. Using other (inexact) number types
(for example, instantiating the template with
\ccStyle{Simple_cartesian<double>}) is also possible, at the user's own
risk.
    
\ccInclude{CGAL/Arr_non_caching_segment_basic_traits_2.h}
 
\ccIsModel
    \ccc{ArrangementLandmarkTraits_2}

\end{ccRefClass}
\ccRefPageEnd
