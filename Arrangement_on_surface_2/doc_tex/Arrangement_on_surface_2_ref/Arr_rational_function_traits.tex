% +------------------------------------------------------------------------+
% | Reference manual page: Arr_rational_function_traits.tex
% +------------------------------------------------------------------------+
% | 
% | Package: Arrangement_2
% | 
% +------------------------------------------------------------------------+

\ccRefPageBegin
\begin{ccRefClass}{Arr_rational_function_traits_2<AlgebraicKernel_d_1>}
\ccCreationVariable{traits}

\ccDefinition
%============

The traits class \ccRefName\ is a model of the \ccc{ArrangementTraits_2}
concept. It handles bounded or unbounded arcs of rational functions,
referred to as {\sl rational arcs} (in particular, such an arc may
correspond to the entire graph of a rational function), and enables the
construction and maintenance of arrangements of such arcs. 
%Rational functions, and polynomial functions in particular, are not only 
%interesting in their own right, they are also very useful for approximating or
%interpolating more complex curves.

A rational function $y = \frac{P(x)}{Q(x)}$ is defined by two polynomials 
$P$ and $Q$ of arbitrary degrees. 
If $Q(x) = 1$ then the function is a simple polynomial function.
Usually the domain is $\R$ but the function may also be 
restricted to a bounded interval $[x_{\rm min}, x_{\rm max}]$ 
or defined over a ray $(-\infty, x_{\rm max}]$ or over $[x_{\rm min}, \infty)$. 
Rational functions are represented by the nested type \ccc{Curve_2}. 
Note that a rational function may be not continuous since roots of $Q$ induce 
vertical asymptotes, which would contradict the notion of an $x$-monotone curve as 
it is introduced by the \ccc{ArrangementTraits_2} concept. 
Thus, continuous portions of rational functions are represented by the nested 
type \ccc{X_monotone_curve_2}, which is different from \ccc{Curve_2}.
Constructors for both classes are provided by the traits. 
A \ccc{Curve_2} may be split up into several \ccc{X_monotone_curve_2}
using \ccc{Make_x_monotone_2}. 

%If $Q(x) = 1$ then the function is a simple polynomial
%function. A bounded rational arc is defined by the graph of a rational
%function over some internal $[x_{\rm min}, x_{\rm max}]$, where $Q$
%does not have any real roots in this interval (thus the arc does not
%contain any vertical asymptotes). Our traits class is also capable of
%representing functions defined over an unbounded $x$-range, namely
%a ``ray'' defined over $(-\infty, x_{\rm max}]$ or over $[x_{\rm min}, \infty)$,
%or a function defined over the entire real $x$-range. Note that a
%rational arc may be unbounded even if it is defined over some bounded interval.
%In these cases $Q$ has zeros in this interval. That is, the user is able to construct 
%rational arcs of type \ccc{Curve_2}, which may contain vertical asymptotes. 
%These may be split up further into \ccc{X_monotone_curve_2} using 
%\ccc{Make_x_monotone_2}. 

The template parameter to the traits must be a model of the 
concept \ccc{AlgebraicKernel_d_1}. 
A rational function is then represented by two polynomials $P$ and $Q$ of type 
\ccc{AlgebraicKernel_d_1::Polynomial_1}. 
A point is represented by a rational function and its $x$-coordinate, which is 
of type \ccc{AlgebraicKernel_d_1::Algebraic_real_1}. 
Note that an explicit representation of the $y$-coordinate is only computed upon 
request, which can be a rather costly operation. 


The constructed rational functions are cached by the traits class. 
The cache is local to each traits class object. 
It is therefore necessary to construct the curves using the constructor 
objects provided by member functions of the traits class. 
%This is also the reason why IO is not handled via the usual stream operators. 
Moreover, a curve must only be used with its own traits. 
The cache is automatically  cleaned up from time to time.
The amortized clean up costs are constant. However, there is also a 
separate member function that cleans up the cache on demand.  


\ccInclude{CGAL/Arr_rational_function_traits_2.h}

\ccIsModel
  \ccc{ArrangementTraits_2}\\
%  \ccc{ArrangementLandmarkTraits_2}\\ %% not a model of this concept since construction of segment is not easy 
  \ccc{ArrangementDirectionalXMonotoneTraits_2}


\ccTypes \ccThree{}{xxxxxxxxxxxxxxxxxxxxxx}{x} 
\ccTypedef{typedef AlgebraicKernel_d_1 Algebraic_kernel_d_1;}{}\ccGlue
\ccTypedef{typedef AlgebraicKernel_d_1::Coefficient Coefficient;}{}\ccGlue
\ccTypedef{typedef AlgebraicKernel_d_1::Polynomial Polynomial_1;}{}\ccGlue
\ccTypedef{typedef AlgebraicKernel_d_1::Algebraic_real_1 Algebraic_real_1;}{}\ccGlue
\ccTypedef{typedef AlgebraicKernel_d_1::Bound Bound;}{}

%\ccTypedef{typedef AlgebraicKernel_d_1::Bound Approximate_number_type;}{}\ccGlue
%\ccNestedType{Approximate_2}{
%A model of \ccc{ArrangementLandmarkTraits_2::Approximate_2}}\ccGlue
%\ccMethod{Approximate_2 approximate_2_object() const;}{ Returns an instance of \ccc{Construct_curve_2}.}


\ccOperations
\ccThree{xxxxxxxxxxxxxxxxxxxxxxxxxxxxxxx}{xxxxxxx}{} 
\ccMethod{Construct_curve_2 construct_curve_2_object() const;}{ Returns an instance of \ccc{Construct_curve_2}.}\ccGlue
%\ccMethod{Curve_importer_2 curve_importer_2_object() const;}{ Returns an instance of \ccc{Curve_importer_2}.}\ccGlue
%\ccMethod{Curve_exporter_2 curve_exporter_2_object() const;}{ Returns an instance of \ccc{Curve_exporter_2}.}\ccGlue
\ccMethod{Construct_x_monotone_curve_2 construct_x_monotone_curve_2_object() const;}{ Returns an instance of \ccc{Construct_x_monotone_curve_2}.}\ccGlue
%\ccMethod{X_monotone_curve_importer_2 x_monotone_curve_importer_2_object() const;}{ Returns an instance of \ccc{X_monotone_curve_importer_2}.}\ccGlue
%\ccMethod{X_monotone_curve_exporter_2 x_monotone_curve_exporter_2_object() const;}{ Returns an instance of \ccc{X_monotone_curve_exporter_2}.}\ccGlue

\ccMethod{void cleanup_cache() const;}{Deletes all curves from the cache that exist only there.}


\subsection*{Class   Arr\_rational\_function\_traits\_2$<$AlgebraicKernel\_d\_1$>$::Curve\_2}
\begin{ccClass}{Arr_rational_function_traits_2<AlgebraicKernel_d_1>::Curve_2}
\ccCreationVariable{curve}

The \ccc{Curve_2} class nested within the traits is used
to represent rational functions which may be restricted to a certain x-range.  

\ccIsModel

\ccc{Assignable}\\
\ccc{CopyConstructible}\\
\ccc{DefaultConstructible}\\
\ccc{EqualityComparable}
% \ccc{LessThanComparable}\\ should we add this, its possible to define some order.. 


\ccTypes \ccThree{}{xxxxxxxxxxxxxxxxxxxxxx}{x}
\ccTypedef{typedef AlgebraicKernel_d_1::Polynomial_1 Polynomial_1; }{}\ccGlue
\ccTypedef{typedef AlgebraicKernel_d_1::Algebraic_real_1 Algebraic_real_1; }{}

\ccOperations
\ccThree{xxxxxxxxxxxxxxxxxxxx}{xxxxxxxxxxxxxxxxxxxxxxxxxxxxxxx}{}
\ccTwo  {xxxxxxxxxxxxxxxxxxxxxxxxxxxxxxxxxxxxxxxxxxxx}{}


\ccMethod{const Polynomial_1& numerator () const;}
  {returns the numerator of the supporting rational function.}

\ccMethod{const Polynomial_1& denominator () const;}
  {returns the denominator of the supporting rational function.}

\ccMethod{bool is_continuous() const;}
  {returns whether \ccVar\ is continuous, namely whether it does not
   contains any vertical asymptotes in its interior.}

\ccMethod{Arr_parameter_space left_boundary_in_x () const;}
  {returns whether the $x$-coordinate of \ccVar's left end is finite or
   whether it is $\pm\infty$.}
\ccGlue
\ccMethod{Arr_parameter_space right_boundary_in_x () const;}
  {returns whether the $x$-coordinate of \ccVar's right end is finite or
   whether it is $\pm\infty$.}
\ccGlue
\ccMethod{Algebraic_real_1 left_x() const;}
  {returns the $x$-coordinate of the left end.
    \ccPrecond{left\_boundary\_in\_x()==ARR\_INTERIOR}}
\ccGlue
\ccMethod{Algebraic_real_1 right_x() const;}
  {returns the $x$-coordinate of the right end.
   \ccPrecond{right\_boundary\_in\_x()==ARR\_INTERIOR}}
\end{ccClass}

\subsection*{Class   Arr\_rational\_function\_traits\_2$<$AlgebraicKernel\_d\_1$>$::X\_monotone\_curve\_2}

The \ccc{X_monotone_curve_2} class nested within the traits is used
to represent $x$-monotone parts of rational functions. In particular, such an $x$-monotone curve 
may not contain a vertical asymptote in its interior $x$-range. 

\begin{ccClass}{Arr_rational_function_traits_2<AlgebraicKernel_d_1>::X_monotone_curve_2}
\ccCreationVariable{xcurve}

\ccIsModel
\ccc{Assignable}\\
\ccc{CopyConstructible}\\
\ccc{DefaultConstructible}\\
\ccc{EqualityComparable}
% \ccc{LessThanComparable}\\ should we add this, its possible to define some order.. 

\ccTypes \ccThree{}{xxxxxxxxxxxxxxxxxxxxxx}{x}
\ccTypedef{typedef AlgebraicKernel_d_1::Polynomial_1 Polynomial_1; }{}\ccGlue
\ccTypedef{typedef AlgebraicKernel_d_1::Algebraic_real_1 Algebraic_real_1; }{}\ccGlue
\ccTypedef{typedef Arr_rational_function_traits_2<AlgebraicKernel_d_1>::Point_2 Point_2;}{}

\ccOperations
\ccThree{xxxxxxxxxxxxxxxxxxxx}{xxxxxxxxxxxxxxxxxxxxxxxxxxxxxxxxxx}{}
\ccTwo  {xxxxxxxxxxxxxxxxxxxxxxxxxxxxxxxxxxxxxxxxxxxx}{}


\ccMethod{const Polynomial_1& numerator () const;}
  {returns the numerator of the supporting rational function.}

\ccMethod{const Polynomial_1& denominator () const;}
  {returns the denominator of the supporting rational function.}


% ======== source 
\ccMethod{Arr_parameter_space source_boundary_in_x () const;}
  {returns whether the $x$-coordinate of the source is finite or
   whether it is $\pm\infty$.}
\ccGlue
\ccMethod{Arr_parameter_space source_boundary_in_y () const;}
  {returns whether the $y$-coordinate of the source is finite or
   whether it is $\pm\infty$.}
\ccGlue
\ccMethod{const Point_2& source() const;}
  {returns the source point of the arc.
   \ccPrecond{The source is finite in both $x$ and $y$.}}
\ccGlue
\ccMethod{Algebraic_real_1 source_x() const;}
  {returns the $x$-coordinate of the source point.
   \ccPrecond{The source point is finite in $x$.}}
%\ccGlue
%\ccMethod{Algebraic_real_1 source_y() const;}
%  {returns the $y$-coordinate of the source point.
%   \ccPrecond{The source point is finite in $y$.}}


% ======== target 
\ccMethod{Arr_parameter_space target_boundary_in_x () const;}
  {returns whether the $x$-coordinate of the target is finite or
   whether it is $\pm\infty$.}
\ccGlue
\ccMethod{Arr_parameter_space target_boundary_in_y () const;}
  {returns whether the $y$-coordinate of the target is finite or
   whether it is $\pm\infty$.}
\ccGlue
\ccMethod{const Point_2& target() const;}
  {returns the target point of the arc.
   \ccPrecond{The target is finite in both $x$ and $y$.}}
\ccGlue
\ccMethod{Algebraic_real_1 target_x() const;}
  {returns the $x$-coordinate of the target point.
   \ccPrecond{The target point is finite in $x$.}}
%\ccGlue
%\ccMethod{Algebraic_real_1 target_y() const;}
%  {returns the $y$-coordinate of the target point.
%   \ccPrecond{The target point is finite in $y$.}}


% ======== left 
\ccMethod{Arr_parameter_space left_boundary_in_x () const;}
  {returns whether the $x$-coordinate of the left curve end is finite or
   whether it is $\pm\infty$.}
\ccGlue
\ccMethod{Arr_parameter_space left_boundary_in_y () const;}
  {returns whether the $y$-coordinate of the left curve end is finite or
   whether it is $\pm\infty$.}
\ccGlue
\ccMethod{const Point_2& left() const;}
  {returns the left point of the arc.
   \ccPrecond{The left is finite in both $x$ and $y$.}}
\ccGlue
\ccMethod{Algebraic_real_1 left_x() const;}
  {returns the $x$-coordinate of the left point.
   \ccPrecond{The left point is finite in $x$.}}
%\ccGlue
%\ccMethod{Algebraic_real_1 left_y() const;}
%  {returns the $y$-coordinate of the left point.
%   \ccPrecond{The left point is finite in $y$.}}


% ======== right 
\ccMethod{Arr_parameter_space right_boundary_in_x () const;}
  {returns whether the $x$-coordinate of the right curve end is finite or
   whether it is $\pm\infty$.}
\ccGlue
\ccMethod{Arr_parameter_space right_boundary_in_y () const;}
  {returns whether the $y$-coordinate of the right curve end is finite or
   whether it is $\pm\infty$.}
\ccGlue
\ccMethod{const Point_2& right() const;}
  {returns the right point of the arc.
   \ccPrecond{The right is finite in both $x$ and $y$.}}
\ccGlue
\ccMethod{Algebraic_real_1 right_x() const;}
  {returns the $x$-coordinate of the right point.
   \ccPrecond{The right point is finite in $x$.}}
%\ccGlue
%\ccMethod{Algebraic_real_1 right_y() const;}
%  {returns the $y$-coordinate of the right point.
%   \ccPrecond{The right point is finite in $y$.}}

\ccMethod{bool is_left_to_right () const;}
  {returns whether the curve is oriented from left to right.}
\end{ccClass}




\subsection*{Class   Arr\_rational\_function\_traits\_2$<$AlgebraicKernel\_d\_1$>$::Point\_2}
\begin{ccClass}{Arr_rational_function_traits_2<AlgebraicKernel_d_1>::Point_2}
\ccCreationVariable{point}

\ccIsModel
\ccc{Assignable}\\
\ccc{CopyConstructible}\\
\ccc{DefaultConstructible}\\
\ccc{EqualityComparable}\\
\ccc{LessThanComparable}

\ccTypes \ccThree{}{xxxxxxxxxxxxxxxxxxxxxx}{x}
\ccTypedef{typedef AlgebraicKernel_d_1::Polynomial_1 Polynomial_1; }{}\ccGlue
\ccTypedef{typedef AlgebraicKernel_d_1::Algebraic_real_1 Algebraic_real_1; }{}\ccGlue
\ccTypedef{typedef AlgebraicKernel_d_1::Bound Bound; }{}

\ccOperations
\ccThree{xxxxxxxxxxxxxxxxxxxxxxxxx}{xxxxxxxxxxxxxxxxxxxxxxxxxxxxxxx}{}
\ccTwo  {xxxxxxxxxxxxxxxxxxxxxxxxxxxxxxxxxxxxxxxxxxxx}{}

\ccMethod{Polynomial_1 numerator () const;}
{returns the numerator of the supporting rational function.}

\ccMethod{Polynomial_1 denominator () const;}
{returns the denominator of the supporting rational function.}

\ccMethod{std::pair<double,double> to_double() const;}{
  returns double-approximations of the x- and y-coordinates. 
}

\ccMethod{Algebraic_real_1 x() const;}
{returns the $x$-coordinate of the point.}
\ccThree{xxxxxxxxxxxxxxxxxxxxxxxx}{xxxxx}{}

\ccMethod{Algebraic_real_1 y() const;}
{obtains the y-coordinates of the point. {\bf Attention:} As described above, points are not stored by their y-coordinate in \ccc{Algebraic_real_1} representation. In fact, this representation must be computed on demand, and might become quite costly for points defined by high-degree polynomials. Therefore, it is recommended to avoid calls to this function as much as possible.  
}

\ccMethod{std::pair<Bound,Bound>  approximate_absolute_x(int a) const;}{
Computes a pair $p$ approximating the $x$-coordinate with 
respect to the given absolute precision $a$. 
\ccPostcond{$p.first \leq x \leq p.second $}
\ccPostcond{$p.second - p.first \leq  2^{-a} $}
}

\ccMethod{std::pair<Bound,Bound> approximate_absolute_y(int a) const;}{
Computes a pair $p$ approximating the $y$-coordinate with 
respect to the given absolute precision $a$. 
\ccPostcond{$p.first \leq y \leq p.second $}
\ccPostcond{$p.second - p.first \leq  2^{-a} $}
}

\ccMethod{std::pair<Bound,Bound> approximate_relative_x(int r) const;}{
Computes a pair $p$ approximating the $x$-coordinate with 
respect to the given relative precision $r$. 
\ccPostcond{$p.first \leq x \leq p.second $}
\ccPostcond{$p.second - p.first \leq  2^{-r}|x| $}
}

\ccMethod{std::pair<Bound,Bound> approximate_relative_y(int r) const;}{
Computes a pair $p$ approximating the $y$-coordinate with 
respect to the given relative precision $r$. 
\ccPostcond{$p.first \leq y \leq p.second $}
\ccPostcond{$p.second - p.first \leq  2^{-r}|y| $}
}

\end{ccClass}
\subsection*{Class   Arr\_rational\_function\_traits\_2$<$AlgebraicKernel\_d\_1$>$::Construct\_curve\_2}
\begin{ccClass}{Arr_rational_function_traits_2<AlgebraicKernel_d_1>::Construct_curve_2}
\ccCreationVariable{construct}

Functor to construct a \ccc{Curve_2}. To enable caching the class is not default 
constructible and must be obtained via the function \ccc{construct_curve_2_object()}, 
which is a member of the traits. 

\ccIsModel
\ccc{Assignable}\\
\ccc{CopyConstructible}\\
\ccc{AdaptableBinaryFunction}\\
\ccc{AdaptableUnaryFunction}

\ccTypes \ccThree{}{xxxxxxxxxxxxxxxxxxxxxx}{x}
\ccTypedef{typedef AlgebraicKernel_d_1::Polynomial_1 Polynomial_1; }{}\ccGlue
\ccTypedef{typedef AlgebraicKernel_d_1::Algebraic_real_1 Algebraic_real_1; }{}\ccGlue
\ccTypedef{typedef Arr_rational_function_traits_2<AlgebraicKernel_d_1>::Curve_2 result_type; }{}

\ccTypedef{typedef Polynomial_1 argument_type; }{}\ccGlue
\ccTypedef{typedef Polynomial_1 first_argument_type; }{}\ccGlue
\ccTypedef{typedef Polynomial_1 second_argument_type; }{}

\ccOperations
\ccThree{xxxxxxxxxxx}{xxxxx}{}
\ccTwo  {xxxxxxxxxxxxxxxxxxxxxxxxxxxxxxxxxxxxxxxxxxxx}{}


\ccMethod{ Curve_2 operator()(Polynomial_1 P) const;}
{
Constructs a curve representing   the polynomial function $y = P(x)$.
}


\ccMethod{Curve_2 operator()(Polynomial_1 P, const Algebraic_real_1& x, bool right) const;}
{
Constructs a curve representing   the polynomial function $y = P(x)$.
The function is defined over the interval $[x,+\infty)$ if $right$ is true and $(-\infty,x]$ otherwise. 
}

\ccMethod{Curve_2 operator()(Polynomial_1 P, const Algebraic_real_1& lower, const Algebraic_real_1& upper) const;}
{
Constructs a curve representing   the polynomial function $y = P(x)$.
The function is defined over the interval $[lower,upper]$. 
}


\ccMethod{ Curve_2 operator()(Polynomial_1 P, Polynomial_1 Q) const;}
{
Constructs a curve representing   the rational function $y = P(x)/Q(x)$.
}


\ccMethod{Curve_2 operator()(Polynomial_1 P, Polynomial_1 Q, const Algebraic_real_1& x, bool right) const;}
{
Constructs a curve representing   the rational function $y = P(x)/Q(x)$.
The function is defined over the interval $I=[x,+\infty)$ if $right$ is true and $I=(-\infty,x]$ otherwise.
}

\ccMethod{Curve_2 operator()(Polynomial_1 P, Polynomial_1 Q, const Algebraic_real_1& lower, const Algebraic_real_1& upper) const;}
{
Constructs a curve representing   the rational function $y = P(x)/Q(x)$.
The function is defined over the interval $I=[lower,upper]$. 
}


\end{ccClass}

\subsection*{Class   Arr\_rational\_function\_traits\_2$<$AlgebraicKernel\_d\_1$>$::Construct\_x\_monotone\_curve\_2}
\begin{ccClass}{Arr_rational_function_traits_2<AlgebraicKernel_d_1>::Construct_x_monotone_curve_2}


Functor to construct a \ccc{X_monotone_curve_2}. To enable caching the class is not default 
constructible and must be obtained via the function \ccc{construct_x_monotone_curve_2_object()}, 
which is a member of the traits. 


\ccCreationVariable{construct}
\ccIsModel
\ccc{Assignable}\\
\ccc{CopyConstructible}\\
\ccc{AdaptableBinaryFunction}\\
\ccc{AdaptableUnaryFunction}

\ccTypes \ccThree{}{xxxxxxxxxxxxxxxxxxxxxx}{x}
\ccTypedef{typedef AlgebraicKernel_d_1::Polynomial_1 Polynomial_1; }{}\ccGlue
\ccTypedef{typedef AlgebraicKernel_d_1::Algebraic_real_1 Algebraic_real_1; }{}\ccGlue
\ccTypedef{typedef Arr_rational_function_traits_2<AlgebraicKernel_d_1>::X_monotone_curve_2 result_type; }{}\ccGlue
\ccTypedef{typedef Polynomial_1 argument_type; }{}\ccGlue
\ccTypedef{typedef Polynomial_1 first_argument_type; }{}\ccGlue
\ccTypedef{typedef Polynomial_1 second_argument_type; }{}

\ccOperations
\ccThree{xxxxxxxxxxxxxxxxxxxx}{xx}{}
\ccTwo  {xxxxxxxxxxxxxxxxxxxxxxxxxxxxxxxxxxxxxxxxxxxx}{}


\ccMethod{ X_monotone_curve_2 operator()(Polynomial_1 P) const;}
{
Constructs an $x$-monotone curve supported by  the polynomial function $y = P(x)$.
}


\ccMethod{X_monotone_curve_2 operator()(Polynomial_1 P, const Algebraic_real_1& x, bool right) const;}
{
Constructs an $x$-monotone curve supported by  the polynomial function $y = P(x)$.
The function is defined over the interval $[x,+\infty)$ if $right$ is true and $(-\infty,x]$ otherwise. 
}

\ccMethod{X_monotone_curve_2 operator()(Polynomial_1 P, const Algebraic_real_1& lower, const Algebraic_real_1& upper); const }
{
Constructs an $x$-monotone curve supported by  the polynomial function $y = P(x)$.
The function is defined over the interval $[lower,upper]$. 
}


\ccMethod{ X_monotone_curve_2 operator()(Polynomial_1 P, Polynomial_1 Q); const }
{
Constructs an $x$-monotone curve supported by  the rational function $y = P(x)/Q(x)$.
\ccPrecond{$Q$ has no real roots.}
}


\ccMethod{X_monotone_curve_2 operator()(Polynomial_1 P, Polynomial_1 Q, const Algebraic_real_1& x, bool right); const}
{
Constructs an $x$-monotone curve supported by  the rational function $y = P(x)/Q(x)$.
The function is defined over the interval $I=[x,+\infty)$ if $right$ is true and $I=(-\infty,x]$ otherwise.
\ccPrecond{$Q$ has no real roots in the interior of $I$.}
}

\ccMethod{X_monotone_curve_2 operator()(Polynomial_1 P, Polynomial_1 Q, const Algebraic_real_1& lower, const Algebraic_real_1& upper); const
}{
Constructs an $x$-monotone curve supported by  the rational function $y = P(x)/Q(x)$.
The function is defined over the interval $I=[lower,upper]$. 
\ccPrecond{$Q$ has no real roots in the interior of $I$.}
}

\end{ccClass}



%\subsection*{Class   Arr\_rational\_function\_traits\_2$<$AlgebraicKernel\_d\_1$>$::Importer}
%\begin{ccClass}{Arr_rational_function_traits_2<AlgebraicKernel_d_1>::Importer}
%\ccCreationVariable{import}

%Functor to import a \ccc{Curve_2} or \ccc{X_monotone_curve_2} from a stream. 
%To enable caching the class is not default constructible and must be obtained 
%via the function \ccc{Importer_object()}, which is a member of the traits. 

%\ccIsModel
%\ccc{Assignable}\\
%\ccc{CopyConstructible}

%\ccTypes
%\ccThree{}{xxxxxxxxxxxxxxxx}{x}
%\ccTypedef{typedef Arr_rational_function_traits_2<AlgebraicKernel_d_1>::Curve_2 Curve_2;}{}\ccGlue
%\ccTypedef{typedef Arr_rational_function_traits_2<AlgebraicKernel_d_1>::X_monotone_curve_2 X_monotone_curve_2;}{}

%\ccOperations
%\ccThree{xxxxxxxxxxxxx}{}{xxxxxxxxxxxxxxxxxxxxxxxxxxx}
%\ccMethod{
%  std::istream& operator() (
%  std::istream& is, const Curve_2& curve);}
%{
%  Imports a \ccc{Curve_2} from the given input stream.
%}
%\ccMethod{
%  std::istream& operator() (
%  std::istream& is, const X_monotone_curve_2& curve);}
%{
%  Imports an \ccc{X_monotone_curve_2} from the given input stream.
%}

%\end{ccClass}

%\subsection*{Class   Arr\_rational\_function\_traits\_2$<$AlgebraicKernel\_d\_1$>$::Exporter}
%\begin{ccClass}{Arr_rational_function_traits_2<AlgebraicKernel_d_1>::Exporter}
%\ccCreationVariable{export}

%Functor to export a \ccc{Curve_2} or \ccc{X_monotone_curve_2} to a stream.

%\ccIsModel
%\ccc{Assignable}\\
%\ccc{CopyConstructible}

%\ccTypes
%\ccThree{}{xxxxxxxxxxxxxxxx}{x}
%\ccTypedef{typedef Arr_rational_function_traits_2<AlgebraicKernel_d_1>::Curve_2 Curve_2;}{}\ccGlue
%\ccTypedef{typedef Arr_rational_function_traits_2<AlgebraicKernel_d_1>::X_monotone_curve_2 X_monotone_curve_2;}{}

%\ccOperations
%\ccThree{xxxxxxxxxxxxx}{}{xxxxxxxxxxxxxxxxxxxxxxxxxxx}
%\ccMethod{
% std::ostream& operator() (
% std::ostream& os, 
% const Curve_2& curve);}
%{
%  Exports a \ccc{Curve_2} 
%  into the given output stream.
%}

%\ccMethod{
% std::ostream& operator() (
% std::ostream& os, 
% const X_monotone_curve_2& curve);}
%{
%  Exports an \ccc{X_monotone_curve_2} 
%  into the given output stream.
%}
%
%\end{ccClass}





\end{ccRefClass}
\ccRefPageEnd


