% Reference manual page: ArrangementTraits.tex
% Package: Arrangement_2

\ccRefPageBegin
\begin{ccRefConcept}{ArrangementOpenBoundaryTraits_2}

\ccDefinition
% ===========
Several predicates are required to handle $x$-monotone curves that
approach infinity and thus approach the boundary of the parameter
space. These predicates are sufficient to handle not only curves
embedded in an unbounded parameter space, but also curves embedded
in a bounded parameter space with open boundaries. Models of the
concept \ccRefName{} handle curves that approach the boundary of a
parameter space. This concept refines the concept
\ccc{ArrangementBasicTraits_2}. The arrangement template
instantiated with a traits class that models this concept can handle
$x$-monotone curves that are unbounded in any direction. The concept
\ccRefName{}, nontheless, also supports planar $x$-monotone curves
that reach the boundary of an open yet bounded parameter space.

An $x$-monotone curve may be \emph{closed}, in which case its endpoints
are representable as \ccc{Point_2} objects, or \emph{open} at the
boundary of the parameter space. It can have one open end and one
closed end (e.g., a ray). The nature of the $x$-monotone curves,
whether they are expected to be closed or not at any one of the four
boundary-sides, is conveyed through the definition of the four nested
types \ccc{Left_side_category}, \ccc{Right_side_category},
\ccc{Bottom_side_category}, and \ccc{Top_side_category}. If some curves
handled by a model of the concept \ccRefName{} are expected to be open
on the left, the nested type \ccc{Left_side_category} must be convertible
to \ccc{Arr_open_side_tag}. Similarly, if some curves handled by the
concept are expected to be open on the right, open at the bottom, or
open at the top, the corresponding nested type must be convertible to
\ccc{Arr_open_side_tag}. A model of the concept \ccRefName{} must have
all the four categories convertible to
\ccc{Arr_open_side_tag}.\footnote{We intend to introduce more concepts
that require only a subset of the categories to be convertible to
\ccc{Arr_open_side_tag}.} In this case the \dcel{} of the arrangement
instantiated with the model is initialized with an implicit bounding
rectangle. When the parameter space is bounded, it is the exact
geometric embedding of the implicit bounding rectangle.

%% An arrangement that supports unbounded $x$-monotone curves maintains
%% an implicit bounding rectangle embedded in the \dcel{} structure.

\ccRefines
\ccc{ArrangementBasicTraits_2}

%% \ccTypes
%% % ======
%% \ccNestedType{Curve_2}{models the concept \ccc{ArrTraits::Curve_2}.}

\ccHeading{Categories}
% ==============
\ccNestedType{Left_side_category}
  {Must be convertible to either \ccc{Arr_oblivious_side_tag} or
        \ccc{Arr_open_side_tag}.}
\ccNestedType{Bottom_side_category}
  {Must be convertible to either \ccc{Arr_oblivious_side_tag} or
        \ccc{Arr_open_side_tag}.}
\ccNestedType{Top_side_category}
  {Must be convertible to either \ccc{Arr_oblivious_side_tag} or
        \ccc{Arr_open_side_tag}.}
\ccNestedType{Right_side_category}
  {Must be convertible to either \ccc{Arr_oblivious_side_tag} or
        \ccc{Arr_open_side_tag}.}

\ccHeading{Functor Types}
% =======================
\ccThree{Parameter_space_in_x_2}{}{\hspace*{14cm}}
\ccThreeToTwo
\ccNestedType{Parameter_space_in_x_2}%
  {models the concept \ccc{ArrTraits::ParameterSpaceInX_2}.
  Required only if the traits class supports unbounded curves that
  approach the left or the right sides (the \ccc{Left_side_category}
  or the \ccc{Right_side_category} categories are convertible to
  \ccc{Arr_open_side_tag}).}

\ccThree{Compare_y_near_boundary_2}{}{\hspace*{14cm}}
\ccThreeToTwo
\ccNestedType{Compare_y_near_boundary_2}%
  {models the concept \ccc{ArrTraits::CompareYNearBoundary_2}.
  Required only if the traits class supports unbounded curves that
  approach the left or the right sides (the \ccc{Left_side_category}
  or the \ccc{Right_side_category} categories are convertible to
  \ccc{Arr_open_side_tag}).}

\ccThree{Parameter_space_in_y_2}{}{\hspace*{14cm}}
\ccThreeToTwo
\ccNestedType{Parameter_space_in_y_2}%
  {models the concept \ccc{ArrTraits::ParameterSpaceInY_2}.
  Required only if the traits class supports unbounded curves that
  approach the bottom or the top sides (the \ccc{Bottom_side_category}
  or the \ccc{Top_side_category} categories are convertible to
  \ccc{Arr_open_side_tag}).}

\ccThree{Compare_x_at_limit_2}{}{\hspace*{14cm}}
\ccThreeToTwo
\ccNestedType{Compare_x_at_limit_2}%
  {models the concept \ccc{ArrTraits::CompareXAtLimit_2}.
  Required only if the traits class supports unbounded curves that
  approach the bottom or the top sides (the \ccc{Bottom_side_category}
  or the \ccc{Top_side_category} categories are convertible to
  \ccc{Arr_open_side_tag}).}
  
\ccThree{Compare_x_near_limit_2}{}{\hspace*{14cm}}
\ccThreeToTwo
\ccNestedType{Compare_x_near_limit_2}%
  {models the concept \ccc{ArrTraits::CompareXNearLimit_2}.
  Required only if the traits class supports unbounded curves that
  approach the bottom or the top sides (the \ccc{Bottom_side_category}
  or the \ccc{Top_side_category} categories are convertible to
  \ccc{Arr_open_side_tag}).}
  
\ccCreationVariable{traits}
% \ccCreation
% ===========

\ccHeading{Accessing Functor Objects}
% ===================================
\ccMethod{Parameter_space_in_x_2 parameter_space_in_x_2_object() const;} {}
\ccMethod{Compare_y_near_boundary_2 compare_y_near_boundary_2_object() const;} {}
\ccMethod{Parameter_space_in_y_2 parameter_space_in_y_2_object() const;} {}
\ccMethod{Compare_x_at_limit_2 compare_x_at_limit_2_object() const;} {}
\ccMethod{Compare_x_near_limit_2 compare_x_near_limit_2_object() const;} {}

\ccHasModels
% ==========
\ccc{CGAL::Arr_linear_traits_2<Kernel>}\\
\ccc{CGAL::Arr_rational_function_traits_2<AlgebraicKernel_d_1>}\\
\ccc{CGAL::Arr_algebraic_segment_traits_2<Coefficient>}\\
\ccc{CGAL::Arr_curve_data_traits_2<Tr,XData,Mrg,CData,Cnv>}\\
\ccc{CGAL::Arr_consolidated_curve_data_traits_2<Traits,Data>}

\ccSeeAlso
% ========
\ccc{ArrangementBasicTraits_2}\lcTex{
  (\ccRefPage{ArrangementBasicTraits_2})}\\
\ccc{ArrangementXMonotoneTraits_2}\lcTex{
  (\ccRefPage{ArrangementXMonotoneTraits_2})}\\
\ccc{ArrangementLandmarkTraits_2}\lcTex{
  (\ccRefPage{ArrangementLandmarkTraits_2})}\\
\ccc{ArrangementTraits_2}\lcTex{
  (\ccRefPage{ArrangementTraits_2})}

\end{ccRefConcept}
\ccRefPageEnd

%%%%%%%% Functors %%%%%%%%
% ========================

%%%%%%%% ParameterSpaceInX_2
% ==========================
\ccRefPageBegin
\begin{ccRefConcept}{ArrTraits::ParameterSpaceInX_2}
\ccRefines{AdaptableBinaryFunction}

\ccHasModels\ccc{ArrangementOpenBoundaryTraits_2::Parameter_space_in_x_2}

\ccCreationVariable{fo}

\ccMethod{Arr_parameter_space operator()(const ArrTraits::X_monotone_curve_2& xcv,
                                         Arr_curve_end ce);}{%

  Given an $x$-monotone curve \ccc{xcv} and an enumeration \ccc{ce}
  that specifies either the minimum or the maximum end of the curve,
  determines the location of the curve end along the $x$-dimension.
  The variable \ccc{xcv} identifies the parametric curve
  $C(t) = (X(t),Y(t))$ defined over an open or half-open interval with
  endpoints~$0$ and~$1$. The enumeration \ccc{ce} identifies an open
  end $d \in \{0,1\}$ of $C$. Formally, determines whether
  $\lim_{t \rightarrow d} X(t)$ evaluates to $b_l$, $b_r$, or a value
  in between, where $b_l$ and $b_r$ are the $x$-coordinates of the
  left and right boundaries of the parameter space, respectively.
  Returns \ccc{ARR_LEFT_BOUNDARY}, \ccc{ARR_RIGHT_BOUNDARY}, or
  \ccc{ARR_INTERIOR}, accordingly.
  \ccPrecond{If \ccc{ArrTraits::Left_side_category} is not convertible to
    \ccc{Arr_open_side_tag} then \ccc{ce} $\neq$ \ccc{ARR_MIN_END}.}
  \ccPrecond{If \ccc{ArrTraits::Right_side_category} is not convertible to
    \ccc{Arr_open_side_tag} then \ccc{ce} $\neq$ \ccc{ARR_MAX_END}.}
  \ccPostcond{If \ccc{ce} = \ccc{ARR_MIN_END} then the result is either
    \ccc{ARR_LEFT_BOUNDARY} or \ccc{ARR_INTERIOR}.}
  \ccPostcond{If \ccc{ce} = \ccc{ARR_MAX_END} then the result is either
    \ccc{ARR_RIGHT_BOUNDARY} or \ccc{ARR_INTERIOR}.}}
\end{ccRefConcept}
\ccRefPageEnd

%%%%%%%% CompareYNearBoundary_2
% =============================
\ccRefPageBegin
\begin{ccRefConcept}{ArrTraits::CompareYNearBoundary_2}

\ccRefines{AdaptableTernaryFunction}

\ccHasModels\ccc{ArrangementOpenBoundaryTraits_2::Compare_y_near_boundary_2}

\ccCreationVariable{fo}

\ccMethod{Comparison_result operator()(const ArrTraits::X_monotone_curve_2& xcv1,
                                       const ArrTraits::X_monotone_curve_2& xcv2,
                                       Arr_curve_end ce);}{%
  Given two $x$-monotone curves \ccc{xcv1} and \ccc{xcv2} and an
  enumeration \ccc{ce} that specifies either the minimum or the maximum
  ends of the curves, compares the $y$-coordinate of the curves near
  their respective ends. Returns \ccc{SMALLER}, \ccc{EQUAL}, or
  \ccc{LARGER} accordingly. More precisely, compares the
  $y$-coordinates of the vertical projection of a point $p$ onto
  \ccc{xcv1} and \ccc{xcv2}. If \ccc{ce} is \ccc{ARR_MIN_END}, the
  predicate \ccc{Parameter_space_in_x_2} evaluates to
  \ccc{ARR_LEFT_BOUNDARY} when applied to \ccc{xcv1} and \ccc{ce} and
  when applied to \ccc{xcv2} and \ccc{ce}. In this case $p$ is
  located far to the left, such that the result is invariant under
  a translation of $p$ farther to the left. If \ccc{ce} is
  \ccc{ARR_MAX_END}, the predicate \ccc{Parameter_space_in_x_2}
  evaluates to \ccc{ARR_RIGHT_BOUNDARY} when applied to \ccc{xcv1} and
  \ccc{ce} and when applied to \ccc{xcv2} and \ccc{ce}. In that case
  $p$ is located far to the right in a similar manner.
  \ccPrecond{If \ccc{ArrTraits::Left_side_category} is not convertible to
    \ccc{Arr_open_side_tag} then \ccc{ce} $\neq$ \ccc{ARR_MIN_END}.}
  \ccPrecond{If \ccc{ArrTraits::Right_side_category} is not convertible to
    \ccc{Arr_open_side_tag} then \ccc{ce} $\neq$ \ccc{ARR_MAX_END}.}
  \ccPrecond{%
     \ccc{parameter_space_in_x_2}(\ccc{xcv2}, \ccc{ce}) =
     \ccc{parameter_space_in_x_2}(\ccc{xcv1}, \ccc{ce}).}
  \ccPrecond{\ccc{parameter_space_in_x_2}(\ccc{xcv1}, \ccc{ce}) $\neq$
     \ccc{ARR_INTERIOR}.}
  \ccPrecond{If \ccc{parameter_space_in_x_2}(\ccc{xcv1}, \ccc{ce}) =
     \ccc{ARR_LEFT_BOUNDARY} then \ccc{ce} = \ccc{ARR_MIN_END}.}
  \ccPrecond{If \ccc{parameter_space_in_x_2}(\ccc{xcv1}, \ccc{ce}) =
     \ccc{ARR_RIGHT_BOUNDARY} then \ccc{ce} = \ccc{ARR_MAX_END}.}}
\end{ccRefConcept}
\ccRefPageEnd

%%%%%%%% ParameterSpaceInY_2
% ==========================
\ccRefPageBegin
\begin{ccRefConcept}{ArrTraits::ParameterSpaceInY_2}
\ccRefines{AdaptableBinaryFunction}

\ccHasModels\ccc{ArrangementOpenBoundaryTraits_2::Parameter_space_in_y_2}

\ccCreationVariable{fo}

\ccMethod{Arr_parameter_space operator()(const ArrTraits::X_monotone_curve_2& xcv,
                                         Arr_curve_end ce);}{%
  Given an $x$-monotone curve \ccc{xcv} and an enumeration \ccc{ce}
  that specifies either the minimum or the maximum end of the curve,
  determines the location of the curve end along the $y$-dimension.
  The variable \ccc{xcv} identifies the parametric curve
  $C(t) = (X(t),Y(t))$ defined over an open or half-open interval with
  endpoints~$0$ and~$1$. The enumeration \ccc{ce} identifies an open
  end $d \in \{0,1\}$ of $C$. Formally, determines whether
  $\lim_{t \rightarrow d} Y(t)$ evaluates to $b_b$, $b_t$, or a value
  in between, where $b_b$ and $b_t$ are the $y$-coordinates of the
  bottom and top boundaries of the parameter space, respectively.
  Returns \ccc{ARR_BOTTOM_BOUNDARY}, \ccc{ARR_TOP_BOUNDARY}, or
  \ccc{ARR_INTERIOR}, accordingly.
  \ccPostcond{If \ccc{ArrTraits::Bottom_side_category} is not convertible to
    \ccc{Arr_open_side_tag} then the result is not \ccc{ARR_BOTTOM_BOUNDARY}.}
  \ccPostcond{If \ccc{ArrTraits::Top_side_category} is not convertible to
    \ccc{Arr_open_side_tag} then the result is not \ccc{ARR_TOP_BOUNDARY}.}}
\end{ccRefConcept}
\ccRefPageEnd

%%%%%%%% CompareXAtLimit_2
% ================================
\ccRefPageBegin
\begin{ccRefConcept}{ArrTraits::CompareXAtLimit_2}
\ccRefines{AdaptableFunctor}

\ccHasModels\ccc{ArrangementOpenBoundaryTraits_2::Compare_x_at_limit_2}

\ccCreationVariable{fo}

\ccMethod{Comparison_result operator()(const ArrTraits::Point_2& p,
                                       const ArrTraits::X_monotone_curve_2& xcv,
                                       Arr_curve_end ce);}{%
  Given a point \ccc{p}, an $x$-monotone curve \ccc{xcv}, and an
  enumeration \ccc{ce} that specifies either the minimum or the
  maximum end of the curve where the curve has a vertical asymptote,
  compares the $x$-coordinate of \ccc{p} and the $x$-coordinate of the
  limit of the curve at its specificed end. The variable \ccc{xcv}
  identifies the parametric curve $C(t) = (X(t),Y(t))$ defined over an
  open or half-open interval with endpoints~$0$ and~$1$. The
  enumeration \ccc{ce} identifies an open end $d \in \{0,1\}$ of $C$.
  Formally, compares the $x$-coordinate of \ccc{p} and
  $\lim_{t \rightarrow d} X(t)$. Returns \ccc{SMALLER}, \ccc{EQUAL}, or
  \ccc{LARGER} accordingly.  
  \ccPrecond{\ccc{parameter_space_in_y_2}(\ccc{xcv}, \ccc{ce}) $\neq$
    \ccc{ARR_INTERIOR}.}
  \ccPrecond{If the parameter space is unbounded, $C$ has a vertical
    asymptote at its $d$-end; that is,
    \ccc{parameter_space_in_x_2}(\ccc{xcv}, \ccc{ce}) = \ccc{ARR_INTERIOR}.}}
%
\ccMethod{Comparison_result operator()(const ArrTraits::X_monotone_curve_2& xcv1,
                                       Arr_curve_end ce1,
                                       const ArrTraits::X_monotone_curve_2& xcv2,
                                       Arr_curve_end ce2);}{%
  Given two $x$-monotone curves \ccc{xcv1} and \ccc{xcv2} and two
  indices \ccc{ce1} and \ccc{ce2} that specify either the minimum
  or the maximum ends of \ccc{xcv1} and \ccc{xcv2}, respectively,
  where the curves have vertical asymptotes, compares the
  $x$-coordinates of the limits of the curves at their specificed
  ends. The variables \ccc{xcv1} and \ccc{xcv2} identify the
  parametric curves $C_1(t) = (X_1(t),Y_1(t))$ and
  $C_2(t) = (X_2(t),Y_2(t))$, respectively, defined over open or
  half-open intervals with endpoints~$0$ and~$1$. The indices
  \ccc{ce1} and \ccc{ce2} identify open ends $d_1 \in \{0,1\}$ and
  $d_2 \in \{0,1\}$ of $C_1$ and $C_2$, respectively. Formally,
  compares $\lim_{t \rightarrow d_1} X_1(t)$ and
  $\lim_{t \rightarrow d_2} X_2(t)$. Returns \ccc{SMALLER}, \ccc{EQUAL},
  or \ccc{LARGER} accordingly. 
  \ccPrecond{%
    \ccc{parameter_space_in_y_2}(\ccc{xcv1}, \ccc{ce1}) $\neq$
    \ccc{ARR_INTERIOR}.}
  \ccPrecond{%
    \ccc{parameter_space_in_y_2}(\ccc{xcv2}, \ccc{ce2}) $\neq$
    \ccc{ARR_INTERIOR}.}
  \ccPrecond{If the parameter space is unbounded, $C_1$ has a vertical
    asymptote at its respective end; that is,\\ 
    \ccc{parameter_space_in_x_2}(\ccc{xcv1}, \ccc{ce1}) =
    \ccc{ARR_INTERIOR}.}
  \ccPrecond{If the parameter space is unbounded, $C_2$ has a vertical
    asymptote at its respective end; that is,\\ 
    \ccc{parameter_space_in_x_2}(\ccc{xcv2}, \ccc{ce2}) =
    \ccc{ARR_INTERIOR}.}}
\end{ccRefConcept}
\ccRefPageEnd

%%%%%%%% CompareXNearLimit_2
% =============================
\ccRefPageBegin
\begin{ccRefConcept}{ArrTraits::CompareXNearLimit_2}
\ccRefines{AdaptableTernaryFunction}

\ccHasModels\ccc{ArrangementOpenBoundaryTraits_2::Compare_x_near_limit_2}

\ccCreationVariable{fo}

\ccMethod{Comparison_result operator()(const ArrTraits::X_monotone_curve_2& xcv1,
                                       const ArrTraits::X_monotone_curve_2& xcv2,
                                       Arr_curve_end ce);}{%
  Given two $x$-monotone curves \ccc{xcv1} and \ccc{xcv2} and an
  enumeration \ccc{ce} that specifies either the minimum ends or the
  maximum ends of the curves where the curves have a vertical
  asymptote, compares the $x$-coordinate of the curves near their
  respective ends. Returns \ccc{SMALLER}, \ccc{EQUAL}, or \ccc{LARGER}
  accordingly. More precisely, compares the $x$-coordinates of the
  horizontal projection of a point $p$ onto \ccc{xcv1} and \ccc{xcv2}.
  If \ccc{xcv1} and \ccc{xcv2} approach the bottom boundary-side, $p$
  is located far to the bottom, such that the result is invariant
  under a translation of $p$ farther to the bottom. If \ccc{xcv1}
  and \ccc{xcv2} approach the top boundary-side, $p$ is located far
  to the top in a similar manner.
  \ccPrecond{The $x$-coordinates of the limits of the curves at their
    respective ends are equal. That is,\\
    \ccc{compare_x_at_limit_2}(\ccc{xcv1}, \ccc{xcv2}, \ccc{ce}) =
    \ccc{EQUAL}.}
  \ccPrecond{%
    \ccc{parameter_space_in_y_2}(\ccc{xcv1}, \ccc{ce}) =
    \ccc{parameter_space_in_y_2}(\ccc{xcv2}, \ccc{ce}).}
  \ccPrecond{\ccc{parameter_space_in_y_2}(\ccc{xcv1}, \ccc{ce}) $\neq$
    \ccc{ARR_INTERIOR}.}}
\end{ccRefConcept}
\ccRefPageEnd
