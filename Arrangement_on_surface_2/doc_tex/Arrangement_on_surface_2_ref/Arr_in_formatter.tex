% +------------------------------------------------------------------------+
% | Reference manual page: Arr_in_formatter.tex
% +------------------------------------------------------------------------+
% | 
% | Package: Arrangement_2
% | 
% +------------------------------------------------------------------------+

\ccRefPageBegin

\begin{ccRefConcept}{ArrangementInputFormatter}

A model for the \ccRefName\ concept supports a set of functions that enable
reading an arrangement from an input stream using a specific format.

\ccTypes
%=======

\ccNestedType{Arrangement_2}{the type of arrangement to input.}

\ccTypedef{typedef typename Arrangement_2::Point_2 Point_2;}
    {the point type.}
\ccGlue
\ccTypedef{typedef typename Arrangement_2::X_monotone_curve_2 X_monotone_curve_2;}
    {the $x$-monotone curve type.}

\ccTypedef{typedef typename Arrangement_2::Size Size;}{}
\ccGlue
\ccTypedef{typedef typename Arrangement_2::Vertex_handle Vertex_handle;}{}
\ccGlue
\ccTypedef{typedef typename Arrangement_2::Halfedge_handle Halfedge_handle;}{}
\ccGlue
\ccTypedef{typedef typename Arrangement_2::Face_handle Face_handle;}{}

\ccCreation
\ccCreationVariable{inf}
%=======================
    
\ccConstructor{Arr_in_formatter();}
   {default constructor.}

\ccConstructor{Arr_in_formatter (std::istream& is);}
   {constructs a formatter that reads from \ccc{is}.}

\ccMethod{void set_in (std::istream& is);}
   {directs \ccVar\ to read from \ccc{is}.}

\ccAccessFunctions
%=================

\ccMethod{std::istream& in ();}
   {returns the stream that \ccVar\ reads from.
    \ccPrecond{\ccVar\ is directed to a valid output stream.}}

\ccHeading{Formatted Input Functions}
%=====================================

\ccMethod{void read_arrangement_begin ();}
    {reads a message indicating the beginning of the arrangement.}
\ccGlue
\ccMethod{void read_arrangement_end ();}
    {reads a message indicating the end of the arrangement.}

\ccMethod{Size read_size (const char *label = NULL);}
    {reads a size value, which is supposed to be preceeded by the given label.}

\ccMethod{void read_vertices_begin();}
    {reads a message indicating the beginning of the vertex records.}
\ccGlue
\ccMethod{void read_vertices_end();}
    {reads a message indicating the end of the vertex records.}

\ccMethod{void read_edges_begin();}
    {reads a message indicating the beginning of the edge records.}
\ccGlue
\ccMethod{void read_edges_end();}
    {reads a message indicating the end of the edge records.}

\ccMethod{void read_faces_begin();}
    {reads a message indicating the beginning of the face records.}
\ccGlue
\ccMethod{void read_faces_end();}
    {reads a message indicating the end of the face records.}

\ccMethod{void read_vertex_begin();}
    {reads a message indicating the beginning of a single vertex record.}
\ccGlue
\ccMethod{void read_vertex_end();}
    {reads a message indicating the end of a single vertex record.}

\ccMethod{std::size_t read_vertex_index ();}
    {reads and returns a vertex index.}

\ccMethod{void read_point (Point_2& p);}
    {reads a point.}

\ccMethod{void read_vertex_data (Vertex_handle v);}
    {reads an auxiliary vertex-data object and associates it with the vertex \ccc{v}.}

\ccMethod{void read_edge_begin();}
    {reads a message indicating the beginning of a single edge record.}
\ccGlue
\ccMethod{void read_edge_end();}
    {reads a message indicating the end of a single edge record.}

\ccMethod{std::size_t read_halfedge_index ();}
    {reads and returns halfedge index.}

\ccMethod{void read_x_monotone_curve (X_monotone_curve_2& c);}
    {reads an $x$-monotone curve.}

\ccMethod{void read_halfegde_data (Halfedge_handle he);}
    {reads an auxiliary halfedge-data object and associates it with the halfedge \ccc{he}.}

\ccMethod{void read_face_begin();}
    {reads a message indicating the beginning of a single face record.}
\ccGlue
\ccMethod{void read_face_end();}
    {reads a message indicating the end of a single face record.}

\ccMethod{void read_outer_ccb_begin();}
    {reads a message indicating the beginning of the outer CCB of the current face.}
\ccGlue
\ccMethod{void read_outer_ccb_end();}
    {reads a message indicating the end of the outer CCB of the current face.}

\ccMethod{void read_holes_begin();}
    {reads a message indicating the beginning of the container of holes inside the
     current face.}
\ccGlue
\ccMethod{void read_holes_end();}
    {reads a message indicating the end of the container of holes inside the
     current face.}

\ccMethod{void read_inner_ccb_begin();}
    {reads a message indicating the beginning of an inner CCB of the current face.}
\ccGlue
\ccMethod{void read_inner_ccb_end();}
    {reads a message indicating the end of an inner CCB of the current face.}

\ccMethod{void read_ccb_halfedges_begin();}
    {reads a message indicating the beginning a connected component boundary.}
\ccGlue
\ccMethod{void read_ccb_halfedges_end();}
    {reads a message indicating the end of a connected component boundary.}

\ccMethod{void read_isolated_vertices_begin();}
    {reads a message indicating the beginning of the container of isolated vertices
     inside the current face.}
\ccGlue
\ccMethod{void read_isolated_vertices_end();}
    {reads a message indicating the end of the container of isolated vertices inside
     the current face.}

\ccMethod{void read_face_data (Face_handle f);}
    {reads an auxiliary face-data object and associates it with the face \ccc{f}.}

\ccHasModels
%===========

\ccc{Arr_text_formatter<Arrangement>}\lcTex{
        (\ccRefPage{CGAL::Arr_text_formatter<Arrangement>})}\\
\ccc{Arr_face_extended_text_formatter<Arrangement>}\lcTex{
        (\ccRefPage{CGAL::Arr_face_extended_text_formatter<Arrangement>})}\\
\ccc{Arr_extended_dcel_text_formatter<Arrangement>}\lcTex{
        (\ccRefPage{CGAL::Arr_extended_dcel_text_formatter<Arrangement>})}

\end{ccRefConcept}

\ccRefPageEnd

