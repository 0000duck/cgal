\ccRefPageBegin

\begin{ccRefConcept}{SweepLineVisitor_2}

\ccDefinition
% ===========

This concept defines the set of methods that are needed to be implemented.



\ccTypes
% ======

\ccNestedType{Traits_2}{the type of the associated traits class.}
\ccNestedType{Sweep_line_2}{the type of the associated sweep-line class.}

\ccTypedef{typedef typename Traits_2::Point_2 Point_2;}
    {the point type.}
\ccGlue
\ccTypedef{typedef typename Traits_2::X_monotone_curve_2 X_monotone_curve_2;}
    {the $x$-monotone curve type.}

\ccTypedef{typedef typename Sweep_line_2::Event Event;}{the event type.}
\ccGlue
\ccTypedef{typedef typename Sweep_line_2::Subcurve Subcurve;}{the subcurve type.}
\ccGlue
\ccTypedef{typedef typename Sweep_line_2::Status_line_iterator Status_line_iterator;}{the iterator type of the status line,
whose value type is \ccc{Subcurve*}. }
\ccGlue


\ccHeading{Notifications from the sweep line}
%=========================================================


\ccMethod{void update_event (Event* e,
                     Subcurve* sc1,
                     Subcurve* sc2,
                     bool is_new);}
    { Update the event to be the intersection point of two subcurves \ccc{sc1} and \ccc{sc2}.\ccc{is_new} is \ccc{true} iff a new
      event is created.}
\ccGlue

\ccMethod{void update_event (Event* e, Subcurve* sc);}
    {Update the event to be on the interior of \ccc{sc}.}
   
   
\ccMethodvoid found_overlap(Subcurve* sc1, Subcurve* sc2, Subcurve* sc);}
    {Notify about an overlap subcurve that was found, where \ccc{sc1} and \ccc{sc2} are the originating subcurves
     and \ccc{sc} is the new subcurve that stores the overlap.}
  
\ccHasModels
%===========
\ccc{CGAL::Sweep_line_empty_visitor<Traits>}

\end{ccRefConcept}

\ccRefPageEnd
