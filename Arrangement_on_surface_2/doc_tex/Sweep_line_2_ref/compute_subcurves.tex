\ccRefPageBegin

\begin{ccRefFunction}{compute_subcurves}

\ccInclude{CGAL/Sweep_line_2_algorithms.h}

\ccGlobalFunction{template <class InputIterator, class OutputIterator>
OutputIterator compute_subcurves (InputIterator curves_begin,
                                  InputIterator curves_end,
                                  OutputIterator subcurves,
                                  bool multiple_overlaps = false);}
{Given a range of curves, compute all $x$-monotone subcurves that are pairwise
 disjoint in their interior, as induced by the input curves.
 If the flag \ccc{multiple_overlaps} is \ccc{true}, then a subcurve that
 represents an overlap of $k$ input curves is reported $k$ times; otherwise,
 each subcurve is reported only once. The value-type of
 \ccc{InputIterator} is a curve type, and the value-type of
 \ccc{OutputIterator} is an $x$-monotone curve type.}
\ccGlue
\ccGlobalFunction{template <class InputIterator, class OutputIterator, class Traits>
OutputIterator compute_subcurves (InputIterator curves_begin,
                                  InputIterator curves_end,
                                  OutputIterator subcurves,
                                  bool multiple_overlaps = false,
                                  Traits traits = Default_traits());}
{Given a range of curves, compute all $x$-monotone subcurves that are pairwise
 disjoint in their interior, as induced by the input curves.
 If the flag \ccc{multiple_overlaps} is \ccc{true}, then a subcurve that
 represents an overlap of $k$ input curves is reported $k$ times; otherwise,
 each subcurve is reported only once. The \ccc{Traits} type must be a model
 of the \ccc{ArrangementTraits_2} concept, such that the value-type of
 \ccc{InputIterator} is \ccc{Traits::Curve_2}, and the value-type of
 \ccc{OutputIterator} is \ccc{Traits::X_monotone_curve_2}.}
 
\end{ccRefFunction}

\ccRefPageEnd

