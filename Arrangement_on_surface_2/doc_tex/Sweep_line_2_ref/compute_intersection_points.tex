\ccRefPageBegin

\begin{ccRefFunction}{compute_intersection_points}

\ccInclude{CGAL/Sweep_line_2_algorithms.h}

\ccGlobalFunction{template <class InputIterator, class OutputIterator>
OutputIterator compute_intersection_points (InputIterator curves_begin,
                                            InputIterator curves_end,
                                            OutputIterator points,
                                            bool report_endpoints = false);}
{Given a range of curves, compute all intersection points between two (or more)
 input curves. When the flag \ccc{report_endpoints} is \ccc{true}, 
  this function reports all the curve endpoints as well. If a curve
  endpoint is also an intersection point, it is reported once (regardless
  of the value of the \ccc{report_endpoints} flag). The value-type of
  \ccc{InputIterator} is a curve type and the value-type of \ccc{OutputIterator}
  is a point type. The output points are reported in an increasing
  $xy$-lexicographical order.}
\ccGlue
\ccGlobalFunction{template <class InputIterator, class OutputIterator, class Traits>
OutputIterator compute_intersection_points (InputIterator curves_begin,
                                            InputIterator curves_end,
                                            OutputIterator points,
                                            bool report_endpoints = false,
                                            Traits traits);}
{Given a range of curves, compute all intersection points between two (or more)
 input curves. When the flag \ccc{report_endpoints} is \ccc{true}, 
  this function reports all the curve endpoints as well. If a curve
  endpoint is also an intersection point, it is reported once (regardless
  of the value of the \ccc{report_endpoints} flag). The \ccc{Traits} type
  must be a model of the \ccc{ArrangementTraits_2} concept, such that the
  value-type of \ccc{InputIterator} is \ccc{Traits::Curve_2}, and the
  value-type of \ccc{OutputIterator} is \ccc{Traits::Point_2}.
 The output points are reported in an increasing $xy$-lexicographical order.}
\end{ccRefFunction}

\ccRefPageEnd

