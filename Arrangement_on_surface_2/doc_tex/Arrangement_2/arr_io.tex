\section{Input/Output Streams\label{arr_sec:io}}
%===============================

In some cases, one would like to save an arrangement object
constructed by some application, so that later on it can be
restored. In other cases one would like to create nice drawings that
represent arrangements constructed by some application. These
drawings can be hard printed or displayed on a computer screen.

\subsection{Input/Output Stream\label{arr_ssec:io_stream}}
%--------------------------------------------
Consider an arrangement that represents a very complicated geographical 
map, and assume that there are various applications that need to answer 
point-location queries on this map. Naturally, you can store the set of 
curves that induces the arrangement, but this implies that you would need 
to construct the arrangement from scratch each time you need to reuse it.
A more efficient solution is to write the arrangement to a file in a format 
that other applications can read.

This package provides an {\em inserter} (the \ccc{<<} operator) and an 
{\em extractor} (the \ccc{>>} operator) for the 
\ccc{Arrangement_2<Traits,Dcel>} class that inserts and an arrangement
object into an output stream and extracts an arrangement object from an
input stream respectively. The arrangement is written using a simple
predefined ASCII format that encodes the arrangement topology, as well
as all geometric entities associated with vertices and edges.

The ability to use the input/output operators, requires that the 
\ccc{Point_2} type and the \ccc{X_monotone_curve_2} type defined by the 
traits class both support the \ccc{<<} and \ccc{>>} operators. The 
\ccc{Arr_conic_traits_2} class (see Section~\ref{arr_ssec:tr_conic}), the 
\ccc{Arr_rational_arc_traits_2} class (see Section~\ref{arr_ssec:tr_ratfunc}),
and the \ccc{Arr_linear_traits_2} class (see Section~\ref{arr_ssec:tr_segs})
currently do not provide these operators for the geometric types they define. 
Thus, only arrangements of line segments or of polylines can be written or 
read.

The following example constructs the arrangement depicted in 
Figure~\ref{arr_fig:ex_5} and writes it to an output file. It also
demonstrates how to re-read the arrangement from a file:

\ccIncludeExampleCode{Arrangement_2/io.cpp}

\begin{ccAdvanced}
\subsection{Arrangements with Auxiliary Data\label{arr_ssec:arr_io_aux_data}}
%--------------------------------------------

The inserter and extractor both ignore any auxiliary data stored with
the arrangement features. Thus, they are ideal for arrangements
instantiated using the \ccc{Arr_default_dcel} class.
However, as explained in Section~\ref{arr_sec:ex_dcel}, one can easily
extend the arrangement faces by using the \ccc{Arr_face_extended_dcel}
template, or extend all \dcel\ records by using the \ccc{Arr_extended_dcel}
template. In such cases, it might be crucial that the auxiliary data fields
are written to the file and read from there.

The arrangement package includes the free functions
\ccc{write(arr, os, formatter)}, which writes the arrangement \ccc{arr}
to an output stream \ccc{os}, and \ccc{read(arr, os, formatter)}, which
reads the arrangement \ccc{arr} from an input stream \ccc{is}. Both
operations are performed using a \ccc{formatter} object, which defines
the I/O format. The package contains three formatter classes:
\begin{itemize}
\item \ccc{Arr_text_formatter<Arrangement>} defines a simple textual
I/O format for the arrangement topology and geometry, disregarding any
auxiliary data that may be associated with the arrangement features.
This is the default formatter used by the arrangement inserter and the
arrangement extractor, as defined above.
%
\item \ccc{Arr_face_extended_text_formatter<Arrangement>} operates on
arrangements whose \dcel\ representation is based on the
\ccc{Arr_face_extended_dcel<Traits,FaceData>} class (see
Section~\ref{arr_ssec:ex_dcel_face}). It supports reading and writing
the auxiliary data objects stored with the arrangement faces provided
that the \ccc{FaceData} class supports an inserter and an extractor.
%
\item \ccc{Arr_extended_dcel_text_formatter<Arrangement>} operates on
arrangements whose \dcel\ representation is based on the
\ccc{Arr_extended_dcel<Traits,VertexData,HalfedgeData,FaceData>} class
(see Section~\ref{arr_ssec:ex_dcel_all}). It supports reading and writing
the auxiliary data objects stored with the arrangement vertices, edges
and faces, provided that the \ccc{VertexData}, \ccc{HalfedgeData} and
\ccc{FaceData} classed all have inserters and extractors.
\end{itemize}

The following example constructs the same arrangement as the
example \ccc{dcel_extension} does
(see Section~\ref{arr_ssec:ex_dcel_all}), depicted in
Figure~\ref{arr_fig:ex_20}, and writes it to an output file. It also
demonstrates how to re-read the arrangement from a file:

\ccIncludeExampleCode{Arrangement_2/dcel_extension_io.cpp}

You may develop your own own formatter classes --- models of the
\ccc{ArrangementInputFormatter} and \ccc{ArrangementOutputFormatter}
concepts, as defined in the Reference Manual.
Doing so, you can define other I/O formats, such as an XML-based
format or a binary format.
\end{ccAdvanced}

\subsection{Arrangements with Curve History\label{arr_ssec:arr_io_hist}}
%-------------------------------------------

Section~\ref{arr_sec:arr_with_hist} introduces the
\ccc{Arrangement_with_history_2<Traits,Dcel>} class, which saves the
set of curves inducing an arrangement and maintains the relations between
these curves and the edges they induce. Naturally, when reading or writing 
an arrangement-with-history instance we would like this information to be
saved to the output stream or restored from the input stream alongside
with the basic arrangement structure.

The arrangement package supplies an inserter and an extractor for the
\ccc{Arrangement_with_history_2<Traits,Dcel>} class. The arrangement is
represented using a simple predefined ASCII format. An object of the
\ccc{Arrangement_with_history_2<Traits,Dcel>} type can be saved and 
restored, as long as the \ccc{Curve_2} type defined by the traits 
class --- as well as the \ccc{Point_2} type and the \ccc{X_monotone_curve_2} 
types --- support the \ccc{<<} and\ccc{>>} operators.

The following example constructs the same arrangement as example
\ccc{curve_history} does
(see Section~\ref{arr_ssec:arr_hist_ex}), depicted in 
Figure~\ref{arr_fig:ex_24}, and writes it to an output file. It also
demonstrates how to re-read the arrangement-with-history from a file:

\ccIncludeExampleCode{Arrangement_2/io_curve_history.cpp}

\begin{ccAdvanced}
The arrangement package also includes the free functions
\ccc{write(arr, os, formatter)} and \ccc{read(arr, os, formatter)} that
operate on a given arrangement-with-history instance \ccc{arr}.
Both functions are parameterized by a \ccc{formatter} object, which define
the I/O format. The package contains a template called,
\ccc{Arr_with_hist_text_formatter<ArranagmentFormatter>}, which extends
an arrangement formatter class (see Section~\ref{arr_ssec:arr_io_aux_data})
and defines a simple textual input/output format.
\end{ccAdvanced}

\subsection{Output \qt{}-widget Stream}\label{arr_ssec:qt_stream}
%--------------------------------------------
You can display the drawings of arrangements in a graphical window
powered by \qt\ interactively. All you need to do is follow the
guidelines for handling \ccc{Qt_widget} objects, and apply the
{\em inserter}, which inserts an arrangement into a \ccc{Qt_widget}
stream to complete the drawing.

The ability to use this output operator, requires that the
\ccc{Point_2} and \ccc{X_monotone_curve_2} types defined by the
traits class both support the \ccc{<<} operator that inserts the
respective geometric object into a \ccc{Qt_widget} stream. The
\ccc{Arr_rational_arc_traits_2} class (see
Section~\ref{arr_ssec:tr_ratfunc}) and the \ccc{Arr_linear_traits_2}
class (see Section~\ref{arr_ssec:tr_segs}) currently do not provide
this operator for the geometric types they define. Thus, only
arrangements of line segments, polylines, or conic ards can be written.
The \ccc{<<} operator for polylines and conic arcs defined in
\ccc{CGAL/IO/Qt_widget_Polyline_2.h} and
\ccc{CGAL/IO/Qt_widget_Conic_arc_2.h}
must be explicitly included, to insert polylines and conic arcs
respectively into \ccc{Qt_widget} streams.

\subsection{Output Postscript Stream\label{arr_ssec:ps_stream}}
%--------------------------------------------
You can create nice drawings of arrangements in Postscript format, which
you can insert into your publications, using the utility class 
\ccc{Arr_postscript_file_stream}. This class is derived from the 
\ccc{Postscript_file_stream} utility class, which in turn is derived from
the \leda\ class \ccc{ps_file}. Note that you must have \leda\ installed
in order to use this feature. 
This package provides an {\em inserter} (the \ccc{<<} operator) for
the \ccc{Arrangement_2<Traits,Dcel>} class that inserts an arrangement
object into a Postscript output stream.

The \ccc{Arr_postscript_file_stream}
class allows you to specify the color of points and curves separately
in addition to the functionality of the based class. 

The ability to use this output operator, requires that the
\ccc{Point_2} and \ccc{X_monotone_curve_2} types defined by the
traits class both support the \ccc{<<} operator that inserts the
respective geometric object into a Postscript stream. The
\ccc{Arr_conic_traits_2} class (see Section~\ref{arr_ssec:tr_conic}), the
\ccc{Arr_rational_arc_traits_2} class (see Section~\ref{arr_ssec:tr_ratfunc}),
and the \ccc{Arr_linear_traits_2} class (see Section~\ref{arr_ssec:tr_segs})
currently do not provide this operator for the geometric types they define.
Thus, only arrangements of line segments or of polylines can be written.
The \ccc{<<} operator for polylines defined in
\ccc{CGAL/IO/Polyline_postscript_file_sctream.h} must be explicitly
included, to insert polylines into Postscript streams.
