% +------------------------------------------------------------------------+
% | Reference manual page: Arr_dcel_vertex.tex
% +------------------------------------------------------------------------+
% | 
% | Package: Arrangement_2
% | 
% +------------------------------------------------------------------------+

\ccRefPageBegin

\begin{ccRefConcept}{ArrangementDcelVertex}

\ccDefinition

A vertex record in a \dcel\ data structure. A vertex is always associated
with a point. However, the vertex record only stores a pointer to the
associated point, and the actual \ccc{Point} object is stored elsewhere.

A vertex usually has several halfedges incident to it, such that it is
possible to access one of these halfedges directly and to traverse all
incident halfedges around the vertex. However, the \dcel\ may also contain
isolated vertices that have no incident halfedges. In this case, the vertex
stores an isolated vertex-information record, indicating the face that
contains this vertex in its interior.

\ccCreationVariable{v}

\ccTypes
%=======

\ccNestedType{Halfedge}{the corresponding \dcel\ halfedge type.}
\ccGlue
\ccNestedType{Isolated_vertex}{the corresponding \dcel\ isolated
                               vertex-information type.}

\ccNestedType{Point}{the point type associated with the vertex.} 

\ccCreation
%==========

\ccConstructor{Arr_dcel_vertex();}
   {default constructor.}

\ccMethod{void assign (const Self& other);}
   {assigns \ccVar{} with the contents of the \ccc{other} vertex.}

\ccHtmlNoLinksFrom{   % to avoid linkage of Vertex, etc to HDS::Vertex

\ccAccessFunctions
%=================

All functions below also have \ccc{const} counterparts, returning non-mutable
pointers or references:

\ccMethod{bool is_isolated() const;}
    {returns whether the vertex is isolated (has no incident halfedges).}

\ccMethod{Halfedge* halfedge();}
    {returns an incident halfedge that has \ccVar{} as its target.
     \ccPrecond{\ccVar{} is {\em not} an isolated vertex.}}

\ccMethod{Isolated_vertex* isolated_vertex();}
    {returns the isolated vertex-information record.
     \ccPrecond{\ccVar{} is an isolated vertex.}}

\ccMethod{bool has_null_point () const;}
    {returns whether the vertex is not associated with a valid point (i.e. it
     lies at infinity).}

\ccMethod{Point& point();}
    {returns the associated point.
     \ccPrecond{\ccVar() is associated with a valid point.}}

\ccMethod{Boundary_type boundary_in_x () const;}
    {return whether the vertex has an infinite $x$-coordinate.}

\ccMethod{Boundary_type boundary_in_y () const;}
    {return whether the vertex has an infinite $y$-coordinate.}

\ccModifiers
%===========

\ccMethod{void set_halfedge (Halfedge* e);}
    {sets the incident halfedge, marking \ccVar{} as a regular vertex.}

\ccMethod{void set_isolated_vertex (Isolated_vertex* iv);}
    {sets the isolated vertex-information record, marking \ccVar{}
     as an isolated vertex.}

\ccMethod{void set_point (Point* p);}
    {sets the associated point.}

\ccMethod{void set_boundary (Boundary_type inf_x, Boundary_type inf_y);}
    {sets \ccVar{} as a vertex at infinity.
     \ccPrecond{Either \ccc{inf_x} or \ccc{inf_y} is not \ccc{NO_BOUNDARY}.}}

} % ccHtmlNoLinksFrom

\ccSeeAlso

\ccc{ArrangementDcel}\lcTex{ 
     (\ccRefPage{ArrangementDcel})}\\
\ccc{ArrangementDcelHalfedge}\lcTex{
     (\ccRefPage{ArrangementDcelHalfedge})}\\
\ccc{ArrangementDcelIsolatedVertex}\lcTex{
     (\ccRefPage{ArrangementDcelIsolatedVertex})}

\end{ccRefConcept}  

\ccRefPageEnd

