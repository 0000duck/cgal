% +------------------------------------------------------------------------+
% | Reference manual page: ArrangementPointLocation_2.tex
% +------------------------------------------------------------------------+
% | 
% | Package: Arrangement_2
% | 
% +------------------------------------------------------------------------+

\ccRefPageBegin

\begin{ccRefConcept}{ArrangementPointLocation_2}

\ccDefinition
%============

A model of the \ccRefName\ concept can be attached to an \ccc{Arrangement_2}
instance and answer point-location queries on this arrangement. Namely, given
a \ccc{Arrangement_2::Point_2} object, representing a point in the plane,
it returns the arrangement cell containing it. In the general case, the
query point is contained inside an arrangement face, but in degenerate
situations it may lie on an edge or coincide with an arrangement vertex.

\ccTypes
%=======

\ccNestedType{Arrangement_2}{the associated arrangement type.}

\ccNestedType{Point_2}{equivalent to \ccc{Arrangement_2::Point_2}.}

\ccCreation
\ccCreationVariable{pl}
%======================

\ccConstructor{ArrangementPointLocation_2();}{default constructor.}

\ccConstructor{ArrangementPointLocation_2 (const Arrangement_2& arr);}
  {constructs a point-location object \ccVar{} attached to the given
   arrangement \ccc{arr}.}

\ccQueryFunctions
%================

\ccMethod{Object locate (const Point_2& q) const;}
  {locates the arrangement cell that contains the query point \ccc{q}
   and returns a handle for this cell.
   The function returns an \ccc{Object} instance that wraps either of the
   following types:
   \parbox[t]{\ccwComment}{
   \begin{itemize}
   \item \ccc{Arrangement_2::Face_const_handle}, in case \ccc{q} is
         contained inside an arrangement face;
   \item \ccc{Arrangement_2::Halfedge_const_handle}, in case \ccc{q} lies
         on an arrangement edge;
   \item \ccc{Arrangement_2::Vertex_const_handle}, in case \ccc{q} coincides
         with an arrangement vertex.
   \end{itemize}}
   \ccPrecond{\ccVar{} is attached to a valid arrangement instance.}}

\ccOperations
%============

\ccMethod{void attach (const Arrangement_2& arr);}
  {attaches \ccVar{} to the given arrangement \ccc{arr}.}

\ccMethod{void detach ();}
  {detaches \ccVar{} from the arrangement it is currently attached to.}

\ccHasModels
%===========

\ccc{Arr_naive_point_location<Arrangement>}\\
\ccc{Arr_walk_along_a_line_point_location<Arrangement>} \\
\ccc{Arr_trapezoid_ric_point_location<Arrangement>}\\
\ccc{Arr_landmarks_point_location<Arrangement,Generator>}\\

\end{ccRefConcept}

\ccRefPageEnd
