\begin{ccRefConcept}{AlgebraicKernelForCircles::ConstructPolynomial_1_2}

\ccCreationVariable{fo}

A model \ccVar\ of this type must provide:

\ccMethod{AlgebraicKernelForCircles::Polynomial_1_2
	operator()(const AlgebraicKernelForCircles::RT &a,
		const AlgebraicKernelForCircles::RT &b,
		const AlgebraicKernelForCircles::RT &c);}
{Constructs polynomial \ccc{ax+by+c}.}

\ccSeeAlso

\ccRefIdfierPage{CircularKernel::ConstructLine_2}\\
\ccRefIdfierPage{CircularKernel::GetEquation}

\end{ccRefConcept}

\begin{ccRefConcept}{AlgebraicKernelForCircles::ConstructPolynomialForCircles_2_2}

\ccCreationVariable{fo}

A model \ccVar\ of this type must provide:

\ccMethod{AlgebraicKernelForCircles::PolynomialForCircles_2_2
	operator()(const AlgebraicKernelForCircles::FT a,
		const AlgebraicKernelForCircles::FT b,
		const AlgebraicKernelForCircles::FT rsq);}
{Constructs polynomial \ccc{(x-a)^2 + (y-b)^2 - rsq}.}

\ccSeeAlso

\ccRefIdfierPage{CircularKernel::ConstructCircle_2}\\
\ccRefIdfierPage{CircularKernel::GetEquation}

\end{ccRefConcept}

\begin{ccRefConcept}{AlgebraicKernelForCircles::Solve}

\ccDefinition

\ccCreationVariable{fo}

A model \ccVar\ of this type must provide:

\ccMethod{template < class OutputIterator >
    OutputIterator
    operator()(const AlgebraicKernelForCircles::Polynomial_1_2 &p1,
	     const AlgebraicKernelForCircles::Polynomial_1_2 &p2,
	     OutputIterator res);}
{Copies in the output iterator the common roots of \ccc{p1} and \ccc{p2}, 
with their multiplicity, as objects of type 
\ccc{std::pair< AlgebraicKernelForCircles::Root_for_circles_2_2, int>}.} 

\ccMethod{template < class OutputIterator >
    OutputIterator
    operator()(const AlgebraicKernelForCircles::Polynomial_1_2 &p1,
	     const AlgebraicKernelForCircles::Polynomial_for_circles_2_2 &p2,
	     OutputIterator res);}
{Same as previous.} 

\ccMethod{template < class OutputIterator >
    OutputIterator
    operator()(const AlgebraicKernelForCircles::Polynomial_for_circles_2_2 &p1,
	     const AlgebraicKernelForCircles::Polynomial_1_2 &p2,
	     OutputIterator res);}
{Same as previous.} 

\ccMethod{template < class OutputIterator >
    OutputIterator
    operator()(const AlgebraicKernelForCircles::Polynomial_for_circles_2_2 &p1,
	     const AlgebraicKernelForCircles::Polynomial_for_circles_2_2 &p2,
	     OutputIterator res);}
{Same as previous.} 

\end{ccRefConcept}
\begin{ccRefConcept}{AlgebraicKernelForCircles::XCriticalPoints}

\ccDefinition

\ccCreationVariable{fo}

A model \ccVar\ of this type must provide:

\ccMethod{template < class OutputIterator >
    OutputIterator
    operator()(const AlgebraicKernelForCircles::Polynomial_for_circles_2_2 &p,
	     OutputIterator res);}
{Copies in the output iterator the \ccc{x}-critical points of polynomial 
\ccc{p}, as objects of type \ccc{AlgebraicKernelForCircles::Root_for_circles_2_2}.} 

\ccMethod{template < class OutputIterator >
    AlgebraicKernelForCircles::Root_for_circles_2_2
    operator()(const AlgebraicKernelForCircles::Polynomial_for_circles_2_2 &p,
	     bool b);}
{Computes the \ccc{x}-critical point with smallest (resp. largest) $x$
  of polynomial \ccc{p} if $b$ is \ccc{true} (resp. \ccc{false}).} 

\ccSeeAlso

\ccRefIdfierPage{AlgebraicKernelForCircles::YCriticalPoints}

\end{ccRefConcept}
\begin{ccRefConcept}{AlgebraicKernelForCircles::YCriticalPoints}

\ccDefinition

\ccCreationVariable{fo}

A model \ccVar\ of this type must provide:

\ccMethod{template < class OutputIterator >
    OutputIterator
    operator()(const AlgebraicKernelForCircles::Polynomial_for_circles_2_2 &p,
	     OutputIterator res);}
{Copies in the output iterator the \ccc{y}-critical points of polynomial 
\ccc{p}, as objects of type \ccc{AlgebraicKernelForCircles::Root_for_circles_2_2}.} 

\ccMethod{template < class OutputIterator >
    AlgebraicKernelForCircles::Root_for_circles_2_2
    operator()(const AlgebraicKernelForCircles::Polynomial_for_circles_2_2 &p,
	     bool i);}
{Computes the \ccc{y}-critical point with smallest (resp. largest) $y$
  of polynomial \ccc{p} if $b$ is \ccc{true} (resp. \ccc{false}).} 

\ccSeeAlso

\ccRefIdfierPage{AlgebraicKernelForCircles::XCriticalPoints}

\end{ccRefConcept}
