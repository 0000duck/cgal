\ccRefChapter{2D Circular Geometry Kernel}

\ccChapterAuthor{Pedro Machado Manh\~{a}es de Castro \and Sylvain Pion \and Monique Teillaud}

%%%%%%%%%%%%%%%%%%%%%%%%%%%%%%%%%%%%%%%%%%%%%%%%%%%%%%%%%
\section{Geometric Concepts}

\ccRefConceptPage{CircularKernel}

	\subsubsection*{Functors} 

\ccRefConceptPage{CircularKernel::ConstructLine_2}\\
\ccRefConceptPage{CircularKernel::ConstructCircle_2}\\
\ccRefConceptPage{CircularKernel::ConstructCircularArcPoint_2}\\
\ccRefConceptPage{CircularKernel::ConstructLineArc_2}\\
\ccRefConceptPage{CircularKernel::ConstructCircularArc_2}

\ccRefConceptPage{CircularKernel::ComputeCircularX_2}\\
\ccRefConceptPage{CircularKernel::ComputeCircularY_2}

\ccRefConceptPage{CircularKernel::ConstructCircularMinVertex_2}\\
\ccRefConceptPage{CircularKernel::ConstructCircularMaxVertex_2}\\
\ccRefConceptPage{CircularKernel::ConstructCircularSourceVertex_2}\\
\ccRefConceptPage{CircularKernel::ConstructCircularTargetVertex_2}

%\footnote{technical remark: the previous functors have a different name 
%``Circular'' because the operators() don't have the same return type 
%as the existing CGAL functors... it would be nice to find a way to avoid 
%this, but I don't know any technique for this.} 

\ccRefConceptPage{CircularKernel::ConstructBbox_2}

\ccRefConceptPage{CircularKernel::CompareX_2}\\
\ccRefConceptPage{CircularKernel::CompareY_2}\\
\ccRefConceptPage{CircularKernel::CompareXY_2}

\ccRefConceptPage{CircularKernel::Equal_2}

\ccRefConceptPage{CircularKernel::CompareYatX_2}\\
\ccRefConceptPage{CircularKernel::CompareYtoRight_2}

\ccRefConceptPage{CircularKernel::HasOn_2}

\ccRefConceptPage{CircularKernel::DoOverlap_2}

\ccRefConceptPage{CircularKernel::DoIntersect_2}

\ccRefConceptPage{CircularKernel::BoundedSide_2}\\
\ccRefConceptPage{CircularKernel::HasOnBoundedSide_2}\\
\ccRefConceptPage{CircularKernel::HasOnUnboundedSide_2}

\ccRefConceptPage{CircularKernel::InXRange_2}
%\ccRefConceptPage{CircularKernel::InYRange_2}

\ccRefConceptPage{CircularKernel::IsVertical_2}

\ccRefConceptPage{CircularKernel::IsXMonotone_2}\\
\ccRefConceptPage{CircularKernel::IsYMonotone_2}

\ccRefConceptPage{CircularKernel::MakeXMonotone_2}\\
\ccRefConceptPage{CircularKernel::MakeXYMonotone_2}

\ccRefConceptPage{CircularKernel::Intersect_2}

\ccRefConceptPage{CircularKernel::Split_2}

\ccRefConceptPage{CircularKernel::GetEquation}

%%%%%%%%%%%%%%%%%%%%%%%%%%%%%%%%%%%%%%%%%%%%%%%%%%%%%%%%%
\section{Geometric Kernels and Classes}

	\subsubsection*{Kernels}

\ccRefIdfierPage{CGAL::Circular_kernel_2<Kernel,AlgebraicKernelForCircles>} 
\\
\ccRefIdfierPage{CGAL::Exact_circular_kernel_2}
%\ccRefIdfierPage{CGAL::Lazy_curved_kernel<??>}%\\
%\ccRefIdfierPage{CGAL::Filtered_hexagon_curved_kernel<CircularKernel>}\\
%\ccRefIdfierPage{CGAL::Filtered_bbox_curved_kernel<CircularKernel>}

	\subsubsection*{Points}
\ccRefIdfierPage{CGAL::Circular_arc_point_2<CircularKernel>}

	\subsubsection*{Arcs}
\ccRefIdfierPage{CGAL::Circular_arc_2<CircularKernel>}\\
\ccRefIdfierPage{CGAL::Line_arc_2<CircularKernel>}

%%%%%%%%%%%%%%%%%%%%%%%%%%%%%%%%%%%%%%%%%%%%%%%%%%%%%%%%%
\section{Geometric Global Functions}	

\ccRefIdfierPage{CGAL::compare_x}\\
\ccRefIdfierPage{CGAL::compare_y}\\
\ccRefIdfierPage{CGAL::compare_xy}

\ccRefIdfierPage{CGAL::compare_y_at_x}\\
\ccRefIdfierPage{CGAL::compare_y_to_right}\\

\ccRefIdfierPage{CGAL::has_in_x_range}\\
\ccRefIdfierPage{CGAL::has_on}

\ccRefIdfierPage[Kernel::do_intersect]{CGAL::do_intersect}\\
\ccRefIdfierPage[Kernel::intersection]{CGAL::intersection}

\ccRefIdfierPage{CGAL::x_extremal_point}\\
\ccRefIdfierPage{CGAL::y_extremal_point}\\
\ccRefIdfierPage{CGAL::x_extremal_points}\\
\ccRefIdfierPage{CGAL::y_extremal_points}

\ccRefIdfierPage{CGAL::make_x_monotone}\\
\ccRefIdfierPage{CGAL::make_xy_monotone}


%%%%%%%%%%%%%%%%%%%%%%%%%%%%%%%%%%%%%%%%%%%%%%%%%%%%%%%%%
\section{Algebraic Concepts}

\ccRefConceptPage{AlgebraicKernelForCircles}

%\ccRefConceptPage{AlgebraicKernelForCircles::Polynomial_1_2}\footnote{General
%remark about the suffix \_d\_v: \_d stands
%for the degree of the polynomials and the algebraic numbers, and
%\_v stands for the number of variables, which is analogous to the
%dimension for CGAL geometric objects.}\\
%\ccRefConceptPage{AlgebraicKernelForCircles::PolynomialForCircles_2_2}\\
%\ccRefConceptPage{RootOf_2}\\
%\ccRefConceptPage{AlgebraicKernelForCircles::RootForCircles_2_2}

	\subsubsection*{Functors} 
% \footnote{no \_2 (or \_2\_2) for functors ????????? problem of compatibility 
% with CK and the current kernel. On the other hand, allows to have only
% one functor for several types of arguments}

\ccRefConceptPage{AlgebraicKernelForCircles::ConstructPolynomial_1_2}\\
\ccRefConceptPage{AlgebraicKernelForCircles::ConstructPolynomialForCircles_2_2}

\ccRefConceptPage{AlgebraicKernelForCircles::CompareX}\\
\ccRefConceptPage{AlgebraicKernelForCircles::CompareY}\\
\ccRefConceptPage{AlgebraicKernelForCircles::CompareXY}

\ccRefConceptPage{AlgebraicKernelForCircles::SignAt}

\ccRefConceptPage{AlgebraicKernelForCircles::XCriticalPoints}\\
\ccRefConceptPage{AlgebraicKernelForCircles::YCriticalPoints}

\ccRefConceptPage{AlgebraicKernelForCircles::Solve}


%%%%%%%%%%%%%%%%%%%%%%%%%%%%%%%%%%%%%%%%%%%%%%%%%%%%%%%%%
\section{Algebraic Kernel and Classes}

	\subsubsection*{Kernel}

\ccRefIdfierPage{CGAL::Algebraic_kernel_for_circles_2_2<RT>}

	\subsubsection*{Polynomials} 

\ccRefIdfierPage{CGAL::Polynomial_1_2<RT>}\\
\ccRefIdfierPage{CGAL::Polynomial_for_circles_2_2<FT>}

	\subsubsection*{Roots of Polynomials}

\ccRefIdfierPage{CGAL::Sqrt_extension<NT,ROOT>}\\
\ccRefIdfierPage{CGAL::Root_for_circles_2_2<FT>}

\ccRefIdfierPage{CGAL::Root_of_traits_2<RT>}
