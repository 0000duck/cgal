
\begin{ccRefFunctionObjectConcept}{CircularKernel::ConstructLine_2}
\ccCreationVariable{fo}

\ccRefines

\ccc{Kernel::ConstructLine_2}

A model \ccVar\ of this type must provide:

\ccMemberFunction{CircularKernel::Line_2 operator()
	(CircularKernel::Polynomial_1_2);}
{Constructs a line from an equation.}

\ccSeeAlso

\ccRefIdfierPage{CircularKernel::GetEquation}

\end{ccRefFunctionObjectConcept}
\begin{ccRefFunctionObjectConcept}{CircularKernel::ConstructCircle_2}
\ccCreationVariable{fo}

\ccRefines

\ccc{Kernel::ConstructCircle_2}

A model \ccVar\ of this type must provide:

\ccMemberFunction{CircularKernel::Circle_2 operator()
	(CircularKernel::Polynomial_for_circles_2_2);}
{Constructs a circle from an equation.}

\ccSeeAlso

\ccRefIdfierPage{CircularKernel::GetEquation}

\end{ccRefFunctionObjectConcept}
\begin{ccRefFunctionObjectConcept}{CircularKernel::ConstructCircularArcPoint_2}
\ccCreationVariable{fo}

A model \ccVar\ of this type must provide:

\ccMemberFunction{CircularKernel::Circular_arc_point_2 operator()
	(const CircularKernel::Root_for_circles_2_2 & r);}
{}

\ccMemberFunction{CircularKernel::Circular_arc_point_2 operator()
	(const CircularKernel::Point_2 & p);}
{}

\end{ccRefFunctionObjectConcept}
\begin{ccRefFunctionObjectConcept}{CircularKernel::ConstructLineArc_2}
\ccCreationVariable{fo}

A model \ccVar\ of this type must provide:

\ccMemberFunction{CircularKernel::Line_arc_2 operator()
	(const CircularKernel::Line_2 &l,
	const CircularKernel::Circular_arc_point_2 &p1,
	const CircularKernel::Circular_arc_point_2 &p2);}
{Constructs the line segment supported by \ccc{l}, whose source 
is \ccc{p1} and whose target is \ccc{p2}.
\ccPrecond{\ccc{p1} and \ccc{p2} lie on \ccc{l}.}}

%\ccMemberFunction{CircularKernel::Line_arc_2 operator()
%	(const CircularKernel::Line_2 &l,
%	const CircularKernel::Point_2 &p1,
%	const CircularKernel::Point_2 &p2);}
%{Same.}

\ccMemberFunction{CircularKernel::Line_arc_2 operator()
	(const CircularKernel::Segment_2 &s);}
{}

\ccMemberFunction{CircularKernel::Line_arc_2 operator()
	(const CircularKernel::Point_2 &p1,
	const CircularKernel::Point_2 &p2);}
{}

\ccMemberFunction{CircularKernel::Line_arc_2 operator()
	(const CircularKernel::Line_2 &l,
	const CircularKernel::Circle_2 &c1, bool b1,
	const CircularKernel::Circle_2 &c2, bool b2);}
{Constructs the line segment whose supporting line is \ccc{l}, whose 
source endpoint is the $b_1^{th}$ intersection of \ccc{l} with \ccc{c1}, 
and whose target endpoint is the $b_2^{th}$ intersection of \ccc{l} 
and \ccc{c2}, where intersections are ordered lexicographically.
\ccPrecond{\ccc{l} intersects both \ccc{c1} and \ccc{c2}, and the arc
defined by the intersections has non-zero length.}}

\ccMemberFunction{CircularKernel::Line_arc_2 operator()
	(const CircularKernel::Line_2 &l,
	const CircularKernel::Line_2 &l1,
	const CircularKernel::Line_2 &l2);}
{Same, for intersections defined by lines instead of circles.}

\end{ccRefFunctionObjectConcept}
\begin{ccRefFunctionObjectConcept}{CircularKernel::ConstructCircularArc_2}
\ccCreationVariable{fo}

A model \ccVar\ of this type must provide:

\ccMemberFunction{CircularKernel::Circular_arc_2 operator()
	(const CircularKernel::Circle_2 &c);}
{Constructs an arc from a full circle.}

\ccMemberFunction{CircularKernel::Circular_arc_2 operator()
	(const CircularKernel::Circle_2 &c,
	const CircularKernel::Circular_arc_point_2 &p1,
	const CircularKernel::Circular_arc_point_2 &p2);}
{Construct the circular arc supported by \ccc{c}, that is oriented 
counterclockwise, whose source is \ccc{p1} and whose target is \ccc{p2}.
\ccPrecond{\ccc{p1} and \ccc{p2} lie on \ccc{c}.}}

%\ccMemberFunction{CircularKernel::Circular_arc_2 operator()
%	(const CircularKernel::Circle_2 &c,
%	const CircularKernel::Point_2 &p1,
%	const CircularKernel::Point_2 &p2);}
%{Same.}

\ccMemberFunction{CircularKernel::Circular_arc_2 operator()
	(const CircularKernel::Circle_2 &c, 
	const CircularKernel::Circle_2 &c1, bool b1,
	const CircularKernel::Circle_2 &c2, bool b2);}
{Constructs the unique circular arc that is oriented counterclockwise,
whose supporting circle is \ccc{c}, and whose source endpoint is the
intersection of \ccc{c} and \ccc{c1} with index $b_1$, and whose target
is the intersection of \ccc{c} and \ccc{c2} of index $b_2$, where
intersections are ordered lexicographically.
\ccPrecond{\ccc{c} intersects both \ccc{c1} and \ccc{c2}, and the arc
defined by the intersections has non-zero length.}}

\ccMemberFunction{CircularKernel::Circular_arc_2 operator()
	(const CircularKernel::Circle_2 &c,
	const CircularKernel::Line_2 &l1, bool b1,
 	const CircularKernel::Line_2 &l2, bool b2);}
{Same, for intersections defined by lines instead of circles.}

\end{ccRefFunctionObjectConcept}
\begin{ccRefFunctionObjectConcept}{CircularKernel::ConstructCircularMinVertex_2} 

\ccCreationVariable{fo}

A model \ccVar\ of this type must provide:

\ccMemberFunction{CircularKernel::Circular_arc_point_2 operator()
	(const CircularKernel::Circular_arc_2 & c);}
{Constructs the $x$-minimal vertex of \ccc{c}.
\ccPrecond{The arc \ccc{c} is $x$-monotone.}}

\ccMemberFunction{CircularKernel::Circular_arc_point_2 operator()
	(const CircularKernel::Line_arc_2 & l);}
{Same, for a line segment.}

\end{ccRefFunctionObjectConcept}
\begin{ccRefFunctionObjectConcept}{CircularKernel::ConstructCircularMaxVertex_2} 

\ccCreationVariable{fo}

A model \ccVar\ of this type must provide:

\ccMemberFunction{CircularKernel::Circular_arc_point_2 operator()
	(const CircularKernel::Circular_arc_2 & c);}
{Constructs the $x$-maximal vertex of \ccc{c}.
\ccPrecond{The arc \ccc{c} is $x$-monotone.}}

\ccMemberFunction{CircularKernel::Circular_arc_point_2 operator()
	(const CircularKernel::Line_arc_2 & l);}
{Same, for a line segment.}

\end{ccRefFunctionObjectConcept}
\begin{ccRefFunctionObjectConcept}{CircularKernel::ConstructCircularSourceVertex_2} 

\ccCreationVariable{fo}

A model \ccVar\ of this type must provide:

\ccMemberFunction{CircularKernel::Circular_arc_point_2 operator()
	(const CircularKernel::Circular_arc_2 & c);}
{Constructs the source vertex of \ccc{c}.}

\ccMemberFunction{CircularKernel::Circular_arc_point_2 operator()
	(const CircularKernel::Line_arc_2 & l);}
{Same, for a line segment.}

\end{ccRefFunctionObjectConcept}
\begin{ccRefFunctionObjectConcept}{CircularKernel::ConstructCircularTargetVertex_2} 

\ccCreationVariable{fo}

A model \ccVar\ of this type must provide:

\ccMemberFunction{CircularKernel::Circular_arc_point_2 operator()
	(const CircularKernel::Circular_arc_2 & c);}
{Constructs the target vertex of \ccc{c}.}

\ccMemberFunction{CircularKernel::Circular_arc_point_2 operator()
	(const CircularKernel::Line_arc_2 & l);}
{Same, for a line segment.}

\end{ccRefFunctionObjectConcept}
\begin{ccRefFunctionObjectConcept}{CircularKernel::ConstructBbox_2}

\ccCreationVariable{fo}

A model \ccVar\ of this type must provide operators to construct 
a bounding box of geometric objects:

\ccMemberFunction{CGAL::Bbox_2 operator()
	(const CircularKernel::Circular_arc_point_2 & p);}
{}

\ccMemberFunction{CGAL::Bbox_2 operator()
	(const CircularKernel::Line_arc_2 & l);}
{}

\ccMemberFunction{CGAL::Bbox_2 operator()
	(const CircularKernel::Circular_arc_2 & c);}
{}

\end{ccRefFunctionObjectConcept}

