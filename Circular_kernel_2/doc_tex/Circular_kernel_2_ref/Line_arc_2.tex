\begin{ccRefClass}{Line_arc_2<CircularKernel>}

\ccInclude{CGAL/Line_arc_2.h}

\ccIsModel

\ccc{CircularKernel::LineArc_2}

\ccCreation
\ccCreationVariable{la}

\ccThree{Circular_arc_point_2}{ca.is_x_monotone()}{}
\ccThreeToTwo

\ccConstructor{Line_arc_2(const Line_2<CircularKernel> &l,
		const Circular_arc_point_2<CircularKernel> &p1,
		const Circular_arc_point_2<CircularKernel> &p2)}
{Construct the line segment supported by \ccc{l}, whose source 
is \ccc{p1} and whose target is \ccc{p2}.
\ccPrecond{\ccc{p1} and \ccc{p2} lie on \ccc{l}.}}

\ccConstructor{Line_arc_2(const Line_2<CircularKernel> &l,
		const Point_2<CircularKernel> &p1,
		const Point_2<CircularKernel> &p2)}
{Same.}

\ccConstructor{Line_arc_2(const Segment_2<CircularKernel> &s)}
{}

\ccAccessFunctions

\ccThree{Circular_arc_point_2<CircularKernel>}{ca.is_x_monotone()}{}
\ccThreeToTwo

\ccMethod{Line_2<CircularKernel> supporting_line();}{}

\ccMethod{Circular_arc_point_2<CircularKernel> source();}{}
\ccGlue
\ccMethod{Circular_arc_point_2<CircularKernel> target();}{}

\ccMethod{Circular_arc_point_2<CircularKernel> left();}{}
\ccGlue
\ccMethod{Circular_arc_point_2<CircularKernel> right();}{}

\ccMethod{Bbox_2 bbox() const;}
{Returns a bounding box containing the line segment.}

\ccQueryFunctions

\ccMethod{bool is_vertical();}{}

\ccOperations

\ccFunction{bool operator==(const Line_arc_2<CircularKernel> &la1,
  const Line_arc_2<CircularKernel> &la2);} 
{Test for equality. Two arcs are equal, iff their non-oriented
  supporting lines are equal (i.e. they contain the same set of
  points) and their endpoints are equal.}

\ccFunction{bool operator!=(const Line_arc_2<CircularKernel> &la1,
  const Line_arc_2<CircularKernel> &la2);} 
{Test for non-equality.}

\ccHeading{I/O}

\ccFunction{istream& operator>> (std::istream& is, Line_arc_2 & ca);}{}
\ccGlue
\ccFunction{ostream& operator<< (std::ostream& os, const Line_arc_2 & ca);}{}

The format for input/output is, for each line arc: a \ccc{Line_2} 
(the supporting line) and two \ccc{Circular_arc_point_2} (the two endpoints), 
under the condition that the endpoints are actually lying on the line. 

\ccSeeAlso

\ccRefIdfierPage{CGAL::Circular_arc_point_2<CircularKernel>}\\
\ccRefIdfierPage{CGAL::Circular_arc_2<CircularKernel>}

\end{ccRefClass}

