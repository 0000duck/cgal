\begin{ccRefClass}{Line_arc_2<CircularKernel>}

\ccInclude{CGAL/Line_arc_2.h}

\ccIsModel

\ccc{CircularKernel::LineArc_2}

\ccCreation
\ccCreationVariable{la}

\ccThree{Circular_arc_point_2}{ca.is_x_monotone()}{}
\ccThreeToTwo

\ccConstructor{Line_arc_2(const CircularKernel::Line_2 &l,
		const CircularKernel::Circular_arc_point_2 &p1,
		const CircularKernel::Circular_arc_point_2 &p2)}
{Construct the line segment supported by \ccc{l}, whose source 
is \ccc{p1} and whose target is \ccc{p2}.
\ccPrecond{\ccc{p1} and \ccc{p2} lie on \ccc{l}.}}

\ccConstructor{Line_arc_2(const CircularKernel::Line_2 &l,
		const CircularKernel::Point_2 &p1,
		const CircularKernel::Point_2 &p2)}
{Same.}

\ccConstructor{Line_arc_2(const CircularKernel::Segment_2 &s)}
{}

\ccAccessFunctions

\ccThree{CircularKernel::Circular_arc_point_2}{ca.is_x_monotone()}{}
\ccThreeToTwo

\ccMethod{CircularKernel::Line_2 supporting_line();}{}

\ccMethod{CircularKernel::Circular_arc_point_2 source();}{}
\ccGlue
\ccMethod{CircularKernel::Circular_arc_point_2 target();}{}

\ccMethod{CircularKernel::Circular_arc_point_2 left();}{}
\ccGlue
\ccMethod{CircularKernel::Circular_arc_point_2 right();}{}

\ccQueryFunctions

\ccMethod{bool is_vertical();}{}

\ccHeading{I/O}

\ccFunction{istream& operator>> (std::istream& is, Line_arc_2 & ca);}{}
\ccGlue
\ccFunction{ostream& operator<< (std::ostream& os, const Line_arc_2 & ca);}{}

\ccSeeAlso

\ccRefIdfierPage{CGAL::Circular_arc_point_2<CircularKernel>}\\
\ccRefIdfierPage{CGAL::Circular_arc_2<CircularKernel>}

\end{ccRefClass}

