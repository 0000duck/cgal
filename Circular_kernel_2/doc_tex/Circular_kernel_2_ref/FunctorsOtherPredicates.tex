\begin{ccRefFunctionObjectConcept}{CircularKernel::HasOn_2}

\ccDefinition

To test whether a point lies on a curve.

\ccRefines

\ccc{Kernel::HasOn_2}

\ccCreationVariable{fo}

An object \ccVar\ of this type must provide:

\ccMemberFunction{bool operator()
	(const CircularKernel::Line_2 & l, 
	const CircularKernel::Circular_arc_point_2 &p);}
{For a line.}

\ccMemberFunction{bool operator()
	(const CircularKernel::Circle_2 & c, 
	const CircularKernel::Circular_arc_point_2 &p);}
{For a circle.}

\ccMemberFunction{bool operator()
	(const CircularKernel::Line_arc_2 & l, 
	const CircularKernel::Circular_arc_point_2 &p);}
{For a line arc.}

\ccMemberFunction{bool operator()
	(const CircularKernel::Circular_arc_2 & c, 
	const CircularKernel::Circular_arc_point_2 &p);}
{For a circular arc.}

\end{ccRefFunctionObjectConcept}
\begin{ccRefFunctionObjectConcept}{CircularKernel::DoOverlap_2}

\ccDefinition

Testing whether the interiors of two curves overlap. 

\ccRefines

\ccc{Kernel::DoOverlap_2}

\ccCreationVariable{fo}

An object \ccVar\ of this type must provide:

\ccMemberFunction{bool operator()
	(const CircularKernel::Line_arc_2 & l0,
	const CircularKernel::Line_arc_2 & l1);}
{For two line arcs.}

\ccMemberFunction{bool operator()
	(const CircularKernel::Circular_arc_2 & a0,
	const CircularKernel::Circular_arc_2 & a1);}
{For two circular arcs.}

\end{ccRefFunctionObjectConcept}
\begin{ccRefFunctionObjectConcept}{CircularKernel::DoIntersect_2}

\ccDefinition

Testing whether two curves intersect.

\ccRefines

\ccc{Kernel::DoIntersect_2}

\ccCreationVariable{fo}

An object \ccVar\ of this type must provide:

\ccMemberFunction{bool operator()
	(const Type1 & obj1, const Type2 & obj2);}
{determines if two geometric objects of type Type1 and Type2 intersect or not.}

for all pairs \ccc{Type1} and \ccc{Type2}, where the types 
\ccc{Type1} and \ccc{Type2} can be any of the following:
\begin{itemize}
\item {} \ccc{CircularKernel::Line_2}
\item {} \ccc{CircularKernel::Line_arc_2}
\item {} \ccc{CircularKernel::Circle_2}
\item {} \ccc{CircularKernel::Circular_arc_2}
\end{itemize} 

\ccSeeAlso

\ccRefIdfierPage{CGAL::do_intersect}

\end{ccRefFunctionObjectConcept}
\begin{ccRefFunctionObjectConcept}{CircularKernel::BoundedSide_2}

\ccRefines

\ccc{Kernel::BoundedSide_2}

\ccCreationVariable{fo}

An object \ccVar\ of this type must provide:

\ccMemberFunction{Bounded_side
	operator()
	(const CircularKernel::Circle_2& s, 
	const CircularKernel::Circular_arc_point_2& p);}
{}

\ccSeeAlso

\ccRefIdfierPage{CircularKernel::HasOnBoundedSide_2}\\
\ccRefIdfierPage{CircularKernel::HasOnUnboundedSide_2}

\end{ccRefFunctionObjectConcept}
\begin{ccRefFunctionObjectConcept}{CircularKernel::HasOnBoundedSide_2}

\ccRefines

\ccc{Kernel::HasOnBoundedSide_2}

\ccCreationVariable{fo}

An object \ccVar\ of this type must provide:

\ccMemberFunction{bool
	operator()
	(const CircularKernel::Circle_2& s, 
	const CircularKernel::Circular_arc_point_2& p);}
{}

\ccSeeAlso

\ccRefIdfierPage{CircularKernel::BoundedSide_2}\\
\ccRefIdfierPage{CircularKernel::HasOnUnboundedSide_2}

\end{ccRefFunctionObjectConcept}
\begin{ccRefFunctionObjectConcept}{CircularKernel::HasOnUnboundedSide_2}

\ccRefines

\ccc{Kernel::HasOnUnboundedSide_2}

\ccCreationVariable{fo}

An object \ccVar\ of this type must provide:

\ccMemberFunction{bool
	operator()
	(const CircularKernel::Circle_2& s, 
	const CircularKernel::Circular_arc_point_2& p);}
{}

\ccSeeAlso

\ccRefIdfierPage{CircularKernel::BoundedSide_2}\\
\ccRefIdfierPage{CircularKernel::HasOnBoundedSide_2}

\end{ccRefFunctionObjectConcept}
\begin{ccRefFunctionObjectConcept}{CircularKernel::InXRange_2}

\ccDefinition

To test whether a point lies in the vertical range of a curve.

\ccCreationVariable{fo}

An object \ccVar\ of this type must provide:

\ccMemberFunction{bool operator()
	(const CircularKernel::Line_arc_2 & l,
	const CircularKernel::Circular_arc_point_2 & p);}
{For a line arc.}

\ccMemberFunction{bool operator()
	(const CircularKernel::Circular_arc_2 & c,
	const CircularKernel::Circular_arc_point_2 & p);}
{For a circular arc. \ccPrecond{$c$ is $x$-monotone.}}

\end{ccRefFunctionObjectConcept}
%\begin{ccRefFunctionObjectConcept}{CircularKernel::InYRange_2}

%\ccDefinition

%To test whether a point lies in the horizontal range of a curve.

%\ccCreationVariable{fo}

%An object \ccVar\ of this type must provide:

%\ccMemberFunction{bool operator()
%	(const CircularKernel::Line_arc_2 & l,
%	const CircularKernel::Circular_arc_point_2 & p);}
%{For a line arc.}

%\ccMemberFunction{bool operator()
%	(const CircularKernel::Circular_arc_2 & c,
%	const CircularKernel::Circular_arc_point_2 & p);}
%{For a circular arc. \ccPrecond{$c$ is $y$-monotone.}}

%\ccHasModels

%\ccc{Circular_kernel_2::In_y_range_2;}

%\ccSeeAlso

%\ccRefIdfierPage{CGAL::in_y_range}

%\end{ccRefFunctionObjectConcept}
\begin{ccRefFunctionObjectConcept}{CircularKernel::IsVertical_2}

\ccCreationVariable{fo}

\ccRefines

\ccc{Kernel::IsVertical_2}

An object \ccVar\ of this type must provide:

\ccMemberFunction{bool operator()
	(const CircularKernel::Line_arc_2 & l);}
{For a line arc.}

\ccMemberFunction{bool operator()
	(const CircularKernel::Circular_arc_2 & c);}
{For a circular arc,  always returns \ccc{false}.}

\end{ccRefFunctionObjectConcept}
\begin{ccRefFunctionObjectConcept}{CircularKernel::IsXMonotone_2}

\ccCreationVariable{fo}

An object \ccVar\ of this type must provide:

\ccMemberFunction{bool operator()
	(const CircularKernel::Circular_arc_2 & c);}
{Tests whether the arc is $x$-monotone.}

\ccMemberFunction{bool operator()
	(const CircularKernel::Line_arc_2 & l);}
{For a line arc, always returns \ccc{true}.}

\end{ccRefFunctionObjectConcept}
\begin{ccRefFunctionObjectConcept}{CircularKernel::IsYMonotone_2}

\ccCreationVariable{fo}

An object \ccVar\ of this type must provide:

\ccMemberFunction{bool operator()
	(const CircularKernel::Circular_arc_2 & c);}
{Tests whether the arc is $y$-monotone.}

\ccMemberFunction{bool operator()
	(const CircularKernel::Line_arc_2 & l);}
{For a line arc, always returns \ccc{true}.}

\end{ccRefFunctionObjectConcept}
