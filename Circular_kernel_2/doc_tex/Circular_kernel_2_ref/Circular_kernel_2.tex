\begin{ccRefClass}{Circular_kernel_2<Kernel,AlgebraicKernelForCircles>}

\ccInclude{CGAL/Circular_kernel_2.h}

\ccIsModel

\ccc{CircularKernel}

\ccParameters

The first parameter of the circular kernel must be instantiated with a
model of the \ccc{Kernel} concept. The \ccc{Circular_kernel_2} class
template derives from this first parameter, in order to reuse all
needed functionalities on basic objects provided by a model of the
Kernel concept.

The second parameter, \ccc{AlgebraicKernelForCircles}, is meant to provide the
circular kernel with all the algebraic functionalities required for the
manipulation of algebraic curves. 

\ccInheritsFrom

\ccc{Kernel}

\ccTypes

\ccThree{typedef Circular_arc_point_2<Circular-Kernel>}{Root_of_4xxx}{}
\ccThreeToTwo

The circular kernel uses basic number types of the algebraic kernel:
\ccTypedef{typedef AlgebraicKernelForCircles::RT RT;}{Ring number type.}
\ccTypedef{typedef AlgebraicKernelForCircles::FT FT;}{Field number type.}
In fact, the two number types \ccc{AlgebraicKernelForCircles::RT} and
\ccc{Kernel::RT} must coincide, as well as
\ccc{AlgebraicKernelForCircles::FT} and \ccc{Kernel::FT}.

The following types are available, as well as all the functionality on
them described in the \ccc{CircularKernel} concept. 

\ccTypedef{typedef Line_arc_2<Circular_kernel_2> Line_arc_2;}{}
\ccGlue
\ccTypedef{typedef Circular_arc_2<Circular_kernel_2> Circular_arc_2;}{}
\ccGlue
\ccTypedef{typedef Circular_arc_point_2<Circular_kernel_2> Circular_arc_point_2;}{}

\ccSeeAlso

\ccRefIdfierPage{Kernel}\\
\ccRefIdfierPage{AlgebraicKernelForCircles}\\
\ccRefIdfierPage{CGAL::Exact_circular_kernel_2}

\end{ccRefClass}
