\begin{ccRefFunction}{x_extremal_point}

\ccInclude{CGAL/global_functions_circular_kernel_2.h}

\ccFunction{Circular_arc_point_2<CircularKernel>
  x_extremal_point(const Circle_2<CircularKernel> & c, bool b);}
{Returns the leftmost (resp. rightmost) point of the circle if $b$ is
  \ccc{true} (resp. \ccc{false}).}

\end{ccRefFunction}
\begin{ccRefFunction}{y_extremal_point}

\ccInclude{CGAL/global_functions_circular_kernel_2.h}

\ccFunction{Circular_arc_point_2<CircularKernel>
  y_extremal_point(const Circle_2<CircularKernel> & c, bool b);}
{Returns the bottommost (ressp. topmost) point of the circle if $b$ is
  \ccc{true} (resp. \ccc{false}).}

\end{ccRefFunction}
\begin{ccRefFunction}{x_extremal_points}

\ccInclude{CGAL/global_functions_circular_kernel_2.h}

\ccFunction{template < class OutputIterator >
  OutputIterator
  x_extremal_points(const Circle_2<CircularKernel> & c,
  OutputIterator res);}
{Copies in the output iterator the $x$-extremal points of the
  circle. \ccc{res} iterates on elements of type
  \ccc{Circular_arc_point_2<CircularKernel>}, sorted in $x$.}

\end{ccRefFunction}
\begin{ccRefFunction}{y_extremal_points}

\ccInclude{CGAL/global_functions_circular_kernel_2.h}

\ccFunction{template < class OutputIterator >
  OutputIterator
  y_extremal_points(const Circle_2<CircularKernel> & c,
                    OutputIterator res);}
{Copies in the output iterator the $y$-extremal points of the
  circle. \ccc{res} iterates on elements of type
  \ccc{Circular_arc_point_2<CircularKernel>}, sorted in $y$.}

\end{ccRefFunction}
\begin{ccRefFunction}{has_on}

\ccInclude{CGAL/global_functions_circular_kernel_2.h}

\ccFunction{bool
  has_on(const Circle_2<CircularKernel> &c, 
         const Circular_arc_point_2<CircularKernel> &p);}
{Checks whether the point lies on the circle.}

\end{ccRefFunction}
\begin{ccRefFunction}{compare_y_to_right}

\ccFunction{CGAL::Comparison_result 
  compare_y_to_right(const Circular_arc_2<CircularKernel> &ca1,
                     const Circular_arc_2<CircularKernel> &ca2,
                     Circular_arc_point_2<CircularKernel> &p);}
{Compares vertically the two arcs, to the right of the point $p$,
\ccPrecond{$p$ is an intersection point of the arcs, and the arcs are
  defined to the right of $p$.}}

\end{ccRefFunction}
\begin{ccRefFunction}{point_in_x_range}

\ccInclude{CGAL/global_functions_circular_kernel_2.h}

\ccFunction{bool 
  point_in_x_range(const Circular_arc_2<CircularKernel> &ca, 
                   const Circular_arc_point_2<CircularKernel> &p);}
{Checks whether the point lies in the vertical range defined by the
  arc.}

\ccFunction{bool 
  point_in_x_range(const Line_arc_2<CircularKernel> &ca, 
                   const Circular_arc_point_2<CircularKernel> &p);}
{Checks whether the point lies in the vertical range defined by the
  line segment.}

\end{ccRefFunction}
\begin{ccRefFunction}{make_x_monotone}

\ccInclude{CGAL/global_functions_circular_kernel_2.h}

\ccFunction{template < class OutputIterator >
  OutputIterator
  make_x_monotone(const Circular_arc_2<CircularKernel> &ca, 
                  OutputIterator res);}
{Copies in the output iterator the $x$-monotone sub-arcs of $ca$.}

\end{ccRefFunction}
\begin{ccRefFunction}{make_xy_monotone}

\ccInclude{CGAL/global_functions_circular_kernel_2.h}

\ccFunction{template < class OutputIterator >
  OutputIterator
  make_xy_monotone(const Circular_arc_2<CircularKernel> &ca, 
                  OutputIterator res);}
{Copies in the output iterator the $xy$-monotone sub-arcs of $ca$.}

\end{ccRefFunction}