\section{Introduction}

The goal of the circular kernel is to offer to the user a large set of
functionalities on circles and circular arcs in the plane. All the
choices (interface, robustness, representation, and so on) made here
are consistent with the choices made in the \cgal\ kernel, for which we
refer the user to the 2D kernel manual. 

In this first release, all functionalities necessary for computing an
arrangement of circular arcs and these line segments are
defined. Three traits classes are provided for the \cgal\ arrangement
package. 

\section{Software Design}

The design is done in such a way that the algebraic concepts and the
geometric concepts are clearly separated. \ccc{Circular_kernel_2}
has therefore two template parameters: 
\begin{itemize}
\item {} the first parameter must model the \cgal\ 
three dimensional \ccc{Kernel} concept. The circular kernel derives
from it, and it provides all elementary geometric objects like points,
lines, circles, and elementary functionality on them. 
%The \cgal\ 2D Kernel can be used as parameter,
%but the user may plug his own kernel 
%instead, as long as it follows the \cgal\ kernel concept. 
\item {} the second parameter is the algebraic kernel, which is 
responsible for computations on polynomials and algebraic numbers. It 
has to be a model of concept \ccc{AlgebraicKernelForCircles}. The
robustness of the package relies on the fact that the algebraic kernel
provides exact computations on algebraic objects.
\end{itemize}

The circular kernel uses the extensibility scheme presented in the 2D
kernel manual (see Section~\ref{section-extensible-kernel}). 
The types of \ccc{Kernel} are inherited
by the circular kernel and some types are taken from the
\ccc{AlgebraicKernelForCircles} parameter. Three new main geometric objects are
introduced by \ccc{Circular_kernel_2}: circular arcs, points of
circular arcs (used in particular for endpoints of arcs and
intersection points between arcs) and line segments whose endpoints
are points of this new type.  

In fact, the circular kernel is documented as a concept, \ccc{CircularKernel}, 
and two models are provided: 
\begin{itemize}
\item {} \ccc{Circular_kernel_2<Kernel,AlgebraicKernelForCircles>}, the basic kernel, 
\item {} and
a predefined filtered kernel \ccc{Exact_circular_kernel_2}, 
that is based on similar techniques as 
\ccc{Exact_predicates_exact_constructions_kernel}. 
\end{itemize}

% those comments may need to be updated
%several filtered kernels are built on top of this basic kernel to
%increase effciency:\\
%\ccc{Lazy_curved_kernel<??>}, \\
%\ccc{Filtered_hexagon_curved_kernel<CircularKernel>}, and\\
%\ccc{Filtered_bbox_curved_kernel<CircularKernel>}. \\
%Kernels are enumerated here from finest to coarsest filtering
%technique. As in the 2D \cgal\ kernel, the idea is that cascading
%filters should give the most efficient result \footnote{In practice,
%benchmarking should still be done and further optimizations will be
%provided for future releases.}

\section{Examples}

The first example shows how to construct circles or circular arcs from
points, and how to compute intersections between them using the global
function.

\ccIncludeExampleCode{Circular_kernel_2/intersecting_arcs.cpp}

The following example shows how to use a functor of the kernel.

\ccIncludeExampleCode{Circular_kernel_2/functor_has_on_2.cpp}

\section{Design and Implementation History}

The first pieces of prototype code were comparisons of algebraic
numbers of degree~2, written by Olivier Devillers
\cite{cgal:dfmt-amafe-00,cgal:dfmt-amafe-02}, and that are still used
in the current implementation of \ccc{CGAL::Root_of_2}.

Some work was then done in the direction of a ``kernel'' for
\cgal.\footnote{Monique Teillaud, First Prototype of a
\cgal\ Geometric Kernel with Circular Arcs, Technical Report
ECG-TR-182203-01, 2002\\Sylvain Pion and Monique Teillaud,
Towards a \cgal-like kernel for curves, Technical Report
ECG-TR-302206-01, 2003} and the first design emerged in
\cite{cgal:ekptt-tock-04}.

The code of this package was initially written by Sylvain Pion and
Monique Teillaud who also wrote the manual. Athanasios Kakargias had
worked on a prototype version of this kernel in 2003. Julien
Hazebrouck participated in the implementation in July and August
2005. The contribution of Pedro Machado Manh\~{a}es de Castro in
summer 2006 improved significantly the efficiency of this kernel.  He
also added more functionality in 2008.

This work was partially supported by the IST Programme of the EU as a
Shared-cost RTD (FET Open) Project under Contract No IST-2000-26473
(\ccAnchor{http://www-sop.inria.fr/prisme/ECG/}{ECG} - Effective
Computational Geometry for Curves and Surfaces) and by the IST
Programme of the 6th Framework Programme of the EU as a STREP (FET
Open Scheme) Project under Contract No IST-006413
(ACS - Algorithms for Complex
Shapes).
