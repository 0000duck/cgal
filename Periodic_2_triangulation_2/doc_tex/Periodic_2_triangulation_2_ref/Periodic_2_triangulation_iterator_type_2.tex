% +------------------------------------------------------------------------+
% | Reference manual page: Periodic_2_triangulation_iterator_type_2.tex
% +------------------------------------------------------------------------+
% | 20.03.2010   Nico Kruithof
% | Package: Periodic_2_triangulation_2
% | 
\RCSdef{\RCSPeriodictriangulationiteratortypeRev}{$Id$}
\RCSdefDate{\RCSPeriodictriangulationiteratortypeDate}{$Date$}
% |
%%RefPage: end of header, begin of main body
% +------------------------------------------------------------------------+

\ccModifierCrossRefOff
\begin{ccRefEnum}[Periodic_2_triangulation_2::]{Iterator_type}

\ccDefinition
  
The enum \ccRefName\ is defined by \ccc{Periodic_2_triangulation_2} to
specify the behavior of geometric iterators.

\ccEnum{enum Iterator_type {STORED=0, UNIQUE, STORED_COVER_DOMAIN,
    UNIQUE_COVER_DOMAIN};}
{}

The elements of the enum have the following meaning:
\begin{itemize}
%
\item{\ccc{STORED}:} Return all geometric primitives as they are
  stored internally in \ccc{Triangulation_data_structure_2}.
%
\item{\ccc{UNIQUE}:} Return only one representative of each geometric
  primitive even if the triangulation is computed in a multiply
  sheeted covering space. Choose the representative whose maximum
  offset is minimal but non-negative in each direction of space.
%
\item{\ccc{STORED\_COVER\_DOMAIN}:} Same as \ccc{STORED} but return
  additionally all primitives whose intersection with the original
  domain of the current covering space is non-empty.
%
\item{\ccc{UNIQUE\_COVER\_DOMAIN}:} Same as \ccc{UNIQUE} but return
  additionally all primitives whose intersection with the original
  domain is non-empty.
%
\end{itemize}

\ccSeeAlso

\ccc{CGAL::Periodic_2_triangulation_2}

\end{ccRefEnum}
\ccModifierCrossRefOn
