% +------------------------------------------------------------------------+
% | Reference manual page: Periodic_2_triangulation_2/intro.tex
% +------------------------------------------------------------------------+
% | 16.03.2010   Nico Kruithof
% | Package: Periodic_2_triangulation_2
% | 
\RCSdef{\RCSPeriodicTriangulationIntroRev}{$Id$}
\RCSdefDate{\RCSPeriodicTriangulationIntroDate}{$Date$}
% |
%%RefPage: end of header, begin of main body
% +------------------------------------------------------------------------+

%\clearpage
%\section{Reference pages for 2D-Triangulations}

\ccRefChapter{2D Periodic Triangulations}
\ccChapterAuthor{Nico Kruithof}

The main classes of the 2D Periodic Triangulation package are
\ccc{Periodic_2_triangulation_2} and
\ccc{Periodic_2_Delaunay_triangulation_2}. They contain functionality
to access triangulations and to run queries on
them. \ccc{Periodic_2_Delaunay_triangulation_2} can construct and
modify Delaunay triangulations. It takes the geometric traits as well
as the triangulation data structure as template parameters. 

The geometric traits class must be a model of the concept
\ccc{Periodic_2TriangulationTraits_2} and
\ccc{Periodic_2DelaunayTriangulationTraits_2} respectively. It
contains all predicates and constructions that are needed by the
functions in the triangulation classes.

The package uses the \ccc{Triangulation_data_structure_2} to represent
the triangulation. The faces need to be a model of the concept
\ccc{Periodic_2TriangulationDSFaceBase_2} and the vertices can be a
model of the Euclidean vertex concept
\ccc{TriangulationDSVertexBase_2}.  A triangulation is stored as a
collection of vertices and faces that are linked together through
incidence and adjacency relations. Each face gives access to its three
incident vertices and to its three adjacent faces. Each vertex gives
access to one of its incident faces.

The three vertices of a face are indexed with 0, 1 and 2 in positive
orientation, the positive orientation being defined by the orientation
of the underlying space $\mathbb T_c^2$. The neighbors of a face are
also indexed with 0, 1, 2 in such a way that the neighbor indexed by
$i$ is opposite to the vertex with the same index. See
Figure~\ref{P2Triangulation2-fig-orient}.

In order to be able to specify the triangle that contain vertices both
inside and outside the original domain we store offsets for each
vertex of a face within the face. These offsets are models of the
concept \ccc{Periodic_2Offset_2}.


\section{Classified Reference Pages}

\subsection*{Concepts}

\ccRefConceptPage{Periodic_2TriangulationTraits_2}\\
\ccRefConceptPage{Periodic_2DelaunayTriangulationTraits_2}

\ccRefConceptPage{TriangulationDataStructure_2}\\
\ccRefConceptPage{Periodic_2TriangulationDSFaceBase_2}\\
\ccRefConceptPage{TriangulationVertexBase_2}

\ccRefConceptPage{Periodic_2Offset_2}

\subsection*{Classes}

\subsubsection*{Main Classes}

\ccRefIdfierPage{CGAL::Periodic_2_triangulation_2<Traits,Tds>}\\
\ccRefIdfierPage{CGAL::Periodic_2_Delaunay_triangulation_2<Traits,Tds>}\\
\ccRefIdfierPage{CGAL::Periodic_2_triangulation_hierarchy_2<PTr>}

\ccRefIdfierPage{CGAL::Periodic_2_triangulation_ds_face_base_2<>}

\ccRefIdfierPage{CGAL::Periodic_2_offset_2}

\subsubsection*{Traits Classes}

\ccRefIdfierPage{CGAL::Periodic_2_triangulation_traits_2<Traits,Periodic_2Offset_2>}

\subsection*{Enums}

\ccRefIdfierPage{CGAL::Periodic_2_triangulation_2::Iterator_type}\\
\ccRefIdfierPage{CGAL::Periodic_2_triangulation_2::Locate_type}
