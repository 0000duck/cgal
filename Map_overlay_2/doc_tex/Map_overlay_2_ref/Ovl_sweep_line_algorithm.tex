% +------------------------------------------------------------------------+
% | Reference manual page: Ovl_sweep_line_algorithm.tex (Map_overlay)
% +------------------------------------------------------------------------+
% | 
% | Package: ovl (Map_overlay_2)
% | 
% +------------------------------------------------------------------------+

\ccRefPageBegin

%%RefPage: end of header, begin of main body
% +------------------------------------------------------------------------+

%**************************************************************************
\renewcommand{\ccRefPageBegin}{\begin{ccAdvanced}}
\renewcommand{\ccRefPageEnd}{\end{ccAdvanced}}

\begin{ccRefClass}{Map_overlay_sweep<Subdivision,Notifier>}
\label{OVL_sec:sweep_line}

The \ccRefName\ class implements the sweep-line algorithm.
It sweeps all curves of both creators simultaneously. 
We implemented and generalized the standard sweep-line algorithm 
presented by Bentley and Ottmann~\cite{bkos-cgaa-97}, which requires 
$O(N\log{N} + k\log{N})$ time, where $N$ is the number of curves the two 
creators have in total, and $k$ is the number of intersections between curves 
from the first creator and curves from the second one.

\ccInclude{CGAL/Map_overlay_sweep.h}

\ccIsModel
  \ccc{MapOverlayAlgorithm_2}

\ccInheritsFrom
  \ccc{Map_overlay_base<Subdivision,Notifier>}

\ccSeeAlso
   Discussion of the different overlay construction strategies in the introduction
of \ccc{Map_overlay_2} reference pages\lcTex{ (\ccRefPage{Ovl_map_overlay})}.

\end{ccRefClass}
\renewcommand{\ccRefPageBegin}{}
\renewcommand{\ccRefPageEnd}{}


% +------------------------------------------------------------------------+
%%RefPage: end of main body, begin of footer
\ccRefPageEnd
% EOF
% +------------------------------------------------------------------------+
