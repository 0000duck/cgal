% +------------------------------------------------------------------------+
% | Reference manual page: spatial_sort.tex
% +------------------------------------------------------------------------+
% | 05.06.2006   Christophe Delage
% | Package: Spatial_sorting
% | 
% |
%%RefPage: end of header, begin of main body
% +------------------------------------------------------------------------+


\begin{ccRefFunction}{spatial_sort}  %% add template arg's if necessary

%% \ccHtmlCrossLink{}     %% add further rules for cross referencing links
%% \ccHtmlIndexC[function]{} %% add further index entries

\ccDefinition
  
The function \ccRefName\ sorts an iterator range of points in a way that
improves space locality. Two points close in the order will be close
geometrically, and two points close geometrically will have a high probability
of being close in the order.

\ccInclude{CGAL/spatial_sort.h}

\ccGlobalFunction{template <class RandomAccessIterator, class Traits, class PolicyTag>
                  void
                  spatial_sort( RandomAccessIterator begin,
                                RandomAccessIterator end,
                               const Traits& traits = Default_traits,
                                PolicyTag policy = Default_policy,
                                std::ptrdiff_t threshold_hilbert=default,
                                std::ptrdiff_t threshold_multiscale=default,
                                double ratio=default);}%
                 {sorts the range [\ccc{begin},\ccc{end}) in place.}
                 
The default traits class \ccc{Default_traits} is the kernel in which the type
\ccc{RandomAccessIterator::value_type} is defined.

The default policy is \ccc{Hilbert_sort_median_policy()} and the 
other option is \ccc{Hilbert_sort_middle_policy()}.

The default values for the thresholds and the ratio depends on the dimension.


\ccHeading{Requirements}
\begin{enumerate}
\item  \ccc{RandomAccessIterator::value_type} is convertible to
\ccc{Traits::Point_2}, \ccc{Traits::Point_3}, or  \ccc{Traits::Point_d}.
\item    \ccc{Traits} is a model for concept \ccc{SpatialSortingTraits_2},
  \ccc{SpatialSortingTraits_3}, or \ccc{SpatialSortingTraits_d}.
\end{enumerate}


\ccImplementation

Creates an instance of \ccc{Multiscale_sort<Hilbert_sort>}
where \ccStyle{Hilbert_sort} is an Hilbert sorting object,
and calls its \ccc{operator()}.

The \ccStyle{threshold_hilbert} is the minimal size of a point set to be 
subdivided recursively during Hilbert sorting, otherwise random order is used.
 The \ccStyle{threshold_multiscale} value is the minimal size for a sample to
call Hilbert sort, otherwise random order is used.
The \ccStyle{ratio} value is used to split the original set in two subsets,
spatial sort is applied on the first subset of size
\ccStyle{ratio} 
times the original size of the set, Hilbert sort is applied on the
second subset. 

%\ccc{Some_other_class},
%\ccc{some_other_function}.


\end{ccRefFunction}

% +------------------------------------------------------------------------+
%%RefPage: end of main body, begin of footer
% EOF
% +------------------------------------------------------------------------+

