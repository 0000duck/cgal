% +------------------------------------------------------------------------+
% | Reference manual page: Hilbert_sort_d.tex
% +------------------------------------------------------------------------+

\begin{ccRefFunctionObjectClass}{Hilbert_sort_d<Traits, PolicyTag>}
\ccDefinition The function object \ccRefName\ sorts iterator ranges of
\ccc{Traits::Point_d} along a Hilbert curve by recursively subdividing  at the median
or the middle depending on the \ccc{PolicyTag}.

\ccRequirements \ccc{Traits} is a model for \ccc{SpatialSortingTraits_d}.

\ccTagFullDeclarations
\ccCreationVariable{o}
\ccCreation
\ccConstructor{Hilbert_sort_d(const Traits &traits = Traits())}{constructs an instance with \ccc{traits} as traits class instance.}
\ccOperations
\ccThree{void;;}{A}{}

\ccMethod{template <class RandomAccessIterator> void operator() (RandomAccessIterator begin, RandomAccessIterator end) const;}{sorts the range [\ccc{begin}, \ccc{end}).
    \ccRequire{ \ccc{RandomAccessIterator::value_type} equals to \ccc{Traits::Point_d}.}}

    \ccTagDefaults

\end{ccRefFunctionObjectClass}
% +------------------------------------------------------------------------+
%%RefPage: end of main body, begin of footer
% EOF
% +------------------------------------------------------------------------+

