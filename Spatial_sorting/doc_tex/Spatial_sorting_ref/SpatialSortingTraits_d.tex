% +------------------------------------------------------------------------+
% | Reference manual page: SpatialSortingTraits_d.tex
% +------------------------------------------------------------------------+


\begin{ccRefConcept}{SpatialSortingTraits_d}

\ccDefinition
  
All higher dimensional spatial sorting algorithms provided in \cgal{} are parameterized
by a traits class \ccStyle{Traits}, which defines the
primitives (objects and predicates) that the sorting algorithms use.
\ccRefName\ defines the complete set of primitives required in these
functions and functors.

\ccTypes
\ccAutoIndexingOff
\ccSetTwoColumns{SpatialSortingTraits_3::Compute_coordinate_d}{}

\ccNestedType{Point_d}%
       {The point type on which the sorting algorithms operate.
       }


\ccNestedType{Compute_coordinate_d(}%
       {Functor object type returning the coordinates of a  \ccc{Point_d}.
       Must provide 
        \ccc{FT operator()(Point_d p, int i)}  returning the $i$th
        coordinate of $p$. \ccc{FT} is a type that can be compared by
        operator  \ccc{<} and \ccc{<=} and can be used as
        argument of \ccc{CGAL::to_double}.
       }

\ccCreation
\ccCreationVariable{traits}  %% choose variable name

Only a copy constructor is required.

\ccConstructor{SpatialSortingTraits_d(const SpatialSortingTraits_d& t);}{}

\ccOperations

The following member functions to create instances of the above predicate
object types must exist. 

\setlength\parskip{0mm}
\ccMemberFunction{Compute_coordinate_d compute_coordinate_d_object(); }{}

\ccParDims
\ccHasModels

Any \cgal{} $d$ dimensional kernel.

\ccParDims
\end{ccRefConcept}

% +------------------------------------------------------------------------+
%%RefPage: end of main body, begin of footer
% EOF
% +------------------------------------------------------------------------+

