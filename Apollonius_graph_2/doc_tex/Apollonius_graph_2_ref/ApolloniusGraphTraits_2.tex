% +------------------------------------------------------------------------+
% | Reference manual page: ApolloniusGraphTraits_2.tex
% +------------------------------------------------------------------------+
% | 12.04.2000   Author
% | Package: Package
% | 
%\RCSdef{\RCSRegulartriangulationtraitsRev}{$Id$}
%\RCSdefDate{\RCSRegulartriangulationtraitsDate}{$Date$}
% |
%%RefPage: end of header, begin of main body
% +------------------------------------------------------------------------+


\begin{ccRefConcept}{ApolloniusGraphTraits_2}

%% \ccHtmlCrossLink{}     %% add further rules for cross referencing links
%% \ccHtmlIndexC[concept]{} %% add further index entries
\ccDefinition
  
The concept \ccc{ApolloniusGraphTraits_2} provides the traits
requirements for the \ccc{Apollonius_graph_2} class. In particular,
it provides a type \ccc{Site_2}, which must be a model of the concept
\ccc{ApolloniusSite_2}. It also provides
constructions for sites and several function object
types for the predicates.

\ccTypes
\ccTwo{ApolloniusGraphTraits_2::Segment_2+}{}
%
\ccNestedType{Point_2}{A type for a point.}
\ccGlue
\ccNestedType{Site_2}{A type for an Apollonius site. Must be a model
  of the concept \ccc{ApolloniusSite_2}.}
\ccGlue
\ccNestedType{Line_2}{A type for a line. Only required if access to
  the dual of the Apollonius graph is required or if the primal
  or dual diagram are inserted in a stream.}
\ccGlue
\ccNestedType{Ray_2}{A type for a ray. Only required if access to
  the dual of the Apollonius graph is required or if the primal
  or dual diagram are inserted in a stream.}
\ccGlue
\ccNestedType{Segment_2}{A type for a segment. Only required if access to
  the dual of the Apollonius graph is required or if the primal
  or dual diagram are inserted in a stream.}
\ccGlue
\ccNestedType{Object_2}{A type representing different types of objects
  in two dimensions, namely: \ccc{Point_2}, \ccc{Site_2},
  \ccc{Line_2}, \ccc{Ray_2} and \ccc{Segment_2}.}
\ccGlue
\ccNestedType{FT}{A type for the field number type of sites.}
\ccGlue
\ccNestedType{RT}{A type for the ring number type of sites.}
\ccGlue
\ccNestedType{Assign_2}
{Must provide \ccc{template <class T> bool operator() ( T& t,
    Object_2 o)} which assigns \ccc{o} to \ccc{t} if \ccc{o} was
  constructed from an object of type \ccc{T}. Returns 
  \ccc{true}, if the assignment was possible.}
\ccGlue
\ccNestedType{Construct_object_2}{Must provide \ccc{template <class T>
    Object_2 operator()( T t)} that constructs an object of type
  \ccc{Object_2} that contains \ccc{t} and returns it.}
\ccGlue
\ccNestedType{Construct_Apollonius_vertex_2}{
  A constructor for a point of the Apollonius diagram equidistant
  from three sites. Must provide
  \ccc{Point_2 operator()(Site_2 s1, Site_2 s2, Site_2 s3)}, which
  constructs a point equidistant from the sites \ccc{s1}, \ccc{s2} and
  \ccc{s3}.
}
%
\ccGlue
\ccNestedType{Construct_Apollonius_site_2}
{A constructor for
a dual Apollonius site (a site whose center is a
vertex of the Apollonius diagram and its weight is the common
distance of its center from the three defining sites).
Must provide \ccc{Site_2 operator()(Site_2 s1,
Site_2 s2, Site_2 s3)}, which constructs a
dual site whose center $c$ is equidistant from \ccc{s1}, \ccc{s2} and
\ccc{s3}, and its weight is equal to the (signed) distance of $c$
from \ccc{s1} (or \ccc{s2} or \ccc{s3}).\\
Must also provide \ccc{Line_2 operator()(Site_2 s1, Site_2 s2)}, which
constructs a line bitangent to \ccc{s1} and \ccc{s2}. This line is the
dual site of \ccc{s1}, \ccc{s2} and the site at infinity; it can be
viewed as a dual Apollonius site whose center is at infinity
and its weight is infinite.
}
%
\ccTwo{ApolloniusGraphTraits_2}{}
%
\ccGlue
\ccNestedType{Compare_x_2}{A predicate object type. Must
provide \ccc{Comparison_result operator()(Site_2 s1,
Site_2 s2)}, which compares the $x$-coordinates of the centers of
\ccc{s1} and \ccc{s2}.}
%
\ccGlue
\ccNestedType{Compare_y_2}{A predicate object type. Must
provide \ccc{Comparison_result operator()(Site_2 s1,
Site_2 s2)}, which compares the $y$-coordinates of the centers of
\ccc{s1} and \ccc{s2}.}
%
\ccGlue
\ccNestedType{Compare_weight_2}{A predicate object type. Must
provide \ccc{Comparison_result operator()(Site_2 s1,
Site_2 s2)}, which compares the weights of \ccc{s1} 
and \ccc{s2}.}
%
\ccGlue
\ccNestedType{Orientation_2}{A predicate object type. Must
provide \ccc{Orientation operator()(Site_2 s1,
Site_2 s2, Site_2 s3)}, which performs the
usual orientation test for the centers of the three sites
\ccc{s1}, \ccc{s2} and \ccc{s3}.\\
Must also provide \ccc{Orientation operator()(Site_2 s1, Site_2 s2,
  Site_2 s3, Site_2 p1, Site_2 p2)}, 
which performs the usual orientation test for the Apollonius vertex of
\ccc{s1}, \ccc{s2}, \ccc{s3} and the centers of \ccc{p1} and
\ccc{p2}.
\ccPrecond{the Apollonius vertex of \ccc{s1}, \ccc{s2}
and \ccc{s3} must exist.}}
%
\ccGlue
\ccNestedType{Is_hidden_2}{A predicate object type. Must
provide \ccc{bool operator()(Site_2 s1,
Site_2 s2)}, which returns \ccc{true} if the circle
corresponding to \ccc{s2} is contained in the closure of the disk
corresponding to \ccc{s1}, \ccc{false} otherwise.}
%
\ccGlue
\ccNestedType{Oriented_side_of_bisector_2}{A predicate object type.
Must provide \ccc{Oriented_side operator()(Site_2 s1,
Site_2 s2, Point_2 p)}, which returns
the oriented side of the bisector of \ccc{s1} and \ccc{s2} that
contains \ccc{p}. Returns \ccc{ON_POSITIVE_SIDE} if \ccc{p} lies in
the half-space of \ccc{s1} (i.e., \ccc{p} is closer to \ccc{s1} than
\ccc{s2}); returns \ccc{ON_NEGATIVE_SIDE} if \ccc{p} lies in the
half-space of \ccc{s2}; returns \ccc{ON_ORIENTED_BOUNDARY} if \ccc{p}
lies on the bisector of \ccc{s1} and \ccc{s2}.}
%
\ccGlue
\ccNestedType{Vertex_conflict_2}{A predicate object type.
Must provide \ccc{Sign operator()(Site_2 s1, Site_2
s2, Site_2 s3, Site_2 q)}, which
returns the sign of the distance of \ccc{q} from the dual Apollonius
site of \ccc{s1}, \ccc{s2}, \ccc{s3}.
\ccPrecond{the dual Apollonius site of \ccc{s1}, \ccc{s2},
\ccc{s3} must exist.}\\
Must also provide \ccc{Sign operator()(Site_2 s1,
Site_2 s2, Site_2 q)}, which returns the sign of the distance of
\ccc{q} from the bitangent line of \ccc{s1}, \ccc{s2} (a degenerate
dual Apollonius site, with its center at infinity).}
%
\ccGlue
\ccNestedType{Finite_edge_interior_conflict_2}{A predicate object
type. Must provide \ccc{bool operator()(Site_2 s1,
Site_2 s2, Site_2 s3, Site_2 s4,
Site_2 q, bool b)}. The sites \ccc{s1}, \ccc{s2},
\ccc{s3} and \ccc{s4} define an Apollonius edge that lies on the
bisector of \ccc{s1} and \ccc{s2} and has as endpoints the Apollonius
vertices defined by the triplets \ccc{s1}, \ccc{s2}, \ccc{s3} and
\ccc{s1}, \ccc{s4} and \ccc{s2}. The boolean \ccc{b} denotes if the
two Apollonius vertices are in conflict with the site
\ccc{q} (in which case \ccc{b} should be \ccc{true}, otherwise
\ccc{false}). In case that \ccc{b} is \ccc{true}, the predicate
returns \ccc{true} if and only if the entire Apollonius edge is in
conflict with \ccc{q}. If \ccc{b} is \ccc{false}, the predicate returns
\ccc{false} if and only if \ccc{q} is not in conflict with the
Apollonius edge.
\ccPrecond{the Apollonius vertices of \ccc{s1}, \ccc{s2},
\ccc{s3}, and \ccc{s1}, \ccc{s4}, \ccc{s2} must exist.}\\
%
Must also provide \ccc{bool operator()(Site_2 s1,
Site_2 s2, Site_2 s3, Site_2 q, bool b)}. The
sites \ccc{s1}, \ccc{s2}, \ccc{s3} and the site at infinity
$s_\infty$ define an Apollonius edge that lies on the bisector of
\ccc{s1} and \ccc{s2} and has as endpoints the Apollonius vertices
defined by the triplets \ccc{s1}, \ccc{s2}, \ccc{s3} and \ccc{s1},
$s_\infty$ and \ccc{s2} (the second Apollonius vertex is actually at
infinity). The boolean \ccc{b} denotes if the two Apollonius vertices
are in conflict with the site \ccc{q} (in which case \ccc{b}
should be \ccc{true}, otherwise \ccc{false}). 
In case that \ccc{b} is \ccc{true}, the predicate
returns \ccc{true} if and only if the entire Apollonius edge is in
conflict with \ccc{q}. If \ccc{b} is \ccc{false}, the predicate returns
\ccc{false} if and only if \ccc{q} is not in conflict with the
Apollonius edge.
\ccPrecond{the Apollonius vertex of \ccc{s1}, \ccc{s2},
\ccc{s3} must exist.}\\
%
Must finally provide \ccc{bool operator()(Site_2 s1,
Site_2 s2, Site_2 q, bool b)}. The
sites \ccc{s1}, \ccc{s2} and the site at infinity
$s_\infty$ define an Apollonius edge that lies on the bisector of
\ccc{s1} and \ccc{s2} and has as endpoints the Apollonius vertices
defined by the triplets \ccc{s1}, \ccc{s2}, $s_\infty$ and \ccc{s1},
$s_\infty$ and \ccc{s2} (both Apollonius vertices are actually at
infinity). The boolean \ccc{b} denotes if the two Apollonius vertices
are in conflict with the site \ccc{q} (in which case \ccc{b}
should be \ccc{true}, otherwise \ccc{false}). 
In case that \ccc{b} is \ccc{true}, the predicate
returns \ccc{true} if and only if the entire Apollonius edge is in
conflict with \ccc{q}. If \ccc{b} is \ccc{false}, the predicate returns 
\ccc{false} if and only if \ccc{q} is not in conflict with the
Apollonius edge.}
%
\ccGlue
\ccNestedType{Infinite_edge_interior_conflict_2}{A predicate
object type. Must provide \ccc{bool operator()(Site_2 s1,
Site_2 s2, Site_2 s3, Site_2 q, bool b)}. The
sites $s_\infty$, \ccc{s1}, \ccc{s2} and \ccc{s3} define an
Apollonius edge that lies on the bisector of $s_\infty$ and \ccc{s1}
and has as endpoints the Apollonius vertices defined by the triplets
$s_\infty$, \ccc{s1}, \ccc{s2} and $s_\infty$, \ccc{s3} and
\ccc{s1}. The boolean \ccc{b} denotes if the two Apollonius vertices
are in conflict with the site \ccc{q} (in which case \ccc{b}
should be \ccc{true}, otherwise \ccc{false}.
In case that \ccc{b} is \ccc{true}, the predicate
returns \ccc{true} if and only if the entire Apollonius edge is in
conflict with \ccc{q}. If \ccc{b} is \ccc{false}, the predicate returns
\ccc{false} if and only if \ccc{q} is not in conflict with the
Apollonius edge.}
%
\ccGlue
\ccNestedType{Is_degenerate_edge_2}{A predicate object type.
Must provide \ccc{bool operator()(Site_2 s1, Site_2
s2, Site_2 s3, Site_2 s4)}. It returns \ccc{true} if
the Apollonius edge defined by \ccc{s1}, \ccc{s2}, \ccc{s3} and
\ccc{s4} is degenerate, \ccc{false} otherwise. An Apollonius edge is
called degenerate if its two endpoints coincide.
\ccPrecond{the Apollonius vertices of \ccc{s1}, \ccc{s2},
\ccc{s3}, and \ccc{s1}, \ccc{s4}, \ccc{s2} must exist.}}
%
 

\ccCreation
\ccCreationVariable{gt}  %% choose variable name

\ccThree{ApolloniusGraphTraits_2}{ traits = other}{}
\ccThreeToTwo
%
\ccConstructor{ ApolloniusGraphTraits_2(); }{Default constructor.}
\ccGlue
\ccConstructor{ ApolloniusGraphTraits_2(ApolloniusGraphTraits_2 other);}
{Copy constructor.}
\ccGlue
\ccMethod{ApolloniusGraphTraits_2 operator=(ApolloniusGraphTraits_2 other);}
{Assignment operator.}



\ccHeading{Access to predicate objects}
%
\ccThree{Infinite_edge_interior_conflict_2}
{gt.infinite_edge_interior_conflict_2_object();}{}
%
\ccMethod{Compare_x_2 compare_x_2_object();}{}
\ccGlue
\ccMethod{Compare_y_2 compare_y_2_object();}{}
\ccGlue
\ccMethod{Compare_weight_2 compare_weight_2_object();}{}
\ccGlue
\ccMethod{Orientation_2 orientation_2_object();}{}
\ccGlue
\ccMethod{Is_hidden_2 is_hidden_2_object();}{}
\ccGlue
\ccMethod{Oriented_side_of_bisector_2
oriented_side_of_bisector_test_2_object();}{}
\ccGlue
\ccMethod{Vertex_conflict_2 vertex_conflict_2_object();}{}
\ccGlue
\ccMethod{Finite_edge_interior_conflict_2
	finite_edge_interior_conflict_2_object();}{}
\ccGlue
\ccMethod{Infinite_edge_interior_conflict_2
	infinite_edge_interior_conflict_2_object();}{}
\ccGlue
\ccMethod{Is_degenerate_edge_2 is_degenerate_edge_2_object();}{}




\ccHeading{Access to contructor objects}
%
\ccMethod{Construct_object_2
construct_object_2_object();}{} 
%
\ccGlue
\ccMethod{Construct_Apollonius_vertex_2
construct_Apollonius_vertex_2_object();}{} 
%
\ccGlue
\ccMethod{Construct_Apollonius_site_2
construct_Apollonius_site_2_object();}{}




\ccHeading{Access to other objects}
%
\ccMethod{Assign_2 assign_2_object();}{} 


\ccHasModels

\ccc{CGAL::Apollonius_graph_traits_2<K,Method_tag>}\\
\ccc{CGAL::Apollonius_graph_filtered_traits_2<CK,CM,EK,EM,FK,FM>}


\ccSeeAlso
\ccc{CGAL::Apollonius_graph_2<Gt,Agds>}\\
\ccc{CGAL::Apollonius_graph_traits_2<K,Method_tag>}\\
\ccc{CGAL::Apollonius_graph_filtered_traits_2<CK,CM,EK,EM,FK,FM>}
\end{ccRefConcept}

% +------------------------------------------------------------------------+
%%RefPage: end of main body, begin of footer
% EOF
% +------------------------------------------------------------------------+

