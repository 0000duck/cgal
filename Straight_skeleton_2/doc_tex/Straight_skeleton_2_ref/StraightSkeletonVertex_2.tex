%% Copyright (c) 2006  SciSoft.  All rights reserved.
%%
%% This file is part of CGAL (www.cgal.org); you may redistribute it under
%% the terms of the Q Public License version 1.0.
%% See the file LICENSE.QPL distributed with CGAL.
%%
%% Licensees holding a valid commercial license may use this file in
%% accordance with the commercial license agreement provided with the software.
%%
%% This file is provided AS IS with NO WARRANTY OF ANY KIND, INCLUDING THE
%% WARRANTY OF DESIGN, MERCHANTABILITY AND FITNESS FOR A PARTICULAR PURPOSE.
%%
%% 
%%
%% Author(s)     : Fernando Cacciola <fernando_cacciola@hotmail.com>

\begin{ccRefConcept}{StraightSkeletonVertex_2}

%% \ccHtmlCrossLink{}     %% add further rules for cross referencing links
%% \ccHtmlIndexC[concept]{} %% add further index entries

\ccDefinition

The concept \ccRefName\ describes the requirements for the vertex type of the
\ccc{StraightSkeleton_2} concept. It is a refinement of the \ccc{HalfedgeDSVertex} concept
with support for storage of the incident halfedge. The \ccRefName\ concept requires the geometric embedding to be a 2D point.

\ccRefines
\ccc{HalfedgeDSVertex}

\ccTypes
  \ccNestedType{Point_2}{The type of the 2D point being the geometric embedding of the vertex}{}
\ccGlue
  \ccNestedType{FT}{A model of the \ccc{SqrtFieldNumberType} concept representing the time of a vertex (an Euclidean distance)}{}
\ccGlue
  \ccNestedType{Halfedge_around_vertex_const_circulator}{}{}
\ccGlue
  \ccNestedType{Halfedge_around_vertex_circulator}{The circulator type used to visit all the incident halfedges around a vertex}
\ccGlue
  \ccNestedType{Defining_contour_halfedge_const_circulator}{}{}
\ccGlue
  \ccNestedType{Defining_contour_halfedge_circulator}{The circulator type used to visit all the defining contour halfedges of a vertex}
\ccGlue
  \ccTypedef{boost::tuple<Halfedge_handle,Halfedge_handle,Halfedge_handle> Triedge;}{A record encapsulating 3 halfedges}

\ccCreation
\ccCreationVariable{v}  %% choose variable name

\ccConstructor{StraightSkeletonVertex_2();}{Default constructor}
\ccGlue
\ccConstructor{StraightSkeletonVertex_2(int id, Point_2 const& p);}{Constructs a contour vertex with ID number \ccc{id}, at the point \ccc{p}}
\ccGlue
\ccConstructor{StraightSkeletonVertex_2(int id, Point_2 const& p, FT time, bool is_split, bool has_infinite_time );}
{Constructs a skeleton vertex with ID number \ccc{id}, at point \ccc{p}, time \ccc{time} and the flags \ccc{is_split} and \ccc{has_infinite_time} set as given.}

\ccAccessFunctions
  \ccMethod{int id() const;}{The ID of the vertex.}
  \ccGlue
  \ccMethod{Point_2 const& point() const;}{The vertex point.}
  \ccGlue
  \ccMethod{FT time() const;}{The time of the vertex: the distance from the vertex point to the lines supporting the defining contour edges}
  \ccGlue
  \ccMethod{Halfedge_handle primary_bisector();}{}
  \ccGlue
  \ccMethod{Halfedge_const_handle primary_bisector() const;}
{Returns the skeleton halfedge incident upon the vertex (called the \textit{primary} bisector).}
  \ccGlue
  \ccMethod{Halfedge_around_vertex_circulator halfedge_around_vertex_begin();}{}
  \ccGlue
  \ccMethod{Halfedge_around_vertex_const_circulator halfedge_around_vertex_begin() const;}
{Returns a bi-directional circulator pointing to one of the incident halfedges (which one is unspecified).\\
There will always be as many incident halfedges as the degree of the vertex.\\
If this is a \textit{contour} vertex, its degree is exactly 3, and from the halfedges pointed to by the circulator, 2 are contour and 1 is a bisector.\\
If this is an \ccc{skeleton} vertex, its degree is at least 3 and all of the halfedges pointed to by the circulator are bisectors.\\
Each halfedge pointed to by this circulator is the one which is oriented towards the vertex (according to the geometric embedding).}
  \ccGlue
  \ccMethod{Defining_contour_halfedge_circulator defining_contour_halfedges_begin();}{}
  \ccGlue
  \ccMethod{Defining_contour_halfedge_const_circulator defining_contour_halfedges_begin() const;}
{Returns a bi-directional circulator pointing to one of the defining contour halfedges of the vertex (which one is unspecified).\\
There will always be as many incident defining contour halfedges as the degree of the vertex.\\
Each halfedge pointed to by this circulator is the one having its left side facing inwards (which happens to be the contour halfedge for which \ccc{is\_border()} is \ccc{false}).}
  \ccGlue
  \ccMethod{Triedge const& event_triedge();}
{Returns a triedge corresponding to the event that created the vertex, that is, the 3 halfedges corresponding to the contour edges that collided.}
  \ccGlue
  \ccMethod{void set_event_triedge(Triedge const& tri );}
{Defines the triedge corresponding to the event that created the vertex..}


\ccHeading{Queries}
\ccMethod{bool is_contour() const;}{Returns \ccc{true} iff this is a contour vertex.}
\ccGlue
\ccMethod{bool is_skeleton() const;}{Returns \ccc{true} iff this is a skeleton vertex.}
\ccGlue
\ccMethod{bool has_infinite_time() const;}{Returns \ccc{true} iff this is an infinite skeleton vertex. 
Initially, all faces are topologically closed with a face-specific infinite vertex. This vertex does not correspond to an event and connects
the two unbounded halfedge bisectors incident upon the source and target endpoints of the face's contour halfedge. An infinite vertex has no geometric embeeding 
and no predicate can be applied to it. In a full straight skeleton, all the initial infinite vertices are replaced by normal skeleton vertices,
so an infinite vertex will only be found on a partial straight skeleton.}
\ccGlue
\ccMethod{bool has_null_point() const;}{Returns \ccc{true} iff this is an infinite vertex (which does not have any geometric embeeding).}
\ccGlue
\ccMethod{bool is_split() const;}{Returns \ccc{true} iff this is a skeleton vertex corresponding to a split event. If this is \ccc{false} then it corresponds to an edge event.}
\ccGlue
\ccMethod{std::size_t degree() const;}{Returns the number of edges incident in this vertex. For a skeleton vertex, 
this is the number of bisectors, which is always at least 3, unless the vertex is infinite, in which case is exactly 2.}

\ccHasModels

\ccc{CGAL::Straight_skeleton_vertex_base_2<Refs,Point,FT>}.

\ccSeeAlso

\ccc{StraightSkeleton_2}\\
\ccc{StraightSkeletonHalfedge_2}\\
\ccc{CGAL::Straight_skeleton_vertex_base_2<Refs,Point,FT>}\\
\ccc{CGAL::Straight_skeleton_halfedge_base_2<Refs>}\\

\end{ccRefConcept}

% +------------------------------------------------------------------------+
%%RefPage: end of main body, begin of footer
% EOF
% +------------------------------------------------------------------------+
