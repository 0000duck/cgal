%% Copyright (c) 2006  SciSoft.  All rights reserved.
%%
%% This file is part of CGAL (www.cgal.org).
%% You can redistribute it and/or modify it under the terms of the GNU
%% General Public License as published by the Free Software Foundation,
%% either version 3 of the License, or (at your option) any later version.
%%
%% Licensees holding a valid commercial license may use this file in
%% accordance with the commercial license agreement provided with the software.
%%
%% This file is provided AS IS with NO WARRANTY OF ANY KIND, INCLUDING THE
%% WARRANTY OF DESIGN, MERCHANTABILITY AND FITNESS FOR A PARTICULAR PURPOSE.
%%
%% 
%%
%% Author(s)     : Fernando Cacciola <fernando_cacciola@hotmail.com>

\begin{ccRefConcept}{StraightSkeletonVertex_2}

%% \ccHtmlCrossLink{}     %% add further rules for cross referencing links
%% \ccHtmlIndexC[concept]{} %% add further index entries

\ccDefinition

The concept \ccRefName\ describes the requirements for the vertex type of the
\ccc{StraightSkeleton_2} concept. It is a refinement of the \ccc{HalfedgeDSVertex} concept
with support for storage of the incident halfedge. The \ccRefName\ concept requires the geometric embedding to be a 2D point.

\ccRefines
\ccc{HalfedgeDSVertex}

\ccTypes
  \ccNestedType{Point_2}{The type of the 2D point being the geometric embedding of the vertex}{}
\ccGlue
  \ccNestedType{FT}{A model of the \ccc{SqrtFieldNumberType} concept representing the time of a vertex (an Euclidean distance)}{}
\ccGlue
  \ccNestedType{Halfedge_around_vertex_const_circulator}{}{}
\ccGlue
  \ccNestedType{Halfedge_around_vertex_circulator}{The circulator type used to visit all the incident halfedges around a vertex}
\ccGlue
  \ccNestedType{Defining_contour_halfedge_const_circulator}{}{}
\ccGlue
  \ccNestedType{Defining_contour_halfedge_circulator}{The circulator type used to visit all the defining contour halfedges of a vertex}

\ccCreation
\ccCreationVariable{v}  %% choose variable name

\ccConstructor{StraightSkeletonVertex_2();}{Default constructor}
\ccGlue
\ccConstructor{StraightSkeletonVertex_2(int id, Point_2 const& p);}{Constructs a contour vertex with ID number \ccc{id}, at the point \ccc{p}}
\ccGlue
\ccConstructor{StraightSkeletonVertex_2(int id, Point_2 const& p, FT time );}
{Constructs a skeleton vertex with ID number \ccc{id}, at point \ccc{p} and time \ccc{time}.}

\ccAccessFunctions
  \ccMethod{int id() const;}{The ID of the vertex.}
  \ccGlue
  \ccMethod{Point_2 const& point() const;}{The vertex point.}
  \ccGlue
  \ccMethod{FT time() const;}{The time of the vertex: the distance from the vertex point to the lines supporting the defining contour edges}
  \ccGlue
  \ccMethod{Halfedge_handle primary_bisector();}{}
  \ccGlue
  \ccMethod{Halfedge_const_handle primary_bisector() const;}
{Returns the skeleton halfedge incident upon the vertex (called the \textit{primary} bisector).}
  \ccGlue
  \ccMethod{Halfedge_around_vertex_circulator halfedge_around_vertex_begin();}{}
  \ccGlue
  \ccMethod{Halfedge_around_vertex_const_circulator halfedge_around_vertex_begin() const;}
{Returns a bi-directional circulator pointing to one of the incident halfedges (which one is unspecified).\\
There will always be as many incident halfedges as the degree of the vertex.\\
If this is a \textit{contour} vertex, its degree is exactly 3, and from the halfedges pointed to by the circulator, 2 are contour and 1 is a bisector.\\
If this is an \ccc{skeleton} vertex, its degree is at least 3 and all of the halfedges pointed to by the circulator are bisectors.\\
Each halfedge pointed to by this circulator is the one which is oriented towards the vertex (according to the geometric embedding).}
  \ccGlue
  \ccMethod{Defining_contour_halfedge_circulator defining_contour_halfedges_begin();}{}
  \ccGlue
  \ccMethod{Defining_contour_halfedge_const_circulator defining_contour_halfedges_begin() const;}
{Returns a bi-directional circulator pointing to one of the defining contour halfedges of the vertex (which one is unspecified).\\
There will always be as many incident defining contour halfedges as the degree of the vertex.\\
Each halfedge pointed to by this circulator is the one having its left side facing inwards (which happens to be the contour halfedge for which \ccc{is\_border()} is \ccc{false}).}


\ccHeading{Queries}
\ccMethod{bool is_contour() const;}{Returns \ccc{true} iff this is a contour vertex.}
\ccGlue
\ccMethod{bool is_skeleton() const;}{Returns \ccc{true} iff this is a skeleton vertex.}

\ccHasModels

\ccc{CGAL::Straight_skeleton_vertex_base_2<Refs,Point,FT>}.

\ccSeeAlso

\ccc{StraightSkeleton_2}\\
\ccc{StraightSkeletonHalfedge_2}\\
\ccc{CGAL::Straight_skeleton_vertex_base_2<Refs,Point,FT>}\\
\ccc{CGAL::Straight_skeleton_halfedge_base_2<Refs>}\\

\end{ccRefConcept}

% +------------------------------------------------------------------------+
%%RefPage: end of main body, begin of footer
% EOF
% +------------------------------------------------------------------------+
