%% Copyright (c) 2004  SciSoft.  All rights reserved.
%%
%% This file is part of CGAL (www.cgal.org).
%% You can redistribute it and/or modify it under the terms of the GNU
%% General Public License as published by the Free Software Foundation,
%% either version 3 of the License, or (at your option) any later version.
%%
%% Licensees holding a valid commercial license may use this file in
%% accordance with the commercial license agreement provided with the software.
%%
%% This file is provided AS IS with NO WARRANTY OF ANY KIND, INCLUDING THE
%% WARRANTY OF DESIGN, MERCHANTABILITY AND FITNESS FOR A PARTICULAR PURPOSE.
%%
%% 
%%
%% Author(s)     : Fernando Cacciola <fernando_cacciola@hotmail.com>


\begin{ccRefFunction}{compute_outer_frame_margin}

%% add template arg's if necessary

%% \ccHtmlCrossLink{}     %% add further rules for cross referencing links
%% \ccHtmlIndexC[class]{} %% add further index entries
\ccDefinition

The function \ccRefName\ computes the separation required between a polygon and the outer frame used to obtain an exterior skeleton suitable for the computation of outer offset polygons at a given distance.

\ccInclude{CGAL/compute_outer_frame_margin.h}

\ccFunction
{template <class InputIterator, class Traits>
boost::optional< typename Traits::FT >
compute_outer_frame_margin( InputIterator       first
                          , InputIterator       beyond
                          , typename Traits::FT offset
                          , Traits const&       traits = Default_traits
                          );
} 
{Given a non-degenerate strictly-simple 2D polygon whose vertices are passed 
in the range [\ccc{first},\ccc{beyond}), calculates the largest euclidean distance
\ccc{d} between each input vertex and its corresponding offset vertex at
a distance \ccc{offset}.\\
If such a distance can be approximately computed, returns an \ccc{optional<FT>} with the value \ccc{d + (offset * 1.05)}. If the distance cannot be computed, not even approximately, due to overflow for instance, returns an empty \ccc{optional<FT>} (an {\em absent result}).\\
This result is the required separation between the input polygon
and the rectangular frame used to construct an exterior offset contour
at distance \ccc{offset} (which is done by placing the polygon as a hole of that frame).\\
Such a separation must be computed in this way because if the frame is 
too close to the polygon, the inward offset contour from the frame could
collide with the outward offset contour of the polygon, resulting in a merged
contour offset instead of two contour offsets, one of them corresponding to the frame.\\
Simply using \ccc{2*offset} as the separation is incorrect since \ccc{offset} is the distance 
between an offset line and its original, not between an offset vertex and its original.
The later, which is calculated by this function and needed to place the frame sufficiently
away from the polygon, can be thousands of times larger than \ccc{offset}.\\
If the result is {\em absent}, any attempt to construct an exterior offset polygon at distance \ccc{offset} will fail. This will occur whenever the polygon has a vertex with an internal angle approaching \ccc{0} (because the offset vertex of a vertex whose internal angle equals 0 is at {\em infinity} ).\\
\ccPrecond offset $>$ 0.
\ccPrecond The range [\ccc{first},\ccc{beyond}) contains the vertices
of a non-degenerate strictly-simple 2D polygon.
}

The default traits class \ccc{Default_traits} is an instance of the
class \ccc{Polygon_offset_builder_traits_2<Kernel>} parameterized on
the kernel in which the type \ccc{InputIterator::value_type} is defined.

\ccHeading{Requirements}
\begin{enumerate}
   \item    \ccc{InputIterator::value_type}
            is equivalent to \ccc{Traits::Point_2}.
   \item    \ccc{Traits} must be a model for \ccc{PolygonOffsetBuilderTraits_2}
\end{enumerate}

\ccSeeAlso
\ccc{PolygonOffsetBuilderTraits_2}\\
\ccc{CGAL::Polygon_offset_builder_traits_2<Kernel>}\\

\end{ccRefFunction}

% +------------------------------------------------------------------------+
%%RefPage: end of main body, begin of footer
% EOF
% +------------------------------------------------------------------------+

