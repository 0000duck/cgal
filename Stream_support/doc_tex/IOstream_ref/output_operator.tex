% +------------------------------------------------------------------------+
% | Reference manual page: output_op.tex
% +------------------------------------------------------------------------+
% | 10.04.2002   Author
% | Package: Package
% | 
\RCSdef{\RCSoutputopRev}{$Id$}
\RCSdefDate{\RCSoutputopDate}{$Date$}
% |
%%RefPage: end of header, begin of main body
% +------------------------------------------------------------------------+


\ccHtmlNoClassLinks
\begin{ccRefFunction}{operator<<}  %% add template arg's if necessary

%% \ccHtmlCrossLink{}     %% add further rules for cross referencing links
%% \ccHtmlIndexC[function]{} %% add further index entries

\ccDefinition
\cgal\  defines output operators for classes that are derived
from the class \ccStyle{ostream}. This allows to write to ostreams
as \ccStyle{cout} or \ccStyle{cerr}, as well as to strstreams
and fstreams. 
The output operator is defined for all classes in the \cgal\ kernel and for the class \ccc{Color} as well.
Let \ccc{os}  be an output stream.

\ccFunction{ostream& operator>>(ostream& os, Class c);}
{Inserts object \ccStyle{c} in the stream \ccc{os}. Returns \ccc{os}.}



\ccSeeAlso
\ccRefIdfierPage{CGAL::set_mode} \\
\ccRefIdfierPage{CGAL::set_ascii_mode} \\
\ccRefIdfierPage{CGAL::set_binary_mode}  \\
\ccRefIdfierPage{CGAL::set_pretty_mode}  \\
\ccRefIdfierPage{CGAL::get_mode} \\
\ccRefIdfierPage{CGAL::is_ascii} \\
\ccRefIdfierPage{CGAL::is_binary} \\
\ccRefIdfierPage{CGAL::is_pretty} \\
\ccRefIdfierPage{CGAL::operator>>}

\ccExample
\begin{cprog}

#include <CGAL/basic.h>
#include <iostream>
#include <fstream>

#include <CGAL/Cartesian.h>
#include <CGAL/Segment_2.h>

typedef CGAL::Point_2< CGAL::Cartesian<double> >     Point;
typedef CGAL::Segment_2< CGAL::Cartesian<double> >   Segment;

int main()
{
    Point p(0,1), q(2,2);
    Segment s(p,q);

    CGAL::set_pretty_mode(std::cout);
    std::cout << p << std::endl << q  << std::endl;

    std::ofstream f("data.txt");
    CGAL::set_binary_mode(f);
    f << s << p ;

    return 1;
}
\end{cprog} 

\end{ccRefFunction}

% +------------------------------------------------------------------------+
%%RefPage: end of main body, begin of footer
% EOF
% +------------------------------------------------------------------------+

