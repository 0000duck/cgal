% +------------------------------------------------------------------------+
% | Reference manual page: AABBTraits.tex
% +------------------------------------------------------------------------+
% | 21.02.2009   Author
% | Package: Package
% |
\RCSdef{\RCSAABBTraitsRev}{$Id: header.tex 40270 2007-09-07 15:29:10Z lsaboret $}
\RCSdefDate{\RCSAABBTraitsDate}{$Date: 2007-09-07 17:29:10 +0200 (Ven, 07 sep 2007) $}
% |
\ccRefPageBegin
%%RefPage: end of header, begin of main body
% +------------------------------------------------------------------------+


\begin{ccRefConcept}{AABBTraits}

%% \ccHtmlCrossLink{}     %% add further rules for cross referencing links
%% \ccHtmlIndexC[concept]{} %% add further index entries

\ccDefinition
  
The concept \ccRefName\ provides the geometric primitive types and methods for the class \ccc{AABB_tree<Traits>}.

% \ccGeneralizes

% ThisConcept \\
% ThatConcept

\ccTypes

\ccNestedType{Primitive}{This type must be a model of the concept
\ccc{AABBPrimitive}.}
\ccGlue
\ccNestedType{Bounding_box}{Bounding box type.}
\ccGlue
\ccNestedType{Intersection_type}{Typically a point or a \ccc{CGAL::Object} or a container of those.}

The following types are required for projection queries.

\ccNestedType{Sphere}{Sphere type.}
\ccGlue
\ccNestedType{Projection_query}{Type for which projection queries are issued, typically a point.}
\ccGlue
\ccNestedType{Projection_return}{Return type of projection queries, typically a point with additional information such as handle.}
	

\ccCreation
\ccCreationVariable{traits}  %% choose variable name

\ccConstructor{AABBTraits();}{default constructor.}

\ccOperations

During the construction of the \ccc{AABB_tree}, the primitives are sorted according to some comparison functions related to the $x$, $y$ or $z$ coordinate axis.

\ccMethod{bool x_less_than(const Primitive & pr1, const Primitive & pr2);}{}
\ccGlue
\ccMethod{bool y_less_than(const Primitive & pr1, const Primitive & pr2);}{}
\ccGlue
\ccMethod{bool z_less_than(const Primitive & pr1, const Primitive & pr2);}{}


\ccMethod{Bounding_box compute_bbox(const_primitive_iterator p_begin, const_primitive_iterator p_end);}
{Returns the bounding box of a set of primitives.}

The following predicates are required for each type \ccc{Query}
for which the class \ccc{AABB_tree<Traits>} may receive a query.

\ccMethod{bool do_intersect(const Query & q, const Bounding_box & box);}
{Returns true if the query intersects the bounding box, false otherwise.}
\ccGlue	
\ccMethod{bool do_intersect(const Query & q, const Primitive::Data_type pr);}
{Returns true if the query intersects the primitive, false otherwise.}
\ccGlue	
\ccMethod{bool intersection(const Query & q, 
const Primitive::Data_type pr, Intersection_type & intersection);}
{Returns true if the query intersects the primitive, and false otherwise. In the positive, stores the intersection in the last parameter. }

% PA: I removed this as we intersect a single primitive here, and we assume dealing only with convex primitives.
% The \ccc{Intersection_type} may be a container for multiple intersections.



The following members are required only if projection queries are issued.
\ccMethod{Sphere sphere_from_query_and_hint(const Projection_query & center, 
                                const Projection_return & hint);}
{Returns a sphere centered at \ccc{center} and passing through \ccc{hint}.}

\ccMethod{ bool intersect(const Sphere & sphere, 
                          const Primitive::Data_type p, 
                          const Projection_return  & pr);}
{Returns true if the primitive is intersecting the ball defined as the interior of the sphere, and false otherwise. In the positive, the point of the primitive closest to the center of the sphere is stored in the last parameter.}

\ccMethod{bool is_smaller(const Sphere & a, const Sphere & b);}{Returns true if sphere $a$ is contained in sphere $b$. Both spheres are assumed to share the same center.}

\ccHasModels

\ccc{AABB_traits<GeomTraits,TrianglePrimitive>}.


\ccSeeAlso
\ccc{AABB_traits<GeomTraits,TrianglePrimitive>}\\
\ccc{AABB_tree<Traits>}


% \ccExample

% A short example program.
% Instead of a short program fragment, a full running program can be
% included using the 
% \verb|\ccIncludeExampleCode{Package/AABBTraits.C}| 
% macro. The program example would be part of the source code distribution and
% also part of the automatic test suite.

% \begin{ccExampleCode}
% void your_example_code() {
% }
% \end{ccExampleCode}

% %% \ccIncludeExampleCode{Package/AABBTraits.C}

\end{ccRefConcept}

% +------------------------------------------------------------------------+
%%RefPage: end of main body, begin of footer
\ccRefPageEnd
% EOF
% +------------------------------------------------------------------------+

