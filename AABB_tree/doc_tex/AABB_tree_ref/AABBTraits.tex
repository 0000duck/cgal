% +------------------------------------------------------------------------+
% | Reference manual page: AABBTraits.tex
% +------------------------------------------------------------------------+
% | 21.02.2009   Author
% | Package: Package
% |
\RCSdef{\RCSAABBTraitsRev}{$Id: header.tex 40270 2007-09-07 15:29:10Z lsaboret $}
\RCSdefDate{\RCSAABBTraitsDate}{$Date: 2007-09-07 17:29:10 +0200 (Ven, 07 sep 2007) $}
% |
\ccRefPageBegin
%%RefPage: end of header, begin of main body
% +------------------------------------------------------------------------+


\begin{ccRefConcept}{AABBTraits}

%% \ccHtmlCrossLink{}     %% add further rules for cross referencing links
%% \ccHtmlIndexC[concept]{} %% add further index entries

\ccDefinition
  
The concept \ccRefName\ provides the geometric primitive types and methods for the class \ccc{AABB_tree<AT>}.

\ccTypes

\ccNestedType{FT}{Value type of the \ccc{Squared_distance} functor.}

\ccNestedType{Point_3}{Type of a 3D point.}

\ccNestedType{Primitive}{Type of primitive.
                         Must be a model of the concepts \ccc{AABBPrimitive} or \ccc{AABBPrimitiveWithSharedData}.}

\ccNestedType{Bounding_box}{Bounding box type.}

\ccNestedType{enum Axis \{ X_AXIS, Y_AXIS, Z_AXIS \}}{}

\ccTypedef{typedef std::pair<Point_3, Primitive::Id> Point_and_primitive_id;}{}

\begin{ccDeprecated}
\ccTypedef{typedef std::pair<Object, Primitive::Id> Object_and_primitive_id;}{}
\end{ccDeprecated}

\ccNestedType{Intersection_and_primitive_id<Query>}{A nested type to aquire the 
a pair of the return type of intersections with an object of type \ccStyle{Query} and a \ccStyle{Primitive::Id} through the member typedef \ccStyle{Type}.}

During the construction of the \ccc{AABB_tree}, the primitives are sorted according to some comparison functions related to the $x$, $y$ or $z$ coordinate axis:

\ccNestedType{Split_primitives_along_x_axis}{A functor object to split a range of primitives into two sub-ranges along the X-axis. Provides the operator:
\ccc{void operator()(InputIterator first, InputIterator beyond);} Iterator \ccc{InputIterator} must be a model of RandomAccessIterator and have \ccc{Primitive} as value type. The operator is used for determining the primitives assigned to the two children nodes of a given node, assuming that the goal is to split the X-dimension of the bounding box of the node. The primitives assigned to this node are passed as argument to the operator. It should modify the iterator range in such a way that its first half and its second half correspond to the two children nodes.}

\ccNestedType{Split_primitives_along_y_axis}{A functor object to split a range of primitives into two sub-ranges along the Y-axis. See \ccc{Split_primitives_along_x_axis} for the detailed description.}

\ccNestedType{Split_primitives_along_z_axis}{A functor object to split a range of primitives into two sub-ranges along the Z-axis. See \ccc{Split_primitives_along_x_axis} for the detailed description.}

\ccNestedType{Compute_bbox}{A functor object to compute the bounding box of a set of primitives. Provides the operator:
\ccc{Bounding_box operator()(Input_iterator begin,                                              Input_iterator beyond);} Iterator \ccc{InputIterator} must have \ccc{Primitive} as value type. }

%remove as not used any where in the code
%\ccNestedType{Splitting_direction}{A functor object to specify the direction along which the bounding box should be split:
%\ccc{Axis operator()(const Bounding_box& bbox);} which returns the direction used for splitting \ccc{bbox}. It is usually the axis aligned with the longest edge of \ccc{bbox}.}

\ccNestedType{Squared_distance}{A functor object to compute the squared distance between two points. Provides the operator:
\ccc{FT operator()(const Point_3& p, const Point_3 & q);} which returns the squared distance between \ccc{p} and \ccc{q}.}

The following predicates are required for each type \ccc{Query}
for which the class \ccc{AABB_tree<AT>} may receive an intersection detection or computation query.

\ccNestedType{Do_intersect}{A functor object to compute intersection predicates between the query and the nodes of the tree. Provides the operators:
\ccc{bool operator()(const Query & q, const Bounding_box & box);} which returns \ccc{true} iff the query intersects the bounding box, and \ccc{bool operator()(const Query & q, const Primitive & primitive);} which returns \ccc{true} iff the query intersects the primitive.}

\begin{ccDeprecated}
\ccNestedType{Intersect}{A functor object to compute the intersection of a query and a primitive. Provides the operator:
\ccc{boost::optional<Object_and_primitive_id> operator()(const Query & q, const Primitive& primitive);} which returns the intersection as a pair composed of an object and a primitive id, iff the query intersects the primitive.}
\end{ccDeprecated}

\ccNestedType{Intersect}{A functor object to compute the intersection of a query and a primitive. Provides the operator:
\ccc{Intersection_and_primitive_id<Query>::type operator()(const Query & q, const Primitive& primitive);}  which returns the intersection as a pair composed of an intersection object and the id of the primitive.}

The following predicates are required for each type \ccc{Query}
for which the class \ccc{AABB_tree<AT>} may receive a distance query.

\ccNestedType{Compare_distance}{A functor object to compute distance comparisons between the query and the nodes of the tree. Provides the operators:
\ccc{bool operator()(const Query & query, const Bounding_box& box, const Point & closest);} which returns \ccc{true} iff the bounding box is closer to \ccc{query} than \ccc{closest} is, and \ccc{bool operator()(const Query & query, const Primitive & primitive, const Point & closest);} which returns \ccc{true} iff \ccc{primitive} is closer to the \ccc{query} than \ccc{closest} is.}

\ccNestedType{Closest_point}{A functor object to compute closest point from the query on a primitive. Provides the operator:
\ccc{Point_3 operator()(const Query& query, const Primitive& primitive, const Point_3 & closest);} which returns the closest point to \ccc{query}, among \ccc{closest} and all points of the primitive.}



\ccCreationVariable{traits}  %% choose variable name

\ccOperations

\ccMethod{Split_primitives_along_x_axis split_primitives_along_x_axis_object();}{Returns the primitive splitting functor for the X axis.}

\ccMethod{Split_primitives_along_y_axis split_primitives_along_y_axis_object();}{Returns the primitive splitting functor for the Y axis.}

\ccMethod{Split_primitives_along_z_axis split_primitives_along_z_axis_object();}{Returns the primitive splitting functor for the Z axis.}

\ccMethod{Compute_bbox compute_bbox_object();}{Returns the bounding box constructor.}

\ccMethod{Do_intersect do_intersect_object();}{Returns the intersection detection functor.}

\ccMethod{Intersect intersect_object();}{Returns the intersection constructor.}

\ccMethod{Compare_distance compare_distance_object();}{Returns the distance comparison functor.}

\ccMethod{Closest_point closest_point_object();}{Returns the closest point constructor.}

\ccMethod{Squared_distance squared_distance_object();}{Returns the squared distance functor.}

In addition, if \ccc{Primitive} is a model of the concept \ccc{AABBPrimitiveWithSharedData}, the following functions are part of the concept:

\ccMethod{template <class ... T> void set_shared_data(T ... t);}
{the signature of that function must be the same as the static function \ccc{Primitive::construct_shared_data}. The type \ccc{Primitive} expects that the data constructed
by a call to \ccc{Primitive::construct_shared_data(t...)} is the one given back when accessing the reference point and the datum of a primitive.}

\ccMethod{const Primitive::Shared_data& shared_data() const;}{Returns the shared data of the primitive constructed after a call to \ccc{set_shared_data}. If no call to \ccc{set_shared_data}
has been done, \ccc{Primitive::Shared_data()} is returned.}

\ccHasModels
\ccc{AABB_traits<GeomTraits,Primitive>}.

\ccSeeAlso
\ccc{AABB_traits<GeomTraits,Primitive>}\\
\ccc{AABB_tree<AT>}
\ccc{AABBPrimitive}

\end{ccRefConcept}

% +------------------------------------------------------------------------+
%%RefPage: end of main body, begin of footer
\ccRefPageEnd
% EOF
% +------------------------------------------------------------------------+

