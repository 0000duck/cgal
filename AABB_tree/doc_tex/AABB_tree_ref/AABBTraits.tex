% +------------------------------------------------------------------------+
% | Reference manual page: AABBTraits.tex
% +------------------------------------------------------------------------+
% | 21.02.2009   Author
% | Package: Package
% |
\RCSdef{\RCSAABBTraitsRev}{$Id: header.tex 40270 2007-09-07 15:29:10Z lsaboret $}
\RCSdefDate{\RCSAABBTraitsDate}{$Date: 2007-09-07 17:29:10 +0200 (Ven, 07 sep 2007) $}
% |
\ccRefPageBegin
%%RefPage: end of header, begin of main body
% +------------------------------------------------------------------------+


\begin{ccRefConcept}{AABBTraits}

%% \ccHtmlCrossLink{}     %% add further rules for cross referencing links
%% \ccHtmlIndexC[concept]{} %% add further index entries

\ccDefinition
  
The concept \ccRefName\ provides the geometric primitive types and methods for the class \ccc{AABB_tree<Traits>}.

\ccTypes

\ccNestedType{Primitive}{Type of primitive. Must be a model of the concept \ccc{AABBPrimitive}.}
\ccGlue
\ccNestedType{Bounding_box}{Bounding box type.}
\ccGlue
\ccNestedType{Point_and_primitive}{Pair composed of a point and a primitive.}

The following types are required for distance queries.

\ccNestedType{Sphere}{Sphere type.}
	

\ccCreation
\ccCreationVariable{traits}  %% choose variable name

\ccConstructor{AABBTraits();}{Default constructor.}

\ccOperations

During the construction of the \ccc{AABB_tree}, the primitives are sorted according to some comparison functions related to the $x$, $y$ or $z$ coordinate axis:
\ccMethod{bool less_x(const Primitive & pr1, const Primitive & pr2);}{}
\ccGlue
\ccMethod{bool less_y(const Primitive & pr1, const Primitive & pr2);}{}
\ccGlue
\ccMethod{bool less_z(const Primitive & pr1, const Primitive & pr2);}{}

\ccMethod{Bounding_box compute_bbox(const_primitive_iterator begin,                                              const_primitive_iterator beyond);}
{Returns the bounding box of a set of primitives.}

The following predicates are required for each type \ccc{Query}
for which the class \ccc{AABB_tree<Traits>} may receive a query.

\ccMethod{bool do_intersect(const Query & q, const Bounding_box & box);}
{Returns \ccc{true} iff the query intersects the bounding box.}
\ccGlue	
\ccMethod{bool do_intersect(const Query & q, const Primitive& primitive);}
{Returns \ccc{true} iff the query intersects the primitive.}
\ccGlue	
\ccMethod{int do_transversally_intersect(const Query & q, const Primitive& primitive);}
{Returns 1 iff the query intersects the primitive transversally, returns 0 iff the query does not intersect the primitive and returns a negative error code in case of a non-transversal intersection.}
\ccGlue	
% TODO: replace return by pair<bool,Intersection>
\ccMethod{bool intersection(const Query & q, 
          const Primitive& primitive, 
          Point_and_primitive& result);}
{Returns \ccc{true} iff the query intersects the primitive. In the positive, stores the intersection in the last parameter as a pair composed of a point and a primitive. }

% TODO: document nearest_point


\ccMethod{bool is_contained(const Sphere & a, const Sphere & b);}{Returns \ccc{true} iff sphere $a$ is contained in sphere $b$. Both spheres are assumed to share the same center.}

\ccHasModels
\ccc{AABB_traits<GeomTraits,TrianglePrimitive>}.

\ccSeeAlso
\ccc{AABB_traits<GeomTraits,TrianglePrimitive>}\\
\ccc{AABB_tree<Traits>}
\ccc{AABBPrimitive}

\end{ccRefConcept}

% +------------------------------------------------------------------------+
%%RefPage: end of main body, begin of footer
\ccRefPageEnd
% EOF
% +------------------------------------------------------------------------+

