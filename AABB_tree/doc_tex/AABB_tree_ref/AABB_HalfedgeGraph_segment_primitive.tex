\ccRefPageBegin

\begin{ccRefClass}{AABB_HalfedgeGraph_segment_primitive<HalfedgeGraph,cache_datum>}

\ccDefinition
  
The class \ccRefName\ is a model of the concept \ccc{AABBPrimitive}.
It wraps an \ccc{edge_descriptor} of a class model of \ccc{HalfedgeGraph} to a 3D segment.
The class model of \ccc{HalfedgeGraph} from which the primitive is built should not be deleted
while the AABB tree holding the primitive is in use.



\ccParameters
\ccc{HalfedgeGraph} is a model of the halfedge graph concept.\\
\ccc{cache_datum} is either \ccc{CGAL::Tag_true} or \ccc{CGAL::Tag_false}. In the former case, the datum is stored in the primitive, while in the latter it is constructed on the fly to reduce
the memory footprint. The default is \ccc{CGAL::Tag_false} (datum is not stored).

\ccInclude{CGAL/AABB_HalfedgeGraph_segment_primitive.h}

\ccTypes
\ccTypedef{typedef boost::property_traits< boost::property_map< HalfedgeGraph, vertex_point_t>::type >::value_type Point;}{The point type.}
\ccGlue
\ccTypedef{typedef Kernel_traits<Point>::Kernel::Segment_3 Datum;}{Geometric data type.}
\ccGlue
\ccTypedef{typedef boost::graph_traits<HalfedgeGraph>::edge_descriptor Id;}{Id type.}

\ccCreation
\ccCreationVariable{p}

\ccConstructor{
template <class Iterator>
AABB_HalfedgeGraph_segment_primitive(Iterator it,const HalfedgeGraph& graph)}
{Constructs a primitive. \ccc{Iterator} is an input iterator with \ccc{Id} as \ccc{value_type}.
The example below gives a way to call this constructor the insert-by-range method of the class \ccc{CGAL::AABB_tree<Traits>}.}


\ccSeeAlso

\ccc{AABBPrimitive},\\
\ccc{AABB_primitive<Id,ObjectPropertyMap,PointPropertyMapPolyhedron,cache_datum>},\\
\ccc{AABB_FaceGraph_triangle_primitive<FaceGraph,cache_datum>}.\\

\ccIncludeExampleCode{AABB_tree/AABB_HalfedgeGraph_edge_example.cpp}

\end{ccRefClass}

\ccRefPageEnd
