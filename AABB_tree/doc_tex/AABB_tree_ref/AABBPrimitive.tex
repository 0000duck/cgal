% +------------------------------------------------------------------------+
% | Reference manual page: AABBPrimitive.tex
% +------------------------------------------------------------------------+
% | 21.02.2009   Author
% | Package: Package
% |
\RCSdef{\RCSAABBPrimitiveRev}{$Id: header.tex 40270 2007-09-07 15:29:10Z lsaboret $}
\RCSdefDate{\RCSAABBPrimitiveDate}{$Date: 2007-09-07 17:29:10 +0200 (Ven, 07 sep 2007) $}
% |
\ccRefPageBegin
%%RefPage: end of header, begin of main body
% +------------------------------------------------------------------------+


\begin{ccRefConcept}{AABBPrimitive}

%% \ccHtmlCrossLink{}     %% add further rules for cross referencing links
%% \ccHtmlIndexC[concept]{} %% add further index entries

\ccDefinition
  
The concept \ccRefName\ describes the requirements for the primitives stored in the AABB data structure. The concept encapsulates a type for the input datum (a geometric object) and an identifier type by which those primitives are referred to. The concept \ccRefName\ also refines the concepts DefaultConstructible and Assignable.

\ccTypes

\ccNestedType{FT}{Field number type.}

\ccNestedType{Datum}{Type of input datum.}

\ccNestedType{Id}{Type of identifiers by which the input objects are referred to.}


% variable name
\ccCreationVariable{primitive}  

\ccOperations

\ccMethod{Datum datum();}
{Returns the datum (geometric object) wrapped inside the Primitive.}

\ccMethod{Id id();}
{Returns the corresponding identifier. This identifier is only used as a reference for the objects in the output of the \ccc{AABB_tree} methods.}

\ccMethod{FT xref();}
{Returns the \ccc{x} reference coordinate of the primitive used for sorting the primitives.}

\ccMethod{FT yref();}
{Returns the \ccc{y} reference coordinate of the primitive used for sorting the primitives.}

\ccMethod{FT zref();}
{Returns the \ccc{z} reference coordinate of the primitive used for sorting the primitives.}

% \ccHasModels
% \ccc{AABB_primitive_base<ID>}

\ccSeeAlso
\ccc{AABB_tree<Traits>}

\ccExample

Usually, the \ccc{Primitive} type is a wrapper around a \ccc{Handle}. Assume for instance that the input objects are the triangle faces of a mesh stored as a \ccc{CGAL::Polyhedron}, the Primitive type would be a wrapper around the \ccc{Face_handle}  type of the polyhedron. The \ccc{Datum} would be a \ccc{Triangle_3}, the \ccc{Id} would be a \ccc{Face_handle}. \ccc{datum()} would return a \ccc{Triangle_3} either constructed on the fly from the id (stored as face handle) or stored internally. \ccc{id()} would return the \ccc{Face_handle}. 

\end{ccRefConcept}

% +------------------------------------------------------------------------+
%%RefPage: end of main body, begin of footer
\ccRefPageEnd
% EOF
% +------------------------------------------------------------------------+

