\section{Performances}
\label{AABB_tree_section_performances}

We provide some performance numbers for the case where the AABB tree contains a set of polyhedron triangle facets. We measure both the tree construction time and the number of queries per second for a variety of intersection and distance queries. The machine used is a PC running Windows XP64 with an Intel CPU Core2 Extreme clocked at 3.06 GHz with 4GB of RAM. The kernel used is \ccc{Simple_cartesian<double>} (the fastest in our experiments). The program has been compiled with Visual C++ 2005 compiler with the O2 option (maximize speed).

\subsection{Intersection}

The surface triangle mesh chosen for benchmarking is the knot model (14,400 triangles) available in the demo data folder. We measure the tree construction time for this model as well as for three denser versions subdivided through the Loop subdivision scheme which increases the number of triangles by a factor of four.

\begin{tabular}{|l|c|}
  \hline
  Number of triangles & Construction time (in ms)\\
  \hline
   14,400 &   156 \\
   57,600 &   328 \\
  230,400 & 1,141 \\
  921,600 & 4,813 \\
  \hline
\end{tabular}

The following table measures the number of intersection queries per second on the 14,400 triangle version of the knot mesh model for ray, line, segment and plane queries. Each ray query is generated by choosing a random source point within the bounding box of the mesh and a random vector. A line or segment query is generated by choosing two random points within the bounding box. A plane query is generated by picking a random point within the bounding box and a random normal vector. Note that a plane query generically intersects many triangles of the input surface mesh. This explains the low performance numbers for the intersection functions which enumerate all intersections.

\begin{tabular}{|l|r|r|r|r|}
  \hline
  Function                            &     Ray &    Line & Segment &   Plane \\
  \hline
  do\_intersect                       & 164,182 & 171,862 & 177,007 & 217,287 \\
  number\_of\_intersected\_primitives &  89,075 &  90,334 & 104,475 &  10,248 \\
  any\_intersection                   & 163,657 & 169,989 & 183,011 & 220,203 \\
  all\_intersections                  &  62,625 &  60,168 &  73,525 &   3,026 \\
  all\_intersected\_primitives        &  83,992 &  85,347 &  97,258 &   6,273 \\
  \hline
\end{tabular}

[inserts here a curve depicting the scalability wrt number of triangles]

\subsection{Distance}


