% Implementation Details
\section{Implementation Details}
\label{AABB_tree_section_details}

The AABB tree construction is initialized by computing the AABB of the whole set of input primitives. All primitives are then sorted along the longest coordinate axis of this box, and the primitives are separated into two equal size sets. This procedure is applied recursively until an AABB contains a single primitive. In practice the design is slightly different than the one described above as the tree is leafless as presented in {\sc opcode} \cite{cgal:t-ocdl-05}.\\

An intersection query traverses the tree by computing intersection tests only with respect to the AABBs during traversal, and with respect to the input primitives at the end of traversal, i.e., in the leafs of the tree.\\

A distance query between a query point $q$ and, e.g., a set of triangle primitives, is turned into a \emph{ball} query centered at the query point. The ball traverses the tree while recursively querying intersections with the AABBs, and computes the closest point $p$ from the query point to the input primitives at the leafs of the tree. The ball radius is then shrunk to the distance between $p$ and $q$ for all remaining recursive traversals of the tree. Efficiency is achieved through setting the initial ball radius to a small value albeit guaranteed to intersect the input primitives. This is achieved by constructing an internal KD-tree used to query closest hint points from point queries as described in Section \ref{AABB_tree_section_interface}.
