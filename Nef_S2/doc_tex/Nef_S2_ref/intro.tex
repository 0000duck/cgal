% +------------------------------------------------------------------------+
% | Cbp Reference Manual:  intro.tex
% +------------------------------------------------------------------------+
% | Nef_polyhedron_S2 reference manual pages
% |
% | 26.5.2004   Peter Hachenberger
\RCSdef{\Nef_polyhedronRefRev}{$Id$}
\RCSdefDate{\Nef_polyhedronRefDate}{$Date$}
% +------------------------------------------------------------------------+

\clearpage
\ccRefChapter{2D Boolean Operations on Nef Polygons Embedded on the Sphere}
\label{chapterNefpolyhedronRef}
\ccChapterAuthor{Peter Hachenberger \and Lutz Kettner \and Michael Seel}

% +------------------------------------------------------------------------+


Nef polyhedra are defined as a subset of the d-dimensional space obtained by
a finite number of set complement and set intersection operations on
halfspaces. 

Due to the fact that all other binary set operations like union,
difference and symmetric difference can be reduced to intersection and
complement calculations, Nef polyhedra are also closed under those
operations. Also, Nef polyhedra are closed under topological unary 
set operations. Given a Nef polyhedron one can determine its interior, its
boundary, and its closure.

Additionally, a d-dimensional Nef polyhedron has the property, that its boundary
is a (d-1)-dimensional Nef polyhedron. This property can be used as a way to
represent 3-dimensional Nef polyhedra by means of planar Nef polyhedra.
This is done by intersecting the neighborhood of a vertex in a 3D Nef polyhedron
with an $\epsilon$-sphere. The result is a planar Nef polyhedron embedded
on the sphere.

The intersection of a halfspace going through the center of the $\epsilon$-sphere,
with the $\epsilon$-sphere, results in a halfsphere which is bounded by
a great circle. A binary operation of two halfspheres cuts the great circles
into great arcs.

The incidence structure of planar Nef polyhedra can be reused. The items
are denoted as $svertex$, $shalfedge$ and $sface$, analogous 
to their counterparts in \ccc{Nef_polyhedron_2}. Additionally, there is the
\emph{shalfloop} representing the great circles.

%\ccHeading{Restricted Spherical Geometry}

We introduce geometric objects that are part of the spherical surface
$S_2$ and operations on them. We define types \ccc{Sphere_point},
\ccc{Sphere_circle}, \ccc{Sphere_segment}, and \ccc{Sphere_direction}.
\ccc{Sphere_point}s are points on $S_2$, \ccc{Sphere_circle}s are
oriented great circles of $S_2$, \ccc{Sphere_segment}s are oriented
parts of \ccc{Sphere_circles} bounded by a pair of
\ccc{Sphere_point}s, and \ccc{Sphere_direction}s are directions that
are part of great circles. (a direction is usually defined to be a
vector without length, that floats around in its underlying space and
can be used to specify a movement at any point of the underlying
space; in our case we use directions only at points that are part of
the great circle that underlies also the direction.)

Note that we have to consider special geometric properties of the
objects. For example two points that are part of a great circle define
two \ccc{Sphere_segment}s, and two arbitrary \ccc{Sphere_segment}s can
intersect in two points.

If we restrict our geometric objects to a so-called perfect hemisphere
of $S_2$\footnote{A perfect hemisphere of $S_2$ is an open half-sphere
  plus an open half-circle in the boundary of the open half-sphere
  plus one endpoint of the half-circle.} then the restricted objects
behave like in classical geometry, e.g., two points define exactly one
segment, two segments intersect in at most one interior point
(non-degenerately), or three non-cocircular sphere points can be
qualified as being positively or negatively oriented.



