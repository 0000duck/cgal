% +------------------------------------------------------------------------+
% | Reference manual page: Nef_polyhedron_S2_SHalfloop.tex
% +------------------------------------------------------------------------+
% | 14.05.2004   Peter Hachenberger
% | Package: Nef_S2
% | 
\RCSdef{\RCSNefS2SHalfloopRev}{$Id$}
\RCSdefDate{\RCSNefS2SHalfloopDate}{$Date$}
% +------------------------------------------------------------------------+

\ccRefPageBegin

%%RefPage: end of header, begin of main body
% +------------------------------------------------------------------------+


\begin{ccRefClass}[Nef_polyhedron_S2<Traits>::]{SHalfloop}

\ccDefinition

A sloop is a great circle on a sphere. A shalfloop is an oriented sloop. It is always paired with a 
shalfloop whose supporting \ccc{Sphere_circle} is pointing in
the opposite direction. The \ccc{twin()} member function returns
this shalfloop of opposite orientation. Each \ccc{Nef_polyhedron_S2} can only have one sloop 
(resp. two shalfloops).

The figure below
depicts the relationship between a shalfloop and sfaces on a sphere map.  


\begin{figure}[bht]
\begin{center}
\begin{ccTexOnly}
          \parbox{0.3\textwidth}{%
              \includegraphics[width=0.3\textwidth]{Nef_S2_ref/fig/shalfloopB}%
          }
\end{ccTexOnly}
\begin{ccHtmlOnly}
    <A HREF="fig/shalfloopB.gif">
        <img src="fig/shalfloopB.gif" 
             alt="Incidences of an SHalfloop"></A><br>
\end{ccHtmlOnly}
\caption{Incidences of an SHalfloop}
\label{figureNefS2SHalfloopIncidences}
\end{center}
\end{figure}

\ccInclude{CGAL/Nef_polyhedron_S2.h}

\ccTypes
\ccThree{SHalfloop_const_handle}{incident_facet() const;;}{}
\ccThreeToTwo

The following types are the same as in \ccc{Nef_polyhedron_S2<Traits>}.

\ccNestedType{Mark}{type of mark.}

\ccNestedType{Sphere_circle}{sphere circle type stored in SHalfloop.}

\ccNestedType{SHalfloop_const_handle}{const handle to SHalfloop.}
\ccGlue
\ccNestedType{SFace_const_handle}{const handle to SFace.}

\ccCreation
\ccCreationVariable{se}

There is no need for a user to create a \ccc{SHalfloop} explicitly. The
class \ccc{Nef_polyhedron_S2<Traits>} manages the needed shalfloops internally.

%\ccConstructor{SHalfloop();}{default constructor.}

\ccOperations

\ccMethod{const Mark& mark() const;}{the mark of \ccVar\ .}

\ccMethod{const Sphere_circle& circle() const;}{the sphere circle of \ccVar\ .}

\ccMethod{SHalfloop_const_handle twin() const;}{the twin of \ccVar\ .}

\ccMethod{SFace_const_handle incident_sface() const;}{the incident sface of \ccVar\ .}

\ccSeeAlso

\ccRefIdfierPage{CGAL::Nef_polyhedron_S2<Traits>::SFace}\\
\ccRefIdfierPage{CGAL::Nef_polyhedron_S2<Traits>::Sphere_circle}

\ccTagDefaults
\end{ccRefClass}

% +------------------------------------------------------------------------+
%%RefPage: end of main body, begin of footer
\ccRefPageEnd
% EOF
% +------------------------------------------------------------------------+
