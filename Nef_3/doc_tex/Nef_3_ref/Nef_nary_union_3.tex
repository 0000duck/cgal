% +------------------------------------------------------------------------+
% | Reference manual page: Nef_nary_union_3.tex
% +------------------------------------------------------------------------+
% | 15.07.2008   Peter Hachenberger
% | Package: Nef_3
% | 
%\RCSdef{\RCSNefNaryUnionThreeRev}{$Id: $}
%\RCSdefDate{\RCSNefNaryUnionThreeDate}{$Date: $}
% +------------------------------------------------------------------------+

\begin{ccRefClass}{Nef_nary_union_3<Nef_polyhedron_3>}

\ccDefinition

This class helps to perform the union of a set of 3D Nef polyhedra
efficiently. It succesively applies the binary union operation of
\ccc{Nef_polyhedron_3}, but schedules these union operations in an
opportune way. The class is most efficient, if the polyhedra are added
in sorted order. Any order that reflects proximity in the
three-dimensional space is helpful. To allow saving memory space, the
sorting is left to the user. This way the user can generate the
polyhedra in a sorted way and add them one by one to \ccc{Nef_nary_union_3}.

\ccInclude{CGAL/Nef_nary_union_3.h}

\ccParameters

\begin{tabbing}
\ccc{template <} \=\ccc{class Nef_polyhedron_3,}\\
    \ccc{class Nef_nary_union_3;}
\end{tabbing}

As a template parameter an instantiation of the template class Nef
polyhedra is needed.

\ccCreation
\ccCreationVariable{U}

\ccThree{Nef_nary_union_3<Nef_polyhedron_3>;}{U}{}
\ccThreeToTwo
\ccConstructor{Nef_nary_union_3<Nef_polyhedron_3>();} {initialization only.}

\ccHeading{Member Functions}

\ccMethod{Nef_polyhedron_3 get_union() const;}
{returns the union of the polyhedra previously added to the class.}

\ccMethod{void add_polyhedron(const Nef_polyhedron_3& N);} 
{adds a polyhedron.}

\ccRefIdfierPage{CGAL::Nef_polyhedron_3<Traits>}

\end{ccRefClass}
