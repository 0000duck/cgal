% +------------------------------------------------------------------------+
% | Reference manual page: Nef_polyhedron_3_SFace.tex
% +------------------------------------------------------------------------+
% | 14.05.2004   Peter Hachenberger
% | Package: Nef_3
% | 
\RCSdef{\RCSSFaceRev}{$Id$}
\RCSdefDate{\RCSSFaceDate}{$Date$}
% +------------------------------------------------------------------------+

\ccRefPageBegin

%%RefPage: end of header, begin of main body
% +------------------------------------------------------------------------+


\begin{ccRefClass}[Nef_polyhedron_3<Traits>::]{SFace}

\ccDefinition

Figure~\ccTexHtml{\ref{figureNef3HalfedgeIncidences} on page 
\pageref{figureNef3HalfedgeIncidences}}{}\begin{ccHtmlOnly}
  <A HREF="#figureNef3HalfedgeIncidences"><IMG 
  SRC="cc_ref_up_arrow.gif" ALT="reference arrow" WIDTH="10" HEIGHT="10"></A>
\end{ccHtmlOnly} and
figure~\ccTexHtml{\ref{figureNef3HalfloopIncidences} on page 
\pageref{figureNef3HalfloopIncidences}}{}\begin{ccHtmlOnly}
  <A HREF="#figureNef3HalfloopIncidences"><IMG 
  SRC="cc_ref_up_arrow.gif" ALT="reference arrow" WIDTH="10" HEIGHT="10"></A>
\end{ccHtmlOnly} illustrate the incidences of an sface. An sface is described
by its boundaries. An entry item to each boundary cycle can be accessed
using the iterator range (\ccc{sface_cycles_begin()}/\ccc{sface_cycles_end()}).
Additionally, \ccc{Nef_polyhedron_S2} provides the macro 
\ccc{CGAL_forall_sface_cylces_of}. The iterators are of type 
\ccc{SFace_cycle_const_iterator} and represent either a shalfedge, a shalfloop,
or a svertex.


\ccInclude{CGAL/Nef_polyhedron_3.h}

\ccTypes
\ccThree{SFace_cycle_const_iterator}{sf.sface_cycle_begin();;}{}
\ccThreeToTwo

The following types are the same as in \ccc{Nef_polyhedron_3<Traits>}.

\ccNestedType{Mark}{type of mark.}

\ccNestedType{Object_list}{list of Object handles.}

\ccNestedType{Vertex_const_handle}{const handle to Vertex.}
\ccGlue
\ccNestedType{Volume_const_handle}{const handle to Volume.}
\ccGlue
\ccNestedType{SFace_const_handle}{const handle to SFace.}
\ccGlue
\ccNestedType{SFace_cycle_const_iterator}
{const iterator over the entries to all sface cycles of a sface.}

\ccCreation
\ccCreationVariable{sf}

There is no need for a user to create a \ccc{SFace} explicitly. The
class \ccc{Nef_polyhedron_3<Traits>} manages the needed sfaces internally.

%\ccConstructor{SFace();}{default constructor.}

\ccOperations

\ccMethod{const Mark& mark() const;}{the mark of \ccVar\ .}

\ccMethod{Vertex_const_handle center_vertex() const;}
{the center vertex of the sphere map \ccVar\ belongs to.}

\ccMethod{Volume_const_handle volume() const;}
{the volume that corresponds to \ccVar\ in the 3D incidence structure.}

\ccMethod{SFace_cycle_const_iterator  sface_cycle_begin() const;}
{iterator over the entries to all sface cycles of \ccVar\ .}

\ccMethod{SFace_cycle_const_iterator  sface_cycle_end() const;}
{past-the-end iterator.}

\ccSeeAlso

\ccRefIdfierPage{CGAL::Nef_polyhedron_3<Traits>::Vertex}\\
\ccRefIdfierPage{CGAL::Nef_polyhedron_3<Traits>::Volume}

\ccTagDefaults
\end{ccRefClass}

% +------------------------------------------------------------------------+
%%RefPage: end of main body, begin of footer
\ccRefPageEnd
% EOF
% +------------------------------------------------------------------------+
