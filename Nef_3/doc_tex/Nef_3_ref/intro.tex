% +------------------------------------------------------------------------+
% | CGAL Reference Manual:  intro.tex
% +------------------------------------------------------------------------+
% | Nef 3 reference manual pages
% |
\RCSdef{\polyhedronRefRev}{$Id$}
\RCSdefDate{\polyhedronRefDate}{$Date$}
% +------------------------------------------------------------------------+

\chapter{3D Boolean Operations on Nef Polyhedra}
\label{chapterNef3Ref}
\ccChapterAuthor{Peter Hachenberger \and Lutz Kettner \and Michael Seel}

% +------------------------------------------------------------------------+


A Nef polyhedron is any point set generated from a finite number of
open halfspaces by set complement and set intersection operations.  In
our implementation of Nef polyhedra in 3-dimensional space, we offer a
B-rep data structures that is closed under boolean operations and with
all their generality. Starting from halfspaces (and also directly from
oriented 2-manifolds), we can work with set union, set intersection,
set difference, set complement, interior, exterior, boundary, closure,
and regularization operations. In essence, we can evaluate a CSG-tree
with halfspaces as primitives and convert it into a B-rep
representation.

In fact, we work with two data structures; one that represents the
local neighborhoods of vertices, which is in itself already a complete
description, and a data structure that connects these neighborhoods up
to a global data structure with edges, facets, and volumes. We offer a
rich interface to investigate these data structures, their different
elements and their connectivity. We provide affine (rigid)
tranformations and a point location query operation. We have a custom
file format for storing and reading Nef polyhedra from files. We offer
a simple OpenGL visualization for debugging and illustrations.

\section{Reference Pages}

\subsection*{Classes}

\ccRefIdfierPage{CGAL::Nef_polyhedron_3<Traits>}\\ 
\ccRefIdfierPage{CGAL::Nef_polyhedron_3<Traits>::Vertex}\\ 
\ccRefIdfierPage{CGAL::Nef_polyhedron_3<Traits>::Halfedge}\\ 
\ccRefIdfierPage{CGAL::Nef_polyhedron_3<Traits>::Halffacet}\\ 
\ccRefIdfierPage{CGAL::Nef_polyhedron_3<Traits>::Volume}\\ 
\ccRefIdfierPage{CGAL::Nef_polyhedron_3<Traits>::SHalfedge}\\ 
\ccRefIdfierPage{CGAL::Nef_polyhedron_3<Traits>::SHalfloop}\\ 
\ccRefIdfierPage{CGAL::Nef_polyhedron_3<Traits>::SFace}\\ 
\ccRefIdfierPage{CGAL::Nef_polyhedron_3<Traits>::SFace_cycle_iterator} 
%\ccRefIdfierPage{CGAL::Nef_polyhedron_3<Traits>::Shell_entry_iterator} 

\subsection*{Functions}
\ccThree{ostream&}{}{page 999.}

\ccHtmlNoLinks\ccHtmlNoIndex
\ccFunction{template<class Nef_polyhedron_3>
            std::size_t OFF_to_nef_3(std::istream& in, Nef_polyhedron_3& N);}
    {\lcRawHtml{<A HREF="Function_operator.html">(go there)</A>}
     \lcTex{\hfill page~\pageref{refNef_polyhedron_operator_leftshift_3}}}

\ccHtmlNoLinks\ccHtmlNoIndex
\ccFunction{template <class Traits>
            ostream& operator<<( ostream& out, 
                                 const CGAL::Nef_polyhedron_3<Traits>& N);}
    {\lcRawHtml{<A HREF="Function_operator.html">(go there)</A>}
     \lcTex{\hfill page~\pageref{refNef_polyhedron_operator_leftshift_3}}}

\ccHtmlNoLinks\ccHtmlNoIndex
\ccFunction{template <class Traits>
            istream& operator>>( istream& in, 
                                 CGAL::Nef_polyhedron_3<Traits>& N);}
    {\lcRawHtml{<A HREF="Function_operator--.html">(go there)</A>}
     \lcTex{\hfill page~\pageref{refNef_polyhedron_operator_rightshift_3}}}


%% EOF %%
