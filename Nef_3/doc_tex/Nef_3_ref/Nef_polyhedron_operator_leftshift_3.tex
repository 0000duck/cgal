% +------------------------------------------------------------------------+
% | Reference manual page: Polyhedron_operator_leftshift.tex
% +------------------------------------------------------------------------+
% | 05.04.2004   Peter Hachenberger
% | Package: Nef_3
% | 
\RCSdef{\RCSNefpolyhedron3operatoristreamRev}{$Id$}
\RCSdefDate{\RCNefpolyhedron3operatoristreamDate}{$Date$}
% |
%%RefPage: end of header, begin of main body
% +------------------------------------------------------------------------+

\ccHtmlNoClassLinks
\begin{ccRefFunction}{operator<<}
\label{refNef_polyhedron_operator_leftshift_3}

\ccDefinition

This operator writes The Nef polyhedron \emph{N} to the output stream \ccc{out} 
using a propriatary file format. It includes the 
complete incidence structure, the geometric data, and the marks of each item.

Using \cgal\ stream modifiers the following output formats can be chosen: 
ASCII(\ccc{set_ascii_mode}), binary(\ccc{set_binary_mode}) or 
pretty(\ccc{set_pretty_mode}). The mandatory format is the ASCII format. It 
is recommended to use this format for file input and output.

As the output depends on the output operators of the geometric primitives 
provided by the traits class, it might not be possible that the input operator
and output operators of different traits classes are not compatible. We recommend
to use the \cgal\ kernels \ccc{Homogeneous}, \ccc{Simple_homogeneous}, or 
\ccc{Extended_homogeneous}
parametrized with any exact number type that models $mathbb{Z}$ (e.g. \ccc{Gmpz}
or \ccc{leda_integer}).

A bounded \ccc{Nef_polyhedron_3<Extended_homogeneous>} is automatically written
as though \ccc{Nef_polyhedron_3<CGAL::Homogeneous>} or 
\ccc{Nef_polyhedron_3<CGAL::Simple_homogeneous>} is used. As a result, the input
operator of each of these types can read the output.

\ccInclude{CGAL/IO/Nef_polyhedron_iostream_3.h}

\ccGlobalFunction{template <class Nef_polyhedronTraits_3>
    ostream& operator<<( ostream& out, 
                         const CGAL::Nef_polyhedron_3<Nef_polyhedronTraits_3>& N);}
  
\ccSeeAlso

\ccRefIdfierPage{CGAL::Nef_polyhedron_3<Traits>}\\ 
\lcTex{\ccc{operator>>} \dotfill\ 
    page~\pageref{refNef_polyhedron_operator_rightshift_3}}%
\lcRawHtml{
    <I><A HREF="Function_operator--.html">operator&gt;&gt;</A></I>.
}

\end{ccRefFunction}

% +------------------------------------------------------------------------+
%%RefPage: end of main body, begin of footer
% EOF
% +------------------------------------------------------------------------+

