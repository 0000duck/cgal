\ccDefGlobalScope{CGAL::}
\begin{ccRefClass}[Qt::]{GraphicsViewCircleInput<K>}

\ccDefinition
An object of type \ccRefName\ creates a circle, defined by either
center and radius, or two or three points on the circle. A new 
vertex is inserted every time the left mouse button is pressed.
The \ccc{Escape} key removes the last entered point.


\ccInclude{CGAL/Qt/GraphicsViewCircleInput.h}

\ccParameters

The template parameter of \ccRefName\ must be a \cgal\ \ccc{Kernel}. 

\ccInheritsFrom
\ccc{Qt::GraphicsViewInput}

\ccGlue

\ccCreation
\ccCreationVariable{in}
\ccSetTwoColumns{GraphicsViewCircleInput}{}

\ccConstructor{GraphicsViewCircleInput<T>(QGraphicsScene* s, int pointsOnCircle = 1);}
{\ccc{s} is the scene where we want to draw the circle. \ccc{pointsOnCircle} is the
  number of points on the circle to generate.}

\ccMethod{void generate(Object o);}{The object \ccc{o} contains a \ccc{std::pair<K::Point_2, K::FT>}
for center and radius, or a \ccc{std::pair<K::Point_2, K::Point_2>} for two points defining
the circle, or \ccc{Triple<K::Point_2, K::Point_2, K::Point_2>} for three points defining
the circle.}

\end{ccRefClass}







