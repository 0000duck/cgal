\ccRefPageBegin

\begin{ccRefConcept}{EnvelopeTraits_3}

\ccDefinition
% ===========

This concept defines the minimal set of geometric predicates and
operations needed to compute the envelope of a set of arbitrary
surfaces in $\mathbb{R}^3$. It refines the 
\ccc{ArrangementXMonotoneTraits_2} concept. In addition to the
\ccc{Point_2} and \ccc{X_monotone_curve_2} types and the
\ccc{Has_boundary_category} category tag listed in the base concept,
it also lists the \ccc{Surface_3} and  \ccc{Xy_monotone_surface_3}
types, which represent arbitrary surfaces and $xy$-monotone surfaces,
respectively, and some constructions and predicates on these types.
Note however, that these operations usually involve the projection of
3D objects onto the $xy$-plane.

\ccRefines
\ccc{ArrangementXMonotoneTraits_2}

\ccTypes
% ======

\ccNestedType{Surface_3}
{represents an arbitrary surface in $\mathbb{R}^3$.}
\ccGlue
\ccNestedType{Xy_monotone_surface_3}
{represents a weakly $xy$-monotone surface in $\mathbb{R}^3$.}

\ccHeading{Functor Types}
% =======================

\ccThree{Compare_y_at_x_2}{}{\hspace*{14cm}}
\ccThreeToTwo

\ccNestedType{Make_xy_monotone_3}
{provides the operator (templated by the \ccc{OutputIterator} type)~:
 \begin{itemize}
 \item \ccc{OutputIterator operator() (Surface_3 S, bool is_lower,
                                       OutputIterator oi)} \\
 which subdivides the given surface \ccc{S} into $xy$-monotone parts
 and inserts them into the output iterator. The value of
 \ccc{is_lower} indicates whether we compute the lower or the upper
 envelope, so that $xy$-monotone surfaces that are irrelevant to the
 lower-envelope (resp. upper-envelope) computation may be discarded.
 The value-type of \ccc{OutputIterator} is \ccc{Xy_monotone_surface_3}.
 The operator returns a past-the-end iterator for the output sequence.
 \end{itemize}}

\ccNestedType{Construct_projected_boundary_2}
{provides the operator (templated by the \ccc{OutputIterator} type)~:
 \begin{itemize}
 \item \ccc{OutputIterator operator() (Xy_monotone_surface_3 s, OutputIterator oi)} \\
 which computes all planar $x$-monotone curves and possibly isolated planar 
 points that form the projection of the boundary of the given $xy$-monotone 
 surface $s$ onto the $xy$-plane, and inserts them into the output iterator. 
 The value-type of \ccc{OutputIterator} is \ccc{Object}, where \ccc{Object}
 wraps either a \ccc{Point_2}, or a
 \ccc{pair<X_monotone_curve_2, Oriented_side>}. In the former case, the
 object represents an isolated point of the projected boundary. In the latter,
 more general, case the object represents an $x$-monotone boundary curve
 along with an enumeration value which is either \ccc{ON_NEGATIVE_SIDE}
 or \ccc{ON_POSITIVE_SIDE}, indicating whether whether the projection of the
 surface onto the $xy$-plane lies below or above this $x$-monotone curve,
 respectively. In degenerate case, namely when the surface itself is vertical,
 and its projection onto the plane is $1$-dimensional, the \ccc{Oriented_side}
 value is \ccc{ON_ORIENTED_BOUNDARY}. The operator returns a past-the-end
 iterator for the output sequence.
 \end{itemize}}
 
\ccNestedType{Construct_projected_intersections_2}
{provides the operator (templated by the \ccc{OutputIterator} type)~:
 \begin{itemize}
 \item \ccc{OutputIterator operator() (Xy_monotone_surface_3 s1,
                                       Xy_monotone_surface_3 s2,
                                       OutputIterator oi)} \\
 which computes the projection of the intersections of the
 $xy$-monotone surfaces \ccc{s1} and \ccc{s2} onto the $xy$-plane,
 and inserts them into the output iterator.
 The value-type of \ccc{OutputIterator} is \ccc{Object}, where
 each \ccc{Object} either wraps a \ccc{pair<X_monotone_curve_2,Multiplicity>}
 instance, which represents a projected intersection curve with its
 multiplicity (in case the multiplicity is undefined or not known, it
 should be set to $0$) or an \ccc{Point_2} instance, representing the
 projected image of a degenerate intersection (the projection of an
 isolated intersection point, or of a vertical intersection curve).
 The operator returns a past-the-end iterator for the output sequence.
 \end{itemize}}

\ccNestedType{Compare_z_at_xy_3}
{provides the operators~:
 \begin{itemize}
 \item \ccc{Comparison_result operator() (Point_2 p,
                                          Xy_monotone_surface_3 s1,
                                          Xy_monotone_surface_3 s2)}\\
 which determines the relative $z$-order of the two given $xy$-monotone
 surfaces at the $xy$-coordinates of the point \ccc{p}, with the
 precondition that both surfaces are defined over \ccc{p}. Namely, it
 returns the comparison result of $s_1(p)$ and $s_2(p)$.
%
 \item \ccc{Comparison_result operator() (X_monotone_curve_2 c,
                                          Xy_monotone_surface_3 s1,
                                          Xy_monotone_surface_3 s2)}\\
 which determines the relative $z$-order of the two given $xy$-monotone
 surfaces over the interior of a given $x$-monotone curve $c$, with the
 precondition that $c$ is fully contained in the $xy$-definition range
 of both $s_1$ and $s_2$, and that the surfaces do not intersect over
 $c$. The functor should therefore return the comparison result of
 $s_1(p')$ and $s_2(p')$ for some point $p'$ in the interior of $c$.
 %
 \item \ccc{Comparison_result operator() (Xy_monotone_surface_3 s1,
                                          Xy_monotone_surface_3 s2)}\\
 which determines the relative $z$-order of the two given unbounded
 $xy$-monotone surfaces, which are defined over the entire $xy$-plane and
 have no boundary, with the precondition that the surfaces do not intersect
 at all.
 The functor should therefore return the comparison result of
 $s_1(p)$ and $s_2(p)$ for some planar point $p \in \mathbb{R}^2$.
 This operator is required iff the category tag \ccc{Has_boundary_category} 
 is defined as \ccc{Tag_true}.
 \end{itemize}}

\ccNestedType{Compare_z_at_xy_above_3}
{provides the operator~:
 \begin{itemize}
 \item \ccc{Comparison_result operator() (X_monotone_curve_2 c,
                                          Xy_monotone_surface_3 s1,
                                          Xy_monotone_surface_3 s2)}\\
 which determines the relative $z$-order of the two given $xy$-monotone
 surfaces immediately above their projected intersection curve
 $c$ (a planar point $p$ is {\em above} an $x$-monotone curve $c$ if it
 is in the $x$-range of $c$, and lies to its left when the curve is
 traversed from its $xy$-lexicographically smaller endpoint to its
 larger endpoint). We have the precondition that both surfaces are
 defined ``above'' $c$, and their relative $z$-order is the same for
 some small enough neighborhood of points above $c$.
 \end{itemize}}

\ccNestedType{Compare_z_at_xy_below_3}
{provides the operator~:
 \begin{itemize}
 \item \ccc{Comparison_result operator() (X_monotone_curve_2 c,
                                          Xy_monotone_surface_3 s1,
                                          Xy_monotone_surface_3 s2)}\\
 which determines the relative $z$-order of the two given $xy$-monotone
 surfaces immediately below their projected intersection curve
 $c$ (a planar point $p$ is {\em below} an $x$-monotone curve $c$ if it
 is in the $x$-range of $c$, and lies to its right when the curve is
 traversed from its $xy$-lexicographically smaller endpoint to its
 larger endpoint). We have the precondition that both surfaces are
 defined ``below'' $c$, and their relative $z$-order is the same for
 some small enough neighborhood of points below $c$.
 \end{itemize}}

\ccCreation
\ccCreationVariable{traits}
% =========================

\ccThree{EnvelopeTraits_3~~~}{}{\hspace*{8cm}}
\ccThreeToTwo

\ccConstructor{EnvelopeTraits_3();}{default constructor.}
\ccGlue
\ccConstructor{EnvelopeTraits_3(EnvelopeTraits_3 other);}
{copy constructor.}
\ccGlue
\ccMethod{EnvelopeTraits_3 operator=(other);}{assignment operator.}

\ccHeading{Accessing Functor Objects}
%====================================

\ccThree{Construct_projected_boundary_2~~~}{}{\hspace*{7cm}}
\ccThreeToTwo

\ccMethod{Make_xy_monotone_3 make_xy_monotone_3_object();}
{}
\ccGlue
\ccMethod{Construct_projected_boundary_2 construct_projected_boundary_2_object();}
{}
\ccGlue
\ccMethod{Construct_projected_intersections_2 construct_projected_intersections_2_object();}
{}
\ccGlue
\ccMethod{Compare_z_at_xy_3 compare_z_at_xy_3_object();}
{}
\ccGlue
\ccMethod{Compare_z_at_xy_above_3 compare_z_at_xy_above_3_object();}
{}
\ccGlue
\ccMethod{Compare_z_at_xy_below_3 compare_z_at_xy_below_3_object();}
{}

\ccHasModels
%===========
\ccc{CGAL::Env_triangle_traits_3<Kernel>}\\
\ccc{CGAL::Env_sphere_traits_3<ConicTraits>}\\
\ccc{CGAL::Env_plane_traits_3<Kernel>}\\
\ccc{CGAL::Env_surface_data_traits_3<Traits,XyData,SData,Cnv>}

\end{ccRefConcept}

\ccRefPageEnd
