% +------------------------------------------------------------------------+
% | Reference manual page: Arr_surface_data_traits.tex
% +------------------------------------------------------------------------+
% | 
% | Package: Envelope_3
% | 
% +------------------------------------------------------------------------+

\ccRefPageBegin
\begin{ccRefClass}{Env_surface_data_traits_2<Traits,XyData,SData,Cnv>}

\ccDefinition
%============

The class \ccRefName\ is a model of the \ccc{EnvelopeTraits_3} concept
and serves as a decorator class that allows the extension of the surfaces
defined by the base traits-class (the \ccc{Traits} parameter), which serves
as a geometric traits-class (a model of the \ccc{EnvelopeTraits_3} concept),
with extraneous (non-geometric) data fields.

The traits-class decorator extends the \ccc{Surface_3} and the
\ccc{Xy_monotone_surface_3} types as detailed below.
Each \ccc{Surface_3} object is associated with a single data field of type
\ccc{SData}, and each \ccc{Xy_monotone_surface_3} object is associated with
a single data field of type \ccc{XyData}. When a surface is
subdivided into $xy$-monotone surfaces, its data field is converted using
the conversion functor, which is specified by the \ccc{Cnv} template-parameter,
and the resulting objects is copied to all \ccc{Xy_monotone_surface_3} objects
induced by this surface. The conversion functor should provide an operator with
the following prototype: \\
\indent \ccc{XyData operator() (const SData& d) const;} \\

By default, the two data types are the same, so the conversion operator
is trivial and just copies the data object:
\begin{tabbing}
\ccc{SData} = \=\ccc{XyData}\\
\ccc{Cnv} = \>\ccc{_Default_convert_functor<SData,XyData>}
\end{tabbing}

\ccInclude{CGAL/Env_surface_data_traits_3.h}
 
\ccIsModel
    \ccc{EnvelopeTraits_3}

\ccTypes
%-------

\ccTypedef{typedef Traits Base_traits_3;}
    {the base traits-class.}
\ccGlue
\ccTypedef{typedef typename Base_traits_3::Surface_3 Base_surface_3;}
    {the base surface.}
\ccGlue
\ccTypedef{typedef typename Base_traits_3::Xy_monotone_surface_3 Base_xy_monotone_surface_3;}
    {the base $xy$-monotone surface surface.}

\ccTypedef{typedef Cnv Convert;}{the conversion functor.}

\ccTypedef{typedef SData Surface_data;}
    {the type of data associated with surfaces.}
\ccGlue
\ccTypedef{typedef XyData Xy_monotone_surface_data;}
    {the type of data associated with $xy$-monotone surfaces.}

\ccInheritsFrom
  \ccc{Base_traits_3}

\subsection*{Class Env\_surface\_data\_traits\_3$<$Traits,XyData,SData,Cnv$>$::Surface\_3}
%==================================================================

The \ccc{Surface_3} class nested within the surface-data traits
extends the \ccc{Base_traits_3::Surface_3} type with an extra data field.

\begin{ccClass}{Env_surface_data_traits_3<Traits,XyData,SData,Cnv>::Surface_3}

\ccInheritsFrom
  \ccc{Base_surface_3}

\ccCreation
\ccCreationVariable{s}
%---------------------

\ccConstructor{Surface_3 ();}
    {default constructor.}

\ccConstructor{Surface_3 (const Base_surface_3& base);}
    {constructs surface from the given \ccc{base} surface with uninitialized
     data field.}

\ccConstructor{Surface_3 (const Base_surface_3& base, const Surface_data& data);}
    {constructs surface from the given \ccc{base} surface with an attached
     \ccc{data} field.}

\ccAccessFunctions
%-----------------

\ccMethod{const Surface_data& data () const;}
  {returns the data field (a non-const version, which returns a reference
   to the data object, is also available).}

\ccMethod{void set_data (const Surface_data& data);}
  {sets the data field.}

\end{ccClass}

\subsection*{Class Env\_surface\_data\_traits\_3$<$Traits,XyData,SData,Cnv$>$::Xy_monotone_surface\_3}
%=============================================================================

The \ccc{Xy_monotone_surface_3} class nested within the surface-data traits
extends the \ccc{Base_traits_3::Xy_monotone_surface_3} type with an extra data
field.

\begin{ccClass}{Arr_surface_data_traits_3<Tr,XData,Mrg,CData,Cnv>::Xy_monotone_surface_3}

\ccInheritsFrom
  \ccc{Base_xy_monotone_surface_3}

\ccCreation
\ccCreationVariable{xys}
%-----------------------

\ccConstructor{Xy_monotone_surface_3 ();}
    {default constructor.}

\ccConstructor{Xy_monotone_surface_3 (const Base_xy_monotone_surface_3& base);}
    {constructs an $xy$-monotone surface from the given \ccc{base} surface with
     uninitialized data field.}

\ccConstructor{Xy_monotone_surface_3 (const Base_xy_monotone_surface_3& base, 
                                      const Xy_monotone_surface_data& data);}
    {constructs an $x$-monotone surface from the given \ccc{base} $x$-monotone
     surface with an attached \ccc{data} field.}

\ccAccessFunctions
%-----------------

\ccMethod{const Xy_monotone_surface_data& data () const;}
  {returns the field (a non-const version, which returns a reference
   to the data object, is also available).}

\ccMethod{void set_data (const Xy_monotone_surface_data& data);}
  {sets the data field.}

\end{ccClass}

\end{ccRefClass}

\ccRefPageEnd
