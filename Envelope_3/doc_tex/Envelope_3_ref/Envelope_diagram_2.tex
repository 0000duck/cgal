% +------------------------------------------------------------------------+
% | Reference manual page: Envelope_diagram.tex
% +------------------------------------------------------------------------+
% | 
% | Package: Envelope_3
% | 
% +------------------------------------------------------------------------+

\ccRefPageBegin

\begin{ccRefClass}{Envlope_diagram_2<EnvTraits>}

\ccDefinition
%============

The class-template \ccClassTemplateName\ represents the minimization
diagram that corresponds to the lower envelope of a set of curves, or the
maximization diagram that corresponds to their upper envelope. It is
parameterized by a traits class that must be a model of the
\ccc{EnvelopeTraits_3} concept, and is basically a planar arrangement of
$x$-monotone curves, as defined by this traits class. These $x$-monotone
curves are the projections of boundary curves of $xy$-monotone surfaces,
or the intersection curves between such surfaces, onto the $xy$-plane.
Thus, it is possible to traverse the envelope diagram using the
methods inherited from the \ccc{Arrangement_2} class.

The envelope diagram extends the arrangment features (namely the vertices,
halfedges, and faces), such that each feature stores a container of
originators --- namely, the $xy$-monotone surfaces (instances of the type
\ccc{EnvTraits::Xy_monotone_surface_3}) that induce the lower envelop
(or the upper envelope, in case of a maximization diagram) over this
feature. The envelope diagram provides access methods to these originators.

\ccInclude{CGAL/envelope_3.h}
 
\ccInheritsFrom
    \ccc{Arrangement_2<EnvTraits>}

\ccTypes
%=======

\ccTypedef{typedef Envelope_diagram_2<EnvTraits> Self;}{}
\ccGlue
\ccTypedef{typedef Arrangement_2<EnvTraits> Base;}{}

\ccNestedType{Surface_const_iterator}
{an iterator for the $xy$-monotone surfaces that induce a diagram feature.
 Its value-type is \ccc{EnvTraits::Xy_monotone_surface_3}.}

\ccCreation
\ccCreationVariable{diag}
%========================

\ccConstructor{Envelope_diagram_2();} 
    {constructs an empty diagram containing one unbounded face,
     which corresponds to the entire plane and has no originators.}
    
\ccConstructor{Envelope_diagram_2 (const Self& other);}
    {copy constructor.}
        
\ccConstructor{Envelope_diagram_2 (EnvTraits *traits);}
    {constructs an empty diagram that uses the given \ccc{traits}
     instance for performing the geometric predicates.}

%~~~~~~~~~~~~~~~~~~~~~~~~~~~~~~~~~~~~~~~~~~~~~~~~~~~~~~~~~~~~~~~~~~~~~~~~~~~

\subsection*{Class Envelope\_diagram\_2$<$EnvTraits$>$::Vertex}
%==============================================================

\begin{ccClass}{Envelope_diagram_2<EnvTraits>::Vertex}

\ccInheritsFrom
    \ccc{Base::Vertex}

\ccCreationVariable{v}
\ccAccessFunctions
%-----------------

\ccMethod{size_t number_of_surfaces () const;}
{returns the number of $xy$-monotone surfaces that induce \ccVar.}

\ccMethod{Surface_const_iterator surfaces_begin () const;}
{returns an iterator for the first $xy$-monotone surface that induces \ccVar.}
\ccGlue
\ccMethod{Surface_const_iterator surfaces_end () const;}
{returns a past-the-end iterator for the $xy$-monotone surfaces that induce
 \ccVar.}
 \ccMethod{Xy_monotone_surface_3 surface () const;}
{returns the first $xy$-monotone surface that induce \ccVar.
 \ccPrecond{The number of surfaces is not 0.}}

\end{ccClass}

%~~~~~~~~~~~~~~~~~~~~~~~~~~~~~~~~~~~~~~~~~~~~~~~~~~~~~~~~~~~~~~~~~~~~~~~~~~~

\subsection*{Class Envelope\_diagram\_2$<$EnvTraits$>$::Halfedge}
%================================================================

\begin{ccClass}{Envelope_diagram_2<EnvTraits>::Halfedge}

\ccInheritsFrom
    \ccc{Base::Halfedge}

\ccCreationVariable{e}
\ccAccessFunctions
%-----------------

\ccMethod{size_t number_of_surfaces () const;}
{returns the number of $xy$-monotone surfaces that induce \ccVar.}

\ccMethod{Surface_const_iterator surfaces_begin () const;}
{returns an iterator for the first $xy$-monotone surface that induces \ccVar.}
\ccGlue
\ccMethod{Surface_const_iterator surfaces_end () const;}
{returns a past-the-end iterator for the $xy$-monotone surfaces that induce
 \ccVar.}
 \ccMethod{Xy_monotone_surface_3 surface () const;}
{returns the first $xy$-monotone surface that induce \ccVar.
 \ccPrecond{The number of surfaces is not 0.}}

\end{ccClass}

%~~~~~~~~~~~~~~~~~~~~~~~~~~~~~~~~~~~~~~~~~~~~~~~~~~~~~~~~~~~~~~~~~~~~~~~~~~~

\subsection*{Class Envelope\_diagram\_2$<$EnvTraits$>$::Face}
%============================================================

\begin{ccClass}{Envelope_diagram_2<EnvTraits>::Face}

\ccInheritsFrom
    \ccc{Base::Face}

\ccCreationVariable{f}
\ccAccessFunctions
%-----------------

\ccMethod{size_t number_of_surfaces () const;}
{returns the number of $xy$-monotone surfaces that induce \ccVar.}

\ccMethod{Surface_const_iterator surfaces_begin () const;}
{returns an iterator for the first $xy$-monotone surface that induces \ccVar.}
\ccGlue
\ccMethod{Surface_const_iterator surfaces_end () const;}
{returns a past-the-end iterator for the $xy$-monotone surfaces that induce
 \ccVar.}
 \ccMethod{Xy_monotone_surface_3 surface () const;}
{returns the first $xy$-monotone surface that induce \ccVar.
 \ccPrecond{The number of surfaces is not 0.}}

\end{ccClass}

%~~~~~~~~~~~~~~~~~~~~~~~~~~~~~~~~~~~~~~~~~~~~~~~~~~~~~~~~~~~~~~~~~~~~~~~~~~~

\end{ccRefClass}

\ccRefPageEnd

