% -*- latex -*-
% =============================================================================
% The CGAL Reference Manual
% Chapter: Geometric Optimisation
% Class  : CGAL::Min_ellipse_2_traits_2<K>
% -----------------------------------------------------------------------------
% file   : doc_tex/basic/Optimisation/..._ref/Min_ellipse_2_traits_2.tex
% package: Min_ellipse_2
% author : Sven Sch�nherr <sven@inf.ethz.ch>
% -----------------------------------------------------------------------------
% $Id$
% $Date$
% =============================================================================

\ccAutoIndexingOff

\begin{ccRefClass}{Approximate_min_ellipsoid_d_traits_3<K,ET>}
\ccIndexTraitsClassBegin{Approximate_min_ellipsoid_d_traits_3}{Approximate_min_ellipsoid_d}{}

\ccAutoIndexingOn

% -----------------------------------------------------------------------------
\ccDefinition

The class \ccRefName\ is a traits class for
\ccc{CGAL::Approximate_min_ellipsoid_d<Traits>} using the
$3$-di\-men\-sional \cgal\ kernel. In order to use this class, you
have to provide an exact number-type \ccc{ET} which
\ccc{Approximate_min_ellipsoid_d<Traits>} will use for its internal
exact computations.

\ccInclude{CGAL/Approximate_min_ellipsoid_d_traits_3.h}

% -----------------------------------------------------------------------------
\ccRequirements

The template parameter \ccc{K} is a model for concept
\ccc{Kernel}. The template parameter \ccc{ET} is a model for concept
\ccc{RingNumberType} with exact arithmetic operations, i.e., the type
\ccc{CGAL::Number_type_traits<ET>::Has_exact_ring_operations} must be
\ccc{CGAL::Tag_true}. In addition, \ccc{ET} must be able to exactly
represent any finite \ccc{double} value. (An example of such a
number-type is \ccc{CGAL::MP_Float}.)

% -----------------------------------------------------------------------------
\ccIsModel

\ccRefConceptPage{ApproximateMinEllipsoid_d_Traits_d}

% -----------------------------------------------------------------------------
\ccTypes

\ccSaveThreeColumns
%\ccSetTwoColumns{Min_ellipse_3_traits_3<K>:: Ellipse}{}

\ccNestedType{FT}{typedef to \ccc{double}.  The kernel's number type
\ccc{K::RT} must be convertible to \ccc{double}.}
\ccNestedType{ET}{typedef to the second template argument, \ccc{ET}.}
\ccNestedType{Point}{typedef to \ccc{K::Point_3}.}
\ccNestedType{Cartesian_const_iterator}{typedef to
\ccc{K::Cartesian_const_iterator_3}.}

% -----------------------------------------------------------------------------

\ccAccessFunctions

The class provides the access functions required by the concept
\ccc{ApproximateMinEllipsoid_d_Traits_d}; they simply map to the
corresponding routines of class \ccc{K::Point_3}:

\ccCreationVariable{traits}

\ccMemberFunction{ int dimension(const Point& p);}{ maps to
\ccc{p.dimension()}.}

\ccMemberFunction{ Cartesian_const_iterator cartesian_begin(const
Point& p);}{ maps to \ccc{p.cartesian_begin()}.}

% -----------------------------------------------------------------------------
\ccCreation
\ccCreationVariable{traits}

\ccConstructor{ApproximateMinEllipsoid_d_Traits_3();}{default constructor.}

% -----------------------------------------------------------------------------
\ccSeeAlso

\ccRefIdfierPage{CGAL::Approximate_min_ellipsoid_d_traits_2<K,ET>}\\%
\ccRefIdfierPage{CGAL::Approximate_min_ellipsoid_d_traits_d<K,ET>}\\%
\ccRefConceptPage{ApproximateMinEllipsoid_d_Traits_d}

% -----------------------------------------------------------------------------

\ccAutoIndexingOff

\ccIndexTraitsClassEnd
\end{ccRefClass}

\ccAutoIndexingOn

% ===== EOF ===================================================================
