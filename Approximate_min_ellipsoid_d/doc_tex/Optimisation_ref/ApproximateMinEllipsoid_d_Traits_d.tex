% -*- latex -*-
% =============================================================================
% The CGAL Reference Manual
% Chapter: Geometric Optimisation
% Content: reference pages of package `Min_ellipse_2'
% -----------------------------------------------------------------------------
% file   : doc_tex/basic/Optimisation/Optimisation_ref/ApproximateMinEllipsoid_d_Traits_d.tex
% package: Approximate_min_ellipsoid_d
% author : Kaspar Fischer <fischerk@inf.ethz.ch>
% -----------------------------------------------------------------------------
% $Id$
% $Date$
% =============================================================================

\begin{ccRefConcept}{ApproximateMinEllipsoid_d_Traits_d}

% -----------------------------------------------------------------------------
\ccDefinition

This concept defines the requirements for traits classes of
\ccGlobalScope\ccc{Approximate_min_ellipsoid_d<Traits>}.

\ccRefines

\ccc{DefaultConstructible} \\%
\ccc{CopyConstructible}\\%
\ccc{Assignable}

% -----------------------------------------------------------------------------
\ccTypes

\ccNestedType{FT}{\ccc{typedef double FT}}

\ccNestedType{ET}{Some model of concept \ccc{RingNumberType} that
        provides exact arithmetic, meaning that
        \ccc{CGAL::Number_type_traits<ET>::Has_exact_ring_operations}
        must be \ccc{CGAL::Tag_true}.  In addition, \ccc{ET} must be
        able to exactly represent any finite \ccc{double} value.  (An
        example for such a type is \ccc{CGAL::MP_Float}.). \\ The type
        \ccc{ET} is to be used by the
        \ccc{Approximate_min_ellipsoid_d<Traits>} class for internal,
        exact computations.}

\ccNestedType{Point}{Type of the input points. \ccc{Point} must
        provide the default and copy constructor, and must be a model
        of \ccc{DefaultConstructible}, \ccc{CopyConstructible}, and
        \ccc{Assignable}.}

\ccNestedType{Cartesian_const_iterator}{Model for the STL concept
\ccc{RandomAccessIterator} whose value type must be convertible to
\ccc{double}.  This type is used to iterate over the Cartesian
coordinates of an instance of type \ccc{Point}, see
\ccc{cartesian_begin()} below.}

\ccAccessFunctions

\ccCreationVariable{traits}

\ccMemberFunction{ int dimension(const Point& p);}{ returns the
    dimension of a point \ccc{p}.}

\ccMemberFunction{Cartesian_const_iterator cartesian_begin(const
Point& p);}{ returns an input iterator over the Euclidean coordinates
of the point $p$.  The range of the iterator must have size
\ccc{dimension(p)}.}

% -----------------------------------------------------------------------------
\ccHasModels

\ccRefIdfierPage{CGAL::Approximate_min_ellipsoid_d_traits_2<K, ET>}
\ccRefIdfierPage{CGAL::Approximate_min_ellipsoid_d_traits_3<K, ET>}
\ccRefIdfierPage{CGAL::Approximate_min_ellipsoid_d_traits_d<K, ET>}%

% -----------------------------------------------------------------------------
\ccSeeAlso

\ccIndexTraitsClassRequirements[C]{Approximate_min_ellipsoid_d_traits_d}
\ccRefIdfierPage{CGAL::Min_ellipse_2<Traits>}\\[1ex]


% -----------------------------------------------------------------------------

\end{ccRefConcept}
\ccAutoIndexingOn

% ===== EOF ===================================================================
