\section{Cutting a Mesh}

\subsection{Computing a Cut}

All surface parameterization methods proposed in this package only
deal with topological discs.  The input mesh can be of any genus and
have any number of connected components, but if it is not a topological
disc, it has to come with a description of a border (an oriented list of
vertices) which is the border of a topological disc.  If no border  is
given as input, we assume that the surface border is the longest border already
in the input mesh (the other borders will be considered as holes).

% pierre: big contradiction here - it can be something else than a
% disk then!

This package does not provide any algorithm to transform a closed mesh
of arbitrary genus into a topological disk, the user being responsible
for computing such a cut. Nevertheless we provide in
\ccc{polyhedron_ex_parameterization.C} a simple cutting algorithm for
the sake of completeness.


\subsection{Applying a Cut}

Parameterization methods in this package only support triangulated
surfaces that are homeomorphic to a disk (models of
\ccc{ParameterizationMesh_3}). This software design simplifies the
implementation of all new parameterization methods based on linear
solvers.

\ccc{Parameterization_mesh_patch_3<ParameterizationPatchableMesh_3>}
class is responsible for virtually
{\em cutting} a patch to a \ccc{ParameterizationPatchableMesh_3} mesh,
to make it similar (from the interface point of view) to a topological
disk with a \ccc{ParameterizationMesh_3} interface.

\ccc{ParameterizationPatchableMesh_3} inherits from concept \ccc{ParameterizationMesh_3},
thus is a concept for a 3D surface mesh.
\ccc{ParameterizationPatchableMesh_3} adds the ability to support patches and
virtual seams. Patches are a subset of a 3D mesh. Virtual seams are the ability
to behave exactly as if the surface was {\em cut} following a certain path.

The \ccc{ParameterizationMesh_3} interface with the Polyhedron is also model of
\ccc{ParameterizationPatchableMesh_3}:

\ccc{CGAL::Parameterization_polyhedron_adaptor_3<Polyhedron_3_>}  \\


\subsection{Cutting a Mesh Example}

\ccc{Mesh_cutting_parameterization.C} virtually {\em cuts} a \ccc{Polyhedron_3} mesh
to make it a topological disk, then applies the default parameterization:

\ccIncludeExampleCode{Surface_mesh_parameterization/Mesh_cutting_parameterization.C}

