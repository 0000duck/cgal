\section{Not Yet Implemented}

The authors of this package plan to add the next improvements in a close
future:

\begin{itemize}

\item Speed up solving of sparse linear systems:

    \begin{itemize}

    \item Add to OpenNL a BICGSTAB and a Conjugate Gradient solvers
    with Jacobi preconditioner, to increase speed and support larger systems.

    \item \ccc{Fixed_border_parameterizer_3<ParameterizationMesh_3, BorderParameterizer_3, SparseLinearAlgebraTraits_d>} 
    should remove border vertices from
    the linear systems, in order to have a symmetric definite positive matrix
    for Tutte Barycentric Mapping and Discrete Conformal Map algorithms.

    \item \ccc{Taucs_matrix<T>} must reallocate the array of non null elements when it's full.

    \item Filling a \ccc{Taucs_matrix<T>} by lines is very slow. We must optimize this code.

    \item Add traits class for SuperLU solver.

    \end{itemize}

\item Implement \ccc{ParameterizationPatchableMesh_3} adaptor for 2D Triangulation
Data Structure enriched with 3D points.

\item Implement Natural Conformal Map parameterization algorithm.

\item Implement arc-length and uniform border parameterizations
      over a convex polygon.

\item Detect at installation time if {\sc Taucs} and SuperLU are installed,
      then set appropriate compilation variables.

\item Demos:

    \begin{itemize}

        \item Qt/OpenGL demo (same features as \ccc{polyhedron_ex_parameterization.C}
        + 3D rendering of textured mesh + 2D rendering of parameterized mesh)

        \item MFC/OpenGL demo (same features as Qt/OpenGL demo)

    \end{itemize}

\item Code cleaning:

    \begin{itemize}

    \item Add to \ccc{SparseLinearAlgebraTraits_d}, \ccc{Taucs_symmetric_solver_traits<T>}
    and SuperLU traits class
    the ability to solve linear systems in the least squares sense. Then, access to
    the solver via the concept interface instead of calls specific to OpenNL.

    \item In all solver traits classes,
    implement \ccc{is_solvable}() by solving the system, then checking that $|$ $|$$|$A$\ast$X$|$$|$/$|$$|$B$|$$|$ - 1 $|$ $<$ epsilon. This method is
    currently not implemented.

    \end{itemize}

\end{itemize}
