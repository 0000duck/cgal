\section{Sparse Linear Algebra \label{sec:Sparse-Linear-Algebra}}

Parameterizing triangle meshes requires both efficient representation
of sparse matrices and efficient iterative or direct linear
solvers. We provide links to libraries (\eigen, {\sc Taucs})
and include a separate package devoted to OpenNL sparse linear solver.

\subsection{List of Solvers}

We provide an interface to several sparse linear solvers, as models
of the \ccc{SparseLinearAlgebraTraits_d} concept:

\begin{itemize}
\item
    An interface to sparse solvers from the \opennl\ library \cite{cgal:l-nmdgp-05} is provided through classes
    \ccc{OpenNL::DefaultLinearSolverTraits<COEFFTYPE, MATRIX, VECTOR, SOLVER>} and 
    \ccc{OpenNL::SymmetricLinearSolverTraits<COEFFTYPE, MATRIX, VECTOR, SOLVER>}. The OpenNL library version shipped with \cgal\
    is a lightweight default sparse linear solver. It does not support large systems, but it is portable and
    supports exact number types.

\item
    An interface to all sparse solvers from the \ccThirdPartyEigen\ library is provided through the class
    \ccc{CGAL::Eigen_solver_traits<T>}. This solver traits class can be used for an iterative or a direct,
    symmetric or general sparse solvers. The \eigen\ solver to be used must be given as template parameter.

% \item
%     An interface to sparse solvers from the\ccThirdPartyTaucs\ library is provided through the classes
%     \ccc{CGAL::Taucs_solver_traits<T>} (out-of-core general sparse solver) and 
%     \ccc{CGAL::Taucs_symmetric_solver_traits<T>} (direct symmetric sparse solver).
%     \ccThirdPartyTaucs\ is no longer maintained and we recommend to use \ccThirdPartyEigen\ instead.\\
\end{itemize}


\subsection{\eigen\ Solver Example}

The example \ccc{examples/Surface_mesh_parameterization/Eigen_parameterization.cpp} computes the
default parameterization method (Floater mean value coordinates with a circular border),
but specifically instantiates an \eigen\ solver. Specifying a specific solver
instead of the default one (OpenNL) means using the third parameter of
\ccc{CGAL::Mean_value_coordinates_parameterizer_3<ParameterizationMesh_3,
BorderParameterizer_3, SparseLinearAlgebraTraits_d>}.  The differences with the first
example \ccc{examples/Surface_mesh_parameterization/Simple_parameterization.cpp} are:

\begin{ccExampleCode}

#include <CGAL/Eigen_solver_traits.h>

...

//***************************************
// Floater Mean Value Coordinates parameterization
// (circular border) with Eigen solver
//***************************************

// Circular border parameterizer (the default)
typedef CGAL::Circular_border_arc_length_parameterizer_3<Parameterization_polyhedron_adaptor>
                                                    Border_parameterizer;
// Eigen solver
typedef CGAL::Eigen_solver_traits<>                 Solver;

// Floater Mean Value Coordinates parameterization
// (circular border) with Eigen solver
typedef CGAL::Mean_value_coordinates_parameterizer_3<Parameterization_polyhedron_adaptor,
                                                        Border_parameterizer,
                                                        Solver>
                                                    Parameterizer;

Parameterizer::Error_code err = CGAL::parameterize(mesh_adaptor, Parameterizer());

...

\end{ccExampleCode}

