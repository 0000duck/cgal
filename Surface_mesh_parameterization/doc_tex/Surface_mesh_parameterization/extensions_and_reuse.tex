\section{Extending the Package and Reusing Code}

\subsection{Reusing Mesh Adaptors}

\ccc{ParameterizationMesh_3} defines a concept to access to a
general polyhedral mesh.
It is optimized for the \ccc{Surface_mesh_parameterization} package
only in the sense that it
defines the accessors to fields specific to the parameterization domain
(\ccc{index}, \ccc{u}, \ccc{v}, \ccc{is_parameterized}).

It may be easily generalized.


\subsection{Reusing Sparse Linear Algebra}

The \ccc{SparseLinearAlgebraTraits_d} concept and the traits classes
for \eigen, OpenNL and {\sc Taucs} are independent of the rest of the
\ccc{Surface_mesh_parameterization} package, and may be reused by
\cgal\ developers for other purposes.


\subsection{Adding New Parameterization Methods}

Implementing a new fixed border linear parameterization is easy.  Most
of the code of the fixed border methods is factorized in the
\ccc{CGAL::Fixed_border_parameterizer_3<ParameterizationMesh_3, BorderParameterizer_3, SparseLinearAlgebraTraits_d>}
class.  Subclasses must mainly
implement a \ccc{compute_w_ij}() method which computes each
$w_{ij}$ = $(i, j)$ coefficient of the matrix $A$ for $v_j$ neighboring
vertices of $v_i$.

Although implementing a new free border linear parameterization
method is more challenging, the Least Squares Conformal Maps
parameterization method provides a good starting point.

Implementing \emph{non} linear parameterizations is a natural extension
to this package, although only the mesh adaptors can be reused.


\subsection{Adding New Border Parameterization Methods}

Implementing a new border parameterization method is easy.
Square, circular and two-points border parameterizations are good starting points.


\subsection{Mesh Cutting}

Obviously, this package would benefit of having robust algorithms
which transform arbitrary meshes into topological disks.

% pierre : references ?
