\section{Output}

All parameterization methods store the result in the $(u,v)$ fields of
the input mesh vertices and halfedges. A $(u,v)$ pair is provided for
each inner vertex, while a $(u,v)$ pair is provided for each border
halfedge (possibly a seam in the mesh). The user has to iterate over
the corresponding mesh elements to get the result.

Note: $(u,v)$ fields do not exist in CGAL \ccc{Polyhedron_3} and
\ccc{Triangulation_data_structure_2}, thus the output traversal is specific to the way the (u,v) fields are implemented.

\subsection{EPS Output Example}

The C++ code below gets the $(u,v)$ fields computed by a
parameterization method in a \ccc{Polyhedron_3} mesh with a
\ccc{Parameterization_polyhedron_adaptor_3} adaptor.  Note that
\ccc{Parameterization_polyhedron_adaptor_3} stores $(u,v)$ pairs in
halfedges (both inner and border).

\begin{ccExampleCode}

// CGAL kernel
typedef CGAL::Cartesian<double>                         Kernel;

// Mesh true type and parameterization adaptors
typedef CGAL::Polyhedron_3<Kernel>                      Polyhedron;
typedef CGAL::Parameterization_polyhedron_adaptor_3<Polyhedron>     
                                                        Parameterization_polyhedron_adaptor;
...

// Dump parameterized mesh to an eps file
static bool write_file_eps(const Parameterization_polyhedron_adaptor& mesh_adaptor,
                           const char *pFilename,
                           double scale = 500.0)
{
    // Get actual mesh from adaptor
    const Polyhedron* mesh = mesh_adaptor.get_adapted_mesh();

    FILE *pFile = fopen(pFilename,"wt");
    ...

    // Get (u,v) pair of each halfedge (either inner or border)
    for (pHalfedge = mesh->halfedges_begin();
         pHalfedge != mesh->halfedges_end();
         pHalfedge++)
    {
        double x1 = scale * mesh_adaptor.info(pHalfedge->prev())->uv().x();
        double y1 = scale * mesh_adaptor.info(pHalfedge->prev())->uv().y();
        double x2 = scale * mesh_adaptor.info(pHalfedge)->uv().x();
        double y2 = scale * mesh_adaptor.info(pHalfedge)->uv().y();
        fprintf(pFile,"%g %g %g %g E\n",x1,y1,x2,y2);
    }

    ...
    fclose(pFile);
}

int main(int argc,char * argv[])
{
    Polyhedron mesh;
    ...

    // The Surface_mesh_parameterization package needs an adaptor to handle Polyhedron_3 meshes
    Parameterization_polyhedron_adaptor mesh_adaptor(&mesh);

    // Parameterization
    ...

    // Write Postscript file
    write_file_eps(mesh_adaptor, output_filename);
}

\end{ccExampleCode}

See the complete example in \ccc{Taucs_parameterization.C}.