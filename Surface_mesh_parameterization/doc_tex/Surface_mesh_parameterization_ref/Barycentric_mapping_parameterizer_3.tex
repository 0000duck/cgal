% +------------------------------------------------------------------------+
% | Reference manual page: Barycentric_mapping_parameterizer_3.tex
% +------------------------------------------------------------------------+
% | 23.08.2005   Laurent Saboret, Pierre Alliez, Bruno Levy
% | Package: Surface_mesh_parameterization
% |
\RCSdef{\RCSBarycentricmappingparameterizerRev}{$Id$}
\RCSdefDate{\RCSBarycentricmappingparameterizerDate}{$Date$}
% |
\ccRefPageBegin
%%RefPage: end of header, begin of main body
% +------------------------------------------------------------------------+


\begin{ccRefClass}{Barycentric_mapping_parameterizer_3<ParameterizationMesh_3, BorderParameterizer_3, SparseLinearAlgebraTraits_d>}

%% \ccHtmlCrossLink{}     %% add further rules for cross referencing links
%% \ccHtmlIndexC[class]{} %% add further index entries


\ccDefinition

% The section below is automatically generated. Do not edit!
%START-AUTO(\ccDefinition)

The class \ccc{Barycentric_mapping_parameterizer_3} implements Tutte Barycentric Mapping algorithm \cite{t-hdg-63}. This algorithm is also called {\em Tutte Uniform Weights} by other authors.

One-to-one mapping is guaranteed if the surface's border is mapped to a convex polygon.

This class is a Strategy \cite{cgal:ghjv-dpero-95} called by the main parameterization algorithm \ccc{Fixed_border_parameterizer_3::parameterize}(). It provides default \ccc{BorderParameterizer_3} and \ccc{SparseLinearAlgebraTraits_d} template parameters that make sense. It implements \ccc{compute_w_ij}() to compute \ccc{w_ij} = (i, j) coefficient of matrix A for j neighbor vertex of i based on Tutte Barycentric Mapping method. It implements an optimized version of \ccc{is_one_to_one_mapping}().

%END-AUTO(\ccDefinition)

% The section below is automatically generated. Do not edit!
%START-AUTO(\ccInclude)

\ccInclude{CGAL/Barycentric_mapping_parameterizer_3.h}

%END-AUTO(\ccInclude)


\ccInheritsFrom

% The section below is automatically generated. Do not edit!
%START-AUTO(\ccInheritsFrom)

\ccc{Fixed_border_parameterizer_3<ParameterizationMesh_3, BorderParameterizer_3, SparseLinearAlgebraTraits_d>}

%END-AUTO(\ccInheritsFrom)


\ccIsModel

% The section below is automatically generated. Do not edit!
%START-AUTO(\ccIsModel)

Model of the \ccc{ParameterizerTraits_3} concept.

%END-AUTO(\ccIsModel)


\ccHeading{Design Pattern}

% The section below is automatically generated. Do not edit!
%START-AUTO(\ccHeading{Design Pattern})

\ccc{Barycentric_mapping_parameterizer_3} class is a Strategy \cite{cgal:ghjv-dpero-95}: it implements a strategy of surface parameterization for models of \ccc{ParameterizationMesh_3}.

%END-AUTO(\ccHeading{Design Pattern})


\ccParameters

The full template declaration is:

% The section below is automatically generated. Do not edit!
%START-AUTO(\ccParameters)

template$<$  \\
class \ccc{ParameterizationMesh_3},   \\
class \ccc{BorderParameterizer_3} = \ccc{Circular_border_arc_length_parameterizer_3<ParameterizationMesh_3>},   \\
class \ccc{SparseLinearAlgebraTraits_d} = \ccc{OpenNL::DefaultLinearSolverTraits<typename ParameterizationMesh_3::NT>}$>$   \\
class \ccc{Barycentric_mapping_parameterizer_3};

\ccCommentHeading{Parameters}  \\
\ccc{ParameterizationMesh_3}: 3D surface mesh. \ccc{BorderParameterizer_3}: Strategy to parameterize the surface border. \ccc{SparseLinearAlgebraTraits_d}: Traits class to solve a sparse linear system. Note: the system is {\bf not} symmetric because \ccc{Fixed_border_parameterizer_3} does not remove (yet) border vertices from the system.

%END-AUTO(\ccParameters)


\ccCreation
\ccCreationVariable{param}  %% variable name used by \ccMethod below

% The section below is automatically generated. Do not edit!
%START-AUTO(\ccCreation)

\ccConstructor{Barycentric_mapping_parameterizer_3(Border_param border_param = Border_param(), Sparse_LA sparse_la = Sparse_LA());}
{
Constructor.
\ccCommentHeading{Parameters}  \\
\ccc{border_param}: Object that maps the surface's border to 2D space. \ccc{sparse_la}: Traits object to access a sparse linear system.
}
\ccGlue

%END-AUTO(\ccCreation)


\ccOperations

% The section below is automatically generated. Do not edit!
%START-AUTO(\ccOperations)

\ccMethod{virtual NT compute_w_ij(const Adaptor& , Vertex_const_handle, Vertex_around_vertex_const_circulator)[protected, virtual];}
{
Compute \ccc{w_ij} = (i, j) coefficient of matrix A for j neighbor vertex of i.  \\
Tutte Barycentric Mapping algorithm is the most simple one: \ccc{w_ij} = 1 for j neighbor vertex of i.
}
\ccGlue
\ccMethod{virtual bool is_one_to_one_mapping(const Adaptor& , const Matrix& , const Vector& , const Vector& )[protected, virtual];}
{
Check if 3D -$>$ 2D mapping is one-to-one.  \\
Theorem: one-to-one mapping is guaranteed if all \ccc{w_ij} coefficients are $>$ 0 (for j vertex neighbor of i) and if the surface border is mapped onto a 2D convex polygon. All \ccc{w_ij} coefficients = 1 (for j vertex neighbor of i), thus mapping is guaranteed if the surface border is mapped onto a 2D convex polygon.
}
\ccGlue

%END-AUTO(\ccOperations)


\ccSeeAlso

\ccRefIdfierPage{CGAL::Parameterizer_traits_3<ParameterizationMesh_3>}  \\
\ccRefIdfierPage{CGAL::Fixed_border_parameterizer_3<ParameterizationMesh_3, BorderParameterizer_3, SparseLinearAlgebraTraits_d>}  \\
\ccRefIdfierPage{CGAL::Discrete_authalic_parameterizer_3<ParameterizationMesh_3, BorderParameterizer_3, SparseLinearAlgebraTraits_d>}  \\
\ccRefIdfierPage{CGAL::Discrete_conformal_map_parameterizer_3<ParameterizationMesh_3, BorderParameterizer_3, SparseLinearAlgebraTraits_d>}  \\
\ccRefIdfierPage{CGAL::LSCM_parameterizer_3<ParameterizationMesh_3, BorderParameterizer_3, SparseLinearAlgebraTraits_d>}  \\
\ccRefIdfierPage{CGAL::Mean_value_coordinates_parameterizer_3<ParameterizationMesh_3, BorderParameterizer_3, SparseLinearAlgebraTraits_d>}  \\


\end{ccRefClass}

% +------------------------------------------------------------------------+
%%RefPage: end of main body, begin of footer
\ccRefPageEnd
% EOF
% +------------------------------------------------------------------------+

