% +------------------------------------------------------------------------+
% | Reference manual page: Two_vertices_parameterizer_3.tex
% +------------------------------------------------------------------------+
% | 21.09.2005   Laurent Saboret, Pierre Alliez, Bruno Levy
% | Package: Surface_mesh_parameterization
% |
\RCSdef{\RCSTwoverticesparameterizerRev}{$Id$}
\RCSdefDate{\RCSTwoverticesparameterizerDate}{$Date$}
% |
\ccRefPageBegin
%%RefPage: end of header, begin of main body
% +------------------------------------------------------------------------+


\begin{ccRefClass}{Two_vertices_parameterizer_3<ParameterizationMesh_3>}

%% \ccHtmlCrossLink{}     %% add further rules for cross referencing links
%% \ccHtmlIndexC[class]{} %% add further index entries


\ccDefinition

\ccc{Two_vertices_parameterizer_3} is the default border parameterizer
for Least Squares Conformal Maps parameterization.

% The section below is automatically generated. Do not edit!
%START-AUTO(\ccDefinition)

The class \ccc{Two_vertices_parameterizer_3} parameterizes two extreme vertices of a 3D surface. This kind of border parameterization is used by free border parameterizations.

Implementation note: To simplify the implementation, \ccc{BorderParameterizer_3} models know only the \ccc{ParameterizationMesh_3} class. They do not know the parameterization algorithm requirements or the kind of sparse linear system used.

%END-AUTO(\ccDefinition)

\ccInclude{CGAL/Two_vertices_parameterizer_3.h}


\ccIsModel

% The section below is automatically generated. Do not edit!
%START-AUTO(\ccIsModel)

Model of the \ccc{BorderParameterizer_3} concept.

%END-AUTO(\ccIsModel)


\ccHeading{Design Pattern}

% The section below is automatically generated. Do not edit!
%START-AUTO(\ccHeading{Design Pattern})

\ccc{BorderParameterizer_3} models are Strategies [GHJV95]: they implement a strategy of border parameterization for models of \ccc{ParameterizationMesh_3}.

%END-AUTO(\ccHeading{Design Pattern})


\ccParameters

The full template declaration is:

% The section below is automatically generated. Do not edit!
%START-AUTO(\ccParameters)

template$<$  \\
class \ccc{ParameterizationMesh_3}$>$   \\
class \ccc{Two_vertices_parameterizer_3};

%END-AUTO(\ccParameters)


\ccTypes

% The section below is automatically generated. Do not edit!
%START-AUTO(\ccTypes)

\ccNestedType{Adaptor}
{
Export \ccc{ParameterizationMesh_3} template parameter.
}
\ccGlue

%END-AUTO(\ccTypes)


\ccCreation
\ccCreationVariable{bp}  %% choose variable name for \ccMethod

\ccConstructor{Two_vertices_parameterizer_3();}{default constructor.}

% The section below is automatically generated. Do not edit!
%START-AUTO(\ccCreation)
%END-AUTO(\ccCreation)


\ccOperations

% The section below is automatically generated. Do not edit!
%START-AUTO(\ccOperations)

\ccMethod{Parameterizer_traits_3<Adaptor>::Error_code parameterize_border(Adaptor& mesh);}
{
Map two extreme vertices of the 3D mesh and mark them as {\em parameterized}.
Map two extreme vertices of the 3D mesh and mark them as {\em parameterized}. Return false on error.
}
\ccGlue
\ccMethod{bool is_border_convex();}
{
Indicate if border's shape is convex. Meaningless for free border parameterization algorithms.
}
\ccGlue

%END-AUTO(\ccOperations)


\ccSeeAlso

\ccRefIdfierPage{CGAL::Circular_border_arc_length_parameterizer_3<ParameterizationMesh_3>}  \\
\ccRefIdfierPage{CGAL::Circular_border_uniform_parameterizer_3<ParameterizationMesh_3>}  \\
\ccRefIdfierPage{CGAL::Square_border_arc_length_parameterizer_3<ParameterizationMesh_3>}  \\
\ccRefIdfierPage{CGAL::Square_border_uniform_parameterizer_3<ParameterizationMesh_3>}  \\


\end{ccRefClass}

% +------------------------------------------------------------------------+
%%RefPage: end of main body, begin of footer
\ccRefPageEnd
% EOF
% +------------------------------------------------------------------------+

