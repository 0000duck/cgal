% +------------------------------------------------------------------------+
% | Reference manual page: Parameterizer_traits_3.tex
% +------------------------------------------------------------------------+
% | 21.09.2005   Laurent Saboret, Pierre Alliez, Bruno Levy
% | Package: Surface_mesh_parameterization
% |
\RCSdef{\RCSParameterizertraitsRev}{$Id$}
\RCSdefDate{\RCSParameterizertraitsDate}{$Date$}
% |
\ccRefPageBegin
%%RefPage: end of header, begin of main body
% +------------------------------------------------------------------------+


\begin{ccRefClass}{Parameterizer_traits_3<ParameterizationMesh_3>}

%% \ccHtmlCrossLink{}     %% add further rules for cross referencing links
%% \ccHtmlIndexC[class]{} %% add further index entries


\ccDefinition

% The section below is automatically generated. Do not edit!
%START-AUTO(\ccDefinition)

The class \ccc{Parameterizer_traits_3} is the base class of all parameterization methods. This class is a pure virtual class, thus cannot be instantiated.

This class doesn't do much. Its main goal is to ensure that subclasses will be proper models of the \ccc{ParameterizerTraits_3} concept:\begin{itemize}
\item \ccc{Parameterizer_traits_3} defines the \ccc{Error_code} list of errors detected by this package\item \ccc{Parameterizer_traits_3} declares a pure virtual method parameterize()\end{itemize}

%END-AUTO(\ccDefinition)

% The section below is automatically generated. Do not edit!
%START-AUTO(\ccInclude)

\ccInclude{CGAL/Parameterizer_traits_3.h}

%END-AUTO(\ccInclude)


\ccIsModel

% The section below is automatically generated. Do not edit!
%START-AUTO(\ccIsModel)

Model of the \ccc{ParameterizerTraits_3} concept (although you cannot instantiate this class).

%END-AUTO(\ccIsModel)


\ccHeading{Design Pattern}

% The section below is automatically generated. Do not edit!
%START-AUTO(\ccHeading{Design Pattern})

\ccc{ParameterizerTraits_3} models are Strategies \cite{cgal:ghjv-dpero-95}: they implement a strategy of surface parameterization for models of \ccc{ParameterizationMesh_3}.

%END-AUTO(\ccHeading{Design Pattern})


\ccParameters

The full template declaration is:

% The section below is automatically generated. Do not edit!
%START-AUTO(\ccParameters)

template$<$  \\
class \ccc{ParameterizationMesh_3}$>$   \\
class \ccc{Parameterizer_traits_3};

%END-AUTO(\ccParameters)


\ccTypes

% The section below is automatically generated. Do not edit!
%START-AUTO(\ccTypes)

\ccNestedType{Adaptor}
{
Export \ccc{ParameterizationMesh_3} template parameter.
}
\ccGlue
\ccNestedType{NT}
{
[protected] \\
}
\ccGlue
\ccNestedType{Point_2}
{
[protected] \\
}
\ccGlue
\ccNestedType{Point_3}
{
[protected] \\
}
\ccGlue
\ccNestedType{Vector_2}
{
[protected] \\
}
\ccGlue
\ccNestedType{Vector_3}
{
[protected] \\
}
\ccGlue
\ccNestedType{Facet}
{
[protected] \\
}
\ccGlue
\ccNestedType{Facet_handle}
{
[protected] \\
}
\ccGlue
\ccNestedType{Facet_const_handle}
{
[protected] \\
}
\ccGlue
\ccNestedType{Facet_iterator}
{
[protected] \\
}
\ccGlue
\ccNestedType{Facet_const_iterator}
{
[protected] \\
}
\ccGlue
\ccNestedType{Vertex}
{
[protected] \\
}
\ccGlue
\ccNestedType{Vertex_handle}
{
[protected] \\
}
\ccGlue
\ccNestedType{Vertex_const_handle}
{
[protected] \\
}
\ccGlue
\ccNestedType{Vertex_iterator}
{
[protected] \\
}
\ccGlue
\ccNestedType{Vertex_const_iterator}
{
[protected] \\
}
\ccGlue
\ccNestedType{Border_vertex_iterator}
{
[protected] \\
}
\ccGlue
\ccNestedType{Border_vertex_const_iterator}
{
[protected] \\
}
\ccGlue
\ccNestedType{Vertex_around_facet_circulator}
{
[protected] \\
}
\ccGlue
\ccNestedType{Vertex_around_facet_const_circulator}
{
[protected] \\
}
\ccGlue
\ccNestedType{Vertex_around_vertex_circulator}
{
[protected] \\
}
\ccGlue
\ccNestedType{Vertex_around_vertex_const_circulator}
{
[protected] \\
}
\ccGlue

%END-AUTO(\ccTypes)


\ccConstants

% The section below is automatically generated. Do not edit!
%START-AUTO(\ccConstants)

\ccEnum{enum Error_code { OK, ERROR_EMPTY_MESH, ERROR_NON_TRIANGULAR_MESH, ERROR_NO_TOPOLOGICAL_DISC, ERROR_BORDER_TOO_SHORT, ERROR_NON_CONVEX_BORDER, ERROR_CANNOT_SOLVE_LINEAR_SYSTEM, ERROR_NO_1_TO_1_MAPPING, ERROR_OUT_OF_MEMORY, ERROR_WRONG_PARAMETER };}
{
List of errors detected by this package.
}
\ccGlue
\ccCommentHeading{Values}
\begin{description}
\item \ccc{OK}: Success. \item \ccc{ERROR_EMPTY_MESH}: Input mesh is empty. \item \ccc{ERROR_NON_TRIANGULAR_MESH}: Input mesh is not triangular. \item \ccc{ERROR_NO_TOPOLOGICAL_DISC}: Input mesh is not a topological disc. \item \ccc{ERROR_BORDER_TOO_SHORT}: This border parameterization requires a longer border. \item \ccc{ERROR_NON_CONVEX_BORDER}: This parameterization method requires a convex border. \item \ccc{ERROR_CANNOT_SOLVE_LINEAR_SYSTEM}: Cannot solve linear system. \item \ccc{ERROR_NO_1_TO_1_MAPPING}: Parameterization failed: no one-to-one mapping. \item \ccc{ERROR_OUT_OF_MEMORY}: Not enough memory. \item \ccc{ERROR_WRONG_PARAMETER}: A method received an unexpected parameter. \end{description}
\ccGlue

%END-AUTO(\ccConstants)


\ccCreation
\ccCreationVariable{param}  %% variable name used by \ccMethod below

% The section below is automatically generated. Do not edit!
%START-AUTO(\ccCreation)
%END-AUTO(\ccCreation)


\ccOperations

% The section below is automatically generated. Do not edit!
%START-AUTO(\ccOperations)

\ccMethod{virtual Error_code parameterize(Adaptor& mesh);}
{
[pure virtual] \\
Compute a one-to-one mapping from a 3D surface \ccc{mesh} to a piece of the 2D space. The mapping is linear by pieces (linear in each triangle). The result is the (u, v) pair image of each vertex of the 3D surface.
\ccCommentHeading{Preconditions}\begin{itemize}
\item \ccc{mesh} must be a surface with one connected component.\item \ccc{mesh} must be a triangular mesh. \end{itemize}
}
\ccGlue
\ccMethod{static const char* get_error_message(int error_code);}
{
[static] \\
Get message (in English) corresponding to an error code
\ccCommentHeading{Parameters}
\ccc{error_code}: The code returned by parameterize() 
\ccCommentHeading{Returns} The string describing the error code
}
\ccGlue

%END-AUTO(\ccOperations)


\ccSeeAlso

\ccRefIdfierPage{CGAL::Fixed_border_parameterizer_3<ParameterizationMesh_3, BorderParameterizer_3, SparseLinearAlgebraTraits_d>}  \\
\ccRefIdfierPage{CGAL::Barycentric_mapping_parameterizer_3<ParameterizationMesh_3, BorderParameterizer_3, SparseLinearAlgebraTraits_d>}  \\
\ccRefIdfierPage{CGAL::Discrete_authalic_parameterizer_3<ParameterizationMesh_3, BorderParameterizer_3, SparseLinearAlgebraTraits_d>}  \\
\ccRefIdfierPage{CGAL::Discrete_conformal_map_parameterizer_3<ParameterizationMesh_3, BorderParameterizer_3, SparseLinearAlgebraTraits_d>}  \\
\ccRefIdfierPage{CGAL::LSCM_parameterizer_3<ParameterizationMesh_3, BorderParameterizer_3, SparseLinearAlgebraTraits_d>}  \\
\ccRefIdfierPage{CGAL::Mean_value_coordinates_parameterizer_3<ParameterizationMesh_3, BorderParameterizer_3, SparseLinearAlgebraTraits_d>}  \\


\end{ccRefClass}

% +------------------------------------------------------------------------+
%%RefPage: end of main body, begin of footer
\ccRefPageEnd
% EOF
% +------------------------------------------------------------------------+

