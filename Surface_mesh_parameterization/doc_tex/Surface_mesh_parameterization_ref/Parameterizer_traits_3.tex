% +------------------------------------------------------------------------+
% | Reference manual page: Parameterizer_traits_3.tex
% +------------------------------------------------------------------------+
% | 21.09.2005   Laurent Saboret, Pierre Alliez, Bruno Levy
% | Package: Surface_mesh_parameterization
% |
\RCSdef{\RCSParameterizertraitsRev}{$Id$}
\RCSdefDate{\RCSParameterizertraitsDate}{$Date$}
% |
\ccRefPageBegin
%%RefPage: end of header, begin of main body
% +------------------------------------------------------------------------+


\begin{ccRefClass}{Parameterizer_traits_3<ParameterizationMesh_3>}

%% \ccHtmlCrossLink{}     %% add further rules for cross referencing links
%% \ccHtmlIndexC[class]{} %% add further index entries


\ccDefinition

% The section below is automatically generated. Do not edit!
%START-AUTO(\ccDefinition)

The class \ccc{Parameterizer_traits_3} is the base class of all parameterization methods. This class is a pure virtual class, thus cannot be instantiated.

This class doesn't do much. Its main goal is to ensure that subclasses will be proper models of the \ccc{ParameterizerTraits_3} concept:\begin{itemize}
\item \ccc{Parameterizer_traits_3} defines the \ccc{Error_code} list of errors detected by this package\item \ccc{Parameterizer_traits_3} declares a pure virtual method parameterize()\end{itemize}

%END-AUTO(\ccDefinition)

\ccInclude{CGAL/Parameterizer_traits_3.h}


\ccIsModel

% The section below is automatically generated. Do not edit!
%START-AUTO(\ccIsModel)

Model of the \ccc{ParameterizerTraits_3} concept (although you cannot instantiate this class).

%END-AUTO(\ccIsModel)


\ccParameters

The full template declaration is:

% The section below is automatically generated. Do not edit!
%START-AUTO(\ccParameters)

template$<$  \\
class \ccc{ParameterizationMesh_3}$>$   \\
class \ccc{Parameterizer_traits_3};

%END-AUTO(\ccParameters)


\ccTypes

% The section below is automatically generated. Do not edit!
%START-AUTO(\ccTypes)

\ccNestedType{Adaptor}
{
Export \ccc{ParameterizationMesh_3} template parameter.
}
\ccGlue

%END-AUTO(\ccTypes)


\ccConstants

% The section below is automatically generated. Do not edit!
%START-AUTO(\ccConstants)

\ccEnum{enum Error_code { OK, ERROR_EMPTY_MESH, ERROR_NON_TRIANGULAR_MESH, ERROR_NO_SURFACE_MESH,  ERROR_INVALID_BORDER, ERROR_CANNOT_SOLVE_LINEAR_SYSTEM, ERROR_NO_1_TO_1_MAPPING, ERROR_NOT_ENOUGH_MEMORY,  ERROR_WRONG_PARAMETER  };}
{
List of errors detected by this package.
}
\ccGlue
\begin{description}
\item[Enumeration values: ]
\begin{description}
\item[OK
]Success. \item[\ccc{ERROR_EMPTY_MESH}
]input mesh is empty \item[\ccc{ERROR_NON_TRIANGULAR_MESH}
]input mesh is not triangular \item[\ccc{ERROR_NO_SURFACE_MESH}
]input mesh is not a surface \item[\ccc{ERROR_INVALID_BORDER}
]parameterization requires a convex border \item[\ccc{ERROR_CANNOT_SOLVE_LINEAR_SYSTEM}
]cannot solve linear system \item[\ccc{ERROR_NO_1_TO_1_MAPPING}
]parameterization does not ensure a one-to-one mapping \item[\ccc{ERROR_NOT_ENOUGH_MEMORY}
]not enough memory \item[\ccc{ERROR_WRONG_PARAMETER}
]a method received an unexpected parameter \end{description}
\end{description}
\ccGlue

%END-AUTO(\ccConstants)


\ccCreation
\ccCreationVariable{param}  %% variable name used by \ccMethod below

\ccConstructor{Parameterizer_traits_3();}{default constructor.}

% The section below is automatically generated. Do not edit!
%START-AUTO(\ccCreation)



%END-AUTO(\ccCreation)


\ccOperations

% The section below is automatically generated. Do not edit!
%START-AUTO(\ccOperations)

\ccMethod{virtual Error_code parameterize (Adaptor & mesh);}
{
Compute a one-to-one mapping from a 3D surface 'mesh' to a piece of the 2D space. The mapping is linear by pieces (linear in each triangle). The result is the (u,v) pair image of each vertex of the 3D surface.
Preconditions:\begin{itemize}
\item 'mesh' must be a surface with one connected component.\item 'mesh' must be a triangular mesh. \end{itemize}
}
\ccGlue
\ccMethod{static const char* get_error_message (int error_code);}
{
Get message (in english) corresponding to an error code 
}
\ccGlue
\begin{description}
\item[Parameters:]
\begin{description}
\item[\ccc{error_code}]The code returned by parameterize() \end{description}
\end{description}
\begin{description}
\item[Returns:]The string describing the error code \end{description}
\ccGlue

%END-AUTO(\ccOperations)


\ccSeeAlso

\ccRefIdfierPage{CGAL::Fixed_border_parameterizer_3<ParameterizationMesh_3, BorderParameterizer_3, SparseLinearAlgebraTraits_d>}  \\
\ccRefIdfierPage{CGAL::Barycentric_mapping_parameterizer_3<ParameterizationMesh_3, BorderParameterizer_3, SparseLinearAlgebraTraits_d>}  \\
\ccRefIdfierPage{CGAL::Discrete_authalic_parameterizer_3<ParameterizationMesh_3, BorderParameterizer_3, SparseLinearAlgebraTraits_d>}  \\
\ccRefIdfierPage{CGAL::Discrete_conformal_map_parameterizer_3<ParameterizationMesh_3, BorderParameterizer_3, SparseLinearAlgebraTraits_d>}  \\
\ccRefIdfierPage{CGAL::LSCM_parameterizer_3<ParameterizationMesh_3, BorderParameterizer_3, SparseLinearAlgebraTraits_d>}  \\
\ccRefIdfierPage{CGAL::Mean_value_coordinates_parameterizer_3<ParameterizationMesh_3, BorderParameterizer_3, SparseLinearAlgebraTraits_d>}  \\


\end{ccRefClass}

% +------------------------------------------------------------------------+
%%RefPage: end of main body, begin of footer
\ccRefPageEnd
% EOF
% +------------------------------------------------------------------------+

