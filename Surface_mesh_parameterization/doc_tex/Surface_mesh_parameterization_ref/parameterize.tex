% +------------------------------------------------------------------------+
% | Reference manual page: parameterize.tex
% +------------------------------------------------------------------------+
% | 23.09.2005   Laurent Saboret, Pierre Alliez, Bruno Levy
% | Package: Surface_mesh_parameterization
% |
\RCSdef{\RCSparameterizeRev}{$Id$}
\RCSdefDate{\RCSparameterizeDate}{$Date$}
% |
\ccRefPageBegin
%%RefPage: end of header, begin of main body
% +------------------------------------------------------------------------+

\begin{ccRefFunction}{parameterize}  %% add template arg's if necessary

%% \ccHtmlCrossLink{}     %% add further rules for cross referencing links
%% \ccHtmlIndexC[function]{} %% add further index entries

\ccDefinition

\ccc{CGAL::parameterize()} is the main entry-point of the \ccc{Surface_mesh_parameterization} package.

It computes a one-to-one mapping from a 3D triangle surface 'mesh' to a simple 2D domain.
The mapping is piecewise linear on the triangle mesh.
The result is a pair (u,v) of parameter coordinates for each vertex of the input mesh.
One-to-one mapping may be guaranteed or not, depending on the chosen \ccc{ParametizerTraits} algorithm.

The \ccc{CGAL::parameterize()} function exists in two flavors, to provide a default parameterization algorithm
of Floater Mean Value Coordinates.

\ccInclude{CGAL/parameterize.h}

% The section below is automatically generated. Do not edit!
%START-AUTO(\ccDefinition)

% Reduce left margin
\ccThree{123456789012345}{6789012}{}

\ccFunction{template<class ParameterizationMesh_3> Parameterizer_traits_3<ParameterizationMesh_3>::Error_code parameterize(ParameterizationMesh_3& mesh);}
{
Compute a one-to-one mapping from a 3D triangle surface \ccc{mesh} to a 2D circle, using Floater Mean Value Coordinates algorithm. A one-to-one mapping is guaranteed.  \\
The mapping is piecewise linear on the input mesh triangles. The result is a (u, v) pair of parameter coordinates for each vertex of the input mesh.
\ccCommentHeading{Preconditions}  \\
\ccc{mesh} must be a surface with one connected component. \ccc{mesh} must be a triangular mesh.
\ccCommentHeading{Parameters}  \\
\ccc{mesh}: 3D mesh, model of \ccc{ParameterizationMesh_3} concept.
}
\ccGlue
\ccFunction{template<class ParameterizationMesh_3, class ParameterizerTraits_3> Parameterizer_traits_3<ParameterizationMesh_3>::Error_code parameterize(ParameterizationMesh_3& mesh, ParameterizerTraits_3 parameterizer);}
{
Compute a one-to-one mapping from a 3D triangle surface \ccc{mesh} to a simple 2D domain. The mapping is piecewise linear on the triangle mesh. The result is a pair (u, v) of parameter coordinates for each vertex of the input mesh.  \\
One-to-one mapping may be guaranteed or not, depending on the chosen \ccc{ParametizerTraits_3} algorithm.
\ccCommentHeading{Preconditions}  \\
\ccc{mesh} must be a surface with one connected component. \ccc{mesh} must be a triangular mesh. The mesh border must be mapped onto a convex polygon (for fixed border parameterizations).
\ccCommentHeading{Parameters}  \\
\ccc{mesh}: 3D mesh, model of \ccc{ParameterizationMesh_3}. \ccc{parameterizer}: Parameterization method for \ccc{mesh}.
}
\ccGlue

%END-AUTO(\ccDefinition)

\ccSeeAlso

\ccRefIdfierPage{CGAL::Barycentric_mapping_parameterizer_3<ParameterizationMesh_3, BorderParameterizer_3, SparseLinearAlgebraTraits_d>}  \\
\ccRefIdfierPage{CGAL::Discrete_authalic_parameterizer_3<ParameterizationMesh_3, BorderParameterizer_3, SparseLinearAlgebraTraits_d>}  \\
\ccRefIdfierPage{CGAL::Discrete_conformal_map_parameterizer_3<ParameterizationMesh_3, BorderParameterizer_3, SparseLinearAlgebraTraits_d>}  \\
\ccRefIdfierPage{CGAL::LSCM_parameterizer_3<ParameterizationMesh_3, BorderParameterizer_3, SparseLinearAlgebraTraits_d>}  \\
\ccRefIdfierPage{CGAL::Mean_value_coordinates_parameterizer_3<ParameterizationMesh_3, BorderParameterizer_3, SparseLinearAlgebraTraits_d>}  \\

\ccExample

See \ccc{Simple_parameterization.cpp} example.

\ccImplementation

This function simply calls the parameterize() method of the parameterization
algorithm chosen.

\end{ccRefFunction}

% +------------------------------------------------------------------------+
%%RefPage: end of main body, begin of footer
\ccRefPageEnd
% EOF
% +------------------------------------------------------------------------+

