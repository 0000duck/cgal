% +------------------------------------------------------------------------+
% | Reference manual page: Parameterization_mesh_feature_extractor.tex
% +------------------------------------------------------------------------+
% | 21.09.2005   Laurent Saboret, Pierre Alliez, Bruno Levy
% | Package: Surface_mesh_parameterization
% |
\RCSdef{\RCSMeshadaptorfeatureextractorRev}{$Id$}
\RCSdefDate{\RCSMeshadaptorfeatureextractorDate}{$Date$}
% |
\ccRefPageBegin
%%RefPage: end of header, begin of main body
% +------------------------------------------------------------------------+


\begin{ccRefClass}{Parameterization_mesh_feature_extractor<ParameterizationMesh_3>}

%% \ccHtmlCrossLink{}     %% add further rules for cross referencing links
%% \ccHtmlIndexC[class]{} %% add further index entries


\ccDefinition

% The section below is automatically generated. Do not edit!
%START-AUTO(\ccDefinition)

The class \ccc{Parameterization_mesh_feature_extractor} computes features (genus, borders, ...) of a 3D surface, model of the \ccc{ParameterizationMesh_3} concept.

%END-AUTO(\ccDefinition)

% The section below is automatically generated. Do not edit!
%START-AUTO(\ccInclude)

\ccInclude{CGAL/Parameterization_mesh_feature_extractor.h}

%END-AUTO(\ccInclude)


\ccParameters

The full template declaration is:

% The section below is automatically generated. Do not edit!
%START-AUTO(\ccParameters)

template$<$  \\
class \ccc{ParameterizationMesh_3}$>$   \\
class \ccc{Parameterization_mesh_feature_extractor};

%END-AUTO(\ccParameters)


\ccTypes

% The section below is automatically generated. Do not edit!
%START-AUTO(\ccTypes)

\ccNestedType{Adaptor}
{
Export \ccc{ParameterizationMesh_3} template parameter.
}
\ccGlue
\ccNestedType{Border}
{
Type representing a border = STL container of vertex handles.
}
\ccGlue
\ccNestedType{Skeleton}
{
Type representing the list of all borders of the mesh = STL container of Border elements.
}
\ccGlue

%END-AUTO(\ccTypes)


\ccCreation
\ccCreationVariable{extractor}  %% choose variable name for \ccMethod

% The section below is automatically generated. Do not edit!
%START-AUTO(\ccCreation)

\ccConstructor{Parameterization_mesh_feature_extractor(Adaptor& mesh);}
{
Constructor.
CAUTION: This class caches the result of feature extractions =$>$ The caller must NOT modify \ccc{mesh} during the \ccc{Parameterization_mesh_feature_extractor} life cycle.
}
\ccGlue

%END-AUTO(\ccCreation)


\ccOperations

% The section below is automatically generated. Do not edit!
%START-AUTO(\ccOperations)

\ccTagFullDeclarations

\ccMethod{int get_nb_borders();}
{
Get number of borders.
}
\ccGlue
\ccMethod{const Skeleton& get_borders();}
{
Get extracted borders. The longest border is the first one.
}
\ccGlue
\ccMethod{const Border& get_longest_border();}
{
Get longest border.
}
\ccGlue
\ccMethod{int get_nb_connex_components();}
{
Get \# of connected components.
}
\ccGlue
\ccMethod{int get_genus();}
{
Get the genus.
}
\ccGlue

%END-AUTO(\ccOperations)


\ccSeeAlso

\ccRefIdfierPage{ParameterizationMesh_3}  \\


\ccExample

See \ccc{Mesh_cutting_parameterization.cpp} example.


\end{ccRefClass}

% +------------------------------------------------------------------------+
%%RefPage: end of main body, begin of footer
\ccRefPageEnd
% EOF
% +------------------------------------------------------------------------+

