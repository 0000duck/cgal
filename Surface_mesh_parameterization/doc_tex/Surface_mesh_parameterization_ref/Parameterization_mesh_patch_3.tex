% +------------------------------------------------------------------------+
% | Reference manual page: Parameterization_mesh_patch_3.tex
% +------------------------------------------------------------------------+
% | 21.09.2005   Laurent Saboret, Pierre Alliez, Bruno Levy
% | Package: Surface_mesh_parameterization
% |
\RCSdef{\RCSMeshadaptorpatchRev}{$Id$}
\RCSdefDate{\RCSMeshadaptorpatchDate}{$Date$}
% |
\ccRefPageBegin
%%RefPage: end of header, begin of main body
% +------------------------------------------------------------------------+


\begin{ccRefClass}{Parameterization_mesh_patch_3<ParameterizationPatchableMesh_3>}

%% \ccHtmlCrossLink{}     %% add further rules for cross referencing links
%% \ccHtmlIndexC[class]{} %% add further index entries


\ccDefinition

% The section below is automatically generated. Do not edit!
%START-AUTO(\ccDefinition)

\ccc{Parameterization_mesh_patch_3} is a Decorator class to {\em virtually} cut a patch in a \ccc{ParameterizationPatchableMesh_3} 3D surface. Only the patch is exported, making the 3D surface look like a topological disk.

The input mesh can be of any genus, but it has to come with a {\em seam} that describes the border of a topological disc. This border may be an actual border of the mesh or a virtual border.

%END-AUTO(\ccDefinition)

% The section below is automatically generated. Do not edit!
%START-AUTO(\ccInclude)

\ccInclude{CGAL/Parameterization_mesh_patch_3.h}

%END-AUTO(\ccInclude)


\ccIsModel

% The section below is automatically generated. Do not edit!
%START-AUTO(\ccIsModel)

Model of \ccc{ParameterizationMesh_3} concept, whose purpose is to allow the \ccc{Surface_mesh_parameterization} package to access meshes in a uniform manner.

%END-AUTO(\ccIsModel)


\ccHeading{Design Pattern}

% The section below is automatically generated. Do not edit!
%START-AUTO(\ccHeading{Design Pattern})

\ccc{Parameterization_mesh_patch_3} is a Decorator \cite{cgal:ghjv-dpero-95}: it changes the behavior of a \ccc{ParameterizationPatchableMesh_3} 3D surface while keeping its \ccc{ParameterizationMesh_3} interface.

%END-AUTO(\ccHeading{Design Pattern})


\ccParameters

The full template declaration is:

% The section below is automatically generated. Do not edit!
%START-AUTO(\ccParameters)

template$<$  \\
class \ccc{ParameterizationPatchableMesh_3}$>$   \\
class \ccc{Parameterization_mesh_patch_3};

\begin{description}
\item[Parameters:]
\begin{description}
\item[\ccc{ParameterizationPatchableMesh_3}]3D surface mesh. \end{description}
\end{description}

%END-AUTO(\ccParameters)


\ccTypes

The following mutable handles, iterators, and circulators have appropriate
non-mutable counterparts, i.e. \ccc{const_handle},
\ccc{const_iterator}, and \ccc{const_circulator}. The mutable types are
assignable to their non-mutable counterparts.  Both circulators are
assignable to the \ccc{Vertex_iterator}. The iterators are
assignable to the respective handle types. Wherever the handles appear
in function parameter lists, the corresponding iterators can be used as
well.

% The section below is automatically generated. Do not edit!
%START-AUTO(\ccTypes)

\ccNestedType{Adaptor}
{
Export template parameter.
}
\ccGlue

\subsubsection{Types implementing the ParameterizationMesh\_3 interface}

\ccNestedType{NT}
{
Number type to represent coordinates.
}
\ccGlue
\ccNestedType{Point_2}
{
2D point that represents (u,v) coordinates computed by parameterization methods. Must provide X() and Y() methods.
}
\ccGlue
\ccNestedType{Point_3}
{
3D point that represents vertices coordinates. Must provide X() and Y() methods.
}
\ccGlue
\ccNestedType{Vector_2}
{
2D vector. Must provide X() and Y() methods.
}
\ccGlue
\ccNestedType{Vector_3}
{
3D vector. Must provide X() and Y() methods.
}
\ccGlue
\ccNestedType{Facet}
{
Opaque type representing a facet of the 3D mesh. No methods are expected.
}
\ccGlue
\ccNestedType{Facet_handle}
{
Handle to a facet. Model of the Handle concept.
}
\ccGlue
\ccNestedType{Facet_const_handle}
{
}
\ccGlue
\ccNestedType{Facet_iterator}
{
Iterator over all mesh facets. Model of the ForwardIterator concept.
}
\ccGlue
\ccNestedType{Facet_const_iterator}
{
}
\ccGlue
\ccNestedType{Vertex}
{
Opaque type representing a vertex of the 3D mesh. No methods are expected.
}
\ccGlue
\ccNestedType{Vertex_handle}
{
Handle to a vertex. Model of the Handle concept.
}
\ccGlue
\ccNestedType{Vertex_const_handle}
{
}
\ccGlue
\ccNestedType{Vertex_iterator}
{
Iterator over all vertices of a mesh. Model of the ForwardIterator concept.
}
\ccGlue
\ccNestedType{Vertex_const_iterator}
{
}
\ccGlue
\ccNestedType{Border_vertex_iterator}
{
Iterator over vertices of the mesh {\em main border}. Model of the ForwardIterator concept.
}
\ccGlue
\ccNestedType{Border_vertex_const_iterator}
{
}
\ccGlue
\ccNestedType{Vertex_around_facet_circulator}
{
Counter-clockwise circulator over a facet's vertices. Model of the BidirectionalCirculator concept.
}
\ccGlue
\ccNestedType{Vertex_around_facet_const_circulator}
{
}
\ccGlue
\ccNestedType{Vertex_around_vertex_circulator}
{
Clockwise circulator over the vertices incident to a vertex. Model of the BidirectionalCirculator concept.
}
\ccGlue
\ccNestedType{Vertex_around_vertex_const_circulator}
{
}
\ccGlue

%END-AUTO(\ccTypes)


\ccHeading{Variables}

% The section below is automatically generated. Do not edit!
%START-AUTO(\ccHeading{Variables})

\ccVariable{Adaptor& m_mesh_adaptor;}
{
The decorated mesh.
}
\ccGlue

%END-AUTO(\ccHeading{Variables})


\ccCreation
\ccCreationVariable{mesh}  %% define variable name used by \ccMethod below

% The section below is automatically generated. Do not edit!
%START-AUTO(\ccCreation)

\ccConstructor{Parameterization_mesh_patch_3(Adaptor& mesh, InputIterator first_seam_vertex, InputIterator end_seam_vertex);}
{
Create a Decorator for an existing \ccc{ParameterizationPatchableMesh_3} mesh. The input mesh can be of any genus, but it has to come with a {\em seam} that describes the border of a topological disc. This border may be an actual border of the mesh or a virtual border.
Preconditions:\begin{itemize}
\item \ccc{first_seam_vertex} -$>$ \ccc{end_seam_vertex} defines the outer seam, i.e. \ccc{Parameterization_mesh_patch_3} will export the {\em right} of the seam.\item The {\em seam} is given as a container of \ccc{Adaptor::Vertex_handle} elements.\item The {\em seam} is implicitely a loop. The first vertex should $\ast$not$\ast$ be duplicated at the end. \end{itemize}
}
\ccGlue

%END-AUTO(\ccCreation)


\ccOperations

The following methods returning a mutable handle, iterator, or circulator have appropriate
non-mutable counterpart methods, i.e. \ccc{const}, returning a \ccc{const_handle},
\ccc{const_iterator}, or \ccc{const_circulator}.

% The section below is automatically generated. Do not edit!
%START-AUTO(\ccOperations)

\ccMethod{Adaptor& get_decorated_mesh();}
{
}
\ccGlue
\begin{description}
\item[Returns:]the decorated mesh. \end{description}
\ccGlue
\ccMethod{const Adaptor& get_decorated_mesh() const;}
{
}
\ccGlue

\subsubsection{Methods implementing the ParameterizationMesh\_3 interface}

\ccMethod{bool is_valid() const;}
{
Indicate if the mesh matches the \ccc{ParameterizationMesh_3} concept.
}
\ccGlue
\ccMethod{Vertex_iterator mesh_vertices_begin();}
{
Get iterator over first vertex of mesh.
}
\ccGlue
\ccMethod{Vertex_const_iterator mesh_vertices_begin() const;}
{
}
\ccGlue
\ccMethod{Vertex_iterator mesh_vertices_end();}
{
Get iterator over past-the-end vertex of mesh.
}
\ccGlue
\ccMethod{Vertex_const_iterator mesh_vertices_end() const;}
{
}
\ccGlue
\ccMethod{int count_mesh_vertices() const;}
{
Count the number of vertices of the mesh.
}
\ccGlue
\ccMethod{void index_mesh_vertices();}
{
Index vertices of the mesh from 0 to \ccc{count_mesh_vertices}()-1.
}
\ccGlue
\ccMethod{Border_vertex_iterator mesh_main_border_vertices_begin();}
{
Get iterator over first vertex of mesh's main border (aka {\em seam}).
}
\ccGlue
\ccMethod{Border_vertex_const_iterator mesh_main_border_vertices_begin() const;}
{
}
\ccGlue
\ccMethod{Border_vertex_iterator mesh_main_border_vertices_end();}
{
Get iterator over past-the-end vertex of mesh's main border (aka {\em seam}).
}
\ccGlue
\ccMethod{Border_vertex_const_iterator mesh_main_border_vertices_end() const;}
{
}
\ccGlue
\ccMethod{std::list<Vertex_handle> get_border(Vertex_handle seed_vertex);}
{
}
\ccGlue
\begin{description}
\item[Returns:]the border containing \ccc{seed_vertex} (or an empty list if not found). \end{description}
\begin{description}
\item[Parameters:]
\begin{description}
\item[\ccc{seed_vertex}]a border vertex. \end{description}
\end{description}
\ccGlue
\ccMethod{Facet_iterator mesh_facets_begin();}
{
Get iterator over first facet of mesh.
}
\ccGlue
\ccMethod{Facet_const_iterator mesh_facets_begin() const;}
{
}
\ccGlue
\ccMethod{Facet_iterator mesh_facets_end();}
{
Get iterator over past-the-end facet of mesh.
}
\ccGlue
\ccMethod{Facet_const_iterator mesh_facets_end() const;}
{
}
\ccGlue
\ccMethod{int count_mesh_facets() const;}
{
Count the number of facets of the mesh.
}
\ccGlue
\ccMethod{bool is_mesh_triangular() const;}
{
Return true of all mesh's facets are triangles.
}
\ccGlue
\ccMethod{int count_mesh_halfedges() const;}
{
Count the number of halfedges of the mesh.
}
\ccGlue
\ccMethod{Vertex_around_facet_circulator facet_vertices_begin(Facet_handle facet);}
{
Get circulator over facet's vertices.
}
\ccGlue
\ccMethod{Vertex_around_facet_const_circulator facet_vertices_begin(Facet_const_handle facet) const;}
{
}
\ccGlue
\ccMethod{int count_facet_vertices(Facet_const_handle facet) const;}
{
Count the number of vertices of a facet.
}
\ccGlue
\ccMethod{Point_3 get_vertex_position(Vertex_const_handle vertex) const;}
{
Get the 3D position of a vertex.
}
\ccGlue
\ccMethod{Point_2 get_vertex_uv(Vertex_const_handle vertex) const;}
{
Get/set the 2D position (u/v pair) of a vertex. Default value is undefined.
}
\ccGlue
\ccMethod{void set_vertex_uv(Vertex_handle vertex, const Point_2& uv);}
{
}
\ccGlue
\ccMethod{bool is_vertex_parameterized(Vertex_const_handle vertex) const;}
{
Get/set {\em is parameterized} field of vertex. Default value is undefined.
}
\ccGlue
\ccMethod{void set_vertex_parameterized(Vertex_handle vertex, bool parameterized);}
{
}
\ccGlue
\ccMethod{int get_vertex_index(Vertex_const_handle vertex) const;}
{
Get/set vertex index. Default value is undefined.
}
\ccGlue
\ccMethod{void set_vertex_index(Vertex_handle vertex, int index);}
{
}
\ccGlue
\ccMethod{int get_vertex_tag(Vertex_const_handle vertex) const;}
{
Get/set vertex' all purpose tag. Default value is undefined.
}
\ccGlue
\ccMethod{void set_vertex_tag(Vertex_handle vertex, int tag);}
{
}
\ccGlue
\ccMethod{bool is_vertex_on_border(Vertex_const_handle vertex) const;}
{
Return true if a vertex belongs to ANY mesh's border.
}
\ccGlue
\ccMethod{bool is_vertex_on_main_border(Vertex_const_handle vertex) const;}
{
Return true if a vertex belongs to the UNIQUE mesh's main border set by the constructor.
}
\ccGlue
\ccMethod{Vertex_around_vertex_circulator vertices_around_vertex_begin(Vertex_handle vertex, Vertex_handle start_position = Vertex_handle());}
{
Get circulator over the vertices incident to \ccc{vertex}. \ccc{start_position} defines the optional initial position of the circulator.
}
\ccGlue
\ccMethod{Vertex_around_vertex_const_circulator vertices_around_vertex_begin(Vertex_const_handle vertex, Vertex_const_handle start_position = Vertex_const_handle()) const;}
{
}
\ccGlue

%END-AUTO(\ccOperations)


\ccSeeAlso

\ccRefIdfierPage{CGAL::Parameterization_polyhedron_adaptor_3<Polyhedron_3_>}  \\


\ccExample

See \ccc{Mesh_cutting_parameterization.cpp} example.


\end{ccRefClass}

% +------------------------------------------------------------------------+
%%RefPage: end of main body, begin of footer
\ccRefPageEnd
% EOF
% +------------------------------------------------------------------------+

