% +------------------------------------------------------------------------+
% | Reference manual page: LSCM_parameterizer_3.tex
% +------------------------------------------------------------------------+
% | 21.09.2005   Laurent Saboret, Pierre Alliez, Bruno Levy
% | Package: Surface_mesh_parameterization
% |
\RCSdef{\RCSLSCMparameterizerRev}{$Id$}
\RCSdefDate{\RCSLSCMparameterizerDate}{$Date$}
% |
\ccRefPageBegin
%%RefPage: end of header, begin of main body
% +------------------------------------------------------------------------+


\begin{ccRefClass}{LSCM_parameterizer_3<ParameterizationMesh_3, BorderParameterizer_3, SparseLinearAlgebraTraits_d>}

%% \ccHtmlCrossLink{}     %% add further rules for cross referencing links
%% \ccHtmlIndexC[class]{} %% add further index entries


\ccDefinition

% The section below is automatically generated. Do not edit!
%START-AUTO(\ccDefinition)

The class \ccc{LSCM_parameterizer_3} implements the Least Squares Conformal Maps (LSCM) parameterization \cite{cgal:lprm-lscm-02}.

This is a conformal parameterization, i.e. it attempts to preserve angles.

This is a free border parameterization. No need to map the surface's border onto a convex polygon (only two pinned vertices are needed to ensure a unique solution), but one-to-one mapping is NOT guaranteed.

%END-AUTO(\ccDefinition)

% The section below is automatically generated. Do not edit!
%START-AUTO(\ccInclude)

\ccInclude{CGAL/LSCM_parameterizer_3.h}

%END-AUTO(\ccInclude)


\ccInheritsFrom

% The section below is automatically generated. Do not edit!
%START-AUTO(\ccInheritsFrom)

\ccc{Parameterizer_traits_3<ParameterizationMesh_3>}

%END-AUTO(\ccInheritsFrom)


\ccIsModel

% The section below is automatically generated. Do not edit!
%START-AUTO(\ccIsModel)

Model of the \ccc{ParameterizerTraits_3} concept.

%END-AUTO(\ccIsModel)


\ccHeading{Design Pattern}

% The section below is automatically generated. Do not edit!
%START-AUTO(\ccHeading{Design Pattern})

\ccc{LSCM_parameterizer_3} class is a Strategy \cite{cgal:ghjv-dpero-95}: it implements a strategy of surface parameterization for models of \ccc{ParameterizationMesh_3}.

%END-AUTO(\ccHeading{Design Pattern})


\ccParameters

The full template declaration is:

% The section below is automatically generated. Do not edit!
%START-AUTO(\ccParameters)

template$<$  \\
class \ccc{ParameterizationMesh_3},   \\
class \ccc{BorderParameterizer_3} = \ccc{Two_vertices_parameterizer_3<ParameterizationMesh_3>},   \\
class \ccc{SparseLinearAlgebraTraits_d} = \ccc{OpenNL::SymmetricLinearSolverTraits<typename ParameterizationMesh_3::NT>}$>$   \\
class \ccc{LSCM_parameterizer_3};

%END-AUTO(\ccParameters)


\ccTypes

% The section below is automatically generated. Do not edit!
%START-AUTO(\ccTypes)

\ccNestedType{Border_param}
{
Export \ccc{BorderParameterizer_3} template parameter.
}
\ccGlue
\ccNestedType{Sparse_LA}
{
Export \ccc{SparseLinearAlgebraTraits_d} template parameter.
}
\ccGlue

%END-AUTO(\ccTypes)


\ccCreation
\ccCreationVariable{param}  %% choose variable name for \ccMethod

% The section below is automatically generated. Do not edit!
%START-AUTO(\ccCreation)

\ccConstructor{LSCM_parameterizer_3(Border_param border_param = Border_param(), Sparse_LA sparse_la = Sparse_LA());}
{
Constructor.
}
\ccGlue
\begin{description}
\item[Parameters: ]
\begin{description}
\item[\ccc{border_param}]Object that maps the surface's border to 2D space \item[\ccc{sparse_la}]Traits object to access a sparse linear system \end{description}
\end{description}
\ccGlue

%END-AUTO(\ccCreation)


\ccOperations

% The section below is automatically generated. Do not edit!
%START-AUTO(\ccOperations)

\ccMethod{LSCM_parameterizer_3<Adaptor, Border_param, Sparse_LA>::Error_code parameterize(Adaptor& mesh);}
{
[virtual] \\
Compute a one-to-one mapping from a triangular 3D surface \ccc{mesh} to a piece of the 2D space. The mapping is linear by pieces (linear in each triangle). The result is the (u,v) pair image of each vertex of the 3D surface.
Preconditions:\begin{itemize}
\item \ccc{mesh} must be a surface with one connected component.\item \ccc{mesh} must be a triangular mesh. \end{itemize}
}
\ccGlue

%END-AUTO(\ccOperations)


\ccSeeAlso

\ccRefIdfierPage{CGAL::Parameterizer_traits_3<ParameterizationMesh_3>}  \\
\ccRefIdfierPage{CGAL::Fixed_border_parameterizer_3<ParameterizationMesh_3, BorderParameterizer_3, SparseLinearAlgebraTraits_d>}  \\
\ccRefIdfierPage{CGAL::Barycentric_mapping_parameterizer_3<ParameterizationMesh_3, BorderParameterizer_3, SparseLinearAlgebraTraits_d>}  \\
\ccRefIdfierPage{CGAL::Discrete_authalic_parameterizer_3<ParameterizationMesh_3, BorderParameterizer_3, SparseLinearAlgebraTraits_d>}  \\
\ccRefIdfierPage{CGAL::Discrete_conformal_map_parameterizer_3<ParameterizationMesh_3, BorderParameterizer_3, SparseLinearAlgebraTraits_d>}  \\
\ccRefIdfierPage{CGAL::Mean_value_coordinates_parameterizer_3<ParameterizationMesh_3, BorderParameterizer_3, SparseLinearAlgebraTraits_d>}  \\


\end{ccRefClass}

% +------------------------------------------------------------------------+
%%RefPage: end of main body, begin of footer
\ccRefPageEnd
% EOF
% +------------------------------------------------------------------------+

