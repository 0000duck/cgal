% +------------------------------------------------------------------------+
% | Reference manual page: Fixed_border_parameterizer_3.tex
% +------------------------------------------------------------------------+
% | 21.09.2005   Laurent Saboret, Pierre Alliez, Bruno Levy
% | Package: Surface_mesh_parameterization
% |
\RCSdef{\RCSFixedborderparameterizerRev}{$Id$}
\RCSdefDate{\RCSFixedborderparameterizerDate}{$Date$}
% |
\ccRefPageBegin
%%RefPage: end of header, begin of main body
% +------------------------------------------------------------------------+


\begin{ccRefClass}{Fixed_border_parameterizer_3<ParameterizationMesh_3, BorderParameterizer_3, SparseLinearAlgebraTraits_d>}

%% \ccHtmlCrossLink{}     %% add further rules for cross referencing links
%% \ccHtmlIndexC[class]{} %% add further index entries


\ccDefinition

% The section below is automatically generated. Do not edit!
%START-AUTO(\ccDefinition)

The class \ccc{Fixed_border_parameterizer_3} is the base class of fixed border parameterization methods (Tutte, Floater, ...).

One-to-one mapping is guaranteed if surface's border is mapped onto a convex polygon.

This class is a pure virtual class, thus cannot be instantiated. Anyway, it implements most of the parameterization algorithm parameterize(). Subclasses are Strategies \cite{cgal:ghjv-dpero-95} that modify the behavior of this algorithm:\begin{itemize}
\item They provide \ccc{BorderParameterizer_3} and \ccc{SparseLinearAlgebraTraits_d} template parameters that make sense.\item They implement \ccc{compute_w_ij}() to compute \ccc{w_ij} = (i, j) coefficient of matrix A for j neighbor vertex of i.\item They may implement an optimized version of \ccc{is_one_to_one_mapping}().\end{itemize}

%END-AUTO(\ccDefinition)

\ccInclude{CGAL/Fixed_border_parameterizer_3.h}


\ccIsModel

% The section below is automatically generated. Do not edit!
%START-AUTO(\ccIsModel)

Model of the \ccc{ParameterizerTraits_3} concept (although you cannot instantiate this class).

%END-AUTO(\ccIsModel)


\ccHeading{Design Pattern}

% The section below is automatically generated. Do not edit!
%START-AUTO(\ccHeading{Design Pattern})

\ccc{Fixed_border_parameterizer_3} class is a Strategy \cite{cgal:ghjv-dpero-95}: it implements (part of) a strategy of surface parameterization for models of \ccc{ParameterizationMesh_3}.

%END-AUTO(\ccHeading{Design Pattern})


\ccParameters

The full template declaration is:

% The section below is automatically generated. Do not edit!
%START-AUTO(\ccParameters)

template$<$  \\
class \ccc{ParameterizationMesh_3},   \\
class \ccc{BorderParameterizer_3} = \ccc{Circular_border_arc_length_parameterizer_3<ParameterizationMesh_3>},   \\
class \ccc{SparseLinearAlgebraTraits_d} = \ccc{OpenNL::DefaultLinearSolverTraits<typename ParameterizationMesh_3::NT>}$>$   \\
class \ccc{Fixed_border_parameterizer_3};

%END-AUTO(\ccParameters)


\ccTypes

% The section below is automatically generated. Do not edit!
%START-AUTO(\ccTypes)

\ccNestedType{Border_param}
{
Export \ccc{BorderParameterizer_3} template parameter.
}
\ccGlue
\ccNestedType{Sparse_LA}
{
Export \ccc{SparseLinearAlgebraTraits_d} template parameter.
}
\ccGlue

%END-AUTO(\ccTypes)


\ccCreation
\ccCreationVariable{param}  %% choose variable name for \ccMethod

% The section below is automatically generated. Do not edit!
%START-AUTO(\ccCreation)

\ccConstructor{Fixed_border_parameterizer_3(Border_param border_param = Border_param(), Sparse_LA sparse_la = Sparse_LA());}
{
Constructor.
}
\ccGlue
\begin{description}
\item[Parameters: ]
\begin{description}
\item[\ccc{border_param}]Object that maps the surface's border to 2D space \item[\ccc{sparse_la}]Traits object to access a sparse linear system \end{description}
\end{description}
\ccGlue

%END-AUTO(\ccCreation)


\ccOperations

% The section below is automatically generated. Do not edit!
%START-AUTO(\ccOperations)

\ccMethod{Fixed_border_parameterizer_3<Adaptor, Border_param, Sparse_LA>::Error_code parameterize(Adaptor& mesh);}
{
[virtual] \\
Compute a one-to-one mapping from a triangular 3D surface \ccc{mesh} to a piece of the 2D space. The mapping is linear by pieces (linear in each triangle). The result is the (u,v) pair image of each vertex of the 3D surface.
Preconditions:\begin{itemize}
\item \ccc{mesh} must be a surface with one connected component.\item \ccc{mesh} must be a triangular mesh.\item the mesh border must be mapped onto a convex polygon. \end{itemize}
}
\ccGlue
\ccMethod{Fixed_border_parameterizer_3<Adaptor, Border_param, Sparse_LA>::Error_code check_parameterize_preconditions(Adaptor& mesh);}
{
[protected, virtual] \\
Check parameterize() preconditions:\begin{itemize}
\item \ccc{mesh} must be a surface with one connected component.\item \ccc{mesh} must be a triangular mesh.\item the mesh border must be mapped onto a convex polygon. \end{itemize}
}
\ccGlue
\ccMethod{void initialize_system_from_mesh_border(Matrix& A, Vector& Bu, Vector& Bv, const Adaptor& mesh);}
{
[protected] \\
Initialize A, Bu and Bv after border parameterization. Fill the border vertices' lines in both linear systems: {\em u = constant} and {\em v = constant}.
Preconditions:\begin{itemize}
\item vertices must be indexed.\item A, Bu and Bv must be allocated.\item border vertices must be parameterized. \end{itemize}
}
\ccGlue
\ccMethod{virtual NT compute_w_ij(const Adaptor& mesh, Vertex_const_handle main_vertex_v_i, Vertex_around_vertex_const_circulator neighbor_vertex_v_j);}
{
[protected, pure virtual] \\
Compute \ccc{w_ij} = (i, j) coefficient of matrix A for j neighbor vertex of i. Implementation note: Subclasses must at least implement \ccc{compute_w_ij}().
}
\ccGlue
\ccMethod{Fixed_border_parameterizer_3<Adaptor, Border_param, Sparse_LA>::Error_code setup_inner_vertex_relations(Matrix& A, Vector& Bu, Vector& Bv, const Adaptor& mesh, Vertex_const_handle vertex);}
{
[protected, virtual] \\
Compute the line i of matrix A for i inner vertex:\begin{itemize}
\item call \ccc{compute_w_ij}() to compute the A coefficient \ccc{w_ij} for each neighbor \ccc{v_j}.\item compute \ccc{w_ii} = - sum of \ccc{w_ijs}.\end{itemize}
Preconditions:\begin{itemize}
\item vertices must be indexed.\item vertex i musn't be already parameterized.\item line i of A must contain only zeros. \end{itemize}
}
\ccGlue
\ccMethod{void set_mesh_uv_from_system(Adaptor& mesh, const Vector& Xu, const Vector& Xv);}
{
[protected] \\
Copy Xu and Xv coordinates into the (u,v) pair of each surface vertex.
}
\ccGlue
\ccMethod{Fixed_border_parameterizer_3<Adaptor, Border_param, Sparse_LA>::Error_code check_parameterize_postconditions(const Adaptor& mesh, const Matrix& A, const Vector& Bu, const Vector& Bv);}
{
[protected, virtual] \\
Check parameterize() postconditions:\begin{itemize}
\item 3D -$>$ 2D mapping is one-to-one. \end{itemize}
}
\ccGlue
\ccMethod{bool is_one_to_one_mapping(const Adaptor& mesh, const Matrix& A, const Vector& Bu, const Vector& Bv);}
{
[protected, virtual] \\
Check if 3D -$>$ 2D mapping is one-to-one. The default implementation checks each normal.
}
\ccGlue
\ccMethod{Border_param& get_border_parameterizer();}
{
[protected] \\
Get the object that maps the surface's border onto a 2D space.
}
\ccGlue
\ccMethod{Sparse_LA& get_linear_algebra_traits();}
{
[protected] \\
Get the sparse linear algebra (traits object to access the linear system).
}
\ccGlue

%END-AUTO(\ccOperations)


\ccSeeAlso

\ccRefIdfierPage{CGAL::Parameterizer_traits_3<ParameterizationMesh_3>}  \\
\ccRefIdfierPage{CGAL::Barycentric_mapping_parameterizer_3<ParameterizationMesh_3, BorderParameterizer_3, SparseLinearAlgebraTraits_d>}  \\
\ccRefIdfierPage{CGAL::Discrete_authalic_parameterizer_3<ParameterizationMesh_3, BorderParameterizer_3, SparseLinearAlgebraTraits_d>}  \\
\ccRefIdfierPage{CGAL::Discrete_conformal_map_parameterizer_3<ParameterizationMesh_3, BorderParameterizer_3, SparseLinearAlgebraTraits_d>}  \\
\ccRefIdfierPage{CGAL::LSCM_parameterizer_3<ParameterizationMesh_3, BorderParameterizer_3, SparseLinearAlgebraTraits_d>}  \\
\ccRefIdfierPage{CGAL::Mean_value_coordinates_parameterizer_3<ParameterizationMesh_3, BorderParameterizer_3, SparseLinearAlgebraTraits_d>}  \\


\end{ccRefClass}

% +------------------------------------------------------------------------+
%%RefPage: end of main body, begin of footer
\ccRefPageEnd
% EOF
% +------------------------------------------------------------------------+

