% +------------------------------------------------------------------------+
% | Reference manual page: Fixed_border_parameterizer_3.tex
% +------------------------------------------------------------------------+
% | 21.09.2005   Laurent Saboret, Pierre Alliez, Bruno Levy
% | Package: Surface_mesh_parameterization
% |
\RCSdef{\RCSFixedborderparameterizerRev}{$Id$}
\RCSdefDate{\RCSFixedborderparameterizerDate}{$Date$}
% |
\ccRefPageBegin
%%RefPage: end of header, begin of main body
% +------------------------------------------------------------------------+


\begin{ccRefClass}{Fixed_border_parameterizer_3<ParameterizationMesh_3, BorderParameterizer_3, SparseLinearAlgebraTraits_d>}

%% \ccHtmlCrossLink{}     %% add further rules for cross referencing links
%% \ccHtmlIndexC[class]{} %% add further index entries


\ccDefinition

% The section below is automatically generated. Do not edit!
%START-AUTO(\ccDefinition)

The class \ccc{Fixed_border_parameterizer_3} is the base class of fixed border parameterization methods (Tutte, Floater, ...).

One-to-one mapping is guaranteed if surface's border is mapped onto a convex polygon.

This class is a pure virtual class, thus cannot be instantiated. Anyway, it implements most of the parameterization algorithm parameterize(). Subclasses are Strategies \cite{cgal:ghjv-dpero-95} that modify the behavior of this algorithm: They provide \ccc{BorderParameterizer_3} and \ccc{SparseLinearAlgebraTraits_d} template parameters that make sense. They implement \ccc{compute_w_ij}() to compute \ccc{w_ij} = (i, j) coefficient of matrix A for j neighbor vertex of i. They may implement an optimized version of \ccc{is_one_to_one_mapping}().

%END-AUTO(\ccDefinition)

% The section below is automatically generated. Do not edit!
%START-AUTO(\ccInclude)

\ccInclude{CGAL/Fixed_border_parameterizer_3.h}

%END-AUTO(\ccInclude)


\ccInheritsFrom

% The section below is automatically generated. Do not edit!
%START-AUTO(\ccInheritsFrom)

\ccc{Parameterizer_traits_3<ParameterizationMesh_3>}

%END-AUTO(\ccInheritsFrom)


\ccIsModel

% The section below is automatically generated. Do not edit!
%START-AUTO(\ccIsModel)

Model of the \ccc{ParameterizerTraits_3} concept (although you cannot instantiate this class).

%END-AUTO(\ccIsModel)


\ccHeading{Design Pattern}

% The section below is automatically generated. Do not edit!
%START-AUTO(\ccHeading{Design Pattern})

\ccc{Fixed_border_parameterizer_3} class is a Strategy \cite{cgal:ghjv-dpero-95}: it implements (part of) a strategy of surface parameterization for models of \ccc{ParameterizationMesh_3}.

%END-AUTO(\ccHeading{Design Pattern})


\ccParameters

The full template declaration is:

% The section below is automatically generated. Do not edit!
%START-AUTO(\ccParameters)

template$<$  \\
class \ccc{ParameterizationMesh_3},   \\
class \ccc{BorderParameterizer_3} = \ccc{Circular_border_arc_length_parameterizer_3<ParameterizationMesh_3>},   \\
class \ccc{SparseLinearAlgebraTraits_d} = \ccc{OpenNL::DefaultLinearSolverTraits<typename ParameterizationMesh_3::NT>}$>$   \\
class \ccc{Fixed_border_parameterizer_3};

%END-AUTO(\ccParameters)


\ccTypes

% The section below is automatically generated. Do not edit!
%START-AUTO(\ccTypes)

\ccNestedType{Border_param}
{
Export \ccc{BorderParameterizer_3} template parameter.
}
\ccGlue
\ccNestedType{Sparse_LA}
{
Export \ccc{SparseLinearAlgebraTraits_d} template parameter.
}
\ccGlue

%END-AUTO(\ccTypes)


\ccCreation
\ccCreationVariable{param}  %% choose variable name for \ccMethod

% The section below is automatically generated. Do not edit!
%START-AUTO(\ccCreation)

\ccConstructor{Fixed_border_parameterizer_3(Border_param border_param = Border_param(), Sparse_LA sparse_la = Sparse_LA());}
{
Constructor.
\ccCommentHeading{Parameters}  \\
\ccc{border_param}: Object that maps the surface's border to 2D space. \ccc{sparse_la}: Traits object to access a sparse linear system.
}
\ccGlue

%END-AUTO(\ccCreation)


\ccOperations

% The section below is automatically generated. Do not edit!
%START-AUTO(\ccOperations)

\ccMethod{Error_code parameterize(Adaptor& mesh);}
{
Compute a one-to-one mapping from a triangular 3D surface \ccc{mesh} to a piece of the 2D space. The mapping is linear by pieces (linear in each triangle). The result is the (u, v) pair image of each vertex of the 3D surface.
\ccCommentHeading{Preconditions}  \\
\ccc{mesh} must be a surface with one connected component. \ccc{mesh} must be a triangular mesh. The mesh border must be mapped onto a convex polygon.
}
\ccGlue
\ccMethod{Error_code check_parameterize_preconditions(Adaptor& mesh)[protected, virtual];}
{
Check parameterize() preconditions: \ccc{mesh} must be a surface with one connected component. \ccc{mesh} must be a triangular mesh. The mesh border must be mapped onto a convex polygon.
}
\ccGlue
\ccMethod{void initialize_system_from_mesh_border(Matrix& A, Vector& Bu, Vector& Bv, const Adaptor& mesh);}
{
Initialize A, Bu and Bv after border parameterization. Fill the border vertices' lines in both linear systems: {\em u = constant} and {\em v = constant}.
\ccCommentHeading{Preconditions}  \\
Vertices must be indexed. A, Bu and Bv must be allocated. Border vertices must be parameterized.
}
\ccGlue
\ccMethod{virtual NT compute_w_ij(const Adaptor& mesh, Vertex_const_handle main_vertex_v_i, Vertex_around_vertex_const_circulator neighbor_vertex_v_j)[protected, pure virtual];}
{
Compute \ccc{w_ij} = (i, j) coefficient of matrix A for j neighbor vertex of i. Implementation note: Subclasses must at least implement \ccc{compute_w_ij}().
}
\ccGlue
\ccMethod{Error_code setup_inner_vertex_relations(Matrix& A, Vector& Bu, Vector& Bv, const Adaptor& mesh, Vertex_const_handle vertex)[protected, virtual];}
{
Compute the line i of matrix A for i inner vertex: call \ccc{compute_w_ij}() to compute the A coefficient \ccc{w_ij} for each neighbor \ccc{v_j}. compute \ccc{w_ii} = - sum of \ccc{w_ijs}.
\ccCommentHeading{Preconditions}  \\
Vertices must be indexed. Vertex i musn't be already parameterized. Line i of A must contain only zeros.
}
\ccGlue
\ccMethod{void set_mesh_uv_from_system(Adaptor& mesh, const Vector& Xu, const Vector& Xv);}
{
Copy Xu and Xv coordinates into the (u, v) pair of each surface vertex.
}
\ccGlue
\ccMethod{Error_code check_parameterize_postconditions(const Adaptor& mesh, const Matrix& A, const Vector& Bu, const Vector& Bv)[protected, virtual];}
{
Check parameterize() postconditions: 3D -$>$ 2D mapping is one-to-one.
}
\ccGlue
\ccMethod{bool is_one_to_one_mapping(const Adaptor& mesh, const Matrix& A, const Vector& Bu, const Vector& Bv)[protected, virtual];}
{
Check if 3D -$>$ 2D mapping is one-to-one. The default implementation checks each normal.
}
\ccGlue
\ccMethod{Border_param& get_border_parameterizer();}
{
Get the object that maps the surface's border onto a 2D space.
}
\ccGlue
\ccMethod{Sparse_LA& get_linear_algebra_traits();}
{
Get the sparse linear algebra (traits object to access the linear system).
}
\ccGlue

%END-AUTO(\ccOperations)


\ccSeeAlso

\ccRefIdfierPage{CGAL::Parameterizer_traits_3<ParameterizationMesh_3>}  \\
\ccRefIdfierPage{CGAL::Barycentric_mapping_parameterizer_3<ParameterizationMesh_3, BorderParameterizer_3, SparseLinearAlgebraTraits_d>}  \\
\ccRefIdfierPage{CGAL::Discrete_authalic_parameterizer_3<ParameterizationMesh_3, BorderParameterizer_3, SparseLinearAlgebraTraits_d>}  \\
\ccRefIdfierPage{CGAL::Discrete_conformal_map_parameterizer_3<ParameterizationMesh_3, BorderParameterizer_3, SparseLinearAlgebraTraits_d>}  \\
\ccRefIdfierPage{CGAL::LSCM_parameterizer_3<ParameterizationMesh_3, BorderParameterizer_3, SparseLinearAlgebraTraits_d>}  \\
\ccRefIdfierPage{CGAL::Mean_value_coordinates_parameterizer_3<ParameterizationMesh_3, BorderParameterizer_3, SparseLinearAlgebraTraits_d>}  \\


\end{ccRefClass}

% +------------------------------------------------------------------------+
%%RefPage: end of main body, begin of footer
\ccRefPageEnd
% EOF
% +------------------------------------------------------------------------+

