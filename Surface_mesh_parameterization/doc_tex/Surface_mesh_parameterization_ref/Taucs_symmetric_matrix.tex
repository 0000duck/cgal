% +------------------------------------------------------------------------+
% | Reference manual page: Taucs_symmetric_matrix.tex
% +------------------------------------------------------------------------+
% | 21.09.2005   Laurent Saboret, Pierre Alliez, Bruno Levy
% | Package: Surface_mesh_parameterization
% |
\RCSdef{\RCSTaucssymmetricmatrixRev}{$Id$}
\RCSdefDate{\RCSTaucssymmetricmatrixDate}{$Date$}
% |
%%RefPage: end of header, begin of main body
% +------------------------------------------------------------------------+


\begin{ccRefClass}{Taucs_symmetric_matrix}  %% add template arg's if necessary

%% \ccHtmlCrossLink{}     %% add further rules for cross referencing links
%% \ccHtmlIndexC[class]{} %% add further index entries


\ccDefinition

% The section below is automatically generated. Do not edit!
%START-AUTO(\ccDefinition)

The class Taucs\_symmetric\_matrix is a C++ wrapper around a TAUCS $\ast$symmetric$\ast$ matrix (type taucs\_ccs\_matrix).

Symmetric matrices store only the lower triangle.

%END-AUTO(\ccDefinition)

\ccInclude{CGAL/Taucs_matrix.h}


\ccIsModel

% The section below is automatically generated. Do not edit!
%START-AUTO(\ccIsModel)

Model of the SparseLinearAlgebraTraits\_d::Matrix concept.

%END-AUTO(\ccIsModel)


\ccParameters

The full template declaration is:

% The section below is automatically generated. Do not edit!
%START-AUTO(\ccParameters)

template$<$  \\
class T$>$   \\
struct Taucs\_symmetric\_matrix;

%END-AUTO(\ccParameters)


\ccTypes

% The section below is automatically generated. Do not edit!
%START-AUTO(\ccTypes)

\ccNestedType{NT}
{
}
\ccGlue

%END-AUTO(\ccTypes)


\ccCreation
\ccCreationVariable{M}  %% choose variable name for \ccMethod

% The section below is automatically generated. Do not edit!
%START-AUTO(\ccCreation)

\ccConstructor{Taucs_symmetric_matrix (int dim);}
{
Create a square SYMMETRIC matrix initialized with zeros. The max number of non 0 elements in the matrix is automatically computed. 
}
\ccGlue
\begin{description}
\item[Parameters: ]
\begin{description}
\item[dim]Matrix dimension. \end{description}
\end{description}
\ccGlue
\ccConstructor{Taucs_symmetric_matrix (int rows, int columns, int nb_max_elements = 0);}
{
Create a square SYMMETRIC matrix initialized with zeros.
}
\ccGlue
\begin{description}
\item[Parameters: ]
\begin{description}
\item[rows]Matrix dimensions. \item[nb\_max\_elements]Max number of non 0 elements in the matrix (automatically computed if 0). \end{description}
\end{description}
\ccGlue

%END-AUTO(\ccCreation)


\ccOperations

% The section below is automatically generated. Do not edit!
%START-AUTO(\ccOperations)



%END-AUTO(\ccOperations)


\ccSeeAlso

\ccRefIdfierPage{CGAL::Taucs_solver_traits}  \\
\ccRefIdfierPage{CGAL::Taucs_symmetric_solver_traits}  \\
\ccRefIdfierPage{CGAL::Taucs_matrix}  \\
\ccRefIdfierPage{CGAL::Taucs_vector}  \\
\ccc{OpenNL::DefaultLinearSolverTraits}  \\
\ccc{OpenNL::SymmetricLinearSolverTraits}  \\


\ccExample

A Taucs\_matrix object can be used as a 2D array of floating points.
Use get\_coef() and set\_coef() to access to the matrix elements.

\begin{ccExampleCode}

typedef CGAL::Taucs_symmetric_matrix<double>  Matrix;

int main()
{
    Matrix m(100);                  // 100 x 100 elements
    m.set_coef(0, 0, 4.3);
    double val = m.get_coef(0, 0);
    ...
}

\end{ccExampleCode}


\end{ccRefClass}

% +------------------------------------------------------------------------+
%%RefPage: end of main body, begin of footer
% EOF
% +------------------------------------------------------------------------+

