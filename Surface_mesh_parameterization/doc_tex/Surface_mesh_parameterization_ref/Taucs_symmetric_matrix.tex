% +------------------------------------------------------------------------+
% | Reference manual page: Taucs_symmetric_matrix.tex
% +------------------------------------------------------------------------+
% | 21.09.2005   Laurent Saboret, Pierre Alliez, Bruno Levy
% | Package: Surface_mesh_parameterization
% |
\RCSdef{\RCSTaucssymmetricmatrixRev}{$Id$}
\RCSdefDate{\RCSTaucssymmetricmatrixDate}{$Date$}
% |
\ccRefPageBegin
%%RefPage: end of header, begin of main body
% +------------------------------------------------------------------------+


\begin{ccRefClass}{Taucs_symmetric_matrix<T>}

%% \ccHtmlCrossLink{}     %% add further rules for cross referencing links
%% \ccHtmlIndexC[class]{} %% add further index entries


\ccDefinition

% The section below is automatically generated. Do not edit!
%START-AUTO(\ccDefinition)

The class \ccc{Taucs_symmetric_matrix} is a C++ wrapper around a TAUCS $\ast$symmetric$\ast$ matrix (type \ccc{taucs_ccs_matrix}).

Symmetric matrices store only the lower triangle.

%END-AUTO(\ccDefinition)

% The section below is automatically generated. Do not edit!
%START-AUTO(\ccInclude)

\ccInclude{CGAL/Taucs_matrix.h}

%END-AUTO(\ccInclude)


\ccInheritsFrom

% The section below is automatically generated. Do not edit!
%START-AUTO(\ccInheritsFrom)

\ccc{Taucs_matrix<T>}

%END-AUTO(\ccInheritsFrom)


\ccIsModel

% The section below is automatically generated. Do not edit!
%START-AUTO(\ccIsModel)

Model of the \ccc{SparseLinearAlgebraTraits_d::Matrix} concept.

%END-AUTO(\ccIsModel)


\ccParameters

The full template declaration is:

% The section below is automatically generated. Do not edit!
%START-AUTO(\ccParameters)

template$<$  \\
class T$>$   \\
struct \ccc{Taucs_symmetric_matrix};

%END-AUTO(\ccParameters)


\ccTypes

% The section below is automatically generated. Do not edit!
%START-AUTO(\ccTypes)

\ccNestedType{NT}
{
}
\ccGlue

%END-AUTO(\ccTypes)


\ccCreation
\ccCreationVariable{M}  %% choose variable name for \ccMethod

% The section below is automatically generated. Do not edit!
%START-AUTO(\ccCreation)

\ccConstructor{Taucs_symmetric_matrix(int dim);}
{
Create a square SYMMETRIC matrix initialized with zeros. The max number of non 0 elements in the matrix is automatically computed.
}
\ccGlue
\begin{description}
\item[Parameters: ]
\begin{description}
\item[dim]Matrix dimension. \end{description}
\end{description}
\ccGlue
\ccConstructor{Taucs_symmetric_matrix(int rows, int columns);}
{
Create a square SYMMETRIC matrix initialized with zeros.
}
\ccGlue
\begin{description}
\item[Parameters: ]
\begin{description}
\item[rows]Matrix dimensions. \end{description}
\end{description}
\ccGlue

%END-AUTO(\ccCreation)


\ccOperations

% The section below is automatically generated. Do not edit!
%START-AUTO(\ccOperations)
%END-AUTO(\ccOperations)


\ccSeeAlso

\ccRefIdfierPage{CGAL::Taucs_solver_traits<T>}  \\
\ccRefIdfierPage{CGAL::Taucs_symmetric_solver_traits<T>}  \\
\ccRefIdfierPage{CGAL::Taucs_matrix<T>}  \\
\ccRefIdfierPage{CGAL::Taucs_vector<T>}  \\
\ccc{OpenNL::DefaultLinearSolverTraits<COEFFTYPE, MATRIX, VECTOR, SOLVER>} in OpenNL package  \\
\ccc{OpenNL::SymmetricLinearSolverTraits<COEFFTYPE, MATRIX, VECTOR, SOLVER>} in OpenNL package  \\


\end{ccRefClass}

% +------------------------------------------------------------------------+
%%RefPage: end of main body, begin of footer
\ccRefPageEnd
% EOF
% +------------------------------------------------------------------------+

