% +------------------------------------------------------------------------+
% | Reference manual page: Taucs_matrix.tex
% +------------------------------------------------------------------------+
% | 21.09.2005   Laurent Saboret, Pierre Alliez, Bruno Levy
% | Package: Surface_mesh_parameterization
% |
\RCSdef{\RCSTaucsmatrixRev}{$Id$}
\RCSdefDate{\RCSTaucsmatrixDate}{$Date$}
% |
\ccRefPageBegin
%%RefPage: end of header, begin of main body
% +------------------------------------------------------------------------+


\begin{ccRefClass}{Taucs_matrix<T>}

%% \ccHtmlCrossLink{}     %% add further rules for cross referencing links
%% \ccHtmlIndexC[class]{} %% add further index entries


\ccDefinition

% The section below is automatically generated. Do not edit!
%START-AUTO(\ccDefinition)

The class \ccc{Taucs_matrix} is a C++ wrapper around TAUCS' matrix type \ccc{taucs_ccs_matrix}.

This kind of matrix can be either symmetric or not. Symmetric matrices store only the lower triangle.

%END-AUTO(\ccDefinition)

\ccInclude{CGAL/Taucs_matrix.h}


\ccIsModel

% The section below is automatically generated. Do not edit!
%START-AUTO(\ccIsModel)

Model of the \ccc{SparseLinearAlgebraTraits_d::Matrix} concept.

%END-AUTO(\ccIsModel)


\ccParameters

The full template declaration is:

% The section below is automatically generated. Do not edit!
%START-AUTO(\ccParameters)

template$<$  \\
class T$>$   \\
struct \ccc{Taucs_matrix};

%END-AUTO(\ccParameters)


\ccTypes

% The section below is automatically generated. Do not edit!
%START-AUTO(\ccTypes)

\ccNestedType{NT}
{
}
\ccGlue

%END-AUTO(\ccTypes)


\ccCreation
\ccCreationVariable{M}  %% choose variable name for \ccMethod

% The section below is automatically generated. Do not edit!
%START-AUTO(\ccCreation)

\ccConstructor{Taucs_matrix(int dim, bool is_symmetric = false);}
{
Create a square matrix initialized with zeros.
}
\ccGlue
\begin{description}
\item[Parameters: ]
\begin{description}
\item[dim]Matrix dimension. \item[\ccc{is_symmetric}]Symmetric/hermitian? \end{description}
\end{description}
\ccGlue
\ccConstructor{Taucs_matrix(int rows, int columns, bool is_symmetric = false);}
{
Create a rectangular matrix initialized with zeros.
}
\ccGlue
\begin{description}
\item[Parameters: ]
\begin{description}
\item[rows]Matrix dimensions. \item[\ccc{is_symmetric}]Symmetric/hermitian? \end{description}
\end{description}
\ccGlue

%END-AUTO(\ccCreation)


\ccOperations

% The section below is automatically generated. Do not edit!
%START-AUTO(\ccOperations)

\ccMethod{int row_dimension() const;}
{
Return the matrix number of rows.
}
\ccGlue
\ccMethod{int column_dimension() const;}
{
Return the matrix number of columns.
}
\ccGlue
\ccMethod{T get_coef(int i, int j) const;}
{
Read access to a matrix coefficient.
Preconditions:\begin{itemize}
\item 0 $<$= i $<$ \ccc{row_dimension}().\item 0 $<$= j $<$ \ccc{column_dimension}(). \end{itemize}
}
\ccGlue
\ccMethod{void set_coef(int i, int j, T val);}
{
Write access to a matrix coefficient: \ccc{a_ij} $<$- val.
Optimization: For symmetric matrices, \ccc{Taucs_matrix} stores only the lower triangle \ccc{set_coef}() does nothing if (i, j) belongs to the upper triangle.
Preconditions:\begin{itemize}
\item 0 $<$= i $<$ \ccc{row_dimension}().\item 0 $<$= j $<$ \ccc{column_dimension}(). \end{itemize}
}
\ccGlue
\ccMethod{void add_coef(int i, int j, T val);}
{
Write access to a matrix coefficient: \ccc{a_ij} $<$- \ccc{a_ij} + val.
Optimization: For symmetric matrices, \ccc{Taucs_matrix} stores only the lower triangle \ccc{add_coef}() does nothing if (i, j) belongs to the upper triangle.
Preconditions:\begin{itemize}
\item 0 $<$= i $<$ \ccc{row_dimension}().\item 0 $<$= j $<$ \ccc{column_dimension}(). \end{itemize}
}
\ccGlue
\ccMethod{const taucs_ccs_matrix* get_taucs_matrix() const;}
{
Construct and return the TAUCS matrix wrapped by this object. Note: the TAUCS matrix returned by this method is valid only until the next call to \ccc{set_coef}(), \ccc{add_coef}() or \ccc{get_taucs_matrix}().
}
\ccGlue

%END-AUTO(\ccOperations)


\ccSeeAlso

\ccRefIdfierPage{CGAL::Taucs_solver_traits<T>}  \\
\ccRefIdfierPage{CGAL::Taucs_symmetric_solver_traits<T>}  \\
\ccRefIdfierPage{CGAL::Taucs_symmetric_matrix<T>}  \\
\ccRefIdfierPage{CGAL::Taucs_vector<T>}  \\


\end{ccRefClass}

% +------------------------------------------------------------------------+
%%RefPage: end of main body, begin of footer
\ccRefPageEnd
% EOF
% +------------------------------------------------------------------------+

