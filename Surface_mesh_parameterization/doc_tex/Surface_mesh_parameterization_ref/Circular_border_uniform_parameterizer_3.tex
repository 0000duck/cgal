% +------------------------------------------------------------------------+
% | Reference manual page: Circular_border_uniform_parameterizer_3.tex
% +------------------------------------------------------------------------+
% | 21.09.2005   Laurent Saboret, Pierre Alliez, Bruno Levy
% | Package: Surface_mesh_parameterization
% |
\RCSdef{\RCSCircularborderuniformparameterizerRev}{$Id$}
\RCSdefDate{\RCSCircularborderuniformparameterizerDate}{$Date$}
% |
\ccRefPageBegin
%%RefPage: end of header, begin of main body
% +------------------------------------------------------------------------+


\begin{ccRefClass}{Circular_border_uniform_parameterizer_3<ParameterizationMesh_3>}

%% \ccHtmlCrossLink{}     %% add further rules for cross referencing links
%% \ccHtmlIndexC[class]{} %% add further index entries


\ccDefinition

% The section below is automatically generated. Do not edit!
%START-AUTO(\ccDefinition)

This class parameterizes the border of a 3D surface onto a circle in a uniform manner: points are equally spaced. Circular\_border\_parameterizer\_3 implements most of the border parameterization algorithm. This class implements only compute\_edge\_length() to compute a segment's length.

%END-AUTO(\ccDefinition)

\ccInclude{CGAL/Circular_border_parameterizer_3.h}


\ccIsModel

% The section below is automatically generated. Do not edit!
%START-AUTO(\ccIsModel)

Model of the BorderParameterizer\_3 concept.

%END-AUTO(\ccIsModel)


\ccHeading{Design Pattern}

% The section below is automatically generated. Do not edit!
%START-AUTO(\ccHeading{Design Pattern})

BorderParameterizer\_3 models are Strategies \cite{cgal:ghjv-dpero-95}: they implement a strategy of border parameterization for models of ParameterizationMesh\_3

%END-AUTO(\ccHeading{Design Pattern})


\ccParameters

The full template declaration is:

% The section below is automatically generated. Do not edit!
%START-AUTO(\ccParameters)

template$<$  \\
class ParameterizationMesh\_3$>$   \\
class Circular\_border\_uniform\_parameterizer\_3;

%END-AUTO(\ccParameters)


\ccTypes

% The section below is automatically generated. Do not edit!
%START-AUTO(\ccTypes)



%END-AUTO(\ccTypes)


\ccCreation
\ccCreationVariable{bp}  %% choose variable name for \ccMethod

\ccConstructor{Circular_border_uniform_parameterizer_3();}{default constructor.}

% The section below is automatically generated. Do not edit!
%START-AUTO(\ccCreation)



%END-AUTO(\ccCreation)


\ccOperations

% The section below is automatically generated. Do not edit!
%START-AUTO(\ccOperations)

\ccMethod{virtual double compute_edge_length (const Adaptor & mesh, Vertex_const_handle source, Vertex_const_handle target);}
{
Compute the length of an edge.
Uniform border parameterization: points are equally spaced.
}
\ccGlue

%END-AUTO(\ccOperations)


\ccSeeAlso

\ccRefIdfierPage{CGAL::Circular_border_arc_length_parameterizer_3<ParameterizationMesh_3>}  \\
\ccRefIdfierPage{CGAL::Square_border_arc_length_parameterizer_3<ParameterizationMesh_3>}  \\
\ccRefIdfierPage{CGAL::Square_border_uniform_parameterizer_3<ParameterizationMesh_3>}  \\
\ccRefIdfierPage{CGAL::Two_vertices_parameterizer_3<ParameterizationMesh_3>}  \\


\end{ccRefClass}

% +------------------------------------------------------------------------+
%%RefPage: end of main body, begin of footer
\ccRefPageEnd
% EOF
% +------------------------------------------------------------------------+

