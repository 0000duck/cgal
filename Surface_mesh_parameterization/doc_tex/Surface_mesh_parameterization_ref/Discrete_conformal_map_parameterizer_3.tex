% +------------------------------------------------------------------------+
% | Reference manual page: Discrete_conformal_map_parameterizer_3.tex
% +------------------------------------------------------------------------+
% | 21.09.2005   Laurent Saboret, Pierre Alliez, Bruno Levy
% | Package: Surface_mesh_parameterization
% |
\RCSdef{\RCSDiscreteconformalmapparameterizerRev}{$Id$}
\RCSdefDate{\RCSDiscreteconformalmapparameterizerDate}{$Date$}
% |
\ccRefPageBegin
%%RefPage: end of header, begin of main body
% +------------------------------------------------------------------------+


\begin{ccRefClass}{Discrete_conformal_map_parameterizer_3<ParameterizationMesh_3, BorderParameterizer_3, SparseLinearAlgebraTraits_d>}

%% \ccHtmlCrossLink{}     %% add further rules for cross referencing links
%% \ccHtmlIndexC[class]{} %% add further index entries


\ccDefinition

% The section below is automatically generated. Do not edit!
%START-AUTO(\ccDefinition)

The class \ccc{Discrete_conformal_map_parameterizer_3} implements the Discrete Conformal Map (DCM) parameterization \cite{cgal:eddhls-maam-95}. This algorithm is also called {\em Discrete Conformal Parameterization (DCP)}, {\em Discrete Harmonic Map} or {\em Fixed Conformal Parameterization} by other authors.

This is a conformal parameterization, i.e. it attempts to preserve angles.

One-to-one mapping is guaranteed if surface's border is mapped onto a convex polygon.

This class is a Strategy \cite{cgal:ghjv-dpero-95} called by the main parameterization algorithm \ccc{Fixed_border_parameterizer_3::parameterize}(). \ccc{Discrete_conformal_map_parameterizer_3}:\begin{itemize}
\item provides default \ccc{BorderParameterizer_3} and \ccc{SparseLinearAlgebraTraits_d} template parameters that make sense.\item implements \ccc{compute_w_ij}() to compute \ccc{w_ij} = (i, j) coefficient of matrix A for j neighbor vertex of i based on Discrete Conformal Map method.\end{itemize}

%END-AUTO(\ccDefinition)

\ccInclude{CGAL/Discrete_conformal_map_parameterizer_3.h}


\ccIsModel

% The section below is automatically generated. Do not edit!
%START-AUTO(\ccIsModel)

Model of the \ccc{ParameterizerTraits_3} concept.

%END-AUTO(\ccIsModel)


\ccHeading{Design Pattern}

% The section below is automatically generated. Do not edit!
%START-AUTO(\ccHeading{Design Pattern})

\ccc{Discrete_conformal_map_parameterizer_3<ParameterizationMesh_3, ...>} class is a Strategy \cite{cgal:ghjv-dpero-95}: it implements a strategy of surface parameterization for models of \ccc{ParameterizationMesh_3}.

%END-AUTO(\ccHeading{Design Pattern})


\ccParameters

The full template declaration is:

% The section below is automatically generated. Do not edit!
%START-AUTO(\ccParameters)

template$<$  \\
class \ccc{ParameterizationMesh_3},   \\
class \ccc{BorderParameterizer_3} = \ccc{Circular_border_arc_length_parameterizer_3<ParameterizationMesh_3>},   \\
class \ccc{SparseLinearAlgebraTraits_d} = \ccc{OpenNL::DefaultLinearSolverTraits<typename ParameterizationMesh_3::NT>}$>$   \\
class \ccc{Discrete_conformal_map_parameterizer_3};

\begin{description}
\item[Parameters:]
\begin{description}
\item[\ccc{ParameterizationMesh_3}]3D surface mesh. \item[\ccc{BorderParameterizer_3}]Strategy to parameterize the surface border. \item[\ccc{SparseLinearAlgebraTraits_d}]Traits class to solve a sparse linear system. Note: the system is NOT symmetric because \ccc{Fixed_border_parameterizer_3} does not remove (yet) border vertices from the system. \end{description}
\end{description}

%END-AUTO(\ccParameters)


\ccTypes

% The section below is automatically generated. Do not edit!
%START-AUTO(\ccTypes)
%END-AUTO(\ccTypes)


\ccCreation
\ccCreationVariable{param}  %% choose variable name for \ccMethod

% The section below is automatically generated. Do not edit!
%START-AUTO(\ccCreation)

\ccConstructor{Discrete_conformal_map_parameterizer_3(Border_param border_param = Border_param(), Sparse_LA sparse_la = Sparse_LA());}
{
Constructor.
}
\ccGlue
\begin{description}
\item[Parameters: ]
\begin{description}
\item[\ccc{border_param}]Object that maps the surface's border to 2D space. \item[\ccc{sparse_la}]Traits object to access a sparse linear system. \end{description}
\end{description}
\ccGlue

%END-AUTO(\ccCreation)


\ccOperations

% The section below is automatically generated. Do not edit!
%START-AUTO(\ccOperations)

\ccMethod{virtual NT compute_w_ij(const Adaptor& mesh, Vertex_const_handle main_vertex_v_i, Vertex_around_vertex_const_circulator neighbor_vertex_v_j);}
{
[protected, virtual] \\
Compute \ccc{w_ij} = (i,j) coefficient of matrix A for j neighbor vertex of i.
}
\ccGlue

%END-AUTO(\ccOperations)


\ccSeeAlso

\ccRefIdfierPage{CGAL::Parameterizer_traits_3<ParameterizationMesh_3>}  \\
\ccRefIdfierPage{CGAL::Fixed_border_parameterizer_3<ParameterizationMesh_3, BorderParameterizer_3, SparseLinearAlgebraTraits_d>}  \\
\ccRefIdfierPage{CGAL::Barycentric_mapping_parameterizer_3<ParameterizationMesh_3, BorderParameterizer_3, SparseLinearAlgebraTraits_d>}  \\
\ccRefIdfierPage{CGAL::Discrete_authalic_parameterizer_3<ParameterizationMesh_3, BorderParameterizer_3, SparseLinearAlgebraTraits_d>}  \\
\ccRefIdfierPage{CGAL::LSCM_parameterizer_3<ParameterizationMesh_3, BorderParameterizer_3, SparseLinearAlgebraTraits_d>}  \\
\ccRefIdfierPage{CGAL::Mean_value_coordinates_parameterizer_3<ParameterizationMesh_3, BorderParameterizer_3, SparseLinearAlgebraTraits_d>}  \\


\end{ccRefClass}

% +------------------------------------------------------------------------+
%%RefPage: end of main body, begin of footer
\ccRefPageEnd
% EOF
% +------------------------------------------------------------------------+

