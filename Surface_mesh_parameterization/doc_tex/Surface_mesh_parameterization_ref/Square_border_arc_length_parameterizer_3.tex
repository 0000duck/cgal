% +------------------------------------------------------------------------+
% | Reference manual page: Square_border_arc_length_parameterizer_3.tex
% +------------------------------------------------------------------------+
% | 21.09.2005   Laurent Saboret, Pierre Alliez, Bruno Levy
% | Package: Surface_mesh_parameterization
% |
\RCSdef{\RCSSquareborderarclengthparameterizerRev}{$Id$}
\RCSdefDate{\RCSSquareborderarclengthparameterizerDate}{$Date$}
% |
\ccRefPageBegin
%%RefPage: end of header, begin of main body
% +------------------------------------------------------------------------+


\begin{ccRefClass}{Square_border_arc_length_parameterizer_3<ParameterizationMesh_3>}

%% \ccHtmlCrossLink{}     %% add further rules for cross referencing links
%% \ccHtmlIndexC[class]{} %% add further index entries


\ccDefinition

% The section below is automatically generated. Do not edit!
%START-AUTO(\ccDefinition)

This class parameterizes the border of a 3D surface onto a square, with an arc-length parameterization: (u,v) values are proportional to the length of border edges.

\ccc{Square_border_parameterizer_3} implements most of the border parameterization algorithm. This class implements only \ccc{compute_edge_length}() to compute a segment's length.

%END-AUTO(\ccDefinition)

% The section below is automatically generated. Do not edit!
%START-AUTO(\ccInclude)

\ccInclude{CGAL/Square_border_parameterizer_3.h}

%END-AUTO(\ccInclude)


\ccInheritsFrom

% The section below is automatically generated. Do not edit!
%START-AUTO(\ccInheritsFrom)

\ccc{Square_border_parameterizer_3<ParameterizationMesh_3>}

%END-AUTO(\ccInheritsFrom)


\ccIsModel

% The section below is automatically generated. Do not edit!
%START-AUTO(\ccIsModel)

Model of the \ccc{BorderParameterizer_3} concept.

%END-AUTO(\ccIsModel)


\ccHeading{Design Pattern}

% The section below is automatically generated. Do not edit!
%START-AUTO(\ccHeading{Design Pattern})

\ccc{BorderParameterizer_3} models are Strategies \cite{cgal:ghjv-dpero-95}: they implement a strategy of border parameterization for models of \ccc{ParameterizationMesh_3}

%END-AUTO(\ccHeading{Design Pattern})


\ccParameters

The full template declaration is:

% The section below is automatically generated. Do not edit!
%START-AUTO(\ccParameters)

template$<$  \\
class \ccc{ParameterizationMesh_3}$>$   \\
class \ccc{Square_border_arc_length_parameterizer_3};

%END-AUTO(\ccParameters)


\ccCreation
\ccCreationVariable{bp}  %% variable name for \ccMethod

% The section below is automatically generated. Do not edit!
%START-AUTO(\ccCreation)
%END-AUTO(\ccCreation)


\ccOperations

% The section below is automatically generated. Do not edit!
%START-AUTO(\ccOperations)

\ccMethod{virtual double compute_edge_length(const Adaptor& mesh, Vertex_const_handle source, Vertex_const_handle target);}
{
[protected, virtual] \\
Compute the length of an edge.
Arc-length border parameterization: (u,v) values are proportional to the length of border edges.
}
\ccGlue

%END-AUTO(\ccOperations)


\ccSeeAlso

\ccRefIdfierPage{CGAL::Circular_border_arc_length_parameterizer_3<ParameterizationMesh_3>}  \\
\ccRefIdfierPage{CGAL::Circular_border_uniform_parameterizer_3<ParameterizationMesh_3>}  \\
\ccRefIdfierPage{CGAL::Square_border_uniform_parameterizer_3<ParameterizationMesh_3>}  \\
\ccRefIdfierPage{CGAL::Two_vertices_parameterizer_3<ParameterizationMesh_3>}  \\


\ccExample

See \ccc{Square_border_parameterization.cpp} example.


\end{ccRefClass}

% +------------------------------------------------------------------------+
%%RefPage: end of main body, begin of footer
\ccRefPageEnd
% EOF
% +------------------------------------------------------------------------+

