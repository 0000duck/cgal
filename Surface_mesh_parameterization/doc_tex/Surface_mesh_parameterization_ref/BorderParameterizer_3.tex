% +------------------------------------------------------------------------+
% | Reference manual page: BorderParameterizer_3.tex
% +------------------------------------------------------------------------+
% | 23.08.2005   Laurent Saboret, Pierre Alliez, Bruno Levy
% | Package: Surface_mesh_parameterization
% |
\RCSdef{\RCSBorderParameterizerRev}{$Id$}
\RCSdefDate{\RCSBorderParameterizerDate}{$Date$}
% |
\ccRefPageBegin
%%RefPage: end of header, begin of main body
% +------------------------------------------------------------------------+


\begin{ccRefConcept}{BorderParameterizer_3}

%% \ccHtmlCrossLink{}     %% add further rules for cross referencing links
%% \ccHtmlIndexC[concept]{} %% add further index entries


\ccDefinition

% The section below is automatically generated. Do not edit!
%START-AUTO(\ccDefinition)

\ccc{BorderParameterizer_3} is a concept of class that parameterizes a given type of mesh, 'Adaptor', which is a model of the \ccc{ParameterizationMesh_3} concept.

Implementation note: To simplify the implementation, \ccc{BorderParameterizer_3} models know only the \ccc{ParameterizationMesh_3} class. They do not know the parameterization algorithm requirements or the kind of sparse linear system used.

%END-AUTO(\ccDefinition)


\ccHeading{Design Pattern}

% The section below is automatically generated. Do not edit!
%START-AUTO(\ccHeading{Design Pattern})

\ccc{BorderParameterizer_3} models are Strategies [GHJV95]: they implement a strategy of border parameterization for models of \ccc{ParameterizationMesh_3}.

%END-AUTO(\ccHeading{Design Pattern})


\ccTypes

% The section below is automatically generated. Do not edit!
%START-AUTO(\ccTypes)

\ccNestedType{Adaptor}
{
Export \ccc{ParameterizationMesh_3} template parameter.
}
\ccGlue
\ccNestedType{Error_code}
{
The various errors detected by this package.
}
\ccGlue

%END-AUTO(\ccTypes)


\ccCreation
\ccCreationVariable{bp}  %% variable name for \ccMethod below

Construction and destruction are undefined.

% The section below is automatically generated. Do not edit!
%START-AUTO(\ccCreation)
%END-AUTO(\ccCreation)


\ccOperations

% The section below is automatically generated. Do not edit!
%START-AUTO(\ccOperations)

\ccMethod{Error_code parameterize_border(Adaptor& mesh);}
{
Assign to mesh's border vertices a 2D position (ie a (u,v) pair) on border's shape. Mark them as {\em parameterized}. Return false on error.
}
\ccGlue
\ccMethod{bool is_border_convex();}
{
Indicate if border's shape is convex.
}
\ccGlue

%END-AUTO(\ccOperations)


\ccHasModels

\ccRefIdfierPage{CGAL::Circular_border_arc_length_parameterizer_3<ParameterizationMesh_3>}  \\
\ccRefIdfierPage{CGAL::Circular_border_uniform_parameterizer_3<ParameterizationMesh_3>}  \\
\ccRefIdfierPage{CGAL::Square_border_arc_length_parameterizer_3<ParameterizationMesh_3>}  \\
\ccRefIdfierPage{CGAL::Square_border_uniform_parameterizer_3<ParameterizationMesh_3>}  \\
\ccRefIdfierPage{CGAL::Two_vertices_parameterizer_3<ParameterizationMesh_3>}  \\


\ccSeeAlso

\ccRefIdfierPage{ParameterizerTraits_3}  \\
\ccRefIdfierPage{ParameterizationMesh_3}  \\


\end{ccRefConcept}

% +------------------------------------------------------------------------+
%%RefPage: end of main body, begin of footer
\ccRefPageEnd
% EOF
% +------------------------------------------------------------------------+

