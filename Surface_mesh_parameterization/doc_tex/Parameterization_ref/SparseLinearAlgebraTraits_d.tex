% +------------------------------------------------------------------------+
% | Reference manual page: SparseLinearAlgebraTraits_d.tex
% +------------------------------------------------------------------------+
% | 21.09.2005   Laurent Saboret, Pierre Alliez, Bruno Levy
% | Package: Surface_mesh_parameterization
% |
\RCSdef{\RCSSparseLinearAlgebraTraitsdRev}{$Id$}
\RCSdefDate{\RCSSparseLinearAlgebraTraitsdDate}{$Date$}
% |
%%RefPage: end of header, begin of main body
% +------------------------------------------------------------------------+


\begin{ccRefConcept}{SparseLinearAlgebraTraits_d}

%% \ccHtmlCrossLink{}     %% add further rules for cross referencing links
%% \ccHtmlIndexC[concept]{} %% add further index entries


\ccDefinition

% The section below is automatically generated. Do not edit!
%START-AUTO(\ccDefinition)

The concept SparseLinearAlgebraTraits\_d is used to solve sparse linear systems {\em A$\ast$X = B}.

\begin{description}
\item[Todo]Add to SparseLinearAlgebraTraits\_d the ability to solve linear systems in the least squares sense.\end{description}

%END-AUTO(\ccDefinition)


\ccRefines

% The section below is automatically generated. Do not edit!
%START-AUTO(\ccRefines)

This is a sub-concept of LinearAlgebraTraits\_d.

%END-AUTO(\ccRefines)


\ccTypes

% The section below is automatically generated. Do not edit!
%START-AUTO(\ccTypes)

\ccNestedType{Matrix}
{
}
\ccGlue
\ccNestedType{Vector}
{
}
\ccGlue
\ccNestedType{NT}
{
}
\ccGlue

%END-AUTO(\ccTypes)


\ccCreation
\ccCreationVariable{sparse_LA}  %% variable name for \ccMethod

% The section below is automatically generated. Do not edit!
%START-AUTO(\ccCreation)

\ccConstructor{SparseLinearAlgebraTraits_d ();}
{
Default constructor.
}
\ccGlue

%END-AUTO(\ccCreation)


\ccOperations

% The section below is automatically generated. Do not edit!
%START-AUTO(\ccOperations)

\ccMethod{bool linear_solver (const Matrix & A, const Vector & B, Vector & X, NT & D);}
{
Solve the sparse linear system {\em A$\ast$X = B}. Return true on success. The solution is then (1/D) $\ast$ X.
Preconditions:\begin{itemize}
\item A.row\_dimension() == B.dimension().\item A.column\_dimension() == X.dimension(). \end{itemize}
}
\ccGlue
\ccMethod{bool is_solvable (const Matrix & A, const Vector & B);}
{
Indicate if the linear system can be solved and if the matrix conditioning is good.
Preconditions:\begin{itemize}
\item A.row\_dimension() == B.dimension(). \end{itemize}
}
\ccGlue

%END-AUTO(\ccOperations)


\ccHasModels

\ccRefIdfierPage{CGAL::Taucs_solver_traits}  \\
\ccRefIdfierPage{CGAL::Taucs_symmetric_solver_traits}  \\
\ccc{OpenNL::DefaultLinearSolverTraits}  \\
\ccc{OpenNL::SymmetricLinearSolverTraits}  \\


\ccSeeAlso

\ccRefIdfierPage{SparseLinearAlgebraTraits_d::Matrix}  \\
\ccRefIdfierPage{SparseLinearAlgebraTraits_d::Vector}  \\


\end{ccRefConcept}

% +------------------------------------------------------------------------+
%%RefPage: end of main body, begin of footer
% EOF
% +------------------------------------------------------------------------+

