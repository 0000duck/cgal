%% Copyright (c) 2005  Foundation for Research and Technology-Hellas (Greece).
%% All rights reserved.
%%
%% This file is part of CGAL (www.cgal.org).
%% You can redistribute it and/or modify it under the terms of the GNU
%% General Public License as published by the Free Software Foundation,
%% either version 3 of the License, or (at your option) any later version.
%%
%% Licensees holding a valid commercial license may use this file in
%% accordance with the commercial license agreement provided with the software.
%%
%% This file is provided AS IS with NO WARRANTY OF ANY KIND, INCLUDING THE
%% WARRANTY OF DESIGN, MERCHANTABILITY AND FITNESS FOR A PARTICULAR PURPOSE.
%%
%% $URL$
%% $Id$
%% 
%%
%% Author(s)     : Menelaos Karavelas <mkaravel@iacm.forth.gr>


\begin{ccRefClass}{Voronoi_diagram_2<DG,AT,AP>}

%% add template arg's if necessary

%% \ccHtmlCrossLink{}     %% add further rules for cross referencing links
%% \ccHtmlIndexC[class]{} %% add further index entries
\ccCreationVariable{traits}
\ccDefinition

The class \ccRefName\ provides an adaptor that enables us
to view a triangulated Delaunay graph as their dual subdivision, the
Voronoi diagram. The class \ccRefName\ is designed to provide an API
that is similar to that of \cgal's arrangements.\\
The first template parameter of the \ccRefName\ class corresponds to the
triangulated Delaunay graph and must be a model of the
\ccc{DelaunayGraph_2} concept. The second template parameter must be a
model of the \ccc{AdaptationTraits_2} concept. The third template
parameter must be a model of the \ccc{AdaptationPolicy_2} concept. The
third template parameter defaults to \ccc{CGAL::Identity_policy_2<DG,AT>}.

\ccInclude{CGAL/Voronoi_diagram_2.h}

\ccRefines
\ccc{DefaultConstructible}, \ccc{CopyConstructible}, \ccc{Assignable}

\ccTypes
%
\ccThree{typedef Adaptation_traits::Vertex_handle}{Delaunay_vertex_handle+}{}
\ccThreeToTwo
%
\ccTypedef{typedef DG Delaunay_graph;}{A type for the dual Delaunay graph.}
\ccGlue
\ccTypedef{typedef AT Adaptation_traits;}{A type for the adaptation
  traits needed by the Voronoi diagram adaptor.}
\ccGlue
\ccTypedef{typedef AP Adaptation_policy;}{A type for the adaptation
  policy used.}
\ccGlue
\ccTypedef{typedef Adaptation_traits::Point_2 Point_2;}{A type a point.}
\ccGlue
\ccTypedef{typedef Adaptation_traits::Site_2 Site_2;}{A type
  for the sites of the Voronoi diagram.}
\ccGlue
\ccTypedef{typedef Delaunay_graph::size_type size_type;}{A type for sizes.}
\ccGlue
\ccTypedef{typedef Delaunay_graph::Geom_traits Delaunay_geom_traits;}
{A type for the geometric traits of the Delaunay graph.}
\ccGlue
\ccTypedef{typedef Delaunay_graph::Vertex_handle Delaunay_vertex_handle;}
{A type for the vertex handles of the Delaunay graph.}
\ccGlue
\ccTypedef{typedef Delaunay_graph::Face_handle Delaunay_face_handle;}
{A type for the face handles of the Delaunay graph.}
\ccGlue
\ccTypedef{typedef Delaunay_graph::Edge Delaunay_edge;}
{A type for the edges of the Delaunay graph.}
\ccTwo{Voronoi_diagram_2<DG,AT,AP>::Halfedge+}{}
\ccGlue
%  It must be a model of the \ccc{VoronoiDiagramLocateResult_2} concept.}
\ccNestedType{Halfedge}{A type for the halfedges of the Voronoi diagram.}
%  It is a model of the \ccc{VoronoiDiagramHalfedge_2} concept.}
\ccGlue
\ccNestedType{Vertex}{A type for the vertices of the Voronoi diagram.}
%  It is a model of the \ccc{VoronoiDiagramVertex_2} concept.}
\ccGlue
\ccNestedType{Face}{A type for the faces of the Voronoi diagram.}
%  It is a model of the \ccc{VoronoiDiagramFace_2} concept.}

The vertices, edges and faces of the Voronoi diagram are accessed
through \ccc{handles}, \ccc{iterators} and \ccc{circulators}. 
The iterators and circulators are all bidirectional and non-mutable.
The circulators and iterators are assignable to the 
corresponding handle types, and they are also convertible to the
corresponding handles.

\ccTwo{Voronoi_diagram_2<DG,AT,AP>::Unbounded_halfedges_iterator+}{}
%
\ccNestedType{Halfedge_handle}{Handle for halfedges.}
\ccGlue
\ccNestedType{Vertex_handle}{Handle for vertices.}
\ccGlue
\ccNestedType{Face_handle}{Handle for faces.}
\ccGlue
\ccNestedType{Edge_iterator}{A type for an iterator over Voronoi
  edges. Edges are considered non-oriented. Its value type is
  \ccc{Halfedge}.}
\ccGlue
\ccNestedType{Halfedge_iterator}{A type for an iterator over Voronoi
  halfedges. Halfedges are oriented and come in pairs. Its value type
  is \ccc{Halfedge}.}
\ccGlue
\ccNestedType{Face_iterator}{A type for an iterator over Voronoi
  faces. Its value type is \ccc{Face}.}
\ccGlue
\ccNestedType{Vertex_iterator}{A type for an iterator over Voronoi
  vertices. Its value type is \ccc{Vertex}.}
\ccGlue
\ccNestedType{Halfedge_around_vertex_circulator}{A type for a
  circulator over the halfedges that have a common vertex as their
  target. Its value type is \ccc{Halfedge}.}
\ccGlue
\ccNestedType{Ccb_halfedge_circulator}{A type for a circulator over
  the halfedges on the boundary of a Voronoi face. Its value type of
  is \ccc{Halfedge}.}
\ccGlue
\ccNestedType{Unbounded_faces_iterator}{A type for an iterator over
  the unbounded faces of the Voronoi diagram. Its value type is
  \ccc{Face}.}
\ccGlue
\ccNestedType{Bounded_faces_iterator}{A type for an iterator over
  the bounded faces of the Voronoi diagram. Its value type is
  \ccc{Face}.}
\ccGlue
\ccNestedType{Unbounded_halfedges_iterator}{A type for an iterator over
  the unbounded halfedges of the Voronoi diagram. Its value type is
  \ccc{Halfedge}.}
\ccGlue
\ccNestedType{Bounded_halfedges_iterator}{A type for an iterator over
  the bounded halfedges of the Voronoi diagram. Its value type is
  \ccc{Halfedge}.}
\ccGlue
\ccNestedType{Site_iterator}{A type for an iterator over the
  sites of the Voronoi diagram. Its value type is \ccc{Site_2}.}
%\ccGlue
\ccThree{typedef boost::variant<Face_handle,Halfedge_handle,Vertex_handle>}
{Locate_result+}{}
\ccThreeToTwo
\ccTypedef{typedef boost::variant<Face_handle,Halfedge_handle,Vertex_handle>
Locate_result;}{The result type of the point location queries.}

\ccCreationVariable{vd}

\ccCreation
\ccThree{Voronoi_diagram_2<DG,AT,AP>}{vd}{}
\ccThreeToTwo
%
\ccConstructor{Voronoi_diagram_2(Adaptation_traits
  at = Adaptation_traits(), Adaptation_policy ap = Adaptation_policy(),
  Delaunay_geom_traits gt = Delaunay_geom_traits());}
{Creates a Voronoi diagram using \ccc{at} as adaptation traits and
  \ccc{ap} as adaptation policy; the underlying Delaunay graph is
  created using \ccc{gt} as geometric traits.}
%
\ccGlue
\ccConstructor{Voronoi_diagram_2(Delaunay_graph dg, bool swap_dg = false,
  Adaptation_traits at = Adaptation_traits(), Adaptation_policy ap =
  Adaptation_policy());}
{Creates a Voronoi diagram from the Delaunay graph \ccc{dg} and using
  \ccc{at} as adaptation traits and \ccc{ap} as adaptation policy. The 
  Delaunay graph \ccc{dg} is fully copied if \ccc{swap_dg} is set to
  \ccStyle{false}, or swapped with the one stored internally if
  \ccc{swap_dg} is set to \ccStyle{true}.}
%
\ccGlue
\ccConstructor{template<class Iterator>
  Voronoi_diagram_2(Iterator first, Iterator beyond, Adaptation_traits
  at = Adaptation_traits(), Adaptation_policy ap = Adaptation_policy(),
  Delaunay_geom_traits gt = Delaunay_geom_traits());}
{Creates a Voronoi diagram using as sites the sites in the iterator
  range \ccStyle{[first, beyond)}, \ccc{at} as adaptation traits and
  \ccc{ap} as adaptation policy; the underlying Delaunay graph is
  created using \ccc{gt} as geometric traits. \ccc{Iterator} must be a
  model of the \ccc{InputIterator} concept and its value type must be
  \ccc{Site_2}.}


\ccHeading{Access Methods}
%
\ccThree{Adaptation_policy}{vd.number_of_connected_components()+}{}
%
\ccMethod{Delaunay_graph dual();}
{Returns a const reference to the dual graph, i.e., the Delaunay graph.}
\ccGlue
\ccMethod{Halfedge_handle dual(Delaunay_edge e);}
{Returns a handle to the halfedge in the Voronoi diagram that is dual
  to the edge \ccc{e} in the Delaunay graph.}
\ccGlue
\ccMethod{Face_handle dual(Delaunay_vertex_handle v);}
{Returns a handle to the face in the Voronoi diagram that is dual to
  the vertex corresponding to the vertex handle \ccc{v} in the
  Delaunay graph.}
\ccGlue
\ccMethod{Vertex_handle dual(Delaunay_face_handle f);}
{Returns a handle to the vertex in the Voronoi diagram that is dual to
  the face corresponding to the face handle \ccc{f} in the Delaunay graph.}
\ccGlue
\ccMethod{Adaptation_traits adaptation_traits();}
{Returns a reference to the Voronoi traits.}
\ccGlue
\ccMethod{Adaptation_policy adaptation_policy();}
{Returns a reference to the adaptation policy.}
\ccGlue
\ccMethod{size_type number_of_vertices();}
{Returns the number of Voronoi vertices.}
\ccGlue
\ccMethod{size_type number_of_faces();}
{Returns the number of Voronoi faces (bounded and unbounded).}
\ccGlue
\ccMethod{size_type number_of_halfedges();}
{Returns the number of halfedges (bounded and unbounded) in the
  Voronoi diagram. This is always an even number.}
\ccGlue
\ccMethod{size_type number_of_connected_components();}
{Returns the number of connected components of the Voronoi skeleton.}
\ccGlue
\ccMethod{Face_handle unbounded_face();}{Returns one of the unbounded
  faces of the Voronoi diagram. If no unbounded faces exist (this can
  happen if the number of sites is zero) the
  default constructed face handle is returned.}
\ccGlue
\ccMethod{Face_handle bounded_face();}{Returns one of the bounded
  faces of the Voronoi diagram. If no bounded faces exist the default
  constructed face handle is returned.}
\ccGlue
\ccMethod{Halfedge_handle unbounded_halfedge();}{Returns one of the unbounded
  halfedges of the Voronoi diagram. If no unbounded halfedges exist the
  default constructed halfedge handle is returned.}
\ccGlue
\ccMethod{Halfedge_handle bounded_halfedge();}{Returns one of the bounded
  halfedges of the Voronoi diagram. If no bounded halfedges exist the
  default constructed halfedge handle is returned.}
%
%% \ccThree{Edge_degeneracy_tester}{vd.edge_tester()+}{}
%% \ccMethod{Edge_degeneracy_tester edge_tester();}{Returns a reference
%%   to the edge degeneracy tester.}
%% \ccGlue
%% \ccMethod{Face_degeneracy_tester face_tester();}{Returns a reference
%%   to the face degeneracy tester.}



\ccHeading{Traversal of the Voronoi diagram}


A Voronoi diagram can be seen as a container of faces, vertices and
halfedges. Therefore the Voronoi diagram provides several iterators
and circulators that allow to traverse it.



\ccHeading{Iterators}

\ccThree{Unbounded_halfedges_iterator}{vd.unbounded_halfedges_begin()+}{}

The following iterators allow respectively to visit the faces (all or
only the unbounded/bounded ones), edges, halfedges (all or only the
unbounded/bounded ones) and vertices of the Voronoi diagram. These
iterators are non-mutable, bidirectional and their 
value types are respectively \ccc{Face}, \ccc{Halfedge},
\ccc{Halfedge} and \ccc{Vertex}. All iterators are convertible to the
corresponding handles and are invalidated by any change in the Voronoi
diagram.

\ccMethod{Face_iterator faces_begin();}{Starts at an arbitrary Voronoi face.}
\ccGlue
\ccMethod{Face_iterator faces_end();}{Past-the-end iterator.}

\ccMethod{Unbounded_faces_iterator unbounded_faces_begin();}
{Starts at an arbitrary unbounded Voronoi face.}
\ccGlue
\ccMethod{Unbounded_faces_iterator unbounded_faces_end();}
{Past-the-end iterator.}

\ccMethod{Bounded_faces_iterator bounded_faces_begin();}
{Starts at an arbitrary bounded Voronoi face.}
\ccGlue
\ccMethod{Bounded_faces_iterator bounded_faces_end();}
{Past-the-end iterator.}

\ccMethod{Edge_iterator edges_begin();}{Starts at an arbitrary Voronoi edge.}
\ccGlue
\ccMethod{Edge_iterator edges_end();}{Past-the-end iterator.}

\ccMethod{Halfedge_iterator halfedges_begin();}
{Starts at an arbitrary Voronoi halfedge.}
\ccGlue
\ccMethod{Halfedge_iterator halfedges_end();}{Past-the-end iterator.}

\ccMethod{Unbounded_halfedges_iterator unbounded_halfedges_begin();}
{Starts at an arbitrary unbounded Voronoi edge.}
\ccGlue
\ccMethod{Unbounded_halfedges_iterator unbounded_halfedges_end();}
{Past-the-end iterator.}

\ccMethod{Bounded_halfedges_iterator bounded_halfedges_begin();}
{Starts at an arbitrary bounded Voronoi edge.}
\ccGlue
\ccMethod{Bounded_halfedges_iterator bounded_halfedges_end();}
{Past-the-end iterator.}

\ccMethod{Vertex_iterator vertices_begin();}
{Starts at an arbitrary Voronoi vertex.}
\ccGlue
\ccMethod{Vertex_iterator vertices_end();}{Past-the-end iterator.}

The following iterator provides access to the sites that define
the Voronoi diagram. Its value type is \ccc{Site_2}. It is
invalidated by any change in the Voronoi diagram.

\ccMethod{Site_iterator sites_begin();}
{Starts at an arbitrary site.}
\ccGlue
\ccMethod{Site_iterator sites_end();}{Past-the-end iterator.}


\ccHeading{Circulators}

The Voronoi diagram adaptor also provides circulators that allow to visit 
all halfedges whose target is a given vertex -- this is the
\ccc{Halfedge_around_vertex_circulator}, as well as all halfedges on
the boundary of a Voronoi face -- this is the
\ccc{Ccb_halfedge_circulator}. These circulators are non-mutable  
and bidirectional.
The operator \ccc{operator++} moves the former circulator counterclockwise
around the vertex while the \ccc{operator--} moves clockwise.
The latter circulator is moved by the operator \ccc{operator++} to the
next halfedge on the boundary in the counterclockwise sense, while
\ccc{operator--} moves clockwise.
When the \ccc{Ccb_halfedge_circulator} is defined over an infinite
Voronoi face \ccc{f}, then applying \ccc{operator++} to a circulator
corresponding to a halfedge whose target is not finite moves to the
next infinite (or semi-infinite) halfedge of \ccc{f} in the
counterclockwise sense. Similarly, applying \ccc{operator++} to a
circulator corresponding to a halfedge whose source is not finite,
moves to the previous infinite (or semi-infinite) halfedge of \ccc{f}
in the clockwise sense.
The \ccc{Halfedge_around_vertex_circulator} circulator is
invalidated by any modification in the faces adjacent to the vertex
over which it is defined. The \ccc{Ccb_halfedge_circulator} is
invalidated by any modification in the face over which it is defined.


\ccThree{Ccb_halfedge_circulator}{vd.ccb_halfedges(Face_handle f)+}{}
\ccThreeToTwo
%
\ccMethod{Ccb_halfedge_circulator ccb_halfedges(Face_handle f);}
{Returns a circulator over the halfedges on the boundary of \ccc{f}.
  The circulator is initialized to an arbitrary halfedge on the
  boundary of the Voronoi face \ccc{f}.}
\ccGlue
\ccMethod{Ccb_halfedge_circulator ccb_halfedges(Face_handle f,
  Halfedge_handle h);}
{Returns a circulator over the halfedges on the boundary of
  \ccc{f}. The circulator is initialized with the halfedge \ccc{h}.
  \ccPrecond{The halfedge \ccc{h} must lie on the boundary of \ccc{f}.}}

\ccThree{Halfedge_around_vertex_circulator}{vd.incident_halfedges+}{}
\ccThreeToTwo
%
\ccMethod{Halfedge_around_vertex_circulator
  incident_halfedges(Vertex_handle v);}
{Returns a circulator over the halfedges whose target is the Voronoi
  vertex \ccc{v}. The circulator is initialized to an arbitrary
  halfedge incident to \ccc{v}.}
\ccGlue
\ccMethod{Halfedge_around_vertex_circulator
  incident_halfedges(Vertex_handle v, Halfedge_handle h);}
{Returns a circulator over the halfedges whose target is the Voronoi
  vertex \ccc{v}. The circulator is initialized with the halfedge
  \ccc{h}.
  \ccPrecond{The vertex \ccc{v} must be the target vertex of the
    halfedge \ccc{h}.}}


\begin{ccTexOnly}
\newpage
\end{ccTexOnly}


\ccHeading{Insertion}
\ccThree{Face_handle}{vd.insert(Site_2 t)+}{}
\ccMethod{Face_handle insert(Site_2 t);}{Inserts the site
  \ccc{t} in the Voronoi diagram. A handle to the face corresponding
  to the Voronoi face of \ccc{t} in the Voronoi diagram is
  returned. If \ccc{t} has an empty Voronoi cell, the default
  constructed face handle is returned. This method is supported only
  if \ccStyle{Voronoi_traits::Has_inserter} is set to
  \ccStyle{CGAL::Tag_true}.}
\ccGlue
\ccMethod{template<class Iterator>
  size_type insert(Iterator first, Iterator beyond);}{Inserts, in the
  Voronoi diagram, the sites in the iterator range \ccc{[first,
    beyond)}. The value type of \ccc{Iterator} must be
  \ccc{Site_2}. The number of sites in the iterator range is
  returned. This method is supported only if
  \ccStyle{Voronoi_traits::Has_inserter} is set to \ccStyle{CGAL::Tag_true}.}


\ccHeading{Queries}
\ccThree{Locate_result}{vd.locate(Point_2 p)+}{}
\ccMethod{Locate_result locate(Point_2 p);}{Performs point location for
  the query point \ccc{p}. In other words, the face, halfedge or
  vertex of the Voronoi diagram is found on which the point \ccc{p}
  lies. This method is supported only if
  \ccStyle{Voronoi_traits::Has_nearest_site_2} is set to
  \ccStyle{CGAL::Tag_true}.
  \ccPrecond{The Voronoi diagram must contain at least one face.}}
%\ccGlue
%\ccMethod{template<class OutputIt> OutputIt nearest_sites(Point_2 p,
%  OutputIt oit);}{\ccc{OutputIt} is an output iterator with
%  \ccc{Face_handle} as value type. This methods outputs in the
%  container pointed to by \ccc{oit} the face handles corresponding to
%  the nearest site(s) of \ccc{p} in the Voronoi diagram. The resulting
%  output iterator is returned.}




\ccHeading{I/O}
\ccThree{std::ostream&}{vd.file_output(std::ostream& os)+}{}
%
\ccMethod{void file_output(std::ostream& os);}
{Writes the current state of the Voronoi diagram to the output
  stream \ccc{os}.\\
  The following operator must be defined:\\
  \ccc{std::ostream& operator<<(std::ostream&, Delaunay_graph)}
}
\ccGlue
\ccMethod{void file_input(std::istream& is);}
{Reads the current state of the Voronoi diagram from the input
  stream \ccc{is}.\\
  The following operator must be defined:\\
  \ccc{std::istream& operator>>(std::istream&, Delaunay_graph)}
}
\ccFunction{
std::ostream& operator<<(std::ostream& os, Voronoi_diagram_2<DG,AT,AP> vd);}
{Writes the current state of the Voronoi diagram to the output
  stream \ccc{os}.\\
  The following operator must be defined:\\
  \ccc{std::ostream& operator<<(std::ostream&, Delaunay_graph)}
}
\ccGlue
\ccFunction{
std::istream& operator>>(std::istream& is, Voronoi_diagram_2<DG,AT,AP> vd);}
{Reads the current state of the Voronoi diagram from the input
  stream \ccc{is}.\\
  The following operator must be defined:\\
  \ccc{std::istream& operator>>(std::istream&, Delaunay_graph)}
}

\ccHeading{Validity check}
\ccThree{bool}{vd.is_valid()+}{}
%
\ccMethod{bool is_valid();}{Checks the validity of the dual Delaunay
  graph and the Voronoi diagram adaptor.}

\ccHeading{Miscellaneous}
\ccThree{void}{vd.swap(other)+}{}
%
\ccMethod{void clear();}{Clears all contents of the Voronoi diagram.}
\ccGlue
\ccMethod{void swap(Voronoi_diagram_2<DG,AT,AP> other);}{The Voronoi
  diagrams \ccc{other} and \ccVar\ are
  swapped. \ccVar.\ccc{swap(other)} should be preferred to
  \ccVar\ccc{ = other} or to 
  \ccVar\ccc{(other)} if \ccc{other} is deleted afterwards.}

\ccSeeAlso
\ccc{DelaunayGraph_2}\\
\ccc{AdaptationTraits_2}\\
\ccc{AdaptationPolicy_2}\\
\ccc{CGAL::Voronoi_diagram_2<DG,AT,AP>::Face}\\
\ccc{CGAL::Voronoi_diagram_2<DG,AT,AP>::Halfedge}\\
\ccc{CGAL::Voronoi_diagram_2<DG,AT,AP>::Vertex}\\
\ccc{CGAL::Delaunay_triangulation_2<Traits,Tds>}\\
\ccc{CGAL::Regular_triangulation_2<Traits,Tds>}\\
\ccc{CGAL::Triangulation_hierarchy_2<Tr>} provided that \ccc{Tr} is a
model of \ccc{DelaunayGraph_2}\\
\ccc{CGAL::Segment_Delaunay_graph_2<Gt,DS>}\\
\ccc{CGAL::Segment_Delaunay_graph_hierarchy_2<Gt,STag,DS>}\\
\ccc{CGAL::Apollonius_graph_2<Gt,Agds>}\\
\ccc{CGAL::Apollonius_graph_hierarchy_2<Gt,Agds>}\\
\ccc{CGAL::Apollonius_graph_adaptation_traits_2<AG2>}\\
\ccc{CGAL::Delaunay_triangulation_adaptation_traits_2<DT2>}\\
\ccc{CGAL::Regular_triangulation_adaptation_traits_2<RT2>}\\
\ccc{CGAL::Segment_Delaunay_graph_adaptation_traits_2<SDG2>}\\
\ccc{CGAL::Identity_policy_2<DG,AT>}\\
\ccc{CGAL::Apollonius_graph_degeneracy_removal_policy_2<AG2>}\\
\ccc{CGAL::Apollonius_graph_caching_degeneracy_removal_policy_2<AG2>}\\
\ccc{CGAL::Delaunay_triangulation_degeneracy_removal_policy_2<DT2>}\\
\ccc{CGAL::Delaunay_triangulation_caching_degeneracy_removal_policy_2<DT2>}\\
\ccc{CGAL::Regular_triangulation_degeneracy_removal_policy_2<RT2>}\\
\ccc{CGAL::Regular_triangulation_caching_degeneracy_removal_policy_2<RT2>}\\
\ccc{CGAL::Segment_Delaunay_graph_degeneracy_removal_policy_2<SDG2>}\\
\ccc{CGAL::Segment_Delaunay_graph_caching_degeneracy_removal_policy_2<SDG2>}

\end{ccRefClass}

%% EOF
