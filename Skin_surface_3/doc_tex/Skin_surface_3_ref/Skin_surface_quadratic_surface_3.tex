% +------------------------------------------------------------------------+
% | Reference manual page: Skin_surface_quadratic_surface_3.tex
% +------------------------------------------------------------------------+
% | 27.09.2005   Nico Kruithof
% | Package: Skin_surface
% | 
\RCSdef{\RCSSkinsurfacequadraticsurfaceRev}{$Id$}
\RCSdefDate{\RCSSkinsurfacequadraticsurfaceDate}{$Date$}
% +------------------------------------------------------------------------+

\ccRefPageBegin

%%RefPage: end of header, begin of main body
% +------------------------------------------------------------------------+


\begin{ccRefClass}{Skin_surface_quadratic_surface_3<Kernel>}
\label{refSkinsurfacequadraticsurface}
\ccDefinition
  
The virtual class \ccRefName\ defines a quadratic surface specifically
designed for the skin surface package.

\ccInclude{CGAL/Skin_surface_quadratic_surface_3.h}

\ccTypes
The geometric types used by the quadratic surface:
\ccThree{typedef Weighted_point<Point, RT> x}{Triangulated}{}
\ccThreeToTwo

\ccTypedef{typedef  Kernel::RT                  RT;}{}
\ccGlue
\ccTypedef{typedef  Kernel::Point_3             Point;}{}
\ccGlue
\ccTypedef{typedef  Kernel::Vector_3            Vector;}{}
\ccGlue
\ccTypedef{typedef  Kernel::Segment_3           Segment;}{}
\ccGlue
\ccTypedef{typedef Weighted_point<Point, RT> Weighted_point;}{}

\ccCreation
\ccCreationVariable{surf}  %% choose variable name

This class is virtual and cannot be constructed without a derived
class that implements the pure virtual functions.

\ccOperations
\ccThree{Point}{to_surfaceconst Point &p0, const }{}
\ccThreeToTwo

\ccMethod{Point to_surface(const Point &p0, const Point &p1);}
{Returns a point close to the intersection point of the line segment
  \ccc{p0p1} with the surface.}
%
\ccGlue 
%
\ccMethod{Point to_surface(const Segment &s);}{Equal to
  \ccc{to_surface(s.source(), s.target())}.}

The surfaces are given in implicit form, where the surface is the
zeroset of a scalar valued function. The following two functions
return the function value and the gradient in a given point.

\ccMethod{virtual RT value(const Point &p) const = 0;}{Returns the
  function value of the quadric.}

\ccGlue \ccMethod{virtual Vector gradient(const Point &p) = 0;
}{Returns the gradient in \ccc{p}. }

\ccSeeAlso
\ccRefIdfierPage{CGAL::Skin_surface_hyperboloid_3<Kernel>}\\
\ccRefIdfierPage{CGAL::Skin_surface_sphere_3<Kernel>}
\end{ccRefClass}

