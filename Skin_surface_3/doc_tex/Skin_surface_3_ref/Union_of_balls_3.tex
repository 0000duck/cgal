% +------------------------------------------------------------------------+
% | Reference manual page: skin_surface_3.tex
% +------------------------------------------------------------------------+
% | 27.09.2005   Nico Kruithof
% | Package: Skin_surface_3
% | 
\RCSdef{\RCSskinsurfaceRev}{$Id$}
\RCSdefDate{\RCSskinsurfaceDate}{$Date$}
% |
%%RefPage: end of header, begin of main body
% +------------------------------------------------------------------------+

\begin{ccRefClass}{Union_of_balls_3<RegularTriangulationTraits_3>}
  \ccDefinition

  The \ccRefName\ is used to represent a skin surface with shrink
  factor equal to one, which is the boundary union of the input balls.
  This case is handled separately since the mixed complex is equal to
  the power diagram, which has a much simpler structure. The functions
  and types have the same name as the \ccc{Skin_surface_3} class to
  make the algorithms work on both surface types.

  The type \ccc{RegularTriangulationTraits_3} is required to be a model of the
  concept \ccc{RegularTriangulationTraits_3}, which means that it provides the
  predicates to construct a regular triangulation of the weighted
  points.
  
  \ccInclude{CGAL/Union_of_balls_3.h}

  %\ccInheritsFrom

  %\ccRefines

  %\ccHasModels

  \ccIsModel \ccc{SkinSurface_3} 
% NGHK: ADD
% (and through \ccc{SkinSurface_3} of \ccc{Surface_3})

  \ccTypes
  \ccThree{typedef Union_of_balls_3::Weighted_point<Point_3,FT>asdf}{x}{}
  \ccThreeToTwo

  \ccTypedef{typedef RegularTriangulationTraits_3 Geometric_traits;}{}
  \ccGlue
  \ccTypedef{typedef Geometric_traits::Weighted_point Weighted_point;}{}
  \ccGlue
  % The following two are not defined in the RegularTriangulationTraits
  \ccTypedef{typedef Weighted_point::Point Bare_point;}{} 
  \ccGlue
  \ccTypedef{typedef Weighted_point::Weight RT;}{}

  \ccNestedType{TMC_Traits}{The traits class used for the triangulated
    mixed complex}
  \ccGlue
  \ccTypedef{typedef Skin_surface_quadratic_surface_3<TMC_Traits>
    Quadratic_surface;}{The definition of a quadratic
    surface. This type should be a model of the concept
    \ccc{QuadraticSurface_3}.}

  \ccNestedType{Triangulated_mixed_complex;}{The type of a 3D
    triangulation that is used to store the power diagram (which is
    the triangulated mixed complex for a shrink factor equal to one).}
  \ccGlue
%
  \ccTypedef{typedef Triangulated_mixed_complex::Vertex_handle
    TMC_Vertex_handle;}{The type of a vertex handle of the
    \ccc{Triangulated_mixed_complex}.}
%
  \ccGlue
%
  \ccTypedef{typedef Triangulated_mixed_complex::Cell_handle
    TMC_Cell_handle;}{The type of a cell handle of the
    \ccc{Triangulated_mixed_complex}.}


  %\ccConstants

  \ccCreation 
  \ccCreationVariable{union} 

  \ccConstructor{Union_of_balls_3 ( WP_iterator begin, WP_iterator
    end, Geometric_traits gt = Geometric_traits(), bool verbose =
    false);}{The mandatory argument to the constructor is an iterator
    range \ccc{[begin,end)} of weighted points. Optionally, it is
    possible to specify whether weights of the input balls have to be
    increased and a \ccc{Geometric_traits} class to be used.}
  

  \ccThree{Triangulate_mixed_complex xx}{x}{}
  \ccThreeToTwo

  %\ccOperations
  \ccAccessFunctions

  \ccMethod{const Triangulated_mixed_complex
    &triangulated_mixed_complex() const;}{Returns the triangulated
    power diagram (which is the triangulated mixed complex for a
    shrink factor equal to one).  }

  \ccQueryFunctions
%
  \ccMethod{TMC_Cell_handle locate(const TMC_Point &p) const;}{Returns
    a handle to the cell of the triangulated mixed complex containing
    the point $p$.}


  \ccSeeAlso
  \ccRefIdfierPage{SkinSurface_3}

  %\ccImplementation

  %\ccExample

\end{ccRefClass}

% +------------------------------------------------------------------------+
%%RefPage: end of main body, begin of footer
% EOF
% +------------------------------------------------------------------------+

