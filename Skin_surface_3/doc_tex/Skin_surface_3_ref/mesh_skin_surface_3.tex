% +------------------------------------------------------------------------+
% | Reference manual page: mesh_skin_surface_3.tex
% +------------------------------------------------------------------------+
% | 27.09.2005   Nico Kruithof
% | Package: Skin_surface_3
% | 
\RCSdef{\RCSskinsurfaceRev}{$Id$}
\RCSdefDate{\RCSskinsurfaceDate}{$Date$}
% |
%%RefPage: end of header, begin of main body
% +------------------------------------------------------------------------+

\begin{ccRefFunction}{mesh_skin_surface_3<SkinSurface_3, Polyhedron_3>}
  \ccDefinition

  The function \ccRefName\ constructs a mesh isotopic to the skin
  surface based on the algorithm in \cite{cgal:kv-mssct-05}. It takes
  as input a \ccc{SkinSurface_3} object, which is a model of the
  \ccc{SkinSurface_3} concept and outputs the mesh in a
  \ccc{Polyhedron_3} object.

  \ccInclude{CGAL/mesh_skin_surface_3.h}

  \ccThree{void}{a}{}
  \ccThreeToTwo

  \ccFunction{void make_skin_surface_mesh_3<SkinSurface_3,Polyhedron_3>
    (const SkinSurface_3 &skin_surface, Polyhedron_3 &p);}{Constructs
    a mesh of the \ccc{skin_surface} in
    \ccc{p}.\ccPrecond{\ccc{SkinSurface_3} is a model of the concept
      \ccc{SkinSurface_3} and \ccc{Polyhedron_3::HDS} can be used as
      the template argument of the
      \ccc{Polyhedron_incremental_builder_3<HDS>}}.}

\end{ccRefFunction}

% +------------------------------------------------------------------------+
%%RefPage: end of main body, begin of footer
% EOF
% +------------------------------------------------------------------------+

