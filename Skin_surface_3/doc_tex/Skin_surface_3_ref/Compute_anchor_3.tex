% +------------------------------------------------------------------------+
% | Reference manual page: Compute_anchor_3.tex
% +------------------------------------------------------------------------+
% | 27.09.2005   Nico Kruithof
% | Package: Skin_surface
% | 
\RCSdef{\RCSComputeanchorRev}{$Revision$}
\RCSdefDate{\RCSComputeanchorDate}{$Date$}
% +------------------------------------------------------------------------+
\ccRefPageBegin
%%RefPage: end of header, begin of main body
% +------------------------------------------------------------------------+

\begin{ccRefClass}{Compute_anchor_3<RegularTriangulation_3>}

\ccDefinition

The affine hulls of a Regular simplex and its dual Voronoi cell
intersect in a single point, called the center. The point in a Regular
simplex (Voronoi cell) closest to the center is called the anchor
point of the simplex (cell). The anchor point coincides with the
center of the simplex (cell) or with the center of one of its faces.

The class \ccRefName\ for a given Regular simplex (Power cell) with
the center of which simplex it coincides.

\ccInclude{CGAL/Compute_anchor_3.h}

\ccTypes
\ccThree{typedef Triangulation_simplex_3<RegularTriangulation_3>xxx}{Simplex;xxx}{}
\ccThreeToTwo

\ccTypedef{typedef RegularTriangulation_3::Vertex_handle Vertex_handle;}{}
\ccGlue
\ccTypedef{typedef RegularTriangulation_3::Edge          Edge;}{}
\ccGlue
\ccTypedef{typedef RegularTriangulation_3::Facet         Facet;}{}
\ccGlue
\ccTypedef{typedef RegularTriangulation_3::Cell_handle   Cell_handle;}{}

\ccGlue
\ccTypedef{typedef Triangulation_simplex_3<RegularTriangulation_3> Simplex;}{}

\ccNestedType{Simplex_iterator}{iterator over simplices}
\ccGlue


\ccCreation
\ccCreationVariable{compute_anchor}  %% choose variable name

\ccThree{Compute_anchor_3}{xxxx}{}
\ccThreeToTwo


\ccConstructor{Compute_anchor_3(const RegularTriangulation_3 &reg);}{
  constructor that takes the regular triangulation from which it will
  construct anchor points as argument.}

\ccOperations

\ccHeading{Computing the anchor point}

The class \ccClassTemplateName provides functions for the computation
of anchor points. These functions return the face of the argument with
minimal dimension and an anchor point equal to the anchor point of the
argument, which is unique (under a generality condition). If no proper
face exists satisfying this condition, the argument is returned.

\ccMethod{Simplex anchor_del( const Vertex_handle v );}{ } \ccGlue
\ccMethod{Simplex anchor_del( const Edge &e );}{ } \ccGlue
\ccMethod{Simplex anchor_del( const Facet &f );}{ } \ccGlue
\ccMethod{Simplex anchor_del( const Cell_handle ch );}{ } \ccGlue
\ccMethod{Simplex anchor_del( const Simplex &s );}{ }


\ccMethod{Simplex anchor_vor( const Vertex_handle v );}{ }
\ccGlue
\ccMethod{Simplex anchor_vor( const Edge &e );}{ }
\ccGlue
\ccMethod{Simplex anchor_vor( const Facet &f );}{ }
\ccGlue
\ccMethod{Simplex anchor_vor( const Cell_handle ch );}{ }
\ccGlue
\ccMethod{Simplex anchor_vor( const Simplex &s );}{ }

\ccHeading{Degenerate situations} 
%
In degenerate cases, the anchor cannot be represented by a simple
simplex. E.g., if a Voronoi edge degenerates into a point, the anchor
of the Voronoi edge is equal to the anchor of the edge and both
vertices.

In the case of degeneracies, the method \ccc{is_degenerate()} returns
true and the functions \ccc{equivalent_anchors_begin()} and
\ccc{equivalent_anchors_end()} define a range of equivalent simplices.

\ccMethod{bool is_degenerate();}{ }
\ccGlue
\ccMethod{Simplex_iterator equivalent_anchors_begin();}{ }
\ccGlue
\ccMethod{Simplex_iterator equivalent_anchors_end();}{ }

%\ccSeeAlso
%\ccRefIdfierPage{CGAL::SkinSurfaceTraits_3<Traits>}

\end{ccRefClass}

% +------------------------------------------------------------------------+
%%RefPage: end of main body, begin of footer
\ccRefPageEnd
% EOF
% +------------------------------------------------------------------------+

