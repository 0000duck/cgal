% +------------------------------------------------------------------------+
% | Reference manual page: skin_surface_3.tex
% +------------------------------------------------------------------------+
% | 27.09.2005   Nico Kruithof
% | Package: Skin_surface_3
% | 
\RCSdef{\RCSskinsurfaceRev}{$Id: Skin_surface_3.tex 33413 2006-08-18 13:45:15Z nicokruithof $}
\RCSdefDate{\RCSskinsurfaceDate}{$Date: 2006-08-18 15:45:15 +0200 (Fri, 18 Aug 2006) $}
% |
%%RefPage: end of header, begin of main body
% +------------------------------------------------------------------------+

\begin{ccRefClass}{Skin_surface_traits_3<K>}
  NGHK: TODO

  \ccDefinition
  
  The first argument \ccc{K} must be a model of the \ccc{Kernel}
  concept.

  The \ccRefName\ is the main class in the
  \ccHtmlNoLinksFrom{\ccc{Skin_surface_3}}\ package. It is a model of
  the concept \ccc{SkinSurface_3}.  
% NGHK: ADD 
% The class also defines an interface to the surface mesher as it is a
% model of the concept \ccc{Surface_3}.

  The type \ccc{SkinSurfaceTraits_3} is required to be a model of the
  concept \ccc{SkinSurfaceTraits_3}, which means that it provides 
  predicates to construct a regular triangulation of the weighted
  points and for point location in the mixed complex.
  
  \ccInclude{CGAL/Skin_surface_3.h}

  %\ccInheritsFrom

  %\ccRefines

  %\ccHasModels

  \ccIsModel \ccc{SkinSurface_3} 
% NGHK: ADD
% (and through \ccc{SkinSurface_3} of \ccc{Surface_3})

  \ccTypes
  \ccThree{typedef Skin_surface_3::Weighted_point<Point_3,FT>asdf}{x}{}
  \ccThreeToTwo

  \ccTypedef{typedef SkinSurfaceTraits_3 Geometric_traits;}{}
  \ccGlue
  \ccTypedef{typedef Geometric_traits::Weighted_point Weighted_point;}{}
  \ccGlue
  % The following two are not defined in the SkinSurfaceTraits
  \ccTypedef{typedef Weighted_point::Point Bare_point;}{} 
  \ccGlue
  \ccTypedef{typedef Weighted_point::Weight RT;}{}

  \ccTypedef{typedef Skin_surface_quadratic_surface_3<TMC_Traits>
    Quadratic_surface;}{The definition of a quadratic
    surface. This type should be a model of the concept
    \ccc{QuadraticSurface_3}.}

  \ccNestedType{TMC_traits}{The traits class used for the triangulated
    mixed complex}
  \ccGlue
  \ccNestedType{Triangulated_mixed_complex;}{The type of a 3D
    triangulation that is used to store the triangulated mixed
    complex.}
  \ccGlue
%
  \ccTypedef{typedef Triangulated_mixed_complex::Vertex_handle
    TMC_Vertex_handle;}{The type of a vertex handle of the
    \ccc{Triangulated_mixed_complex}.}
%
  \ccGlue
%
  \ccTypedef{typedef Triangulated_mixed_complex::Cell_handle
    TMC_Cell_handle;}{The type of a cell handle of the
    \ccc{Triangulated_mixed_complex}.}


  %\ccConstants

  \ccCreation 
  \ccCreationVariable{skin} 

  \ccConstructor{Skin_surface_3 ( WP_iterator begin, WP_iterator end,
    RT shrink_factor, bool grow_balls = true, Geometric_traits gt =
    Geometric_traits(), bool verbose = false);}{The mandatory
    arguments to the constructor are an iterator range
    \ccc{[begin,end)} of weighted points and a shrink factor between 0
    and 1. Optionally, it is possible to specify whether weights of
    the input balls have to be increased and a \ccc{Geometric_traits}
    class to be used.}
  

  %\ccOperations
  \ccAccessFunctions

  \ccMethod{const Triangulated_mixed_complex
    &triangulated_mixed_complex() const;}{Returns the triangulated
    mixed complex.  }

  \ccQueryFunctions
%
  \ccMethod{TMC_Cell_handle locate(const TMC_Point &p) const;}{Returns
    a handle to the cell of the triangulated mixed complex containing the point $p$.}


  \ccSeeAlso
  \ccRefIdfierPage{SkinSurface_3}

  %\ccImplementation

  %\ccExample

\end{ccRefClass}

% +------------------------------------------------------------------------+
%%RefPage: end of main body, begin of footer
% EOF
% +------------------------------------------------------------------------+

