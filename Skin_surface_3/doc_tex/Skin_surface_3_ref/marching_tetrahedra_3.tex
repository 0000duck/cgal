% +------------------------------------------------------------------------+
% | Reference manual page: marching_tetrahedra_3.tex
% +------------------------------------------------------------------------+
% | 27.09.2005   Nico Kruithof
% | Package: Skin_surface_3
% | 
\RCSdef{\RCSmarchingtetrahedraRev}{$Id$}
\RCSdefDate{\RCSmarchingtetrahedraDate}{$Date$}
% |
%%RefPage: end of header, begin of main body
% +------------------------------------------------------------------------+


\begin{ccRefFunction}{marching_tetrahedra_3}

\ccDefinition

The function \ccRefName\ implements the marching tetrahedra algorithm
\cite{cgal:tpg-rmtiise-99}. To allow for maximal flexibility, it uses
the traits class \ccc{MarchingTetrahedraTraits_3} for labeling the
vertices as inside and outside and the construction of vertices on
edges of the triangulation. The observer class
\ccc{MarchingTetrahedraObserver_3} is used to add callbacks after the
insertion of an vertex or facet of the polyhedron.

The input is a triangulation, the traits class and an observer, which
is empty by default. The triangulation has to define the types
\ccc{Finite_cells_iterator} and \ccc{Vertex_handle}. The vertices are
accessible by the function \ccc{vertex(i)}, with $i=0,1,2,3$ and each
tetrahedron has a positive orientation. A Strict Weak Ordering and
equality has to be defined for \ccc{Vertex_handle}. These requirements
are satisfied by \ccc{Triangulation_3}.

The output is a \ccc{Polyhedron_3} and is constructed using the
\ccc{Polyhedral_incremental_builder_3}.

\ccInclude{CGAL/marching_tetrahedra_3.h}

\ccGlobalFunction{
template <class Triangulation_3,
	  class Polyhedron_3,
	  class MarchingTetrahedraTraits_3,
          class MarchingTetrahedraObserver_3 >
void marching_tetrahedra_3(
  const Triangulation_3 &triangulation,
  Polyhedron_3 &polyhedron,
  const MarchingTetrahedraTraits_3 &marching_traits,
  const MarchingTetrahedraObserver_3 &marching_observer = 
    CGAL::Marching_tetrahedra_observer_default_3());}

\ccSeeAlso
\ccRefIdfierPage{MarchingTetrahedraTraits_3}\\
\ccRefIdfierPage{MarchingTetrahedraObserver_3}\\
\ccRefIdfierPage{Polyhedron_3}\\
\ccRefIdfierPage{Triangulation_3}

\end{ccRefFunction}

% +------------------------------------------------------------------------+
%%RefPage: end of main body, begin of footer
% EOF
% +------------------------------------------------------------------------+

