% +------------------------------------------------------------------------+
% | Reference manual page: skin_surface_3.tex
% +------------------------------------------------------------------------+
% | 27.09.2005   Nico Kruithof
% | Package: Skin_surface_3
% | 
\RCSdef{\RCSskinsurfaceRev}{$Id$}
\RCSdefDate{\RCSskinsurfaceDate}{$Date$}
% |
%%RefPage: end of header, begin of main body
% +------------------------------------------------------------------------+

\begin{ccRefConcept}{SkinSurface_3}
  \ccDefinition

% NGHK: ADD INSTEAD OF NEXT LINE
%  The concept \ccRefName\ defines a skin surface and provides an
%  interface for both the dedicated skin surface mesher as well as the
%  general Delaunay based mesher, described in
%  Chapter~\ref{chapter_SurfaceMesher}. 
  The concept \ccRefName\ defines a skin surface and provides an
  interface for the dedicated skin surface mesher.
%
  The concept requires a constructor from an iterator range of
  weighted points and a shrink factor. By default the input balls are
  grown in such that the skin surface wraps around the input balls.

  %\ccRefines

  \ccThree{Triangulated_mixed_complex}{Skin_surface_3}{}
  \ccThreeToTwo

  \ccHasModels
  \ccc{CGAL::Skin_surface_3<SkinSurfaceTraits_3>}\\
  \ccc{CGAL::Union_of_balls_3<SkinSurfaceTraits_3>}

  \ccTypes

  \ccNestedType{Geometric_traits}{The geometric traits used for the
    construction of the regular triangulation.}  
  \ccGlue
  \ccNestedType{Weighted_point}{The \ccc{Weighted_point} type defined
    by the \ccc{Geometric_traits}.}
  \ccGlue
  \ccNestedType{Bare_point}{The \ccc{Bare_point} type defined by the
    \ccc{Geometric_traits}.}
  \ccGlue

  \ccNestedType{RT}{The \ccc{RT} type defined by the
    \ccc{Geometric_traits}.}

  %\ccConstants
  \ccCreation 
  \ccCreationVariable{skin} 

  \ccConstructor{Skin_surface_3 ( WP_iterator begin, WP_iterator end,
    RT shrink_factor);}{The mandatory arguments to the constructor are
    an iterator range \ccc{[begin,end)} of weighted points and a
    shrink factor between 0 and 1.}

  \ccOperations
%
  \ccMethod{template <class Polyhedron_3> void
    mesh_skin_surface_3 (Polyhedron_3 &p);}{Constructs a
    coarse mesh in \ccc{p}.\ccPrecond{\ccc{Polyhedron_3::HDS} can be
      used as the template argument of the
      \ccc{Polyhedron_incremental_builder_3<HDS>}}.  }

  \ccMethod{template <class Polyhedron_3> void
    subdivide_skin_surface_mesh_3 (Polyhedron_3 &p, int nSubdiv =
    1);}{Subdivides the skin surface using \ccc{nSubdiv} 1-4 split
    operations (each triangle is split into four sub-triangles) and
    the new vertices are moved towards the skin surface.}
 
  %\ccAccessFunctions

%  \ccQueryFunctions
%
%  \ccMethod{Simplex locate(const TMC_Point &p) const;}{Returns
%    a handle to the cell of the triangulated mixed complex containing the point $p$.}

%  \ccPredicates
%  \ccModifiers %	Modifiers 	insert, delete, update
%  \ccSeeAlso %	See Also 	other classes, functions
%  \ccImplementation %	Implementation 	running time, memory
%  \ccExample

\end{ccRefConcept}

% +------------------------------------------------------------------------+
%%RefPage: end of main body, begin of footer
% EOF
% +------------------------------------------------------------------------+

