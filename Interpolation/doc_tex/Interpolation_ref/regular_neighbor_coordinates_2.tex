% +------------------------------------------------------------------------+
% | Reference manual page: regular_neighbor_coordinates_2.tex
% +------------------------------------------------------------------------+
% | 
% | August 2003  Julia Floetotto
% | Package:   Interpolation
% | 
% |
% +------------------------------------------------------------------------+

%\renewcommand{\ccRefPageBegin}{\begin{ccAdvanced}}
%\renewcommand{\ccRefPageEnd}{\end{ccAdvanced}}
\begin{ccRefFunction}{regular_neighbor_coordinates_2}  %% add template arg's if necessary
\ccIndexSubitemBegin{Interpolation}{regular_neighbor_coordinates_2}

\ccDefinition
  
The function \ccRefName\ computes natural neighbor coordinates, also
called Sibson's coordinates, for weighted $2D$ points provided a
two-dimensional regular triangulation and a (weighted) query point
inside the convex hull of the vertices of the triangulation. We call these 
coordinates regular neighbor coordinates.

\ccInclude{CGAL/regular_neighbor_coordinates_2.h}


\ccFunction{template < class Rt, class OutputIterator > CGAL::Triple<
  OutputIterator, typename Rt::Geom_traits::FT, bool >
  regular_neighbor_coordinates_2(const Rt& rt, const typename
  Rt::Weighted_point& p, OutputIterator out, typename
  Rt::Face_handle start = typename Rt::Face_handle());} {
  computes the regular neighbor coordinates for \ccc{p} with respect
  to the weighted points in the two-dimensional regular triangulation
  \ccc{rt}.  The template class \ccc{Rt} should be of type
  \ccc{Regular_triangulation_2<Traits, Tds>}.  The value type of the
  \ccc{OutputIterator} is a pair of \ccc{Rt::Weighted_point} and the
  coordinate value of type \ccc{Rt::Geom_traits::FT}. The sequence of
  point/coordinate pairs that is computed by the function is placed
  starting at \ccc{out}. The function returns a triple with an
  iterator that is placed past-the-end of the resulting sequence of
  point/coordinate pairs, the normalization factor of the coordinates
  and a boolean value which is set to true iff the coordinate
  computation was successful, i.e. if \ccc{p} lies inside the
  convex hull of the points in \ccc{rt}. 
  }  

\ccFunction{template <class Rt, class OutputIterator,
  class EdgeIterator, class VertexIterator > CGAL::Triple<
  OutputIterator, typename Traits::FT, bool >
  regular_neighbor_coordinates_2(const Rt& rt, const typename
  Traits::Weighted_point& p, OutputIterator out, EdgeIterator
  hole_begin, EdgeIterator hole_end, VertexIterator
  hidden_vertices_begin, VertexIterator hidden_vertices_end);} 
{ The same as above.  \ccc{hole_begin} and
  \ccc{hole_end} determines the iterator range over the boundary edges
  of the conflict zone of \ccc{p} in the triangulation \ccc{rt}.
  \ccc{hidden_vertices_begin} and \ccc{hidden_vertices_end} determines
  the iterator range over the hidden vertices of the conflict zone of
  \ccc{p} in\ccc{rt}. It is the result of the function
  \ccc{T.get_boundary_of_conflicts(p,std::back_inserter(hole),
    std::back_inserter(hidden_vertices),
    start)}, see \ccc{Regular_triangulation_2<Traits, Tds>}.}


\ccFunction{template <class Rt, class OutputIterator>
  CGAL::Triple< OutputIterator, typename Rt::Geom_traits::FT, bool >
  regular_neighbor_coordinates_2(const Rt& rt, typename
  Rt::Vertex_handle vh, OutputIterator out);}{This function computes
  the regular neighbor coordinates of the point \ccc{vh->point()} with respect to
  the vertices of \ccc{rt} excluding \ccc{vh->point()}. The same as above for the
  remaining parameters.}  

\ccHeading{Requirements}
\begin{enumerate}
\item \ccc{Rt} are equivalent to the class
  \ccc{Regular_triangulation_2<Traits, Tds>}.
\item The traits class \ccc{Traits} of \ccc{Rt} is a model of the
  concept \ccc{RegularTriangulationTraits_2}. It provides the number
  type \ccc{FT} which is a model for \ccc{FieldNumberType} and it must
  meet the requirements for the traits class of the
  \ccc{polygon_area_2} function. A model of this traits class is
  \ccc{Regular_triangulation_euclidean_traits_2<K, Weight>}.
\item \ccc{OutputIterator::value_type} is equivalent to
  \ccc{std::pair<Rt::Weighted_point, Rt::Geom_traits::FT>}, i.e.\ a pair
  associating a point and its regular neighbor coordinate.
\end{enumerate}

\ccImplementation This function computes the areas stolen from the
Voronoi cells of points in \ccc{rt} by the insertion of \ccc{p}. The
total area of the Voronoi cell of \ccc{p} is also computed and
returned by the function. If \ccc{p} lies outside the convex hull, the
coordinate values cannot be computed and the third value of the result
triple is set to \ccc{false}.

\ccSeeAlso
\ccRefIdfierPage{CGAL::natural_neighbor_coordinates_2}
% The following does not exist (!) :
% \ccRefIdfierPage{CGAL::Regular_neighbor_coordinates_traits_2<R, Weight>}

\ccIndexSubitemEnd{Interpolation}{regular_neighbor_coordinates_2}
\end{ccRefFunction}
