% =============================================================================
% The CGAL Reference Manual
% Chapter: Geometric Object Generators
% Section: Random Numbers Generator
% -----------------------------------------------------------------------------
% file  : doc_tex/support/Generator/Random.tex
% author: Sven Sch�nherr <sven@inf.ethz.ch>
% -----------------------------------------------------------------------------
% $CGAL_Chapter: Geometric Object Generators $
% $CGAL_Package: Random_numbers WIP $
% $Id$
% $Date$
% =============================================================================

\begin{ccRefClass}{Random}
\label{sec:random_numbers_generator}

% -----------------------------------------------------------------------------
\ccDefinition

The class \ccRefName is a random numbers generator. It generates
uniformly distributed random \ccc{bool}s, \ccc{int}s and \ccc{double}s. 
It can be used as the random number generating function object in the 
\stl\ algorithm \ccc{random_shuffle}.

Instances of \ccClassName\ can be seen as input streams. Different
streams are \emph{independent} of each other, i.e.\ the sequence of
numbers from one stream does \emph{not} depend upon how many numbers
were extracted from the other streams. At each time, an instance has 
a \emph{state} that uniquely determines the subsequent numbers being
produced.

It can be very useful, e.g.\ for debugging, to reproduce a sequence of
random numbers. This can be done by either initialising with a fixed
seed, or by using the state functions as described below.

\ccInclude{CGAL/Random.h}

% -----------------------------------------------------------------------------
\ccTypes

\ccUnchecked
\ccNestedType{State}{The State type.}

% -----------------------------------------------------------------------------
\ccCreation
\ccCreationVariable{random}

\ccConstructor{ Random( );}{
        introduces a variable \ccVar\ of type \ccClassTemplateName. The
        seed is chosen ``randomly'', depending on the system time.}

\ccConstructor{ Random( unsigned int seed);}{
        introduces a variable \ccVar\ of type \ccClassTemplateName\
        and initializes its internal state using \ccc{seed}. Equal
        values for \ccc{seed} result in equal sequences of random
        numbers.}

% -----------------------------------------------------------------------------
\ccOperations

\ccMemberFunction{ bool get_bool( );}{
        returns a random \ccc{bool}.}

\ccMemberFunction{ template <int b> int get_bits();}{
        returns a random \ccc{int} value from the interval
        $[\mbox{\ccc{0},\ccc{2^b}})$.  This is supposed to
	be efficient.}

\ccMemberFunction{ int get_int( int lower, int upper);}{
        returns a random \ccc{int} from the interval
        $[\mbox{\ccc{lower},\ccc{upper}})$.}

\ccMemberFunction{ double get_double( double lower = 0.0,
                                      double upper = 1.0);}{
        returns a random \ccc{double} from the interval
        $[\mbox{\ccc{lower},\ccc{upper}})$.}

\ccMemberFunction{ int operator() ( int upper);}{
        returns \ccVar\ccc{.get_int( 0, upper)}.}

% -----------------------------------------------------------------------------
\ccHeading{Seed and State Functions}

\ccMemberFunction{ unsigned int get_seed() const;}{
        returns the seed used for initialization.}

\ccMemberFunction{ void save_state( State& state) const;}{
        saves the current internal state in \ccc{state}.}

\ccMemberFunction{ void restore_state( State const& state);}{
        restores the internal state from \ccc{state}.}

% -----------------------------------------------------------------------------
\ccHeading{Equality Test}

\ccMemberFunction{ bool  operator == ( Random const& random2) const;}{
        returns \ccc{true}, iff \ccVar\ and \ccc{random2} have equal
        internal states.}

% -----------------------------------------------------------------------------
\ccImplementation

We use the C library function \ccc{rand} to generate the random
numbers, \textit{i.e.}, the sequence of numbers depends on the implementation
of \ccc{rand} on your specific platform.



\ccSeeAlso

\ccRefIdfierPage{CGAL::default_random}\\

\end{ccRefClass}

% ===== EOF ===================================================================
