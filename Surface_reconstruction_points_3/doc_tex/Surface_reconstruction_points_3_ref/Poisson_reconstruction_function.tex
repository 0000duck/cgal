% +------------------------------------------------------------------------+
% | Reference manual page: Poisson_reconstruction_function.tex
% +------------------------------------------------------------------------+
% | 07.09.2007   Pierre Alliez, Laurent Saboret, Gael Guennebaud
% | Package: Surface_reconstruction_points_3
% |
\RCSdef{\RCSPoissonreconstructionfunctionRev}{$Id$}
\RCSdefDate{\RCSPoissonreconstructionfunctionDate}{$Date$}
% |
\ccRefPageBegin
%%RefPage: end of header, begin of main body
% +------------------------------------------------------------------------+


\begin{ccRefClass}{Poisson_reconstruction_function<GeomTraits, ReconstructionTriangulation_3>}

%% \ccHtmlCrossLink{}     %% add further rules for cross referencing links
%% \ccHtmlIndexC[class]{} %% add further index entries

\ccDefinition

% The section below is automatically generated. Do not edit!
%START-AUTO(\ccDefinition)

Given a set of 3D points with oriented normals sampled on the boundary of a 3D solid, the Poisson Surface Reconstruction method \cite{Kazhdan06} solves for an approximate indicator function of the inferred solid, whose gradient best matches the input normals. The output scalar function, represented in an adaptive octree, is then iso-contoured using an adaptive marching cubes.

\ccc{Poisson_reconstruction_function} implements a variant of this algorithm which solves for a piecewise linear function on a 3D Delaunay triangulation instead of an adaptive octree and uses the TAUCS sparse linear solver. 

% In order to get a unique solution, one vertex outside of the surface is constrained to a value of 0.0.

%END-AUTO(\ccDefinition)

% The section below is automatically generated. Do not edit!
%START-AUTO(\ccInclude)

\ccInclude{CGAL/Poisson_reconstruction_function.h}

%END-AUTO(\ccInclude)

\ccParameters

The class template declaration is:

% The section below is automatically generated. Do not edit!
%START-AUTO(\ccParameters)

template$<$  \\
class Gt,   \\
class \ccc{ReconstructionTriangulation_3} = \ccc{Reconstruction_triangulation_3<Gt>}$>$   \\
class \ccc{Poisson_reconstruction_function};

\ccCommentHeading{Parameters}
\begin{description}
\item \ccc{Gt}: Geometric traits class. \item \ccc{ReconstructionTriangulation_3}: 3D Delaunay triangulation class derived from \ccc{Reconstruction_triangulation_3}. \end{description}

%END-AUTO(\ccParameters)

\ccIsModel

% The section below is automatically generated. Do not edit!
%START-AUTO(\ccIsModel)

Model of the \ccc{ImplicitFunction} concept.

%END-AUTO(\ccIsModel)

\ccTypes

% The section below is automatically generated. Do not edit!
%START-AUTO(\ccTypes)

\ccNestedType{Triangulation}
{
}
\ccGlue
\ccNestedType{Geom_traits}
{
Kernel's geometric traits.
}
\ccGlue
\ccNestedType{FT}
{
}
\ccGlue
\ccNestedType{Point}
{
== \ccc{Point_3<Gt>}
}
\ccGlue
\ccNestedType{Vector}
{
== \ccc{Vector_3<Gt>}
}
\ccGlue
\ccNestedType{Sphere}
{
}
\ccGlue

%END-AUTO(\ccTypes)

\ccCreation
\ccCreationVariable{fct}  %% variable name for \ccMethod below

% The section below is automatically generated. Do not edit!
%START-AUTO(\ccCreation)

\ccConstructor{template<typename InputIterator, typename PointPMap, typename NormalPMap> Poisson_reconstruction_function(InputIterator first, InputIterator beyond, PointPMap point_pmap, NormalPMap normal_pmap);}
{
Creates a Poisson implicit function from the [first, beyond) range of vertices.
\ccCommentHeading{Template Parameters}
\begin{description}
\item \ccc{InputIterator}: iterator over input points. \item \ccc{PointPMap}: is a model of \ccc{boost::ReadablePropertyMap} with a \ccc{value_type} = \ccc{Point_3<Gt>}. It can be omitted if InputIterator \ccc{value_type} is convertible to \ccc{Point_3<Gt>}. \item \ccc{NormalPMap}: is a model of \ccc{boost::ReadablePropertyMap} with a \ccc{value_type} = \ccc{Vector_3<Gt>}. \end{description}
\ccCommentHeading{Parameters}
\begin{description}
\item \ccc{first}: iterator over the first input point. \item \ccc{beyond}: past-the-end iterator. \item \ccc{point_pmap}: property map InputIterator -$>$ \ccc{Point_3}. \item \ccc{normal_pmap}: property map InputIterator -$>$ \ccc{Vector_3}. \end{description}
}
\ccGlue

%END-AUTO(\ccCreation)

\ccOperations

% The section below is automatically generated. Do not edit!
%START-AUTO(\ccOperations)

\ccMethod{ReconstructionTriangulation_3& triangulation();}
{
Get embedded triangulation.
}
\ccGlue
\ccMethod{const ReconstructionTriangulation_3& triangulation() const;}
{
}
\ccGlue
\ccMethod{Sphere bounding_sphere() const;}
{
Returns a sphere bounding the inferred surface.
}
\ccGlue
\ccMethod{bool compute_implicit_function();}
{
The function \ccc{compute_implicit_function}() must be called after each insertion of oriented points. It computes the piecewise linear scalar function \ccc{f} by:\begin{itemize}
\item applying Delaunay refinement.\item solving for \ccc{f} at each vertex of the triangulation with a sparse linear solver.\item shifting and orienting \ccc{f} such that 'f = 0' at all input points and 'f$<$0' inside the inferred surface.\end{itemize}
Returns false if the linear solver fails.
}
\ccGlue
\ccMethod{FT operator()(const Point& p) const;}
{
Evaluates the implicit function at a given 3D query point.
}
\ccGlue
\ccMethod{FT operator()(const Point& p) const;}
{
\ccc{ImplicitFunction} interface: evaluate implicit function for any 3D point.
}
\ccGlue
\ccMethod{Point get_inner_point() const;}
{
Returns a point located inside the inferred surface.
}
\ccGlue

%END-AUTO(\ccOperations)

\ccSeeAlso

\ccRefIdfierPage{CGAL::APSS_reconstruction_function<GeomTraits>}  \\

\ccExample

See \ccc{poisson_reconstruction_example.cpp}.

\end{ccRefClass}

% +------------------------------------------------------------------------+
%%RefPage: end of main body, begin of footer
\ccRefPageEnd
% EOF
% +------------------------------------------------------------------------+

