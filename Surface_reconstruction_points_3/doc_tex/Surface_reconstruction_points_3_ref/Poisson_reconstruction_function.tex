% +------------------------------------------------------------------------+
% | Reference manual page: Poisson_reconstruction_function.tex
% +------------------------------------------------------------------------+
% | 07.09.2007   Pierre Alliez, Laurent Saboret, Gael Guennebaud
% | Package: Surface_reconstruction_points_3
% |
%\RCSdef{\RCSPoissonreconstructionfunctionRev}{$Id$}
%\RCSdefDate{\RCSPoissonreconstructionfunctionDate}{$Date$}
% |
\ccRefPageBegin
%%RefPage: end of header, begin of main body
% +------------------------------------------------------------------------+


\begin{ccRefClass}{Poisson_reconstruction_function<GeomTraits>}

%% \ccHtmlCrossLink{}     %% add further rules for cross referencing links
%% \ccHtmlIndexC[class]{} %% add further index entries

\ccDefinition

% The section below is automatically generated. Do not edit!
%START-AUTO(\ccDefinition)

Given a set of 3D points with oriented normals sampled on the boundary of a 3D solid, the Poisson Surface Reconstruction method \cite{Kazhdan06} solves for an approximate indicator function of the inferred solid, whose gradient best matches the input normals. The output scalar function, represented in an adaptive octree, is then iso-contoured using an adaptive marching cubes.

\ccc{Poisson_reconstruction_function} implements a variant of this algorithm which solves for a piecewise linear function on a 3D Delaunay triangulation instead of an adaptive octree.

%END-AUTO(\ccDefinition)

% The section below is automatically generated. Do not edit!
%START-AUTO(\ccInclude)

\ccInclude{CGAL/Poisson_reconstruction_function.h}

%END-AUTO(\ccInclude)

\ccParameters

% The section below is automatically generated. Do not edit!
%START-AUTO(\ccParameters)

template$<$class Gt$>$   \\
class \ccc{Poisson_reconstruction_function};

\ccCommentHeading{Parameters}  \\
\ccc{Gt}: Geometric traits class.

%END-AUTO(\ccParameters)

\ccIsModel

% The section below is automatically generated. Do not edit!
%START-AUTO(\ccIsModel)

Model of the \ccc{ImplicitFunction} concept.

%END-AUTO(\ccIsModel)

\ccTypes

% Enlarge left margin
\ccTwo{123456789012345678901234567890123456789012345678}{}

% The section below is automatically generated. Do not edit!
%START-AUTO(\ccTypes)

\ccNestedType{Geom_traits}
{
Geometric traits class.
}
\ccGlue
\ccNestedType{FT}
{
typedef to \ccc{Geom_traits::FT}
}
\ccGlue
\ccNestedType{Point}
{
typedef to \ccc{Geom_traits::Point_3}
}
\ccGlue
\ccNestedType{Vector}
{
typedef to \ccc{Geom_traits::Vector_3}
}
\ccGlue
\ccNestedType{Sphere}
{
typedef to \ccc{Geom_traits::Sphere_3}
}
\ccGlue

%END-AUTO(\ccTypes)

\ccCreation
\ccCreationVariable{fct}  %% variable name for \ccMethod below

% Reduce left margin
\ccTwo{1234567890123456789012}{}

% The section below is automatically generated. Do not edit!
%START-AUTO(\ccCreation)

\ccConstructor{template<typename InputIterator, typename PointPMap, typename NormalPMap> Poisson_reconstruction_function(InputIterator first, InputIterator beyond, PointPMap point_pmap, NormalPMap normal_pmap);}
{
Creates a Poisson implicit function from the [first, beyond) range of points.
\ccCommentHeading{Template Parameters}  \\
\ccc{InputIterator}: iterator over input points. \ccc{PointPMap}: is a model of \ccc{boost::ReadablePropertyMap} with a \ccc{value_type} = \ccc{Point_3}. It can be omitted if \ccc{InputIterator} \ccc{value_type} is convertible to \ccc{Point_3}. \ccc{NormalPMap}: is a model of \ccc{boost::ReadablePropertyMap} with a \ccc{value_type} = \ccc{Vector_3}.
\ccCommentHeading{Parameters}  \\
\ccc{first}: iterator over the first input point. \ccc{beyond}: past-the-end iterator over the input points. \ccc{point_pmap}: property map to access the position of an input point. \ccc{normal_pmap}: property map to access the {\bf oriented} normal of an input point.
}
\ccGlue

%END-AUTO(\ccCreation)

\ccOperations

% Workaround bug in cgal_manual which, by default,
% removes "const" from "Reconstruction" in return values.
\ccTagFullDeclarations

% Enlarge left margin
\ccThree{1234567890}{12345678901234567890123}{}

% The section below is automatically generated. Do not edit!
%START-AUTO(\ccOperations)

\ccMethod{Sphere bounding_sphere() const;}
{
Returns a sphere bounding the inferred surface.
}
\ccGlue
\ccMethod{template<class SparseLinearAlgebraTraits_d> bool compute_implicit_function(SparseLinearAlgebraTraits_d solver = SparseLinearAlgebraTraits_d());}
{
The function \ccc{compute_implicit_function}() must be called after the insertion of oriented points. It computes the piecewise linear scalar function operator() by: applying Delaunay refinement, solving for operator() at each vertex of the triangulation with a sparse linear solver, and shifting and orienting operator() such that it is 0 at all input points and negative inside the inferred surface.
\ccCommentHeading{Template parameters}  \\
\ccc{SparseLinearAlgebraTraits_d}: Symmetric definite positive sparse linear solver. The default solver is TAUCS Multifrontal Supernodal Cholesky Factorization.
\ccCommentHeading{Returns} false if the linear solver fails.
\ccCommentHeading{Parameters}  \\
\ccc{solver}: sparse linear solver.
}
\ccGlue
\ccMethod{FT operator()(const Point& p) const;}
{
\ccc{ImplicitFunction} interface: evaluates the implicit function at a
given 3D query point. The function \ccc{compute_implicit_function} must be
called before the first call to \ccc{operator()}.
}
\ccGlue
\ccMethod{Point get_inner_point() const;}
{
Returns a point located inside the inferred surface.
}
\ccGlue

%END-AUTO(\ccOperations)

\ccExample

See \ccReferToExampleCode{Surface_reconstruction_points_3/poisson_reconstruction_example.cpp}.

\end{ccRefClass}

% +------------------------------------------------------------------------+
%%RefPage: end of main body, begin of footer
\ccRefPageEnd
% EOF
% +------------------------------------------------------------------------+
