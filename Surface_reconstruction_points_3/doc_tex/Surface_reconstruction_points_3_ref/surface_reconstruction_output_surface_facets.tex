% +------------------------------------------------------------------------+
% | Reference manual page: surface_reconstruction_output_surface_facets.tex
% +------------------------------------------------------------------------+
% | 07.01.2009   Pierre Alliez, Laurent Saboret, Gael Guennebaud
% | Package: Surface_reconstruction_points_3
% |
\RCSdef{\RCSsurfacereconstructionoutputsurfacefacetsRev}{$Id$}
\RCSdefDate{\RCSsurfacereconstructionoutputsurfacefacetsDate}{$Date$}
% |
\ccRefPageBegin
%%RefPage: end of header, begin of main body
% +------------------------------------------------------------------------+


\begin{ccRefFunction}{surface_reconstruction_output_surface_facets}  %% add template arg's if necessary

%% \ccHtmlCrossLink{}     %% add further rules for cross referencing links
%% \ccHtmlIndexC[function]{} %% add further index entries

\ccDefinition

\ccc{surface_reconstruction_output_surface_facets()} converts a surface reconstructed by \ccc{make_surface_mesh()} to a triangles soup.

\ccInclude{CGAL/surface_reconstruction_output_surface_facets.h}

% The section below is automatically generated. Do not edit!
%START-AUTO(\ccDefinition)

\ccFunction{template<class SurfaceMeshComplex_2InTriangulation_3, typename OutputIterator> void surface_reconstruction_output_surface_facets(const SurfaceMeshComplex_2InTriangulation_3& c2t3, OutputIterator output_iterator);}
{
Get reconstructed surface out of a \ccc{SurfaceMeshComplex_2InTriangulation_3} object.
This variant exports the surface as a triangle soup.
\ccCommentHeading{Template Parameters}
\begin{description}
\item \ccc{SurfaceMeshComplex_2InTriangulation_3}: model of the \ccc{SurfaceMeshComplex_2InTriangulation_3} concept. \item \ccc{OutputIterator}: \ccc{value_type} must be convertible from \ccc{Triangle_3<Kernel>}.\end{description}
\ccCommentHeading{Returns} true on success.
\ccCommentHeading{Parameters}
\begin{description}
\item \ccc{c2t3}: Input surface. \item \ccc{output_iterator}: Output iterator. \end{description}
}
\ccGlue
\ccFunction{template<class SurfaceMeshComplex_2InTriangulation_3, class PolyhedronTraits_3, class PolyhedronItems_3, template<class T, class I, class A> class HalfedgeDS, class Alloc> void surface_reconstruction_output_surface_facets(const SurfaceMeshComplex_2InTriangulation_3& c2t3, Polyhedron_3<PolyhedronTraits_3, PolyhedronItems_3, HalfedgeDS, Alloc>& output_polyhedron);}
{
Get reconstructed surface out of a \ccc{SurfaceMeshComplex_2InTriangulation_3} object.
This variant exports the surface as a polyhedron. It requires the surface to be manifold. For this purpose, you may call \ccc{make_surface_mesh}() with \ccc{Manifold_tag} or \ccc{Manifold_with_boundary_tag} parameter.
\ccCommentHeading{Template Parameters}
\begin{description}
\item \ccc{SurfaceMeshComplex_2InTriangulation_3}: model of the \ccc{SurfaceMeshComplex_2InTriangulation_3} concept. \item \ccc{PolyhedronTraits_3, PolyhedronItems_3, HalfedgeDS, Alloc}: see \ccc{Polyhedron_3} declaration.\end{description}
\ccCommentHeading{Returns} true on success.
\ccCommentHeading{Parameters}
\begin{description}
\item \ccc{c2t3}: Input surface. \item \ccc{output_polyhedron}: Output polyhedron. \end{description}
}
\ccGlue

%END-AUTO(\ccDefinition)

\ccSeeAlso

\ccRefIdfierPage{CGAL::output_surface_facets_to_off}  \\

\ccExample

\begin{ccExampleCode}
typedef CGAL::Exact_predicates_inexact_constructions_kernel Kernel;
typedef CGAL::Surface_mesh_default_triangulation_3 STr;
typedef CGAL::Surface_mesh_complex_2_in_triangulation_3<STr> C2t3;
typedef CGAL::Implicit_surface_3<Kernel, Poisson_reconstruction_function> Surface_3;
typedef Kernel::Triangle_3 Triangle;

Surface_3 surface =...; // Implicit function to contour

// Contour surface
STr tr; // 3D-Delaunay triangulation
C2t3 c2t3 (tr); // 2D-complex in 3D-Delaunay triangulation
CGAL::make_surface_mesh(c2t3, surface, ...);

// Convert reconstructed surface to a triangles soup
std::deque<Triangle> triangles;
CGAL::surface_reconstruction_output_surface_facets(c2t3, std::back_inserter(triangles));
\end{ccExampleCode}

\end{ccRefFunction}

% +------------------------------------------------------------------------+
%%RefPage: end of main body, begin of footer
\ccRefPageEnd
% EOF
% +------------------------------------------------------------------------+

