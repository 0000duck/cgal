% +------------------------------------------------------------------------+
% | Reference manual page: ReconstructionVertexBase_3.tex
% +------------------------------------------------------------------------+
% | 07.09.2007   Pierre Alliez, Laurent Saboret, Gael Guennebaud
% | Package: Surface_reconstruction_points_3
% |
\RCSdef{\RCSReconstructionVertexBaseRev}{$Id$}
\RCSdefDate{\RCSReconstructionVertexBaseDate}{$Date$}
% |
\ccRefPageBegin
%%RefPage: end of header, begin of main body
% +------------------------------------------------------------------------+


\begin{ccRefConcept}{ReconstructionVertexBase_3}

%% \ccHtmlCrossLink{}     %% add further rules for cross referencing links
%% \ccHtmlIndexC[concept]{} %% add further index entries

\ccDefinition

% The section below is automatically generated. Do not edit!
%START-AUTO(\ccDefinition)

The \ccc{ReconstructionVertexBase_3} concept defines the interface of the vertex class of the \ccc{ReconstructionTriangulation_3} concept. It provides the interface requested by the \ccc{Poisson_reconstruction_function} class:\begin{itemize}
\item Each vertex stores a normal vector.\item A vertex is either an input point or a Steiner point added by Delaunay refinement.\item In order to solve a linear system over the triangulation, a vertex may be constrained or not (i.e. contributes to the right or left member of the linear system), and has a unique index.\end{itemize}



\ccPrecond The geometric traits class 's \ccc{Point_3} type must be a model of \ccc{PointWithNormal_3}.

%END-AUTO(\ccDefinition)

\ccRefines

% The section below is automatically generated. Do not edit!
%START-AUTO(\ccRefines)

\ccc{DelaunayTriangulationVertexBase_3}, and DefaultConstructible

%END-AUTO(\ccRefines)

\ccTypes

% The section below is automatically generated. Do not edit!
%START-AUTO(\ccTypes)

\ccNestedType{FT}
{
}
\ccGlue
\ccNestedType{Point}
{
Model of \ccc{PointWithNormal_3}.
}
\ccGlue
\ccNestedType{Point_with_normal}
{
Model of \ccc{PointWithNormal_3}.
}
\ccGlue
\ccNestedType{Normal}
{
Model of \ccc{Kernel::Vector_3} concept.
}
\ccGlue

%END-AUTO(\ccTypes)

\ccCreation
\ccCreationVariable{v}  %% variable name for \ccMethod below

% The section below is automatically generated. Do not edit!
%START-AUTO(\ccCreation)

\ccConstructor{ReconstructionVertexBase_3();}
{
}
\ccGlue
\ccConstructor{ReconstructionVertexBase_3(const Point& p);}
{
}
\ccGlue
\ccConstructor{ReconstructionVertexBase_3(const Point& p, Cell_handle c);}
{
}
\ccGlue
\ccConstructor{ReconstructionVertexBase_3(Cell_handle c);}
{
}
\ccGlue

%END-AUTO(\ccCreation)

\ccOperations

% The section below is automatically generated. Do not edit!
%START-AUTO(\ccOperations)

\ccMethod{bool constrained() const;}
{
Is vertex constrained, i.e. does it contribute to the right or left member of the linear system? Default value is false.
}
\ccGlue
\ccMethod{bool& constrained();}
{
}
\ccGlue
\ccMethod{FT f() const;}
{
Get/set the value of the implicit function. Default value is 0.0.
}
\ccGlue
\ccMethod{FT& f();}
{
}
\ccGlue
\ccMethod{double average_spacing() const;}
{
Get/set average spacing at each input point.
}
\ccGlue
\ccMethod{double& average_spacing();}
{
}
\ccGlue
\ccMethod{unsigned char type() const;}
{
Get/set the type = INPUT or STEINER.
}
\ccGlue
\ccMethod{unsigned char& type();}
{
}
\ccGlue
\ccMethod{unsigned int index() const;}
{
Get/set the index in matrix.
}
\ccGlue
\ccMethod{unsigned int& index();}
{
}
\ccGlue
\ccMethod{const Normal& normal() const;}
{
Get/set normal (vector + orientation). Default value is null vector.
}
\ccGlue
\ccMethod{Normal& normal();}
{
}
\ccGlue
\ccMethod{int tag() const;}
{
General purpose tag.
}
\ccGlue
\ccMethod{int& tag();}
{
}
\ccGlue

%END-AUTO(\ccOperations)

\ccHasModels

% The section below is automatically generated. Do not edit!
%START-AUTO(\ccHasModels)

\ccc{Reconstruction_vertex_base_3<GeomTraits, TriangulationVertexBase_3>}

%END-AUTO(\ccHasModels)

\end{ccRefConcept}

% +------------------------------------------------------------------------+
%%RefPage: end of main body, begin of footer
\ccRefPageEnd
% EOF
% +------------------------------------------------------------------------+

