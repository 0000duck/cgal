% +------------------------------------------------------------------------+
% | Reference manual page: erase_small_polyhedron_connected_components.tex
% +------------------------------------------------------------------------+
% | 16.04.2009   Pierre Alliez, Laurent Saboret, Gael Guennebaud
% | Package: Surface_reconstruction_points_3
% |
\RCSdef{\RCSerasesmallpolyhedronconnectedcomponentsRev}{$Id$}
\RCSdefDate{\RCSerasesmallpolyhedronconnectedcomponentsDate}{$Date$}
% |
\ccRefPageBegin
%%RefPage: end of header, begin of main body
% +------------------------------------------------------------------------+


\begin{ccRefFunction}{erase_small_polyhedron_connected_components<Polyhedron>}

%% \ccHtmlCrossLink{}     %% add further rules for cross referencing links
%% \ccHtmlIndexC[function]{} %% add further index entries

\ccDefinition

\ccc{erase_small_polyhedron_connected_components<Polyhedron>} erases all connected components of a polyhedron but the largest.

% The section below is automatically generated. Do not edit!
%START-AUTO(\ccDefinition)

\ccFunction{template<class Polyhedron> unsigned int erase_small_polyhedron_connected_components(Polyhedron& polyhedron);}
{
Erase small connected components of a polyhedron: erase all connected components but the largest.
\ccCommentHeading{Template Parameters}
\ccc{Polyhedron}: an instance of \ccc{Polyhedron_3<>} that supports vertices and removal operation.
\ccCommentHeading{Returns} the number of connected components erased.
}
\ccGlue

%END-AUTO(\ccDefinition)

\ccInclude{erase_small_polyhedron_connected_components.h}

\ccSeeAlso

\ccRefIdfierPage{CGAL::Polyhedron_3<>}  \\
\ccRefIdfierPage{CGAL::get_polyhedron_connected_components<Polyhedron>}  \\

\ccExample

See \ccc{APSS_reconstruction.cpp} example.

\end{ccRefFunction}

% +------------------------------------------------------------------------+
%%RefPage: end of main body, begin of footer
\ccRefPageEnd
% EOF
% +------------------------------------------------------------------------+

