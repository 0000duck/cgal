% +------------------------------------------------------------------------+
% | Reference manual page: jet_normal_estimation.tex
% +------------------------------------------------------------------------+
% | 19.02.2008   Pierre Alliez, Laurent Saboret, Gael Guennebaud
% | Package: Surface_reconstruction_points_3
% |
\RCSdef{\RCSjetnormalestimationRev}{$Id$}
\RCSdefDate{\RCSjetnormalestimationDate}{$Date$}
% |
\ccRefPageBegin
%%RefPage: end of header, begin of main body
% +------------------------------------------------------------------------+


\begin{ccRefFunction}{jet_normal_estimation}  %% add template arg's if necessary

%% \ccHtmlCrossLink{}     %% add further rules for cross referencing links
%% \ccHtmlIndexC[function]{} %% add further index entries

\ccDefinition

\ccc{CGAL::jet_normal_estimation()} estimates normal directions of all points from the input set by fitting jet surfaces over the $k$ nearest neighbors. The default jet surface is a quadric, and the result is an unoriented normal vector for each input point. The complete version of this function requires the kernel, while another deduces it from the input data.

\ccInclude{CGAL/jet_normal_estimation.h}

% The section below is automatically generated. Do not edit!
%START-AUTO(\ccDefinition)

\ccFunction{template<typename InputIterator, typename OutputIterator, typename Kernel> OutputIterator jet_normal_estimation(InputIterator first, InputIterator beyond, OutputIterator normals, unsigned int k, const Kernel& , unsigned int degre_fitting);}
{
Estimate normal directions using jet fitting on the k nearest neighbors. This variant requires the kernel.
\ccPrecond k $>$= 2.
\ccCommentHeading{Template Parameters}
\begin{description}
\item \ccc{InputIterator}: \ccc{value_type} is \ccc{Point_3}. \item \ccc{OutputIterator}: \ccc{value_type} is \ccc{Point_3}. \item \ccc{Kernel}: Geometric traits class.\end{description}
\ccCommentHeading{Returns} past-the-end output iterator.
\ccCommentHeading{Parameters}
\begin{description}
\item \ccc{first}: iterator over the first input point \item \ccc{beyond}: past-the-end iterator over input points \item \ccc{normals}: output normals \item \ccc{k}: number of neighbors \end{description}
}
\ccGlue
\ccFunction{template<typename InputIterator, typename OutputIterator> OutputIterator jet_normal_estimation(InputIterator first, InputIterator beyond, OutputIterator normals, unsigned int k, unsigned int degre_fitting = 2);}
{
Estimate normal directions using jet fitting on the k nearest neighbors. This variant deduces the kernel from iterator types.
\ccPrecond k $>$= 2.
\ccCommentHeading{Template Parameters}
\begin{description}
\item \ccc{InputIterator}: \ccc{value_type} is \ccc{Point_3}. \item \ccc{OutputIterator}: \ccc{value_type} is \ccc{Point_3}.\end{description}
\ccCommentHeading{Returns} past-the-end output iterator.
\ccCommentHeading{Parameters}
\begin{description}
\item \ccc{first}: iterator over the first input point \item \ccc{beyond}: past-the-end iterator over input points \item \ccc{normals}: output normals \item \ccc{k}: number of neighbors \end{description}
}
\ccGlue

%END-AUTO(\ccDefinition)

\ccSeeAlso

\ccRefIdfierPage{CGAL::pca_normal_estimation}  \\
\ccRefIdfierPage{CGAL::mst_normal_orientation}  \\

\ccExample

See \ccc{normal_estimation.cpp} example.

\end{ccRefFunction}

% +------------------------------------------------------------------------+
%%RefPage: end of main body, begin of footer
\ccRefPageEnd
% EOF
% +------------------------------------------------------------------------+

