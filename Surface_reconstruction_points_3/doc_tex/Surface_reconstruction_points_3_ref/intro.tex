% +------------------------------------------------------------------------+
% | Reference manual page: Surface_reconstruction_points_3/intro.tex
% +------------------------------------------------------------------------+
% | 06.01.2009   Pierre Alliez, Laurent Saboret, Gael Guennebaud, Mariette Yvinec
% | Package: Surface_reconstruction_points_3
% |
% |
% +------------------------------------------------------------------------+

\ccRefChapter{Surface Reconstruction from Point Sets
\label{chap:surface_reconstruction_points_3_ref}}

\ccChapterAuthor{Pierre Alliez, Laurent Saboret, Gael Guennebaud, Mariette Yvinec}

This \cgal\ component implements two state-of-the-art surface reconstruction methods. The input is an unorganized point set, possibly with attributes such as unoriented or oriented normals. The output is either an implicit function (ready for evaluation by any contouring algorithm), or a surface mesh generated by the \cgal\ surface mesh generator.

Note that since reconstruction methods often require pre-processing a point set (through reduction, smoothing, outlier removal, normal estimation, normal orientation), we provide components devoted to these tasks in the point set processing component.

\section{Classified Reference Pages}

\subsection{Classes}

\ccRefIdfierPage{CGAL::Poisson_reconstruction_function<GeomTraits, ReconstructionTriangulation_3>}  \\
\ccRefIdfierPage{CGAL::APSS_reconstruction_function<GeomTraits>}  \\


\subsection{Functions}

\ccRefIdfierPage{CGAL::output_surface_facets_to_polyhedron}  \\
\ccRefIdfierPage{CGAL::keep_largest_connected_components<Polyhedron>}  \\

