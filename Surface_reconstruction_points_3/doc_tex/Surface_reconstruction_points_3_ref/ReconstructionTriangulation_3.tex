% +------------------------------------------------------------------------+
% | Reference manual page: ReconstructionTriangulation_3.tex
% +------------------------------------------------------------------------+
% | 22.02.2008   Pierre Alliez, Laurent Saboret, Gael Guennebaud
% | Package: Surface_reconstruction_points_3
% |
\RCSdef{\RCSReconstructionTriangulationRev}{$Id$}
\RCSdefDate{\RCSReconstructionTriangulationDate}{$Date$}
% |
\ccRefPageBegin
%%RefPage: end of header, begin of main body
% +------------------------------------------------------------------------+


\begin{ccRefConcept}{ReconstructionTriangulation_3}

%% \ccHtmlCrossLink{}     %% add further rules for cross referencing links
%% \ccHtmlIndexC[concept]{} %% add further index entries

\ccDefinition

% The section below is automatically generated. Do not edit!
%START-AUTO(\ccDefinition)

The \ccc{ReconstructionTriangulation_3} concept defines the interface of a 3D Delaunay triangulation requested by the \ccc{Poisson_reconstruction_function} class. The vertex class must be a model of \ccc{ReconstructionVertexBase_3}.

It provides the interface requested by the \ccc{Poisson_reconstruction_function} class:\begin{itemize}
\item Each vertex stores a normal vector.\item A vertex is either an input point or a Steiner point added by Delaunay refinement.\item In order to solve a linear system over the triangulation, a vertex may be constrained or not (i.e. may contribute to the right or left member of the linear system), and has a unique index.\end{itemize}

%END-AUTO(\ccDefinition)

\ccRefines

% The section below is automatically generated. Do not edit!
%START-AUTO(\ccRefines)

\ccc{DelaunayTriangulation_3}, and DefaultConstructible

%END-AUTO(\ccRefines)

\ccTypes

% The section below is automatically generated. Do not edit!
%START-AUTO(\ccTypes)

\ccNestedType{Geom_traits}
{
Geometric traits class / \ccc{Point_3} is a model of \ccc{PointWithNormal_3}.
}
\ccGlue
\ccNestedType{FT}
{
}
\ccGlue
\ccNestedType{Vector}
{
}
\ccGlue
\ccNestedType{Sphere}
{
}
\ccGlue
\ccNestedType{Point}
{
Model of \ccc{PointWithNormal_3}.
}
\ccGlue
\ccNestedType{Point_with_normal}
{
Model of \ccc{PointWithNormal_3}.
}
\ccGlue
\ccNestedType{Normal}
{
Model of \ccc{Kernel::Vector_3} concept.
}
\ccGlue
\ccNestedType{Normal_iterator}
{
Iterator over all normals.
}
\ccGlue
\ccNestedType{Input_vertices_iterator}
{
Iterator over input vertices.
}
\ccGlue
\ccNestedType{Input_point_iterator}
{
Iterator over input points.
}
\ccGlue

%END-AUTO(\ccTypes)

\ccConstants

% The section below is automatically generated. Do not edit!
%START-AUTO(\ccConstants)

\ccEnum{enum Point_type { INPUT, STEINER };}
{
Point type.
}
\ccGlue
\ccCommentHeading{Values}
\begin{description}
\item \ccc{INPUT}: Input point. \item \ccc{STEINER}: Steiner point created by Delaunay refinement. \end{description}
\ccGlue

%END-AUTO(\ccConstants)

%\ccCreation
\ccCreationVariable{tr}  %% variable name for \ccMethod calls below

\ccOperations

% The section below is automatically generated. Do not edit!
%START-AUTO(\ccOperations)

\ccMethod{Normal_iterator normals_begin();}
{
Get first iterator over finite vertices normals.
}
\ccGlue
\ccMethod{Normal_iterator normals_end();}
{
Get past-the-end iterator over finite vertices normals.
}
\ccGlue
\ccMethod{Input_vertices_iterator input_vertices_begin() const;}
{
Get first iterator over input vertices.
}
\ccGlue
\ccMethod{Input_vertices_iterator input_vertices_end() const;}
{
Get past-the-end iterator over input vertices.
}
\ccGlue
\ccMethod{Input_point_iterator input_points_begin() const;}
{
Get iterator over the first input point.
}
\ccGlue
\ccMethod{Input_point_iterator input_points_end() const;}
{
Get past-the-end iterator over input points.
}
\ccGlue
\ccMethod{Sphere bounding_sphere() const;}
{
Get the bounding sphere of all points.
}
\ccGlue
\ccMethod{Sphere input_points_bounding_sphere() const;}
{
Get the bounding sphere of input points.
}
\ccGlue
\ccMethod{Vertex_handle insert(const Point& p, Point_type type = INPUT, Cell_handle start = Cell_handle());}
{
Insert point (model of \ccc{PointWithNormal_3}) in the triangulation. Default type is INPUT.
}
\ccGlue
\ccMethod{template<class InputIterator> int insert(InputIterator first, InputIterator beyond, Point_type type = INPUT);}
{
Insert points in the triangulation using a spatial sort. Default type is INPUT.
\ccPrecond InputIterator \ccc{value_type} must be convertible to \ccc{Point_with_normal}.
\ccCommentHeading{Parameters}
\begin{description}
\item \ccc{first}: Iterator over first point to add. \item \ccc{beyond}: Past-the-end iterator to add. \end{description}
\ccCommentHeading{Returns} the number of inserted points.
}
\ccGlue
\ccMethod{template<class CellIt> Vertex_handle insert_in_hole(const Point& p, CellIt cell_begin, CellIt cell_end, Cell_handle begin, int i, Point_type type = STEINER);}
{
Delaunay refinement callback: insert STEINER point in the triangulation.
}
\ccGlue
\ccMethod{unsigned int index_unconstrained_vertices();}
{
Index unconstrained vertices following the order of \ccc{Finite_vertices_iterator}.
\ccCommentHeading{Returns} the number of unconstrained vertices.
}
\ccGlue

%END-AUTO(\ccOperations)

\ccHasModels

% The section below is automatically generated. Do not edit!
%START-AUTO(\ccHasModels)

\ccc{Reconstruction_triangulation_3<GeomTraits, TriangulationDataStructure_3>}

%END-AUTO(\ccHasModels)


\end{ccRefConcept}

% +------------------------------------------------------------------------+
%%RefPage: end of main body, begin of footer
\ccRefPageEnd
% EOF
% +------------------------------------------------------------------------+

