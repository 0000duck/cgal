% +------------------------------------------------------------------------+
% | Reference manual page: surface_reconstruction_write_off_point_cloud.tex
% +------------------------------------------------------------------------+
% | 07.01.2009   Pierre Alliez, Laurent Saboret, Gael Guennebaud
% | Package: Surface_reconstruction_points_3
% |
\RCSdef{\RCSsurfacereconstructionwriteoffpointcloudRev}{$Id$}
\RCSdefDate{\RCSsurfacereconstructionwriteoffpointcloudDate}{$Date$}
% |
\ccRefPageBegin
%%RefPage: end of header, begin of main body
% +------------------------------------------------------------------------+


\begin{ccRefFunction}{surface_reconstruction_write_off_point_cloud}  %% add template arg's if necessary

%% \ccHtmlCrossLink{}     %% add further rules for cross referencing links
%% \ccHtmlIndexC[function]{} %% add further index entries

\ccDefinition

\ccc{CGAL::surface_reconstruction_write_off_point_cloud()} saves points (positions + optionally normals) to a .off file (ASCII).

The \ccc{CGAL::surface_reconstruction_write_off_point_cloud()} function exists in two flavors:
the input can be either a container of \ccc{Point_3} or of \ccc{PointWithNormal_3} points.

\ccInclude{CGAL/surface_reconstruction_write_off_point_cloud.h}

% The section below is automatically generated. Do not edit!
%START-AUTO(\ccDefinition)

\ccFunction{bool surface_reconstruction_write_off_point_cloud(const char * pFilename, InputIterator first, InputIterator beyond);}
{
Save points (positions + normals) to a .off file (ASCII).
\ccCommentHeading{Template Parameters}
\ccc{InputIterator}: \ccc{value_type} must be a model of the \ccc{PointWithNormal_3} concept.
\ccCommentHeading{Returns} true on success.
\ccCommentHeading{Parameters}
\begin{description}
\item \ccc{first}: first input point \item \ccc{beyond}: past-the-end input point \end{description}
}
\ccGlue
\ccFunction{bool surface_reconstruction_write_off_point_cloud(const char * pFilename, InputIterator first, InputIterator beyond, bool write_normals);}
{
Save points (positions + optionally normals) to a .off file (ASCII).
\ccCommentHeading{Template Parameters}
\ccc{InputIterator}: \ccc{value_type} must be a model of \ccc{PointWithNormal_3} if \ccc{write_normals} is true, else a model of \ccc{Kernel::Point_3}.
\ccCommentHeading{Returns} true on success.
\ccCommentHeading{Parameters}
\begin{description}
\item \ccc{first}: first input point \item \ccc{beyond}: past-the-end input point \end{description}
}
\ccGlue

%END-AUTO(\ccDefinition)

\ccSeeAlso

\ccRefIdfierPage{CGAL::surface_reconstruction_read_xyz}  \\
\ccRefIdfierPage{CGAL::surface_reconstruction_write_xyz}  \\
\ccRefIdfierPage{CGAL::surface_reconstruction_read_off_point_cloud}  \\

\ccExample

\begin{ccExampleCode}
typedef CGAL::Exact_predicates_inexact_constructions_kernel Kernel;
typedef CGAL::Point_with_normal_3<Kernel> Point_with_normal;
std::deque<Point_with_normal> points;
char* filename = ...;

// Save the point set to OFF file
if(!CGAL::surface_reconstruction_write_off_point_cloud(filename,
                                                       points.begin(), points.end()))
{
  std::cerr << "Error: cannot write file " << filename << std::endl;
  return EXIT_FAILURE;
}
\end{ccExampleCode}

\end{ccRefFunction}

% +------------------------------------------------------------------------+
%%RefPage: end of main body, begin of footer
\ccRefPageEnd
% EOF
% +------------------------------------------------------------------------+

