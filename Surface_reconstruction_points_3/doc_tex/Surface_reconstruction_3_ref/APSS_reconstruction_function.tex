% +------------------------------------------------------------------------+
% | Reference manual page: APSS_reconstruction_function.tex
% +------------------------------------------------------------------------+
% | 02.06.2008   Pierre Alliez, Laurent Saboret, Gael Guennebaud
% | Package: Surface_reconstruction_points_3
% |
\RCSdef{\RCSAPSSreconstructionfunctionRev}{$Id$}
\RCSdefDate{\RCSAPSSreconstructionfunctionDate}{$Date$}
% |
\ccRefPageBegin
%%RefPage: end of header, begin of main body
% +------------------------------------------------------------------------+


\begin{ccRefClass}{APSS_reconstruction_function<GeomTraits>}

%% \ccHtmlCrossLink{}     %% add further rules for cross referencing links
%% \ccHtmlIndexC[class]{} %% add further index entries

\ccDefinition

% The section below is automatically generated. Do not edit!
%START-AUTO(\ccDefinition)

\ccc{APSS_reconstruction_function} computes an implicit function that defines a Point Set Surface (PSS) based on moving least squares (MLS) fitting of algebraic spheres. See {\em Algebraic Point Set Surfaces} by Guennebaud and Gross \cite{Guennebaud07}.

%END-AUTO(\ccDefinition)

% The section below is automatically generated. Do not edit!
%START-AUTO(\ccInclude)

\ccInclude{CGAL/APSS_reconstruction_function.h}

%END-AUTO(\ccInclude)

\ccParameters

The full template declaration is:

% The section below is automatically generated. Do not edit!
%START-AUTO(\ccParameters)

template$<$  \\
class Gt$>$   \\
class \ccc{APSS_reconstruction_function};

\ccCommentHeading{Parameters}
\ccc{Gt}: Geometric traits class.

%END-AUTO(\ccParameters)

\ccIsModel

% The section below is automatically generated. Do not edit!
%START-AUTO(\ccIsModel)

Model of the ImplicitFunction concept.

%END-AUTO(\ccIsModel)

\ccTypes

% The section below is automatically generated. Do not edit!
%START-AUTO(\ccTypes)

\ccNestedType{Geom_traits}
{
Kernel's geometric traits.
}
\ccGlue
\ccNestedType{FT}
{
}
\ccGlue
\ccNestedType{Point}
{
}
\ccGlue
\ccNestedType{Vector}
{
}
\ccGlue
\ccNestedType{Iso_cuboid}
{
}
\ccGlue
\ccNestedType{Sphere}
{
}
\ccGlue
\ccNestedType{Point_with_normal}
{
== \ccc{Point_with_normal_3<Gt>}
}
\ccGlue
\ccNestedType{Normal}
{
== \ccc{Vector_3}
}
\ccGlue

%END-AUTO(\ccTypes)

\ccCreation
\ccCreationVariable{fct}  %% variable name for \ccMethod below

% The section below is automatically generated. Do not edit!
%START-AUTO(\ccCreation)

\ccConstructor{template<class InputIterator> APSS_reconstruction_function(InputIterator first, InputIterator beyond, unsigned int k, FT projection_error = 3.16e-4);}
{
Create an APSS implicit function from a point set.
\ccPrecond the value type of InputIterator must be convertible to \ccc{Point_with_normal_3}.
\ccCommentHeading{Parameters}
\begin{description}
\item \ccc{first}: First point to add. \item \ccc{beyond}: Past-the-end point to add. \item \ccc{k}: Number of nearest neighbors. \item \ccc{projection_error}: Dichotomy error when projecting point. \end{description}
}
\ccGlue
\ccConstructor{APSS_reconstruction_function(const APSS_reconstruction_function<Gt>& other);}
{
Copy constructor.
}
\ccGlue

%END-AUTO(\ccCreation)

\ccOperations

% The section below is automatically generated. Do not edit!
%START-AUTO(\ccOperations)

\ccMethod{APSS_reconstruction_function& operator =(const APSS_reconstruction_function<Gt>& other);}
{
operator =()
}
\ccGlue
\ccMethod{void set_numbers_of_neighbors(unsigned int k);}
{
}
\ccGlue
\ccMethod{Iso_cuboid bounding_box() const;}
{
Get the bounding box.
}
\ccGlue
\ccMethod{const Sphere& bounding_sphere() const;}
{
Returns a sphere bounding the inferred surface.
}
\ccGlue
\ccMethod{Sphere region_of_interest() const;}
{
Get the region of interest, ignoring the outliers. This method is used to define the OpenGL arcball sphere.
}
\ccGlue
\ccMethod{FT operator()(const Point& p) const;}
{
[ImplicitFunction interface]
Evaluate implicit function for any 3D point.
}
\ccGlue
\ccMethod{Point get_inner_point() const;}
{
Returns a point located inside the inferred surface.
}
\ccGlue

%END-AUTO(\ccOperations)

\ccSeeAlso

\ccRefIdfierPage{CGAL::Poisson_reconstruction_function<GeomTraits, ReconstructionTriangulation_3>}  \\

\ccExample

See \ccc{APSS_reconstruction.cpp} example.

\end{ccRefClass}

% +------------------------------------------------------------------------+
%%RefPage: end of main body, begin of footer
\ccRefPageEnd
% EOF
% +------------------------------------------------------------------------+

