%\ignore{

\section{Introduction}

The 3D quadrical kernel aims to provide the user basic functionality
on quadrics and their spatial intersection curves. It is 
consistent to most existing design choices taken in \cgal. To better
understand its principles we refer to the kernel manual if \cgal.

\section{Software Design}

The quadrical kernel is documented as a concept that can be 
implemented by various models, like the projection approach, or
or the parameterization method. TODO cites!

The quadrical kernel is instanciated with two template parameters:
\begin{itemize}
\item {} the \ccc{SphericalKernel}, from which the quadrical
kernel inherits basic elementary linear and also
spherical objects.
\item {} the second parameter is the algebraic kernel, which is 
responsible for computations on polynomials and algebraic numbers. It 
has to be a model of concept \ccc{AlgebraicKernelForQuadrics}. The
robustness of the package relies on the fact that the algebraic kernel
provides exact computations on algebraic objects.
\end{itemize}

Note that the 3D quadrical provides most basic types taking from one
of its two template parameters. Main new geometric objects
are \ccc{Quadric_3}, \ccc{Curve_3}, \ccc{Curve_point_3} and \ccc{Curve_arc_3}.

\section{Design Rationales and Discussion}

The proposed concept follows main discussions of underlying
geometry and algebraic kernels. In contrast to the circular
and spherical world, quadrics poses additional problems that do 
not appear before. 

One point is the correct handling of infinite objects. Quadrics
and also their intersection curves can approach infinity in each variable.
Due to the affine view of the geometric kernel, we can exclude
for sure that the later concept of \ccc{Curve_point_3} never has 
to represent curve-ends at infinity. The situation will be different
when designing a homogeneous kernel. 

Another difficulty is hidden in the spatial intersection curves.
Spheres intersect in spatial circles, that might be degenerated
to a point in case the spheres only touch. Spatial circles
have three advantages: (1)~they are bounded, (2)~consists of one irreducible
algebraic component which implies (3)~consists of a single
connected component.
In general, all properties get lost when switching to arbitrary quadrics.
It is possible to think of an instance of \ccc{Curve_3} being either
(1)~The one-dimensional zero set of a polynomial
system, i.e., the algebraic space curve, or (2)~an algebraically 
irreducible factor of the former system, or (3)~a maximal connected component
of the spatial curve. 
The concept of \ccc{Curve_3} should either be fixed to one of the
three definitions, we can think of additional types, which might be
confusing, or the type is polymorphic, i.e., it is possible
to represent all definitions. But then proper decomposition
methods are required, i.e., at least from (1) to (2) and from (2) to (3).

For curve arcs the situation is similar. Since spatial curves can be infinite,
consist of several algebraic or connected components and event can have 
self-intersection, we carefully have to decide what is described by a curve
arc. Three different objects can be be distinguished: (a)~Both ends of an 
arc are finite, (b)~one end of an arc is finite, like a ``ray'', or 
(c)~both ends are infinite, like a ``branch''. It is unclear
whether different types should be introduced, or
all can be encoded by a single ``polymorphic'' type. In the latter
case, end-points can only be accessed if they are finite. As mentioned before,
in a homogenous version, this condition can be removed. To find the 
proper splitting of a \ccc{Curve_3} into maximal \ccc{Curve_arc_3}s depends
on this question.

\section{Acknowledgements}

This work is partially supported by the IST Programme of the 6th
Framework Programme of the EU as a STREP (FET Open Scheme) Project
under Contract No IST-006413 (\ccAnchor{http://acs.cs.rug.nl/}{ACS} -
Algorithms for Complex Shapes).
