% +------------------------------------------------------------------------+
% | Reference manual page: Umbilic.tex
% +------------------------------------------------------------------------+
% | 15.09.2006   Marc Pouget and Fr�d�ric Cazals
% | Package: Ridges_3
% | 
\RCSdef{\RCSUmbilicRev}{$Id$}
\RCSdefDate{\RCSUmbilicDate}{$Date$}
% |
%%RefPage: end of header, begin of main body
% +------------------------------------------------------------------------+


\begin{ccRefClass}{Umbilic<TriangularPolyhedralSurface>}  %% add template arg's if necessary

%% \ccHtmlCrossLink{}     %% add further rules for cross referencing links
%% \ccHtmlIndexC[class]{} %% add further index entries

\ccDefinition
  
The class \ccRefName\ stores the description of an umbilic.

\ccInclude{CGAL/Umbilic.h}

\ccTypes

\ccNestedType{  typedef typename TriangularPolyhedralSurface::Vertex_handle    Vertex_handle;}{}
\ccGlue
\ccNestedType{ typedef typename TriangularPolyhedralSurface::Halfedge_handle  Halfedge_handle;}{}

\ccCreation
\ccCreationVariable{umbilic} %hoose variable name

\ccConstructor{Umbilic(Vertex_handle v_init,
	  std::list<Halfedge_handle> contour_init);}{default constructor.}

\ccAccessFunctions
% +--------------------------------------------------------------
\ccMethod{const Vertex_handle vertex();}{}
\ccMethod{const Umbilic_type umbilic_type();}{}
\ccGlue
\ccMethod{Umbilic_type& umbilic_type();}{}
\ccMethod{const std::list<Halfedge_handle>& contour_list() ;}{}
\ccGlue
\ccMethod{std::list<Halfedge_handle>& contour_list();}{}


\ccOperations
The usual insert operator ($<<$) is overloaded for \ccc{Umbilic}, it
gives the location (3d coordinates of the vertex) and the type.

%\ccFunction{template <class TriangularPolyhedralSurface>
%std::ostream& 
%operator<<(std::ostream& out_stream, const Umbilic<TriangularPolyhedralSurface>& umbilic);}
%{Outputs umbilic location and type.}


\ccSeeAlso

\ccc{Umbilic_approximation}


\end{ccRefClass}

% +------------------------------------------------------------------------+
%%RefPage: end of main body, begin of footer
% EOF
% +------------------------------------------------------------------------+

