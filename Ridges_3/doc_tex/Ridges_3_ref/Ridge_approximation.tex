% +------------------------------------------------------------------------+
% | Reference manual page: Ridge_approximation.tex
% +------------------------------------------------------------------------+
% | 15.09.2006   Marc Pouget and Fr�d�ric Cazals
% | Package: Ridges_3
% | 
\RCSdef{\RCSRidgeapproximationRev}{$Id$}
\RCSdefDate{\RCSRidgeapproximationDate}{$Date$}
% |
%%RefPage: end of header, begin of main body
% +------------------------------------------------------------------------+


\begin{ccRefClass}{Ridge_approximation<TriangularPolyhedralSurface,Vertex2FTPropertyMap,Vertex2VectorPropertyMap>} 
%% add templatearg's if necessary

%% \ccHtmlCrossLink{}     %% add further rules for cross referencing links
%% \ccHtmlIndexC[class]{} %% add further index entries

\ccDefinition
  
The class \ccRefName\ computes the approximation of
ridges of a triangular polyhedral surface. 

\ccInclude{CGAL/Ridges.h}

\ccParameters
The class \ccRefName\ has three template parameters. 
\ccc{TriangularPolyhedralSurface} provides  the surface. 
 Parameters  \ccc{Vertex2FTPropertyMap}
and \ccc{Vertex2VectorPropertyMap} provide the differential properties of
the surface associated to its vertices.

Requirements (checked at compile time) : the types
\ccc{TriangularPolyhedralSurface::Traits::FT} and
\ccc{Vertex2FTPropertyMap::value_type} must coincide; the types
\ccc{TriangularPolyhedralSurface::Traits::Vector_3} and
\ccc{Vertex2VectorPropertyMap::value_type} must coincide; the types
\ccc{TriangularPolyhedralSurface::Vertex_handle},
\ccc{Vertex2FTPropertyMap::key_type} and
\ccc{Vertex2VectorPropertyMap::key_type} must coincide;

\ccTypes

%\ccNestedType{TYPE}{some nested types}
\ccEnum{ enum Tag_order {Tag_3, Tag_4};} {Order of differential
  quantities used to distinguish elliptic and hyperbolic ridges. Third
  (\ccc{Tag_3}) or fourth (\ccc{Tag_4}) order quantities may be used as
  explained in section \ref{ridge-mesh} of the user manual.}


\ccCreation
\ccCreationVariable{ridge_approximation}  %% choose variable name, given by \ccVar

\ccConstructor{Ridge_approximation(const TriangularPolyhedralSurface &P,
		      const Vertex2FTPropertyMap& vertex2k1_pm, const
		      Vertex2FTPropertyMap& vertex2k2_pm, const
		      Vertex2FTPropertyMap& vertex2b0_pm, const
		      Vertex2FTPropertyMap& vertex2b3_pm, const
		      Vertex2VectorPropertyMap& vertex2d1_pm, const
		      Vertex2VectorPropertyMap& vertex2d2_pm, const
		      Vertex2FTPropertyMap& vertex2P1_pm, const
		      Vertex2FTPropertyMap& vertex2P2_pm);}
		      {%Precondition : all faces of P must be triangular. \\
                        The two last property maps may
		      not be used if computations are performed with
		      the parameter \ccc{Tag_3}, in which case these
		      property maps shall be initialized with their
		      default constructors.}

%\ccOperations

\ccMethod{ template <class OutputIterator> OutputIterator compute_all_ridges(OutputIterator it, Tag_order ord = Tag_3);}
{Computes blue, red and crest ridges. Parameter \ccc{it}
 is a  {\sc Stl} output iterator whose
\ccc{value_type} is \ccc{Ridge_line*}.}
\ccGlue
\ccMethod{ template <class OutputIterator>  OutputIterator compute_ridges(Ridge_interrogation_type r_type, 
			  OutputIterator ridge_lines_it,
			  Tag_order ord = Tag_3);}
{Computes blue, red or crest ridges. Parameter \ccc{it}
 is a  {\sc Stl} output iterator whose
\ccc{value_type} is \ccc{Ridge_line*}.}

\ccSeeAlso

\ccc{Ridge_line}

\end{ccRefClass}

% +------------------------------------------------------------------------+
%%RefPage: end of main body, begin of footer
% EOF
% +------------------------------------------------------------------------+

