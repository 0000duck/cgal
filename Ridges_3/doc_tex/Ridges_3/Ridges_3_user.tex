%note on pictures

% ellipsoid_ridges 
% ./blind -f data/ellipsoid_u_0.02.off -d4 -m4 -a3 -t3 
% ../../demo/Ridges_3/introspect-qt data/ellipsoid_u_0.02.off data/data_ellipsoid_u_0.02.offRIDGES-d4-m4-t3-a3-p0.4ogl.txt 0 0





\newtheorem{definition}{Definition.}
\newcommand{\hot}{h.o.t}%[0]


This chapter describes the \cgal's package for the extraction of
ridges and umbilics on meshes.  Given a smooth surface, a ridge is a
curve along which one of the principal curvatures has an extremum
along its curvature line. An umbilic is a point at which both
principal curvatures are equal. Umbilics are special points on the
ridge lines. Ridges are curves of {\em extremal} curvature and
therefore encode important informations used in segmentation,
registration, matching and surface analysis.  Based on the results of
the article
\cite{cgal:cp-tdare-05}
, we propose algorithms to identify and extract
different parts of this singular ridge curve as well as umbilics on a
surface given as a triangular mesh. Differential quantities associated
to the mesh vertices are assumed to be given for these algorithms;
such quantities may be computed by the package {\em Estimation of
Local Differential Properties of Sampled Surfaces via Polynomial
Fitting}.


\subsection{Overview}
%%%%%%%%%%%%%%%%%%%%%%

2 main independent classes

\begin{itemize}
\item
\ccc{Umbilic_approximation}
\item
\ccc{Ridge_approximation}
\end{itemize}

And two ``container'' classes.

%%%%%%%%%%%%%%%%%%%%%%%
\section{Introduction}
\label{sec:intro}
%%%%%%%%%%%%%%%%%%%%%%%

\subsection{Ridges and umbilics of a smooth surface}
%%%%%%%%%%%%%%%%%%%%%%%%%%%%%%%%%%%%%%%

A comprehensive description of ridges can be found in
\cite{cgal:hgygm-ttdpf-99,cgal:p-gd-01,cgal:cp-ssulc-05}
, and in the sequel, we just introduce the basic notions so as to
explain our algorithms.  Consider a smooth embedded surface, and
denote $k_1$ and $k_2$ the principal curvatures, with $k_1\geq
k_2$. Umbilics are the points where $k_1=k_2$.  For any non umbilical
point, the corresponding principal directions of curvature are well
defined, and we denote them $d_1$ and $d_2$.
%%
Anything related to the maximal (minimal) curvature is qualified blue
(red), for example we shall speak of the blue curvature for $k_1$ or
the red direction for $d_2$.
%%
In local coordinates, we denote $\langle , \rangle$
the inner product induced by the ambient Euclidean space, and $dk_1$,
$dk_2$ the gradients of the principal curvatures. Ridges, illustrated
on Figs \ref{pict:ellipsoid_ridges} and \ref{fig:ridges_ellipsoid},
are defined by:

\begin{definition}
\label{def:ridge-extrema}
A non umbilical point is called
\begin{itemize}
\item
a blue ridge point if the {\em extremality coefficient} $b_0=\langle
dk_1,d_1 \rangle$ vanishes, i.e. $b_0=0$.

\item
a red ridge point if the {\em extremality coefficient} $b_3=\langle
dk_2,d_2 \rangle$ vanishes, i.e. $b_3=0$.

\end{itemize}
\end{definition}

%%
The previous characterization of ridges involves third-order
differential properties. Using fourth-order differential quantities, a
ridge point can further be qualified as {\em elliptic} if it
corresponds to a maximum of $k_1$ or a minimum of $k_2$, or {\em
hyperbolic} otherwise. Hence we end with four types of ridges, namely
: \ccc{BLUE_ELLIPTIC_RIDGE}, \ccc{BLUE_HYPERBOLIC_RIDGE}, \ccc{RED_ELLIPTIC_RIDGE},
\ccc{RED_HYPERBOLIC_RIDGE}.
In addition, a subset of elliptic ridges, called the crest lines,
which can be seen as the visually most salient curves on a surface are
also of interest. A crest line is an elliptic ridge which is a maximum
of $\max(|k_1|,|k_2|)$. Hence we provide two additional ridge types :
\ccc{BLUE_CREST_RIDGE} and \ccc{RED_CREST_RIDGE}.



At any point of the surface which is not an umbilic, principal
directions $d_1, d_2$ are well defined, and these (non oriented)
directions together with the normal vector $n$ define two direct
orthonormal frames. If $v_1$ is a unit vector of direction $d_1$ then
there exists a unique unit vector $v_2$ so that $(v_1,v_2,n)$ is
direct (and the other possible frame is $(-v_1,-v_2,n)$). In the
coordinate systems $(v_1,v_2,n)$, the surface has the following
canonical form~:
%
\begin{eqnarray}
\label{eq:monge}
z(x,y) =  & \frac{1}{2}(k_1x^2 + k_2y^2)+
	\frac{1}{6}(b_0x^3+3b_1x^2y+3b_2xy^2+b_3y^3) \\
  &  +\frac{1}{24}(c_0x^4+4c_1x^3y+6c_2x^2y^2+4c_3xy^3+c_4y^4) + \hot
\end{eqnarray}

\noindent The Taylor expansion of $k_1$ (resp. $k_2$) along the blue
(resp. red) curvature line going through the origin and parameterized
by $x$ (resp. $y$) are:
\begin{equation}
\label{eq:taylor_along_line}
k_1(x) = k_1 + b_0x + \frac{P_1}{2(k_1-k_2)}x^2 +\hot \quad \quad \quad 
P_1= 3b_1^2+(k_1-k_2)(c_0-3k_1^3).
\end{equation}
%
\begin{equation}
\label{eq:taylor_along_red_line}
k_2(y) = k_2 + b_3y + \frac{P_2}{2(k_2-k_1)}y^2 +\hot \quad \quad \quad 
P_2= 3b_2^2+(k_2-k_1)(c_4-3k_2^3).
\end{equation}

\noindent Notice also that switching from one of the two coordinate systems
mentioned to the other reverts the sign of all the odd
coefficients on the Monge form of the surface.
\medskip

Hence ridge types are characterized by 
\begin{itemize}
\item
\ccc{BLUE_RIDGE} if $b_0=0$
\item
\ccc{BLUE_ELLIPTIC_RIDGE} if  $b_0=0$ and $P_1<0$
\item
 \ccc{BLUE_HYPERBOLIC_RIDGE} if  $b_0=0$ and $P_1>0$
\item
 \ccc{RED_RIDGE} if  $b_3=0$
\item
 \ccc{RED_ELLIPTIC_RIDGE} if  $b_3=0$ and $P_2<0$
\item
\ccc{RED_HYPERBOLIC_RIDGE} if  $b_3=0$ and $P_2>0$
\item
\ccc{BLUE_CREST_RIDGE} if  $b_0=0$  and $P_1<0$ and $|k_1|>|k_2|$
\item
\ccc{RED_CREST_RIDGE} if  $b_3=0$ and $P_2<0$ and $|k_2|>|k_1|$
\end{itemize}


\subsection{Extraction of ridges on Triangular Meshes}
%%%%%%%%%%%%%%%%%%%%%%%%%%%%%%%%%

As 0-level set of the extremality coefficients $b_0$ and $b_3$, ridges
are extracted by a marching line algorithm on the surface.  As the
signs of these extremalities depends on the orientation of the
principal directions, we expect both extremalities and vectors
orienting the principal direction to be given at each point vertex of
the mesh. Except in the neighborhood of umbilics, if the mesh is dense
enough, a coherent orientation of principal directions at both
endpoints of an edge is chosen such than the angle between the two
vectors is acute-- we call this method the ``acute rule''. Moreover,
we only check for ridges in triangles such than one can find an
orientation of its three vertices (we call theses triangles
``regular'') such that the three edges are coherently oriented.

These choices lead to the following observations.
\begin{itemize}
\item
A regular triangle can have either 0 or 2 edges crossed by a ridge of
a given color (meaning that $b_0$ or $b_3$ change sign on the
edge). We say that the triangle contains a ridge segment, for which we
compute its type.  Then we define a
\ccc{Ridge_line} as a maximal sequence of ridge segments of the 
same type and connected together.  The topology of a \ccc{Ridge_line}
is either that of an interval or a circle.
\item
In the neighborhood of umbilics, triangle are less likely to be
regular and the detection of ridges cannot be relevant by this method.
This is why we propose another method to detect umbilics
independently.
\end{itemize}


explain

tag E/H with 3 or 4 order

extraction of umbilics, present the class \ccc{Umbilic}

strength sharpness

\subsection{Extraction of Umbilics on Triangular Meshes}
%%%%%%%%%%%%%%%%%%%%%%%%%%%%%%%%%

First we need to define patches of triangles of the mesh $T$
containing an umbilic.  The method combines a minimization and an
index computation on the neighborhood of each triangle of $T$. The
size of the neighborhood is the only parameter of the algorithm.

\paragraph{Finding patches around vertices.}
Given a vertex $v$, we aim at defining a collection of triangles
around $v$ so that this collection defines a topological disk on the
triangulation $T$. First we collect the 1-ring triangles. We define
the size of this 1-ring patch as the distance from $v$ to its farthest
1-ring vertex neighbor. Then we collect recursively adjacent triangles
so that the patch remains a topological disk and such that these
triangles are at distance less than (1-ring patch size) times a user
given parameter.

Umbilics are vertices $v$ satisfying the 2 conditions
\begin{itemize}
\item
the function $k_1-k_2$ has its minimum at $v$ amongst all the
vertices of the patch;
\item
the deviation $\delta$ of the principal directions along the contour
of the patch travelled in the CCW sens is
\begin{itemize}
\item
$\delta \in ]\Pi/2,3\Pi/2[$, then the umbilic is called a wedge,
\item
$\delta \in ]-3\Pi/2,-\Pi/2[$, then the umbilic is called a trisector,
\item
$\delta \geq 3\Pi/2$ or $\delta \leq -3\Pi/2$ then the umbilic is called non-generic.
\end{itemize}
\end{itemize}

%%%%%%%%%%%%%%%%%%%%%%%
\section{Software Design}
%%%%%%%%%%%%%%%%%%%%%%%

usage of pm, triangular meshes.

\subsection{Options and interface specifications}
%%%%%%%%%%%%%%%%%%%%%
ridges: approx and container class

ridges compute with para type and tag

Umbilics : approx and container, 

umbilic compute with para 
size

\subsection{Template parameters}
%%%%%%%%%%%%%%%%%%%%%%%%%%%%%%%%%

On a des concepts: Vertex2FTPropertyMap et Vertex2VectorPropertyMap
qui sont specialisent le concept de Propety Map de boost, avec
\ccc{key_type CGAL::Polyhedron_3::Vertex_handle} et de \ccc{value_type
CGAL::Polyhedron_3::Traits::FT ou Vector_3 }. On a donc les fct get,
put, []; preciser LvaluePropertyMap ? Utilisation stockage
d'information scalaire ou vectorielle pour les vertex d'un polyhedron,
peut-etre dans le vertex lui-meme ou externe avec une std::map par
exemple. 

Poly est un concept qui a pour model \ccc{CGAL::Polyhedron_3} (faut-il
detaille les requirements?) TriangularPolyhedralSurface?

OutputIt est un concept de stl Output Iterator avec \ccc{value_type
CGAL::Ridge_line*} .\\
ou Outputit est un Output Iterator de la stl sur des Umbilic*


\subsection{Output}
%%%%%%%%%%%%%%%%%%%%

Classes \ccc{Umbilic} and \ccc{Ridge_line}

%%%%%%%%%%%%%%%%%%%%%%%%%%%%%%%%%%%%%%%%%%%%%%%%%%%%%%%%%%%%%%%%%%%%%%%%%%%%
\section{Examples} 
%%%%%%%%%%%%%%%%%%%%%%%%%%%%%%%%%%%%%%%%%%%%%%%%%%%%%%%%%%%%%%%%%%%%%%%%%%%%
