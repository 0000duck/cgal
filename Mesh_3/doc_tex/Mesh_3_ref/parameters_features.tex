% +------------------------------------------------------------------------+
% | Reference manual page: parameters_features.tex
% +------------------------------------------------------------------------+
% | 18.06.2010   Author
% | Package: Package
% |
\RCSdef{\RCSparametersfeaturesRev}{$Id$}
\RCSdefDate{\RCSparametersfeaturesDate}{$Date$}
% |
\ccRefPageBegin
%%RefPage: end of header, begin of main body
% +------------------------------------------------------------------------+


\begin{ccRefFunction}{parameters::features}  %% add template arg's if necessary

%% \ccHtmlCrossLink{}     %% add further rules for cross referencing links
%% \ccHtmlIndexC[function]{} %% add further index entries

\ccDefinition
  
The function \ccRefName\  provides a  value of internal type \ccc{Features}
to specify if  $0$ and $1$-dimensional features have to be taken into account.
The provided value is a default value that triggers the representation
of corners and  curve segments in the mesh when the domain is a model
of  \ccc{MeshDomainWithFeatures_3}.

\ccInclude{CGAL/refine_mesh_3.h}

\ccGlobalFunction{%
parameters::internal::Features_options parameters::features(
            MeshDomain_3 domain);}
Provides a \ccc{Features_options} value such that
 $0$ and $1$-dimensional input features are taken into account
 if domain is a model of  the refined concept \ccc{MeshDomainWithFeatures_3}.

\ccSeeAlso

\ccRefIdfierPage{CGAL::make_mesh_3}, \\
\ccRefIdfierPage{CGAL::refine_mesh_3}, \\
\ccRefIdfierPage{CGAL::parameters::no_features}. \\

% \ccExample

% A short example program.
% Instead of a short program fragment, a full running program can be
% included using the 
% \verb|\ccIncludeExampleCode{Package/parameters_features.C}| 
% macro. The program example would be part of the source code distribution and
% also part of the automatic test suite.

% \begin{ccExampleCode}
% void your_example_code() {
% }
% \end{ccExampleCode}

%% \ccIncludeExampleCode{Package/parameters_features.C}

\end{ccRefFunction}

% +------------------------------------------------------------------------+
%%RefPage: end of main body, begin of footer
\ccRefPageEnd
% EOF
% +------------------------------------------------------------------------+

