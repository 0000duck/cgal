% +------------------------------------------------------------------------+
% | Reference manual page: parameters_perturb.tex
% +------------------------------------------------------------------------+
% | 20.10.2009   Stephane Tayeb
% | Package: Mesh_3
% |
\RCSdef{\RCSperturbRev}{$Id$}
\RCSdefDate{\RCSperturbDate}{$Date$}
% |
\ccRefPageBegin
 %%RefPage: end of header, begin of main body
% +------------------------------------------------------------------------+

\begin{ccRefFunction}{parameters::perturb}  %% add template arg's if necessary

%% \ccHtmlCrossLink{}     %% add further rules for cross referencing links
%% \ccHtmlIndexC[function]{} %% add further index entries

\ccDefinition
  
The function \ccRefName\ allows the user to trigger a call to
\ccc{perturb_mesh_3} in
\ccc{make_mesh_3} and \ccc{refine_mesh_3} mesh generation functions. It also
allows the user to pass parameters 
to the optimization function \ccc{perturb_mesh_3} through these mesh generation functions.

\ccInclude{CGAL/refine_mesh_3.h}

\ccGlobalFunction{parameters::internal::Perturb parameters::perturb(
  double time_limit = 0,
  double sliver_bound = 0);}

\ccParameters

The parameters are named parameters. They are the same (i.e. they have the same
name and the same default values) as the parameters of \ccc{perturb_mesh_3}
function. See its manual page for further details.


\ccSeeAlso

\ccc{no_perturb} \\
\ccc{perturb_mesh_3} \\
\ccc{make_mesh_3} \\
\ccc{refine_mesh_3} 


\ccExample

\begin{ccExampleCode}
// Mesh generation with a perturbation step
C3t3 c3t3 = make_mesh_3<c3t3>(domain, criteria, parameters::perturb());
refine_mesh_3(c3t3, domain, criteria, parameters::perturb(parameters::time_limit=10));
\end{ccExampleCode}


\end{ccRefFunction}

% +------------------------------------------------------------------------+
%%RefPage: end of main body, begin of footer
\ccRefPageEnd
% EOF
% +------------------------------------------------------------------------+

