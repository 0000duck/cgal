% +------------------------------------------------------------------------+
% | Reference manual page: Mesh_domain_with_polyline_features_3.tex
% +------------------------------------------------------------------------+
% | 28.07.2009   Stephane Tayeb
% | Package: Mesh_3
% |
\RCSdef{\RCSMeshdomainwithpolylinefeaturesRev}{$Id$}
\RCSdefDate{\RCSMeshdomainwithpolylinefeaturesDate}{$Date$}
% |
\ccRefPageBegin
%%RefPage: end of header, begin of main body
% +------------------------------------------------------------------------+


\begin{ccRefClass}{Mesh_domain_with_polyline_features_3<MeshDomain_3>}  %% add template arg's if necessary

%% \ccHtmlCrossLink{}     %% add further rules for cross referencing links
%% \ccHtmlIndexC[class]{} %% add further index entries

\ccDefinition
  
The class \ccRefName\ is designed to allow the user
 to add some 0- and 1-dimensional
features into any model of the \ccc{MeshDomain_3} concept.
The  1-dimensional features are described as polylines
whose endpoints are the added corners.



\ccParameters

The template parameter \ccc{MeshDomain_3} is the type
of the domain which should be extended.
It has to be  a model of the \ccc{MeshDomain_3} concept.


\ccInclude{CGAL/Mesh_domain_with_polyline_features_3.h}

\ccIsModel

\ccc{MeshDomainWithFeatures_3}

\ccInheritsFrom
\ccHtmlNoLinksFrom{\ccc{MeshDomain_3}}

\ccTypes

\ccTypedef{typedef int Corner_index;}{\ccc{Corner_index} type.}
\ccTypedef{typedef int Curve_segment_index;}{\ccc{Curve_segment_index} type.}

\ccCreation
\ccCreationVariable{md}  %% choose variable name

\ccConstructor{template <typename ...T>
Mesh_domain_with_polyline_features_3(T ...t);}{Constructor. Forwards the arguments to the constructor
of the base class.}

\ccOperations

\ccMethod{template <typename InputIterator>
  void add_features(InputIterator begin, InputIterator beyond);}{
  Add 1-dimensional features in the domain. \ccc{InputIterator} value type must
  be a model of the concept \ccc{MeshPolyline_3}.}

\ccMethod{template <typename InputIterator>
  void add_features_and_incidences(InputIterator begin, InputIterator beyond);}{
  Add 1-dimensional features in the domain with their incidences with 2-dimensional
  features of the domain. The \ccc{InputIterator} value type must be 
  \ccc{std::pair<Polyline, std::pair<InputSurfacePatchIndexIterator, InputSurfacePatchIndexIterator> >} 
  where \ccc{Polyline} must be a model of the concept \ccc{MeshPolyline_3} 
  and the internal pair gives a range on surface patches indices which are incident
  to the polyline.}

\ccSeeAlso

\ccRefConceptPage{MeshDomain_3}\\
\ccRefConceptPage{MeshDomainWithFeatures_3}\\
\ccRefConceptPage{MeshPolyline_3}\\
\ccRefIdfierPage{CGAL::Implicit_mesh_domain_3<Function,BGT>} \\
\ccRefIdfierPage{CGAL::Polyhedral_mesh_domain_3<Polyhedron,IGT,TriangleAccessor>} \\
\ccRefIdfierPage{CGAL::Labeled_image_mesh_domain_3<Image,BGT>} 


\end{ccRefClass}

% +------------------------------------------------------------------------+
%%RefPage: end of main body, begin of footer
\ccRefPageEnd
% EOF
% +------------------------------------------------------------------------+

