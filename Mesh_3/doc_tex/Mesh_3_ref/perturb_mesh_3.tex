% +------------------------------------------------------------------------+
% | Reference manual page: perturb_mesh_3.tex
% +------------------------------------------------------------------------+
% | 28.07.2009   Stephane Tayeb
% | Package: Mesh_3
% |
\RCSdef{\RCSperturbmeshRev}{$Id$}
\RCSdefDate{\RCSperturbmeshDate}{$Date$}
% |
\ccRefPageBegin
%%RefPage: end of header, begin of main body
% +------------------------------------------------------------------------+


\begin{ccRefFunction}{perturb_mesh_3}  %% add template arg's if necessary

%% \ccHtmlCrossLink{}     %% add further rules for cross referencing links
%% \ccHtmlIndexC[function]{} %% add further index entries

\ccDefinition
  
The function \ccRefName{} is a mesh optimizer that
 improves the quality of a Delaunay mesh
by changing the mesh vertices positions.

The perturber tries to improve the dihedral angles of the worst cells in the mesh
degree by degree: the
step  number \emph{n} is considered as successfull 
if after this step the worst tetrahedron of the mesh has a minimal dihedral
angle larger than \emph{n} degrees.
The perturber exits if this is not the case.

\ccInclude{CGAL/perturb_mesh_3.h}

\ccGlobalFunction{
	template<typename C3T3, typename MeshDomain>
	Mesh_optimization_return_code
        perturb_mesh_3(C3T3& c3t3,
          MeshDomain domain,
          double parameters::time_limit=0,
          double parameters::sliver_bound=0);}{
 \ccPrecond{\ccc{time_limit} $\geq$ 0 and 0 $\leq$ \ccc{sliver_bound} $\leq$ 180} }


\ccParameters

Parameter \ccc{C3T3} is required to be a model of the concept
\ccc{MeshComplex_3InTriangulation_3}.
The argument \ccc{c3t3}, passed by
reference, provides the initial mesh  
and is modified  by the algorithm 
to represent the final optimized mesh.

Parameter \ccc{MeshDomain} is required to be a model of the concept
\ccc{MeshDomain_3}. The argument \ccc{domain} must be the \ccc{MeshDomain}
object used to create \ccc{c3t3} parameter.

The function has two optional parameters which are named parameters (we use the Boost.Parameters library). 
Therefore, when calling the function,  the parameters can be provided in any order 
provided that the names of the parameters are used
 (see example at the bottom of this page).
\begin{itemize}
\item
Parameter \ccc{time_limit}  whose name is \ccc{parameters::time_limit},
 is used to set up, in seconds,
 a CPU time limit after which the optimization process is stopped. This time is
 measured using \ccc{CGAL::Timer}.
The default value is \ccc{0} and means that there is no time limit.
\item 
Parameter \ccc{sliver_bound}, whose name is \ccc{parameters::sliver_bound},
is designed to give, in degree,  a targeted
lower bound on dihedral angles of mesh cells.
The function \ccRefName{} function runs as long as step are successfull
 and  step number \ccc{sliver_bound} (after which
the worst tetrahedron in the mesh has a smallest angle larger than 
\ccc{sliver_bound}  degrees) has not been reached.
The default value is \ccc{0} and means that  there is no targeted bound: 
the perturber  then runs as long as
steps are successfull.
\end{itemize}




\ccHeading{Return Values}
The function \ccRefName{} returns a value of type \ccc{Mesh_optimization_return_code}
which is:
\begin{itemize}
\item \ccc{BOUND_REACHED} when the targeted bound for the smallest dihedral angle in the mesh is reached.
\item\ccc{TIME_LIMIT_REACHED} when the time limit is reached.
\item \ccc{CANT_IMPROVE_ANYMORE} when exudation process stops because the last step is unsuccessfull.
\end{itemize}

\ccSeeAlso

\ccc{Mesh_optimization_return_code} \\
\ccc{make_mesh_3} \\
\ccc{refine_mesh_3} \\
\ccc{exude_mesh_3} \\
\ccc{lloyd_optimize_mesh_3} \\
\ccc{odt_optimize_mesh_3} \\


\ccExample
\begin{ccExampleCode}
// Perturb until every dihedral angle of the mesh is >= 10 degrees
// No time bound is set
perturb_mesh_3(c3t3, domain, parameters::sliver_bound=10);
\end{ccExampleCode}

\end{ccRefFunction}

% +------------------------------------------------------------------------+
%%RefPage: end of main body, begin of footer
\ccRefPageEnd
% EOF
% +------------------------------------------------------------------------+

