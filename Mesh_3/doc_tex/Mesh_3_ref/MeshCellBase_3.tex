% +------------------------------------------------------------------------+
% | Reference manual page: MeshCellBase_3.tex
% +------------------------------------------------------------------------+
% | 19.05.2008   Laurent Rineau
% | Package: Mesh_3
% |
\RCSdef{\RCSMeshCellBaseRev}{$Id$}
\RCSdefDate{\RCSMeshCellBaseDate}{$Date$}
% |
\ccRefPageBegin
%%RefPage: end of header, begin of main body
% +------------------------------------------------------------------------+


\begin{ccRefConcept}{MeshCellBase_3}

%% \ccHtmlCrossLink{}     %% add further rules for cross referencing links
%% \ccHtmlIndexC[concept]{} %% add further index entries

\ccDefinition
  
The concept \ccRefName\ describes the  requirements 
for the \ccc{Cell} type  of the triangulation
used in the 3D mesh generation process. The type \ccRefName\
refines the concept \ccc{RegularTriangulationCellBase_3}.
As a \cgal{} 3D triangulation triangulates the whole convex hull of its vertices,
it is needed to specify whether a given cell of the triangulation
is inside the domain or not and which subdomain it belongs to
in case of a multi-domain. 

Moreover, this concept adds four markers per cell to mark the facets
of the triangulation that are surface facets,
and four additional helper markers
used in some operations to mark for instance
the facets that have been visited.

This concept also provides storage for  the centers of  Delaunay surface
balls.
Each surface facet has a  Delaunay surface ball, i. e. 
a circumscribing ball, centered 
on an input complex  surface patch,  
 and empty of triangulation vertices. 
In the following we call \ccc{surface center}
of a surface facet, the center of its biggest Delaunay surface ball.

\ccRefines
\ccc{RegularTriangulationCellBase_3} \\

\ccTypes

\ccNestedType{Point}{Point type, required to match the point type
of the 3D triangulation in which the mesh is embedded.}

% \ccNestedType{Subdomain_index}
% {Type of indices for cells of the
%   input complex. Must be a model of CopyConstructible,
%   Assignable, DefaultConstructible and EqualityComparable. 
% This type must be convertible to  the nested type \ccc{MeshDomain_3::Subdomain_index}.
% The default constructed value must
%   be the label of the exterior of the domain (which contains at least the
%   unbounded component).}

% \ccNestedType{Surface_index}{Type of indices for surface patches of the
%   input complex. Must be a model of CopyConstructible,
%   Assignable, DefaultConstructible and EqualityComparable.
% This type must be convertible 
% to the nested type \ccc{MeshDomain_3::Surface_index}. The default constructed
% value must be different from values taken by indices of surface facets,
% as it may be used to
% distinguish between a surface facet and an internal (or outside) one.}

\ccNestedType{Subdomain_index;}
{Type of indices for cells of the input complex. Must match the type \ccc{MeshDomain_3::Subdomain_index}.}
\ccGlue
\ccNestedType{Surface_index;}
{Type of indices for surface patches of the input complex. Must match the type \ccc{MeshDomain_3::Surface_index}.}

\ccCreationVariable{cell}  %% choose variable name

\ccOperations

\ccMethod{Subdomain_index subdomain_index();}{Returns the
  index of the input subdomain that contains the cell \ccc{cell}
  of the triangulation.}

\ccMethod{void set_subdomain_index(Subdomain_index index);}{Sets
  the subdomain index of the cell.}

\ccMethod
{bool is_facet_on_surface(int i);}
{returns \ccc{true} iff \ccc{facet(i)} is a surface facet. }
\ccGlue
\ccMethod
{void set_facet_on_surface(int i, Surface_index index);}
{Sets \ccc{facet(i)} as a surface facet  iff \ccc{index} is
  different from \ccc{Surface_index} default value.}

\ccMethod
{bool is_facet_visited (int i);}
{Returns \ccc{true} iff \ccc{facet(i)} has been visited.}
\ccGlue
\ccMethod
{void set_facet_visited (int i);} 
{Marks \ccc{facet(i)} as visited.}

\ccMethod
{const Point& get_facet_surface_center(int i);}
{Returns a const reference to the surface center of \ccc{facet(i)}.}
\ccGlue
\ccMethod
{void set_facet_surface_center (int i, Point p);} 
{Sets point \ccc{p} as the surface center of \ccc{facet(i)}.}


\ccHasModels
\ccc{Mesh_cell_base_3<MD,Gt,Cb>}


\ccSeeAlso

\ccc{make_mesh_3} \\
\ccc{MeshDomain_3} \\
\ccc{Complex_3InTriangulation_3}

\end{ccRefConcept}

% +------------------------------------------------------------------------+
%%RefPage: end of main body, begin of footer
\ccRefPageEnd
% EOF
% +------------------------------------------------------------------------+

