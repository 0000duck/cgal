% +------------------------------------------------------------------------+
% | Reference manual page: MeshVertexBase_3.tex
% +------------------------------------------------------------------------+
% | 19.05.2008   Laurent Rineau
% | Package: Mesh_3
% |
\RCSdef{\RCSMeshVertexBaseRev}{$Id$}
\RCSdefDate{\RCSMeshVertexBaseDate}{$Date$}
% |
\ccRefPageBegin
%%RefPage: end of header, begin of main body
% +------------------------------------------------------------------------+


\begin{ccRefConcept}{MeshVertexBase_3}

%% \ccHtmlCrossLink{}     %% add further rules for cross referencing links
%% \ccHtmlIndexC[concept]{} %% add further index entries

\ccDefinition
  
The concept \ccRefName\ describes the  requirements 
for the \ccc{Vertex} type  of the triangulation
used by a 3D mesh generation process. The type \ccRefName\
refines both the concept \ccc{RegularTriangulationVertexBase_3}
and 
 the concept \ccc{SurfaceMeshVertexBase_3}.
It provides additional members  to store and retrieve
information about the location of the vertex with respect 
to the input domain describing the domain to be discretized.
More specifically, the concept \ccRefName\ provides  read-write access 
to an integer representing the dimension of the lowest dimensional face
of the input 3D complex on which the  vertex lies,
and to an index characteristic of this face.


\ccRefines
\ccc{RegularTriangulationVertexBase_3}\\
\ccc{SurfaceMeshVertexBase_3} 

\ccTypes

\ccNestedType{Index;}
{Index type. Must match the type \ccc{MeshDomain_3::Index}.}
\ccNestedType{FT;}
{Numerical type.}
% \ccNestedType{Index}{ Index type. 
% This type must be a model  of CopyConstructible,
%   Assignable, DefaultConstructible and  Comparable.
% This type must be convertible to the nested type \ccc{MeshDomain_3::Index}
% of the \ccc{MeshDomain} template parameter used by the mesh generation process.}

\ccCreationVariable{vertex}  %% choose variable name

\ccOperations

\ccMethod{int in_dimension() const;}{Returns the dimension of the lowest dimensional face of the input 3D complex that contains the vertex.}
\ccGlue
\ccMethod{void set_dimension(int);}{Sets the dimension of the lowest dimensional face of the input 3D complex that contains the vertex.}

\ccMethod{Index index();}{Returns the index of the lowest dimensional face of the input 3D complex that contains the vertex.}
\ccGlue
\ccMethod{void set_index(Index);}{Sets the index of the lowest dimensional face of the input 3D complex that contains the vertex.}

\ccHeading{Internal}

\ccMethod{FT meshing_info() const;}{}
\ccGlue
\ccMethod{void set_meshing_info(FT);}{}
{These functions are used internally by mesh optimizers. The user is
not encouraged to use them directly as they may change in the future.}

\ccHasModels
\ccc{Mesh_vertex_base_3<MD,Gt,Vb>}


\ccSeeAlso

\ccRefIdfierPage{CGAL::make_mesh_3} \\
\ccRefIdfierPage{CGAL::refine_mesh_3} \\
\ccRefConceptPage{MeshDomain_3}

\end{ccRefConcept}

% +------------------------------------------------------------------------+
%%RefPage: end of main body, begin of footer
\ccRefPageEnd
% EOF
% +------------------------------------------------------------------------+

