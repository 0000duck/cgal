% +------------------------------------------------------------------------+
% | Reference manual page: Mesh_polyhedron_3.tex
% +------------------------------------------------------------------------+
% | 28.07.2009   Stephane Tayeb
% | Package: Mesh_3
% |
\RCSdef{\RCSMeshpolyhedronRev}{$Id$}
\RCSdefDate{\RCSMeshpolyhedronDate}{$Date$}
% |
\ccRefPageBegin
%%RefPage: end of header, begin of main body
% +------------------------------------------------------------------------+


\begin{ccRefClass}{Mesh_polyhedron_3<Gt>}  %% add template arg's if necessary

%% \ccHtmlCrossLink{}     %% add further rules for cross referencing links
%% \ccHtmlIndexC[class]{} %% add further index entries

\ccDefinition
  
The class \ccRefName\ provides a customized \ccc{CGAL::Polyhedron_3} type. This type uses
as \ccc{PolyhedronItems_3} a customized type which adds data to the Vertex, Face and
Halfedge class. Those data are required to use our sharp features
detection algorithm.

\ccInclude{CGAL/Mesh_polyhedron_3.h}

\ccParameters

Template parameter \ccc{Gt} stands for the geometric traits associated
to the meshing process. It should be a model of concept \ccc{PolyhedronTraits_3}.

\ccTypes

\ccNestedType{type}{\ccc{CGAL::Polyhedron_3<Gt>} type with customized \ccc{PolyhedronItems_3}
designed to handle sharp feature detection.}

\ccSeeAlso

\ccRefIdfierPage{CGAL::Polyhedral_mesh_domain_with_features_3<IGT>}

\end{ccRefClass}

% +------------------------------------------------------------------------+
%%RefPage: end of main body, begin of footer
\ccRefPageEnd
% EOF
% +------------------------------------------------------------------------+

