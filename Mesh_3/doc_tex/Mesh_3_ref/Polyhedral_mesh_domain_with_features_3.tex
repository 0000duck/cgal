% +------------------------------------------------------------------------+
% | Reference manual page: Polyhedral_mesh_domain_with_features_3.tex
% +------------------------------------------------------------------------+
% | 28.07.2009   Stephane Tayeb
% | Package: Mesh_3
% |
\RCSdef{\RCSPolyhedralmeshdomainwithfeaturesRev}{$Id$}
\RCSdefDate{\RCSPolyhedralmeshdomainwithfeaturesDate}{$Date$}
% |
\ccRefPageBegin
%%RefPage: end of header, begin of main body
% +------------------------------------------------------------------------+


\begin{ccRefClass}{Polyhedral_mesh_domain_with_features_3<IGT>}  %% add template arg's if necessary

%% \ccHtmlCrossLink{}     %% add further rules for cross referencing links
%% \ccHtmlIndexC[class]{} %% add further index entries

\ccDefinition

The class \ccRefName\ implements a domain whose 
boundary is a simplicial polyhedral surface.
This surface must be closed and free of intersection.
It is a model of the concept \ccc{MeshDomainWithFeatures_3}. It also
provides a member function to automatically detect sharp features from
the input polyhedral surface.


\ccParameters

The parameter \ccc{IGT} stands for a geometric traits class providing the types
and functors required to implement the intersection tests and intersection computations
for polyhedral boundary surfaces. This parameter has to be 
instantiated with a model of the concept \ccc{IntersectionGeometricTraits_3}.

\ccInclude{CGAL/Polyhedral_mesh_domain_with_features_3.h}

\ccIsModel

\ccc{MeshDomainWithFeatures_3}

\ccInheritsFrom
\ccc{CGAL::Mesh_domain_with_polyline_features_3<CGAL::Polyhedral_mesh_domain_3<CGAL::Mesh_polyhedron_3<IGT>::type,IGT> >}

\ccTypes

\ccNestedType{FT}{Numerical type.}

\ccCreation
\ccCreationVariable{md}  %% choose variable name

\ccConstructor{template <typename Polyhedron>
  Polyhedral_mesh_domain_with_features_3(const Polyhedron& p);}{
Constructs a \ccc{Polyhedral_mesh_domain_with_features_3} from a \ccc{Polyhedron}. 
The only requirement on type \ccc{Polyhedron} is that \ccc{CGAL::Mesh_polyhedron_3<IGT>::type} should 
be constructible from \ccc{Polyhedron}. 
No feature detection is done at this level. Note that a copy of \ccc{p} will be done.}

\ccConstructor{Polyhedral_mesh_domain_with_features_3(const std::string& filename);}{
Constructs a \ccc{Polyhedral_mesh_domain_with_features_3} from an off file. No feature
detection is done at this level.}

\ccOperations

\ccMethod{void detect_features(FT angle_bound=120);}{Detects sharp features of the internal polyhedron
and inserts them in as features of the domain. \ccc{angle_bound} gives the maximum dihedral angle
(in degrees) between two triangles of the internal polyhedron which is used to consider that the edge between
those triangles is a feature edge.}

\ccSeeAlso

\ccRefConceptPage{MeshDomainWithFeatures_3} \\
\ccRefIdfierPage{CGAL::Mesh_domain_with_polyline_features_3<MeshDomain>} \\
\ccRefIdfierPage{CGAL::Polyhedral_mesh_domain_3<Polyhedron,IGT,TriangleAccessor>} \\
\ccRefIdfierPage{CGAL::Mesh_polyhedron_3<IGT>}

\end{ccRefClass}

% +------------------------------------------------------------------------+
%%RefPage: end of main body, begin of footer
\ccRefPageEnd
% EOF
% +------------------------------------------------------------------------+

