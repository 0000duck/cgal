% +------------------------------------------------------------------------+
% | Reference manual page: MeshDomainField_3.tex
% +------------------------------------------------------------------------+
% | 28.07.2009   Stephane Tayeb
% | Package: Mesh_3
% |
\RCSdef{\RCSMeshDomainFieldRev}{$Id$}
\RCSdefDate{\RCSMeshDomainFieldDate}{$Date$}
% |
\ccRefPageBegin
%%RefPage: end of header, begin of main body
% +------------------------------------------------------------------------+


\begin{ccRefFunctionObjectConcept}{MeshDomainField_3}

%% \ccHtmlCrossLink{}     %% add further rules for cross referencing links
%% \ccHtmlIndexC[concept]{} %% add further index entries

\ccDefinition
  
The concept \ccRefName\ describes a scalar field which could be queried
at any point of the space. 



\ccTypes

\ccNestedType{FT}{Numerical type.}
\ccNestedType{Point_3}{Point type.}
\ccNestedType{Index}{Index type for points. Must match the type \ccc{MeshDomain_3::Index}.}

\ccCreationVariable{msf}  %% choose variable name
\ccOperations
The  \ccc{operator()} returns the field value at a query point.
The field  value may depend  on the query point location and/or
on the input feature including the query point.

\ccMethod{FT operator()(const Point_3& p, int dimension, const Index& index) const;}{
returns the value of the sizing field at the point \ccc{p}, 
assumed to be included in the input complex feature with  dimension \ccc{dimension}
and mesh vertex index \ccc{index}.}

\ccHasModels

\ccc{Mesh_constant_field_3<Gt,Index>}

\ccSeeAlso
\ccRefConceptPage{MeshDomain_3} \\
\ccRefConceptPage{MeshDomainWithFeatures_3} \\
\ccRefIdfierPage{CGAL::Mesh_edge_criteria_3<Tr>} \\
\ccRefIdfierPage{CGAL::Mesh_facet_criteria_3<Tr>} \\
\ccRefIdfierPage{CGAL::Mesh_cell_criteria_3<Tr>} \\


\end{ccRefFunctionObjectConcept}

% +------------------------------------------------------------------------+
%%RefPage: end of main body, begin of footer
\ccRefPageEnd
% EOF
% +------------------------------------------------------------------------+

