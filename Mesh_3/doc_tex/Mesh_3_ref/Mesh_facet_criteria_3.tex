% +------------------------------------------------------------------------+
% | Reference manual page: Mesh_facet_criteria_3.tex
% +------------------------------------------------------------------------+
% | 16.02.2009   Author
% | Package: Package
% |
\RCSdef{\RCSMeshfacetcriteriaRev}{$Id$}
\RCSdefDate{\RCSMeshfacetcriteriaDate}{$Date$}
% |
\ccRefPageBegin
%%RefPage: end of header, begin of main body
% +------------------------------------------------------------------------+


\begin{ccRefFunctionObjectClass}{Mesh_facet_criteria_3<Tr>}  %% add template arg's if necessary

%% \ccHtmlCrossLink{}     %% add further rules for cross referencing links
%% \ccHtmlIndexC[class]{} %% add further index entries

\ccDefinition
  
The class \ccRefName\ is a model of \ccc{MeshFacetCriteria_3}. 
It provides a uniform  bound for the shape criterion,
a uniform or variable sizing field
for the size criterion and/or 
a uniform or variable distance field
for the  approximation error criterion.


% \begin{itemize}
% \item a lower bound on the minimum angle in degrees of the surface
%   mesh facets.
% \item an upper bound on the radius of surface Delaunay balls. A
%   surface Delaunay ball is a ball circumscribing a facet, centered on
%   the surface and empty of vertices. Such a ball exists for each facet
%   of the current surface mesh. Indeed the current surface mesh is the
%   Delaunay triangulation of the current sampling restricted to the
%   surface which is just the set of facets in the three dimensional
%   Delaunay triangulation of the sampling that have a Delaunay surface
%   ball.
% \item an upper bound on the center-center distances of the surface
%   mesh facets. The center-center distance of a surface mesh facet is
%   the distance between the facet circumcenter and the center of its
%   surface Delaunay ball.
% \end{itemize}

\ccInclude{CGAL/Mesh_facet_criteria_3.h}

\ccParameters
The parameter \ccc{Tr} must be identical to the nested type
\ccc{Triangulation} of the instance used as model of
\ccc{MeshComplex_3InTriangulation_3}.

\ccIsModel
\ccc{MeshFacetCriteria_3}


\ccTypes

\ccTypedef{typedef Tr::Geom_traits::FT FT;}{Numerical type}

\ccCreation
\ccCreationVariable{mfc}  %% choose variable name

\ccConstructor{Mesh_facet_criteria_3(
  FT angle_bound,
  FT radius_bound,
  FT distance_bound,
  Mesh_facet_topology topology = FACET_VERTICES_ON_SURFACE);}
{Returns an object to serve as criteria for facets. The argument 
\ccc{angle_bound} is the lower bound for the angle in degrees of the
surface mesh facets. The argument \ccc{radius_bound} is a uniform upper bound
for the radius of the surface Delaunay balls. The argument
\ccc{distance_bound} is an upper bound for the center-center distances
of the surface mesh facets. \ccc{topology} is the set of topological constraints
which have to be verified by each surface facet. See
section~\ref{introsec:param} for further details.
Note that if one parameter is set to 0, then its corresponding criteria is ignored.}

\ccConstructor{
	template <class SizingField, class DistanceField>
	Mesh_facet_criteria_3(
  FT angle_bound,
  SizingField radius_bound,
  DistanceField distance_bound,
  Mesh_facet_topology topology = FACET_VERTICES_ON_SURFACE);}
{Returns an object to serve as criteria for facets. The types \ccc{SizingField} and
\ccc{DistanceField} must
be models of the concept \ccc{MeshDomainField_3}. The behavior and semantic of the arguments are the same 
as above, except that the radius and distance bound parameters are functionals instead of constants.}



\ccSeeAlso

\ccRefIdfierPage{CGAL::Mesh_facet_topology} \\
\ccRefConceptPage{MeshCriteria_3} \\
\ccRefConceptPage{MeshFacetCriteria_3} \\
\ccRefConceptPage{Mesh_criteria_3<Tr>} \\
\ccRefConceptPage{MeshDomainField_3} \\
\ccRefIdfierPage{CGAL::make_mesh_3}




\end{ccRefFunctionObjectClass}

% +------------------------------------------------------------------------+
%%RefPage: end of main body, begin of footer
\ccRefPageEnd
% EOF
% +------------------------------------------------------------------------+
