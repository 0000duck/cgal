% +------------------------------------------------------------------------+
% | Reference manual page: MeshDomain_3.tex
% +------------------------------------------------------------------------+
% | 18.05.2008   Laurent Rineau
% | Package: Mesh_3
% |
\RCSdef{\RCSMeshDomainRev}{$Id$}
\RCSdefDate{\RCSMeshDomainDate}{$Date$}
% |
\ccRefPageBegin
%%RefPage: end of header, begin of main body
% +------------------------------------------------------------------------+


\begin{ccRefConcept}{MeshDomain_3}

%% \ccHtmlCrossLink{}     %% add further rules for cross referencing links
%% \ccHtmlIndexC[concept]{} %% add further index entries

\ccDefinition
  
The concept \ccRefName\ describes the knowledge required on the
object to be discretized. 
%Its generalize the concept \ccc{SurfaceMeshTraits_3}. 
%The concept \ccRefName\ 
The concept \ccRefName\  is the concept to be used when the input
domain does not have $0$ or $1$-dimensional features that need to be
accurately approximated by the mesh. 
In such a case, the queries issued by the meshing process concern only the faces of the input domain with dimension $3$
and $2$, that are respectively called {\em subdomains} and {\em surface patches}.

More specifically the concept \ccRefName\  provides 
a method to localize a point with respect to the  input domain
and its subdomains.
Moreover, as the concept \ccc{SurfaceMeshTraits_3}, 
it also provides  
 predicates to test whether a query segment
(or a ray,  or a line) intersects the boundary of the domain or 
of the subdomains, and constructors 
 to compute some intersection point if any. 
It also includes a method able to provide
a small set of initial points on the boundary. 

In the following we consider only proper intersection with the domain and
subdomain boundaries. 
A segment, ray or line is said to intersect properly the domain boundary
(resp. a subdomain boundary)
if it includes points which are strictly inside
and strictly outside  the domain (resp. the subdomain).




% \ccRefines

% \ccc{SurfaceMeshTraits_3}

\ccTypes

\ccNestedType{R}{Geometric traits class. This type is defined to ensure compatibility with
 \ccc{CGAL::Kernel_traits<T>}.}

\ccNestedType{Point_3}{Point type.}
\ccGlue
\ccNestedType{Segment_3}{Segment type.}
\ccGlue
\ccNestedType{Ray_3}{Ray type.}
\ccGlue
\ccNestedType{Line_3}{Line type.}


\ccTypedef{ typedef CGAL::Tag_false  Has_features;} {A type to distinguish
\ccc{MeshDomain_3} models from \ccc{MeshDomainWithFeatures_3} models.}

\ccNestedType{Subdomain_index}{Type of indices for subdomains of the
  input domain. Must be a model of CopyConstructible,
  Assignable, DefaultConstructible and EqualityComparable.
  The default constructed value must match the label of the exterior of
  the domain (which contains at least the unbounded component).}
\ccGlue
\ccNestedType{Surface_patch_index}{Type of indices for surface patches
  (boundaries and interfaces) of the
  input domain. Must be a model of CopyConstructible,
  Assignable, DefaultConstructible and EqualityComparable.
  The default constructed value must be the index value assigned 
  to a non surface facet.}
\ccGlue
\ccNestedType{Index}{Type of indices to be stored at mesh vertices
to characterize the lowest dimensional face of the input complex
on which the vertex lies. Must be a model of CopyConstructible,
Assignable, DefaultConstructible and EqualityComparable.
%
% Moreover, \ccc{Index} must be constructible from \ccc{Subdomain_index} and
% from \ccc{Surface_patch_index}.
%
% It must define the following methods:\\
% \ccc{Surface_patch_index surface_patch_index()}\\
% \ccc{Subdomain_index subdomain_index()}\\
% which return respectively \ccc{Surface_patch_index} and
% \ccc{Subdomain_index} which correspond to \ccc{Index} object
% if it exists (default value otherwise).
}

%\ccNestedType{Subdomain}{Return type of \ccc{Is_in_domain} queries. Must
%  be convertible to \ccc{bool}. If it converts to \ccc{false}, the
%  query point is outside the domain. Otherwise the query point
%  subdomain index must be accessible using \ccc{operator*()}.
%  Note that \ccc{boost::optional<Subdomain_index>} is a natural model of this concept.}%boost::optional
%\ccGlue
%\ccNestedType{Surface_patch}{Return type of \ccc{Do_intersect_surface} queries. Must
%  be convertible to \ccc{bool}. If it converts to \ccc{false}, no
%  surface patch is intersected by the query. Otherwise the  index of
%  one of the intersected surface patches must be accessible using \ccc{operator*()}.
%  Note that \ccc{boost::optional<Surface_patch_index>} is a natural model of this concept.}%boost::optional
%\ccGlue
\ccTypedef{typedef CGAL::cpp11::tuple<Point_3, Index, int> Intersection;}
{Return type of \ccc{Construct_intersection} queries.
  \ccc{int} represents the
  dimension of the lower dimensional face of the input  complex on which the intersection
  point lies and \ccc{Index} is the index of this face.}

\ccNestedType{Construct_initial_points}{A function object to construct
a set of initial points on the surface of the domain. Provides the
following operators:\\
\ccc{template<typename OutputIterator>\\
 OutputIterator operator()(OutputIterator pts)}\\
\ccc{template<typename OutputIterator>\\
 OutputIterator operator()(int n, OutputIterator pts)}\\
Those two operators output a set of (\ccc{n}) surface points to the
output iterator \ccc{pts}, as objects of type \ccc{std::pair<Point_3,
  Index>}. If \ccc{n} is not given, the functor must provide enough
points to initialize the mesh generation process.}


\ccNestedType{Is_in_domain}{A function object to query whether a point is in
  the input domain or not. In the positive case, it outputs the
  subdomain which includes the query point. Provides the operator:\\
\ccc{boost::optional<Subdomain_index> operator()(Point_3 p)}}

\ccNestedType{Do_intersect_surface}{A function object which answers
  intersection queries between the surface patches of the domain and
  objects of type \ccc{Segment_3}, \ccc{Ray_3} or
  \ccc{Line_3}. Provides the operators:\\
%\ccc{Surface_patch operator()(Segment_3 s)}\\
%\ccc{Surface_patch operator()(Ray_3 r)}\\
%\ccc{Surface_patch operator()(Line_3 l)}\\
\ccc{boost::optional<Surface_patch_index> operator()(Segment_3 s)}\\
\ccc{boost::optional<Surface_patch_index> operator()(Ray_3 r)}\\
\ccc{boost::optional<Surface_patch_index> operator()(Line_3 l)}\\
The return type of the operators tell whether or not the query intersects a
surface patch. In the positive case, it provides (through operator*())  the
\ccc{Surface_patch_index} of one of the intersected surface patches.}


\ccNestedType{Construct_intersection}{A function object to construct the
  intersection between an object of type \ccc{Segment_3}, \ccc{Ray_3} or
  \ccc{Line_3} and an interface. Provides the operators:\\
\ccc{Intersection operator()(Segment_3 s)}\\
\ccc{Intersection operator()(Ray_3 r)}\\
\ccc{Intersection operator()(Line_3 l)}
\ccPrecond{do\_intersect\_surface(s/r/l) == true}}


\ccCreationVariable{domain}  %% choose variable name

\ccOperations
The following functions give access to the function objects:

\ccMethod{Construct_initial_points construct_initial_points_object();}{}
\ccGlue
\ccMethod{Is_in_domain is_in_domain_object();}{}
\ccGlue
\ccMethod{Do_intersect_surface do_intersect_surface_object();}{}
\ccGlue
\ccMethod{Construct_intersection construct_intersection_object();}{}

These methods are designed to convert indices:

\ccMethod{Index index_from_surface_patch_index(Surface_patch_index surface_patch_index);}{Returns
 the index to be stored at a vertex lying on the surface patch identified by \ccc{surface_patch_index}.}
\ccGlue
\ccMethod{Index index_from_subdomain_index(Subdomain_index subdomain_index);}{Returns
 the index to be stored at a vertex lying in  the subdomain identified by \ccc{subdomain_index}.}
\ccGlue
\ccMethod{Surface_patch_index surface_patch_index(Index index);}
{Returns the \ccc{Surface_patch_index} of the surface patch 
where lies a vertex with dimension 2 and  index  \ccc{index}.} 
\ccGlue
\ccMethod{Subdomain_index subdomain_index(Index index);}{Returns the index 
of the subdomain containing a vertex with dimension 3 and  index \ccc{index}.}

\ccHasModels
\ccc{Polyhedral_mesh_domain_3<Polyhedron,IGT,TriangleAccessor>} \\
\ccc{Implicit_mesh_domain_3<Function,BGT>} \\
\ccc{Labeled_image_mesh_domain_3<Image,BGT>}
 
\ccSeeAlso
\ccRefConceptPage{MeshVertexBase_3}\\
\ccRefConceptPage{MeshCellBase_3} \\
\ccRefIdfierPage{CGAL::make_mesh_3} \\
\ccRefIdfierPage{CGAL::refine_mesh_3}


\end{ccRefConcept}

% +------------------------------------------------------------------------+
%%RefPage: end of main body, begin of footer
\ccRefPageEnd
% EOF
% +------------------------------------------------------------------------+

