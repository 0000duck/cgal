\begin{ccRefFunctionObjectConcept}{CircularKernel::MakeXMonotone_2}

\ccCreationVariable{fo}

A model \ccVar\ of this type must provide:

\ccMemberFunction{template < class OutputIterator >
    OutputIterator
    operator()(const CircularKernel::Circular_arc_2 &ca, OutputIterator oit);}
{Splits the arc \ccc{ca} into monotone arcs that are returned through the 
output iterator.}

For the sake of completeness, the \ccc{operator()} must also be defined for a 
\ccc{Line_arc_2}. In this case, the input line arc itself is the only
arc returned through the \ccc{OutputIterator}. 

\end{ccRefFunctionObjectConcept}
\begin{ccRefFunctionObjectConcept}{CircularKernel::Intersect_2}

\ccRefines

\ccc{Kernel::Intersect_2}

\ccCreationVariable{fo}

A model \ccVar\ of this type must provide:

\ccMemberFunction{template < class OutputIterator >
    OutputIterator
    operator()(const Type1 &obj1, const Type2 &obj2,
	OutputIterator intersections);}
{Copies in the output iterator the intersection elements between the
two objects. \ccc{intersections} iterates on
elements of type \ccc{CGAL::Object}.}

where \ccc{Type_1} and \ccc{Type_2} can both be either
\begin{itemize}
\item {} \ccc{CircularKernel::Line_arc_2} or
\item {} \ccc{CircularKernel::Circle_2} or
\item {} \ccc{CircularKernel::Circular_arc_2}.
\end{itemize} 

Depending on the types \ccc{Type_1} and \ccc{Type_2}, these elements can be
assigned to 
\begin{itemize}
\item {} \ccc{std::pair<CircularKernel::Circular_arc_point_2, unsigned>},
where the unsigned integer is the multiplicity of the corresponding
intersection point between \ccc{obj_1} and \ccc{obj_2} or
\item {} \ccc{CircularKernel::Circular_arc_2} in case of an overlap.
\end{itemize} 

%\ccMemberFunction{template < class OutputIterator >
%    OutputIterator
%    operator()(const CircularKernel::Line_arc_2 &l, const Type &c, 
%	OutputIterator points);}
%{Copies in the output iterators the intersection points between the
%two objects, sorted lexicographically. \ccc{points} iterates on
%elements of type \ccc{std::pair<CircularKernel::Circular_arc_point_2,
%int>}, where the integer is the multiplicity of the corresponding
%intersection point between \ccc{l} and \ccc{c}. }
%and
%\ccMemberFunction{template < class OutputIterator >
%    OutputIterator
%    operator()(const Type &c, const CircularKernel::Line_arc_2 &l);}
%{same}

%where \ccc{Type} is either \ccc{CircularKernel::Circle_2} or
%\ccc{CircularKernel::Circular_arc_2}. 

%\ccMemberFunction{template < class OutputIteratorPoints, 
%				class OutputIteratorArcs >
%    std::pair< OutputIteratorPoints, OutputIteratorArcs >
%    operator()(const Type1 &obj1, const Type2 &obj2, 
%	OutputIteratorPoints points, OutputIteratorArcs arcs);}
%{Copies in the output iterators the intersections between the two
%objects, sorted lexicographically. \ccc{points} iterates on elements
%of type \ccc{std::pair<CircularKernel::Circular_arc_point_2, int>},
%where the integer is the multiplicity of the corresponding
%intersection point between \ccc{obj1} and \ccc{obj2}, and \ccc{arcs}
%iterates on elements of type \ccc{CircularKernel::Circular_arc_2} %(or
%%\ccc{CircularKernel::Circle_2} in the special case of two equal input circles)
%that are arcs on which \ccc{obj1} and \ccc{obj2} overlap.} 

%where \ccc{Type1} and \ccc{Type2} can both be either
%\ccc{CircularKernel::Circle_2} or
%\ccc{CircularKernel::Circular_arc_2}. 

\end{ccRefFunctionObjectConcept}

\begin{ccRefFunctionObjectConcept}{CircularKernel::Split_2}

\ccCreationVariable{fo}

A model \ccVar\ of this type must provide:

\ccMemberFunction{void operator()
	(const CircularKernel::Circular_arc_2 &a, 
	const CircularKernel::Circular_arc_point_2 &p,
	CircularKernel::Circular_arc_2 &a1, 
	CircularKernel::Circular_arc_2 &a2);}
{Splits arc $a$ at point $p$, which creates arcs $a1$ and $a2$.}

\ccMemberFunction{void operator()
	(const CircularKernel::Line_arc_2 &l, 
	 const CircularKernel::Circular_arc_point_2 &p,
	 CircularKernel::Line_arc_2 &l1, CircularKernel::Line_arc_2 &l2);}
{Same for a line arc.}

\end{ccRefFunctionObjectConcept}
