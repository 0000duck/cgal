\begin{ccRefConcept}{AlgebraicKernelForCircles::ConstructPolynomial_1_2}

\ccDefinition

\ccCreationVariable{fo}

A model \ccVar\ of this type must provide:

\ccMethod{AlgebraicKernelForCircles::Polynomial_1_2
	operator()(const AlgebraicKernelForCircles::RT &a,
		const AlgebraicKernelForCircles::RT &b,
		const AlgebraicKernelForCircles::RT &c);}
{Constructs polynomial \ccc{ax+by+c}.}

\end{ccRefConcept}
\begin{ccRefConcept}{AlgebraicKernelForCircles::ConstructPolynomialForCircles_2_2}

\ccCreationVariable{fo}

A model \ccVar\ of this type must provide:

\ccMethod{AlgebraicKernelForCircles::PolynomialForCircles_2_2
	operator()(const AlgebraicKernelForCircles::FT a,
		const AlgebraicKernelForCircles::FT b,
		const AlgebraicKernelForCircles::FT rsq);}
{Constructs polynomial \ccc{(x-a)^2 + (y-b)^2 - rsq}.}

\end{ccRefConcept}

\begin{ccRefConcept}{AlgebraicKernelForCircles::Solve}

\ccDefinition

\ccCreationVariable{fo}

A model \ccVar\ of this type must provide:

\ccMethod{template < class OutputIterator >
    OutputIterator
    operator()(const AlgebraicKernelForCircles::Polynomial_1_2 &p1,
	     const AlgebraicKernelForCircles::Polynomial_1_2 &p2,
	     OutputIterator res);}
{Copies in the output iterator the common roots of \ccc{p1} and \ccc{p2}, 
with their multiplicity, as objects of type 
\ccc{std::pair< AlgebraicKernelForCircles::RootForCircles_2_2, int>}.} 

\ccMethod{template < class OutputIterator >
    OutputIterator
    operator()(const AlgebraicKernelForCircles::Polynomial_1_2 &p1,
	     const AlgebraicKernelForCircles::PolynomialForCircles_2_2 &p2,
	     OutputIterator res);}
{Same as previous.} 

\ccMethod{template < class OutputIterator >
    OutputIterator
    operator()(const AlgebraicKernelForCircles::PolynomialForCircles_2_2 &p1,
	     const AlgebraicKernelForCircles::Polynomial_1_2 &p2,
	     OutputIterator res);}
{Same as previous.} 

\ccMethod{template < class OutputIterator >
    OutputIterator
    operator()(const AlgebraicKernelForCircles::PolynomialForCircles_2_2 &p1,
	     const AlgebraicKernelForCircles::PolynomialForCircles_2_2 &p2,
	     OutputIterator res);}
{Same as previous.} 

\ccHasModels

\ccc{Algebraic_kernel_for_circles_2_2::Solve;}

\ccSeeAlso

\ccRefIdfierPage{CGAL::solve}

\end{ccRefConcept}

\begin{ccRefConcept}{AlgebraicKernelForCircles::XCriticalPoints}

\ccDefinition

\ccCreationVariable{fo}

A model \ccVar\ of this type must provide:

\ccMethod{template < class OutputIterator >
    OutputIterator
    operator()(const AlgebraicKernelForCircles::PolynomialForCircles_2_2 &p,
	     OutputIterator res);}
{Copies in the output iterator the \ccc{x}-critical points of polynomial 
\ccc{p}, as objects of type \ccc{AlgebraicKernelForCircles::RootForCircles_2_2}.} 

\ccMethod{template < class OutputIterator >
    AlgebraicKernelForCircles::RootForCircles_2_2
    operator()(const AlgebraicKernelForCircles::PolynomialForCircles_2_2 &p,
	     bool i);}
{Computes the \ccc{i}th \ccc{x}-critical point of polynomial \ccc{p}.} 

\ccHasModels

\ccc{Algebraic_kernel_for_circles_2_2::X_critical_points;}

\ccSeeAlso

\ccRefIdfierPage{CGAL::x_critical_points}

\end{ccRefConcept}
\begin{ccRefConcept}{AlgebraicKernelForCircles::YCriticalPoints}

\ccDefinition

\ccCreationVariable{fo}

A model \ccVar\ of this type must provide:

\ccMethod{template < class OutputIterator >
    OutputIterator
    operator()(const AlgebraicKernelForCircles::PolynomialForCircles_2_2 &p,
	     OutputIterator res);}
{Copies in the output iterator the \ccc{y}-critical points of polynomial 
\ccc{p}, as objects of type \ccc{AlgebraicKernelForCircles::RootForCircles_2_2}.} 

\ccMethod{template < class OutputIterator >
    AlgebraicKernelForCircles::RootForCircles_2_2
    operator()(const AlgebraicKernelForCircles::PolynomialForCircles_2_2 &p,
	     bool i);}
{Computes the \ccc{i}th \ccc{y}-critical point of polynomial \ccc{p}.} 

\ccHasModels

\ccc{Algebraic_kernel_for_circles_2_2::Y_critical_points;}

\ccSeeAlso

\ccRefIdfierPage{CGAL::y_critical_points}

\end{ccRefConcept}
