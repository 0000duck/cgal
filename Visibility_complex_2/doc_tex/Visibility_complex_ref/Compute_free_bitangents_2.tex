% +------------------------------------------------------------------------+
% | Reference manual page: Visibility_complex.tex
% +------------------------------------------------------------------------+
% | Package: Visibility_complex
% +------------------------------------------------------------------------+

\ccRefPageBegin

% +----------------------------------------------------------------------------+
\begin{ccRefFunctionObjectClass}{Compute_free_bitangents_2<VC2GeomTraits,VC2Items>}
% +----------------------------------------------------------------------------+

% +----------------------------------------------------------------------------+
\ccDefinition
  
The class \ccRefName{} computes the free bitangents of a scene of convex
objects and constraints. The first template parameter is a geometric traits
class, model of concept \ccc{VC2GeomTraits}, and the second, defaulting to
\ccc{Visibility_complex_2_items}, is a model of \ccc{VC2Items}.

\ccInclude{CGAL/Visibility_complex_2.h}

\ccTypes
\ccTwo{VisibilityComplex::Vertex_iterator}{}

\ccNestedType{Gt}{the geometric traits class.}
\ccGlue
\ccNestedType{Bitangent_2}{Bitangent type from \ccc{Gt}. Model of
\ccc{VC2Vertex}.}
\ccGlue
\ccTypedef{typedef std::pair<typename Bitangent_2::Type,std::pair<size_t,size_t> > Constraint_input;}
{Used to input the constraints. Gives the type of theconstraint and the
indexes in the sequence of disks of the source and target disk.}

\ccOperations
\ccMethod
{template <class DiskIterator, class ConstraintIterator, class OutputIterator>
  OutputIterator operator()
    (DiskIterator first, DiskIterator last,
     ConstraintIterator firstc, ConstraintIterator lastc,
     OutputIterator out); const}
{Computes the free bitangents of the scene passed as argument, and outputs
them through the output iterator, which should receive values of type
\ccc{Bitangent_2}. The input iterator for the disks should be a forward
iterator satisfying the extra requirement that a pointer obtained by
\ccc{operator->()} remains valid after a \ccc{++}. The input iterator for
constraints should yield values of type \ccc{Constraint_input}.}
\ccGlue
\ccMethod
{template <class DiskIterator, class OutputIterator>
  OutputIterator operator()
    (DiskIterator first, DiskIterator last,
     OutputIterator out); const}{Same as above, without constraints.}
\ccTagDefaults


\ccImplementation

This class maintains a small subset of the visibility complex, and outputs
copies of the bitangents as it computes them. If the output iterator does
not store the bitangents, the memory used will be linear in the size of the
scene.

\ccSeeAlso

\ccRefConceptPage{VC2Items}\\
\ccRefIdfierPage{CGAL::Visibility_complex_2<VC2GeomTraits,VC2Items>}\\
\ccRefIdfierPage{CGAL::Visibility_complex_2_segment_traits<Kernel,DistanceNT,RToDistanceNT>}\\
\ccRefIdfierPage{CGAL::Visibility_complex_2_point_traits<Kernel,DistanceNT,RToDistanceNT>}\\
\ccRefIdfierPage{CGAL::Visibility_complex_2_polygon_traits<Kernel,Container,DistanceNT,RToDistanceNT>}\\
\ccRefIdfierPage{CGAL::Visibility_complex_2_circle_traits<Kernel>}\\
\ccRefIdfierPage{CGAL::Visibility_complex_2_items}


\end{ccRefFunctionObjectClass} 
