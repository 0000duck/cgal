% +------------------------------------------------------------------------+
% | CEP User Manual:  Leda_rat_kernel.tex
% +------------------------------------------------------------------------+


\chapter{LEDA Rational kernel traits}
\label{chapterLeda_rat_kernel}
\ccChapterAuthor{Matthias B\"asken}

%---------------------------------------------
% Introduction
%---------------------------------------------
\section{Introduction}

The Computational Geometry Algorithms Library (CGAL) defines the concept of a geometry
kernel. Such a kernel provides types, construction objects and generalized
predicates. Most implementations of CG algorithms and data structures 
in the basic library of CGAL were done in a way that the classes or functions can be
parametrized with a geometric traits class. 

In most cases this geometric traits class must be a model of a CGAL geometry kernel
(but there are some exceptions).

The CGAL distribution comes with a number of models (or geometry kernels), for instance
the cartesian kernel (\ccc{CGAL::Cartesian}) or the homogeneous kernel (\ccc{CGAL::Homogeneous}).

This package provides another model for a CGAL geometry kernel - the rational LEDA kernel.
This model uses the classes and functions of the rational geometry kernel of the LEDA
library.

LEDA (Library of Efficient Data types and Algorithms) is a popular software library
for combinatorial and geometric computing and contains two geometry kernels,
a floating-point kernel (cartesian double coordinates) and a rational kernel (homogeneous LEDA integer
coordinates). CGAL already comes with support for LEDA visualization classes (\ccc{GeoWin} and
\ccc{leda_window}) and for the LEDA number types (for example the integer or rational numbers
of the LEDA library). This package provides support for the LEDA rational kernel, so that
many packages of the basic library can be used directly with geometric LEDA types and functions.
The LEDA rational kernel guarantees robustness (the floating-point kernel does not), so in most cases
this is the recommended geometry kernel when you use LEDA. This is also the reason why this
package supports the rational kernel and not the floating-point kernel.

If you want to use this package you must have CGAL (version 2.4) and LEDA (version 4.3 or 4.4) 
installed. The package only consists of header files, so you don't have to build an additional
library.

The package supports the two- and three-dimensional part of the CGAL kernel concept. The d-dimensional
part is not supported because LEDA has no d-d kernel.

Some benefits of the LEDA rational kernel traits are:
\begin{itemize}
\item easy interfacing of CGAL and LEDA geometry components
\item fast geometric predicates (the LEDA rational kernel uses floating-point filtering)
\item developers of CGAL or CGALized code can use the package for testing kernel traits compatibility of
      their code with another kernel model (very useful to find little incompatibilities)
\item LEDA users get access to a number of algorithms and data structures not present in LEDA, and these
      can be used with existing LEDA types
\end{itemize}


\section{Examples}

Two example programs show the usage of the LEDA rational kernel traits. The first example uses this traits
class for computing the two-dimensional convex hull of some points. The second example shows  the usage
of the LEDA rational kernel traits with the polyhedral surface package of CGAL. This package needs the 3-d
part of a kernel, so we give an example for using the 3-d LEDA rational kernel with a CGAL package.
Both programs can be found in the demo directory of the LEDA rational kernel traits package.

\subsection{Two-dimensional convex hull}

The example program for 2-d convex hull computation uses the \ccc{GeoWin} visualization library for
graphical output and input. 
The example has two scenes of geometric objects. One scene ($my\_scene$) contains the input point set, the other
scene is dependent from this scene and contains the output (the convex hull of the input point set).
An update object of type $geo\_hull$ is used for recomputing the output when the content of the input scene
changes.
\ccHtmlLinksOff

{\bf CGAL\_convex\_hull\_2.C :}

\begin{ccExampleCode}
#include <CGAL/basic.h>
#include <CEP/Leda_rat_kernel/leda_rat_kernel_traits.h>
#include <CEP/Leda_rat_kernel/geowin_leda_rat_kernel.h>
#include <CGAL/convex_hull_2.h>
#include <CGAL/geowin_support.h>

#if defined(LEDA_NAMESPACE)
using namespace leda;
#endif

typedef CGAL::leda_rat_kernel_traits      K;
typedef K::Point_2                        Point;
typedef K::Segment_2                      Segment;

class geo_hull : public geowin_update<std::list<Point>, std::list<Segment> >
{
public:
 void update(const std::list<Point>& L, std::list<Segment>& Sl)
 {
  Sl.clear();
  std::list<Point> out;
  
  K traits;
  CGAL::convex_hull_points_2(L.begin(),L.end(), std::back_inserter(out), traits);   

  if( out.size() > 1 ) {
    Point pakt,prev,pstart;

    std::list<Point>::const_iterator it=out.begin();
    prev= *it; pstart=prev;
    it++;

    for(; it != out.end(); ++it) {
       pakt= *it;
       Sl.push_back(Segment(prev,pakt));
       prev=pakt;
    }
    Sl.push_back(Segment(pakt,pstart));
  }
 }
 
};

int main()
{
  geowin_init_default_type((std::list<Point>*)0, leda_string("LEDA-rat_point"));
 
  std::list<Point> L;

  GeoWin GW("2d convex hull");
  geo_hull update_obj;
  geo_scene my_scene= GW.new_scene(L);  
  geo_scene result  = GW.new_scene(update_obj,my_scene,leda_string("Convex Hull")); 
  GW.set_visible(result,true);
 
  GW.edit(my_scene);
  
  return 0;  
}
\end{ccExampleCode}

\ccHtmlLinksOn

\subsection{Polyhedral surfaces example}

This example is a version of the example program $polyhedron\_prog\_simple.C$ from the Polyhedron examples
of CGAL, but this time we use the LEDA rational kernel (the original example program uses a CGAL kernel).
Please note that we define a macro \ccc{CGAL\_PROVIDE\_LEDA\_RAT\_KERNEL\_TRAITS\_3} at the beginning of the
program. This macro indicates that we want to use not only the 2-d part of the LEDA rational kernel
traits, but also the 3-d part.
If you want to use the 3-d part of the LEDA rational kernel traits you have to define this macro before
including \ccc{leda\_rat\_kernel\_traits.h} and you have to link with the 3-d library of LEDA ($-lD3$ on
UNIX systems).
\ccHtmlLinksOff

{\bf polyhedron\_prog\_simple\_leda.C :}

\begin{ccExampleCode}
#define CGAL_PROVIDE_LEDA_RAT_KERNEL_TRAITS_3

#include <CGAL/Cartesian.h>
#include <CEP/Leda_rat_kernel/leda_rat_kernel_traits.h>
#include <CGAL/Polyhedron_3.h>

#if defined(LEDA_NAMESPACE)
using namespace leda;
#endif

typedef CGAL::leda_rat_kernel_traits       Kernel;

typedef CGAL::Polyhedron_3<Kernel>         Polyhedron;
typedef Polyhedron::Halfedge_handle        Halfedge_handle;

int main() {
    Polyhedron P;
    Halfedge_handle h = P.make_tetrahedron();
    if ( P.is_tetrahedron(h))
        return 0;
    return 1;
}
\end{ccExampleCode}

\ccHtmlLinksOn
