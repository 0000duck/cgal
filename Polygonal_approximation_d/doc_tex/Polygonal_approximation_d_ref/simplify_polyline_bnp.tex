% +------------------------------------------------------------------------+
% | Reference manual page: simplify_polyline_bnp.tex
% +------------------------------------------------------------------------+
% | 03.10.2005   Ovidiu Grigore
% | Package: Polygonal_approximation_d
% | 
% +------------------------------------------------------------------------+


\begin{ccRefFunction}{simplify_polyline_bound_number_of_points}

\ccDefinition
  
The function \ccRefName\ simplifies a
sequence of points that forms a simple polyline or polygon. A
function argument specifies the desired number of
points of the resulting polyline.

The template parameter \ccc{Method} may be one of
\ccc{Dynamic_programming}, \ccc{Graph_search},
\ccc{Convex_hull_graph_search}, \ccc{Iterative_graph_search}, or
\ccc{Recursive_split}.  The first four methods are optimal, that is the
obtained result is a minimum error approximation of the input curve.

The template parameter \ccc{ErrorAssessment} must be \ccc{Local} or \ccc{Global}.
In the local approach the error evaluates the local
distortion between the input polyline and the simplified polyline,
being computed as the maximum value of the local errors measured
between each approximation line segment and its corresponding input
curve fragment.
In the global approach the error evaluates the
global distortion between the input polyline and the simplified
polyline, being computed as the sum of all local errors measured
between each approximation line segment and its corresponding input
curve fragment.

The \ccc{Convex_hull_graph_search} method only supports \ccc{Local},
the \ccc{Iterative_graph_search} method olny \ccc{Global}.

\ccInclude{CGAL/simplify_polyline.h}

\ccFunction{template<class Method, class ErrorAssessment, class Error, class InputIterator,class OutputIterator>
OutputIterator simplify_polyline_bound_number_of_points(InputIterator begin, 
                                                        InputIterator beyond, 
                                                        std::size_t n_pt_bound, 
                                                        Error::FT &error, 
                                                        OutputIterator result);}
            {simplifies the polyline given by an iterator range and writes the simplified
             polyline with \ccc{pt_bound} points into \ccc{result}. After the function call 
             the reference parameter \ccc{error} holds the maximal error.}



\ccHeading{Requirements}
\begin{enumerate}
   \item    The value type of \ccc{InputIterator} and \ccc{OutputIterator}
            are equivalent to \ccc{Error::Point}.
   \item    \ccc{Error} must be a model of the concept \ccc{Error}. For
            the method \ccc{RecursiveSplit} it must provide the operator
            which computes a split point. For the methods \ccc{Convex_hull_graph_search}
            and \ccc{Iterative_graph_search} the traits classes must be
            \ccc{Convex_hull_graph_search_error_2} and \ccc{Iterative_graph_search_error_2},
	    respectively.
\end{enumerate}

\ccSeeAlso

\ccRefIdfierPage{CGAL::simplify_polyline_bound_error} \\
\ccRefIdfierPage{CGAL::Error} 

\ccImplementation


In the \ccc{Convex_hull_graph_search} method, the maximum Euclidean
distances are measured using the method of Toussaint \cite{t-capcp-85} based on the
on-line convex hull algorithm. With these distances, it is built the
graph \ccc{G(C,error_bound)} that contains only the arcs corresponding to those
segments that approximate the curve portion with an error less then
the input \ccc{error_bound}. The optimal polygonal approximation that
minimizes the number of the vertices corresponds to the shortest path
linking the starting and the ending nodes in \ccc{G(C,error_bound)}. It can be found
in linear time using the breadth-first search procedure. Using this
method, the solution can be found in $O(n^2 \log n)$ time.

In the \ccc{Iterative_graph_search} method, the Euclidean distances
are measured using the incremental technique introduced by Perez and
Vidal \cite{cgal:pv-opadc-94}. With these distances, it is built the graph \ccc{G(C,error)},
containing only the arcs corresponding to those segments that
approximate the curve portion with an error less then a threshold. As
at the beginning the value of \ccc{error} is unknown, its value is adaptively
modified, starting with the largest value, which is obtained when the
approximation is the segment linking the two end points of the
processed portion of curve, and decreasing it step by step to obtain
the optimal approximation error in the end of the processing
procedure. The optimal polygonal approximation that minimizes the sum
of the distances corresponds to that path linking the starting and the
ending nodes in \ccc{G(C,error)}, which has the length equal to the desired
number of vertices and cumulates along it the minimum error. Using
this method, the solution can be found in $O(n^3)$ time.

\end{ccRefFunction}

% +------------------------------------------------------------------------+
%%RefPage: end of main body, begin of footer
% EOF
% +------------------------------------------------------------------------+

