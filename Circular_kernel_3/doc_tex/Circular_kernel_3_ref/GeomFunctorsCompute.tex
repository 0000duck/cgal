\begin{ccRefFunctionObjectConcept}{SphericalKernel::ComputeAreaDividedByPi_3}
\ccCreationVariable{fo}

TO BE MOVED TO KERNEL

An object \ccVar\ of this type must provide:

\ccMemberFunction{SphericalKernel::Algebraic_kernel::FT
	operator()(const Circle_3 & c);}
{Computes the area of the circle, divided by $\pi$.}

\ccSeeAlso

\ccRefIdfierPage{SphericalKernel::ComputeApproximateArea_3}

\end{ccRefFunctionObjectConcept}
\begin{ccRefFunctionObjectConcept}{SphericalKernel::ComputeSquaredLengthDividedByPiSquare_3}
\ccCreationVariable{fo}

TO BE MOVED TO KERNEL

An object \ccVar\ of this type must provide:

\ccMemberFunction{SphericalKernel::Algebraic_kernel::FT
	operator()(const Circle_3 & c);}
{Computes the squared length of the circle, divided by $\pi^2$.}

\ccSeeAlso

\ccRefIdfierPage{SphericalKernel::ComputeApproximateSquaredLength_3}

\end{ccRefFunctionObjectConcept}
\begin{ccRefFunctionObjectConcept}{SphericalKernel::ComputeApproximateArea_3}
\ccCreationVariable{fo}

TO BE MOVED TO KERNEL

An object \ccVar\ of this type must provide:

\ccMemberFunction{double
	operator()(const Circle_3 & c);}
{Computes an approximation of the area of the circle.}

\ccSeeAlso

\ccRefIdfierPage{SphericalKernel::ComputeAreaDividedByPi_3}

\end{ccRefFunctionObjectConcept}
\begin{ccRefFunctionObjectConcept}{SphericalKernel::ComputeApproximateSquaredLength_3}
\ccCreationVariable{fo}

TO BE MOVED TO KERNEL

An object \ccVar\ of this type must provide:

\ccMemberFunction{double
	operator()(const Circle_3 & c);}
{Computes an approximation of the squared length of the circle.}

\ccMemberFunction{double
	operator()(const Circular_arc_3 & c);}
{Computes an approximation of the squared length of the arc.}

\ccSeeAlso

\ccRefIdfierPage{SphericalKernel::ComputeSquaredLengthDividedByPiSquare_3}

\end{ccRefFunctionObjectConcept}
\begin{ccRefFunctionObjectConcept}{SphericalKernel::ComputeApproximateAngle_3}
\ccCreationVariable{fo}

TO BE MOVED TO KERNEL

An object \ccVar\ of this type must provide:

\ccMemberFunction{double
	operator()(const Circular_arc_3 & c);}
{Computes an approximation of the angle of the arc in radians.}

\end{ccRefFunctionObjectConcept}
