\begin{ccRefFunctionObjectConcept}{SphericalKernel::ComputeAreaDividedByPi}
\ccCreationVariable{fo}

An object \ccVar\ of this type must provide:

\ccMemberFunction{SphericalKernel::Algebraic_kernel::FT
	operator()(const Circle_3 & c);}
{Computes the area of the circle, divided by $\pi$.}

\end{ccRefFunctionObjectConcept}
\begin{ccRefFunctionObjectConcept}{SphericalKernel::ComputeSquaredLengthDividedByPiSquare}
\ccCreationVariable{fo}

An object \ccVar\ of this type must provide:

\ccMemberFunction{SphericalKernel::Algebraic_kernel::FT
	operator()(const Circle_3 & c);}
{Computes the squared length of the circle, divided by $\pi^2$.}

\end{ccRefFunctionObjectConcept}
\begin{ccRefFunctionObjectConcept}{SphericalKernel::ComputeApproximateArea}
\ccCreationVariable{fo}

An object \ccVar\ of this type must provide:

\ccMemberFunction{double
	operator()(const Circle_3 & c);}
{Computes an approximation of the area of the circle.}

\end{ccRefFunctionObjectConcept}
\begin{ccRefFunctionObjectConcept}{SphericalKernel::ComputeApproximateSquaredLength}
\ccCreationVariable{fo}

An object \ccVar\ of this type must provide:

\ccMemberFunction{double
	operator()(const Circle_3 & c);}
{Computes an approximation of the squared length of the circle.}

\ccMemberFunction{double
	operator()(const Circular_arc_3 & c);}
{Computes an approximation of the squared length of the arc.}

\end{ccRefFunctionObjectConcept}
\begin{ccRefFunctionObjectConcept}{SphericalKernel::ComputeApproximateAngle}
\ccCreationVariable{fo}

An object \ccVar\ of this type must provide:

\ccMemberFunction{double
	operator()(const Circular_arc_3 & c);}
{Computes an approximation of the angle of the arc.}

\end{ccRefFunctionObjectConcept}
