\begin{ccRefFunctionObjectConcept}{SphericalKernel::ComputeCircularX_3}

\ccCreationVariable{fo}

\ccMemberFunction{SphericalKernel::Root_of_2 
		 operator()(const SphericalKernel::Circular_arc_point_3 & p) const;}
       {returns the $x$-coordinate of the point \ccc{p}.}

\end{ccRefFunctionObjectConcept}
\begin{ccRefFunctionObjectConcept}{SphericalKernel::ComputeCircularY_3}

\ccCreationVariable{fo}

\ccMemberFunction{SphericalKernel::Root_of_2 
		 operator()(const SphericalKernel::Circular_arc_point_3 & p) const;}
       {returns the $y$-coordinate of the point \ccc{p}.}

\end{ccRefFunctionObjectConcept}
\begin{ccRefFunctionObjectConcept}{SphericalKernel::ComputeCircularZ_3}

\ccCreationVariable{fo}

\ccMemberFunction{SphericalKernel::Root_of_2 
		 operator()(const SphericalKernel::Circular_arc_point_3 & p) const;}
       {returns the $z$-coordinate of the point \ccc{p}.}

\end{ccRefFunctionObjectConcept}

\begin{ccRefFunctionObjectConcept}{SphericalKernel::ComputeApproximateSquaredLength_3}

\ccRefines

\ccc{Kernel::ComputeApproximateSquaredLength_3}

\ccCreationVariable{fo}

In addition, an object \ccVar\ of this type must provide:

\ccMemberFunction{double
	operator()(const SphericalKernel::Circular_arc_3 & a);}
{Computes an approximation of the squared length of the arc \ccc{a}.}

\end{ccRefFunctionObjectConcept}
\begin{ccRefFunctionObjectConcept}{SphericalKernel::ComputeApproximateAngle_3}
\ccCreationVariable{fo}

An object \ccVar\ of this type must provide:

\ccMemberFunction{double
	operator()(const SphericalKernel::Circular_arc_3 & a);}
{Computes an approximation of the angle of the arc in radians \ccc{a}.}

\end{ccRefFunctionObjectConcept}
