\begin{ccRefFunctionObjectConcept}{SphericalKernel::Intersect_3}

\ccRefines

\ccc{Kernel::Intersect_3}

\ccCreationVariable{fo}

A model \ccVar\ of this type must provide:

\ccMemberFunction{template < class OutputIterator >
    OutputIterator
    operator()(const Type1 &obj1, const Type2 &obj2,
	OutputIterator intersections);}
{Copies in the output iterator the intersection elements between the
two objects. \ccc{intersections} iterates on
elements of type \ccc{CGAL::Object}, in lexicographic order 
when this ordering is defined on the computed objects.}

\ccMemberFunction{template < class OutputIterator >
    OutputIterator
    operator()(const Type1 &obj1, const Type2 &obj2,
	const Type3 &obj3,
	OutputIterator intersections);}
{Copies in the output iterator the intersection elements between the
two objects. \ccc{intersections} iterates on
elements of type \ccc{CGAL::Object}, in lexicographic order 
when this ordering is defined on the computed objects.}

For the \textbf{first operator}, \ccc{Type1} and \ccc{Type2} can both 
be either

\textbf{!!!!!!! we coded here the intersection \ccc{Line_3}-\ccc{Line_3}, 
since it was apparently forgotten in \ccc{CGAL::Kernel}, this code
should be moved there}

\begin{itemize}
\item {} \ccc{SphericalKernel::Sphere_3},
\item {} \ccc{SphericalKernel::Plane_3},
\item {} \ccc{SphericalKernel::Line_3},
\item {} \ccc{SphericalKernel::Circle_3},
\item {} \ccc{SphericalKernel::Line_arc_3} or
\item {} \ccc{SphericalKernel::Circular_arc_3},
\end{itemize} 

and depending on the types \ccc{Type1} and \ccc{Type2}, the computed 
\ccc{CGAL::Object}s can be assigned to 
\begin{itemize}
\item {} \ccc{std::pair<SphericalKernel::Circular_arc_point_3, unsigned>},
where the unsigned integer is the multiplicity of the corresponding
intersection point between \ccc{obj_1} and \ccc{obj_2},
\item {} \ccc{Type1}, when \ccc{Type1} and \ccc{Type2} are equal, and 
if the two objets \ccc{obj1} and \ccc{obj2} are equal,
\item {} \ccc{SphericalKernel::Line_3} or \ccc{SphericalKernel::Circle_3} 
when \ccc{Type1} and \ccc{Type2} are two-dimensional objets intersecting 
along a curve (2 planes, or 2 spheres, or one plane and one sphere),
\item {} \ccc{SphericalKernel::Circular_arc_3} in case of an overlap of 
two circular arcs or
\item {} \ccc{SphericalKernel::Line_arc_3} in case of an overlap of two 
line segments. 
\end{itemize} 

For the \textbf{second operator}, \ccc{Type1}, \ccc{Type2} and \ccc{Type3}
can be either
\textbf{!!!!!!! case of 3 planes is missing, again it should be in Kernel}

\begin{itemize}
\item {} \ccc{SphericalKernel::Sphere_3} or
\item {} \ccc{SphericalKernel::Plane_3}
\end{itemize}

and depending of these types, the computed \ccc{CGAL::Object}s can be 
assigned to 
\begin{itemize}
\item {} \ccc{std::pair<SphericalKernel::Circular_arc_point_3, unsigned>},
where the unsigned integer is the multiplicity of the corresponding
intersection point,
\item {} \ccc{SphericalKernel::Circle_3} or
\item {} \ccc{Type1}, when \ccc{Type1}, \ccc{Type2} and \ccc{Type3} 
are equal, and if the three objets \ccc{obj1} and \ccc{obj2} and \ccc{obj3}
are equal.
\end{itemize} 

\end{ccRefFunctionObjectConcept}
