\begin{ccRefClass}{Circular_arc_3<SphericalKernel>}

\ccInclude{CGAL/Circular_arc_3.h}

\ccIsModel

\ccc{SphericalKernel::CircularArc_3}

\ccCreation
\ccCreationVariable{ca}

\ccThree{Circular_arc_point_3}{ca.is_x_monotone()}{}
\ccThreeToTwo

\ccConstructor{Circular_arc_3(const SphericalKernel::Circle_3 &c)}
{Constructs an arc from a full circle.}

\ccConstructor{Circular_arc_3(const SphericalKernel::Circle_3 &c,
		const SphericalKernel::Circular_arc_point_3 &p1,
		const SphericalKernel::Circular_arc_point_3 &p2)}
{Constructs the circular arc supported by \ccc{c}, that is oriented 
counterclockwise in the \textit{positive} plane containing the circle, 
whose source is \ccc{p1} and whose target is \ccc{p2}.
\ccPrecond{\ccc{p1} and \ccc{p2} lie on \ccc{c}.}}

Here, we say that a plane is \textit{positive} if its equation is 
of the form $ax+by+cz+d=0$ with $(a,b,c)>(0,0,0)$ 
(i.e. $(a>0) || (a==0) \&\& (b>0) || (a==0)\&\&(b==0)\&\&(c>0)$). 

\ccAccessFunctions

\ccThree{SphericalKernel::Circular_arc_point_3}{ca.is_x_monotone()}{}
\ccThreeToTwo

\ccMethod{SphericalKernel::Circle_3 supporting_circle();}{}

A circular arc is supposed to be oriented counterclockwise, from 
\ccc{source} to \ccc{target}. 

\ccMethod{SphericalKernel::Circular_arc_point_3 source();}{}
\ccGlue
\ccMethod{SphericalKernel::Circular_arc_point_3 target();}{}

When the methods \ccc{source} and \ccc{target} return the same point, then 
the arc is in fact a full circle. %\footnote{so far, arcs of zero length are
%points, not arcs. But see Menelaos' remark: add functors to test whether an
%arc is degenerate... to be fixed}

\ccHeading{I/O}

\ccFunction{istream& operator>> (std::istream& is, Circular_arc_3 & ca);}{}
\ccGlue
\ccFunction{ostream& operator<< (std::ostream& os, const Circular_arc_3 & ca);}{}

\ccSeeAlso

\ccRefIdfierPage{CGAL::Circular_arc_point_3<SphericalKernel>}\\
\ccRefIdfierPage{CGAL::Line_arc_3<SphericalKernel>}

\end{ccRefClass}
