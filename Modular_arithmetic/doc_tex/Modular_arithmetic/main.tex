\cleardoublepage
\ccUserChapter{Modular Arithmetic\label{chap:modular_arithmetic}}
\ccChapterAuthor{Michael Hemmer}


\begin{ccPkgDescription}{3D Convex Hulls\label{Pkg:ConvexHull3}}
\ccPkgHowToCiteCgal{cgal:hs-ch3-07}
\ccPkgSummary{This package provides functions 
for computing convex hulls in three dimensions as well as functions
for checking if sets of points are strongly convex are not. One can
compute the convex hull of a set of points in three dimensions in one
of three ways: using a static algorithm, using an incremental
construction algorithm, or using a triangulation to get a fully
dynamic computation.}

\ccPkgDependsOn{All algorithms produce as output a \ccRef[3D Polyhedron]{Pkg:Polyhedron}. 
                The dynamic algorithms depend on \ccRef[3D Triangulations]{Pkg:Triangulation3}}
\ccPkgIntroducedInCGAL{1.1}
\ccPkgLicense{\ccLicenseQPL}
\ccPkgIllustration{Convex_hull_3/bunny.png}{Convex_hull_3/bunny.png}
\end{ccPkgDescription}


\section{Introduction}

Modular arithmetic is a fundamental tool in modern algebra systems. 
In conjunction with the Chinese remainder theorem it serves as the 
workhorse in several algorithms computing the gcd, resultant etc. 
Moreover, it can serve as a very efficient filter, since it is often 
possible to exclude that some value is zero by computing its modular 
correspondent with respect to one prime only. 

First of all, this package introduces a type \ccc{CGAL::Residue}.
It represents $\Z_{/p\Z}$ for some prime $p$. 
The prime number $p$ is stored in a static member variable. 
The class provides static member functions to change this value. 
{\bf Note that changing the prime invalidates already existing objects 
of this type.}
However, already existing objects do not lose their value with respect to the 
old prime and can be reused after restoring the old prime. 
Since the type is based on double 
arithmetic the prime is restricted to values less than $2^{26}$. 
The initial value of $p$ is 67111067. 

Please note that the implementation of class \ccc{CGAL::Residue} requires that 
doubles are rounded to nearest, that is, it indispensable that the 
rounding mode is set accordingly before performing any arithmetic operations. 
For instance you may use \ccc{CGAL::Protect_FPU_rounding} with 
\ccc{CGAL_FE_TONEAREST}. 

\begin{ccAdvanced}      
In case the flag \ccc{CGAL_HAS_THREADS} 
is undefined the prime is just stored in a static member 
of the class, that is, \ccc{CGAL::Residue} is not thread-safe in this case.  
In case \ccc{CGAL_HAS_THREADS}
the implementation of the class is thread safe using 
\ccc{boost::thread_specific_ptr}. However, this may cause some performance 
penalty. Hence, it may be advisable to configure \ccc{CGAL} with 
\ccc{CGAL_HAS_NO_THREADS}. 
\end{ccAdvanced} 

Moreover, the package introduces the concept \ccc{Modularizable}. 
An algebraic structure \ccc{T} is considered as \ccc{Modularizable} if there 
is a mapping from \ccc{T} into an algebraic structure that is based on 
the type \ccc{CGAL::Residue}.  
For scalar types, e.g. Integers, this mapping is just the canonical 
homomorphism into $\Z_{/p\Z}$ represented by \ccc{CGAL::Residue}. 
For compound types, e.g. Polynomials, the mapping is applied to the 
coefficients of the compound type. 
The mapping is provided by the class \ccc{CGAL::Modular_traits<T>}.
The class \ccc{CGAL::Modular_traits<T>} is designed such that the concept 
\ccc{Modularizable} can be considered as optional, i.e., 
\ccc{CGAL::Modular_traits<T>} provides a tag that can be used for dispatching. 

\subsection{Example}

In the following example modular arithmetic is used as a filter. 
\ccIncludeExampleCode{Modular_arithmetic/modular_filter.cpp}

\section{Design and Implementation History}

The class \ccc{CGAL::Residue} is based on the C-code of Sylvain Pion et. al. 
as it was presented in \cite{bepp-sdrns-99}. 

The remaining part of the package is the result of the integration process
of the NumeriX library of EXACUS \cite{beh+-eeeafcs-05} into CGAL. 

