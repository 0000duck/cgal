\cleardoublepage
\ccUserChapter{Modular Arithmetic}
\label{chap:modular_arithmetic}
\ccChapterAuthor{Michael Hemmer}


\begin{ccPkgDescription}{3D Convex Hulls\label{Pkg:ConvexHull3}}
\ccPkgHowToCiteCgal{cgal:hs-ch3-07}
\ccPkgSummary{This package provides functions 
for computing convex hulls in three dimensions as well as functions
for checking if sets of points are strongly convex are not. One can
compute the convex hull of a set of points in three dimensions in one
of three ways: using a static algorithm, using an incremental
construction algorithm, or using a triangulation to get a fully
dynamic computation.}

\ccPkgDependsOn{All algorithms produce as output a \ccRef[3D Polyhedron]{Pkg:Polyhedron}. 
                The dynamic algorithms depend on \ccRef[3D Triangulations]{Pkg:Triangulation3}}
\ccPkgIntroducedInCGAL{1.1}
\ccPkgLicense{\ccLicenseQPL}
\ccPkgIllustration{Convex_hull_3/bunny.png}{Convex_hull_3/bunny.png}
\end{ccPkgDescription}


\section{Introduction}

This package introduces a type \ccc{CGAL::Modular} 
representing a finite field over some prime. 
This prime can be changed at runtime. From there, the type may serve
as the workhorse for algorithms base on chinese remainder.  

Moreover, the package introduces the \ccc{CGAL::Modular_traits<T>} 
providing a mapping from some algebraic strucutre \ccc{T} into algebraic 
structure that is based on the type \ccc{CGAL::Modular}.  
For scalar types, e.g. Integers, this mapping is just the kanonical homomorphism
into the type \ccc{CGAL::Modular}. 
For compount types, e.g. Polynomials, the mapping is applied to the 
coefficients of the compount type. 

\section{Software Design}

The class \ccc{CGAL::Modular_traits<T>} is designed such that the concept 
\ccc{Modularizable} can be considered as optional, i.e., 
\ccc{CGAL::Modular_traits<T>} provides a tag that can be used for dispatching. 

\subsection{Examples}

In the following example the modular arithmetic is used as a filter. 
\ccIncludeExampleCode{Modular_arithmetic/modular_filter.cpp}

