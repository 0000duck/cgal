% +------------------------------------------------------------------------+
% | Reference manual page: Convex_hull_d_ref/intro.tex
% +------------------------------------------------------------------------+

%\clearpage
%\section{Reference Pages for dD Convex Hulls and Delaunay Triangulations}
\ccRefChapter{dD Convex Hulls and Delaunay Triangulations\label{chap:convex_hull_d_ref}}
\ccChapterAuthor{Susan Hert \and Michael Seel}

A subset $S \subseteq \R^3$ is convex if for any two points $p$ and $q$
in the set the line segment with endpoints $p$ and $q$ is contained
in $S$. The convex hull\ccIndexMainItemDef{convex hull} of a set $S$ is 
the smallest convex set containing
$S$. The convex hull of a set of points $P$ is a convex 
polytope with vertices in $P$.  A point in $P$ is an extreme point 
(with respect to $P$)\ccIndexMainItemDef{extreme point} if it is a vertex 
of the convex hull of $P$.

\cgal\ provides functions for computing convex hulls in two, three 
and arbitrary dimensions as well as functions for testing if a given set of 
points in is strongly convex or not.  This chapter describes the class
available for arbitrary dimensions and its companion class for 
computing the nearest and furthest side Delaunay triangulation. 

\section{Classified Reference Pages}

\ccHeading{Concepts}

\ccRefConceptPage{ConvexHullTraits_d} \\
\ccRefConceptPage{DelaunayLiftedTraits_d} \\
\ccRefConceptPage{DelaunayTraits_d} \\

\ccHeading{Classes}

\ccRefIdfierPage{CGAL::Convex_hull_d_traits_3<R>} \\
\ccRefIdfierPage{CGAL::Convex_hull_d<R>}  \\
\ccRefIdfierPage{CGAL::Delaunay_d< R, Lifted_R >} 

\clearpage


