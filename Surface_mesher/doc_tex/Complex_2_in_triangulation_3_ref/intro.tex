\ccRefChapter{2D Complex embedded in a 3D Triangulation\label{ref_chapter_Complex2inTriangulation3}}
\label{ref_chapter_C2inT3}

\vspace*{-15mm}
\minitoc
\vspace*{30mm}

This package provides a data structure to  represent 
2D simplicial complex as a subcomplex of a 3D triangulation.
A priori the represented 2D complex is a  simplicial complex
because its faces are part of a 3D triangulation.
It does not need to be connected, nor to be a pure complex.
which  means that the complex can have antenna edges 
that is edges which do not belong to a facet or
isolated vertices that is vertices  which do not belong to
an edge.

The data structure allows to mark certain edges of the 2D complex.
This feature enables  to have  constrained edges
and to represent constrained 3D triangulations
where the constraints are described 
as a non simplicial complex like the piecewise linear complex
introduced  e.g. in \cite{s-cgehd-98}.

\subsection*{Classes}
\ccRefIdfierPage{CGAL::Complex_2_in_triangulation_3<Triangulation, Facet_Info, Edge_info>}
