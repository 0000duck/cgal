% +------------------------------------------------------------------------+
% | Reference manual page: PureComplex_2InTriangulation_3.tex
% +------------------------------------------------------------------------+
% | 07.12.2005   Author
% | Package: Package
% | 
\RCSdef{\RCSSurfaceMeshCTRev}{$Id$}
\RCSdefDate{\RCSSurfaceMeshCTDate}{$Date$}
% |
%%RefPage: end of header, begin of main body
% +------------------------------------------------------------------------+


\begin{ccRefConcept}{SurfaceMeshComplex_2InTriangulation_3}

%% \ccHtmlCrossLink{}     %% add further rules for cross referencing links
%% \ccHtmlIndexC[concept]{} %% add further index entries

\ccDefinition
  
The concept \ccRefName\ describes a data structure
designed  to represent a  two dimensional pure complex 
embedded in a three dimensional triangulation.

A {\em complex} is a set $C$ of faces such that~:\\
- any subface of a face in $C$ is a face of $C$ \\
- two faces of $C$ are disjoint or share a common subface \\
The complex is {\em two dimensional}, if its faces have dimension at most
two. It is {\em pure} if any face in the complex is a subface
of some face of maximal dimension.
Thus, a two dimensional pure complex is a set of facets
together with their edges and vertices.
A  two dimensional pure complex  embedded 
in a three dimensional triangulation
is a subset of the  facets 
of this triangulation, together with their edges and vertices.



The concept \ccRefName\  is particularly suited to handle
surface meshes obtained as the restriction to a surface of 
a  three dimensional Delaunay triangulation.
A model of this concept is a type to be plugged as first template
parameter in the global
function \ccc{make_surface_mesh<C2T3,SurfaceOracle,Criteria,Tag>}.


The concept \ccRefName\   is a simplification of more general concepts called
 respectively \ccc{PureComplex2InTriangulation3} and
\ccc{Complex2InTriangulation3}.
\ccc{PureComplex2InTriangulation3} is  designed to represent  pure complex
and \ccc{Complex2InTriangulation3} is designed to represent any
two dimensional complex. 
Both concepts include member functions to analyse the complex,
i.e. find its size, number of connected components, genus etc... 

%\ccGeneralizes

%ThisConcept \\
%ThatConcept

\ccTypes
\ccRefName\ provides the following types.

\ccNestedType{Triangulation_3}{The type of the
embedding 3D triangulation.}
\ccGlue
\ccTypedef{typedef Triangulation_3::Vertex_handle Vertex_handle;}{ The type of
the embedding triangulation vertex handles.}
\ccGlue
\ccTypedef{typedef Triangulation_3::Cell_handle  Cell_handle;}{ The  type of
the  embedding triangulation cell handles.}
\ccGlue
\ccTypedef{typedef Triangulation_3::Facet Facet;}{The  type of
the  embedding triangulation facets.}
\ccGlue
\ccTypedef{typedef Triangulation_3::Edge Edge;} {The  type of
the  embedding triangulation edges.}


\ccEnum{enum Face_status {NOT_IN_COMPLEX, BOUNDARY, REGULAR,
SINGULAR};}
{A type to describe the status of a face (facet, edge, or vertex) with respect to 
the 2D pure complex. A \ccc{NOT_IN_COMPLEX} face does not belong to the 2D complex.
Facets can only be  \ccc{NOT_IN_COMPLEX}  or \ccc{REGULAR}
depending on whether they belong to the 2D complex on not.
Edges and vertices can be  \ccc{NOT_IN_COMPLEX}, \ccc{BOUNDARY},
\ccc{REGULAR} or \ccc{SINGULAR}.
An edge in the complex is 
\ccc{BOUNDARY},
\ccc{REGULAR}, or \ccc{SINGULAR},
 if it is incident to respectively 1, 2,  or 3 or
more facets in the complex.
The status of a vertex is determined by 
the adjacency graph of the facets of the 2D complex
incident to that vertex.
The vertex  of the 2D complex is \ccc{BOUNDARY}, if this adjacency graph
is a simple path,  it is \ccc{REGULAR} if the adjacency graph is cyclic,
and \ccc{SINGULAR}in any other case.
}


\ccNestedType{Facets_iterator}{An iterator type to visit the facets
of the 2D complex.}
\ccGlue
\ccNestedType{Edges_iterator}{An iterator type to visit  the
edges of the 2D complex.}
\ccGlue
\ccNestedType{Vertices_iterator}{An iterator type to visit 
vertices of the 2D complex.}
%\ccNestedType{Facets_circulator}{An iterator type to visit the facets
%of the 2D complex around a boundary or regular vertex.}
%\ccGlue
%\ccNestedType{Edges_circulator}{An iterator type to visit  the
%edges of the 2D complex around a boundary or regular vertex.}
%\ccGlue
%\ccNestedType{Vertices_circulator}{An iterator type to visit 
%vertices of the 2D complex around a boundary or regular vertex.}




\ccCreation
\ccCreationVariable{c2t3}  %% choose variable name

%\ccConstructor{SurfaceMeshComplex_2InTriangulation_3();}{default constructor.}
%\ccGlue
\ccConstructor{SurfaceMeshComplex_2InTriangulation_3(Triangulation_3& t3);}
 {Builds an empty 2D complex embedded in the triangulation~\ccc{t3}}
\ccGlue
\ccConstructor{template < class FacetSelector>
SurfaceMeshComplex_2InTriangulation_3(Triangulation_3& t3, 
         FacetSelector select);}
{Builds a 2D complex embedded in the triangulation~\ccc{t3},
including in the 2D complex the facets of~\ccc{t3} for 
 which the predicate \ccc{select} returns \ccc{true}.\\
The type \ccc{FacetSelector} must be
a function object with an operator to select facets:
\ccc{bool operator()(Facet f);}.}


\ccHeading{Modifications}
\ccMethod{void add_to_complex(Facet f);}{Adds facet \ccc{f} to the 2D complex.}
\ccGlue
\ccMethod{void add_to_complex(Cell_handle c, int i);}
{Adds facet \ccc{(c,i)} to the 2D complex.}
\ccGlue
\ccMethod{void remove_from_complex(Facet f);}
{Removes facet \ccc{f} from  the 2D complex.}
\ccGlue
\ccMethod{void remove_from_complex(Cell_handle c, int i);}
{Removes facet \ccc{(c,i)} from the 2D complex.}

\ccHeading{Queries}
Queries on the status  of individual face with respect to the 2D complex.
\ccMethod{Face_status   face_status(Facet f);}
{Returns the status of the facet \ccc{f}  with respect to  the 2D complex.}
\ccGlue
\ccMethod{Face_status  face_status(Cell_handle c, int i);}
{Returns the status of the facet \ccc{(c,i)} with respect to  the 2D complex.}
\ccGlue
\ccMethod{Face_status  face_status(Edge e);}
{Returns the status of edge \ccc{e} in the 2D complex.}
\ccGlue
\ccMethod{Face_status  face_status(Cell_handle c, int
i, int j);}
{Returns the status of edge \ccc{(c,i,j)}  in the 2D complex.}
\ccGlue
\ccMethod{Face_status  face_status(Vertex_handle v);}
{Returns the status of vertex \ccc{v} in the 2D complex.}
\ccGlue
\ccMethod{bool is_in_complex(Facet f);}
{Returns \ccc{true}, if the facet \ccc{f} belongs to the 2D complex.}
\ccGlue
\ccMethod{bool is_in_complex(Cell_handle c, int i);}
{Returns \ccc{true}, if the facet \ccc{(c,i)} belongs to the 2D complex.}
\ccGlue
\ccMethod{bool is_in_complex(Edge e);}
{Returns \ccc{true}, if the edge  \ccc{e} belongs to the 2D complex.}
\ccGlue
\ccMethod{bool is_in_complex(Cell_handle c, int i, int j);}
{Returns \ccc{true}, if the edge \ccc{(c,i,j)} belongs to the 2D complex.}
\ccGlue
\ccMethod{bool is_in_complex(Vertex_handle v);}
{Returns \ccc{true}, if the vertex \ccc{v} belongs to the 2D complex.}
\ccGlue
\ccMethod{bool is_regular(Facet f);}
{Returns \ccc{true}, if the facet \ccc{f} belongs to the 2D complex.}
\ccGlue
\ccMethod{bool is_regular(Edge e);}
{Returns \ccc{true}, if the edge  \ccc{e} is regular.
\ccPrecond{\ccc{e} is an edge of the complex.} }
\ccGlue
\ccMethod{bool is_regular(Cell_handle c, int i, int j);}
{Returns \ccc{true}, if the edge \ccc{(c,i,j)} is regular.
\ccPrecond{\ccc{(c,i,j)} is an edge of the complex.} }
\ccGlue
\ccMethod{bool is_regular(Vertex_handle v);}
{returns \ccc{true}, if the vertex \ccc{v} is regular.
\ccPrecond{\ccc{v} is a vertex of the complex.} }
\ccGlue
\ccMethod{bool is_on_boundary(Edge e);}
{returns \ccc{true}, if the edge  \ccc{e} is on the boundary.
\ccPrecond{\ccc{e} is an edge of the complex.} }
\ccGlue
\ccMethod{bool is_on_boundary(Cell_handle c, int i, int j);}
{returns \ccc{true}, if the edge \ccc{(c,i,j)} is on the boundary.
\ccPrecond{\ccc{(c,i,j)} is an edge of the complex.} }
\ccGlue
\ccMethod{bool is_on_boundary(Vertex_handle v);}
{returns \ccc{true}, if the vertex \ccc{v} is on the boundary.
\ccPrecond{\ccc{v} is a vertex of the complex.} }
\ccGlue
\ccMethod{bool is_singular(Edge e);}
{returns \ccc{true}, if the edge  \ccc{e} is regular.
\ccPrecond{\ccc{e} is an edge of the complex.} }
\ccGlue
\ccMethod{bool is_singular(Cell_handle c, int i, int j);}
{returns \ccc{true}, if the edge \ccc{(c,i,j)} is regular.
\ccPrecond{\ccc{(c,i,j)} is an edge of the complex.} }
\ccGlue
\ccMethod{bool is_singular(Vertex_handle v);}
{returns \ccc{true}, if the vertex \ccc{v} is regular.
\ccPrecond{\ccc{v} is a vertex of the complex.} }



%Queries on the 2D complex. 

%\ccMethod{bool is_manifold();}
%{returns \ccc{true}, if the complex is manifold.}
%\ccGlue
%\ccMethod{bool is_connected_component_manifold(Facet f);}
%{Returns true, if the   
%connected component including  facet \ccc{f} is manifold.}
%\ccGlue
%\ccMethod{std::size_type number_of_connected_components();}
%{Returns the number of connected components of the
%complex.}
%\ccGlue
%\ccMethod{std::size_type 
%number_of_connected_components_of_boundary();}
%{Returns the number of connected components of the boundary
%  the complex.}
%\ccGlue
%\ccMethod{std::size_type
%number_of_connected_components_of_boundary(Facet f);}
%{Returns the number of connected components of the boundary
%of connected component of the complex including facet \ccc{f}.}
%\ccGlue
%\ccMethod{std::size_type
%number_of_connected_components_of_boundary(Cell_handle c, int i);}
%{Returns the number of connected components of the boundary
%of connected component of the complex including facet
%\ccc{(c,i)}}
%\ccGlue
%\ccMethod{std::size_type
%genus(Facet f);}
%{Returns the topological genus 
%of the connected component of the complex including facet \ccc{f}.}
%\ccGlue
%\ccMethod{std::size_type
%genus(Cell_handle c, int i);}
%{Returns the topological genus 
%of connected component of the complex including facet \ccc{(c,i)}.}





\ccHeading{Traversal of the complex}

The data structure provides iterators  to visit  
the facets, edges and vertices of the complex.
% and circulators to visit the faces
%incident to a given vertex of the complex.
All those iterators are bidirectional and
non mutable.

\ccMethod{Facets_iterator facets_begin();}
{Returns an iterator with value type \ccc{Facet} to visit the facets
of the 2D complex.}
\ccGlue
\ccMethod{Facets_iterator facets_end();}
{Returns the  past the end iterator for the above iterator.}

\ccMethod{Edges_iterator edges_begin();}
{Returns an iterator with value type \ccc{Edge} to visit the
edges of the 2D complex which are not isolated.}
\ccGlue
\ccMethod{Edges_iterator edges_end();}
{Returns the past the end iteror for the above iterator.}
\ccGlue
\ccMethod{Boundary_edges_iterator boundary_edges_begin();}
{Returns an iterator with value type \ccc{Edge} to visit the
boundary  edges of the complex.}
\ccGlue
\ccMethod{Boundary_edges_iterator boundary_edges_end();}
{Returns the past the end iterator for the above iterator.}
\ccGlue
\ccMethod{Vertices_iterator vertices_begin();}
{Returns an iterator with value type \ccc{Vertex_handle} to visit the
vertices of the 2D complex.}
\ccGlue
\ccMethod{Vertices_iterator vertices_end();}
{Returns the past the end iterator for the above iterator.}
\ccGlue
\ccMethod{Boundary_vertices_iterator 
         boundary_vertices_begin();}
{Returns an iterator with value type \ccc{Vertex_handle} to visit the 
vertices of the 2D complex that are  boundary vertices.}
\ccGlue
\ccMethod{Boundary_vertices_iterator 
 boundary_vertices_end();}
{Returns the past the end iterator for the above iterator.}


%\ccMethod{Facets_circulator
%incident_facets(Vertex_handle v);}
%{Returns a circulator to visit the facets of the complex around
%vertex \ccc{v}. Returns an empty circulator, if the vertex is 
%singular. For a boundary  vertex,  the circulator will wrap from the last
%incident facet  to the first one.}
%\ccGlue
%\ccMethod{Egdes_circulator
%incident_edges(Vertex_handle v);}
%{Returns a circulator to visit the edges of the complex around
%vertex \ccc{v}. Returns an empty circulator, if the vertex is 
%singular. For a boundary  vertex,  the circulator will wrap from the last
%incident edge  to the first one.}
%\ccGlue
%\ccMethod{Vertices_circulator
%incident_vertices(Vertex_handle v);}
%{Returns a circulator to visit the vertices of the complex around
%vertex \ccc{v}. Returns an empty circulator, if the vertex  or
%singular. For a boundary  vertex,  the circulator will wrap from the last
%incident vertex  to the first one.}

%The following function is the basic function to walk on the 2D
%complex

%\ccMethod{Facet neighbor(Facet f, int j);}
%{Returns the facet of the complex which is the neighbor of 
%the facet \ccc{f} opposite to the vertex with index \ccc{j} of
%\ccc{f}.
%The vertices of the facet \ccc{f = (cell  c, i)} are numbered
%(0,1,2) (according to the \ccc{vertex_triple_index(i,j)} member function
%of \ccc{Triangulation_3})
%in such a way that facet \ccc{f} is oriented by the
%outward normal of tetraedra \ccc{c}. 
%If there is no such neighbor, or if the edge is singular the functions returns \ccc{Facet()}.}
%\ccGlue
%\ccMethod{Facet neighbor(Cell_handle c, int i, int j);}
%{Returns the facet of the complex which is the neighbor of 
%the facet \ccc{f} opposite to the vertex with index \ccc{j} of \ccc{f}.
%See above.}




\ccHasModels

\ccc{Surface_mesh_complex_2_in_triangulation_3<Tr>}


\ccSeeAlso

%\ccc{Complex_2InTriangulation\_3}, \\
%\ccc{PureComplex_2InTriangulation_3}\\
\ccc{make_surface_mesh}.

%\ccExample

%A short example program.
%Instead of a short program fragment, a full running program can be
%included using the 
%\verb|\ccIncludeExampleCode{Package/SurfaceMeshComplex_2InTriangulation_3.C}| 
%macro. The program example would be part of the source code distribution and
%also part of the automatic test suite.

%\begin{ccExampleCode}
%void your_example_code() {
%}
%\end{ccExampleCode}

%% \ccIncludeExampleCode{Package/SurfaceMeshComplex_2InTriangulation_3.C}

\end{ccRefConcept}

% +------------------------------------------------------------------------+
%%RefPage: end of main body, begin of footer
% EOF
% +------------------------------------------------------------------------+

