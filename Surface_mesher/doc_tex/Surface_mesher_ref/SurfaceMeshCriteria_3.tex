% +------------------------------------------------------------------------+
% | Reference manual page: SurfaceMeshCriteria.tex
% +------------------------------------------------------------------------+
% | 02.12.2005   Mariette Yvinec, Laurent Rineau
% | Package: Surface_mesher
% |  ( $URL$ )
% | 
\RCSdef{\RCSSurfaceMeshCriteriaRev}{$Id$}
\RCSdefDate{\RCSSurfaceMeshCriteriaDate}{$Date$}
% |
%%RefPage: end of header, begin of main body
% +------------------------------------------------------------------------+


\begin{ccRefConcept}{SurfaceMeshCriteria_3}

\ccDefinition
  
The Delaunay refinement process involved in  the 
global function \ccc{make_surface_mesh} 
is guided by a set of refinement criteria.
The concept \ccRefName\ describes the type  which
handles those criteria. 
It corresponds to the requirements for the third template parameter
\ccc{Criteria} of the surface mesher.

Typically the meshing criteria are a set
of elementary criterion, each of which
has to be met by  the facets of the final mesh.
The meshing algorithm eliminates in turn  {\em bad} facets, i.e.,
facets that do not meet all the criteria.

The size and quality of the final mesh 
depends on the order according to which bad facets
are handled. Therefore, the meshing algorithm 
needs to be able to quantify the facet qualities and to compare
the qualities of different faces.
The concept \ccRefName\ 
defines a type \ccc{Quality} designed to measure
the quality of a mesh facet. 
Typically this quality
is a multicomponent variable.  Each component corresponds to
one criterion and measures how much the facet deviates from
meeting this criterion. Then, the  comparison operator on qualities
is just a lexicographical comparison.

\ccHasModel

\ccc{Surface_mesh_default_criteria_3<Tr>}

\end{ccRefConcept}

% +------------------------------------------------------------------------+
%%RefPage: end of main body, begin of footer
% EOF
% +------------------------------------------------------------------------+

