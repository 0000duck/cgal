% +------------------------------------------------------------------------+
% | Reference manual page: Manifold_with_boundary_tag.tex
% +------------------------------------------------------------------------+
% | 09.12.2005   Author
% | Package: Package
% | 
\RCSdef{\RCSTagmanifoldwithboundaryRev}{$Id$}
\RCSdefDate{\RCSTagmanifoldwithboundaryDate}{$Date$}
% |
%%RefPage: end of header, begin of main body
% +------------------------------------------------------------------------+


\begin{ccRefClass}{Manifold_with_boundary_tag}  %% add template arg's if necessary

%% \ccHtmlCrossLink{}     %% add further rules for cross referencing links
%% \ccHtmlIndexC[class]{} %% add further index entries

\ccDefinition
  
The class \ccRefName\ is a tag class used to monitor the
surface meshing algorithm. When instantiated with the tag
\ccRefName, the global function 
\ccc{make_surface_mesh<C2T3,SurfaceOracle,Criteria,Tag>}
ensures that the output mesh  is a  manifold surface
but it may have boundaries.

\ccInclude{CGAL/Surface_mesher.h}

%\ccIsModel

%Concept

%\ccTypes

%\ccNestedType{TYPE}{some nested types}

%\ccCreation
%\ccCreationVariable{a}  %% choose variable name

%\ccConstructor{Manifold_with_boundary_tag();}{default constructor.}

%\ccOperations

%\ccMethod{void foo();}{some member functions}

\ccSeeAlso

\ccc{make_surface_mesh<C2T3,SurfaceOracle,Criteria,Tag>} \\
\ccc{Manifold_tag} \\
%\ccc{Manifold_with_boundary_tag}
\ccc{Non_manifold_tag}
%\ccExample

%A short example program.
%Instead of a short program fragment, a full running program can be
%included using the 
%\verb|\ccIncludeExampleCode{Package/Manifold_with_boundary_tag.C}| 
%macro. The program example would be part of the source code distribution and
%also part of the automatic test suite.

%\begin{ccExampleCode}
%void your_example_code() {
%}
%\end{ccExampleCode}

%%% \ccIncludeExampleCode{Package/Manifold_with_boundary_tag.C}

\end{ccRefClass}

% +------------------------------------------------------------------------+
%%RefPage: end of main body, begin of footer
% EOF
% +------------------------------------------------------------------------+

