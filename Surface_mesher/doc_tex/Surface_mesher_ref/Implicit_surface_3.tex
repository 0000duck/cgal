% +------------------------------------------------------------------------+
% | Reference manual page: Implicit_surface_3.tex
% +------------------------------------------------------------------------+
% | 09.12.2005   Mariette Yvinec
% | Package: Surface_mesher
% | 
\RCSdef{\RCSImplicitsurface3Rev}{$Id$}
\RCSdefDate{\RCSImplicitsurface3Date}{$Date$}
% |
%%RefPage: end of header, begin of main body
% +------------------------------------------------------------------------+


\begin{ccRefClass}{Implicit_surface_3<Traits, Function>}

%% \ccHtmlCrossLink{}     %% add further rules for cross referencing links
%% \ccHtmlIndexC[class]{} %% add further index entries

\ccDefinition
  
The class \ccRefName\  implements a surface described 
as the zero level
set  of a function \begin{math}f : \R^3 \longrightarrow \R\end{math}.


The library provides a partial specialization
of the  surface mesher traits generator:
\ccc{Surface_mesh_traits_generator_3<Implicit_surface_3<Traits, Function> >}
that provides a model of \ccc{SurfaceMeshTraits_3} for this type of surfaces.


The parameter \ccc{Traits} is a traits class 
that has to be implemented with a model of 
\ccc{ImplicitSurfaceTraits_3}.
Actually, this traits class implements the oracle needed by the
surface mesher:
the types, predicates and constructors objects provided
in \ccc{Traits} are
used in 
the partial specialisation 
\ccc{Surface_mesh_traits_generator_3<Implicit_surface_3<Traits, Function> >}
to generate the model of  \ccc{SurfaceMeshTraits_3}.

The template parameter \ccc{Function}  stands for a model
of the concept \ccc{ImplicitFunction}.
The number type \ccc{Function::FT} has to match
the type \ccc{Traits::FT}.

\ccInclude{CGAL/Implicit_surface_3.h}

\ccCreation
\ccCreationVariable{surface}

\ccConstructor{
Implicit_surface_3(Function f,
                   const Sphere_3 bounding_sphere,
                   const FT error_bound);}
{\ccc{f} is the object of type \ccc{Function} that represents the implicit
  surface.\\
 \ccc{bounding_sphere} is a bounding sphere of the implicit surface. The
 evaluation of \ccc{f} at the center \ccc{c} of this sphere must be
 negative: $f(c)<0$.\\
 \ccc{error_bound} is a bound that is used if the partial specialisation 
 \ccc{Surface_mesh_traits_generator_3<Implicit_surface_3<Traits, Function> >}
 to compute intersection points between the implicit surface and some
 segments.}
% TODO: modify the initial points generation with bipolar oracle, because
% f(c) < 0 

\ccSeeAlso
\ccc{make_surface_mesh},\\
\ccc{Surface_mesh_traits_generator_3<Surface>},\\
\ccc{ImplicitSurfaceTraits}, \\
\ccc{ImplicitFunction}.

\end{ccRefClass}

% +------------------------------------------------------------------------+
%%RefPage: end of main body, begin of footer
% EOF
% +------------------------------------------------------------------------+

