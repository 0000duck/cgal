% +------------------------------------------------------------------------+
% | Reference manual page: Gray_level_image_3.tex
% +------------------------------------------------------------------------+
% | 06.03.2006   Laurent Rineau
% | Package: Surface_mesher
% | 
\RCSdef{\RCSGraylevelimage3Rev}{$Id$}
%\RCSdefDate{\RCSGraylevelimage3Date}{$Date$}
% |
%%RefPage: end of header, begin of main body
% +------------------------------------------------------------------------+


\begin{ccRefClass}{Gray_level_image_3<FT, Point>}

%% \ccHtmlCrossLink{}     %% add further rules for cross referencing links
%% \ccHtmlIndexC[class]{} %% add further index entries

\ccDefinition

 A~3D gray image is a
tri-dimensional array that associates a scalar value to each triple of
integer $(x, y, z)$ in the range of the image. A~trilinear interpolation
algorithm provides a map \begin{math}f : \R^3 \longrightarrow \R\end{math}.

The class \ccRefName\ is a 3D gray image loader and a model
of the concept \ccc{ImplicitFunction}.
An object of the class \ccRefName\  is created with a parameter
\ccc{iso} and then its \ccc{operator()}  implements
the function \ccc{sign of (f(p) - iso)}, for $p \in \R^3$.
Plugging such a function in the creation of the \ccc{Implicit_surface_3}
object given as parameter  to \ccc{make_surface_mesh} yields
a mesh approximating the level with value \ccc{iso}
in the input 3D gray image.
 

%An object of type \ccRefName\ represents the implicit surface of equation
%$f(x,y,z) = iso$, on $\R^3$, where $iso$ is a given isovalue. The parameter
%\ccc{FT} is the numerical types. \ccRefName\ provides an \ccc{operator()(FT x, FT y
%  ,FT z)} that returns \ccc{FT(-1)} if $f(x,y,z)>0$, \ccc{FT(0)} if
%$f(x,y,z)=0$, and \ccc{FT(1)} if $f(x,y,z)<0$. \ccRefName\ is a model of
%\ccc{ImplicitFunction}.

\ccRefName\ provides an interface with an auxiliary library called
\emph{ImageIO}. An executable that uses \ccRefName\ must be linked with the \emph{ImageIO}
library. This library is shipped with \cgal\ in the
\ccc{examples/Surface_mesher/} subdirectory.

The library \emph{ImageIO} and therefore \ccRefName\ support
several types of 3D images: INRIMAGE (extension .inr[.gz]), GIS (extension
.dim, of .ima[.gz]), and ANALYZE (extension .hdr, or .img[.gz]).


\ccInclude{CGAL/Gray_level_image_3.h}

\ccIsModel

\ccc{ImplicitFunction}

\ccTypes

\ccNestedType{FT}{the numerical type \ccc{FT}}
\ccNestedType{FT}{the point type \ccc{Point}}

\ccCreation
\ccCreationVariable{image}

\ccConstructor{Gray_level_image_3(const char* filename, FT iso_value);}
{\ccc{filename} is the path to a file of a type supported by \emph{ImageIO}.\\
\ccc{iso_value} is an isovalue of $f$.
}

\ccSeeAlso

\ccc{ImplicitFunction},\\
\ccc{Implicit_surface_3<Traits, Function>}, \\
\ccc{make_surface_mesh}

\end{ccRefClass}

% +------------------------------------------------------------------------+
%%RefPage: end of main body, begin of footer
% EOF
% +------------------------------------------------------------------------+

