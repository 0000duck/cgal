% +------------------------------------------------------------------------+
% | Reference manual page: Surface_mesh_default_criteria.tex
% +------------------------------------------------------------------------+
% | 08.12.2005   Mariette Yvinec, Laurent Rineau
% | Package: Surface_mesher
% |
\RCSdef{\RCSSurfacemeshdefaultcriteria3Rev}{$Id$}
\RCSdefDate{\RCSSurfacemeshdefaultcriteria3Date}{$Date$}
% |
%%RefPage: end of header, begin of main body
% +------------------------------------------------------------------------+

\begin{ccRefClass}{Surface_mesh_default_criteria_3<Tr>}

\ccDefinition
  
The class \ccRefName\  implements  the most commonly used combination
of meshing criteria. It involves mainly three criteria which are
in order~: \\
- a lower bound on the minimum angle of the surface mesh facets. \\
- an upper bound on the radius of surface Delaunay balls.
 A surface Delaunay ball is a  ball circumscribing  a facet,
centered on the surface and empty of vertices.
 Such a ball exists for each facet
of the current surface mesh.
Indeed  the current surface mesh
is  the Delaunay triangulation of the current sampling restricted to
the surface 
which is just the set of facets in the three dimensional  Delaunay triangulation of
the sampling  that  have a Delaunay surface ball. \\
- an upper bound on the center-center distances of the surface mesh facets.
  The center-center distance of a surface mesh facet 
  is the distance between the facet circumcenter and the 
  center of its  surface Delaunay ball. 

\ccInclude{CGAL/Surface_mesh_default_criteria_3.h}

\ccIsModel

\ccc{SurfaceMeshCriteria_3}

\ccTypes

\ccTypedef{typedef Tr::FT FT;}{The numerical type.}

\ccCreation
\ccCreationVariable{criteria}

\ccConstructor{Surface_mesh_default_criteria_3(FT angle_bound, 
                                               FT radius_bound,
                                               FT distance_bound);}
{Returns a \ccRefName\ with \ccc{angle_bound}, \ccc{radius_bound},
\ccc{distance_bound} as bounds for the minimum facet angle,
the radius of the surface Delaunay balls 
and the center-center distances respectively.}

\ccSeeAlso

\ccc{make_surface_mesh}

\end{ccRefClass}

% +------------------------------------------------------------------------+
%%RefPage: end of main body, begin of footer
% EOF
% +------------------------------------------------------------------------+

