s
% +------------------------------------------------------------------------+
% | Reference manual page: ImplicitFunction.tex
% +------------------------------------------------------------------------+
% | 09.12.2005   Mariette Yvinec, Laurent Rineau
% | Package: Surface_mesher
% |
\RCSdef{\RCSImplicitFunctionRev}{$Id$}
\RCSdefDate{\RCSImplicitFunctionDate}{$Date$}
% |
%%RefPage: end of header, begin of main body
% +------------------------------------------------------------------------+


\begin{ccRefConcept}{ImplicitFunction}

\ccDefinition
  
The concept \ccRefName\ describes a function object
whose \ccc{operator()} computes the values of a function
\begin{math}f : \R^3 \longrightarrow \R\end{math}.

\ccTypes

\ccNestedType{FT}{Number type}


\ccCreationVariable{function} 

\ccOperations

\ccMethod{FT operator()(FT x, FT y, FT z);}{Returns the value $f(x,y,z)$.}

\ccHasModels

\ccc{Gray_level_image_function},\\
any pointer to a function of type \ccc{FT (*)(FT, FT, FT)}.

\ccSeeAlso
\ccc{Implicit_surface_3<Traits, Function>},\\
\ccc{make_surface_mesh}



%\ccExample

%A short example program.
%Instead of a short program fragment, a full running program can be
%included using the 
%\verb|\ccIncludeExampleCode{Package/ImplicitFunction.C}| 
%macro. The program example would be part of the source code distribution and
%also part of the automatic test suite.

%\begin{ccExampleCode}
%void your_example_code() {
%}
%\end{ccExampleCode}

%%% \ccIncludeExampleCode{Package/ImplicitFunction.C}

\end{ccRefConcept}

% +------------------------------------------------------------------------+
%%RefPage: end of main body, begin of footer
% EOF
% +------------------------------------------------------------------------+

