% +------------------------------------------------------------------------+
% | Reference manual page: SurfaceMeshVertexBase_3.tex
% +------------------------------------------------------------------------+
% | 09.12.2005   Author
% | Package: Package
% | 
\RCSdef{\RCSSurfaceMeshVertexBaseRev}{$Id$}
\RCSdefDate{\RCSSurfaceMeshVertexBaseDate}{$Date$}
% |
%%RefPage: end of header, begin of main body
% +------------------------------------------------------------------------+

\begin{ccRefConcept}{SurfaceMeshVertexBase_3}

%% \ccHtmlCrossLink{}     %% add further rules for cross referencing links
%% \ccHtmlIndexC[concept]{} %% add further index entries

\ccDefinition
  
The concept \ccRefName\ describes the vertex base type
of the three dimensional triangulation used
to embed the surface mesh.

More precisely,
the first template parameter \ccc{SurfaceMeshC2T3} of the surface mesher
\ccc{make_surface_mesh}
is a model of the concept 
\ccc{SurfaceMeshComplex_2InTriangulation_3} 
which describes a data structure to store
a pure two dimensional complex 
embedded in a three dimensional triangulation.
In particular, the type \ccc{SurfaceMeshC2T3} is required to provide
a three dimensional triangulation type
\ccc{SurfaceMeshC2T3::Triangulation_3}
The concept \ccRefName\ describes the vertex base type
required in this triangulation type.





\ccGeneralizes

\ccc{TriangulationVertexBase_3}


The surface mesher algorithm issues frequent queries
about the status of the vertices  with respect to the
two dimensional complex that represents the current
surface approximation.
The class \ccRefName\ offers a caching mechanism
to answer  more efficiently these queries.
The caching mechanism includes two cached integers, which, when they
are valid, store respectively the number of complex
facets incident to the vertex and the number
of connected components  of the adjacency graph of those facets.






%\ccTypes

%\ccNestedType{TYPE}{some nested types}

\ccCreation
\ccCreationVariable{vb}  %% choose variable name


\ccOperations


\ccMethod{bool is_c2t3_cache_valid();}
{Returns \ccc{true} if the cache is valid.}
\ccGlue
\ccMethod{void invalidate_c2t3_cache();}
{Invalidates the cash.}
\ccGlue
\ccMethod{int cached_number_of_incident_facets();}
{Returns the cached number of facets of the complex incident to the vertex.}
\ccGlue
\ccMethod{int cached_number_of_components();}
{ This method concerns the adjacency
graph of  the facets of the complex incident to the vertex
and returns a cached value for the number of connected components this graph.}

\ccHasModels

\ccc{Surface_mesh_vertex_base_3<Gt,Vb>}


\ccSeeAlso

\ccc{SurfaceMesherComplex_2InTriangulation_3} \\
\ccc{Surface_mesh_complex_2_in_triangulation_3<Tr>}.

%\ccExample

%A short example program.
%Instead of a short program fragment, a full running program can be
%included using the 
%\verb|\ccIncludeExampleCode{Package/SurfaceMeshVertexBase_3.C}| 
%macro. The program example would be part of the source code distribution and
%also part of the automatic test suite.

%\begin{ccExampleCode}
%void your_example_code() {
%}
%\end{ccExampleCode}

%%% \ccIncludeExampleCode{Package/SurfaceMeshVertexBase_3.C}

\end{ccRefConcept}

% +------------------------------------------------------------------------+
%%RefPage: end of main body, begin of footer
% EOF
% +------------------------------------------------------------------------+

