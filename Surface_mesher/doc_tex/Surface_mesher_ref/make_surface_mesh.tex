% +------------------------------------------------------------------------+
% | Reference manual page: surface_mesher.tex
% +------------------------------------------------------------------------+
% | 01.12.2005   Author
% | Package: Package
% | 
\RCSdef{\RCSsurfacemesherRev}{$Id$}
\RCSdefDate{\RCSsurfacemesherDate}{$Date$}
% |
%%RefPage: end of header, begin of main body
% +------------------------------------------------------------------------+


\begin{ccRefFunction}{make_surface_mesh<C2T3, Surface, Criteria, Tag>}  %% add template arg's if necessary

%% \ccHtmlCrossLink{}     %% add further rules for cross referencing links
%% \ccHtmlIndexC[function]{} %% add further index entries

\ccDefinition
  
The function \ccRefName\ is a surface mesher,
that is a function to build a two dimensional mesh 
approximating  a surface.

The first template parameter  \ccc{C2T3}
is required to be a model of the concept
\ccc{SurfaceMeshComplex_2InTriangulation_3},
a data structure able to represent a two dimensional
complex  embedded in a three dimensional triangulation.
The argument \ccc{c2t3} of type \ccc{C2T3},  passed by reference
to the surface mesher, 
is used to maintain  the current approximating mesh and it stores
the final mesh at the end of the procedure.
The type \ccc{C2T3} is in particuler required to
provide a type \ccc{C2T3::Triangulation_3}
for the three dimensional triangulation
embedding the surface mesh.
The vertex and cell base classes of the triangulation
\ccc{C2T3::Triangulation_3} are required
to be  models  of the concepts 
\ccc{SurfaceMeshVertexBase_3} and 
\ccc{SurfaceMeshCellBase_3} respectively.


The second template parameter describes the required knowledge
on the surface to be meshed. The plugged in class has to be 
either a model of the concept \ccc{SurfaceMeshTraits_3}
or a class for which a specialization of the
template class \ccc{Surface_mesher_traits_3<Surface> exists.
Such partial specialization are currently provided
for implicit surfaces (\ccc{Implicit_surface_3<Traits, Func> ) and 
grey level images (\ccc{Grey_level_image_3<Traits, Func> )
has to be a model
of the concept
\ccc{SurfaceOracle} which describes the required knowledge
on the surface to be meshed. 


The third template parameter \ccc{Criteria} has to be a model
of the concept \ccc{SurfaceMeshCriteria}.
The argument of type  \ccc{Criteria} passed to the surface
mesher specifies the size and shape  requirements
on the output surface mesh.

The fourth template parameter is a tag whose type affects the behaviour of the
meshing algorithm. The function \ccRefName\  has specialized versions
for the following  tag types~: \\
- \ccc{Manifold_tag}: the output mesh is guaranteed to be a manifold
surface without boundary.\\
- \ccc{Manifold_with_boundary_tag}~:the output mesh is guaranteed to be
manifold but may have boundaries.\\
- \ccc{Non_manifold_tag} : the algorithm relies on the given criteria and
guarantees nothing else.



\ccInclude{CGAL/make_surface_mesh.h}

\ccGlobalFunction{template <class C2T3,
                            class Surface,
                            class Criteria,
                            class Tag >
void make_surface_mesh(C2T3& c2t3,
                       Surface surface,
                       Criteria criteria,
                       Tag = Non_manifold_tag());}


The Delaunay refinement
process is started with an initial set of points which is the union 
of two sets: the
set of vertices in the initial  triangulation pointed to by the
\ccc{c2t3} argument   and a set of
points output by the \ccc{Construct_initial_points()} 
of the surface.
This initial set of points is required to include at least one point
on each connected components of the surface to be meshed.



 

%\ccIsModel

%Concept

\ccSeeAlso
%\ccc{Complex2InTriangulation3} \\
\ccc{SurfaceMeshComplex_2InTriangulation_3} \\
\ccc{SurfaceMeshCellBase_3} \\
\ccc{SurfaceMeshVertexBase_3} \\
\ccc{SurfaceOracle} \\
\ccc{SurfaceMeshCriteria} \\



%\ccExample

%A short example program.
%Instead of a short program fragment, a full running program can be
%included using the 
%\verb|\ccIncludeExampleCode{Package/surface_mesher.C}| 
%macro. The program example would be part of the source code distribution and
%also part of the automatic test suite.

%\begin{ccExampleCode}
%void your_example_code() {
%}
%\end{ccExampleCode}

%% \ccIncludeExampleCode{Package/make_surface_mesh.C}

\end{ccRefFunction}

% +------------------------------------------------------------------------+
%%RefPage: end of main body, begin of footer
% EOF
% +------------------------------------------------------------------------+

