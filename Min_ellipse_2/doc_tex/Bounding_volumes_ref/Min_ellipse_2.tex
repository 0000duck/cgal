
\begin{ccRefClass}{Min_ellipse_2<Traits>}

\ccIndexSubitem       [t]{ellipse}{smallest enclosing}
\ccIndexSubitemSeeAlso[t]{circle}{smallest enclosing ellipse}
\ccIndexSubitem       [t]{smallest enclosing}{ellipse}
\ccIndexSubitem       [t]{bounding volumes}{smallest enclosing ellipse}

\ccSaveThreeColumns
\cgalMinEllipseLayout

% -----------------------------------------------------------------------------
\ccDefinition

An object of the class \ccRefName\ is the unique ellipse of smallest area
enclosing a finite (multi)set of points in two-dimensional euclidean
space~$\E_2$.  For a point set $P$ we denote by $me(P)$ the smallest
ellipse that contains all points of $P$. Note that $me(P)$ can be
degenerate\lcTex{\ccIndexSubitem[t]{degeneracies}{\ccFont Min_ellipse_2}},
i.e.~$me(P)=\emptyset$ if
$P=\emptyset$, $me(P)=\{p\}$ if $P=\{p\}$,
and $me(P) = \{ \mbox{\ccTexHtml{$(1-\lambda)p + \lambda q \mid 0 \leq
    \lambda \leq 1$}{(1-l)p + l q | 0 <= l <= 1}} \}$ if $P=\{p,q\}$.

An inclusion-minimal subset $S$ of $P$ with $me(S)=me(P)$ is called a
\emph{support set}\lcTex{\ccIndexSubitem[t]{support set}{\ccFont Min_ellipse_2}},
the points in $S$ are the \emph{support points}.  A support set has size at
most five, and all its points lie on the boundary of $me(P)$. In general,
neither the support set nor its size are necessarily unique.

The underlying algorithm can cope with all kinds of input, e.g.~$P$ may be
empty or points may occur more than once. The algorithm computes a support
set $S$ which remains fixed until the next insert or clear operation.

\ccInclude{CGAL/Min_ellipse_2.h}

% -----------------------------------------------------------------------------
\ccRequirements
\ccIndexRequirements

The template parameter \ccc{Traits} is a model for \ccc{MinEllipse2Traits}.

We provide the model \ccc{CGAL::Min_ellipse_2_traits_2<K>} using the 
two-dimensional \cgal~kernel.

% -----------------------------------------------------------------------------
\ccTypes
\ccIndexClassTypes

\ccSetTwoColumns{\ccRefName:: Ellipse}{}

\ccNestedType{ Point  }{ Typedef to \ccc{Traits::Point  }.}

\ccNestedType{ Ellipse}{ Typedef to \ccc{Traits::Ellipse}. If you 
        are using the predefined traits class 
        \ccc{CGAL::Min_ellipse_2_traits_2<K>},
        you can access the coefficients of the ellipse, see the 
        documentation of \ccc{CGAL::Min_ellipse_2_traits_2<K>} and 
        the example below.}

\ccNestedType{ Point_iterator}{
        Non-mutable model of the \stl\ concept \emph{BidirectionalIterator}
        with value type \ccc{Point}. Used to access the points
        of the smallest enclosing ellipse.}

\ccNestedType{ Support_point_iterator}{
        Non-mutable model of the \stl\ concept \emph{RandomAccessIterator}
        with value type \ccc{Point}. Used to access the support points
        of the smallest enclosing ellipse.}

\ccPropagateThreeToTwoColumns

% -----------------------------------------------------------------------------
\ccCreation
\ccIndexClassCreation
\ccCreationVariable{min_ellipse}

A \ccRefName\ object can be created from an arbitrary point set $P$ and by
specialized construction methods expecting no, one, two, three, four or
five points as arguments. The latter methods can be useful for
reconstructing $me(P)$ from a given support set\lcTex{\ccIndexSubitem[t]{support
  set}{\ccFont Min_ellipse_2}} $S$ of $P$.

\ccConstructor{ template < class InputIterator >
                Min_Ellipse_2( InputIterator  first,
                               InputIterator  last,
                               bool           randomize,
                               Random&        random    = default_random,
                               const Traits&  traits    = Traits()   );}{
        initializes \ccVar\ to $me(P)$ with $P$ being the set of points
        in the range [\ccc{first},\ccc{last}). If \ccc{randomize} is
        \ccc{true}, a random permutation of $P$ is computed in
        advance, using the random numbers generator \ccc{random}.
        Usually, this will not be necessary, however, the algorithm's
        efficiency depends on the order in which the points are
        processed, and a bad order might lead to extremely poor
        performance (see example below).
        \ccRequire  The value type of \ccc{first} and \ccc{last}
        is \ccc{Point}.}

\ccConstructor{ Min_ellipse_2( const Traits& traits = Traits());}{
        creates a variable \ccVar\ of type \ccClassTemplateName.
        It is initialized to
        $me(\emptyset)$, the empty set.
        \ccPostcond \ccVar\ccc{.is_empty()} = \ccc{true}.}

\ccConstructor{ Min_ellipse_2( const Point&   p,
                                    const Traits&  traits = Traits());}{
        initializes \ccVar\ to $me(\{p\})$, the set $\{p\}$.
        \ccPostcond \ccVar\ccc{.is_degenerate()} = \ccc{true}.}

\ccConstructor{ Min_ellipse_2( const Point&   p,
                               const Point&   q,
                               const Traits&  traits = Traits());}{
        initializes \ccVar\ to $me(\{p,q\})$,
\ccTexHtml{\\}{}
                                              the set
        $\{ \mbox{\ccTexHtml{$(1-\lambda)p + \lambda q \mid
          0 \leq \lambda \leq 1$}{(1-l)p + l q | 0 <= l <= 1}} \}$.
        \ccPostcond \ccVar\ccc{.is_degenerate()} = \ccc{true}.}

\ccConstructor{ Min_ellipse_2( const Point&   p1,
                               const Point&   p2,
                               const Point&   p3,
                               const Traits&  traits = Traits());}{
        initializes \ccVar\ to $me(\{p1,p2,p3\})$.}

\ccConstructor{ Min_ellipse_2( const Point&   p1,
                               const Point&   p2,
                               const Point&   p3,
                               const Point&   p4,
                               const Traits&  traits = Traits());}{
        initializes \ccVar\ to $me(\{p1,p2,p3,p4\})$.}

\ccConstructor{ Min_ellipse_2( const Point&   p1,
                               const Point&   p2,
                               const Point&   p3,
                               const Point&   p4,
                               const Point&   p5,
                               const Traits&  traits = Traits());}{
        initializes \ccVar\ to $me(\{p1,p2,p3,p4,p5\})$.}

% -----------------------------------------------------------------------------
\ccAccessFunctions
\begin{ccIndexMemberFunctions}
\ccIndexMemberFunctionGroup{access}

\ccMemberFunction{ int  number_of_points( ) const;}{
        returns the number of points of \ccVar, i.e.\ $|P|$.}

\ccMemberFunction{ int  number_of_support_points( ) const;}{
        returns the number of support points of \ccVar, i.e.\ $|S|$.}
\lcTex{\ccIndexSubitem[t]{support set}{\ccFont Min_ellipse_2}}

\ccGlueBegin
\ccMemberFunction{ Point_iterator  points_begin() const;}{
        returns an iterator referring to the first point of \ccVar.}
%
\ccMemberFunction{ Point_iterator  points_end() const;}{
        returns the corresponding past-the-end iterator.}
\ccGlueEnd

\ccGlueBegin
\ccMemberFunction{ Support_point_iterator  support_points_begin() const;}{
        returns an iterator referring to the first support point of \ccVar.}
\lcTex{\ccIndexSubitem[t]{support set}{\ccFont Min_ellipse_2}}
%
\ccMemberFunction{ Support_point_iterator  support_points_end() const;}{
        returns the corresponding past-the-end iterator.}
\lcTex{\ccIndexSubitem[t]{support set}{\ccFont Min_ellipse_2}}
\ccGlueEnd

\ccMemberFunction{ const Point&  support_point( int i) const;}{
        returns the \ccc{i}-th support point of \ccVar. Between two
        modifying operations (see below) any call to
        \ccVar\ccc{.support_point(i)} with the same \ccc{i} returns
        the same point.
        \ccPrecond $0 \leq i< \mbox{\ccVar\ccc{.number_of_support_points()}}$.}
\lcTex{\ccIndexSubitem[t]{support set}{\ccFont Min_ellipse_2}}

\ccMemberFunction{ const Ellipse&  ellipse( ) const;}{
        returns the current ellipse of \ccVar.}

% -----------------------------------------------------------------------------
\ccPredicates
\ccIndexMemberFunctionGroup{predicates}

By definition, an empty \ccRefName\ has no boundary and no bounded side,
i.e.\ its unbounded side equals the whole space $\E_2$.

\ccMemberFunction{ CGAL::Bounded_side
                   bounded_side( const Point& p) const;}{
        returns \ccGlobalScope\ccc{ON_BOUNDED_SIDE},
        \ccGlobalScope\ccc{ON_BOUNDARY}, or
        \ccGlobalScope\ccc{ON_UNBOUNDED_SIDE} iff \ccc{p} lies properly
        inside, on the boundary of, or properly outside of \ccVar, resp.}

\ccMemberFunction{ bool  has_on_bounded_side( const Point& p) const;}{
        returns \ccc{true}, iff \ccc{p} lies properly inside \ccVar.}

\ccMemberFunction{ bool  has_on_boundary( const Point& p) const;}{
        returns \ccc{true}, iff \ccc{p} lies on the boundary
        of \ccVar.}

\ccMemberFunction{ bool  has_on_unbounded_side( const Point& p) const;}{
        returns \ccc{true}, iff \ccc{p} lies properly outside of \ccVar.}

\ccMemberFunction{ bool  is_empty( ) const;}{
        returns \ccc{true}, iff \ccVar\ is empty (this implies
        degeneracy).}

\ccMemberFunction{ bool  is_degenerate( ) const;}{
        returns \ccc{true}, iff \ccVar\ is degenerate,
        i.e.~if \ccVar\ is empty, equal to a single point or equal to a
        segment, equivalently if the number of support points is less
        than~3.}
\lcTex{\ccIndexSubitem[t]{degeneracies}{\ccFont Min_ellipse_2}}

% -----------------------------------------------------------------------------
\ccModifiers
\ccIndexMemberFunctionGroup{modifiers}

New points can be added to an existing \ccVar, allowing to build $me(P)$
incrementally, e.g.~if $P$ is not known in advance. Compared to the direct
creation of $me(P)$, this is not much slower, because the construction
method is incremental itself.

\ccMemberFunction{ void  insert( const Point& p);}{
        inserts \ccc{p} into \ccVar\ and recomputes the smallest
        enclosing ellipse.}

\ccMemberFunction{ template < class InputIterator >
                   void  insert( InputIterator  first,
                                 InputIterator  last );}{
        inserts the points in the range [\ccc{first},\ccc{last})
        into \ccVar\ and recomputes the smallest enclosing ellipse by
        calling \ccc{insert(p)} for each point \ccc{p} in
        [\ccc{first},\ccc{last}).
        \ccRequire  The value type of \ccc{first} and \ccc{last} is
        \ccc{Point}.}

\ccMemberFunction{ void clear( );}{
        deletes all points in \ccVar\ and sets \ccVar\ to the empty set.
        \ccPostcond  \ccVar\ccc{.is_empty()} = \ccc{true}.}

% -----------------------------------------------------------------------------
\ccHeading{Validity Check}
\ccIndexMemberFunctionGroup{validity check}
\lcTex{\ccIndexSubitem[t]{validity check}{\ccFont Min_ellipse_2}}

An object \ccVar\ is valid, iff
\begin{itemize}
  \item \ccVar\ contains all points of its defining set $P$,
  \item \ccVar\ is the smallest ellipse spanned by its support set $S$, and
  \item $S$ is minimal, i.e.\ no support point is redundant.
\end{itemize}
\emph{Note:} In this release only the first item is considered by the
validity check.

\ccMemberFunction{ bool is_valid( bool verbose = false,
                                int  level   = 0    ) const;}{
      returns \ccc{true}, iff \ccVar\ contains all points of its
      defining set $P$. If \ccc{verbose} is \ccc{true}, some
      messages concerning the performed checks are written to
      standard error stream. The second parameter \ccc{level} is not
      used, we provide it only for consistency with interfaces of
      other classes.}

% -----------------------------------------------------------------------------
\ccHeading{Miscellaneous}
\ccIndexMemberFunctionGroup{miscellaneous}

\def\ccTagRmConstRefPair{\ccFalse}

\ccMemberFunction{ const Traits&  traits( ) const;}{
        returns a const reference to the traits class object.}

\def\ccTagRmConstRefPair{\ccTrue}

\end{ccIndexMemberFunctions}

% -----------------------------------------------------------------------------
\ccHeading{I/O}
\begin{ccIndexGlobalFunctions}

\ccIndexGlobalFunctionGroup{output}
\ccFunction{ std::ostream&
             operator << ( std::ostream& os,
                           const Min_ellipse_2<Traits>& min_ellipse);}{
        writes \ccVar\ to output stream \ccc{os}.
        \ccRequire  The output operator is defined for \ccc{Point}
        (and for \ccc{Ellipse}, if pretty printing is used).}

\ccIndexGlobalFunctionGroup{input}
\ccFunction{ std::istream&
             operator >> ( std::istream& is,
                           Min_ellipse_2<Traits> min_ellipse&);}{
        reads \ccVar\ from input stream \ccc{is}.
        \ccRequire  The input operator is defined for \ccc{Point}.}

\end{ccIndexGlobalFunctions}

% -----------------------------------------------------------------------------
\ccSeeAlso

\ccRefIdfierPage{CGAL::Min_circle_2<Traits>}\\[1ex]
\ccRefIdfierPage{CGAL::Min_ellipse_2_traits_2<K>}\\[1ex]
\ccRefConceptPage{MinEllipse2Traits}

% -----------------------------------------------------------------------------
\ccImplementation
\ccIndexImplementation

We implement the incremental algorithm\lcTex{\ccIndexSubitem[t]{incremental
  algorithm}{\ccFont Min_ellipse_2}} of Welzl, with move-to-front
heuristic~\cite{w-sedbe-91a}\lcTex{\ccIndexSubitem[t]{move-to-front
  heuristic}{\ccFont Min_ellipse_2}}, using the primitives as described
in~\cite{gs-epsee-97,cgal:gs-seefe-97a}. The whole implementation is described
in~\cite{cgal:gs-seeeg-98}.

If randomization\lcTex{\ccIndexSubitem[t]{randomization}{\ccFont Min_ellipse_2}} is
chosen, the creation time is almost always linear in the number of points.
Access functions and predicates take constant time, inserting a point might
take up to linear time, but substantially less than computing the new
smallest enclosing ellipse from scratch. The clear operation and the check
for validity each takes linear time.

% -----------------------------------------------------------------------------
\ccExample
\ccIndexSubitem[C]{Min_ellipse_2}{example}

To illustrate the usage of \ccRefName\ and to show that randomization
can be useful in certain cases, we give an example. The example also
shows how the coefficents of the constructed ellipse can be accessed. 

\ccIncludeExampleCode{Min_ellipse_2/min_ellipse_2.cpp}

% -----------------------------------------------------------------------------

\ccRestoreThreeColumns

\end{ccRefClass}


