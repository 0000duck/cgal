\begin{ccRefClass}{Triangulation_conformer_2<CDT>}

The class \ccRefName{} is an auxiliary class of
\ccc{Delaunay_mesher_2<CDT>}. It permits to refine a constrained
Delaunay triangulation into a conforming Delaunay or conforming
Gabriel triangulation. For standard needs, consider using the global
functions \ccc{make_conforming_Gabriel_2} and
\ccc{make_conforming_Delaunay_2}.

\ccParameters 

The template parameter \ccc{CDT} should be a model of the concept
\ccc{ConstrainedDelaunayTriangulation_2}.

The geometric traits class of the instance of \ccc{CDT} has to be
a model of the concept \ccc{ConformingDelaunayTriangulationTraits_2}.

\ccHeading{Using this class}

The constructor of the class \ccRefName\ takes a reference to a \ccc{CDT}
as an argument. A call to the method \ccc{make_conforming_Delaunay()} or
\ccc{make_conforming_Gabriel()} will refine this contrained Delaunay
triangulation into a conforming Delaunay or conforming Gabriel
triangulation. Note that if, during the life time of the \ccRefName\
object, the triangulation is externally modified, any further call to its
member methods may lead to undefined behavior. Consider reconstructing a
new \ccRefName\ object if the triangulation has been modified.

The conforming methods insert points into constrained edges, thereby splitting
them into several sub-constraints. You have access to the initial inserted
constraints if you instantiate the template parameter by a
\ccc{CGAL::Constrained_triangulation_plus_2<CDT>}.

\ccInclude{CGAL/Triangulation_conformer_2.h}

\ccCreation
\ccCreationVariable{m}

\ccConstructor{Triangulation_conformer_2(CDT& t);}
{Create a new conforming maker, working on \ccc{t}.}

\ccOperations

\ccHeading{Conforming methods}

\ccMethod{ void make_conforming_Delaunay(); }
{ Refines the triangulation into a conforming Delaunay triangulation.
  After a call to this method, all triangles fulfill the Delaunay property,
  that is the empty circle
  property. }

\ccMethod{ void make_conforming_Gabriel(); }
{ Refines the triangulation into a conforming Gabriel triangulation.
  After a call to this method, all constrained edges $e$ have the
  \emph{Gabriel property}: the circle with diameter $e$ 
  does not contain any vertex of the triangulation. }

\ccHeading{Checking}
The following methods verify that the constrained triangulation is
conforming Delaunay or conforming Gabriel. These methods scan the
whole triangulation and their complexity is proportional to the number
of edges.

\ccMethod{ bool is_conforming_Delaunay(); }
{ Returns \ccc{true} iff all triangles fulfill the Delaunay property.}

\ccMethod{ bool is_conforming_Gabriel(); }
{ Returns \ccc{true} iff all constrained edges have the Gabriel property:
  their circumsphere is empty. }

\begin{ccAdvanced}

\ccHeading{Step by step operations}

The \ccRefName{} class allows, for debugging or demos, to play the
conforming algorithm step by step, using the following methods. They exist
in two versions, depending on whether you want the triangulation to be
conforming Delaunay or conforming Gabriel, respectively. Any call to a
\ccc{step_by_step_conforming_XX} function requires a previous call to the
corresponding function \ccc{init_XX} and Gabriel and Delaunay methods can
not be mixed between two calls of \ccc{init_XX}.

\ccMethod{ void init_Delaunay(); }
{ The method must be called after all points and constrained segments
  are inserted and before any call to the following methods. If some
  points or segments are then inserted in the triangulation, this
  method must be called again. }

\ccMethod{ bool step_by_step_conforming_Delaunay (); }
{ Applies one step of the algorithm, by inserting one point, if the
  algorithm is not done. Returns \ccc{false} iff no point has been inserted
  because the algorithm is done. }

\ccMethod{ void init_Gabriel(); }{Analog to
    \ccc{init_Delaunay} for Gabriel conforming.}

\ccMethod{ bool step_by_step_conforming_Gabriel (); }{Analog to
  \ccc{step_by_step_conforming_Delaunay()} for Gabriel conforming.}

\ccMethod{ bool is_conforming_done(); }
{ Tests if the step by step conforming algorithm is done. If it
  returns \ccc{true}, the following calls to
  \ccc{step_by_step_conforming_XX} will not insert any points, until some
  new constrained segments or points are inserted in the triangulation and
  \ccc{init_XX} is called again. }

\end{ccAdvanced}

\end{ccRefClass}

%%% For emacs/AucTeX:
%%% Local Variables: ***
%%% mode:latex ***
%%% TeX-master: "../Mesh_2.tex"  ***
%%% End: ***
