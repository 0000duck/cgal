% ======================================================================
%
% Copyright (c) 1999 The GALIA Consortium
%
% This software and related documentation is part of the
% Computational Geometry Algorithms Library (CGAL).
%
% Every use of CGAL requires a license. Licenses come in three kinds:
%
% - For academic research and teaching purposes, permission to use and
%   copy the software and its documentation is hereby granted free of  
%   charge, provided that
%   (1) it is not a component of a commercial product, and
%   (2) this notice appears in all copies of the software and
%       related documentation.
% - Development licenses grant access to the source code of the library 
%   to develop programs. These programs may be sold to other parties as 
%   executable code. To obtain a development license, please contact
%   the GALIA Consortium (at cgal@cs.uu.nl).
% - Commercialization licenses grant access to the source code and the
%   right to sell development licenses. To obtain a commercialization 
%   license, please contact the GALIA Consortium (at cgal@cs.uu.nl).
%
% This software and documentation is provided "as-is" and without
% warranty of any kind. In no event shall the CGAL Consortium be
% liable for any damage of any kind.
%
% The GALIA Consortium consists of Utrecht University (The Netherlands),
% ETH Zurich (Switzerland), Free University of Berlin (Germany),
% INRIA Sophia-Antipolis (France), Martin-Luther-University Halle-Wittenberg
% (Germany), Max-Planck-Institute Saarbruecken (Germany),
% and Tel-Aviv University (Israel).
%
% ----------------------------------------------------------------------
%
% package       : Alpha_shapes_2
% author(s)     : Tran Kai Frank DA <Frank.Da@sophia.inria.fr>
%
% coordinator   : INRIA Sophia-Antipolis (<Mariette.Yvinec@sophia.inria.fr>)
%
% ======================================================================

\RCSdef{\alphashapeRevision}{$Id$}
\RCSdefDate{\alphashapeDate}{$Date$}

%----------------------------------------------------------------------

\begin{ccRefConcept} {AlphaShapeTraits_2}

\ccCreationVariable{t}

\ccDefinition
A model of the concept \ccRefName\  must provide the following predicate and
operations in addition to the requirements for the underlying triangulation
traits class.
It means, the metric has to be Euclidean for Delaunay triangulation or the power 
metric for regular triangulation.

\ccRefines
\ccc{TriangulationTraits_2}


\ccHasModels

The kernels supplied by \cgal\ are models of \ccRefName. 

\ccRefConceptPage{Projection_traits_xy_3<K>}

\ccTypes

\ccNestedType{FT}{A type to hold a coordinate type class. 
The type must provide a copy constructor, assignment, comparison
operators, negation, multiplication, division and allow the
declaration and initialization with a small integer constant
(cf. requirements for number types). 
\ccPrecond An obvious choice would be coordinate type of the point class.}

\ccCreation

Only a default constructor is required. Note that further constructors
can be provided. 

\ccConstructor{Gt();}
{A default constructor.}

\ccHeading{Constructions by function objects}

\ccMethod{Compute_squared_radius_2 compute_squared_radius_2_object();} 
{Returns an object, which has to be able to compute the squared radius of the
circle of the points \ccc{p0, p1, p2} or the squared radius of smallest  circle
of the points \ccc{p0, p1}, as \ccc{FT} associated with {\em the metric used
by \ccc{Dt}}.} 

\ccHeading{Predicate by function object}

\ccMethod{Side_of_bounded_circle_2 side_of_bounded_circle_2_object();}
{Returns an object, which has to be able to compute the relative position of
point \ccc{test} to the smallest circle of the points \ccc{p0, p1}, using
{\em the same metric as \ccc{Dt}}.}

\end{ccRefConcept}

