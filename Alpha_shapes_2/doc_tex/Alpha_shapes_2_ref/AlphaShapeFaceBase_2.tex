% ======================================================================
%
% Copyright (c) 1999 The GALIA Consortium
%
% This software and related documentation is part of the
% Computational Geometry Algorithms Library (CGAL).
%
% Every use of CGAL requires a license. Licenses come in three kinds:
%
% - For academic research and teaching purposes, permission to use and
%   copy the software and its documentation is hereby granted free of  
%   charge, provided that
%   (1) it is not a component of a commercial product, and
%   (2) this notice appears in all copies of the software and
%       related documentation.
% - Development licenses grant access to the source code of the library 
%   to develop programs. These programs may be sold to other parties as 
%   executable code. To obtain a development license, please contact
%   the GALIA Consortium (at cgal@cs.uu.nl).
% - Commercialization licenses grant access to the source code and the
%   right to sell development licenses. To obtain a commercialization 
%   license, please contact the GALIA Consortium (at cgal@cs.uu.nl).
%
% This software and documentation is provided "as-is" and without
% warranty of any kind. In no event shall the CGAL Consortium be
% liable for any damage of any kind.
%
% The GALIA Consortium consists of Utrecht University (The Netherlands),
% ETH Zurich (Switzerland), Free University of Berlin (Germany),
% INRIA Sophia-Antipolis (France), Martin-Luther-University Halle-Wittenberg
% (Germany), Max-Planck-Institute Saarbruecken (Germany),
% and Tel-Aviv University (Israel).
%
% ----------------------------------------------------------------------
%
% package       : Alpha_shapes_2
% author(s)     : Tran Kai Frank DA <Frank.Da@sophia.inria.fr>
%
% coordinator   : INRIA Sophia-Antipolis (<Mariette.Yvinec@sophia.inria.fr>)
%
% ======================================================================

\RCSdef{\alphashapeRevision}{$Id$}
\RCSdefDate{\alphashapeDate}{$Date$}

%----------------------------------------------------------------------

\begin{ccRefConcept}{AlphaShapeFace_2}

\ccDefinition


\ccRefines
\ccc{TriangulationFaceBase_2}.


\ccCreationVariable{f}
\ccTypes

\ccNestedType{Interval_3}{A container type to get (and put) the three special values 
($\alpha_1, \alpha_2, \alpha_3$) associated with an alpha shape edge.}

\ccNestedType{FT}{A type to hold a coordinate type class. 
The type must provide a copy constructor, assignment, comparison
operators, negation, multiplication, division and allow the
declaration and initialization with a small integer constant
(cf. requirements for number types). An obvious choice would be
coordinate type of the point class} 

\ccCreation

\begin{ccAdvanced}

\ccConstructor{AlphaShapeFace_2();}{default constructor.}
\ccGlue
\ccConstructor{AlphaShapeFace_2(void* v0, void* v1, void* v2);}{constructor setting the incident vertices.}
\ccGlue
\ccConstructor{AlphaShapeFace_2(void* v0, void* v1, void*  v2, void* n0, void* n1, void* n2);} 
{constructor setting the incident vertices and the neighboring faces.}

\end{ccAdvanced}

\ccHeading{Access Functions}
\ccMethod{Interval_3 get_ranges(const int& i);}
{returns the interval associated with the edge indexed with $i$, which contains
three alpha values
 $\alpha_1 \leq \alpha_2 \leq \alpha_3$, such as for
$\alpha$  between $\alpha_1$ and $\alpha_2$, the edge indexed with $i$ is
attached but singular, 
for $\alpha$ between $\alpha_2$ and $\alpha_3$, the edge is regular, and for $\alpha$
greater than $\alpha_3$, the edge is interior.}
\ccMethod{FT get_alpha();}
{return the alpha value, under which the alpha shape contains the
face.}

\ccModifiers

\begin{ccAdvanced}

\ccMethod{void set_ranges(const int& i, const Interval_3& V);}
{sets the interval associated with the edge indexed with $i$, which contains three
alpha values
 $\alpha_1 \leq \alpha_2 \leq \alpha_3$, such as for
$\alpha$  between $\alpha_1$ and $\alpha_2$, the edge indexed with $i$ is
attached but singular, 
for $\alpha$ between $\alpha_2$ and $\alpha_3$, the edge is regular, and for $\alpha$
greater than $\alpha_3$, the edge is interior.}
\ccMethod{void set_alpha(FT A);}
{sets the alpha value, under which the alpha shape contains the
face.}

\end{ccAdvanced}

\end{ccRefConcept}
