% ======================================================================
%
% Copyright (c) 1999 
% Utrecht University (The Netherlands),
% ETH Zurich (Switzerland),
% INRIA Sophia-Antipolis (France),
% Max-Planck-Institute Saarbruecken (Germany),
% and Tel-Aviv University (Israel).
%
%
% This software and related documentation is part of the
% Computational Geometry Algorithms Library (CGAL).
%
% Every use of CGAL requires a license. Licenses come in three kinds:
%
% - For academic research and teaching purposes, permission to use and
%   copy the software and its documentation is hereby granted free of  
%   charge, provided that
%   (1) it is not a component of a commercial product, and
%   (2) this notice appears in all copies of the software and
%       related documentation.
% - Development licenses grant access to the source code of the library 
%   to develop programs. These programs may be sold to other parties as 
%   executable code. To obtain a development license, please contact
%   the GALIA Consortium (at cgal@cs.uu.nl).
% - Commercialization licenses grant access to the source code and the
%   right to sell development licenses. To obtain a commercialization 
%   license, please contact the GALIA Consortium (at cgal@cs.uu.nl).
%
% This software and documentation is provided "as-is" and without
% warranty of any kind. In no event shall the CGAL Consortium be
% liable for any damage of any kind.
%
%
% ----------------------------------------------------------------------
%
% package       : Alpha_shapes_2
% author(s)     : Tran Kai Frank DA <Frank.Da@sophia.inria.fr>
%
% coordinator   : INRIA Sophia-Antipolis (<Mariette.Yvinec@sophia.inria.fr>)
%
% ======================================================================

\RCSdef{\alphashapeRevision}{$Id$}
\RCSdefDate{\alphashapeDate}{$Date$}

%----------------------------------------------------------------------

\begin{ccRefClass} {Alpha_shape_2<Dt,ExactAlphaComparisonTag>}

\ccDefinition

The class \ccClassTemplateName\ represents the family of
$\alpha$-shapes of points in a plane for {\em all} positive
$\alpha$. It maintains the underlying triangulation \ccc{Dt} which
represents connectivity and order among its faces. Each
$k$-dimensional face of the \ccc{Dt} is associated with
an interval that specifies for which values of $\alpha$ the face
belongs to the $\alpha$-shape. There are links between the intervals
and the $k$-dimensional faces of the triangulation.

Note that this class is at the same time used for {\em basic} and
for {\em weighted} Alpha Shapes\ccIndexMainItem[C]{Weighted_alpha_shapes_2}.

\ccInclude{CGAL/Alpha_shape_2.h}

\ccParameters

The template parameter \ccc{Dt} has to be either \ccc{Delaunay_triangulation_2} or \ccc{Regular_triangulation_2}.
Note that \ccc{DT::Geom_traits}, \ccc{DT::Vertex} and \ccc{DT::Face} must model the concepts \ccc{AlphaShapeTraits_2},
\ccc{AlphaShapeVertex_2} and \ccc{AlphaShapeFace_2} respectively.

The template parameter \ccc{ExactAlphaComparisonTag} is a tag that, when set to
\ccc{CGAL::Tag_true}, triggers exact comparisons between alpha values. This is useful
when the underlying triangulation is instantiated with an exact predicates inexact constructions
kernel. By default the \ccc{ExactAlphaComparisonTag} is set to \ccc{CGAL::Tag_false} as it induces a small
overhead. Note that since such a strategy does not make sense if used together with a traits class with exact constructions,
the tag \ccc{ExactAlphaComparisonTag} is not taken into account if \ccc{Dt::Geom_traits::FT} is not a floating point number type.


\ccInheritsFrom

\ccc{Dt}

This class is the underlying triangulation class.

The modifying functions \ccc{insert} and \ccc{remove} will overwrite
the inherited functions. At the moment, only the static version is implemented.

\ccTypes
\ccSetThreeColumns{Oriented_side}{}{\hspace*{10cm}}
\ccThreeToTwo

\ccNestedType{Gt}{the alpha shape traits type.}
it has to derive from a  triangulation traits class.  
For example \ccc{Dt::Point} is a Point class. 

\ccTypedef{typedef Gt::FT FT;}{the number type for computation.}

\ccNestedType{size_type}{The size type.}

\ccNestedType{Alpha_iterator}{A bidirectional and non-mutable iterator that allow to traverse 
the increasing sequence of different $\alpha$-values.
\ccPrecond Its \ccc{value_type} is \ccc{FT}}

\ccNestedType{Alpha_shape_vertices_iterator}{A bidirectional and non-mutable iterator that allow to traverse 
the vertices which belongs to the $\alpha$-shape for the current $\alpha$. 
\ccPrecond Its \ccc{value_type} is \ccc{Dt::Vertex_handle}}

\ccNestedType{Alpha_shape_edges_iterator}{A bidirectional and non-mutable iterator that allow to traverse 
the edges which belongs to the $\alpha$-shape for the current $\alpha$. 
\ccPrecond Its \ccc{value_type} is \ccc{Dt::Edge}.}

\ccEnum{enum Classification_type {EXTERIOR, SINGULAR, REGULAR, INTERIOR};}
{Distinguishes the different cases for classifying a $k$-dimensional face
 of the underlying triangulation of the $\alpha$-shape. \\
\ccc{EXTERIOR} if the face does not belong to the $\alpha$-complex.\\
\ccc{SINGULAR} if the face belongs to the boundary of the $\alpha$-shape,
 but is not incident to any 2-dimensional face of the $\alpha$-complex\\
\ccc{REGULAR} if the face belongs to the boundary of the $\alpha$-shape
 and is incident to a 2-dimensional face of the $\alpha$-complex\\
\ccc{INTERIOR} if the face belongs to the $\alpha$-complex, but does
not belong to the boundary of the $\alpha$-shape.\\}

\ccEnum{enum Mode {GENERAL, REGULARIZED};}
{ In general, an alpha shape can be disconnected and contain many singular edges 
or vertices. Its regularized version is formed by the set of regular edges
and their vertices.}

\ccCreation
\ccCreationVariable{A}

\ccConstructor{Alpha_shape_2(FT alpha = 0,
				  Mode m = GENERAL);}
{Introduces an empty $\alpha$-shape \ccVar\ for a positive $\alpha$-value
 \ccc{alpha}.
\ccPrecond \ccc{alpha}~$\geq~0$.}

\ccConstructor{Alpha_shape_2(
          Dt& dt,
          FT alpha = 0,
				  Mode m = GENERAL);}
{Builds an alpha shape of mode \ccc{m} from the triangulation \ccc{dt}
for a positive $\alpha$-value \ccc{alpha}.
Be careful that this operation destroys the triangulation.
\ccPrecond \ccc{alpha}~$\geq~0$.}

\ccConstructor{template < class InputIterator >
		Alpha_shape_2(
			InputIterator first,
			InputIterator last,
                const FT& alpha = 0,
	        Mode m = GENERAL);}
{Initializes the family of alpha-shapes with the points in the range
$\left[\right.$\ccc{first}, \ccc{last}$\left.\right)$ and 
introduces an $\alpha$-shape \ccVar\ for a positive $\alpha$-value
\ccc{alpha}.  
\ccPrecond The \ccc{value_type} of \ccc{first} and
\ccc{last} is \ccc{Point}.\\
\ccc{alpha} $\geq 0$.}

\ccOperations

\ccMethod{template < class InputIterator >
		std::ptrdiff_t make_alpha_shape(
			InputIterator first,
			InputIterator last);}
{Initialize the family of alpha-shapes with the points in the range
$\left[\right.$\ccc{first}, \ccc{last}$\left.\right)$. Returns the number of
inserted points. \\ 
If the function is applied to an non-empty family of alpha-shape, it is cleared
before initialization.
\ccPrecond The \ccc{value_type} of \ccc{first} and
\ccc{last} is \ccc{Point}.}

\ccMethod{void
	clear();}
{Clears the structure.}

\ccMethod{FT
	set_alpha(const FT& alpha);}
{Sets the $\alpha$-value to \ccc{alpha}.
 Returns the previous $\alpha$-value.
\ccPrecond \ccc{alpha} $\geq 0$.}

\ccMethod{const FT&
	get_alpha(void) const;}
{Returns the current $\alpha$-value.}

\ccMethod{const FT&  get_nth_alpha(size_type n) const;}
{Returns the $n$-th alpha-value, sorted in an increasing order.
 \ccPrecond \ccc{n} $<$ number of alphas.}

\ccMethod{size_type number_of_alphas() const;}
{Returns the number of different alpha-values.}

% dynamic version	
% 
% \ccMethod{Vertex_handle insert(const Point& p);}
% {Inserts point \ccc{p} in the alpha shape and returns the
% corresponding vertex of the underlying Delaunay triangulation.\\ If
% point \ccc{p} coincides with an already existing vertex, this
% vertex is returned and the alpha shape remains unchanged.\\ Otherwise,
% the vertex is inserted in the underlying Delaunay triangulation and
% the associated intervals are updated. }
% 
% \ccMethod{void remove(Vertex *v);}
% {Removes the vertex from the underlying Delaunay triangulation. The
% created hole is retriangulated and the associated intervals are
% updated.}
% 

\ccMethod{Mode
	set_mode(Mode m = GENERAL );}
{Sets \ccVar\ to its general or regularized version. 
Returns the previous mode.}

\ccMethod{Mode
	get_mode(void) const;}
{Returns whether \ccVar\ is general or regularized.}

\ccMethod{Alpha_shape_vertices_iterator alpha_shape_vertices_begin();}{Starts at
an arbitrary finite vertex which belongs to the $\alpha$-shape for the current $\alpha$.}

\ccMethod{Alpha_shape_vertices_iterator alpha_shape_vertices_end();}{Past-the-end iterator.}

\ccMethod{Alpha_shape_edges_iterator alpha_shape_edges_begin();}{Starts at
an arbitrary finite edge which belongs to the $\alpha$-shape for the current
$\alpha$. In regularized mode, edges are represented as a pair (f,i), where f is 
an interior face of the $\alpha$-shape.}

\ccMethod{Alpha_shape_edges_iterator alpha_shape_edges_end();}{Past-the-end iterator.}

%\ccMethod{template  < class OutputIterator >
%	  OutputIterator get_alpha_shape_vertices(
%					 OutputIterator result);}
%{Writes the vertices of the alpha shape \ccVar\ for the current $\alpha$-value
%to the container where \ccc{result} refers to. 
%The \ccc{value_type} of \ccc{result} is \ccc{Vertex_handle}. 
%Returns an output iterator which is the end of the constructed range.}
%
%\ccMethod{template  < class OutputIterator >
%	  OutputIterator get_alpha_shape_edges(
%					 OutputIterator result);}
%{Writes the edges 
%of the alpha shape \ccVar\ for the current $\alpha$-value
%to the container where \ccc{result} refers to. 
%The \ccc{value_type} of \ccc{result} is \ccc{pair<Face_handle, int>}.
%Returns an output iterator which is the end of the constructed range.}

\ccHeading{Predicates}


\ccMethod{Classification_type
           classify(const Point& p, 
	const FT& alpha = get_alpha()) const;}
{Locates a point  \ccc{p} in the underlying triangulation and Classifies the 
associated k-face with respect to \ccVar.}

\ccMethod{Classification_type
           classify(Face_handle f, const FT& alpha = get_alpha()) const;}
{Classifies the face \ccc{f} of the underlying triangulation with respect to \ccVar.}

\ccMethod{Classification_type
           classify(Edge e, const FT& alpha = get_alpha()) const;}
{Classifies the edge \ccc{e} of the underlying triangulation with respect to \ccVar.}

\ccMethod{Classification_type
           classify(Face_handle f, int i, const FT& alpha = get_alpha()) const;}
{Classifies the edge of the face \ccc{f} opposite to the vertex with index
\ccc{i} 
of the underlying triangulation with respect to \ccVar.}	

\ccMethod{Classification_type
           classify(Vertex_handle v, const FT& alpha = get_alpha()) const;}
{Classifies the vertex \ccc{v} of the underlying triangulation with respect to \ccVar.}


\ccHeading{Traversal of the $\alpha$-Values}

\ccMethod{Alpha_iterator alpha_begin() const;}
{Returns an iterator that allows to traverse the
sorted sequence of $\alpha$-values of the family of alpha shapes.}

\ccMethod{Alpha_iterator alpha_end() const;}
{Returns the corresponding past-the-end iterator.}

\ccMethod{Alpha_iterator alpha_find(const FT& alpha) const;}
{Returns an iterator pointing to an element with $\alpha$-value
\ccc{alpha}, or the corresponding past-the-end iterator if such 
an element is not found.}

\ccMethod{Alpha_iterator alpha_lower_bound(const FT& alpha) const;}
{Returns an iterator pointing to the first element with
$\alpha$-value not less than \ccc{alpha}.}

\ccMethod{Alpha_iterator alpha_upper_bound(const FT& alpha) const;}
{Returns an iterator pointing to the first element with $\alpha$-value
greater than \ccc{alpha}.}

\ccHeading{Operations}

\ccMethod{size_type number_of_solid_components(const FT& alpha = get_alpha()) const;}
{Returns the number of solid components of \ccVar, that is, the number of
components of its 
regularized version.}

\ccMethod{Alpha_iterator find_optimal_alpha(size_type nb_components) const;}
{Returns an iterator pointing to the first element with $\alpha$-value
such that \ccVar\ satisfies the following two properties:\\
\ccc{nb_components} equals the number of solid components and \\
all data points are either on the boundary or in the interior of the regularized version of \ccVar.\\
If no such value is found, the iterator points to the first element with 
$\alpha$-value such that \ccVar\ satisfies the second property.}

\ccHeading{I/O}

The I/O operators are defined for \ccc{iostream}. The format for the iostream
is an internal format. 

\ccInclude{CGAL/IO/io.h}

\ccFunction{ostream& operator<<(ostream& os,
                  const Alpha_shape_2<Dt>& A);}
{Inserts the alpha shape \ccVar\ for the current $\alpha$-value into the stream \ccc{os}.
\ccPrecond The insert operator must be defined for \ccc{Point}.}


\ccImplementation
The set of intervals associated with the
$k$-dimensional faces of the underlying triangulation are
stored in \ccc{multimaps}.

The cross links between the intervals and the $k$-dimensional faces of the
triangulation are realized using methods in the $k$-dimensional faces
themselves.

\ccc{A.alpha_find} uses linear search, while 
\ccc{A.alpha_lower_bound} and \ccc{A.alpha_upper_bound} 
use binary search.
\ccc{A.number_of_solid_components} performs a graph traversal and takes time
linear in the number of faces of the underlying triangulation.
\ccc{A.find_optimal_alpha} uses binary search and takes time
$O(\mbox{ \em n } \log{\mbox{ \em n } })$, where  $n$ is the number of points.

\end{ccRefClass}

