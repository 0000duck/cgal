% +------------------------------------------------------------------------+
% | Reference manual page: ch_jarvis_march.tex
% +------------------------------------------------------------------------+
% | 09.05.2001   Susan Hert and Stefan Schirra
% | Package: Convex_hull_2
% |
% +------------------------------------------------------------------------+

\begin{ccRefFunction}{ch_jarvis_march}  %% add template arg's if necessary
\ccIndexSubitemBegin{convex hull, 2D}{Jarvis march}
\ccIndexSubitemBegin{convex hull, 2D}{gift-wrapping}
\ccIndexSubitemBegin{extreme points, 2D}{between two points}

\ccDefinition
  
The function \ccRefName\ generates the counterclockwise sequence of extreme
points from a given set of input points that line between two input points.


\ccInclude{CGAL/ch_jarvis.h}

\ccFunction{template <class ForwardIterator, class OutputIterator, class Traits>
            OutputIterator
            ch_jarvis_march(ForwardIterator first, ForwardIterator beyond,
                            const Traits::Point_2& start_p,
                            const Traits::Point_2& stop_p,
                            OutputIterator  result,
                            const Traits& ch_traits = Default_traits);}
            {generates the counterclockwise subsequence of
             extreme points between \ccc{start_p} and \ccc{stop_p} of the 
             points in the range [\ccc{first},\ccc{beyond}), starting at 
             position \ccc{result} with point \ccc{start_p}.  The last point 
             generated is the point preceding \ccc{stop_p} in the 
             counterclockwise order of extreme points.\\
            \ccPrecond%\ccIndexSubitem[C]{ch_jarvis_march}{preconditions}
            \ccc{start_p} and \ccc{stop_p} are extreme points with respect to
            the points in the range [\ccc{first},\ccc{beyond}) and \ccc{stop_p}
            is an element of range [\ccc{first},\ccc{beyond}).}


The default traits class \ccc{Default_traits} is the kernel in which the
type \ccc{ForwardIterator::value_type} is defined.


\ccHeading{Requirements}
\begin{enumerate}
   \item    \ccc{ForwardIterator::value_type} and 
            \ccc{OutputIterator::value_type}
            are equivalent to \ccc{Traits::Point_2}.
   \item    \ccc{Traits} defines the following subset of types from
            the concept ConvexHullTraits\_2 and their corresponding member
            %\ccIndexMainItem[c]{ConvexHullTraits_2}
            functions that return instances of these types:
            \begin{itemize}
                \item \ccc{Traits::Point_2},
		\item \ccc{Traits::Equal_2},
                \item \ccc{Traits::Less_rotate_ccw_2}.
            \end{itemize}
\end{enumerate}

\ccSeeAlso

\ccRefIdfierPage{CGAL::ch_jarvis} \\
\ccRefIdfierPage{CGAL::lower_hull_points_2} \\
\ccRefIdfierPage{CGAL::upper_hull_points_2} \\

\ccIndexSubitemEnd{convex hull, 2D}{Jarvis march}
\ccIndexSubitemEnd{convex hull, 2D}{gift-wrapping}
\ccIndexSubitemEnd{extreme points, 2D}{between two points}

\ccImplementation

The function uses the Jarvis march (gift-wrapping)
algorithm \cite{j-ichfs-73}. This algorithm requires $O(n h)$ time 
in the worst case for $n$ input points with $h$ extreme points.  
\end{ccRefFunction}

% +------------------------------------------------------------------------+
%%RefPage: end of main body, begin of footer
% EOF
% +------------------------------------------------------------------------+

