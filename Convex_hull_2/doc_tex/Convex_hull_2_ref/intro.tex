% +------------------------------------------------------------------------+
% | Reference manual page: ConvexHull/intro.tex
% +------------------------------------------------------------------------+
% | 10.5.2001   Susan Hert and Stefan Schirra
% | Package: ConvexHull
% | 
% |
% +------------------------------------------------------------------------+

%\clearpage
%\section{Reference Pages for 2D Convex Hulls and Extreme Points}
\ccRefChapter{2D Convex Hulls and Extreme Points}
\label{chap:convex_hull_2_ref}
\ccChapterAuthor{Susan Hert \and Stefan Schirra}



A subset $S \subseteq \R^2$ is convex if for any two points $p$ and $q$
in the set the line segment with endpoints $p$ and $q$ is contained
in $S$. The convex hull\ccIndexMainItemDef{convex hull} of a set $S$ is 
the smallest convex set containing
$S$. The convex hull of a set of points $P$ is a convex 
polygon with vertices in $P$.  A point in $P$ is an extreme point 
(with respect to $P$)\ccIndexMainItemDef{extreme point} if it is a vertex 
of the convex hull of $P$.

\cgal\ provides functions for computing convex hulls in two dimensions
as well as functions for testing if a given set of points is strongly 
convex or not.  There are also a number of functions available for computing 
particular extreme points in 2D and subsequences of the hull points, such as 
the lower hull or upper hull of a set of points.

\section{Classified Reference Pages}  

\ccHeading{Assertions}
\ccIndexSubitem{convex hull, 2D}{assertion flags}
\ccIndexSubitem{assertion flags}{convex hull, 2D}
The assertion flags for the convex hull and extreme point algorithms
use \ccc{CH} in their names (\textit{e.g.}, \ccc{CGAL_CH_NO_POSTCONDITIONS}).
For the convex hull algorithms, the postcondition
check tests only convexity (if not disabled), but not containment of the
input points in the polygon or polyhedron defined by the output points.
The latter is considered an expensive checking and can be enabled by
defining \ccc{CGAL_CH_CHECK_EXPENSIVE}%
\ccIndexAssertionFlagName{CGAL_CH_CHECK_EXPENSIVE}.


\ccHeading{Concepts}


\ccRefConceptPage{ConvexHullTraits_2}


\ccHeading{Traits Classes}


\ccRefIdfierPage{CGAL::Convex_hull_constructive_traits_2<R>}  \\
\ccRefIdfierPage{CGAL::Convex_hull_projective_xy_traits_2<Point_3>}  \\
\ccRefIdfierPage{CGAL::Convex_hull_projective_xz_traits_2<Point_3>}  \\
\ccRefIdfierPage{CGAL::Convex_hull_projective_yz_traits_2<Point_3>}  \\
\ccRefIdfierPage{CGAL::Convex_hull_traits_2<R>} 


\ccHeading{Convex Hull Functions}

\ccRefIdfierPage{CGAL::ch_akl_toussaint} \\
\ccRefIdfierPage{CGAL::ch_bykat} \\
\ccRefIdfierPage{CGAL::ch_eddy} \\
\ccRefIdfierPage{CGAL::ch_graham_andrew} \\
\ccRefIdfierPage{CGAL::ch_jarvis} \\
\ccRefIdfierPage{CGAL::ch_melkman} \\
\ccRefIdfierPage{CGAL::convex_hull_2} 

\ccHeading{Convexity Checking Functions}

\ccRefIdfierPage{CGAL::is_ccw_strongly_convex_2} \\
\ccRefIdfierPage{CGAL::is_cw_strongly_convex_2} 

\ccHeading{Hull Subsequence Functions}

\ccRefIdfierPage{CGAL::ch_graham_andrew_scan} \\
\ccRefIdfierPage{CGAL::ch_jarvis_march} \\
\ccRefIdfierPage{CGAL::lower_hull_points_2} \\
\ccRefIdfierPage{CGAL::upper_hull_points_2}


\ccHeading{Extreme Point Functions}

\ccRefIdfierPage{CGAL::ch_e_point} \\
\ccRefIdfierPage{CGAL::ch_nswe_point} \\
\ccRefIdfierPage{CGAL::ch_n_point} \\
\ccRefIdfierPage{CGAL::ch_ns_point} \\
\ccRefIdfierPage{CGAL::ch_s_point} \\
\ccRefIdfierPage{CGAL::ch_w_point} \\
\ccRefIdfierPage{CGAL::ch_we_point} 

\clearpage


