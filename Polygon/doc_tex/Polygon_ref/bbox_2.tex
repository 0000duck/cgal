% +------------------------------------------------------------------------+
% | Reference manual page: bbox_2.tex
% +------------------------------------------------------------------------+
% | 21.06.2001   Author
% | Package: Polygon
% | 
\RCSdef{\RCSleftvertexRev}{$Id$}
\RCSdefDate{\RCSleftvertexDate}{$Date$}
% |
%%RefPage: end of header, begin of main body
% +------------------------------------------------------------------------+


\begin{ccRefFunction}{bbox_2}

%% \ccHtmlCrossLink{}     %% add further rules for cross referencing links
%% \ccHtmlIndexC[function]{} %% add further index entries

\ccDefinition
  
%The function \ccRefName\ does this and that.

\ccInclude{CGAL/Polygon_2_algorithms.h}

\ccFunction{template <class InputIterator, class Traits>
Bbox_2 bbox_2(InputIterator first,
                                   InputIterator last,
                                   const Traits& traits);}
{
Returns the bounding box of the range \ccStyle{[first,last)}. 
}

\ccHeading{Requirements}
%\ccIndexSubitem[C]{bbox_2}{requirements}
\begin{enumerate}
    \item \ccc{Traits} is a model of the concept 
	  PolygonTraits\_2\ccIndexMainItem[c]{PolygonTraits_2}.
	  In fact, only the members \ccc{Construct_bbox_2} and
	  \ccc{construct_bbox_2_object} are used.
    \item \ccc{InputIterator::value_type} should be \ccc{Traits::Point_2},
\end{enumerate}


\ccSeeAlso
\ccRefIdfierPage{CGAL::Polygon_2<PolygonTraits_2, Container>}

%\ccExample
%
%% \ccIncludeExampleCode{examples/Polygon/bbox_2_prog.C}

\end{ccRefFunction}

% +------------------------------------------------------------------------+
%%RefPage: end of main body, begin of footer
% EOF
% +------------------------------------------------------------------------+

