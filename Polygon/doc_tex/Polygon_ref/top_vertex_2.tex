% +------------------------------------------------------------------------+
% | Reference manual page: top_vertex_2.tex
% +------------------------------------------------------------------------+
% | 21.06.2001   Author
% | Package: Polygon
% | 
\RCSdef{\RCStopvertexRev}{$Id$}
\RCSdefDate{\RCStopvertexDate}{$Date$}
% |
%%RefPage: end of header, begin of main body
% +------------------------------------------------------------------------+


\begin{ccRefFunction}{top_vertex_2}  %% add template arg's if necessary

%% \ccHtmlCrossLink{}     %% add further rules for cross referencing links
%% \ccHtmlIndexC[function]{} %% add further index entries

\ccDefinition
  
%The function \ccRefName\ does this and that.

\ccInclude{CGAL/Polygon_2_algorithms.h}

\ccFunction{template <class ForwardIterator, class Traits>
ForwardIterator top_vertex_2(ForwardIterator first,
                                   ForwardIterator last,
                                   const Traits& traits);}
{
Returns an iterator to the topmost point from the range
\ccStyle{[first,last)}. In case of a tie, the point
with the largest \ccStyle{x}-coordinate is taken.
}

\ccHeading{Requirements}
\ccIndexSubitem[C]{top_vertex_2}{requirements}
\begin{enumerate}
    \item \ccc{Traits} is a model of the concept 
	  PolygonTraits\_2\ccIndexMainItem[c]{PolygonTraits_2}.
	  In fact, only the members \ccc{Less_yx_2} and
	  \ccc{less_yx_2_object} are used.
    \item \ccc{ForwardIterator::value_type} should be \ccc{Traits::Point_2},
\end{enumerate}


\ccSeeAlso

\ccRefIdfierPage{CGAL::left_vertex_2} \\
\ccRefIdfierPage{CGAL::right_vertex_2} \\
\ccRefIdfierPage{CGAL::bottom_vertex_2} \\
\ccRefIdfierPage{CGAL::Polygon_2<PolygonTraits_2, Container>}

%\ccExample
%
%% \ccIncludeExampleCode{examples/Polygon/top_vertex_2_prog.cpp}

\end{ccRefFunction}

% +------------------------------------------------------------------------+
%%RefPage: end of main body, begin of footer
% EOF
% +------------------------------------------------------------------------+

