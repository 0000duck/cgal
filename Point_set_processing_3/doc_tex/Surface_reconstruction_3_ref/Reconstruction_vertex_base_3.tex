% +------------------------------------------------------------------------+
% | Reference manual page: Reconstruction_vertex_base_3.tex
% +------------------------------------------------------------------------+
% | 05.03.2008   Pierre Alliez, Laurent Saboret, Gael Guennebaud
% | Package: Surface_reconstruction_3
% |
\RCSdef{\RCSReconstructionvertexbaseRev}{$Id$}
\RCSdefDate{\RCSReconstructionvertexbaseDate}{$Date$}
% |
\ccRefPageBegin
%%RefPage: end of header, begin of main body
% +------------------------------------------------------------------------+


\begin{ccRefClass}{Reconstruction_vertex_base_3<GeomTraits, TriangulationVertexBase_3>}

%% \ccHtmlCrossLink{}     %% add further rules for cross referencing links
%% \ccHtmlIndexC[class]{} %% add further index entries

\ccDefinition

% The section below is automatically generated. Do not edit!
%START-AUTO(\ccInclude)

\ccInclude{CGAL/Reconstruction_triangulation_3.h}

%END-AUTO(\ccInclude)

% The section below is automatically generated. Do not edit!
%START-AUTO(\ccDefinition)

The \ccc{Reconstruction_vertex_base_3} class is the default vertex class of the \ccc{Reconstruction_triangulation_3} class.

It provides the interface requested by the \ccc{Poisson_reconstruction_function} class:\begin{itemize}
\item Each vertex owns a normal vector.\item A vertex is either an input point or a Steiner point added by Delaunay refinement.\item In order to solve a linear system over the triangulation, a vertex may be constrained or not, and has a unique index.\end{itemize}

%END-AUTO(\ccDefinition)

\ccParameters

% The section below is automatically generated. Do not edit!
%START-AUTO(\ccParameters)

template$<$  \\
typename Gt,   \\
typename Vb = \ccc{Triangulation_vertex_base_3<Gt>}$>$   \\
class \ccc{Reconstruction_vertex_base_3};

\ccCommentHeading{Parameters}
\begin{description}
\item \ccc{Gt}: Geometric traits class / \ccc{Point_3} is a model of \ccc{PointWithNormal_3}. \item \ccc{Cb}: Vertex base class, model of \ccc{TriangulationVertexBase_3}. \end{description}

%END-AUTO(\ccParameters)

\ccInheritsFrom

% The section below is automatically generated. Do not edit!
%START-AUTO(\ccInheritsFrom)

Vb

%END-AUTO(\ccInheritsFrom)

\ccIsModel

% The section below is automatically generated. Do not edit!
%START-AUTO(\ccIsModel)

Model of the \ccc{ReconstructionVertexBase_3} concept.

%END-AUTO(\ccIsModel)

\ccTypes

% The section below is automatically generated. Do not edit!
%START-AUTO(\ccTypes)

\ccNestedType{FT}
{
}
\ccGlue
\ccNestedType{Point}
{
Model of \ccc{PointWithNormal_3}.
}
\ccGlue
\ccNestedType{Point_with_normal}
{
Model of \ccc{PointWithNormal_3}.
}
\ccGlue
\ccNestedType{Normal}
{
Model of \ccc{Kernel::Vector_3} concept.
}
\ccGlue

%END-AUTO(\ccTypes)

\ccCreation
\ccCreationVariable{v}  %% choose variable name

% The section below is automatically generated. Do not edit!
%START-AUTO(\ccCreation)

\ccConstructor{Reconstruction_vertex_base_3();}
{
}
\ccGlue
\ccConstructor{Reconstruction_vertex_base_3(const Point& p);}
{
}
\ccGlue
\ccConstructor{Reconstruction_vertex_base_3(const Point& p, Cell_handle c);}
{
}
\ccGlue
\ccConstructor{Reconstruction_vertex_base_3(Cell_handle c);}
{
}
\ccGlue

%END-AUTO(\ccCreation)

\ccOperations

% The section below is automatically generated. Do not edit!
%START-AUTO(\ccOperations)

\ccMethod{bool constrained() const;}
{
Is vertex constrained?
}
\ccGlue
\ccMethod{bool& constrained();}
{
}
\ccGlue
\ccMethod{FT f() const;}
{
Get/set the value of the implicit function.
}
\ccGlue
\ccMethod{FT& f();}
{
}
\ccGlue
\ccMethod{double average_spacing() const;}
{
Get/set average spacing at each input point.
}
\ccGlue
\ccMethod{double& average_spacing();}
{
}
\ccGlue
\ccMethod{unsigned char type() const;}
{
Get/set the type = INPUT or STEINER.
}
\ccGlue
\ccMethod{unsigned char& type();}
{
}
\ccGlue
\ccMethod{unsigned int index() const;}
{
Get/set the index in matrix.
}
\ccGlue
\ccMethod{unsigned int& index();}
{
}
\ccGlue
\ccMethod{const Normal& normal() const;}
{
Get/set normal (vector + orientation).
}
\ccGlue
\ccMethod{Normal& normal();}
{
}
\ccGlue
\ccMethod{int tag() const;}
{
General purpose tag.
}
\ccGlue
\ccMethod{int& tag();}
{
}
\ccGlue

%END-AUTO(\ccOperations)

\ccSeeAlso

\ccRefIdfierPage{CGAL::Reconstruction_triangulation_3<GeomTraits, TriangulationDataStructure_3>}  \\
\ccRefIdfierPage{CGAL::Reconstruction_cell_base_3<GeomTraits, TriangulationCellBase_3>}  \\
\ccRefIdfierPage{CGAL::Poisson_reconstruction_function<GeomTraits, ReconstructionTriangulation_3>}  \\

\end{ccRefClass}

% +------------------------------------------------------------------------+
%%RefPage: end of main body, begin of footer
\ccRefPageEnd
% EOF
% +------------------------------------------------------------------------+

