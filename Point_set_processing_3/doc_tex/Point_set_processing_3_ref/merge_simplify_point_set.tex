% +------------------------------------------------------------------------+
% | Reference manual page: grid_simplify_point_set.tex
% +------------------------------------------------------------------------+
% | 02.06.2008   Pierre Alliez, Laurent Saboret, Gael Guennebaud
% | Package: Point_set_processing_3
% |
\RCSdef{\RCSmergesimplificationRev}{$Id$}
\RCSdefDate{\RCSmergesimplificationDate}{$Date$}
% |
\ccRefPageBegin
%%RefPage: end of header, begin of main body
% +------------------------------------------------------------------------+


\begin{ccRefFunction}{grid_simplify_point_set}  %% add template arg's if necessary

%% \ccHtmlCrossLink{}     %% add further rules for cross referencing links
%% \ccHtmlIndexC[function]{} %% add further index entries

\ccDefinition

\ccc{CGAL::grid_simplify_point_set()} iteratively iteratively merges pairs of points which are epsilon-closed. More specifically, 
it merges points which belong to the same cell of an axis-aligned grid of cell size the \ccc{epsilon} parameter. This method modifies the order of input points so as to pack all remaining points first, and returns an iterator over the first point to remove (see erase-remove idiom). For this reason it should not be called on sorted containers.

\ccInclude{CGAL/grid_simplify_point_set.h}

% The section below is automatically generated. Do not edit!
%START-AUTO(\ccDefinition)

\ccFunction{template<typename ForwardIterator, typename PointPMap, typename Kernel> ForwardIterator grid_simplify_point_set(ForwardIterator first, ForwardIterator beyond, PointPMap point_pmap, double epsilon, const Kernel& kernel);}
{
Merge points which belong to the same cell of a grid of cell size = epsilon.
This method modifies the order of input points, and returns an iterator over the first point to remove (see erase-remove idiom). Warning: this method should not be called on sorted containers.
\ccPrecond epsilon $>$ 0.
\ccCommentHeading{Template Parameters}
\begin{description}
\item \ccc{ForwardIterator}: iterator over input points. \item \ccc{PointPMap}: is a model of \ccc{boost::ReadablePropertyMap} with a \ccc{value_type} = \ccc{Point_3<Kernel>}. It can be omitted if ForwardIterator \ccc{value_type} is convertible to \ccc{Point_3<Kernel>}. \item \ccc{Kernel}: Geometric traits class. It can be omitted and deduced automatically from PointPMap \ccc{value_type}.\end{description}
\ccCommentHeading{Returns} iterator over the first point to remove.
\ccCommentHeading{Parameters}
\begin{description}
\item \ccc{first}: iterator over the first input point. \item \ccc{beyond}: past-the-end iterator. \item \ccc{point_pmap}: property map ForwardIterator -$>$ \ccc{Point_3} \item \ccc{epsilon}: tolerance value when merging 3D points. \item \ccc{kernel}: geometric traits. \end{description}
}
\ccGlue

%END-AUTO(\ccDefinition)

\ccSeeAlso

\ccRefIdfierPage{CGAL::random_simplify_point_set}  \\

\ccExample

\begin{ccExampleCode}
typedef CGAL::Exact_predicates_inexact_constructions_kernel Kernel;
typedef Kernel::Point_3 Point;
std::vector<Point> points = ...;
double epsilon = 0.001;

std::vector<Point>::iterator first_iterator_to_remove =
  CGAL::grid_simplify_point_set(points.begin(), points.end(),
                                 epsilon);
std::erase(std::remove(first_iterator_to_remove, points.end()), // erase-remove idiom
           points.end());
\end{ccExampleCode}

\end{ccRefFunction}

% +------------------------------------------------------------------------+
%%RefPage: end of main body, begin of footer
\ccRefPageEnd
% EOF
% +------------------------------------------------------------------------+

