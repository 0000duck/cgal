% +------------------------------------------------------------------------+
% | Reference manual page: write_xyz_points.tex
% +------------------------------------------------------------------------+
% | 07.01.2009   Pierre Alliez, Laurent Saboret, Gael Guennebaud
% | Package: Point_set_processing_3
% |
\RCSdef{\RCSwritexyzpointsetRev}{$Id$}
\RCSdefDate{\RCSwritexyzpointsetDate}{$Date$}
% |
\ccRefPageBegin
%%RefPage: end of header, begin of main body
% +------------------------------------------------------------------------+


\begin{ccRefFunction}{write_xyz_points}  %% add template arg's if necessary

%% \ccHtmlCrossLink{}     %% add further rules for cross referencing links
%% \ccHtmlIndexC[function]{} %% add further index entries

\ccDefinition

\ccc{CGAL::write_xyz_points()} saves points (positions only) to a .xyz ASCII stream.
\ccc{CGAL::write_xyz_points_and_normals()} saves points (positions + normals) to a .xyz ASCII stream.

\ccInclude{CGAL/write_xyz_points.h}

% The section below is automatically generated. Do not edit!
%START-AUTO(\ccDefinition)

\ccFunction{template<typename ForwardIterator, typename PointPMap, typename Kernel> bool write_xyz_points(std::ostream& stream, ForwardIterator first, ForwardIterator beyond, PointPMap point_pmap, const Kernel& kernel);}
{
Saves the [first, beyond) range of points (positions only) to a .xyz ASCII stream.
\ccCommentHeading{Template Parameters}
\begin{description}
\item \ccc{ForwardIterator}: iterator over input points. \item \ccc{PointPMap}: is a model of \ccc{boost::ReadablePropertyMap} with a \ccc{value_type} = \ccc{Point_3<Kernel>}. It can be omitted if ForwardIterator \ccc{value_type} is convertible to \ccc{Point_3<Kernel>}. \item \ccc{Kernel}: Geometric traits class. It can be omitted and deduced automatically from PointPMap \ccc{value_type}.\end{description}
\ccCommentHeading{Returns} true on success.
\ccCommentHeading{Parameters}
\begin{description}
\item \ccc{stream}: output stream. \item \ccc{first}: iterator over the first input point. \item \ccc{beyond}: past-the-end iterator over the input points. \item \ccc{point_pmap}: property map ForwardIterator -$>$ \ccc{Point_3}. \item \ccc{kernel}: geometric traits. \end{description}
}
\ccGlue

%END-AUTO(\ccDefinition)

% \end{ccRefFunction}
%
% \begin{ccRefFunction}{write_xyz_points_and_normals}  %% add template arg's if necessary

% The section below is automatically generated. Do not edit!
%START-AUTO(\ccDefinition)

\ccFunction{template<typename ForwardIterator, typename PointPMap, typename NormalPMap, typename Kernel> bool write_xyz_points_and_normals(std::ostream& stream, ForwardIterator first, ForwardIterator beyond, PointPMap point_pmap, NormalPMap normal_pmap, const Kernel& kernel);}
{
Saves the [first, beyond) range of points (positions + normals) to a .xyz ASCII stream.
\ccPrecond normals must be unit vectors.
\ccCommentHeading{Template Parameters}
\begin{description}
\item \ccc{ForwardIterator}: iterator over input points. \item \ccc{PointPMap}: is a model of \ccc{boost::ReadablePropertyMap} with a \ccc{value_type} = \ccc{Point_3<Kernel>}. It can be omitted if ForwardIterator \ccc{value_type} is convertible to \ccc{Point_3<Kernel>}. \item \ccc{NormalPMap}: is a model of \ccc{boost::WritablePropertyMap} with a \ccc{value_type} = \ccc{Vector_3<Kernel>}. \item \ccc{Kernel}: Geometric traits class. It can be omitted and deduced automatically from PointPMap \ccc{value_type}.\end{description}
\ccCommentHeading{Returns} true on success.
\ccCommentHeading{Parameters}
\begin{description}
\item \ccc{stream}: output stream. \item \ccc{first}: iterator over the first input point. \item \ccc{beyond}: past-the-end iterator over the input points. \item \ccc{point_pmap}: property map ForwardIterator -$>$ \ccc{Point_3}. \item \ccc{normal_pmap}: property map ForwardIterator -$>$ \ccc{Vector_3}. \item \ccc{kernel}: geometric traits. \end{description}
}
\ccGlue

%END-AUTO(\ccDefinition)

\ccSeeAlso

\ccRefIdfierPage{CGAL::read_xyz_points}  \\
\ccRefIdfierPage{CGAL::read_off_points}  \\
\ccRefIdfierPage{CGAL::write_off_points}  \\

\ccExample

See \ccc{read_write_xyz_point_set_example.cpp}.

\end{ccRefFunction}

% +------------------------------------------------------------------------+
%%RefPage: end of main body, begin of footer
\ccRefPageEnd
% EOF
% +------------------------------------------------------------------------+

