% +------------------------------------------------------------------------+
% | Reference manual page: jet_smooth_point_set.tex
% +------------------------------------------------------------------------+
% | 02.06.2008   Pierre Alliez, Laurent Saboret, Gael Guennebaud
% | Package: Point_set_processing_3
% |
\RCSdef{\RCSjetsmoothingRev}{$Id$}
\RCSdefDate{\RCSjetsmoothingDate}{$Date$}
% |
\ccRefPageBegin
%%RefPage: end of header, begin of main body
% +------------------------------------------------------------------------+


\begin{ccRefFunction}{jet_smooth_point_set}  %% add template arg's if necessary

%% \ccHtmlCrossLink{}     %% add further rules for cross referencing links
%% \ccHtmlIndexC[function]{} %% add further index entries

\ccDefinition

\ccc{CGAL::jet_smooth_point_set()} smooths a point set by fitting for each point a jet surface and projecting it onto the jet. The default jet surfaces are quadrics. The function exists in four variants: it may either alter (mutate) the input point set or create a copy; and it may either require a kernel or deduce it from input data.

\ccInclude{CGAL/jet_smooth_point_set.h}

% The section below is automatically generated. Do not edit!
%START-AUTO(\ccDefinition)

\ccFunction{template<typename InputIterator, typename OutputIterator, typename Kernel> OutputIterator jet_smooth_point_set(InputIterator first, InputIterator beyond, OutputIterator output, const unsigned int k, const Kernel& kernel, const unsigned int degree_fitting = 2, const unsigned int degree_monge = 2);}
{
Smooth a point set using jet fitting on the k nearest neighbors and reprojection onto the jet. This variant requires the kernel.
\ccPrecond k $>$= 2.
\ccCommentHeading{Template Parameters}
\begin{description}
\item \ccc{InputIterator}: \ccc{value_type} must be convertible to \ccc{Point_3<Kernel>}. \item \ccc{OutputIterator}: \ccc{value_type} must be convertible from InputIterator's \ccc{value_type}. \item \ccc{Kernel}: Geometric traits class.\end{description}
\ccCommentHeading{Returns} past-the-end output iterator.
\ccCommentHeading{Parameters}
\begin{description}
\item \ccc{first}: iterator over the first input point. \item \ccc{beyond}: past-the-end iterator over input points. \item \ccc{output}: iterator over the first output point. \item \ccc{k}: number of neighbors. \item \ccc{kernel}: geometric traits. \end{description}
}
\ccGlue
\ccFunction{template<typename ForwardIterator, typename Kernel> void jet_smooth_point_set(ForwardIterator first, ForwardIterator beyond, unsigned int k, const Kernel& kernel, const unsigned int degree_fitting = 2, const unsigned int degree_monge = 2);}
{
Smooth a point set using jet fitting on the k nearest neighbors and reprojection onto the jet. This function is mutating the input point set. This variant requires the kernel.
Warning: This method moves the points, thus should not be called on containers sorted wrt points position.
\ccPrecond k $>$= 2.
\ccCommentHeading{Template Parameters}
\begin{description}
\item \ccc{ForwardIterator}: \ccc{value_type} must be convertible to \ccc{Point_3<Kernel>}. \item \ccc{Kernel}: Geometric traits class. \end{description}
\ccCommentHeading{Parameters}
\begin{description}
\item \ccc{first}: iterator over the first input/output point. \item \ccc{beyond}: past-the-end iterator. \item \ccc{k}: number of neighbors. \item \ccc{kernel}: geometric traits. \end{description}
}
\ccGlue
\ccFunction{template<typename InputIterator, typename OutputIterator> OutputIterator jet_smooth_point_set(InputIterator first, InputIterator beyond, OutputIterator output, unsigned int k, const unsigned int degree_fitting = 2, const unsigned int degree_monge = 2);}
{
Smooths points by fitting jet surfaces over their k nearest neighbors and projecting onto the jets. This variant deduces the kernel from iterator types.
\ccPrecond k $>$= 2.
\ccCommentHeading{Template Parameters}
\begin{description}
\item \ccc{InputIterator}: \ccc{value_type} must be convertible to \ccc{Point_3<Kernel>}. \item \ccc{OutputIterator}: \ccc{value_type} must be convertible from InputIterator's \ccc{value_type}.\end{description}
\ccCommentHeading{Returns} past-the-end output iterator.
\ccCommentHeading{Parameters}
\begin{description}
\item \ccc{first}: iterator over the first input point \item \ccc{beyond}: past-the-end iterator over input points \item \ccc{output}: iterator over the first output point \item \ccc{k}: number of neighbors \end{description}
}
\ccGlue
\ccFunction{template<typename ForwardIterator> void jet_smooth_point_set(ForwardIterator first, ForwardIterator beyond, unsigned int k, const unsigned int degree_fitting = 2, const unsigned int degree_monge = 2);}
{
Smooths points by fitting jet surfaces over their k nearest neighbors and projecting onto the jets. This function is mutating the input point set. This variant deduces the kernel from iterator types.
Warning: As this method relocates the points, it should not be called on containers sorted w.r.t. point locations.
\ccPrecond k $>$= 2.
\ccCommentHeading{Template Parameters}
\ccc{ForwardIterator}: \ccc{value_type} must be convertible to \ccc{Point_3<Kernel>}. 
\ccCommentHeading{Parameters}
\begin{description}
\item \ccc{first}: iterator over the first input/output point \item \ccc{beyond}: past-the-end iterator \item \ccc{k}: number of neighbors \end{description}
}
\ccGlue

%END-AUTO(\ccDefinition)

\ccExample

\ccIncludeExampleCode{Point_set_processing_3/jet_smoothing_example.cpp}

\end{ccRefFunction}

% +------------------------------------------------------------------------+
%%RefPage: end of main body, begin of footer
\ccRefPageEnd
% EOF
% +------------------------------------------------------------------------+

