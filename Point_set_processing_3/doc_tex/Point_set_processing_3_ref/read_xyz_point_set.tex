% +------------------------------------------------------------------------+
% | Reference manual page: read_xyz_point_set.tex
% +------------------------------------------------------------------------+
% | 07.01.2009   Pierre Alliez, Laurent Saboret, Gael Guennebaud
% | Package: Point_set_processing_3
% |
\RCSdef{\RCSreadxyzpointsetRev}{$Id$}
\RCSdefDate{\RCSreadxyzpointsetDate}{$Date$}
% |
\ccRefPageBegin
%%RefPage: end of header, begin of main body
% +------------------------------------------------------------------------+


\begin{ccRefFunction}{read_xyz_point_set}  %% add template arg's if necessary

%% \ccHtmlCrossLink{}     %% add further rules for cross referencing links
%% \ccHtmlIndexC[function]{} %% add further index entries

\ccDefinition

\ccc{CGAL::read_xyz_point_set()} reads points (positions + normals, if available) from a .xyz file (ASCII).

The \ccc{CGAL::read_xyz_point_set()} function exists in two flavors: 
the output can be either a container of \ccc{Point_3} or of \ccc{PointWithNormal_3} points.

\ccInclude{CGAL/read_xyz_point_set.h}

% The section below is automatically generated. Do not edit!
%START-AUTO(\ccDefinition)

\ccFunction{bool read_xyz_point_set(const char * pFilename, OutputIterator output);}
{
Read points (positions + normals, if available) from a .xyz file (ASCII).
\ccCommentHeading{Template Parameters}
\ccc{OutputIterator}: \ccc{value_type} must be a model of the \ccc{PointWithNormal_3} concept.
\ccCommentHeading{Returns} true on success.
}
\ccGlue
\ccFunction{bool read_xyz_point_set(const char * pFilename, OutputIterator output, bool read_normals);}
{
Read points (positions only) from a .xyz file (ASCII). Normals are ignored.
\ccCommentHeading{Template Parameters}
\ccc{OutputIterator}: \ccc{value_type} must be \ccc{Point_3}.
\ccCommentHeading{Returns} true on success.
}
\ccGlue

%END-AUTO(\ccDefinition)

\ccSeeAlso

\ccRefIdfierPage{CGAL::write_xyz_point_set}  \\
\ccRefIdfierPage{CGAL::read_off_point_set}  \\
\ccRefIdfierPage{CGAL::write_off_point_set}  \\

\ccExample

\ccIncludeExampleCode{Point_set_processing_3/read_write_xyz_point_set_example.cpp}

\end{ccRefFunction}

% +------------------------------------------------------------------------+
%%RefPage: end of main body, begin of footer
\ccRefPageEnd
% EOF
% +------------------------------------------------------------------------+

