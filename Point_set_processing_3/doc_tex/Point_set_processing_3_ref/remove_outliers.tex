% +------------------------------------------------------------------------+
% | Reference manual page: remove_outliers.tex
% +------------------------------------------------------------------------+
% | 02.06.2008   Pierre Alliez, Laurent Saboret, Gael Guennebaud
% | Package: Point_set_processing_3
% |
\RCSdef{\RCSoutlierremovalRev}{$Id$}
\RCSdefDate{\RCSoutlierremovalDate}{$Date$}
% |
\ccRefPageBegin
%%RefPage: end of header, begin of main body
% +------------------------------------------------------------------------+


\begin{ccRefFunction}{remove_outliers}  %% add template arg's if necessary

%% \ccHtmlCrossLink{}     %% add further rules for cross referencing links
%% \ccHtmlIndexC[function]{} %% add further index entries

\ccDefinition


\ccc{CGAL::remove_outliers()} deletes a user-specified fraction of outliers from the input point set. More specifically, it sorts the input points in increasing order of average squared distances to the $k$ nearest neighbors and deletes the points with largest value. This method modifies the order of input points so as to pack all remaining points first, and returns and returns an iterator over the first point to remove (see erase-remove idiom). For this reason it should not be called on sorted containers.

\ccInclude{CGAL/remove_outliers.h}

% The section below is automatically generated. Do not edit!
%START-AUTO(\ccDefinition)

\ccFunction{template<typename ForwardIterator, typename PointPMap, typename Kernel> ForwardIterator remove_outliers(ForwardIterator first, ForwardIterator beyond, PointPMap point_pmap, unsigned int k, double threshold_percent, const Kernel& kernel);}
{
Removes outliers:\begin{itemize}
\item computes average squared distance to the K nearest neighbors,\item sorts the points in increasing order of average distance.\end{itemize}
This method modifies the order of input points so as to pack all remaining points first, and returns an iterator over the first point to remove (see erase-remove idiom). For this reason it should not be called on sorted containers.
\ccPrecond k $>$= 2.
\ccCommentHeading{Template Parameters}
\begin{description}
\item \ccc{ForwardIterator}: iterator over input points. \item \ccc{PointPMap}: is a model of \ccc{boost::ReadablePropertyMap} with a \ccc{value_type} = \ccc{Point_3<Kernel>}. It can be omitted if \ccc{ForwardIterator} \ccc{value_type} is convertible to \ccc{Point_3<Kernel>}. \item \ccc{Kernel}: Geometric traits class. It can be omitted and deduced automatically from \ccc{PointPMap} \ccc{value_type}.\end{description}
\ccCommentHeading{Returns} iterator over the first point to remove.
\ccCommentHeading{Parameters}
\begin{description}
\item \ccc{first}: iterator over the first input point. \item \ccc{beyond}: past-the-end iterator over the input points. \item \ccc{point_pmap}: property map \ccc{ForwardIterator} -$>$ \ccc{Point_3} \item \ccc{k}: number of neighbors. \item \ccc{threshold_percent}: percentage of points to remove. \item \ccc{kernel}: geometric traits. \end{description}
}
\ccGlue

%END-AUTO(\ccDefinition)

\ccExample

See \ccc{remove_outliers_example.cpp}.

\end{ccRefFunction}

% +------------------------------------------------------------------------+
%%RefPage: end of main body, begin of footer
\ccRefPageEnd
% EOF
% +------------------------------------------------------------------------+

