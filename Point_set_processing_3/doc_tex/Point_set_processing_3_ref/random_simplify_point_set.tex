% +------------------------------------------------------------------------+
% | Reference manual page: random_simplify_point_set.tex
% +------------------------------------------------------------------------+
% | 07.01.2009   Pierre Alliez, Laurent Saboret, Gael Guennebaud
% | Package: Point_set_processing_3
% |
\RCSdef{\RCSrandomsimplificationRev}{$Id$}
\RCSdefDate{\RCSrandomsimplificationDate}{$Date$}
% |
\ccRefPageBegin
%%RefPage: end of header, begin of main body
% +------------------------------------------------------------------------+


\begin{ccRefFunction}{random_simplify_point_set}  %% add template arg's if necessary

%% \ccHtmlCrossLink{}     %% add further rules for cross referencing links
%% \ccHtmlIndexC[function]{} %% add further index entries

\ccDefinition

\ccc{CGAL::random_simplify_point_set()} randomly deletes a user-specified fraction of the input points.

\ccInclude{CGAL/random_simplify_point_set.h}

% The section below is automatically generated. Do not edit!
%START-AUTO(\ccDefinition)

\ccFunction{template<typename ForwardIterator, typename Kernel> ForwardIterator random_simplify_point_set(ForwardIterator first, ForwardIterator beyond, double threshold_percent, const Kernel& kernel);}
{
Randomly deletes a user-specified fraction of the input points.
This method modifies the order of input points, and returns an iterator over the first point to remove (see erase-remove idiom). Warning: this method should not be called on sorted containers.
\ccCommentHeading{Template Parameters}
\begin{description}
\item \ccc{ForwardIterator}: \ccc{value_type} must be convertible to \ccc{Point_3<Kernel>}. \item \ccc{Kernel}: Geometric traits class. It can be omitted and deduced automatically from the iterator type.\end{description}
\ccCommentHeading{Returns} iterator over the first point to remove.
\ccCommentHeading{Parameters}
\begin{description}
\item \ccc{first}: iterator over the first input/output point. \item \ccc{beyond}: past-the-end iterator. \item \ccc{threshold_percent}: percentage of points to remove. \item \ccc{kernel}: geometric traits. \end{description}
}
\ccGlue

%END-AUTO(\ccDefinition)

\ccSeeAlso

\ccRefIdfierPage{CGAL::merge_simplify_point_set}  \\

\ccExample

\ccIncludeExampleCode{Point_set_processing_3/random_simplification_example.cpp}

\end{ccRefFunction}

% +------------------------------------------------------------------------+
%%RefPage: end of main body, begin of footer
\ccRefPageEnd
% EOF
% +------------------------------------------------------------------------+

