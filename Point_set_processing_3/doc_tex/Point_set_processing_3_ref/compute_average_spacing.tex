% +------------------------------------------------------------------------+
% | Reference manual page: compute_average_spacing.tex
% +------------------------------------------------------------------------+
% | 02.06.2008   Pierre Alliez, Laurent Saboret, Gael Guennebaud
% | Package: Point_set_processing_3
% |
\RCSdef{\RCSaveragespacingRev}{$Id$}
\RCSdefDate{\RCSaveragespacingDate}{$Date$}
% |
\ccRefPageBegin
%%RefPage: end of header, begin of main body
% +------------------------------------------------------------------------+


\begin{ccRefFunction}{compute_average_spacing}  %% add template arg's if necessary

%% \ccHtmlCrossLink{}     %% add further rules for cross referencing links
%% \ccHtmlIndexC[function]{} %% add further index entries

\ccDefinition

\ccc{CGAL::compute_average_spacing()} computes the average spacing of all points from the input set to their $k$ nearest neighbors.
One version of this function requires a traits class. Another deduces it from the input data points.

% question: can it take oriented points as well?

\ccInclude{CGAL/compute_average_spacing.h}

% The section below is automatically generated. Do not edit!
%START-AUTO(\ccDefinition)

\ccFunction{template<typename InputIterator, typename Kernel> Kernel::FT compute_average_spacing(InputIterator first, InputIterator beyond, unsigned int k, const Kernel& kernel);}
{
Compute average spacing from k nearest neighbors.
\ccPrecond k $>$= 2.
\ccCommentHeading{Template Parameters}
\begin{description}
\item \ccc{InputIterator}: \ccc{value_type} must be convertible to \ccc{Point_3<Kernel>}. \item \ccc{Kernel}: Geometric traits class. It can be omitted and deduced automatically from the iterator type.\end{description}
\ccCommentHeading{Returns} average spacing (scalar).
\ccCommentHeading{Parameters}
\begin{description}
\item \ccc{first}: iterator over the first input point. \item \ccc{beyond}: past-the-end iterator over input points. \item \ccc{k}: number of neighbors. \item \ccc{kernel}: geometric traits. \end{description}
}
\ccGlue

%END-AUTO(\ccDefinition)

\ccExample

See \ccc{average_spacing_example.cpp}.

\end{ccRefFunction}

% +------------------------------------------------------------------------+
%%RefPage: end of main body, begin of footer
\ccRefPageEnd
% EOF
% +------------------------------------------------------------------------+

