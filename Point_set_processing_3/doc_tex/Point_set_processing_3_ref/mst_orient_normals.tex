% +------------------------------------------------------------------------+
% | Reference manual page: mst_orient_normals.tex
% +------------------------------------------------------------------------+
% | 02.06.2008   Pierre Alliez, Laurent Saboret, Gael Guennebaud
% | Package: Point_set_processing_3
% |
\RCSdef{\RCSmstnormalorientationRev}{$Id$}
\RCSdefDate{\RCSmstnormalorientationDate}{$Date$}
% |
\ccRefPageBegin
%%RefPage: end of header, begin of main body
% +------------------------------------------------------------------------+


\begin{ccRefFunction}{mst_orient_normals}  %% add template arg's if necessary

%% \ccHtmlCrossLink{}     %% add further rules for cross referencing links
%% \ccHtmlIndexC[function]{} %% add further index entries

\ccDefinition

\ccc{CGAL::mst_orient_normals()} orients the normals of a point set using the propagation of a seed orientation through a minimum spanning tree of the Riemannian graph \cite{cgal:hddms-srup-92}.

This method modifies the order of input points, and returns an iterator over the first point with an unoriented normal (see erase-remove idiom). \\
Warning: this method should not be called on sorted containers.

\ccInclude{CGAL/mst_orient_normals.h}

% The section below is automatically generated. Do not edit!
%START-AUTO(\ccDefinition)

\ccFunction{template<typename ForwardIterator, typename PointPMap, typename NormalPMap, typename IndexPMap, typename Kernel> ForwardIterator mst_orient_normals(ForwardIterator first, ForwardIterator beyond, PointPMap point_pmap, NormalPMap normal_pmap, IndexPMap index_pmap, unsigned int k, const Kernel& kernel);}
{
Orient the normals of the [first, beyond) range of vertices using the method described by Hoppe, DeRose, Duchamp, McDonald and Stuetzle in {\em Surface reconstruction from unorganized points} \cite{cgal:hddms-srup-92}.
This method modifies the order of input points, and returns an iterator over the first point with an unoriented normal (see erase-remove idiom). Warning: this method should not be called on sorted containers.
\ccCommentHeading{Preconditions}\begin{itemize}
\item Normals must be unit vectors.\item k $>$= 2.\end{itemize}
\ccCommentHeading{Template Parameters}
\begin{description}
\item \ccc{ForwardIterator}: iterator over input points. \item \ccc{PointPMap}: is a model of \ccc{boost::readable_property_map} with a \ccc{value_type} = \ccc{Point_3<Kernel>}. It can be omitted if ForwardIterator's \ccc{value_type} is convertible to \ccc{Point_3<Kernel>}. \item \ccc{NormalPMap}: is a model of \ccc{boost::lvalue_property_map} with a \ccc{value_type} = \ccc{Vector_3<Kernel>}. \item \ccc{IndexPMap}: must be a model of \ccc{boost::readable_property_map} with an integral \ccc{value_type}. It can be omitted and will default to a \ccc{std::map<ForwardIterator, int>}. \item \ccc{Kernel}: Geometric traits class. It can be omitted and deduced automatically from the iterator type.\end{description}
\ccCommentHeading{Returns} iterator over the first point with an unoriented normal.
\ccCommentHeading{Parameters}
\begin{description}
\item \ccc{first}: iterator over the first input point. \item \ccc{beyond}: past-the-end iterator. \item \ccc{point_pmap}: property map ForwardIterator -$>$ \ccc{Point_3} \item \ccc{normal_pmap}: property map ForwardIterator -$>$ \ccc{Vector_3} \item \ccc{index_pmap}: property map ForwardIterator -$>$ index \item \ccc{k}: number of neighbors \item \ccc{kernel}: geometric traits. \end{description}
}
\ccGlue

%END-AUTO(\ccDefinition)

\ccSeeAlso

\ccRefIdfierPage{CGAL::pca_estimate_normals}  \\
\ccRefIdfierPage{CGAL::jet_estimate_normals}  \\

\ccExample

See \ccc{pca_estimate_normals_example.cpp}.

\end{ccRefFunction}

% +------------------------------------------------------------------------+
%%RefPage: end of main body, begin of footer
\ccRefPageEnd
% EOF
% +------------------------------------------------------------------------+

