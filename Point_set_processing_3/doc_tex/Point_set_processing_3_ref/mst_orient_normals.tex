% +------------------------------------------------------------------------+
% | Reference manual page: mst_orient_normals.tex
% +------------------------------------------------------------------------+
% | 02.06.2008   Pierre Alliez, Laurent Saboret, Gael Guennebaud
% | Package: Point_set_processing_3
% |
\RCSdef{\RCSmstnormalorientationRev}{$Id$}
\RCSdefDate{\RCSmstnormalorientationDate}{$Date$}
% |
\ccRefPageBegin
%%RefPage: end of header, begin of main body
% +------------------------------------------------------------------------+


\begin{ccRefFunction}{mst_orient_normals}  %% add template arg's if necessary

%% \ccHtmlCrossLink{}     %% add further rules for cross referencing links
%% \ccHtmlIndexC[function]{} %% add further index entries

\ccDefinition

\ccc{CGAL::mst_orient_normals()} orients the normals of a point set using the propagation of a seed orientation through a minimum spanning tree of the Riemannian graph \cite{cgal:hddms-srup-92}. This method modifies the order of the input points so as to pack all successfully oriented normals first, and returns an iterator over the first point with an unoriented normal (see erase-remove idiom). For this reason it should not be called on sorted containers.

\ccInclude{CGAL/mst_orient_normals.h}

% The section below is automatically generated. Do not edit!
%START-AUTO(\ccDefinition)

% Reduce left margin
\ccThree{123456789012345}{6789012}{}

\ccFunction{template<typename ForwardIterator, typename PointPMap, typename NormalPMap, typename IndexPMap, typename Kernel> ForwardIterator mst_orient_normals(ForwardIterator first, ForwardIterator beyond, PointPMap point_pmap, NormalPMap normal_pmap, IndexPMap index_pmap, unsigned int k, const Kernel& kernel);}
{
Orients the normals of the [first, beyond) range of points using the propagation of a seed orientation through a minimum spanning tree of the Riemannian graph \cite{cgal:hddms-srup-92}. This method modifies the order of input points so as to pack all sucessfully oriented points first, and returns an iterator over the first point with an unoriented normal (see erase-remove idiom). For this reason it should not be called on sorted containers.
\ccCommentHeading{Preconditions}  \\
Normals must be unit vectors. k $>$= 2.
\ccCommentHeading{Template Parameters}  \\
\ccc{ForwardIterator}: iterator over input points. \ccc{PointPMap}: is a model of \ccc{boost::ReadablePropertyMap} with a \ccc{value_type} = \ccc{Point_3<Kernel>}. It can be omitted if \ccc{ForwardIterator} \ccc{value_type} is convertible to \ccc{Point_3<Kernel>}. \ccc{NormalPMap}: is a model of \ccc{boost::ReadWritePropertyMap} with a \ccc{value_type} = \ccc{Vector_3<Kernel>}. \ccc{IndexPMap}: must be a model of \ccc{boost::ReadablePropertyMap} with an integral \ccc{value_type}. It can be omitted and will default to a \ccc{std::map<ForwardIterator, int>}. \ccc{Kernel}: Geometric traits class. It can be omitted and deduced automatically from \ccc{PointPMap} \ccc{value_type}.
\ccCommentHeading{Returns} iterator over the first point with an unoriented normal.
\ccCommentHeading{Parameters}  \\
\ccc{first}: iterator over the first input point. \ccc{beyond}: past-the-end iterator over the input points. \ccc{point_pmap}: property map \ccc{ForwardIterator} -$>$ \ccc{Point_3}. \ccc{normal_pmap}: property map \ccc{ForwardIterator} -$>$ \ccc{Vector_3}. \ccc{index_pmap}: property map \ccc{ForwardIterator} -$>$ index. \ccc{k}: number of neighbors. \ccc{kernel}: geometric traits.
}
\ccGlue

%END-AUTO(\ccDefinition)

\ccSeeAlso

\ccRefIdfierPage{CGAL::pca_estimate_normals}  \\
\ccRefIdfierPage{CGAL::jet_estimate_normals}  \\

\ccExample

See \ccc{pca_estimate_normals_example.cpp}.

\end{ccRefFunction}

% +------------------------------------------------------------------------+
%%RefPage: end of main body, begin of footer
\ccRefPageEnd
% EOF
% +------------------------------------------------------------------------+

