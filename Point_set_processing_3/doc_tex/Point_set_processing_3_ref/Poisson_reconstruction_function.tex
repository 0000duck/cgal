% +------------------------------------------------------------------------+
% | Reference manual page: Poisson_reconstruction_function.tex
% +------------------------------------------------------------------------+
% | 07.09.2007   Pierre Alliez, Laurent Saboret, Gael Guennebaud
% | Package: Point_set_processing_3
% |
\RCSdef{\RCSPoissonreconstructionfunctionRev}{$Id$}
\RCSdefDate{\RCSPoissonreconstructionfunctionDate}{$Date$}
% |
\ccRefPageBegin
%%RefPage: end of header, begin of main body
% +------------------------------------------------------------------------+


\begin{ccRefClass}{Poisson_reconstruction_function<GeomTraits, ReconstructionTriangulation_3>}

%% \ccHtmlCrossLink{}     %% add further rules for cross referencing links
%% \ccHtmlIndexC[class]{} %% add further index entries

\ccDefinition

% Insert image poisson.jpg/eps
\begin{center}
    \label{Point_set_processing_3-fig-poisson}
    % Image
    \begin{ccTexOnly}
        \includegraphics[width=0.7\textwidth]{Point_set_processing_3/poisson} % omit .eps suffix
    \end{ccTexOnly}
    \begin{ccHtmlOnly}
        <img width="70%" border=0 src="../Point_set_processing_3/poisson.jpg"><P>
    \end{ccHtmlOnly}
    % Title
    \begin{figure}[h]
        \caption{Poisson reconstruction}
    \end{figure}
\end{center}

Kazhdan, Bolitho and Hoppe introduced the Poisson Surface Reconstruction algorithm \cite{Kazhdan06}.
Given a set of 3D points with oriented normals (denoted oriented points in the sequel) sampled on the boundary of a 3D solid,
this method solves for an approximate indicator function of the inferred solid, whose gradient best matches the input normals.
The output scalar function, represented in an adaptive octree, is then iso-contoured using an adaptive marching cubes.
\cgal\ implements a variant of this algorithm which solves for a piecewise linear function on a 3D Delaunay triangulation instead of an adaptive octree.

% The section below is automatically generated. Do not edit!
%START-AUTO(\ccDefinition)

\ccc{Poisson_reconstruction_function} computes an indicator function f() piecewise-linear over the tetrahedra. We solve the Poisson equation Laplacian(f) = divergent(normals field) at each vertex of the triangulation via the TAUCS sparse linear solver. One vertex must be constrained.

%END-AUTO(\ccDefinition)

% The section below is automatically generated. Do not edit!
%START-AUTO(\ccInclude)

\ccInclude{CGAL/Poisson_reconstruction_function.h}

%END-AUTO(\ccInclude)

\ccParameters

The class template declaration is:

% The section below is automatically generated. Do not edit!
%START-AUTO(\ccParameters)

template$<$  \\
class Gt,   \\
class \ccc{ReconstructionTriangulation_3}$>$   \\
class \ccc{Poisson_reconstruction_function};

\ccCommentHeading{Parameters}
\begin{description}
\item \ccc{Gt}: Geometric traits class \item \ccc{ReconstructionTriangulation_3}: 3D Delaunay triangulation, model of \ccc{ReconstructionTriangulation_3} concept. \end{description}

%END-AUTO(\ccParameters)

\ccIsModel

% The section below is automatically generated. Do not edit!
%START-AUTO(\ccIsModel)

Model of the ImplicitFunction concept.

%END-AUTO(\ccIsModel)

\ccTypes

% The section below is automatically generated. Do not edit!
%START-AUTO(\ccTypes)

\ccNestedType{Triangulation}
{
}
\ccGlue
\ccNestedType{Geom_traits}
{
Kernel's geometric traits.
}
\ccGlue
\ccNestedType{FT}
{
}
\ccGlue
\ccNestedType{Point}
{
}
\ccGlue
\ccNestedType{Vector}
{
}
\ccGlue
\ccNestedType{Iso_cuboid}
{
}
\ccGlue
\ccNestedType{Sphere}
{
}
\ccGlue
\ccNestedType{Point_with_normal}
{
Model of \ccc{PointWithNormal_3}.
}
\ccGlue
\ccNestedType{Normal}
{
Model of \ccc{Kernel::Vector_3} concept.
}
\ccGlue

%END-AUTO(\ccTypes)

\ccCreation
\ccCreationVariable{fct}  %% variable name for \ccMethod below

% The section below is automatically generated. Do not edit!
%START-AUTO(\ccCreation)

\ccConstructor{Poisson_reconstruction_function(ReconstructionTriangulation_3& pdt);}
{
Creates a scalar function from a set of oriented points. Inserts the iterator range first...beyond into the triangulation pdt, refines it and solves for a piecewise linear scalar function which gradient best matches the input normals. If pdt is empty, create an empty implicit function.
\ccCommentHeading{Parameters}
\ccc{pdt}: \ccc{ReconstructionTriangulation_3} base of the Poisson indicator function. 
}
\ccGlue
\ccConstructor{template<class InputIterator> Poisson_reconstruction_function(ReconstructionTriangulation_3& pdt, InputIterator first, InputIterator beyond);}
{
Creates a scalar function from a set of oriented points. Inserts the iterator range first...beyond into the triangulation pdt, refines it and solves for a piecewise linear scalar function which gradient best matches the input normals.
\ccPrecond the value type of InputIterator must be convertible to \ccc{Point_with_normal}.
\ccCommentHeading{Parameters}
\begin{description}
\item \ccc{pdt}: \ccc{ReconstructionTriangulation_3} base of the Poisson indicator function. \item \ccc{first}: First point to add. \item \ccc{beyond}: Past-the-end point to add. \end{description}
}
\ccGlue

%END-AUTO(\ccCreation)

\ccOperations

% The section below is automatically generated. Do not edit!
%START-AUTO(\ccOperations)

\ccMethod{template<class InputIterator> int insert(InputIterator first, InputIterator beyond);}
{
Insert points.
\ccPrecond the value type of InputIterator must be convertible to \ccc{Point_with_normal}.
\ccCommentHeading{Parameters}
\begin{description}
\item \ccc{first}: First point to add. \item \ccc{beyond}: Past-the-end point to add. \end{description}
\ccCommentHeading{Returns} the number of inserted points.
}
\ccGlue
\ccMethod{void clear();}
{
Remove all points.
}
\ccGlue
\ccMethod{ReconstructionTriangulation_3& triangulation();}
{
Get embedded triangulation.
}
\ccGlue
\ccMethod{const ReconstructionTriangulation_3& triangulation() const;}
{
}
\ccGlue
\ccMethod{Iso_cuboid bounding_box() const;}
{
Returns a bounding box of the inferred surface.
}
\ccGlue
\ccMethod{Sphere bounding_sphere() const;}
{
Returns a sphere bounding the inferred surface.
}
\ccGlue
\ccMethod{Sphere region_of_interest() const;}
{
Get the region of interest, ignoring the outliers. This method is used to define the OpenGL arcball sphere.
}
\ccGlue
\ccMethod{bool compute_implicit_function();}
{
The function \{\ccc{compute_implicit_function}\}() must be called after each insertion of oriented points. It computes the piecewise linear scalar function \ccc{f} by:\begin{itemize}
\item applying Delaunay refinement.\item solving for \ccc{f} at each vertex of the triangulation with a sparse linear solver.\item shifting and orienting \ccc{f} such that 'f = 0' at all input points and 'f$<$0' inside the inferred surface.\end{itemize}
Returns false on error.
}
\ccGlue
\ccMethod{void average_spacing_avg_knn_sq_distance_3();}
{
}
\ccGlue
\ccMethod{unsigned int delaunay_refinement(FT radius_edge_ratio_bound, FT cell_radius_bound, unsigned int max_vertices, FT enlarge_ratio);}
{
Delaunay refinement (break bad tetrahedra, where bad means badly shaped or too big). The normal of Steiner points is set to zero. Return the number of vertices inserted.
\ccCommentHeading{Parameters}
\begin{description}
\item \ccc{radius_edge_ratio_bound}: radius edge ratio bound (ignored if zero) \item \ccc{cell_radius_bound}: cell radius bound (ignored if zero) \item \ccc{max_vertices}: number of vertices bound \item \ccc{enlarge_ratio}: bounding box enlarge ratio \end{description}
}
\ccGlue
\ccMethod{unsigned int delaunay_refinement_shell(FT size_shell, FT sizing, unsigned int max_vertices);}
{
}
\ccGlue
\ccMethod{void extrapolate_normals();}
{
Extrapolate the normals field: compute null normals by averaging neighbor normals.
}
\ccGlue
\ccMethod{FT gaussian_function(FT sigma, FT distance);}
{
}
\ccGlue
\ccMethod{int extrapolate_normals_using_gaussian_kernel();}
{
Extrapolate the normals field. Return the number of normals computed.
}
\ccGlue
\ccMethod{bool solve_poisson(double lambda, double * duration_assembly, double * duration_factorization, double * duration_solve, bool is_normalized = false);}
{
Poisson reconstruction. Return false on error.
}
\ccGlue
\ccMethod{void SaveAsMeshFile();}
{
}
\ccGlue
\ccMethod{FT set_contouring_value(FT contouring_value);}
{
Shift and orient the implicit function such that:\begin{itemize}
\item the implicit function = 0 for points / f() = \ccc{contouring_value},\item the implicit function $<$ 0 inside the surface.\end{itemize}
Return the minimum value of the implicit function.
}
\ccGlue
\ccMethod{FT f(const Point& p) const;}
{
Evaluates the implicit function at a given 3D query point.
}
\ccGlue
\ccMethod{FT operator()(const Point& p) const;}
{
[ImplicitFunction interface]
Evaluates the implicit function at a given 3D query point.
}
\ccGlue
\ccMethod{Point get_inner_point() const;}
{
Returns a point located inside the inferred surface.
}
\ccGlue
\ccMethod{FT average_value_at_input_vertices() const;}
{
Get average value of the implicit function over input vertices.
}
\ccGlue
\ccMethod{FT median_value_at_input_vertices() const;}
{
Get median value of the implicit function over input vertices.
}
\ccGlue
\ccMethod{FT min_value_at_input_vertices() const;}
{
Get min value of the implicit function over input vertices.
}
\ccGlue
\ccMethod{FT max_value_at_input_vertices() const;}
{
Get max value of the implicit function over input vertices.
}
\ccGlue
\ccMethod{FT median_value_at_convex_hull() const;}
{
Get median value of the implicit function over convex hull vertices.
}
\ccGlue
\ccMethod{FT average_value_at_convex_hull() const;}
{
Get average value of the implicit function over convex hull vertices.
}
\ccGlue

%END-AUTO(\ccOperations)

\ccSeeAlso

\ccRefIdfierPage{CGAL::Reconstruction_triangulation_3<GeomTraits, TriangulationDataStructure_3>}  \\
\ccRefIdfierPage{CGAL::Reconstruction_cell_base_3<GeomTraits, TriangulationCellBase_3>}  \\
\ccRefIdfierPage{CGAL::Reconstruction_vertex_base_3<GeomTraits, TriangulationVertexBase_3>}  \\
\ccRefIdfierPage{CGAL::APSS_reconstruction_function<GeomTraits>}  \\

\ccExample

\ccIncludeExampleCode{Point_set_processing_3/poisson_reconstruction_example.cpp}

\end{ccRefClass}

% +------------------------------------------------------------------------+
%%RefPage: end of main body, begin of footer
\ccRefPageEnd
% EOF
% +------------------------------------------------------------------------+

