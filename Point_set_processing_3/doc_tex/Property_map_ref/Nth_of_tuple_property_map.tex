% +------------------------------------------------------------------------+
% | Reference manual page: Nth_of_tuple_property_map.tex
% +------------------------------------------------------------------------+
% | 16.06.2009   Pierre Alliez, Laurent Saboret, Gael Guennebaud
% | Package: Property_map
% |
\RCSdef{\RCSNthoftuplepropertymapRev}{$Id$}
\RCSdefDate{\RCSNthoftuplepropertymapDate}{$Date$}
% |
\ccRefPageBegin
%%RefPage: end of header, begin of main body
% +------------------------------------------------------------------------+


\begin{ccRefClass}{Nth_of_tuple_property_map<N, Tuple>}

%% \ccHtmlCrossLink{}     %% add further rules for cross referencing links
%% \ccHtmlIndexC[class]{} %% add further index entries

\ccDefinition

% The section below is automatically generated. Do not edit!
%START-AUTO(\ccDefinition)

Property map that accesses the Nth item of a \ccc{boost::tuple} Tuple from a Tuple$\ast$ pointer (or in general an iterator over Tuple elements).

%END-AUTO(\ccDefinition)

% The section below is automatically generated. Do not edit!
%START-AUTO(\ccInclude)

\ccInclude{CGAL/property_map.h}

%END-AUTO(\ccInclude)

\ccParameters

% The section below is automatically generated. Do not edit!
%START-AUTO(\ccParameters)

template$<$int N,   \\
typename Tuple$>$   \\
struct \ccc{Nth_of_tuple_property_map};

\ccCommentHeading{Parameters}  \\
\ccc{Index}: of the item to access. \ccc{Tuple}: Instance of \ccc{boost::tuple}.

%END-AUTO(\ccParameters)

\ccInheritsFrom

% The section below is automatically generated. Do not edit!
%START-AUTO(\ccInheritsFrom)

\ccc{boost::put_get_helper<boost::tuples::element<N, Tuple>::type}\& , \ccc{Nth_of_tuple_property_map<N, Tuple>} $>$

%END-AUTO(\ccInheritsFrom)

\ccIsModel

% The section below is automatically generated. Do not edit!
%START-AUTO(\ccIsModel)

Model of \ccc{boost::LvaluePropertyMap} concept.

%END-AUTO(\ccIsModel)

\ccTypes

% The section below is automatically generated. Do not edit!
%START-AUTO(\ccTypes)

\ccNestedType{key_type}
{
typedef to \ccc{Tuple*}
}
\ccGlue
\ccNestedType{value_type}
{
typedef to \ccc{boost::tuples::element<N, Tuple>::type}
}
\ccGlue
\ccNestedType{reference}
{
typedef to \ccc{value_type}\&
}
\ccGlue
\ccNestedType{category}
{
\ccc{boost::lvalue_property_map_tag}
}
\ccGlue

%END-AUTO(\ccTypes)

\ccCreation
\ccCreationVariable{pmap}

\ccConstructor{Nth_of_tuple_property_map();}
{
Constructor.
}

\ccOperations

% The section below is automatically generated. Do not edit!
%START-AUTO(\ccOperations)

\ccMethod{template<class Iter> reference operator[](Iter tuple) const;}
{
Access a property map element.
\ccCommentHeading{Template Parameters}  \\
\ccc{Iter}: Type convertible to \ccc{key_type}.
}
\ccGlue

%END-AUTO(\ccOperations)

\ccHeading{Related Functions}

% The section below is automatically generated. Do not edit!
%START-AUTO(\ccHeading{Related Functions})

\ccFunction{Nth_of_tuple_property_map<N, typename value_type_traits<Iter>::type> make_nth_of_tuple_property_map(Iter);}
{
Free function to create a \ccc{Nth_of_tuple_property_map} property map.
}
\ccGlue

%END-AUTO(\ccHeading{Related Functions})

\ccExample

See \ccc{property_map.cpp} example.

\end{ccRefClass}

% +------------------------------------------------------------------------+
%%RefPage: end of main body, begin of footer
\ccRefPageEnd
% EOF
% +------------------------------------------------------------------------+

