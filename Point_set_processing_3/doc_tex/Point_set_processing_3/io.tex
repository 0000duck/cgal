\section{Input/Output}

The algorithms of this component take as input parameters iterator ranges of 3D points, or points with normals.
As the latter might be represented in various ways, e.g., as a class derived from the \cgal\ point class, or as a
\ccc{std::pair<Point_3<K>, Vector_3<K> >}, or as a \ccc{boost::tuple<..,Point_3<K>, ..., Vector_3<K> >}, the
algorithms use property maps to access the point or vector information in the input data.

Users of this package may use other types to represent positions and normals if they implement the corresponding property maps. \\

We provide functions to read and write sets of points or points with normals from the following ASCII file formats: \\
XYZ (three point coordinates \ccc{x y z} per line or three point coordinates and three normal vector coordinates \ccc{x y z nx ny nz} per line) \\
and Object File Format (OFF)~\cite{cgal:p-gmgv16-96}.
\\
\ccc{CGAL::read_xyz_points}  \\
\ccc{CGAL::read_off_points}  \\
\ccc{CGAL::write_off_points}  \\
\ccc{CGAL::write_xyz_points}  \\

\ccExample

The following example reads a point set from an input file stream and writes it to a file stream, both in the xyz format.
\ccIncludeExampleCode{Point_set_processing_3/read_write_xyz_point_set_example.cpp}