\section{Input/Output}


\subsection{Property Maps}

The algorithms of this component take as input parameters iterator ranges of 3D points, or of 3D points with normals. The property maps are used to access the point or normal information from the input data, so as to let the user decide upon the implementation of a point with normal. The latter can be represented as, e.g., a class derived from the \cgal\ 3D point, or as a
\ccc{std::pair<Point_3<K>, Vector_3<K> >}, or as a \ccc{boost::tuple<..,Point_3<K>, ..., Vector_3<K> >}.\\
\\
The following classes described in Chapter \ref {chapter-Property_map} provide property maps for the implementations of points with normals listed above:  \\
\ccc{CGAL::Dereference_property_map<T>}  \\
\ccc{CGAL::First_of_pair_property_map<Pair>} and \ccc{CGAL::Second_of_pair_property_map<Pair>}  \\
\ccc{CGAL::Nth_of_tuple_property_map<N, Tuple>}  \\
\\
\ccc{Dereference_property_map<Point_3>} is the default value of the position property map expected by all functions in this component. \\
See below examples using pair and tuple property maps. \\
Users of this package may use other types to represent positions and normals if they implement the corresponding property maps.


\subsection{Streams}

We provide functions to read and write sets of points or sets of points with normals from the following ASCII file formats: XYZ (three point coordinates \ccc{x y z} per line or three point coordinates and three normal vector coordinates \ccc{x y z nx ny nz} per line), and OFF (Object File Format) \cite{cgal:p-gmgv16-96}.

\ccc{CGAL::read_xyz_points}  \\
\ccc{CGAL::read_off_points}  \\
\ccc{CGAL::write_off_points} \\
\ccc{CGAL::write_xyz_points} \\


\subsection{Example}

The following example reads a point set from an input file and writes it to a file, both in the XYZ format. Positions and normals are stored in pairs and accessed through property maps.
\ccIncludeExampleCode{Point_set_processing_3/read_write_xyz_point_set_example.cpp}
