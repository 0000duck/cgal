\section{Input/Output}

The algorithms of this component takes as input parameters iterator ranges of 3D points plus property maps to access each point's position and normal (3D vector). \cgal\ provides the next classes to implement points and normals, which come with the appropriate property maps:

\begin{itemize}
\item \ccRefIdfierPage{CGAL::Point_3<GeomTraits>} \\
\cgal\ 3D position.
\item \ccRefIdfierPage{CGAL::Vector_3<GeomTraits>} \\
\cgal\ 3D vector.
\item \ccRefIdfierPage{CGAL::Point_with_normal_3<GeomTraits, Normal_3>} \\
3D point location plus normal.
\end{itemize}

Users of this package may use other types to represent positions and normals if they implement the corresponding property maps. \\

We provide functions to read and write sets of points or points with normals from the following ASCII file formats: XYZ (three point coordinates \ccc{x y z} per line or three point coordinates and three normal vector coordinates \ccc{x y z nx ny nz} per line) and OFF (Geomview Object File Format), see \cite{cgal-off-format}. % TOFIX: add ref to http://www.geomview.org/
\\
\ccRefIdfierPage{CGAL::read_xyz_point_set}  \\
\ccRefIdfierPage{CGAL::read_off_point_set}  \\
\ccRefIdfierPage{CGAL::write_off_point_set}  \\
\ccRefIdfierPage{CGAL::write_xyz_point_set}  \\

The following example reads a point set from an input file stream and writes it to a file stream, both in the xyz format.
\ccIncludeExampleCode{Point_set_processing_3/read_write_xyz_point_set_example.cpp}