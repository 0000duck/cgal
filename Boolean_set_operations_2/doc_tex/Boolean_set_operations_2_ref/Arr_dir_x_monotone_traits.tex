% +------------------------------------------------------------------------+
% | Reference manual page: ArrangementDirectionalXMonotoneTraits.tex
% +------------------------------------------------------------------------+

\ccRefPageBegin

\begin{ccRefConcept}{ArrangementDirectionalXMonotoneTraits_2}

\ccDefinition
% ===========
This concept refines the basic arrangement $x$-monotone traits concept.
A model of this concept is able to handle \emph{directed} $x$-monotone curves
that intersect in their interior. Namely, an instance of the
\ccc{X_monotone_curve_2} type defined by a model of the concept
\ccc{ArrangementXMonotoneTraits_2} is only required to have a \emph{left}
(lexicographically smaller) endpoint and a \emph{right} endpoint.
If the traits class is also a model of
\ccc{ArrangementDirectionalXMonotoneTraits_2}, the $x$-monotone curve is
also required to have a direction, namely one of these two endpoint serves
as its \emph{source} and the other as its \emph{target}.

\ccRefines
\ccc{ArrangementXMonotoneTraits_2}

\ccHeading{Functor Types}
%========================

\ccThree{Compare_y_at_x_2~~~}{}{\hspace*{12cm}}
\ccThreeToTwo

\ccNestedType{Compare_endpoints_xy_2}
{models the concept \ccc{ArrDirectionalTraits::CompareEndpointsXy_2}.}

\ccNestedType{Construct_opposite_2}
{models the concept \ccc{ArrDirectionalTraits::ConstructOpposite_2}.}

\ccNestedType{Intersect_2}
{models the concept \ccc{ArrDirectionalTraits::Intersect_2}.}

\ccNestedType{Split_2}
{models the concept \ccc{ArrDirectionalTraits::Split_2}.}

\ccNestedType{Are_mergeable_2}
{models the concept \ccc{ArrDirectionalTraits::AreMergeable_2}.}

\ccNestedType{Merge_2}
{models the concept \ccc{ArrDirectionalTraits::Merge_2}.}

\ccCreation
\ccCreationVariable{traits}
%==========================

\ccThree{Construct_x_monotone_curve_2~~~}{}{\hspace*{7cm}}
\ccThreeToTwo

\ccConstructor{ArrangementDirectionalXMonotoneTraits_2();}{default constructor.}
\ccGlue
\ccConstructor{ArrangementDirectionalXMonotoneTraits_2(ArrangementDirectionalXMonotoneTraits_2 other);}
{copy constructor}
\ccGlue
\ccMethod{ArrangementDirectionalXMonotoneTraits_2  operator=(other);}{assignment operator.}


\ccHeading{Accessing Functor Objects}
%====================================

\ccMethod{Compare_endpoints_xy_2 compare_endpoints_xy_2_object();} {}
\ccGlue
\ccMethod{Construct_opposite_2 construct_opposite_2_object();} {}

\ccHasModels
%===========

\ccc{CGAL::Arr_segment_traits_2<Kernel>} \\
\ccc{CGAL::Arr_non_caching_segment_traits_2<Kernel>} \\
\ccc{CGAL::Arr_circle_segment_traits_2<Kernel>} \\
\ccc{CGAL::Arr_conic_traits_2<RatKernel,AlgKernel,NtTraits>} \\
\ccc{CGAL::Arr_rational_function_traits_2<AlgKernel,NtTraits>}\\
\ccc{CGAL::Arr_Bezier_curve_traits_2<RatKernel,AlgKernel,NtTraits>}\\
\ccc{CGAL::Arr_algebraic_segment_traits_2<Coefficient>}

\ccSeeAlso
%=========

\ccc{ArrangementXMonotoneTraits_2}\lcTex{ 
      (\ccRefPage{ArrangementXMonotoneTraits_2})}

\end{ccRefConcept}

\ccRefPageEnd

%%%%%%%% Functors  %%%%%%%%

%%%%%%%% Intersect_2
% ==================
\ccRefPageBegin
\begin{ccRefConcept}{ArrDirectionalTraits::Intersect_2}
\ccRefines{Functor}

\ccHasModels\ccc{ArrangementDirectionalXMonotoneTraits_2::Intersect_2}

\ccCreationVariable{fo}

\ccMethod{Output_iterator operator()(ArrDirectionalTraits::X_monotone_curve_2 xc1,
                                     ArrDirectionalTraits::X_monotone_curve_2 xc2,
				     Output_iterator& oi);}
  {computes the intersections of \ccc{xc1} and \ccc{xc2} and 
  inserts them \emph{in an ascending lexicographic $xy$-order} into the 
  output iterator \ccc{oi}. The value-type of \ccc{Output_iterator} is 
  \ccc{CGAL::Object}, where each \ccc{Object} wraps either a 
  \ccc{pair<ArrDirectionalTraits::Point_2, ArrDirectionalTraits::Multiplicity>} object, which
  represents an intersection point with its multiplicity (in case the
  multiplicity is undefined or unknown, it is set to $0$) or an 
  \ccc{ArrDirectionalTraits::X_monotone_curve_2} object, representing an
  overlapping subcurve of \ccc{xc1} and \ccc{xc2}. In the latter case,
  the overlapping subcurves are given the direction of \ccc{xc1} and
  \ccc{xc2} if their directions are identical. Otherwise, the overlapping
  subcurves are given an arbitrary direction. The operator returns a
  past-the-end iterator for the output sequence.}
\end{ccRefConcept}
\ccRefPageEnd

%%%%%%%% Split_2
% ==============
\ccRefPageBegin
\begin{ccRefConcept}{ArrDirectionalTraits::Split_2}
\ccRefines{Functor}

\ccHasModels\ccc{ArrangementDirectionalXMonotoneTraits_2::Split_2}

\ccCreationVariable{fo}
\def\ccTagRmConstRefPair{\ccFalse}%
\ccMethod{void operator()(ArrDirectionalTraits::X_monotone_curve_2 xc,
                          ArrDirectionalTraits::Point_2 p,
                          ArrDirectionalTraits::X_monotone_curve_2& xc1,
                          ArrDirectionalTraits::X_monotone_curve_2& xc2);}
  {accepts an input curve \ccc{xc} and a split point \ccc{p} in its
  interior. It splits \ccc{xc} at the split point into two subcurves
  \ccc{xc1} and \ccc{xc2}, such that \ccc{p} is \ccc{xc1}'s \emph{right}
  endpoint and \ccc{xc2}'s \emph{left} endpoint. The direction of \ccc{xc}
  is preserved. That is, in case \ccc{xc} is directed from left to right,
  \ccc{p} becomes \ccc{xc1}'s target and \ccc{c2}'s source;
  otherwise, \ccc{p} becomes \ccc{xc2}'s target and \ccc{xc1}'s source.}%
\ccTagDefaults
\end{ccRefConcept}
\ccRefPageEnd

%%%%%%%% AreMergeable_2
% =====================
\ccRefPageBegin
\begin{ccRefConcept}{ArrDirectionalTraits::AreMergeable_2}
\ccRefines{Functor}

\ccHasModels\ccc{ArrangementDirectionalXMonotoneTraits_2::Are_mergeable_2}

\ccCreationVariable{fo}
\def\ccTagRmConstRefPair{\ccFalse}%
\ccMethod{bool operator()(ArrDirectionalTraits::X_monotone_curve_2 xc1,
                          ArrDirectionalTraits::X_monotone_curve_2 xc2);}
  {accepts two $x$-monotone curves \ccc{xc1} and \ccc{xc2} and determines
  whether they can be merged to form a single $x$-monotone curve.
  \ccc{xc1} and \ccc{xc2} are mergeable if their underlying curves are
  identical, they share a common endpoint, and they do not bend to form
  a non-$x$-monotone curve.
  \ccPrecond{The target point of \ccc{xc1} and the source point \ccc{xc2}
             coincide or the source point of \ccc{xc2} and the target
             point \ccc{xc2} coincide.}}
\end{ccRefConcept}
\ccRefPageEnd

%%%%%%%% Merge_2
% ==============
\ccRefPageBegin
\begin{ccRefConcept}{ArrDirectionalTraits::Merge_2}
\ccRefines{Functor}

\ccHasModels\ccc{ArrangementDirectionalXMonotoneTraits_2::Merge_2}

\ccCreationVariable{fo}
\def\ccTagRmConstRefPair{\ccFalse}%
\ccMethod{void operator()(ArrDirectionalTraits::X_monotone_curve_2 xc1,
                          ArrDirectionalTraits::X_monotone_curve_2 xc2,
		          ArrDirectionalTraits::X_monotone_curve_2& xc);}
  {accepts two \emph{mergeable} $x$-monotone curves \ccc{xc1} and
  \ccc{xc2} and asigns \ccc{xc} with the merged curve. If the target
  point of \ccc{xc1} and the source point of \ccc{xc2} coincide; then
  the source point of \ccc{xc1} and the target point of \ccc{xc2} become
  the source and target points of \ccc{xc}, respectively. If the target
  point of \ccc{xc2} and the source point of \ccc{xc1} coincide; then
  the source point of \ccc{xc2} and the target point of \ccc{xc1} become
  the source and target points of \ccc{xc}, respectively.
  \ccPrecond{\ccc{are_mergeable_2}(\ccc{xc1}, \ccc{xc2}) is true.}}
\ccTagDefaults
\end{ccRefConcept}
\ccRefPageEnd

%%%%%%%% CompareEndpointsXy_2
% ===========================
\ccRefPageBegin
\begin{ccRefConcept}{ArrDirectionalTraits::CompareEndpointsXy_2}
\ccRefines{Functor}

\ccHasModels\ccc{ArrangementDirectionalXMonotoneTraits_2::CompareEndpointsXy_2}

\ccCreationVariable{fo}
\def\ccTagRmConstRefPair{\ccFalse}%
\ccMethod{Comparison_result operator()(ArrDirectionalTraits::X_monotone_curve_2 xc);}
  {accepts an input curve \ccc{xc} and compares its source and target
  points. It returns \ccc{SMALLER} if the curve is directed from
  lexicographically left to right, and \ccc{LARGER} if it is directed
  from lexicographically right to left.}

\ccTagDefaults
\end{ccRefConcept}
\ccRefPageEnd

%%%%%%%% ConstructOpposite_2
% ===========================
\ccRefPageBegin
\begin{ccRefConcept}{ArrDirectionalTraits::ConstructOpposite_2}
\ccRefines{Functor}

\ccHasModels\ccc{ArrangementDirectionalXMonotoneTraits_2::ConstructOpposite_2}

\ccCreationVariable{fo}
\def\ccTagRmConstRefPair{\ccFalse}%
\ccMethod{ArrDirectionalTraits::X_monotone_curve_2 operator()(ArrDirectionalTraits::X_monotone_curve_2 xc);}
  {accepts an $x$-monotone curve \ccc{xc} and returns its opposite curve,
  namely a curve whose graph is the same as \ccc{xc}'s, and whose source and
  target are swapped with respect to \ccc{xc}'s source and target.}

\ccTagDefaults
\end{ccRefConcept}
\ccRefPageEnd
