\ccRefPageBegin

\begin{ccRefConcept}{GeneralPolygonSetDcel}

A doubly-connected edge-list (\dcel\ for short) data-structure. It consists
of three containers of records: vertices $V$, halfedges $E$, and faces $F$.
It maintains the incidence relation among them. The halfedges are ordered
in pairs sometimes referred to as twins, such that each halfedge pair
represent an edge.

A model of the \ccRefName\ concept must provide the following types and 
operations. (In addition to the requirements here, the local types 
\ccHtmlNoLinksFrom{\ccStyle{Vertex},\ccStyle{Halfedge}, \ccStyle{Face}
\ccStyle{Hole} and \ccStyle{Isolated_vertex}}
must be models of the concepts 
\ccc{ArrangementDcelVertex}\lcTex{ 
(\ccRefPage{ArrangementDcelVertex})},
\ccc{ArrangementDcelHalfedge}\lcTex{
(\ccRefPage{ArrangementDcelHalfedge})},
\ccc{ArrangementDcelFace}\lcTex{
(\ccRefPage{ArrangementDcelFace})},
\ccc{ArrangementDcelHole}\lcTex{
(\ccRefPage{ArrangementDcelHole})} and
\ccc{ArrangementDcelIsolatedVertex}\lcTex{
(\ccRefPage{ArrangementDcelIsolatedVertex})} respectively.)

% \ccTypes
%=======

\ccCreation
\ccCreationVariable{dcel}
%========================  
\ccConstructor{Gps_dcel();}
   {constructs an empty \dcel.}

% \ccAccessFunctions
%=================

% \ccModifiers

\ccHasModels
  \ccc{Gps_default_dcel<Traits>}\lcTex{
    (\ccRefPage{CGAL::Gps_default_dcel<Traits>})}

\ccSeeAlso
  \ccc{ArrangementDcel}\lcTex{(\ccRefPage{ArrangementDcel})}

\end{ccRefConcept}

\ccRefPageEnd

