\ccRefPageBegin

\begin{ccRefConcept}{GeneralPolygonSetTraits_2}

\ccThree{GeneralPolygonSetTraits_2}{Polygon_with_holes_2}{}
\ccThreeToTwo

\ccDefinition
% ===========
This concept defines the minimal set of geometric predicates needed to
perform the Boolean-set operations. It refines the directional $x$-monotone
arrangement-traits concept. In addition to the \ccc{Point_2} and
\ccc{X_monotone_curve_2} types defined in the generalized concept, it defines
a type that represents a general polygon and another one that represents
general polygon with holes. It also requires operations that operate on these
types.

\ccRefines
% ========
\ccc{ArrangementDirectionalXMonotoneTraits_2}

\ccTypes
% ======

\ccNestedType{Polygon_2}{represents a general polygon. Must be a model of the \ccc{GpsTraitsGeneralPolygon_2} concept.}
\ccGlue
\ccNestedType{Polygon_with_holes_2}{represents a general polygon with holes. Must be a model of the \ccc{GpsTraitsGeneralPolygonWithHoles_2} concept.}

\ccNestedType{Curve_const_iterator}
{A const iterator of curves. Its value type is const
\ccc{X_monotone_curve_2}.}

\ccHeading{Functor Types}
% =======================

\ccThree{Construct_polygon_2}{}{\hspace*{\ccwComment}}
\ccThreeToTwo

\ccNestedType{Construct_polygon_2}
{a functor that constructs a general polygon from a range of
$x$-monotone curves. It uses the operator\\
 \ccc{void operator() (InputIterator begin, Input iterator end,
                       Polygon_2 & pgn)}, \\
 parameterized by the \ccc{InputIterator} type.}

\ccNestedType{Construct_general_polygon_with_holes_2}{a functor that constructs a general polygon with holes from a general polygon and, optionally, a range of holes. It uses the operator\\
 \ccc{void operator() (const General_polygon_2& pgn_boundary)} or \\
\ccc{void operator() (const General_polygon_2& pgn_boundary, HolesInputIterator h_begin, HolesInputIterator h_end)} parameterized by the \ccc{HolesInputIterator} type.}

\ccNestedType{Construct_outer_boundary} {A functor for constructing the outer boundary of a polygon with holes.  It uses the operator\\
 \ccc{General_polygon_2 operator()(const  General_polygon_with_holes_2& pol_wh)}.}

\ccNestedType{Construct_holes} {A functor for constructing the container of holes of a polygon with holes. It returns the begin/end iterators for the holes It uses the operator\\
 \ccc{std::pair<Hole_const_iterator, Hole_const_iterator>  operator()(const General_polygon_with_holes_2& pol_wh)}.}

\ccNestedType{Is_unbounded} {A functor for checking if polygon with holes has no outer boundary.  It uses the operator\\
\ccc{bool operator()(const  General_polygon_with_holes_2& pol_wh)}.}

\ccCreation
\ccCreationVariable{traits}
% =========================

\ccConstructor{GeneralPolygonSetTraits_2();}{default constructor.}
\ccGlue
\ccConstructor{GeneralPolygonSetTraits_2(GeneralPolygonSetTraits_2 other);}
{copy constructor}
\ccGlue
\ccMethod{GeneralPolygonSetTraits_2  operator=(other);}{assignment operator.}


\ccHeading{Accessing Functor Objects}
%====================================

\ccMethod{Construct_polygon_2 construct_polygon_2_object();}
{returns a functor that constructs a polygon.}

\ccMethod{Construct_curves_2 construct_curves_2_object();}
{returns a functor that obtains the curves of a polygon.}

\ccMethod{Construct_general_polygon_with_holes_2 construct_polygon_with_holes_2_object() const;}
{returns a functor that constructs a polygon with holes.}

\ccMethod{Construct_outer_boundary construct_outer_boundary_object() const;}
{returns a functor that obtains the outer boundary of a polygon with holes.}

\ccMethod{Construct_holes construct_holes_object() const;}
{returns a functor that obtains begin/end iterators over a container of holes.}

\ccMethod{Is_unbounded construct_is_unbounded_object();} {returns a functor that determines if the polygon with holes is unbounded}


\ccHasModels
%===========
\ccc{CGAL::Gps_segment_traits_2<Kernel,Container,ArrSegmentTraits>}\\
\ccc{CGAL::Gps_circle_segment_traits_2<Kernel>}\\
\ccc{CGAL::Gps_traits_2<ArrTraits,GeneralPolygon>}

\ccSeeAlso
% ========
  \ccc{ArrangementDirectionalXMonotoneTraits_2}\lcTex{(\ccRefPage{ArrangementDirectionalXMonotoneTraits_2})}

\end{ccRefConcept}

\ccRefPageEnd
