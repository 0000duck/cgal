\ccRefPageBegin

\begin{ccRefClass}{Polygon_set_2<Kernel,Container,Dcel>}

\ccThree{Polygon_set_2<Kernel,Container,Dcel>}{Polygon_2}{}
\ccThreeToTwo
    
\ccDefinition
%============

The class \ccRefName{} represents sets of linear polygons with holes.
It is parameterized with three types. The types (\ccc{Kernel} and 
\ccc{Container}) are used to instantiate the type 
\ccc{CGAL::Polygon_2<Kernel,Container>}. The type \ccc{Container} is used 
to represents the outer boundary and the boundary of the holes of the set 
members.

The \ccc{Dcel} template-parameter should be instantiated with a
model of the concept \ccc{GeneralPolygonSetDcel}. It is instantiated
by default with the type \ccc{Gps_default_dcel<Traits>}. You can override 
this default, with a different {\sc Dcel} class, typically an extension
of the \ccc{Gps_default_dcel} class template. Overriding the default is 
necessary only if you intend to obtain the undelying internal arrangement 
and process it further.

\ccInclude{CGAL/Polygon_set_2.h}

\ccInheritsFrom
  \ccc{General_polygon_set_2<Gps_segment_traits_2<Kernel,Container> >}

\ccSeeAlso
  \ccc{General_polygon_set_2}\lcTex{(\ccRefPage{CGAL::General_polygon_set_2<Traits>})}\\
  \ccc{Gps_segment_traits_2}\lcTex{(\ccRefPage{CGAL::Gps_segment_traits_2<Kernel,Container,ArrSegmentTraits>})}

\end{ccRefClass}
\ccRefPageEnd
