\ccRefPageBegin

\begin{ccRefClass}{Boolean_set_operations_2<Polygonal_region_with_holes_2, Traits>}
    
\ccDefinition
%============
An instance of the \ccClassTemplateName\ class-template represents a
point set in the plane. An \ccc{Arrangement_2} data structure is used
internally to represent the point set. The class template provides
methods to apply Boolean-set operations on pairs of instances of  
\ccc{Boolean_set_operations_2} and a few other utility methods. At this
point only regularized Boolean-set operations are implemented.
The input and output of these methods consist of one or more
polygonal-like regions, some of which may have holes.

The \ccClassTemplateName\ template has two parameters:
\begin{itemize}
\item The \ccc{Polygon_region_with_holes_2} template-parameter should
be instantiated with a class that is a model of the
\ccc{PolygonRegionWithHoles_2} concept.
\item The \ccc{Traits} template-parameter should be instantiated with
a model of the \ccc{PolygonalSetTraits_2} concept. The traits
class defines the types of points, $x$-monotone curves, polygonal-like
regions, and polygonal-like regions with holes, that is \ccc{Point_2},
\ccc{X_monotone_curve_2}, \ccc{Polygonal_region_2}, and
\ccc{Polygonal_region_with_holes_2} respectively. \ccc{Point_2} must
be the type of the endpoints of \ccc{X_monotone_curve_2}, and
\ccc{X_monotone_curve_2} must be the type of the curves that comprise
the boundaries of the polygonal regions. The traits class supports
geometric operations on the types above. For some classes that model
\ccc{PolygonalRegionWithHoles_2} default traits class are
extracted. For other, the template parameter
\ccc{Polygonal_region_with_holes_2} must match the corresponding
nested type of the \ccc{PolygonalSetTraits_2} parameter.
\end{itemize}

\ccTypes
%=======
\ccNestedType{Polygonal_region_2}{the polygonal-like region type.}
\ccGlue
\ccNestedType{Polygonal_region_with_holes_2}{the polygonal-like region
with holes type.} 

\ccNestedType{Traits_2}{the traits class in use.}

\ccCreation
\ccCreationVariable{ps}
%=======================
    
\ccConstructor{Boolean_set_operations_2<Polygonal_region_with_holes_2,Traits>();} 
  {constructs an empty set of polygonal regions represented by an
  empty arrangement.}
    
\ccConstructor{Boolean_set_operations_2<Polygonal_region_with_holes_2,Traits>(const Self& other);}
  {copy constructor.}
        
\ccConstructor{Boolean_set_operations_2<Polygonal_region_with_holes_2,Traits>(Traits_2 * traits);}
  {constructs an empty set of polygonal regions that uses the given
  \ccc{traits} instance for performing the geometric operations.}

\ccConstructor{Boolean_set_operations_2<Polygonal_region_with_holes_2,Traits>(const Polygonal_region_2 & pgn);}
  {constructs a set of polygonal regions from the polygonal region
  \ccc{pgn}.}

\ccConstructor{Boolean_set_operations_2<Polygonal_region_with_holes_2,Traits>(const Polygonal_region_with_holes_2 & pgn_with_holes);}
  {constructs a set of polygonal regions from the polygonal region
  with holes \ccc{pgn_with_holes}.}

\ccConstructor{template <class PolygonalRegionIterator> Boolean_set_operations_2<Polygonal_region_with_holes_2,Traits>(PolygonalRegionIterator pgn_begin, PolygonalRegionIterator pgn_end);}
  {constructs a set of polygonal regions that from the given range of
  \ccc{Polygonal_region_2} instances.}
  
\ccConstructor{template <class PolygonalRegionIterator,
                         class PolygonalRegionWithHolesIterator>
  Boolean_set_operations_2<Polygonal_region_with_holes_2,Traits>
  (PolygonalRegionIterator pgn_begin, PolygonalRegionIterator pgn_end,
  PolygonalRegionWithHolesIterator pgn_with_holes_begin,
  PolygonalRegionWithHolesIterator pgn_with_holes_end);}
  {constructs a set of polygonal regions from the given range of
  \ccc{Polygonal_region_2} and
  \ccc{Polygonal_region_with_holes_2}instances.}

\ccAccessFunctions
% ================
\ccMethod{template <class OutputIterator>
  void polygonal_regions(OutputIterator out);}
  {obtains all the polygonal regions as polygonal regions with
  holes. This operation takes time $O(n)$, where $n$ is the number of
  polygonal regions.}
  
\ccMethod{template <class OutputIterator1, class OutputIterator2>
  void polygonal_regions(OutputIterator1 out1, OutputIterator2 out2);}
  {obtains all the polygonal regions. The polygonal regions (without
  holes) are inserted to the range given by the first iterator, and
  those with holes are inserted to the second one. This operation
  takes time $O(n)$, where $n$ is the number of polygonal regions.}

\ccMethod{Size number_of_polygons() const;}
  {returns the number of polygonal regions with holes represented by
  this.}

\ccMethod{Size number_of_polygons_with_holes() const;}
  {returns the number of polygonal regions with holes represented by
  this.}

\ccMethod{Traits & traits() const;}
  {obtains an instance of the traits. If the traits was passed as a
  parameter to the constructor of this, it is returned. Otherwise, a
  newly created instance is returned.}

\ccModifiers
% ==========
\ccMethod{void clear();}
  {clears this.}

\ccMethod{void intersection(const Boolean_set_operations_2& bops);}
  {computes the intersection of this and \ccc{bops}.}

\ccMethod{void union(const Boolean_set_operations_2& bops);}
  {computes the union of this and \ccc{bops}.}

\ccMethod{void difference(const Boolean_set_operations_2& bops);}
  {computes the difference between this and \ccc{bops}.}

\ccMethod{void symmetric_difference(const Boolean_set_operations_2& bops);}
  {computes the symmetric difference between this and \ccc{bops}.}
  
\ccPredicates
% ===========
\ccMethod{void do_intersect(const Boolean_set_operations_2& bops);}
  {returns \ccc{true} if this and \ccc{bops} intersect in their
  interior, and \ccc{false} otherwise.}

\ccMethod{bool is_empty() const;}
  {returns \ccc{true} if this is empty and contains nothing.}

\ccMethod{Bounded_side bounded_side(const Point_2& q);}
  {returns the symbolic constant \ccc{ON_BOUNDED_SIDE},
  \ccc{ON_BOUNDARY} or \ccc{ON_UNBOUNDED_SIDE}, depending on the
  location of point \ccc{q}.}

\ccMethod{Polygonal_region_with_holes_2 locate(const Point_2 & p);}
  {returns the location of the query point \ccc{p}.}

\ccSeeAlso
  \ccc{Arrangement_2}\lcTex{(\ccRefPage{Arrangement_2})}\\
  \ccc{ArrangementTraits_2}\lcTex{(\ccRefPage{ArrangementTraits_2})}

\end{ccRefClass}

\ccRefPageEnd
