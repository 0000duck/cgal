\ccRefPageBegin

\begin{ccRefConcept}{GeneralPolygon_2}

\ccRefines
% ========
\ccc{GpsTraits::GeneralPloygon_2}

\ccTypes
%=======

\ccNestedType{X_monotone_curve_2}{represents a planar (weakly) $x$-monotone
  curve. The type of the geometric mapping of the polygonal edges.}

\ccNestedType{Curve_iterator}{an iterator over the geometric mapping of the 
polygon edges. Its value type is \ccc{X_monotone_curve_2}.}
\ccGlue
\ccNestedType{Curve_const_iterator}{a const iterator over the geometric 
mapping of the polygon edges. Its value type is \ccc{X_monotone_curve_2}.}
 
\ccDefinition
% ===========
A model of this concept represents a simple general-polygon. The
geometric mapping of the edges of the polygon must be $x$-monotone curves.
The concept requires the ability to access these curves. 
The general polygon represented must be simple. That is, the
only points of the plane belonging to two curves are the geometric mapping
of the polygon vertices. In addition, the vertices of the represented 
polygon must be ordered consistently, and the curved must be directed 
accordingly. Only counterclockwise oriented polygons are valid operands
of Boolean set-operations. General polygon that represent holes must be
clockwise oriented.

\ccCreation
\ccCreationVariable{polygon}

\ccThree{Edge_const_iterator~~~}{}{\hspace*{7cm}}
\ccThreeToTwo

\ccConstructor{GeneralPolygon_2();}{default constructor.}
\ccGlue
\ccConstructor{GeneralPolygon_2(GeneralPolygon_2 other);}
{copy constructor.}
\ccGlue
\ccMethod{GeneralPolygon_2  operator=(other);}{assignment operator.}
\ccGlue
\ccConstructor{template <class InputIterator>
GeneralPolygon_2(InputIterator begin, InputIterator end);}
{constructs a general polygon from a given range of curves.}

\ccAccessFunctions
%=================

\ccMethod{Curve_iterator curves_begin();} 
{returns the begin iterator of the curves.}
\ccGlue
\ccMethod{Curve_iterator curves_end();} 
{returns the past-the-end iterator of the curves.}

\ccMethod{Curve_const_iterator curves_begin();} 
{returns the begin const iterator of the curves.}
\ccGlue
\ccMethod{Curve_const_iterator curves_end();} 
{returns the past-the-end const iterator of the curves.}

\ccModifiers
% ==========

\ccMethod{template <class Iterator>
  void init(Iterator begin, Iterator end);}
{initializes the polygon with the polygonal chain given by the range. 
The value type of \ccc{Iterator} must be \ccc{X_monotone_curve_2}.
\ccPrecond{The curves in the range must define a simple polygon.}}
 
% \ccPredicates
% % ===========
% 
% \ccMethod{bool is_simple();}{returns \ccc{true} if the polygon is simple, 
% and \ccc{false} otherwise. Used as precondition for some of the operations.}
% \ccGlue
% \ccMethod{bool is_strictly_simple();}{returns \ccc{true} if the polygon
% is strictly simple, and \ccc{false} otherwise. Used as precondition for
% some of the operations.
% \ccPrecond{\ccStyle{is_simple()}.}}

\ccHasModels
% ==========
\ccc{CGAL::General_polygon_2<ArrTraits>} \\
\ccc{CGAL::Polygon_2<PolygonTraits_2, Container>}

\end{ccRefConcept}

\ccRefPageEnd
