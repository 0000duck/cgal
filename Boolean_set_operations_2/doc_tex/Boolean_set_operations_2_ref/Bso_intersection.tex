\ccRefPageBegin
\label{ref_bso_intersection}

\begin{ccRefFunction}{intersection}

\ccThree{OutputIterator}{intersection}{}
\ccThreeToTwo

\ccDefinition

\ccInclude{CGAL/Boolean_set_operations_2.h}

\ccGlobalFunction{
OutputIterator intersection(const Type1 & p1, const Type2 & p2,
                            OutputIterator oi);}
{Each one of these functions computes the intersection of two given
polygons \ccc{p1} and \ccc{p2}, inserts the resulting polygons with
holes into an output container through a given output iterator
\ccc{oi}, and returns the output iterator. The value type of the
\ccc{OutputIterator} is either \ccc{Polygon_with_holes_2} or 
\ccc{General_polygon_with_holes_2}.}

\begin{ccTexOnly}
\begin{longtable}[c]{|l|l|}
\multicolumn{2}{l}{\sl \ \ }
\endfirsthead
\multicolumn{2}{l}{\sl continued}
\endhead
\hline
\textbf{Arg 1 Type} & \textbf{Arg 2 Type}\\
\hline
\hline
\ccc{Polygon_2} & \ccc{Polygon_2}\\
\hline
\ccc{Polygon_2} & \ccc{Polygon_with_holes_2}\\
\hline
\ccc{Polygon_with_holes_2} & \ccc{Polygon_2}\\
\hline
\ccc{Polygon_with_holes_2} & \ccc{Polygon_with_holes_2}\\
\hline
\ccc{General_polygon_2} & \ccc{General_polygon_2}\\
\hline
\ccc{General_polygon_2} & \ccc{General_polygon_with_holes_2}\\
\hline
\ccc{General_polygon_with_holes_2} & \ccc{General_polygon_2}\\
\hline
\ccc{General_polygon_with_holes_2} & \ccc{General_polygon_with_holes_2}\\
\hline
\end{longtable}
\end{ccTexOnly}

\begin{ccHtmlOnly}
<div align="center">
<table cellpadding=3 border="1">
<tr><th> Arg 1 type</th><th>Arg 2 type</th></tr>
<tr><td valign="center">Polygon_2</td><td valign="center">Polygon_2</td></tr>
<tr><td valign="center">Polygon_2</td><td valign="center">Polygon_with_holes_2</td></tr> 
<tr><td valign="center">Polygon_with_holes_2</td><td valign="center">Polygon_2</td></tr>
<tr><td valign="center">Polygon_with_holes_2</td><td valign="center">Polygon_with_holes_2</td></tr>
<tr><td valign="center">General_polygon_2</td><td valign="center">General_polygon_2</td></tr>
<tr><td valign="center">General_polygon_2</td><td valign="center">General_polygon_with_holes_2</td></tr>
<tr><td valign="center">General_polygon_with_holes_2</td><td valign="center">General_polygon_2</td></tr>
<tr><td valign="center">General_polygon_with_holes_2</td><td valign="center">General_polygon_with_holes_2</td></tr>
</table>
</div>
\end{ccHtmlOnly}

\ccGlobalFunction{template <class Kernel, class Container, class OutputIterator>
OutputIterator intersection(const Polygon_2<Kernel, Container> & p1,
                            const Polygon_2<Kernel, Container> & p2,
                            OutputIterator oi);}
\ccGlue
\ccGlobalFunction{template <class Kernel, class Container, class OutputIterator>
OutputIterator intersection(const Polygon_2<Kernel, Container> & p1,
                            const Polygon_with_holes_2<Kernel, Container> & p2,
                            OutputIterator oi);}
\ccGlue
\ccGlobalFunction{template <class Kernel, class Container, class OutputIterator>
OutputIterator intersection(const Polygon_with_holes_2<Kernel, Container> & p1,
                            const Polygon_2<Kernel, Container> & p2,
                            OutputIterator oi);}
\ccGlue
\ccGlobalFunction{template <class Kernel, class Container, class OutputIterator>
OutputIterator intersection(const Polygon_with_holes_2<Kernel, Container> & p1,
                            const Polygon_with_holes_2<Kernel, Container> & p2,
                            OutputIterator oi);}
\ccGlue
\ccGlobalFunction{template <class Traits, class OutputIterator>
OutputIterator intersection(const General_polygon_2<Traits> & p1,
                            const General_polygon_2<Traits> & p2,
                            OutputIterator oi);}
\ccGlue
\ccGlobalFunction{template <class Traits, class OutputIterator>
OutputIterator intersection(const General_polygon_with_holes_2<General_polygon_2<Traits> > & p1,
                            const General_polygon_2<Traits> & p2,
                            OutputIterator oi);}
\ccGlue
\ccGlobalFunction{template <class Traits, class OutputIterator>
OutputIterator intersection(const General_polygon_2<Traits> & p1,
                            const General_polygon_with_holes_2<General_polygon_2<Traits> > & p2,
                            OutputIterator oi);}
\ccGlue
\ccGlobalFunction{template <class Polygon, class OutputIterator>
OutputIterator intersection(const General_polygon_with_holes_2<Polygon> & p1,
                            const General_polygon_with_holes_2<Polygon> & p2,
                            OutputIterator oi);}

\ccGlobalFunction{template <class InputIterator, class OutputIterator>
OutputIterator intersection(InputIterator begin, InputIterator end,
                            OutputIterator oi);}
{Computes the intersection of the general polygons (or general polygons with
holes) in the given range. (The value type of the input iterator is
used to distinguish between the two.) The result, represented by a set
of general polygon with holes, is inserted into an output container
through a given output iterator \ccc{oi}. The output iterator is
returned. The value type of the \ccc{OutputIterator} is
\ccc{Traits::Polygon_with_holes_2}.}

\ccGlobalFunction{template <class InputIterator1, class InputIterator2,
		  class OutputIterator>
OutputIterator intersection(InputIterator1 pgn_begin1,
                            InputIterator1 pgn_end1,
                            InputIterator2 pgn_begin2,
	                    InputIterator2 pgn_end2,
		            OutputIterator oi);}
{Computes the intersection of the general polygons and general polygons
with holes in the given two ranges. The result, represented by a set
of general polygon with holes, is inserted into an output container
through a given output iterator \ccc{oi}. The output iterator is
returned. The value type of the \ccc{OutputIterator} is
\ccc{Traits::Polygon_with_holes_2}.}

\ccSeeAlso
\ccRefIdfierPage{CGAL::do_intersect}\\
\ccRefIdfierPage{CGAL::join}\\
\ccRefIdfierPage{CGAL::difference}\\
\ccRefIdfierPage{CGAL::symmetric_difference}

\end{ccRefFunction}

\ccRefPageEnd
