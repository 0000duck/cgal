% +------------------------------------------------------------------------+
% | CGAL Reference Manual:  Real_timer.tex
% +------------------------------------------------------------------------+

\begin{ccRefClass}{Real_timer}

\ccDefinition

\ccInclude{CGAL/Real_timer.h}
\ccCreationVariable{t}

The class \ccRefName\ is a timer class for measuring real time.
A timer \ccVar\ of type \ccRefName\ is an object with a state. It is
either {\em running\/} or it is {\em stopped}. The state is controlled
with \ccStyle{t.start()} and \ccStyle{t.stop()}. The timer counts the
time elapsed since its creation or last reset. It counts only the time
where it is in the running state. The time information is given in seconds.
The timer counts also the number of intervals it was running, i.e. it 
counts the number of calls of the \ccc{start()} member function since the 
last reset. If the reset occures while the timer is running it counts as the
first interval.

\ccCreation

\ccSetThreeColumns{Timer&}{t.is_running();}{}
\ccPropagateThreeToTwoColumns
%\ccSetTwoColumns{Timer t( Timer r);}{}

\ccConstructor{Real_timer();}{state is {\em stopped.}}

\ccOperations

\ccMethod{void     start();}{\ccPrecond state is {\em stopped}.}
\ccGlue
\ccMethod{void     stop ();}{\ccPrecond state is {\em running}.}
\ccGlue
\ccMethod{void     reset();}{reset timer to zero. The state is unaffected.}
\ccGlue
\ccMethod{bool     is_running();}{\ccc{true} if the current state is running.}

\ccMethod{double   time();}{real time in seconds, or 0 if the 
                            underlying system call failed.}
\ccGlue
\ccMethod{int      intervals();}{number of start/stop-intervals since
                                 the last reset.} 
\ccGlue
\ccMethod{double   precision();}{smallest possible time step in seconds,
                                 or -1 if the system call failed.}
\ccGlue
\ccMethod{double   max();}{maximal representable time in seconds.}

\ccImplementation


The timer class is based in the C function \ccc{gettimeofday()} on
POSIX systems and the C function \ccc{_ftime()} on MS Visual C++.
The system calls to these
timers might fail, in which case a warning message will be issued
through the \cgal\ error handler and the functions return with the
error codes indicated above.  The \ccc{precision} method computes the
precision dynamically at runtime at its first invocation.


\end{ccRefClass}
