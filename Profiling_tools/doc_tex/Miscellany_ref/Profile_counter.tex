% +------------------------------------------------------------------------+
% | CGAL Reference Manual:  Profile_counter.tex
% +------------------------------------------------------------------------+

\begin{ccRefClass}{Profile_counter}

\ccDefinition

\ccInclude{CGAL/Profile_counter.h}
\ccCreationVariable{pc}

The class \ccRefName\ provides a way to count the number of times a given
line of code is executed during the execution of a program, and print this
number at the end of the execution of the program.  Such counters can be added
at critical places in the code, and at the end of the execution of a program,
the count is printed on \ccc{std::cerr}, together with an identification
string passed to the constructor.  The macro \ccc{CGAL_PROFILER} can be
used to conveniently place these counters anywhere.  They are disabled by
default and activated by the global macro \ccc{CGAL_PROFILE}.


\ccCreation

\ccSetThreeColumns{Profile_counter&}{t.is_running();}{}
\ccPropagateThreeToTwoColumns

\ccConstructor{Profile_counter(std::string s);}{The internal counter is
initialized to 0, and the string \ccc{s} is stored for further printing
by the destructor.}

\ccConstructor{~Profile_counter();}{The value of the counter is printed
to \ccc{std::cerr} together with the string.}

\ccOperations

\ccMethod{void operator++();}{Increments the internal counter.}

%%  XXX TODO : Use a \ccRefMacro
%%             - CGAL_PROFILE and CGAL_PROFILER need to end up in the index.
%%             - rename CGAL_PROFILER to CGAL_PROFILE_COUNTER ?
\ccc{\#define CGAL_PROFILER(MSG)}
{If \ccc{CGAL_PROFILE} is not defined, then \ccc{CGAL_PROFILER} is defined
to an empty statement.  Otherwise, it is defined to \ccc{static
CGAL::Profile_counter tmp(MSG); ++tmp;}.}

\ccIncludeExampleCode{Profiling_tools/Profile_counter.cpp}

will print at exit:

\begin{ccExampleCode}
[CGAL::Profile_counter]     10 iterations of the for-loop
\end{ccExampleCode}

\end{ccRefClass}
