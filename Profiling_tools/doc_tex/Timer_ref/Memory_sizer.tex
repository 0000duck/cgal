% +------------------------------------------------------------------------+
% | CGAL Reference Manual:  Memory_sizer.tex
% +------------------------------------------------------------------------+

\begin{ccRefClass}{Memory_sizer}

\ccDefinition

The class \ccRefName\ allows to measure the memory size used by the process.
Both the virtual memory size and the resident size are available (the resident
size does not account for swapped out memory nor for the memory which is not
yet paged-in).

\ccInclude{CGAL/Memory_sizer.h}
\ccCreationVariable{m}

\ccTypes
\ccThreeToTwo

\ccTypedef{typedef std::size_t   size_type;}{}

\ccCreation

\ccSetThreeColumns{Timer&}{t.is_running();}{}
\ccPropagateThreeToTwoColumns
%\ccSetTwoColumns{Timer t( Timer r);}{}

\ccConstructor{Memory_sizer();}{Default constructor.}

\ccOperations

\ccMethod{size_type virtual_size() const;}{Returns the virtual memory size in bytes.}
\ccGlue
\ccMethod{size_type resident_size() const;}{Returns the resident memory size in bytes.}


\ccImplementation

Accessing this information requires the use of non-portable code.
Currently, there is support for Linux platforms and the Microsoft and Intel
compiler on Windows.  If a platform is not supported, the two member
functions return 0.

\end{ccRefClass}
