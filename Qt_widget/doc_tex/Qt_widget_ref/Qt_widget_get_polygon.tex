% +------------------------------------------------------------------------+
% | CGAL Reference Manual: Reference manual for Qt_widget.tex
% +------------------------------------------------------------------------+
% |
% | 03.01.2001  Radu Ursu
% | 
% | \RCSdef{\qtwidgetRev}{$Id$}
% | \RCSdefDate{\qtwidgetDate}{$Date$}
% +------------------------------------------------------------------------+

% +-----------------------------------------------------+
\begin{ccRefClass}{Qt_widget_get_polygon<Polygon>}

\ccDefinition
An object of type \ccRefName\ creates a \cgal\ polygon. A new 
vertex is inserted every time the left mouse button is pressed.
The polygon is returned on a right click. You can use the \ccc{Escape}
key if you want to remove your last entered point in the polygon.
The use of \ccRefName\ requires that the mouse tracking is
enabled for widgets attaching it.

\ccInclude{CGAL/IO/Qt_widget_get_polygon.h}

\ccParameters

The full template declaration of \ccc{Qt_widget_get_polygon} states one parameter:

\begin{tabbing}
\ccc{template <} \=\ccc{class Polygon >}\\
        \ccc{class Qt_widget_get_polygon;}
\end{tabbing}

Polygon should be a \cgal\ Polygon, or should provide the following
types:

\ccTypes
\ccTypedef{typedef Polygon::Point_2 Point_2;}{}
\ccTypedef{typedef Polygon::Segment_2 Segment2;}{}
\ccTypedef{typedef Polygon::Edge_const_iterator ECI;}{}
\ccTypedef{typedef Polygon::Vertex_iterator VI;}{}
\ccTypedef{typedef Polygon::FT FT;}{}

\ccInheritsFrom
\ccc{Qt_widget_tool}

\ccGlue

\ccCreation
\ccCreationVariable{getpoly}
\ccSetTwoColumns{Qt_widget_get_polygon}{}

\ccConstructor{Qt_widget_get_polygon<T>(const QCursor
c=QCursor(Qt::crossCursor), QObject* parent = 0, const char* name =
0);}{\ccc{c} is the cursor that this layer will use when is
active. \ccc{parent} is the parent widget and \ccc{name} is the name you 
give to this layer.}

\end{ccRefClass}

% +-----------------------------------------------------+
% EOF








