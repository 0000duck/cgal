% +------------------------------------------------------------------------+
% | CGAL Reference Manual: Reference manual for Qt_widget.tex
% +------------------------------------------------------------------------+
% |
% | 03.01.2001	Radu Ursu
% | 
% | \RCSdef{\qtwidgetRev}{$Id$}
% | \RCSdefDate{\qtwidgetDate}{$Date$}
% +------------------------------------------------------------------------+

% +-----------------------------------------------------+
\begin{ccRefClass}{Qt_widget_get_point<T>}
\ccDefinition
An object of type \ccRefName\ creates a \cgal\ point, every time 
the left mouse button is clicked.

\ccInclude{CGAL/IO/Qt_widget_get_point.h}

\ccParameters

The full template declaration of \ccc{Qt_widget_get_point} states one parameter:

\begin{tabbing}
\ccc{template <} \=\ccc{class T >}\\
	\ccc{class Qt_widget_get_point;}
\end{tabbing}

If T is one of the \cgal\ kernels you don't need additional types. If
not, the parameter T has to provide this types:

\ccTypes
\ccTypedef{typedef T::Point_2 Point_2;}{This should be a Point type}
\ccTypedef{typedef T::FT FT;}{This should be a Field type}

\ccInheritsFrom
\ccc{Qt_widget_layer}

\ccGlue

\ccCreation
\ccCreationVariable{getpoint}
\ccSetTwoColumns{Qt_widget_get_point}{}

\ccConstructor{Qt_widget_get_point<T>(const QCursor
c=QCursor(Qt::crossCursor), QObject* parent = 0, const char* name =
0);}{\ccc{c} is the cursor that this layer will use when is
active. \ccc{parent} is the parent widget and \ccc{name} is the name you 
give to this layer.}

\end{ccRefClass}

% +-----------------------------------------------------+
% EOF








