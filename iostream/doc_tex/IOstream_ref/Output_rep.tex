\begin{ccRefClass}{Output_rep<T,F>}

\ccDefinition

In some situations, you want to control the output formatting for a type \texttt{T}. For external types (third party libraries etc.), there might be problems if the stream output operator does not honor the CGAL output mode. The purpose of \ccc{Output_rep} is to provide a way to control output formatting that works independently of the object's stream output operator.

\ccc{Output_rep} is used in conjunction with the free function \ccc{oformat}. Both provide an intermediate layer between the actual output stream and the type \texttt{T}. \ccc{Output_rep<T>} can be regarded as a wrapper around \texttt{T} and \ccc{oformat} is just a convenience function to construct an instance of \ccc{Output_rep<T>}.

By default, the object's stream output operator is called. If you want another behaviour for your type \texttt{T}, you have to create a template specialization for that type. Furthermore, you can provide specializations with a second template parameter (a formatting tag). The second template parameter defaults to \ccc{Null_tag}.

Specializations of \ccc{Output_rep} should provide the following features:

\begin{ccExampleCode}
template< class F >
struct Output_rep< My_special_type, F > {
    Output_rep( const My_special_type& t );
    std::ostream& operator()( std::ostream& out ) const;
};
\end{ccExampleCode}



\ccExample
For example, if you want special formatting for \ccc{CORE::BigRat}, your \ccc{Output_rep} specialization could look like this:
\begin{ccExampleCode}
template <class F>
class Output_rep< ::CORE::BigRat, F> {
    const ::CORE::BigRat& t;
public:
    Output_rep( const ::CORE::BigRat& tt) : t(tt) {}

    std::ostream& operator()( std::ostream& out) const {
        switch (get_mode(out)) {
        case IO::PRETTY:{
            if(CGAL_CORE_DENOMINATOR(t) == ::CORE::BigRat(1))
                return out <<CGAL_CORE_NUMERATOR(t);
            else
                return out << CGAL_CORE_NUMERATOR(t)
                           << "/"
                           << CGAL_CORE_DENOMINATOR(t);
            break;
        }

        default:
            return out << CGAL_CORE_NUMERATOR(t)
                       << "/"
                       << CGAL_CORE_DENOMINATOR(t);
        }
    }
};
\end{ccExampleCode}


\end{ccRefClass}