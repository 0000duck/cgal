% +------------------------------------------------------------------------+
% | Reference manual page: input_op.tex
% +------------------------------------------------------------------------+
% | 10.04.2002   Author
% | Package: Package
% | 
\RCSdef{\RCSinputopRev}{$Id$}
\RCSdefDate{\RCSinputopDate}{$Date$}
% |
%%RefPage: end of header, begin of main body
% +------------------------------------------------------------------------+


\ccHtmlNoClassLinks
\begin{ccRefFunction}{operator>>}  %% add template arg's if necessary

%% \ccHtmlCrossLink{}     %% add further rules for cross referencing links
%% \ccHtmlIndexC[function]{} %% add further index entries

\ccDefinition
\cgal\  defines input operators for classes that are derived
from the class \ccStyle{istream}. This allows to read from istreams
as \ccStyle{cin}, as well as from strstreams and fstreams.
The input operator is defined for all classes in the \cgal\ kernel.


\ccFunction{istream& operator>>(istream& is, Class c);}
   {Extracts object \ccStyle{c} from the stream \ccc{is}. Returns \ccc{is}.}           
\ccSeeAlso
\ccRefIdfierPage{CGAL::set_mode} \\
\ccRefIdfierPage{CGAL::set_ascii_mode} \\
\ccRefIdfierPage{CGAL::set_binary_mode}  \\
\ccRefIdfierPage{CGAL::set_pretty_mode}  \\
\ccRefIdfierPage{CGAL::get_mode} \\
\ccRefIdfierPage{CGAL::is_ascii} \\
\ccRefIdfierPage{CGAL::is_binary} \\
\ccRefIdfierPage{CGAL::is_pretty} \\
\ccRefIdfierPage{CGAL::operator<<}

\ccExample
\begin{cprog}

#include <CGAL/basic.h>
#include <iostream>
#include <fstream>

#include <CGAL/Cartesian.h>
#include <CGAL/Segment_2.h>

typedef CGAL::Point_2< CGAL::Cartesian<double> >     Point;
typedef CGAL::Segment_2< CGAL::Cartesian<double> >   Segment;

int
main()
{
    Point p, q;
    Segment s;

    CGAL::set_ascii_mode(std::cin);
    std::cin >> p >> q;

    std::ifstream f("data.txt");
    CGAL::set_binary_mode(f);
    f >> s >> p;

    return 1;
}
\end{cprog} 

\end{ccRefFunction}

% +------------------------------------------------------------------------+
%%RefPage: end of main body, begin of footer
% EOF
% +------------------------------------------------------------------------+

