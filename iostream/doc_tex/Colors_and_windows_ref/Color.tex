% +------------------------------------------------------------------------+
% | Reference manual page: Color.tex
% +------------------------------------------------------------------------+
% | 05.04.2002   Author
% | Package: Package
% | 
\RCSdef{\RCSColorRev}{$Id$}
\RCSdefDate{\RCSColorDate}{$Date$}
% |
%%RefPage: end of header, begin of main body
% +------------------------------------------------------------------------+


\begin{ccRefClass}{Color}  %% add template arg's if necessary

%% \ccHtmlCrossLink{}     %% add further rules for cross referencing links
%% \ccHtmlIndexC[class]{} %% add further index entries

\ccDefinition
An object of the class \ccRefName\ is a color available
for drawing operations in \cgal\ output streams.
Each color is defined by a triple of integers $(r,g,b)$ with
$0\le r,g,b \le 255$, the so-called {\em rgb-value} of the color.  

\ccInclude{CGAL/IO/Color.h}

%\ccIsModel

%Concept

%\ccTypes

%\ccNestedType{TYPE}{some nested types}

\ccCreation
\ccCreationVariable{c}  %% choose variable name


\ccConstructor{Color();}
             {creates a color with rgb-value $(0,0,0)$, i.e.\ black.}

\ccConstructor{Color(int red, int green, int blue);}
             {creates a color with rgb-value \ccStyle{(red,green,blue)}.}


\ccOperations

\ccHidden \ccMethod{Color & operator=(const Color &q);}
        {Assignment.}

\ccMethod{bool operator==(const Color &q) const;}
       {Test for equality: Two colors are equal, iff their 
        rgb-values are equal.}
\ccGlue
\ccMethod{bool operator!=(const Color &q) const;}
       {Test for inequality.}

\ccMethod{int red() const;}
       {returns the  red component of \ccVar.}
\ccGlue
\ccMethod{int green() const;}
       {returns the  green component of \ccVar.}
\ccGlue
\ccMethod{int blue() const;}
       {returns the blue component of \ccVar.}

\ccConstants

The following constants are predefined:

\ccThree{const Color &}{WHITE = Color(255, 255, 255);}{}

\ccVariable{const Color BLACK  = Color(0, 0, 0);}{Black.}
\ccGlue
\ccVariable{const Color WHITE  = Color(255, 255, 255);}{White.}
\ccGlue
\ccVariable{const Color RED    = Color(255, 0, 0);}{Red.}
\ccGlue
\ccVariable{const Color GREEN  = Color(0, 255, 0);}{Green.}
\ccGlue
\ccVariable{const Color BLUE   = Color(0, 0, 255);}{Blue.}
\ccGlue
\ccVariable{const Color VIOLET = Color(255, 0, 255);}{Violet.}
\ccGlue
\ccVariable{const Color ORANGE = Color(255, 170, 0);}{Orange.}

\ccSeeAlso

\ccc{CGAL::Window_stream } \\
\ccc{CGAL::PostScript_File_Stream}

%\ccExample

%A short example program.
%Instead of a short program fragment, a full running program can be
%included using the 
%\verb|\ccIncludeExampleCode{Package/Color_prog.C}| 
%macro. The program example would be part of the source code distribution and
%also part of the automatic test suite.

%\begin{ccExampleCode}
%void your_example_code() {
%}
%\end{ccExampleCode}

%%% \ccIncludeExampleCode{Package/Color_prog.C}

\end{ccRefClass}

% +------------------------------------------------------------------------+
%%RefPage: end of main body, begin of footer
% EOF
% +------------------------------------------------------------------------+

