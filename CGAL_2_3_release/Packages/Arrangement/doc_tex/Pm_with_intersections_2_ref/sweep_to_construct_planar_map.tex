% +------------------------------------------------------------------------+
% | Reference manual page: Pm_bounding_box_traits.tex (Planar_map)
% +------------------------------------------------------------------------+
% | 
% | Package: pm (Planar_map_2)
% | 
% +------------------------------------------------------------------------+

\ccRefPageBegin

%%RefPage: end of header, begin of main body
% +------------------------------------------------------------------------+

%sweep_to_construct_planar_map
\begin{ccRefFunction}{sweep_to_construct_planar_map}

\ccCreationVariable{sweep_to_construct_planar_map}

\ccDefinition
\ccc{sweep_to_construct_planar_map} is a global function defined for constructing the \ccc{Planar map} 
induced by the input curves by the sweep line technique.


\ccInclude{CGAL/Sweep_curves_to_planar_map.h}

\def\ccLongParamLayout{\ccTrue} 

\ccGlobalFunction{
   template <class Curve_iterator, class Planar_map>
   void 
   sweep_to_construct_planar_map(
   Curve_iterator curves_begin, 
   Curve_iterator curves_end, 
   Planar_map &result);
}

constructs a \ccc{Planar map} from the curves given in the range 
\ccStyle{[curves_begin, curves_end)}. The parameter \ccStyle{result} holding the resulting \ccc{Planar map}, if \ccStyle{result} holds a non empty \ccc {Planar map} before calling to \ccc{sweep_to_construct_planar_map} - then \ccStyle{result} will be cleared and the curves \ccStyle{result} was holding would be sweeped together with the curves in \ccStyle{[curves_begin, curves_end)}.

\ccc{sweep_to_construct_planar_map} supports curves which are interior intersect, overlap or non x-monotone, and hence the user may provide his curves as input without any resriction of the kind mentioned above. 
This function is also updating the \ccc{Planar map} efficiently, since it makes no point location operations when inserting curves to the \ccc{Planar map}. Hence, inserting curves to a \ccc{Planar map}, even when there are no interior intersectrions between those curves and only iteratively inseting them to a \ccc{Planar map} is legal, will be more efficient when using \ccc{sweep_to_construct_planar_map}.

\ccSeeAlso
\ccRefPage{CGAL::sweep_to_produce_planar_map_subcurves}

\ccImplementation 
The implementation uses the container $map$ defined in STL for implementing 
the event queue and the status line.
The implementation also keeps per each input curve an associated container of all 
its intersection points ordered from left to right, and also keeps per each event point all 
its outcoming curves.   
Due to these structures, when constructing a subdivision no point location operation has to be done. 
The complexity of this algorithm is $O(nlogn + k)$ where $n$ is the number of the input curves 
and $k$ is the number of intersection points induced by these curves.

\end{ccRefFunction} % sweep_to_construct_planar_map

% +------------------------------------------------------------------------+
%%RefPage: end of main body, begin of footer
\ccRefPageEnd
% EOF
% +------------------------------------------------------------------------+











%%% Local Variables: 
%%% mode: latex
%%% TeX-master: "Pm_file_scanner"
%%% End: 
