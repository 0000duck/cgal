% +------------------------------------------------------------------------+
% | Reference manual page: Arr_subcurve_node.tex (Arrangement)
% +------------------------------------------------------------------------+
% | 
% | Package: arr (Arrangement_2)
% | 
% +------------------------------------------------------------------------+

\ccRefPageBegin

%%RefPage: end of header, begin of main body
% +------------------------------------------------------------------------+

% +========================================================================+
%   Arrangement_2::Subcurve_node
% +========================================================================+
\begin{ccRefClass}[Arrangement_2<Dcel,Traits,Base_node>::]{Subcurve_node}

\ccDefinition An object $sn$ of the class
    \ccClassTemplateName\ is a node holding a subcurve in the
    hierarchy tree of the arrangement.

\ccInheritsFrom
    \ccc{Base_node}

    The \ccClassTemplateName\ is the base class for the nodes of the tree.
    Its functions are inherited by the \ccc{Curve_node} and \ccc{Edge_node}
    classes.

    \ccCreationVariable{sn}
    
\ccOperations

    \ccMethod{virtual bool is_edge_node();}{returns false;}

    \ccMethod{Curve_iterator curve_node();}{returns an iterator to the
       \ccc{Curve_node} that is the root of the hierarchy tree \ccVar{}
       is in.}

    \ccMethod{Subcurve_iterator parent();}{returns an iterator to the node that
       is the parent of \ccVar{} in the hierarchy tree.
       Note that the return value is
       always a \ccc{Subcurve_iterator} even if the actual type is a
       \ccc{Curve_node}; in order to get the \ccc{Curve_node} which is the 
       root of the tree use the member function \ccc{curve_node()}.}
       %(or cast the \ccc{Subcurve_iterator} to \ccc{Curve_iterator}).}
    
  \ccMethod{Subcurve_iterator children_begin();}{returns a begin iterator to the
     children of \ccVar{} in the hierarchy tree.}
 
  \ccMethod{Subcurve_iterator children_end();}{returns a past-the-end iterator
     of the children of \ccVar{} in the hierarchy tree.}

  \ccMethod{Edge_iterator edges_begin();}{returns a begin iterator to the
     edges in the subtree ``hanging'' from \ccVar, i.e., the subtree that
     \ccVar{} is a root of.
     \ccPrecond{\ccVar{} has a non-empty edge level.}}

  \ccMethod{Edge_iterator edges_end();}{returns a past-the-end iterator to the
     edges in the subtree ``hanging'' from \ccVar.
     \ccPrecond{\ccVar{} has a non-empty edge level.}}


\end{ccRefClass}
% +------------------------------------------------------------------------+
%%RefPage: end of main body, begin of footer
\ccRefPageEnd
% EOF
% +------------------------------------------------------------------------+
