% +------------------------------------------------------------------------+
% | Reference manual page: Arr_file_writer.tex (Arrangement)
% +------------------------------------------------------------------------+
% | 
% | Package: arr (Arrangement)
% | 
% +------------------------------------------------------------------------+

\ccRefPageBegin

%%RefPage: end of header, begin of main body
% +------------------------------------------------------------------------+

%Arr_file_writer
\begin{ccRefClass}{Arr_file_writer<Arrangement>}

\ccCreationVariable{arr_file_writer}

\ccDefinition
This class contains functions for writing the \ccc{Arrangement} components to a given output stream.
Users have to make sure that methods for writing \ccStyle{Point} 
and \ccStyle{X\_curve} objects defined in the \ccc{Arrangement traits} 
class are provided when using the output stream they choose to work with..
This class is templated with \ccc{Arrangement}, and hence the user can work with \ccRefName\ 
with its own \ccc{Arrangement}. 
 
\ccInheritsFrom
\ccc{Pm_file_writer}

%The following requirement catalog lists the primitives, i.e., types, member functions
%etc., that must be defined for any \ccc{Planar map} file writer class that can be used to parameterize the reading function of \ccc{Planar map}. 

\ccTypes

The following handles and iterators have appropriate constant counterparts. 
The mutable types are assignable to their constant counterparts. 
The iterators are assignable to the respective handle types. 
Wherever the handles appear in function parameter lists, the appropriate iterator can be used as well. 

\ccNestedType{Curve_iterator}{A bidirectional iterator over all curve nodes of the \ccc{arrangement}.}
\ccNestedType{Subcurve_iterator}{A bidirectional iterator over all subcurve nodes of the \ccc{arrangement}.}
\ccNestedType{Edge_iterator}{A bidirectional iterator over all edge nodes of the \ccc{arrangement}.}

\ccCreation
As described in the introduction, the \ccc{Arrangement} users can choose a verbose format for 
printing their arrangement, this is done by putting this option in the constructor.

\ccConstructor{Arr_file_writer(std::ostream& o, const Arrangement& arr, bool verbose = false);} 
{constructs \ccRefName\ for a given output stream, the specified \ccc{Arrangement} and a verbose choice.}

\ccConstructor{Arr_file_writer(std::ostream& o, const Arrangement& arr, const File_header& h)} 
{constructs \ccRefName\ for a given output stream, the specified \ccc{Arrangement} and \ccc{File header}.}

\ccOperations

The operation \ccRefName \/ supports are all the operations \ccc{Pm_file_writer} supported plus the following:

\ccMethod{void write_curve (Curve_iterator cv);}
{writes a curve node to the output stream.}

\ccMethod{void write_subcurve (Subcurve_iterator scv);}
{writes a subcurve node to the output stream.}

\ccMethod{void write_edge (Edge_iterator edge);}
{writes an edge node to the output stream.}

\end{ccRefClass} % Arr_file_writer

% +------------------------------------------------------------------------+
%%RefPage: end of main body, begin of footer
\ccRefPageEnd
% EOF
% +------------------------------------------------------------------------+











%%% Local Variables: 
%%% mode: latex
%%% TeX-master: "Pm_drawer"
%%% End: 
