% +------------------------------------------------------------------------+
% | Reference manual page: Pm_dynamic_open_bounding_box.tex (Planar_map)
% +------------------------------------------------------------------------+
% | 
% | Package: pm (Planar_map_2)
% | 
% +------------------------------------------------------------------------+

\ccRefPageBegin

%%RefPage: end of header, begin of main body
% +------------------------------------------------------------------------+


\begin{ccRefClass}{Pm_dynamic_open_bounding_box<Planar_map>}
\label{Pm_ref:dynamic_open}

\ccDefinition
    The \ccRefName\ class 
    implements a dynamic bounding box, namely the bounding box increases 
    keeping all curves
    bound inside it. We name this strategy \ccc{open} because the curves on
    the bounding box boundary are not considered as part of the map.
    The \ccc{Pm_dynamic_open_bounding_box} strategy maintains the planar map
    along with a bounding box, large enough to bound all the finite curves
    inserted into 
    the map, and to intersect all the unbounded curves. In such a case we 
    say that the 
    bounding box {\it bound inside} the curve.
    Each curve is represented as its intersection with the bounding box, 
    allowing the planar map to simulate insertion of bounded curves. 

\ccInclude{CGAL/Pm_dynamic_open_bounding_box.h}

\ccIsModel
  \ccc{PlanarMapBoundingBox_2}

\ccInheritsFrom
  \ccc{Pm_bounding_box_base<Planar_map>}

\ccSeeAlso
   Discussion of the different bounding box strategies in the introduction
of \ccc{Planar_map} reference pages\lcTex{ (\ccRefPage{Pm_Ref_intro})}.

\end{ccRefClass} % Pm_dynamic_open_bounding_box<Planar_map>}

% +------------------------------------------------------------------------+
%%RefPage: end of main body, begin of footer
\ccRefPageEnd
% EOF
% +------------------------------------------------------------------------+
