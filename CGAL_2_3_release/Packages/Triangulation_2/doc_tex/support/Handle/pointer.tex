%% $RCSfile$
% $Revision$
% $Date$

\chapter{Handles} \label{I1_ChapterPointer}


\begin{ccClassTemplate}{Handle<T>}
\section{Requirements for a handle}

We present in this chapter the concept of a handle to an object
of type \ccc{T}. Basically a handle to an object of type \ccc{T}
behaves like a pointer to an object of type \ccc{T} except that it has
no increment nor decrement operators and does not deal with memory
allocation or deallocation. Essentially a
 handle has two operators
* and \ccc{->} which gives  access to the object pointed to.
 

%\begin{ccClassTemplate}{Handle<T>}

\ccTypes

\ccTypedef{typedef T value_type;}{}

\ccCreation
\ccCreationVariable{ht}
\ccThree{Handle<T>& }{operator==(const Pointer &hr);}{}
\ccThreeToTwo

\ccConstructor{Handle()}{\ccVar\ references no object.}
\ccGlue
\ccConstructor{Handle(const Handle<T>& hr);}{copy constructor}
\ccGlue
\ccConstructor{Handle(const T*& pt);}{constructs a reference from a
pointer \ccc{T*}.}

\ccMethod{Handle<T>& operator=(const T*& pr);} {\ccVar\ points to the object *pr}
\ccGlue
\ccMethod{Handle<T>&  operator=(const Handle<T>& pr);} {\ccVar\ and pr point
to the same object.}
\ccGlue
%\ccMethod{void clear();} {sets \ccVar\ to reference no object.}

\ccHeading{Dereference}

\ccMethod{T& operator*();}{returns the object pointed to.}
\ccGlue
\ccMethod{T* operator->();}{returns a pointer to the object pointed to.}

%\ccHeading{Memory allocation}

%\ccMethod{void Delete();} {deallocate the memory referenced by \ccVar.}


\ccHeading{Equality and Comparison}
%\ccMethod { bool is_null() const; }{ true if \ccVar\ references no
%object.}
%\ccGlue
\ccMethod{ bool operator==(const Handle<T>& hr) const;}
{true if \ccVar\ and \ccc{q} reference the same object.}
\ccGlue
\ccMethod{ bool operator!=(const Handle<T>& hr) const;}{true if
\ccVar\ and \ccc{hr} reference distinct objects.}

The following operator is not quite compulsory, but very conveniemt if 
present to define structures like \ccc{set<Handle<T> >}
\ccMethod{bool operator<(const Handle<T>& p) const;} {true if 
\ccc{ptr() < p.ptr()}}

\end{ccClassTemplate} 

\section{Model for a handle}



\begin{ccClassTemplate} {Pointer<T>}


\ccThree{typedef Pointer<T>}{Pointer}{}
\ccThreeToTwo



The class \ccClassTemplateName\ provides a model for a handle
pointing to an object
of class \ccc{T}. 

\ccInclude{CGAL/Pointer.h}

Note that a simple pointer \ccc{T*},
an iterator or a circulator with value type \ccc{T},
can also be used as handles for object of type  \ccc{T}.


\end{ccClassTemplate} 
