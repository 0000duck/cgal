
% =============================================================================
% The CGAL Reference Manual
% Chapter: Geometric Optimisation
% Section: Smallest Enclosing Sphere
% -----------------------------------------------------------------------------
% file  : doc_tex/basic/Optimisation/Optimisation_ref/Min_sphere_d.tex
% author: Bernd G�rtner (gaertner@inf.ethz.ch)
% -----------------------------------------------------------------------------
% $Revision$
% $Date$
% $CGAL_Package: Min_sphere_d WIP $
% =============================================================================

    \begin{ccRefClass}{Min_sphere_d<Traits>}
    \ccIndexSubitem[t]{sphere}{smallest enclosing}
    \ccIndexSubitem[t]{smallest enclosing}{sphere}
    \ccIndexSubitem[t]{bounding volumes}{smallest enclosing sphere}
    \ccIndexSubitemSeeAlso[t]{annulus}{smallest enclosing sphere}
    \ccIndexSubitemSeeAlso[t]{circle}{smallest enclosing sphere}
\cgalSetMinSphereLayout

% -----------------------------------------------------------------------------
\ccDefinition

An object of the class \ccRefName\ is the unique sphere of
smallest volume enclosing a finite (multi)set of points in $d$-dimensional
Euclidean space $\E_d$. For a set $P$ we denote by $ms(P)$ the
smallest sphere that contains all points of $P$. $ms(P)$ can
be degenerate, i.e.\ $ms(P)=\mbox{\ccTexHtml{$\;\emptyset$}{&Oslash;}}$
if $P=\mbox{\ccTexHtml{$\;\emptyset$}{&Oslash;}}$ and $ms(P)=\{p\}$ if
$P=\{p\}$.

An inclusion-minimal subset $S$ of $P$ with $ms(S)=ms(P)$ is called a
\emph{support set}, the points in $S$ are the \emph{support points}.
A support set has size at most $d+1$, and all its points lie on the
boundary of $ms(P)$. In general, neither the support set nor its size
are unique.

The algorithm
computes a support set $S$ which remains fixed until the next insert
or clear operation.

    \ccInclude{CGAL/Min_sphere_d.h}

    \ccRequirements
    \ccIndexRequirements

    The class \ccRefName\ expects a model of the concept
    \ccc{OptimisationDTraits} as its template argument.
      We provide the models \ccc{Optimisation_d_traits_2},
      \ccc{Optimisation_d_traits_3} and \ccc{Optimisation_d_traits_d}
    for two-, three-, and $d$-dimensional points respectively.

\ccTypes
\ccIndexClassTypes

\ccSetThreeColumns{typedef Traits::Point}{Sphere;}{}

\ccNestedType{Traits}{}
\ccNestedType{FT}{typedef to \ccc{Traits::FT}.}
\ccNestedType{Point}{typedef to \ccc{Traits::Point}.}


\ccSetTwoColumns{Min_sphere_d<Traits>:: Support_point_iterator}{}

\ccGlueBegin
\ccUnchecked
    \ccNestedType{Point_iterator}{non-mutable model of the STL
    concept \ccc{BidirectionalIterator} with value type \ccc{Point}. Used
    to access the points used to build the smallest enclosing sphere.}
    \ccNestedType{Support_point_iterator}{non-mutable model of the STL
    concept \ccc{BidirectionalIterator} with value type \ccc{Point}. Used
    to access the support points defining the smallest enclosing sphere.}
\ccGlueEnd

\cgalSetMinSphereLayout

% -----------------------------------------------------------------------------
\ccCreation
\ccIndexClassCreation
\ccCreationVariable{min_sphere}

\ccUnchecked
\ccConstructor{ Min_sphere_d (const Traits& traits = Traits());}{
        creates a variable of type \ccRefName\ and
        initializes it to $ms(\mbox{\ccTexHtml{$\;\emptyset$}{&Oslash;}})$.
          If the traits parameter is not supplied, the class \ccc{Traits}
          must provide a default constructor.}

\ccUnchecked
\ccConstructor{ template < class InputIterator >
                Min_sphere_d( InputIterator  first,
                                   InputIterator  last,
                                   const Traits&  traits = Traits());}{
        creates a variable \ccVar\ of type \ccRefName.
        It is initialized to $ms(P)$ with $P$ being the set of points
        in the range [\ccc{first},\ccc{last}).
        \ccRequire The value type of \ccc{first} and \ccc{last}
        is \ccc{Point}. If the traits parameter is not supplied,
        the class \ccc{Traits} must provide a default constructor.
        \ccPrecond All points have the same dimension.}

\ccUnchecked
\ccHidden
\ccConstructor{ Min_sphere_d( const Min_sphere_d<Traits>&);}{
        copy constructor.}

\ccHidden
\ccConstructor{ ~Min_sphere_d( );}{
        destructor.}

\ccUnchecked
\ccHidden
\ccMemberFunction{ Min_sphere_d<Traits>&
                   operator = ( const Min_sphere_d<Traits>&);}{
        assignment operator.}

% -----------------------------------------------------------------------------\ccAccessFunctions
\begin{ccIndexMemberFunctions}
\ccIndexMemberFunctionGroup{access}

\ccIndexSubitem[t]{support set}{\ccFont Min_sphere_d}
\ccMemberFunction{ int  number_of_points( ) const;}{
        returns the number of points of \ccVar, i.e.\ $|P|$.}

\ccMemberFunction{ int  number_of_support_points( ) const;}{
        returns the number of support points of \ccVar, i.e.\ $|S|$.}

\ccGlueBegin
\ccMemberFunction{ Point_iterator  points_begin() const;}{
        returns an iterator referring to the first point of \ccVar.}
%
\ccMemberFunction{ Point_iterator  points_end() const;}{
        returns the corresponding past-the-end iterator.}
\ccGlueEnd

\ccGlueBegin
\ccMemberFunction{ Support_point_iterator  support_points_begin() const;}{
        returns an iterator referring to the first support point of \ccVar.}
%
\ccMemberFunction{ Support_point_iterator  support_points_end() const;}{
        returns the corresponding past-the-end iterator.}
\ccGlueEnd

\ccMemberFunction{ int ambient_dimension() const;}{
        returns the dimension of the points in $P$. If \ccVar\
        is empty, the ambient dimension is $-1$.}

\ccMemberFunction{ const Point&  center( ) const;}{
        returns the center of \ccVar.
        \ccPrecond \ccVar\ is not empty.}

        \ccMemberFunction{FT squared_radius( ) const;}{
        returns the squared radius of \ccVar.
        \ccPrecond \ccVar\ is not empty.}

% -----------------------------------------------------------------------------
\ccPredicates
\ccIndexMemberFunctionGroup{predicates}

By definition, an empty \ccRefName\ has no boundary and no
bounded side, i.e.\ its unbounded side equals the whole space $\E_d$.

\ccMemberFunction{ Bounded_side
                   bounded_side( const Point& p) const;}{
        returns \ccc{CGAL::ON_BOUNDED_SIDE}, \ccc{CGAL::ON_BOUNDARY}, or
        \ccc{CGAL::ON_UNBOUNDED_SIDE} iff \ccc{p} lies properly inside,
        on the boundary, or properly outside of \ccVar, resp.
        \ccPrecond if \ccVar\ is not empty, the dimension of $p$
        equals \ccc{ambient_dimension()}.}

\ccMemberFunction{ bool  has_on_bounded_side( const Point& p) const;}{
        returns \ccc{true}, iff \ccc{p} lies properly inside \ccVar.
        \ccPrecond if \ccVar\ is not empty, the dimension of $p$
        equals \ccc{ambient_dimension()}.}

\ccMemberFunction{ bool  has_on_boundary( const Point& p) const;}{
        returns \ccc{true}, iff \ccc{p} lies on the boundary
        of \ccVar.
        \ccPrecond if \ccVar\ is not empty, the dimension of $p$
        equals \ccc{ambient_dimension()}.}

\ccMemberFunction{ bool  has_on_unbounded_side( const Point& p) const;}{
        returns \ccc{true}, iff \ccc{p} lies properly outside of \ccVar.
        \ccPrecond if \ccVar\ is not empty, the dimension of $p$
        equals \ccc{ambient_dimension()}.}

\ccMemberFunction{ bool  is_empty( ) const;}{
        returns \ccc{true}, iff \ccVar\ is empty (this implies
        degeneracy).}

\ccIndexSubitem[t]{degeneracies}{\ccFont Min_sphere_d}
\ccMemberFunction{ bool  is_degenerate( ) const;}{
        returns \ccc{true}, iff \ccVar\ is degenerate, i.e.\ if
        \ccVar\ is empty or equal to a single point, equivalently if
        the number of support points is less than 2.}

% -----------------------------------------------------------------------------
\ccModifiers
\ccIndexMemberFunctionGroup{modifiers}

\ccMemberFunction{ void clear ();}{
        resets \ccVar\ to $ms(\mbox{\ccTexHtml{$\;\emptyset$}{&Oslash;}})$.}

\ccUnchecked
\ccMemberFunction{ template < class InputIterator >
                   void set( InputIterator first,
                             InputIterator last );}{
        sets \ccVar\ to the $ms(P)$, where $P$ is the set of points
        in the range [\ccc{first},\ccc{last}).
        \ccRequire The value type of \ccc{first} and \ccc{last} is
        \ccc{Point}.
        \ccPrecond All points have the same dimension.}

\ccMemberFunction{ void  insert( const Point& p);}{
        inserts \ccc{p} into \ccVar. If \ccc{p} lies inside the
        current sphere, this is a constant-time operation, otherwise
        it might take longer, but usually substantially less than
        recomputing the smallest enclosing sphere from scratch.
        \ccPrecond The dimension of \ccc{p} equals \ccc{ambient_dimension()}
        if \ccVar\ is not empty.}

\ccUnchecked
\ccMemberFunction{ template < class InputIterator >
                   void  insert( InputIterator  first,
                                 InputIterator  last );}{
        inserts the points in the range [\ccc{first},\ccc{last})
        into \ccVar\ and recomputes the smallest enclosing sphere, by
        calling \ccc{insert} for all points in the range.
        \ccRequire  The value type of \ccc{first} and \ccc{last} is
      \ccc{Point}.
        \ccPrecond All points have the same dimension. If
        \ccVar\ is not empty, this dimension must be equal to
        \ccc{ambient_dimension()}.}

% -----------------------------------------------------------------------------
    \ccHeading{Validity Check}
    \ccIndexMemberFunctionGroup{validity check}
  \ccIndexSubitem[t]{validity check}{\ccFont Min_sphere_d}
  An object \ccVar\ is valid, iff
  \begin{itemize}
    \item \ccVar\ contains all points of its defining set $P$,
    \item \ccVar\ is the smallest sphere containing its support set $S$, and
    \item $S$ is minimal, i.e.\ no support point is redundant.
  \end{itemize}

  \emph{Note:} Under inexact arithmetic, the result of the
  validation is not realiable, because the checker itself can suffer
  from numerical problems.

  \ccMemberFunction{ bool is_valid( bool verbose = false,
                                    int  level   = 0    ) const;}{
        returns \ccc{true}, iff \ccVar\ is valid. If \ccc{verbose}
        is \ccc{true}, some messages concerning the performed checks
        are written to standard error stream. The second parameter
        \ccc{level} is not used, we provide it only for consistency
        with interfaces of other classes.}

% -----------------------------------------------------------------------------
\ccHeading{Miscellaneous}
\ccIndexMemberFunctionGroup{miscellaneous}

\def\ccTagRmConstRefPair{\ccFalse}
\ccMemberFunction{ const Traits&  traits( ) const;}{
        returns a const reference to the traits class object.}
        \def\ccTagRmConstRefPair{\ccTrue}

 \end{ccIndexMemberFunctions}

% -----------------------------------------------------------------------------
\ccHeading{I/O}
\begin{ccIndexGlobalFunctions}

\ccIndexGlobalFunctionGroup{output}
\ccFunction{ std::ostream& operator << ( std::ostream& os,
                                    const Min_sphere_d<Traits>&
                                        min_sphere);}{
        writes \ccVar\ to output stream \ccc{os}.
        \ccRequire The output operator is defined for \ccc{Point}.}

\ccIndexGlobalFunctionGroup{input}

\ccFunction{ std::istream& operator >> ( std::istream& is,
                                    Min_sphere_d<Traits> min_sphere&);}{
        reads \ccVar\ from input stream \ccc{is}.
          \ccRequire The input operator is defined for \ccc{Point}.}

\end{ccIndexGlobalFunctions}

% -----------------------------------------------------------------------------
\ccSeeAlso

    \ccRefIdfierPage{CGAL::Optimisation_d_traits_2<K,ET,NT>}\\
    \ccRefIdfierPage{CGAL::Optimisation_d_traits_3<K,ET,NT>}\\
    \ccRefIdfierPage{CGAL::Optimisation_d_traits_d<K,ET,NT>}\\
    \ccRefConceptPage{OptimisationDTraits}\\
    \ccRefIdfierPage{CGAL::Min_circle_2<Traits>}\\
    \ccRefIdfierPage{CGAL::Min_annulus_d<Traits>}

% -----------------------------------------------------------------------------
\ccImplementation
\ccIndexImplementation

    \ccIndexSubitem[t]{incremental algorithm}{\ccFont Min_sphere_d}
    \ccIndexSubitem[t]{move-to-front heuristic}{\ccFont Min_sphere_d}
We implement the algorithm of Welzl with move-to-front
heuristic~\cite{w-sedbe-91a} for small point sets, combined with a new
efficient method for large sets, which is particularly tuned for
moderately large dimension ($d \leq 20$) \cite{g-frseb-99}.
The creation time is almost
always linear in the number of points. Access functions and predicates
take constant time, inserting a point might take up to linear time,
but substantially less than computing the new smallest enclosing
sphere from scratch. The clear operation and the check for validity
    each take linear time.

    \ccExample
    \ccIncludeVerbatim{Optimisation_ref/min_sphere_d_example.C}

\end{ccRefClass}

% ===== EOF ===================================================================

