% +------------------------------------------------------------------------+
% | Reference manual page: HalfedgeDSFace.tex
% +------------------------------------------------------------------------+
% | 22.03.1999   Lutz Kettner
% | Package: HalfedgeDS
% | 
\RCSdef{\RCSFaceRev}{$Revision$}
\RCSdefDate{\RCSFaceDate}{$Date$}
% +------------------------------------------------------------------------+

\ccRefPageBegin

%%RefPage: end of header, begin of main body
% +------------------------------------------------------------------------+


\begin{ccRefConcept}{HalfedgeDSFace}
\label{pageHalfedgeDSItemsFaceRef}

\ccDefinition
  
The concept \ccRefName\ defines the requirements for the local \ccc{Face} 
type in the \ccc{HalfedgeDS} concept. It is also required in 
the \ccc{Face_wrapper<Refs,Traits>} member class template of an
items class, see the \ccc{HalfedgeDSItems} concept.

A face optionally stores a reference to an incident halfedge that
points to the face.  A type tag indicates whether the related member
functions are supported. 
Figure~\ccTexHtml{\ref{figureHalfedgeDSOptionalMethods} 
on page \pageref{figureHalfedgeDSOptionalMethods}}{}\begin{ccHtmlOnly}
  <A HREF="Halfedge.html#figureHalfedgeDSOptionalMethods"><IMG 
  SRC="cc_ref_up_arrow.gif" ALT="reference arrow" WIDTH="10" HEIGHT="10"></A>
\end{ccHtmlOnly}
depicts the relationship between a halfedge and its incident
halfedges, vertices, and faces.

For the protection of the integrity of the data structure classes such as
\ccc{CGAL::Polyhedron_3} are allowed to redefine the modifying member 
functions to be private. In order to make them accessible for the 
halfedge data structure they must be derived from a base class \ccc{Base}
where the modifying member functions are still public. (The protection
can be bypassed by the user, but not by accident.)

\ccTypes

\ccThree{Halfedge_const_handle}{v.set_halfedge( Halfedge_handle h);}{}
\ccThreeToTwo
%\ccTwo{HalfedgeDSFace:: Halfedge_const_handleXX}{}

\ccNestedType{HalfedgeDS}
    {instantiated \ccc{HalfedgeDS} ( $\equiv$ \ccc{Refs}).}
\ccGlue
\ccNestedType{Base}{base class that allows modifications.}
\ccGlue
\ccNestedType{Vertex}{model of \ccc{HalfedgeDSVertex}.}
\ccGlue
\ccNestedType{Halfedge}{model of \ccc{HalfedgeDSHalfedge}.}
\ccGlue
\ccNestedType{Vertex_handle}{handle to vertex.}
\ccGlue
\ccNestedType{Halfedge_handle}{handle to halfedge.}
\ccGlue
\ccNestedType{Face_handle}{handle to face.}
\ccGlue
\ccNestedType{Vertex_const_handle}{}
\ccGlue
\ccNestedType{Halfedge_const_handle}{}
\ccGlue
\ccNestedType{Face_const_handle}{}
\ccGlue
\ccNestedType{Supports_face_halfedge}{\ccc{CGAL::Tag_true} or 
  \ccc{CGAL::Tag_false}.}

\ccCreation
\ccCreationVariable{f}

\ccConstructor{Face();}{default constructor.}

\ccTagFullDeclarations

\ccHeading{Operations available if \ccc{Supports_face_halfedge} $\equiv$ 
           \ccc{CGAL::Tag_true}}

\ccMethod{Halfedge_handle       halfedge();}{}
\ccGlue
\ccMethod{Halfedge_const_handle halfedge() const;}{
    incident halfedge that points to \ccVar.}
\ccGlue
\ccMethod{void set_halfedge( Halfedge_handle h);}{
    sets incident halfedge to $h$.}


\ccHasModels

\ccRefIdfierPage{CGAL::HalfedgeDS_face_base<Refs>}\\
\ccRefIdfierPage{CGAL::HalfedgeDS_face_min_base<Refs>}

\ccSeeAlso

\ccRefConceptPage{HalfedgeDS<Traits,Items,Alloc>}\\
\ccRefConceptPage{HalfedgeDSItems}\\
\ccRefConceptPage{HalfedgeDSVertex}\\
\ccRefConceptPage{HalfedgeDSHalfedge}

\ccTagDefaults
\end{ccRefConcept}

% +------------------------------------------------------------------------+
%%RefPage: end of main body, begin of footer
\ccRefPageEnd
% EOF
% +------------------------------------------------------------------------+

