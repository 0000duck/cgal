% +------------------------------------------------------------------------+
% | Reference manual page: Extended_cartesian.tex
% +------------------------------------------------------------------------+
% | 26.06.2001 Michael Seel
% | Package:   Nef_2
% | 
% +------------------------------------------------------------------------+

\ccAutoIndexingOff
\begin{ccRefClass}{Extended_cartesian<FT>}
\ccAutoIndexingOn
\ccIndexTraitsClassBegin{Extended_cartesian}{}{Nef polyhedron, 2D}

\ccDefinition
  
The class \ccRefName\ serves as a traits class for the class
\ccc{CGAL::Nef_polyhedron_2<T>}.  It uses a polynomial component
representation based on a field number type \ccc{FT}.

\ccInclude{CGAL/Extended_cartesian.h}

\ccIsModel

\ccRefConceptPage{ExtendedKernelTraits_2}%
\ccIndexSubitem[c]{ExtendedKernelTraits_2}{model} \\

\ccCreation
\ccCreationVariable{traits}  %% choose variable name

\ccConstructor{Extended_cartesian();}{default constructor.}

\ccHeading{Requirements} 

To make a field number type \ccc{FT_model} work with this class you
have to provide the following number type traits class in global
namespace.

\begin{ccExampleCode}
template <>
struct ring_or_field<FT_model> {
  typedef field_with_div kind;
};    
\end{ccExampleCode}

\ccOperations

Fits all operation requirements of the concept.

\ccSeeAlso

\ccRefIdfierPage{CGAL::Extended_homogeneous<RT>} \\
\ccRefIdfierPage{CGAL::Filtered_extended_homogeneous<RT>} 


\ccIndexTraitsClassEnd
\ccAutoIndexingOff
\end{ccRefClass}
\ccAutoIndexingOn

% EOF
