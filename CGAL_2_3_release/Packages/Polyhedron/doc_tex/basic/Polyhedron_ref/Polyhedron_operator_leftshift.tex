% +------------------------------------------------------------------------+
% | Reference manual page: Polyhedron_operator_leftshift.tex
% +------------------------------------------------------------------------+
% | 09.09.2000   Lutz Kettner
% | Package: Polyhedron
% | 
\RCSdef{\RCSPolyhedronoperatoristreamRev}{$Revision$}
\RCSdefDate{\RCSPolyhedronoperatoristreamDate}{$Date$}
% |
%%RefPage: end of header, begin of main body
% +------------------------------------------------------------------------+


\ccHtmlNoClassLinks
\begin{ccRefFunction}{operator<<}
\label{refPolyhedron_operator_leftshift}

\ccDefinition

This operator writes the polyhedral surface $P$ to the output stream
\ccc{out} using the Object File Format, OFF, with file extension 
{\tt .off}, which is also understood by
GeomView~\cite{p-gmgv15-94}. The output is in ASCII format. From the
polyhedral surface, only the point coordinates and facets are
written. Neither normal vectors nor color attributes are used.

For OFF an ASCII and a binary format exist. The format can be selected with
the \cgal\ modifiers for streams, \ccc{set_ascii_mode} and
\ccc{set_binary_mode} respectively. The modifier \ccc{set_pretty_mode}
can be used to allow for (a few) structuring comments in the
output. Otherwise, the output would be free of comments.  The default
for writing is ASCII without comments.

\ccInclude{CGAL/IO/Polyhedron_iostream.h}

\ccGlobalFunction{template <class PolyhedronTraits_3>
    ostream& operator<<( ostream& out, 
                         const CGAL::Polyhedron_3<PolyhedronTraits_3>& P);}
  
\ccSeeAlso

\ccRefIdfierPage{CGAL::Polyhedron_3<Traits>}\\ 
\lcTex{\ccc{operator>>} \dotfill\ 
    page~\pageref{refPolyhedron_operator_rightshift}}%
\lcRawHtml{
    <I><A HREF="Function_operator--.html">operator&gt;&gt;</A></I>.
}


\end{ccRefFunction}

% +------------------------------------------------------------------------+
%%RefPage: end of main body, begin of footer
% EOF
% +------------------------------------------------------------------------+

