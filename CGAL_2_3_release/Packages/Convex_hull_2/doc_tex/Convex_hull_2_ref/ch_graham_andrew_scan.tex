% +------------------------------------------------------------------------+
% | Reference manual page: ch_graham_andrew.tex
% +------------------------------------------------------------------------+
% | 09.05.2001   Susan Hert and Stefan Schirra
% | Package: Convex_hull_2
% |
% +------------------------------------------------------------------------+


\begin{ccRefFunction}{ch_graham_andrew_scan}  %% add template arg's if necessary
\ccIndexSubitemBegin{convex hull, 2D}{Graham-Andrew scan}
\ccIndexSubitemBegin{extreme points, 2D}{right of line}

\ccDefinition
  
The function \ccRefName\ generates the counterclockwise sequence of extreme
points from a given set of input points that are not left of the line defined
by the first and last points in this sequence.  


\ccInclude{CGAL/ch_graham_andrew.h}

\ccFunction{template <class BidirectionalIterator, class OutputIterator, 
                      class Traits>
            OutputIterator
            ch_graham_andrew_scan( BidirectionalIterator first,
                                   BidirectionalIterator beyond,
                                   OutputIterator        result,
                                   const Traits& ch_traits = Default_traits);}
           {generates the counterclockwise sequence of extreme points that are
            not left of $pq$, where $p$ is the value of \ccc{first} and $q$ is
            the value of \ccc{beyond} $-1$. The resulting sequence is placed
            starting at \ccc{result} with $p$; point $q$ is omitted.  The
            past-the-end iterator for the sequence is returned.
            \ccPrecond %\ccIndexSubitem[C]{ch_graham_andrew_scan}{preconditions}
            The range [\ccc{first},\ccc{beyond}) contains at least
            two different points.
            The points in [\ccc{first},\ccc{beyond}) are ``sorted'' with respect
            to $pq$, {\it i.e.}, the sequence of points in 
            [\ccc{first},\ccc{beyond}) define a counterclockwise polygon, 
            for which the Graham-Sklansky-procedure \cite{s-mcrm-72} works.}

The default traits class \ccc{Default_traits} is the kernel in which the
type \ccc{BidirectionalIterator::value_type} is defined.


\ccHeading{Requirements}
\begin{enumerate}
   \item    \ccc{BidirectionalIterator::value_type} and 
            \ccc{OutputIterator::value_type}
            are equivalent to \ccc{Traits::Point_2}.
   \item    \ccc{Traits} defines the following two types from
            the concept ConvexHullTraits\_2 and their corresponding member
            %\ccIndexMainItem[c]{ConvexHullTraits_2}
            functions that return instances of these types:
            \begin{itemize}
                \item \ccc{Traits::Point_2},
                \item \ccc{Traits::Leftturn_2}.
            \end{itemize}
\end{enumerate}


\ccSeeAlso

\ccRefIdfierPage{CGAL::ch_graham_andrew} \\
\ccRefIdfierPage{CGAL::lower_hull_points_2} \\
\ccRefIdfierPage{CGAL::upper_hull_points_2} 

\ccImplementation

The function uses Andrew's 
variant of the Graham scan algorithm \cite{a-aeach-79} . This algorithm 
requires $O(n \log n)$ time in the worst case for $n$ input points.  

\ccExample

In the following example \ccc{ch_graham_andrew_scan()} is used to
realize Anderson's variant \cite{a-readc-78} of the Graham Scan 
\cite{g-eadch-72}.  The points are sorted counterclockwise around the leftmost 
point using the \ccc{Less_rotate_ccw_2} predicate, as defined in
the concept ConvexHullTraits\_2. According to the definition 
of \ccc{Less_rotate_ccw_2}, the leftmost point is the last point in the sorted 
sequence and its predecessor on the convex hull is the first point in the 
sorted sequence.  It is not hard to see that the preconditions of
\ccc{ch_graham_andrew_scan()} are satisfied.  Anderson's variant of the 
Graham scan is usually inferior to Andrew's variant because of its higher 
arithmetic demand.

\begin{verbatim}
template <class InputIterator, class OutputIterator, class Traits>
OutputIterator
ch_graham_anderson( InputIterator  first, InputIterator  beyond,
                    OutputIterator result, const Traits&  ch_traits)
{
  typedef typename Traits::Less_xy_2          Less_xy_2;
  typedef typename Traits::Point_2            Point_2;
  typedef typename Traits::Less_rotate_ccw_2  Less_rotate_ccw_2;

  if (first == beyond) return result;
  std::vector< Point_2 >  V;
  copy( first, beyond, back_inserter(V) );
  typename std::vector< Point_2 >::iterator it = 
               std::min_element(V.begin(), V.end(), Less_xy_2());
  std::sort( V.begin(), V.end(), Less_rotate_ccw_2(*it) );
  if ( *(V.begin()) == *(V.rbegin()) )
  {
      *result = *(V.begin());  ++result;
      return result;
  }
  return ch_graham_andrew_scan( V.begin(), V.end(), result, ch_traits);
}
\end{verbatim}


\ccIndexSubitemEnd{convex hull, 2D}{Graham-Andrew scan}
\ccIndexSubitemEnd{extreme points, 2D}{right of line}
\end{ccRefFunction}

% +------------------------------------------------------------------------+
%%RefPage: end of main body, begin of footer
% EOF
% +------------------------------------------------------------------------+

