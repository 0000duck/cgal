% =============================================================================
% The CGAL Reference Manual
% Chapter: Geometric Object Generators
% Section: Random Numbers Generator
% -----------------------------------------------------------------------------
% file  : doc_tex/support/Generator/Random.tex
% author: Sven Sch�nherr <sven@inf.ethz.ch>
% -----------------------------------------------------------------------------
% $CGAL_Chapter: Geometric Object Generators $
% $CGAL_Package: Random_numbers WIP $
% $Revision$
% $Date$
% =============================================================================

\begin{ccClass}{Random}
\ccSection{Random Numbers Generator}
\label{sec:random_numbers_generator}

% -----------------------------------------------------------------------------
\ccDefinition

An instance of type \ccClassName\ is a random numbers generator. It
allows to generate uniformly distributed random \ccc{bool}s,
\ccc{int}s and \ccc{double}s. It can be used as the random number
generating function object in the STL algorithm \ccc{random_shuffle}.

It can be very useful, e.g.\ for debugging, to reproduce a sequence of
random numbers.  This can be done by initialising deterministically.

\ccInclude{CGAL/Random.h}

% -----------------------------------------------------------------------------
\ccHeading{Global Variables}

\ccVariable{ Random default_random;}{
          This global variable is used as the
          default random numbers generator.}

% -----------------------------------------------------------------------------
\ccCreation
\ccCreationVariable{random}

\ccConstructor{ Random( );}{
        introduces a variable \ccVar\ of type \ccClassTemplateName.}

\ccConstructor{ Random( unsigned int  seed);}{
        introduces a variable \ccVar\ of type \ccClassTemplateName\
        and initializes the seed using \ccc{seed}.}

% -----------------------------------------------------------------------------
\ccOperations

\ccMemberFunction{ bool get_bool( );}{
        returns a random \ccc{bool}.}

\ccMemberFunction{ int get_int( int lower, int upper);}{
        returns a random \ccc{int} from the interval
        $[\mbox{\ccc{lower},\ccc{upper}})$.}

\ccMemberFunction{ double get_double( double lower = 0.0,
                                      double upper = 1.0);}{
        returns a random \ccc{double} from the interval
        $[\mbox{\ccc{lower},\ccc{upper}})$.}

\ccMemberFunction{ int operator() ( int upper);}{
        returns \ccVar\ccc{.get_int( 0, upper)}.}

% -----------------------------------------------------------------------------
\ccImplementation

We use the function \ccc{rand} from the standard C library to generate
the random numbers, i.e.\ the sequence of numbers depends on the
implementation of \ccc{rand} on your specific platform.

\end{ccClass}

% ===== EOF ===================================================================
