\begin{ccRefConcept}{DelaunayTraits_d}

\ccDefinition Requirements of the first traits class to be used with the
class \ccc{Delaunay_d}.

\ccTypes
\ccAutoIndexingOff
\ccNestedType{Point_d}{the dD point type on which the delaunay 
algorithm operates}
\ccGlue
\ccNestedType{Sphere_d}{a dD sphere}
\ccGlue
\ccNestedType{FT}{an arithmetic field type}

\ccNestedType{Point_of_sphere_d}{Predicate object type that provides
\ccc{Point_d operator()(Sphere_d s, int i)}, which returns the $i$th
point defining sphere \ccc{s}.}

\ccNestedType{Construct_sphere_d}{Predicate object type that provides
\ccc{Sphere_d operator()(int d, ForwardIterator first, ForwardIterator
last)}, which returns a dD sphere through the points in
\ccc{tuple[first,last)}.}

\ccNestedType{Contained_in_simplex_d}{Predicate object type that
provides \ccc{bool operator()(ForwardIterator first, ForwardIterator
last, Point_d p)}, which determines if \ccc{p} is contained in 
the closed simplex defined by the points in \ccc{tuple[first,last)}.}

\ccNestedType{Squared_distance_d}{Predicate object type that provides
\ccc{FT operator()(Point_d p,Point_d q)}, which determines the
squared distance from \ccc{p} to \ccc{q}.}

\ccNestedType{Affinely_independent_d}{Predicate object type that
provides \ccc{bool operator()(ForwardIterator first, ForwardIterator
last)}, which determines if the points in \ccc{tuple[first,last)} are
affinely independent.}

\ccCreation
\ccCreationVariable{traits}

A default constructor and copy constructor is required.

\ccOperations

For each of the above function and predicate object types,
\ccc{Func_obj_type}, a function must exist with the name
\ccc{func_obj_type_object} that creates an instance of the function or
predicate object type.  For example:

\ccMethod{Construct_sphere_d construct_sphere_d_object();}{}

\ccAutoIndexingOn

\ccHasModels
% cross references between manuals not currently possible
%\ccRefIdfierPage{CGAL::Cartesian_d<FT,LA>}\\
%\ccRefIdfierPage{CGAL::Homogeneous_d<RT,LA>}\\
\ccc{CGAL::Cartesian_d<FT,LA>}\\
\ccc{CGAL::Homogeneous_d<RT,LA>}\\

%\ccSeeAlso

\end{ccRefConcept}
