% file doc_tex/basic/ConvexHull/convex_hull_d.tex
\section{dD Convex Hull}
\begin{ccPackage}{convex hull, dD}
  
The class \ccc{CGAL::Convex_hull_d<R>} is used to represent the
convex hull of a set of points in $d$-dimensional space.  This class
supports incremental construction of hulls, and provides a rich
interface for exploration. There are also output routines for hulls
of dimension 2 and 3.

The convex hull class is parameterized by a traits class that provides 
$d$-dimensional data types and predicates.  The class 
\ccc{Convex_hull_d_traits_3} adapts any low-dimensional standard kernel 
model \textit{e.g.}, \ccc{Homogeneous<RT>} or \ccc{Cartesian<FT>} for use 
with \ccc{Convex_hull_d}, where the dimension is fixed to three.
The validity of the computed convex hull can be checked using the
member funciton \ccc{is_valid}, which implements the algorithm
of Mehlhorn \textit{et al.}\cite{mnssssu-cgpvg-96} to determine if
the vertices of a given polytope constitute a strongly convex point
set or not.
  
The implementation follows the papers \cite{cms:fourresults-93} and
\cite{BMS:degeneracy-94}.  

\end{ccPackage}

