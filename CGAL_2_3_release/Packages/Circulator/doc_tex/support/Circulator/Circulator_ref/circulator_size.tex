% +------------------------------------------------------------------------+
% | CGAL Reference Manual: circulator_size
% +------------------------------------------------------------------------+
% | Requirements for circulators in analogy to STL iterators.
% | Adaptors between circulators and iterators.
% | Proposal for CGAL.
% |
% | 11.10.1996   Lutz Kettner
% | 
% | \RCSdef{\circRev}{$Revision$}
% | \RCSdefDate{\circDate}{$Date$}
% +------------------------------------------------------------------------+

% +-----------------------------------------------------+
\begin{ccRefFunction}{circulator_size}

\ccDefinition

\ccSetThreeColumns{template <class C> sieze_t}{XXX}{}

The size of a circulator is the size of the data structure it refers
to. It is zero for a circulator on an empty sequence. The size can be
computed in linear time for forward and bidirectional circulators, and
in constant time for random access circulators using the minimal
circulator. The function \ccStyle{circulator_size(c)}
returns the circulator size. It uses the
\ccStyle{c.min_circulator()} function if $c$ is a random
access circulator.

\ccInclude{CGAL/circulator.h}

\ccGlobalFunction{template <class C> C::size_type circulator_size(C c);}

\ccSeeAlso

\ccc{circulator_distance},
\ccc{iterator_distance},
\ccc{is_empty_range},
\ccc{Circulator}.

\end{ccRefFunction}

% +-----------------------------------------------------+
% EOF
