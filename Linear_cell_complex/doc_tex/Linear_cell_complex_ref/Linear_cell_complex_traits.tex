% +------------------------------------------------------------------------+
% | Reference manual page: LinearCellComplexTraits.tex
% +------------------------------------------------------------------------+
% | 04.02.2010   Guillaume Damiand
% | Package: Combinatorial_map
% +------------------------------------------------------------------------+
\ccRefPageBegin
%%RefPage: end of header, begin of main body
% +------------------------------------------------------------------------+

\begin{ccRefClass}{Linear_cell_complex_traits<d,K>}

\ccInclude{CGAL/Linear_cell_complex_traits.h}

\ccDefinition

This geometric traits concept is used in the
\ccc{Linear_cell_complex} class.  It can take as parameter any model of the
concept \ccc{Kernel} (for example any \cgal\ kernel), and define inner
types and functors corresponding to the given dimension.

\ccIsModel
\ccRefConceptPage{LinearCellComplexTraits}

\ccInheritsFrom
\ccc{K}.

\ccParameters
\ccc{d} the dimension of the kernel\\
\ccc{K} a model of the concept \ccc{Kernel} if \ccc{d==2} or 
 \ccc{d==3}; a model of the concept \ccc{Kernel_d} otherwise. 

\ccConstants
\ccVariable{static unsigned int ambient_dimension = d;}{}

% \ccTypes
% \ccTypedef{typedef K Kernel;}{}

\ccSeeAlso

%\ccRefConceptPage{LinearCellComplex}\\
\ccRefIdfierPage{CGAL::Linear_cell_complex<d,d2,Traits_,Items_,Alloc_>}\\
\ccRefConceptPage{LinearCellComplexItems}

\end{ccRefClass}
% +------------------------------------------------------------------------+
%%RefPage: end of main body, begin of footer
\ccRefPageEnd
% EOF
% +------------------------------------------------------------------------+

%for example \ccc{CGAL::Cartesian<double>} or \ccc{CGAL::Simple_cartesian<CGAL::Gmpq>}.

% \ccRefines
% \ccc{CopyConstructable}, \ccc{Assignable}.

% ... Question is all these typedef required ?


% \ccTypes

% % \ccNestedType{Kernel}{kernel type.}

% \ccTypedef{Kernel::FT FT;}{Number type.}

% \subsection{If \ccc{Dimension==2}}

% \ccTypes

% \ccTypedef{Kernel::Point_2 Point;}{point type.}
% \ccGlue
% \ccTypedef{Kernel::Vector_2 Vector;}{vector type.}
% % \ccGlue
% % \ccTypedef{Kernel::Iso_rectangle_2 Iso_rectangle}{iso rectangle type.}

% \ccHeading{Constructions}

% \ccTypedef{Kernel::Construct_translated_point_2 Construct_translated_point;}{}
% \ccGlue
% \ccTypedef{Kernel::Construct_vector_2 Construct_vector;}{}
% \ccGlue
% \ccTypedef{Kernel::Construct_sum_of_vectors_2 Construct_sum_of_vectors;}{}
% \ccGlue
% \ccTypedef{Kernel::Construct_scaled_vector_2 Construct_scaled_vector;}{}
% \ccGlue
% \ccTypedef{Kernel::Construct_midpoint_2 Construct_midpoint;}{}
% \ccGlue
% \ccTypedef{Kernel::Construct_direction_2 Construct_direction;}{}

% ...

% \subsection{If \ccc{Dimension==3}}

% \ccTypes

% \ccTypedef{Kernel::Point_3 Point;}{point type.}
% \ccGlue
% \ccTypedef{Kernel::Vector_3 Vector;}{vector type.}
% % \ccGlue
% % \ccTypedef{Kernel::Iso_cuboid_3 }{iso cuboid type.}

% \ccHeading{Constructions}

% \ccTypedef{Kernel::Construct_translated_point_3 Construct_translated_point;}{}
% \ccGlue
% \ccTypedef{Kernel::Construct_vector_3 Construct_vector;}{}
% \ccGlue
% \ccTypedef{Kernel::Construct_sum_of_vectors_3 Construct_sum_of_vectors;}{}
% \ccGlue
% \ccTypedef{Kernel::Construct_scaled_vector_3 Construct_scaled_vector;}{}
% \ccGlue
% \ccTypedef{Kernel::Construct_midpoint_3 Construct_midpoint;}{}
% \ccGlue
% \ccTypedef{Kernel::Construct_direction_3 Construct_direction;}{}

% ...

% \subsection{If \ccc{Dimension>3}}

% \ccTypes

% \ccTypedef{Kernel::Point_d;}{point type.}
% \ccGlue
% \ccTypedef{Kernel::Vector_d;}{vector type.}

% \ccHeading{Constructions}

% \ccTypedef{Kernel::Construct_vector_d;}{a model of \ccc{Kernel::ConstructVector_d}}
% \ccGlue
% \ccTypedef{Kernel::Construct_midpoint_d;}{a model of \ccc{Kernel::ConstructMidpoint_d}}
% \ccGlue
% \ccTypedef{Kernel::Point_to_vector_d;}{a model of \ccc{Kernel::Point_to_vector_d}}

% \ccHeading{Generalized Predicates}

% \ccTypedef{Kernel::Compare_lexicographically_d;}{a model of \ccc{Kernel::Compare_lexicographically_d}}

% \ccHeading{Operators}

% Because there is no construction for these operations.

% \ccTypedef{Vector_d(int,Base_vector,FT);}{}
% \ccGlue
% \ccTypedef{operator+(Point_d,Point_d);}{}
% \ccGlue
% \ccTypedef{operator+(Point_d,Vector_d);}{}
% \ccGlue
% \ccTypedef{operator+(Vector_d,Vector_d);}{}
% \ccGlue
% \ccTypedef{operator*(Vector_d,FT);}{}


