% +------------------------------------------------------------------------+
% | Reference manual page: Linear_cell_complex_min_items.tex
% +------------------------------------------------------------------------+
% | 04.02.2010   Guillaume Damiand
% | Package: Linear_cell_complex
% +------------------------------------------------------------------------+
\ccRefPageBegin
%%RefPage: end of header, begin of main body
% +------------------------------------------------------------------------+

\begin{ccRefClass}{Linear_cell_complex_min_items<d>}

\ccInclude{CGAL/Linear_cell_complex_min_items.h}

\ccDefinition
  
The class \ccRefName\ defines the type of darts, which is a
\ccc{Dart_wrapper::Dart<d,LCC>}, and the traits class used.  In
this class, 0-attributes are enabled and associated with
\ccc{Cell_attribute_with_point}.

\ccIsModel
\ccRefConceptPage{LinearCellComplexItems}

\ccParameters
\ccc{d} the dimension of the combinatorial map.

\ccExample

The following example shows one implementation of the
\ccRefName\ class.

\begin{ccExampleCode}
  template <unsigned int d>
  struct Linear_cell_complex_min_items
  {
    template <class LCC>
    struct Dart_wrapper
    {
      typedef CGAL::Dart<d, LCC> Dart;

      typedef CGAL::Cell_attribute_with_point<LCC> Vertex_attrib;    
      typedef CGAL::cpp0x::tuple<Vertex_attrib> Attributes;
    };
  };
\end{ccExampleCode}

\ccSeeAlso
\ccRefIdfierPage{CGAL::Linear_cell_complex<d,d2,LCCTraits,CMItems,Alloc>}\\
\ccRefIdfierPage{CGAL::Dart<d,CMap>}

\end{ccRefClass}

% +------------------------------------------------------------------------+
%%RefPage: end of main body, begin of footer
\ccRefPageEnd
% EOF
% +------------------------------------------------------------------------+

