% +------------------------------------------------------------------------+
% | Reference manual page: Linear_cell_complex_operations.tex
% +------------------------------------------------------------------------+
% | 04.02.2010   Guillaume Damiand
% | Package: Combinatorial_map
% +------------------------------------------------------------------------+
\ccRefPageBegin
%%RefPage: end of header, begin of main body
% +------------------------------------------------------------------------+

% \begin{ccRefFunction}{barycenter<LCC,i>}
% \ccInclude{Linear_cell_complex_operations.h}\\
% \ccFunction{template<class LCC, unsigned int i>
%   typename LCC::Point barycenter(const LCC& lcc, 
%   typename LCC::Dart_const_handle dh);}
% {Returns the barycenter of the \emph{i}-cell containing \ccc{dh}.
%   \ccPrecond{0\myleq{}\emph{i}\myleq{}\ccc{LCC::dimension} and \ccc{*dh}\myin{}\ccc{lcc.darts()}.}
% }

%   for example $i=2$ for facet, or $i=3$ for volume).\\
% \ccCommentHeading{Template parameter}\\
% \ccc{LCC} must be a model of the \ccc{CombinatorialLCCWithPoints} concept.
% \ccCommentHeading{Parameters} \\
% \ccc{lcc}: the combinatorial map used;\\
% \ccc{adart}: a dart belonging to the cell;\\
% \ccCommentHeading{Returns} \\
%    the barycenter of the cell.
% }
% \ccSeeAlso
% \ccRefIdfierPage{CGAL::compute_normal_of_cell_0<LCC>}\\
% \ccRefIdfierPage{CGAL::compute_normal_of_cell_2<LCC>}\\
% \ccRefIdfierPage{CGAL::insert_center_cell_0_in_cell_2<LCC>}\\
% \end{ccRefFunction}
%--------------------------------------------------------------------------------
\begin{ccRefFunction}{compute_normal_of_cell_0<LCC>}
\ccInclude{Linear_cell_complex_operations.h}\\
\ccFunction{template <class LCC>
typename LCC::Vector compute_normal_of_cell_0(const LCC& lcc, 
typename LCC::Dart_const_handle dh);}
{Returns the normal vector of the 0-cell containing \ccc{dh}.
  \ccPrecond{\ccc{LCC::ambient_dimension}==3 and \ccc{*dh}\myin{}\ccc{lcc.darts()}.}
}

\ccSeeAlso
%\ccRefIdfierPage{CGAL::barycenter<LCC,i>}\\
\ccRefIdfierPage{CGAL::compute_normal_of_cell_2<LCC>}\\
\end{ccRefFunction}
%--------------------------------------------------------------------------------
\begin{ccRefFunction}{compute_normal_of_cell_2<LCC>}
\ccInclude{Linear_cell_complex_operations.h}\\
\ccFunction{template <class LCC>
typename LCC::Vector compute_normal_of_cell_2(const LCC& lcc, 
typename LCC::Dart_const_handle dh);}
{Returns the normal vector of the 2-cell containing \ccc{dh}.
  \ccPrecond{\ccc{LCC::ambient_dimension}==3 and \ccc{*dh}\myin{}\ccc{lcc.darts()}.}
}

\ccSeeAlso
%\ccRefIdfierPage{CGAL::barycenter<LCC,i>}\\
\ccRefIdfierPage{CGAL::compute_normal_of_cell_0<LCC>}\\
\end{ccRefFunction}
%--------------------------------------------------------------------------------
% \begin{ccRefFunction}{insert_barycenter_in_cell<LCC,i>}
% \ccInclude{Combinatorial_map_operations.h}\\

% \ccFunction{template <class LCC, unsigned int i>
%   typename LCC::Dart_handle insert_barycenter_in_cell(LCC& lcc,
%                                         typename LCC::Dart_handle dh);}
% {Inserts a point in the barycenter of the \emph{i}-cell containing \ccc{dh}.
%   Returns an handle on one dart of this cell.  
%   \ccPrecond{\ccc{LCC::dimension}\mygeq{}1 and \ccc{*dh}\myin{}\ccc{lcc.darts()}.}\\
% %  \begin{ccAdvanced}
%     If \emph{i}-attributes are non void, 
%     \ccc{Attribute_type<i>::type::On_split}(\emph{a},\emph{a'}) is called,
%     with \emph{a} the original \emph{i}-attribute associated
%     with \emph{dh} and \emph{a'} each new \emph{i}-attribute created during the operation.
% %  \end{ccAdvanced}
% }

% \ccSeeAlso
% \ccRefIdfierPage{CGAL::insert_cell_0_in_cell_1<LCC>}\\
% \ccRefIdfierPage{CGAL::insert_cell_0_in_cell_2<LCC>}\\
% \ccRefIdfierPage{CGAL::insert_barycenter_in_cell<LCC,i>}\\
% \ccRefIdfierPage{CGAL::insert_dangling_cell_1_in_cell_2<LCC>}\\
% \end{ccRefFunction}
%--------------------------------------------------------------------------------
% \begin{ccRefFunction}{insert_point_in_cell<LCC,i>}
% \ccInclude{Combinatorial_map_operations.h}\\

% \ccFunction{template <class LCC, unsigned int i>
%   typename LCC::Dart_handle insert_point_in_cell(LCC& lcc,
%                                         typename LCC::Dart_handle dh,
%                                         typename LCC::Point p);}
% {Inserts a point, copy of \ccc{p}, in the \emph{i}-cell containing \ccc{dh}.
%   Returns an handle on one dart of this cell.  
%   \ccPrecond{\ccc{LCC::dimension}\mygeq{}1 and \ccc{*dh}\myin{}\ccc{lcc.darts()}.}\\
% %  \begin{ccAdvanced}
%     If \emph{i}-attributes are non void, 
%     \ccc{Attribute_type<i>::type::On_split}(\emph{a},\emph{a'}) is called, 
%     with $a$ the original \emph{i}-attribute associated
%     with $dh$ and $a'$ each new \emph{i}-attribute created during the operation.
% %  \end{ccAdvanced}
% }

% \ccSeeAlso
% \ccRefIdfierPage{CGAL::insert_barycenter_in_cell<LCC,i>}\\
% \ccRefIdfierPage{CGAL::insert_dangling_cell_1_in_cell_2<LCC>}\\
% \end{ccRefFunction}
%--------------------------------------------------------------------------------
% \begin{ccRefFunction}{insert_cell_0_in_cell_2<LCC>}
% \ccInclude{Linear_cell_complex_operations.h}\\
% \ccFunction{template <class LCC>
%       typename LCC::Dart_handle insert_cell_0_in_cell_2(LCC & lcc,
%       typename LCC::Dart_handle dh,
%       typename LCC::Point p);}
% {Inserts a 0-cell in the 2-cell containing \ccc{dh}, associated with
%   a 0-attribute having \ccc{p} as point.
% The 2-cell is splitted in triangles, one for each initial edge of the facet.
% Returns an handle on one dart belonging to the new 0-cell.
% \ccPrecond{\ccc{LCC::dimension}\mygeq{}2 and \ccc{*dh}\myin{}\ccc{lcc.darts()}.}\\
% %  \begin{ccAdvanced}
%     If 2-attributes are non void, 
%     \ccc{Attribute_type<2>::type::On_split}(\emph{a},\emph{a'}) is called, 
%     with \emph{a} the original 2-attribute associated
%     with \emph{dh} and \emph{a'} each new 2-attribute created during the operation.
% %  \end{ccAdvanced}
% }

% \ccSeeAlso
% \ccRefIdfierPage{CGAL::insert_middle_cell_0_in_cell_1<LCC>}\\
% \ccRefIdfierPage{CGAL::insert_cell_0_in_cell_1<LCC>}\\
% \ccRefIdfierPage{CGAL::insert_center_cell_0_in_cell_2<LCC>}\\
% \ccRefIdfierPage{CGAL::insert_dangling_cell_1_in_cell_2<LCC>}\\
% \end{ccRefFunction}
%--------------------------------------------------------------------------------
% \begin{ccRefFunction}{insert_center_cell_0_in_cell_2<LCC>}
% \ccInclude{Linear_cell_complex_operations.h}\\
% \ccFunction{template <class LCC>
%       typename LCC::Dart_handle insert_center_cell_0_in_cell_2(LCC & lcc,
%       typename LCC::Dart_handle dh);}
% {Inserts a 0-cell in the barycenter of the 2-cell containing \ccc{dh}.
% The 2-cell is splitted in triangles, one for each initial edge of the facet.
% Returns an handle on one dart belonging to the new 0-cell.
% \ccPrecond{\ccc{LCC::dimension}\mygeq{}2 and \ccc{*dh}\myin{}\ccc{lcc.darts()}.}\\
% %  \begin{ccAdvanced}
%     If 2-attributes are non void,
%     \ccc{Attribute_type<2>::type::On_split}(\emph{a},\emph{a'}) is called, 
%     with \emph{a} the original 2-attribute associated
%     with \emph{dh} and \emph{a'} each new 2-attribute created during the operation.
% %  \end{ccAdvanced}
% }

% \ccSeeAlso
% \ccRefIdfierPage{CGAL::barycenter<LCC,i>}\\
% \ccRefIdfierPage{CGAL::insert_middle_cell_0_in_cell_1<LCC>}\\
% \ccRefIdfierPage{CGAL::insert_cell_0_in_cell_1<LCC>}\\
% \ccRefIdfierPage{CGAL::insert_cell_0_in_cell_2<LCC>}\\
% \ccRefIdfierPage{CGAL::insert_dangling_cell_1_in_cell_2<LCC>}\\
% \end{ccRefFunction}
%--------------------------------------------------------------------------------
% \begin{ccRefFunction}{insert_dangling_cell_1_in_cell_2<LCC>}
% \ccInclude{Combinatorial_map_operations.h}\\

% \ccFunction{template <class LCC>
%   typename LCC::Dart_handle insert_dangling_cell_1_in_cell_2(LCC& lcc,
%                                          typename LCC::Dart_handle dh,
%                                          typename LCC::Point p);}
% {Inserts a 1-cell in a the 2-cell containing \ccc{adart}, the 1-cell
%   being attached only by one of its vertex to the 0-cell containing \ccc{dh}.
%   The second vertex is associated with a new 0-attribute containing a copy of
%   \ccc{p} as point. Returns an handle on one dart belonging to the new 0-cell.
%   \ccPrecond{\ccc{LCC::dimension}\mygeq{}2 and \ccc{*dh}\myin{}\ccc{lcc.darts()}.}
% }
% \ccSeeAlso
% \ccRefIdfierPage{CGAL::insert_middle_cell_0_in_cell_1<LCC>}\\
% \ccRefIdfierPage{CGAL::insert_cell_0_in_cell_1<LCC>}\\
% \ccRefIdfierPage{CGAL::insert_cell_0_in_cell_2<LCC>}\\
% \ccRefIdfierPage{CGAL::insert_center_cell_0_in_cell_2<LCC>}\\
% \end{ccRefFunction}
%--------------------------------------------------------------------------------

% +------------------------------------------------------------------------+
%%RefPage: end of main body, begin of footer
\ccRefPageEnd
% EOF
% +------------------------------------------------------------------------+
