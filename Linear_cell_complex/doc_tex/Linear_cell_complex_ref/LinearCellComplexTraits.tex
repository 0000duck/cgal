% +------------------------------------------------------------------------+
% | Reference manual page: LinearCellComplexTraits.tex
% +------------------------------------------------------------------------+
% | 04.02.2010   Guillaume Damiand
% | Package: Linear_cell_complex
% +------------------------------------------------------------------------+
\ccRefPageBegin
%%RefPage: end of header, begin of main body
% +------------------------------------------------------------------------+

\begin{ccRefConcept}{LinearCellComplexTraits}

Required types and functors for the \ccRefName\ concept. This
geometric traits concept is used in the \ccc{Linear_cell_complex}
class.  

\ccConstants
\ccVariable{static unsigned int ambient_dimension;}
{The ambient dimension, must be \mygt{}1.}

\ccTypes

\ccNestedType{FT}{a number type that is a model of FieldNumberType.}
\ccGlue
\ccNestedType{Point}{point type.}
\ccGlue
\ccNestedType{Vector}{vector type.}

\ccHeading{Constructions}

\ccNestedType{Construct_translated_point}
{Functor that provides \ccc{Point operator() (const Point& p, const Vector& v)}, 
 which constructs the translation of point \ccc{p} by vector \ccc{v}, and
\ccc{Point operator() (const CGAL::Origin&, const Vector& v)}, 
 which constructs the translation of a point at the origin by vector \ccc{v}
(used in \ccc{Linear_cell_complex::barycenter}).}
\ccGlue
\ccNestedType{Construct_vector}
{Functor that provides \ccc{Vector operator() (const Point& p1, const Point& p2)} 
 which constructs a vector as the difference of points \ccc{p2-p1}, and
 \ccc{Vector operator() (const CGAL::Origin&, const Point& p)} 
 which constructs a vector as the difference of point \ccc{p} and a point at the origin
(used in \ccc{Linear_cell_complex::barycenter} and \ccc{CGAL::import_from_plane_graph}).}
\ccGlue
\ccNestedType{Construct_sum_of_vectors}
{Functor that provides \ccc{Vector operator() (const Vector& v1, const Vector& v2)} 
 which constructs a vector as the sum of vectors \ccc{v1+v2}
(used in \ccc{Linear_cell_complex::barycenter}, \ccc{CGAL::compute_normal_of_cell_0} 
 and \ccc{CGAL::compute_normal_of_cell_2}).}
\ccGlue
\ccNestedType{Construct_scaled_vector}
{Functor that provides \ccc{Vector operator() (const Vector& v, FT scale)}
  which constructs a vector equal to vector \ccc{v} scaled by \ccc{scale} factor
(used in \ccc{Linear_cell_complex::barycenter} , \ccc{CGAL::compute_normal_of_cell_0}
 and \ccc{CGAL::compute_normal_of_cell_2}).}
\ccGlue
\ccNestedType{Construct_midpoint}
{Functor that provides \ccc{Point operator() (const Point& p1, const Point& p2)}
  which constructs the midpoint of points \ccc{p1} and \ccc{p2}
(used in \ccc{Linear_cell_complex::barycenter}).}

\textbf{If \ccc{ambient_dimension==2}}\\
\ccNestedType{Direction_2}{a model of \ccc{Direction_2}.}
\ccGlue
\ccNestedType{Construct_direction_2}
{a model of \ccc{ConstructDirection_2} (used in \ccc{CGAL::import_from_plane_graph}).}

\textbf{If \ccc{ambient_dimension==3}}\\
\ccNestedType{Construct_normal_3}
{a model of \ccc{ConstructNormal_3} (used in \ccc{CGAL::compute_normal_of_cell_2}).}
\ccNestedType{Collinear_3}
{a model of \ccc{Collinear_3} (used in \ccc{CGAL::compute_normal_of_cell_2}).}

\ccHasModels

\ccRefIdfierPage{CGAL::Linear_cell_complex_traits<d,K>}.

\ccSeeAlso
\ccRefIdfierPage{CGAL::Linear_cell_complex<d,d2,LCCTraits,CMItems,Alloc>}

\end{ccRefConcept}
% +------------------------------------------------------------------------+
%%RefPage: end of main body, begin of footer
\ccRefPageEnd
% EOF
% +------------------------------------------------------------------------+

