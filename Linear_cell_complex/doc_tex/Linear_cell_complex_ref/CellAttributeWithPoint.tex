% +------------------------------------------------------------------------+
% | Reference manual page: CellAttributeWithPoint.tex
% +------------------------------------------------------------------------+
% | 04.02.2010   Guillaume Damiand
% | Package: Linear_cell_complex
% +------------------------------------------------------------------------+
\ccRefPageBegin
%%RefPage: end of header, begin of main body
% +------------------------------------------------------------------------+
\begin{ccRefConcept}{CellAttributeWithPoint}

\ccDefinition
  
The concept \ccRefName\ is a refinement of the \ccc{CellAttribute}
concept, to represent a cell attribute containing a point.

\ccRefines
\ccRefConceptPage{CellAttribute}  % \\

\ccTypes
%\ccParameters
% \ccc{Refs} must be a model of the \ccc{CombinatorialMap} concept.
% \ccc{T} must be \ccc{Tag_true} to enable the storage of a
% \ccc{Dart_handle} within the class (to be set to a dart which is part of the cell),
% and \ccc{Tag_false} otherwise.
% \ccNestedType{Traits}{The traits class, a model of the \ccc{LinearCellComplexTraits} concept.}
% \ccGlue
\ccNestedType{Point}{Type of the used point.} % Equals to \ccc{Traits::Point}.}

% A model of
%  \ccc{Kernel::Point_2} if \ccc{ambient_dimension==2},
%  a model of \ccc{Kernel::Point_3} if \ccc{ambient_dimension==3},
%  or a model of \ccc{Kernel::Point_d} otherwise.}

% \ccc{FunctorOnMerge} functor used when two cell attributes are merged. Must contains a method \ccc{operator ()} taking two \ccc{CellAttribute} as parameters.
% \ccc{FunctorOnSplit} functor used when one cell attribute was split in two. Must contains a method \ccc{operator ()} taking two \ccc{CellAttribute} as parameters.

% This concept does not have any restriction on the number
% of additional template parameters.

% \ccTypes

% \ccNestedType{Supports_cell_dart}
%     {equal to T (\ccc{Tag_true} or \ccc{Tag_false}).}
% +-----------------------------------+
% \ccConstants
% \ccVariable{static unsigned int ambient_dimension;}{The dimension of the ambient space.}

% +-----------------------------------+
\ccCreation
\ccCreationVariable{cawp}

\ccConstructor{CellAttributeWithPoint();}{Default constructor.}

\ccConstructor{CellAttributeWithPoint(const Point&apoint);}
   {Constructor initializing the point of \ccc{cawp} by the 
    copy contructor \ccc{Point(apoint)}.}

\ccConstructor{CellAttributeWithPoint(const Point&apoint, const Info& info);}
   {Constructor initializing the point of \ccc{cawp} by the 
    copy contructor \ccc{Point(apoint)} and initializing the
    information of \ccc{cawp} by the 
    copy contructor \ccc{Info(info)}.
    Defined only if \ccc{Info} is different from \ccc{void}.}

% +-----------------------------------+
\ccHeading{Access Member Functions}

\ccMethod{Point& point();} 
     {Returns the point of \ccc{cawp}.}

\ccMethod{const Point& point() const;} 
     {Returns the point of \ccc{cawp}, when \ccc{cawp} is const.}

\ccHasModels
\ccRefIdfierPage{CGAL::Cell_attribute_with_point<LCC,Info_,Tag,OnMerge,OnSplit>}

\ccSeeAlso
%\ccRefConceptPage{LinearCellComplex}\\
\ccRefConceptPage{LinearCellComplexItems}
%\ccRefConceptPage{LinearCellComplexTraits}\\

\end{ccRefConcept}
% +------------------------------------------------------------------------+
%%RefPage: end of main body, begin of footer
\ccRefPageEnd
% EOF
% +------------------------------------------------------------------------+
