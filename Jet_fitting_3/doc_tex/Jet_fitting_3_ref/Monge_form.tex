% +------------------------------------------------------------------------+
% | Reference manual page: Monge_form.tex
% +------------------------------------------------------------------------+
% | 09.02.2006   Marc Pouget and Fr�d�ric Cazals
% | Package: Jet_fitting_3
% | 
\RCSdef{\RCSMongeformRev}{$Id$}
\RCSdefDate{\RCSMongeformDate}{$Date$}
% |
%%RefPage: end of header, begin of main body
% +------------------------------------------------------------------------+


\begin{ccRefClass}[Monge_via_jet_fitting< DataKernel, LocalKernel, SvdTraits>::]{Monge_form}  %% add template arg's if necessary

%% \ccHtmlCrossLink{}     %% add further rules for cross referencing links
%% \ccHtmlIndexC[class]{} %% add further index entries

\ccDefinition
  
The class \ccRefName\ stores the Monge representation, i.e. the Monge
coordinate system and the coefficients of the Monge form in this
system.

\ccInclude{CGAL/Monge_via_jet_fitting.h}

\ccTypes
% +--------------------------------------------------------------
%\ccNestedType{TYPE}{some nested types}
\ccTypedef{  typedef typename DataKernel::FT        FT; }{}
\ccGlue
\ccTypedef{  typedef typename DataKernel::Point_3   Point_3; }{}
\ccGlue
\ccTypedef{   typedef typename DataKernel::Vector_3  Vector_3;}{}
\ccGlue



\ccCreation
% +--------------------------------------------------------------
\ccCreationVariable{monge_form}  %% choose variable name

\ccConstructor{Monge_form();}{default constructor.}

\ccAccessFunctions
% +--------------------------------------------------------------

\ccMemberFunction{Point_3 origin();}{Point on the fitted surface where
differential quantities are computed.}

The Monge basis is given by :

\ccMemberFunction{Vector_3 maximal_principal_direction();}{}
\ccGlue
\ccMemberFunction{Vector_3 minimal_principal_direction(); }{}
\ccGlue
\ccMemberFunction{Vector_3 normal_direction(); }{}

The Monge coefficients are given by :

\ccMemberFunction{FT principal_curvatures(size_t i);}
{$i=0$ for the maximum and $i=1$ for the minimum.}
\ccGlue
\ccMemberFunction{FT third_order_coefficients(size_t i); }{$0 \leq i \leq 3$}
\ccGlue
\ccMemberFunction{FT fourth_order_coefficients(size_t i); }{$0 \leq i \leq 4$}

\ccOperations
% +--------------------------------------------------------------
%\ccMemberFunction{void set_up(int degree);}{Set the number of coefficients according 
%to the degree of the Monge representation given. The
%$(degree+1)(degree+2)/2-4$ coefficients are set to 0.}
\ccMemberFunction{void comply_wrt_given_normal(const Vector_3 given_normal);}
{ change principal basis and Monge coefficients so that the
given\_normal and the Monge normal make an acute angle.\\ If
given\_normal.monge\_normal $< 0$ then change the orientation~: if
$z=g(x,y)$ in the basis (d1,d2,n) then in the basis (d2,d1,-n)
$z=h(x,y)=-g(y,x)$. }

% \ccMemberFunction{void dump_verbose(std::ofstream& out_stream);}
% { Outputs the data in a human readable way, an alternative is to use the
% operator $<<$ defined for the class.}
% \ccMemberFunction{void dump_4ogl(std::ofstream& out_stream, const FT scale);}
% { Outputs the data for further visualization with successively~: the
% coordinates of m\_origin\_pt, the coordinates of m\_d1 scaled by scale,
% the coordinates of m\_d2 scaled by scale, the maximal principal
% curvature, the minimal principal curvature.}



The usual insert operator (\ccc{operator<<}) is overloaded for
\ccc{Monge_form}, it gives the Monge coordinate system (the origin and
an orthonormal basis) and the coefficients of the Monge form in this
system.

\ccSeeAlso

\ccc{Monge_via_jet_fitting}
%\ccc{some_other_function}.

%\ccExample


%\begin{ccExampleCode}
%void your_example_code() {
%}
%\end{ccExampleCode}

%%% \ccIncludeExampleCode{Jet_fitting_3/Monge_rep.cpp}

\end{ccRefClass}

% +------------------------------------------------------------------------+
%%RefPage: end of main body, begin of footer
% EOF
% +------------------------------------------------------------------------+

