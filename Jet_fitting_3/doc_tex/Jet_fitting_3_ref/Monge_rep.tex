% +------------------------------------------------------------------------+
% | Reference manual page: Monge_rep.tex
% +------------------------------------------------------------------------+
% | 09.02.2006   Marc Pouget and Fr�d�ric Cazals
% | Package: Jet_fitting_3
% | 
\RCSdef{\RCSMongerepRev}{$Revision$}
\RCSdefDate{\RCSMongerepDate}{$Date$}
% |
%%RefPage: end of header, begin of main body
% +------------------------------------------------------------------------+


\begin{ccRefClass}{Monge_rep<DataKernel>}  %% add template arg's if necessary

%% \ccHtmlCrossLink{}     %% add further rules for cross referencing links
%% \ccHtmlIndexC[class]{} %% add further index entries

\ccDefinition
  
The class \ccRefName\ stores the Monge representation. The
\ccc{DataKernel} template parameter must be the 
same for the classes \ccc{Monge_rep} and
\ccc{Monge_via_jet_fitting}.

\ccInclude{Monge_via_jet_fitting.h}

\ccTypes
% +--------------------------------------------------------------
%\ccNestedType{TYPE}{some nested types}
\ccTypedef{  typedef typename DataKernel::FT        DFT; }{}
\ccGlue
\ccTypedef{  typedef typename DataKernel::Point_3   DPoint; }{}
\ccGlue
\ccTypedef{   typedef typename DataKernel::Vector_3  DVector;}{}
\ccGlue

\ccHeading{Data members}
% +--------------------------------------------------------------
\ccVariable{  DPoint m_origin_pt;}{Point on the fitted surface where
differential quantities are computed} 

The basis \ccc{(m_d1, m_d2, m_n)} is direct orthonormal.

\ccVariable{DVector m_d1;}{maximal ppal dir} 
\ccVariable{DVector m_d2;}{minimal ppal dir} 
\ccVariable{DVector m_n;}{normal direction} 
\ccVariable{std::vector<DFT> m_coefficients;}{coeff = (k1, k2 (ppal curv),
          b0, b1, b2, b3 (third order), c0, c1, c2, c3, c4 (fourth
          order)). If the degree of the Monge representation is 1 there
          is no coefficient}

\ccCreation
% +--------------------------------------------------------------
\ccCreationVariable{monge_rep}  %% choose variable name

\ccConstructor{Monge_rep();}{default constructor.}

\ccAccessFunctions
% +--------------------------------------------------------------
\ccMemberFunction{const DPoint origin_pt();}{}
\ccMemberFunction{DPoint& origin_pt();}{}
\ccGlue
\ccMemberFunction{const DVector d1();}{}
\ccMemberFunction{DVector& d1(); }{}
\ccGlue
\ccMemberFunction{const DVector d2();}{}
\ccMemberFunction{DVector& d2();}{}
\ccGlue
\ccMemberFunction{const DVector n();}{}
\ccMemberFunction{DVector& n();}{}
\ccGlue
\ccMemberFunction{const std::vector<DFT> coefficients();}{}
\ccMemberFunction{std::vector<DFT>& coefficients();}{}

\ccOperations
% +--------------------------------------------------------------
\ccMemberFunction{void set_up(int degree);}{Set the number of coefficients according 
to the degree of the Monge representation given. The
$(degree+1)(degree+2)/2-4$ coefficients are set to 0.}

\ccSeeAlso

\ccc{Monge_via_jet_fitting},
%\ccc{some_other_function}.

%\ccExample

%A short example program.
%Instead of a short program fragment, a full running program can be
%included using the 
%\verb|\ccIncludeExampleCode{Jet_fitting_3/Monge_rep.C}| 
%macro. The program example would be part of the source code distribution and
%also part of the automatic test suite.

%\begin{ccExampleCode}
%void your_example_code() {
%}
%\end{ccExampleCode}

%%% \ccIncludeExampleCode{Jet_fitting_3/Monge_rep.C}

\end{ccRefClass}

% +------------------------------------------------------------------------+
%%RefPage: end of main body, begin of footer
% EOF
% +------------------------------------------------------------------------+

