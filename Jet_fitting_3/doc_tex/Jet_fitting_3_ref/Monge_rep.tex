% +------------------------------------------------------------------------+
% | Reference manual page: Monge_rep.tex
% +------------------------------------------------------------------------+
% | 09.02.2006   Marc Pouget and Fr�d�ric Cazals
% | Package: Jet_fitting_3
% | 
\RCSdef{\RCSMongerepRev}{$Id$}
\RCSdefDate{\RCSMongerepDate}{$Date$}
% |
%%RefPage: end of header, begin of main body
% +------------------------------------------------------------------------+


\begin{ccRefClass}{Monge_rep<DataKernel>}  %% add template arg's if necessary

%% \ccHtmlCrossLink{}     %% add further rules for cross referencing links
%% \ccHtmlIndexC[class]{} %% add further index entries

\ccDefinition
  
The class \ccRefName\ stores the Monge representation. The
\ccc{DataKernel} template parameter must be the 
same for the classes \ccc{Monge_rep} and
\ccc{Monge_via_jet_fitting}.

\ccInclude{CGAL/Monge_via_jet_fitting.h}

\ccTypes
% +--------------------------------------------------------------
%\ccNestedType{TYPE}{some nested types}
\ccTypedef{  typedef typename DataKernel::FT        DFT; }{}
\ccGlue
\ccTypedef{  typedef typename DataKernel::Point_3   DPoint; }{}
\ccGlue
\ccTypedef{   typedef typename DataKernel::Vector_3  DVector;}{}
\ccGlue



\ccCreation
% +--------------------------------------------------------------
\ccCreationVariable{monge_rep}  %% choose variable name

\ccConstructor{Monge_rep();}{default constructor.}

\ccAccessFunctions
% +--------------------------------------------------------------

\ccMemberFunction{const DPoint origin_pt();}{Point on the fitted surface where
differential quantities are computed}
\ccGlue
\ccMemberFunction{DPoint& origin_pt();}{}

The basis \ccc{(m_d1, m_d2, m_n)} is direct orthonormal.
\ccMemberFunction{const DVector d1();}{maximal principal direction}
\ccGlue
\ccMemberFunction{DVector& d1(); }{}
\ccMemberFunction{const DVector d2();}{minimal principal direction}
\ccGlue
\ccMemberFunction{DVector& d2();}{}
\ccMemberFunction{const DVector n();}{normal direction}
\ccGlue
\ccMemberFunction{DVector& n();}{}
\ccMemberFunction{const std::vector<DFT> coefficients();}{coeff = k1, k2 (principal curvatures);
          b0, b1, b2, b3 (third order coefficients); c0, c1, c2, c3, c4 (fourth
          order coefficients). If the degree of the Monge representation is 1 there
          is no coefficient.}
\ccGlue
\ccMemberFunction{std::vector<DFT>& coefficients();}{}

\ccOperations
% +--------------------------------------------------------------
\ccMemberFunction{void set_up(int degree);}{Set the number of coefficients according 
to the degree of the Monge representation given. The
$(degree+1)(degree+2)/2-4$ coefficients are set to 0.}
\ccMemberFunction{void comply_wrt_given_normal(const DVector given_normal);}
{ change principal basis and Monge coefficients so that the
given\_normal and the Monge normal make an acute angle.\\ If
given\_normal.monge\_normal $< 0$ then change the orientation~: if
$z=g(x,y)$ in the basis (d1,d2,n) then in the basis (d2,d1,-n)
$z=h(x,y)=-g(y,x)$. }
\ccMemberFunction{void dump_verbose(std::ofstream& out_stream);}
{ Outputs the data in a human readable way.}
\ccMemberFunction{void dump_4ogl(std::ofstream& out_stream, const DFT scale);}
{ Outputs the data for further visualization with successively~: the
coordinates of m\_origin\_pt, the coordinates of m\_d1 scaled by scale,
the coordinates of m\_d2 scaled by scale, the maximal principal
curvature, the minimal principal curvature.}

\ccSeeAlso

\ccc{Monge_via_jet_fitting},
%\ccc{some_other_function}.

%\ccExample

%A short example program.
%Instead of a short program fragment, a full running program can be
%included using the 
%\verb|\ccIncludeExampleCode{Jet_fitting_3/Monge_rep.C}| 
%macro. The program example would be part of the source code distribution and
%also part of the automatic test suite.

%\begin{ccExampleCode}
%void your_example_code() {
%}
%\end{ccExampleCode}

%%% \ccIncludeExampleCode{Jet_fitting_3/Monge_rep.C}

\end{ccRefClass}

% +------------------------------------------------------------------------+
%%RefPage: end of main body, begin of footer
% EOF
% +------------------------------------------------------------------------+

