% +------------------------------------------------------------------------+
% | Reference manual page: SvdTraits.tex
% +------------------------------------------------------------------------+
% | 09.02.2006   Marc Pouget and Fr�d�ric Cazals
% | Package: Jet_fitting_3
% | 
\RCSdef{\RCSSvdTraitsRev}{$Id$}
\RCSdefDate{\RCSSvdTraitsDate}{$Date$}
% |
%%RefPage: end of header, begin of main body
% +------------------------------------------------------------------------+


\begin{ccRefConcept}{SvdTraits}

%% \ccHtmlCrossLink{}     %% add further rules for cross referencing links
%% \ccHtmlIndexC[concept]{} %% add further index entries

\ccDefinition
% +------------------------------------------------------------------
The concept \ccRefName\ describes the set of requirements to be
fulfilled by any class used to instantiate the third template
parameter of the class
\ccc{Monge_via_jet_fitting<DataKernel,LocalKernel,SvdTraits>}.

It describes the linear algebra types and algorithms needed by the
class \ccc{Monge_via_jet_fitting}. 

\ccRequirements
% +------------------------------------------------------------------
The scalar type, \ccc{SvdTraits::FT}, must be the same as that of
the \ccc{LocalKernel} concept~: \ccc{LocalKernel::FT}.

\ccTypes
% +------------------------------------------------------------------
\ccNestedType{FT}{The scalar type.}
\ccGlue
\ccNestedType{Vector }{The Vector type.}
\ccGlue
\ccNestedType{Matrix }{The Matrix type.}

%\ccCreation
%\ccCreationVariable{a}  %% choose variable name

%\ccConstructor{SvdTraits();}{default constructor.}
\ccOperations
\ccCreationVariable{vector} %choose variable name

\ccConstructor{Vector(size_t n);} { initialize all the elements of the vector to zero.} 
The Vector has the access methods

\ccMethod{size_t size();}{}
\ccGlue
\ccMethod{FT operator()(size_t i); }{return the $i^{th}$ entry, $i$ from 0 to $size()-1$}
\ccGlue
\ccMethod{void set(size_t i, const FT value);}{set the $i^{th}$ entry to $value$}

The Matrix has the access methods

\ccCreationVariable{matrix} %choose variable name
\ccConstructor{Matrix(size_t n1, size_t n2);} { initialize all the entries of the matrix to zero.} 
\ccMethod{size_t number_of_rows();}{}
\ccGlue
\ccMethod{size_t number_of_columns();}{}
\ccGlue
\ccMethod{FT operator()(size_t i, size_t j); }
{return the entry at row $i$ and column $j$, $i$ from 0 to \ccc{number_of_rows - 1}, 
$j$ from 0 to \ccc{number_of_columns - 1}}
\ccGlue
\ccMethod{void set(size_t i, size_t j, const FT value); }
    {set the entry at row $i$ and column $j$ to $value$}

The SvdTraits has a linear solver using a singular value decomposition
algorithm.
\ccCreationVariable{traits} %choose variable name

\ccMethod{FT solve(Matrix& M, Vector& B);}
{ Solves the system MX=B (in the least square sense if M is not
square) using a Singular Value Decomposition and returns the condition
number of M. The solution is stored in B.}

\ccHasModels
% +------------------------------------------------------------------
\ccc{Lapack_svd}. 

\ccSeeAlso
% +------------------------------------------------------------------
\ccc{LocalKernel}

\end{ccRefConcept}

% +------------------------------------------------------------------------+
%%RefPage: end of main body, begin of footer
% EOF
% +------------------------------------------------------------------------+

