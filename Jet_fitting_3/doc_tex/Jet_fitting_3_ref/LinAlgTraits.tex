% +------------------------------------------------------------------------+
% | Reference manual page: LinAlgTraits.tex
% +------------------------------------------------------------------------+
% | 09.02.2006   Marc Pouget and Fr�d�ric Cazals
% | Package: Jet_fitting_3
% | 
\RCSdef{\RCSLinAlgTraitsRev}{$Id$}
\RCSdefDate{\RCSLinAlgTraitsDate}{$Date$}
% |
%%RefPage: end of header, begin of main body
% +------------------------------------------------------------------------+


\begin{ccRefConcept}{LinAlgTraits}

%% \ccHtmlCrossLink{}     %% add further rules for cross referencing links
%% \ccHtmlIndexC[concept]{} %% add further index entries

\ccDefinition
% +------------------------------------------------------------------
The concept \ccRefName\ describes the set of requirements to be
fulfilled by any class used to instantiate the third template
parameter of the class
\ccc{Monge_via_jet_fitting<DataKernel,LocalKernel,LinAlgTraits>}.

It describes the linear algebra types and algorithms needed by the
class \ccc{Monge_via_jet_fitting}. 

\ccRequirements
% +------------------------------------------------------------------
The scalar type, \ccc{LinAlgTraits::FT}, must be the same than that of
the \ccc{LocalKernel} concept~: \ccc{LocalKernel::FT}.

\ccTypes
% +------------------------------------------------------------------
\ccNestedType{FT}{The scalar type.}
\ccGlue
\ccNestedType{Matrix }{The Matrix type.}

%\ccCreation
%\ccCreationVariable{a}  %% choose variable name

%\ccConstructor{LinAlgTraits();}{default constructor.}
\ccCreationVariable{matrix} %choose variable name

\ccOperations
The Matrix has classical access to its elements.

\ccMethod{void set_elt(size_t i, size_t j, const FT value);}{}
\ccGlue
\ccMethod{FT get_elt(size_t i, size_t j);}{}

The LinAlgTraits has a singular value decomposition algorithm.
\ccCreationVariable{traits} %choose variable name

\ccMethod{void solve_ls_svd_algo(Matrix& M, FT* B, FT* cond_nb);}
{ Solves the system MX=B (in the least square sense if M is not
square) using a Singular Value Decomposition and gives the condition
number of M. The solution is stored in B.}

\ccHasModels
% +------------------------------------------------------------------
\ccc{Lapack}. 

\ccSeeAlso
% +------------------------------------------------------------------
\ccc{LocalKernel}

%\ccExample

%A short example program.
%Instead of a short program fragment, a full running program can be
%included using the 
%\verb|\ccIncludeExampleCode{Jet_fitting_3/LinAlgTraits.cpp}| 
%macro. The program example would be part of the source code distribution and
%also part of the automatic test suite.

%\begin{ccExampleCode}
%void your_example_code() {
%}
%\end{ccExampleCode}

%% \ccIncludeExampleCode{Jet_fitting_3/LinAlgTraits.cpp}

\end{ccRefConcept}

% +------------------------------------------------------------------------+
%%RefPage: end of main body, begin of footer
% EOF
% +------------------------------------------------------------------------+

