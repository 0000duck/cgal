% +------------------------------------------------------------------------+
% | Reference manual page: Lapack.tex
% +------------------------------------------------------------------------+
% | 09.02.2006   Marc Pouget and Fr�d�ric Cazals
% | Package: Jet_fitting_3
% | 
\RCSdef{\RCSLapackRev}{$Id$}
\RCSdefDate{\RCSLapackDate}{$Date$}
% |
%%RefPage: end of header, begin of main body
% +------------------------------------------------------------------------+

\begin{ccRefClass}{Lapack} %%add template arg's if necessary

%% \ccHtmlCrossLink{}     %% add further rules for cross referencing links
%% \ccHtmlIndexC[class]{} %% add further index entries

\ccDefinition
  
The class \ccRefName\ provides an algorithm to solve in the least
square sense a linear system.

\ccInclude{CGAL/Lapack/Linear_algebra_lapack.h}

\ccIsModel
\ccc{LinAlgTraits}

\ccTypes
% +--------------------------------------------------------------
%\ccNestedType{TYPE}{some nested types}
\ccTypedef{  typedef Lapack_matrix Matrix; }{}

\ccCreation
% +--------------------------------------------------------------
\ccCreationVariable{lapack}  %% choose variable name

\ccConstructor{Lapack();}{default constructor.}

\ccOperations
% +--------------------------------------------------------------
\ccMemberFunction{ void solve_ls_svd_algo(Matrix& M, double* B, double &cond_nb);
}{Solve MX=B using SVD and give the condition number of M. The
solution is stored in B.}

\ccSeeAlso

\ccc{LinAlgTraits}\\
\ccc{Lapack_matrix}
%\ccc{some_other_function}.

\end{ccRefClass}

% +------------------------------------------------------------------------+
%%RefPage: end of main body, begin of footer
% EOF
% +------------------------------------------------------------------------+

