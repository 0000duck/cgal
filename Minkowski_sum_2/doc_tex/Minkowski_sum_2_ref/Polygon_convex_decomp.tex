% +------------------------------------------------------------------------+
% | Reference manual page: Polygon_convex_decomp.tex
% +------------------------------------------------------------------------+
% | 
% | Package: Minkowski_sum_2
% | 
% +------------------------------------------------------------------------+

\ccRefPageBegin

\begin{ccRefConcept}{PolygonConvexDecomposition}

A model of the \ccRefName\ concept is capable of decomposing an input
polygon $P$ into a set of convex sub-polygons $P_1, \ldots, P_k$, such
that $\cup_{i=1}^{k}{P_k} = P$.

\ccTypes
%=======

\ccNestedType{Kernel}{the geometric kernel type.}
\ccGlue
\ccNestedType{Point_2}{the point type, used to represent polygon vertices.}
\ccGlue
\ccNestedType{Polygon_2}{the polygon type.}

\ccCreation
\ccCreationVariable{decomp}
%========================
    
\ccConstructor{PolygonConvexDecomposition ();}
   {default constructor.}

\ccOperations
%============

\ccMethod{template <class OutputIterator>
          OutputIterator decompose (const Polygon_2& P,
                                    OutputIterator oi) const;}
    {decomposes the input polygon \ccc{P} into convex sub-polygons,
     and writes them to the output iterator \ccc{oi}. The value-type of the
     output iterator must be \ccc{Polygons_2}.
     The function returns a past-the-end iterator for the convex sub-polygons.}

\ccHasModels

\ccc{Small_side_angle_bisector_decomposition<Kernel,Container>}\lcTex{
        (\ccRefPage{CGAL::Small_side_angle_bisector_decomposition<Kernel,Container>})}\\
\ccc{Optimal_convex_decomposition<Kernel,Container>}\lcTex{
        (\ccRefPage{CGAL::Optimal_convex_decomposition<Kernel,Container>})}\\
\ccc{Hertel_Mehlhorn_convex_decomposition<Kernel,Container>}\lcTex{
   (\ccRefPage{CGAL::Hertel_Mehlhorn_convex_decomposition<Kernel,Container>})}\\
\ccc{Greene_convex_decomposition<Kernel,Container>}\lcTex{
   (\ccRefPage{CGAL::Greene_convex_decomposition<Kernel,Container>})}

\end{ccRefConcept}

\ccRefPageEnd

