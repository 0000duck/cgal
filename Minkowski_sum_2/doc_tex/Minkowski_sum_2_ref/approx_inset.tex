% +------------------------------------------------------------------------+
% | Reference manual page: approx_inset.tex
% +------------------------------------------------------------------------+
% | 
% | Package: Minkowski_sum_2
% | 
% +------------------------------------------------------------------------+

\ccRefPageBegin

\begin{ccRefFunction}{approximated_inset_2}

\ccInclude{CGAL/approximated_offset_2.h}

\ccFunction{template<class Kernel, class Container, class OutputIterator>
            OutputIterator
            approximated_inset_2 (const Polygon_2<Kernel, Container>& pgn,
                                  const typename Kernel::FT& r,
                                  const double& eps,
                                  OutputIterator oi);}
   {Provides a guaranteed approximation of the inset, or inner offset, of
    the given polygon \ccc{P} by a given radius \ccc{r}. Namely, the
    function computes the set of points inside the polygon whose distance
    from $P$'s boundary is at least $r$:
    $\{ p \in P \;|\; {\rm dist}(p, \partial P) \geq r \}$,
    with the approximation error bounded by \ccc{eps}.
    Note that as the input polygon may not be convex, its inset may comprise
    several disconnected components. The result is therefore represented as a
    sequence of generalized polygons, whose edges are either line segments or
    circular arcs.
    The output sequence is returned via the output iterator \ccc{oi}, whose
    value-type must be \ccc{Gps_circle_segment_traits_2<Kernel>::Polygon_2}.
    \ccPrecond{\ccc{P} is a simple polygon.}}

\end{ccRefFunction}

\ccRefPageEnd
