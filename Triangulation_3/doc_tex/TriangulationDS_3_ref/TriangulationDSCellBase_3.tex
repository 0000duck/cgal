% +------------------------------------------------------------------------+
% | Reference manual page: TriangulationDSCellBase_3.tex
% +------------------------------------------------------------------------+
% | 29.3.2000   Monique Teillaud
% | Package: Triangulation3
% | 
\RCSdef{\RCSTriangulationcbRev}{$Id$}
\RCSdefDate{\RCSTriangulationcbDate}{$Date$}
% |
%%RefPage: end of header, begin of main body
% +------------------------------------------------------------------------+


\begin{ccRefConcept}{TriangulationDSCellBase_3}

%% \ccHtmlCrossLink{}     %% add further rules for cross referencing links
%% \ccHtmlIndexC[concept]{} %% add further index entries

\ccDefinition
  
At the base level 
(see Sections~\ref{Triangulation3-sec-design} and~\ref{TDS3-sec-design}),
a cell stores handles to its four vertices and to its four neighbor cells. 
The vertices and neighbors are indexed 0, 1, 2 and 3. Neighbor $i$
lies opposite to vertex $i$.

Since the Triangulation data structure is the class which defines the handle
types, the cell base class has to be somehow parameterized by the Triangulation
data structure.  But since it is itself parameterized by the cell and vertex
base classes, there is a cycle in the definition of these classes.  In order
to break the cycle, the base classes for vertex and cell which are given as
arguments for the Triangulation data structure use \ccc{void} as Triangulation
data structure parameter, and the Triangulation data structure then uses a
{\it rebind}-like mechanism (similar to the one specified in
\ccc{std::allocator}) in order to put itself as parameter to the vertex and
cell classes.  The {\it rebound} base classes so obtained are the classes
which are used as base classes for the final vertex and cell classes.
More information can be found in Section~\ref{TDS3-sec-design}.

\ccTypes
\ccThree{typedef TriangulationDataStructure_3::Vertex_handle}{}{}
\ccThreeToTwo
The concept \ccRefName\ has to provide the following types.

\ccNestedType{
  template <typename TDS2>
  struct Rebind_TDS}
{This nested template class has to define a type \ccc{Other} which is the
{\it rebound} cell, that is, the one whose \ccc{Triangulation_data_structure}
will be the actually used one.  The \ccc{Other} type will be the real base
class of \ccc{Triangulation_data_structure_3::Cell}.}

\ccTypedef{typedef TriangulationDataStructure_3 Triangulation_data_structure;}{}
\ccGlue
\ccTypedef{typedef TriangulationDataStructure_3::Vertex_handle Vertex_handle;}{}
\ccGlue
\ccTypedef{typedef TriangulationDataStructure_3::Cell_handle Cell_handle;}{}

\ccCreation
\ccCreationVariable{c}  %% choose variable name

\ccConstructor{Cell_base();}
{Default constructor}
\ccGlue
\ccConstructor{Cell_base( Vertex_handle v0, Vertex_handle v1,
                          Vertex_handle v2, Vertex_handle v3);}
{Initializes the vertices with \ccc{v0, v1, v2, v3}. Neighbors are
initialized to the default constructed handle.} 
\ccGlue
\ccConstructor{Cell_base( Vertex_handle v0, Vertex_handle v1,
                          Vertex_handle v2, Vertex_handle v3,
			  Cell_handle n0, Cell_handle n1,
                          Cell_handle n2, Cell_handle n3)} 
{Initializes the vertices with \ccc{v0, v1, v2, v3} and the neighbors with
\ccc{n0, n1, n2, n3}.}
                    
\ccAccessFunctions
\ccThree{Vertex_handlex}{xxxxxxxxxxxxxxxxxxxxxxxxxxxxxxxxxxxx}{}

\ccMethod{Vertex_handle vertex(int i) const;}
{Returns the vertex \ccc{i} of \ccVar.
\ccPrecond{$i \in \{0, 1, 2, 3\}$.}}
\ccGlue
\ccMethod{int index(Vertex_handle v) const;}
{Returns the index of \ccc{v}.
\ccPrecond{\ccc{v} is a vertex of \ccVar}}
\ccGlue
\ccMethod{bool has_vertex(Vertex_handle v);}
{True iff \ccc{v} is a vertex of \ccVar.}
\ccGlue
\ccMethod{bool has_vertex(Vertex_handle v, int & i) const;}
{Returns \ccc{true} if \ccc{v} is a vertex of \ccVar, and
computes its index \ccc{i} in \ccVar.}

\ccMethod{Cell_handle neighbor(int i) const;}
{Returns  the neighbor \ccc{i} of \ccVar.
\ccPrecond{$i \in \{0, 1, 2, 3\}$.}}
\ccGlue
\ccMethod{int index(Cell_handle n) const;}
{Returns the index of cell \ccc{n} in \ccVar.
\ccPrecond{\ccc{n} is a neighbor of \ccVar.}}
\ccGlue
\ccMethod{bool has_neighbor(Cell_handle n);}
{Returns \ccc{true} if \ccc{n} is a neighbor of \ccVar.}
\ccGlue
\ccMethod{bool has_neighbor(Cell_handle n, int & i) const;}
{Returns \ccc{true} if \ccc{n} is a neighbor of \ccVar,  and
computes its index \ccc{i} in \ccVar.}

\ccHeading{Setting}

\ccMethod{void set_vertex(int i, Vertex_handle v);}
{Sets vertex \ccc{i} to \ccc{v}.
\ccPrecond{$i \in \{0, 1, 2, 3\}$.}}
\ccGlue
\ccMethod{ void set_vertices();}
{Sets the vertices to the default constructed handle.}
\ccGlue
\ccMethod{void set_vertices( Vertex_handle v0, Vertex_handle v1,
		       	     Vertex_handle v2, Vertex_handle v3);}
{Sets the vertices.}

\ccMethod{void set_neighbor(int i, Cell_handle n);}
{Sets neighbor \ccc{i} to \ccc{n}.
\ccPrecond{$i \in \{0, 1, 2, 3\}$.}}
\ccGlue
\ccMethod{void set_neighbors();} 
{Sets the neighbors to the default constructed handle.}
\ccGlue
\ccMethod{void set_neighbors(Cell_handle n0, Cell_handle n1,
		     	     Cell_handle n2, Cell_handle n3);}
{Sets the neighbors.}

\ccHeading{Checking}
\ccMethod{bool is_valid(bool verbose = false, int level = 0) const;}
{Performs any desired geometric test on a cell.\\
When \ccc{verbose} is set to \ccc{true}, messages are printed to give
a precise indication of the kind of invalidity encountered. \ccc{level}
increases the level of testing.}

\ccHeading{Various}

\ccMethod{void * for_compact_container() const;}{}
\ccGlue
\ccMethod{void * & for_compact_container();}{}
{ These member functions are required by \ccc{Triangulation_data_structure_3}
  because it uses \ccc{Compact_container} to store its cells.  See the
  documentation of \ccc{Compact_container} for the exact requirements.}

\ccMethod{void set_in_conflict_flag(unsigned char f);}{}
\ccGlue
\ccMethod{unsigned char get_in_conflict_flag() const;}{}
{ These functions are used internally to mark cells with a flag.  The user is
not encouraged to use them directly as they may change in the future.}

\ccHeading{I/O}

\ccFunction{istream& operator>> (istream& is, TriangulationDSCellBase_3 & c);}
{Inputs the possible non combinatorial information given by the cell.}
\ccGlue
\ccFunction{ostream& operator<< (ostream& os, const TriangulationDSCellBase_3 & c);}
{Outputs the possible non combinatorial information given by the cell.}

\ccHasModels

\ccc{CGAL::Triangulation_ds_cell_base_3}\\
\ccc{CGAL::Triangulation_cell_base_3}\\
\ccc{CGAL::Triangulation_cell_base_with_info_3}

\ccSeeAlso

\ccc{TriangulationDSVertexBase_3}\\
\ccc{TriangulationVertexBase_3}\\
\ccc{TriangulationHierarchyVertexBase_3}\\
\ccc{TriangulationCellBase_3}

%% \ccExample

%% \ccIncludeExampleCode{Triangulation3/Triangulation_cb_3_prog.C}

\end{ccRefConcept}
