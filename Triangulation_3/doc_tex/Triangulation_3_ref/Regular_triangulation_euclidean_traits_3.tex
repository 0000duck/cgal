% +------------------------------------------------------------------------+
% | Reference manual page: Regular_triangulation_euclidean_traits_3.tex
% +------------------------------------------------------------------------+
% | 27.3.2000   Monique Teillaud
% | Package: Triangulation3
% | 
\RCSdef{\RCSRegulartriangulationeuclideantraitsRev}{$Id$}
\RCSdefDate{\RCSRegulartriangulationeuclideantraitsDate}{$Date$}
% |
%%RefPage: end of header, begin of main body
% +------------------------------------------------------------------------+


\begin{ccRefClass}{Regular_triangulation_euclidean_traits_3<K,Weight>}  %% add template arg's if necessary

%% \ccHtmlCrossLink{}     %% add further rules for cross referencing links
%% \ccHtmlIndexC[class]{} %% add further index entries

\ccDefinition
  
The class \ccRefName\ is designed as a default traits class for the
class \ccc{Regular_triangulation_3<RegularTriangulationTraits_3,TriangulationDataStructure_3>}. 
It provides \ccc{Weighted_point_3}, a class for weighted points, which derives
from the three dimensional point class \ccc{K::Point_3}.

The first argument \ccc{K} must be a model of the \ccc{Kernel} concept. 

The second argument \ccc{Weight} of the class
\ccc{Regular_triangulation_euclidean_traits_3<K,Weight>} is in fact
optional: if is it not provided, \ccc{K::RT} will be used.

The class is a model of the concept \ccc{RegularTriangulationTraits_3}
but it also contains predicates and constructors on weighted points
that are not required in the
concept\ccc{RegularTriangulationTraits_3}.

Note that this template class is specialized for
\ccc{CGAL::Exact_predicates_inexact_constructions_kernel}, so that it is as if
\ccc{Regular_triangulation_filtered_traits_3} was used, i.e. you get
filtered predicates automatically.

 


\ccInclude{CGAL/Regular_triangulation_euclidean_traits_3.h}

\ccIsModel
\ccc{RegularTriangulationTraits_3}

\ccInheritsFrom{\ccc{K}}

\ccTypes
\ccThree{typedef Weighted_point <Bare_point, Weight>}{Weighted_point_3;}{}

\ccTypedef{typedef K::Point_3 Bare_point;}
{The type for point $p$ of a weighted point ${p}^{(w)}=(p,w_p)$.}
\ccGlue
\ccTypedef{typedef Weighted_point <Bare_point, Weight>
Weighted_point_3;}
{The type for weighted points.}

\ccHeading{Types for predicate functors}
\ccTwo{Regular}{}
\ccNestedType{Power_test_3}
{A predicate type for power test. Belongs to the 
\ccc{RegularTriangulationTraits_3} concept.}

\ccNestedType{Compare_power_distance_3}
{A predicate type to compare power distance. Belongs to the
\ccc{RegularTriangulationTraits_3} concept.}

\ccNestedType{In_smallest_orthogonal_sphere_3}
{A predicate  type.  The operator() takes weighted points as arguments
 and returns  the  sign of the power distance of the last one
with respect to the smallest sphere orthogonal to the others. \\
\ccc{Sign  operator()(  Weighted_point_3  p,
		        Weighted_point_3  q,
		        Weighted_point_3  r,
		        Weighted_point_3  s,
		        Weighted_point_3  t)  ;} \\
\ccc{Sign  operator()(  Weighted_point_3  p,
		        Weighted_point_3  q,
		        Weighted_point_3  r,
		        Weighted_point_3  s)  ;} \\
\ccc{Sign  operator()(  Weighted_point_3  p,
		        Weighted_point_3  q,
		        Weighted_point_3  r)  ;} \\
\ccc{Sign  operator()(  Weighted_point_3  p,
		        Weighted_point_3  q)  ;} \\
}


\ccNestedType{Side_of_bounded_orthogonal_sphere_3}
{A predicate  type.  The operator()  is  similar to the operator()  of
\ccc{In_smallest_orthogonal_sphere_3}
except that the  returned type  is not a \ccc{Sign}
but belongs to the  enum \ccc{Bounded_side}
(A \ccc{NEGATIVE}, ccc{NULL} and \ccc{POSITIVE})
 corresponding respectively to 
\ccc{ON_BOUNDED_SIDE}, \ccc{ON_BOUNDARY} and \ccc{ON_UNBOUNDED_SIDE})).
 \\
\ccc{Bounded_side operator() (  Weighted_point_3  p,
			     Weighted_point_3  q,
			     Weighted_point_3  r,
			     Weighted_point_3  s,
			     Weighted_point_3  t) ; } \\
\ccc{Bounded_side operator() (  Weighted_point_3  p,
			     Weighted_point_3  q,
			     Weighted_point_3  r,
			     Weighted_point_3  s) ; } \\
\ccc{Bounded_side operator() (  Weighted_point_3  p,
			     Weighted_point_3  q,
			     Weighted_point_3  r) ; } 
}


\ccNestedType{Does_simplex_intersect_dual_support_3}
{A predicate type. The operator() 
takes weighted points as arguments, 
considers the subspace of points with equal power distance 
with respect to its arguments
and the intersection of this subspace with the affine hull of the bare
points associated to the arguments.
The operator()  returns \ccc{ON_BOUNDED_SIDE},
\ccc{ON_BOUNDARY} or  \ccc{ON_UNBOUNDED_SIDE}
according to the position of this intersection 
with respect to the simplex formed by the bare
points.
This predicate is usefull for flow computations. \\
\ccc{Bounded_side operator()( Weighted_point_3  p,
		           Weighted_point_3  q,
		           Weighted_point_3  r,
		           Weighted_point_3  s) ; } \\
\ccc{Bounded_side operator()( Weighted_point_3  p,
		           Weighted_point_3  q,
		           Weighted_point_3  r) ; }\\
\ccc{Bounded_side operator()( Weighted_point_3  p,
		           Weighted_point_3  q) ; }
}

\ccHeading{Types for constructor functors}

\ccNestedType{Construct_weighted_circumcenter_3}
{A constructor type. The operator()  constructs the bare point
 which is the center of the smallest orthogonal sphere to the input
weighted points. \\
\ccc{Bare_point operator() (  Weighted_point_3  p,
			   Weighted_point_3  q,
			   Weighted_point_3  r,
			   Weighted_point_3  s);} \\
\ccc{Bare_point operator() (  Weighted_point_3  p,
			   Weighted_point_3  q,
			   Weighted_point_3  r);} \\
\ccc{Bare_point operator() (  Weighted_point_3  p,
			   Weighted_point_3  q);}
}



\ccNestedType{Compute_power_product_3}
{A functor type. The operator() computes the power distance between its
arguments. \\
\ccc{FT operator() ( Weighted_point_3  p,
		  Weighted_point_3  q) ; }
}


\ccNestedType{Compute_squared_radius_smallest_orthogonal_sphere_3}
{A functor type. The operator() computes the squared radius of the
smallest sphere orthogonal to the arguments. 
\\
\ccc{FT operator() (  Weighted_point_3  p,
		   Weighted_point_3  q,
		   Weighted_point_3  r,
		   Weighted_point_3  s);} \\
\ccc{FT operator() (  Weighted_point_3  p,
		   Weighted_point_3  q,
		   Weighted_point_3  r);} \\
\ccc{FT operator() (  Weighted_point_3  p,
		    Weighted_point_3  q);} \\
}

\ccNestedType{Compute_critical_squared_radius_3}
{A functor type. The operator() takes weighted points as arguments
and computes the squared radius 
of the sphere centered in the last point and orthogonal
to the other weighted points. The last argument is a weighted point 
but its weight does not matter. 
This construction is adhoc for pumping slivers.
For robustness issue, a predicate to compare critical squared radii
for a given last point should be needed. \\
\ccc{FT operator() ( Weighted_point_3  p,
			   Weighted_point_3  q,
			   Weighted_point_3  r,
			   Weighted_point_3  s,
			   Weighted_point_3  t);}
}





\ccOperations
The following functions give access to the predicate and constructor
functors.
\ccCreationVariable{traits}

\ccThree{Does_simplex_intersect_dual_support_3}
{does_simplex_intersect_dual_support_3_object()}{}
\ccMethod{Power_test_3 power_test_3_object();}{}
\ccGlue
\ccMethod{Compare_power_distance_3
compare_power_distance_3_object();}{}
\ccGlue
\ccMethod{In_smallest_orthogonal_sphere_3 
  in_smallest_orthogonal_sphere_3_object();}{}
\ccGlue
\ccMethod{Side_of_bounded_orthogonal_sphere_3
  side_of_bounded_orthogonal_sphere_3_object();}{}
\ccGlue
\ccMethod{Does_simplex_intersect_dual_support_3 
  does_simplex_intersect_dual_support_3_object();}{}
\ccGlue
\ccMethod{Construct_weighted_circumcenter_3 
  construct_weighted_circumcenter_3_object();}{}
\ccGlue
\ccMethod{Compute_power_product_3
  compute_power_product_3_object() ;}{}
\ccGlue
\ccMethod{Compute_squared_radius_smallest_orthogonal_sphere_3
  compute_squared_radius_smallest_orthogonal_sphere_3_object() ;}{}
\ccGlue
\ccMethod{Compute_critical_squared_radius_3
  compute_critical_squared_radius_3_object();}{}


%% \ccExample

%% \ccIncludeExampleCode{examples/Triangulation3/Regular_triangulation_euclidean_traits_3_prog.C}

\ccSeeAlso

\ccc{CGAL::Regular_triangulation_filtered_traits_3}.

\end{ccRefClass}
