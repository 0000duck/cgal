% +------------------------------------------------------------------------+
% | Reference manual page: LargestEmptyIsoRectangleTraits_2.tex
% +------------------------------------------------------------------------+
% | 06.04.2000   Author
% | Package: Package
% | 
\RCSdef{\RCSSnapRoundingTraits_2Rev}{$Revision$}
\RCSdefDate{\RCSSnapRoundingTraits_2Date}{$Date$}
% |
%%RefPage: end of header, begin of main body
% +------------------------------------------------------------------------+

\begin{ccRefConcept}{SnapRoundingTraits_2}

%% \ccHtmlCrossLink{}     %% add further rules for cross referencing links
%% \ccHtmlIndexC[concept]{} %% add further index entries

\ccDefinition
  
The concept \ccRefName\ lists the set of requirements that must be fulfilled by
an instance of the \ccc{Traits} template-parameter of
the function \ccc{snap_rounding_2<Traits,InputIterator,OutputContainer>()}.
This concept provides the types of the geometric primitives used in
this class and some function object types for the required
predicates on those primitives.

\ccRefines
This concept refines the standard concepts DefaultConstructible, Assignable and
CopyConstructible. It also refines the concept
\ccc{SweepLineTraits_2} \lcTex{(\ccRefPage{SweepLineTraits_2})}.
An instance of this concept is used as the traits class for the
\ccc{Sweep_line_2.get_intersection_points()} operation. The requirements listed
below are induced by compnents of the \ccc{snap_rounding_2()} function other
than the call to \ccc{Sweep_line_2.get_intersection_points()}. Naturally, some
of them may already be listed in \ccc{SweepLineTraits_2}.

\ccTypes
\ccNestedType{FT}{The number type. This type must fulfill the requirements on
\ccc{FieldNumberType}}
\ccGlue
\ccNestedType{Point_2}{The point type.}
\ccGlue
\ccNestedType{Segment_2}{The segment type.}
\ccGlue
\ccNestedType{Iso_rectangle_2}{The iso-rectangle type.}

\ccNestedType{Construct_vertex_2}{Function object. Must provide the operator
\ccc{Point_2 operator()(Segment_2 seg, int i)}, which returns the source or
target of \ccc{seg}. If \ccc{i} modulo 2 is 0, the source is returned,
otherwise the target is returned. }
\ccGlue
\ccNestedType{Construct_segment_2}{Function object. Must provide the operator
\ccc{Segment_2 operator()(Point_2 p, Point_2 q)}, which  introduces a segment
with source \ccc{p} and target \ccc{q}. The segment is directed from the
source towards the target.}
\ccGlue
\ccNestedType{Construct_iso_rectangle_2}{Function object. Must provide the
operator
\ccc{Iso_rectangle_2 operator()(Point_2 left, Point_2 right, Point_2 bottom,
Point_2 top)}, which  introduces an iso-oriented rectangle fo whose minimal
$x$ coordinate is the one of \ccc{left}, the maximal $x$ coordinate is the one
of \ccc{right}, the minimal $y$ coordinate is the one of \ccc{bottom}, the
maximal $y$ coordinate is the one of \ccc{top}. }

\ccNestedType{To_double}{Function object. Must provide the operator
\ccc{double  operator()(FT)}, which computes an approximation of a given number
of type \ccc{FT}. The precision of this operation is of not high significance,
as it is only used in the implementation of the heuristic technique to exploit
a cluster of kd-trees rather than just one.}
\ccGlue
\ccNestedType{Compare_x_2}{Function object. Must provide the operator
\ccc{Comparison_result operator()(Point_2 p, Point_2 q)}
which returns 
\ccc{SMALLER, EQUAL} or \ccc{LARGER} according to the
$x$-ordering of points \ccc{p} and \ccc{q}.}
\ccGlue
\ccNestedType{Compare_y_2}{Function object. Must provide the operator
\ccc{Comparison_result operator()(Point_2 p, Point_2 q)}
which returns
\ccc{SMALLER, EQUAL} or \ccc{ LARGER}
according to the
$y$-ordering of points \ccc{p} and \ccc{q}.}
\ccGlue

\ccNestedType{Snap_2}{Rounds a point to a center of a pixel (unit square)
in the grid used by the Snap Rounding algorithm. Note that no conversion
to an integer grid is done yet. Must have the syntax
\ccc{void operator()(Point_2 p,FT pixel_size,FT &x,FT &y)} where $p$ is the
input point, \ccc{pixel_size} is the size of the pixel of the grid,
and $x$ and $y$ are the $x$ and $y$-coordinates of the rounded point
respectively.}

\ccNestedType{Integer_grid_point_2}{Convert coordinates
into an integer representation where one unit is equal to pixel size.
For instance, if a point has the coordinates $(3.7,5.3)$ and the pixel
size is $0.5$, then the new point will have the coordinates of $(7,10)$.
Note, however, that the number type remains the same here, although
integers are represented.
Must have the syntax \ccc{Point_2 operator()(Point_2 p,NT pixel_size)}
where $p$ is the converted point and \ccc{pixel_size} is the size of the pixel
of the grid.}

\ccNestedType{Minkowski_sum_with_pixel_2}{Returns the vertices of a polygon,
which is the Minkowski sum of a segment and a square centered at the origin
with edge size \ccc{pixel edge}.
Must have the syntax
\ccc{void operator()(std::list<Point_2>& vertices_list, Segment_2 s,
NT unit_square)}
where \ccc{vertices_list} is the list of the vertices of the Minkowski sum
polygon, $s$ is the input segment and \ccc{unit_square} is the edge size of
the pixel.}

\ccCreation
\ccCreationVariable{traits}  %% choose variable name
This concept refines the standard concepts DefaultConstructible, Assignable and
CopyConstructible.

% \ccConstructor{SnapRoundingTraits_2();}{Default constructor.}
% \ccGlue
% \ccConstructor{SnapRoundingTraits_2(SnapRoundingTraits_2);}
% {Copy constructor}
% \ccMethod{SnapRoundingTraits_2 operator=(SnapRoundingTraits_2 gtr);}
% {Assignment operator.}

\ccOperations

The following functions construct the required function objects occasionally
referred as functors listed above.
% Each of these objects provide a necessary operation to carry out the
% snap-rounding procedure.

\ccThree{Minkowski_sum_with_pixel_2}{minkowski_sum_with_pixel_2_object();}{}
\ccMethod{Construct_vertex_2 construct_vertex_2_object();}{}
\ccGlue
\ccMethod{Construct_segment_2 construct_segment_2_object();}{}
\ccGlue
\ccMethod{Construct_iso_rectangle_2 construct_iso_rectangle_2_object();}{}

\ccMethod{Compare_x_2 compare_x_2_object();}{}                               
\ccGlue
\ccMethod{Compare_y_2 compare_y_2_object();}{}
\ccGlue
\ccMethod{Snap_2 snap_2_object();}{}
\ccGlue
\ccMethod{Integer_grid_point_2 integer_grid_point_2_object();}{}
\ccGlue
\ccMethod{Minkowski_sum_with_pixel_2 minkowski_sum_with_pixel_2_object();}{}
\ccGlue


\ccHasModels
\ccc{CGAL::Snap_rounding_traits<Kernel>}

\ccSeeAlso
\ccc{CGAL::Snap_rounding_2<Traits>}

\end{ccRefConcept}

% +------------------------------------------------------------------------+
%%RefPage: end of main body, begin of footer
% EOF
% +------------------------------------------------------------------------+

