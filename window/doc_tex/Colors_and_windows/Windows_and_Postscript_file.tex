\ccParDims

\section{Introduction}

The \leda\ library comes with its own window toolkit and with a
Postscipt file stream. \cgal\ provides two classes 
 \ccStyle{CGAL::Window_stream} \ccStyle{Postscript_file_stream} 
which  are a thin glue layer that allows I/O for the geometric 
classes of the \cgal\ Kernel.


\section{Window output}

An object of type \ccStyle{CGAL::Window_stream} is a two-dimensional window for
graphical IO. The input and output operations perform a mapping from
object coordinates to window coordinates.
Actually, \ccStyle{CGAL::Window_stream} is identical to the \leda\ class \ccStyle{leda_window},
and \leda\ must be installed. The class additionlly provides input and output 
operators for the classes which are defined in the \cgal\ kernel.
\ccStyle{CGAL::Window_stream} should be initialized with a coordinate frame 
(by calling member function
\ccc{init}) and must be displayed (by calling member function \ccc{display}) before
drawing.

\begin{ccTexOnly}
For further information on \ccStyle{leda_window}, we refer to \cite{cgal:mnsu-lum}.
\end{ccTexOnly}
\begin{ccHtmlOnly}
For further information on <i>leda_window</i>, we refer to the 
<A HREF="http://www.algorithmic-solutions.info/leda_manual/MANUAL.html">LEDA manual</A>.
\end{ccHtmlOnly}

Input and output operators for \ccStyle{CGAL::Window_stream} are defined for 
the geometric classes in the \cgal\ kernel.
The coordinates of the geometric objects are in object space. Data are
entered with the left mouse button.

It is important that the window stream header file gets included
{\em after} the inclusion of the header files of geometric classes
that get inserted to or extracted from the window stream. See the following 
small example program.

\begin{verbatim}
#include <CGAL/Cartesian.h>
#include <CGAL/Segment_2.h>
#include <CGAL/IO/Window_stream.h>

typedef CGAL::Point_2< CGAL::Cartesian<double> >     Point;
typedef CGAL::Segment_2< CGAL::Cartesian<double> >   Segment;

int main()
{
    Point p(0,1), q(2,2);
    Segment s(p,q);

    CGAL::Window_stream W(100,100);
    W.init(0,10,10);
    CGAL::cgalize( W);
    W.display();

    W << CGAL::RED << s << CGAL::BLACK << p << q ;
    W >> s >> p;
    return 0;
}
\end{verbatim} 

3D objects can be used as well with the \ccc{Window_stream}. The stream output
operator \ccc{<<} projects them to the $xy$-plane. The stream input
operator \ccc{>>} sets the $z$ coordinate to zero.
To get these operators you have to 

\ccInclude{CGAL/IO/window_stream_xy_3.h}


\section{Postscript output}

An object of type \ccStyle{Postscript_file_stream} provides graphical output in PostScript
format.
\ccStyle{Postscript_file_stream} is derived from the \leda\ class \ccStyle{ps_file}.
Therefore, the functionality of \ccStyle{ps_file} is available on \ccStyle{Postscript_file_stream}
as well.
In addition, the class \ccStyle{Postscript_file_stream} provides output 
operators for the classes which are defined in the \cgal\ kernel.
The interface of \ccStyle{Postscript_file_stream} is very similar to the interface 
of \ccStyle{leda_window}. 
To increase similarity, some functions have been added with respect to \ccStyle{ps_file}.
\begin{ccTexOnly}
For further information on \ccStyle{ps_file}, we refer to \cite{cgal:mnsu-lum}.
\end{ccTexOnly}
\begin{ccHtmlOnly}
For further information on <i>ps_file</i>, we refer to the 
<A HREF="http://www.algorithmic-solutions.info/leda_manual/MANUAL.html">LEDA manual</A>.
\end{ccHtmlOnly}

The output operator to \ccStyle{Postscript_file_stream} is defined for all geometric classes 
in the two-dimensional \cgal\ kernel. There are no input operators of course.

It is important that the postscript file stream header file gets included
{\em after} the inclusion of the header files of geometric classes
that get inserted to the postscript stream file. See the following example program.

\begin{verbatim}
#include <CGAL/Cartesian.h>
#include <CGAL/Segment_2.h>
#include <CGAL/IO/Postscript_file_stream.h>

typedef CGAL::Point_2< CGAL::Cartesian<double> >     Point;
typedef CGAL::Segment_2< CGAL::Cartesian<double> >   Segment;

int main()
{
    Point p(0,1), q(2,2);
    Segment s(p,q);

    CGAL::Postscript_file_stream PS(100,100);
    PS.init(0,10,0);
    CGAL::cgalize( PS);
    PS.display();

    PS << CGAL::RED << s << CGAL::BLACK << p << q ;
    return 0;
}
\end{verbatim} 

