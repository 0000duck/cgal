\begin{ccRefClass}{Triangulation_ds_vertex<TriangulationDataStructure>}

\ccDefinition

The class \ccRefName\ serves as the default vertex template parameter in the
class \ccc{Triangulation_data_structure<Dimensionality, TriangulationDSVertex,
TriangulationDSFullCell>}.

This class does not contain any geometric information but only combinatorial
(adjacency) information. Thus, if the \ccc{Triangulation_data_structure} is
used as a parameter of a (embedded) \ccc{Triangulation}, then its vertex template parameter
has to fulfill additional geometric requirements, {i.e.}, it has to be a
model of the refined concept \ccc{TriangulationVertex}.

This class can be used directly or can serve as a base to derive other classes
with some additional attributes tuned for a specific application (a color for
example).

\ccInclude{CGAL/Triangulation_ds_vertex.h}

\ccParameters

The template parameter \ccc{TriangulationDataStructure} must be a model of the
\ccc{TriangulationDataStructure} concept.
% It defaults to \ccc{void}, which is used to break some dependency cycles.

\ccIsModel

\ccc{TriangulationDSVertex}

\ccCreationVariable{v}

\begin{ccDebug}
\ccHeading{Validity check}

The \ccc{is_valid} method is only minimally defined in the
\ccc{TriangulationDSVertex} concept, so that we document it more precisely
here, for the model \ccRefName:

\ccMethod{bool is_valid(bool verbose=false) const;}{%
Implements the validity checks required by the concept
\ccc{TriangulationDSVertex}. Does not implement additional checks.%
}
\end{ccDebug}

\begin{ccAdvanced}
\ccHeading{Rebind mechanism}
In case of derivation from that class, the nested class
\ccc{Rebind_TDS} need to be provided in the derived class.
\end{ccAdvanced}



\ccSeeAlso

\ccc{Triangulation_ds_full_cell<TriangulationDataStructure, TDSFullCellStoragePolicy>}\\
\ccc{Triangulation_data_structure<Dimensionality, TriangulationDSVertex, TriangulationDSFullCell>>}
\end{ccRefClass}
