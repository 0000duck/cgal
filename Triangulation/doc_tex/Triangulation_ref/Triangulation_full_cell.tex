\begin{ccRefClass}{Triangulation_full_cell<TriangulationTraits, Data, TriangulationDSFullCell>}

\ccDefinition

The class \ccRefName\ is a model of the concept \ccc{TriangulationFullCell}. It
is used by default for representing full cells in the class
\ccc{Triangulation<TriangulationTraits, TriangulationDataStructure>}.

A \ccRefName\ stores handles to the vertices of the cell as well as handles
to its adjacent cells.

\ccInclude{CGAL/Triangulation_full_cell.h}

\ccParameters

\ccc{TriangulationTraits} must be a model of the concept \ccc{TriangulationTraits}. It
provides geometric types and predicates for use in the
\ccc{Triangulation<TriangulationTraits, TriangulationDataStructure>} class.

\ccc{Data} is an optional type of data to be stored in the full cell class. The
class template \ccRefName\ accepts that no second parameter be specified. In
this case, \ccc{Data} defaults to \ccc{CGAL::No_full_cell_data}.
\ccc{CGAL::No_full_cell_data} can explicitely be specified to access the third parameter.

Parameter \ccc{TriangulationDSFullCell} must be a model of the concept
\ccc{TriangulationDSFullCell}.
The class template \ccRefName\ accepts that no third parameter be specified.
It also accepts the tag \ccc{CGAL::Default} as third parameter. In both
cases, \ccc{TriangulationDSFullCell} defaults to \ccc{CGAL::Triangulation_ds_full_cell<>}.

\ccInheritsFrom

\ccc{TriangulationDSFullCell} (the third template parameter)

\ccIsModel

\ccc{TriangulationFullCell}

Additionally, the class \ccRefName\ also provides the following type,
constructors and methods:

\ccTypes

\ccTypedef{typedef Data Data;}{The type of the additional data stored in the
cell. If you read a \ccRefName\ from a stream (a file) or write a \ccRefName
to a stream, then streaming operators \ccc{<<} and \ccc{>>} must be provided for this
type.}

\ccCreation
\ccCreationVariable{c}

\ccConstructor{template< typename T> Triangulation_full_cell(int dmax, const T
& t);}{Sets the maximum possible dimension of the cell to \ccc{dmax}.
The parameter \ccc{t} is passed to the \ccc{Data} constructor.}

\ccHeading{Data access}

\ccMethod{const Data & data() const;}{Returns a const reference to the stored data.}
\ccGlue\ccMethod{Data & data();}{Returns a non-const reference to the stored data.}

\ccHeading{Input/Output}

\ccFunction{istream & operator>>(istream & is, Triangulation_full_cell & v);}%
{Inputs the non-combinatorial information given by the cell, {i.e.},
the point and other possible information. The data of type \ccc{Data} is
{also} read.}

\ccFunction{ostream & operator<<(ostream & os, const Triangulation_full_cell & v);}%
{Outputs the non-combinatorial information given by the cell, {i.e.},
the point and other possible information. The data of type \ccc{Data} is
{also} written.}

\ccSeeAlso

\ccc{Triangulation_vertex<TriangulationTraits, Data, TriangulationDSVertex>}\\
\ccc{Triangulation_data_structure<Dimensionality, TriangulationDSVertex, TriangulationDSFullCell>}\\
\ccc{Triangulation<TriangulationTraits,TriangulationDataStructure>}\\
\ccc{Delaunay_triangulation<DelaunayTriangulationTraits, TriangulationDataStructure>}
\end{ccRefClass}
