\begin{ccRefConcept}{TriangulationDSVertex}

\ccDefinition

The concept \ccRefName\ describes what a vertex is in a model of the concept
\ccc{TriangulationDataStructure}. It sets requirements of combinatorial nature
only, as geometry is not concerned here. In particular, we only require that
the vertex hold a handle to a full cell incident to it in the triangulation.

This concept is a \emph{sub-concept} of the \ccc{TriangulationDataStructure}
concept.

\ccHasModels

\ccc{CGAL::Triangulation_ds_vertex<TriangulationDataStructure>}\\
\ccc{CGAL::Triangulation_vertex<TriangulationTraits, Data, TriangulationDSVertex>}

\ccTypes

\ccNestedType{Full_cell_handle}{A handle to a cell. It must be the same as the
nested type \ccc{TriangulationDataStructure::Full_cell_handle} of the \ccc{TriangulationDataStructure} in which the
\ccc{TriangulationDSVertex} is defined/used.}

\ccNestedType{template<typename TDS2> Rebind_TDS}{This nested template
class must define a nested type \ccc{Other} which is the rebound vertex
class template, that is: \ccc{Other == TriangulationDSVertex<TDS2>}.}

\ccCreation	
\ccCreationVariable{v}

\ccConstructor{TriangulationDSVertex();}{The default constructor (no incident
full cell is set).}
\ccGlue
\ccConstructor{TriangulationDSVertex(Full_cell_handle c);}{Sets the incident
full cell to \ccc{c}. \ccPrecond \ccc{c} must not be the default-constructed
\ccc{Full_cell_handle}.}

\ccOperations

\ccMethod{void set_full_cell(Full_cell_handle c);}{Set \ccc{c} as the vertex's
incident full cell. \ccPrecond \ccc{c} must not be the default-constructed
\ccc{Full_cell_handle}.}

\ccMethod{Full_cell_handle full_cell() const;}{Returns a handle to a
  full cell incident to the vertex.}

\ccHeading{Validity check}

\ccMethod{bool is_valid(bool verbose=false, int level=0) const;}{Performs any
desired test on a vertex. Al least, checks that the pointer to an incident
full cell is not the default constructed handle ({i.e.}, is not
\ccc{NULL}). The parameter \ccc{level} is not used, but can be used in derived
classes.}

% \ccHeading{Memory management}

% \ccMethod{void*   for_compact_container() const;}{}
% \ccGlue\ccMethod{void* & for_compact_container();}{}

% These member functions are required by the classes
% \ccc{Triangulation_data_structure<Dimensionality, TriangulationDSVertex, TriangulationDSFullCell>} and
% \ccc{Triangulation<TriangulationTraits, TriangulationDataStructure>} (and its derived classes) because they use
% the \cgal\ container class \ccc{Compact_container} to store their vertices and
% full cells. See the documentation of \ccc{Compact_container} for the exact
% requirements.

\ccHeading{Input/Output}

\ccFunction{template<class TriangulationDataStructure> istream& operator>>(istream & is,
        Triangulation_ds_vertex<TriangulationDataStructure> & v);}
{Reads (possible) non-combinatorial information about a vertex from the stream \ccc{is}
into \ccc{v}.}

\ccFunction{template<class TriangulationDataStructure> ostream& operator<<(ostream & os, const
        Triangulation_ds_vertex<TriangulationDataStructure> & v);}
{Writes (possible) non-combinatorial information about vertex \ccc{v} to the stream
    \ccc{os}.}

\ccSeeAlso

\ccc{TriangulationDSFullCell}\\
\ccc{TriangulationDataStructure}

\end{ccRefConcept}
