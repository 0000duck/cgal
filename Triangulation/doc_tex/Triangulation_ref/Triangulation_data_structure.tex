\begin{ccRefClass}{Triangulation_data_structure<Dimensionality, TDSVertex, TDSFullCell>}

\ccDefinition 
This class is used for storing the combinatorial information of a triangulation
of dimension $k\leq d$.

\ccInclude{CGAL/Triangulation_data_structure.h}

\ccParameters

\ccc{Dimensionality} can be either \begin{itemize}

\item \ccPureGlobalScope\ccc{Dimension_tag<d>} for some integer \ccc{d}. This
indicates that the pure complex can store simplices of dimension at most
\ccc{d}. The maximum dimension \ccc{d} is known by the compiler, which
triggers some optimizations. Or

\item \ccPureGlobalScope\ccc{Dynamic_dimension_tag}. In this case, the maximum
dimension of the simplices is passed as an integer argument to an instance
constructor (see \ccc{TriangulationDataStructure}).\end{itemize}

\ccc{TDSVertex} is the class to be used as the base \ccc{Vertex} type in the
triangulation data structure. It must be a model of the concept
\ccc{TriangulationDSVertex}. The class template \ccRefName\ accepts that no
second parameter be specified. It also accepts the tag \ccc{CGAL::Default} as
second parameter. In both cases, \ccc{TDSVertex} defaults to
\ccc{CGAL::Triangulation_ds_vertex<>}.


\ccc{TDSFullCell} is the class to be used as the base \ccc{Full_cell} type in
the triangulation data structure. It must be a model of the concept
\ccc{TriangulationDSFullCell}. The class template \ccRefName\ accepts that no
third parameter be specified. It also accepts the tag \ccc{CGAL::Default} as
third parameter. In both cases, \ccc{TDSFullCell} defaults to
\ccc{CGAL::Triangulation_ds_full_cell<>}.

\ccIsModel

\ccc{TriangulationDataStructure}.

\ccCreationVariable{t}

\begin{ccAdvanced}

\ccTypes

\ccNestedType{Full_cell_data}
{A data member of type \ccc{Full_cell_data} is stored in every full cell (models
of the concept \ccc{TriangulationDSFullCell}).  It is used to mark some full
cells, during modifications of the triangulation data structure.}

\ccHeading{Vertex insertion} % - - - - - - - - - - - - - - - - - - INSERTIONS

\ccMethod{template< OutputIterator > Full_cell_handle insert_in_tagged_hole(
        Vertex_handle v, Facet f, OutputIterator new_simplices);}
{A set \ccc{C} of simplices satisfying the same condition as in method
\ccRefName\ccc{::insert_in_hole()} is assumed to be marked. This
method creates new simplices from \ccc{Vertex} v to the boundary of \ccc{C}.
The boundary is recognized by checking the mark of the simplices.
This method is used by \ccRefName\ccc{::insert_in_hole()}.}

\end{ccAdvanced}

\end{ccRefClass}
