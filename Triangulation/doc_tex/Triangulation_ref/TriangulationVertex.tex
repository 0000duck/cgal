\begin{ccRefConcept}{TriangulationVertex}

\ccDefinition

The concept \ccRefName\ describes the requirements on the type used by the
class \ccc{Triangulation<TriangulationTraits, TriangulationDataStructure>}, and its derived classes, to
represent a vertex.

\ccRefines

\ccc{TriangulationDSVertex}

We only list below the additional specific requirements of \ccRefName.

Compared to \ccc{TriangulationDSVertex}, the main difference is the addition of
an embedding of the vertex into a geometric point.

\ccHasModels

\ccc{CGAL::Triangulation_vertex<TriangulationTraits, Data, TriangulationDSVertex>}

\ccTypes

\ccNestedType{Point}{The type of the point stored in the vertex. It must be
the same as the point type \ccc{TriangulationTraits::Point} (or its refined
concepts) when the \ccc{TriangulationVertex} is used in the class
\ccc{Triangulation<TriangulationTraits, TriangulationDataStructure>} (or its derived classes).}

\ccCreation
\ccCreationVariable{v}

\ccConstructor{TriangulationVertex(Full_cell_handle c, const Point & p);}%
{Constructs a vertex with incident full cell \ccc{c}. The vertex is embedded at point \ccc{p}.}

\ccGlue\ccConstructor{TriangulationVertex(const Point & p);}%
{Same as above, but without incident full cell.}%{ (Who would have guessed?)}

\ccGlue\ccConstructor{TriangulationVertex();}%
{Same as above, but with a default-constructed \ccc{Point}.}

\ccOperations

\ccMethod{void set_point(const Point & p);}%
{The parameter \ccc{p} becomes the new geometrical position of the vertex.}

\ccMethod{const Point & point() const;}%
{Returns the vertex's position.}

\ccHeading{Input/Output}

\ccFunction{istream & operator>>(istream & is, TriangulationVertex & v);}%
{Inputs the non-combinatorial information given by the vertex, {i.e.},
the point and other possible information.}

\ccFunction{ostream & operator<<(ostream & os, const TriangulationVertex & v);}%
{Outputs the non-combinatorial information given by the vertex, {i.e.},
the point and other possible information.}

\ccSeeAlso

\ccc{Triangulation_vertex<TriangulationTraits, Data, TriangulationDSVertex>}\\
\ccc{TriangulationFullCell}\\
\ccc{Triangulation<TriangulationTraits, TriangulationDataStructure>}

\end{ccRefConcept}
