\begin{ccRefConcept}{TriangulationTraits}

\ccDefinition

The concept \ccRefName\ describes the various types and functions that a class
has to provide as the first parameter (\ccc{TrTraits}) to the class template
\ccc{Triangulation<TrTraits, TDS>}. It brings the geometric ingredient to the
definition of a triangulation, while the combinatorial ingredient is brought by
the second template parameter, \ccc{TDS}.

Inserting a range of points in a triangulation is optimized using
spatial sorting, thus besides the requirements below, 
a class provided as \ccc{TrTraits} should also satisfies  the concept 
\ccc{SpatialSortingTraits_d}.

\ccTypes

\ccNestedType{Point_d}%
{A type representing a point in euclidean space.}

\ccNestedType{Orientation_d}{A predicate object that must provide the
templated operator\\\ccc{template<typename ForwardIterator> Orientation
operator()(ForwardIterator start, ForwardIterator end)}.\\The operator returns
\ccc{POSITIVE}, if the simplex defined by the points in the range \ccc{[start,
end)} has a positive orientation, \emph{i.e.}, the last point \ccc{p} of the
range lies on the positive side of the oriented hyperplane \ccc{h} defined by
the other points. It returns \ccc{NEGATIVE} if \ccc{p} lies on the negative
side of \ccc{h} and returns \ccc{COPLANAR} if \ccc{p} lies on \ccc{h}.
%\\ The
%range's size must of course be one more than the dimension of the euclidean
%space the points live in.
 \ccPrecond The number of points in the range
must be equal to one more than the dimension of the euclidean space the points
live in, \emph{i.e.}, \ccc{std::distance(start,end)} is equal to \ccc{start->dimension()+1}.}

\ccNestedType{Coaffine_orientation_d}{Same as above, but the operator can use
less points as input. The orientation is taken for some
arbitrary---\textbf{but consistent}---orientation of the affine subspace
spanned by the input points. By consistent, it is meant that an affine
subspace will always have the same orientation whatever the set of points
spanning it. \ccPrecond
$2\leq$\ccc{std::distance(start,end)}$\leq$\ccc{start->dimension()+1}.}

\ccNestedType{Compare_lexicographically_d}{A predicate object that must
provide the operator\\\ccc{Comparison_result operator()(const Point_d & p,
const Point_d & q)}.\\The operator returns \ccc{SMALLER} if \ccc{p} is
lexicographically smaller than point \ccc{q}, \ccc{EQUAL} if both points are
the same and \ccc{LARGER} otherwise.}

\ccNestedType{Contained_in_affine_hull_d}{A predicate object that must provide
the templated operator\\\ccc{template<typename ForwardIterator> bool
operator()(ForwardIterator start, ForwardIterator end, const Point_d &
p)}.\\The operator returns \ccc{true} if and only if point \ccc{p} is
contained in the affine space spanned by the points in the range \ccc{[start,
end)}. That affine space is also called the \textbf{affine hull} of the points
in the range.}

\ccNestedType{Affinely_independent_d}{A predicate object that must provide the
templated operator\\\ccc{template<typename ForwardIterator> bool
operator()(ForwardIterator start, ForwardIterator end)}.\\The operator returns
\ccc{true} if and only if the dimension of the affine hull of the points in
the range \ccc{R=[start, end)} is one less than the number of points in
\ccc{R}.}

\ccNestedType{Center_of_sphere_d}{A construction object that must
provide the templated operator\\\ccc{template<typename ForwardIterator> bool
operator()(ForwardIterator start, ForwardIterator end)}.\\The operator
constructs the center of the sphere circumscribing the points in
the range \ccc{R=[start, end)}. \ccPrecond The number of points in the range
must be equal to one more than the dimension of the euclidean space the points
live in, \emph{i.e.}, \ccc{std::distance(start,end)} is equal to \ccc{start->dimension()+1}.}

\ccCreation
\ccCreationVariable{traits}

\ccConstructor{TriangulationTraits();}{The default constructor.}

\ccOperations

The following methods permit access to the traits class's predicates:

\ccMethod{Orientation_d orientation_d_object() const;}%
{}
\ccGlue
\ccMethod{Coaffine_orientation_d coaffine_orientation_d_object() const;}%
{}
\ccGlue
\ccMethod{Compare_lexicographically_d compare_lexicographically_d_object()
const;}%
{}
\ccGlue
\ccMethod{Contained_in_affine_hull_d contained_in_affine_hull_d_object()
const;}%
{}
\ccGlue
\ccMethod{ Affinely_independent_d affinely_independent_d_object() const;}%
{}
\ccGlue
\ccMethod{ Center_of_sphere_d center_of_sphere_d_object()
const;}%
{}

\ccHasModels

\ccc{Cartesian_d<FT, Dim, LA>},\\
\ccc{Simple_cartesian_d<FT, Dim, LA>},\\
\ccc{Filtered_kernel_d<K>} (recommended).
\note{a revoir en fonction des disponibilitees}

\ccSeeAlso

\ccc{DelaunayTriangulationTraits}


\end{ccRefConcept}
