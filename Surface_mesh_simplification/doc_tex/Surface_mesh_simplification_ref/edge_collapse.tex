%% Copyright (c) 2004  SciSoft.  All rights reserved.
%%
%% This file is part of CGAL (www.cgal.org); you may redistribute it under
%% the terms of the Q Public License version 1.0.
%% See the file LICENSE.QPL distributed with CGAL.
%%
%% Licensees holding a valid commercial license may use this file in
%% accordance with the commercial license agreement provided with the software.
%%
%% This file is provided AS IS with NO WARRANTY OF ANY KIND, INCLUDING THE
%% WARRANTY OF DESIGN, MERCHANTABILITY AND FITNESS FOR A PARTICULAR PURPOSE.
%%
%% 
%%
%% Author(s)     : Fernando Cacciola <fernando_cacciola@hotmail.com>


\begin{ccRefFunction}{Surface_mesh_simplification::edge_collapse}

%% add template arg's if necessary

%% \ccHtmlCrossLink{}     %% add further rules for cross referencing links
%% \ccHtmlIndexC[class]{} %% add further index entries
\ccDefinition

The function \ccRefName\ simplifies in-place a triangulated surface mesh by iteratively collapsing edges.

\ccInclude{CGAL/Surface_mesh_simplification/edge_collapse.h}

\ccFunction
{
template<class EdgeCollapsableMesh,class StopPredicate, class P, class T, class R>
int edge_collapse ( EdgeCollapsableMesh&           surface
                  , StopPredicate           const& should_stop
                  , sms_named_params<P,T,R> const& named_parameters
                  ) ;
}
{Simplifies \ccc{surface} in-place by collapsing edges, and returns
the number of edges effectively removed.
}


\ccHeading{Non-named parameters}

\ccc{surface} defines the type of the surface. 
It must be a model of the \ccc{EdgeCollapsableMesh} concept.

\ccc{should_stop} defines the type of the stop-condition policy.
It must be a model of the \ccc{StopPredicate} concept.

\ccHeading{Named parameters}

\ccc{named_parameters} holds the list of all the additional parameters 
used by the \ccc{edge_collapse} function (including default parameters).

The named parameters list is a composition of function calls, whose names corresponds
to the name of the parameter, separated by a dot ($.$), as in:

\ccc{param0_name(actual_param0).param1_name(actual_param1)}

each named parameter function corresponds to one named parameter.

All named parameters have default values and the order of the named parameters 
is irrelevant so you only need to compose those for which the default
is inappropriate.

\subsubsection*{vertex\_point\_map(VertexPointMap vpm)}

Provides lvalue access to the point of a vertex of the surface.
\ccc{VertexPointMap} must be an 
\ccAnchor{http://www.boost.org/libs/property_map/LvaluePropertyMap.html}{LValuePropertyMap} 
whose \ccc{key_type} is
\ccc{boost::graph_traits<EdgeCollapsableMesh>::vertex_descriptor}
and whose \ccc{value_type} is 
\ccc{boost::halfedge_graph_traits<EdgeCollapsableMesh>::Point}

\textbf{Default}: the property map obtained by calling \ccc{get(vertex_point,surface)}.



\subsubsection*{edge\_index\_map(EdgeIndexMap eim)} 

Maps each {\em directed} edge in the surface into an integer number
in the range \ccc{[0,num_edges(surface))}.

\ccc{EdgeIndexMap} must be a either a
\ccAnchor{http://www.boost.org/libs/property_map/ReadablePropertyMap.html}{ReadablePropertyMap} 
whose \ccc{key_type} is
\ccc{boost::graph_traits<EdgeCollapsableMesh const>::edge_descriptor}
and whose \ccc{value_type} is 
\ccc{boost::graph_traits<EdgeCollapsableMesh>::size_type},
or the same as \ccc{EdgeCollapsableMesh}, in which case 
\ccc{eim} is in fact the surface to be simplified.

If \ccc{eim} is a valid property map, it is
passed as the argument, but if it is the surface, a
temporary property map is passed instead.
Such temporary map externally relates each edge 
in the surface (\ccc{eim}) to its index,
without requiring any user intervention or any special
provision in the surface edge.

\textbf{Default}: the property map obtained by calling \ccc{get(edge_index,surface)}.

\subsubsection*{edge\_is\_border\_map(EdgeIsBorderMap ebm)} 

Maps each {\em directed} edge in the surface into a Boolean value
which indicates if the edge belongs to the boundary of the surface
(facing the outside).
\ccc{EdgeIsBorderMap} must be a 
\ccAnchor{http://www.boost.org/libs/property_map/ReadablePropertyMap.html}{ReadablePropertyMap} 
whose \ccc{key_type} is
\ccc{boost::graph_traits<EdgeCollapsableMesh const>::edge_descriptor}
and whose \ccc{value_type} is \ccc{bool}.

\textbf{Default}: the property map obtained by calling \ccc{get(edge_is_border,surface)}.


\subsubsection*{vertex\_is\_fixed\_map(VertexIsFixedMap vfm)}


Maps each vertex in the surface into a boolean value
which indicates if the vertex is fixed and cannot be modified 
by the \ccc{edge_collapse} function.\\
\ccc{VertexIsFixedMap} must be a 
\ccAnchor{http://www.boost.org/libs/property_map/ReadablePropertyMap.html}{ReadablePropertyMap} 
whose \ccc{key_type} is
\ccc{boost::graph_traits<EdgeCollapsableMesh const>::vertex_descriptor}
and whose \ccc{value_type} is \ccc{bool}.

\textbf{Default}: 
\ccc{Vertex_is_fixed_property_map_always_false<EdgeCollapsableMesh>()}.


\subsubsection*{set\_cache(SetCache sc)} 

The policy which indicates the caching level used 
by the \ccc{edge_collapse} function.\\
The type of \ccc{sc} must be a model of the \ccc{SetCache} concept.

\textbf{Default}: 
\ccc{CGAL::Surface_mesh_simplification::LindstromTurk_set_cost_cache<EdgeCollapsableMesh>()}


\subsubsection*{get\_cost(GetCost gc)}


The policy which returns the collapse cost for an edge.\\
The type of \ccc{gc} must be a model of the \ccc{GetCost} concept.

\textbf{Default}: 
\ccc{CGAL::Surface_mesh_simplification::Cached_cost<EdgeCollapsableMesh>}.


\subsubsection*{get\_cost\_params(GetCost::Params const* gp)}

The model-specific parameters to \ccc{get_cost}.

\textbf{Default}: \ccc{NULL}.


\subsubsection*{get\_placement(GetPlacement gp)}

The policy which returns the placement (position of the replacemet vertex)
for an edge.\\
The type of \ccc{gp} must be a model of the \ccc{GetPlacement} concept.

\textbf{Default}: 
\ccc{CGAL::Surface_mesh_simplification::LindstromTurk_placement<EdgeCollapsableMesh>}



\subsubsection*{get\_placement\_params(GetPlacement::Params const* gpp)}

The model-specific parameters to \ccc{get_placement}.

\textbf{Default}: A pointer to a local default constructed instance of \ccc{CGAL::LindstromTurk_params}.



\subsubsection*{visitor(EdgeCollapseSimplificationVisitor const* v)}

The visitor that is called by the \ccc{edge_collapse} function
in certain points to allow the user to track the simplification process.\\
The type of object pointed to by \ccc{v} must be a model of the \ccc{EdgeCollapseSimplificationVisitor} concept.

\textbf{Default}: \ccc{NULL}.

All these functions naming parameters are defined in
\ccc{namespace CGAL}. Being non-member functions, they could clash
with equally named functions in some other namespace. If that happens
simply qualify the named parameter with \ccc{CGAL::} as shown in the
examples in the user manual.

\ccHeading{Semantics}

The simplification process continues until the \ccc{should_stop} policy returns \ccc{true}
or the surface cannot be simplified any further due to topological constraints.

\ccc{vertex_is_fixed_map} is used to indicate that some edges should not be
collapsed: edges incident upon fixed vertices are not collapsed.

\ccc{set_cache}, along with \ccc{get_cost} and \ccc{get_placement},
are the policies which control the {\em cost-strategy}, that is, 
the order in which edges are collapsed and the replacement vertex is positioned.\\
\ccc{get_cost_params} and \ccc{get_placement_params} are the runtime 
parameters used by the these policies. They can be null pointers
if the policies do not need them. This strategy is the driving factor that determines the accuracy of the
simplified surface with respect to the original.

\ccc{visitor} is an optional object (can be null) which can be used
to keep track of the simplification process.

\end{ccRefFunction}


% +------------------------------------------------------------------------+
%%RefPage: end of main body, begin of footer
% EOF
% +------------------------------------------------------------------------+

