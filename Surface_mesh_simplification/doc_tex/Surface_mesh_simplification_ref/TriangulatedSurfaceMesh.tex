%% Copyright (c) 2004  SciSoft.  All rights reserved.
%%
%% This file is part of CGAL (www.cgal.org); you may redistribute it under
%% the terms of the Q Public License version 1.0.
%% See the file LICENSE.QPL distributed with CGAL.
%%
%% Licensees holding a valid commercial license may use this file in
%% accordance with the commercial license agreement provided with the software.
%%
%% This file is provided AS IS with NO WARRANTY OF ANY KIND, INCLUDING THE
%% WARRANTY OF DESIGN, MERCHANTABILITY AND FITNESS FOR A PARTICULAR PURPOSE.
%%
%% 
%%
%% Author(s)     : Fernando Cacciola <fernando_cacciola@hotmail.com>

\begin{ccRefConcept}{TriangulatedSurfaceMesh}

%% \ccHtmlCrossLink{}     %% add further rules for cross referencing links
%% \ccHtmlIndexC[concept]{} %% add further index entries

\ccDefinition

The concept \ccRefName\ describes the requirements for the Triangulated Surface Mesh data structure.

\ccRefines
\ccc{HalfedgeGraph}

\ccHeading{Methods}
\ccc{EdgeIsBorderPropertyMap}

\ccOperations

  \ccFunction
  {template<class Graph>
  void
  collapse_triangulation_edge(typename boost::graph_traits<Graph>::edge_descriptor p_q
                             ,typename boost::graph_traits<Graph>::edge_descriptor p_t
                             ,typename boost::graph_traits<Graph>::edge_descriptor q_b
                             ,Graph& surface
                             );}  {Collapses the undirected edge p-q replacing it with one single vertex,
This function assumes the surface to be an {\em open 2-manifold} and it must
  never be called in a context were the collape would result in the surface 
  becoming a {\em non-manifold}. Thus, the following preconditions hold:\\
%
  source(p-q) $==$ source(p-t).\\
  target(p-q) $==$ source(q-b).\\
  One of p-q or q-p is not a border edge.\\
  p-t is a null edge iff p-q is a border edge.\\
  q-b is a null edge iff q-p is a border edge.\\
  if p-t is non-null, next(t-p) $==$ p-q.\\
  if q-b is non-null, next(b-q) $==$ q-p.\\
%
  The net effect of the operation is equivalent to removing one of the vertices
  and re-triangulating the resulting facet.\\
%
  The actual collapse operation is required to remove at most 
  2 triangles, 3 {\em undirected} edges and 1 vertex, as follows:\\
  %
  The triangle above p-q is called top-triangle.\\
  The triangle below p-q is called bottom-triangle.\\
  The triangle adjacent to the top-triangle along p-t is called top-left-triangle.\\
  The triangle adjacent to the bottom-triangle along q-b is called bottom-right-triangle.\\
  %
  The preconditions guarantee that the top-left-triangle and top-bottom-triangle
  are disjoint, that t-p and p-q are both incident on the top-triangle and that
  b-q and q-p are both incident on the bottom-triangle.\\
%      
  Either p or q is removed, which in turn removes the pair of opposing egdes 
  p-q and q-p (that is, the undirected edge p-q).\\
%
  If there is a top-triangle, that is, p-q is not a border edge and p-t $!=$ null, it is removed.\\
  If there is a bottom triange, that is, q-p is not a border edge and q-b $!=$ null, it is removed.\\
  (at least one of these triangles exist so at least 1 triangle, and at most 2, are removed).\\
%  
  Removing the top-triangle neccesarily removes the pair of opposing edges p-t and t-p.\\
  Removing the bottom-triangle neccesarily removes the pair of opposing edges q-b and b-q.\\
%      
  If the bottom-triangle is removed and neither the top-triangle nor
  the bottom-right-triangle exist, the vertex removed is q. In all other
  cases, the vertex removed is p.\\
  %
  If any triangle (other than the top and bottom, which are removed) is defined by
  the removed vertex, it must be redefined with the other vertex, but it cannot
  be replaced by a new triangle. (A model must follow the as-is rule: if it really
  needs to do replace triangles, the new triangles must aquire any property 
  that could be internally or externally attached to the old triangle)\\
 %
  The function returns the vertex that is not removed.}

\ccHasModels
\ccRefIdfierPage{CGAL::Polyhedron_3<Traits>}
  (through the various traits and function overloads that externally adapt a \cgal\
  polyhedron to the involved concepts)

\end{ccRefConcept}

% +------------------------------------------------------------------------+
%%RefPage: end of main body, begin of footer
% EOF
% +------------------------------------------------------------------------+
