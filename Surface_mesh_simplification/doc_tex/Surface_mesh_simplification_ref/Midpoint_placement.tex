%% Copyright (c) 2004  SciSoft.  All rights reserved.
%%
%% This file is part of CGAL (www.cgal.org); you may redistribute it under
%% the terms of the Q Public License version 1.0.
%% See the file LICENSE.QPL distributed with CGAL.
%%
%% Licensees holding a valid commercial license may use this file in
%% accordance with the commercial license agreement provided with the software.
%%
%% This file is provided AS IS with NO WARRANTY OF ANY KIND, INCLUDING THE
%% WARRANTY OF DESIGN, MERCHANTABILITY AND FITNESS FOR A PARTICULAR PURPOSE.
%%
%% 
%%
%% Author(s)     : Fernando Cacciola <fernando_cacciola@hotmail.com>


\begin{ccRefFunctionObjectClass}{Surface_mesh_simplification::Midpoint_placement<ECM>}

%% add template arg's if necessary

%% \ccHtmlCrossLink{}     %% add further rules for cross referencing links
%% \ccHtmlIndexC[class]{} %% add further index entries
\ccCreationVariable{d}
\ccDefinition

The class \ccRefName\ provides a model for the \ccc{GetPlacement} concept. 
It computes the placement as the midpoint position along the edge.

The class \ccRefName\ has one template arguments: the type of surface being simplified. 
It be a model of the \ccc{EdgeCollapsableMesh} concept.

\ccInclude{CGAL/Surface_mesh_simplification/Policies/Edge_collapse/Midpoint_placement.h}

\ccIsModel
\ccc{GetPlacement}

\ccCreation
\ccCreationVariable{gp}  
\ccConstructor{Midpoint_placement<ECM>();}{Default constructor} 

\ccHeading{Methods}
\ccTagFullDeclarations
\ccCreationVariable{gp}  
  \ccMemberFunction{result_type operator()( Profile const& edge_profile ) const; }
  {Returns the {\em placement} (vertex position) as the midpoint between 
  the points of the source and target vertices
  (\ccc{profile.p0()} and \ccc{profile.p1()})}

\end{ccRefFunctionObjectClass}

% +------------------------------------------------------------------------+
%%RefPage: end of main body, begin of footer
% EOF
% +------------------------------------------------------------------------+

