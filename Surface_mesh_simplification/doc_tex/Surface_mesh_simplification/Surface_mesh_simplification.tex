%% Copyright (c) 2004  SciSoft.  All rights reserved.
%%
%% This file is part of CGAL (www.cgal.org); you may redistribute it under
%% the terms of the Q Public License version 1.0.
%% See the file LICENSE.QPL distributed with CGAL.
%%
%% Licensees holding a valid commercial license may use this file in
%% accordance with the commercial license agreement provided with the software.
%%
%% This file is provided AS IS with NO WARRANTY OF ANY KIND, INCLUDING THE
%% WARRANTY OF DESIGN, MERCHANTABILITY AND FITNESS FOR A PARTICULAR PURPOSE.
%%
%% 
%%
%% Author(s)     : Fernando Cacciola <fernando_cacciola@hotmail.com>

\section{Introduction}

This \cgal\ package provides an algorithm for the simplification of triangulated surface meshes.

Simplification is the process of reducing the number of triangles used in the surface but keeping the overall shape, volumen and boundaries preserved as much as possible. It is the opposite of subdivision.

%% figures of simplified surfaces

The algorithm presented here can simplify any {\em open oriented 2-manifold surface}, with any number of connected components, with or without boundaries (border or holes) and handles (arbitrary genus), using a method known as {\em edge collapse}.\\
Roughly speaking, the method consists of iteratively replacing an edge with a single vertex, removing 2 triangles per collapse.

%% figure of edge collapse

Edges are collapsed according to a priority given by a user-supplied {\em cost} function, and the coordinates of the replacing vertex are determined by another user-supplied {\em placement} function. The algorithm terminates when a user-supplied {\em stop predicate} is met, such as reaching the desired number of edges.

The algorithm implemented here is generic in the sense that it does not require the surface to be of a particular type. Instead, it defines the concept of a \ccc{EdgeCollapsableMesh} and any surface that is a model of that concept can be simplified. Furthermore, the concept is defined not in terms of a monolithic class, but in terms of a set of functions and traits, making it easy to adapt any concrete surface type. In particular, the concept definition follows the design of the 
 \ccAnchor{http://www.boost.org/libs/graph/doc/index.html}{ Boost Graph Library ({\sc Bgl})}.

The design is \ccAnchor{http://en.wikipedia.org/wiki/Policy-based_design}{{\em policy-based}},
meaning that you can customize some aspects of the process by passing a set of {\em policy objects}. Each policy object specifies a particular aspect of the algorithm, such as how edges are selected and where the replacement vertex is placed. All policies have a sensible default so you don't need to learn about them unless you need a non-default behaviour. Furthermore, the API uses the so-called \ccc{named-parameters} technique which allows you to pass only the relevant parameters, in any order, ommitting
those parameters whose default is appropriate.

\section{Overview of the Simplification Process}

The free function that implements the simplifcation algorithm takes not only the surface and the desired stop predicate but a number of additional parameters which control and monitor the simplification process. This section briefly describes the process in order to set the background for the discussion of algorithm parameters.

An edge collapse operation replaces the edge with a vertex, removing the two incident triangles (hence two additional edges, one from each triangle). Therefore, each edge collapse decreases the number of triangles, edges, and vertices by 2, 3, and 1, respectively (if the collapsed edge is a border, by 1,2, and 1, respectively, since only one triangle is removed).

Naturally, the surface that results from an edge collapse deviates from the initial surface by some amount, and since the goal of simplification is to reduce the number of triangles while retaining the overall look of the surface as much as possible, it is necessary to measure such a deviation. Some methods attempt to measure the total deviation from the initial surface to the completely simplified surface, for example, by tracking an accumulated error while keeping a history of the simplification changes. Other methods, like the one implemented in this package, attempt to measure only a step {\em cost} (the local deviation introduced by a single simplification step) and plan the entire process as a sequence of steps of increasing cost. 

Global error tracking methods produce highly accurate simplifications but take up a lot of additional space. Cost-driven methods, like the one in this package, produce slightly less accurate simplifications but take up much less additional space, even none in some cases.

The cost-driven method implemented in this package is mainly based on \cite[Memoryless simplification]{cgal:lt-fmeps-98,cgal:lt-ems-99}, with contributions from \cite[Mesh Optimization]{hddms-mo-93}, \cite[Surface simplification using quadric error metrics]{gh-ssqem-97}
and \cite[Topology preserving edge contraction]{degn-tpec-98}.

The algorithm proceeds in two stages. In the collection stage, an initial {\em collapse cost} is assigned to each and every edge in the surface. In the collapsing stage, edges are {\em processed} in order of increasing cost. Some processed edges are {\em collapsed} while some are just discarded. Collapsed edges are replaced by a vertex and the collapse cost of all the edges now incident on the replacement vertex is recalculated, affecting the order of the edges left to process.

Not all edges selected for processing are collapsed. A processed edge can be discarded without being collapsed for one of three reasons: its cost could not be computed (for whatever reason), collapsing the edge would result in an inconsistent surface, or the edge is incident upon a vertex which the user marked as fixed.

If an edge is effectively collapsed, is replaced by a vertex whose position, called {\em placement}, must be choosen carefully to avoid too much deviation from the initial surface. In many cases, the collapse cost is a function of the vertex position, so the placement might be calculated each time the cost is calculated and not just when an edge is effectively collapsed.

The algorithm presented in \cite{gh-ssqem-97} collapses arbitrary vertex pairs and not only edges by considering certain vertices as forming a pseudo-edge, proceeding to collapse both edges and pseudo-edges in the same way as in \cite{cgal:lt-fmeps-98,cgal:lt-ems-99}, which is the algorithm implemented here. Since it is always possible to physically add and remove those pusedo-edges (and the required pseudo-triangles needed to keep the surface structurally valid), this \cgal\ package can be used in the implementation of \cite{gh-ssqem-97}, provided the appropriate cost-strategy policies are implemented (they are not implemented in the current version)

\section{Cost Strategy}

The specific way in which the collapse cost and vertex placement is
calculated is called the {\em cost strategy}. The user can choose 
different strategies in the form of policies and related parameters,
passed to the algorithm.
 
The current version of the package version provides two strategies: 
Lindstrom-Turk (the default) and edge-length cost with a midpoint placement 
(much faster but less accurate).

\subsection{Lindstrom-Turk cost and placement strategy\label{SurfaceMeshSimplification:LindstromTurkStrategy}}

\subsection{Cache Level}

The cost for each edge is initially fixed and changes only in those edges 
affected by a neighboring collapse. A given edge can be expected to be 
affected by a neighboring collapse only a small number of times 
(in the order of the average vertex degree), so the total number of 
cost {\em computations} (and recomputations) is a small factor of
the total number of edges. 
On the other hand, edges are processed in order of increasing cost, and as is 
the case with any sorting situation, the cost of any given edge must be accessed 
several times while it is compared against the cost of other edges, even if it 
remains constant. That is, costs are {\em queried} several times more than are computed 
and updated, so it make sense to cache the cost of an edge to avoid recomputing it.

The placement for each edge is queried only when the edge is effectively collapsed, 
however, most cost strategies are based on the placement, so each cost computation 
typically involves a hidden placement computation. Therefore, it might also make sense
to cache not only the cost but also the placement (which is a Point object).

The cost for all edges are computed at least once, and on avergage updated a few times,
but the placement is queried only when the edge is effectively collapsed. Caching only
the cost as opposed to caching both the cost and the placement makes a small difference
in the total running time, while, not caching the cost at all causes a significant slowdown.
Still, caching the cost and/or placement in the edge adds to the size of the surface 
a significant amount. For example, if the cost is of type {\tt double}, each million edges
require about 7Mb of additional storage for the cost, and 30Mb for the cost and placement.
It might be better in a specific context to avoid any caching altoghether, even 
at the expense of increased running time. The most sensible default choice is to cache
only the cost.

The cost strategy used by the algorithm is selected by means of 3 policies: 
\ccc{SetCache}, \ccc{GetCost} and \ccc{GetPlacement}. 

The \ccc{SetCache} policy controls whether the cache is empty (\ccc{NoCache}), 
stores the cost but not the placement(\ccc{CostCache}), or stores both 
(\ccc{CostAndPlacementCache)}.

The \ccc{GetCost} and \ccc{GetPlacement} policies are called each time
the algorithm needs to access the cost or placement for an edge.
If an empty cache is used, these functions must effectively
calculate the values and return it, but otherwise they can 
(and normally would) simply return the cached value.

If \ccc{SetCache} actually stores any value in the cache, it computes
such a value by calling itself to a GetCost and/or a GetPlacement
policy (other than the ones passed to the algorithm)

The three policies are related in that if \ccc{SetCache}
assigns a \ccc{NoCache} record, the \ccc{GetCost} and \ccc{GetPlacement} policies
must be actually computing the values, while if \ccc{SetCache} sets a 
\ccc{CostCache} or a \ccc{CostAndPlacementCache} record, 
the other 2 policies would just return the cached values.

\section{API}

\subsection{API Overview}

The simplification algorithm is implemented as the free template function 
\ccc{edge_collapse}.

\subsubsection{Main paramters}

There are two main parameters to the algorithm: the surface to be simplified (in-place) and the stop predicate.

The surface to simplify must be a model of the \ccc{EdgeCollapsableMesh} concept. Many concrete surface types, such as \ccc{CGAL::Polyhedron_3}, become models of that concept via a technique known as 
{\em external adaptation} (see \ccAnchor{http://www.boost.org/libs/graph/doc/leda_conversion.html}{this Bgl page for details}). External adaptation is a way to add an interface to an object without coercing the type of the object (which happens when you adapt it by means of a wrapper). That is, the formal parameter to the \ccc{edge_collapse} function that implements the simplification is the concrete surface object itself, not an adaptor which delegates the functionality to the concrete type.

The stop predicate is called after each edge is selected for processing, {\em before} it is classfied as collapsable or not (thus before it is collapsed). If the stop predicate returns true the algorithm terminates.

\subsubsection{Named parameters}

The rest of the parameters are optional. 
A {\em named parameters technique}, like that used in the 
\ccAnchor{http://www.boost.org/libs/graph/doc/bgl_named_params.html}{Bgl algorithms},
is used to allow you to specify only those parameters which are really needed,
and by name, making the parameter ordering unimportant. 

One of the named parameters is a visitor object which can be used to track the simplification process. Some of the named parameters are property maps which supply information about the surface,
and the rest correspond to the cost-strategy policies. 

The following sample call briefly introduces all of the parameters.
Motice that the list of named parameters is really a composition
of function calls separated by a dot ($.$). You can ommit any of these named parameters, and list them in any order.

\begin{cprog}
int r = edge_collapse(surface
                     ,stop_predicate
                     
                     ,vertex_point_map   (vertex_point_map_arg)
                     .edge_index_map     (edge_index_map_arg)
                     .edge_is_border_map (edge_is_border_map_arg)
                     .vertex_is_fixed_map(vertex_is_fixed_map_arg)
                     
                     .set_cache          (set_cache_arg)
                     .get_cost           (get_cost_arg)
                     .get_placement      (get_placement_arg)
                     .cost_params        (cost_params_arg)
                     .placement_params   (placement_params_arg)
                     
                     .visitor            (visitor_arg)
                     );
\end{cprog}

\subsection{Examples}

The following examples introduce the API in increasing complexity. All the examples follow this general structure:

\ccIncludeExampleCode{Surface_mesh_simplification/general_form.cpp}

As you can see, each example varies from this general form in the code sections between the marks.
Each example is coded to be compilable so some neccessary parts are repeated in many of them. To simplify the exposition, repeated parts are described in detail only in the first example they appear.

\subsubsection{Example Using Default Parameters}

The following example illustrates the simplest of the cases.\\
It uses an {\em ordinary} Polyhedron, that is, a polyhedron without any special provision 
for the sake of the algorithm (which is possible as illustrated in the next example).\\
Many parameters are ommited, causing the algorithm to run in default mode:

\begin{enumerate}
\item \ccc{There are no fixed vertices in the surface.}
\item \ccc{The cost strategy is that of Lindstrom-Turk, computed with default parameters and caching only the cost (partial cache).}
\item \ccc{No visitor is used to track the algorithm progress.}
\end{enumerate}

\ccIncludeExampleCode{Surface_mesh_simplification/LT_edge_collapse_polyhedron.cpp}

\subsubsection{Example Using Enriched Polyhedron}

The following example is equivalent to the previous example but instead of using an external map to associate the extra pointer to each edge, it directly stores the pointer in the edge by enriching the polyhedron definition.

\ccIncludeExampleCode{Surface_mesh_simplification/LT_edge_collapse_enriched_polyhedron.cpp}

\subsubsection{Example using Non-Default Cost Strategy}

The following example shows how to use non-default cost strategy: the edge length as cost and the edge
midpoint as placement. It also shows how to specify that neither the cost nor the placement should be cached (to save memory).

\ccIncludeExampleCode{Surface_mesh_simplification/MP_edge_collapse_polyhedron.cpp}

\subsubsection{Example using a full cache for the Collapse Data}

The following example shows how to use a full collapse-data cache for efficiency. Using a full collapse data is slightly different for the default cost-strategy (Lindstrom-Turk) than for alternative strategies, so both cases are shown in the same example.

\ccIncludeExampleCode{Surface_mesh_simplification/Edge_collapse_fully_cached_polyhedron.cpp}

\subsubsection{Example using a Visitor}

The following example shows how to use a visitor to track the simplification process. It also shows how to setup the Lindstrom-Turk cost strategy but with a partial case (unlike a full cache as in the previous example). A partial cache is the default policy so you don't need to explicitely specify it unless, as in this case, you have to pass an argument further up in the list.

\ccIncludeExampleCode{Surface_mesh_simplification/LT_edge_collapse_visited_polyhedron.cpp}


\subsubsection{Example using Fixed Vertices via an external Property Map}

The following example shows how to fix vertices to prevent edges from being collapsed. In this case, an ordinary polyhedron is used so the is-fixed vertex flags are stored in an external map.

\ccIncludeExampleCode{Surface_mesh_simplification/LT_edge_collapse_with_fixed_vertices_polyhedron.cpp}

\subsubsection{Example using Fixed Vertices via an Enriched Polyhedron}

The following example, just like the previous example, shows how to fix vertices to prevent edges from being collapsed. In this case, however, an enriched polyhedron is defined which stores the is-fixed flag right in the vertex.

\ccIncludeExampleCode{Surface_mesh_simplification/LT_edge_collapse_with_fixed_vertices_enriched_polyhedron.cpp}

% +------------------------------------------------------------------------+
%%RefPage: end of main body, begin of sfsooter
% EOF
% +------------------------------------------------------------------------+


