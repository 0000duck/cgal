% +------------------------------------------------------------------------+
% | Reference manual page: Fixed_border_parameterizer_3.tex
% +------------------------------------------------------------------------+
% | 21.09.2005   Laurent Saboret, Pierre Alliez, Bruno Levy
% | Package: Parameterization
% |
\RCSdef{\RCSFixedborderparameterizerRev}{$Id$}
\RCSdefDate{\RCSFixedborderparameterizerDate}{$Date$}
% |
%%RefPage: end of header, begin of main body
% +------------------------------------------------------------------------+


\begin{ccRefClass}{Fixed_border_parameterizer_3}  %% add template arg's if necessary

%% \ccHtmlCrossLink{}     %% add further rules for cross referencing links
%% \ccHtmlIndexC[class]{} %% add further index entries


\ccDefinition

% The section below is automatically generated. Do not edit!
%START-AUTO(\ccDefinition)

The class Fixed\_border\_parameterizer\_3 is the base class of fixed border parameterization methods (Tutte, Floater, ...).

1 to 1 mapping is guaranteed if surface's border is mapped onto a convex polygon.

This class is a pure virtual class, thus cannot be instantiated. Anyway, it implements most of the parameterization algorithm parameterize(). Subclasses are Strategies \cite{cgal:ghjv-dpero-95} that modify the behavior of this algorithm:\begin{itemize}
\item They provide BorderParameterizer\_3 and SparseLinearAlgebraTraits\_d template parameters that make sense.\item They implement compute\_w\_ij() to compute w\_ij = (i,j) coefficient of matrix A for j neighbor vertex of i.\item They may implement an optimized version of is\_one\_to\_one\_mapping().\end{itemize}


\begin{description}
\item[Todo]Fixed\_border\_parameterizer\_3 should remove border vertices from the linear systems in order to have a symmetric definite positive matrix for Tutte Barycentric Mapping and Discrete Conformal Map algorithms.\end{description}

%END-AUTO(\ccDefinition)

\ccInclude{CGAL/Fixed_border_parameterizer_3.h}


\ccIsModel

% The section below is automatically generated. Do not edit!
%START-AUTO(\ccIsModel)

Model of the ParameterizerTraits\_3 concept (although you cannot instantiate this class).

%END-AUTO(\ccIsModel)


\ccParameters

The full template declaration is:

% The section below is automatically generated. Do not edit!
%START-AUTO(\ccParameters)

template$<$  \\
class ParameterizationMesh\_3,   \\
class BorderParameterizer\_3 = Circular\_border\_arc\_length\_parameterizer\_3$<$ParameterizationMesh\_3$>$,   \\
class SparseLinearAlgebraTraits\_d = OpenNL::DefaultLinearSolverTraits$<$typename ParameterizationMesh\_3::NT$>$$>$   \\
class Fixed\_border\_parameterizer\_3;

%END-AUTO(\ccParameters)


\ccTypes

% The section below is automatically generated. Do not edit!
%START-AUTO(\ccTypes)

\ccNestedType{Border_param}
{
Export BorderParameterizer\_3 template parameter.
}
\ccGlue
\ccNestedType{Sparse_LA}
{
Export SparseLinearAlgebraTraits\_d template parameter.
}
\ccGlue

%END-AUTO(\ccTypes)


\ccCreation
\ccCreationVariable{param}  %% choose variable name for \ccMethod

% The section below is automatically generated. Do not edit!
%START-AUTO(\ccCreation)

\ccConstructor{Fixed_border_parameterizer_3 (Border_param border_param = Border_param(), Sparse_LA sparse_la = Sparse_LA());}
{
Constructor.
}
\ccGlue
\begin{description}
\item[Parameters: ]
\begin{description}
\item[border\_param]Object that maps the surface's border to 2D space \item[sparse\_la]Traits object to access a sparse linear system \end{description}
\end{description}
\ccGlue

%END-AUTO(\ccCreation)


\ccOperations

% The section below is automatically generated. Do not edit!
%START-AUTO(\ccOperations)

\ccMethod{Parameterizer_traits_3< Adaptor >::Error_code parameterize (Adaptor * mesh);}
{
Compute a 1 to 1 mapping from a triangular 3D surface 'mesh' to a piece of the 2D space. The mapping is linear by pieces (linear in each triangle). The result is the (u,v) pair image of each vertex of the 3D surface.
Preconditions:\begin{itemize}
\item 'mesh' must be a surface with 1 connected component.\item 'mesh' must be a triangular mesh.\item the mesh border must be mapped onto a convex polygon. \end{itemize}
}
\ccGlue
\ccMethod{Parameterizer_traits_3< Adaptor >::Error_code check_parameterize_preconditions (Adaptor * mesh);}
{
Check parameterize() preconditions:\begin{itemize}
\item 'mesh' must be a surface with 1 connected component.\item 'mesh' must be a triangular mesh.\item the mesh border must be mapped onto a convex polygon. \end{itemize}
}
\ccGlue
\ccMethod{void initialize_system_from_mesh_border (Matrix * A, Vector * Bu, Vector * Bv, const Adaptor & mesh);}
{
Initialize A, Bu and Bv after border parameterization. Fill the border vertices' lines in both linear systems: {\em u = constant} and {\em v = constant}.
Preconditions:\begin{itemize}
\item vertices must be indexed.\item A, Bu and Bv must be allocated.\item border vertices must be parameterized. \end{itemize}
}
\ccGlue
\ccMethod{virtual NT compute_w_ij (const Adaptor & mesh, Vertex_const_handle main_vertex_v_i, Vertex_around_vertex_const_circulator neighbor_vertex_v_j);}
{
Compute w\_ij = (i,j) coefficient of matrix A for j neighbor vertex of i. Implementation note: Subclasses must at least implement compute\_w\_ij().
}
\ccGlue
\ccMethod{Parameterizer_traits_3< Adaptor >::Error_code setup_inner_vertex_relations (Matrix * A, Vector * Bu, Vector * Bv, const Adaptor & mesh, Vertex_const_handle vertex);}
{
Compute the line i of matrix A for i inner vertex:\begin{itemize}
\item call compute\_w\_ij() to compute the A coefficient w\_ij for each neighbor v\_j.\item compute w\_ii = - sum of w\_ijs.\end{itemize}
Preconditions:\begin{itemize}
\item vertices must be indexed.\item vertex i musn't be already parameterized.\item line i of A must contain only zeros. \end{itemize}
}
\ccGlue
\ccMethod{void set_mesh_uv_from_system (Adaptor * mesh, const Vector & Xu, const Vector & Xv);}
{
Copy Xu and Xv coordinates into the (u,v) pair of each surface vertex.
}
\ccGlue
\ccMethod{Parameterizer_traits_3< Adaptor >::Error_code check_parameterize_postconditions (const Adaptor & mesh, const Matrix & A, const Vector & Bu, const Vector & Bv);}
{
Check parameterize() postconditions:\begin{itemize}
\item {\em A$\ast$Xu = Bu} and {\em A$\ast$Xv = Bv} systems are solvable with a good conditioning.\item 3D -$>$ 2D mapping is 1 to 1. \end{itemize}
}
\ccGlue
\ccMethod{bool is_one_to_one_mapping (const Adaptor & mesh, const Matrix & A, const Vector & Bu, const Vector & Bv);}
{
Check if 3D -$>$ 2D mapping is 1 to 1. The default implementation checks each normal.
}
\ccGlue
\ccMethod{Border_param& get_border_parameterizer ();}
{
Get the object that maps the surface's border onto a 2D space.
}
\ccGlue
\ccMethod{Sparse_LA& get_linear_algebra_traits ();}
{
Get the sparse linear algebra (traits object to access the linear system).
}
\ccGlue

%END-AUTO(\ccOperations)


\ccSeeAlso

\ccRefIdfierPage{CGAL::Parameterizer_traits_3}  \\
\ccRefIdfierPage{CGAL::Barycentric_mapping_parameterizer_3}  \\
\ccRefIdfierPage{CGAL::Discrete_authalic_parameterizer_3}  \\
\ccRefIdfierPage{CGAL::Discrete_conformal_map_parameterizer_3}  \\
\ccRefIdfierPage{CGAL::LSCM_parameterizer_3}  \\
\ccRefIdfierPage{CGAL::Mean_value_coordinates_parameterizer_3}  \\


\end{ccRefClass}

% +------------------------------------------------------------------------+
%%RefPage: end of main body, begin of footer
% EOF
% +------------------------------------------------------------------------+

