% +------------------------------------------------------------------------+
% | Reference manual page: Taucs_symmetric_solver_traits.tex
% +------------------------------------------------------------------------+
% | 21.09.2005   Laurent Saboret, Pierre Alliez
% | Package: Parameterization
% |
\RCSdef{\RCSTaucssymmetricsolvertraitsRev}{$Revision$}
\RCSdefDate{\RCSTaucssymmetricsolvertraitsDate}{$Date$}
% |
%%RefPage: end of header, begin of main body
% +------------------------------------------------------------------------+


\begin{ccRefClass}{Taucs_symmetric_solver_traits}  %% add template arg's if necessary

%% \ccHtmlCrossLink{}     %% add further rules for cross referencing links
%% \ccHtmlIndexC[class]{} %% add further index entries


\ccDefinition

% The section below is automatically generated. Do not edit!
%START-AUTO(\ccDefinition)

The class Taucs\_symmetric\_solver\_traits is a traits class for solving SYMMETRIC DEFINIE POSITIVE sparse linear systems using TAUCS solvers family. The default solver is the Multifrontal Supernodal Cholesky Factorization.

\begin{description}
\item[Todo]Add to Taucs\_symmetric\_solver\_traits the ability to solve linear systems in the least squares sense.\end{description}

%END-AUTO(\ccDefinition)

\ccInclude{CGAL/Taucs_solver_traits.h}


\ccIsModel

% The section below is automatically generated. Do not edit!
%START-AUTO(\ccIsModel)

Model of the SparseLinearAlgebraTraits\_d concept.

%END-AUTO(\ccIsModel)


\ccParameters

The full template declaration is:

% The section below is automatically generated. Do not edit!
%START-AUTO(\ccParameters)

template$<$  \\
class T$>$   \\
class Taucs\_symmetric\_solver\_traits;

%END-AUTO(\ccParameters)


\ccTypes

% The section below is automatically generated. Do not edit!
%START-AUTO(\ccTypes)

\ccNestedType{Matrix}
{
}
\ccGlue
\ccNestedType{Vector}
{
}
\ccGlue
\ccNestedType{NT}
{
}
\ccGlue

%END-AUTO(\ccTypes)


\ccCreation
\ccCreationVariable{solver}  %% choose variable name for \ccMethod

% The section below is automatically generated. Do not edit!
%START-AUTO(\ccCreation)

\ccConstructor{Taucs_symmetric_solver_traits (const char * options[] = NULL, const void * arguments[] = NULL);}
{
Create a TAUCS sparse linear solver for SYMMETRIC DEFINIE POSITIVE matrices. The default solver is the Multifrontal Supernodal Cholesky Factorization. See taucs\_linsolve() documentation for the meaning of the 'options' and 'arguments' parameters. 
}
\ccGlue
\begin{description}
\item[Parameters: ]
\begin{description}
\item[options]must be persistent \item[arguments]must be persistent \end{description}
\end{description}
\ccGlue

%END-AUTO(\ccCreation)


\ccOperations

% The section below is automatically generated. Do not edit!
%START-AUTO(\ccOperations)

\ccMethod{bool linear_solver (const Matrix & A, const Vector & B, Vector & X, NT & D);}
{
Solve the sparse linear system {\em A$\ast$X = B}. Return true on success. The solution is then (1/D) $\ast$ X.
Preconditions:\begin{itemize}
\item A.row\_dimension() == B.dimension().\item A.column\_dimension() == X.dimension(). \end{itemize}
}
\ccGlue
\ccMethod{bool is_solvable (const Matrix & A, const Vector & B);}
{
Indicate if the linear system can be solved and if the matrix conditioning is good.
Preconditions:\begin{itemize}
\item A.row\_dimension() == B.dimension().\end{itemize}
}
\ccGlue
\begin{description}
\item[Todo]Implement Taucs\_symmetric\_solver\_traits::is\_solvable() by solving the system, then checking that $|$ $|$$|$A$\ast$X$|$$|$/$|$$|$B$|$$|$ - 1 $|$ $<$ epsilon. \end{description}
\ccGlue

%END-AUTO(\ccOperations)


\ccSeeAlso

\ccRefIdfierPage{CGAL::Taucs_solver_traits}  \\
\ccRefIdfierPage{CGAL::Taucs_matrix}  \\
\ccRefIdfierPage{CGAL::Taucs_symmetric_matrix}  \\
\ccRefIdfierPage{CGAL::Taucs_vector}  \\
\ccc{OpenNL::DefaultLinearSolverTraits}  \\
\ccc{OpenNL::SymmetricLinearSolverTraits}  \\


\ccExample

Currently, the Parameterization package creates non symmetric square linear
systems, thus no example is available.


\end{ccRefClass}

% +------------------------------------------------------------------------+
%%RefPage: end of main body, begin of footer
% EOF
% +------------------------------------------------------------------------+

