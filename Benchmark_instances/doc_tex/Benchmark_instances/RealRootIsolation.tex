\section{Real Root Isolation\label{bi_sec:RealRootIsolation}}

The isolation of real roots is a fundamental technique in computer algebra.
In computational geometry it is also used as basic ingredient. 

\subsection{Univariate Case\label{bi_ssec:univariateisolation}}

Univariate polynomials of degree $d$ can have up to $d$ roots. The task
of the real root isolation is to find isolating intervals, such that each
interval contains exactly one of these roots. The following sets
contains such univariate polynomials.

\ccBenchmarkInstance{Polynomial/Polynomial_1/Integer/random/}
\ccBenchmarkInstance{Polynomial/Polynomial_1/Sqrt_extension/random/}
\ccBenchmarkInstance{Polynomial/Polynomial_1/Sqrt_extension/quadric_resultants/}

The sorting of these roots can be seen as another task.

\subsection{Bivariate Case\label{bi_ssec:bivariateisolation}}

In the bivariate case, one aims to find isolating boxes for value-pairs
where pairs of polynomials vanish. The following file contains
a selection of such polynomials.

\ccBenchmarkInstance{Polynomial/Polynomial_2_pair/*.bff}
