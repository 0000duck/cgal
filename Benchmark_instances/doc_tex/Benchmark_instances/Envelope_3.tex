\chapter{Envelope\_3\label{bi_cha:Envelope3}}

Lower envelopes are a fundamental structure in computational geometry.
CGAL provides with its {\tt Envelope\_3} package an implementation
of a generic divide-and-conquer algorithm to compute the lower envelope
of three-dimensional objects. 
The following sections collect different families of data sets that are used
to benchmark this implementation.

\section{Quadric\_3\label{bi_sec:Envelope3Quadric3}}

Quadrics are the simplest non-linear objects. 
Note that, if non-bounded quadrics
are involved, quite certainly several unbounded faces will show up in the
minimization diagram.

\subsection{Random data\label{bi_subsec:Envelope3Quadric3Randomdata}}

The following instances contains quadrics that are randomly generated. 
The coefficients of each quadric consist of 10-bit integers.

The first set contains instance with up to 1000 quadrics.
\ccBenchmarkInstance{Envelope_3/data/Quadric_3/random/quadrics/}

The set is restricted to instances with with up to 1000 ellipsoids.
\ccBenchmarkInstance{Envelope_3/data/Quadric_3/random/ellipsoids}

Ellipsoids can also be excluded like in this set, which 
contains instances of up to 1000 random non-ellipsoids.
\ccBenchmarkInstance{Envelope_3/data/Quadric_3/random/non-ellipsoids}

%Another instance, more specific:

%\ccBenchmarkInstance{Envelope_3/data/Quadric_3/random/non-ellipsoids/rnd_non_ellipsoids_0800_*.bff}
