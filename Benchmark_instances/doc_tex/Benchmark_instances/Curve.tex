\section{Curve\label{bi_cha:Curve}}

A two-dimensional arrangement is the subdivision of the plane into
maximal connected 0-dimensional (vertices), 1-dimensional (edges), 
and 2-dimensional (faces) cells induced by a set of input objects, like
curves. {\sc Cgal}'s arrangement package provides a mature implementation
to compute arrangements of curves, incrementally and by the sweep-line
paradigm. This chapter collects instances to benchmark the 
{\tt Arrangement\_2} of {\sc Cgal}.

\subsection{Conics\label{bi_sec:Arrangement2Conics}}

A conic is defined by the zero set of a two-variate polynomial
of degree 2. A circle is a degenerated conic.

\subsubsection{Random data\label{bi_sec:Arrangement2ConicsRandomdata}}

The following files contain contain randomly generated conics. 
Each instance of one file inlcudes the same number of conics. Not that
a generic conic may also be unbounded.

\ccBenchmarkInstance{Arrangement_2/data/Conic_2/random/number10/}
\ccBenchmarkInstance{Arrangement_2/data/Conic_2/random/number20/}
\ccBenchmarkInstance{Arrangement_2/data/Conic_2/random/number30/}
\ccBenchmarkInstance{Arrangement_2/data/Conic_2/random/number40/}
\ccBenchmarkInstance{Arrangement_2/data/Conic_2/random/number75/}
\ccBenchmarkInstance{Arrangement_2/data/Conic_2/random/number100/}
\ccBenchmarkInstance{Arrangement_2/data/Conic_2/random/number125/}
\ccBenchmarkInstance{Arrangement_2/data/Conic_2/random/number150/}
\ccBenchmarkInstance{Arrangement_2/data/Conic_2/random/number175/}
\ccBenchmarkInstance{Arrangement_2/data/Conic_2/random/number200/}
\ccBenchmarkInstance{Arrangement_2/data/Conic_2/random/number300/}
\ccBenchmarkInstance{Arrangement_2/data/Conic_2/random/number400/}
\ccBenchmarkInstance{Arrangement_2/data/Conic_2/random/number500/}

\subsection{Cubics}

\subsection{Curves of Arbitrary Degree\label{bi_sec:Arrangement2AlgebraicCurve2}}

Algebraic curves of arbitrary degree can also be considered. Full families
of instances will be provided in the future. The attached archive
lists selected curves that have interesting features like covertical 
$x$-extremal points, several cusps, high-degree singularities, et cetera.

\ccBenchmarkInstance{Arrangement_2/data/Algebraic_curve_2/singleton_examples/}


