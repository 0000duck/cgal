% +------------------------------------------------------------------------+
% | Reference manual page: surface_mesher.tex
% +------------------------------------------------------------------------+
% | 01.12.2005   Author
% | Package: Package
% | 
\RCSdef{\RCSsurfacemesherRev}{$Revision$}
\RCSdefDate{\RCSsurfacemesherDate}{$Date$}
% |
%%RefPage: end of header, begin of main body
% +------------------------------------------------------------------------+


\begin{ccRefFunction}{surface_mesher<C2T3, Surface_oracle,
Surface_mesh_criteria, Tag>}  %% add template arg's if necessary

%% \ccHtmlCrossLink{}     %% add further rules for cross referencing links
%% \ccHtmlIndexC[function]{} %% add further index entries

\ccDefinition
  
The function \ccRefName\ computes a two dimensional mesh 
approximating  a surface.

The surface is given as an object of type \ccc{Surface_oracle}.
The class \ccc{Surface_oracle}
has to be a model of the concept \ccc{SurfaceOracle}.

The output mesh satifies the requirements specified by the user
in an object of the class \ccc{Surface_mesh_criteria}.
The class \ccc{Surface_mesh_criteria} has to be
a model of the concept \ccc{SurfaceMeshCriteria}.

 The class \ccc{C2T3}
is required to be a model of the concept
\ccc{Complex2InTriangulation3}.
The  computed  mesh is output as an object \ccc{c2t3} of the class \ccc{C2T3}
which is passed by reference.
The Delaunay refinement
process is started with a set of points which is the union of the
set of vertices in the initial complex \ccc{c2t3} and the set of
points output by the method \ccc{surf.initial_points()}.

The fourth argument is a tag whose type precises the behaviour of the
meshing algorithm. The function \ccRefName\  has specialized versions
for the following  tag types~: \\
\ccc{Manifold} : the output mesh is guaranteed to be a manifold
surface without boundary.\\
\ccc{Manifold_with_boundary}~:the output mesh is guaranteed to be
manifold but may have boundaries.\\
\ccc{Non_manifold} : the algorithm relies on the given criteria and
guarantees nothing else.

The class \ccc{C2T3} has a pointer to 
a \ccc{Delaunay_triangulation_3} class.
The vertex and cell base classes plugged 
in the \ccc{Triangulation_data_structure_3} of the
\ccc{Delaunay_triangulation_3} pointed to by the \ccc{C2T3}
complex are required  to be repectively  models  of the concepts 
\ccc{Complex2InTriangulation3VertexBase_3} and 
\ccc{Complex2InTriangulation3SurfaceMeshCellBase_3}.

\ccInclude{CGAL/surface_mesher.h}

\ccGlobalFunction{template <class C2T3,
                            class Surface_oracle,
                            class Surface_mesh_criteria,
                            class Tag >
void surface_mesher(C2T3& c2t3,
                    Surface_oracle surf,
                    Surface_mesh_criteria criteria,
                    Tag = Non_manifold());}


 

%\ccIsModel

%Concept

\ccSeeAlso
\ccc{Complex2InTriangulation3} \\
\ccc{Complex2InTriangulation3SurfaceMeshCellBase_3} \\
\ccc{Complex2InTriangulation3VertexBase_3} \\
\ccc{SurfaceOracle} \\
\ccc{SurfaceMeshCriteria} \\



%\ccExample

%A short example program.
%Instead of a short program fragment, a full running program can be
%included using the 
%\verb|\ccIncludeExampleCode{Package/surface_mesher.C}| 
%macro. The program example would be part of the source code distribution and
%also part of the automatic test suite.

%\begin{ccExampleCode}
%void your_example_code() {
%}
%\end{ccExampleCode}

%% \ccIncludeExampleCode{Package/surface_mesher.C}

\end{ccRefFunction}

% +------------------------------------------------------------------------+
%%RefPage: end of main body, begin of footer
% EOF
% +------------------------------------------------------------------------+

