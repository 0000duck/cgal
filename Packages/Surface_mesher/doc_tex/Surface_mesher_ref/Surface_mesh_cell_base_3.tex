% +------------------------------------------------------------------------+
% | Reference manual page: Surface_mesh_cell_base_3.tex
% +------------------------------------------------------------------------+
% | 09.12.2005   Author
% | Package: Package
% | 
\RCSdef{\RCSSurfacemeshcellbaseRev}{$Revision$}
\RCSdefDate{\RCSSurfacemeshcellbaseDate}{$Date$}
% |
%%RefPage: end of header, begin of main body
% +------------------------------------------------------------------------+


\begin{ccRefClass}{Surface_mesh_cell_base_3<Gt,Vb>}  %% add template arg's if necessary

%% \ccHtmlCrossLink{}     %% add further rules for cross referencing links
%% \ccHtmlIndexC[class]{} %% add further index entries

\ccDefinition
  
The class \ccRefName\ is a model of the concept
\ccc{SurfaceMeshCellBase_3}.
It is designed to serve  as vertex base  class
in a  triangulation class \ccc{Tr} 
plugged in a  \ccc{Surface_mesh_c2t3<Tr>} class.

\ccInclude{Surface_mesh_cell_base_3.h}

\ccIsModel

\ccc{SurfaceMeshCellBase_3}

%\ccTypes

%\ccNestedType{TYPE}{some nested types}

%\ccCreation
%\ccCreationVariable{a}  %% choose variable name

%\ccConstructor{Surface_mesh_cell_base_3();}{default constructor.}

%\ccOperations

%\ccMethod{void foo();}{some member functions}

\ccSeeAlso

\ccc{Surface_mesh_c2t3<Tr>} 


%\ccExample

%A short example program.
%Instead of a short program fragment, a full running program can be
%included using the 
%\verb|\ccIncludeExampleCode{Package/Surface_mesh_cell_base_3.C}| 
%macro. The program example would be part of the source code distribution and
%also part of the automatic test suite.

%\begin{ccExampleCode}
%void your_example_code() {
%}
%\end{ccExampleCode}

%%% \ccIncludeExampleCode{Package/Surface_mesh_cell_base_3.C}

\end{ccRefClass}

% +------------------------------------------------------------------------+
%%RefPage: end of main body, begin of footer
% EOF
% +------------------------------------------------------------------------+

