% +------------------------------------------------------------------------+
% | Reference manual page: SurfaceOracle.tex
% +------------------------------------------------------------------------+
% | 02.12.2005   Author
% | Package: Package
% | 
\RCSdef{\RCSSurfaceOracleRev}{$Revision$}
\RCSdefDate{\RCSSurfaceOracleDate}{$Date$}
% |
%%RefPage: end of header, begin of main body
% +------------------------------------------------------------------------+


\begin{ccRefConcept}{SurfaceOracle}

%% \ccHtmlCrossLink{}     %% add further rules for cross referencing links
%% \ccHtmlIndexC[concept]{} %% add further index entries

\ccDefinition
  
The concept \ccRefName\ describes the knowledge that is required on the
surface to be meshed. Basically a model of this concept
implements an oracle that is able to tell if a segment
(ray or line) intersects the surface or not
and to compute the intersection
points if any. The concept \ccRefName\ also include methods  to provide
a small set of initial points on the surface.

%\ccGeneralizes
%ThisConcept \\
%ThatConcept

\ccTypes

\ccNestedType{Point}{The type of points. This type is required to be
the type of points of the triangulation pointed to by the 
object (whose type is a model of \ccc{PureCcomplex_2InTriangulation_3}) storing the mesh.}
\ccNestedType{Segment}{The type of segments.}
\ccNestedType{Ray}{The type of rays.}
\ccNestedType{Line}{The type of lines.}



\ccCreation
\ccCreationVariable{surfor}  %% choose variable name

\ccConstructor{SurfaceOracle();}{default constructor.}

\ccOperations


\ccMethod{Object intersect_segment_surface(Segment s);}
{ Returns an \ccc{Object}  which is either a \ccc{Point} where
the surface and the segment \ccc{s} intersect or
a trivial \ccc{Object} if the intersection is empty.}
\ccGlue
\ccMethod{Object intersect_ray_surface(Ray r);}
{ Returns an \ccc{Object}  which is is either a \ccc{Point} where
the surface and the ray \ccc{r} intersect or
a trivial \ccc{Object} if the intersection is empty.}
\ccGlue
\ccMethod{Object intersect_line_surface(Line r);}
{ Returns an \ccc{Object}  which is is either a \ccc{Point} where
 the surface and the line \ccc{l} intersect or
a trivial \ccc{Object} if the intersection is empty.}

\ccMethod{template <class OutputIteratorPoints>
OutputIteratorPoints
initial_points(OutputIteratorPoints pts);}
{Outputs a set of points on the surface.}
\ccGlue
\ccMethod{template <class OutputIteratorPoints>
OutputIteratorPoints
initial_points(OutputIteratorPoints pts, int n);}
{Outputs a set of \ccc{n} points on the surface.}


\ccHasModels

\ccc{Implicit_surface_oracle<Func>}


%\ccSeeAlso

%Some\_other\_concept,
%\ccc{some_other_function}.

%\ccExample

%A short example program.
%Instead of a short program fragment, a full running program can be
%included using the 
%\verb|\ccIncludeExampleCode{Package/SurfaceOracle.C}| 
%macro. The program example would be part of the source code distribution and
%also part of the automatic test suite.

%\begin{ccExampleCode}
%void your_example_code() {
%}
%\end{ccExampleCode}

%%% \ccIncludeExampleCode{Package/SurfaceOracle.C}

\end{ccRefConcept}

% +------------------------------------------------------------------------+
%%RefPage: end of main body, begin of footer
% EOF
% +------------------------------------------------------------------------+

