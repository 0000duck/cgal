% +------------------------------------------------------------------------+
% | Reference manual page: Implicit_surface_oracle.tex
% +------------------------------------------------------------------------+
% | 09.12.2005   Author
% | Package: Package
% | 
\RCSdef{\RCSImplicitsurfaceoracleRev}{$Revision$}
\RCSdefDate{\RCSImplicitsurfaceoracleDate}{$Date$}
% |
%%RefPage: end of header, begin of main body
% +------------------------------------------------------------------------+


\begin{ccRefClass}{Implicit_surface_oracle<Traits, Func>}  %% add template arg's if necessary

%% \ccHtmlCrossLink{}     %% add further rules for cross referencing links
%% \ccHtmlIndexC[class]{} %% add further index entries

\ccDefinition
  
The class \ccRefName\  is a model of the concept
\ccc{SurfaceOracle}. 
This class implements the surface oracle
required by the surface mesher 
\ccc{make_surface_mesh<C2T3, SurfaceOracle,
Criteria, Tag>}
in the case
where the surface to be meshed is described as the zero level
set  of a function $f : \R^3 \longrightarrow \R$.

The parameter \ccc{GT} is a traits class providing 
a type \ccc{Point_3} and a numerical type \ccc{FT}.
This parameter  must be instantiated by 
a model of the concept \ccc{ImplicitSurfaceTraits}.
The \ccc{Point_3} type provided by the traits class  \ccc{GT}
has to match  the \ccc{Point_3} type of the
triangulation of  type \ccc{C2t3::Triangulation_3} emmbedding
the two dimensional complex.

The template parameter \ccc{Func}  stands for a model
of the concept \ccc{ImplicitFunction}
The numerical type \ccc{Func::FT} has to match
the type {GT::FT}.


\ccInclude{CGAL/Implicit_surface_oracle.h}

\ccIsModel

\ccc{SurfaceOracle}

%\ccTypes

%\ccNestedType{TYPE}{some nested types}

%\ccCreation
%\ccCreationVariable{a}  %% choose variable name

%\ccConstructor{Implicit_surface_oracle();}{default constructor.}

%\ccOperations

%\ccMethod{void foo();}{some member functions}

\ccSeeAlso

\ccc{ImplicitFunction}


%\ccExample

%A short example program.
%Instead of a short program fragment, a full running program can be
%included using the 
%\verb|\ccIncludeExampleCode{Package/Implicit_surface_oracle.C}| 
%macro. The program example would be part of the source code distribution and
%also part of the automatic test suite.

%\begin{ccExampleCode}
%void your_example_code() {
%}
%\end{ccExampleCode}

%%% \ccIncludeExampleCode{Package/Implicit_surface_oracle.C}

\end{ccRefClass}

% +------------------------------------------------------------------------+
%%RefPage: end of main body, begin of footer
% EOF
% +------------------------------------------------------------------------+

