% +------------------------------------------------------------------------+
% | Reference manual page: SurfaceMeshCellBase_3.tex
% +------------------------------------------------------------------------+
% | 09.12.2005   Author
% | Package: Package
% | 
\RCSdef{\RCSSurfaceMeshCellBaseRev}{$Revision$}
\RCSdefDate{\RCSSurfaceMeshCellBaseDate}{$Date$}
% |
%%RefPage: end of header, begin of main body
% +------------------------------------------------------------------------+


\begin{ccRefConcept}{SurfaceMeshCellBase_3}

%% \ccHtmlCrossLink{}     %% add further rules for cross referencing links
%% \ccHtmlIndexC[concept]{} %% add further index entries

\ccDefinition
  
The \ccc{Surface_mesh_C2t3<Tr>} data structure is able to store
a pure two dimensional complex embedded in a triangulation.
The concept \ccRefName\ describes the cell base type
to be used  in the  triangulation class  \ccc{Tr}  plugged in 
\ccc{Surface_mesh_C2t3<Tr>} data structure.


\ccGeneralizes
\ccc{TriangulationCellBase_3} \\
The concept \ccRefName\ adds four boolean marker to mark the facets
of the triangulation that belong to the two dimensional complex,
four boolean to mark the visited facets.
This class is also able to store the center of a Delaunay surfacic
ball for each facet.  In the following we call \ccc{surface center}
of a facet, the center of its biggest Delaunay surfacic ball.

\ccTypes

\ccNestedType{Point}{The point type, required to be the point type
of the three dimensional
triangulation in which the surface mesh is embedded.}

\ccCreation
\ccCreationVariable{cell}  %% choose variable name



\ccOperations

\ccMethod
{bool is_facet_on_surface(int i);}
{returns true if \ccc{facet(i)} is in the 2D complex. }
\ccGlue
\ccMethod
{void set_facet_on_surface(int i, bool b );}
{sets \ccc{facet(i)} in the complex if \ccc{b} is true,
and \ccc{NOT_IN_COMPLEX} if \ccc{b} is false.}


\ccMethod
{bool is_facet_visited (int i);}
{returns true if \ccc{facet(i)} has been visited,
false otherwise.}
\ccGlue
\ccMethod
{void set_facet_visited (int i, bool b);} 
{marks \ccc{facet(i)} as visited if \ccc{b} is true
 and non visited otherwise.}


\ccMethod
{const Point&   get_facet_surface_center(int i);}
{returns a reference to the surface center of \ccc{facet(i).}}
\ccGlue
\ccMethod
{void set_facet_surface_center (int i, Point p);} 
{sets point \ccc{p} as the surface center of \ccc{facet(i)}}.


    




\ccHasModels

\ccc{Surface_mesh_cell_base_3<Gt,Vb>}


\ccSeeAlso

\ccc{SurfaceMeshC2T3} \\
\ccc{Surface_mesh_C2T3<Tr>}


%\ccExample

%A short example program.
%Instead of a short program fragment, a full running program can be
%included using the 
%\verb|\ccIncludeExampleCode{Package/SurfaceMeshCellBase_3.C}| 
%macro. The program example would be part of the source code distribution and
%also part of the automatic test suite.

%\begin{ccExampleCode}
%void your_example_code() {
%}
%\end{ccExampleCode}

%% \ccIncludeExampleCode{Package/SurfaceMeshCellBase_3.C}

\end{ccRefConcept}

% +------------------------------------------------------------------------+
%%RefPage: end of main body, begin of footer
% EOF
% +------------------------------------------------------------------------+

