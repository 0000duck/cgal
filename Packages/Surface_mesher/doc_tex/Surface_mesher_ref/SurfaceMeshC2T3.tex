% +------------------------------------------------------------------------+
% | Reference manual page: SurfaceMeshC2T3.tex
% +------------------------------------------------------------------------+
% | 07.12.2005   Author
% | Package: Package
% | 
\RCSdef{\RCSSurfaceMeshCTRev}{$Revision$}
\RCSdefDate{\RCSSurfaceMeshCTDate}{$Date$}
% |
%%RefPage: end of header, begin of main body
% +------------------------------------------------------------------------+


\begin{ccRefConcept}{SurfaceMeshC2T3}

%% \ccHtmlCrossLink{}     %% add further rules for cross referencing links
%% \ccHtmlIndexC[concept]{} %% add further index entries

\ccDefinition
  
The concept \ccRefName\ describes a data structure
designed  to represent a  two dimensional pure complex 
embedded in a three dimensional triangulation.

A complex is a set $C$ of faces such that~:\\
- any subface of a face in $C$ is a face of $C$ \\
- two faces of $C$ are disjoint or share a common subface \\
The complex is two dimensional if its faces have dimension at most
two. It is pure if any face in the complex is a subface
of some face of maximal dimension.
Thus a two dimensional pure complex is just a set of facets
together with their edges and vertices.
A  two dimensional pure complex  embedded 
in a three dimensional triangulation
it is just a subset of the  facets 
of this triangulation.

This complex is a simplification of a more general concept called
\ccc{Complex2InTriangulation3} designed to represent any
two dimensional complex 
embedded in a three dimensional triangulation.

The concept \ccRefName\  is specially suited to handled
surface meshes obtained as the restriction to a surface of 
a  three dimensional Delaunay triangulation.
A model of this concept is to  be used as first parameter in the global
function \ccc{surface_mesher<C2T3,Oracle,Criteria,Tag>}.


 

%\ccGeneralizes

%ThisConcept \\
%ThatConcept

\ccTypes
\ccRefName\ provides the following types.

\ccNestedType{Triangulation_3}{The type of the
underlying 3D triangulation.}
\ccGlue
\ccTypedef{typedef Triangulation_3::Vertex_handle Vertex_handle;}{ The type of
the underlying triangulation vertex handles.}
\ccGlue
\ccTypedef{typedef Triangulation_3::Cell_handle  Cell_handle;}{ The  type of
the  underlying triangulation cell handles.}
\ccGlue
\ccTypedef{typedef Triangulation_3::Facet Facet;}{The  type of
the  underlying triangulation facets.}
\ccGlue
\ccTypedef{typedef Triangulation_3::Edge Edge;} {The  type of
the  underlying triangulation edges.}


\ccEnum{enum Face_type {NOT_IN_COMPLEX, ISOLATED, BOUNDARY, REGULAR,
SINGULAR};}
{A type to describe the type of a face (facet, edge or vertex) with respect to 
the 2D complex. A \ccc{NOT_IN_COMPLEX} face does not belong to the 2D complex.
Facets can only be  \ccc{NOT_IN_COMPLEX}  or \ccc{REGULAR}
depending on wether they belong to the 2D complex on not.
An edge in the complex is said to be respectively \ccc{ISOLATED}, 
\ccc{BOUNDARY},
\ccc{REGULAR} or \ccc{SINGULAR}
 if it is incident to respectively 0, 1, 2  or 3 or
more facets in the complex. A vertex of the complex is
\ccc{ISOLATED} if it is incident incident to no edge of the complex.
It is said to be \ccc{BOUNDARY}  
if the adjacency graph of incidents facets 
is a simple path  and \ccc{REGULAR} if this graph is  a cycle.
Any other vertex of the complex is \ccc{SINGULAR}.
In the present concept which described only pure two dimensional
complex,
there is no \ccc{ISOLATED} vertices or edges. Nevertheless for generality
we maintain
this value in the \ccc{Face_type}.}


\ccNestedType{Facets_iterator}{An iterator type to visit the facets
of the 2D complex around a boundary or regular vertex.}
\ccGlue
\ccNestedType{Edges_iterator}{An iterator type to visit  the
edges of the 2D complex around a boundary or regular vertex.}
\ccGlue
\ccNestedType{Vertices_iterator}{An iterator type to visit 
vertices of the 2D complex around a boundary or regular vertex.}

\ccNestedType{Facets_circulator}{An iterator type to visit the facets
of the 2D complex around a boundary or regular vertex.}
\ccGlue
\ccNestedType{Edges_circulator}{An iterator type to visit  the
edges of the 2D complex around a boundary or regular vertex.}
\ccGlue
\ccNestedType{Vertices_circulator}{An iterator type to visit 
vertices of the 2D complex around a boundary or regular vertex.}




\ccCreation
\ccCreationVariable{c2t3}  %% choose variable name

\ccConstructor{SurfaceMeshC2T3();}{default constructor.}
\ccGlue
\ccConstructor{SurfaceMeshC2T3((Triangulation_3& t3));}
 {Build an empty 2D complex embedded in the triangulation \ccc{t3}}
\ccGlue
\ccConstructor{template < class FacetSelector>
SurfaceMeshC2T3((Triangulation_3& t3), FacetSelector select);}
{Build an empty 2D complex embedded in the triangulation \ccc{t3}
including the facets of \ccc{t3} which passes the test \ccc{select}.
The type \ccc{FacetSelector} supposed to be 
a function object with an operator to select facets~:
\ccc{bool operator()(Facet f);}.}


\ccHeading{Modifications}
\ccMethod{void set_in_complex(Facet f);}{Sets facet \ccc{f} in the 2D complex.}
\ccGlue
\ccMethod{void set_in_complex(Cell_handle c, int i);}
{Sets facet \ccc{(c,i)} in the 2D complex.}
\ccGlue
\ccMethod{void remove_from_complex(Facet f);}
{Removes facet \ccc{f} from  the 2D complex.}
\ccGlue
\ccMethod{void remove_from_complex(Cell_handle c, int i);}
{Removes facet \ccc{(c,i)} in the 2D complex.}

\ccHeading{Queries}
Queries on the type of individual face with respect to the 2D complex.
\ccMethod{Face_type  face_type(Facet f);}
{Returns the type of the facet \ccc{f}  with respect to  the 2D complex.}
\ccGlue
\ccMethod{Face_type  face_type(Cell_handle c, int i);}
{Returns the type of the facet \ccc{(c,i)} with respect to  the 2D complex.}
\ccGlue
\ccMethod{Face_type  face_type(Edge e);}
{Returns the type of edge \ccc{e} in the 2D complex.}
\ccGlue
\ccMethod{Face_type  face_type(Cell_handle c, int
i, int j);}
{Returns the type of edge \ccc{(c,i,j)}  in the 2D complex.}
\ccGlue
\ccMethod{Face_type  face_type(Vertex_handle v);}
{Returns the type of vertex \ccc{v} in the 2D complex.}
\ccGlue
\ccMethod{bool is_in_complex(Facet f);}
{returns \ccc{true} if the facet \ccc{f} belongs to the 2D complex.}
\ccGlue
\ccMethod{bool is_in_complex(Cell_handle c, int i);}
{returns \ccc{true} if the facet \ccc{(c,i)} belongs to the 2D complex.}
\ccGlue
\ccMethod{bool is_in_complex(Edge e);}
{returns \ccc{true} if the edge  \ccc{e} belongs to the 2D complex.}
\ccGlue
\ccMethod{bool is_in_complex(Cell_handle c, int i, int j);}
{returns \ccc{true} if the edge \ccc{(c,i,j)} belongs to the 2D complex.}
\ccGlue
\ccMethod{bool is_in_complex(Vertex_handle v);}
{returns \ccc{true} if the vertex \ccc{v} belongs to the 2D complex.}
\ccGlue
\ccMethod{bool is_regular(Edge e);}
{returns \ccc{true} if the edge  \ccc{e} is regular.
\ccPrecond{\ccc{e} is an edge of the complex.} }
\ccGlue
\ccMethod{bool is_regular(Cell_handle c, int i, int j);}
{returns \ccc{true} if the edge \ccc{(c,i,j)} is regular.
\ccPrecond{\ccc{(c,i,j)} is an edge of the complex.} }
\ccGlue
\ccMethod{bool is_regular(Vertex_handle v);}
{returns \ccc{true} if the vertex \ccc{v} is regular.
\ccPrecond{\ccc{v} is a vertex of the complex.} }
\ccGlue
\ccMethod{bool is_on_boundary(Edge e);}
{returns \ccc{true} if the edge  \ccc{e} is on boundary.
\ccPrecond{\ccc{e} is an edge of the complex.} }
\ccGlue
\ccMethod{bool is_on_boundary(Cell_handle c, int i, int j);}
{returns \ccc{true} if the edge \ccc{(c,i,j)} is on boundary.
\ccPrecond{\ccc{(c,i,j)} is an edge of the complex.} }
\ccGlue
\ccMethod{bool is_on_boundary(Vertex_handle v);}
{returns \ccc{true} if the vertex \ccc{v} is on boundary.
\ccPrecond{\ccc{v} is a vertex of the complex.} }

Queries on the 2D complex. 

\ccMethod{bool is_manifold();}
{returns \ccc{true} if the complex is manifold.}
\ccGlue
\ccMethod{bool is_connected_component_manifold(Facet f);}
{Returns true if the   
connected component including  facet \ccc{f} is manifold.}
\ccGlue
\ccMethod{std::size_type number_of_connected_components();}
{Returns the number of connected components of the
complex.}
\ccGlue
\ccMethod{std::size_type 
number_of_connected_components_of_boundary();}
{Returns the number of connected components of the boundary
  the complex.}
\ccGlue
\ccMethod{std::size_type
number_of_connected_components_of_boundary(Facet f);}
{Returns the number of connected components of the boundary
of connected component of the complex including facet \ccc{f}.}
\ccGlue
\ccMethod{std::size_type
number_of_connected_components_of_boundary(Cell_handle c, int i);}
{Returns the number of connected components of the boundary
of connected component of the complex including facet
\ccc{(c,i)}}
\ccGlue
\ccMethod{std::size_type
genus(Facet f);}
{Returns the topological genus 
of the connected component of the complex including facet \ccc{f}.}
\ccGlue
\ccMethod{std::size_type
genus(Cell_handle c, int i);}
{Returns the topological genus 
of connected component of the complex including facet \ccc{(c,i)}.}





\ccHeading{Traversal of the complex}

The data structures provides iterators  to visit  
various faces of the complex and circulators to visit the faces
incident to a given vertex of the complex.
All those iterators and circulators are bidirectional and
non mutable.

\ccMethod{Facets_iterator facets_begin();}
{Returns an iterator with value type \ccc{Facet} to visit the facets
of the 2D complex.}
\ccGlue
\ccMethod{Facets_iterator facets_end();}
{Returns past the end value for the above iterator.}

\ccMethod{Edges_iterator edges_begin();}
{Returns an iterator with value type \ccc{Edge} to visit the
edges of the 2D complex which are not isolated.}
\ccGlue
\ccMethod{Edges_iterator edges_end();}
{Returns past the end value for the above iterator.}
\ccGlue
\ccMethod{Boundary_edges_iterator boundary_edges_begin();}
{Returns an iterator with value type \ccc{Edge} to visit the
boundary  edges of the complex..}
\ccGlue
\ccMethod{Boundary_edges_iterator boundary_edges_end();}
{Returns past the end value for the above iterator.}
\ccGlue
\ccMethod{Vertices_iterator vertices_begin();}
{Returns an iterator with value type \ccc{Vertex_handle} to visit the
vertices of the 2D complex which are not isolated.}
\ccGlue
\ccMethod{Vertices_iterator vertices_end();}
{Returns past the end value for the above iterator.}
\ccGlue
\ccMethod{Boundary_vertices_iterator 
         boundary_vertices_begin();}
{Returns an iterator with value type \ccc{Vertex_handle} to visit the 
vertices of the 2D complex that are  boundary vertices.}
\ccGlue
\ccMethod{Boundary__vertices_iterator 
 boundary_vertices_end();}
{Returns past the end value for the above iterator.}


\ccMethod{Facets_circulator
incident_facets(Vertex_handle v);}
{Returns a circulator to visit the facets of the complex incident to
vertex \ccc{v}. Returns an empty circulator if the vertex is 
singular. For a boundary  vertex,  the circulator will wrap from the last
incident facet  to the first one.}
\ccGlue
\ccMethod{Egdes_circulator
incident_edges(Vertex_handle v);}
{Returns a circulator to visit the edges of the complex incident to
vertex \ccc{v}. Returns an empty circulator if the vertex is 
singular. For a boundary  vertex,  the circulator will wrap from the last
incident edge  to the first one.}
\ccGlue
\ccMethod{Vertices_circulator
incident_vertices(Vertex_handle v);}
{Returns a circulator to visit the vertices of the complex adjacent to
vertex \ccc{v}. Returns an empty circulator if the vertex  or
singular. For a boundary  vertex,  the circulator will wrap from the last
incident vertex  to the first one.}

The following function is the basic function to walk on the 2D
complex
\ccMethod{Facet neighbor(Facet f, int j);}
{Returns the facet of the complex which is the neighbor of 
the facet \ccc{f} opposite to the vertex with index \ccc{j} of
\ccc{f} .
The vertices of the facet \ccc{f = (cell  c, i)} are numbered
(0,1,2) (according to the \ccc{vertex_triple_index(i,j)} member function
of \ccc{Triangulation_3})
in such a way that facet \ccc{f} is oriented by the
outward normal of tetraedra \ccc{c}.}
\ccGlue
\ccMethod{Facet neighbor(Cell_handle c, int i, int j);}
{Returns the facet of the complex which is the neighbor of 
the facet \ccc{f} opposite to the vertex with index \ccc{j} of \ccc{f}.
See above.}




\ccHasModels

\ccc{Surface_mesh_C2T3<Tr>}


\ccSeeAlso

{Complex2InTriangulation\_3}, \\
\ccc{surface_mesher<C2T3,Oracle,Criteria,Tag>}.

%\ccExample

%A short example program.
%Instead of a short program fragment, a full running program can be
%included using the 
%\verb|\ccIncludeExampleCode{Package/SurfaceMeshC2T3.C}| 
%macro. The program example would be part of the source code distribution and
%also part of the automatic test suite.

%\begin{ccExampleCode}
%void your_example_code() {
%}
%\end{ccExampleCode}

%% \ccIncludeExampleCode{Package/SurfaceMeshC2T3.C}

\end{ccRefConcept}

% +------------------------------------------------------------------------+
%%RefPage: end of main body, begin of footer
% EOF
% +------------------------------------------------------------------------+

