% =============================================================================
% The CGAL Reference Manual
% Section: 2D Smallest Enclosing Circle
% -----------------------------------------------------------------------------
% file  : spec/Min_circle_2.tex
% author: Bernd G�rtner, Sven Sch�nherr (sven@inf.fu-berlin.de)
% $Id$
% =============================================================================
 
\begin{ccClassTemplate}{CGAL_Min_circle_2<R>}
\ccSection{2D Smallest Enclosing Circle}

\ccDefinition

An object of the class \ccClassTemplateName\ is the unique circle of
smallest area enclosing a finite set of points in two-dimensional
euclidean space $\E_2$. For a point set $P$ we denote by $mc(P)$ the
smallest circle that contains all points of $P$. Note that $mc(P)$ can
be degenerate, i.e.\ $P=$\ccTexHtml{$\emptyset$}{&Oslash;} if
$P=$\ccTexHtml{$\emptyset$}{&Oslash;} and $mc(P)=\{p\}$ if $P=\{p\}$.

An inclusion-minimal subset $S$ of $P$ with $mc(S)=mc(P)$ is called
a {\em support set}, the points in $S$ are the {\em support points}.
A support set has size at most three, and all its points lie on the
boundary of $mc(P)$.

If $mc(P)$ has more than three points on the boundary,
neither the support set nor its size are necessarily unique.

The underlying algorithm can cope with all kinds of input, e.g.\
$P$ may be empty or points may occur more than once. The algorithm
computes a support set $S$ which remains fixed until the next insert
operation.

\ccCreation
\ccCreationVariable{min_circle}

A \ccClassTemplateName\ object can be created from an arbitrary point
set $P$ and by specialized construction methods expecting no, one, two
or three points as arguments. The latter methods can be useful for
reconstructing $mc(P)$ from a given support set $S$ of $P$.

\ccSetThreeColumns{CGAL_Bounded_side}{}{
  returns \ccStyle{CGAL_ON_BOUNDED_SIDE}, \ccStyle{CGAL_ON_BOUNDARY},}
\ccPropagateThreeToTwoColumns

\ccStyle{#include <CGAL/Min_circle_2.h>}

\ccConstructor{ CGAL_Min_circle_2( );}{
        introduces a variable \ccVar\ of type \ccClassTemplateName.
	It is initialized to
	$mc($\ccTexHtml{$\emptyset$}{&Oslash;}$)$, the empty set.
	\ccPostcond \ccStyle{\ccVar.is_empty()} = \ccStyle{true}.}

\ccHidden
\ccConstructor{ CGAL_Min_circle_2( const CGAL_Min_circle_2<R>&);}{
        copy constructor.}

\ccConstructor{ CGAL_Min_circle_2( const CGAL_Point_2<R>& p);}{
        introduces a variable \ccVar\ of type \ccClassTemplateName.
        It is initialized to $mc(\{p\})$, the set $\{p\}$.
        \ccPostcond \ccStyle{\ccVar.is_degenerate()} = \ccStyle{true}.}

\ccConstructor{ CGAL_Min_circle_2( const CGAL_Point_2<R>& p1,
                                   const CGAL_Point_2<R>& p2);}{
        introduces a variable \ccVar\ of type \ccClassTemplateName.
        It is initialized to $mc(\{p1,p2\})$, the circle with diameter
        equal to the segment connecting $p1$ and $p2$.}

\ccConstructor{ CGAL_Min_circle_2( const CGAL_Point_2<R>& p1,
                                   const CGAL_Point_2<R>& p2,
                                   const CGAL_Point_2<R>& p3);}{
        introduces a variable \ccVar\ of type \ccClassTemplateName.
        It is initialized to $mc(\{p1,p2,p3\})$.}

\ccConstructor{ CGAL_Min_circle_2( const CGAL_Point_2<R>* first,
                                   const CGAL_Point_2<R>* last,
                                   bool randomize = false);}{
        introduces a variable \ccVar\ of type \ccClassTemplateName. It
        is initialized to $mc(P)$ with $P$ being the set of points in
        the range [\ccStyle{first},\ccStyle{last}). If
        \ccStyle{randomize} is \ccStyle{true}, a random permutation of
        $P$ is computed in advance. Usually, this will not be
        necessary, however, the algorithm's efficiency depends on the
        order in which the points are processed, and a bad order might
        lead to extremely poor performance (see example below).}

\ccHidden
\ccMemberFunction{ CGAL_Min_circle_2<R>&
                   operator = ( const CGAL_Min_circle_2<R>&);}{
        assignment operator.}


\ccHeading{Access operations}

\ccMemberFunction{ int number_of_points( ) const;}{
        returns the number of points of \ccVar, i.e.\ $|P|$.}

\ccMemberFunction{ int number_of_support_points( ) const;}{
        returns the number of support points of \ccVar, i.e.\ $|S|$.}

\ccMemberFunction{ const CGAL_Point_2<R>& point( int i) const;}{
        returns the \ccStyle{i}-th point of \ccVar. Between two insert
        operations any call to \ccStyle{\ccVar.point(i)} with the same
        \ccStyle{i} returns the same point.
        \ccPrecond $0 \leq i <$ \ccStyle{\ccVar.number_of_points()}.}

\ccMemberFunction{ const CGAL_Point_2<R>& support_point( int i) const;}{
        returns the \ccStyle{i}-th support point of \ccVar. Between two
        insert operations any call to \ccStyle{\ccVar.support_point(i)}
        with the same \ccStyle{i} returns the same point.
        \ccPrecond $0 \leq i <$ \ccStyle{\ccVar.number_of_support_points()}.}

\ccMemberFunction{ const CGAL_Point_2<R>& operator [] ( int i) const;}{
        returns \ccStyle{\ccVar.point(i)}.}

\ccMemberFunction{ const CGAL_Circle_2<R>& circle( ) const;}{
        returns an oriented circle with same center $c$ and same
        squared radius $r$ as \ccVar\ and positive orientation.
        \ccPrecond \ccStyle{\ccVar.is_empty()} = \ccStyle{false}.}

\ccMemberFunction{ CGAL_Bbox_2 bbox( ) const;}{
        returns a bounding box containing \ccVar.
        \ccPrecond \ccStyle{\ccVar.is_empty()} = \ccStyle{false}.}


\ccHeading{Update operations}

New points can be added to an existing $\ccVar$, allowing to build
$mc(P)$ incrementally, e.g.\ if $P$ is not known in advance. Compared
to the direct creation of $mc(P)$, this is not much slower, because
the construction method is incremental itself.

\ccMemberFunction{ void insert( const CGAL_Point_2<R>& p);}{
        inserts \ccStyle{p} in \ccVar\ and recomputes the smallest
        enclosing circle.}

\ccMemberFunction{ void reserve( int n);}{
        reserves storage for at least \ccStyle{n} points in \ccVar.
        It can be used, if the number of insert operations is known in
        advance.}


\ccHeading{Check operation}

\ccMemberFunction{ bool check( bool verbose = false) const;}{
        checks \ccVar\ for consistency. It returns \ccStyle{true}, iff
        (a) \ccVar\ contains all points of its defining set $P$, (b)
        \ccVar\ is the smallest circle spanned by its support set $S$,
        and (c) $S$ is minimal, i.e. no support point is redundant. If
        \ccStyle{verbose} is \ccStyle{true}, error messages are
        written to standard error stream.}


\ccHeading{Predicates}

The following predicates imitate the corresponding ones of the class
\ccStyle{CGAL_Circle_2<R>}, with the exception of \ccStyle{is_empty()}
which is not present in \ccStyle{CGAL_Circle_2<R>}, because objects of
this class cannot be empty. By definition, an empty
\ccClassTemplateName\ has no boundary and no bounded side, i.e.\ its
unbounded side equals the whole plane $\E_2$.

\ccMemberFunction{ CGAL_Bounded_side
                   bounded_side( const CGAL_Point_2<R>& p) const;}{
        returns \ccStyle{CGAL_ON_BOUNDED_SIDE},
        \ccStyle{CGAL_ON_BOUNDARY}, or
        \ccStyle{CGAL_ON_UNBOUNDED_SIDE} iff \ccStyle{p} lies inside,
        on the boundary, or outside of \ccVar, respectively.}

\ccMemberFunction{ bool has_on_bounded_side( const CGAL_Point_2<R>& p) const;}{
        returns \ccStyle{true}, iff \ccStyle{p} lies inside \ccVar.}

\ccMemberFunction{ bool has_on_boundary( const CGAL_Point_2<R>& p) const;}{
        returns \ccStyle{true}, iff \ccStyle{p} lies on the boundary
        of \ccVar.}

\ccMemberFunction{ bool
                   has_on_unbounded_side( const CGAL_Point_2<R>& p) const;}{
        returns \ccStyle{true}, iff \ccStyle{p} lies outside of \ccVar.}

\ccMemberFunction{ bool is_empty( ) const;}{
        returns \ccStyle{true}, iff \ccVar\ is empty (this implies
        degeneracy).}

\ccMemberFunction{ bool is_degenerate( ) const;}{
        returns \ccStyle{true}, iff \ccVar\ is degenerate.}


\ccImplementation

We implement the algorithm of Welzl, with move-to-front
heuristic~\cite{Welzl}. If randomization is chosen, the creation time
is almost always linear in the number of points. Access operations and
predicates take constant time, inserting a point might take up to
linear time, but substantially less than computing the new smallest
enclosing circle from scratch. For the member function \ccStyle{reserve}
see the container \ccStyle{vector} from STL~\cite{STL}.


\ccExample

To illustrate the creation of \ccClassTemplateName\ and to show that
randomization can be useful in certain cases, we give an example.

\begin{cprog}
    #include <CGAL/Integer.h>
    #include <CGAL/Homogeneous.h>
    #include <CGAL/Min_circle_2.h>

    typedef  CGAL_Homogeneous<integer>  R;
    typedef  CGAL_Point_2<R>            Point;
    typedef  CGAL_Min_circle_2<R>       Min_circle;

    int     n = 1000;
    Point*  P = new Point[ n];

    for ( int i = 0; i < n; ++i)
        P[ i] = Point( (i%2 == 0 ? i : -i), 0);
    /* (0,0), (-1,0), (2,0), (-3,0), ... */

    Min_circle  mc1( P, P+n);           /* very slow */
    Min_circle  mc2( P, P+n, true);     /* fast      */
\end{cprog}

\end{ccClassTemplate}

% ===== EOF ===================================================================
