% =============================================================================
% The CGAL Reference Manual
% Section: 2D Smallest Enclosing Circle
% -----------------------------------------------------------------------------
% file  : Library/spec/Min_circle_2.tex
% author: Bernd G�rtner, Sven Sch�nherr (sven@inf.fu-berlin.de)
% -----------------------------------------------------------------------------
% $Revision$
% $Date$
% =============================================================================

\begin{ccClassTemplate}{CGAL_Min_circle_2<I>}
\ccSection{2D Smallest Enclosing Circle}

% -----------------------------------------------------------------------------
\ccDefinition

An object of the class \ccClassTemplateName\ is the unique circle of
smallest area enclosing a finite set of points in two-dimensional
euclidean space $\E_2$. For a point set $P$ we denote by $mc(P)$ the
smallest circle that contains all points of $P$. Note that $mc(P)$ can
be degenerate, i.e.\ $mc(P)=\mbox{\ccTexHtml{$\;\emptyset$}{&Oslash;}}$
if $P=\mbox{\ccTexHtml{$\;\emptyset$}{&Oslash;}}$ and $mc(P)=\{p\}$ if
$P=\{p\}$.

An inclusion-minimal subset $S$ of $P$ with $mc(S)=mc(P)$ is called a
{\em support set}, the points in $S$ are the {\em support points}.  A
support set has size at most three, and all its points lie on the
boundary of $mc(P)$. If $mc(P)$ has more than three points on the
boundary, neither the support set nor its size are necessarily unique.

The underlying algorithm can cope with all kinds of input, e.g.\
$P$ may be empty or points may occur more than once. The algorithm
computes a support set $S$ which remains fixed until the next insert
operation.

\emph{Note:} In this release correct results are only guaranteed if
exact arithmetic is used, see Section~\ref{sec:opt_I_Impl}.

\ccInclude{CGAL/Min_circle_2.h}

% -----------------------------------------------------------------------------
\ccHeading{Interface}

The template parameter \ccc{I} is the interface class for optimisation
algorithms. It is a traits class that defines the interface between
optimisation algorithms and the primitives they use. For example
\ccc{I::Point} is a mapping on a point class. Think of it as 2D points
in the Euclidean plane.

\cgal\ provides a ready-made implementation for the interface class as
described in Section~\ref{sec:opt_I_Impl}. Customizing own interface
classes for optimisation algorithms can be done according to the
requirements for interface classes listed in Section~\ref{sec:opt_I_Req}.

% -----------------------------------------------------------------------------
\ccTypes

\ccNestedType{I}{Interface type.}

\ccGlueBegin
\ccUnchecked
\ccTypedef{typedef I::Point     Point;   }{Point    type.}
\ccUnchecked
\ccTypedef{typedef I::Distance  Distance;}{Distance type.}
\ccGlueEnd

\ccUnchecked
\ccTypedef{typedef I::Circle    Circle;  }{Circle   type.}

The following types denote iterators that allow to traverse all points
and support points of the smallest enclosing circle, resp. The
iterators are non-mutable and their value type is \ccc{Point}. The
iterator category is given in parentheses.

\ccSetTwoColumns{CGAL_Min_circle_2<I>:: Support_point_iterator}{}

\ccNestedType{Point_iterator}{(bidirectional).}

\ccNestedType{Support_point_iterator}{(random access).}

\ccPropagateThreeToTwoColumns

% -----------------------------------------------------------------------------
\ccCreation
\ccCreationVariable{min_circle}

A \ccClassTemplateName\ object can be created from an arbitrary point
set $P$ and by specialized construction methods expecting no, one, two
or three points as arguments. The latter methods can be useful for
reconstructing $mc(P)$ from a given support set $S$ of $P$.

\ccConstructor{ CGAL_Min_circle_2( I const& i = I());}{
        introduces a variable \ccVar\ of type \ccClassTemplateName.
        It is initialized to
        $mc(\mbox{\ccTexHtml{$\emptyset$}{&Oslash;}})$, the empty set.
        \ccPostcond \ccVar\ccc{.is_empty()} = \ccc{true}.}

\ccUnchecked
\ccHidden
\ccConstructor{ CGAL_Min_circle_2( CGAL_Min_circle_2<I> const&);}{
        copy constructor.}

\ccConstructor{ CGAL_Min_circle_2( Point const& p, I const& i = I());}{
        introduces a variable \ccVar\ of type \ccClassTemplateName.
        It is initialized to $mc(\{p\})$, the set $\{p\}$.
        \ccPostcond \ccVar\ccc{.is_degenerate()} = \ccc{true}.}

\ccConstructor{ CGAL_Min_circle_2( Point const& p1,
                                   Point const& p2,
                                   I const& i = I());}{
        introduces a variable \ccVar\ of type \ccClassTemplateName.
        It is initialized to $mc(\{p1,p2\})$, the circle with diameter
        equal to the segment connecting $p1$ and $p2$.}

\ccConstructor{ CGAL_Min_circle_2( Point const& p1,
                                   Point const& p2,
                                   Point const& p3,
                                   I const& i = I());}{
        introduces a variable \ccVar\ of type \ccClassTemplateName.
        It is initialized to $mc(\{p1,p2,p3\})$.}

\ccUnchecked
\ccConstructor{ template < class InputIterator >
                CGAL_Min_circle_2( InputIterator first,
                                   InputIterator last,
                                   bool          randomize = false,
                                   CGAL_Random&  random    = CGAL_random,
                                   I const&      i         = I()        );}{
        introduces a variable \ccVar\ of type \ccClassTemplateName. It
        is initialized to $mc(P)$ with $P$ being the set of points in
        the range [\ccc{first},\ccc{last}). If \ccc{randomize} is
        \ccc{true}, a random permutation of $P$ is computed in
        advance, using the random numbers generator \ccc{random}.
        Usually, this will not be necessary, however, the algorithm's
        efficiency depends on the order in which the points are
        processed, and a bad order might lead to extremely poor
        performance (see example below).
        \ccPrecond The value type of \ccc{first} and \ccc{last} is
        \ccc{Point}.}

{\em Note:} Since most compilers do not support member templates yet,
we provide specialized constructors instead. In the current release
there are constructors for C arrays (using pointers as iterators) and
for the STL sequence containers \ccc{vector<Point>} and \ccc{list<Point>}.

\ccHidden
\ccConstructor{ CGAL_Min_circle_2( const Point* first,
                                   const Point* last,
                                   bool         randomize = false,
                                   CGAL_Random& random    = CGAL_random,
                                   I const&     i         = I()        );}{
        STL-like constructor for random access iterators.}

\ccHidden
\ccConstructor{ CGAL_Min_circle_2( list<Point>::const_iterator first,
                                   list<Point>::const_iterator last,
                                   bool         randomize = false,
                                   CGAL_Random& random    = CGAL_random,
                                   I const&     i         = I()        );}{
        STL-like constructor for sequence container list<Point>.}

\ccHidden
\ccConstructor{ ~CGAL_Min_circle_2( );}{
        destructor.}

\ccUnchecked
\ccHidden
\ccMemberFunction{ CGAL_Min_circle_2<I>&
                   operator = ( CGAL_Min_circle_2<I> const&);}{
        assignment operator.}

% -----------------------------------------------------------------------------
\ccAccessFunctions

\ccMemberFunction{ int number_of_points( ) const;}{
        returns the number of points of \ccVar, i.e.\ $|P|$.}

\ccMemberFunction{ int number_of_support_points( ) const;}{
        returns the number of support points of \ccVar, i.e.\ $|S|$.}

\ccMemberFunction{ Point_iterator  points_begin() const;}{
        returns an iterator referring to the first point of \ccVar.}

\ccMemberFunction{ Point_iterator  points_end() const;}{
        returns the corresponding past-the-end iterator.}

\ccMemberFunction{ Support_point_iterator  support_points_begin() const;}{
        returns an iterator referring to the first support point of \ccVar.}

\ccMemberFunction{ Support_point_iterator  support_points_end() const;}{
        returns the corresponding past-the-end iterator.}

\ccMemberFunction{ Point const&  support_point( int i) const;}{
        returns the \ccc{i}-th support point of \ccVar. Between two
        insert operations any call to \ccVar\ccc{.support_point(i)}
        with the same \ccc{i} returns the same point.
        \ccPrecond $0 \leq i< \mbox{\ccVar\ccc{.number_of_support_points()}}$.}

\ccMemberFunction{ Circle  circle( ) const;}{
        returns a circle with same center and same squared radius
        as \ccVar, if \ccVar is not empty.  Otherwise the default
        circle, i.e.\ \ccc{Circle()}, is returned.}


% -----------------------------------------------------------------------------
\ccPredicates

By definition, an empty \ccClassTemplateName\ has no boundary and no
bounded side, i.e.\ its unbounded side equals the whole plane $\E_2$.

\ccMemberFunction{ CGAL_Bounded_side
                   bounded_side( Point const& p) const;}{
        returns \ccc{CGAL_ON_BOUNDED_SIDE},\ccTexHtml{$\:$}{ }%
        \ccc{CGAL_ON_BOUNDARY}, or \ccc{CGAL_ON_UNBOUNDED_SIDE}
        iff \ccc{p} lies properly inside, on the boundary, or properly
        outside of \ccVar, resp.}

\ccMemberFunction{ bool has_on_bounded_side( Point const& p) const;}{
        returns \ccc{true}, iff \ccc{p} lies properly inside \ccVar.}

\ccMemberFunction{ bool has_on_boundary( Point const& p) const;}{
        returns \ccc{true}, iff \ccc{p} lies on the boundary
        of \ccVar.}

\ccMemberFunction{ bool has_on_unbounded_side( Point const& p) const;}{
        returns \ccc{true}, iff \ccc{p} lies properly outside of \ccVar.}

\ccMemberFunction{ bool is_empty( ) const;}{
        returns \ccc{true}, iff \ccVar\ is empty (this implies
        degeneracy).}

\ccMemberFunction{ bool is_degenerate( ) const;}{
        returns \ccc{true}, iff \ccVar\ is degenerate, i.e.\ if
        \ccVar\ is empty or equal to a single point, equivalently if
        the number of support points is less than 2.}

% -----------------------------------------------------------------------------
\ccHeading{Modifiers}

New points can be added to an existing \ccVar, allowing to build
$mc(P)$ incrementally, e.g.\ if $P$ is not known in advance. Compared
to the direct creation of $mc(P)$, this is not much slower, because
the construction method is incremental itself.

\ccMemberFunction{ void insert( Point const& p);}{
        inserts \ccc{p} into \ccVar\ and recomputes the smallest
        enclosing circle.}

% -----------------------------------------------------------------------------
\ccHeading{Validity Check}

An object \ccVar\ is valid, iff
\begin{itemize}
  \item \ccVar\ contains all points of its defining set $P$,
  \item \ccVar\ is the smallest circle spanned by its support set $S$, and
  \item $S$ is minimal, i.e.\ no support point is redundant.
\end{itemize}
Using the ready-made implementation for the interface class with exact
arithmetic as described in Section~\ref{sec:opt_I_Impl} guarantees
validity of \ccVar. The following function is mainly intended for
debugging user supplied interface classes.

\begin{ccAdvanced}
\ccMemberFunction{ bool is_valid( bool verbose = false,
                                  int  level   = 0    ) const;}{
        returns \ccc{true}, iff \ccVar\ is valid. If \ccc{verbose}
        is \ccc{true}, some messages concerning the performed checks
        are written to standard error stream. The second parameter
        \ccc{level} is not used, we provide it only for consistency
        with interfaces of other classes.}
\end{ccAdvanced}

% -----------------------------------------------------------------------------
\ccHeading{I/O}

\ccFunction{ ostream& operator << ( ostream& os,
                                    CGAL_Min_circle_2<I> const& min_circle);}{
        writes \ccVar\ to output stream \ccc{os}.
        \ccPrecond The output operator is defined for \ccc{I::Point}
        (and for \ccc{I::Circle}, if pretty printing is used).}

\ccFunction{ istream& operator >> ( istream& is,
                                    CGAL_Min_circle_2<I> &min_circle);}{
        reads \ccVar\ from input stream \ccc{is}.
        \ccPrecond The input operator is defined for \ccc{I::Point}.}

\ccInclude{CGAL/IO/Window_stream.h}

\ccUnchecked
\ccFunction{ CGAL_Window_stream&
             operator << ( CGAL_Window_stream& ws,
                           CGAL_Min_circle_2<I> const& min_circle);}{
        writes \ccVar\ to window stream \ccc{ws}.
        \ccPrecond The window stream output operator is defined for
        \ccc{I::Point} and \ccc{I::Circle}.}

% -----------------------------------------------------------------------------
\ccImplementation

We implement the algorithm of Welzl, with move-to-front
heuristic~\cite{w-sedbe-91a}. If randomization is chosen, the creation
time is almost always linear in the number of points. Access functions
and predicates take constant time, inserting a point might take up to
linear time, but substantially less than computing the new smallest
enclosing circle from scratch. The check for validity takes linear
time.

% -----------------------------------------------------------------------------
\ccExample

To illustrate the creation of \ccClassTemplateName\ and to show that
randomization can be useful in certain cases, we give an example.

\begin{cprog}
    #include <CGAL/Integer.h>
    #include <CGAL/Homogeneous.h>
    #include <CGAL/Min_circle_2.h>

    typedef  CGAL_Homogeneous<integer>  R;
    typedef  CGAL_Point_2<R>            Point;
    typedef  CGAL_Min_circle_2< CGAL_Optimisation_default_interface<R> >
                                        Min_circle;

    int     n = 1000;
    Point*  P = new Point[ n];

    for ( int i = 0; i < n; ++i)
        P[ i] = Point( (i%2 == 0 ? i : -i), 0);
    /* (0,0), (-1,0), (2,0), (-3,0), ... */

    Min_circle  mc1( P, P+n);           /* very slow */
    Min_circle  mc2( P, P+n, true);     /* fast      */

    delete[] P;
\end{cprog}

\end{ccClassTemplate}

% ===== EOF ===================================================================
