% =============================================================================
% The CGAL Reference Manual
% Section: 2D Smallest Enclosing Circle
% -----------------------------------------------------------------------------
% file  : Library/spec/Min_circle_2.tex
% author: Bernd G�rtner, Sven Sch�nherr (sven@inf.fu-berlin.de)
% -----------------------------------------------------------------------------
% $Revision$
% $Date$
% =============================================================================

\begin{ccClassTemplate}{CGAL_Min_circle_2<I>}
\ccSection{2D Smallest Enclosing Circle}

% -----------------------------------------------------------------------------
\ccDefinition

An object of the class \ccClassTemplateName\ is the unique circle of
smallest area enclosing a finite set of points in two-dimensional
euclidean space $\E_2$. For a point set $P$ we denote by $mc(P)$ the
smallest circle that contains all points of $P$. Note that $mc(P)$ can
be degenerate, i.e.\ $mc(P)=$\ccTexHtml{$\;\emptyset$}{&Oslash;} if
$P=$\ccTexHtml{$\;\emptyset$}{&Oslash;} and $mc(P)=\{p\}$ if $P=\{p\}$.

An inclusion-minimal subset $S$ of $P$ with $mc(S)=mc(P)$ is called a
{\em support set}, the points in $S$ are the {\em support points}.  A
support set has size at most three, and all its points lie on the
boundary of $mc(P)$. If $mc(P)$ has more than three points on the
boundary, neither the support set nor its size are necessarily unique.

The underlying algorithm can cope with all kinds of input, e.g.\
$P$ may be empty or points may occur more than once. The algorithm
computes a support set $S$ which remains fixed until the next insert
operation.

\emph{Note:} In this release correct results are only guaranteed if
exact arithmetic is used, see Section~\ref{sec:opt_I_Impl}.

\ccInclude{CGAL/Min_circle_2.h}

% -----------------------------------------------------------------------------
\ccHeading{Interface}

The template parameter \ccc{I} is the interface class for optimisation
algorithms. It is a traits class that defines the interface between
optimisation algorithms and the primitives they use. For example
\ccc{I::Point} is a mapping on a point class. Think of it as 2D points
in the Euclidean plane.

\cgal\ provides a ready-made implementation for the interface class as
described in Section~\ref{sec:opt_I_Impl}. Customizing own interface
classes for optimisation algorithms can be done according to the
requirements for interface classes listed in
Section~\ref{sec:opt_I_Req}.

% -----------------------------------------------------------------------------
\ccTypes

\ccSetThreeColumns{typedef I::Circle}{Circle;M}{}
\ccPropagateThreeToTwoColumns

\ccNestedType{I}{Interface type.}

\ccTypedef{typedef I::Point  Point; }{Point  type.}
%\ccTypedef{typedef I::Circle Circle;}{Circle type.}

The following types denote iterators that allow to traverse all points
and support points of the smallest enclosing circle, resp. The
iterators are non-mutable and their value type is \ccc{Point}. The
iterator category is given in parentheses.

\ccSetTwoColumns{CGAL_Min_circle_2<I>:: Support_point_const_iterator}{}

\ccNestedType{Point_const_iterator}{(bidirectional).}

\ccNestedType{Support_point_const_iterator}{(random access).}

% -----------------------------------------------------------------------------
\ccCreation
\ccCreationVariable{min_circle}

A \ccClassTemplateName\ object can be created from an arbitrary point
set $P$ and by specialized construction methods expecting no, one, two
or three points as arguments. The latter methods can be useful for
reconstructing $mc(P)$ from a given support set $S$ of $P$.

\ccSetThreeColumns{CGAL_Bounded_side}{}{
  returns \ccc{CGAL_ON_BOUNDED_SIDE}, \ccc{CGAL_ON_BOUNDARY},}
\ccPropagateThreeToTwoColumns

\ccConstructor{ CGAL_Min_circle_2( );}{
        introduces a variable \ccVar\ of type \ccClassTemplateName.
        It is initialized to
        $mc($\ccTexHtml{$\emptyset$}{&Oslash;}$)$, the empty set.
        \ccPostcond \ccc{\ccVar.is_empty()} = \ccc{true}.}

\ccHidden
\ccConstructor{ CGAL_Min_circle_2( const CGAL_Min_circle_2<I>&);}{
        copy constructor.}

\ccConstructor{ CGAL_Min_circle_2( Point const& p);}{
        introduces a variable \ccVar\ of type \ccClassTemplateName.
        It is initialized to $mc(\{p\})$, the set $\{p\}$.
        \ccPostcond \ccc{\ccVar.is_degenerate()} = \ccc{true}.}

\ccConstructor{ CGAL_Min_circle_2( Point const& p1,
                                   Point const& p2);}{
        introduces a variable \ccVar\ of type \ccClassTemplateName.
        It is initialized to $mc(\{p1,p2\})$, the circle with diameter
        equal to the segment connecting $p1$ and $p2$.}

\ccConstructor{ CGAL_Min_circle_2( Point const& p1,
                                   Point const& p2,
                                   Point const& p3);}{
        introduces a variable \ccVar\ of type \ccClassTemplateName.
        It is initialized to $mc(\{p1,p2,p3\})$.}

\ccUnchecked
\ccConstructor{ template < class InputIterator >
                CGAL_Min_circle_2( InputIterator first,
                                   InputIterator last,
                                   bool randomize = false,
                                   CGAL_Random& random = CGAL_Random());}{
        introduces a variable \ccVar\ of type \ccClassTemplateName. It
        is initialized to $mc(P)$ with $P$ being the set of points in
        the range [\ccc{first},\ccc{last}). If \ccc{randomize} is
        \ccc{true}, a random permutation of $P$ is computed in
        advance, using the random numbers generator \ccc{random}.
        Usually, this will not be necessary, however, the algorithm's
        efficiency depends on the order in which the points are
        processed, and a bad order might lead to extremely poor
        performance (see example below).
        \ccPrecond The value type of \ccc{first} and \ccc{last} is
        \ccc{Point}.}

\emph{Note:} Since most compilers do not support member templates yet,
we provide specialized constructors instead. In the current release
there are constructors for C arrays (using pointers as iterators) and
for STL sequence containers \ccc{vector<Point>} and \ccc{list<Point>}.

\ccHidden
\ccConstructor{ CGAL_Min_circle_2( const Point* first,
                                   const Point* last,
                                   bool randomize = false,
                                   CGAL_Random& random = CGAL_Random());}{
        STL-like constructor for random access iterators.}

\ccHidden
\ccConstructor{ CGAL_Min_circle_2( list<Point>::const_iterator first,
                                   list<Point>::const_iterator last,
                                   bool randomize = false,
                                   CGAL_Random& random = CGAL_Random());}{
        STL-like constructor for sequence container list<Point>.}

\ccHidden
\ccMemberFunction{ CGAL_Min_circle_2<I>&
                   operator = ( const CGAL_Min_circle_2<I>&);}{
        assignment operator.}

% -----------------------------------------------------------------------------
\ccAccessFunctions

\ccMemberFunction{ int number_of_points( ) const;}{
        returns the number of points of \ccVar, i.e.\ $|P|$.}

\ccMemberFunction{ int number_of_support_points( ) const;}{
        returns the number of support points of \ccVar, i.e.\ $|S|$.}

\ccMemberFunction{ Point_const_iterator  points_begin() const;}{
        returns an iterator referring to the first point of \ccVar.}

\ccMemberFunction{ Point_const_iterator  points_end() const;}{
        returns the corresponding past-the-end iterator.}

\ccMemberFunction{ Support_point_const_iterator support_points_begin() const;}{
        returns an iterator referring to the first support point of \ccVar.}

\ccMemberFunction{ Support_point_const_iterator support_points_end() const;}{
        returns the corresponding past-the-end iterator.}

\ccMemberFunction{ Point const& support_point( int i) const;}{
        returns the \ccc{i}-th support point of \ccVar. Between two
        insert operations any call to \ccc{\ccVar.support_point(i)}
        with the same \ccc{i} returns the same point.
        \ccPrecond $0 \leq i <$ \ccc{\ccVar.number_of_support_points()}.}

%\ccMemberFunction{ Circle const& circle( ) const;}{
%        returns an oriented circle with same center $c$ and same
%        squared radius $r$ as \ccVar\ and positive orientation.
%        \ccPrecond \ccc{\ccVar.is_empty()} = \ccc{false}.}

\ccMemberFunction{ CGAL_Bbox_2 bbox( ) const;}{
        returns a bounding box containing \ccVar.
        \ccPrecond \ccc{\ccVar.is_empty()} = \ccc{false}.}

% -----------------------------------------------------------------------------
\ccHeading{Predicates}

%The following predicates imitate the corresponding ones of the class
%\ccc{Circle}, with the exception of \ccc{is_empty()}
%which is not present in \ccc{Circle}, because objects of
%this class cannot be empty.

By definition, an empty \ccClassTemplateName\ has no boundary and no
bounded side, i.e.\ its unbounded side equals the whole plane $\E_2$.

\ccMemberFunction{ CGAL_Bounded_side
                   bounded_side( Point const& p) const;}{
        returns \ccc{CGAL_ON_BOUNDED_SIDE},
        \ccc{CGAL_ON_BOUNDARY}, or
        \ccc{CGAL_ON_UNBOUNDED_SIDE} iff \ccc{p} lies inside,
        on the boundary, or outside of \ccVar, resp.}

\ccMemberFunction{ bool has_on_bounded_side( Point const& p) const;}{
        returns \ccc{true}, iff \ccc{p} lies inside \ccVar.}

\ccMemberFunction{ bool has_on_boundary( Point const& p) const;}{
        returns \ccc{true}, iff \ccc{p} lies on the boundary
        of \ccVar.}

\ccMemberFunction{ bool
                   has_on_unbounded_side( Point const& p) const;}{
        returns \ccc{true}, iff \ccc{p} lies outside of \ccVar.}

\ccMemberFunction{ bool is_empty( ) const;}{
        returns \ccc{true}, iff \ccVar\ is empty (this implies
        degeneracy).}

\ccMemberFunction{ bool is_degenerate( ) const;}{
        returns \ccc{true}, iff \ccVar\ is degenerate, i.e. if
        \ccVar\ is empty or equal to a single point, equivalently if
        the number of support points is less than 2.}

% -----------------------------------------------------------------------------
\ccHeading{Modifiers}

New points can be added to an existing $\ccVar$, allowing to build
$mc(P)$ incrementally, e.g.\ if $P$ is not known in advance. Compared
to the direct creation of $mc(P)$, this is not much slower, because
the construction method is incremental itself.

\ccMemberFunction{ void insert( Point const& p);}{
        inserts \ccc{p} in \ccVar\ and recomputes the smallest
        enclosing circle.}

% -----------------------------------------------------------------------------
\ccHeading{Validity Check}

\ccMemberFunction{ bool is_valid( bool verbose = false,
                                  int  level   = 0    ) const;}{
        checks \ccVar\ for validity. It returns \ccc{true}, iff
        (a) \ccVar\ contains all points of its defining set $P$, (b)
        \ccVar\ is the smallest circle spanned by its support set $S$,
        and (c) $S$ is minimal, i.e.\ no support point is redundant.
        If \ccc{verbose} is \ccc{true}, error messages are written to
        standard error stream. The second parameter \ccc{level} is not
        used, we provide it only for consistency with interfaces of
        other classes.}

% -----------------------------------------------------------------------------
\ccImplementation

We implement the algorithm of Welzl, with move-to-front
heuristic~\cite{Welzl}. If randomization is chosen, the creation time
is almost always linear in the number of points. Access functions and
predicates take constant time, inserting a point might take up to
linear time, but substantially less than computing the new smallest
enclosing circle from scratch.

%We provide a specialization with homogeneous representation and
%numbertype \ccc{integer} that uses floating-point filter for the sign
%calculations in the predicates (see~\cite{MehlhornNaeher94}). This
%decreaes computation time by a factor of four compared to the generic
%implementation. If both \ccc{<CGAL/Integer.h>} and
%\ccc{<CGAL/Homogeneous.h>} are included before
%\ccc{<CGAL/Min_circle_2.h>} then the specialization for
%\ccc{CGAL_Min_circle_2< CGAL_Optimisation_default_interface<
%  CGAL_Homogeneous<integer> > >} is used instead of the generic
%implementation.

% -----------------------------------------------------------------------------
\ccExample

To illustrate the creation of \ccClassTemplateName\ and to show that
randomization can be useful in certain cases, we give an example.

\begin{cprog}
    #include <CGAL/Integer.h>
    #include <CGAL/Homogeneous.h>
    #include <CGAL/Min_circle_2.h>

    typedef  CGAL_Homogeneous<integer>  R;
    typedef  CGAL_Point_2<R>            Point;
    typedef  CGAL_Min_circle_2< CGAL_Optimisation_default_interface<R> > Min_circle;

    int     n = 1000;
    Point*  P = new Point[ n];

    for ( int i = 0; i < n; ++i)
        P[ i] = Point( (i%2 == 0 ? i : -i), 0);
    /* (0,0), (-1,0), (2,0), (-3,0), ... */

    Min_circle  mc1( P, P+n);           /* very slow */
    Min_circle  mc2( P, P+n, true);     /* fast      */
\end{cprog}

\end{ccClassTemplate}

% ===== EOF ===================================================================
