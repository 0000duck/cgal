% =============================================================================
% The CGAL Reference Manual
% Section: 2D Smallest Enclosing Circle
% -----------------------------------------------------------------------------
% source: Min_circle_2.tex
% author: Sven Sch�nherr (sven@inf.fu-berlin.de)
% $Id$
% =============================================================================
 
\begin{CCclassTemplate}{CGAL_Min_circle_2<R>}
\ccSection{2D Smallest Enclosing Circle}

\ccDefinition

An object of the class \ccClassTemplateName\ is the unique smallest
enclosing circle of a set of points in two-dimensional euclidean space
$\E_2$.  For point sets $P$ and $B$ we denote by $\textit{mc}(P,B)$
the smallest circle that contains all points of $P$ and has (at least)
the points of $B$ on the boundary. Note that $\textit{mc}(P,B)$ can be
degenerate, i.e.\ $\textit{mc}(P,B) = \emptyset$ if $P \cup B =
\emptyset$ and $\textit{mc}(P,B) = \{p\}$ if $P \cup B = p$. If $B
\neq \emptyset$ then $\textit{mc}(P,B)$ may be undefined, i.e.\ there
is no circle containing $P$ with $B$ on the boundary.

The smallest enclosing circle of a point set $P$ is determined by at
most three points on the boundary. A minimal subset $S$ of $P$ with
$\textit{mc}(S,\emptyset) = \textit{mc}(P,\emptyset)$ is called a {\em
support set}, the points in $S$ are the {\em support points}.  Note
that in general the set $S$ is not unique.

The underlying algorithm can cope with all kinds of input, e.g.\ one
or both of the point sets $P$ or $B$ may be empty, $B$ may contain
more than three points, or some points may occure more than once in
$P$ or $B$. The algorithm computes a support set $S$, which remains
fixed until the next update operation.


\ccCreation
\ccCreationVariable{min_circle}

\ccSetThreeColumns
  {CGAL_Bounded_side}
  {}
  {returns \ccStyle{CGAL_ON_BOUNDED_SIDE}, \ccStyle{CGAL_ON_BOUNDARY},}
\ccPropagateThreeToTwoColumns

\ccStyle{#include <CGAL/Min_circle_2.h>}

\ccConstructor{ CGAL_Min_circle_2( );}{
        introduces a variable \ccVar\ of type \ccClassTemplateName.
        It is initialized to $\textit{mc}(\emptyset,\emptyset)$,
        i.e.\ to the empty set.
        \ccPostcond \ccVar\ccStyle{.is_degenerate()}.}

\ccHidden
\ccConstructor{ CGAL_Min_circle_2( const CGAL_Min_circle_2<R>& min_circle2);}{
        copy constructor.}

\ccConstructor{ CGAL_Min_circle_2( const Point_2<R>& p);}{
        introduces a variable \ccVar\ of type \ccClassTemplateName.
        It is initialized to $\textit{mc}(\emptyset,\{\ccStyle{p}\})$,
        i.e.\ to the set $\{\ccStyle{p}\}$.
        \ccPostcond \ccVar\ccStyle{.is_degenerate()}.}

\ccConstructor{ CGAL_Min_circle_2( const Point_2<R>& p1,
                                   const Point_2<R>& p2);}{
        introduces a variable \ccVar\ of type \ccClassTemplateName.
        It is initialized to
        $\textit{mc}(\emptyset,\{\ccStyle{p1},\ccStyle{p2}\})$,
        i.e.\ to the circle with diameter
        $\overline{\ccStyle{p1}\ccStyle{p2}}$.}

\ccConstructor{ CGAL_Min_circle_2( const Point_2<R>& p1,
                                   const Point_2<R>& p2,
                                   const Point_2<R>& p3);}{
        introduces a variable \ccVar\ of type \ccClassTemplateName.
        It is initialized to $\textit{mc}(\emptyset,
        \{\ccStyle{p1},\ccStyle{p2},\ccStyle{p3}\})$,
        i.e.\ to the unique circle with \ccStyle{p1}, \ccStyle{p2} and
        \ccStyle{p3} on the boundary.}

\ccConstructor{ CGAL_Min_circle_2( forward_iterator< Point_2<R> > first,
                                   forward_iterator< Point_2<R> > last,
                                   bool randomize = false);}{
        introduces a variable \ccVar\ of type \ccClassTemplateName. It
        is initialized to $\textit{mc}(P,\emptyset)$ with $P$ being
        the set of points in the range
        $[\ccStyle{first},\ccStyle{last})$. If \ccStyle{randomize} is
        \ccStyle{true}, a random permutation of $P$ is computed in
        advance.}

\ccConstructor{ CGAL_Min_circle_2( forward_iterator< Point_2<R> > p_first,
                                   forward_iterator< Point_2<R> > p_last,
                                   forward_iterator< Point_2<R> > b_first,
                                   forward_iterator< Point_2<R> > b_last,
                                   bool randomize = false);}{
        introduces a variable \ccVar\ of type \ccClassTemplateName. It
        is initialized to $\textit{mc}(P,B)$ with $P$ being the set of
        points in the range $[\ccStyle{p_first},\ccStyle{p_last})$ and
        $B$ being the set of points in the range
        $[\ccStyle{b_first},\ccStyle{b_last})$. If \ccStyle{randomize}
        is \ccStyle{true}, a random permutation of $P$ is computed in
        advance.}


\ccHeading{Access operations}

\ccMemberFunction{ int number_of_points( ) const;}{
        returns the number of points of \ccVar, i.e.\ $|P|+|B|$.}

\ccMemberFunction{ int number_of_support_points( ) const;}{
        returns the number of support points of \ccVar, i.e.\ $|S|$.}

\ccMemberFunction{ const CGAL_Point_2<R>& point( int i) const;}{
        returns the \ccStyle{i}'th point of \ccVar. Between two update
        operations any call to \ccVar\ccStyle{.point(i)} with the same
        \ccStyle{i} returns the same point.
        \ccPrecond $0 \leq \ccStyle{i} < \ccVar\ccStyle{.number_of_points()}$.}

\ccMemberFunction{ const CGAL_Point_2<R>& support_point( int i) const;}{
        returns the \ccStyle{i}'th support point of \ccVar. Between
        two update operations any call to
        \ccVar\ccStyle{.support_point(i)} with the same \ccStyle{i}
        returns the same point.
        \ccPrecond $0 \leq \ccStyle{i} <
        \ccVar\ccStyle{.number_of_support_points()}$.}

\ccMemberFunction{ const CGAL_Point_2<R>& operator [] ( int i) const;}{
        returns \ccVar\ccStyle{.point( i)}.}

\ccMemberFunction{ CGAL_Circle_2<R> circle( ) const;}{
        returns an oriented circle with same center $c$ and same
        squared radius $r$ as \ccVar\ and positive orientation. If
        \ccVar\ is the empty set, $c$ is undefined und $r$ is set to
        zero. If \ccVar\ contains exactly one point $p$, $c$ is set to
        $p$ and $r$ is set to zero.}

\ccMemberFunction{ CGAL_Bbox_2 bbox( ) const;}{
        returns a bounding box containing \ccVar.}


\ccHeading{Update operations}

\ccMemberFunction{ void insert( const CGAL_Point_2<R>& p);}{
        inserts \ccStyle{p} in \ccVar\ and recomputes the smallest
        enclosing circle.}


\ccHeading{Tests}

\ccMemberFunction{ CGAL_Bounded_side
                   bounded_side( const CGAL_Point_2<R>& p) const;}{
        returns \ccStyle{CGAL_ON_BOUNDED_SIDE},
        \ccStyle{CGAL_ON_BOUNDARY}, or
        \ccStyle{CGAL_ON_UNBOUNDED_SIDE} iff \ccStyle{p} lies inside,
        on the boundary, or outside of \ccVar, respectively.}

\ccMemberFunction{ bool has_on_bounded_side( const CGAL_Point_2<R>& p) const;}{
        returns \ccStyle{true}, iff \ccStyle{p} lies inside \ccVar.}

\ccMemberFunction{ bool has_on_boundary( const CGAL_Point_2<R>& p) const;}{
        returns \ccStyle{true}, iff \ccStyle{p} lies on the boundary
        of \ccVar.}

\ccMemberFunction{ bool
                   has_on_unbounded_side( const CGAL_Point_2<R>& p) const;}{
        returns \ccStyle{true}, iff \ccStyle{p} lies outside of \ccVar.}

\ccMemberFunction{ bool is_degenerate( ) const;}{
        returns \ccStyle{true}, iff \ccVar\ is degenerate.}

\ccMemberFunction{ bool is_undefined( ) const;}{
        returns \ccStyle{true}, iff \ccVar\ is undefined.}

\end{CCclassTemplate}

% ===== EOF ===================================================================
