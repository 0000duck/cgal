% =============================================================================
% The CGAL Reference Manual
% Chapter: Optimisation
% Section: 2D Smallest Enclosing Circle
% -----------------------------------------------------------------------------
% file  : doc_tex/basic/Optimisation/Min_circle_2.tex
% author: Bernd G�rtner, Sven Sch�nherr (sven@inf.fu-berlin.de)
% -----------------------------------------------------------------------------
% $Revision$
% $Date$
% =============================================================================

\begin{ccClassTemplate}{CGAL_Min_circle_2<Traits>}
\ccSection{2D Smallest Enclosing Circle}
\label{sec:min_circle_2_spec}

% -----------------------------------------------------------------------------
\ccDefinition

An object of the class \ccClassTemplateName\ is the unique circle of
smallest area enclosing a finite set of points in two-dimensional
Euclidean plane $\E_2$. For a point set $P$ we denote by $mc(P)$ the
smallest circle that contains all points of $P$. Note that $mc(P)$ can
be degenerate, i.e.\ $mc(P)=\mbox{\ccTexHtml{$\;\emptyset$}{&Oslash;}}$
if $P=\mbox{\ccTexHtml{$\;\emptyset$}{&Oslash;}}$ and $mc(P)=\{p\}$ if
$P=\{p\}$.

An inclusion-minimal subset $S$ of $P$ with $mc(S)=mc(P)$ is called a
\emph{support set}, the points in $S$ are the \emph{support points}.
A support set has size at most three, and all its points lie on the
boundary of $mc(P)$. If $mc(P)$ has more than three points on the
boundary, neither the support set nor its size are necessarily unique.

The underlying algorithm can cope with all kinds of input, e.g.\
$P$ may be empty or points may occur more than once. The algorithm
computes a support set $S$ which remains fixed until the next insert
or clear operation.

\emph{Note:} In this release correct results are only guaranteed if
exact arithmetic is used, see Section~\ref{sec:opt_traits_impl}.

\ccInclude{CGAL/Min_circle_2.h}

% -----------------------------------------------------------------------------
\ccHeading{Traits Class}

The template parameter \ccc{Traits} is a traits class that defines the
abstract interface between the optimisation algorithm and the
primitives it uses.  For example \ccc{Traits::Point} is a mapping on a
point class. Think of it as 2D points in the Euclidean plane.

We provide a traits class implementation using the \cgal\ 2D kernel as
described in Section~\ref{sec:opt_traits_impl}. Traits class adapters
to user supplied point classes are available, see
Section~\ref{sec:opt_traits_adapt}. Customizing own traits classes for
optimisation algorithms can be done according to the requirements for
traits classes listed in Section~\ref{sec:opt_traits_req}.

% -----------------------------------------------------------------------------
\ccTypes

\ccNestedType{Traits}{}

\ccGlueBegin
\ccUnchecked
\ccTypedef{typedef Traits::Point   Point; }{Point  type.}
\ccUnchecked
\ccTypedef{typedef Traits::Circle  Circle;}{Circle type.}
\ccGlueEnd

The following types denote iterators that allow to traverse all points
and support points of the smallest enclosing circle, resp. The
iterators are non-mutable and their value type is \ccc{Point}. The
iterator category is given in parentheses.

\ccSetTwoColumns{CGAL_Min_circle_2<Traits>:: Support_point_iterator}{}

\ccNestedType{Point_iterator}{(bidirectional).}

\ccNestedType{Support_point_iterator}{(random access).}

\ccPropagateThreeToTwoColumns

% -----------------------------------------------------------------------------
\ccCreation
\ccCreationVariable{min_circle}

A \ccClassTemplateName\ object can be created from an arbitrary point
set $P$ and by specialized construction methods expecting no, one, two
or three points as arguments. The latter methods can be useful for
reconstructing $mc(P)$ from a given support set $S$ of $P$.

\ccUnchecked
\ccConstructor{ template < class InputIterator >
                CGAL_Min_circle_2( InputIterator  first,
                                   InputIterator  last,
                                   bool           randomize = false,
                                   CGAL_Random&   random    = CGAL_random,
                                   Traits const&  traits    = Traits()   );}{
        creates a variable \ccVar\ of type \ccClassTemplateName%
\SaveSpaceByHand{$\!$}{}.
        It is initialized to $mc(P)$ with $P$ being the set of points
        in the range [\ccc{first},\ccc{last}). If \ccc{randomize} is
        \ccc{true}, a random permutation of $P$ is computed in
        advance, using the random numbers generator \ccc{random}.
        Usually, this will not be necessary, however, the algorithm's
        efficiency depends on the order in which the points are
        processed, and a bad order might lead to extremely poor
        performance (see example below).
        \ccPrecond The value type of \ccc{first} and \ccc{last}
        is \ccc{Point}.}

\emph{Note:} Since most compilers do not support template member functions
yet, we provide specialized constructors instead. In the current release
there are constructors for C arrays (using pointers as iterators), for the
\stl\ sequence containers \ccc{vector<Point>} and \ccc{list<Point>} and for
the \stl\ input stream iterator \ccc{istream_iterator<Point>}.

\ccHidden
\ccConstructor{ CGAL_Min_circle_2( const Point*   first,
                                   const Point*   last,
                                   bool           randomize = false,
                                   CGAL_Random&   random    = CGAL_random,
                                   Traits const&  traits    = Traits()   );}{
        STL-like constructor for random access iterators.}

\ccHidden
\ccConstructor{ CGAL_Min_circle_2( list<Point>::const_iterator  first,
                                   list<Point>::const_iterator  last,
                                   bool           randomize = false,
                                   CGAL_Random&   random    = CGAL_random,
                                   Traits const&  traits    = Traits()    );}{
        STL-like constructor for sequence container list<Point>.}

\ccHidden
\ccConstructor{ CGAL_Min_circle_2( istream_iterator<Point,ptrdiff_t>  first,
                                   istream_iterator<Point,ptrdiff_t>  last,
                                   bool           randomize = false,
                                   CGAL_Random&   random    = CGAL_random,
                                   Traits const&  traits    = Traits()    );}{
        STL-like constructor for input stream iterator
        istream_iterator<Point,ptrdiff_t>.}


\ccConstructor{ CGAL_Min_circle_2( Traits const&  traits = Traits());}{
        creates a variable \ccVar\ of type \ccClassTemplateName%
\SaveSpaceByHand{$\!$}{}.
        It is initialized to
        $mc(\mbox{\ccTexHtml{$\emptyset$}{&Oslash;}})$, the empty set.
        \ccPostcond  \ccVar\ccc{.is_empty()} = \ccc{true}.}

\ccConstructor{ CGAL_Min_circle_2( Point  const&  p,
                                   Traits const&  traits = Traits());}{
        creates a variable \ccVar\ of type \ccClassTemplateName%
\SaveSpaceByHand{$\!$}{}.
        It is initialized to $mc(\{p\})$, the set $\{p\}$.
        \ccPostcond  \ccVar\ccc{.is_degenerate()} = \ccc{true}.}

\ccConstructor{ CGAL_Min_circle_2( Point  const&  p1,
                                   Point  const&  p2,
                                   Traits const&  traits = Traits());}{
        creates a variable \ccVar\ of type \ccClassTemplateName%
\SaveSpaceByHand{$\!$}{}.
        It is initialized to $mc(\{p1,p2\})$, the circle with diameter
        equal to the segment connecting $p1$ and $p2$.}

\ccConstructor{ CGAL_Min_circle_2( Point  const&  p1,
                                   Point  const&  p2,
                                   Point  const&  p3,
                                   Traits const&  traits = Traits());}{
        creates a variable \ccVar\ of type \ccClassTemplateName%
\SaveSpaceByHand{$\!$}{}.
        It is initialized to $mc(\{p1,p2,p3\})$.}

\ccUnchecked
\ccHidden
\ccConstructor{ CGAL_Min_circle_2( CGAL_Min_circle_2<Traits> const&);}{
        copy constructor.}

\ccHidden
\ccConstructor{ ~CGAL_Min_circle_2( );}{
        destructor.}

\ccUnchecked
\ccHidden
\ccMemberFunction{ CGAL_Min_circle_2<Traits>&
                   operator = ( CGAL_Min_circle_2<Traits> const&);}{
        assignment operator.}

% -----------------------------------------------------------------------------
\ccAccessFunctions

\ccMemberFunction{ int  number_of_points( ) const;}{
        returns the number of points of \ccVar, i.e.\ $|P|$.}

\ccMemberFunction{ int  number_of_support_points( ) const;}{
        returns the number of support points of \ccVar, i.e.\ $|S|$.}

\ccGlueBegin
\ccMemberFunction{ Point_iterator  points_begin() const;}{
        returns an iterator referring to the first point of \ccVar.}
%
\ccMemberFunction{ Point_iterator  points_end() const;}{
        returns the corresponding past-the-end iterator.}
\ccGlueEnd

\ccGlueBegin
\ccMemberFunction{ Support_point_iterator  support_points_begin() const;}{
        returns an iterator referring to the first support point of
\linebreakByHand
                                                                    \ccVar.}
%
\ccMemberFunction{ Support_point_iterator  support_points_end() const;}{
        returns the corresponding past-the-end iterator.}
\ccGlueEnd

\ccMemberFunction{ Point const&  support_point( int i) const;}{
        returns the \ccc{i}-th support point of \ccVar. Between two
        modifying operations (see below) any call to
        \ccVar\ccc{.support_point(i)} with the same \ccc{i} returns
        the same point.
        \ccPrecond $0 \leq i< \mbox{\ccVar\ccc{.number_of_support_points()}}$.}

\ccMemberFunction{ Circle const&  circle( ) const;}{
        returns the current circle of \ccVar.}

% -----------------------------------------------------------------------------
\ccPredicates

By definition, an empty \ccClassTemplateName\ has no boundary and no
bounded side, i.e.\ its unbounded side equals the whole plane $\E_2$.

\ccMemberFunction{ CGAL_Bounded_side
                   bounded_side( Point const& p) const;}{
        returns \ccc{CGAL_ON_BOUNDED_SIDE},%
\SaveSpaceByHand{}{ }%
                                            \ccc{CGAL_ON_BOUNDARY}%
\SaveSpaceByHand{$\!$}{},
                                                                    or
        \ccc{CGAL_ON_UNBOUNDED_SIDE} iff \ccc{p} lies properly inside,
        on the boundary, or properly outside of \ccVar, resp.}

\ccMemberFunction{ bool  has_on_bounded_side( Point const& p) const;}{
        returns \ccc{true}, iff \ccc{p} lies properly inside \ccVar.}

\ccMemberFunction{ bool  has_on_boundary( Point const& p) const;}{
        returns \ccc{true}, iff \ccc{p} lies on the boundary
        of \ccVar.}

\ccMemberFunction{ bool  has_on_unbounded_side( Point const& p) const;}{
        returns \ccc{true}, iff \ccc{p} lies properly outside of \ccVar.}

\ccMemberFunction{ bool  is_empty( ) const;}{
        returns \ccc{true}, iff \ccVar\ is empty (this implies
        degeneracy).}

\ccMemberFunction{ bool  is_degenerate( ) const;}{
        returns \ccc{true}, iff \ccVar\ is degenerate, i.e.\ if
        \ccVar\ is empty or equal to a single point, equivalently if
        the number of support points is less than 2.}

% -----------------------------------------------------------------------------
\ccModifiers

New points can be added to an existing \ccVar, allowing to build
$mc(P)$ incrementally, e.g.\ if $P$ is not known in advance. Compared
to the direct creation of $mc(P)$, this is not much slower, because
the construction method is incremental itself.

\ccMemberFunction{ void  insert( Point const& p);}{
        inserts \ccc{p} into \ccVar\ and recomputes the smallest
        enclosing circle.}

\ccUnchecked
\ccMemberFunction{ template < class InputIterator >
                   void  insert( InputIterator  first,
                                 InputIterator  last );}{
        inserts the points in the range [\ccc{first},\ccc{last})
        into \ccVar\ and recomputes the smallest enclosing circle by
        calling \ccc{insert(p)} for each point \ccc{p} in
        [\ccc{first},\ccc{last}).
        \ccPrecond  The value type of \ccc{first} and \ccc{last} is
        \ccc{Point}.}

\emph{Note:} Since most compilers do not support template member functions
yet, we provide specialized \ccc{insert} functions instead. In the current
release there are \ccc{insert} functions for C arrays (using pointers as
iterators), for the \stl\ sequence containers \ccc{vector<Point>} and
\ccc{list<Point>} and for the \stl\ input stream iterator
\ccc{istream_iterator<Point>}.

\ccHidden
\ccMemberFunction{ void  insert( const Point*  first,
                                 const Point*  last );}{
        STL-like `insert' function for random access iterators.}

\ccHidden
\ccMemberFunction{ void  insert( list<Point>::const_iterator  first,
                                 list<Point>::const_iterator  last );}{
        STL-like `insert' function for sequence container list<Point>.}

\ccHidden
\ccMemberFunction{ void  insert( istream_iterator<Point,ptrdiff_t>  first,
                                 istream_iterator<Point,ptrdiff_t>  last );}{
        STL-like `insert' function for input stream iterator
        istream_iterator<Point,ptrdiff_t>.}

\ccMemberFunction{ void clear( );}{
        deletes all points in \ccVar\ and sets \ccVar\ to the empty set.
        \ccPostcond  \ccVar\ccc{.is_empty()} = \ccc{true}.}

% -----------------------------------------------------------------------------
\ccHeading{Validity Check}

An object \ccVar\ is valid, iff
\begin{itemize}
  \item \ccVar\ contains all points of its defining set $P$,
  \item \ccVar\ is the smallest circle spanned by its support set $S$, and
  \item $S$ is minimal, i.e.\ no support point is redundant.
\end{itemize}
Using the traits class implementation for the \cgal\ kernel with exact
arithmetic as described in Section~\ref{sec:opt_traits_impl}
guarantees validity of \ccVar. The following function is mainly
intended for debugging user supplied traits classes but also for
convincing the anxious user that the traits class implementation is
correct.
\begin{ccAdvanced}
\ccMemberFunction{ bool is_valid( bool verbose = false,
                                  int  level   = 0    ) const;}{
        returns \ccc{true}, iff \ccVar\ is valid. If \ccc{verbose}
        is \ccc{true}, some messages concerning the performed checks
        are written to standard error stream. The second parameter
        \ccc{level} is not used, we provide it only for consistency
        with interfaces of other classes.}
\end{ccAdvanced}

% -----------------------------------------------------------------------------
\ccHeading{Miscellaneous}

\ccMemberFunction{ Traits const&  traits( ) const;}{
        returns a const reference to the traits class object.}

% -----------------------------------------------------------------------------
\ccHeading{I/O}

\ccFunction{ ostream& operator << ( ostream& os,
                                    CGAL_Min_circle_2<Traits> const&
                                        min_circle);}{
        writes \ccVar\ to output stream \ccc{os}.
        \ccPrecond  The output operator is defined for \ccc{Point}
        (and for \ccc{Circle}, if pretty printing is used).}

\ccFunction{ istream& operator >> ( istream& is,
                                    CGAL_Min_circle_2<Traits> &min_circle);}{
        reads \ccVar\ from input stream \ccc{is}.
        \ccPrecond  The input operator is defined for \ccc{Point}.}

\ccInclude{CGAL/IO/Window_stream.h}

\ccUnchecked
\ccFunction{ CGAL_Window_stream&
             operator << ( CGAL_Window_stream& ws,
                           CGAL_Min_circle_2<Traits> const& min_circle);}{
        writes \ccVar\ to window stream \ccc{ws}.
        \ccPrecond  The window stream output operator is defined for
        \ccc{Point} and \ccc{Circle}.}

% -----------------------------------------------------------------------------
\ccImplementation

We implement the algorithm of Welzl, with move-to-front
heuristic~\cite{w-sedbe-91a}. If randomization is chosen, the creation
time is almost always linear in the number of points. Access functions
and predicates take constant time, inserting a point might take up to
linear time, but substantially less than computing the new smallest
enclosing circle from scratch. The clear operation and the check for
validity each takes linear time.

% -----------------------------------------------------------------------------
\ccExample

To illustrate the creation of \ccClassTemplateName\ and to show that
randomization can be useful in certain cases, we give an example.

\begin{cprog}
  #include <CGAL/Gmpz.h>
  #include <CGAL/Homogeneous.h>
  #include <CGAL/Point_2.h>
  #include <CGAL/Optimisation_traits_2.h>
  #include <CGAL/Min_circle_2.h>

  typedef  CGAL_Gmpz                      NT;
  typedef  CGAL_Homogeneous<NT>           R;
  typedef  CGAL_Point_2<R>                Point;
  typedef  CGAL_Optimisation_traits_2<R>  Traits;
  typedef  CGAL_Min_circle_2<Traits>      Min_circle;

  int     n = 1000;
  Point*  P = new Point[ n];

  for ( int i = 0; i < n; ++i)
      P[ i] = Point( (i%2 == 0 ? i : -i), 0);
  /* (0,0), (-1,0), (2,0), (-3,0), ... */

  Min_circle  mc1( P, P+n);           /* very slow */
  Min_circle  mc2( P, P+n, true);     /* fast      */

  delete[] P;
\end{cprog}

\end{ccClassTemplate}

% ===== EOF ===================================================================
