% =============================================================================
% The CGAL Reference Manual
% Chapter: Geometric Optimisation
% Section: Smallest Enclosing Circles
% Subsec.: Requirements of Traits Classes for 2D Smallest Enclosing Circle
% -----------------------------------------------------------------------------
% file  : doc_tex/basic/Optimisation/requirements_Min_circle_2.tex
% author: Sven Sch�nherr <sven@inf.fu-berlin.de>
% -----------------------------------------------------------------------------
% $CGAL_Package: Min_circle_2 WIP $
% $Revision$
% $Date$
% =============================================================================

\ccHtmlNoClassToc
\begin{ccHtmlClassFile}{requirements_Min_circle_2.html}{%
    Requirements of Traits Classes for 2D Smallest Enclosing Circle}
\begin{ccTexOnly}
  \subsection{Requirements of Traits Classes for 2D Smallest Enclosing Circle}
  \label{sec:min_circle_2_traits_req}
\end{ccTexOnly}
\begin{ccHtmlOnly}
  <H2>Requirements of Traits Classes for 2D Smallest Enclosing Circle</H2><P>
\end{ccHtmlOnly}

\cgalSetMinCircleLayout

The class template \ccc{Min_circle_2} is parameterized with a
\ccc{Traits} class which defines the abstract interface between the
optimisation algorithm and the primitives it uses. The following
requirements catalog lists the primitives, i.e.\ types, member
functions etc., that must be defined for a class that can be used to
parameterize \ccc{Min_circle_2}.  A traits class implementation
using the 2D \cgal-kernel is available and described in
Section~\ref{sec:min_circle_2_traits_impl_2D_kernel}. In addition, we
provide traits class adapters to user supplied point classes, see
Sections~\ref{sec:min_circle_2_traits_adapterC2}
and~\ref{sec:min_circle_2_traits_adapterH2}. Both, the implementation
and the adapters, can be used as a starting point for customizing own
traits classes, e.g.\ through derivation and specialization.

% =============================================================================

\ccHtmlNoClassToc
\ccHtmlNoClassLinks
\ccHtmlNoClassIndex
\begin{ccClass}{Traits}
\subsection*{Traits Class (\ccClassTemplateName)}

% -----------------------------------------------------------------------------
\ccDefinition

A class that satisfies the requirements of a traits class for
\ccc{Min_circle_2} must provide the following primitives.

% -----------------------------------------------------------------------------
\ccTypes

\ccNestedType{Point}{
        The point type must provide default and copy constructor,
        assignment and equality test.}

\ccNestedType{Circle}{
        The circle type must fulfill the requirements listed below
        in the next section.}

In addition, if I/O is used, the corresponding I/O operators for
\ccc{Point} and \ccc{Circle} have to be provided, see topic
\textbf{I/O} in Section~\ref{sec:min_circle_2_spec}.

% -----------------------------------------------------------------------------
\ccHeading{Variables}

\ccVariable{ Circle  circle;}{
        The actual circle. This variable is maintained by the algorithm,
        the user should neither access nor modify it directly.}

% -----------------------------------------------------------------------------
\ccCreation
\ccCreationVariable{traits}

Only default and copy constructor are required. Note that further
constructors can be provided.

\ccConstructor{ Traits( );}{A default constructor.}

\ccConstructor{ Traits( const Traits&);}{A copy constructor.}

% -----------------------------------------------------------------------------
\ccOperations

The following predicate is only needed, if the member function
\ccc{is_valid} of \ccc{Min_circle_2} is used.

\ccMemberFunction{ Orientation  orientation( const Point& p,
                                                  const Point& q,
                                                  const Point& r) const;}{
        returns constants \ccc{LEFTTURN}, \ccc{COLLINEAR},
        or \ccc{RIGHTTURN} iff \ccc{r} lies properly to the left of,
        on, or properly to the right of the oriented line through
        \ccc{p} and \ccc{q}, resp.}

\end{ccClass}

% =============================================================================

\ccHtmlNoClassToc
\ccHtmlNoClassLinks
\ccHtmlNoClassIndex
\begin{ccClass}{Circle}
\subsection*{Circle Type (\ccClassTemplateName)}

% -----------------------------------------------------------------------------
\ccDefinition
 
An object of the class \ccClassName\ is a circle in two-dimensional
Euclidean plane $\E_2$. Its boundary splits the plane into a bounded
and an unbounded side. By definition, an empty \ccClassName\ has no
boundary and no bounded side, i.e.\ its unbounded side equals the
whole plane $\E_2$. A \ccClassName\ containing exactly one point~$p$
has no bounded side, its boundary is $\{p\}$, and its unbounded side
equals $\E_2\mbox{\ccTexHtml{$\setminus$}{-}}\{p\}$.

% -----------------------------------------------------------------------------
\ccTypes
 
\ccNestedType{Point}{Point type.}

The following type is only needed, if the member function \ccc{is_valid}
of \ccc{Min_circle_2} is used.

\ccNestedType{Distance}{
        Distance type. The function \ccc{squared_radius} (see below)
        returns an object of this type.}
 
% -----------------------------------------------------------------------------
\ccCreation
\ccCreationVariable{circle}
 
\ccMemberFunction{ void  set( );}{
        sets \ccVar\ to the empty circle.}

\ccMemberFunction{ void  set( const Point& p);}{
        sets \ccVar\ to the circle containing exactly $\{\mbox{\ccc{p}}\}$.}

\ccMemberFunction{ void  set( const Point& p,
                              const Point& q);}{
        sets \ccVar\ to the circle with diameter equal to the segment
        connecting \ccc{p} and \ccc{q}. The algorithm guarantees that
        \ccc{set} is never called with two equal points.}

\ccMemberFunction{ void  set( const Point& p,
                              const Point& q,
                              const Point& r);}{
        sets \ccVar\ to the circle through \ccc{p},\ccc{q},\ccc{r}.
        The algorithm guarantees that \ccc{set} is never called with
        three collinear points.}

% -----------------------------------------------------------------------------
\ccPredicates

\ccMemberFunction{ bool  has_on_unbounded_side( const Point& p) const;}{
        returns \ccc{true}, iff \ccc{p} lies properly outside of \ccVar.}

Each of the following predicates is only needed, if the corresponding
predicate of \ccc{Min_circle_2} is used.

\ccMemberFunction{ Bounded_side
                   bounded_side( const Point& p) const;}{
        returns \ccc{ON_BOUNDED_SIDE}, \ccc{ON_BOUNDARY,}
\newlineByHand
        or \ccc{ON_UNBOUNDED_SIDE} iff \ccc{p} lies properly inside,
\newlineByHand
        on the boundary, or properly outside of \ccVar, resp.}

\ccMemberFunction{ bool  has_on_bounded_side( const Point& p) const;}{
        returns \ccc{true}, iff \ccc{p} lies properly inside \ccVar.}

\ccMemberFunction{ bool  has_on_boundary( const Point& p) const;}{
        returns \ccc{true}, iff \ccc{p} lies on the boundary
        of \ccVar.}

\ccMemberFunction{ bool  is_empty( ) const;}{
        returns \ccc{true}, iff \ccVar\ is empty (this implies
        degeneracy).}

\ccMemberFunction{ bool  is_degenerate( ) const;}{
        returns \ccc{true}, iff \ccVar\ is degenerate, i.e.\ if
        \ccVar\ is empty or equal to a single point.}

% -----------------------------------------------------------------------------
\ccHeading{Additional Operations for Checking}

The following operations are only needed, if the member function
\ccc{is_valid} of \ccc{Min_circle_2} is used.

\ccMemberFunction{ bool  operator == ( const Circle& circle2) const;}{
        returns \ccc{true}, iff \ccVar\ and \ccc{circle2} are equal.}

\ccMemberFunction{ Point  center( ) const;}{
        returns the center of \ccVar.}
 
\ccMemberFunction{ Distance  squared_radius( ) const;}{
        returns the squared radius of \ccVar.}
 
% -----------------------------------------------------------------------------
\ccHeading{I/O}

The following I/O operators are only needed, if the corresponding I/O
operators of \ccc{Min_circle_2} are used.

\ccHtmlNoIndex
\ccFunction{ ostream& operator << ( ostream& os, const Circle& circle);}{
        writes \ccVar\ to output stream \ccc{os}.}

\ccHtmlNoIndex
\ccFunction{ Window_stream&
             operator << ( Window_stream& ws, const Circle& circle);}{
        writes \ccVar\ to window stream \ccc{ws}.}

% -----------------------------------------------------------------------------

\end{ccClass}

% =============================================================================

\end{ccHtmlClassFile}

% ===== EOF ===================================================================
