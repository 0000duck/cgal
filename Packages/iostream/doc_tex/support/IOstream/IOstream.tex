
All classes in the \cgal\ kernel provide input and output operators for
IO streams.  The basic task of such an operator is to produce a
representation of an object that can be written as a sequence of
characters on devices as a console, a file, or a pipe. In \cgal\ 
we distinguish between  a raw ascii, a raw binary and a 
pretty printing format. 

\ccEnum{Mode = {ASCII = 0, BINARY, PRETTY};}{}


In \ccc{ASCII} mode, objects are written as
a set of  numbers, e.g.\ the coordinates of a point or
the coefficients of a line, in a machine independent format. 
In \ccc{BINARY} mode,
data are written in a binary format, e.g.\ a double is represented
as a sequence of four byte. The format depends on the machine. 
The mode  \ccc{PRETTY}
serves mainly for debugging as the type of the geometric
object is written, as well as the data defining the object. For example
for a point at the origin with Cartesian double coordinates, the output
would be \ccc{PointC2(0.0, 0.0)}.  At the moment \cgal\ does not
provide input operations for pretty printed data. By default a stream
is in {\sc ascii} mode.

\ccThree{IO::Mode}{set_mode(std::ios& s, IO::Mode m);}{}

\cgal\ provides the following functions to modify the mode of an IO stream.

\ccFunction{IO::Mode  set_mode(std::ios& s, IO::Mode m);}{}

\ccFunction{IO::Mode  set_ascii_mode(std::ios& s);}{}
\ccGlue
\ccFunction{IO::Mode  set_binary_mode(std::ios& s);}{}
\ccGlue
\ccFunction{IO::Mode  set_pretty_mode(std::ios& s);}{}

The following functions allow to test whether a stream is in a certain mode.

\ccFunction{IO::Mode  get_mode(std::ios& s);}{}

\ccFunction{bool is_ascii(std::ios& s);}{}
\ccGlue
\ccFunction{bool is_binary(std::ios& s);}{}
\ccGlue
\ccFunction{bool is_pretty(std::ios& s);}{}


%\newpage

\ccThree{ostream&}{operator<<( const Class &c);}{}

%\begin{ccClass} {ostream}
\section{Output Operator}
%\ccCreationVariable{os}

%\ccDefinition
\cgal\  defines output operators for classes that are derived
from the class \ccStyle{ostream}. This allows to write to ostreams
as \ccStyle{cout} or \ccStyle{cerr}, as well as to strstreams
and fstreams. 
The output operator is defined for all classes in the \cgal\ kernel and for the class \ccc{Color} as well.
Let \ccc{os}  be an output stream.

\ccFunction{ostream& operator<<(ostream& os, Class c);}
{Inserts object \ccStyle{c} in the stream \ccc{os}. Returns \ccc{os}.}


\ccExample

\begin{cprog}

#include <CGAL/basic.h>
#include <iostream>
#include <fstream>

#include <CGAL/Cartesian.h>
#include <CGAL/Segment_2.h>

typedef CGAL::Point_2< CGAL::Cartesian<double> >     Point;
typedef CGAL::Segment_2< CGAL::Cartesian<double> >   Segment;

int main()
{
    Point p(0,1), q(2,2);
    Segment s(p,q);

    CGAL::set_pretty_mode(std::cout);
    std::cout << p << std::endl << q  << std::endl;

    std::ofstream f("data.txt");
    CGAL::set_binary_mode(f);
    f << s << p ;

    return 1;
}
\end{cprog} 
%\end{ccClass} 


%\newpage
%\begin{ccClass} {istream}
\section{Input Operator}


%\ccDefinition
\cgal\  defines input operators for classes that are derived
from the class \ccStyle{istream}. This allows to read from istreams
as \ccStyle{cin}, as well as from strstreams and fstreams.
The input operator is defined for all classes in the \cgal\ kernel.
Let \ccc{is} be an input stream.

\ccFunction{istream& operator>>(istream& is, Class c);}
   {Extracts object \ccStyle{c} from the stream \ccc{is}. Returns \ccc{is}.} 


\ccExample

\begin{cprog}

#include <CGAL/basic.h>
#include <iostream>
#include <fstream>

#include <CGAL/Cartesian.h>
#include <CGAL/Segment_2.h>

typedef CGAL::Point_2< CGAL::Cartesian<double> >     Point;
typedef CGAL::Segment_2< CGAL::Cartesian<double> >   Segment;

int
main()
{
    Point p, q;
    Segment s;

    CGAL::set_ascii_mode(std::cin);
    std::cin >> p >> q;

    std::ifstream f("data.txt");
    CGAL::set_binary_mode(f);
    f >> s >> p;

    return 1;
}
\end{cprog} 
%\end{ccClass} 



