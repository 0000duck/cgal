\chapter{Sweep lime}

\subsection*{Introduction}
The construction of a planar map can be done using a sweep line
technique rather than iteratively, inserting one curve after another.
The sweep line algorithm performs better, but is done statically and
therefore requires the list of curves in advance.  Due to the sweep
line technique, inserting curves which are interior intersect,
overlap or non $x$-monotone is supported. The interface is given in
chapter ~\ref{I1_ChapterPlanarMap}.

\begin{ccTexOnly}

\subsection*{Concepts}
\ccRefConceptPage{SweepLineTraits_2}\\

\subsection*{Classes}
\ccRefIdfierPage{CGAL::Sweep_curves_base_2<CurveInputIterator,SweepLineTraits_2,Point_plus_2,X_curve_plus_2>}\\
\ccRefIdfierPage{CGAL::Sweep_curves_to_planar_map_2<CurveInputIterator,SweepLineTraits_2,Planar_map_2,Change_notification_2>}\\
\ccRefIdfierPage{CGAL::Sweep_curves_to_subcurves_2<CurveInputIterator,SweepLineTraits_2>}\\
\ccRefIdfierPage{CGAL::Point_plus_handle<Planar_map_2>}\\

\subsection*{Functions}
\ccRefIdfierPage{CGAL::sweep_to_construct_planar_map_2}\\
\ccRefIdfierPage{CGAL::sweep_to_produce_subcurves_2}\\
\ccRefIdfierPage{CGAL::sweep_to_produce_points_2}\\

\end{ccTexOnly}    
