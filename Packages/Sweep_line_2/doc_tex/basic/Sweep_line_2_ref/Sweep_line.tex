% +------------------------------------------------------------------------+
% | Reference manual page: Planar_map.tex (Planar_map)
% +------------------------------------------------------------------------+
% | 
% | Package: pm (Planar_map_2)
% | 
% +------------------------------------------------------------------------+

\ccRefPageBegin

%%RefPage: end of header, begin of main body
% +------------------------------------------------------------------------+


% +========================================================================+
%   Planar Map
% +========================================================================+
\begin{ccRefClass}{Sweep_line_2<CurveInputIterator, SweepLineTraits_2>}
\label{SL_sec:sl}
    
\ccDefinition
    \ccStyle{Sweep_line_2} is a utility class that calculates 
    intersection
    points between given curves. The following operationas are supported:

\begin{itemize}
\item calculate the intersection points between the curves.
\item calculate the interior-disjoint segments induced by $C$.
\item calculate the intersection points and return a list of curves 
participating at each intersection point.
\item query for the existance of an intersection point between any two curves.
\end{itemize}

The \ccStyle{Sweep_line_2} class is implemented in such a way that no
assumptions are made regarding the given curves. For example, the curves may
be overlapping or vertical. There is also no assumtion that the curves are
x-monotone.

The \ccStyle{Sweep_line_2} is parametrized by two parameters. The input 
iterator and  the traits parameter. The 
\ccStyle{CurveInputIterator::type_value} is equivalent to 
\ccStyle{Traits::Curve_2}.

\ccInclude{CGAL/Sweep_line_2.h}

\ccCreation
  \ccCreationVariable{sweepLine}
    
  \ccConstructor{Sweep_line_2<CurveInputIterator, SweepLineTraits_2>();} 
   {constructs an instance of the \ccStyle{Sweep_line_2} class.}
    
  \ccConstructor{Sweep_line_2<CurveInputIterator, SweepLineTraits_2>(SweepLineTraits_2 *traits);}
   {constructs an instance of the \ccStyle{Sweep_line_2} class.}
    

\ccTypes
\ccNestedType{Traits}{traits class.}
\ccNestedType{Curve_2}{a 2D curve as defined in the traits class.}
\ccNestedType{X_monotone_curve_2}{an x-monotone 2D curve as defined in the traits class.}
\ccNestedType{Point_2}{a 2D point as defined in the traits class.}

\ccQueryFunctions

\ccMethod{template <class OutpoutIterator>
          void  get_subcurves(CurveInputIterator curves_begin, 
		      CurveInputIterator curves_end, 
		      OutpoutIterator subcurves, 
		      bool overlapping = false);}
{given a range of curves, this method returns a container of 
interior-disjoint curves that are the product of intersecting 
the specified curves. 
\\The overlapping parameter is an input parameter and is relevant 
only in case there are overlapping curves in the input. If this parameter 
is false (default), any overlapping subcurve is reported only once. If the 
parameter is set to true, the subcurve is reported as many times as it 
is overlapped.
\\\ccStyle{OutpoutIterator::value_type} is equvalent to 
\ccStyle{Traits::X_monotone_curve_2}.}

\ccMethod{  template <class OutpoutIterator>
  void  get_intersection_points(CurveInputIterator curves_begin, 
				CurveInputIterator curves_end, 
				OutpoutIterator points,
				bool endpoints = true,
				bool overlapping = false);}
{given a range of curves, this method returns a list of all intersection
points of the curves. 
\\If endpoints is true (default) the end points
of the curves are reported as intersection points. Otherwise they are 
ommited. 
\\The overlapping parameter is an input parameter and is relevant 
only in case there are overlapping curves in the input. If this parameter 
is false (default), any overlapping subcurve is reported only once. If the 
parameter is set to true, the subcurve is reported as many times as it 
is overlapped.
\\The \ccStyle{OutputIterator::value_type} is equivalent to
\ccStyle{Traits::Point_2}.}


\ccMethod{template <class OutputIterator>
  void  get_intersecting_curves(CurveInputIterator curves_begin, 
				CurveInputIterator curves_end, 
				OutputIterator intersecting_curves,
				bool endpoints = true);}
{given a range of curves this method calculates the intersection points
between the curves. It returns the list of intersection points,
and for each intersection point the list of curves intersecting
at that point is given. 
\\If endpoints is true (default) the end points
of the curves are reported as intersection points. Otherwise they are 
ommited.
\\The \ccStyle{OutputIterator::value_type} is equivalent to a pair of 
\ccStyle{Point_2} and a container (e.g., list, vector) of \ccStyle{Curve_2}.}
 

\ccMethod{  bool  do_curves_intersect(CurveInputIterator curves_begin, 
			    CurveInputIterator curves_end);}
{given a range of curves this method returns true if any two curves intersect,
false otherwise.}

\end{ccRefClass}

% FILL IN THE REST OF THE FILES
% +------------------------------------------------------------------------+
%%RefPage: end of main body, begin of footer
\ccRefPageEnd
% EOF
% +------------------------------------------------------------------------+
