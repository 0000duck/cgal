\chapter{Sweep line}

\subsection*{Introduction}
%The construction of a planar map can be done using a sweep line
%technique rather than iteratively, inserting one curve after another.
%The sweep line algorithm performs better, but is done statically and
%therefore requires the list of curves in advance.  Due to the sweep
%line technique, inserting curves which are interior intersect,
%overlap or non $x$-monotone is supported. 

Given a collection $C$ of possibly intersecting 
(not necessarily $x$-monotone) curves in the plane, 
the sweep line algorithm can produce the collections of their 
intersections and the disjoint-interior subcurves induced by $C$, or rather 
construct the planar map induced by $C$.

A \ccc{Planar Map with Intersections} can be built incrementally by
inserting one curve after the other into the map. 
However, in most cases it is much faster to perform the sweep line algorithm on
the collection of input curves in order to produce their planar map 
rather using the \ccc{Planar_map_with_intersections_2<Planar_map>} 
or \ccc{Arrangement_2<Dcel,Traits,Base_node>} packages.
Constructing a planar map by the sweep line algorithm is done statically and
therefore requires the pre-knowledge of all curves in the input.

%due to an 
%efficient calculation of intersections and avoiding the usage of 
%point location operations when inserting curves to the planar map. 


The interface is given in chapter ~\ref{I1_ChapterPlanarMap}.

\begin{ccTexOnly}

\subsection*{Concepts}
\ccRefConceptPage{SweepLineTraits_2}

\subsection*{Functions}
\ccRefIdfierPage{CGAL::sweep_to_produce_subcurves_2}\\
\ccRefIdfierPage{CGAL::sweep_to_produce_points_2}\\
\ccRefIdfierPage{CGAL::sweep_to_construct_planar_map_2}\\

\end{ccTexOnly}    

