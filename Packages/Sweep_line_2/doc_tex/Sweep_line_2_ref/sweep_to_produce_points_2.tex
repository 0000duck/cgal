% +------------------------------------------------------------------------+
% | Reference manual page: sweep_to_produce_points_2.tex
% +------------------------------------------------------------------------+
% | 17.02.2002   Author
% | Package: Sweep_line_2
% | 
\RCSdef{\RCSsweeptoproducepointsRev}{$Revision$}
\RCSdefDate{\RCSsweeptoproducepointsDate}{$Date$}
% |
%%RefPage: end of header, begin of main body
% +------------------------------------------------------------------------+


\begin{ccRefFunction}[CGAL::]{sweep_to_produce_points_2}  %% add template arg's if necessary

%% \ccHtmlCrossLink{}     %% add further rules for cross referencing links
%% \ccHtmlIndexC[function]{} %% add further index entries

\ccDefinition
  
The function \ccRefName\ generates the set of intersection points of a
given set of input curves. It is also possible to ask the function to
include the endpoints of the input curves within the output.

\ccInclude{sweep_to_produce_points_2.h}

\ccFunction{
   template<class Curve_iterator, class Traits, class Container>
   void 
   sweep_to_produce_planar_map_points_2(
   Curve_iterator curves_begin, 
   Curve_iterator curves_end, 
   Traits& traits, 
   Container &subcurves);}
   {Explanation will come here.}

\ccRequirements

\ccSeeAlso

\ccc{Some_other_class},
\ccc{some_other_function}.

\ccImplementation
  % Say how it was implemented
  % See documentation of convex_hull_2

\end{ccRefFunction}

% +------------------------------------------------------------------------+
%%RefPage: end of main body, begin of footer
% EOF
% +------------------------------------------------------------------------+

