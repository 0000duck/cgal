\chapter{Sweep line}

\subsection*{Introduction}
%The construction of a planar map can be done using a sweep line
%technique rather than iteratively, inserting one curve after another.
%The sweep line algorithm performs better, but is done statically and
%therefore requires the list of curves in advance.  Due to the sweep
%line technique, inserting curves which are interior intersect,
%overlap or non $x$-monotone is supported. 

Given a collection $C$ of possibly intersecting 
(not necessarily $x$-monotone) curves in the plane, 
the \ccStyle{Sweep_line_2} class calculates the intersection points between 
the curves of $C$. It also calculates the disjoint-interior subcurves
induced by $C$, or rather returns an indication whether any two curves in 
$C$ intersect.

In order to use the sweep line algorithm, an instance of the 
\ccStyle{Sweep_line_2} class is to be constructed. 

The sweep line algorithm is also used in 
\ccc{Planar_map_with_intersections_2<Planar_map>}. 

\begin{ccTexOnly}

\subsection*{Concepts}
\ccRefConceptPage{SweepLineTraits_2}

\subsection*{Functions}

\end{ccTexOnly}    

