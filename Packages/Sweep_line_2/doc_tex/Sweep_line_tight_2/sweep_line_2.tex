\section{Sweep Line}

% Describe the functionality of the package

Given a collection $C$ of possibly intersecting 
(not necessarily $x$-monotone~\footnote{
We stress this fact because some implementation, such as, 
\ccc{Planar_map_2<Dcel,Traits>} assume all curves are $x$-monotone 
in order to gain simplicity and speed.}) 
curves in a plane, the \ccStyle{Sweep_line_2} class provides the following 
operations that can be performed on $C$:

\begin{itemize}
\item calculate the intersection points between the curves.
\item calculate the interior-disjoint segments induced by $C$.
\item calculate the intersection points and return a list of curves 
participating at each intersection point.
\item query for the existance of an intersection point between any two curves.
\end{itemize}


The \ccStyle{Sweep_line_2} implementation supports not only line segments 
but also 
general curves. Hence users may define their own geometry and by providing a 
suitable {\em Traits} class they can gain all sweep line functionality. 
In addition, the sweep line implementation supports degenerate cases such as  
vertical curves, tangency between two different curves and overlapping curves. 

 
% Put examples here

\section{Example using \ccStyle{Sweep_Line_2}}
\label{ssec:example1_sweep}
The following example demonstrates the usage of the 
\ccc {Sweep_line_2} algorithm to find all subcurves created by 
intersecting a set of curves.

The input is a set of curves $(10,1)-(20,1)$, $(10,-4)-(20,6)$,
$(12,-4)-(12,3)$, $(20,6)-(20,1)$. These curves intersect at 3 
interior points ont he curves and 2 end points. A total of 10 
segments are calculated after intersecting the curves.

% \ccIncludeExampleCode{Sweep_line_2/example1.C}

The output of the program looks like this:

\ccIncludeExampleCode{Sweep_line_2/example1.cout}

% EOF ------------------------------------------------------------------------80






