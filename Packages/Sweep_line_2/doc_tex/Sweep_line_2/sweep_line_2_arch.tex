\section{Software Design}

The \ccStyle{Sweep_line_2<InputIterator,Traits>} class is parameterized with
two objects. The sequence of input curves are obtained through the
\ccc{InputIterator}. The type of curves processed by the
\ccStyle{Sweep_line_2<InputIterator,Traits>} class is dictated
by the injected traits class, a model of the \ccStyle{SweepLineTraits}
concept. Inject the suitable traits class, to handle your desired family of
curves. The {\em planar map with intersections} comes with a collection of
traits classes that handle various types of curves, such as segments,
polylines, and conics. Naturally, users may author new traits classes
that handle other types or that possesses different carachteristics.

% \subsection{Operations}
% need to list all operations.

\subsection{Traits Classes}
The \ccStyle{Sweep_line_2<InputIterator,Traits>} class is parameterized by a
traits class, that defines the abstract interface between the sweep-line
algorithm and the primitive it uses. It must define three types of objects,
namely \ccc{Curve_2}, \ccStyle{X_monotone_curve_2}, and \ccStyle{Point_2},
where the type of the endpoints of an \ccStyle{X_monotone_curve_2}-type curve
is \ccStyle{Point_2}. In addition, the traits class must provide a set of
operations on these two types.

The \ccc{PlanarMapWithIntersectionsTraits_2} concept is a refinement of the 
\ccc{SweepLineTraits_2} concept, as some of the functions required to
perform some of the {\em planar map} operations are not needed, by the sweep
line algorithm. Hence, all the models of the
Planar Map With Intersections Traits \ccc{PlanarMapWithIntersectionsTraits_2}
concept are also models of the \ccc{SweepLineTraits_2} concept. 

