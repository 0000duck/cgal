\section{Introduction}

% Put here the definition of the algorithm (theory)
Let $C:=\{c_1, \ldots c_n\}$ be the set of curves for which we want 
to compute all intersections. We want to avoid testing pairs of segments 
that are fat apart. To find the intersecting pairs we imagine sweeping  
a line $l$ from left to right over the plane, starting from a position 
left to all curves. While we sweep the imaginary line, we keep track of all 
curves intersecting it.

This type of algorithm is called a {\em plane sweep algorithm} and the line 
$l$ is called the {\em sweep line}. The {\em status} of the sweep line if 
the set of curves intersecting it. The status changes while the sweep line 
moves to right, but nor continuously. Only at particular points is an update of the status required. We call these points the {\em event points} of the 
plane sweep algorithm. In this algorithm we initialize the event points with 
the endpoints of the curves.

% Put here a description of what's in this chapter

This chapter describes the functions provided in
\cgal\ for constructing planar map induced by a collection of planar curves 
as well as functions for producing the corresponding disjoint-interior 
subcurves and intersection points.
