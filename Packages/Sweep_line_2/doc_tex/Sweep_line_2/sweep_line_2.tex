\section{Sweep Line}

% Describe the functionality of the package

Given a collection $C$ of possibly intersecting 
(not necessarily $x$-monotone~\footnote{
We stress this fact because some implementation, such as, 
\ccc{Planar_map_2<Dcel,Traits>} assume all curves are $x$-monotone 
in order to gain simplicity and speed.}) 
curves in the plane, the sweep line algorithm can produce the collections of their 
intersections and the disjoint-interior subcurves induced by $C$, or rather 
construct the planar map induced by $C$. Use the functions \ccc{sweep_to_produce_points_2},
\ccc{sweep_to_produce_subcurves_2} and \ccc{sweep_to_construct_planar_map_2} respectively.

The sweep line implementation supports not only segments but rather 
general curves, and hence users may define their own geometry by providing a 
suitable {\em Traits} class and gain all sweep line functionality. 
In addition, the sweep line implementation supports degenerate cases as  
vertical curves, tangency between two different curves and overlapping curves. 
In the latter case, the sweep line implementation can report all overlapping cases 
in a given collection of curves. 
 
Am important advancement gained by using the sweep line algorithm 
is the ability to construct a planar map much faster.
A \ccc{Planar Map with Intersections} can be built incrementally by
inserting one curve after the other into the map. 
In general, it is much faster to perform the sweep line algorithm on
the collection of input curves in order to produce their planar map 
rather using the \ccc{Planar_map_with_intersections_2<Planar_map>} 
or \ccc{Arrangement_2<Dcel,Traits,Base_node>} packages, due to an 
efficient calculation of intersections and avoiding the usage of 
point location operations when inserting curves to the planar map. 
The sweep line implementation has been also optimized in order to 
save geometric predicates. The above optimization speeds up performance. 
The utility can either report the pairwise interior disjoint subcurves
computed, report the intersection points of all curves, 
or rather build the induced planar map, 

%If no additional insertions of intersecting curves are planned following the building of the map it is possible to perform a sweep
%line operation that will build the simpler class \ccc{Planar_map_2}.
%This is possible since the output of the sweep line operation is a
%collection of $x$-monotone pairwise interior disjoint, which are
%supported by the \ccc{Planar Map} package. 
When constructing a planar map by the sweep line process the user 
provides (possibly an empty) planar map as the resulting parameter. 
The result of applying the sweep line algorithm to a 
collection of curves and a non-empty map is equal to that 
of applying the algorithm to the union of the
planar map curves and the collection of input curves. Simply put,
the intersections of input curves and planar map curves are also
calculated.

% Put examples here
\section{Example using Sweep Line to construct a Planar map}
\label{ssec:example1_sweep}
The following example demonstrates the usage of the 
\ccc {Sweep line} algorithm. 
It reports all disjoint-interior subsegments out of four segments
--- $(0,0)-(1,1)$ , $(0,1)-(1,0)$ , $(0,0)-(1,0)$ and $(0,1)-(1,1)$
(an hourglass shape), two of them are intersecting in their
interior. A resulting container which will contain all the disjoint
interior sub-segments obtained by the calculation of the sweep
line algorithm is constructed. For clarity, we printed all the halfedges of the
resulting planar map to the standard output using the I/O functions
for the \ccc{Planar map} package.

\ccIncludeExampleCode{Sweep_line_2/example3.C}

The output of the program looks like this:

\ccIncludeExampleCode{Sweep_line_2/example3.cout}

% EOF ------------------------------------------------------------------------80






