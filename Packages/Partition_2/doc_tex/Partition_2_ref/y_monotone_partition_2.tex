\begin{ccRefFunction}{y_monotone_partition_2}

\ccDefinition

Function that produces a set of $y$-monotone polygons that 
represent a partitioning of a polygon defined on a sequence of points.
\ccIndexSubitem{polygon partitioning}{y-monotone}

\ccInclude{CGAL/partition_2.h}

\ccFunction{
template <class InputIterator, class OutputIterator, class Traits>
OutputIterator y_monotone_partition_2(InputIterator first, 
                                      InputIterator beyond,
                                      OutputIterator result, 
                                      const Traits& traits = Default_traits);
}
{
computes a partition of the polygon defined 
by the points in the range [\ccc{first}, \ccc{beyond}) into $y$-monotone 
polygons. The counterclockwise-oriented partition polygons are written to
the sequence starting at position \ccc{result}.  The past-the-end iterator for 
the resulting sequence of polygons is returned.
\ccPrecond The points in the range [\ccc{first}, \ccc{beyond}) define a
simple, counterclockwise-oriented polygon.
%\ccIndexSubitem[C]{y_monotone_partition_2}{preconditions}
}

\ccHeading{Requirements}
\begin{enumerate}
    \item \ccc{Traits} is a model of the concept
          \ccc{YMonotonePartitionTraits_2}
          % \ccIndexMainItem[c]{YMonotonePartitionTraits_2}
          and, for the purposes
          of checking the postcondition that the partition is valid, it should 
          also be a model of \ccc{YMonotonePartitionIsValidTraits_2}.
          % \ccIndexMainItem[c]{YMonotonePartitionIsValidTraits_2}.
    \item \ccc{OutputIterator::value_type} should be 
          \ccc{Traits::Polygon_2}.
    \item \ccc{InputIterator::value_type} should be \ccc{Traits::Point_2},
          which should also be the type of the points stored in an object
          of type \ccc{Traits::Polygon_2}.
\end{enumerate}

The default traits class \ccc{Default_traits} is \ccc{Partition_traits_2},
%\ccIndexTraitsClassDefault{y_monotone_partition_2}
with the representation type determined by \ccc{InputIterator::value_type}.

\ccSeeAlso
\ccRefIdfierPage{CGAL::approx_convex_partition_2} \\
\ccRefIdfierPage{CGAL::greene_approx_convex_partition_2} \\
\ccRefIdfierPage{CGAL::optimal_convex_partition_2} \\
\ccRefIdfierPage{CGAL::partition_is_valid_2} \\
\ccRefIdfierPage{CGAL::y_monotone_partition_is_valid_2}

\ccImplementation

This function implements the algorithm presented by de Berg \textit{et al.}
\cite{bkos-cgaa-97} which requires $O(n \log n)$ time
and $O(n)$ space for a polygon with $n$ vertices.

\ccExample


The following program computes a $y$-monotone partitioning
of a polygon using the default
traits class and stores the partition polygons in the list 
\ccc{partition_polys}.  It then asserts that each partition polygon
produced is, in fact, $y$-monotone and that the partition is valid.
(Note that these assertions are superfluous unless the postcondition
checking for \ccc{y_monotone_partition_2} has been turned off.)

\ccIncludeExampleCode{Partition_2/y_monotone_ex.C}



\end{ccRefFunction}
