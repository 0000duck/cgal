\begin{ccRefFunction}{is_y_monotone_2}

\ccDefinition

Function for testing the $y$-monotonicity of a sequence of points.
\ccIndexSubitem{polygon partitioning}{y-monotone}
\ccIndexMainItem{y-monotone polygon}

\ccInclude{CGAL/is_y_monotone_2.h}

\ccFunction{
template<class InputIterator, class Traits>
bool
is_y_monotone_2(InputIterator first, InputIterator beyond,
                const Traits& traits);
}
{
Determines if the sequence of points in the range 
[\ccc{first}, \ccc{beyond}) define a $y$-monotone 
polygon or not. If so, the function returns \ccc{true}, otherwise it
returns \ccc{false}. 
}

\ccHeading{Requirements}
\ccIndexSubitem[C]{is_y_monotone_2}{preconditions}
\begin{enumerate}
    \item \ccc{Traits} is a model of the concept IsYMonotoneTraits\_2.%
          \ccIndexMainItem[c]{IsYMonotoneTraits_2}
    \item \ccc{InputIterator::value_type} should be \ccc{Traits::Point_2}.
\end{enumerate}

The default traits class \ccc{Default_traits} is the kernel in which the
type \ccc{InputIterator::value_type} is defined.%
\ccIndexTraitsClassDefault{is_y_monotone_2}

\ccSeeAlso
\ccRefIdfierPage{CGAL::Is_y_monotone_2<Traits>} \\
\ccRefIdfierPage{CGAL::y_monotone_partition_2} \\
\ccRefIdfierPage{CGAL::y_monotone_partition_is_valid_2}

\ccImplementation

This function requires $O(n)$ time for a polygon with $n$ vertices.

\ccExample

The following program computes a $y$-monotone partitioning
of a polygon using the default
traits class and stores the partition polygons in the list 
\ccc{partition_polys}.  It then asserts that each of the partition 
polygons is, in fact, a $y$-monotone polygon and that the partition
is valid.  (Note that the
assertions are superfluous unless the postcondition checking done
by \ccc{y_monotone_partition_2} has been turned off during compilation.)

\ccIncludeExampleCode{Partition_2/y_monotone_ex.C}



\end{ccRefFunction}
