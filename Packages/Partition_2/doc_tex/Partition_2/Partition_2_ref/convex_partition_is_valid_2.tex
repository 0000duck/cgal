\renewcommand\ccRefPageBegin{\ccParDims\cgalColumnLayout\begin{ccAdvanced}}
\renewcommand\ccRefPageEnd{\ccParDims\cgalColumnLayout\end{ccAdvanced}}
\begin{ccRefFunction}{convex_partition_is_valid_2}

\ccDefinition
Function that determines if a given set of polygons represents
a valid convex partitioning for a given sequence of points that represent a
simple, counterclockwise-oriented polygon.  
A convex partition is valid if the 
polygons do not overlap, the union of the polygons is the same as the original
polygon given by the sequence of points, and if each partition polygon is 
convex. 
\ccIndexSubsubitem{polygon partitioning}{convex}{valid}

\ccInclude{CGAL/partition_is_valid_2.h}

\ccFunction{
template<class InputIterator, class ForwardIterator, class Traits>
bool
convex_partition_is_valid_2 (InputIterator point_first, 
                             InputIterator point_beyond,
                             ForwardIterator poly_first, 
                             ForwardIterator poly_beyond,
                             const Traits& traits = Default_traits);
}
{
determines if the polygons in the range [\ccc{poly_first}, \ccc{poly_beyond})
define a valid convex partition of the polygon defined by the points in the 
range [\ccc{point_first}, \ccc{point_beyond}). 
The function returns \ccc{true} iff the partition is valid and otherwise
returns \ccc{false}.
\ccPrecond The points in the range [\ccc{point_first}, \ccc{point_beyond}) 
define a simple, counterclockwise-oriented polygon.
}

\ccHeading{Requirements}
\ccIndexSubitem[C]{convex_partition_is_valid_2}{preconditions}
\begin{enumerate}
    \item \ccc{Traits} is a model of the concept
          ConvexPartitionIsValidTraits\_2%
          \ccIndexMainItem[c]{ConvexPartitionIsValidTraits_2}.
    \item \ccc{InputIterator::value_type} should be \ccc{Traits::Point_2},
          which should also be the type of the points stored in an object
          of type \ccc{Traits::Polygon_2}.
    \item \ccc{ForwardIterator::value_type} should be 
          \ccc{Traits::Polygon_2}.
\end{enumerate}

The default traits class \ccc{Default_traits} is \ccc{Partition_traits_2},%
\ccIndexTraitsClassDefault{convex_partition_is_valid_2} 
with the representation type determined by \ccc{InputIterator::value_type}.

\ccSeeAlso

\ccRefIdfierPage{CGAL::approx_convex_partition_2} \\
\ccRefIdfierPage{CGAL::greene_approx_convex_partition_2} \\
\ccRefIdfierPage{CGAL::optimal_convex_partition_2}\\
\ccRefIdfierPage{CGAL::partition_is_valid_2} \\
\ccc{CGAL::is_convex_2}

\ccImplementation

This function calls \ccc{partition_is_valid_2} using the function object
\ccc{Is_convex_2} to determine the convexity of each partition polygon.
Thus the time required by this function is $O(n \log n + e \log e)$ where
$n$ is the total number of vertices in the partition polgons and $e$ the
total number of edges.

\ccExample

See the example presented with the function \ccc{approx_convex_partition_2}
for an illustration of the use of this function.

\end{ccRefFunction}
\renewcommand\ccRefPageBegin{\ccParDims\cgalColumnLayout}
\renewcommand\ccRefPageEnd{\ccParDims\cgalColumnLayout}
