% +------------------------------------------------------------------------+
% | Reference manual page: YMonotonePartitionTraits_2.tex
% +------------------------------------------------------------------------+
% | 10.05.2000   Susan Hert
% | Package: Partition_2
% | 
% |
%%RefPage: end of header, begin of main body
% +------------------------------------------------------------------------+


\begin{ccRefConcept}{YMonotonePartitionTraits_2}
\ccIndexSubitemBegin[C]{y_monotone_partition_2}{traits class}

\ccDefinition
  
Requirements of a traits class to be
used with the function \ccc{y_monotone_partition_2}.  

\ccGeneralizes

\ccRefConceptPage{PartitionTraits_2}

\ccTypes

\ccIndexMainItem[c]{PartitionTraits_2}
In addition to the types defined for the concept PartitionTraits\_2, the
following types are also required:


\ccNestedType{Line_2}{}
%                      that has a function \ccc{FT x_at_y(FT x)} that returns
%                      the $y$-coordinate of the point on non-vertical line 
%                      $l$ with the given $x$-coordinate.}

\ccNestedType{Compare_x_at_y_2}{Predicate object type that provides
\ccc{CGAL::Comparision_result operator()(Point_2 p, Line_2 h)} to compare
the $x$ coordinate of \ccc{p} and the horizontal projection of \ccc{p}
on \ccc{h}.}

\ccNestedType{Construct_line_2}{Function object type that provides
\ccc{Line_2 operator()(Point_2 p, Point_2 q)}, which constructs and
returns the line defined by the points $p$ and $q$.}

\ccCreation
\ccCreationVariable{traits}  %% choose variable name

Only a copy constructor is required.

\ccConstructor{YMonotonePartitionTraits(const YMonotonePartitionTraits& tr);}{}

\ccOperations

In addition to the functions required for the concept PartitionTraits\_2,
\ccIndexMainItem[c]{PartitionTraits_2}
the following functions that create instances of the above function 
object types must exist.

\ccMethod{Construct_line_2 construct_line_2_object();}{}

\ccMethod{Compare_x_at_y_2 compare_x_at_y_2_object();}{}


\ccHasModels

\ccRefIdfierPage{CGAL::Partition_traits_2<R>}

%\ccSeeAlso
%
%\ccc{y_monotone_partition_2}

\ccIndexSubitemEnd[C]{y_monotone_partition_2}{traits class}
\end{ccRefConcept}

% +------------------------------------------------------------------------+
%%RefPage: end of main body, begin of footer
% EOF
% +------------------------------------------------------------------------+

