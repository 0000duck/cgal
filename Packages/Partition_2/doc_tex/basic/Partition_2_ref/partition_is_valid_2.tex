\renewcommand\ccRefPageBegin{\ccParDims\cgalColumnLayout\begin{ccAdvanced}}
\renewcommand\ccRefPageEnd{\ccParDims\cgalColumnLayout\end{ccAdvanced}}
\begin{ccRefFunction}{partition_is_valid_2}

\ccDefinition

Function that determines if a given set of polygons represents
a valid partition for a given sequence of points that 
define a simple, counterclockwise-oriented polygon.  A valid partition is one in
which the polygons are nonoverlapping and the union of the polygons is the 
same as the original polygon.
\ccIndexSubitem{polygon partitioning}{valid}
\ccIndexSubsubitem{polygon partitioning}{convex}{valid}
\ccIndexSubsubitem{polygon partitioning}{y-monotone}{valid}

\ccInclude{CGAL/partition_is_valid_2.h}

\ccFunction{
template<class InputIterator, class ForwardIterator, class Traits>
bool
partition_is_valid_2 (InputIterator point_first, InputIterator point_beyond,
                      ForwardIterator poly_first, ForwardIterator poly_beyond,
                      const Traits& traits = Default_traits);
}
{
returns \ccc{true} iff the polygons in the range [\ccc{poly_first}, 
\ccc{poly_beyond}) define a valid partition of the polygon defined by the 
points in the range [\ccc{point_first}, \ccc{point_beyond}) and 
\ccc{false} otherwise.  
Each polygon must also satisfy the property 
tested by \ccc{Traits::Is_valid()}. 
\ccPrecond Points in the range [\ccc{point_first}, \ccc{point_beyond}) define
a simple, counterclockwise-oriented polygon.
%\ccIndexSubitem[C]{partition_is_valid_2}{preconditions}
}

\ccHeading{Requirements}
\begin{enumerate}
    \item \ccc{Traits} is a model of the concept 
          \ccc{PartitionIsValidTraits_2}%
          \ccIndexMainItem[c]{PartitionIsValidTraits_2} and the
          concept defining the requirements for the validity test 
          implemented by \ccc{Traits::Is_valid()}.
    \item \ccc{InputIterator::value_type} should be \ccc{Traits::Point_2},
          which should also be the type of the points stored in an object
          of type \ccc{Traits::Polygon_2}.
    \item \ccc{ForwardIterator::value_type} should be 
          \ccc{Traits::Polygon_2}.
\end{enumerate}

The default traits class \ccc{Default_traits} is \ccc{Partition_traits_2},%
\ccIndexTraitsClassDefault{partition_is_valid_2}
with the representation type determined by \ccc{InputIterator::value_type}.

\ccSeeAlso

\ccRefIdfierPage{CGAL::approx_convex_partition_2} \\
\ccRefIdfierPage{CGAL::greene_approx_convex_partition_2} \\
\ccRefIdfierPage{CGAL::is_y_monotone_2} \\
\ccRefIdfierPage{CGAL::optimal_convex_partition_2} \\
\ccRefIdfierPage{CGAL::Partition_is_valid_traits_2<Traits, PolygonIsValid>} \\
\ccRefIdfierPage{CGAL::y_monotone_partition_2} \\
\ccc{CGAL::is_convex_2} 

\ccImplementation

This function requires $O(n \log n + e \log e + \Sigma_{i=1}^p m_i)$ where $n$ 
is the total number of vertices of the $p$ partition polygons, $e$ is the 
total number of edges of the partition polygons and $m_i$ is the time required 
by \ccc{Traits::Is_valid()} to test if partition polygon $p_i$ is valid.

\ccExample

See the example presented with the function \ccc{optimal_convex_partition_2}
for an illustration of the use of this function.

\end{ccRefFunction}
\renewcommand\ccRefPageBegin{\ccParDims\cgalColumnLayout}
\renewcommand\ccRefPageEnd{\ccParDims\cgalColumnLayout}
