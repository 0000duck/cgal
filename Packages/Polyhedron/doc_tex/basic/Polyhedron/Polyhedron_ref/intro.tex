% +------------------------------------------------------------------------+
% | Cbp Reference Manual:  intro.tex
% +------------------------------------------------------------------------+
% | Polyhedron reference manual pages
% |
% | 14.05.1998   Lutz Kettner
\RCSdef{\polyhedronRefRev}{$Revision$}
\RCSdefDate{\polyhedronRefDate}{$Date$}
% +------------------------------------------------------------------------+

\clearpage
%% \chapter{Polyhedral Surface}
\section{Reference Pages for Polyhedral Surface}
\label{chapterPolyhedronRef}
%% \ccChapterRelease{\polyhedronRefRev. \ \polyhedronRefDate}\\
%% \ccChapterAuthor{Lutz Kettner}


% +------------------------------------------------------------------------+
\section*{Release Note}

Beginning with \cgal\ R2.3, this package has a new design.  The old
design is still available for backwards compatibility and to support
older compiler, such as MSVC++6.0. However their use is deprecated and
the manual page are not converted into this new manual
format. Instead, see their old documentation in the manual of 
deprecated packages.

For the polyhedral surface, old and new design cannot be used
simultaneously (they have identical include file names and class
names). The include files select automatically the old design for
MSVC++6.0 and the new design otherwise. This automatism can be
overwritten by defining appropriate macros before the include
files. The old design is selected with the 
\texttt{CGAL\_USE\_POLYHEDRON\_DESIGN\_ONE} macro. The new design 
is selected with the 
\texttt{CGAL\_USE\_POLYHEDRON\_DESIGN\_TWO} macro.

The new design differs from the old design in the following ways: The
\ccc{Polyhedron_3} has a different number of template parameters that
require different concepts, such as the new \ccc{HalfedgeDS} concept,
the new \ccc{PolyhedronItems_3} concept, and the new
\ccc{PolyhedronTraits_3} concept. The \ccc{Polyhedron_3} interface is
backwards compatible with the old design except that the
\ccc{normal} member function in the facet and related types disappear.
Please use the plane equation instead, see also
\ccc{Polyhedron_traits_with_normals_3}.

\section*{Summary}

Polyhedral surfaces in three dimensions are composed of vertices,
edges, facets and an incidence relationship on them. The organization
beneath is a halfedge data structure, which restricts the class of
representable surfaces to orientable 2-manifolds -- with and without
boundary. If the surface is closed we call it a {\em polyhedron}.

The polyhedral surface is realized as a container class managing
vertices, halfedges, facets with their incidences, and maintaining the
combinatorial integrity of them. Its local types for the vertices,
halfedges and facets are documented separately.  A default traits
class, a default items class and an incremental builder conclude the
references. The polyhedral surface is based on the highly flexible
design of the halfedge data structure, see the reference for
\ccc{HalfedgeDS} in Chapter~\ref{chapterHalfedgeDSRef}
or~\cite{k-ugpdd-99}, but the default instantiation of the polyhedral
surface can be used without knowing the halfedge data structure.

\subsection*{Concepts}

\ccRefIdfierPage{PolyhedronTraits_3}\\ 
\ccRefIdfierPage{PolyhedronItems_3}

\subsection*{Classes}

\ccRefIdfierPage{CGAL::Polyhedron_3<Traits>}\\ 
\ccRefIdfierPage{CGAL::Polyhedron_3<Traits>::Vertex}\\ 
\ccRefIdfierPage{CGAL::Polyhedron_3<Traits>::Halfedge}\\ 
\ccRefIdfierPage{CGAL::Polyhedron_3<Traits>::Facet}\\ 
\ccRefIdfierPage{CGAL::Polyhedron_traits_3<Kernel>}\\ 
\ccRefIdfierPage{CGAL::Polyhedron_traits_with_normals_3<Kernel>}\\ 
\ccRefIdfierPage{CGAL::Polyhedron_items_3}\\ 
\ccRefIdfierPage{CGAL::Polyhedron_incremental_builder_3<HDS>}

\subsection*{Functions}
\ccThree{ostream&}{}{page 999.}

\ccHtmlNoLinks\ccHtmlNoIndex
\ccFunction{template <class Traits>
            ostream& operator<<( ostream& out, 
	                         const CGAL::Polyhedron_3<Traits>& P);}
    {\lcRawHtml{<A HREF="Function_operator.html">(go there)</A>}
     \lcTex{\hfill page~\pageref{refPolyhedron_operator_leftshift}}}

\ccHtmlNoLinks\ccHtmlNoIndex
\ccFunction{template <class Traits>
            istream& operator>>( istream& in, 
	                         CGAL::Polyhedron_3<Traits>& P);}
    {\lcRawHtml{<A HREF="Function_operator--.html">(go there)</A>}
     \lcTex{\hfill page~\pageref{refPolyhedron_operator_rightshift}}}

\lcHtml{\subsection*{Links to the Reference Sections}}

%% EOF %%
