% +------------------------------------------------------------------------+
% | Cbp Reference Manual:  intro.tex
% +------------------------------------------------------------------------+
% | Polyhedron reference manual pages
% |
% | 14.05.1998   Lutz Kettner
\RCSdef{\polyhedronRefRev}{$Revision$}
\RCSdefDate{\polyhedronRefDate}{$Date$}
% +------------------------------------------------------------------------+

\clearpage
\chapter{3D Polyhedral Surfaces}
%% \section{Reference Pages for Polyhedral Surface}
\label{chapterPolyhedronRef}
%% \ccChapterRelease{\polyhedronRefRev. \ \polyhedronRefDate}\\
%% \ccChapterAuthor{Lutz Kettner}


% +------------------------------------------------------------------------+
\section*{Summary}

Polyhedral surfaces in three dimensions are composed of vertices,
edges, facets and an incidence relationship on them. The organization
beneath is a halfedge data structure, which restricts the class of
representable surfaces to orientable 2-manifolds -- with and without
boundary. If the surface is closed we call it a {\em polyhedron}.

The polyhedral surface is realized as a container class managing
vertices, halfedges, facets with their incidences, and maintaining the
combinatorial integrity of them. Its local types for the vertices,
halfedges and facets are documented separately.  A default traits
class, a default items class and an incremental builder conclude the
references. The polyhedral surface is based on the highly flexible
design of the halfedge data structure, see the reference for
\ccc{HalfedgeDS} in Chapter~\ref{chapterHalfedgeDSRef}
or~\cite{k-ugpdd-99}, but the default instantiation of the polyhedral
surface can be used without knowing the halfedge data structure.

\subsection*{Concepts}

\ccRefIdfierPage{PolyhedronTraits_3}\\ 
\ccRefIdfierPage{PolyhedronItems_3}

\subsection*{Classes}

\ccRefIdfierPage{CGAL::Polyhedron_3<Traits>}\\ 
\ccRefIdfierPage{CGAL::Polyhedron_3<Traits>::Vertex}\\ 
\ccRefIdfierPage{CGAL::Polyhedron_3<Traits>::Halfedge}\\ 
\ccRefIdfierPage{CGAL::Polyhedron_3<Traits>::Facet}\\ 
\ccRefIdfierPage{CGAL::Polyhedron_traits_3<Kernel>}\\ 
\ccRefIdfierPage{CGAL::Polyhedron_traits_with_normals_3<Kernel>}\\ 
\ccRefIdfierPage{CGAL::Polyhedron_items_3}\\ 
\ccRefIdfierPage{CGAL::Polyhedron_min_items_3}\\ 
\ccRefIdfierPage{CGAL::Polyhedron_incremental_builder_3<HDS>}

\subsection*{Functions}
\ccThree{ostream&}{}{page 999.}

\ccHtmlNoLinks\ccHtmlNoIndex
\ccFunction{template <class Traits>
            ostream& operator<<( ostream& out, 
                                 const CGAL::Polyhedron_3<Traits>& P);}
    {\lcRawHtml{<A HREF="Function_operator.html">(go there)</A>}
     \lcTex{\hfill page~\pageref{refPolyhedron_operator_leftshift}}}

\ccHtmlNoLinks\ccHtmlNoIndex
\ccFunction{template <class Traits>
            istream& operator>>( istream& in, 
                                 CGAL::Polyhedron_3<Traits>& P);}
    {\lcRawHtml{<A HREF="Function_operator--.html">(go there)</A>}
     \lcTex{\hfill page~\pageref{refPolyhedron_operator_rightshift}}}

\lcHtml{\subsection*{Links to the Reference Sections}}

%% EOF %%
