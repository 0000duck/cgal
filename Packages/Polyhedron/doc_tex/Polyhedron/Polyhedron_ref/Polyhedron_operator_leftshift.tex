% +------------------------------------------------------------------------+
% | Reference manual page: Polyhedron_operator_leftshift.tex
% +------------------------------------------------------------------------+
% | 09.09.2000   Lutz Kettner
% | Package: Polyhedron
% | 
\RCSdef{\RCSPolyhedronoperatoristreamRev}{$Revision$}
\RCSdefDate{\RCSPolyhedronoperatoristreamDate}{$Date$}
% |
%%RefPage: end of header, begin of main body
% +------------------------------------------------------------------------+


\ccHtmlNoClassLinks
\begin{ccRefFunction}{operator<<}
\label{refPolyhedron_operator_leftshift}

\ccDefinition

This operator writes the polyhedral surface $P$ to the output stream
\ccc{out} using the Object File Format, OFF, with file extension 
{\tt .off}, which is also understood by
GeomView~\cite{p-gmgv15-94}. The output is in ASCII format. From the
polyhedral surface, only the point coordinates and facets are
written. Neither normal vectors nor color attributes are used.

\ccInclude{CGAL/IO/Polyhedron_iostream.h}

\ccGlobalFunction{template <class Traits>
    ostream& operator<<( ostream& out, const CGAL::Polyhedron_3<Traits>& P);}
  
\ccSeeAlso

\ccc{CGAL::Polyhedron_3},
\lcTex{\ccc{operator>>}}
\lcRawHtml{
    <I><A HREF="Function_operator--.html">operator&gt;&gt;</A></I>
}.


\end{ccRefFunction}

% +------------------------------------------------------------------------+
%%RefPage: end of main body, begin of footer
% EOF
% +------------------------------------------------------------------------+

