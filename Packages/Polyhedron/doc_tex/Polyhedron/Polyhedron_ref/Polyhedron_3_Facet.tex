% +------------------------------------------------------------------------+
% | Reference manual page: Facet.tex
% +------------------------------------------------------------------------+
% | 17.03.1999   Lutz Kettner
% | Package: Polyhedron
% | 
\RCSdef{\RCSFacetRev}{$Revision$}
\RCSdefDate{\RCSFacetDate}{$Date$}
% +------------------------------------------------------------------------+

\ccRefPageBegin

%%RefPage: end of header, begin of main body
% +------------------------------------------------------------------------+


\begin{ccRefClass}[Polyhedron_3<Traits>::]{Facet}

\ccDefinition
  
A facet optionally stores a normal vector, a plane equation, and a
reference to an incident halfedge that points to the facet.  If a
plane equation is supported the normal vector is taken from the plane
equation. Type tags indicate whether these member functions are
supported.  Figure~\ccTexHtml{\ref{figurePolyOptionalMethods} on page
  \pageref{figurePolyOptionalMethods}}{}\begin{ccHtmlOnly}
  <A HREF="Halfedge.html#figurePolyOptionalMethods"><IMG
  SRC="cc_ref_up_arrow.gif" ALT="reference arrow" WIDTH="10" HEIGHT="10"></A>
\end{ccHtmlOnly}
depicts the relationship between a halfedge and its incident
halfedges, vertices, and facets. The circulator is assignable to the
\ccc{Halfedge_handle}. The circulator is bidirectional if the
halfedge provided to the polyhedron with the \ccc{Items} template
argument provides a member function \ccc{prev()}, otherwise it is
of the forward category.


\ccInclude{CGAL/Polyhedron_3.h}

\ccTypes
\ccThree{Halfedge_const_handle}{h.halfedge() const;;}{}
\ccThreeToTwo

\ccNestedType{Vertex}{type of incident vertices.}
\ccGlue
\ccNestedType{Halfedge}{type of incident halfedges.}
\ccGlue
\ccNestedType{Plane}{plane equation type stored in facets.}
\ccGlue
\ccNestedType{Normal}{normal vector type stored in facets.}

\ccNestedType{Vertex_handle}{handle to vertex.}
\ccGlue
\ccNestedType{Halfedge_handle}{handle to halfedge.}
\ccGlue
\ccNestedType{Facet_handle}{handle to facet.}
\ccGlue
\ccNestedType{Halfedge_around_facet_circulator}{circulator of
  halfedges around a facet.}

\ccNestedType{Vertex_const_handle}{}
\ccGlue
\ccNestedType{Halfedge_const_handle}{}
\ccGlue
\ccNestedType{Facet_const_handle}{}
\ccGlue
\ccNestedType{Halfedge_around_facet_const_circulator}{}

\ccNestedType{Supports_facet_halfedge}{either \ccc{CGAL::Tag_true} or 
  \ccc{CGAL::Tag_false}.}
\ccGlue
\ccNestedType{Supports_facet_plane}{\~{}}
\ccGlue
\ccNestedType{Supports_facet_normal}{\~{}}


\ccCreation
\ccCreationVariable{f}

\ccConstructor{Facet();}{default constructor.}

\ccTagFullDeclarations
\ccHeading{Operations available if \ccc{Supports_facet_plane} $\equiv$ 
           \ccc{CGAL::Tag_true}}
         
Note, this implies \ccc{Supports_facet_normal} $\equiv$ \ccc{CGAL::Tag_true}, but
the return type of the \ccc{normal()} member function will not be by reference.

\ccMethod{Plane&       plane();}{}
\ccGlue
\ccMethod{const Plane& plane() const;}{plane equation.}
\ccGlue
\ccMethod{Normal normal() const;}{normal vector.}

\ccHeading{Operations available if \ccc{Supports_facet_normal} $\equiv$ 
           \ccc{CGAL::Tag_true}}

\ccMethod{Normal&       normal();}{}
\ccGlue
\ccMethod{const Normal& normal() const;}{normal vector.}


\ccHeading{Operations available if \ccc{Supports_facet_halfedge} $\equiv$ 
           \ccc{CGAL::Tag_true}}

\ccMethod{Halfedge_handle       halfedge();}{}
\ccGlue
\ccMethod{Halfedge_const_handle halfedge() const;}{
    an incident halfedge that points to \ccVar.}

%%\ccThree{Halfedge_const_handleh.halfedge() const;}{;}{}
\ccMethod{Halfedge_around_facet_circulator       facet_begin();}{}

\ccMethod{Halfedge_around_facet_const_circulator facet_begin() const;}
    {circulator of halfedges around the facet (counterclockwise).}

\ccMethod{void set_halfedge( Halfedge_handle h);}
    {sets incident halfedge to \ccc{h}
     \ccPrecond \ccc{h} is incident, i.e., \ccc{h->facet() ==} \ccVar.}

\ccSeeAlso

\ccc{Vertex}, \ccc{Halfedge}, \ccc{CGAL::Polyhedron_3}.

\ccTagDefaults
\end{ccRefClass}

% +------------------------------------------------------------------------+
%%RefPage: end of main body, begin of footer
\ccRefPageEnd
% EOF
% +------------------------------------------------------------------------+
