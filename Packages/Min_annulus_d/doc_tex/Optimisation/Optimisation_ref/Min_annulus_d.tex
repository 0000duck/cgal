% -*- latex -*-
% =============================================================================
% The CGAL Reference Manual
% Chapter: Geometric Optimisation
% Class  : CGAL::Min_annulus_d<Traits>
% -----------------------------------------------------------------------------
% file   : doc_tex/basic/Optimisation/Optimisation_ref/Min_annulus_d.tex
% package: Min_annulus_d
% author : Sven Sch�nherr <sven@inf.ethz.ch>
% -----------------------------------------------------------------------------
% $Revision$
% $Date$
% =============================================================================

\begin{ccRefClass}{Min_annulus_d<Traits>}

\ccIndexSubitem       [t]{annulus}{smallest enclosing}
\ccIndexSubitemSeeAlso[t]{sphere}{smallest enclosing annulus}
\ccIndexSubitem       [t]{smallest enclosing}{annulus}
\ccIndexSubitem       [t]{bounding volumes}{smallest enclosing annulus}

\ccSaveThreeColumns
\cgalMinAnnulusLayout

% -----------------------------------------------------------------------------
\ccDefinition

An object of the class \ccRefName\ is the unique annulus (region between
two concentric spheres with radii $r$ and $R$, $r \leq R$) enclosing a
finite set of points in $d$-dimensional Euclidean space $\E_d$, where the
difference $R^2-r^2$ is minimal. For a point set $P$ we denote by $ma(P)$
the smallest annulus that contains all points of $P$.  Note that $ma(P)$
can be degenerate\ccIndexSubitem[t]{degeneracies}{\ccFont Min_annulus_d},
i.e.~$ma(P)=\mbox{\ccTexHtml{$\;\emptyset$}{&Oslash;}}$ if
$P=\mbox{\ccTexHtml{$\;\emptyset$}{&Oslash;}}$ and $ma(P)=\{p\}$ if
$P=\{p\}$.

An inclusion-minimal subset $S$ of $P$ with $ma(S)=ma(P)$ is called a
\emph{support set}\ccIndexSubitem[t]{support set}{\ccFont Min_annulus_d},
the points in $S$ are the \emph{support points}.  A support set has size at
most $d+2$, and all its points lie on the boundary of $ma(P)$. In general,
the support set is not necessarily unique.

The underlying algorithm can cope with all kinds of input, e.g.~$P$ may be
empty or points may occur more than once. The algorithm computes a support
set $S$ which remains fixed until the next set, insert, or clear operation.

\ccInclude{CGAL/Min_annulus_d.h}

% -----------------------------------------------------------------------------
\ccRequirements
\ccIndexRequirements

The template parameter \ccc{Traits} is a model for \ccc{Optimisation_d_traits}.

We provide the models \ccc{Optimisation_d_traits_2},
\ccc{Optimisation_d_traits_3}, and \ccc{Optimisation_d_traits_d} using the
two-, three-, and $d$-dimensional \cgal~kernel, respectively.

% -----------------------------------------------------------------------------
\ccTypes
\ccIndexClassTypes

\ccSetTwoColumns{\ccRefName:: Point}{}

\ccNestedType{ Point}{ typedef to \ccc{Traits::Point_d}.
        Point type used to represent the input points.}

\ccNestedType{ FT}{ typedef to \ccc{Traits::FT}.
        Number type used to return the squared radii of the smallest
        enclosing annulus.}

\ccNestedType{ ET}{ typedef to \ccc{Traits::ET}.
        Number type used to do the exact computations in the underlying
        solver for quadratic programs (cf.~\textbf{Implementation}).}

\ccNestedType{ Point_iterator}{
        non-mutable model of the \stl\ concept \emph{RandomAccessIterator}
        with value type \ccc{Point}. Used to access the points
        of the smallest enclosing annulus.}

\ccNestedType{ Support_point_iterator}{
        non-mutable model of the \stl\ concept \emph{RandomAccessIterator}
        with value type \ccc{Point}. Used to access the support points
        of the smallest enclosing annulus.}

\ccNestedType{ Inner_support_point_iterator}{
        non-mutable model of the \stl\ concept \emph{RandomAccessIterator}
        with value type \ccc{Point}. Used to access the inner support points
        of the smallest enclosing annulus.}

\ccNestedType{ Outer_support_point_iterator}{
        non-mutable model of the \stl\ concept \emph{RandomAccessIterator}
        with value type \ccc{Point}. Used to access the outer support points
        of the smallest enclosing annulus.}

\ccNestedType{ Coordinate_iterator}{
        non-mutable model of the \stl\ concept \emph{RandomAccessIterator}
        with value type \ccc{ET}. Used to access the coordinates of
        the center of the smallest enclosing annulus.}

\ccPropagateThreeToTwoColumns

% -----------------------------------------------------------------------------
\ccCreation
\ccIndexClassCreation
\ccCreationVariable{min_annulus}

\ccConstructor{ Min_annulus_d( const Traits&  traits  = Traits(),
                               int            verbose = 0,
                               std::ostream&  stream  = std::cout);}{
        initializes \ccVar\ to $ma(\mbox{\ccTexHtml{$\emptyset$}{&Oslash;}})$.}

\ccConstructor{ template < class InputIterator >
                Min_annulus_d( InputIterator  first,
                               InputIterator  last,
                               const Traits&  traits = Traits(),
                               int            verbose = 0,
                               std::ostream&  stream  = std::cout);}{
        initializes \ccVar\ to $ma(P)$ with $P$ being the set of points
        in the range [\ccc{first},\ccc{last}).
        \ccRequire  The value type of \ccc{InputIterator} is \ccc{Point}.
        \ccPrecond  All points have the same dimension.}

\begin{ccAdvanced}
  If \ccc{verbose} is set to $1$, $2$, or $3$ then some, more, or full
  verbose output of the underlying solver for linear programs is
  written to \ccc{stream}, resp.
\end{ccAdvanced}

% -----------------------------------------------------------------------------
\ccAccessFunctions
\begin{ccIndexMemberFunctions}
\ccIndexMemberFunctionGroup{access}

\ccMemberFunction{ int  ambient_dimension( ) const;}{
        returns the dimension of the points in $P$.
        If \ccVar\ is empty, the ambient dimension is $-1$.}

\ccMemberFunction{ int  number_of_points( ) const;}{
        returns the number of points of \ccVar, i.e.~$|P|$.}

\ccMemberFunction{ int  number_of_support_points( ) const;}{
        returns the number of support points of \ccVar, i.e.~$|S|$.}
\ccIndexSubitem[t]{support set}{\ccFont Min_annulus_d}

\ccMemberFunction{ int  number_of_inner_support_points( ) const;}{
        returns the number of support points of \ccVar{}
        which lie on the inner sphere.}

\ccMemberFunction{ int  number_of_outer_support_points( ) const;}{
        returns the number of support points of \ccVar{}
        which lie on the outer sphere.}

\medskip
\ccGlueBegin
\ccMemberFunction{ Point_iterator  points_begin( ) const;}{
        returns an iterator referring to the first point of \ccVar.}
%
\ccMemberFunction{ Point_iterator  points_end( ) const;}{
        returns the corresponding past-the-end iterator.}
\ccGlueEnd

\medskip
\ccGlueBegin
\ccMemberFunction{ Support_point_iterator  support_points_begin( ) const;}{
        returns an iterator referring to the first support point of \ccVar.}
%
\ccMemberFunction{ Support_point_iterator  support_points_end( ) const;}{
        returns the corresponding past-the-end iterator.}
\ccGlueEnd

\medskip
\ccGlueBegin
\ccMemberFunction{ Inner_support_point_iterator
                   inner_support_points_begin( ) const;}{
        returns an iterator referring to the first inner support point
        of \ccVar.}
%
\ccMemberFunction{ Inner_support_point_iterator
                   inner_support_points_end( ) const;}{
        returns the corresponding past-the-end iterator.}
\ccGlueEnd

\medskip
\ccGlueBegin
\ccMemberFunction{ Outer_support_point_iterator
                   outer_support_points_begin( ) const;}{
        returns an iterator referring to the first outer support point
        of \ccVar.}
%
\ccMemberFunction{ Outer_support_point_iterator
                   outer_support_points_end( ) const;}{
        returns the corresponding past-the-end iterator.}
\ccGlueEnd

\ccMemberFunction{ Point  center( ) const;}{
        returns the center of \ccVar.
        \ccRequire  An implicit conversion from \ccc{ET} to \ccc{RT} is
        available.
        \ccPrecond  \ccVar\ is not empty.}

\ccMemberFunction{ FT  squared_inner_radius( ) const;}{
        returns the squared inner radius of \ccVar.
        \ccRequire  An implicit conversion from \ccc{ET} to \ccc{RT} is
        available.
        \ccPrecond  \ccVar\ is not empty.}

\ccMemberFunction{ FT  squared_outer_radius( ) const;}{
        returns the squared outer radius of \ccVar.
        \ccRequire  An implicit conversion from \ccc{ET} to \ccc{RT} is
        available.
        \ccPrecond  \ccVar\ is not empty.}

\medskip
\ccGlueBegin
\ccMemberFunction{ Coordinate_iterator
                   center_coordinates_begin() const;}{
        returns an iterator referring to the first coordinate
        of the center of \ccVar.
        \ccCommentHeading{Note} The coordinates have a rational
        representation, i.e.~the first $d$ elements of the iterator
        range are the numerators and the $(d\!+\!1)$-st element is the
        common denominator.}
%
\ccMemberFunction{ Coordinate_iterator
                   center_coordinates_end() const;}{
        returns the corresponding past-the-end iterator.}
\ccGlueEnd

\ccMemberFunction{ ET  squared_inner_radius_numerator( ) const;}{
        returns the numerator of the squared inner radius of \ccVar.}

\ccMemberFunction{ ET  squared_outer_radius_numerator( ) const;}{
        returns the numerator of the squared outer radius of \ccVar.}

\ccMemberFunction{ ET  squared_radii_denominator( ) const;}{
        returns the denominator of the squared radii of \ccVar.}

% -----------------------------------------------------------------------------
\ccPredicates
\ccIndexMemberFunctionGroup{predicates}

The bounded area of the smallest enclosing annulus lies between the inner
and the outer sphere. The boundary is the union of both spheres. By
definition, an empty annulus has no boundary and no bounded side, i.e.~its
unbounded side equals the whole space $\E_d$.

\ccMemberFunction{ CGAL::Bounded_side
                   bounded_side( const Point& p) const;}{
        returns \ccGlobalScope\ccc{ON_BOUNDED_SIDE},
        \ccGlobalScope\ccc{ON_BOUNDARY}, or
        \ccGlobalScope\ccc{ON_UNBOUNDED_SIDE} iff \ccc{p} lies
        properly inside, on the boundary, or properly outside of
        \ccVar, resp.
        \ccPrecond  The dimension of \ccc{p} equals
        \ccVar\ccc{.ambient_dimension()} if \ccVar\ is not empty.}

\ccMemberFunction{ bool  has_on_bounded_side( const Point& p) const;}{
        returns \ccc{true}, iff \ccc{p} lies properly inside \ccVar.
        \ccPrecond  The dimension of \ccc{p} equals
        \ccVar\ccc{.ambient_dimension()} if \ccVar\ is not empty.}

\ccMemberFunction{ bool  has_on_boundary( const Point& p) const;}{
        returns \ccc{true}, iff \ccc{p} lies on the boundary of \ccVar.
        \ccPrecond  The dimension of \ccc{p} equals
        \ccVar\ccc{.ambient_dimension()} if \ccVar\ is not empty.}

\ccMemberFunction{ bool  has_on_unbounded_side( const Point& p) const;}{
        returns \ccc{true}, iff \ccc{p} lies properly outside of \ccVar.
        \ccPrecond  The dimension of \ccc{p} equals
        \ccVar\ccc{.ambient_dimension()} if \ccVar\ is not empty.}

\ccMemberFunction{ bool  is_empty( ) const;}{
        returns \ccc{true}, iff \ccVar\ is empty (this implies degeneracy).}

\ccMemberFunction{ bool  is_degenerate( ) const;}{
        returns \ccc{true}, iff \ccVar\ is degenerate, i.e.~if \ccVar\
        is empty or equal to a single point.}
\ccIndexSubitem[t]{degeneracies}{\ccFont Min_annulus_d}

% -----------------------------------------------------------------------------
\ccModifiers
\ccIndexMemberFunctionGroup{modifiers}

\ccMemberFunction{ void  clear( );}{
        resets \ccVar\ to $ma(\mbox{\ccTexHtml{$\emptyset$}{&Oslash;}})$.}

\ccMemberFunction{ template < class InputIterator >
                   void  set( InputIterator first,
                              InputIterator last );}{
        sets \ccVar\ to $ma(P)$, where $P$ is the set of points in
        the range [\ccc{first},\ccc{last}).
        \ccRequire  The value type of \ccc{InputIterator} is \ccc{Point}.
        \ccPrecond  All points have the same dimension.}

\ccMemberFunction{ void  insert( const Point& p);}{
        inserts \ccc{p} into \ccVar.
        \ccPrecond  The dimension of \ccc{p} equals
        \ccVar\ccc{.ambient_dimension()} if \ccVar\ is not empty.}

\ccMemberFunction{ template < class InputIterator >
                   void  insert( InputIterator  first,
                                 InputIterator  last );}{
        inserts the points in the range [\ccc{first},\ccc{last}) into
        \ccVar\ and recomputes the smallest enclosing annulus.
        \ccRequire  The value type of \ccc{InputIterator} is \ccc{Point}.
        \ccPrecond  All points have the same dimension.
        If \ccVar\ is not empty, this dimension must be equal to
        \ccVar\ccc{.ambient_dimension()}.}

% -----------------------------------------------------------------------------
\ccHeading{Validity Check}
\ccIndexMemberFunctionGroup{validity check}
\ccIndexSubitem[t]{validity check}{\ccFont Min_annulus_d}

An object \ccVar\ is valid, iff
\begin{itemize}
  \item \ccVar\ contains all points of its defining set $P$,
  \item \ccVar\ is the smallest annulus containing its support set $S$, and
  \item $S$ is minimal, i.e.\ no support point is redundant.
\end{itemize}
\emph{Note:} In this release only the first item is considered by the
validity check.

\ccMemberFunction{ bool is_valid( bool verbose = false,
                                  int  level   = 0    ) const;}{
      returns \ccc{true}, iff \ccVar\ is valid. If \ccc{verbose} is
      \ccc{true}, some messages concerning the performed checks are
      written to standard error stream. The second parameter
      \ccc{level} is not used, we provide it only for consistency
      with interfaces of other classes.}

% -----------------------------------------------------------------------------
\ccHeading{Miscellaneous}
\ccIndexMemberFunctionGroup{miscellaneous}

\def\ccTagRmConstRefPair{\ccFalse}

\ccMemberFunction{ const Traits&  traits( ) const;}{
        returns a const reference to the traits class object.}

\def\ccTagRmConstRefPair{\ccTrue}

\end{ccIndexMemberFunctions}

% -----------------------------------------------------------------------------
\ccHeading{I/O}
\begin{ccIndexGlobalFunctions}

\ccIndexGlobalFunctionGroup{output}
\ccFunction{ std::ostream&
             operator << ( std::ostream& os,
                           const Min_annulus_d<Traits>& min_annulus);}{
        writes \ccVar\ to output stream \ccc{os}.
        \ccRequire  The output operator is defined for \ccc{Point}.}

\ccIndexGlobalFunctionGroup{input}
\ccFunction{ std::istream&
             operator >> ( std::istream& is,
                           Min_annulus_d<Traits> min_annulus&);}{
        reads \ccVar\ from input stream \ccc{is}.
        \ccRequire  The input operator is defined for \ccc{Point}.}

\end{ccIndexGlobalFunctions}

% -----------------------------------------------------------------------------
\ccSeeAlso

\ccRefIdfierPage{CGAL::Min_sphere_d<Traits>}\\[1ex]
\ccRefIdfierPage{CGAL::Optimisation_d_traits_2<R,ET,NT>}\\
\ccRefIdfierPage{CGAL::Optimisation_d_traits_3<R,ET,NT>}\\
\ccRefIdfierPage{CGAL::Optimisation_d_traits_d<R,ET,NT>}\\[1ex]
\ccRefConceptPage{Optimisation_d_traits}

% -----------------------------------------------------------------------------
\ccImplementation
\ccIndexImplementation

The problem of finding the smallest enclosing annulus of a finite point set
can be formulated as an optimization problem with linear constraints and a
linear objective function\ccIndexSubitem[t]{linear program}{\ccFont
  Min_annulus_d}. The solution is obtained using our exact
solver for linear and quadratic programs~\cite{gs-eegqp-00}.

The creation time is almost always linear in the number of points.  Access
functions and predicates take constant time, inserting a point takes almost
always linear time. The clear operation and the check for validity each
take linear time.

% -----------------------------------------------------------------------------

\ccRestoreThreeColumns

\end{ccRefClass}

% ===== EOF ===================================================================
