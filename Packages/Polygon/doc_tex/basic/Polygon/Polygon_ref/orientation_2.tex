% +------------------------------------------------------------------------+
% | Reference manual page: orientation_2.tex
% +------------------------------------------------------------------------+
% | 21.06.2001   Author
% | Package: Polygon
% | 
\RCSdef{\RCSorientationRev}{$Revision$}
\RCSdefDate{\RCSorientationDate}{$Date$}
% |
%%RefPage: end of header, begin of main body
% +------------------------------------------------------------------------+


\begin{ccRefFunction}{orientation_2}  %% add template arg's if necessary

%% \ccHtmlCrossLink{}     %% add further rules for cross referencing links
%% \ccHtmlIndexC[function]{} %% add further index entries

\ccDefinition
  
The function \ccRefName\ does this and that.

\ccInclude{orientation_2.h}

\ccGlobalFunction{void orientation_2();}

\ccIsModel

Concept

\ccSeeAlso

\ccc{Some_other_class},
\ccc{some_other_function}.

\ccExample

A short example program.
Instead of a short program fragment, a full running program can be
included using the 
\verb|\ccIncludeExampleCode{examples/Polygon/orientation_2_prog.C}| 
macro. The program example would be part of the source code distribution and
also part of the automatic test suite.

\begin{ccExampleCode}
void your_example_code() {
}
\end{ccExampleCode}

%% \ccIncludeExampleCode{examples/Polygon/orientation_2_prog.C}

\end{ccRefFunction}

% +------------------------------------------------------------------------+
%%RefPage: end of main body, begin of footer
% EOF
% +------------------------------------------------------------------------+

