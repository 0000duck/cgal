% +------------------------------------------------------------------------+
% | Reference manual page: area_2.tex
% +------------------------------------------------------------------------+
% | 21.06.2001   Author
% | Package: Polygon
% | 
\RCSdef{\RCSareaRev}{$Revision$}
\RCSdefDate{\RCSareaDate}{$Date$}
% |
%%RefPage: end of header, begin of main body
% +------------------------------------------------------------------------+


\begin{ccRefFunction}{area_2}  %% add template arg's if necessary

%% \ccHtmlCrossLink{}     %% add further rules for cross referencing links
%% \ccHtmlIndexC[function]{} %% add further index entries

\ccDefinition
  
The function \ccRefName\ computes the signed area of a polygon.

\ccInclude{CGAL/Polygon_2_algorithms.h}

\ccGlobalFunction{
template <class ForwardIterator, class Traits>
void 
area_2( ForwardIterator first, ForwardIterator last,
   	typename Traits::FT &result,
        const Traits& traits);}
{
Computes the signed area of the polygon defined by the range of points
\ccc{first} \ldots\ \ccc{last}. The area is returned in the parameter
\ccc{result}. The sign is positive for counterclockwise polygons, negative for
clockwise polygons. If the polygon is not simple, the area is not well defined.
The functionality is also available by the \ccc{polygon_area_2} function, which
returns the area instead of taking it as a parameter.
}

\ccHeading{Requirements}
\ccIndexSubitem[C]{bottom_vertex_2}{requirements}
\begin{enumerate}
    \item \ccc{Traits} is a model of the concept 
	  PolygonTraits\_2\ccIndexMainItem[c]{PolygonTraits_2}.
	  Only the following members of this traits class are used:
          \begin{itemize}
            \item \ccc{Compute_area_2}
            \item \ccc{FT}
            \item \ccc{compute_area_2_object}
          \end{itemize}
    \item \ccc{ForwardIterator::value_type} should be \ccc{Traits::Point_2},
\end{enumerate}

\ccSeeAlso

\ccRefIdfierPage{CGAL::polygon_area_2} \\
\ccRefIdfierPage{PolygonTraits_2} \\
\ccRefIdfierPage{CGAL::orientation_2} \\
\ccRefIdfierPage{CGAL::Polygon_2<PolygonTraits_2, Container>} \\

%% \ccIncludeExampleCode{examples/Polygon/area_2_prog.C}

\end{ccRefFunction}

% +------------------------------------------------------------------------+
%%RefPage: end of main body, begin of footer
% EOF
% +------------------------------------------------------------------------+

