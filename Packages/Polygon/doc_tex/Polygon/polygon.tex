
\cleardoublepage
\chapter{Polygon}\label{Polygon}

\begin{ccClassTemplate}{Polygon_2<Traits,Container>}
\ccSection{2D Polygon}

\ccDefinition

The class \ccClassName\ represents a simple polygon in the two-dimensional
Euclidean plane $\E_2$. A polygon is called {\em simple} if there is no pair
of nonconsecutive edges sharing a point (see \cite{ps-cgi-85}).

An object \ccStyle{p} of the data type \ccClassName\ is defined by the sequence
of its vertices. A simple polygon \ccStyle{p} is oriented, i.e., its boundary has
clockwise or counterclockwise orientation. The side to the left of the boundary
is called the positive side and the side to the right of the boundary is called
the negative side.  As any Jordan curve, the boundary of a polygon divides the
plane into two open regions, a bounded one and an unbounded one.

An object \ccStyle{p} of \ccClassName\ is a dynamic data
structure, i.e. vertices can be added and removed. These operations may
destroy the simplicity of the polygon, which is a precondition to most
predicates of polygons.

The data type \ccClassName\ is parameterized with two template parameters: 
a traits class \ccStyle{Traits} and a container class \ccStyle{Container}. 
The parameter \ccStyle{Traits} defines the types and predicates
that are used in the polygon class and the polygon algorithms.
For example \ccStyle{Traits::Point_2} denotes the type of the vertices
of the polygon. 
A default polygon traits class \ccStyle{Polygon_traits_2<R>} is provided 
(see Section~\ref{sec:poly_default}), where \ccStyle{R} is a representation 
class.
The parameter \ccStyle{Container} specifies the type of container that is 
used to store the sequence of vertices of the polygon, e.g.\ a list, a vector, 
a tree, etc.
The type \ccStyle{Container} should fulfill the requirements of a sequence 
container given in \cite{ms-strg-96}. 
The value type of the container should be the same as the point type of the 
traits class.

{\em Note}: Currently, a polygon declaration looks like
\ccStyle{Polygon_2<Traits, list<Traits::Point_2> >}.
When nested templates become available this might be simplified to
\ccStyle{Polygon_2<Traits, list>}.

\subsubsection*{Assertions}
The polygon code uses infix \ccc{POLYGON} in the assertions,
for example defining the compiler flag
\ccc{CGAL_POLYGON_NO_PRECONDITIONS} switches precondition
checking off, cf.~Section~2 of the Reference Manual Part 0, General 
Introduction.

%\subsubsection*{Changes}
%The most important changes with respect to the previous release are:

%\begin{itemize}
%\item There is now a proper traits parameter. It is no longer possible to use
%the representation class \ccc{R} parameter for the template parameter \ccc{Traits}.
%Instead the default traits class \ccc{Polygon_traits_2<R>} should be used,
%or when appropriate a user defined one.

%\item The method \ccc{append} is now called \ccc{push_back}. This was done to
%support back insert iterators.

%\item The \ccc{transform} method has been removed. It is replaced by
%a global \ccc{transform} function.

%\item Random access methods have been added. These methods are only available
%for random access containers.

%\item The window stream output operator has been added.

%\item A few bugs have been fixed (in the area computation and the simplicity test).
%\end{itemize}

\ccHeading{Types}
\ccThree{Polygon_2<Traits,Container>}{}{\hspace*{7cm}}
\ccThreeToTwo
\ccNestedType{Traits}{The traits type.}
\ccGlue
\ccNestedType{Container}{The container type.}

\ccTypedef{typedef Traits::FT FT;}
{The number type, which is the {\em field type} of the points of the polygon.} 
\ccGlue
\ccTypedef{typedef Traits::Point_2 Point_2;}{The point type of the polygon.}
\ccGlue
\ccTypedef{typedef Traits::Segment_2 Segment_2;}{The type of a segment between two points 
of the polygon.}

The following types denote iterators that allow to traverse the vertices and
edges of a polygon. 
Since it is questionable whether a polygon should be viewed as a circular or 
as a linear data structure both circulators and iterators are defined.
The circulators and iterators with `const' in their name are non-mutable, the 
others are mutable.
The iterator category is in all cases bidirectional, except for
\ccStyle{Vertex_iterator} and \ccStyle{Vertex_const_iterator}, which have the
same iterator category as \ccStyle{Container::iterator}.
{\bf N.B.} In fact all of them should have the same iterator category as
\ccStyle{Container::iterator}. However, due to compiler problems this is currently
not possible. This will be corrected when iterator traits become available.
The consequence of using iterators / circulators with an incorrect iterator
category is that when an STL algorithm is applied to such a range,
the wrong (i.e. inefficient) version of an STL algorithm may be selected. 

For vertices we define

\ccNestedType{Vertex_iterator}{}
\ccGlue
\ccNestedType{Vertex_const_iterator}{}
\ccGlue
\ccNestedType{Vertex_circulator}{}
\ccGlue
\ccNestedType{Vertex_const_circulator}{}

Their value type is \ccStyle{Point_2}.

For edges we define

\ccNestedType{Edge_const_circulator}{}
\ccGlue
\ccNestedType{Edge_const_iterator}{}

Their value type is \ccStyle{Segment_2}.

\ccCreation
\ccCreationVariable{p}
\ccThree{C}{}{\hspace*{10cm}}
\ccThreeToTwo

  \ccHidden\ccConstructor{ Polygon_2( );}
    { Introduces an empty polygon \ccVar.}

  \ccHidden\ccConstructor{ Polygon_2( const Polygon_2<Traits,Container>& q);}
    { Introduces a polygon \ccVar\ with vertices from \ccStyle{q}.}

  \ccConstructor{template <class InputIterator>
      Polygon_2( InputIterator first, InputIterator last ); }
    { Introduces a polygon \ccVar\ with vertices from the sequence defined by
      the range \ccStyle{[first,last)}.
      \ccPrecond The value type of points in the range \ccStyle{[first,last)} is
                 \ccStyle{Point_2}.
    }

  \ccConstructor{template <class ForwardCirculator>
      Polygon_2( ForwardCirculator start); }
    { Introduces a polygon \ccVar\ with vertices from the sequence defined by
      the range \ccStyle{[start,start]}.
      \ccPrecond The value type of points in the range \ccStyle{[first,last)} is
                 \ccStyle{Point_2}.
    }

%  \ccHidden\ccDestructor{ ~Polygon_2();}{ destructor }

  \ccHidden\ccMethod{ Polygon_2<Traits,Container>& operator=(const
     Polygon_2<Traits,Container>& polygon);}
   {assignment operator}

\newpage
\ccHeading{Operations}
\ccThree{Vertex_const_circulator}{}{\hspace*{10cm}}

The following operations allow to modify a polygon.

  \ccMethod{ Vertex_iterator insert(Vertex_iterator i, const Point_2& q);}
    { Inserts the vertex \ccStyle{q} before \ccStyle{i}.
      The return value points to the inserted vertex. }

  \ccMethod{template <class InputIterator>
      void insert(Vertex_iterator i, InputIterator first, InputIterator last);}
    { Inserts the vertices in the range \ccStyle{[first, last)} before
      \ccStyle{i}.
      \ccPrecond The value type of points in the range \ccStyle{[first,last)} is
                 \ccStyle{Point_2}.
      %Note: this method is only available if the compiler supports member templates.
    }

  \ccMethod{ void push_back(const Point_2& q);}
    { Has the same semantics as \ccStyle{p.insert(p.vertices_end(), q)}.}

  \ccMethod{ void erase(Vertex_iterator i);}
    { Erases the vertex pointed to by \ccStyle{i}.}

  \ccMethod{ void erase(Vertex_iterator first, Vertex_iterator last);}
    { Erases the vertices in the range \ccStyle{[first, last)}.}

  \ccMethod{ void reverse_orientation(); }
    { Reverses the orientation of the polygon. The vertex pointed to by
      \ccStyle{p.vertices_begin()} remains the same. }

\ccHeading{Traversal of a polygon}
The following methods of the class \ccClassName\ return
circulators and iterators that allow to traverse the vertices and edges.

  \ccMethod{ Vertex_iterator vertices_begin(); }
    { Returns a mutable iterator that allows to traverse the vertices of
      the polygon \ccStyle{p}.}

  \ccMethod{ Vertex_iterator vertices_end(); }
    { Returns the corresponding past-the-end iterator. }

  \ccMethod{ Vertex_const_iterator vertices_begin(); }
    { Returns a non-mutable iterator that allows to traverse the vertices of
      the polygon \ccStyle{p}.}

  \ccMethod{ Vertex_const_iterator vertices_end() const; }
    { Returns the corresponding past-the-end iterator. }

  \ccMethod{ Vertex_circulator vertices_circulator(); }
    { Returns a mutable circulator that allows to traverse the vertices of
      the polygon \ccStyle{p}.}

  \ccMethod{ Vertex_const_circulator vertices_circulator() const;}
    { Returns a non-mutable circulator that allows to traverse the vertices of
      the polygon \ccStyle{p}.}

  \ccMethod{ Edge_const_iterator edges_begin() const;}
    { Returns a non-mutable iterator that allows to traverse the edges of
      the polygon \ccStyle{p}.}

  \ccMethod{ Edge_const_iterator edges_end() const;}
    { Returns the corresponding past-the-end iterator. }

  \ccMethod{ Edge_const_circulator edges_circulator() const;}
    { Returns a non-mutable circulator that allows to traverse the edges of
      the polygon \ccStyle{p}.}

\ccHeading{Predicates}

  \ccMethod{bool is_simple() const;}
    { Returns whether \ccStyle{p} is a simple polygon.}

  \ccMethod{bool is_convex() const;}
    { Returns whether \ccStyle{p} is convex. }

  \ccMethod{Orientation orientation() const;}
    { Returns the orientation of \ccVar. If the number of vertices 
      $\ccStyle{p.size()} < 3$ then \ccStyle{COLLINEAR} is returned.
      \ccPrecond \ccStyle{p.is_simple()}.
    }

  \ccMethod{Oriented_side oriented_side(const Point_2& q) const;}
    { Returns \ccStyle{POSITIVE_SIDE}, or \ccStyle{NEGATIVE_SIDE},
       or \ccStyle{ON_ORIENTED_BOUNDARY}, 
       depending on where point \ccStyle{q} is.
      \ccPrecond \ccStyle{p.is_simple()}.
    }

  \ccMethod{Bounded_side bounded_side(const Point_2& q) const;}
    { Returns the symbolic constant \ccStyle{ON_BOUNDED_SIDE}, 
      \ccStyle{ON_BOUNDARY}
      or \ccStyle{ON_UNBOUNDED_SIDE}, depending on where point
      \ccStyle{q} is.
      \ccPrecond \ccStyle{p.is_simple()}.
    }

  \ccMethod{Bbox_2 bbox() const;}
    { Returns the smallest bounding box containing \ccVar.}
    
  \ccMethod{Traits::FT area() const;}
    { Returns the signed area of the polygon \ccVar. This means that the area is
      positive for counter clockwise polygons and negative for clockwise polygons.
    }
      
  \ccMethod{Vertex_iterator left_vertex();}
    { Returns the leftmost vertex of the polygon \ccStyle{p} with the smallest
     \ccStyle{y}-coordinate. }
    
  \ccMethod{Vertex_iterator right_vertex();}
    { Returns the rightmost vertex of the polygon \ccStyle{p} with the largest
     \ccStyle{y}-coordinate. }
    
  \ccMethod{Vertex_iterator top_vertex();}
    { Returns topmost vertex of the polygon \ccStyle{p} with the largest
     \ccStyle{x}-coordinate. }
    
  \ccMethod{Vertex_iterator bottom_vertex();}
    { Returns the bottommost vertex of the polygon \ccStyle{p} with the smallest
     \ccStyle{x}-coordinate. }
    
\newpage
For convenience we provide the following boolean functions:    
\ccThree{Segment_2}{}{\hspace*{10cm}}

\ccMethod{bool is_counterclockwise_oriented() const;}
    {}

\ccGlue
\ccMethod{bool is_clockwise_oriented() const;}
    {}

\ccGlue
\ccMethod{bool is_collinear_oriented() const;}
    {}

\ccGlue
\ccMethod{bool has_on_positive_side(const Point_2& q) const;}
    {}

\ccGlue
\ccMethod{bool has_on_negative_side(const Point_2& q) const;}
    {}

\ccGlue
\ccMethod{bool has_on_boundary(const Point_2& q) const;}
    {}
  
\ccGlue
\ccMethod{bool has_on_bounded_side(const Point_2& q) const;}
    {}

\ccGlue
\ccMethod{bool has_on_unbounded_side(const Point_2& q) const;}
    {}

\ccHeading{Random access methods}
These methods are only available for random access containers.

  \ccMethod{const Point_2& vertex(int i) const;}
    { Returns a (const) reference to the $i$-th vertex. }

  \ccMethod{const Point_2& operator[](int i) const;}
    { Returns a (const) reference to the $i$-th vertex. }

  \ccMethod{Segment_2 edge(int i) const;}
    { Returns a const reference to the $i$-th edge. }

\ccHeading{Miscellaneous}

  \ccMethod{int size() const;}
    { Returns the number of vertices of the polygon \ccVar.}

  \ccMethod{bool is_empty() const;}
    { Returns $\ccStyle{p.size()} == 0$.}

  \ccMethod{const Container& container() const;}
    { Returns a const reference to the sequence of vertices of the polygon 
      \ccVar. }

  \ccHidden\ccMethod{bool identical(const Polygon_2<Traits,Container> &) const;}
    {}

\ccHeading{Globally defined operators}

\ccStyle{template <class Traits, class Container1, class Container2>}
  \ccFunction{
     bool operator==(const Polygon_2<Traits,Container1>& p1, 
                     const Polygon_2<Traits,Container2>& p2);}
     { Test for equality: two polygons are equal iff there exists a cyclic
       permutation of the vertices of \ccStyle{p2} such that they are equal to the
       vertices of \ccStyle{p1}. Note that the template argument
       \ccStyle{Container} of \ccStyle{p1} and \ccStyle{p2} may be different. }

\ccStyle{template <class Traits, class Container1, class Container2>}
  \ccFunction{
     bool operator!=(const Polygon_2<Traits,Container1>& p1, 
                     const Polygon_2<Traits,Container2>& p2);}
     { Test for inequality. }

\ccStyle{template <class Transformation, class Traits, class Container>}

  \ccFunction{
     Polygon_2<Traits,Container>
     transform(const Transformation& t, const Polygon_2<Traits,Container>& p);}
  { Returns the image of the polygon \ccc{p} under the transformation \ccc{t}. }

\ccHeading{I/O}

The I/O operators are defined for \ccStyle{iostream}, and for
the window stream provided by \cgal. The format for the iostream
is an internal format.

  \ccFunction{ostream& operator<<(ostream& os, const Polygon_2<Traits, Container>& p);}
     { Inserts the polygon \ccVar\ into the stream \ccStyle{os}.
       \ccPrecond The insert operator is defined for class \ccStyle{Point_2}.}

  \ccFunction{istream& operator>>(istream& is,
                  const Polygon_2<Traits, Container>& p);}
     { Reads a polygon from stream \ccStyle{is} and assigns it
       to \ccVar.}

\ccInclude{CGAL/IO/Window_stream.h}

\ccFunction{
Window_stream& operator<<(Window_stream& W,
                               const Polygon_2<Traits, Container>& p);
}
{
Inserts the triangulation \ccStyle{p} into the window stream \ccStyle{W}.
The insert operator must be defined for \ccStyle{Point_2}.
}

\ccExample

The following code fragment creates a polygon and checks if it is convex.

\begin{cprog}
#include <CGAL/basic.h>
#include <CGAL/Cartesian.h>
#include <CGAL/Polygon_2.h>
#include <list>

typedef CGAL::Cartesian<double> R;
typedef CGAL::Polygon_traits_2<R> Traits;
typedef Traits::Point_2 Point;
typedef std::list<Point> Container;
typedef CGAL::Polygon_2<Traits,Container> Polygon;

#include <iostream>

int main()
{
  Polygon p;

  p.push_back(Point(0,0));
  p.push_back(Point(1,0));
  p.push_back(Point(1,1));
  p.push_back(Point(0,1));

  cout << "The polygon is " << (p.is_convex() ? "" : "not ") << "convex." << endl;

  return 0;
}
\end{cprog}

\ccHeading{Implementation}
The methods
\ccStyle{is_simple},
\ccStyle{is_convex},
\ccStyle{orientation},
\ccStyle{oriented_side},
\ccStyle{bounded_side},
\ccStyle{bbox},
\ccStyle{area},
\ccStyle{left_vertex},
\ccStyle{right_vertex},
\ccStyle{top_vertex} and
\ccStyle{bottom_vertex}
are all implemented using the algorithms on sequences of 2D points described
in section \ref{sec:poly_algo}. There you can find information about which
algorithms were used and what their complexity they have.

\end{ccClassTemplate}

\section{Algorithms on sequences of 2D points \label{sec:poly_algo}}

A number of algorithms on sequences of 2D points are supplied as global functions.
In all algorithms the points in the range \ccStyle{[first,last)} are interpreted
as the vertices of a polygon. The vertices in this range should have value type
\ccStyle{Traits::Point_2}. Most functions take a traits class as their last argument.
This is just a technical matter: the template argument \ccStyle{Traits} must
appear in the arguments of the function. Modern compilers should be able to
optimize this traits argument away. In all cases a default version of the function
exists without the \ccStyle{Traits} argument. These default versions use the predefined
polygon traits class from section \ref{sec:poly_default}.

\ccThree{Bounded_side}{}{\hspace*{10.5cm}}

\ccFunction{
template <class ForwardIterator, class Traits>
ForwardIterator left_vertex_2(ForwardIterator first,
                                   ForwardIterator last,
                                   const Traits& traits);
}
{
Returns the leftmost point from the range \ccStyle{[first,last)}
with the smallest \ccStyle{y}-coordinate.
{\sc traits}: uses \ccc{Traits::Less_xy}.
}

\ccFunction{
template <class ForwardIterator, class Traits>
ForwardIterator right_vertex_2(ForwardIterator first,
                                   ForwardIterator last,
                                   const Traits& traits);
}
{
Returns the rightmost point in the range \ccStyle{[first,last)}
with the largest \ccStyle{y}-coordinate. \\
{\sc traits}: uses \ccc{Traits::Less_xy}.
}

{\samepage
\ccFunction{
template <class ForwardIterator, class Traits>
ForwardIterator top_vertex_2(ForwardIterator first,
                                   ForwardIterator last,
                                   const Traits& traits);
}
{
Returns the topmost point from the range \ccStyle{[first,last)}
with the largest \ccStyle{x}-coordinate. \\
{\sc traits}: uses \ccc{Traits::Less_yx}.
}
}

\ccFunction{
template <class ForwardIterator, class Traits>
ForwardIterator bottom_vertex_2(ForwardIterator first,
                                   ForwardIterator last,
                                   const Traits& traits);
}
{
Returns the bottommost point from the range \ccStyle{[first,last)}
with the smallest \ccStyle{x}-coordinate.\\
{\sc traits}: uses \ccc{Traits::Less_yx}.
}

\ccFunction{
template <class InputIterator>
Bbox_2 bbox_2(InputIterator first, InputIterator last);
}
{
Returns the smallest bounding box of the points in the range
\ccStyle{[first,last)}.
\ccPrecond The range \ccStyle{[first,last)} is not empty.
}

\ccFunction{
template <class ForwardIterator, class Numbertype, class Traits>
void area_2(ForwardIterator first,
                 ForwardIterator last,
                 Numbertype& result,
                 const Traits& traits);
}
{
Computes the signed area of the polygon formed by the points in the range
\ccStyle{[first,last)}.\\
{\sc traits}: uses \ccc{Traits::determinant_2}.
}

\ccFunction{
template <class ForwardIterator, class Traits>
bool is_convex_2(ForwardIterator first,
                      ForwardIterator last,
                      const Traits& traits);
}
{
Returns \ccStyle{true} if the polygon formed by the points in the range
\ccStyle{[first,last)} is convex.\\
{\sc traits}: \ccc{Traits::lexicographically_xy_smaller},
                   \ccc{Traits::orientation}.
}


{\samepage
\ccFunction{
template <class ForwardIterator, class Traits>
bool is_simple_2(ForwardIterator first,
                      ForwardIterator last,
                      const Traits& traits);
}
{
Returns \ccStyle{true} if the polygon formed by the points in the range
\ccStyle{[first,last)} is simple.\\
{\sc traits}:\\ 
  \ccc{Traits::lexicographically_yx_smaller_or_equal}, \ccTexHtml{\quad}{}
  \ccc{Traits::do_intersect},
  \ccc{Traits::have_equal_direction}.
  \ccc{Traits::compare_x},
  \ccc{Traits::compare_y},
  \ccc{Traits::cross_product_2},
  \ccc{Traits::is_negative},
}
} 

\ccFunction{
template <class ForwardIterator, class Point, class Traits>
Oriented_side oriented_side_2(ForwardIterator first,
                                        ForwardIterator last,
                                        const Traits::Point_2& point,
                                        const Traits& traits);
}
{
This determines the location of the point \ccStyle{q} with respect to the 
polygon formed by the points in the range \ccStyle{[first,last)}.
It returns \ccStyle{NEGATIVE_SIDE}, or \ccStyle{POSITIVE_SIDE}, or 
\ccStyle{ON_ORIENTED_BOUNDARY}, depending on where point \ccStyle{q} is.
\ccPrecond The points in the range \ccStyle{[first,last)}
form the vertices of a simple polygon.\\
{\sc traits}: uses \ccc{Traits::Less_xy},
                   \ccc{Traits::compare_x},
                   \ccc{Traits::compare_y},
                   \ccc{Traits::determinant_2},
                   \ccc{Traits::orientation} and
                   \ccc{Traits::sign}.
}

\ccFunction{
template <class ForwardIterator, class Point, class Traits>
Bounded_side bounded_side_2(ForwardIterator first,
                                      ForwardIterator last,
                                      const Traits::Point_2& point,
                                      const Traits& traits);
}
{
Determines the location of the point \ccStyle{q} with respect to the polygon
formed by the points in the range \ccStyle{[first,last)}.
It returns either \ccStyle{ON_BOUNDED_SIDE}, \ccStyle{ON_BOUNDARY}
or \ccStyle{ON_UNBOUNDED_SIDE}, depending on where point \ccStyle{q} is.
\ccPrecond The points in the range \ccStyle{[first,last)}
form the vertices of a simple polygon.\\
{\sc traits}: \\ uses \ccc{Traits::compare_x},
                   \ccc{Traits::compare_y},
                   \ccc{Traits::determinant_2} and
                   \ccc{Traits::determinant_2}.
}

{\samepage
\ccFunction{
template <class ForwardIterator, class Traits>
Orientation orientation_2(ForwardIterator first,
                                    ForwardIterator last,
                                    const Traits& traits);
}
{
Returns the orientation of the polygon formed by the points in the range
\ccStyle{[first,last)}. 
If the number of points is smaller than three, \ccStyle{COLLINEAR} is
returned.
\ccPrecond The points in the range \ccStyle{[first,last)}
form the vertices of a simple polygon.\\
{\sc traits}: uses \ccc{Traits::Less_xy} and \ccc{Traits::orientation}.
}
}

\ccImplementation
For the implementation of \ccc{oriented_side_2} 
a {\em crossing test} algorithm is used with complexity
$O(n)$, see \cite{h-pps-94}.
The implementation of \ccc{is_convex_2} checks if the polygon is locally convex
and if there is only one local minimum and one local maximum with respect to
the lexicographical ordering determined by
\ccc{Traits::lexicographically_xy_smaller}.
The complexity of the algorithm is $O(n)$, where $n$ is the number of points
in the range \ccStyle{[first,last)}.
The simplicity test \ccc{is_simple_2} of a polygon is a line-segment 
intersection test. For the implementation a plane sweep algorithm is used, see also 
\cite{ps-cgi-85}, p.276.
The complexity of the algorithm is $O(n \log n)$ (worst case), where $n$ is 
the number of points in the range \ccStyle{[first,last)}.


\section{Polygon Traits class requirements}

The polygon algorithms in section \ref{sec:poly_algo} are parameterized with
a traits class \ccStyle{Traits} that defines the basic types and predicates
that the algorithms use. This section describes the minimal requirements for
the traits class. N.B. Currently the list of requirements contains several
redundant predicates that can be easily expressed in others. For example,
the lexicographical comparison functions can be expressed in the functions
\ccStyle{compare_x} and \ccStyle{compare_y}). These predicates will probably
be removed from the traits class.

\begin{ccClass} {Traits}
\ccCreationVariable{traits}
%\ccThree{Polygon_2<Traits,Container>}{}{\hspace*{7cm}}
%\ccThreeToTwo
\ccTwo{Traits:: Segment_2}{}

\ccTypes
\ccNestedType{FT}
        {The coordinate type of the points of the polygon (i.e. a {\em field type}).}

\ccGlue
\ccNestedType{Point_2}
        {The point type on which the polygon algorithms operate.}

\ccGlue
\ccNestedType{Segment_2}
        {The segment type on which the polygon algorithms operate.}

\ccGlue
\ccNestedType{Vector_2}
        {The vector type on which the polygon algorithms operate.}

\ccGlue
\ccNestedType{Less_xy}
       {Binary predicate object type comparing \ccc{Point_2}s lexicographically.
        Must provide \ccc{bool operator()(Point_2 p, Point_2 q)} where \ccc{true}
        is returned iff $p <_{xy} q$.
        We have $p<_{xy}q$, iff $p_x < q_x$ or $p_x = q_x$ and $p_y < q_y$,
        where $p_x$ and $p_y$ denote $x$ and $y$ coordinate of point $p$ resp.
       }

\ccGlue
\ccNestedType{Less_yx}
       {Same as \ccc{Less_xy} with the roles of $x$ and $y$ interchanged.}

\ccCreation
\ccCreationVariable{traits}

\ccConstructor{ Traits( );}{Default constructor.}

\ccGlue
\ccConstructor{ Traits( Traits const& t);}{Copy constructor.}

\ccOperations
\ccThree{bool}{}{\hspace*{10cm}}

\ccMemberFunction{ 
    bool lexicographically_xy_smaller(const Point_2& p, const Point_2& q) const;
}
{
Returns \ccc{true} iff $p_x < q_x$ or $p_x = q_x$ and $p_y < q_y$.
}

\ccMemberFunction{ 
    bool lexicographically_yx_smaller_or_equal(const Point_2& p,
                                               const Point_2& q) const;
}
{
Returns \ccc{true} iff $p_y \leq q_y$ or $p_y = q_y$ and $p_x \leq q_x$.
}

\ccMemberFunction{ 
    FT cross_product_2(const Vector_2& p, const Vector_2& q) const;
}
{
Returns $p_x q_y - p_y q_x$.
}

\ccMemberFunction{ 
    FT determinant_2(const Point_2& p, const Point_2& q, const Point_2& r) const;
}
{
Returns
$
\begin{ccTexOnly}
  \left|
  \begin{array}{ccc}
     p_x & q_x & r_x \\
     p_y & q_y & r_y \\
     1   & 1   & 1
  \end{array}
  \right|
  =
\end{ccTexOnly}
  (p_x-q_x)(q_y-r_y) - (p_y-q_y)(q_x-r_x).$
}

\ccMemberFunction{ 
    int sign(const FT& x) const;
}
{
\begin{ccTexOnly}
Returns
$
  \left\{
  \begin{array}{lll}
     \ccStyle{-1} & \mbox{if} & x < 0 \\
     \ccStyle{0}  & \mbox{if} & x = 0 \\
     \ccStyle{1}  & \mbox{if} & x > 0
  \end{array}
  \right.
$
\end{ccTexOnly}
}
\begin{ccHtmlOnly}
Returns -1 if x < 0, 0 if x = 0 and 1 if x > 0
\end{ccHtmlOnly}

\ccMemberFunction{
    bool is_negative(const FT& x) const;
}
{
Returns \ccStyle{true} iff \ccStyle{x<0}.
}

\ccMemberFunction{ 
    bool do_intersect(const Point_2& p1,
                      const Point_2& q1,
                      const Point_2& p2,
                      const Point_2& q2) const;
}
{
Returns \ccc{true} iff the segments \ccStyle{[p1,q1]} and \ccStyle{[p2,q2]} intersect.
}

\ccThree{Comparison_result}{}{\hspace*{10.5cm}}

\ccMemberFunction{ 
    Orientation orientation(const Point_2& p,
                                 const Point_2& q,
                                 const Point_2& r) const;
}
{
Returns \ccStyle{LEFTTURN}, if $r$ lies to the left of the oriented
line $l$ defined by $p$ and $q$, returns \ccStyle{RIGHTTURN} if $r$
lies to the right of $l$, and returns \ccStyle{COLLINEAR} if $r$ lies
on $l$.
}

\ccMemberFunction{ 
    Comparison_result compare_x(const Point_2 &p, const Point_2 &q) const;
}
{
\begin{ccTexOnly}
Returns
$
  \left\{
  \begin{array}{lll}
     \ccStyle{SMALLER} & \mbox{if} & p_x < q_x \\
     \ccStyle{EQUAL}   & \mbox{if} & p_x = q_x \\
     \ccStyle{LARGER}  & \mbox{if} & p_x > q_x
  \end{array}
  \right.
$
\end{ccTexOnly}
}
\begin{ccHtmlOnly}
Returns SMALLER if p_x < q_x, EQUAL if p_x = q_x and LARGER if p_x > q_x
\end{ccHtmlOnly}

\ccMemberFunction{ 
    Comparison_result compare_y(const Point_2 &p, const Point_2 &q) const;
}
{
\begin{ccTexOnly}
Returns
$
  \left\{
  \begin{array}{lll}
     \ccStyle{SMALLER} & \mbox{if} & p_y < q_y \\
     \ccStyle{EQUAL}   & \mbox{if} & p_y = q_y \\
     \ccStyle{LARGER}  & \mbox{if} & p_y > q_y
  \end{array}
  \right.
$
\end{ccTexOnly}
}
\begin{ccHtmlOnly}
Returns SMALLER if p_y < q_y, EQUAL if p_y = q_y and LARGER if p_y > q_y
\end{ccHtmlOnly}

\ccMemberFunction{
    bool have_equal_direction(const Vector_2& v1,
                              const Vector_2& v2 ) const;
}
{
Returns \ccc{true} iff the vectors \ccStyle{v1} and \ccStyle{v2} have the
same direction.
}

\end{ccClass}

\begin{ccClassTemplate}{Polygon_traits_2<R>}
\ccSection{Polygon default Traits class} \label{sec:poly_default}

The polygon class \ccStyle{Polygon_2} and the polygon algorithms are
parameterized with a traits class \ccStyle{Traits}. The default polygon
traits class \ccStyle{Polygon_traits_2<R>} is parameterized with a
representation class \ccc{R}.

\ccInclude{CGAL/Polygon_traits_2.h}

\ccTypes
\ccThree{typedef p_Less_yx<Point_2>  Less_yx;}{}{\hspace*{5cm}}

\ccTypedef{typedef R::FT FT;}{}
\ccGlue
\ccTypedef{typedef Point_2<_R>         Point_2;}{}
\ccGlue
\ccTypedef{typedef Segment_2<_R>       Segment_2;}{}
\ccGlue
\ccTypedef{typedef Vector_2<_R>        Vector_2;}{}
\ccGlue
\ccTypedef{typedef p_Less_xy<Point_2>  Less_xy;}{}
\ccGlue
\ccTypedef{typedef p_Less_yx<Point_2>  Less_yx;}{}

\ccCreation
\ccCreationVariable{traits}
\ccGlue
\ccConstructor{ Polygon_traits_2<R>( );}{}

\ccOperations
\ccThree{bool}{}{\hspace*{11cm}}

\ccMemberFunction{
    bool lexicographically_xy_smaller(const Point_2& p, const Point_2& q) const;
}
{
  Returns \ccc{lexicographically_xy_smaller(p,q)}.
}

\ccMemberFunction{
    bool lexicographically_yx_smaller_or_equal(const Point_2& p,
                                               const Point_2& q) const;
}
{
  Returns \ccc{lexicographically_yx_smaller_or_equal(p,q)}.
}

\ccMemberFunction{
    FT cross_product_2(const Vector_2& p, const Vector_2& q) const;
}
{
  Returns \ccc{p.x() * q.y() - q.x() * p.y()}.
}

\ccMemberFunction{
    FT determinant_2(const Point_2& p, const Point_2& q, const Point_2& r) const;
}
{
  Returns \ccc{cross_product_2(p-q, p-r)}.
}

\ccMemberFunction{
    int sign(const FT& x) const;
}
{
  Returns \ccc{sign(x)}.
}

\ccMemberFunction{
    bool is_negative(const FT& x) const;
}
{
  Returns \ccc{is_negative(x)}.
}

\ccMemberFunction{
    bool do_intersect(const Point_2& p1,
                      const Point_2& q1,
                      const Point_2& p2,
                      const Point_2& q2) const;
}
{
  Returns \ccc{do_intersect(Segment_2(p1,q1), Segment_2(p2,q2)}.
}

\ccThree{Comparison_result}{}{\hspace*{11cm}}

\ccMemberFunction{
    Orientation orientation(const Point_2& p,
                                 const Point_2& q,
                                 const Point_2& r) const;
}
{
  Returns \ccc{orientation(p, q, r)}.
}

\ccMemberFunction{
    Comparison_result compare_x(const Point_2 &p, const Point_2 &q) const;
}
{
  Returns \ccc{compare_x(p, q)}.
}

\ccMemberFunction{
    Comparison_result compare_y(const Point_2 &p, const Point_2 &q) const;
}
{
  Returns \ccc{compare_y(p, q)}.
}

\ccMemberFunction{
    bool have_equal_direction(const Vector_2& v1,
                              const Vector_2& v2 ) const;
}
{
  Returns the value of the comparison \ccc{Direction_2<R>(v1) == Direction_2<R>(v2)}.
}

\end{ccClassTemplate}

