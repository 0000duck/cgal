\begin{ccRefConcept}{CircularKernel}

\ccRefines
\ccc{Kernel}

\ccHasModels
\ccc{CGAL::Circular_kernel<BasicGeometricKernel,AlgebraicKernel_2_2>}\\
\ccc{CGAL::Lazy_curved_kernel<CircularKernel>}\\
\ccc{CGAL::Filtered_hexagon_curved_kernel<CircularKernel>}\\
\ccc{CGAL::Filtered_bbox_curved_kernel<CircularKernel>}

\ccTypes

A model of \ccc{CircularKernel} is supposed to provide some basic types

\ccNestedType{Linear_kernel}{Model of \ccc{BasicGeometricKernel}.}
\ccGlue
\ccNestedType{Algebraic_kernel}{Model of \ccc{AlgebraicKernel_2_2}.}

\ccNestedType{RT}{Model of \ccc{RingNumberType}.}
\ccGlue
\ccNestedType{FT}{Model of \ccc{FieldNumberType}.}

\ccNestedType{Root_of_2}{Model of \ccc{AlgebraicKernel_2_2::RootOf_2}.}
\ccGlue
\ccNestedType{Root_for_circles_2_2}{Model of \ccc{AlgebraicKernel_2_2::RootForCircles_2_2}.}
\ccGlue
\ccNestedType{Polynomial_1_2}{Model of \ccc{AlgebraicKernel_2_2::Polynomial_1_2}.}
\ccGlue
\ccNestedType{Polynomial_for_circles_2_2}{Model of \ccc{AlgebraicKernel_2_2::PolynomialForCircles_2_2}.}

and to define the following geometric objects

\ccNestedType{Line_arc_2}{Model of \ccc{CircularKernel::LineArc_2}.}
\ccGlue
\ccNestedType{Circle_2}{Model of \ccc{CircularKernel::Circle_2}.}
\ccGlue
\ccNestedType{Circular_arc_2}{Model of \ccc{CircularKernel::CircularArc_2}.}
\ccGlue
\ccNestedType{Circular_arc_point_2}{Model of \ccc{CircularKernel::CircularArcPoint_2}.}

Moreover, a model of \ccc{CircularKernel} must provide predicates, 
constructions and other functionalities. 

\ccPredicates

\ccNestedType{Compare_x_2}{Model of \ccc{CircularKernel::CompareX_2}.}
\ccGlue
\ccNestedType{Compare_y_2}{Model of \ccc{CircularKernel::CompareY_2}.}
\ccGlue
\ccNestedType{Compare_xy_2}{Model of \ccc{CircularKernel::CompareXY_2}.}

\ccNestedType{Equal_2}{Model of \ccc{CircularKernel::Equal_2}.}

\ccNestedType{Compare_y_at_x_2}{Model of \ccc{CircularKernel::CompareYatX_2}.}
\ccGlue
\ccNestedType{Compare_y_to_right_2}{Model of \ccc{CircularKernel::CompareYtoRight_2}.}

\ccNestedType{Has_on_2}{Model of \ccc{CircularKernel::HasOn_2}.}

\ccNestedType{Do_overlap}{Model of \ccc{CircularKernel::DoOverlap_2}.}

\ccNestedType{In_range_2}{Model of \ccc{CircularKernel::InRange_2}.}

\ccNestedType{Is_vertical_2}{Model of \ccc{CircularKernel::IsVertical_2}.}

\ccNestedType{Is_x_monotone_2}{Model of \ccc{IsXMonotone_2}.}
\ccGlue
\ccNestedType{Is_y_monotone_2}{Model of \ccc{IsYMonotone_2}.}

\ccHeading{Constructions}

\ccNestedType{Construct_circle_2;}{Model of \ccc{CircularKernel::ConstructCircle_2}.}
\ccNestedType{Construct_circular_arc_2}{Model of \ccc{CircularKernel::ConstructCircularArc_2}.}
\ccNestedType{Construct_line_arc_2}{Model of \ccc{CircularKernel::ConstructLineArc_2}.}
\ccNestedType{Construct_circular_arc_point_2}{Model of \ccc{CircularKernel::ConstructCircularArcPoint_2}.}

\ccNestedType{Construct_circular_min_vertex_2}{Model of \ccc{CircularKernel::ConstructCircularMinVertex_2}.}
\ccGlue
\ccNestedType{Construct_circular_max_vertex_2}{Model of \ccc{CircularKernel::ConstructCircularMaxVertex_2}.}
\ccGlue
\ccNestedType{Construct_circular_source_vertex_2}{Model of \ccc{CircularKernel::ConstructCircularSourceVertex_2}.}
\ccGlue
\ccNestedType{Construct_circular_target_vertex_2}{Model of \ccc{CircularKernel::ConstructCircularTargetVertex_2}.}


\ccHeading{Link with the algebraic kernel}

\ccNestedType{Get_equation}{Model of \ccc{CircularKernel::GetEquation}.}

\ccOperations

As in the \ccc{Kernel} concept, for each of the function objects
above, there must exist a member function that requires no arguments
and returns an instance of that function object. The name of the
member function is the uncapitalized name of the type returned with
the suffix \ccc{_object} appended. For example, for the function object 
\ccc{CircularKernel::Construct_circular_arc_2} the following member function 
must exist: 

\ccCreationVariable{ck}
\ccMethod{Construct_circular_arc_2 construct_circular_arc_2_object() const;}{}

\end{ccRefConcept}
