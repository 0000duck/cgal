\begin{ccRefConcept}{CircularKernel}

\ccDefinition

The \ccc{CircularKernel} concept is ............

\ccHasModels
\ccc{CGAL::Circular_kernel<BasicGeometricKernel,AlgebraicKernel>}\\
\ccc{CGAL::Lazy_curved_kernel<CircularKernel>}\\
\ccc{CGAL::Filtered_hexagon_curved_kernel<CircularKernel>}\\
\ccc{CGAL::Filtered_bbox_curved_kernel<CircularKernel>}

\ccTypes

A model of \ccc{CircularKernel} is supposed to provide some basic types

\ccNestedType{Linear_kernel}{Model of \ccc{BasicGeometricKernel}.}
\ccGlue
\ccNestedType{Algebraic_kernel}{Model of \ccc{AlgebraicKernel_2_2}.}

\ccNestedType{RT}{Model of \ccc{RingNumberType}.}
\ccGlue
\ccNestedType{FT}{Model of \ccc{FieldNumberType}.}

\ccNestedType{Root_of_2}{Model of \ccc{AlgebraicKernel_2_2::RootOf_2}.}
\ccGlue
\ccNestedType{Root_for_circles_2_2}{Model of \ccc{AlgebraicKernel_2_2::RootForCircles_2_2}.}
\ccGlue
\ccNestedType{Polynomial_1_2}{Model of \ccc{AlgebraicKernel_2_2::Polynomial_1_2}.}
\ccGlue
\ccNestedType{Polynomial_for_circles_2_2}{Model of \ccc{AlgebraicKernel_2_2::PolynomialForCircles_2_2}.}

and to define the following geometric objects

\ccNestedType{Line_arc_2}{Model of \ccc{CircularKernel::LineArc_2}.}
\ccGlue
\ccNestedType{Circle_2}{Model of \ccc{CircularKernel::Circle_2}.}
\ccGlue
\ccNestedType{Circular_arc_2}{Model of \ccc{CircularKernel::CircularArc_2}.}
\ccGlue
\ccNestedType{Circular_arc_point_2}{Model of \ccc{CircularKernel::CircularArcPoint_2}.}

Moreover, a model of \ccc{CircularKernel} must provide predicates, 
constructions and other functionalities. 

\ccPredicates

\ccNestedType{Compare_x_2}{Model of \ccc{CircularKernel::CompareX_2}.}
\ccGlue
\ccNestedType{Compare_y_2}{Model of \ccc{CircularKernel::CompareY_2}.}
\ccGlue
\ccNestedType{Compare_xy_2}{Model of \ccc{CircularKernel::CompareXY_2}.}

\ccHeading{Constructions}

\ccNestedType{Construct_circle_2;}{Model of ...}

\ccHeading{Link with the algebraic kernel}

\ccNestedType{Get_equation}{Model of \ccc{CircularKernel::GetEquation}.}

\end{ccRefConcept}
