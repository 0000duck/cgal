\begin{ccRefConcept}{AlgebraicKernelForCircles::PolynomialForCircles_2_2}

\ccDefinition

The \ccc{AlgebraicKernelForCircles::PolynomialForCircles_2_2} represents
bivariate polynomials of degree up to~2 capable of storing equations
of circles, whose center has coordinates of type \ccc{FT}, a model of
\ccc{FieldNumberType}, as well as the square of the radius.
\footnote{Of course the name looks bad since it mixes geometry with algebra. 
suggestions welcome}

\ccCreation
\ccCreationVariable{pol}

\ccConstructor{PolynomialForCircles_2_2();}{Default constructor.}

%\ccConstructor{PolynomialForCircles_2_2(const FT & a, const FT & b, const FT & rsq);}{Constructs polynomial \ccc{(x-a)^2 + (y-b)^2 - rsq}.}

%\ccAccessFunctions

%\ccMethod{const FT & a();}{\ccc{x}-coordinate of the center of the circle.}
%\ccGlue
%\ccMethod{const FT & b();}{\ccc{y}-coordinate of the center of the circle.}
%\ccGlue
%\ccMethod{const FT & r_sq();}{Square radius of the center of the circle.}

\ccOperations

The comparison operator \ccc{==} must be provided. 

\ccFunction{bool operator ==
(AlgebraicKernelForCircles::const PolynomialForCircles_2_2 & p,
	const AlgebraicKernelForCircles::PolynomialForCircles_2_2 & q);}{}

\ccHasModels

\ccc{Polynomial_for_circles_2_2}

\ccSeeAlso

\ccRefIdfierPage{AlgebraicKernelForCircles}

\end{ccRefConcept}

