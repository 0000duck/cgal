\begin{ccRefClass}{Curved_kernel<BasicGeometricKernel,AlgebraicKernel>}

\ccDefinition

\ccInclude{CGAL/Circular_kernel.h}

\ccIsModel{CircularKernel}\footnote{for Lutz: macro ``ccIsmodel'' is
apparently assuming that there is only one possible concept...}

\ccParameters

The curved kernel is parameterized by a \ccc{BasicGeometricKernel} parameter
(and derives from it), in order to reuse all needed functionalities on
basic linear objects provided by one of the CGAL kernels. It also
allows other implementations of these basic functionalities.

The second parameter, \ccc{AlgebraicKernel}, is meant to provide the
curved kernel with all the algebraic functionalities required for the
manipulation of algebraic curves. 
\footnote{default arguments?}

\ccInheritsFrom

\ccc{BasicGeometricKernel}

\ccTypes

\ccThree{typedef Circular_arc_point_2<Curved-Kernel>}{Root_of_4xxx}{}
\ccThreeToTwo

The curved kernel takes its basic number types from the algebraic kernel. 
\ccTypedef{typedef AlgebraicKernel::RT RT;}{Ring number type RT.}
\ccTypedef{typedef AlgebraicKernel::FT FT;}{Field number type FT.}
In fact, the two number types \ccc{AlgebraicKernel::RT} and
\ccc{BasicGeometricKernel::RT} must coincide, as well as
\ccc{AlgebraicKernel::FT} and \ccc{BasicGeometricKernel::FT}.
\footnote{how to avoid inconsistencies? a user might take an AK and a
BGK with different RTs...}. 

The types of \ccc{BasicGeometricKernel} are inherited by the curved kernel.
Some types are taken from the \ccc{AlgebraicKernel} parameter, and
some types are defined by the \ccc{Curved_kernel} itself.

\ccc{Curved_kernel} will be a model of \ccc{CircularKernel} if the 
\ccc{AlgebraicKernel} parameter follows the requirements of the 
concept \ccc{AlgebraicKernel_2_2}. The following types will then be
usable, as well as all the functioanlity on them described in the
\ccc{CircularKernel} concept. 

\ccTypedef{typedef Line_arc_2<Curved_kernel> Line_arc_2;}{}
\ccGlue
\ccTypedef{typedef Circle_2<Curved_kernel> Circle_2;}{} 
\ccGlue
\ccTypedef{typedef Circular_arc_2<Curved_kernel> Circular_arc_2;}{}
\ccGlue
\ccTypedef{typedef Circular_arc_point_2<Curved_kernel> Circular_arc_point_2;}{}

\footnote{Here I do as if that Circle\_2 would be moved to Curved
kernel (it is a curve...), which would in fact raise several issues:
backwards compatibility of the code, and also of the license if CK is
not released as the CGAL kernel. I guess it won't be moved...}

\begin{ccAdvanced}
what about the future...? shall we have a different class for each
type of curve (circles, conics, and so on...) or everything in the
same kernel (then how would it work...? several solutions are
posssible)

Do we want to go in the following direction?

\ccc{Curved_kernel} will be a model of \ccc{ConicKernel} if the 
\ccc{AlgebraicKernel} parameter follows the requirements of the 
concept \ccc{AlgebraicKernel_4_2}. The following types will then be usable.

\ccTypedef{typedef Conic_2<Curved_kernel> Conic_2;}{}
\ccGlue
\ccTypedef{typedef Conic_arc_2<Curved_kernel> Conic_arc_2;}{}
\ccGlue
\ccTypedef{typedef Conic_arc_point_2<Curved_kernel> Conic_arc_point_2;}{}
\end{ccAdvanced}

\ccCreationVariable{ck}

\ccPredicates

\ccMethod{Compare_x_2 compare_x_2_object() const;}{}
\ccGlue
\ccMethod{Compare_y_2 compare_y_2_object() const;}{}
\ccGlue
\ccMethod{Compare_xy_2 compare_xy_2_object() const;}{}

\ccHeading{Constructions}

\ccMethod{Construct_circle_2 construct_circle_2_object() const;}{}
\ccGlue
\ccMethod{Construct_circular_arc_2 construct_circular_arc_2_object() const;}{}
\ccGlue
\ccMethod{Construct_circular_arc_endpoint_2 construct_circular_arc_endpoint_2_object() const;}{}

\ccMethod{Construct_conic_2 construct_conic_2_object() const;}{}
\ccGlue
\ccMethod{Construct_conic_arc_2 construct_conic_arc_2_object() const;}{}
\ccGlue
\ccMethod{Construct_conic_arc_endpoint_2 construct_conic_arc_endpoint_2_object() const;}{}

\ccMethod{Make_x_monotone_2 make_x_monotone_2_object() const;}{}
\ccGlue
\ccMethod{Construct_intersections_2 construct_intersections_2_object() const;}{}

\ccHeading{Link with the algebraic kernel}

\ccMethod{Get_equation get_equation_object() const;}{}

\ccSeeAlso

\ccRefIdfierPage{BasicGeometricKernel}\\
\ccRefIdfierPage{AlgebraicKernel_2_2}

\end{ccRefClass}
