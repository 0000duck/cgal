\begin{ccRefFunction}{compare_x}

\ccDefinition

\ccFunction{template < class CircularKernel > 
	Comparison_result compare_x
	(const Circular_arc_point_2<CircularKernel> &p,
	const Circular_arc_point_2<CircularKernel> &q);}
        {Calls the operator() of \ccc{CircularKernel::CompareX_2}.}

\ccSeeAlso

\ccRefConceptPage{CircularKernel::CompareX_2}
\end{ccRefFunction}
\begin{ccRefFunction}{compare_y}

\ccDefinition

\ccFunction{template < class CircularKernel > 
	Comparison_result compare_y
	(const Circular_arc_point_2<CircularKernel> &p,
	const Circular_arc_point_2<CircularKernel> &q);}
        {Calls the operator() of \ccc{CircularKernel::CompareY_2}.}

\ccSeeAlso

\ccRefConceptPage{CircularKernel::CompareY_2}
\end{ccRefFunction}
\begin{ccRefFunction}{compare_xy}

\ccDefinition

\ccFunction{template < class CircularKernel > 
	Comparison_result compare_xy
	(const Circular_arc_point_2<CircularKernel> &p,
	const Circular_arc_point_2<CircularKernel> &q);}
        {Calls the operator() of \ccc{CircularKernel::CompareXY_2}.}

\ccSeeAlso

\ccRefConceptPage{CircularKernel::CompareXY_2}
\end{ccRefFunction}
\begin{ccRefFunction}{make_x_monotone}

\ccDefinition

\ccFunction{template < class CircularKernel, class OutputIterator >
    OutputIterator make_x_monotone
	(const Circular_arc_2<CircularKernel> &ca,
	OutputIterator res);}
        {Calls the operator() of \ccc{CircularKernel::MakeXMonotone_2}.}

\ccSeeAlso

\ccRefConceptPage{CircularKernel::MakeXMonotone_2}
\end{ccRefFunction}
\begin{ccRefFunction}{intersect}

\ccDefinition

\ccFunction{template < class CircularKernel, class OutputIterator >
    OutputIterator intersect
	(const Circle_2<CircularKernel> &ca1,
	const Circle_2<CircularKernel> &ca2, 
	OutputIterator res);}
        {Calls the operator() of \ccc{CircularKernel::Intersect_2}.}

\ccFunction{template < class CircularKernel, class OutputIterator >
    OutputIterator intersect
	(const Circular_arc_2<CircularKernel> &ca1,
	const Circular_arc_2<CircularKernel> &ca2, 
	OutputIterator res);}
        {Calls the operator() of \ccc{CircularKernel::Intersect_2}.}

\ccSeeAlso

\ccRefConceptPage{CircularKernel::Intersect_2}
\end{ccRefFunction}

\begin{ccRefFunction}{get_equation}

\ccFunction{template < class CircularKernel >
	CircularKernel::AlgebraicKernelForCircles::PolynomialForCircles_2_2
	get_equation(const CircularKernel::Circle_2 & c);}
	{Calls the operator() of \ccc{CircularKernel::GetEquation}.}

\ccSeeAlso

\ccRefConceptPage{CircularKernel::GetEquation}
\end{ccRefFunction}
