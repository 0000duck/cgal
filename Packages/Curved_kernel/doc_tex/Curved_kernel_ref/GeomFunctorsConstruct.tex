\begin{ccRefFunctionObjectConcept}{CircularKernel::MakeXMonotone_2}

\ccDefinition

To split curves into monotone pieces. 

\ccCreationVariable{mkxmon}

A model \ccVar\ of this type must provide:

\ccMemberFunction{template < class OutputIterator >
    OutputIterator
    operator()(const CircularKernel::Circular_arc_2 &ca, OutputIterator res);}
{Splits the arc \ccc{ca} into monotone arcs that are returned into the 
output iterator.}

\ccHasModels

\ccc{Circular_kernel::Make_x_monotone_2;}

\ccSeeAlso

\ccRefIdfierPage{CGAL::make_x_monotone}

\end{ccRefFunctionObjectConcept}
\begin{ccRefFunctionObjectConcept}{CircularKernel::Intersect_2}

\ccDefinition

To compute intersections of objects.

\ccCreationVariable{inters}

A model \ccVar\ of this type must provide:

\ccMemberFunction{template < class OutputIterator >
    OutputIterator
    operator()(const CircularKernel::Circle_2 &c1, 
	const CircularKernel::Circle_2 &c2, 
	OutputIterator points);}
{Copies in the output iterators the intersections between the two circles, sorted lexicographically. \ccc{points} iterates on elements of type \ccc{std::pair<CircularKernel::Circular_arc_point_2, int>}, where the integer is the multiplicity of the corresponding intersection point between \ccc{c1} and \ccc{c2}. 
\ccPrecond{\ccc{c1} and \ccc{c2} are not equal.}}

\ccMemberFunction{template < class OutputIterator >
    OutputIterator
    operator()(const CircularKernel::Circle_2 &c, 
	const CircularKernel::Line_arc_2 &l, 
	OutputIterator points);}
{Same as previous, for a circle and a line arc.}

\ccMemberFunction{template < class OutputIterator >
    OutputIterator
    operator()(const CircularKernel::Line_arc_2 &c, 
	const CircularKernel::Circle_2 &l, 
	OutputIterator points);}
{Same as previous.}

\ccMemberFunction{template < class OutputIterator >
    OutputIterator
    operator()(const CircularKernel::Circular_arc_2 &c, 
	const CircularKernel::Line_arc_2 &l, 
	OutputIterator points);}
{Same as previous, for a circular arc and a line arc.}


\ccMemberFunction{template < class OutputIterator >
    OutputIterator
    operator()(const CircularKernel::Line_arc_2 &c, 
	const CircularKernel::Circular_arc_2 &l, 
	OutputIterator points);}
{Same as previous.}

\ccMemberFunction{template < class OutputIteratorPoints, class OutputIteratorArcs >
    std::pair< OutputIteratorPoints, OutputIteratorArcs >
    operator()(const CircularKernel::Circular_arc_2 &ca1, 
	const CircularKernel::Circular_arc_2 &ca2, 
	OutputIteratorPoints points, OutputIteratorArcs arcs);}
{Copies in the output iterators the intersections between the two arcs, sorted lexicographically. \ccc{points} iterates on elements of type \ccc{std::pair<CircularKernel::Circular_arc_point_2, int>}, where the integer is the multiplicity of the corresponding intersection point between \ccc{ca1} and \ccc{ca2}, and \ccc{arcs} iterates on elements of type \ccc{CircularKernel::Circular_arc_2} that are arcs on which \ccc{ca1} and \ccc{ca2} overlap.}

\footnote{note that the multiplicity is not defined for the case of a point that is a common endpoint of the two arcs. Does it make sense to decide a convention: multiplicity=0...? to be discussed}

\ccMemberFunction{template < class OutputIteratorPoints, class OutputIteratorArcs >
    std::pair< OutputIteratorPoints, OutputIteratorArcs >
    operator()(const CircularKernel::Line_arc_2 &ca1, 
	const CircularKernel::Line_arc_2 &ca2, 
	OutputIteratorPoints points, OutputIteratorArcs arcs);}
{Same as previous, for line arcs.}

\ccHasModels

\ccc{Circular_kernel::Intersect_2;}

\ccSeeAlso

\ccRefIdfierPage{CGAL::intersect}

\end{ccRefFunctionObjectConcept}

\begin{ccRefFunctionObjectConcept}{CircularKernel::Split_2}

\ccDefinition

To split curves at a given point.

\ccCreationVariable{sp}

A model \ccVar\ of this type must provide:

\ccMemberFunction{void operator()
	(const CircularKernel::Circular_arc_2 &a, 
	const CircularKernel::Circular_arc_point_2 &p,
	CircularKernel::Circular_arc_2 &a1, 
	CircularKernel::Circular_arc_2 &a2);}
{Splits arc $a$ at point $p$, which creates arcs $a1$ and $a2$.
\ccPrecond{$a$ is $x$-monotone, and $p$ lies on $a$.}}

\ccMemberFunction{void operator()
	(const CircularKernel::Line_arc_2 &l, 
	 const CircularKernel::Circular_arc_point_2 &p,
	 CircularKernel::Line_arc_2 &l1, CircularKernel::Line_arc_2 &l2);}
{Same for a line arc.}

\ccHasModels

\ccc{Circular_kernel::Split_2}

\ccSeeAlso

\ccRefIdfierPage{CGAL::split}

\end{ccRefFunctionObjectConcept}
