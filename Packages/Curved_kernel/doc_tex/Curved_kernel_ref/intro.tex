\chapter{Curved Kernel}

\textbf{Submission - Monique - with Sylvain's help...}

The main class is: 

\ccRefIdfierPage{CGAL::Curved_kernel<BasicGeometricKernel,AlgebraicKernel>}

\begin{ccAdvanced}
I use the ``Advanced'' environment in this document to distinguish
between my current submission to the CGAL editorial board and plans
for the future, related to ACS. The ``Advanced'' parts will disappear
if/when this is released. 
\end{ccAdvanced}

%%%%%%%%%%%%%%%%%%%%%%%%%%%%%%%%%%%%%%%%%%%%%%%%%%%%%%%%%
\section*{Concepts}

\footnote{Note that I have a big question here. Should the kernel here
be described as a concept, as is done currently in CGAL? formally, it
must be, since the CK appears as a template parameter of geometric
object classes. However, it is hard for a user to browse the manual
written that way.}

\ccRefConceptPage{CircularKernel_2}

\ccRefConceptPage{BasicGeometricKernel}

\begin{ccAdvanced}
\ccRefConceptPage{ConicKernel_2}
\end{ccAdvanced}

	\subsubsection*{Functors} 

\ccRefConceptPage{CircularKernel::CompareX_2}\\
\ccRefConceptPage{CircularKernel::CompareY_2}\\
\ccRefConceptPage{CircularKernel::CompareXY_2}

\ccRefConceptPage{CircularKernel::ConstructCircle_2}\\
\ccRefConceptPage{CircularKernel::ConstructCircularArc_2}\\
\ccRefConceptPage{CircularKernel::ConstructCircularArcEndpoint_2}\\

\ccRefConceptPage{CircularKernel::IsXMonotone_2}\\
\ccRefConceptPage{CircularKernel::MakeXMonotone_2}\\
\ccRefConceptPage{CircularKernel::ConstructIntersections_2}\\

\ccRefConceptPage{CircularKernel::GetEquation}\\

\begin{ccAdvanced}
\ccRefConceptPage{ConicKernel::CompareX_2}\\
\ccRefConceptPage{ConicKernel::CompareY_2}\\
\ccRefConceptPage{ConicKernel::CompareXY_2}

\ccRefConceptPage{ConicKernel::ConstructConic_2}\\
\ccRefConceptPage{ConicKernel::ConstructConicArc_2}\\
\ccRefConceptPage{ConicKernel::ConstructConicArcEndpoint_2}

\ccRefConceptPage{ConicKernel::IsXMonotone_2}\\
\ccRefConceptPage{ConicKernel::MakeXMonotone_2}\\
\ccRefConceptPage{ConicKernel::ConstructIntersections_2}

\ccRefConceptPage{ConicKernel::GetEquation}
\end{ccAdvanced}

%%%%%%%%%%%%%%%%%%%%%%%%%%%%%%%%%%%%%%%%%%%%%%%%%%%%%%%%%
\section*{Classes}

\footnote{same big question here about concept/class for the
kernel. the lazy stuff is a good reason to document the kernel as a
concept. But again this does not help the reader...}

\ccRefIdfierPage{CGAL::Curved_kernel<BasicGeometricKernel,AlgebraicKernel>}\\
\ccRefIdfierPage{!!!!!!!!!!!!!!! LAZY STUFF !!!!!!!!!!!!!}

	\subsubsection*{Curves and Surfaces}
\ccRefIdfierPage{CGAL::Circle_2<CircularKernel>}\\

\begin{ccAdvanced}
\ccRefIdfierPage{CGAL::Conic_2<ConicKernel>}\\
\ccRefIdfierPage{CGAL::Ellipse_2<ConicKernel>}\footnote{useful for ETH?}\\
\ccRefIdfierPage{CGAL::Sphere_3<SphericalKernel>}\\
\ccRefIdfierPage{CGAL::Quadric_3<Quadr...Kernel>}
\end{ccAdvanced}

	\subsubsection*{Arcs}
\ccRefIdfierPage{CGAL::Circular_arc_2<CircularKernel>}

\begin{ccAdvanced}
\ccRefIdfierPage{CGAL::Conic_arc_2<ConicKernel>}\\
\ccRefIdfierPage{CGAL::Sphere_patch_3<SphericalKernel>}\\
\ccRefIdfierPage{CGAL::Quadric_patch_3<SphericalKernel>}
\end{ccAdvanced}

	\subsubsection*{Points}
\ccRefIdfierPage{CGAL::Circular_arc_endpoint_2<CircularKernel>}

\begin{ccAdvanced}
\ccRefIdfierPage{CGAL::Conic_arc_endpoint_2<ConicKernel>}
\end{ccAdvanced}

%%%%%%%%%%%%%%%%%%%%%%%%%%%%%%%%%%%%%%%%%%%%%%%%%%%%%%%%%
\section*{Functions}

	\subsubsection*{Predicates}
\ccRefIdfierPage{CGAL::compare_x}\\
\ccRefIdfierPage{CGAL::compare_y}\\
\ccRefIdfierPage{CGAL::compare_xy}

	\subsubsection*{Constructions}

\ccRefIdfierPage{CGAL::make_x_monotone}\\
\ccRefIdfierPage{CGAL::construct_intersections}

	\subsubsection*{others?} 

\ccRefIdfierPage{CGAL::get_equation}

%%%%%%%%%%%%%%%%%%%%%%%%%%%%%%%%%%%%%%%%%%%%%%%%%%%%%%%%%
\section*{Traits class for CGAL Arrangements}

\ccRefIdfierPage{CGAL::Circular_arc_traits}

\begin{ccAdvanced}
\ccRefIdfierPage{CGAL::Conic_arc_traits}
\end{ccAdvanced}

%%%%%%%%%%%%%%%%%%%%%%%%%%%%%%%%%%%%%%%%%%%%%%%%%%%%%%%%%
% \section*{Enums}
