\chapter{Circular Kernel}

\textbf{Submission - Monique - with Sylvain's help...}

%%%%%%%%%%%%%%%%%%%%%%%%%%%%%%%%%%%%%%%%%%%%%%%%%%%%%%%%%
\section*{Geometric concepts}

\ccRefConceptPage{CircularKernel}

\ccRefConceptPage{LinearKernel}

	\subsubsection*{Functors} 

\ccRefConceptPage{CircularKernel::CompareX_2}\\
\ccRefConceptPage{CircularKernel::CompareY_2}\\
\ccRefConceptPage{CircularKernel::CompareXY_2}

\ccRefConceptPage{CircularKernel::Equal_2}

\ccRefConceptPage{CircularKernel::CompareYatX_2}\\
\ccRefConceptPage{CircularKernel::CompareYtoRight_2}

\ccRefConceptPage{CircularKernel::HasOn_2}

\ccRefConceptPage{CircularKernel::DoOverlap_2}

\ccRefConceptPage{CircularKernel::InXRange_2}\\
\ccRefConceptPage{CircularKernel::InYRange_2}

\ccRefConceptPage{CircularKernel::IsVertical_2}

\ccRefConceptPage{CircularKernel::IsXMonotone_2}\\
\ccRefConceptPage{CircularKernel::IsYMonotone_2}

\ccRefConceptPage{CircularKernel::ConstructCircularMinVertex_2}\\
\ccRefConceptPage{CircularKernel::ConstructCircularMaxVertex_2}\\
\ccRefConceptPage{CircularKernel::ConstructCircularSourceVertex_2}\\
\ccRefConceptPage{CircularKernel::ConstructCircularTargetVertex_2}

\footnote{technical remark: the previous functors have a different name 
``Circular'' because the operators() don't have the same return type 
as the existing CGAL functors... it would be nice to find a way to avoid 
this, but I don't know any technique for this.} 

\ccRefConceptPage{CircularKernel::ConstructCircularArc_2}\\
\ccRefConceptPage{CircularKernel::ConstructLineArc_2}\\
\ccRefConceptPage{CircularKernel::ConstructCircularArcPoint_2}

\ccRefConceptPage{CircularKernel::MakeXMonotone_2}\\
\ccRefConceptPage{CircularKernel::Intersect_2}\\
\ccRefConceptPage{CircularKernel::Split_2}

\ccRefConceptPage{CircularKernel::ConstructBbox_2}

\ccRefConceptPage{CircularKernel::GetEquation}\\

%\begin{ccAdvanced}
%\ccRefConceptPage{ConicKernel::CompareX_2}\\
%\ccRefConceptPage{ConicKernel::CompareY_2}\\
%\ccRefConceptPage{ConicKernel::CompareXY_2}

%\ccRefConceptPage{ConicKernel::ConstructConic_2}\\
%\ccRefConceptPage{ConicKernel::ConstructConicArc_2}\\
%\ccRefConceptPage{ConicKernel::ConstructConicArcPoint_2}

%\ccRefConceptPage{ConicKernel::IsXMonotone_2}\\
%\ccRefConceptPage{ConicKernel::MakeXMonotone_2}\\
%\ccRefConceptPage{ConicKernel::Intersect_2}

%\ccRefConceptPage{ConicKernel::GetEquation}
%\end{ccAdvanced}

%%%%%%%%%%%%%%%%%%%%%%%%%%%%%%%%%%%%%%%%%%%%%%%%%%%%%%%%%
\section*{Algebraic concepts}

\ccRefConceptPage{AlgebraicKernelForCircles}

\ccRefConceptPage{AlgebraicKernelForCircles}\\
\ccRefConceptPage{AlgebraicKernelForCircles::Polynomial_1_2}\\
\ccRefConceptPage{AlgebraicKernelForCircles::PolynomialForCircles_2_2}\\
\ccRefConceptPage{AlgebraicKernelForCircles::RootOf_2}\\
\ccRefConceptPage{AlgebraicKernelForCircles::RootForCircles_2_2}

\footnote{General remark about the suffix ${}_d{}_v$: $_d$ stands
for the degree of the polynomials and the algebraic numbers, and
$_v$ stands for the number of variables, which is analogous to the
dimension for CGAL geometric objects.}

	\subsubsection*{} 
\ccRefConceptPage{AlgebraicKernelForCircles::ConstructPolynomial_1_2}\\
\ccRefConceptPage{AlgebraicKernelForCircles::ConstructPolynomialForCircles_2_2}

\ccRefConceptPage{AlgebraicKernelForCircles::Solve}

\ccRefConceptPage{AlgebraicKernelForCircles::SignAt}

\ccRefConceptPage{AlgebraicKernelForCircles::XCriticalPoints}\\
\ccRefConceptPage{AlgebraicKernelForCircles::YCriticalPoints}

\ccRefConceptPage{AlgebraicKernelForCircles::CompareX}\\
\ccRefConceptPage{AlgebraicKernelForCircles::CompareY}\\
\ccRefConceptPage{AlgebraicKernelForCircles::CompareXY}

%%%%%%%%%%%%%%%%%%%%%%%%%%%%%%%%%%%%%%%%%%%%%%%%%%%%%%%%%
\section*{Classes}

	\subsubsection*{Kernels}

\ccRefIdfierPage{CGAL::Circular_kernel<LinearKernel,AlgebraicKernelForCircles>}
\\
\ccRefIdfierPage{CGAL::Lazy_curved_kernel<??>}\\
\ccRefIdfierPage{CGAL::Filtered_hexagon_curved_kernel<CircularKernel>}\\
\ccRefIdfierPage{CGAL::Filtered_bbox_curved_kernel<CircularKernel>}

\ccRefIdfierPage{CGAL::Algebraic_kernel_for_circles_2_2<RT>}

	\subsubsection*{Arcs}
\ccRefIdfierPage{CGAL::Circular_arc_2<CircularKernel>}\\
\ccRefIdfierPage{CGAL::Line_arc_2<CircularKernel>}

	\subsubsection*{Points}
\ccRefIdfierPage{CGAL::Circular_arc_point_2<CircularKernel>}

	\subsubsection*{Algebraic objects} 
\ccRefIdfierPage{CGAL::Polynomial_1_2<RT>}\\
\ccRefIdfierPage{CGAL::Polynomial_for_circles_2_2<FT>}

\ccRefIdfierPage{CGAL::Root_of_2<RT>}\\
\ccRefIdfierPage{CGAL::Root_for_circles_2_2<FT>}

\ccRefIdfierPage{CGAL::Root_of_traits_2<RT>}

%\begin{ccAdvanced}
%\ccRefIdfierPage{CGAL::Conic_arc_point_2<ConicKernel>}
%\end{ccAdvanced}

%%%%%%%%%%%%%%%%%%%%%%%%%%%%%%%%%%%%%%%%%%%%%%%%%%%%%%%%%
\section*{Functions}

	\subsubsection*{Predicates}
\ccRefIdfierPage{CGAL::compare_x}\\
\ccRefIdfierPage{CGAL::compare_y}\\
\ccRefIdfierPage{CGAL::compare_xy}

\ccRefIdfierPage{CGAL::compare_y_at_x}\\
\ccRefIdfierPage{CGAL::compare_y_to_right}

\ccRefIdfierPage{CGAL::sign_at}

	\subsubsection*{Constructions}
\ccRefIdfierPage{CGAL::make_x_monotone}\\
\ccRefIdfierPage{CGAL::intersect}

\ccRefIdfierPage{CGAL::make_root_of_2}

\ccRefIdfierPage{CGAL::solve}

\ccRefIdfierPage{CGAL::x_critical_points}\\
\ccRefIdfierPage{CGAL::y_critical_points}

	\subsubsection*{Accessors} 
\ccRefIdfierPage{CGAL::get_equation}

%%%%%%%%%%%%%%%%%%%%%%%%%%%%%%%%%%%%%%%%%%%%%%%%%%%%%%%%%
\section*{Traits classes for CGAL Arrangements}

\ccRefIdfierPage{CGAL::Circular_arc_traits<CircularKernel>}\\
\ccRefIdfierPage{CGAL::Line_arc_traits<CircularKernel>}\\
\ccRefIdfierPage{CGAL::Variant_traits<CircularKernel>}

%\begin{ccAdvanced}
%\ccRefIdfierPage{CGAL::Conic_arc_traits}
%\end{ccAdvanced}

%%%%%%%%%%%%%%%%%%%%%%%%%%%%%%%%%%%%%%%%%%%%%%%%%%%%%%%%%
% \section*{Enums}

\bigskip
