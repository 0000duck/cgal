\begin{ccRefConcept}{AlgebraicKernelForCircles::CompareX}

\ccDefinition

\ccCreationVariable{cmpx}

A model \ccVar\ of this type must provide:

\ccMethod{template < class OutputIterator >
    CGAL::Comparison_result
    operator()(const AlgebraicKernelForCircles::RootForCircles_2_2 & r1,
	     const AlgebraicKernelForCircles::RootForCircles_2_2 & r2);}
{Compares the \ccc{x} variables of two \ccc{RootForCircles_2_2}.}

\ccHasModels

\ccc{Algebraic_kernel_for_circles_2_2::Compare_x;}

\ccSeeAlso

\ccRefIdfierPage{CGAL::compare_x}

\end{ccRefConcept}
\begin{ccRefConcept}{AlgebraicKernelForCircles::CompareY}

\ccDefinition

\ccCreationVariable{cmpy}

A model \ccVar\ of this type must provide:

\ccMethod{template < class OutputIterator >
    CGAL::Comparison_result
    operator()(const AlgebraicKernelForCircles::RootForCircles_2_2 & r1,
	     const AlgebraicKernelForCircles::RootForCircles_2_2 & r2);}
{Compares the \ccc{y} variables of two \ccc{RootForCircles_2_2}.}

\ccHasModels

\ccc{Algebraic_kernel_for_circles_2_2::Compare_y;}

\ccSeeAlso

\ccRefIdfierPage{CGAL::compare_y}

\end{ccRefConcept}
\begin{ccRefConcept}{AlgebraicKernelForCircles::CompareXY}

\ccDefinition

\ccCreationVariable{cmpxy}

A model \ccVar\ of this type must provide:

\ccMethod{template < class OutputIterator >
    CGAL::Comparison_result
    operator()(const AlgebraicKernelForCircles::RootForCircles_2_2 & r1,
	     const AlgebraicKernelForCircles::RootForCircles_2_2 & r2);}
{Compares two \ccc{RootForCircles_2_2} lexicographically.}

\ccHasModels

\ccc{Algebraic_kernel_for_circles_2_2::Compare_xy;}

\ccSeeAlso

\ccRefIdfierPage{CGAL::compare_xy}

\end{ccRefConcept}

\begin{ccRefConcept}{AlgebraicKernelForCircles::SignAt}

\ccDefinition

\ccCreationVariable{sign}

A model \ccVar\ of this type must provide:

\ccMethod{template < class OutputIterator >
    CGAL::Sign
    operator()(const AlgebraicKernelForCircles::Polynomial_1_2 & p,
	     const AlgebraicKernelForCircles::RootForCircles_2_2 & r);}
{Computes the sign of polynomial \ccc{p} evaluated at a root \ccc{r}.}

\ccMethod{template < class OutputIterator >
    CGAL::Sign
    operator()(const AlgebraicKernelForCircles::PolynomialForCircles_2_2 & p,
	     const AlgebraicKernelForCircles::RootForCircles_2_2 & r);}
{Same as previous.}

\ccHasModels

\ccc{Algebraic_kernel_for_circles_2_2::Sign_at;}

\ccSeeAlso

\ccRefIdfierPage{CGAL::sign_at}

\end{ccRefConcept}

