\begin{ccRefConcept}{AlgebraicKernel_2_2} 

\ccDefinition

The \ccc{AlgebraicKernel_2_2} concept is meant to provide the
curved kernel with all the algebraic functionalities required for the
manipulation of circular arcs. 

\ccTypes

A model of \ccc{AlgebraicKernel_2_2} is supposed to provide

\ccNestedType{RT}{A model of \ccc{RingNumberType}. In addition, the class \ccc{Root_of_traits_2<RT>} must be defined and provide a nested type \ccc{Type} which must be the same as \ccc{Root_of_2} and a function \ccc{make_root_of_2(RT,RT,RT,int)} whose return type is \ccc{Type}.}\footnote{\ccc{make_root_of_2} to be replaced by solve}
\ccGlue
\ccNestedType{FT}{A model of \ccc{FieldNumberType<RT>}.} \footnote{concept template...?}
\ccc{RT} is supposed to be \ccc{Rational_traits<FT>}.

\ccNestedType{Root_of_2}{A model of \ccc{RootOf_2}, for algebraic numbers 
		of degree up to~2.}
\ccGlue
\ccNestedType{Polynomial_circle_2_2}{A model of \ccc{PolynomialCircle_2_2}, for bivariate polynomials of degree up to~2 that can store equations of circles.}

\ccNestedType{Construct_polynomial_circle_2_2}{A model of \ccc{ConstructPolynomialCircle_2_2}.}

\ccNestedType{Solve}{A model of the concept \ccc{AlgebraicKernel_2_2::Solve}.}

\ccHasModels

\footnote{to be done}

\ccSeeAlso

\ccRefIdfierPage{RootOf_2}\\
\ccRefIdfierPage{PolynomialCircle_2_2}\\
\ccRefIdfierPage{CGAL::Curved_kernel<BasicGeometricKernel,AlgebraicKernel>}

\end{ccRefConcept}
