\chapter{Circular kernel}
\label{chapter-circular-kernel}

\ccChapterAuthor{Monique Teillaud, Sylvain Pion}

\section{Introduction}

The CGAL kernel provide the user mainly with \textit{linear} objects
(points, line segments, lines...) and predicates on them. It also
defines circles but does not offer a lot of functionalities of them

The goal of this circular kernel is to offer to the user a large set
of functionalities on circles and circular arcs, with an interface
that is similar to the CGAL kernel interface. 

Two new main geometric objects are introduced: circular arcs, points
of circular arcs (used in particular for endpoints and intersection
points) and line segments whose endpoints are these new points.  

In fact, the circular kernel is documented as a concept, and several
models are provided: \ccc{Circular_kernel} is the basic kernel, and
several filtered kernels are built on top of this basic kernel to
increase effciency: \footnote{to be written}

In this first release, all functionalities necessary for computing an
arrangement of circular arcs and these line segments are
defined. Three traits classes are provided for the CGAL arrangement
package. 

\section{Software design}

The design is done in such a way that the algebraic concepts and the
geometric concepts are clearly separated: the \ccc{Circular_kernel}
has two template parameters: 
\begin{itemize}
\item {} the \ccc{BasicGeometricKernel}, from which the circular kernel derives,
provides all basic geometric objects like points, lines, circles, and
elementary functionality on them. In fact it is meant to be
instantiated by a CGAL kernel. 
\item {} the \ccc{AlgebraicKernel} is responsible for computations on
polynomials and algebraic numbers. It has to be a model of concept 
\ccc{AlgebraicKernel_2_2} described in a separate chapter. The
robustness of the package relies on the fact that the algebraic kernel
provides exact computations on algebraic objects.
\end{itemize}

The types of \ccc{BasicGeometricKernel} are inherited by the circular kernel.
Some types are taken from the \ccc{AlgebraicKernel} parameter, and
some types are defined by the \ccc{Circular_kernel} itself.

\section{Examples}

to be included

\section{Design and Implementation History}
