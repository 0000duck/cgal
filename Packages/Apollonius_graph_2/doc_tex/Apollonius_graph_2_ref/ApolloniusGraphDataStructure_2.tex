% +------------------------------------------------------------------------+
% | Reference manual page: Apollonius_graph_data_structure_2.tex
% +------------------------------------------------------------------------+
% | 12.04.2000   Author
% | Package: Package
% | 
%\RCSdef{\RCSRegulartriangulationtraitsRev}{$Revision$}
%\RCSdefDate{\RCSRegulartriangulationtraitsDate}{$Date$}
% |
%%RefPage: end of header, begin of main body
% +------------------------------------------------------------------------+


\begin{ccRefConcept}{ApolloniusGraphDataStructure_2}

%% \ccHtmlCrossLink{}     %% add further rules for cross referencing links
%% \ccHtmlIndexC[concept]{} %% add further index entries
\ccCreationVariable{agds}
\ccDefinition

The concept \ccc{ApolloniusGraphDataStructure_2} refines the concept
\ccc{TriangulationDataStructure_2}
described in \ccRefPage{TriangulationDataStructure_2}. In addition
it provides two methods for the insertion and removal of a degree 2
vertex in the data structure. We only describe the additional 
requirements with respect to the \ccc{TriangulationDataStructure_2}
concept.

\ccRefines
\ccc{TriangulationDataStructure_2}

\ccHeading{Insertion}
\ccThree{Vertex_handle}{agds.insert_degree_2(Face_handle f, int i)+}{}
%
\ccMethod{Vertex_handle insert_degree_2(Face_handle f, int i);}{Inserts 
a degree two vertex and two faces adjacent to it that have two common
edges. The edge defined by the face handle \ccc{f} and the integer
\ccc{i} is duplicated. It returns a handle to the vertex created.}

\ccHeading{Removal}
\ccMethod{void remove_degree_2(Vertex_handle v);}{Removes a degree 2
vertex and the two faces adjacent to it. The two edges of the star of
\ccc{v} that are not incident to it are collapsed.}


\ccHasModels
\ccc{CGAL::Apollonius_graph_data_structure_2<Vb,Fb>}

\ccSeeAlso
\ccc{TriangulationDataStructure_2}\\
\ccc{ApolloniusGraphVertexBase_2}\\
\ccc{ApolloniusGraphFaceBase_2}

\end{ccRefConcept}

% +------------------------------------------------------------------------+
%%RefPage: end of main body, begin of footer
% EOF
% +------------------------------------------------------------------------+
