% +------------------------------------------------------------------------+
% | Reference manual page: Apollonius_graph_2.tex
% +------------------------------------------------------------------------+
% | 12.04.2000   Author
% | Package: Package
% | 
%\RCSdef{\RCSRegulartriangulationtraitsRev}{$Revision$}
%\RCSdefDate{\RCSRegulartriangulationtraitsDate}{$Date$}
% |
%%RefPage: end of header, begin of main body
% +------------------------------------------------------------------------+


\begin{ccRefClass}{Apollonius_graph_2<Gt,StoreHidden,Agds>}
%% add template arg's if necessary

%% \ccHtmlCrossLink{}     %% add further rules for cross referencing links
%% \ccHtmlIndexC[class]{} %% add further index entries
\ccCreationVariable{traits}
\ccDefinition

The class \ccRefName\ represents the
Apollonius graph. It supports insertions and deletions of weighted
points. It is templated by three template arguments \ccc{Gt}, which
must be a model of \ccc{ApolloniusGraphTraits_2}, \ccc{StoreHidden},
which is a boolean indicating if hidden weighted points are to be
stored, and \ccc{Agds}, which must be a model of
\ccc{ApolloniusGraphDataStructure_2}. The second template argument
defaults to \ccc{true}, but if only insertions of weighted points are
to be made then the recommended value is \ccc{false}. The third template
argument defaults to \ccc{CGAL::Apollonius_graph_data_structure_2<
CGAL::Apollonius_graph_vertex_base_2<Gt>,
CGAL::Apollonius_graph_face_base_2<Gt> >}.

\ccInclude{CGAL/Apollonius_graph_2.h}

\ccTypes

\ccThree{typedef Gt::Weighted_point_2}{Weighted_point_2+}{}
%
\ccTypedef{typedef Agds  Data_structure;}{A type for the underlying
data structure.}
\ccGlue
\ccTypedef{typedef Gt Geom_traits;}{A type for the geometric traits.}
\ccGlue
\ccTypedef{typedef Gt::Bare_point_2 Point_2;}{A type for the bare
point defined in the geometric traits.}
\ccGlue
\ccTypedef{typedef Gt::Weighted_point_2  Weighted_point_2;}
{A type for the weighted point defined in the geometric traits.}
\ccGlue
\ccTypedef{typedef Gt::Weight          Weight;}
{A type for the weight of the weighted point; using the definition in
the geometric traits.}
%

% MK:: in the following copy text from TDS

The vertices and faces of the Apollonius graph are accessed through 
\ccc{handles}, 
\ccc{iterators} and \ccc{circulators}. 
The iterators and circulators
are all bidirectional and non mutable.
The circulators and iterators are assignable to the 
corresponding handle types, and they are also convertible to the
corresponding handles.
The edges of the Apollonius graph can also be visited through iterators
and circulators,
the edge circulators and iterators
are also bidirectional and non mutable.
In the following, we called {\it infinite} any face or edge 
incident  to the infinite vertex and the infinite vertex itself.
 Any other feature (face, edge or vertex) of the Apollonius graph is said 
to be {\it finite}.
Some iterators (the \ccc{All} iterators ) allow to visit finite or 
infinite features while the others (the \ccc{Finite} iterators) visit only
finite features. Circulators visit infinite features as well as finite 
ones.

\ccThree{typedef std::pair<Face_handle, int>}{}{the apollonius graph,
the apollonius graph, the graph}
%\ccThreeToTwo
%
% MK:: change the edge definition as follows?
\ccTypedef{typedef pair<Face_handle, int> Edge;} {the edge type.
The \ccc{Edge(f,i)} is the edge common to faces \ccc{f} and 
\ccc{f.neighbor(i)}. It is also the edge joining the vertices
\ccc{vertex(cw(i))} and \ccc{vertex(ccw(i))} of \ccc{f}.}
%
\ccTwo{Apollonius_graph_2<Gt,StoreHidden,Agds>::Finite_vertices_iterator+}
{}
%
\ccGlue
\ccNestedType{Vertex_handle}{handle to a vertex.}
\ccGlue
\ccNestedType{Face_handle}{handle to a face.}
\ccGlue
\ccNestedType{Vertex_circulator}{circulator over vertices incident to
a given vertex.}
\ccGlue
\ccNestedType{Face_circulator}{circulator over faces incident to a
given vertex.}
\ccGlue
\ccNestedType{Edge_circulator}{circulator over edges incident to a
given vertex.}
\ccGlue
\ccNestedType{All_vertices_iterator}{iterator over all vertices.}
\ccGlue
\ccNestedType{Finite_vertices_iterator}{iterator over finite vertices.}
\ccGlue
\ccNestedType{All_faces_iterator}{iterator over all faces.}
\ccGlue
\ccNestedType{Finite_faces_iterator}{iterator over finite faces.}
\ccGlue
\ccNestedType{All_edges_iterator}{iterator over all edges.}
\ccGlue
\ccNestedType{Finite_edges_iterator}{iterator over finite edges.}

\ccCreationVariable{ag}

\ccCreation
\ccThree{Apollonius_graph_2<Gt,StoreHidden,Agds>}{ag = other other}{}
\ccThreeToTwo
%
\ccConstructor{Apollonius_graph_2(Gt gt=Gt());}{Creates an
Apollonius graph using \ccc{gt} as geometric traits.}
%
\ccConstructor{template< class Input_iterator >
Apollonius_graph_2(Input_iterator first, Input_iterator beyond,
Gt gt=Gt());} {Creates an Apollonius graph using \ccc{gt} as
geometric traits and inserts all weighted points in the range
[\ccc{first}, \ccc{beyond}).
\ccPrecond{\ccc{Input_iterator} must be a model of
\ccc{InputIterator}. The value type of
\ccc{Input_iterator} must be \ccc{Gt::Weighted_point_2}.}}
%
\ccConstructor{Apollonius_graph_2(const Apollonius_graph_2<Gt,StoreHidden,Agds>& other)}{Copy
constructor. All faces and vertices are duplicated. After the construction,
\ccVar\ and \ccc{other} refer to two different Apollonius graphs~: if
\ccc{other} is modified, \ccVar\ is not.}
%
\ccMethod{Apollonius_graph_2<Gt,StoreHidden,Agds>
operator=(const Apollonius_graph_2<Gt,StoreHidden,Agds>& other);}
{Assignment. If \ccc{*this} and \ccc{other} are the same object
  nothing is done. Otherwise, all the vertices and faces are
  duplicated. After the assignment, \ccVar\ and \ccc{other} refer to
  different Apollonius graphs~: if \ccc{other} is modified, \ccVar\ is
  not.}




\ccAccessFunctions
% MK:: in the following copy text from TDS
\ccThree{Vertex_handle}{ag.number_of_vertices()+}{}
%
\ccMethod{const Geom_traits& geom_traits();}
{Returns a const reference to the Apollonius graph traits object.}
\ccGlue
\ccMethod{int number_of_vertices() const;}
{Returns the number of finite vertices.}
\ccGlue
\ccMethod{Face_handle infinite_face() const;}
{a  face incident to the \ccc{infinite_vertex}.}
\ccGlue
\ccMethod{Vertex_handle
          infinite_vertex();}
{the \ccc{infinite_vertex}.}
\ccGlue
\ccMethod{Vertex_handle finite_vertex() const;}
{a vertex distinct from  the \ccc{infinite_vertex}.}




\ccHeading{Traversal of the Apollonius graph}


An Apollonius graph can be seen as a container of faces and vertices.
Therefore the Apollonius graph provides several iterators and circulators
that allow to traverse it (completely or partially).



\ccHeading{Face, Edge and Vertex Iterators}

\ccThree{Finite_vertices_iterator}{ag.finite_vertices_begin()+}{}

The following iterators allow respectively to visit 
finite faces,  finite edges and  finite vertices
of the Apollonius graph. These iterators are non-mutable,
bidirectional and their value types are respectively
\ccc{Face}, \ccc{Edge} and \ccc{Vertex}. 
They are all invalidated by any change in the Apollonius graph.

\ccMethod{Finite_vertices_iterator finite_vertices_begin() const;}{Starts at an arbitrary finite vertex}
\ccGlue
\ccMethod{Finite_vertices_iterator finite_vertices_end() const;}{Past-the-end iterator}

\ccMethod{Finite_edges_iterator finite_edges_begin() const;}{Starts at an arbitrary finite edge}
\ccGlue
\ccMethod{Finite_edges_iterator finite_edges_end() const;}{Past-the-end iterator}

\ccMethod{Finite_faces_iterator finite_faces_begin() const;}{Starts at an arbitrary finite face}
\ccGlue
\ccMethod{Finite_faces_iterator finite_faces_end()
const;}{Past-the-end iterator}

The following iterators allow respectively to visit all
(finite or infinite) faces, edges and vertices
of the Apollonius graph. These iterators are non-mutable, bidirectional
and their value types are respectively
\ccc{Face}, \ccc{Edge} and \ccc{Vertex}. 
They are all invalidated by any change in the Apollonius graph.


\ccMethod{All_vertices_iterator all_vertices_begin() const;}{Starts at an arbitrary  vertex}
\ccGlue
\ccMethod{All_vertices_iterator all_vertices_end() const;}{Past-the-end iterator}

\ccMethod{All_edges_iterator all_edges_begin() const;}{Starts at an arbitrary edge}
\ccGlue
\ccMethod{All_edges_iterator all_edges_end() const;}{Past-the-end iterator}

\ccMethod{All_faces_iterator all_faces_begin() const;}{Starts at an arbitrary face}
\ccGlue
\ccMethod{All_faces_iterator all_faces_end() const;}{Past-the-end iterator}


\ccThree{Vertex_circulator}{t.number_of_vertices()x}{}
\ccThreeToTwo



\ccHeading{Face, Edge and Vertex Circulators}

The Apollonius graph also provides circulators that allow to visit 
respectively all faces or edges incident to a given vertex
or all vertices adjacent to a given vertex.
These circulators are non-mutable and bidirectional.
 The operator \ccc{operator++} moves the circulator
counterclockwise around the vertex while
the \ccc{operator--} moves clockwise.
A face circulator is invalidated by any modification of the face pointed to.
Edge and a vertex circulators are invalidated by any modification
of one of the two faces incident to the edge pointed to.

\ccMethod{Face_circulator incident_faces(Vertex_handle v) const;}
{Starts at an arbitrary face incident
to \ccc{v}.}
\ccGlue
\ccMethod{Face_circulator incident_faces(Vertex_handle v, Face_handle f) const;}
{Starts at face \ccc{f}.
\ccPrecond Face \ccc{f} is incident to vertex \ccc{v}.}
\ccGlue
\ccMethod{Edge_circulator incident_edges(Vertex_handle v) const;}
{Starts at an arbitrary edge incident
to \ccc{v}.}
\ccGlue
\ccMethod{Edge_circulator incident_edges(Vertex_handle v, Face_handle f) const;}
{Starts at the first edge of \ccc{f} incident to 
\ccc{v}, in counterclockwise order around \ccc{v}.
\ccPrecond Face \ccc{f} is incident to vertex \ccc{v}.}
\ccGlue
\ccMethod{Vertex_circulator incident_vertices(Vertex_handle v) const;}
{Starts at an arbitrary  vertex incident
to \ccc{v}.}
\ccGlue
\ccMethod{Vertex_circulator incident_vertices(Vertex_handle v, Face_handle f) ;}
{Starts at the first vertex of \ccc{f} adjacent  to \ccc{v}
in  counterclockwise order around \ccc{v}.
\ccPrecond Face \ccc{f} is incident to vertex \ccc{v}.}




\ccHeading{Traversal of the Convex Hull}

Applied on the \ccc{infinite_vertex}
the above  functions  allow to visit the vertices on the convex hull and
the infinite edges and faces. Note that a counterclockwise
traversal of the vertices adjacent to the \ccc{infinite_vertex} is
a clockwise traversal of the convex hull.

\ccMethod{Face_circulator incident_faces(ag.infinite_vertex()) const;}{}
\ccGlue
\ccMethod{Face_circulator incident_faces(ag.infinite_vertex(), Face_handle f) const;}{}
\ccGlue
\ccMethod{Edge_circulator incident_edges(ag.infinite_vertex()) const;}{}
\ccGlue
\ccMethod{Edge_circulator incident_edges(ag.infinite_vertex(), Face_handle f);}{}
\ccGlue
\ccMethod{Vertex_circulator incident_vertices(ag.infinite_vertex()) ;} {}
\ccGlue
\ccMethod{Vertex_circulator incident_vertices(ag.infinite_vertex(), Face_handle f) ;}{}






\ccPredicates
The class \ccRefName\ provides methods to test
the finite or infinite character of any feature.
\ccThree{bool }{ag.is_infinite( Face_handle f, int i)x}{}

%
\ccMethod{bool
          is_infinite(Vertex_handle v) const;}
{\ccc{true}, iff \ccc{v} is the \ccc{infinite_vertex}.}
\ccGlue
\ccMethod{bool
          is_infinite(Face_handle f) const;}
{\ccc{true}, iff face \ccc{f} is infinite.}
\ccGlue
\ccMethod{bool is_infinite(Face_handle f, int i) const;}
{\ccc{true}, iff edge \ccc{(f,i)} is infinite.}
\ccGlue
\ccMethod{bool
          is_infinite(Edge e) const;}
{\ccc{true}, iff edge \ccc{e} is infinite.}
\ccGlue
\ccMethod{bool
          is_infinite(Edge_circulator ec) const;}
{\ccc{true}, iff edge \ccc{*ec} is infinite.}



\ccHeading{Insertion}
\ccThree{Vertex_handle }{ag.insert(Weighted_point_2 p)x}{}
%

\ccMethod{template< class Input_iterator >
void insert(Input_iterator first, Input_iterator beyond);}
{Inserts the weighted points in the range
[\ccc{first},\ccc{beyond}).
% MK:: Implement this method with copying and sorting
\ccPrecond{\ccc{Input_iterator} must be a model of
  \ccc{InputIterator} and its vlue type must be
  \ccc{Weighted_point_2}.}}
%
\ccGlue
\ccMethod{Vertex_handle  insert(Weighted_point_2 p);}{Inserts the
weighted point \ccc{p} in the Apollonius graph. If \ccc{p} is
not hidden then the vertex handle of \ccc{p} is returned, otherwise
the vertex handle returned is \ccc{NULL}.}
%
\ccGlue
\ccMethod{Vertex_handle insert(Weighted_point_2 p, Vertex_handle
vnear);}{Inserts \ccc{p} in the Apollonius graph using the weighted
point associated with \ccc{vnear} as
an estimate for the nearest neighbor of the center of \ccc{p}. If
\ccc{p} is not hidden then the vertex handle of \ccc{p} is returned,
otherwise the vertex handle returned is pointing to \ccc{NULL}.}


\ccHeading{Removal}
\ccThree{void }{ag.remove(Vertex_handle v)+}{}
%
\ccMethod{void  remove(Vertex_handle v);}{Removes the weighted point
associated to the vertex handle \ccc{v} from the Apollonius
graph.
\ccPrecond{\ccc{v} must correspond to a valid finite
vertex of the Apollonius graph.}}


\ccHeading{Nearest neighbor location}
\ccThree{Vertex_handle }{ag.nearest_neighbor(Point_2 p)+}{}
%
\ccMethod{Vertex_handle  nearest_neighbor(Point_2 p);}
{Finds the nearest neighbor of the point \ccc{p}. In other words it
finds the weighted point the Apollonius region of which contains
\ccc{p}. Ties are broken arbitrarily and one of the nearest neighbors
of \ccc{p} is returned.}
\ccGlue
\ccMethod{Vertex_handle  nearest_neighbor(Point_2 p,
Vertex_handle vnear);}{Finds the nearest neighbor of the point
\ccc{p} using the weighted point associated with \ccc{vnear} as an
estimate for the nearest neighbor of \ccc{p}. Ties are broken
arbitrarily and one of the nearest neighbors 
of \ccc{p} is returned.}


\ccHeading{Access to the dual}
\ccThree{Gt::Object_2}{ag.dual(Finite_edges_iterator ei)+}{}
%
\ccMethod{Point_2  dual(const Face_handle& f) const;}{Returns the
Apollonius vertex corresponding to the face handle \ccc{f}.
\ccPrecond{\ccc{f} must be a finite face.}}
\ccGlue
\ccMethod{Gt::Object_2 dual(Edge e);}{Returns the dual of
the edge \ccc{e}. The returned object can be assignable to one of the
following: \ccc{Gt::Line_2}, \ccc{Gt::Hyperbola_2}, \ccc{Gt::Ray_2},
\ccc{Gt::Hyperbola_ray_2}, \ccc{Gt::Segment_2},
\ccc{Gt::Hyperbola_segment_2}.
\ccPrecond{\ccc{e} must be a finite edge.}}
\ccGlue
\ccMethod{Gt::Object_2 dual(const Edge_circulator& ec) const;}
{Returns the dual of the edge to which the edge circulator \ccc{ec}
points to. The returned object can be assignable to one of the
following: \ccc{Gt::Line_2}, \ccc{Gt::Hyperbola_2}, \ccc{Gt::Ray_2},
\ccc{Gt::Hyperbola_ray_2}, \ccc{Gt::Segment_2},
\ccc{Gt::Hyperbola_segment_2}.
\ccPrecond{The edge corresponding to \ccc{ec} must be a finite edge.}}
\ccGlue
\ccMethod{Gt::Object_2 dual(const Finite_edges_iterator& ei) const;}
{Returns the dual of the edge to which the finite edge iterator
\ccc{ei} points to. The returned object can be assignable to one of the
following: \ccc{Gt::Line_2}, \ccc{Gt::Hyperbola_2}, \ccc{Gt::Ray_2},
\ccc{Gt::Hyperbola_ray_2}, \ccc{Gt::Segment_2},
\ccc{Gt::Hyperbola_segment_2}.}


\ccHeading{I/O}
\ccThree{Stream& }{}
{\ccc{Stream& operator<<(Stream&, Gt::Hyperbola_segment_2)+}}
%
\ccMethod{template < class Stream >
Stream& write_non_hidden_weighted_points(Stream& str);}{Writes
all non-hidden weighted points of the Apollonius graph to the stream
\ccc{str}.
\ccPrecond The following operator must be defined~:\\
\ccc{Stream& operator<<(Stream&, Weighted_point_2)}.}
\ccGlue
\ccMethod{template < class Stream >
Stream& write_all_weighted_points(Stream& str);}{Writes all
weighted points of the Apollonius graph, both hidden and non-hidden,
to the stream \ccc{str}. If the template argument \ccc{StoreHidden}
is set to \ccc{false} only the non-hidden weighted points are
written.
\ccPrecond The following operator must be defined~:\\
\ccc{Stream& operator<<(Stream&, Weighted_point_2)}.}
\ccGlue
\ccMethod{template < class Stream >
Stream& write_hidden_weighted_points(Stream& str);}{Writes all
hidden weighted points to the stream \ccc{str}. If the template
argument \ccc{StoreHidden} is set to false this method does nothing.
\ccPrecond The following operator must be defined~:\\
\ccc{Stream& operator<<(Stream&, Weighted_point_2)}.}
\ccGlue
\ccMethod{template < class Stream >
Stream& write_hidden_weighted_point_centers(Stream& str);}{Writes
the centers of all hidden weighted points to the stream \ccc{str}. If
the template argument \ccc{StoreHidden} is set to \ccc{false} this
method does nothing.
\ccPrecond The following operator must be defined~:\\
\ccc{Stream& operator<<(Stream&, Point_2)}.}
\ccGlue
\ccMethod{template < class Stream >
Stream& write_non_hidden_weighted_points(Stream& str);}{Writes all
non-hidden weighted points to the stream \ccc{str}.
\ccPrecond The following operator must be defined~:\\
\ccc{Stream& operator<<(Stream&, Weighted_point_2)}.}
\ccGlue
\ccMethod{template < class Stream >
Stream& write_non_hidden_weighted_point_centers(Stream& str);}
{Writes the centers of all non-hidden weighted points to the stream
\ccc{str}.
\ccPrecond The following operator must be defined~:\\
\ccc{Stream& operator<<(Stream&, Point_2)}.}
\ccGlue
\ccMethod{template< class Stream >
Stream& write_primal(Stream& str);}{Writes the Apollonius graph to
the stream \ccc{str}.
\ccPrecond{The following operators must be defined~:\\
\ccc{Stream& operator<<(Stream&, Gt::Ray_2)},\\
\ccc{Stream& operator<<(Stream&, Gt::Segment_2)},\\
\ccc{Stream& operator<<(Stream&, Gt::Hyperbola_segment_2)},\\
\ccc{Stream& operator<<(Stream&, Gt::Parabola_segment_2)}.}
}
\ccGlue
\ccMethod{template < class Stream > 
Stream& write_dual(Stream& str);}{Writes the dual of the
Apollonius graph, i.e., the Apollonius diagram, to the stream
\ccc{str}.
\ccPrecond{The following operators must be defined~:\\
\ccc{Stream& operator<<(Stream&, Gt::Line_2)},\\
\ccc{Stream& operator<<(Stream&, Gt::Hyperbola_2)},\\
\ccc{Stream& operator<<(Stream&, Gt::Ray_2)},\\
\ccc{Stream& operator<<(Stream&, Gt::Hyperbola_ray_2)},\\
\ccc{Stream& operator<<(Stream&, Gt::Segment_2)},\\
\ccc{Stream& operator<<(Stream&, Gt::Hyperbola_segment_2)}.}
}
\ccGlue
\ccMethod{template < class Stream > 
Stream& write_dual_weighted_points(Stream& str);}{Writes the
weighted points corresponding to the Apollonius vertices of the
Apollonius diagram.
\ccPrecond{The following operators must be defined~:\\
\ccc{Stream& operator<<(Stream&, Weighted_point_2)},\\
\ccc{Stream& operator<<(Stream&, Gt::Line_2)}.}
}


\ccHeading{Validity check}
\ccThree{bool}{ag.is_valid(bool verbose = false, int level = 1)+}{}
%
% MK:: change the output stream in the code
\ccMethod{bool is_valid(bool verbose = false, int level = 1);}
{Checks the validity of the Apollonius graph. If \ccc{verbose} is
\ccc{true} a short message is sent to \ccc{std::cerr}. If \ccc{level}
is 0, only the data structure is validated. If \ccc{level} is 1, then
both the data structure and the Apollonius graph are
validated. Negative values of \ccc{level} always return true, and
values greater then 1 are equivalent to \ccc{level} being 1.}

\ccHeading{Miscellaneous}
\ccThree{void}{ag.swap(& other)+}{}
%
\ccMethod{void clear();}{Clears all contents of the Apollonius graph.}
\ccGlue
% MK:: add this method make copy(...) protected and see doc of TDS
\ccMethod{void swap(Apollonius_graph_2<Gt,StoreHidden,Agds>
other);}{The Apollonius graphs
\ccc{other} and \ccVar\ are swapped. \ccVar.\ccc{swap(other)} should
be preferred to \ccVar\ccc{ = other} or to \ccVar\ccc{(other)} if
\ccc{other} is deleted afterwards.}




\ccSeeAlso
\ccc{ApolloniusGraphDataStructure_2}\\
\ccc{ApolloniusGraphTraits_2}\\
\ccc{ApolloniusGraphVertexBase_2}\\
\ccc{ApolloniusGraphFaceBase_2}\\
\ccc{CGAL::Apollonius_graph_euclidean_traits_2<R,W,Method_tag>}\\
\ccc{CGAL::Apollonius_graph_data_structure_2<Vb,Fb>}\\
\ccc{CGAL::Apollonius_graph_vertex_base_2<Gt,StoreHidden>}\\
\ccc{CGAL::Apollonius_graph_face_base_2<Gt>}

\end{ccRefClass}

% +------------------------------------------------------------------------+
%%RefPage: end of main body, begin of footer
% EOF
% +------------------------------------------------------------------------+
