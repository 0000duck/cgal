% // ============================================================================
%//
%// Copyright (c) 1999 The CGAL Consortium
%//
%// This software and related documentation is part of an INTERNAL release
%// of the Computational Geometry Algorithms Library (CGAL). It is not
%// intended for general use.
%//
%// ----------------------------------------------------------------------------
%//
%// release       :
%// release_date  :
%//
%// file          : /doc_tex/basic/Apollonius_graph_2/Apollonius_2.tex
%// revision      : $Revision$
%//
%// author(s)     : Menelaos Karavelas <mkaravel@cse.nd.edu>
%//
%// coordinator   : INRIA Sophia Antipolis (Mariette Yvinec <Mariette.Yvinec@sophia.inria.fr>)
%//
%//============================================================================

\chapter{2D Apollonius graph}
\label{chapter-apollonius2}

\vspace*{1cm}

\begin{ccTexOnly}
\minitoc
\newpage
\end{ccTexOnly}

\begin{figure}[htb]
\begin{ccTexOnly}
\begin{center}
\includegraphics[width=0.4\textwidth]%
{Apollonius_graph_2/apollonius_diagram.eps}
\hspace*{2cm} 
\includegraphics[width=0.4\textwidth]%
{Apollonius_graph_2/apollonius_graph.eps} 
\end{center}
\end{ccTexOnly}
\caption{The Apollonius diagram (left) and its dual the Apollonius
  graph (right).}
\label{fig:apollonius}
\begin{ccHtmlOnly}
<center>
<img border=0 src="./apollonius_diagram.gif" align=center
alt="The Apollonius diagram">
<img border=0 src="./apollonius_graph.gif" align=center
alt="The Apollonius graph (dual of the Apollonius diagram)">
</center>
\end{ccHtmlOnly}
\end{figure}

The 2D Apollonius graph class of \cgal\ is designed to compute the
dual of the {\em Apollonius diagram} or, as it is also known, the
{\em Additively weighted Voronoi diagram}. The algorithm that has been
implemented is dynamic, which means that we can perform insertions and
deletions on line. The corresponding \cgal\ class is called
\ccc{Apollonius_graph_2<ApolloniusGraphEuclideanTraits_2,StoreHidden,ApolloniusGraphDataStructure_2>}
and will be discussed in
more detail below. The interested reader may want to refer to the
paper by Karavelas and Yvinec \cite{} for the general idea as well as
the details of the algorithm implemented.


Before describing the details of the implementation we make a brief
introduction to the theory of Apollonius diagrams.
The Apollonius diagram is defined over a set of weighted points
$P_i=(c_i,w_i)$, $n=1,\ldots,n$, where $c_i$ is the point and $w_i$
the weight of $P_i$. It is a subdivision of the plane into connected
regions, called {\em cells}, associated with the weighted points (see
Fig. \ref{fig:apollonius}(left)). The cell of a 
weighted point $P_i$ is the locus of points on the plane that are
closer to $P_i$ than any other weighted point $P_j$, $j\ne i$.
It can easily be seen that it is a generalization of the Voronoi
diagram for points, which can actually be obtained if all the weights
$w_i$ are equal. If all weights $w_i$ are non-negative, the Apollonius
graph can be viewed as the Voronoi diagram of the set of circles
$\{P_1,\ldots, P_n\}$, where $c_i$ is the center of the circle $P_i$
and $w_i$ its radius. The distance $\delta(x, P_i)$ of a point $x$ in
the plane from a circle $P_i$ is then defined as:
\[      \delta(x,P_i)=\|x-c_i\|-w_i,\]
where $\|\cdot\|$ denotes the Euclidean norm.
Unlike the case of points, however, it is possible
that a weighted point $P_i$ might have an empty cell. If this is the
case we call the weighted point {\em hidden} (these are the black
circles in Fig. \ref{fig:apollonius}). From now on we will assume for
simplicity that all weights are positive and we will refer to the
weighted points as circles.


The Apollonius diagram is planar graph, and so is its dual, the
Apollonius graph. There are many ways to embed it on the plane and one
such way is shown in Fig. \ref{fig:apollonius}(right).
The Apollonius graph is uniquely defined once we have
the Apollonius diagram. If the circles are in {\em general position},
then the Apollonius graph is a graph with triangular faces away from
the convex hull of the set of circles (by triangular we mean that
every face has exactly three edges). Near the convex hull we may
have some spikes (i.e., vertices of degree 1). To unify our approach
and handling of the Apollonius graph we add to the set of (finite)
circles a fictious circle at infinity, which we call the
{\em point at infinity}. We can then connect all vertices of the outer
face of the Apollonius graph with the point at infinity which gives us
a graph with the property that all its faces are now
triangular. However, the Apollonius graph is not a triangulation for
two main reasons: we cannot always embed it on the plane with straight
lines segments that yield a triangulation and moreover we may have two
faces of the graph that have two edges in common, which is not allowed
in a triangulation. Both of these particularities appear when we
consider the Apollonius graph of the set of circles in
Fig. \ref{fig:apollonius}(right).


We would like to finish our brief introduction to the theory of
Apollonius graphs by discussing the concept of general position. We say
that a set of circles is in general position if no two triples of
circles have the same tritangent circle. This statement is rather
technical and it is best understood in the context of points. The
equivalent statement of points is that we have no two triples of
points that define the same circle, or equivalently that no four
circles are co-circular. The statement about general position made
above is a direct generalization of the (much simpler to understand)
statement about points. On the contrary, when we have circles in
degenerate position, the Apollonius graph has faces with more than
three edges on their boundary. We can get a triangulated version of
the graph by simply {\em triangulating} the corrsponding face in an
arbitrary way. In fact the algorithm that has been implemented in
\cgal\ has the property that it always returns a valid
{\em triangulated} version of the Apollonius graph. By valid we mean
that is contains the actual Apollonius graph (i.e., the actual dual of
the Apollonius diagram) and whenever there are faces with more than
three faces then they are triangulated. The way that they are
triangulated depends on the order of insertion and deletion of the
circles in the diagram.



One final point has to be made about hidden circles. First of all we
would like to be more precise about our definition of hidden circles:
we say that a circle is hidden if its cell has empty interior. This
definition allows us to guarantee that all non-hidden circles have
cells that are two-dimensional regions.
Geometrically the fact that a circle is hidden means that it is
contained in the closure of the disk of another circle (see again
Fig. \ref{fig:apollonius}). Hidden circles pose an additional
difficulty to our algorithm and software design. Since we allow
circles to be inserted and deleted at wish, it is possible that a
circle that was hidden at some point, may become non-hidden at another
point; for example this can happen if we delete the circle that
contains it. For this purpose we store hidden circles and have them
reappear when they become non-hidden. We will discuss this issue in
detail below. For the time being it suffices to say that the user has
the ability to control this behavior. More specifically it is possible
to discard of the circles that become hidden. This choice is totally
natural when for example we expect to do only insertions, since in
this case a circle that becomes hidden will never reapper. On the
other hand if deletions are expected as well, then we lose the ability
to have the hidden circles reapper.


\section{Representation}
\label{sec:apollonius2-representation}


