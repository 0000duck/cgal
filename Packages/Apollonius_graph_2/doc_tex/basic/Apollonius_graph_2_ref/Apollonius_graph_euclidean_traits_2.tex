% +------------------------------------------------------------------------+
% | Reference manual page: Apollonius_graph_euclidean_traits_2.tex
% +------------------------------------------------------------------------+
% | 12.04.2000   Author
% | Package: Package
% | 
%\RCSdef{\RCSRegulartriangulationtraitsRev}{$Revision$}
%\RCSdefDate{\RCSRegulartriangulationtraitsDate}{$Date$}
% |
%%RefPage: end of header, begin of main body
% +------------------------------------------------------------------------+


\begin{ccRefClass}{Apollonius_graph_euclidean_traits_2<K,Method_tag>}
%% add template arg's if necessary

%% \ccHtmlCrossLink{}     %% add further rules for cross referencing links
%% \ccHtmlIndexC[class]{} %% add further index entries
\ccDefinition
  
The class \ccRefName\ provides a model for the
\ccc{ApolloniusGraphTraits_2} concept.
This class is templated by a kernel class \ccc{K} and a
\ccc{Method_tag} that corresponds to how predicates are
evaluated. There are two predefined 
possible values for \ccc{Method_tag}, namely \ccc{CGAL::Sqrt_tag} and
\ccc{CGAL::No_sqrt_tag}. The first one must be used when the number
type used in the representation supports the exact evaluation of signs
of expressions involving all four basic operations and 
square roots, whereas the second one requires the exact evaluation of
signs of expressions involving only additions, subtractions and
multiplications. 

\ccInclude{CGAL/Apollonius_graph_euclidean_traits_2.h}

\ccIsModel
\ccc{ApolloniusGraphTraits_2}

\ccCreationVariable{traits}
\ccCreation
\ccThree{Apollonius_graph_euclidean_traits_2<K,Method_tag>}
{traits = other}{}
\ccThreeToTwo
%
\ccConstructor{ \ccRefName(); }{Default constructor.}
\ccGlue
\ccConstructor{
\ccRefName(Apollonius_graph_euclidean_traits_2<K,Method_tag> other);}
{Copy constructor.}
\ccGlue
\ccMethod{Apollonius_graph_euclidean_traits_2<K,Method_tag>
operator=(Apollonius_graph_euclidean_traits_2<K,Method_tag>
other);}{Assignment operator.}


\ccSeeAlso
\ccc{ApolloniusGraphTraits_2} \\
\ccc{CGAL::Apollonius_graph_2<Gt,StoreHidden,Agds>}


\end{ccRefClass}

% +------------------------------------------------------------------------+
%%RefPage: end of main body, begin of footer
% EOF
% +------------------------------------------------------------------------+

