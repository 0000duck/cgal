% +------------------------------------------------------------------------+
% | Reference manual page: ApolloniusGraphFaceBase_2.tex
% +------------------------------------------------------------------------+
% | 12.04.2000   Author
% | Package: Package
% | 
%\RCSdef{\RCSRegulartriangulationtraitsRev}{$Revision$}
%\RCSdefDate{\RCSRegulartriangulationtraitsDate}{$Date$}
% |
%%RefPage: end of header, begin of main body
% +------------------------------------------------------------------------+

\begin{ccRefConcept}{ApolloniusGraphFaceBase_2}

%% \ccHtmlCrossLink{}     %% add further rules for cross referencing links
%% \ccHtmlIndexC[concept]{} %% add further index entries
\ccDefinition
The \ccc{ApolloniusGraphFaceBase_2} refines the
\ccc{TriangulationFaceBase_2} concept by adding three face handles
and three integers. These triplets are intended to support an
in-place edge list. Since edges are not represented explicitly, but
rather as pairs of a \ccc{Face_handle} and an integer, the only way
to support this in-place edge list is by adding this functionality in
the face base of the Apollonius graph data structure.

\ccRefines
\ccc{TriangulationFaceBase_2}

\ccTypes
\ccTwo{ApolloniusGraphFaceBase_2::Vertex_handle+}{}
\ccNestedType{Geom_traits}{A type for the geometric traits of the
  Apollonius graph.}
%
\ccGlue
\ccNestedType{Apollonius_graph_data_structure_2}{A type for the
  Apollonius graph data structure, to which the face belongs to.}
%
\ccGlue
\ccNestedType{Vertex_handle}{A type for the vertex handle of the
  Apollonius graph data structure.}
%
\ccGlue
\ccNestedType{Face_handle}{A type for the face handle of the
  Apollonius graph data structure.}
%
\ccGlue
\ccNestedType{Edge}{A type for the edge of the Apollonius graph data
  structure.}


\ccCreation
\ccCreationVariable{fb}
\ccTwo{ApolloniusGraphFaceBase_2 fb()+}{}
%
\ccConstructor{ApolloniusGraphFaceBase_2();}{Default constructor.}
%\ccGlue
\ccConstructor{ApolloniusGraphFaceBase_2(Vertex_handle v0,
  Vertex_handle v1, Vertex_handle  v2);}
{Initializes the vertices of the face with \ccc{v0}, \ccc{v1} and
\ccc{v2}. Neighbors are initialized to \ccc{Face_handle(NULL)}.}
%\ccGlue
\ccConstructor{ApolloniusGraphFaceBase_2(Vertex_handle v0,
  Vertex_handle v1, Vertex_handle v2,
  Face_handle n0, Face_handle n1, Face_handle n2);}
	      {Initializes the vertices of the face with \ccc{v0},
		\ccc{v1} and \ccc{v2}. Initializes the neighbors of
		the face with \ccc{n0}, \ccc{n1} and \ccc{n2}.}


\ccHeading{Operations}
We only cite the additional operations with respect to those of the
\ccc{TriangulationFaceBase_2} concept.

%
\ccThree{Edge}{fb.set_previous(int i, Edge prev)+}{}
%
\ccMethod{bool is_in_list(int i);}
{Checks if the edge corresponding to the pair (\ccc{fb}, \ccc{i})
is currently in the in-place edge list. Returns \ccc{true} if
(\ccc{fb}, \ccc{i}) is in the list, \ccc{false} otherwise.
\ccPrecond{\ccc{i} must be either \ccc{0}, \ccc{1} or \ccc{2}.}}
\ccGlue
\ccMethod{void set_next(int i, Edge next);}
{Sets the edge \ccc{next} as the successor of
(\ccc{fb}, \ccc{i}) in the list.
\ccPrecond \ccc{next} must correspond to a valid edge, and \ccc{i}
must be either \ccc{0}, \ccc{1} or \ccc{2}.}
\ccGlue
\ccMethod{void set_previous(int i, Edge prev);}
{Sets the edge \ccc{prev} as the predecessor of
(\ccc{fb}, \ccc{i}) in the list.
\ccPrecond{\ccc{prev} must correspond to a valid edge, and \ccc{i}
must be either \ccc{0}, \ccc{1} or \ccc{2}.}}
\ccGlue
\ccMethod{Edge next(int i);}
{Returns the successor of (\ccc{fb}, \ccc{i}) in the list.
\ccPrecond{\ccc{i} must be either \ccc{0}, \ccc{1} or \ccc{2}.}}
\ccGlue
\ccMethod{Edge previous(int i);}
{Returns the predecessor of (\ccc{fb}, \ccc{i}) in the list.
\ccPrecond{\ccc{i} must be either \ccc{0}, \ccc{1} or \ccc{2}.}}


\ccHasModels
\ccc{CGAL::Apollonius_graph_face_base_2<Gt>}


\ccSeeAlso
\ccc{TriangulationFaceBase_2}\\
\ccc{ApolloniusGraphDataStructure_2}\\
\ccc{ApolloniusGraphTraits_2}\\
\ccc{CGAL::Apollonius_graph_face_base_2<Gt>}

\end{ccRefConcept}

% +------------------------------------------------------------------------+
%%RefPage: end of main body, begin of footer
% EOF
% +------------------------------------------------------------------------+
