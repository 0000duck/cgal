% +------------------------------------------------------------------------+
% | Reference manual page: Apollonius_graph_hierarchy_2.tex
% +------------------------------------------------------------------------+
% | 12.04.2000   Author
% | Package: Package
% | 
%\RCSdef{\RCSRegulartriangulationtraitsRev}{$Revision$}
%\RCSdefDate{\RCSRegulartriangulationtraitsDate}{$Date$}
% |
%%RefPage: end of header, begin of main body
% +------------------------------------------------------------------------+

\begin{ccRefClass}{Apollonius_graph_hierarchy_2<Gt,Agds>}
%% add template arg's if necessary

%% \ccHtmlCrossLink{}     %% add further rules for cross referencing links
%% \ccHtmlIndexC[class]{} %% add further index entries
\ccDefinition

We provide an alternative to the class
\ccc{Apollonius_graph_2<Gt,Agds>} for the dynamic
construction of the Apollonius graph. The \ccRefName\ class maintains
a hierarchy of Apollonius graphs. The bottom-most level of the
hierarchy contains the full Apollonius diagram. A site that
is in level $i$, is in level $i+1$ with probability $1/\alpha$
where $\alpha > 1$ is some constant. The difference between the
\ccc{Apollonius_graph_2<Gt,Agds>} class and the
\ccRefName\ is on how the nearest neighbor location is done. Given a
point $p$ the location is done as follows: at the top most level we
find the nearest neighbor of $p$ as in the
\ccc{Apollonius_graph_2<Gt,Agds>} class. At every subsequent level $i$
we use the nearest neighbor found at level $i+1$ to find the nearest
neighbor at level $i$. This is a variant of the corresponding
hierarchy for points found in \cite{d-iirdt-98}.
%
The class has two template parameters which have essentially the same
meaning as in the \ccc{Apollonius_graph_2<Gt,Agds>} class. The first
template parameter must be a model of the
\ccc{ApolloniusGraphTraits_2} concept.
The second template parameter must be a model of the
\ccc{ApolloniusGraphDataStructure_2} concept. However, the vertex base
class that is to be used in the Apollonius graph data structure must
be a model of the \ccc{ApolloniusGraphHierarchyVertexBase_2} concept.
The second template parameter defaults to
\ccc{Apollonius_graph_data_structure_2<
Apollonius_graph_hierarchy_vertex_base_2< 
Apollonius_graph_vertex_base_2<Gt,true> >,
Apollonius_graph_face_base_2<Gt> >}. 

The \ccRefName\ class derives publicly from the
\ccc{Apollonius_graph_2<Gt,Agds>} class. The interface is
the same with its base class. In the sequel only the methods
overridden are documented.


\ccInclude{CGAL/Apollonius_graph_hierarchy_2.h}



\ccInheritsFrom
\ccc{CGAL::Apollonius_graph_2<Gt,Agds>}


\ccTypes
\ccRefName\ does not introduce other types than those introduced by
its base class \ccc{Apollonius_graph_2<Gt,Agds>}.

\ccCreation
\ccCreationVariable{agh}
%
\ccThree{Apollonius_graph_hierarchy_2<Gt,Agds>}
{agh(Gt gt=Gt());}{}
\ccThreeToTwo
%
\ccConstructor{Apollonius_graph_hierarchy_2(Gt
gt=Gt())}{Creates an hierarchy of Apollonius graphs using \ccc{gt} as
geometric traits.}
\ccGlue
\ccConstructor{template< class Input_iterator >
Apollonius_graph_hierarchy_2<Gt,Agds>(Input_iterator
first, Input_iterator beyond, Gt gt=Gt())}
{Creates an Apollonius graph hierarchy using 
\ccc{gt} as geometric traits and inserts all sites in the
range [\ccc{first}, \ccc{beyond}).}
\ccGlue
\ccConstructor{Apollonius_graph_hierarchy_2<Gt,Agds>
(Apollonius_graph_hierarchy_2<Gt,Agds> other)}
{Copy constructor. All faces, vertices and inter-level pointers
are duplicated. After the construction, \ccVar\ and \ccc{other} refer
to two different Apollonius graph hierarchies: if 
\ccc{other} is modified, \ccVar\ is not.}
\ccGlue
\ccMethod{Apollonius_graph_hierarchy_2<Gt,Agds>
operator=(Apollonius_graph_hierarchy_2<Gt,Agds>
other);}{Assignment. All faces, vertices and inter-level pointers
are duplicated. After the construction, \ccVar\ and \ccc{other} refer
to two different Apollonius graph hierarchies: if 
\ccc{other} is modified, \ccVar\ is not.} 



\ccHeading{Insertion}
\ccThree{Vertex_handle }{agh.insert(Site_2 s)+}{}
%
\ccMethod{template< class Input_iterator >
unsigned int insert(Input_iterator first, Input_iterator beyond);}
{Inserts the sites in the range
[\ccc{first},\ccc{beyond}). The number of sites in the range
  [\ccc{first}, \ccc{beyond}) is returned.
% MK:: Implement this method with copying and sorting
\ccPrecond{\ccc{Input_iterator} must be a model of \ccc{InputIterator}
  and its value type must be \ccc{Site_2}.}}
%
\ccGlue
\ccMethod{Vertex_handle  insert(Site_2 s);}{Inserts the
site \ccc{s} in the Apollonius graph hierarchy. If \ccc{s}
is visible then the vertex handle of \ccc{s} is returned, otherwise 
\ccc{Vertex_handle(NULL)} is returned.}
%
\ccGlue
\ccMethod{Vertex_handle insert(Site_2 s, Vertex_handle
vnear);}{Inserts \ccc{s} in the Apollonius graph hierarchy using the
site associated with \ccc{vnear} as
an estimate for the nearest neighbor of the center of \ccc{s}.
If \ccc{s} is visible then the vertex handle of \ccc{s} is
returned, otherwise \ccc{Vertex_handle(NULL)} is returned.
A call to this method is equivalent to \ccc{agh.insert(s);} and it has
been added for the sake of conformity with the interface of the
\ccc{Apollonius_graph_2<Gt,Agds>} class.}

\ccHeading{Removal}
\ccThree{void }{agh.remove(Vertex_handle v)+}{}
%
\ccMethod{void  remove(Vertex_handle v);}{Removes the site
associated to the vertex handle \ccc{v} from the Apollonius
graph hierarchy.
\ccPrecond{\ccc{v} must correspond to a valid finite
vertex of the Apollonius graph hierarchy.}}


\ccHeading{Nearest neighbor location}
\ccThree{Vertex_handle }{agh.nearest_neighbor(Point p)+}{}
%
\ccMethod{Vertex_handle  nearest_neighbor(Point p);}
{Finds the nearest neighbor of the point \ccc{p}. In other words it
finds the site whose Apollonius cell contains \ccc{p}. Ties are broken
arbitrarily and one of the nearest neighbors of \ccc{p} is
returned. If there are no visible sites in the Apollonius diagram
\ccc{Vertex_handle(NULL)} is returned.}
\ccGlue
\ccMethod{Vertex_handle  nearest_neighbor(Point p,
Vertex_handle vnear);}{Finds the nearest neighbor of the point
\ccc{p}. If there are no visible sites in the Apollonius diagram
\ccc{Vertex_handle(NULL)} is returned.
A call to this method is equivalent to 
\ccc{agh.nearest_neighbor(p);} and it has been added for the sake of
conformity with the interface of the
\ccc{Apollonius_graph_2<Gt,Agds>} class.}




\ccHeading{Validity check}
\ccMethod{bool is_valid(bool verbose = false, int level = 1) const;}
{Checks the validity of the Apollonius graph hierarchy. If
\ccc{verbose} is \ccc{true} a short message is sent to
\ccc{std::cerr}. If \ccc{level} is 0, the data structure at all levels 
is validated, as well as the inter-level pointers. If \ccc{level} is
1, then the data structure at all levels is validated, the inter-level
pointers are validated and all levels of the Apollonius graph
hierarchy are also validated. Negative values of \ccc{level} always
return \ccc{true}, and values greater then 1 are equivalent to
\ccc{level} being 1.}

\ccHeading{Miscellaneous}
\ccThree{void}{agh.swap(other)+}{}
%
\ccMethod{void clear();}{Clears all contents of the Apollonius graph
hierarchy.}
\ccGlue
% MK:: add this method make copy(...) protected and see doc of TDS
\ccMethod{void swap(Apollonius_graph_hierarchy_2<Gt,Agds>
other);}
{The Apollonius graph hierarchies \ccc{other} and \ccVar\ are
swapped. \ccVar.\ccc{swap(other)} should be preferred to \ccVar\ccc{ =
other} or to \ccVar\ccc{(other)} if \ccc{other} is deleted afterwards.}


\ccSeeAlso
\ccc{ApolloniusGraphDataStructure_2}\\
\ccc{ApolloniusGraphTraits_2}\\
\ccc{ApolloniusGraphHierarchyVertexBase_2}\\
\ccc{CGAL::Apollonius_graph_2<Gt,Agds>}\\
\ccc{CGAL::Apollonius_graph_data_structure_2<Vb,Fb>}\\
\ccc{CGAL::Apollonius_graph_traits_2<K,Method_tag>}\\
\ccc{CGAL::Apollonius_graph_hierarchy_vertex_base_2<Agvb>}




\end{ccRefClass}

% +------------------------------------------------------------------------+
%%RefPage: end of main body, begin of footer
% EOF
% +------------------------------------------------------------------------+
