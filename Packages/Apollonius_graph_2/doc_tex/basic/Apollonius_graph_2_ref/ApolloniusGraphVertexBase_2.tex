% +------------------------------------------------------------------------+
% | Reference manual page: ApolloniusGraphVertexBase_2.tex
% +------------------------------------------------------------------------+
% | 12.04.2000   Author
% | Package: Package
% | 
%\RCSdef{\RCSRegulartriangulationtraitsRev}{$Revision$}
%\RCSdefDate{\RCSRegulartriangulationtraitsDate}{$Date$}
% |
%%RefPage: end of header, begin of main body
% +------------------------------------------------------------------------+

\begin{ccRefConcept}{ApolloniusGraphVertexBase_2}

%% \ccHtmlCrossLink{}     %% add further rules for cross referencing links
%% \ccHtmlIndexC[concept]{} %% add further index entries
\ccCreationVariable{vb}
\ccDefinition

The concept \ccRefName\ describes the requirements for the 
the vertex base class of the \ccc{ApolloniusGraphDataStructure_2}
concept. At the bottom layer, a vertex stores a weighted point and
provides access to one of its incident face through a \ccc{void *}
pointer. In addition to that it maintains a container of weighted
points. The container stores the hidden weighted points related to the 
vertex.

\ccTypes
\ccTwo{ApolloniusGraphVertexBase_2::Hidden_weighted_point_container+}{}
\ccNestedType{Geom_traits}{A type for the geometric traits that define 
the weighted point stored. \ccPrecond{The type \ccc{Geom_traits} must
define the type \ccc{Weighted_point}.}}
%
\ccGlue
\ccNestedType{Weighted_point}{A type for the weighted point stored.
\ccPrecond{This type must coincide with the type
\ccc{Geom_traits::Weighted_point}.}}
%
\ccGlue
\ccNestedType{Hidden_weighted_point_container}{A type for the list of
trivial weighted points stored.
\ccPrecond The container must be a model of \ccc{Back Insertion
Container}.}
%
\ccGlue
\ccNestedType{Hidden_weighted_point_iterator}{An iterator that
iterates over the hidden weighted points in the hidden weighted point
container.
\ccPrecond Must be a model of \ccc{Unidirectional Iterator}.}


\ccCreation
\ccCreationVariable{v}  %% choose variable name

\ccTwo{ApolloniusGraphVertexBase_2 v(Weighted_point p, void* f);++}{}
%
\ccConstructor{ApolloniusGraphVertexBase_2();}{Default constructor.}
\ccGlue
\ccConstructor{ApolloniusGraphVertexBase_2(Weighted_point p);}
{Constructs a vertex associated with the weighted point \ccc{p} and
embedded at the center of \ccc{p}.}
\ccGlue
\ccConstructor{ApolloniusGraphVertexBase_2(Weighted_point p,
                     void* f);}{Constructs a vertex associated with
the weighted point \ccc{p}, embedded at the center of \ccc{p},
and pointing to face \ccc{f}.}





\ccAccessFunctions
\ccThree{Hidden_weighted_point_iterator}
{v.numberof_hidden_weighted_points()+}{}
%
\ccMethod{Weighted_point point();}
{Returns  the weighted point.}
\ccGlue
\ccMethod{void* face();}{Returns a pointer to the incident face.}
\ccGlue
\ccMethod{unsigned int number_of_hidden_weighted_points();}
{Returns the number of hidden weighted points in the hidden
weighted point container.}
\ccGlue
\ccMethod{Hidden_weighted_point_iterator
hidden_weighted_points_begin();}
{Starts at an arbitrary hidden weighted point.}
\ccGlue
\ccMethod{Hidden_weighted_point_iterator
hidden_weighted_points_end();}
{Past-the-end iterator.}




\ccHeading{Setting and unsetting}
\ccThree{void}{v.add_hidden_weighted_point(Weighted_point p)+}{}
\ccMethod{void set_point(Weighed_point p);}
{Sets the weighted point.}
\ccGlue
\ccMethod{void set_face(void* f);}{Sets the incident face.}
\ccGlue
\ccMethod{void add_hidden_weighted_point(Weighted_point p);}
{Add a hidden weighted point to the container of hidden weighted
points.}
\ccGlue
\ccMethod{void clear_hidden_weighted_point_container();}
{Clears the container of hidden weighted points.}




\ccHeading{Checking}
\ccMethod{bool is_valid(bool verbose, int level) const;}
{Performs any required tests on a vertex.}


\ccHasModels

\ccc{CGAL::Apollonius_graph_vertex_base_2<Gt,StoreHidden>}.


\ccSeeAlso

\ccc{ApolloniusGraphDataStructure_2} \\
%\ccc{TriangulationDataStructure_2::Vertex} \\
\ccc{CGAL::Apollonius_graph_vertex_base_2<Gt,StoreHidden>}


\end{ccRefConcept}

% +------------------------------------------------------------------------+
%%RefPage: end of main body, begin of footer
% EOF
% +------------------------------------------------------------------------+
