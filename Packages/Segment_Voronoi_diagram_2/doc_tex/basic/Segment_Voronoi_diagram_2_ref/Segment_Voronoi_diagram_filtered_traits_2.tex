% +------------------------------------------------------------------------+
% | Reference manual page: Apollonius_graph_euclidean_traits_2.tex
% +------------------------------------------------------------------------+
% | 12.04.2000   Author
% | Package: Package
% | 
%\RCSdef{\RCSRegulartriangulationtraitsRev}{$Revision$}
%\RCSdefDate{\RCSRegulartriangulationtraitsDate}{$Date$}
% |
%%RefPage: end of header, begin of main body
% +------------------------------------------------------------------------+


\begin{ccRefClass}
{Segment_Voronoi_diagram_filtered_traits_2<CK,CM,EK,EM,FK,FM>}
%% add template arg's if necessary

%% \ccHtmlCrossLink{}     %% add further rules for cross referencing links
%% \ccHtmlIndexC[class]{} %% add further index entries

{\sc The type of \ccc{EM} must be consistent with whether
  intersections are supported or not. If intersections are supported
  we need \ccc{Field_tag}}

\ccDefinition
  
The class \ccRefName\ provides a model for the
\ccc{SegmentVoronoiDiagramTraits_2} concept.

The class \ccRefName\ uses the filtering technique \cite{bbp-iayed-01}
to achieve traits for the \ccc{Segment_Voronoi_diagram_2<Gt,SC,DS>}
class with efficient and exact predicates given an exact
kernel \ccc{EK} and a filtering kernel \ccc{FK}. The geometric
constructions associated provided by this class are equivalent
to those provided by the traits class
\ccc{Segment_Voronoi_diagram_traits_2<CK,CM>}, which means that they
may be inexact.

This class has six template parameters. The first, third and fifth
template parameters must be a models of the \ccc{Kernel} concept. The
second, fourth and sixth template parameters correspond to how
predicates are evaluated. There are two predefined possible values for
\ccc{Method_tag}, namely \ccc{CGAL::Sqrt_field_tag} and
\ccc{CGAL::Ring_tag}. The first one must be used when the number type
used in the representation supports the exact evaluation of signs of
expressions involving all four basic operations and square roots,
whereas the second one requires the exact evaluation of signs of
ring-type expressions, i.e., expressions involving only additions,
subtractions and multiplications.
%
The way the predicates are evaluated is discussed in
\cite{b-ecvdl-96,k-peasv-04}.

The default values for the template parameters are as follows:
\ccc{CM = CGAL::Sqrt_field_tag},
\ccc{EM = CGAL::Ring_tag},
\ccc{FK = CGAL::Simple_cartesian<CGAL::Interval_nt<false> >},
\ccc{FM = CM}. If the \textsc{Gmp} package is installed with \cgal,
the template parameter \ccc{EK} has the default value:
\ccc{EK = CGAL::Simple_cartesian<CGAL::Gmpq>}, otherwise its default
value is \ccc{EK = CGAL::Simple_cartesian<CGAL::MP_Float>}.


\ccInclude{CGAL/Segment_Voronoi_diagram_filtered_traits_2.h}

\ccIsModel
\ccc{SegmentVoronoiDiagramTraits_2}

\ccCreationVariable{traits}
\ccCreation
\ccThree{Segment_Voronoi_diagram_filtered_traits_2<CK,CK,CK,CK,CK,CK>+}{traits
  = other;+}{}
\ccThreeToTwo
%
\ccConstructor{Segment_Voronoi_diagram_filtered_traits_2<CK,CM,EK,EM,FK,FM>();}
{Default\\ constructor.}
\ccGlue
\ccConstructor{
Segment_Voronoi_diagram_filtered_traits_2
(Segment_Voronoi_diagram_filtered_traits_2<CK,CM,EK,EM,FK,FM> other);}
{Copy\\ constructor.}
\ccGlue
%\ccThree{Apollonius_graph_filtered_traits_2<CK,CK,CK,CK,CK,CK>+}{traits
%= other;+}{}
\ccMethod{Segment_Voronoi_diagram_filtered_traits_2<CK,CM,EK,EM,FK,FM>
  operator=(Segment_Voronoi_diagram_filtered_traits_2<CK,CM,EK,EM,FK,FM>
  other);}{Assignment\\ operator.}



\ccSeeAlso
\ccc{Kernel}\\
\ccc{SegmentVoronoiDiagramTraits_2} \\
\ccc{CGAL::Ring_tag}\\
\ccc{CGAL::Sqrt_field_tag}\\
\ccc{CGAL::Segment_Voronoi_diagram_2<Gt,SC,DS>}\\
\ccc{CGAL::Segment_Voronoi_diagram_traits_2<K,Method_tag>}


\end{ccRefClass}

% +------------------------------------------------------------------------+
%%RefPage: end of main body, begin of footer
% EOF
% +------------------------------------------------------------------------+

