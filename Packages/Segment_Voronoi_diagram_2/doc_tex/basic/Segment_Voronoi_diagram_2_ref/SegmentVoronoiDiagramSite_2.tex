%% Copyright (c) 2003,2004  INRIA Sophia-Antipolis (France) and
%% Notre Dame University (U.S.A.).  All rights reserved.
%%
%% This file is part of CGAL (www.cgal.org); you may redistribute it under
%% the terms of the Q Public License version 1.0.
%% See the file LICENSE.QPL distributed with CGAL.
%%
%% Licensees holding a valid commercial license may use this file in
%% accordance with the commercial license agreement provided with the software.
%%
%% This file is provided AS IS with NO WARRANTY OF ANY KIND, INCLUDING THE
%% WARRANTY OF DESIGN, MERCHANTABILITY AND FITNESS FOR A PARTICULAR PURPOSE.
%%
%% $Source$
%% $Revision$ $Date$
%% $Name$
%%
%% Author(s)     : Menelaos Karavelas <mkaravel@cse.nd.edu>



\begin{ccRefConcept}{SegmentVoronoiDiagramSite_2} 
%% add template arg's if necessary

%% \ccHtmlCrossLink{}     %% add further rules for cross referencing links
%% \ccHtmlIndexC[class]{} %% add further index entries

\ccDefinition
  
The concept \ccc{SegmentVoronoiDiagramSite_2} provides the
requirements for the sites of a segment Voronoi diagram.


\ccTwo{SegmentVoronoiDiagramSite_2::Segment_2+}{}
\ccTypes
\ccNestedType{Point_2}{The point type.}
\ccGlue
\ccNestedType{Segment_2}{The segment type.}
\ccGlue
\ccNestedType{FT}{The field number type.}
\ccGlue
\ccNestedType{RT}{The ring number type.}




\ccCreation
\ccTwo{SegmentVoronoiDiagramSite_2 s(Point_2 p)++}{}


\ccCreationVariable{s}  %% choose variable name
\ccConstructor{SegmentVoronoiDiagramSite2 s()}{Default constructor}
%
\ccGlue
\ccConstructor{SegmentVoronoiSite2(Point_2 p)}{Constructor from a
  point: the site represents the point \ccc{p}.}
%
\ccGlue
\ccConstructor{SegmentVoronoiSite2(Point_2 p1, Point_2
  p2)}{Constructor from two points: the site represents the (open)
  segment \ccc{(p1,p2)}.}
%
\ccGlue
\ccConstructor{SegmentVoronoiSite2(Point_2 p1, Point_2 p2, Point_2 q1,
  Point_2 q2)}{Constructor from four points: the site represents the
  point of intersection of the segments \ccc{(p1,p2)} and
  \ccc{(q1,q2)}.}
%
\ccGlue
\ccConstructor{SegmentVoronoiSite2(Point_2 p1, Point_2 p2, Point_2 q1,
  Point_2 q2, bool b)}{Constructor from four points and a boolean: the
  site represents a segment. If \ccc{b} is \ccc{true} the endpoints
  are \ccc{p1} and $p_\times$, otherwise $p_\times$ and
  \ccc{p2}. $p_\times$ is the point of intersection of the segments
  \ccc{(p1,p2)},\ccc{(q1,q2)}.}

%
\ccGlue
\ccConstructor{SegmentVoronoiSite2(Point_2 p1, Point_2 p2, Point_2 q1,
  Point_2 q2, Point_2 r1, Point_2 r2)}{Constructor from six
  points: the site represents the segment with endpoints the points of
  intersection of the pairs of segments \ccc{(p1,p2)},\ccc{(q1,q2)}
  and \ccc{(p1,p2)},\ccc{(r1,r2)}.}
%
\ccGlue
\ccConstructor{SegmentVoronoiDiagramSite_2(SegmentVoronoiDiagramSite_2
  other)}{Copy constructor.}

\ccPredicates
\ccThree{bool}{s.is_exact(unsigned int i)+}{}
\ccMethod{bool is_defined();}{Returns \ccc{true} if the site
  represents a valid point or segment.}
%
\ccGlue
\ccMethod{bool is_point();}{Returns \ccc{true} if the site represents
  a point.}
%
\ccGlue
\ccMethod{bool is_segment();}{Returns \ccc{true} if the site
  represents a segment.}
%
\ccGlue
\ccMethod{bool is_exact();}{Returns \ccc{true} if the site
  represents an input point or an input segment. Returns \ccc{false}
  if it represents a point of intersection of two segments, or if it
  represents a segment, at least one endpoint of which is a point of
  intersection of two segments.}
%
\ccGlue
\ccMethod{bool is_exact(unsigned int i);}{Returns \ccc{true} if the
  \ccc{i}-th endpoint of the site is an input point. Returns \ccc{false}
  if the \ccc{i}-th endpoint of the site is the intersection of two
  segments.
  \ccPrecond{\ccc{i} must be at most $1$, and \ccc{s.is_segment()}
    must be \ccc{true}.}}




\ccAccessFunctions
%\ccThree{Segment_2}{s.point(unsigned int i)+}{}
\ccThree{Segment_2}{s.segment()+}{}
\ccMethod{Point_2 point(unsigned int i) const;}{
  Returns the \ccc{i}-th point of the site's representation. The valid
  values for \ccc{i} are $0$ through the number of points in the
  constructor or the \ccc{set_point}/\ccc{set_segment} methods used to
  construct or define, respectively the site.
  \ccPrecond{\ccc{i} must be at most $5$. The site must be of the
    correct type in order to access the corresponding point.}}
%
\ccGlue
\ccMethod{Point_2 point() const;}{Returns the point represented by the
  site \ccc{s}.
  \ccPrecond{ \ccc{s.is_point()} must be \ccc{true}.}}
%
\ccGlue
\ccMethod{Segment_2 segment() const;}{Returns the segment represented
  by the site \ccc{s}.
  \ccPrecond{ \ccc{s.is_segment()} must be \ccc{true}.}}
%
\ccGlue
\ccMethod{Point_2 source() const;}{Returns the source endpoint of the
  segment. Note that this method can construct an inexact point if the
  number type used is inexact.
  \ccPrecond{\ccc{s.is_segment()} must be \ccc{true}.}}
%
\ccGlue
\ccMethod{Point_2 target() const;}{Returns the target endpoint of the
  segment. Note that this method can construct an inexact point if the
  number type used is inexact.
  \ccPrecond{\ccc{s.is_segment()} must be \ccc{true}.}}
%
%\ccThree{SegmentVoronoiDiagramSite_2}{s.ps}{}
\ccThree{SegmentVoronoiDiagramSite_2}{}{}
\ccGlue
\ccMethod{SegmentVoronoiDiagramSite_2 supporting_site();}
	 {Returns a segment site object representing the exact segment
	   that supports the segment represented by the site.
  \ccPrecond{\ccc{s.is_segment()} must be \ccc{true}.}
  \ccPostcond{The returned site is exact.}}
%
\ccGlue
\ccMethod{SegmentVoronoiDiagramSite_2 supporting_site(unsinged int i);}
	 {Returns a segment site object representing the \ccc{i}-th
	   exact segment that supports the point of intersection represented
	   by the site.
  \ccPrecond{\ccc{i} must be at most $1$, \ccc{s.is_point()} must be
    \ccc{true} and \ccc{s.is_exact()} must be \ccc{false}.}
  \ccPostcond{The returned site is exact.}}
%
\ccGlue
\ccMethod{SegmentVoronoiDiagramSite_2 crossing_site(unsinged int i);}
	 {Returns a segment site object representing the \ccc{i}-th exact
	   segment that supports the $i$-th endpoint of the site
	   which is not the supporting segment of the site.
	   \ccPrecond{\ccc{i} must be at most $1$,
	     \ccc{s.is_segment()} must be \ccc{true} and
	     \ccc{s.is_exact(i)} must be \ccc{false}.}
	   \ccPostcond{The returned site is exact.}}
%
\ccGlue
\ccMethod{SegmentVoronoiDiagramSite_2 source_site();}
	 {Returns a point site object representing the source point of
	   the site.
	 \ccPrecond{\ccc{s.is_segment()} must be \ccc{true}.}}
%
\ccGlue
\ccMethod{SegmentVoronoiDiagramSite_2 target_site();}
	 {Returns a point site object representing the target point of
	   the site.
	 \ccPrecond{\ccc{s.is_segment()} must be \ccc{true}.}}
%
\ccGlue
\ccMethod{SegmentVoronoiDiagramSite_2 opposite_site();}
	 {Returns a segment site object representing the segment site
	   with its endpoints reversed.
	 \ccPrecond{\ccc{s.is_segment()} must be \ccc{true}.}}




\ccHeading{Set methods}
\ccThree{void}{s.set_segment(Point_2 p1, Point_2 p2)+}{}
\ccMethod{void set_point(Point_2 p);}
	 {Sets the site to be an point site with \ccc{p} as the
	   point.}
%
\ccGlue
\ccMethod{void set_point(Point_2 p1, Point_2 p2, Point_2 q1, Point_2 q2);}
	 {Sets the site to be an point site with the point being the
	   intersection point of the segments \ccc{(p1,p2)} and
	   \ccc{(q1,q2)}.}
%
\ccGlue
\ccMethod{void set_segment(Point_2 p1, Point_2 p2);}
	 {Sets the site to be an exact segment site with \ccc{(p1,p2)}
	   as the segment.}



\ccHasModels
\ccc{CGAL::Segment_Voronoi_diagram_site_2<K>}


\ccSeeAlso
\ccc{SegmentVoronoiDiagramTraits_2}\\
\ccc{CGAL::Segment_Voronoi_diagram_site_2<K>}\\
\ccc{CGAL::Segment_Voronoi_diagram_traits_2<K,Method_tag>}\\
\ccc{CGAL::Segment_Voronoi_diagram_filtered_traits_2<CK,CM,EK,EM,FK,FM>}

\end{ccRefConcept}

% +------------------------------------------------------------------------+
%%RefPage: end of main body, begin of footer
% EOF
% +------------------------------------------------------------------------+

