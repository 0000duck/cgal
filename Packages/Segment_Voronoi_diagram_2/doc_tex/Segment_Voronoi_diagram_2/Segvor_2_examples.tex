%% Copyright (c) 2003,2004  INRIA Sophia-Antipolis (France) and
%% Notre Dame University (U.S.A.).  All rights reserved.
%%
%% This file is part of CGAL (www.cgal.org); you may redistribute it under
%% the terms of the Q Public License version 1.0.
%% See the file LICENSE.QPL distributed with CGAL.
%%
%% Licensees holding a valid commercial license may use this file in
%% accordance with the commercial license agreement provided with the software.
%%
%% This file is provided AS IS with NO WARRANTY OF ANY KIND, INCLUDING THE
%% WARRANTY OF DESIGN, MERCHANTABILITY AND FITNESS FOR A PARTICULAR PURPOSE.
%%
%% $Source$
%% $Revision$ $Date$
%% $Name$
%%
%% Author(s)     : Menelaos Karavelas <mkaravel@cse.nd.edu>

\section{Examples}
\label{sec:segvor2-examples}

\subsection{First example}

The following example shows to use the segment Voronoi diagram traits
in conjunction with the \ccc{Filtered_exact<CT,ET>} mechanism. In
addition it shows how to use a few of the iterators provided by the
\ccc{Segment_Voronoi_diagram_2} class in order to count a few
site-related quantities.

\ccIncludeExampleCode{Segment_Voronoi_diagram_2/svd-count-sites.C}

\subsection{Second example}

The following example shows how to use the segment Voronoi diagram
hierarchy along with the filtered traits class that supports
intersecting sites.

\ccIncludeExampleCode{Segment_Voronoi_diagram_2/svd-filtered-traits.C}

%\subsection{Second example}
%\ccIncludeExampleCode{Segment_Voronoi_diagram_2/example2.C}

%\subsection{Third example}
%\ccIncludeExampleCode{Segment_Voronoi_diagram_2/example3.C}
