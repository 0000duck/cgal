%% Copyright (c) 2003,2004  INRIA Sophia-Antipolis (France) and
%% Notre Dame University (U.S.A.).  All rights reserved.
%%
%% This file is part of CGAL (www.cgal.org); you may redistribute it under
%% the terms of the Q Public License version 1.0.
%% See the file LICENSE.QPL distributed with CGAL.
%%
%% Licensees holding a valid commercial license may use this file in
%% accordance with the commercial license agreement provided with the software.
%%
%% This file is provided AS IS with NO WARRANTY OF ANY KIND, INCLUDING THE
%% WARRANTY OF DESIGN, MERCHANTABILITY AND FITNESS FOR A PARTICULAR PURPOSE.
%%
%% $Source$
%% $Revision$ $Date$
%% $Name$
%%
%% Author(s)     : Menelaos Karavelas <mkaravel@cse.nd.edu>




\begin{ccRefClass}{Segment_Voronoi_diagram_2<Gt,DS>}
%% add template arg's if necessary

%% \ccHtmlCrossLink{}     %% add further rules for cross referencing links
%% \ccHtmlIndexC[class]{} %% add further index entries
\ccCreationVariable{traits}
\ccDefinition

The class \ccRefName\ represents the segment Voronoi diagram by its
dual. Currently it supports only insertions of sites.
%and deletions of sites.
It is templated by two template arguments \ccc{Gt}, which
must be a model of \ccc{SegmentVoronoiDiagramTraits_2}
and \ccc{DS},
which must be a model of \ccc{SegmentVoronoiDiagramDataStructure_2}.
The second template argument defaults to 
\ccc{CGAL::Triangulation_data_structure_2< 
CGAL::Segment_Voronoi_diagram_vertex_base_2<Gt>,
CGAL::Triangulation_face_base_2<Gt> >}.

\ccInclude{CGAL/Segment_Voronoi_diagram_2.h}

\ccIsModel
\ccc{DefaultConstructible}\\
\ccc{CopyConstructible}\\
\ccc{Assignable}

\ccTypes

\ccThree{typedef typename Point_container::iterator}{Point_container+}{}
%
\ccTypedef{typedef Gt Geom_traits;}{A type for the geometric traits.}
\ccGlue
\ccTypedef{typedef DS  Data_structure;}{A type for the underlying
data structure.}
\ccGlue
\ccTypedef{typedef typename DS::size_type size_type;}
{Size type (an unsigned integral type)}
\ccGlue
\ccTypedef{typedef typename Gt::Point_2 Point_2;}{A type for the
point defined in the geometric traits.}
\ccGlue
\ccTypedef{typedef typename Gt::Site_2  Site_2;}
{A type for the segment Voronoi diagram site, defined in the geometric
  traits.}
\ccGlue
\ccTypedef{typedef std::set<Point_2> Point_container;}{A type for the container of points.}
\ccGlue
\ccTypedef{typedef typename Point_container::iterator Point_handle;}
{A handle for points in the point container.}
%

% MK:: in the following copy text from TDS

The vertices and faces of the dual of the segment Voronoi diagram are
accessed through \ccc{handles}, 
\ccc{iterators} and \ccc{circulators}. 
The iterators and circulators
are all bidirectional and non-mutable.
The circulators and iterators are assignable to the 
corresponding handle types, and they are also convertible to the
corresponding handles.
The edges of the dual of the segment Voronoi diagram can also be
visited through iterators and circulators,
the edge circulators and iterators
are also bidirectional and non-mutable.
In the following, we call {\it infinite} any face or edge 
incident  to the infinite vertex and the infinite vertex itself.
Any other feature (face, edge or vertex) of the dual of the segment
Voronoi diagram is said to be {\it finite}.
Some iterators (the \ccc{All} iterators ) allow to visit finite or 
infinite features while the others (the \ccc{Finite} iterators) visit only
finite features. Circulators visit both infinite and finite features.

\ccThree{typedef typename DS::Vertex_circulator}{All_vertices_iterator;}{}
%
\ccTypedef{typedef typename DS::Edge Edge;}{The edge type.
The \ccc{Edge(f,i)} is the edge common to faces \ccc{f} and 
\ccc{f.neighbor(i)}. It is also the edge joining the vertices
\ccc{vertex(cw(i))} and \ccc{vertex(ccw(i))} of \ccc{f}.
\ccPrecond{\ccc{i} must be \ccc{0}, \ccc{1} or \ccc{2}.}}
%
\ccGlue
\ccTypedef{typedef typename DS::Vertex_handle Vertex_handle;}
{A type for a handle to a vertex.}
\ccGlue
\ccTypedef{typedef typename DS::Face_handle Face_handle;}{A type for a handle to a face.}
\ccGlue
\ccTypedef{typedef typename DS::Vertex_circulator Vertex_circulator;}
{A type for a circulator over vertices incident to a given vertex.}
\ccGlue
\ccTypedef{typedef typename DS::Face_circulator Face_circulator;}
{A type for a circulator over faces incident to a given vertex.}
\ccGlue
\ccTypedef{typedef typename DS::Edge_circulator Edge_circulator;}
{A type for a circulator over edges incident to a given vertex.}
\ccGlue
\ccTypedef{typedef typename DS::Vertex_iterator All_vertices_iterator;}
{A type for an iterator over all vertices.}
\ccGlue
\ccTypedef{typedef typename DS::Face_iterator All_faces_iterator;}
{A type for an iterator over all faces.}
\ccGlue
\ccTypedef{typedef typename DS::Edge_iterator All_edges_iterator;}
{A type for an iterator over all edges.}
\ccTwo{Segment_Voronoi_diagram_2<Gt,DS>::Finite_vertices_iterator+}{}
\ccGlue
\ccNestedType{Finite_vertices_iterator;}
{A type for an iterator over finite vertices.}
\ccGlue
\ccNestedType{Finite_faces_iterator;}
{A type for an iterator over finite faces.}
\ccGlue
\ccNestedType{Finite_edges_iterator;}
{A type for an iterator over finite edges.}


In addition to iterators and circulators for vertices and faces,
iterators for sites are provided. In particular there are iterators
for the set of input sites and the set of output sites. The set of
input sites is the set of sites inserted by the user using the
\ccc{insert} methods of this class. If a site is inserted multiple
times, every instance of this site will be reported. The set of output
sites is the set of sites in the segment Voronoi diagram. The value
type of these iterators is \ccc{Site_2}.

\ccTwo{Segment_Voronoi_diagram_2<Gt,DS>::Output_sites_iterator+}{}
%\ccGlue
\ccNestedType{Input_sites_iterator}
{A type for a bidirectional iterator over all input sites.}
\ccGlue
\ccNestedType{Output_sites_iterator}
{A type for a bidirectional iterator over all output sites (the sites
  in the diagram).}

\ccCreationVariable{svd}

\ccCreation
\ccThree{Segment_Voronoi_diagram_2<Gt,DS>}{svd(Gt gt=Gt());}{}
\ccThreeToTwo
%

In addition to the default and copy constructors the following
constructors are defined:

\ccConstructor{Segment_Voronoi_diagram_2(Gt gt=Gt());}{Creates the
  dual of the segment Voronoi diagram using \ccc{gt} as geometric
  traits.}
%
\ccConstructor{template< class Input_iterator >
Segment_Voronoi_diagram_2(Input_iterator first, Input_iterator beyond,
Gt gt=Gt());}
{Creates the dual of the segment Voronoi diagram using \ccc{gt} as
geometric traits and inserts all sites in the range
[\ccc{first}, \ccc{beyond}).
\ccPrecond{\ccc{Input_iterator} must be a model of
\ccc{InputIterator}. The value type of
\ccc{Input_iterator} must be either \ccc{Point_2} or \ccc{Site_2}.}}
%
%% \ccConstructor{Segment_Voronoi_diagram_2(const
%%   Segment_Voronoi_diagram_2<Gt,DS>& other)}
%% {Copy constructor. All faces and vertices are duplicated. After the
%%   construction, 
%%   \ccVar\ and \ccc{other} refer to two different segment Voronoi
%%   diagrams~: if \ccc{other} is modified, \ccVar\ is not.}
%% %
%% \ccMethod{Segment_Voronoi_diagram_2<Gt,DS>
%% operator=(const Segment_Voronoi_diagram_2<Gt,DS>& other);}
%% {Assignment. If \ccc{svd} and \ccc{other} are the same object
%%   nothing is done. Otherwise, all the vertices and faces are
%%   duplicated. After the assignment, \ccVar\ and \ccc{other} refer to
%%   different segment Voronoi diagrams~: if \ccc{other} is modified,
%%   \ccVar\ is not.}




\ccAccessFunctions
% MK:: in the following copy text from TDS
\ccThree{Point_container}{svd.number_of_output_sites()+}{}
%
\ccMethod{Geom_traits geom_traits();}
{Returns a reference to the segment Voronoi diagram traits object.}
\ccGlue
\ccMethod{int dimension();}
{Returns the dimension of the segment Voronoi diagram. The dimension
  is $-1$ if the diagram contains no sites, $0$ if the diagram
  contains one site, $1$ if it contains two sites and $2$ if it
  contains three or more sites.}
\ccGlue
\ccMethod{size_type number_of_vertices();}
{Returns the number of finite vertices of the dual of the segment
  Voronoi diagram.}
\ccGlue
\ccMethod{size_type number_of_faces();}
{Returns the number of faces (both finite and infinite) of the
  dual of the segment Voronoi diagram.}
\ccGlue
\ccMethod{size_type number_of_input_sites();}
{Return the number of input sites.}
\ccGlue
\ccMethod{size_type number_of_output_sites();}
{Return the number of output sites. This is equal to the number of
vertices in the dual of the segment Voronoi diagram.}
\ccGlue
\ccMethod{Face_handle infinite_face();}
{Returns a face incident to the \ccc{infinite_vertex}.}
\ccGlue
\ccMethod{Vertex_handle
          infinite_vertex();}
{Returns the \ccc{infinite_vertex}.}
\ccGlue
\ccMethod{Vertex_handle finite_vertex();}
{Returns a vertex distinct from  the \ccc{infinite_vertex}.
\ccPrecond{The number of sites in the segment Voronoi diagram must
  be at least one.}}
\ccGlue
\ccMethod{Data_structure data_structure();}{Returns a reference to the
  segment Voronoi diagram data structure object.}
\ccGlue
\ccMethod{Point_container point_container();}{Returns a reference to
  the point container object.}


\ccHeading{Traversal of the dual of the segment Voronoi diagram}


A segment Voronoi diagram can be seen as a container of faces and
vertices. Therefore the \ccRefName\ class provides several iterators
and circulators that allow to traverse it (completely or partially).



\ccHeading{Face, Edge and Vertex Iterators}

\ccThree{Finite_vertices_iterator}{svd.finite_vertices_begin()+}{}

The following iterators allow respectively to visit 
finite faces,  finite edges and  finite vertices of the dual of the
segment Voronoi diagram. These iterators are non-mutable,
bidirectional and their value types are respectively
\ccc{Face}, \ccc{Edge} and \ccc{Vertex}. 
They are all invalidated by any change in the segment Voronoi diagram.

\ccMethod{Finite_vertices_iterator finite_vertices_begin();}
{Starts at an arbitrary finite vertex.}
\ccGlue
\ccMethod{Finite_vertices_iterator finite_vertices_end();}
{Past-the-end iterator.}

\ccMethod{Finite_edges_iterator finite_edges_begin();}
{Starts at an arbitrary finite edge.}
\ccGlue
\ccMethod{Finite_edges_iterator finite_edges_end();}
{Past-the-end iterator.}

\ccMethod{Finite_faces_iterator finite_faces_begin();}
{Starts at an arbitrary finite face.}
\ccGlue
\ccMethod{Finite_faces_iterator finite_faces_end()
const;}{Past-the-end iterator.}

The following iterators allow respectively to visit all
(both finite and infinite) faces, edges and vertices
of the dual of the segment Voronoi diagram. These iterators are
non-mutable, bidirectional and their value types are respectively
\ccc{Face}, \ccc{Edge} and \ccc{Vertex}. 
They are all invalidated by any change in the segment Voronoi diagram.


\ccMethod{All_vertices_iterator all_vertices_begin();}
{Starts at an arbitrary  vertex.}
\ccGlue
\ccMethod{All_vertices_iterator all_vertices_end();}
{Past-the-end iterator.}

\ccMethod{All_edges_iterator all_edges_begin();}
{Starts at an arbitrary edge.}
\ccGlue
\ccMethod{All_edges_iterator all_edges_end();}
{Past-the-end iterator.}

\ccMethod{All_faces_iterator all_faces_begin();}
{Starts at an arbitrary face.}
\ccGlue
\ccMethod{All_faces_iterator all_faces_end();}
{Past-the-end iterator.}



\ccHeading{Site iterators}

The following iterators allow respectively to visit 
all sites. These iterators are non-mutable, bidirectional and their
value type is \ccc{Site_2}. They are all invalidated by any change in
the segment Voronoi diagram.


\ccMethod{Input_sites_iterator input_sites_begin();}
{Starts at an arbitrary input site.}
\ccGlue
\ccMethod{Input_sites_iterator input_sites_end();}
{Past-the-end iterator.}
\ccGlue
\ccMethod{Output_sites_iterator output_sites_begin();}
{Starts at an arbitrary output site.}
\ccGlue
\ccMethod{Output_sites_iterator output_sites_end();}
{Past-the-end iterator.}


\ccThree{Vertex_circulator}{t.number_of_vertices()x}{}
\ccThreeToTwo



\ccHeading{Face, Edge and Vertex Circulators}

The \ccRefName\ class also provides circulators that allow to visit
respectively all faces or edges incident to a given vertex or all
vertices adjacent to a given vertex. These circulators are non-mutable
and bidirectional. The operator \ccc{operator++} moves the circulator
counterclockwise around the vertex while the \ccc{operator--} moves
clockwise. A face circulator is invalidated by any modification of the
face pointed to. An edge circulator is invalidated by any modification
in one of the two faces incident to the edge pointed to. A vertex
circulator is invalidated by any modification in any of the faces
adjacent to the vertex pointed to.

\ccMethod{Face_circulator incident_faces(Vertex_handle v);}
{Starts at an arbitrary face incident
to \ccc{v}.}
\ccGlue
\ccMethod{Face_circulator incident_faces(Vertex_handle v, Face_handle f);}
{Starts at face \ccc{f}.
\ccPrecond Face \ccc{f} is incident to vertex \ccc{v}.}
\ccGlue
\ccMethod{Edge_circulator incident_edges(Vertex_handle v);}
{Starts at an arbitrary edge incident
to \ccc{v}.}
\ccGlue
\ccMethod{Edge_circulator incident_edges(Vertex_handle v, Face_handle f);}
{Starts at the first edge of \ccc{f} incident to 
\ccc{v}, in counterclockwise order around \ccc{v}.
\ccPrecond Face \ccc{f} is incident to vertex \ccc{v}.}
\ccGlue
\ccMethod{Vertex_circulator incident_vertices(Vertex_handle v);}
{Starts at an arbitrary  vertex incident
to \ccc{v}.}
\ccGlue
\ccMethod{Vertex_circulator incident_vertices(Vertex_handle v, Face_handle f);}
{Starts at the first vertex of \ccc{f} adjacent  to \ccc{v}
in  counterclockwise order around \ccc{v}.
\ccPrecond Face \ccc{f} is incident to vertex \ccc{v}.}



\ccHeading{Traversal of the Convex Hull}

Applied on the \ccc{infinite_vertex}
the above methods  allow to visit the vertices on the convex hull and
the infinite edges and faces. Note that a counterclockwise
traversal of the vertices adjacent to the \ccc{infinite_vertex} is
a clockwise traversal of the convex hull.

\ccMethod{Vertex_circulator incident_vertices(svd.infinite_vertex());}{}
\ccGlue
\ccMethod{Vertex_circulator incident_vertices(svd.infinite_vertex(),
  Face_handle f);}{}
\ccGlue
\ccMethod{Face_circulator incident_faces(svd.infinite_vertex());}{}
\ccGlue
\ccMethod{Face_circulator incident_faces(svd.infinite_vertex(),
  Face_handle f);}{}
\ccGlue
\ccMethod{Edge_circulator incident_edges(svd.infinite_vertex());}{}
\ccGlue
\ccMethod{Edge_circulator incident_edges(svd.infinite_vertex(),
  Face_handle f);}{}




\ccPredicates
The class \ccRefName\ provides methods to test
the finite or infinite character of any feature.
\ccThree{bool }{svd.is_infinite( Face_handle f, int i)x}{}

%
\ccMethod{bool
          is_infinite(Vertex_handle v) const;}
{\ccc{true}, iff \ccc{v} is the \ccc{infinite_vertex}.}
\ccGlue
\ccMethod{bool
          is_infinite(Face_handle f) const;}
{\ccc{true}, iff face \ccc{f} is infinite.}
\ccGlue
\ccMethod{bool is_infinite(Face_handle f, int i) const;}
{\ccc{true}, iff edge \ccc{(f,i)} is infinite.}
\ccGlue
\ccMethod{bool
          is_infinite(Edge e) const;}
{\ccc{true}, iff edge \ccc{e} is infinite.}
\ccGlue
\ccMethod{bool
          is_infinite(Edge_circulator ec) const;}
{\ccc{true}, iff edge \ccc{*ec} is infinite.}



\ccHeading{Insertion}
\ccThree{Vertex_handle }{svd.insert(Point_2 s)+}{}
%

\ccMethod{template< class Input_iterator >
size_type insert(Input_iterator first, Input_iterator beyond);}
{Inserts the sites in the range
[\ccc{first},\ccc{beyond}). The number of additional sites inserted in
  the diagram is returned. \ccc{Input_iterator} must be a model of
  \ccc{InputIterator} and its value type must be
  either \ccc{Point_2} or \ccc{Site_2}.}
%
\ccMethod{template< class Input_iterator >
size_type insert(Input_iterator first, Input_iterator beyond, Tag_false);}
{Same as the previous method. \ccc{Input_iterator} must be a model of
  \ccc{InputIterator} and its value type must be
  either \ccc{Point_2} or \ccc{Site_2}.}
%
\ccMethod{template< class Input_iterator >
size_type insert(Input_iterator first, Input_iterator beyond, Tag_true);}
{Inserts the sites in the range
[\ccc{first},\ccc{beyond}) after performing a random shuffle on
  them. The number of additional sites inserted in the diagram is
  returned. \ccc{Input_iterator} must be a model of
  \ccc{InputIterator} and its value type must be
  either \ccc{Point_2} or \ccc{Site_2}.}
%
\ccGlue
\ccMethod{Vertex_handle  insert(Point_2 p);}{Inserts the
point \ccc{p} in the segment Voronoi diagram. If \ccc{p} has already
been inserted, then the vertex handle of its already inserted copy is
returned. If \ccc{p} has not been inserted yet, the vertex handle of
\ccc{p} is returned.}
%
\ccGlue
\ccMethod{Vertex_handle insert(Point_2 p, Vertex_handle vnear);}
{Inserts \ccc{p} in the segment Voronoi diagram using the site
associated with \ccc{vnear} as an estimate for the nearest neighbor
of \ccc{p}. The vertex handle returned has the same semantics as
the vertex handle returned by the method
\ccc{Vertex_handle insert(Point_2 p)}.}
%
\ccGlue
\ccMethod{Vertex_handle  insert(Point_2 p1, Point_2 p2);}{Inserts the
  closed segment with endpoints \ccc{p1} and \ccc{p2} in the segment
  Voronoi diagram. If the segment has already been inserted in the
  diagram then the vertex handle of its already inserted copy is
  returned. If the segment does not intersect any segment in the
  existing diagram, the vertex handle corresponding to its
  corresponding open segment is returned. Finally, if the segment
  intersects other segments in the existing Voronoi diagram, the
  vertex handle to one of its open subsegments is returned.}
%
\ccGlue
\ccMethod{Vertex_handle insert(Point_2 p1, Point_2 p2, Vertex_handle
vnear);}{Inserts the segment whose endpoints are \ccc{p1} and \ccc{p2}
  in the segment Voronoi diagram using the site
  associated with \ccc{vnear} as an estimate for the nearest neighbor
  of \ccc{p1}. The vertex handle returned has the same semantics as the
  vertex handle returned by the method
  \ccc{Vertex_handle insert(Point_2 p1, Point_2 p2)}.}
%
\ccGlue
\ccMethod{Vertex_handle insert(Site_2 s);}{Inserts the site \ccc{s} in the
  segment Voronoi diagram. The vertex handle returned has the same
  semantics as the vertex handle returned by the methods
  \ccc{Vertex_handle insert(Point_2 p)} and \ccc{Vertex_handle
    insert(Point_2 p1, Point_2 p2)}, depending on whether \ccc{s}
  represents a point or a segment respectively.
\ccPrecond{\ccc{s.is_input()} must be \ccc{true}.}}
%
\ccGlue
\ccMethod{Vertex_handle insert(Site_2 s, Vertex_handle
vnear);}{Inserts \ccc{s} in the segment Voronoi diagram using the site
associated with \ccc{vnear} as an estimate for the nearest neighbor of
\ccc{s}, if \ccc{s} is a point, or the first endpoint of \ccc{s}, if
\ccc{s} is a segment. The vertex handle returned has the same
semantics as the vertex handle returned by the method
\ccc{Vertex_handle insert(Site_2 s)}.
\ccPrecond{\ccc{s.is_input()} must be \ccc{true}.}}


\ccHeading{Nearest neighbor location}
\ccThree{Vertex_handle }{svd.nearest_neighbor(Point_2 p)+}{}
%
\ccMethod{Vertex_handle  nearest_neighbor(Point_2 p);}
{Finds the nearest neighbor of the point \ccc{p}. In other words it
finds the site whose segment Voronoi diagram cell contains
\ccc{p}. Ties are broken arbitrarily and one of the nearest neighbors
of \ccc{p} is returned. If there are no sites in the
segment Voronoi diagram \ccc{Vertex_handle()} is returned.}
\ccGlue
\ccMethod{Vertex_handle  nearest_neighbor(Point_2 p,
Vertex_handle vnear);}{Finds the nearest neighbor of the point
\ccc{p} using the site associated with \ccc{vnear} as an
estimate for the nearest neighbor of \ccc{p}. Ties are broken
arbitrarily and one of the nearest neighbors of \ccc{p} is
returned. If there are no sites in the segment Voronoi diagram
\ccc{Vertex_handle()} is returned.}


\begin{ccTexOnly}
% to make the last I/O method appear in the next page
%\vspace*{20mm}
\end{ccTexOnly}

\ccHeading{I/O}
\ccThree{Stream& }{svd.draw_dual(Stream& str)+}{}
%
\ccMethod{template < class Stream > 
Stream& draw_dual(Stream& str);}{Draws the segment Voronoi
diagram to the stream \ccc{str}. The following operators must be
defined:\\
\ccc{Stream& operator<<(Stream&, Gt::Segment_2)}\\
\ccc{Stream& operator<<(Stream&, Gt::Ray_2)}\\
\ccc{Stream& operator<<(Stream&, Gt::Line_2)}
}
%
\ccGlue
\ccMethod{template < class Stream > 
Stream& draw_skeleton(Stream& str);}{Draws the segment Voronoi
diagram to the stream \ccc{str}, except the edges of the diagram
corresponding to a segment and its endpoints.
The following operators must be defined:\\
\ccc{Stream& operator<<(Stream&, Gt::Segment_2)}\\
\ccc{Stream& operator<<(Stream&, Gt::Ray_2)}\\
\ccc{Stream& operator<<(Stream&, Gt::Line_2)}
}
%
\ccGlue
\ccMethod{template< class Stream >
Stream& draw_dual_edge(Edge e, Stream& str);}{Draws the edge \ccc{e} of
  the segment Voronoi diagram to the stream \ccc{str}.
The following operators must be defined:\\
\ccc{Stream& operator<<(Stream&, Gt::Segment_2)}\\
\ccc{Stream& operator<<(Stream&, Gt::Ray_2)}\\
\ccc{Stream& operator<<(Stream&, Gt::Line_2)}
\ccPrecond{\ccc{e} must be a finite edge.}}
%
\ccGlue
\ccMethod{template< class Stream >
Stream& draw_dual_edge(Edge_circulator ec, Stream& str);}
{Draws the edge \ccc{*ec} of
  the segment Voronoi diagram to the stream \ccc{str}.
The following operators must be defined:\\
\ccc{Stream& operator<<(Stream&, Gt::Segment_2)}\\
\ccc{Stream& operator<<(Stream&, Gt::Ray_2)}\\
\ccc{Stream& operator<<(Stream&, Gt::Line_2)}
\ccPrecond{\ccc{*ec} must be a finite edge.}}
%
\ccGlue
\ccMethod{template< class Stream >
Stream& draw_dual_edge(All_edges_iterator eit, Stream& str);}
{Draws the edge \ccc{*eit} of
  the segment Voronoi diagram to the stream \ccc{str}.
The following operators must be defined:\\
\ccc{Stream& operator<<(Stream&, Gt::Segment_2)}\\
\ccc{Stream& operator<<(Stream&, Gt::Ray_2)}\\
\ccc{Stream& operator<<(Stream&, Gt::Line_2)}
\ccPrecond{\ccc{*eit} must be a finite edge.}}
%
\ccGlue
\ccMethod{template< class Stream >
Stream& draw_dual_edge(Finite_edges_iterator eit, Stream& str);}
{Draws the edge \ccc{*eit} of
  the segment Voronoi diagram to the stream \ccc{str}.
The following operators must be defined:\\
\ccc{Stream& operator<<(Stream&, Gt::Segment_2)}\\
\ccc{Stream& operator<<(Stream&, Gt::Ray_2)}\\
\ccc{Stream& operator<<(Stream&, Gt::Line_2)}}
%
\ccThree{std::ostream&}{svd.file_output(std::ostream& os)+}{}
\ccThreeToTwo
\ccGlue
\ccMethod{void file_output(std::ostream& os);}{Writes the current
  state of the segment Voronoi diagram to an output stream. In particular,
  all sites in the diagram are written to the stream (represented
  through appropriate input sites), as well as the underlying
  combinatorial data structure.}
\ccGlue
\ccMethod{void file_input(std::istream& is);}{Reads the state of the
  segment Voronoi diagram from an input stream.}
%
\ccGlue
\ccFunction{std::ostream& operator<<(std::ostream& os,
  Segment_Voronoi_diagram_2<Gt,DS> svd);}
{Writes the current state of the segment Voronoi diagram to an output stream.}
\ccGlue
\ccFunction{std::istream& operator>>(std::istream& is,
  Segment_Voronoi_diagram_2<Gt,DS> svd);}
{Reads the state of the segment Voronoi diagram from an input stream.}


\ccHeading{Validity check}
%\ccThree{bool}{svd.is_valid(bool verbose, int level)+}{}
%
% MK:: change the output stream in the code
\ccMethod{bool is_valid(bool verbose = false, int level = 1);}
{Checks the validity of the segment Voronoi diagram. If \ccc{verbose}
  is \ccc{true} a short message is sent to \ccc{std::cerr}. If
  \ccc{level} is 0, only the data structure is validated. If
  \ccc{level} is 1, then both the data structure and the segment
  Voronoi diagram are validated. Negative values of \ccc{level} always
  return true, and values greater than 1 are equivalent to \ccc{level}
  being 1.}

\ccHeading{Miscellaneous}
\ccThree{void}{svd.swap(& other)+}{}
%
\ccMethod{void clear();}{Clears all contents of the segment Voronoi
  diagram.}
\ccGlue
% MK:: add this method make copy(...) protected and see doc of TDS
\ccMethod{void swap(Segment_Voronoi_diagram_2<Gt,DS>
other);}{The segment Voronoi diagrams
\ccc{other} and \ccVar\ are swapped. \ccVar.\ccc{swap(other)} should
be preferred to \ccVar\ccc{ = other} or to \ccVar\ccc{(other)} if
\ccc{other} is deleted afterwards.}




\ccSeeAlso
\ccc{SegmentVoronoiDiagramTraits_2}\\
\ccc{SegmentVoronoiDiagramDataStructure_2}\\
\ccc{SegmentVoronoiDiagramVertexBase_2}\\
\ccc{TriangulationFaceBase_2}\\
\ccc{CGAL::Segment_Voronoi_diagram_hierarchy_2<Gt,STag,DS>}\\
\ccc{CGAL::Segment_Voronoi_diagram_traits_2<K,MTag>}\\
\ccc{CGAL::Segment_Voronoi_diagram_traits_without_intersections_2<K,MTag>}\\
\ccc{CGAL::Segment_Voronoi_diagram_filtered_traits_2<CK,CM,EK,EM,FK,FM>}\\
\ccc{CGAL::Segment_Voronoi_diagram_filtered_traits_without_intersections_2<CK,CM,EK,EM,FK,FM>}\\
\ccc{CGAL::Triangulation_data_structure_2<Vb,Fb>}\\
\ccc{CGAL::Segment_Voronoi_diagram_vertex_base_2<Gt,SSTag>}\\
\ccc{CGAL::Triangulation_face_base_2<Gt>}

\end{ccRefClass}

% +------------------------------------------------------------------------+
%%RefPage: end of main body, begin of footer
% EOF
% +------------------------------------------------------------------------+
