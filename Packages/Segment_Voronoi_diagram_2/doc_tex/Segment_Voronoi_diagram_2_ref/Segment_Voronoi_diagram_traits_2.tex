%% Copyright (c) 2003,2004  INRIA Sophia-Antipolis (France) and
%% Notre Dame University (U.S.A.).  All rights reserved.
%%
%% This file is part of CGAL (www.cgal.org); you may redistribute it under
%% the terms of the Q Public License version 1.0.
%% See the file LICENSE.QPL distributed with CGAL.
%%
%% Licensees holding a valid commercial license may use this file in
%% accordance with the commercial license agreement provided with the software.
%%
%% This file is provided AS IS with NO WARRANTY OF ANY KIND, INCLUDING THE
%% WARRANTY OF DESIGN, MERCHANTABILITY AND FITNESS FOR A PARTICULAR PURPOSE.
%%
%% $Source$
%% $Revision$ $Date$
%% $Name$
%%
%% Author(s)     : Menelaos Karavelas <mkaravel@cse.nd.edu>



\begin{ccRefClass}{Segment_Voronoi_diagram_traits_2<ITag,K,MTag>}
%% add template arg's if necessary

%% \ccHtmlCrossLink{}     %% add further rules for cross referencing links
%% \ccHtmlIndexC[class]{} %% add further index entries

\ccDefinition
  
The class \ccRefName\ provides a model for the
\ccc{SegmentVoronoiDiagramTraits_2} concept.
This class has two template parameters. The first template parameter
indicates whether intersecting segment are supported or not. The
possible values of this parameter are \ccc{CGAL::Tag_true} and
\ccc{CGAL::Tag_false}. \ccc{CGAL::Tag_true} indicates that
intersecting segments are supported. The second template parameter
must be a model of the \ccc{Kernel} concept. The third template
parameter corresponds to how predicates are evaluated. There are three
possible values for \ccc{MTag}, namely
\ccc{CGAL::Sqrt_field_tag}, \ccc{CGAL::Field_tag} and
\ccc{CGAL::Ring_tag}. The first one must be used when the number type
used in the representation supports the exact evaluation of signs of
expressions involving all four basic operations and square roots,
whereas the second one requires the exact evaluation of signs of
ring-type expressions, i.e., expressions involving only additions,
subtractions and multiplications. The default value for
\ccc{MTag} is \ccc{CGAL::Field_tag}. Note that if the parameter
\ccc{ITag} is set to \ccc{CGAL::Tag_true}, the value
\ccc{CGAL::Ring_tag} is no longer a valid possible value since
divisions must also be supported.
%
The way the predicates are evaluated is discussed in
\cite{b-ecvdl-96,k-peasv-04}.


\ccInclude{CGAL/Segment_Voronoi_diagram_traits_2.h}

\ccIsModel
\ccc{SegmentVoronoiDiagramTraits_2}

\ccTypes
The \ccRefName\ class introduces a few additional types with respect
to the \ccc{SegmentVoronoiDiagramTraits_2} concept. These are:

\ccTypedef{typedef K Kernel;}{A typedef for the template parameter
  \ccc{K}.}
\ccGlue
\ccTypedef{typedef MTag Method_tag;}{A typedef for the template
  parameter \ccc{MTag}.}

\ccCreationVariable{traits}
\ccCreation
\ccThree{Segment_Voronoi_diagram_traits_2<ITag,K,MTag>}
{traits = other}{}
\ccThreeToTwo
%
\ccConstructor{Segment_Voronoi_diagram_traits_2<ITag,K,MTag>();}
	      {Default constructor.}
\ccGlue
\ccConstructor{Segment_Voronoi_diagram_traits_2<ITag,K,MTag>(
  Segment_Voronoi_diagram_traits_2<ITag,K,MTag> other);}
{Copy constructor.}
\ccGlue
\ccMethod{Segment_Voronoi_diagram_traits_2<ITag,K,MTag>
operator=(Segment_Voronoi_diagram_traits_2<ITag,K,MTag>
other);}{Assignment operator.}


\ccSeeAlso
\ccc{Kernel}\\
\ccc{SegmentVoronoiDiagramTraits_2} \\
\ccc{CGAL::Ring_tag}\\
\ccc{CGAL::Field_tag}\\
\ccc{CGAL::Sqrt_field_tag}\\
\ccc{CGAL::Segment_Voronoi_diagram_2<Gt,PC,DS>}\\
\ccc{CGAL::Segment_Voronoi_diagram_filtered_traits_2<ITag,CK,CM,EK,EM,FK,FM>}

\end{ccRefClass}

% +------------------------------------------------------------------------+
%%RefPage: end of main body, begin of footer
% EOF
% +------------------------------------------------------------------------+

