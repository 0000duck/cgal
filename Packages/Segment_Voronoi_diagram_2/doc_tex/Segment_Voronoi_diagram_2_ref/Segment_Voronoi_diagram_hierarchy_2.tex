%% Copyright (c) 2003,2004  INRIA Sophia-Antipolis (France) and
%% Notre Dame University (U.S.A.).  All rights reserved.
%%
%% This file is part of CGAL (www.cgal.org); you may redistribute it under
%% the terms of the Q Public License version 1.0.
%% See the file LICENSE.QPL distributed with CGAL.
%%
%% Licensees holding a valid commercial license may use this file in
%% accordance with the commercial license agreement provided with the software.
%%
%% This file is provided AS IS with NO WARRANTY OF ANY KIND, INCLUDING THE
%% WARRANTY OF DESIGN, MERCHANTABILITY AND FITNESS FOR A PARTICULAR PURPOSE.
%%
%% $Source$
%% $Revision$ $Date$
%% $Name$
%%
%% Author(s)     : Menelaos Karavelas <mkaravel@cse.nd.edu>



\begin{ccRefClass}{Segment_Voronoi_diagram_hierarchy_2<Gt,STag,DS>}
%% add template arg's if necessary

%% \ccHtmlCrossLink{}     %% add further rules for cross referencing links
%% \ccHtmlIndexC[class]{} %% add further index entries
\ccDefinition

We provide an alternative to the class
\ccc{Segment_Voronoi_diagram_2<Gt,DS>} for the incremental
construction of the segment Voronoi diagram. The \ccRefName\ class
maintains a hierarchy of Voronoi diagrams. There are two possibilities
as to how this hierarchy is constructed.

In the first case the bottom-most level of the hierarchy contains the
full segment Voronoi diagram. The upper levels of the hierarchy
contain only points that are either point sites or endpoints of
segment sites in the bottom-most Voronoi diagram. 
A point that is in level $i$ (either as an individdual point or as the
endpoint of a segment), is inserted in level $i+1$ with probability
$1/\alpha$ where $\alpha>1$ is some constant.
In the second case the upper levels of the hierarchy contains not only
points but also segments. A site that is in level $i$, is in level
$i+1$ with probability $1/\beta$ where $\beta > 1$ is some constant.

The difference between the \ccc{Segment_Voronoi_diagram_2<Gt,DS>}
class and the \ccRefName\ class (both versions of it) is on how the
nearest neighbor location is done. Given a point $p$ the location is
done as follows: at the top most level we find the nearest neighbor of
$p$ as in the \ccc{Segment_Voronoi_diagram_2<Gt,DS>} class. At
every subsequent level $i$ we use the nearest neighbor found at level
$i+1$ to find the nearest neighbor at level $i$. This is a variant of
the corresponding hierarchy for points found in \cite{d-iirdt-98}. The
details are described in \cite{k-peasv-04}.

The class has four template parameters. The first, third and fourth
have essentially the same semantics as in the
\ccc{Segment_Voronoi_diagram_2<Gt,DS>} class. The
first template parameter must be a model of the
\ccc{SegmentVoronoiDiagramTraits_2} concept.
The third template parameter must be a model of the
\ccc{SimpleContainer} concept. The fourth template parameter must be a
model of the \ccc{SegmentVoronoiDiagramDataStructure_2}
concept. However, the vertex base class that is to be used in the
segment Voronoi diagram data structure must
be a model of the \ccc{SegmentVoronoiDiagramHierarchyVertexBase_2}
concept. The third template parameter defaults to
\ccc{CGAL::Simple_container_wrapper< std::list<typename Gt::Point_2>
>}, whereas the fourth template parameter defaults to
\ccc{Segment_Voronoi_diagram_data_structure_2<
Segment_Voronoi_diagram_hierarchy_vertex_base_2< 
Segment_Voronoi_diagram_vertex_base_2<Gt> >,
Triangulation_face_base_2<Gt> >}. The second template
parameter controls whether or not segments are added in the upper
levels of the hierarchy. It's possible values are \ccc{CGAL::Tag_true}
and \ccc{CGAL::Tag_false}. If it is set to \ccc{CGAL::Tag_true},
segments are also inserted in the upper levels of the hierarchy. The
value \ccc{CGAL::Tag_false} indicates that only points are to be
inserted in the upper levels of the hierarchy.


The \ccRefName\ class derives publicly from the
\ccc{Segment_Voronoi_diagram_2<Gt,DS>} class. The interface is
the same with its base class. In the sequel only additional types
and methods defined are documented.


\ccInclude{CGAL/Segment_Voronoi_diagram_hierarchy_2.h}



\ccInheritsFrom
\ccc{CGAL::Segment_Voronoi_diagram_2<Gt,DS>}


\ccTypes
\ccRefName\ introduces the following types in addition to those
introduced by its base class
\ccc{Segment_Voronoi_diagram_2<Gt,DS>}.

\ccThree{typedef CGAL::Segment_Voronoi_diagram_2<Gt,DS>}{Base+}{}
\ccThreeToTwo
\ccTypedef{typedef STag Segments_in_hierarchy_tag;}{A type for the
  \ccc{STag} template parameter.}
\ccGlue
\ccTypedef{typedef CGAL::Segment_Voronoi_diagram_2<Gt,DS> Base;}
	  {A type for the base class.}

\ccCreation
\ccCreationVariable{svdh}
%
\ccThree{Segment_Voronoi_diagram_hierarchy_2<Gt,STag,DS>}
{svdh(Gt gt=Gt());}{}
\ccThreeToTwo
\ccTwo{Segment_Voronoi_diagram_hierarchy}{}
%
\ccConstructor{Segment_Voronoi_diagram_hierarchy_2(Gt
gt=Gt())}{Creates a hierarchy of segment Voronoi diagrams using
  \ccc{gt} as geometric traits.}
%
\ccGlue
\ccConstructor{template< class Input_iterator >
Segment_Voronoi_diagram_hierarchy_2<Gt,STag,DS>(Input_iterator
first, Input_iterator beyond, Gt gt=Gt())}
{Creates a segment Voronoi diagram hierarchy using 
\ccc{gt} as geometric traits and inserts all sites in the
range [\ccc{first}, \ccc{beyond}). \ccc{Input_iterator} must be a
  model of \ccc{InputIterator}. The value type of \ccc{Input_iterator}
  must be either \ccc{Point_2} or \ccc{Site_2}.}
%
\ccGlue
\ccConstructor{Segment_Voronoi_diagram_hierarchy_2<Gt,STag,DS>
(Segment_Voronoi_diagram_hierarchy_2<Gt,STag,DS> other)}
{Copy constructor. All faces, vertices and inter-level pointers
are duplicated. After the construction, \ccVar\ and \ccc{other} refer
to two different hierarchies: if \ccc{other} is modified, \ccVar\ is
not.}
%
\ccThree{Segment_Voronoi_diagram_hierarchy<Gt,STag,PC,DS>++}{svdh = other;}{}
\ccThreeToTwo
\ccGlue
\ccMethod{Segment_Voronoi_diagram_hierarchy_2<Gt,STag,DS>
operator=(Segment_Voronoi_diagram_hierarchy_2<Gt,STag,DS>
other);}{Assignment. All faces, vertices and inter-level pointers
are duplicated. After the construction, \ccVar\ and \ccc{other} refer
to two different hierarchies: if \ccc{other} is modified, \ccVar\ is
not.}

\ccSeeAlso
\ccc{SegmentVoronoiDiagramDataStructure_2}\\
\ccc{SegmentVoronoiDiagramTraits_2}\\
\ccc{SegmentVoronoiDiagramHierarchyVertexBase_2}\\
\ccc{SimpleContainer}\\
\ccc{CGAL::Segment_Voronoi_diagram_2<Gt,DS>}\\
\ccc{CGAL::Segment_Voronoi_diagram_data_structure_2<Vb,Fb>}\\
\ccc{CGAL::Segment_Voronoi_diagram_traits_2<K,MTag>}\\
\ccc{CGAL::Segment_Voronoi_diagram_traits_without_intersections_2<K,MTag>}\\
\ccc{CGAL::Segment_Voronoi_diagram_filtered_traits_2<CK,CM,EK,EM,FK,FM>}\\
\ccc{CGAL::Segment_Voronoi_diagram_filtered_traits_without_intersections_2<CK,CM,EK,EM,FK,FM>}\\
\ccc{CGAL::Segment_Voronoi_diagram_hierarchy_vertex_base_2<Vbb>}

\end{ccRefClass}

% +------------------------------------------------------------------------+
%%RefPage: end of main body, begin of footer
% EOF
% +------------------------------------------------------------------------+
