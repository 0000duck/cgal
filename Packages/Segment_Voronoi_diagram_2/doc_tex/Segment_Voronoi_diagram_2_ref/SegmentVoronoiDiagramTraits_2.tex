%% Copyright (c) 2003,2004  INRIA Sophia-Antipolis (France) and
%% Notre Dame University (U.S.A.).  All rights reserved.
%%
%% This file is part of CGAL (www.cgal.org); you may redistribute it under
%% the terms of the Q Public License version 1.0.
%% See the file LICENSE.QPL distributed with CGAL.
%%
%% Licensees holding a valid commercial license may use this file in
%% accordance with the commercial license agreement provided with the software.
%%
%% This file is provided AS IS with NO WARRANTY OF ANY KIND, INCLUDING THE
%% WARRANTY OF DESIGN, MERCHANTABILITY AND FITNESS FOR A PARTICULAR PURPOSE.
%%
%% $Source$
%% $Revision$ $Date$
%% $Name$
%%
%% Author(s)     : Menelaos Karavelas <mkaravel@cse.nd.edu>




\begin{ccRefConcept}{SegmentVoronoiDiagramTraits_2}

%% \ccHtmlCrossLink{}     %% add further rules for cross referencing links
%% \ccHtmlIndexC[concept]{} %% add further index entries
\ccDefinition

The concept \ccc{SegmentVoronoiDiagramTraits_2} provides the traits
requirements for the \ccc{Segment_Voronoi_diagram_2} class. In particular,
it provides a type \ccc{Site_2}, which must be a model of the concept
\ccc{SegmentVoronoiDiagramSite_2}. It also provides
constructions for sites and several function object
types for the predicates.

{\sc Add also the stuff about intersections...}

\ccTypes
\ccTwo{SegmentVoronoiDiagramTraits_2::Segment_2+}{}
%
\ccNestedType{Point_2}{A type for a point.}
\ccGlue
\ccNestedType{Site_2}{A type for a site of the segment Voronoi
  diagram. Must be a model of the concept
  \ccc{SegmentVoronoiDiagramSite_2}.}
\ccGlue
\ccNestedType{Line_2}{A type for a line. Only required if the segment
  Voronoi diagram is inserted in a stream.}
\ccGlue
\ccNestedType{Ray_2}{A type for a ray. Only required if the segment
  Voronoi diagram is inserted in a stream.}
\ccGlue
\ccNestedType{Segment_2}{A type for a segment. Only required if 
  if the segment Voronoi diagram is inserted in a stream.}
\ccGlue
\ccNestedType{Object_2}{A type representing different types of objects
  in two dimensions, namely: \ccc{Point_2}, \ccc{Site_2},
  \ccc{Line_2}, \ccc{Ray_2} and \ccc{Segment_2}.}
\ccGlue
\ccNestedType{FT}{A type for the field number type of sites, points, etc..}
\ccGlue
\ccNestedType{RT}{A type for the ring number type of sites, points, etc.}
\ccGlue
\ccNestedType{Assign_2}
{Must provide \ccc{template <class T> bool operator() ( T& t,
    Object_2 o)} which assigns \ccc{o} to \ccc{t} if \ccc{o} was
  constructed from an object of type \ccc{T}. Returns 
  \ccc{true}, if the assignment was possible.}
\ccGlue
\ccNestedType{Construct_object_2}{Must provide \ccc{template <class T>
    Object_2 operator()( T t)} that constructs an object of type
  \ccc{Object_2} that contains \ccc{t} and returns it.}
\ccGlue
\ccNestedType{Construct_svd_vertex_2}{
  A constructor for a point of the segment Voronoi diagram equidistant
  from three sites. Must provide
  \ccc{Point_2 operator()(Site_2 s1, Site_2 s2, Site_2 s3)}, which
  constructs a point equidistant from the sites \ccc{s1}, \ccc{s2} and
  \ccc{s3}.
}
%
%\ccGlue
%\ccNestedType{Construct_Apollonius_site_2}
%{A constructor for
%a dual Apollonius site (a site whose center is a
%vertex of the Apollonius diagram and its weight is the common
%distance of its center from the three defining sites).
%Must provide \ccc{Site_2 operator()(Site_2 s1,
%Site_2 s2, Site_2 s3)}, which constructs a
%dual site whose center $c$ is equidistant from \ccc{s1}, \ccc{s2} and
%\ccc{s3}, and its weight is equal to the (signed) distance of $c$
%from \ccc{s1} (or \ccc{s2} or \ccc{s3}).\\
%Must also provide \ccc{Line_2 operator()(Site_2 s1, Site_2 s2)}, which
%constructs a line bitangent to \ccc{s1} and \ccc{s2}. This line is the
%dual site of \ccc{s1}, \ccc{s2} and the site at infinity; it can be
%viewed as a dual Apollonius site whose center is at infinity
%and its weight is infinite.
%}
%
\ccTwo{SegmentVoronoiDiagram_2}{}
%
%\ccGlue
%\ccNestedType{Compare_x_2}{A predicate object type. Must
%provide \ccc{Comparison_result operator()(Site_2 s1,
%Site_2 s2)}, which compares the $x$-coordinates of the centers of
%\ccc{s1} and \ccc{s2}.}
%
%\ccGlue
%\ccNestedType{Compare_y_2}{A predicate object type. Must
%provide \ccc{Comparison_result operator()(Site_2 s1,
%Site_2 s2)}, which compares the $y$-coordinates of the centers of
%\ccc{s1} and \ccc{s2}.}
%
\ccGlue
\ccNestedType{Orientation_2}{A predicate object type. Must
provide \ccc{Orientation operator()(Site_2 s1,
Site_2 s2, Site_2 s3)}, which performs the
usual orientation test for three points.
\ccc{s1}, \ccc{s2} and \ccc{s3}.
\ccPrecond{the sites \ccc{s1}, \ccc{s2} and \ccc{s3} must be points.}}
%
\ccGlue
\ccNestedType{Oriented_side_of_bisector_2}{A predicate object type.
Must provide \ccc{Oriented_side operator()(Site_2 s1,
Site_2 s2, Point_2 p)}, which returns
the oriented side of the bisector of \ccc{s1} and \ccc{s2} that
contains \ccc{p}. Returns \ccc{ON_POSITIVE_SIDE} if \ccc{p} lies in
the half-space of \ccc{s1} (i.e., \ccc{p} is closer to \ccc{s1} than
\ccc{s2}); returns \ccc{ON_NEGATIVE_SIDE} if \ccc{p} lies in the
half-space of \ccc{s2}; returns \ccc{ON_ORIENTED_BOUNDARY} if \ccc{p}
lies on the bisector of \ccc{s1} and \ccc{s2}.}
%
\ccGlue
\ccNestedType{Vertex_conflict_2}{A predicate object type.
Must provide \ccc{Sign operator()(Site_2 s1, Site_2
s2, Site_2 s3, Site_2 q)}, which
returns the sign of the distance of \ccc{q} from the Voronoi circle
of \ccc{s1}, \ccc{s2}, \ccc{s3}.
\ccPrecond{the Voronoi circle of \ccc{s1}, \ccc{s2},
\ccc{s3} must exist.}\\
Must also provide \ccc{Sign operator()(Site_2 s1,
Site_2 s2, Site_2 q)}, which returns the sign of the distance of
\ccc{q} from the bitangent line of \ccc{s1}, \ccc{s2} (a degenerate
Voronoi circle, with its center at infinity).}
%
\ccGlue
\ccNestedType{Finite_edge_interior_conflict_2}{A predicate object
type. Must provide \ccc{bool operator()(Site_2 s1,
Site_2 s2, Site_2 s3, Site_2 s4,
Site_2 q, Sign sgn)}. The sites \ccc{s1}, \ccc{s2},
\ccc{s3} and \ccc{s4} define a Voronoi edge that lies on the
bisector of \ccc{s1} and \ccc{s2} and has as endpoints the Voronoi
vertices defined by the triplets \ccc{s1}, \ccc{s2}, \ccc{s3} and
\ccc{s1}, \ccc{s4} and \ccc{s2}. The sign \ccc{sgn} is the common sign
of the distance of the site \ccc{q} from the Voronoi circle of the
triplets \ccc{s1}, \ccc{s2}, \ccc{s3} and \ccc{s1}, \ccc{s4} and
\ccc{s2}. In case that \ccc{sgn} is equal to \ccc{NEGATIVE}, the
predicate returns \ccc{true} if and only if the entire Voronoi edge is
in conflict with \ccc{q}. If \ccc{sgn} is equal to \ccc{POSITIVE} or
\ccc{ZERO}, the predicate returns \ccc{false} if and only if \ccc{q}
is not in conflict with the Voronoi edge.
\ccPrecond{the Voronoi vertices of \ccc{s1}, \ccc{s2},
\ccc{s3}, and \ccc{s1}, \ccc{s4}, \ccc{s2} must exist.}\\
%
Must also provide \ccc{bool operator()(Site_2 s1,
Site_2 s2, Site_2 s3, Site_2 q, Sign sgn)}. The
sites \ccc{s1}, \ccc{s2}, \ccc{s3} and the site at infinity
$s_\infty$ define a Voronoi edge that lies on the bisector of
\ccc{s1} and \ccc{s2} and has as endpoints the Voronoi vertices
$v_{123}$ and $v_{1\infty{2}}$ defined by the triplets \ccc{s1},
\ccc{s2}, \ccc{s3} and \ccc{s1}, $s_\infty$ and \ccc{s2} (the second
vertex is actually at infinity). The sign \ccc{sgn} is the common sign
of the distance of the site \ccc{q} from the two Voronoi circles
centered at the Voronoi vertices $v_{123}$ and $v_{1\infty{2}}$.
In case that \ccc{sgn} is \ccc{NEGATIVE}, the predicate
returns \ccc{true} if and only if the entire Voronoi edge is in
conflict with \ccc{q}. If \ccc{sgn} is \ccc{POSITIVE} or \ccc{ZERO},
the predicate returns \ccc{false} if and only if \ccc{q} is not in
conflict with the Voronoi edge.
\ccPrecond{the Voronoi vertex $v_{123}$ of \ccc{s1}, \ccc{s2},
\ccc{s3} must exist.}\\
%
Must finally provide \ccc{bool operator()(Site_2 s1,
Site_2 s2, Site_2 q, Sign sgn)}. The
sites \ccc{s1}, \ccc{s2} and the site at infinity
$s_\infty$ define a Voronoi edge that lies on the bisector of
$v_{12\infty}$ and $v_{1\infty{}2}$
\ccc{s1} and \ccc{s2} and has as endpoints the Voronoi vertices
defined by the triplets \ccc{s1}, \ccc{s2}, $s_\infty$ and \ccc{s1},
$s_\infty$ and \ccc{s2} (both vertices are actually at
infinity). The sign \ccc{sgn} denotes the common sign of the distance
of the site \ccc{q} from the Voronoi circles centered at
$v_{12\infty}$ and $v_{1\infty{}2}$.
If \ccc{sgn} is \ccc{NEGATIVE}, the predicate
returns \ccc{true} if and only if the entire Voronoi edge is in
conflict with \ccc{q}. If \ccc{POSITIVE} or \ccc{ZERO} is \ccc{false},
the predicate returns \ccc{false} if and only if \ccc{q} is not in
conflict with the Voronoi edge.}
%
\ccGlue
\ccNestedType{Infinite_edge_interior_conflict_2}{A predicate
object type. Must provide \ccc{bool operator()(Site_2 s1,
Site_2 s2, Site_2 s3, Site_2 q, Sign sgn)}. The
sites $s_\infty$, \ccc{s1}, \ccc{s2} and \ccc{s3} define a
Voronoi edge that lies on the bisector of $s_\infty$ and \ccc{s1}
and has as endpoints the Voronoi vertices $v_{\infty{}12}$ and
$v_{\infty{}31}$ defined by the triplets
$s_\infty$, \ccc{s1}, \ccc{s2} and $s_\infty$, \ccc{s3} and
\ccc{s1}. The sign \ccc{sgn} is the common sign of the distances of
\ccc{q} from the Voronoi circles centered at the the vertices
$v_{\infty{}12}$ and $v_{\infty{}31}$. If \ccc{sgn} is \ccc{NEGATIVE},
the predicate returns \ccc{true} if and only if the entire Voronoi
edge is in conflict with \ccc{q}. If \ccc{sgn} is \ccc{POSITIVE} or
\ccc{ZERO}, the predicate returns \ccc{false} if and only if \ccc{q}
is not in conflict with the Voronoi edge.}
%
%\ccGlue
%\ccNestedType{Is_degenerate_edge_2}{A predicate object type.
%Must provide \ccc{bool operator()(Site_2 s1, Site_2
%s2, Site_2 s3, Site_2 s4)}. It returns \ccc{true} if
%the Apollonius edge defined by \ccc{s1}, \ccc{s2}, \ccc{s3} and
%\ccc{s4} is degenerate, \ccc{false} otherwise. An Apollonius edge is
%called degenerate if its two endpoints coincide.
%\ccPrecond{the Apollonius vertices of \ccc{s1}, \ccc{s2},
%\ccc{s3}, and \ccc{s1}, \ccc{s4}, \ccc{s2} must exist.}}
%
 

\ccCreation
\ccCreationVariable{gt}  %% choose variable name

\ccThree{SegmentVoronoiDiagramTraits_2}{ traits = other}{}
\ccThreeToTwo
%
\ccConstructor{ SegmentVoronoiDiagramTraits_2(); }{Default constructor.}
\ccGlue
\ccConstructor{SegmentVoronoiDiagramTraits_2(SegmentVoronoiDiagramTraits_2
other);}{Copy constructor.}
\ccGlue
\ccMethod{SegmentVoronoiDiagramTraits_2
  operator=(SegmentVoronoiDiagramTraits_2 other);}
{Assignment operator.}



\ccHeading{Access to predicate objects}
%
\ccThree{Infinite_edge_interior_conflict_2}
{gt.infinite_edge_interior_conflict_2_object();}{}
%
%\ccMethod{Compare_x_2 compare_x_2_object();}{}
%\ccGlue
%\ccMethod{Compare_y_2 compare_y_2_object();}{}
%\ccGlue
\ccMethod{Orientation_2 orientation_2_object();}{}
\ccGlue
\ccMethod{Oriented_side_of_bisector_2
oriented_side_of_bisector_test_2_object();}{}
\ccGlue
\ccMethod{Vertex_conflict_2 vertex_conflict_2_object();}{}
\ccGlue
\ccMethod{Finite_edge_interior_conflict_2
	finite_edge_interior_conflict_2_object();}{}
\ccGlue
\ccMethod{Infinite_edge_interior_conflict_2
	infinite_edge_interior_conflict_2_object();}{}
%\ccGlue
%\ccMethod{Is_degenerate_edge_2 is_degenerate_edge_2_object();}{}




\ccHeading{Access to contructor objects}
%
\ccMethod{Construct_object_2
construct_object_2_object();}{} 
%
\ccGlue
\ccMethod{Construct_svd_vertex_2
construct_svd_vertex_2_object();}{} 
%
%\ccGlue
%\ccMethod{Construct_Apollonius_site_2
%construct_Apollonius_site_2_object();}{}




\ccHeading{Access to other objects}
%
\ccMethod{Assign_2 assign_2_object();}{} 


\ccHasModels

\ccc{CGAL::Segment_Voronoi_diagram_traits_2<K,Method_tag>}\\
\ccc{CGAL::Segment_Voronoi_diagram_filtered_traits_2<CK,CM,EK,EM,FK,FM>}


\ccSeeAlso
\ccc{SegmentVoronoiDiagramSite_2}\\
\ccc{CGAL::Segment_Voronoi_diagram_2<Gt,SC,DS>}\\
\ccc{CGAL::Segment_Voronoi_diagram_traits_2<K,Method_tag>}\\
\ccc{CGAL::Segment_Voronoi_diagram_filtered_traits_2<CK,CM,EK,EM,FK,FM>}
\end{ccRefConcept}

% +------------------------------------------------------------------------+
%%RefPage: end of main body, begin of footer
% EOF
% +------------------------------------------------------------------------+

