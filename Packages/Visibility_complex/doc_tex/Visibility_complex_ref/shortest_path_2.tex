% +------------------------------------------------------------------------+
% | Reference manual page: shortest_path_2.tex
% +------------------------------------------------------------------------+
% | Last revision: 09.07.2002   
% | Author : Pierre Angelier <Pierre.Angelier@ens.fr>
% | Maintainer: Laurent Rineau <Laurent.Rineau@ens.fr>
% | Package: Visibility_complex
% +------------------------------------------------------------------------+

\begin{ccRefFunction}{shortest_path_2}
\label{pageShortest_path_2Ref}

% ------------------------------------------------------------------------------
\ccDefinition

Function that computes the shortest path between two points $p,q$ in a planar
scene (see the basic definitions paragraph in the introduction for what we
understand by a scene). The scene is given as two pairs of iterators describing
the set of disks and constraints of the scene.

The output iterator returned by the function is an iterator on the vertices of
the shortest path. Here a vertex is a vertex of the visibility complex of the
scene plus the two points $p,q$. Recall that vertices correspond to free
bitangents. 

Currently only scene containing points, segments or polygons are supported. In
other words circles are not.
% ------------------------------------------------------------------------------

% ------------------------------------------------------------------------------
\ccInclude{CEP/Visibility_complex/shortest_path_2.h}
% ------------------------------------------------------------------------------

% ------------------------------------------------------------------------------
\ccFunction{
template < class DiskIterator , class ConstraintIterator , 
	   class OutputIterator , class Traits >
OutputIterator shortest_path_2(DiskIterator first , 
			       DiskIterator last  ,
			       ConstraintIterator firstc,
			       ConstraintIterator lastc ,
			       Traits::Point_2 p , Traits::Point_2 q ,
			       OutputIterator result , 
			       Traits t);
}
{
computes the shortest path between \ccc{p} and \ccc{q} in the scene containing
the disks in the range [\ccc{first},\ccc{last}) and the constraints in the range
[\ccc{firstc},\ccc{lastc}). The past-the-end iterator for the list of vertices
of the shortest path is returned. Here vertex refers to the visibility complex
of the input scene plus the two points \ccc{p} and \ccc{q}.
}

\ccFunction{
template < class DiskIterator , class OutputIterator , class Traits >
OutputIterator shortest_path_2(DiskIterator first ,
			       DiskIterator last  ,
			       Traits::Point_2 p , Traits::Point_2 q ,
                               OutputIterator result ,
                               Traits t);
}
{
Same as the previous function but with an empty list of constraints.
}
% ------------------------------------------------------------------------------

% ------------------------------------------------------------------------------
\ccHeading{Requirements}
\begin{enumerate}
    \item \ccc{Traits} is a model for the concept ShortestPathTraits.
    \ccIndexMainItem[c]{ShortestPathTraits}
    \item \ccc{OutputIterator::value_type} should be the same as
    \ccc{ConstraintIterator::value_type} and the same as
    \ccc{Visibility_complex_2<Traits>::Vertex}.
    \item \ccc{DiskIterator::value_type} should be \ccc{Traits::Disk}.
\end{enumerate}
% ------------------------------------------------------------------------------

% ------------------------------------------------------------------------------
\ccSeeAlso

\ccRefIdfierPage{CGAL::Shortest_path_point_traits<R,E>} \\
\ccRefIdfierPage{CGAL::Shortest_path_segment_traits<R,E>} \\
\ccRefIdfierPage{CGAL::Shortest_path_polygon_traits<R,E>}
% ------------------------------------------------------------------------------

% ------------------------------------------------------------------------------
\ccImplementation

The visibility graph of the input scene plus the two points is computed in time
$O(k + n \log n)$ and $O(k)$ space where $n$ and $k$ are respectively the size
of the input and output.
A Dijsktra algorithm is then applied to get the shortest path. We have
implemented the Dijsktra algorithm using a \ccc{stl::set} thus resulting in a
$O(k \log n)$ time complexity.
% ------------------------------------------------------------------------------

% ------------------------------------------------------------------------------
\ccExample

The following program read a list of pairwise disjoint segments from the file
\texttt{segments} and the outputs the bitangent segments of the shortest path to
standard output in a one-per-line fashion.

\ccIncludeExampleCode{Visibility_complex/shortest_path.C}
% ------------------------------------------------------------------------------

% +----------------------------------------------------------------------------+
\ccTagDefaults
\end{ccRefFunction}
\ccRefPageEnd
% +----------------------------------------------------------------------------+
