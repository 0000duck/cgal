% +------------------------------------------------------------------------+
% | Reference manual page: VisibilityComplexAntichain.tex
% +------------------------------------------------------------------------+
% | Last revision: 09.07.2002   
% | Author : Pierre Angelier <Pierre.Angelier@ens.fr>
% | Maintainer: Laurent Rineau <Laurent.Rineau@ens.fr>
% | Package: Visibility_complex
% +------------------------------------------------------------------------+

\ccRefPageBegin

%%RefPage: end of header, begin of main body
% +----------------------------------------------------------------------------+

\begin{ccRefConcept}{VisibilityComplexAntichain}
\label{pageVisibilityComplexAntichainRef}

% +----------------------------------------------------------------------------+
\ccDefinition
  
The concept of a \ccRefName\ defines the sweep structure used to sweep the
visibility complex. It is a planar st-graph: a node corresponds to an edge of
visibility complex and an edge corresponds to a face of the visibility complex.
The antichain maintains a list of potential candidates which are ready to be
swept. 
% +----------------------------------------------------------------------------+

% +----------------------------------------------------------------------------+
\ccTypes
\ccTwo{VisibilityComplexAntichain:: Vertex_iterator }{}

\ccNestedType{Gt}{the geoemtric traits class. Model of a \ccc{VisibilityComplexTraits}. }
\ccGlue
\ccNestedType{Disk}{the disk type. Same as \ccc{VisibilityComplexTraits::Disk}. }
\ccGlue
\ccNestedType{Vertex}{vertex, model for \ccc{VisibilityComplexVertex}\  
		      concept.}
\ccGlue
\ccNestedType{Edge}{edge, model for \ccc{VisibilityComplexEdge}\  concept.}
\ccGlue
\ccNestedType{Face}{face, model for \ccc{VisibilityComplexFace}\  concept.}
\ccNestedType{Vertex_handle}{handle to vertex.}
\ccGlue
\ccNestedType{Edge_handle}{handle to edge.}
\ccGlue
\ccNestedType{Face_handle}{handle to face.}
\ccNestedType{Vertex_iterator}{iterator over all vertices of the antichain.}
\ccGlue
\ccNestedType{Edge_iterator}{iterator over all edges of the antichain.}
\ccGlue
\ccNestedType{Face_iterator}{iterator over all faces of the antichain.}
\ccGlue
\ccNestedType{Minimals_iterator}{iterator over all the minimal vertices of the antichain. }
% +----------------------------------------------------------------------------+

% +----------------------------------------------------------------------------+
\ccCreation
\ccCreationVariable{a}

\ccConstructor{VisibilityComplex_Antichain();}{default constructor.}
\ccConstructor{VisibilityComplex_Antichain(InputIterator first, InputIterator last,
					   ConstraintIt firstc, ConstraintIt lastc);}
{creates the antichain of the disks in the range $[$\ccc{first,last}$)$ and
constraints in the range $[$\ccc{firstc,lastc}$)$ corresponding to the filter
$I(0)$ of bitangents with positive slope. The value type of \ccc{InputIterator}
must be \ccc{Disk} and the value type of \ccc{ConstraintIt} must be
\ccc{Vertex}. }
% +----------------------------------------------------------------------------+

% +----------------------------------------------------------------------------+
\ccOperations

\ccThree{Vertex_iterator}{a.vertices_cw_begin(); }{}
\ccThreeToTwo
\ccTagFullDeclarations
\ccMethod{Vertex_iterator vertices_begin();}{iterator over all the sinks of the
					    faces of the antichain. }
\ccGlue
\ccMethod{Vertex_iterator vertices_end();}{}
\ccGlue
\ccMethod{Vertex_iterator vertices_cw_begin();}{iterator over all the sources of the
						faces of the antichain.}
\ccGlue
\ccMethod{Vertex_iterator vertices_cw_end();}{}
\ccGlue
\ccMethod{Edge_iterator edges_begin();}{iterator over all the edges of the
					antichain.}
\ccGlue
\ccMethod{Edge_iterator edges_end();}{}
\ccGlue
\ccMethod{Face_iterator faces_begin();}{iterator over all the faces of the
					antichain.}
\ccGlue
\ccMethod{Face_iterator faces_end();}{}

\ccThree{bool }{v.is_right_minimal( Vertex_handle v);}{}
\ccMethod{void set_linear_space(bool b);} {If \ccc{b} is true then the antichain
will destroy the elements which are no longer necessary after each sweep of a
vertex. This method is used to sweep the visibility complex while keeping the
storage linear (i.e. in the size of the input). The default value is true.}
\ccGlue
\ccMethod{bool linear_space(); } {returns true if the storage is linear. }
\ccMethod{void set_straight(bool b); } {if \ccc{b} is false (resp. true) then
the antichain will sweep the visibility complex using a topological (straight)
sweep. The default value is false. }
\ccGlue
\ccMethod{bool is_straight(); } { returns true if the antichain is straight. }
\ccMethod{bool is_right_minimal(Vertex_handle v);}{returns true if the vertex
						   $v$ is right minimal.}
\ccGlue
\ccMethod{bool is_left_minimal(Vertex_handle v);} {returns true if the vertex
						   $v$ is left minimal.}
\ccGlue
\ccMethod{bool is_minimal(Vertex_handle v);}{returns true if the vertex $v$ is
					     minimal.}

\ccThree{Minimals_iterator }{v.sweep( Vertex_handle v); }{}
\ccMethod{Minimals_iterator minimals_begin();} {}
\ccGlue
\ccMethod{Minimals_iterator minimals_end();}{ iterator over all the minimal 
					      bitangents in the filter associated 
					      with the antichain. If the antichain 
					      is straight then the vertices are 
					      ordered by increasing slope. }
\ccGlue
\ccMethod{void sweep_all_minimals();}{sweeps all the minimals vertices of the
				      antichain.}
\ccGlue
\ccMethod{void sweep(Vertex_handle v);}{sweeps the vertex $v$. precondition:
					$v$ is minimal. If the antichain is
					not straight, this operation is 
					amortized constant otherwise it takes 
					$O(\log n)$ time. The vertex is removed
					from the list of minimals. }
\ccTagDefaults
% +----------------------------------------------------------------------------+

% ------------------------------------------------------------------------------
\ccHasModels
\ccRefIdfierPage{CGAL::Visibility_complex_antichain<Gt,It>}
% ------------------------------------------------------------------------------

% +----------------------------------------------------------------------------+
\ccSeeAlso

\ccRefConceptPage{VisibilityComplex} \\
\ccRefConceptPage{VisibilityComplexTraits}  \\
\ccRefConceptPage{VisibilityComplexVertex}\\
\ccRefConceptPage{VisibilityComplexEdge}\\
\ccRefConceptPage{VisibilityComplexFace}
% +----------------------------------------------------------------------------+

\end{ccRefConcept}

% +----------------------------------------------------------------------------+
%%RefPage: end of main body, begin of footer
\ccRefPageEnd
% EOF
% +----------------------------------------------------------------------------+
