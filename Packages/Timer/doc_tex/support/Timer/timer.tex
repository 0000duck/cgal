% +------------------------------------------------------------------------+
% | CGAL Reference Manual:  timer.tex
% +------------------------------------------------------------------------+

\ccParDims

\chapter{Timer}
\label{chapterTimer}
\ccChapterAuthor{Lutz Kettner, Matthias B\"asken}

\section{Introduction}

This chapter decribes classes for measuring user process time and real time.

\begin{ccRefClass}{Timer}
\ccSection{A Timer for User-Process Time}

\ccDefinition
\ccCreationVariable{t}

A timer \ccVar\ of type \ccRefName\ is an object with a state. It is
either {\em running\/} or it is {\em stopped}. The state is controlled
with \ccStyle{t.start()} and \ccStyle{t.stop()} . The timer counts the
time elapsed since its creation or last reset. It counts only the time
where it is in the running state. The time information is given in seconds.
The timer counts also the number of intervals it was running, i.e. it 
counts the number of calls of the \ccc{start()} member function since the 
last reset. If the reset occures while the timer is running it counts as the
first interval.

\ccInclude{CGAL/Timer.h}

\ccCreation

\ccSetThreeColumns{Timer&}{t.is_running();}{}
\ccPropagateThreeToTwoColumns
%\ccSetTwoColumns{Timer t( Timer r);}{}

\ccConstructor{Timer();}{state is {\em stopped.}}

\ccOperations

\ccMethod{void     start();}{\ccPrecond state is {\em stopped}.}
\ccGlue
\ccMethod{void     stop ();}{\ccPrecond state is {\em running}.}
\ccGlue
\ccMethod{void     reset();}{reset timer to zero. The state is unaffected.}
\ccGlue
\ccMethod{bool     is_running();}{\ccc{true} if the current state is running.}

\ccMethod{double   time();}{user process time in seconds.}
\ccGlue
\ccMethod{int      intervals();}{number of start/stop-intervals since
                                 the last reset.} 
\ccGlue
\ccMethod{double   precision();}{smallest possible time step in seconds.}
\ccGlue
\ccMethod{double   max();}{maximal representable time in seconds.}


\ccImplementation

The timer class is based in the C function \ccStyle{std::clock(\ldots)}
which measures the user and system times of the current process and
its subprocesses.  The time granularity is reasonable fine with
currently 10 ms on IRIX and Solaris. However the counter wraps around
after only about 36 minutes. On Solaris machines the man page states
that the timer could fail. In that case an error message is printed
and the program aborted.

\end{ccRefClass}

% +----------------------------------------------------------+
\begin{ccRefClass}{Real_timer}
\ccSection{A Timer Measuring Real-Time}

\ccDefinition
\ccCreationVariable{t}

A timer \ccVar\ of type \ccRefName\ is an object with a state. It is
either {\em running\/} or it is {\em stopped}. The state is controlled
with \ccStyle{t.start()} and \ccStyle{t.stop()} . The timer counts the
time elapsed since its creation or last reset. It counts only the time
where it is in the running state. The time information is given in seconds.
The timer counts also the number of intervals it was running, i.e. it 
counts the number of calls of the \ccc{start()} member function since the 
last reset. If the reset occures while the timer is running it counts as the
first interval.

\ccInclude{CGAL/Real_timer.h}

\ccCreation

\ccSetThreeColumns{Real_timer&}{t.is_running();}{}
\ccPropagateThreeToTwoColumns
%\ccSetTwoColumns{Timer t( Timer r);}{}

\ccConstructor{Real_timer();}{state is {\em stopped.}}

\ccOperations

\ccMethod{void     start();}{\ccPrecond state is {\em stopped}.}
\ccGlue
\ccMethod{void     stop ();}{\ccPrecond state is {\em running}.}
\ccGlue
\ccMethod{void     reset();}{reset timer to zero. The state is unaffected.}
\ccGlue
\ccMethod{bool     is_running();}{\ccc{true} if the current state is running.}

\ccMethod{double   time();}{real time in seconds.}
\ccGlue
\ccMethod{int      intervals();}{number of start/stop-intervals since
                                 the last reset.} 
\ccGlue
\ccMethod{double   precision();}{smallest possible time step in seconds; returns -1 if not available.}
\ccGlue
\ccMethod{double   max();}{maximal representable time in seconds.}

\ccImplementation

The timer class is based in the C function \ccStyle{gettimeofday(\ldots)} on UNIX systems and
\ccStyle{_ftime(\ldots)} on MS Visual C++.
\end{ccRefClass}

% +--------------------------------------------------------+
% restore default column and paragraph layout
\ccParDims

% EOF
