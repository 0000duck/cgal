% +------------------------------------------------------------------------+
% | CGAL User Manual:  Timer.tex
% +------------------------------------------------------------------------+

\ccParDims

%\chapter{Timers}
%\label{chapterTimer}
%\ccChapterAuthor{Lutz Kettner, Matthias B\"asken}

\section{Timers}

CGAL provides classes for measuring the user process time and the real time.
The class \ccc{CGAL::Timer} is the version for the user process time and
the class \ccc{CGAL::Real\_timer} is the version for the real time.

Instantiations of both classes are objects with a state. The state is
either {\em running\/} or it is {\em stopped}. The state of an object
\ccc{t} is controlled
with \ccStyle{t.start()} and \ccStyle{t.stop()} . The timer counts the
time elapsed since its creation or last reset. It counts only the time
where it is in the running state. The time information is given in seconds.
The timer counts also the number of intervals it was running, i.e. it 
counts the number of calls of the \ccc{start()} member function since the 
last reset. If the reset occures while the timer is running it counts as the
first interval.


\ccParDims

% EOF
