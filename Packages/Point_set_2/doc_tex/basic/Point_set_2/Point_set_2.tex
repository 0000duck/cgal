% +------------------------------------------------------------------------+
% | CGAL User Manual:  Point_set_2.tex
% +------------------------------------------------------------------------+

\ccParDims

\chapter{Two-dimensional range searches and nearest neighbor queries}
\label{chapterPoint_set_2}
\ccChapterAuthor{Matthias B\"asken}


%---------------------------------------------
% Introduction
%---------------------------------------------
\section{Introduction}

Geometric queries are fundamental to many applications in computational
geometry. The task is to maintain a dynamic set of geometric objects
in such a way that certain queries can be performed efficiently.
Typical examples of queries are:
find out whether a given object is contained in the set,
find all objects of the set lying in a given area (e.g. rectangle),
find the object closest to a given point or
find the pair of objects in the set lying closest to each other. 
Furthermore, the set should be dynamic in the sense that deletions and 
insertions of objects can be performed efficiently.

In computational geometry literature one can find many different data structures for
maintaining sets of geometric objects. Most of them are data structures 
that have been developed to support a single very special kind of query 
operation.
Examples are Voronoi diagrams for answering nearest neighbor
searches, range trees for orthogonal range queries, partition trees
for more general range queries, hierarchical triangulations for point
location and segment trees for intersection queries \dots.

In many applications, different types of queries have to be
performed on the same set of objects. A naive approach to this
problem would use a collection of the above mentioned data structures to
represent the set of objects and delegate every query operation to
the corresponding structure.
However, this is completely impractical since it uses too much
memory and requires the maintenance of all these data structures in the presence of
update operations.

Data structures that are non-optimal in theory seem to perform quite well in
practice for many of these queries.
For example, the Delaunay diagram turns out to be a very powerful
data structure for storing dynamic sets of points under range and nearest
neighbor queries. A first implementation and computational
study of using Delaunay diagrams for geometric queries is described by
Mehlhorn and N\"aher in ~\cite{LEDAbook}.

In this section we present a generic variant of a two dimensional point set
data type supporting various geometric queries.

The \ccc{CGAL::Point\_set\_2} class in this section is inherited
from the two-dimensional CGAL {\em Delaunay Triangulation} data type.

The \ccc{CGAL::Point\_set\_2} class depends on two template parameters T1 and T2.
They are used as template parameters for the \ccc{CGAL::Delaunay\_triangulation\_2}
class \ccc{CGAL::Point\_set\_2} is inherited from. T1 is a model for the
geometric traits and T2 is a model for the triangulation data structure that the Delaunay triangulation
expects.

The \ccc{CGAL::Point\_set\_2} class supports the following kinds of queries:
\begin{itemize}
\item circular range search
\item triangular range search
\item isorectangular range search
\item (k) nearest neighbor(s)
\end{itemize}
For details about the running times see ~\cite{LEDAbook}.

\section{Examples}

\subsection{Range search operations}

The following example program demonstrates the various range search operations
of the two dimensional point set.
First we construct a two dimensional point set $PSet$ and initialize it with a few points.
After the $init$ - call $PSet$ is the point set of the points $p1$, ..., $p10$ .
Then we perform circular, triangular and isorectangular range search operations on the
point set.

\ccHtmlLinksOff

{\bf rs\_example.C :}

\ccIncludeExampleCode{Point_set_2/rs_example.C}

\ccHtmlLinksOn

