\begin{ccRefFunction}{range_search}

\ccDefinition  

There are six versions of the function template \ccRefName\ that
perform range searches on Delaunay triangulations. The first performs
circular range searches, the second triangular range searches and the
third performs iso-rectangular range searches. The other three range search
function templates perform enhanced variants of the three beforementioned
operations.\\
They get a user-defined object that has to control the range search operation.
This way one can for instance stop the search, when $n$ points were found. 

\ccInclude{CGAL/range_search_delaunay_2.h}

\ccFunction{template<class Dt, class Circle, class OutputIterator>
            OutputIterator range_search(Dt& delau, const Circle& C, OutputIterator res);}
{ computes handles to all vertices contained in the closure of disk $C$.
The computed vertex handles will be placed as a sequence of objects in a container of value type
of $res$
which points to the first object in the sequence. The function
returns an output iterator pointing to the position beyond the end
of the sequence.
$delau$ is the \cgal\ Delaunay triangulation on which we perform the range search operation.}

\ccHeading{Requirements}
\begin{itemize}
\item \ccc{Dt} is a \cgal\ Delaunay triangulation and contains the following subset of types from the concept PointSetTraits and from
the Delaunay triangulation data type:
 \begin{itemize}
  \item \ccc{Dt::Geom_traits} 
  \item \ccc{Dt::Vertex_handle}
  \item \ccc{Dt::Vertex} 
  \item \ccc{Dt::Vertex_circulator}   
  \item \ccc{Dt::Vertex_iterator} 
  \item \ccc{Dt::Point}  
  \item \ccc{Dt::Geom_traits::Bounded_side_2} 
  \item \ccc{Dt::Geom_traits::Construct_center_2} 
 \end{itemize}
\item the template parameter \ccc{Circle} corresponds to \ccc{Dt::Geom_traits::Cricle_2}
\end{itemize}

\ccFunction{template<class Dt, class OutputIterator>
            OutputIterator range_search(Dt& delau, const Dt::Point& a, const Dt::Point& b, 
            const Dt::Point& c,OutputIterator res);}
{computes handles to all vertices contained in the closure of the triangle $(a,b,c)$.\\
\ccPrecond $a$, $b$, and $c$ must not be collinear. 
The computed vertex handles will be placed as a sequence of objects in a container of value type
of $res$
which points to the first object in the sequence. The function
returns an output iterator pointing to the position beyond the end
of the sequence. 
$delau$ is the \cgal\ Delaunay triangulation on which we perform the range search operation.}

\ccHeading{Requirements}
\ccc{Dt} is a \cgal\ Delaunay triangulation and contains the following subset of types from the concept PointSetTraits and from
the Delaunay triangulation data type:
\begin{itemize}
  \item \ccc{Dt::Geom_traits} 
  \item \ccc{Dt::Vertex_handle}
  \item \ccc{Dt::Vertex} 
  \item \ccc{Dt::Vertex_circulator}   
  \item \ccc{Dt::Vertex_iterator} 
  \item \ccc{Dt::Point}  
  \item \ccc{Dt::Geom_traits::Circle_2}  
  \item \ccc{Dt::Geom_traits::Bounded_side_2} 
  \item \ccc{Dt::Geom_traits::Construct_center_2} 
  \item \ccc{Dt::Geom_traits::Orientation_2} 
  \item \ccc{Dt::Geom_traits::Construct_circle_2}  
\end{itemize}


\ccFunction{template<class Dt, class OutputIterator>
            OutputIterator range_search(Dt& delau, const Dt::Point& a, const Dt::Point& b, const Dt::Point& c,const Dt::Point& d,OutputIterator res) ;}
{computes handles to all vertices contained in the closure of the iso-rectangle $(a,b,c,d)$.\\
\ccPrecond $a$ is the upper left point, $b$ the lower left, $c$ the lower
right and $d$ the upper right point of the iso rectangle.
The computed vertex handles will be placed as a sequence of objects in a container of value type
of $res$
which points to the first object in the sequence. The function
returns an output iterator pointing to the position beyond the end
of the sequence. $delau$ is the \cgal\ Delaunay triangulation on which we perform the range search operation.} 

\ccHeading{Requirements}
\ccc{Dt} is a \cgal\ Delaunay triangulation and contains the following subset of types from the concept PointSetTraits and from
the Delaunay triangulation data type:
\begin{itemize}
  \item \ccc{Dt::Geom_traits} 
  \item \ccc{Dt::Vertex_handle}
  \item \ccc{Dt::Vertex} 
  \item \ccc{Dt::Vertex_circulator}   
  \item \ccc{Dt::Vertex_iterator} 
  \item \ccc{Dt::Point}  
  \item \ccc{Dt::Geom_traits::Circle_2}  
  \item \ccc{Dt::Geom_traits::Bounded_side_2} 
  \item \ccc{Dt::Geom_traits::Construct_center_2} 
  \item \ccc{Dt::Geom_traits::Orientation_2} 
  \item \ccc{Dt::Geom_traits::Construct_circle_2}  
\end{itemize}


\ccFunction{template<class Dt, class Circle, class OutputIterator, class Pred>
            OutputIterator range_search(Dt& delau, const Circle& C, OutputIterator res,
	                                Pred& pred, bool return_if_succeded);}
{ computes handles to all vertices contained in the closure of disk $C$.
The computed vertex handles will be placed as a sequence of objects in a container of value type
of $res$
which points to the first object in the sequence. The function
returns an output iterator pointing to the position beyond the end
of the sequence.
$delau$ is the \cgal\ Delaunay triangulation on that we perform the range search operation.
$pred$ controls the search operation. If $return\_if\_succeded$ is $true$, we will end the search
after the first success of the predicate, otherwise we will continue till the search is finished.}

\ccHeading{Requirements}
For the requirements of \ccc{Dt} see the description for the non-predicate version.\\
Requirements of \ccc{Pred}:
\begin{itemize}
  \item \ccc{void set_result(bool);}
  \item \ccc{bool operator()(const Point&);}
\end{itemize}
The \ccc{operator()} is used for testing the current point in the search operation.
If this operator returns $true$ and $return\_if\_succeded$ is $true$, the range search will stop.
Otherwise the range search operation will continue. Member function  $set\_result$ can be used to
store the result of the range search in the function object. The result will be $true$ if the last
call to the  \ccc{operator()} of the predicate returned $true$, $false$ otherwise.

\ccFunction{template<class Dt, class OutputIterator, class Pred>
            OutputIterator range_search(Dt& delau, const Dt::Point& a, const Dt::Point& b, 
            const Dt::Point& c,OutputIterator res, Pred& pred, bool return_if_succeded);}
{computes handles to all vertices contained in the closure of the triangle $(a,b,c)$.\\
\ccPrecond $a$, $b$, and $c$ must not be collinear. 
The computed vertex handles will be placed as a sequence of objects in a container of value type
of $res$
which points to the first object in the sequence. The function
returns an output iterator pointing to the position beyond the end
of the sequence. 
$delau$ is the \cgal\ Delaunay triangulation on which we perform the range search operation.
$pred$ controls the search operation. If $return\_if\_succeded$ is $true$, we will end the search
after the first success of the predicate, otherwise we will continue till the search is finished.}

\ccHeading{Requirements}
For the requirements of \ccc{Dt} see the description for the non-predicate version.\\
For the requirements of \ccc{Pred} see the description above. \\

\ccFunction{template<class Dt, class OutputIterator, class Pred>
            OutputIterator range_search(Dt& delau, const Dt::Point& a, const Dt::Point& b, const Dt::Point& c,const Dt::Point& d,
	                                OutputIterator res, Pred& pred, bool return_if_succeded);}
{computes handles to all vertices contained in the closure of the iso-rectangle $(a,b,c,d)$.\\
\ccPrecond $a$ is the upper left point, $b$ the lower left, $c$ the lower
right and $d$ the upper right point of the iso rectangle.
The computed vertex handles will be placed as a sequence of objects in a container of value type
of $res$
which points to the first object in the sequence. The function
returns an output iterator pointing to the position beyond the end
of the sequence. $delau$ is the \cgal\ Delaunay triangulation on which we perform the range search operation.
$pred$ controls the search operation. If $return\_if\_succeded$ is $true$, we will end the search
after the first success of the predicate, otherwise we will continue till the search is finished.} 

\ccHeading{Requirements}
For the requirements of \ccc{Dt} see the description for the non-predicate version.\\
For the requirements of \ccc{Pred} see the description above. \\

\end{ccRefFunction}


