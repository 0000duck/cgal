% =============================================================================
% The CGAL Reference Manual
% Chapter: Geometric Optimisation
% Concept: Min_annulus_d_traits
% -----------------------------------------------------------------------------
% file  : doc_tex/basic/Optimisation/Optimisation_ref/Min_annulus_d_traits.tex
% author: Sven Sch�nherr <sven@inf.ethz.ch>
% -----------------------------------------------------------------------------
% $CGAL_Chapter: Geometric Optimisation $
% $CGAL_Package: Min_annulus_d WIP $
% $Revision$
% $Date$
% =============================================================================

\begin{ccRefConcept}{Min_annulus_d_traits}

% -----------------------------------------------------------------------------
\ccDefinition

This concept defines the requirements for traits classes of
\ccGlobalScope\ccc{Min_annulus_d<Traits>}.

% -----------------------------------------------------------------------------
\ccTypes

\ccSaveThreeColumns
\ccSetTwoColumns{Min_annulus_d_traits:: Rep_tag}{}

\ccNestedType{Point_d}{point type used to represent the input points.}

\ccNestedType{Rep_tag}{%
  compile time tag to distinguish between Cartesian and homogeneous
  representation of the input points. \ccc{Rep_tag} has to be either
  \ccGlobalScope\ccc{Cartesian_tag} or
  \ccGlobalScope\ccc{Homogeneous_tag}.}

\ccNestedType{RT}{%
  number type used to represent the coordinates of the input points.
  It has to be a model for \ccc{RingNumberType}.}

\ccNestedType{FT}{%
  number type used to return the squared radii of the smallest
  enclosing annulus. \ccc{FT} has to be either \ccc{RT} or
  \ccGlobalScope\ccc{Quotient<RT>} if the input points have Cartesian
  or homogeneous representation, respectively (cf.~\ccc{Rep_tag}).}

\ccNestedType{Access_dimension_d}{%
  data accessor object used to access the dimension of the input points.}

\ccNestedType{Access_coordinates_begin_d}{%
  data accessor object used to access the coordinates of the input points.}

\ccNestedType{Construct_point_d}{%
  constructor object used to construct the center of the smallest
  enclosing annulus from its coordinates.}

\ccNestedType{ET}{%
  exact number type used to do the exact computations in the
  underlying solver for linear programs. It has to to be a model for
  \ccc{RingNumberType}. There must be an implicit conversion from
  \ccc{RT} to \ccc{ET} available.}

\ccNestedType{NT}{%
  fast (possibly inexact) number type used to speed up the pricing step in
  the underlying solver for linear programs. It has to be a model for
  \ccc{RingNumberType}. There must be implicit conversions from \ccc{RT} to
  \ccc{NT} and from \ccc{NT} to \ccc{ET} available.}

\ccRestoreThreeColumns

% -----------------------------------------------------------------------------
\ccCreation
\ccCreationVariable{traits}

Only default and copy constructor are required.

\ccConstructor{ Min_annulus_d_traits( );}{}

\ccConstructor{ Min_annulus_d_traits( const Min_annulus_d_traits&);}{}

% -----------------------------------------------------------------------------
\ccOperations

\ccSetThreeColumns{Access_coordinates_begin_d}{}{\hspace*{10cm}}

The following functions just return the corresponding function class
object.

\ccMemberFunction{ Access_dimension_d
                   access_dimension_d_object() const;}{}

\ccMemberFunction{ Access_coordinates_begin_d
                   access_coordinates_begin_d_object() const;}{}

\ccMemberFunction{ Construct_point_d
                   construct_point_d_object() const;}{}

\ccRestoreThreeColumns

% -----------------------------------------------------------------------------
\ccHasModels

\ccRefIdfierPage{\ccPureGlobalScope Min_annulus_d_traits_2<R,ET,NT>}\\
\ccRefIdfierPage{\ccPureGlobalScope Min_annulus_d_traits_3<R,ET,NT>}\\
\ccRefIdfierPage{\ccPureGlobalScope Min_annulus_d_traits_d<R,ET,NT>}

% -----------------------------------------------------------------------------
\ccSeeAlso

\ccRefIdfierPage{\ccPureGlobalScope Min_annulus_d<Traits>}

% -----------------------------------------------------------------------------

\end{ccRefConcept}

% ===== EOF ===================================================================
