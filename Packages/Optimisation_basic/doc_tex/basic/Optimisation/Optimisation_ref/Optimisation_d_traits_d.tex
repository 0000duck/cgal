% -*- latex -*-
% =============================================================================
% The CGAL Reference Manual
% Chapter: Geometric Optimisation
% Class  : CGAL::Optimisation_d_traits_d<Traits>
% -----------------------------------------------------------------------------
% file   : doc_tex/basic/Optimisation/..._ref/Optimisation_d_traits_d.tex
% package: Optimisation_basic
% author : Sven Sch�nherr <sven@inf.ethz.ch>
% -----------------------------------------------------------------------------
% $Revision$
% $Date$
% =============================================================================

\ccAutoIndexingOff

\begin{ccRefClass}{Optimisation_d_traits_d<R,ET,NT>}
\ccIndexTraitsClassBegin{Optimisation_d_traits_d}{%
  Min_sphere_d;Min_annulus_d;Polytope_distance_d}{}

\ccAutoIndexingOn

% -----------------------------------------------------------------------------
\ccDefinition

The class \ccRefName\ is a traits class for the $d$-dimensional
optimisation algorithms using the $d$-dimensional \cgal\ kernel.
 
\ccInclude{CGAL/Optimisation_d_traits_d.h}

% -----------------------------------------------------------------------------
\ccRequirements

The template parameter \ccc{R} is a model for \ccc{Kernel}. Template
parameters \ccc{ET} and \ccc{NT} are models for \ccc{RingNumberType}.

The second and third template parameter have default type \ccc{R::RT}.

% -----------------------------------------------------------------------------
\ccIsModel

\ccRefConceptPage{Optimisation_d_traits}

% -----------------------------------------------------------------------------
\ccTypes

\ccSaveThreeColumns
\ccSetTwoColumns{Optimisation_d_traits_d<R,ET,NT>:: Rep_tag}{}

\ccNestedType{Point_d}{typedef to \ccc{R::Point_d}.}

\ccNestedType{Rep_tag}{typedef to \ccc{R::Rep_tag}.}

\ccNestedType{RT}{typedef to \ccc{R::RT}.}
\ccNestedType{FT}{typedef to \ccc{R::FT}.}

\ccNestedType{Access_dimension_d}{%
  typedef to \ccc{R::Access_dimension_d}.}
\ccNestedType{Access_coordinates_begin_d}{%
  typedef to \ccc{R::Access_coordinates_begin_d}.}

\ccNestedType{Construct_point_d}{%
  typedef to \ccc{R::Construct_point_d}.}

\ccNestedType{ET}{second template parameter (default is \ccc{R::RT}).}
\ccNestedType{NT}{third  template parameter (default is \ccc{R::RT}).}

% -----------------------------------------------------------------------------
\ccCreation
\ccCreationVariable{traits}

\ccConstructor{ Optimisation_d_traits_d( );}{default constructor.}
\ccConstructor{ Optimisation_d_traits_d(
                  const Optimisation_d_traits_d<R,ET,NT>&);}{copy constructor.}

% -----------------------------------------------------------------------------
\ccOperations

\ccSetThreeColumns{Access_coordinates_begin_d}{}{\hspace*{8.5cm}}

The following functions just return the corresponding function class
object.

\ccMemberFunction{ Access_dimension_d
                   access_dimension_d_object() const;}{}

\ccMemberFunction{ Access_coordinates_begin_d
                   access_coordinates_begin_d_object() const;}{}

\ccMemberFunction{ Construct_point_d
                   construct_point_d_object() const;}{}

\ccRestoreThreeColumns

% -----------------------------------------------------------------------------
\ccSeeAlso

\ccRefIdfierPage{CGAL::Min_sphere_d<Traits>}\\
\ccRefIdfierPage{CGAL::Min_annulus_d<Traits>}\\
\ccRefIdfierPage{CGAL::Polytope_distance_d<Traits>}\\[1ex]
\ccRefIdfierPage{CGAL::Optimisation_d_traits_2<R,ET,NT>}\\
\ccRefIdfierPage{CGAL::Optimisation_d_traits_3<R,ET,NT>}\\[1ex]
\ccRefConceptPage{Optimisation_d_traits}

% -----------------------------------------------------------------------------

\ccAutoIndexingOff

\ccIndexTraitsClassEnd
\end{ccRefClass}

\ccAutoIndexingOn

% ===== EOF ===================================================================
