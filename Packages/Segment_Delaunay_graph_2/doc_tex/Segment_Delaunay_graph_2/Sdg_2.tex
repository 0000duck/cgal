%% Copyright (c) 2004,2005  INRIA Sophia-Antipolis (France) and
%% Notre Dame University (U.S.A.).  All rights reserved.
%%
%% This file is part of CGAL (www.cgal.org); you may redistribute it under
%% the terms of the Q Public License version 1.0.
%% See the file LICENSE.QPL distributed with CGAL.
%%
%% Licensees holding a valid commercial license may use this file in
%% accordance with the commercial license agreement provided with the software.
%%
%% This file is provided AS IS with NO WARRANTY OF ANY KIND, INCLUDING THE
%% WARRANTY OF DESIGN, MERCHANTABILITY AND FITNESS FOR A PARTICULAR PURPOSE.
%%
%% $Source$
%% $Revision$ $Date$
%% $Name$
%%
%% Author(s)     : Menelaos Karavelas <mkaravel@cse.nd.edu>

This chapter describes the two-dimensional segment Voronoi diagram
package of \cgal. We start with a few definitions in 
Section~\ref{sec:sdg2-definitions}.
The software design of the 2D segment Voronoi diagram package is
described in Section~\ref{sec:sdg2-design}.
In Section~\ref{sec:sdg2-traits} we discuss the geometric
traits of the 2D segment Voronoi diagram package and in Section
\ref{sec:sdg2-hierarchy} the segment Voronoi diagram hierarchy, a
data structure suitable for fast nearest neighbor queries, is briefly
described.


\section{Definitions}
\label{sec:sdg2-definitions}

\begin{figure}[htb]
\begin{ccTexOnly}
\begin{center}
\includegraphics[width=0.4\textwidth]%
{Segment_Delaunay_graph_2/svd-weakly-bw}
\hfil
\includegraphics[width=0.4\textwidth]%
{Segment_Delaunay_graph_2/svd-strongly-bw} 
\end{center}
\end{ccTexOnly}
\begin{ccHtmlOnly}
  <center>
  <img border=1 src="./svd-weakly-bw.gif" align=center
  alt="The segment Voronoi diagram for a set of weakly intersecting sites"
  title="The segment Voronoi diagram for a set of weakly intersecting sites">
  <!-- -->
  <img border=1 src="./svd-strongly-bw.gif" align=center
  alt="The segment Voronoi diagram for a set of strongly intersecting sites"
  title="The segment Voronoi diagram for a set of strongly intersecting sites">
</center>
\end{ccHtmlOnly}
\begin{ccHtmlOnly}
<font size=-1>
\end{ccHtmlOnly}
\begin{center}
\caption{The segment Voronoi diagram for a set of weakly (left) and
  strongly (right) intersecting sites.}\label{fig-sdg}
\end{center}
\begin{ccHtmlOnly}
</font>
\end{ccHtmlOnly}
\end{figure}

The 2D segment Voronoi diagram package of \cgal\ is designed to
compute the dual of the Voronoi diagram of a set of possibly
intersecting segments on the plane. Although we compute the dual, we
will often refer to the diagram itself, since it is easier to
explain an d understand. The algorithm that has been
implemented is incremental. The corresponding \cgal\ class is called
\ccc{Segment_Delaunay_graph_2<SegmentDelaunayGraphTraits_2,SegmentDelaunayGraphStructure_2>}
and will be discussed in more detail in the sequel. The interested
reader may want to refer to the paper by Karavelas
\cite{cgal:k-reisv-04} for the general idea as well as the details of
the algorithm implemented.

\paragraph{Definitions.}
Before describing the details of the implementation we make a brief
introduction to the theory of segment Voronoi diagrams.
The segment Voronoi diagram is defined over a set of non-intersecting
sites, which can either be points or linear segments, which we assume
that are given through their endpoints. The segment Voronoi diagram a
subdivision of the plane into connected regions, called {\em cells},
associated with the sites. The cell of a site $t_i$ is the locus of
points on the plane that are closer to $t_i$ than any other site
$t_j$, $j\neq i$.
The distance $\delta(x, t_i)$ of a point $x$ in the plane to a 
site $t_i$ is defined as the minimum of the Euclidean distances of $x$
from the points in $t_i$. Hence, if $t_i$ is a point $p_i$, then
\[      \delta(x,t_i)=\|x-t_i\|,  \]
whereas if $t_i$, is a segment, then
\[      \delta(x,t_i)=\min_{y\in t_i}\|x-y\|, \]
where $\|\cdot\|$ denotes the Euclidean norm.
It can easily be seen that it is a generalization of the Voronoi
diagram for points.

In many applications the restriction that sites are non-intersecting
is too strict. Often we want to allow segments that touch at their
endpoints, or even segments that overlap or intersect properly at
their interior (for example, see Fig. \ref{fig-sdg}). Allowing such
configurations poses certain problems. More specifically, when we allow
segments to touch at their endpoints we may end up with pairs of
segments whose bisector is two-dimensional. If we allow pairs of
segments that intersect properly at their interior, the interiors of
their Voronoi cells are no longer simply connected. In both cases
above the resulting Voronoi diagrams are no longer instances of
abstract Voronoi diagrams (cf. \cite{k-cavd-89}), which has a direct
consequence on the efficient computation of the corresponding Voronoi
diagram. The remedy to these problems is to consider linear segments
not as one object, but rather as three, namely the two endpoints and
the interior. This choice guarantees that all bisectors in the Voronoi
diagram are one-dimensional and that all Voronoi cells are simply
connected. Moreover, we further distinguish between two cases,
according to the type of intersecting pair that our input data set
contains. A pair of sites is called \emph{weakly intersecting} if they
a single common point and this common point does not lie in the
interior of any of the two sites. A pair of sites is called
\emph{strongly intersecting} if they intersect and they either have
more than one common point or their common point lies in the interior
of at least one of the two sites. As it will be seen later the two
cases have different representation (and thus storage) requirements,
as well as they require a somehow different treatment on how the
predicates are evaluated. Having made the distinction between weakly
and strongly intersecting sites, and having said that segment sites
are treated as three objects, we are now ready to precisely define the
Voronoi diagram we compute. Given a set $\mathcal{S}$ of input sites,
let $\mathcal{S}_{\mathcal{A}}$ be the set of points and
(open) segments in the arrangement $\mathcal{A}(\mathcal{S})$ of
$\mathcal{S}$. The 2D segment Voronoi diagram package of \cgal\
computes the dual of the Voronoi diagram of the sites in the set 
$\mathcal{S}_{\mathcal{A}}$.

The dual of the segment Voronoi diagram is uniquely defined once we
have the segment Voronoi diagram. If the all sites are in
\emph{general position}, then dual graph is a graph with triangular
faces away from the convex hull of the set of sites. To unify our approach
and handling of the dual graph we add to the set of (finite) sites
a fictitious site at infinity, which we call the
{\em site at infinity}. We can then connect all vertices of the outer
face of the dual graph to the site at infinity which gives us
a graph with the property that all of its faces are now
triangular. However, the dual graph is not a triangulation for
two main reasons: we cannot always embed it on the plane with straight
line segments that yield a triangulation and, moreover, we may have two
faces of the graph that have two edges in common, which is not allowed
in a triangulation.

We would like to finish our brief introduction to the theory of
segment Voronoi diagrams by discussing the concept of general
position. We say that a set of sites is in general position if no two
triplets of sites have the same tritangent Voronoi circle. This
statement is rather technical and it is best understood in the context
of points. The equivalent statement for points is that we have no two
triplets of points that define the same circumcircle, or equivalently
that no four points are co-circular. The statement about general
position made above is a direct generalization of the (much simpler to
understand) statement about points. On the contrary, when we have
sites in degenerate position, the dual graph has faces with more than
three edges on their boundary. We can get a triangulated version of
the graph by simply {\em triangulating} the corresponding faces in an
arbitrary way. In fact the algorithm that has been implemented in
\cgal\ has the property that it always returns a valid
{\em triangulated} version of the dual graph of the segment Voronoi
diagram. By valid we mean that it contains the actual dual graph
(i.e., the actual dual of the segment Voronoi diagram) and whenever
there are faces with more than three faces then they are
triangulated. The way that they are triangulated depends on the order
of insertion of the sites in the diagram.

One final remark has to be made with respect to the difference between
the set of \emph{input sites} and the set of \emph{output sites}. The
set of input sites consists of the closed sites that the user inserts
in the diagram. Since segment sites are treated as three objects,
internally our algorithm sees only points and open segments. As a
result, from the point of view of the algorithm, the input sites have
no real meaning. What has real meaning is the set of sites that
correspond to cells of the Voronoi diagrams and this is the set of
output sites.

\paragraph{Degenerate dimensions.}
The dimension of the segment Voronoi diagram is in general 2. The
exceptions to this rule are as follows:
\begin{itemize}
\item The dimension is $-1$ if the segment Voronoi diagram contains no
  sites.
\item The dimension is $0$ if the segment Voronoi diagram contains exactly
  one (output) site.
\item The dimension is $1$ is the segment Voronoi diagram contains exactly
  two (output) sites.
\end{itemize}


\section{Software Design}
\label{sec:sdg2-design}

The 2D segment Voronoi diagram class
\ccc{Segment_Delaunay_graph_2<SegmentDelaunayGraphTraits_2,SegmentDelaunayGraphDataStructure_2>}
follows the design of the triangulation package of \cgal. It is
parametrized by two arguments:
\begin{itemize}
\item the {\bf geometric traits} class. It provides the basic
  geometric objects involved in the algorithm, such as sites, points
  etc. It also provides the geometric predicates for the computation
  of the segment Voronoi diagram, as well as some basic constructions
  that can be used, for example, to visualize the diagram.
  The geometric traits for the segment Voronoi diagram will be
  discussed in more detail in the next section.
\item the {\bf segment Voronoi diagram data structure}. This is
  essentially the same as the Apollonius graph data structure (discussed
  in Chapter \ref{sec:apollonius2-design}), augmented with some
  additional operations that are specific to segment Voronoi
  diagrams. The corresponding concept is that of
  \ccc{SegmentDelaunayGraphDataStructure_2}, which in fact is a refinement
  of the \ccc{ApolloniusGraphDataStructure_2} concept. The class
  \ccc{Triangulation_data_structure_2<Vb,Fb>} is a model of
  the concept \ccc{SegmentDelaunayGraphDataStructure_2}. A default value
  for the corresponding template parameter is provided, so the user
  does not need to specify it.
\end{itemize}


\paragraph{Strongly intersecting sites and their representation.}
As we have mentioned above, the segment Voronoi diagram package of
\cgal\ is designed to support the computation of the segment Voronoi
diagram even when the input segment sites are intersecting. This
choice poses certain issues for the design of the software
package. The major concern is the representation of the subsegments
that appear in the arrangement of the these sites, because the sites
in the arrangement are the ones over which the diagram is actually
defined. A direct consequence of the choice of representation is the
algebraic degree of the predicates involved in the computation of the
segment Voronoi diagram, as well as the storage requirements for the
subsegments and points on intersection in the arrangement.

The case of weakly intersecting sites does not require any special
treatment. We can simply represent points by their coordinates and
segments by their endpoints.
In the case of strongly intersecting sites,
the obvious choice to use the afore-mentioned representation
%represent points by their coordinates and segments by their endpoints 
has severe disadvantages. Consider two
strongly intersecting segments $t_i$ and $t_j$, whose endpoints have
homogeneous coordinates of size $b$. Their intersection point will
have homogeneous coordinates of bit size $6b+O(1)$. This effect can be
cascaded, which implies that after inserting $k$ (input) segments we
can arrive at having points of intersection whose bit sizes are
exponential with respect to $k$, i.e., their homogeneous coordinates
will have bit size $\Omega(2^kb)$. Not only the points of
intersection, but also the adjacent subsegments will be represented by
quantities of arbitrarily high bit size, and as a result we would not
be able to give a bound on the bit sizes of the coordinates of the
points of intersection. As a result, we would not be able to give
a bound on the memory needed to store these coordinates. An equally
important consequence is that we would also not be able to give a
bound on the algebraic degree of the algebraic expressions involved in
the evaluation of the predicates.

Such a behavior is obviously undesirable. For robustness, efficiency,
and scalability purposes, it is critical that the bit size of the
algebraic expressions in the predicates does not depend on the input
size. For this reason, as well as for others to be discussed below,
we decided to represent sites in a implicit manner, which somehow
encodes the history of their construction. In particular, we exploit
the fact that points of intersection always lie on two input segments,
and that segments that are not part of the input are always supported
by input segments.

For example, let us consider the configuration in
Fig. \ref{fig:sdg2-siterep}. We assume that the segments
$t_i=p_iq_i$, $i=1,2,3$, are inserted in that order. Upon the
insertion of $t_2$, our algorithm will split the segment $t_1$ into
the subsegments
$p_1s_1$ and $s_1q_1$, then add $s_1$, and finally insert the
subsegments $p_2s_1$ and $s_1q_2$. How do we represent the five new
sites? $s_1$ will be represented by its two defining segments $t_1$
and $t_2$. The segment $p_1s_1$ will be represented by two segments, a
point, and a boolean. The first segment is $t_1$, which is always the
segment with the same support as the newly created segment. The second
segment is $t_2$ and the point is $p_1$. The boolean indicates whether
the first endpoint of $p_1s_1$ is an input point; in this case the
boolean is equal to \ccc{true}. The segment $s_1q_1$ will also be
represented by two segments, a point, and a boolean, namely, $t_1$
(the supporting segment of $s_1q_1$), $t_2$ and \ccc{false} (it is the
second endpoint of $s_1q_1$ that is an input point). Subsegments
$p_2s_2$ and $s_2q_2$ are represented analogously.
Consider now what happens when we insert $t_3$. The point 
$s_2$ will again be represented by two segments, but not $s_1q_1$ and
$t_3$. In fact, it will be represented by $t_1$ (the supporting
segment of $s_1q_1$) and $t_3$. $s_2q_1$ will be represented
by two segments, a point, and a boolean ($t_1$, $t_3$ and
\ccc{false}), and similarly for $p_3s_2$ and $s_2q_3$. On the other
hand, both endpoints of $s_1s_2$ are non-input points. In such a
case we represent the segment by three input segments.
More precisely, $s_1s_2$ is represented by the segments $t_1$ (the
supporting segment of $s_1q_1$), $t_2$ (it defines $s_1$ along with
$t_1$) and $t_3$ (it defines $s_2$ along with $t_1$).

\begin{figure}[!t]
  \begin{ccTexOnly}
    \begin{center}
      \includegraphics[width=0.3\textwidth]%
      {Segment_Delaunay_graph_2/sdg-rep}
    \end{center}
  \end{ccTexOnly}
  \begin{ccHtmlOnly}
    <center>
    <img border=1 src="./sdg-rep.gif" align=center
    alt="Site representation" title="Site representation">
    </center>
  \end{ccHtmlOnly}
  \begin{ccHtmlOnly}
    <font size=-1>
  \end{ccHtmlOnly}
  \begin{center}
    \caption{Site representation. The point $s_1$ is represented by the
      four points $p_1$, $q_1$, $p_2$ and $q_2$. The segment $p_1s_1$ is
      represented by the points $p_1$, $q_1$, $p_2$, $q_2$ and a boolean
      which is set to \emph{true} to indicate that the first endpoint in
      not a point of intersection. The segment $s_1s_2$ is represented by
      the six points: $p_1$, $q_1$, $p_2$, $q_2$, $p_3$ and $q_3$. The
      remaining (non-input) points and segments in the figure are
      represented similarly.}\label{fig:sdg2-siterep}
  \end{center}
  \begin{ccHtmlOnly}
    </font>
  \end{ccHtmlOnly}
\end{figure}

The five different presentations, two for points (coordinates; two
input segments) and three for segments (two input points; two input
segments, an input point and a boolean; three input segments),
form a closed set of representations and thus represent
any point of intersection or subsegment regardless of the number of
input segments. Moreover, every point (input or intersection) has
homogeneous coordinates of bit size at most $3b+O(1)$.
The supporting lines of the segments (they are needed in some of
the predicates) have coefficients which are always of bit size
$2b+O(1)$. As a result, the bit size of the expressions involved in
our predicates will always be $O(b)$, independently of the
size of the input.
%
The \ccc{SegmentDelaunayGraphSite_2} concept encapsulates the ideas
presented above. A site is represented in this concept by up to four
points and a boolean, or up to six points, depending on its type. The
class \ccc{Segment_Delaunay_graph_site_2<K>} implements this
concept.

%% Notice, however a drawback in our approach: we only need up
%% to two points to represent a site, when the input data set consists of
%% only weakly intersecting sites. In other words, we penalize the
%% representation of sites with an overhead of up to four points, which
%% dramatically increases the storage requirements of our
%% implementation. For this reason we have a model for the
%% \ccc{SegmentDelaunayGraphSite_2} concept that implements the concept
%% in a custom way, when we only have weakly intersecting sites. The
%% corresponding class is called
%% \ccc{Segment_Delaunay_graph_simple_site_2<K>} and it only stores 2
%% points, in contrast to the 6 points stored by the
%% \ccc{Segment_Delaunay_graph_2<K>} class.

Even this representation, however, has some degree of redundancy. The
endpoint of a segment appears in both the representation of the (open)
segment site as well as the representation of the point site
itself. The situation becomes even worse in the presence of strongly
intersecting sites: a 
point may appear in the representation of multiple subsegments and/or
points of intersection. To avoid this redundancy, input points are
stored in a container, and the various types of sites (input points
and segments, points of intersection, subsegments with one or two
points of intersection as endpoints) only store handles to the points
in the container. This is achieved by the
\ccc{Segment_Delaunay_graph_storage_site_2<Gt>} class which is a
model of the corresponding concept:
\ccc{SegmentDelaunayGraphStorageSite_2}. This concept enforces a site
to be represented by up to 6 handles (which are very lightweight
objects) instead of 6 points, which are, compared to handles of
course, very heavy objects.

\paragraph{Optimizing memory allocation.}
There are applications where we know beforehand that the input
consists of only weakly intersecting sites. In these cases the site
representation described above poses a significant overhead in the
memory requirements of our implementation: instead of representing
sites with up to two points (or ultimately with to two handles), we
require sites to store six points (respectively, six handles). To
avoid this overhead we have introduced two series of traits classes:
\begin{itemize}
\item One that supports the full-fledged sites, and is suitable when
  the input consists of strongly intersecting sites. This series
  consists of the 
  \ccc{Segment_Delaunay_graph_traits_2<K,MTag>} and
  \ccc{Segment_Delaunay_graph_filtered_traits_2<CK,CM,EK,EM,FK,FM>} 
  classes.
\item One that is customized for input that contain only weakly
  intersecting sites. This series consists of the
  \ccc{Segment_Delaunay_graph_traits_without_intersections_2<K,MTag>} and
  \ccc{Segment_Delaunay_graph_filtered_traits_without_intersections_2<CK,CM,EK,EM,FK,FM>}
  classes.
\end{itemize}
The advantages of having different traits classes are as follows:
\begin{itemize}
\item When the user chooses to use one of the traits classes in the
  second series we only store two handles per site. This implies a
  reduction by a factor of three in the memory allocated per site with
  respect to the first series of traits classes.
\item In the case of the first series of traits classes, we can better
  exploit the knowledge that have strongly intersecting sites, in
  order to further apply geometric filters (see below) during the
  evaluation of the predicates. On the contrary, if the second series
  of traits classes is used, we can avoid geometric filtering tests
  that have meaning only in the case of strongly intersecting sites.
\end{itemize}


\section{The Geometric Traits}
\label{sec:sdg2-traits}

The predicates required for the computation of the segment Voronoi
diagram are rather complicated. It is not the purpose of this document to
discuss them in detail. The interested reader may refer to Burnikel's
thesis \cite{b-ecvdl-96}, where it is shown that in the case of weakly
intersecting sites represented in homogeneous coordinates of bit size
$b$, the maximum bit size of the algebraic expressions involved in the
predicates is $40b+O(1)$. Given our site representation given above we
can guarantee that even in the case of strongly intersecting sites,
the algebraic degree of the predicates remains $O(b)$, independently
of the size of the input. What we want to focus in the remainder of
this section are the different kinds of filtering techniques that we
have employed in our implementation.

\paragraph{Geometric filtering.}
Our representation of sites is coupled very naturally, with what we
call {\em geometric filtering}. The technique
amounts to performing simple geometric tests exploiting the
representation of our data, as well as the geometric structure
inherent in our problem, in order to evaluate predicates in seemingly
degenerate configurations. Geometric filtering can be seen as a
preprocessing step before performing arithmetic filtering.
Roughly speaking, by arithmetic filtering we mean that we first try to
evaluate the predicates using a fixed-precision floating-point number
type (such as {\tt double}), and at the same time keep error bounds on
the numerical errors of the computations we perform. If the numerical
errors are too big and do not permit us to evaluate the predicate, we
switch to an exact number type, and repeat the evaluation of the
predicate. Geometric filtering can help by eliminating situations in
which the arithmetic filter will fail, thus decreasing the number
of times we need to evaluate a predicate using exact arithmetic.

To illustrate the application and effectiveness of this approach,
let us consider a very simple example usage.
Suppose we want to determine if two non-input points are
identical (we assume here that the input sites are represented by
\ccc{double}s). In order to do that we need to compute their
coordinates and compare them. If the two points are identical, the
answer to our question using \ccc{double} arithmetic may be 
wrong (due to numerical errors), in which case we will have to reside
to the more expensive exact computation. Instead, before testing the
coordinates for equality, we can use the representation of the points
to potentially answer the question. More specifically, and this is the
geometric filtering part of the computation, we can first
test if the defining segments of the two points are the same. If they
are not, then we proceed to comparing their coordinates as
usual. Testing the defining segments for equality does not involve any
arithmetic operations on the input, but rather only comparisons on
\ccc{double}s. By performing this very simple test we avoid a
numerically difficult computation, which could be performed
thousands of times during the computation of a Voronoi diagram.

Geometric filtering has been implemented in all our models of the
\ccc{SegmentDelaunayGraphTraits_2} concept. These models are the
classes:
\ccc{Segment_Delaunay_graph_traits_2<K,MTag>},
\ccc{Segment_Delaunay_graph_traits_without_intersections_2<K,MTag>},
\ccc{Segment_Delaunay_graph_filtered_traits_2<CK,CM,EK,EM,FK,FM>} and
\ccc{Segment_Delaunay_graph_filtered_traits_without_intersections_2<CK,CM,EK,EM,FK,FM>}.



\paragraph{Arithmetic filtering.}
As mentioned above, performing computations with exact arithmetic
can be very costly. For this reason we have devoted considerable
effort in implementing different kinds of arithmetic filtering
mechanisms. Presently, there two ways of performing arithmetic
filtering for the predicates involved in the computation of
segment Voronoi diagrams:
\begin{enumerate}
\item The user can define his/her kernel using as number type, a
  number type of the form \ccc{CGAL::Filtered_exact<CT,ET>}. Then this
  kernel can be entered as the first template parameter in the
  \ccc{Segment_Delaunay_graph_2<K,MTag>} or
  \ccc{Segment_Delaunay_graph_with_intersections_2<K,MTag>} class.
\item The user can define up to three different kernels \ccc{CK},
  \ccc{FK} and \ccc{EK} (default values are provided for most
  parameters). The first kernel \ccc{CK} is used only for
  constructions. The second kernel \ccc{FK} is the filtering kernel:
  the traits class will attempt to compute the predicates using this
  kernel. If the filtering kernel fails to successfully compute a
  predicate, the exact kernel \ccc{EK} will be used. These three
  kernels are then used in the 
  \ccc{Segment_Delaunay_graph_filtered_traits_2<CK,CM,EK,EM,FK,FM>} and
  \ccc{Segment_Delaunay_graph_filtered_traits_without_intersections_2<CK,CM,EK,EM,FK,FM>}
  classes, which have been implemented using the
  \ccc{Filtered_predicate<EP,FP>} mechanism.
\end{enumerate}
Our experience so far has shown that for all reasonable and valid
values of the template parameters, the second method for arithmetic
filtering is more efficient among the two.

Let's consider once more the classes
\ccc{Segment_Delaunay_graph_2<K,MTag>} and 
\ccc{Segment_Delaunay_graph_with_intersections_2<K,MTag>}.
The template parameter \ccc{MTag} provides another degree of freedom
to the user, who can indicate the type of arithmetic operations to
be used in the evaluation of the predicates. More specifically, in
both classes, \ccc{MTag} can be \ccc{CGAL::Sqrt_field_tag}, in
which case the predicates will be evaluated using all four basic
arithmetic operations plus square roots; this requires, of course,
that the number type used in the kernel \ccc{K} supports these
operations exactly. The second choices are \ccc{CGAL::Field_tag} for
the \ccc{Segment_Delaunay_graph_2<K,MTag>} class, and
\ccc{CGAL::Ring_tag} for the
\ccc{Segment_Delaunay_graph_with_intersections_2<K,MTag>}
class. In the first case we indicate that we want the predicates to be
computed using only the four basic arithmetic operations, whereas in
the second case we evaluate the predicates using only ring operations.
Again, for the predicates to be evaluated correctly, the number type
used in the kernel \ccc{K} must support the corresponding operations
exactly.

The semantics for the template parameters \ccc{CM}, \ccc{FM} and
\ccc{EM} in the
\ccc{Segment_Delaunay_graph_filtered_traits_2<CK,CM,EK,EM,FK,FM>} and 
\ccc{Segment_Delaunay_graph_filtered_traits_without_intersections_2<CK,CM,EK,EM,FK,FM>}
classes are analogous. With each of these template parameters we can
control the type of arithmetic operations that are going to be used in
calculations involving each of the corresponding kernels \ccc{CK},
\ccc{FK} and \ccc{EK}. When the
\ccc{Segment_Delaunay_graph_filtered_traits_2<CK,CM,EK,EM,FK,FM>} is
used the possible values for \ccc{CM}, \ccc{FM} and \ccc{EM} are
\ccc{CGAL::Sqrt_field_tag} and \ccc{CGAL::Field_tag}, whereas if the 
\ccc{Segment_Delaunay_graph_filtered_traits_without_intersections_2<CK,CM,EK,EM,FK,FM>}
class is used, the possible values are \ccc{CGAL::Sqrt_field_tag} and
\ccc{CGAL::Ring_tag}. The semantics are the same as in the case of the 
\ccc{Segment_Delaunay_graph_2<K,MTag>} and 
\ccc{Segment_Delaunay_graph_with_intersections_2<K,MTag>} classes.






\section{The segment Voronoi diagram hierarchy}
\label{sec:sdg2-hierarchy}


The
\ccc{Segment_Delaunay_graph_hierarchy_2<SegmentDelaunayGraphTraits_2,
  SSTag, SegmentDelaunayGraphDataStructure_2>} class is the analogue
of the \ccc{Triangulation_hierarchy_2} or the
\ccc{Apollonius_graph_hierarchy_2} classes, applied to the segment
Voronoi diagram. It consists of a hierarchy of
segment Voronoi diagrams constructed in a manner analogous to the
Delaunay hierarchy by Devillers \cite{cgal:d-dh-02}. Unlike the
triangulation hierarchy or the Apollonius graph hierarchy, the
situation here is more complicated because of two factors: firstly,
segments are treated as three objects instead of one (the two
endpoints and the interior of the segments), and secondly, the
presence of strongly intersecting sites complicates significantly the
way the hierarchy is constructed. The interested reader may refer to
the paper by Karavelas \cite{cgal:k-reisv-04} for the details of the
construction of the hierarchy.
Another alternative is to have a hybrid hierarchy that consists of the
segment Voronoi diagram at the bottom-most level and point Voronoi
diagrams at all other levels. This choice seems to work very well in
practice , primarily because it avoids the overhead of maintaining a
Voronoi diagram for segments at the upper levels of the
hierarchy. However, it seems much less likely to be possible to give
any theoretical guarantees for its performance, in contrast to the
hierarchy with segment Voronoi diagrams at all levels
(cf. \cite{cgal:k-reisv-04}). The user can choose between the two
types of hierarchies by means of the template parameter
\ccc{SSTag}. If \ccc{SSTag} is set to \ccc{false} (which is also the
default value), the upper levels of the hierarchy consist of point
Voronoi diagrams. If \ccc{SSTag} is set to \ccc{true}, we have segment
Voronoi diagrams at all levels of the hierarchy.

The class
\ccc{Segment_Delaunay_graph_hierarchy_2<SegmentDelaunayGraphTraits_2,
  SSTag, SegmentDelaunayGraphDataStructure_2>}
has exactly the same interface and functionality as the
\ccc{Segment_Delaunay_graph_2<SegmentDelaunayGraphTraits_2,SegmentDelaunayGraphDataStructure_2>}
class. Using the segment Voronoi diagram hierarchy involves an
additional cost in space and time for maintaining the hierarchy. Our
experiments have shown that it usually pays off to use the hierarchy
for inputs consisting of more than about 1,000 sites. 

%This threshold holds
%for both the construction of the  diagram itself, as well as for
%nearest neighbor queries.

%EOF
