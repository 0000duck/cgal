%% Copyright (c) 2003,2004,2005  INRIA Sophia-Antipolis (France) and
%% Notre Dame University (U.S.A.).  All rights reserved.
%%
%% This file is part of CGAL (www.cgal.org); you may redistribute it under
%% the terms of the Q Public License version 1.0.
%% See the file LICENSE.QPL distributed with CGAL.
%%
%% Licensees holding a valid commercial license may use this file in
%% accordance with the commercial license agreement provided with the software.
%%
%% This file is provided AS IS with NO WARRANTY OF ANY KIND, INCLUDING THE
%% WARRANTY OF DESIGN, MERCHANTABILITY AND FITNESS FOR A PARTICULAR PURPOSE.
%%
%% $Source$
%% $Revision$ $Date$
%% $Name$
%%
%% Author(s)     : Menelaos Karavelas <mkaravel@cse.nd.edu>

\section{Examples}
\label{sec:sdg2-examples}

\subsection{First example}

The following example shows to use the segment Delaunay graph traits
in conjunction with the \ccc{Filtered_exact<CT,ET>} mechanism. In
addition it shows how to use a few of the iterators provided by the
\ccc{Segment_Delaunay_graph_2} class in order to count a few
site-related quantities.

\ccIncludeExampleCode{Segment_Delaunay_graph_2/sdg-count-sites.C}

\subsection{Second example}

The following example shows how to use the segment Delaunay graph
hierarchy along with the filtered traits class that supports
intersecting sites.

\ccIncludeExampleCode{Segment_Delaunay_graph_2/sdg-filtered-traits.C}

\subsection{Third example}

The following example demonstrates how to recover the defining sites
for the edges of the Voronoi diagram (which are the duals of the
edges of the segment Delaunay graph computed).

\ccIncludeExampleCode{Segment_Delaunay_graph_2/sdg-voronoi-edges.C}

