%\chapter{PostScript Stream} \label{Chapter_Postscript}

\chapter{The PS\_edge\_3 class}

\begin{ccClass} {PS_edge_3}

\section{Introduction}

The class \ccClassName\ is used to represent edges in 3D.

\ccSection{The PS\_edge\_3 Class}
\ccDefinition 

	An object of type \ccClassName\ is an edge in the space.
Indeed, this class is used to represent the edge in 3D. \\
A \ccClassName\ object is composed of two points of the \ccc{Point\_3} type which determines
the extremities of the edge. \\
Moreover, it is possible to specify the color of the edge and its
visibility. 
The visibility determines if the edge is \ccc{VISIBLE} or \ccc{INVISIBLE}. \\
Indeed, during the cutting of a facet we will introduce edge which cut a
face in two part, but we don't want to display this edge. So if an edge
is set to  \ccc{INVISIBLE}, the edge exist but will not be displayed. \\
By default, if the color and the attribut are not specified, they are
set to  \ccc{CGAL::BLACK} and  \ccc{VISIBLE}.

\ccInclude{CGAL/IO/PS_edge_3.h}

\ccTypes

\ccTypedef{typedef double coord_type;}{}
\ccGlue
\ccTypedef{typedef CGAL::Cartesian< coord_type > CT;}{}
\ccGlue
\ccTypedef{typedef CGAL::Point_3< R > Point3;}{}
\ccGlue
\ccTypedef{typedef CGAL::Aff_transformation_3< CT >
Transformation;.}{}
\ccGlue
\ccEnum   {enum Visibility {INVISIBLE,VISIBLE} ;}{}

\ccCreation
\ccCreationVariable{ps_edge_3}

\ccConstructor{PS_edge_3(Point3 &PA,Point3 &PB,Color
color=CGAL::BLACK,Visibility v=VISIBLE);} {Creates an edge by giving
the two Point\_3 \ccc{PA} and \ccc{PB}. The edge color is given by the 
 \ccc{color} parameter. The parameter  \ccc{v} allows to choose the visibility of
the edge. The parameters  \ccc{color} and  \ccc{v} are set by default to
 \ccc{CGAL::BLACK} and \ccc{VISIBLE}. Thus, if the user don't enter any 
color or visibility, they are set to the default value.}

\ccConstructor{PS_edge_3(const PS_edge_3& edge);} {This is the copy
constructor which is use to duplicate an existing edge.}

\ccAccessFunctions
\par

\ccMethod{inline Visibility visibility() const;}{Return the visibility of the 
edge that is to say if the edge is  \ccc{VISIBLE} or  \ccc{INVISIBLE}.}

\ccMethod{inline Color color() const;}{Return the edge color.}

\ccMethod{inline Point3 first_point() const;}{Return the first point of 
the edge.}

\ccMethod{inline Point3 second_point() const;}{Return the second point of 
the edge.}


\ccOperations

\ccMethod{PS_edge_3 operator= (PS_edge_3& edge);}{This is an
assignment operator.}

\ccFunction{friend ostream& operator<< (ostream& os, const PS_edge_3&
ps_edge_3);}{This operator allows to display the settings of the edge
on a terminal. \\
The display is the next : [(PointA),(PointB)] Color: 255 0 0 | VISIBLE.\\
It displays the two points of the edge, the color in \ccc{RGB} and the visibility of the edge.}

\ccFunction{double xmin();}{Return the minimum according to the x
coordinate of the two points of the edge.}

\ccFunction{double ymin();}{Return the minimum according to the y
coordinate of the two points of the edge.}

\ccFunction{double zmin();}{Return the minimum according to the z
coordinate of the two points of the edge.}

\ccFunction{double xmax();}{Return the maximum according to the x
coordinate of the two points of the edge.}

\ccFunction{double ymax();}{Return the maximum according to the y
coordinate of the two points of the edge.}

\ccFunction{double zmax();}{Return the maximum according to the z
coordinate of the two points of the edge.}

\ccFunction{void transformation(Transformation &t);}{Function to
transform the two points of the edge according the transformation t;}

\par
\par

\ccHeading{Setting functions}
\par

\ccFunction{inline void set_visibility(Visibility v);}{Allows to
change the visibility(\ccc{VISIBLE} or  \ccc{INVISIBLE}) of the edge.}

\ccFunction{inline void set_color(Color c);}{Allows to change the edge 
color by the new color \ccc{c}.}

\ccFunction{inline void set_first_point(Point3 pa);}{Allows to change the
coordinates of the first point of the edge.}

\ccFunction{inline void set_second_point(Point3 pb);}{Allows to change the
coordinates of the second point of the edge.}

\par

\end{ccClass}




