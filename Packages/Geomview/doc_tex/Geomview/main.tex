\newcommand{\Section}[1]{Section~{\protect\ref{#1}}}
\newcommand{\Chapter}[1]{Chapter~{\protect\ref{#1}}}
\newcommand{\new}[1]{\marginpar{\sf #1}}

\newcommand{\what}{\mbox{?\hspace*{2cm}?}}


\parindent0em
\setlength{\parskip}{1ex minus 0.9ex}
\sloppy

\newcommand{\note}[1]{{\bf NOTE: #1}}

\chapter{Geomview} \label{ChapterGeomview}


\section{Introduction}

This chapter presents the \cgal\ interface to geomview, which is a
viewer for three-dimensional objects, developed at the Geometry Center
in Minneapolis\footnote{\tt http://www.geom.umn.edu/}.

\noindent {\bf Important:} The last line in the startup file \ccc{.geomview}
must be \ccc{(echo "started")}. The two commands \ccc{rsh} and \ccc{geomview}
must be in the user's path. Otherwise the program will not be able to execute.

\begin{ccClass}{CGAL_Geomview_stream}
\ccDefinition
An object of the class \ccClassName\ is a stream in which geometric
objects can be insrted and where geometric objects can be extracted
from. The constructor starts geomview either on the local either on
a remote machine.

\ccInclude{CGAL/IO/Geomview_stream.h}

\ccSetThreeColumns{CGAL_Oriented_side}{}{\hspace*{10cm}}
\ccThreeToTwo
\ccCreation
\ccCreationVariable{G}

\ccConstructor{CGAL_Geomview_stream(const CGAL_Bbox_3 &bbox 
	                                 = CGAL_Bbox_3(0,0,0, 1,1,1),
				    const char *machine = NULL,
                                    const char *login = NULL);}
{Introduces a geomview stream \ccVar\ with a camera that sees the
bounding box. If machine and login are non-null, geomview is started
on the remote machine.}



\ccOperations


\ccHeading{Output Operators for CGAL Kernel Classes}

At the moment not all classes of the \cgal\ kernel have output
operators. 2D objects are embedded in the \ccc{xy}-plane.

\ccFunction{template <class R>
            CGAL_Geomview_stream&
            operator<<(CGAL_Geomview_stream& G,
                       const CGAL_Point_2<R>& p);}
{Inserts the point \ccc{p} into the stream \ccVar.}


\ccFunction{template <class R>
            CGAL_Geomview_stream&
            operator<<(CGAL_Geomview_stream& G,
                       const CGAL_Point_3<R>& p);}
{Inserts the point \ccc{p} into the stream \ccVar.}


\ccFunction{template <class R>
            CGAL_Geomview_stream&
            operator<<(CGAL_Geomview_stream& G,
                       const CGAL_Segment_2<R>& s);}
{Inserts the segment \ccc{s} into the stream \ccVar.}

\ccFunction{template <class R>
            CGAL_Geomview_stream&
            operator<<(CGAL_Geomview_stream& G,
                       const CGAL_Segment_3<R>& s);}
{Inserts the segment \ccc{s} into the stream \ccVar.}


\ccFunction{template <class R>
            CGAL_Geomview_stream&
            operator<<(CGAL_Geomview_stream& G,
                       const CGAL_Triangle_2<R>& t);}
{Inserts the triangle \ccc{t} into the stream \ccVar.}


\ccFunction{template <class R>
            CGAL_Geomview_stream&
            operator<<(CGAL_Geomview_stream& G,
                       const CGAL_Triangle_3<R>& t);}
{Inserts the triangle \ccc{t} into the stream \ccVar.}


\ccFunction{template <class R>
            CGAL_Geomview_stream&
            operator<<(CGAL_Geomview_stream& G,
                       const CGAL_Tetrahedron_3<R>& t);}
{Inserts the tetrahedron \ccc{t} into the stream \ccVar.}



\ccMethod{CGAL_Geomview_stream&
          operator<<(const CGAL_Bbox_2& b);}
{Inserts the bounding box \ccc{b} into the stream \ccVar.}

\ccMethod{CGAL_Geomview_stream&
          operator<<(const CGAL_Bbox_3& b);}
{Inserts the bounding box \ccc{b} into the stream \ccVar.}


\ccHeading{Input Operators for CGAL Kernel Classes}

At the moment input is only provided for points. The user has to select
a point on the {\it pickplane} with the right mouse button. The pickplane
can be moved anywhere with the left mouse button, before a point is entered.

\ccFunction{template <class R>
            CGAL_Geomview_stream&
            operator>>(CGAL_Geomview_stream& G,
                       CGAL_Point_3<R>& p);}
{Extracts the point \ccc{p} from the stream \ccVar. The point is
 echoed.}


\ccHeading{Output Operators for CGAL Basic Library Classes}

\ccFunction{template <class Traits, class HDS>
CGAL_Geomview_stream&
operator<<(CGAL_Geomview_stream &G,
           const CGAL_Polyhedron_3<Traits,HDS> &P);}
{Inserts the polyhedron \ccc{P} into the stream \ccVar.}

%% There is no tetrahedralization class at the moment
%% 
%% \ccFunction{template <class Traits>
%%             CGAL_Geomview_stream&
%%             operator<<(CGAL_Geomview_stream& G,
%%                        const CGAL_Tetrahedralization_3<Traits>& T);}
%% {Inserts the tetrahedralization \ccc{T} into the stream \ccVar.}
%% 
%% For geometric structures as a tetrahedalization only one geomview
%% object is created. If you want to select a single simplex you have
%% to insert the simplices individually with the following operator.
%% 
%% \ccFunction{template <class V>
%%             CGAL_Geomview_stream&
%%             operator<<(CGAL_Geomview_stream& G,
%%                        const CGAL_Tetrahedralization_simplex<V>* s);}
%% {Inserts the tetrahedralization simplex \ccc{s} into the stream \ccVar.}
%% 
%% 
%% \ccFunction{template < class V >
%%             CGAL_Geomview_stream&
%%             operator>>(CGAL_Geomview_stream &G,
%%                        CGAL_Tetrahedralization_simplex<V>*& s);}
%% {Assigns the pointer to the tetrahedralization simplex the user selected
%%  with the right mouse button to variable \ccc{s}. 
%% \ccPrecond It is in the responsibility of the user to guarantee that the
%% pointer still points to memory allocated for a simplex.}
%% 
%% 
%% \ccFunction{template <class P>
%%             CGAL_Geomview_stream&
%%             operator<<(CGAL_Geomview_stream& G,
%%                        const CGAL_Tetrahedralization_vertex<P>* v);}
%% {Inserts the tetrahedralization vertex \ccc{v} into the stream \ccVar.
%%  Vertices are drawn as a cube with twice \ccc{G.get_vertex_radius}
%%  as side length. This value is 1/100'th of the \ccc{x}-length of the
%% initial bounding box.}
%% 
%% 
%% \ccFunction{template < class P>
%%             CGAL_Geomview_stream&
%%             operator>>(CGAL_Geomview_stream &G,
%%                        CGAL_Tetrahedralization_vertex<V>*& v);}
%% {Assigns the pointer to the tetrahedralization vertex the user selected
%%  with the right mouse button to variable \ccc{v}. 
%% \ccPrecond It is in the responsibility of the user to guarantee that the
%% pointer still points to memory allocated for a vertex.}
%% 
%% end of commented section

\ccHeading{Colors}

geomview distinguishes between edge and face colors. The edge color
is at the same time the color of vertices. 

\ccMethod{CGAL_Geomview_stream&
            operator<<(const CGAL_Color& c);}
{Makes \ccc{c} the color of vertices, edges and faces in subsequent IO 
 operations.}

\ccMethod{CGAL_Color
          set_bg_color(const CGAL_Color& c);}
{Changes the background color. Returns the old value.}

\ccMethod{CGAL_Color
          set_vertex_color(const CGAL_Color& c);}
{Changes the vertex color. Returns the old value.}

\ccMethod{CGAL_Color
          set_edge_color(const CGAL_Color& c);}
{Changes the edge color. Returns the old value.}


\ccMethod{CGAL_Color
          set_face_color(const CGAL_Color& c);}
{Changes the face color. Returns the old value.}


\ccHeading{Miscellaneous}

\ccMethod{void
	  clear();}
{Deletes all objects.}

\ccMethod{void
	  look_recenter();}
{Positions the camera in a way that all objects can be seen.}

\ccMethod{int
          get_line_width() const;}
{Returns the line width.}

\ccMethod{int
          set_line_width(int w);}
{Sets the line width to \ccc{w}. Returns the previous value.}

\ccMethod{double
          get_vertex_radius() const;}
{Returns the radius of vertices.}

\ccMethod{double
          set_vertex_radius(double r) const;}
{Sets the radius of vertices to \ccc{d}. Returns the previous value.}

\ccMethod{bool
          set_trace(bool b);} 
{Sets tracing on. The data that are sent to \ccc{geomview} are also
 sent to \ccc{cerr}.  Returns the previous value. By default tracing is
 off.}

\ccMethod{bool
          get_trace();}
{Returns \ccc{true} iff tracing is on.}

\begin{ccAdvanced}

\ccHeading{Advanced and Developers Features}

The following functions are helpful if you develop your own insert
and extract functions. The following functions allow to pass a string
from geomview and to read data sent back by geomview.

\ccMethod{CGAL_Geomview_stream&
            operator<<(const char* s);}
{Inserts string \ccc{s} into the stream.}


\ccMethod{CGAL_Geomview_stream&
            operator>>(char* s);}
{Extracts a string \ccc{s} from the stream.
\ccPrecond You have to allocate enough memory.}

\ccMethod{CGAL_Geomview_stream&
            operator<<(int i);}
{Inserts integer \ccc{i} into the stream. Puts whitespace around if the
stream is in ascii mode.}


\ccMethod{CGAL_Geomview_stream&
            operator<<(double d);}
{Inserts double \ccc{d} into the stream. Puts whitespace around if the
stream is in ascii mode.}

\ccMethod{bool in_ascii_mode();}
{Returns \ccc{true} iff \ccVar\ is in ascii mode.}

\ccMethod{bool in_binary_mode();}
{Returns \ccc{true} iff \ccVar\ is in binary mode.}

\ccMethod{void set_binary_mode();}{}
\ccMethod{void set_ascii_mode();}{}

For convenience we offer the manipulators \ccc{ascii} and \ccc{binary}
that can be inserted in the stream.

\ccMethod{CGAL_Geomview_stream&
            operator<<(ascii);}
{Sets the stream in ascii mode.}

\ccMethod{CGAL_Geomview_stream&
            operator<<(binary);}
{Sets the stream in binary mode.}

\end{ccAdvanced}

%\ccExample

\begin{cprog}

\end{cprog} 

\ccImplementation

The constructor forks a process and establishes two pipes between the
processes. The forked process is then overlayed with geomview. The
file descriptors \ccc{stdin} and \ccc{stdout} of geomview are hooked
on the two pipes.

All insert operators construct expressions in \ccc{gcl}, the geomview
comand language, which is a subset of {\sc Lisp}. These expressions
are sent to geomview via the pipe. The extract operators notify \ccc{interest}
for a certain kind of events. When such an event happens geomview
sends a description of the event in gcl and the extract operator has
to parse this expression.

In order to implement further insert and extract operators you should
take a look at the implementation \cite{GV_Impl} and at the geomview
manual \cite{GVManual}.
\end{ccClass}

\begin{thebibliography}{10}

\bibitem{GVManual}
Mark Phillips et al.
\newblock {\em Geomview Manual}.
\newblock Version 1.6.1 for Unix Workstations.
\newblock December 10, 1996.


\bibitem{GV_Impl}
Andreas Fabri.
\newblock {How to Implement Geomview Stream IO Operators.}
\newblock July 1997. \ccc{http://www.cs.ru.nl/CGAL/}

\end{thebibliography}

