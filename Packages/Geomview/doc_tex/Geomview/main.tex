\newcommand{\Section}[1]{Section~{\protect\ref{#1}}}
\newcommand{\Chapter}[1]{Chapter~{\protect\ref{#1}}}
\newcommand{\new}[1]{\marginpar{\sf #1}}

\newcommand{\what}{\mbox{?\hspace*{2cm}?}}


\parindent0em
\setlength{\parskip}{1ex minus 0.9ex}
\sloppy

\newcommand{\note}[1]{{\bf NOTE: #1}}

\chapter{Geomview} \label{ChapterGeomview}


\section{Definition}

This chapter presents the \cgal\ interface to Geomview\footnote{\tt
http://www.geomview.org/}, which is a
viewer for three-dimensional objects, originally developed at the Geometry
Center in Minneapolis\footnote{\tt http://www.geom.umn.edu/}.

Geomview 1.8.1 is required.

\noindent {\bf Note:} In releases up to and including 2.2, CGAL used to have
the following requirement : the last line in the startup file \ccc{.geomview}
must be \ccc{(echo "started")}.  This is no longer necessary.


An object of the class \ccc{Geomview_stream}\ is a stream in which geometric
objects can be inserted and where geometric objects can be extracted
from. The constructor starts Geomview either on the local either on
a remote machine.



Not all but most classes of the \cgal\ kernel have output
operators for the  \ccc{Geomview_stream}. 
2D objects are embedded in the \ccc{xy}-plane.
Input is only provided for points.
Polyhedron and @D and 3D triangulations  have output
operators for the  \ccc{Geomview_stream}. 


\section{Implementation}

The constructor forks a process and establishes two pipes between the
processes. The forked process is then overlaid with Geomview. The
file descriptors \ccc{stdin} and \ccc{stdout} of Geomview are hooked
on the two pipes.

All insert operators construct expressions in \ccc{gcl}, the Geomview
command language, which is a subset of {\sc Lisp}. These expressions
are sent to Geomview via the pipe. The extract operators notify \ccc{interest}
for a certain kind of events. When such an event happens Geomview
sends a description of the event in \ccc{gcl} and the extract operator has
to parse this expression.

In order to implement further insert and extract operators you should
take a look at the implementation and at the Geomview manual.


\section{Example}
The following program ouputs successively a 2D
  Delaunay triangulation (projected), a3D Delaunay, and terrain
from the set set of points.
\ccIncludeExampleCode{demo/Geomview/terrain.C}