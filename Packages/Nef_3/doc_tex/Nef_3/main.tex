% +------------------------------------------------------------------------+
% | CGAL User Manual: 
% +------------------------------------------------------------------------+
% |
% | 05.04.2004   Peter Hachenberger
% | 
\RCSdef{\Nef_polyhedronRev}{$Revision$}
\RCSdefDate{\Nef_polyhedronDate}{$Date$}
% +------------------------------------------------------------------------+

\ccParDims

\chapter{3D Nef Polyhedron}
\label{chapterNef_3}
\ccChapterRelease{\Nef_polyhedronRev. \ \Nef_polyhedronDate}\\
\ccChapterAuthor{Peter Hachenberger}

\minitoc

% +------------------------------------------------------------------------+
\section{Introduction}

A Nef polyhedron in 3-dimensional space is
any point set generated from a finite number of open halfspaces by set
complement and set intersection operations. It is also closed under 
all other binary set operations and under topological (unary) set operations. 

A 3D Nef polyhedron can be represented by the local pyramids of the minimal
elments of its incidence structure. Without going into to much detail, a local
pyramid essentially reflects the topologic and geometric situation at a certain
location in a point set. For finite polyhedra the minimal elements 
of the incidence structure are vertices only. This means, that it suffices to 
model the topologic and geometric situation of the vertices. For 
\ccc{Nef_polyhedron_3} the local pyramid of a vertex is represented by 
a planar Nef polyhedra embedded on a sphere.

% +------------------------------------------------------------------------+
\section{Infimaximal Box}
\label{sectionNef_3InfiBox}

Looking at infinte point sets, the minimal elements of the incidence structure 
does not have to be a vertex. To overcome this difference between finite and 
infinite point sets, we intersect Nef polyhedra with
bounding volume of size $[-R,R]^3$, where $R$ is a symbolical unspecified finte 
value, which is larger than the value of any concrete real number. As a result,
each 3D Nef polyhedron becomes finite although it might represent an infinite
point set. We call the boundary of the bounding volume the infimaximal box
~\cite{cgal:sm-iftml-00}.

Clipping lines and rays at the infimaximal box leads to non-standard points,
i.e. points whose coordinates are $-R$ or $R$ for at least one axis, and
linear functions $f(R)$ for the other coordinates. For this purpose, extended
kernels - \ccc{Extended_cartesian} and \ccc{Extended_homogeneous} - are provided. Their
design is the same as for other \cgal kernels, but the coordinates are
modeled by polynomials.

As long as any extended kernel is used, the full functionality provided 
by the class \ccc{Nef_polyhedron_3} is available. If another kernel is
used, which does not use polynomials to represent coordinates, it is not
possible to create Nef polyhedra which intersect with the infimaximal box.
We will refer to such Nef polyhedra as infinite or unbounded Nef polyhedra, 
since its boundary includes unbounded/infinte large planes. Note, that
there are bounded Nef polyhedra, which represent an infinite
point set, e.g. the complete 3D space.

% +========================================================================+%
\section{Regularized Set Operations}
% +========================================================================+



% +========================================================================+
\section{Example Programs}
% +========================================================================+

The following example gives a first impression of how to instantiate and use
\ccc{Nef_polyhedron_3}. The \cgal kernel \ccc{Cartesian} is used as the traits
class. The other \cgal kernels would also be appropriate as long as it uses an
exact number type like \ccc{Gmpz} or \ccc{leda_integer}. 

The example creates two Nef polyhedra - \ccc{N0} is the empty set, while 
\ccc{N1} is set of all points in the 3-dimensional space. The assertion assures
that the empty set is the complement of the complete space.

\ccIncludeExampleCode{Nef_3/simple.C}

% +------------------------------------------------------------------------+
\subsection{Construction and Composition}

This example shows the various constructors. There are only two
types of constructors. With the first one either all points of the 3-dimensional
space belong to the point set or all points do not belong to the point set. The
latter one also is the default constructor. The second type creates a halfsphace
defined by the plane bounding it. It is only available if an extended kernel is
used. This constructor has a second parameter which
specifies whether the defining plane belongs to the point set 
(\ccc{Nef_polyhedron::INCLUDED}) or not (\ccc{Nef_polyhedron::EXCLUDED}). The 
default value for this parameter is \ccc{Nef_polyhedron::INCLUDED}.

\ccIncludeExampleCode{Nef_3/construction.C}

% +------------------------------------------------------------------------+
\subsection{Comparing Nef polyhedra}

The point sets of two Nef polyhedra can be compared. The operators \ccc{==}, 
\ccc{!=},
\ccc{<=}, \ccc{>=}, \ccc{<} and \ccc{>} 
are provided. The first two test whether the point sets 
are equal or different, the others test whether one point set is a (proper) 
subset of the other one. The usage is shown in the following example:

\ccIncludeExampleCode{Nef_3/comparison.C}

% +------------------------------------------------------------------------+
\subsection{Boolean Set Operations}

As explained in the introduction, Nef polyhedra are closed under all boolean 
set operations. The class \ccc{Nef_polyhedron} provides functions and
operators for the most common ones: complement (\ccc{operator!}), union (\ccc{operator+}), 
difference (\ccc{operator-}), intersection (\ccc{operator*}) and 
symmetric difference (\ccc{operator^}). Additionally, the operators 
\ccc{*=}, \ccc{-=}, \ccc{*=} and \ccc{^=} are defined.

\ccIncludeExampleCode{Nef_3/set_operations.C}

% +------------------------------------------------------------------------+
\subsection{Unary Operations}

Additionally to boolean set operations, \ccc{Nef_polyhedron_3} provides
the topological operations \ccc{interior()}, \ccc{closure()} and 
\ccc{boundary()}. \ccc{interior()} deselcts all boundary items, 
\ccc{boundary()} deselects all volumes and \ccc{closure()} selects all
boundary items. 

The following example shows the usage of unary operations and some relations 
between them.

\ccIncludeExampleCode{Nef_3/topological_operations.C}

% +------------------------------------------------------------------------+
\subsection{Using an Extended Kernel}

The provided extended kernels are used the same way as any other \cgal kernel.
The essential difference is, that coordinates are not represented by the number
type that was used to parametrize the kernel type, but by a \ccc{Nef_polynomial}
parametrized by that number type.

The example iterates all vertices of a given Nef polyhedron and decides whether
it is an standard vertex or a vertex on the infimaximal box. Furthermore, it 
tests whether any of the vertices is at (R,R,R). 

\ccIncludeExampleCode{Nef_3/extended_kernel.C}

% +========================================================================+
\section{File I/O}
% +========================================================================+
\label{sectionNef_3IO}

3D Nef polyhedra provide input and output functions for two interfaces for file
input and output. First, there is an interface for converting polyhedral surfaces
represented by the \cgal class \ccc{Polyhedron} into a \ccc{Nef_polyhedron}. The 
conversion in the opposite direction is also supported. \ccc{Polyhedron} can read
the ({\tt .off}) file format and can write the ({\tt .off}),
OpenInventor ({\tt .iv}), VRML 1.0 and 2.0 ({\tt .wrl}) and Wavefront Advanced
Visualizer object format ({\tt .obj}), see Section~\ref{sectionPolyIO}.
 
Both conversions between \ccc{Polyhedron} and \ccc{Nef_polyhedron_3} can only 
be done if the the object represents a two-manifold. \ccc{Nef_polyhedron_3}
provides the function \ccc{is_simple()}, which test for this property. The
usage is illustrated by the following example program:

\ccIncludeExampleCode{Nef_3/offIO.C}

Additionally, \ccc{Nef_polyhedron_3} provides an input and an output operator 
for a propriatary file format. Note, that it is not possible to read an unbounded
Nef polyhedron, if no extended kernel is used. On the other side, it is
possible to write a bounded Nef polyhedron using an extended kernel and 
reading it afterwards using a simple kernel. Furthermore, the output depends
on whether one uses a homogeneous or a cartesian kernel. It is recommended 
to use homogeneous kernels only if input and output is used.

\ccIncludeExampleCode{Nef_3/nefIO.C}

% +------------------------------------------------------------------------+
\subsection{Exploring Shells}

A \emph{shell} of a Nef polyhedron is connected part of the surface incident
to a certain volume. Each Halffacet, SFace and SHalfedge belongs to a single
shell. \ccc{Nef_polyhedron_3} offers a visitor interface to explore a shell.
The Interface is illustrated by the following example.

\ccIncludeExampleCode{Nef_3/shell_exploration.C}

The function \ccc{visit_shell_objects(SFace_const_handle sf, Visitor& V} 
explores a shell starting at the \ccc{sf}. The second argument expects any
class providing the (possibly empty) functions \ccc{visit(Vertex_const_handle)},
\ccc{visit(Halfedge_const_handle)} (remember that Halfedge is the same type as
SVertex), 
\ccc{visit(Halffacet_const_handle)}, \ccc{visit(SHalfedge_const_handle)}, 
\ccc{visit(SHalfloop_const_handle)} and \ccc{visit(SFace_const_handle)}. 
\ccc{visit_shell_objects} will call \ccc{visit} for each item belonging
to the shell once. There are no further requirements on that class. 

In the example, the class \ccc{Shell_explorer} is passed as second argument
to \ccc{visit_shell_objects}. Its task is to find the lexicographically
smallest vertex of a shell. Its internal state consists of three variables. 
The first one is a reference to the explored Nef polyhedron. This reference
is necessary often to retrieve information from the Nef polyhedron. The
second variable \ccc{v_min} stores the smallest vertex found so far, and
the third variable \ccc{first} is initialized to \ccc{false} to signal that no
vertex has been visited so far. After the first vertex has been visited 
\ccc{first} is changed to \ccc{true}.

\ccc{Shell_explorer} provides to further member functions. After the exploration
of a shell \ccc{minimal_vertex} retrieves the smallest vertex. \ccc{reset_minimal_vertex} allows one to use the same instance of 
 \ccc{Shell_explorer} on multiple shells. In this case, \ccc{reset_minimal_vertex}
has to be called between the exploration of two shells.

The example program uses the \ccc{Shell_explorer} for each shell of the given
Nef polyhedron once and reports the smallest veritces to standard output.

% +------------------------------------------------------------------------+
\subsection{Point Location}

The function \ccc{locate(Point_3 p)} locates the point \ccc{p} in the 
Nef polyhedron and returns the item the point belongs to. \ccc{locate}
return an instance of \ccc{Object_handle}, which is a generic handle
type representing any handle type, no matter if it is mutable or const. 
For futher usage of the result, the \ccc{Object_handle} has to be casted 
to the handle type represented. The static \cgal function \ccc{assign}
helps performing the cast. It returns a boolean signalising the success
of the cast. Here, \ccc{Object_handle} can represent a 
\ccc{Vertex_const_handle}, a \ccc{Halfedge_const_handle}, 
a \ccc{Halffacet_handle}, or a \ccc{Volume_const_handle}. If neither of
the four casts work, something went wrong.

\ccIncludeExampleCode{Nef_3/point_location.C}

% +------------------------------------------------------------------------+
\subsection{Visualization}

With the class \ccc{Nef_Visualizor_OpenGL_3} an interface to OpenGL 
visualization is offered. The usage is rather easy as can be seen in the
following example.

\ccIncludeExampleCode{Nef_3/visualization_SNC.C}

% +------------------------------------------------------------------------+
\subsection{Interface between \ccc{Nef_polyhedron_3} and \ccc{Nef_polyhedron_S2}}

It can be interesting to perform algorithms written for instances of 
\ccc{Nef_polyhedron_S2} on a sphere map of a \ccc{Nef_polyhedron_3}. For this
purpose \ccc{Nef_polyhedron_3} provides the function 
\ccc{get_sphere_map(Vertex_const_handle)} which returns the sphere map of
the specified vertex as a \ccc{Nef_polyhedron_S2}. Note, that one has to use
the type \ccc{Nef_polyhedron_S2} as specified in \ccc{Nef_polyhedron_3} as 
is shown in the following example.

\ccIncludeExampleCode{Nef_3/visualization_SM.C}

%% EOF %%