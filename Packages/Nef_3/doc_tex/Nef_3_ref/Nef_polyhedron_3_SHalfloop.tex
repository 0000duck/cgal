% +------------------------------------------------------------------------+
% | Reference manual page: SHalfloop.tex
% +------------------------------------------------------------------------+
% | 14.05.2004   Peter Hachenberger
% | Package: Nef_3
% | 
\RCSdef{\RCSSHalfloopRev}{$Revision$}
\RCSdefDate{\RCSSHalfloopDate}{$Date$}
% +------------------------------------------------------------------------+

\ccRefPageBegin

%%RefPage: end of header, begin of main body
% +------------------------------------------------------------------------+


\begin{ccRefClass}[Nef_polyhedron_3<Traits>::]{SHalfloop}

\ccDefinition

A shalfloop is a great circle on a sphere map. 
Figure~\ccTexHtml{\ref{figureNef3HalfloopIncidences} on page 
\pageref{figureNef3HalfloopIncidences}}{}\begin{ccHtmlOnly}
  <A HREF="#figureNef3HalfedgeIncidences"><IMG 
  SRC="cc_ref_up_arrow.gif" ALT="reference arrow" WIDTH="10" HEIGHT="10"></A>
\end{ccHtmlOnly}
depicts the relationship between a shalfloop and its incident
shalfloops, and sfaces on a sphere map.  A shalfloop is 
an oriented sloop. It is always paired with a 
shalfloop whose supporting \ccc{Sphere_circle} is pointing in
the opposite direction. The \ccc{twin()} member function returns
this shalfloop of opposite orientation.

\begin{ccTexOnly}
    \begin{figure}[bht]
        \begin{center}
          \parbox{0.3\textwidth}{%
              \includegraphics[width=0.3\textwidth]{Nef_3_ref/fig/shalfloopB}%
          }
        \end{center}
        \label{figureNef3HalfloopIncidences}
    \end{figure}
\end{ccTexOnly}

\begin{ccHtmlOnly}
    <CENTER>
    <A NAME="figureNef3SHalfloopIncidences">
    <A HREF="fig/shalfloopB.gif">
        <img src="fig/shalfloopB.gif" 
             alt="Incidences of an SHalfloop"></A><BR>
    <A HREF="fig/shalfloopB.gif">Figure:</A>
    </CENTER>
\end{ccHtmlOnly}

A sphere map having a shalfloop models the neighborhood of a vertex which is 
isolated on a facet. That facet is returned by the member function
\ccc{incident_facet}. 

\ccInclude{CGAL/Nef_polyhedron_3.h}

\ccTypes
\ccThree{SHalfloop_const_handle}{incident_facet() const;;}{}
\ccThreeToTwo

The following types are the same as in \ccc{Nef_polyhedron_3<Traits>}.

\ccNestedType{Mark}{type of mark.}

\ccNestedType{Sphere_circle}{sphere circle type stored in SHalfloop.}

\ccNestedType{Halffacet_const_handle}{const handle to Halffacet.}
\ccGlue
\ccNestedType{SHalfloop_const_handle}{const handle to SHalfloop.}
\ccGlue
\ccNestedType{SFace_const_handle}{const handle to SFace.}

\ccCreation
\ccCreationVariable{se}

There is no need for a user to create a \ccc{SHalfloop} explicitly. The
class \ccc{Nef_polyhedron_3<Traits>} manages the needed shalfloops internally.

%\ccConstructor{SHalfloop();}{default constructor.}

\ccOperations

\ccMethod{const Mark& mark() const;}{the mark of \ccVar\ .}

\ccMethod{const Sphere_circle& circle() const;}{the sphere circle of \ccVar\ .}

\ccMethod{SHalfloop_const_handle twin() const;}{the twin of \ccVar\ .}

\ccMethod{Halffacet_const_handle incident_facet() const;}{the incident facet of \ccVar\ .}

\ccMethod{SFace_const_handle incident_sface() const;}{the incident sface of \ccVar\ .}

\ccSeeAlso

\ccRefIdfierPage{CGAL::Nef_polyhedron_3<Traits>::Halffacet}\\
\ccRefIdfierPage{CGAL::Nef_polyhedron_3<Traits>::SFace}\\
\ccRefIdfierPage{CGAL::Nef_polyhedron_S2<Traits>::Sphere_point}

\ccTagDefaults
\end{ccRefClass}

% +------------------------------------------------------------------------+
%%RefPage: end of main body, begin of footer
\ccRefPageEnd
% EOF
% +------------------------------------------------------------------------+
