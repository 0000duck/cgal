% +------------------------------------------------------------------------+
% | Reference manual page: Polyhedron_operator_leftshift.tex
% +------------------------------------------------------------------------+
% | 05.04.2004   Peter Hachenberger
% | Package: Nef_3
% | 
\RCSdef{\RCSNefpolyhedron3operatoristreamRev}{$Revision$}
\RCSdefDate{\RCNefpolyhedron3operatoristreamDate}{$Date$}
% |
%%RefPage: end of header, begin of main body
% +------------------------------------------------------------------------+

\ccHtmlNoClassLinks
\begin{ccRefFunction}{operator<<}
\label{refNef_polyhedron_operator_leftshift_3}

\ccDefinition

This operator writes the Nef polyhedron $N$ to the output stream \ccc{out} using
a propriatary file format. The output is in ASCII format. It includes the complete
incidence structure, the geometric data, and the marks of each item.

If the output is stored in a file, the Nef polyhedron can be restored from 
that file using the input operator. For this purpose, a standard output mode 
has to be chosen. Therefore, use the \cgal \ccc{set_ascii_mode} modifier and 
do not use the \ccc{set_pretty_mode} modifier.

The output of the geometric data depends on the used kernel, i.e. the ASCII 
output of vertex coordinates depends on the output operator of the $Point$ 
type. Finite Nef polyhedron can be represented with a normal as well as with an
extended kernel. Therefore, the output operator checks for finiteness. If the 
Nef polyhedron is finite but an extended kernel is used, the operator converts 
the geometric data and writes the data as if a normal kernel is used.

\ccInclude{CGAL/IO/Nef_polyhedron_iostream.h}

\ccGlobalFunction{template <class Nef_polyhedronTraits_3>
    ostream& operator<<( ostream& out, 
                         const CGAL::Nef_polyhedron_3<Nef_polyhedronTraits_3>& N);}
  
\ccSeeAlso

\ccRefIdfierPage{CGAL::Nef_polyhedron_3<Traits>}\\ 
\lcTex{\ccc{operator>>} \dotfill\ 
    page~\pageref{refNef_polyhedron_operator_rightshift_3}}%
\lcRawHtml{
    <I><A HREF="Function_operator--.html">operator&gt;&gt;</A></I>.
}

\end{ccRefFunction}

% +------------------------------------------------------------------------+
%%RefPage: end of main body, begin of footer
% EOF
% +------------------------------------------------------------------------+

