% +------------------------------------------------------------------------+
% | Reference manual page: Halfedge.tex
% +------------------------------------------------------------------------+
% | 14.05.2004   Peter Hachenberger
% | Package: Nef_3
% | 
\RCSdef{\RCSHalfedgeRev}{$Revision$}
\RCSdefDate{\RCSHalfedgeDate}{$Date$}
% +------------------------------------------------------------------------+

\ccRefPageBegin

%%RefPage: end of header, begin of main body
% +------------------------------------------------------------------------+


\begin{ccRefClass}[Nef_polyhedron_3<Traits>::]{Halfedge}

\ccDefinition
 
A Halfedge has a double meaning. In the global incidence structure of a
\ccc{Nef_polyhedron_3} it is an oriented edge going from one vertex to another.
A halfedge also coincides with an svertex of the sphere map of its source 
vertex. Because of this, we offer the types \ccc{Halfedge} and \ccc{SVertex}
which are the same. Furthermore, the redundant functions \ccc{center_vertex()}
and \ccc{source()} are provided. The reason is, that we get the same vertex
either if we want to have the source vertex of a halfedge, or if we want to 
have the vertex in the center of the sphere map a svertex lies on. 
Figure~\ccTexHtml{\ref{figureNef3HalfedgeIncidences} on page 
\pageref{figureNef3HalfedgeIncidences}}{}\begin{ccHtmlOnly}
  <A HREF="#figureNef3HalfedgeIncidences"><IMG 
  SRC="cc_ref_up_arrow.gif" ALT="reference arrow" WIDTH="10" HEIGHT="10"></A>
\end{ccHtmlOnly} and
figure~\ccTexHtml{\ref{figureNef3FacetIncidences} on page 
\pageref{figureNef3FacetIncidences}}{}\begin{ccHtmlOnly}
  <A HREF="#figureNef3HalffacetIncidences"><IMG 
  SRC="cc_ref_up_arrow.gif" ALT="reference arrow" WIDTH="10" HEIGHT="10"></A>
\end{ccHtmlOnly} illustrate the incidence of a svertex on a sphere map and of
a halfedge in the global structure. 

As part of the global incidence structure, the member fuctions \ccc{source} 
and \ccc{target} return the source and target vertex of an edge. The member
function \ccc{twin()} returns the opposite halfedge.

Looking at the incidence structure on a sphere map, the member function 
\ccc{out_sedge} returns the first outgoing shalfedge, and \ccc{incident_sface}
returns the the incident sface.

\ccInclude{CGAL/Nef_polyhedron_3.h}

\ccTypes
\ccThree{Halfedge_const_handle}{h.halfedge() const;;}{}
\ccThreeToTwo

The following types are the same as in \ccc{Nef_polyhedron_3<Traits>}.

\ccNestedType{Mark}{type of mark.}

\ccNestedType{Sphere_point}{sphere point type stored in Halfedge.}

\ccNestedType{Vertex_const_handle}{const handle to vertex.}
\ccGlue
\ccNestedType{Halfedge_const_handle}{const handle to halfedge.}
\ccGlue
\ccNestedType{SHalfedge_const_handle}{const handle to SHalfedge.}
\ccGlue
\ccNestedType{SFace_const_handle}{const handle to SFace.}

\ccCreation
\ccCreationVariable{e}

There is no need for a user to create a \ccc{SHalfedge} explicitly. The
class \ccc{Nef_polyhedron_3<Traits>} manages the needed shalfedges internally.

%\ccConstructor{Halfedge();}{default constructor.}

%\ccConstructor{Halfedge(Mark m);}
%{creates Halfedge with an initial value for its mark.}

\ccOperations

\ccMethod{const Mark& mark() const;}{the mark of \ccVar\ .}

\ccMethod{const Sphere_point& point() const;}{the sphere point of \ccVar\ .}

\ccMethod{bool is_isolated() const;}
{returns |true| if \ccVar\ has no adjacent sedges.}

\ccMethod{Vertex_const_handle center_vertex() const;}
{the center vertex of the sphere map \ccVar\ belongs to.}

\ccMethod{Vertex_const_handle source() const;}
{the source vertex of \ccVar\ .}

\ccMethod{Vertex_const_handle target() const;}
{the target vertex \ccVar.}

\ccMethod{Halfedge_const_handle twin() const;}{the twin of \ccVar\ .}

\ccMethod{SHalfedge_const_handle out_sedge() const;}
{the first out sedege of \ccVar\ .}

\ccMethod{SFace_const_handle incident_sface() const;}
{the incident sface of \ccVar\ .}

\ccSeeAlso

\ccRefIdfierPage{CGAL::Nef_polyhedron_3<Traits>::Vertex}\\
\ccRefIdfierPage{CGAL::Nef_polyhedron_3<Traits>::SHalfedge}\\
\ccRefIdfierPage{CGAL::Nef_polyhedron_3<Traits>::SFace} \\
\ccRefIdfierPage{CGAL::Nef_polyhedron_S2<Traits>::Sphere_point}

\ccTagDefaults
\end{ccRefClass}

% +------------------------------------------------------------------------+
%%RefPage: end of main body, begin of footer
\ccRefPageEnd
% EOF
% +------------------------------------------------------------------------+
