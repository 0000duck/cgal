% Interval Arithmetic documentation.
%   $Revision$
%   $Date$

\begin{ccClass} {Interval_nt}
\ccSubsection{Interval Arithmetic}
\label{interval}
% \ccChapterAuthor{Sylvain Pion}

This section describes briefly what interval arithmetic is, its implementation
in {\cgal}, and its possible use by geometric programs.
The main reason for having interval arithmetic in {\cgal} is its integration
into the filtered robust and fast predicates scheme, but we also provide a
number type so that you can use it separately if you find any use for it
(such as interval analysis, or to represent data with tolerance).

% This control is
% provided via the comparison operators.  When there is an unsafe comparison
% between intervals, either a warning or an exception (see below) is raised.
% Otherwise, it means that if you do the same computation with doubles, your
% program will have the correct output (with the exact same input data).

The purpose of interval arithmetic is to provide an efficient way to bound
the roundoff errors made by floating point computations.
You can choose the behaviour of your program depending on these errors; that
is what is done for the filtered robust predicates (see Section~\ref{filter}).
You can find more theoretical information on this topic in
\cite{bbp-iayea-98}.
% \cite{bbp-iayed-98scg}. % worked here, but changed by Remco.

\ccDefinition
Interval arithmetic is a large concept and we will only consider here a 
simple arithmetic based on intervals whose bounds are {\it double}s.
So each variable is an interval representing any value inside the interval.
All arithmetic operations (+, -, $*$, $/$, $\sqrt{}$, \ccc{square()},
\ccc{min()}, \ccc{max()} and \ccc{abs()}) on intervals preserve the inclusion.
This property can be expressed by the following formula ($x$ and $y$ are
reals, $X$ and $Y$ are intervals, $\mathcal{OP}$ is an arithmetic operation):

$$
\forall\ x \in X, \forall\ y \in Y, (x\ \mathcal{OP}\ y)
\in (X\ \mathcal{OP}\ Y)
$$

For example, if the final result of a sequence of arithmetic operations is
an interval that does not contain zero, then you can safely determine its sign.

\ccInclude{CGAL/Interval_arithmetic.h}

\ccCreation

\ccConstructor{Interval_nt(double d);}
{introduces the interval [\ccc{d};\ccc{d}].}

\ccConstructor{Interval_nt(double i, double s);}
{introduces the interval [\ccc{i};\ccc{s}].}


\ccOperations
\ccSetThreeColumns{Interval_nt}{}{\hspace*{8cm}}

All functions required by a class to be considered as a {\cgal} number type
(see~\ref{Numbertype}) are present, as well as the utility functions,
sometimes with a particular semantic which is described below.  There are also
a few additional functions.

\ccCreationVariable{I}

\ccMethod{Interval_nt operator/(Interval_nt J);} {returns
[$-\infty$;$+\infty$] when the denominator contains 0.}

\ccFunction{Interval_nt sqrt(Interval_nt I);} {returns
[0;$\sqrt{upper\_bound(I)}$] when only the lower bound is negative (expectable
case with roundoff errors), and is unspecified when the upper bound also is
negative (unexpected case).}

\ccFunction{double to_double(Interval_nt I);} {returns the
middle of the interval, as a double approximation of the interval.}

\ccMethod{double inf();} {returns the lower bound of the interval.}
\ccMethod{double sup();} {returns the upper bound of the interval.}
\ccMethod{bool is_point();} {returns whether both bounds are equal.}
\ccMethod{bool is_same(Interval_nt J);} {returns whether both intervals have
the same bounds.}
\ccMethod{bool do_overlap(Interval_nt J);} {returns whether both intervals
have a non empty intersection.}

The two following operators can be used for interval analysis, they are not
directly useful for \ccc{Interval_nt} as a number type.

\ccMethod{Interval_nt operator||(Interval_nt J);} {returns the smallest
interval containing the two intervals.}
\ccMethod{Interval_nt operator&&(Interval_nt J);} {returns the biggest
interval contained in the two intervals.  The result is unspecified if
the two intervals don't intersect.}


The comparison operators ($<$, $>$, $<=$, $>=$, $==$, $!=$, \ccc{sign()}
and \ccc{compare()}) have the following semantic: it is the intuitive
one when for all couples of values in both intervals, the comparison
is identical (case of non-overlapping intervals).  This can be expressed
by the following formula ($x$ and $y$ are reals, $X$ and $Y$ are
intervals, $\mathcal{OP}$ is a comparison operator):

$$
\left(\forall x \in X, \forall y \in Y, (x\ \mathcal{OP}\ y) = true\right)
\Rightarrow (X\ \mathcal{OP}\ Y) = true
$$
and
$$
\left(\forall x \in X, \forall y \in Y, (x\ \mathcal{OP}\ y) = false\right)
\Rightarrow (X\ \mathcal{OP}\ Y) =false
$$

Otherwise, the comparison is not safe, and we first increment the counter
\ccc{Interval_nt_advanced::number_of_failures}, and then throw the exception
\ccc{Interval_nt_advanced::unsafe_comparison}.

\ccImplementation

\ccc{Interval_nt} derives from \ccc{Interval_nt_advanced}.
The operations on \ccc{Interval_nt} are automatically protected against
rounding modes, and are thus slower than those on \ccc{Interval_nt_advanced},
but easier to use.

Users that need performance are encouraged to use \ccc{Interval_nt_advanced}
instead (see Section~\ref{interval-adv}).

\end{ccClass}

\begin{ccAdvanced}
\samepage
\begin{ccClass} {Interval_nt_advanced}
\ccSubsection {Advanced Interval Arithmetic}
\label{interval-adv}

The advanced class allows you to make faster computations, but you need to set
the rounding mode of the FPU to 'round to infinity' (see below for how to do
that) before doing any computation with this number type, and each function
(arithmetic operators and conversion functions)
leaves the rounding mode in this state if it needs to modify it internally.

Changing the rounding mode affects all floating point computations, and might
cause problems with parts of your code, or external libraries (even \cgal),
that expect the rounding mode to be the default (round to the nearest).

\ccInclude{CGAL/Interval_arithmetic.h}

\ccSetThreeColumns{Interval_nt_advanced}{}{\hspace*{8.5cm}}
\ccFunction{Interval_nt_advanced convert_to<Interval_nt_advanced>(NT);}
{Several such functions provide a cast from the following numerical types
to an \ccc{Interval_nt_advanced} containing the value.  The following types are
supported: \ccc{leda_real}, \ccc{leda_rational}, \ccc{leda_integer},
\ccc{leda_bigfloat}, \ccc{Gmpz}, \ccc{Fixed_precision_nt}, all built-in
types that fit exactly in a \ccc{double} (which excludes \ccc{long double}
and 64 bits integers, which must be treated separately), and all
\ccc{Quotient<RT>} where RT is a type listed above.
The user can add such functions for his own number types, see the files
\ccc{CGAL/Interval_arithmetic/IA_*.h} for examples.}

% FPU rounding modes~:

We provide the following interface to change the rounding mode:

\ccTypedef{typedef int FPU_CW_t;}
       {The type used by the following functions to deal with rounding modes.
       This is usually an \ccc{int}.}
\ccFunction{void FPU_set_cw (FPU_CW_t R);}
       {sets the rounding mode to \ccc{R}.}
\ccFunction{FPU_CW_t FPU_get_cw (void);}
       {returns the current rounding mode.}
\ccFunction{FPU_CW_t FPU_get_and_set_cw (FPU_CW_t R);}
       {sets the rounding mode to \ccc{R} and returns the old one. }

The macros \ccc{CGAL_FE_TONEAREST}, \ccc{CGAL_FE_TOWARDZERO},
\ccc{CGAL_FE_UPWARD} and \ccc{CGAL_FE_DOWNWARD} are the values corresponding
to the rounding modes.

\ccExample

The correct way to protect an area of code that uses operations on the class
\ccc{Interval_nt_advanced} is the following:

\begin{verbatim}
  FPU_CW_t backup = FPU_get_and_set_cw(CGAL_FE_UPWARD);
  ... // The code to be protected.
  FPU_set_cw(backup);
\end{verbatim}

\ccImplementation

The basic idea is to use the directed rounding modes specified by the 
{\it IEEE 754} standard, which are implemented by almost all processors 
nowadays.
It states that you have the possibility, concerning the basic floating point
operations ($+,-,*,/,\sqrt{}$) to specify the rounding mode of each operation
instead of using the default, which is set to 'round to the nearest'.
This feature allows us to compute easily on intervals.  For example, to
add the two intervals [a.i;a.s] and [b.i;b.s], compute $c.i=a.i+b.i$ rounded
towards minus infinity, and $c.s=a.s+b.s$ rounded towards plus infinity, and
the result is the interval [c.i;c.s].  This method can be extended easily to
the other operations.

The problem is that we have to change the rounding mode very often, and the
functions of the C library doing this operation are slow and not portable.
That's why assembly versions are used as often as possible.
Another trick is to store the opposite of the lower bound, instead of the
lower bound itself, which allows us to never change the rounding mode inside
simple operations.  Therefore, all basic operations, which are in the class 
\ccc{Interval_nt_advanced} assume that the rounding mode is set to 
'round to infinity', and everything works with this correctly set.  
The class \ccc{Interval_nt} takes care of this, but is a bit slower.

So, if the user needs the speed of \ccc{Interval_nt_advanced}, he must
take care of setting the rounding mode to 'round to infinity' before each
block of operations on this number type.  And if other operations might be
affected by this, he must take care to reset it to 'round to the nearest'
before they are executed.

% Note also that NaNs are not handled, so be careful with that
% (especially if you `divide by zero').

Platform support:\\
This part of {\cgal} must be explicitly ported to each non yet supported
platform.  The list of currently tested platforms is:
\begin{itemize}
\item Intel: GCC under Linux and CygWin, and Visual C++ and Borland C++ under
      Windows.
\item Sparc: GCC under Solaris.
\item PowerPC: GCC under Linux.
\item Mips: GCC and MipsPro 7.3 under IRIX 6.
\end{itemize}
It is possible that more platforms actually work, but they have not been
tested enough.

Notes:\\
\begin{itemize}
\item On Intel platforms (with any operating system and compiler), due to a
      misfeature of the floating point unit, which does not handle exactly
      IEEE compliant operations on doubles, we are forced to use a workaround
      which slows down the code, but is only useful when the intervals can
      overflow or underflow.  If you know that the intervals will never
      overflow nor underflow for your code, then you can disable this
      workaround with the flag \ccc{CGAL_IA_NO_X86_OVER_UNDER_FLOW_PROTECT}.
      Other platforms are not affected by this flag.
\item When optimizing, compilers usually propagate the value of variables when
      they know it's a constant.  This can break the interval routines because
      the compiler then does some floating point operations on these constants
      with the default rounding mode, which is wrong.  This kind of problem
      is avoided by stopping constant propagation in the interval routines.
      However, this solution slows down the code and is rarely useful, so you
      can disable it by setting the flag
      \ccc{CGAL_IA_DONT_STOP_CONSTANT_PROPAGATION}.
      Note that GCC offers a feature (\ccc{__builtin_constant_p}) which allows
      to stop constant propagation at no cost for the general case.
      A simple way to stop this optimization in a particular case is to put
      the constant in a global and non static variable, instead of a local
      one.
\item By default, all arithmetic operations over intervals are inline.  This
      can slow down some large predicates, because the code of the
      multiplication is large.  You can turn off inlining of the
      multiplication and division with the flag \ccc{CGAL_IA_NO_INLINE}.
      On the other hand, this slows down the small predicates, which is why
      it is not enabled for all cases.
      You might want to tune this for your particular application.
\end{itemize}

\end{ccClass}

\end{ccAdvanced}


\begin{ccClassTemplate}{Filtered_exact<CT, ET>} %, Cache_t = No_Filter_Cache>}
\ccSubsection{Robust filtered predicates}
\label{filter}

\ccInclude{CGAL/Filtered_exact.h}

The class \ccc{Filtered_exact<CT,ET>} is a wrapper type for the number
type \ccc{CT}, with the difference
that all predicates are specialized such that they are guaranteed to be exact.
Speed is achieved via a filtering scheme using interval arithmetic (see
Section~\ref{interval-adv}).  Here are the necessary requirements:

\begin{itemize}
\item \ccc{CT} is the construction and storage type.  The only data member of
      the class \ccc{Filtered_exact<CT,ET>} is the \ccc{value} of type
      \ccc{CT}.  All arithmetic operations performed {\em outside} of the
      predicates will be executed with this number type.  You can disallow
      these operations compiling with the flag
      \ccc{CGAL_DENY_INEXACT_OPERATIONS_ON_FILTER} (it allows you
      to spot the inexact operations that should be incorporated in the
      predicates).
      The arithmetic operations called from inside the predicates are always
      computed exactly.
\item The \ccc{ET} type must be able to compute exactly the operations involved
      in the predicates called.
\item A \ccc{to_interval(CT)} function must
      be provided, that returns an interval containing the value of the
      argument of type \ccc{CT}, see Section~\ref{interval}.
\item A \ccc{convert_to<ET>(CT)} function must also be provided, that
      returns a number of type \ccc{ET} representing exactly the argument of
      type \ccc{CT}.  It's a conversion function that is used for the exact
      computation, when the filter fails.
      This conversion has to be done exactly to ensure robustness.
\end{itemize}

The following member functions are used to access the numerical value for the
different number types:

\ccMethod{CT value();} {returns the wrapped value.}
\ccMethod{ET exact();} {returns the converted value to \ccc{ET}.}
\ccMethod{Interval_nt_advanced interval();} {returns the converted value
                                             to \ccc{Interval_nt_advanced}.}

% For efficiency reasons, the conversion of the \ccc{value} to an
% \ccc{Interval_nt_advanced} may be cached using \ccc{Filter_Cache} as
% optionnal third template parameter of the class.  This is only wishful if
% \ccc{CT} is a complicated type for which the conversion to an interval takes
% a somewhat long time, and is asked for the same variable several times.  You
% might want to experiment with this feature, depending on your needs.

This type actually has additional parameters for experimental features.  They
will be documented when they will be considered stable, in a next release.

\ccExample

You might use at the beginning of your program a \ccc{typedef} as follows:

\begin{verbatim}
    #include<CGAL/Filtered_exact.h>
    #include<CGAL/leda_real.h>
    #include<CGAL/double.h>
    typedef Filtered_exact<double, leda_real> NT;
\end{verbatim}

Or if you are sure that the predicates involved do not use divisions nor
square roots:

\begin{verbatim}
    #include<CGAL/Filtered_exact.h>
    #include<CGAL/Gmpz.h>
    #include<CGAL/int.h>
    typedef Filtered_exact<int, Gmpz> NT;
\end{verbatim}

And if you know that the double variables contain integer values, you can
use:

\begin{verbatim}
    #include<CGAL/Filtered_exact.h>
    #include<CGAL/Gmpz.h>
    #include<CGAL/double.h>
    typedef Filtered_exact<double, Gmpz> NT;
\end{verbatim}

As a general rule, we advise the use of
\ccc{Filtered_exact<double, leda_real>}.

% If you need exact computations also outside the predicates, you can use the
% following:
% 
% \begin{verbatim}
    % #include<CGAL/Filtered_exact.h>
    % #include<CGAL/lera_real.h>
    % typedef Filtered_exact<leda_real, leda_real, Filter_Cache> NT;
% \end{verbatim}


\ccImplementation

The template definition of the low level predicates of {\cgal} are overloaded
for the type \ccc{Filtered_exact<CT,ET>}.  It is a partial specialisation,
which implies that this is not supported by the compilers that do not support
correctly this C++ feature (for which we set \ccc{CGAL_CFG_MATCHING_BUG_2}).
A workaround is in place that allows you to use
one such filtered number type at once, by explicitely fully specializing the
predicates for it.  To do so, just define the macros \ccc{CGAL_IA_CT} and
\ccc{CGAL_IA_ET} to the corresponding number types before any inclusion of
{\cgal} files, as well as \ccc{CGAL_IA_CACHE}.  For example~:

\begin{verbatim} 
#define CGAL_IA_CT double
#define CGAL_IA_ET leda_real
#define CGAL_IA_CACHE No_Filter_Cache
\end{verbatim}

For each predicate file, the overloaded code is generated automatically by a
\ccc{PERL} script (\ccc{scripts/filtered_predicates_generator.pl}) that you can
use for your own predicates.
This script parses the template functions and generates the
overloaded code the following way:
\begin{itemize}
\item convert the entries to intervals using
    \ccc{to_interval(CT)}, via the \ccc{interval()} member function,
\item call the original template function with the type
    \ccc{Interval_nt_advanced},
\item if no exception is thrown, return the value,
\item if an exception is thrown (the filter failed), convert the original
    entries using \ccc{convert_to<ET>(CT)}, using the \ccc{exact()} member
    function,
\item and call the original template function with the type \ccc{ET}.
\end{itemize}

\ccExample

The low level template predicates of {\cgal} are in files named
\ccc{CGAL/predicates/kernel_ftC2.h} (resp. ftC3), the script is used
to produce the files \ccc{CGAL/Arithmetic_filter/predicates/kernel_ftC2.h}
(resp. ftC3).

For the moment, only the predicates of the Cartesian kernel are supported,
as well as the power tests used by the regular triangulations.

\end{ccClassTemplate}
