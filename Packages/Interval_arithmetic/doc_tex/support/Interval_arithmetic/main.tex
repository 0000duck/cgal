\begin{ccClass} {CGAL_Interval_nt}
\ccSubsection{Interval Arithmetic}
\label{interval}
%\ccChapterAuthor{Sylvain Pion}

This chapter describes briefly what interval arithmetic is, its implementation
in {\cgal}, and its possible use by geometric programs.
The main reason for having interval arithmetic in {\cgal} is its integration
into a geometric predicate pre-compiler, but we also provide a number type so
that the users can use it separately if they find any use for it.
The purpose of interval arithmetic is to provide an efficient way to
control the errors made by floating point computations.

This type is nothing more than doubles, but you can, when you ask for it, have
a control on the computation error.  By default, when the flag
\ccc{CGAL_IA_NO_WARNINGS} is not defined, warnings will be emitted if you try
to compare two overlapping intervals.  This means that if you do the same
computation with doubles, there is a chance that the comparison may return a
false result, otherwise the computation with doubles will be safe.

You can select the behaviour of the algorithm in such cases, when you program
filtered predicates for example.  You can find more information on this topic
in \cite{bbp-ia-98}.

\ccDefinition
Interval arithmetic is a large concept and we will only consider here a 
simple arithmetic based on intervals whose bounds are {\it double}s.
So each variable is an interval representing any value inside the interval.
All arithmetic operations made on intervals give a result that contains all
possible values.
For example, if the final result of a sequence of arithmetic operations is
an interval that does not contain zero, then you can surely decide the sign 
of the result.

\ccInclude{CGAL/Interval_arithmetic.h}

\ccCreation

\ccConstructor{CGAL_Interval_nt(const double& d = 0);} {introduces an interval
whose bounds are \ccc{d}.}

\ccConstructor{CGAL_Interval_nt(const double& d, const double& e);}
{introduces the interval [d;e].}

Several functions should provide a cast from any numerical type used by {\cgal},
to an interval \ccc{CGAL_Interval_nt}.


%\ccTypes

%The two following classes provide the same functionality (in fact, the
%first one derives from the second one).

%\ccTypedef{class CGAL_Interval_nt;} {Represents the interval number type.}

%\ccTypedef{class CGAL_Interval_nt_advanced;} {Is faster but requires more 
%work from the user.}

\ccOperations
\ccSetThreeColumns{CGAL_Interval_nt}{}{\hspace*{8cm}}

All functions required by a class to be considered a {\cgal} number type are
present, sometimes with a particular semantic, notably~:

\ccCreationVariable{I}

\ccMethod{CGAL_Interval_nt operator/(CGAL_Interval_nt J);} {is defined with the
following meaning for the degenerate case where the denominator is an interval
containing zero: it returns the interval ]$-\infty$;$+\infty$[ (the `division
by zero' exception can be raised, but the default is to ignore it, see next
paragraph for details).}

\ccFunction{CGAL_Interval_nt sqrt(const CGAL_Interval_nt& I);} {is defined and
produces an error if the lower bound of the interval is a negative number.}

\ccMethod{bool operator<(CGAL_Interval_nt J);} {returns {\it true} when any
value in $I$ is inferior to any value in $J$.}

\ccMethod{bool operator==(CGAL_Interval_nt J);} {returns {\it true} when $I$
and $J$ overlap (or intersect).}

\ccMethod{bool operator>(CGAL_Interval_nt J);} {is defined logically from the
previous ones}
\ccMethod{bool operator<=(CGAL_Interval_nt J);} {idem}
\ccMethod{bool operator>=(CGAL_Interval_nt J);} {idem}
\ccMethod{bool operator!=(CGAL_Interval_nt J);} {idem}

\ccFunction{double CGAL_to_double(const CGAL_Interval_nt& I);} {returns the
middle of the interval, as a double approximation of the interval.}

\ccMethod{double lower_bound();} {returns the lower bound of the interval.}
\ccMethod{double upper_bound();} {returns the upper bound of the interval.}

\ccImplementation

\ccc{CGAL_Interval_nt} derives from \ccc{CGAL_Interval_nt_advanced}, 
see Section~\ref{interval-adv}.

\ccExample

Here a simple example is given showing how to make safe and fast predicates.
This kind of problem is going to be resolved automatically by a predicate
compiler, but before it is available, you can do it by hand.
Let's consider the \ccc{CGAL_orientation} predicate:

\begin{verbatim}
template < class FT >
CGAL_Orientation CGAL_orientation(const FT &px, const FT &py,
                                  const FT &qx, const FT &qy,
                                  const FT &rx, const FT &ry)
{
  FT d1 = (px-qx)*(qy-ry);
  FT d2 = (py-qy)*(qx-rx);
  return  (d1 < d2) ? CGAL_RIGHTTURN
       : ((d1 == d2)? CGAL_COLLINEAR : CGAL_LEFTTURN);
}
\end{verbatim}

This function becomes the following, if you want it to be exact~:

\begin{verbatim}
#include <CGAL/Interval_arithmetic.h>
typedef CGAL_Interval_nt Dis;

CGAL_Orientation CGAL_orientation(const FT &px, const FT &py,
                                  const FT &qx, const FT &qy,
                                  const FT &rx, const FT &ry)
{
  Dis d1 = ((Dis)px-(Dis)qx)*((Dis)qy-(Dis)ry);
  Dis d2 = ((Dis)py-(Dis)qy)*((Dis)qx-(Dis)rx);

  if (d1 == d2) {
    return CGAL_orientation((leda_real) px, (leda_real) py,
			    (leda_real) qx, (leda_real) qy,
			    (leda_real) rx, (leda_real) ry);
  } else {
    return (d1 < d2) ? CGAL_RIGHTTURN : CGAL_LEFTTURN;
  };
}
\end{verbatim}

\end{ccClass}

\begin{ccAdvanced}

\begin{ccClass} {CGAL_Interval_nt_advanced}
\ccSubsection {Advanced Interval Arithmetic}
\label{interval-adv}

\ccc{CGAL_Interval_nt} derives from \ccc{CGAL_Interval_nt_advanced}.
It allows you to make faster computations, but you need to set the correct
rounding mode of the FPU (towards infinity) before doing any computation with
this number type.  
See for details above, and in the test programs.

\ccInclude{CGAL/Interval_arithmetic.h}

\ccImplementation

The basic idea is to use the directed rounding modes specified by the 
{\it IEEE 754} standard, which are implemented by almost all processors 
nowadays.
It states that you have the possibility, concerning the basic floating point
operations ($+,-,*,/,\sqrt{}$) to specify the rounding mode of each operation
instead of using the default, which is usually to round to the nearest.
This feature allows us to compute easily on intervals.  For example, to add
the two intervals [a.i;a.s] and [b.i;b.s], compute $c.i=a.i+b.i$ rounded
towards minus infinite, and $c.s=a.s+b.s$ rounded towards plus infinite, and
the result is the interval [c.i;c.s].  This method can be extended easily to
the other operations.

The problem is that we have to change the rounding mode very often, and the
functions of the C library doing this operation are slow and different from
one architecture to another.  That's why assembly versions are used as often
as possible.
Another trick is to compute the opposite of the lower bound, instead of the
normal lower bound, which allows us to change the rounding mode once less.
Therefore, all basic operations, which are in the class 
\ccc{CGAL_Interval_nt_advanced} assume that the rounding mode is set to 
positive infinite, and everything works with this correctly set.  
The class \ccc{CGAL_Interval_nt} takes care of this, but is a bit slower.

Note also that NaNs are not handled, so be careful with that
(especially if you `divide by zero').

FPU rounding modes~:

We provide the following functions (in C or assembly) to change the 
rounding mode.
\begin{itemize}
\item \ccc{CGAL_FPU_set_rounding_to_zero()}
\item \ccc{CGAL_FPU_set_rounding_to_nearest()}
\item \ccc{CGAL_FPU_set_rounding_to_infinity()}
\item \ccc{CGAL_FPU_set_rounding_to_minus_infinity()}
\end{itemize}

Platform support~:

This part of {\cgal} must be ported to each non-supported platform, because
of the use of directed rounding modes.  
Here is the current state of supported platforms:

\begin{itemize}
\item Sparc
%\begin{itemize}\itemsep0pt\topsep0pt\partopsep0pt\parskip0pt\parsep0pt
\begin{itemize}
\item g++ uses the assembly versions, independent of the OS (Solaris tested).
\item CC (Solaris) uses the C library functions to change the rounding modes.
\end{itemize}
\item Intel
\begin{itemize}
\item g++ uses the assembly versions, independent of the OS (tested for Linux).
\item For Windows (NT, 95), Solaris, and others the code is not (yet) ported.
\end{itemize}
\item Alpha
\begin{itemize}
\item g++ 2.8 supports the "-mfp-rounding-mode=d" command line option
      to set the dynamic rounding mode.
\item For CC the situation is unknown (try command line option like "-fprm d").
\end{itemize}
\item Mips
\begin{itemize}
\item g++ and CC use the C library functions.
\end{itemize}
\end{itemize}


%\newpage

%\subsection{Files}

%The files involved are~:
%\begin{itemize}
%\item {\it CGAL/Interval\_arithmetic.h} contains the definition of the
%      two classes.
%\item {\it CGAL/\_FPU.h} contains the C and assembly definitions for the
%      rounding mode selection.
%\end{itemize}

%Note that symbols in \_FPU.h are all prefixed by {\it CGAL\_FPU\_}.

\end{ccClass}

\end{ccAdvanced}
