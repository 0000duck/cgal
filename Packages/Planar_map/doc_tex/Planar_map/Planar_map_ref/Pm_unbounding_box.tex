% +------------------------------------------------------------------------+
% | Reference manual page: Pm_unbounding_box (Planar_map)
% +------------------------------------------------------------------------+
% | 
% | Package: pm (Planar_map_2)
% | 
% +------------------------------------------------------------------------+

\ccRefPageBegin

%%RefPage: end of header, begin of main body
% +------------------------------------------------------------------------+

\begin{ccRefClass}{Pm_unbounding_box<Planar_map>}
\label{Pm_ref:unbounding_box}

\ccDefinition 
    The \ccRefName\ class
    implements the naive approach that deals only with bounded curves.
    The curves are inserted into the map as is. 
    This approach is highly efficient though limited to only some of the 
    traits.

    This is the default strategy used by \ccc{Planar_map_2<Dcel, Traits>}.

\ccThree{Curve}{}{\hspace*{10cm}}
\ccThreeToTwo

\ccInclude{CGAL/Pm_unbounding_box.h}

\ccIsModel
  \ccc{PlanarMapBoundingBox_2}

\ccInheritsFrom
  \ccc{Pm_bounding_box_base<Planar_map>}

%\ccCreationVariable{pl}

%\ccConstructor{Pm_unbounding_box_location<Planar_map>();}
%{}
  
\ccSeeAlso
   Discussion of the different bounding box strategies in the introduction
of \ccc{Planar_map} reference pages\lcTex{ (\ccRefPage{Pm_Ref_intro})}.

\end{ccRefClass} % Pm_unbounding_box

% +------------------------------------------------------------------------+
%%RefPage: end of main body, begin of footer
\ccRefPageEnd
% EOF
% +------------------------------------------------------------------------+

