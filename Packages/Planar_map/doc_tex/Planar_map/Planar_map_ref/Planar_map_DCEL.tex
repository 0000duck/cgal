% +------------------------------------------------------------------------+
% | Reference manual page: Planar_map_DCEL.tex (Planar_map)
% +------------------------------------------------------------------------+
% | 
% | Package: pm (Planar_map_2)
% | 
% +------------------------------------------------------------------------+

\ccRefPageBegin

%%RefPage: end of header, begin of main body
% +------------------------------------------------------------------------+

\begin{ccRefConcept}{Planar_map_DCEL}

A planar map class is parameterized with the interface classes 
\ccStyle{Dcel} and \ccStyle{Traits} . The \ccStyle{Dcel} class is
a container class that defines the 
underlying combinatorial data structure (halfedge data structure) 
used by the planar map. 
The \ccStyle{Traits} class defines the abstract interface
between planar maps and the {\em geometric} primitives they use. 

The
% \ccc{\ccRefName} does no convert to html. The converter add \ccc by 
% default to \ccRefName
%{\ccFont \ccRefName}
\lcTex{\ccc{\ccRefName}}\lcHtml{\ccRefName}
% \ccTexHtml{\ccc{\ccRefName}}{\ccRefName} <== Does not work.
concept is a refinement of the \ccc{Topological_map_DCEL}. Hence, the requirements for the latter are requirements for the first. 
In addition 
the 
\ccHtmlNoLinksFrom{\ccStyle{Vertex} and \ccStyle{Halfedge}}
classes for the planar map are
required to have geometric functionality. 
The exact requirements of those are stated in the following two sections.
\end{ccRefConcept} % Planar_map_DCEL

% +------------------------------------------------------------------------+
%%RefPage: end of main body, begin of footer
\ccRefPageEnd
% EOF
% +------------------------------------------------------------------------+
