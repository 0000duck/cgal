% +------------------------------------------------------------------------+
% | Reference manual page: Pm_bounding_box_base.tex (Planar_map)
% +------------------------------------------------------------------------+
% | 
% | Package: pm (Planar_map_2)
% | 
% +------------------------------------------------------------------------+

\ccRefPageBegin

%%RefPage: end of header, begin of main body
% +------------------------------------------------------------------------+

%**************************************************************************
\renewcommand{\ccRefPageBegin}{\begin{ccAdvanced}}
\renewcommand{\ccRefPageEnd}{\end{ccAdvanced}}

\begin{ccRefConcept}{Planar_map_bounding_box}
% \begin{ccRefClass}{Pm_bounding_box_base<Planar_map>}
\label{Pm_ref:bounding_box_base}
%\begin{ccAdvanced}
\ccDefinition
    \ccRefName\ is the bounding box strategy used by \ccc{Planar_map_2<Dcel, Traits>}.

    In order for the advanced user to write a model of a~\ccRefName\ the user's class 
    should inherit the \cgal\/ supplied class 
    \ccc{Pm_bounding_box_base<Planar_map>}.
    This class is an abstract base class for the 
    different bounding box strategies and it provides the interface for any bounding box
    strategy. 

    A bounding box strategy is a class derived
    from \ccc{Pm_bounding_box_base<Planar_map>} that implements the (\ccc{pure virtual})
    member functions which are the requirements of the \ccRefName\ concept.
   
    \ccc{Planar_map_2<Dcel, Traits>} is the
    type that aggregates the strategy. The return values of the queries
    are given in terms of the  \ccc{Planar_map_2<Dcel, Traits>} handles.

\ccInclude{CGAL/Pm_bounding_box_base.h}

\ccThree{typedef typename Planar_map::Halfedge_handle }{}{\hspace*{6cm}}

\ccTypes
\ccTypedef{typedef typename Planar_map::Traits Traits;}{}
\ccTypedef{typedef typename Planar_map::Locate_type Locate_type;}{}
\ccTypedef{typedef typename Planar_map::Halfedge_handle Halfedge_handle;}{}

\ccTypedef{typedef typename Traits::X_curve X_curve;}{}
\ccTypedef{typedef typename Traits::Point Point;}{}

\ccThree{virtual Halfedge_handle}{}{\hspace*{10cm}}

\ccOperations
\ccCreationVariable{pl}
\ccMethod{virtual void init(Planar_map& pmp, Traits& tr);}{initializes
the strategy; will be called by the planar map directly after its 
construction, with \ccc{*this} and the traits instantiation inside the 
planar map, as its parameters.}

%\ccHeading{Query Functions}
\ccQueryFunctions

\ccMethod{virtual bool locate(const Point& p, Locate_type& lt,Halfedge_handle& h);}{}
\ccMethod{virtual bool vertical_ray_shoot(const Point& p, Locate_type& lt, bool up,Halfedge_handle& h);}{}
Reevaluates the resulting \ccc{Point} and \ccc{Locate_type}, while considering the bounding box.

%\ccHeading{Updating Functions}
\ccModifiers
\ccMethod{virtual bool insert(const X_curve& cv);}
{enlarge the bounding box (if possible, e.g., not static) to bound the curve.}

\ccMethod{virtual bool insert(const Point& p);}
{enlarge the bounding box (if possible, e.g., not static) to contain the point.}

%\ccMethod{virtual void split_edge(const X_curve &cv,Halfedge_handle e1,
%               Halfedge_handle e2,const X_curve& cv1, const X_curve& cv2;}
%{}
\ccMethod{virtual void split_boundary_edge(const Halfedge_handle &h,
                Halfedge_handle h1,
                Halfedge_handle h2,
                const Point& p);}
{Split the edge on the boundary.}

%\ccMethod{virtual void merge_edge(const X_curve &cv1,
%               const X_curve &cv2,
%               Halfedge_handle e 
%               ,const X_curve& cv
%               );}
%{Merges the curves }

\ccMethod{virtual void remove_edge(Halfedge_handle e);}
{Removes the curve from the bounding box.}

\ccMethod{virtual void clear();}
{Clears the bounding box.}

\ccHasModels
    \ccc{Pm_unbounding_box<Planar_map>}\\
    \ccc{Pm_dynamic_open_bounding_box<Planar_map>}

%\end{ccAdvanced} % bounding box
\end{ccRefConcept}

\renewcommand{\ccRefPageBegin}{}
\renewcommand{\ccRefPageEnd}{}

% +------------------------------------------------------------------------+
%%RefPage: end of main body, begin of footer
\ccRefPageEnd
% EOF
% +------------------------------------------------------------------------+
