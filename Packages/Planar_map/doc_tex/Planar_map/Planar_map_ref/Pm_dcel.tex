% +------------------------------------------------------------------------+
% | Reference manual page: Pm_dcel.tex (Planar_map)
% +------------------------------------------------------------------------+
% | 
% | Package: pm (Planar_map_2)
% | 
% +------------------------------------------------------------------------+

\ccRefPageBegin

%%RefPage: end of header, begin of main body
% +------------------------------------------------------------------------+

\begin{ccRefClass}{Pm_dcel<V,H,F>}

\ccDefinition

    The \ccClassTemplateName\ is a model of the Dcel concept
    defined above. Its template parameters are the vertex, halfedge and face
    classes which conform to the requirements given above.

\ccInclude{CGAL/Pm_default_dcel.h}

\ccIsModel
    \ccc{Planar_map_DCEL}

\ccHeading{Models for Dcel Vertex, Halfedge and Face}
The \ccStyle{Pm_vertex_base<Point>} is a model for the 
Dcel vertex concept defined above. 
The \ccStyle{Pm_halfedge_base<Curve>} is a
model for the Dcel halfedge concept defined above.
The \ccStyle{Pm_face_base} is a
model for the Dcel face concept defined above. 

\ccInclude{CGAL/Pm_default_dcel.h}

\ccHeading{The Default Dcel Structure}
The \ccStyle{Pm_default_dcel<Traits>} is a model of the Dcel concept
defined above. Its template parameter is the traits class defined below.
It is a wrapper for \ccStyle{Pm_dcel} instantiated 
with \ccStyle{Pm_vertex_base<Traits::Point>},
\ccStyle{Pm_halfedge_base<Traits::X_curve>} and 
\ccStyle{Pm_face_base}.

\ccInclude{CGAL/Pm_default_dcel.h}

\end{ccRefClass}
% +------------------------------------------------------------------------+
%%RefPage: end of main body, begin of footer
\ccRefPageEnd
% EOF
% +------------------------------------------------------------------------+
