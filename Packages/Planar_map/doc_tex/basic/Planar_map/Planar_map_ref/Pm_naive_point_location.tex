% +------------------------------------------------------------------------+
% | Reference manual page: Pm_naive_point_location.tex (Planar_map)
% +------------------------------------------------------------------------+
% | 
% | Package: pm (Planar_map_2)
% | 
% +------------------------------------------------------------------------+

\ccRefPageBegin

%%RefPage: end of header, begin of main body
% +------------------------------------------------------------------------+

%**************************************************************************
\renewcommand{\ccRefPageBegin}{\begin{ccAdvanced}}
\renewcommand{\ccRefPageEnd}{\end{ccAdvanced}}

\begin{ccRefClass}{Pm_naive_point_location<Planar_map>}
\label{PL_sec:naive}

The \ccRefName\ class 
implements the naive algorithm which goes over all the vertices and halfedges
of the planar map. Therefore, the time complexity of a query is linear in
the complexity of the planar map. Since it does not use an additional
structure, the updating operations are empty and therefore 
modifying operations such as 
\ccc{split_edge} and \ccc{merge_edge} take constant time.

\ccInclude{CGAL/Pm_naive_point_location.h}

\ccIsModel
  \ccc{Planar_map_point_location}

\ccInheritsFrom
  \ccc{Pm_point_location_base<Planar_map>}

\ccSeeAlso
   Discussion of the different point location strategies in the introduction
of \ccc{Planar_map} reference pages\lcTex{ (\ccRefPage{Pm_Ref_intro})}.

\end{ccRefClass}
\renewcommand{\ccRefPageBegin}{}
\renewcommand{\ccRefPageEnd}{}


% +------------------------------------------------------------------------+
%%RefPage: end of main body, begin of footer
\ccRefPageEnd
% EOF
% +------------------------------------------------------------------------+
