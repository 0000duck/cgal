% +------------------------------------------------------------------------+
% | Reference manual page: Pm_file_writer.tex (Planar_map)
% +------------------------------------------------------------------------+
% | 
% | Package: pm (Planar_map_2)
% | 
% +------------------------------------------------------------------------+

\ccRefPageBegin

%%RefPage: end of header, begin of main body
% +------------------------------------------------------------------------+

\renewcommand{\ccRefPageBegin}{\begin{ccAdvanced}}
\renewcommand{\ccRefPageEnd}{\end{ccAdvanced}}

%Pm_file_writer
\begin{ccRefClass}{Pm_file_writer<Planar_map>}

\ccCreationVariable{pm_file_writer}

\ccDefinition
The \ccRefName\ class is inherited from the \ccc{File_header} class. 
This class contains functions for writing the \ccc{Planar map} components to a given output stream.
Users have to make sure that methods for writing \ccStyle{Point} 
and \ccStyle{X\_curve} objects defined in the \ccc{Planar map traits} 
class are provided when using the output stream they choose to work with..
This class is templated with \ccc{Planar map}, and hence the user can work with \ccRefName\ 
with its own \ccc{Planar map}. 
 
%The following requirement catalog lists the primitives, i.e., types, member functions
%etc., that must be defined for any \ccc{Planar map} file writer class that can be used to parameterize the reading function of \ccc{Planar map}. 

\ccTypes

The following handles and iterators have appropriate constant counterparts. 
The mutable types are assignable to their
constant counterparts. The iterators are assignable to the respective handle types. 
Wherever the handles appear in function parameter lists, the appropriate iterator can be used as well. 

\ccNestedType{Vertex_handle}{handle to the \ccc{Planar map} vertex.}
\ccNestedType{Halfedge_handle}{handle to the \ccc{Planar map} halfedge.}
\ccNestedType{Face_handle}{handle to the \ccc{Planar map} face}.
\ccNestedType{Vertex_iterator}{a bidirectional iterator over the vertices of the \ccc{Planar map}.}
\ccNestedType{Halfedge_iterator}{a bidirectional iterator over the halfedges of the \ccc{Planar map}.}
\ccNestedType{Face_iterator}{a bidirectional iterator over the faces of the \ccc{Planar map}.}

\ccCreation
As described in the introduction, the \ccc{Planar map} users can choose a verbose format for 
printing their planar map, this is done by putting this option in the constructor.

\ccConstructor{Pm_file_writer(std::ostream& o, const PM& pm, bool verbose = false);} 
{constructs \ccRefName\ for a given output stream, the specified \ccc{Planar map} and a verbose choice.}

\ccConstructor{Pm_file_writer(std::ostream& o, const PM& pm, const File_header& h)} 
{constructs \ccRefName\ for a given output stream, the specified \ccc{Planar map} and {File header}.}

\ccOperations

In the following operations the methods for writing lines in 
comments are defined only if the comments flag is set to true:

\ccMethod{void write_title (const char *feature_name); }
{writes a title in comments to the output stream.}

\ccMethod{void write_pm_vhf_sizes ();}
{writes the number of vertices, halfedges and faces in the \ccc{Planar map} to the output stream.}

\ccMethod{void write_vertices_header ();}
{writes a title in comments that begins the vertices list to the output stream.}

\ccMethod{void write_halfedges_header ();}
{writes a title in comments that begins the halfedges list to the output stream.}

\ccMethod{void write_faces_header ();}
{writes a title in comments that begins the faces list to the output stream.}

\ccMethod{void write_comment (const char *str);}
{writes a line by comments with the information noted in {\em str} to the output stream.}

\ccMethod{void write_comment (const char *str, int i);}
{writes a line by comments with the information noted in {\em str} and {\em i} to the output stream.}

\ccMethod{void write_value (unsigned int val, 
                            char delimiter = newline);}
{writing an integer value followed by a delimiter to the output stream.
 The delimiter has a default value set to the newline character 
 '$\backslash$n'.}

\ccMethod{ostream& out();}
{returns the output stream the \ccc{Planar map} is written to.} 

\ccMethod{void write_vertex (Vertex_handle v);}
{writes a vertex of the \ccc{Planar map} to the output stream.}

\ccMethod{void write_halfedge (Halfedge_handle h);}
{writes an halfedge of the \ccc{Planar map} to the output stream.}

\ccMethod{void write_face (Face_handle f);}
{writes a face of the \ccc{Planar map} to the output stream.}

\ccMethod{void write_vertices (Vertex_iterator Vertices_begin, Vertex_iterator Vertices_end);}
{writes all the vertices in the range [Vertices\_begin, Vertices\_end) to the output stream.}

\ccMethod{void write_halfedges (Halfedge_const_iterator Halfedges_begin, Halfedge_const_iterator Halfedges_end);}
{writes all the halfedges in the range [Halfedges\_begin, Halfedges\_end) to the output stream.}

\ccMethod{void write_faces (Face_iterator Faces_begin, Face_iterator Faces_end);}
{writes all the faces in the range [Faces\_begin, Faces\_end) to the output stream.}

\end{ccRefClass} % Pm_file_writer

\renewcommand{\ccRefPageBegin}{}
\renewcommand{\ccRefPageEnd}{}

% +------------------------------------------------------------------------+
%%RefPage: end of main body, begin of footer
\ccRefPageEnd
% EOF
% +------------------------------------------------------------------------+











%%% Local Variables: 
%%% mode: latex
%%% TeX-master: "Pm_drawer"
%%% End: 
