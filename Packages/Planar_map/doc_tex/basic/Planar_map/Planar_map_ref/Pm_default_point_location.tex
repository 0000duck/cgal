% +------------------------------------------------------------------------+
% | Reference manual page: Pm_default_point_location.tex (Planar_map)
% +------------------------------------------------------------------------+
% | 
% | Package: pm (Planar_map_2)
% | 
% +------------------------------------------------------------------------+

\ccRefPageBegin

%%RefPage: end of header, begin of main body
% +------------------------------------------------------------------------+

%**************************************************************************

\renewcommand{\ccRefPageBegin}{\begin{ccAdvanced}}
\renewcommand{\ccRefPageEnd}{\end{ccAdvanced}}

\begin{ccRefClass}{Pm_default_point_location<Planar_map>}
\label{PL_sec:default}

\ccDefinition
The \ccRefName\ class
implements the incremental randomized algorithm introduced by
Mulmuley ~\cite{m-fppa-90} as presented by Seidel ~\cite{s-sfira-91},~\cite{bkos-cgaa-97}.
It uses a trapezoidal map and a search structure to 
achieve an expected query time of $O(\log n)$.
Updating of these structures takes expected $O(\log n)$ time. 
%Therefore,
%this is also a lower bound for every member function in the class.
There is also a possibility to assure a {\em deterministic} worst-case query 
time  of $O(\log n)$, with a preprocessing step. The trade-off is in a longer
building time.

\ccThree{Curve}{}{\hspace*{10cm}}
\ccThreeToTwo

\ccInclude{CGAL/Pm_default_point_location.h}

\ccCreationVariable{pl}

\ccIsModel
  \ccc{PlanarMapPointLocation_2}

\ccInheritsFrom
  \ccc{Pm_point_location_base<Planar_map>}

\ccCreation

\ccConstructor{Pm_default_point_location<Planar_map>(bool preprocess = false);}
{ constructs the point location object. If \ccc{preprocess} is \ccc{true} then
preprocessing is done to guarantee a deterministic worst case query time of
$O(\log n)$. However, the preprocessing step can slow down considerably
the building time of the structure.}

\ccSeeAlso
   Discussion of the different point location strategies in the introduction
of \ccc{Planar_map} reference pages\lcTex{ (\ccRefPage{Pm_Ref_intro})}.

\end{ccRefClass}
\renewcommand{\ccRefPageBegin}{}
\renewcommand{\ccRefPageEnd}{}

% +------------------------------------------------------------------------+
%%RefPage: end of main body, begin of footer
\ccRefPageEnd
% EOF
% +------------------------------------------------------------------------+
