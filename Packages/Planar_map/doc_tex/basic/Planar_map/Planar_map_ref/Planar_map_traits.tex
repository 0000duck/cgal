% +------------------------------------------------------------------------+
% | Reference manual page: Planar_map_traits.tex (Planar_map)
% +------------------------------------------------------------------------+
% | 
% | Package: pm (Planar_map_2)
% | 
% +------------------------------------------------------------------------+

\ccRefPageBegin

%%RefPage: end of header, begin of main body
% +------------------------------------------------------------------------+

\begin{ccRefConcept}{Planar_map_traits}
\label{PM_sec:req:interface} 
 
%\label{I1_S}} 

The planar map class is parameterized with the
interface class \ccStyle{Planar_map_traits} which defines the abstract interface
between planar maps and the primitives they use. 
The following
requirement catalog lists the primitives, i.e., types, member functions
etc., that must be defined for a Traits class that can be used to
parameterize planar maps. %Ready-made implementation is available
%and described in Section~\ref{PM_sec:def:interface}. They can be used as a
%starting point for customizing own interface classes, for example
%through derivation and specialization.

\ccCreationVariable{pm_traits}


\ccDefinition
    A class \ccClassTemplateName\ that satisfies the requirements of an
    interface class for a planar map class must provide the
    following types and operations. 

\ccTypes

    The geometric types defined below must have a default constructor,
    copy constructor, assignment operator and \ccStyle{operator==}.
    
    \ccNestedType{Point}{A type to hold the location of a vertex in the plane.}
    \ccNestedType{X_curve}{A type to hold an $x$-monotone curve in the plane.}
  
    \ccEnum{enum Curve_point_status {UNDER_CURVE = -1,
          CURVE_NOT_IN_RANGE ,
          ABOVE_CURVE ,
          ON_CURVE };} {a constant describing the relative position between
       a curve and a point.}


    %\ccNestedType{Curve}{A type to hold a curve in the plane. Not
    %   necessarily monotone. This type is used by the
    %   \ccStyle{Arrangement_2} class.}
    
%    \ccNestedType{Point_handle}{A handle of a \ccStyle{Point}.}
    
%    \ccNestedType{X_curve_handle}{A handle of an \ccStyle{X_curve}.}
    
    %\ccNestedType{Curve_handle}{A handle of \ccStyle{Curve}.}

    %\ccNestedType{FT}{- Field type associated with the geometric data. }
    

\ccCreation
    
%    Only a default constructor is required. Note that further
%    constructors can be provided.

    \ccConstructor{Traits();}
    {A default constructor.}

    \ccOperations
%    \ccSetTwoOfThreeColumns{4cm}{4cm}

%predicates
%----------
    \ccMethod{Comparison_result compare_x(const Point& p0,
       const Point& p1);} {compares the \ccStyle{x}-coordinates
       of $p0$ and $p1$.  Return \ccStyle{LARGER} if $x(p0) >
       x(p1)$, \ccStyle{SMALLER} if $x(p0) < x(p1)$, or else
       \ccStyle{EQUAL}.  }
    
    \ccMethod{Comparison_result compare_y(const Point& p0,
       const Point& p1);} {compares the
       \ccStyle{y}-coordinates of $p0$ and $p1$.}


    \ccMethod{bool curve_is_vertical(const X_curve& cv);} 
   {returns \ccStyle{true} if \ccStyle{cv} is a vertical segment.}
        
%redundant - we have curve point status ?
    \ccMethod{bool curve_is_in_x_range( const X_curve& cv,
       const Point& pnt );}{returns \ccStyle{true} if \ccStyle{pnt} is
       in the $x$ range of \ccStyle{cv}.}
    
    \ccMethod{Comparison_result curve_compare_at_x(
       const X_curve& cv1, const X_curve& cv2, const Point& pnt );}
    {compares the $y$-coordinate of $cv1$ and $cv2$ at the $x$-coordinate 
     of $pnt$. Return \ccStyle{EQUAL} if at least one of $cv1$ and $cv2$  
     is not defined at $pnt$'s $x$-coordinate.}
   
  \ccMethod{Comparison_result curve_compare_at_x_left(
     const X_curve& cv1, const X_curve& cv2, const Point& pnt);}
  {compares the $y$-coordinate of $cv1$ and $cv2$ immediately to the
     left of the $x$-coordinate of $pnt$. Return \ccStyle{EQUAL}
     if at least one of $cv1$ and $cv2$ is not defined to the left of $pnt$'s
     $x$-coordinate.}
    
  \ccMethod{Comparison_result
     curve_compare_at_x_right(const X_curve& cv1, const X_curve& cv2,
     const Point& pnt);} {compares the $y$-coordinate of $cv1$ and
     $cv2$ immediately to the right of the $x$-coordinate of $pnt$.
     Return \ccStyle{EQUAL} if at least one of $cv1$ and $cv2$ is
     not defined to the right of $pnt$'s $x$-coordinate.}

    
    \ccMethod{Curve_point_status curve_get_point_status(const X_curve& cv,
       const Point& pnt);} {returns a constant describing the
       relative position between $cv$ and $pnt$:
       \begin{itemize}
       \item[]\ccStyle{CURVE_NOT_IN_RANGE} - $pnt$ is not in the $x$-range of
       $cv$, 
%       \item[]\ccStyle{CURVE_VERTICAL} - $cv$ is a vertical segment,
       \item[]\ccStyle{UNDER_CURVE} - $pnt$ is below $cv$,
       \item[]\ccStyle{ABOVE_CURVE} - $pnt$ is above $cv$, and
       \item[]\ccStyle{ON_CURVE} - $pnt$ is on $cv$.
       \end{itemize}}
       
       \ccMethod{bool curve_is_between_cw(const X_curve& cv,
          const X_curve& first, const X_curve& second, const Point&
          pnt);} {given three curves that share an endpoint $pnt$,
          returns $true$ if $cv$ is between $first$ and $second$, when
          going in the clockwise direction around $pnt$ from $first$
          to $second$. }


       \ccMethod{bool curve_is_same(const X_curve &cv1, const X_cur
          ve &cv2);}{returns \ccc{true} if \ccc{cv1} is the same as \ccc{cv2}.}

%construction function
%---------------------   

    \ccMethod{Point curve_source(const X_curve& cv);}{returns the source of 
       \ccc{cv}.}
    \ccMethod{Point curve_target(const X_curve& cv);}{returns the target of 
       \ccc{cv}.}

%additions needed for the default point location strategy
\ccMethod{Point point_to_left(const Point& p);}{returns a point to 
   the left of \ccc{p}.}
\ccMethod{Point point_to_right(const Point& p);}{returns a point to the right
   of \ccc{p}.} %}


%      \smallskip {\noindent \bf \large Requirements for
%         \ccStyle{Info_face}, \ccStyle{Info_edge}, \ccStyle{Info_vertex}}
      
%      All info types should have a constructor and a copy
%      constructor.

\ccHasModels
  \ccc{Pm_segment_exact_traits<R>}\\
  \ccc{Pm_segment_epsilon_traits<R>}\\
  \ccc{Pm_leda_segment_exact_traits}

  The \ccc{Arrangement_2_traits} concept is a refinement of the \ccc{Planar_map_traits}
  concept, in other words, a model of the former is a model of the latter.
  The meaning of that is that all supplied (or user written ) arrangement traits classes
  can be used ass the \ccc{Traits} template parameter for 
  \ccc{Planar_map_2<Dcel,Traits>}.
  See the list of the arrangement traits classes as part of the 
  \ccc{Arrangement_2 Reference Pages}.
  
\end{ccRefConcept} % Pm_traits

% +------------------------------------------------------------------------+
%%RefPage: end of main body, begin of footer
\ccRefPageEnd
% EOF
% +------------------------------------------------------------------------+
