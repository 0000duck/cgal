% +------------------------------------------------------------------------+
% | Reference manual page: Pm_bounding_box_traits.tex (Planar_map)
% +------------------------------------------------------------------------+
% | 
% | Package: pm (Planar_map_2)
% | 
% +------------------------------------------------------------------------+

\ccRefPageBegin

%%RefPage: end of header, begin of main body
% +------------------------------------------------------------------------+

%Pm_file_scanner
\begin{ccRefClass}{Pm_file_scanner<Planar_map>}

\ccCreationVariable{pm_file_scanner}

\ccDefinition
The \ccRefName\ class is inherited from the \ccc{File_header} class. 
This class contains functions for reading the \ccc{Planar map} components from a given input stream. 
The user has to make sure that methods for reading {\em Point} and {\em X\_curve} types of \ccc{Planar_map} are defined in the input stream he chooses to work with.
This class is templated with \ccc{Planar map}, and hence the user can work with \ccRefName\ with its own \ccc{Planar map}. 
 
%The following requirement catalog lists the primitives, i.e., types, member functions
%etc., that must be defined for any \ccc{Planar map} file scanner class that can be used to parameterize the reading function of \ccc{Planar map}. 

\ccTypes

\ccNestedType{Traits}{Representation of a bounded curve.}
\ccNestedType{Point}{A type to hold the location of a vertex in the plane.}
\ccNestedType{X_curve}{A type to hold an $x$-monotone curve in the plane.}

\ccNestedType{D_vertex}{The \ccc{Dcel} vertex type.}
\ccNestedType{D_halfedge}{The \ccc{Dcel} halfedge type.}
\ccNestedType{D_face}{The \ccc{Dcel} face type.}

\ccCreation
    
\ccConstructor{Pm_file_scanner (istream& in, bool verbose = false)} {A constructor for a given input stream and a vervose choice.}
\ccConstructor{Pm_file_scanner (istream& in, const File_header& header)} {A constructor for a given input stream and File header.}

\ccOperations
\ccMethod{void scan_pm_vhf_sizes ();}{scanning the number of vertices, halfedges and faces in the \ccc{Planar map}.}
\ccMethod{void scan_index (size_t& index);}{scanning one index.}
\ccMethod{istream& in ();}{return the input stream the \ccc{Planar map} is scanned of.} 

\ccMethod{void scan_vertex (D_vertex* v);}{scanning a vertex of the \ccc{Dcel}.}

\ccMethod{X_curve scan_halfedge (D_halfedge* h);}
{scanning an halfedge of the \ccc{Dcel}. This function also returns the scanned curve, the reason for returning that value is because the \ccc{Arrangement} does not update its halfedge with the curve.}

\ccMethod{void scan_face (D_face* f);}{scanning a face of the \ccc{Dcel}.}

\end{ccRefClass} % Pm_file_scanner

% +------------------------------------------------------------------------+
%%RefPage: end of main body, begin of footer
\ccRefPageEnd
% EOF
% +------------------------------------------------------------------------+











%%% Local Variables: 
%%% mode: latex
%%% TeX-master: "draw_pm"
%%% End: 
