% +------------------------------------------------------------------------+
% | Reference manual chapter: planar_ref.tex (Planar Map)
% +------------------------------------------------------------------------+
% | 
% | Package: pm (Planar Map)
% | 
% +------------------------------------------------------------------------+

%+----------------------------------------------------------------------------80
%| update log
%|
%| 01 Jun 2000 - Shai Hirsch
%|    Separated from planar.tex (See previous changes in change log there).
%|     
%+----------------------------------------------------------------------------80


% +========================================================================+
%   Introduction
% +========================================================================+
%\clearpage
%\section{Reference Pages for 2D Planar Maps}
%\ccRefLabel{Pm_Ref_intro}

\chapter{2D Planar Maps}
\label{chap:planar_map_2_ref}
\ccRefLabel{Pm_Ref_intro}
\subsection*{Introduction}

A planar subdivision (or planar map) is an embedding of a planar 
topological map $T$ 
into the plane, such that each edge of $T$ is embedded as a
bounded $x$-monotone curve and each vertex is embedded as a planar point.
In this embedding no
edge intersects another edge interior.

A {\em face} of the subdivision is a maximal connected region of the
plane that does not contain any vertex or edge. 
We consider a face to be open, and its boundary is
formed by vertices and halfedges of the subdivision.
The halfedges are oriented around a face so that the face they bound
is to their left. This means that halfedges on the outer boundary
of a face are traversed in counterclockwise order, and halfedges on the inner
boundaries (holes) of a face are traversed in clockwise order. Edges around a
vertex are also traversed in clockwise order. 

\subsection*{Planar Map Traits}
The planar map class is parameterized with the
interface class \ccStyle{PlanarMapTraits_2} which defines the abstract interface
between planar maps and the primitives they use. 

We supply a default traits class for exact arithmetic ---
\ccStyle{Pm_segment_traits_2<R>} % and a class for finite
%precision arithmetic --- \ccStyle{Pm_segment_epsilon_traits<R>},
where \ccStyle{R} is the kernel representation type
(\ccc{Homogeneous} or \ccc{Cartesian}).  
%There is also a class
%for finite precision arithmetic --- \ccStyle{Pm_segment_epsilon_traits<R>},
%which is not recommended unless it is known that robustness issues will not
%arise. 
%Both traits handle
The traits class handle finite line segments in the plane
and use the CGAL kernel (\ccStyle{Point_2} is of type 
\ccStyle{R::Point_2} and \ccStyle{X_curve_2} is of type
\ccStyle{R::Segment_2}).
We also supply a traits class that uses LEDA's rational kernel and makes use
of its efficient predicates. The different supplied traits classes are described at the end of this chapter.

Models of PlanarMapTraits\_2 are meant to serve as arguments for the respective
template parameter of \ccc{CGAL::Planar_map_2<Dcel,Traits>}. However, it should be noted that
each model of PlanarMapTraits\_2 defines a family of curves and
primitive geometric operations thereof. Sometimes, the only
implementation available for the manipulation of a certain family of
curves is one of the supplied traits classes. A scenario where one
uses a traits class object to manipulate such curves without
maintaining planar maps is certainly possible.

The \ccc{PlanarMapWithIntersectionsTraits_2} and \ccc{ArrangementTraits_2} 
concepts are refinements of the \ccc{PlanarMapTraits_2} concept.
Therefore, all models of the formers are models of the latter. 
There are several supplied traits classes for the \ccc{Arrangement} which the user can use. 
These classes are described at the end of Chapter \ref{I1_ChapterArrangement_2}
(\ccc{2D Arrangements}).

\subsection*{I/O functions}
The \ccc{Planar Map} package supports I/O functions, which include reading a planar map representation from 
the standard input or writing it to the standard output, 
and also sending a planar map to a graphic stream.

The motivation for using I/O functions is not only to be able to draw 
the planar map to a window for instance, 
but also to be capable to save planar maps in a text file
and reload it from a text file when needed.

The format of the output file is defined in a way the reading function
will be able to construct the planar map by updating directly the
\ccc{Dcel} without using the insertion functions of the \ccc{Planar
map}.  Consequently, the reading function constructs the planar map
very efficiently. Obviously, for very big maps or for repetitive use
of the same map it would be extremely faster to build the map only once 
(incrementally or via a sweep line utility)
, save it and reload the ready map when necessary.

Reading a planar map from the standard input or printing it to the
standard output may be done simply with the \ccc{Extractor} (\ccc{ >>
}) and \ccc{Insertor} (\ccc{ << }) operators defined for
\ccc{Planar_map_2}, respectively. 

\ccInclude{CGAL/IO/Pm_iostream.h}

The ability of sending the \ccc{Planar map} 
into a graphic stream as \ccStyle{leda_window}, Postscript file or
Geomview is also provided, users simply have to apply the Insertor
operator on the graphic stream and their planar map instance.

Users of I/O functions for the planar map package are required to define I/O 
operators for the curves defined in their \ccc{Traits} classes. 
When using \ccc{Traits} classes in which this operators are already defined 
(as \ccc{Segment Traits} or \ccc{Circle Traits} ) the operator definition is not obligated, 
however using \ccc{Traits} classes as the \ccc{Polyline Traits} will force the user to define 
I/O operators on his polyline curve.

\begin{ccAdvanced}
\subsection*{I/O for User Defined Planar Maps and the I/O Format}

Users may wish to add their own attributes planar map components. If
those attributes are to be written as part of the planar map
representation (respectively, are to be re-read later) a specialized
reader (scanner) class (writer class, resp.) should be defined for the special
planar map. This is done preferably by making it a sub class of the class
\ccStyle{Pm_file_scanner} (\ccStyle{Pm_file_writer}, resp.) and
overriding all the relevant function for scanning (writing, resp.) the
changed components.

After the definition of the inherited class users have to call the function 
\ccStyle{read} of \ccc{Planar map} (resp., the global function \ccStyle{write_pm} ) with the inherited class as a parameter.

The same goes for extending the output graphic streams to include
additional attributes only for this purpose a new \ccc{drawer}\/ class
has to be defined.  This is done preferably by making this class
inherit the class
\ccStyle{Pm_drawer}. In order to send the special planar map to the graphic stream one should call the global function \ccStyle{draw_pm} 
with this class and their planar map as parameters.

\paragraph{Format}
\ccRefLabel{Pm_IO_format}
The chosen format does not follow an existing standard format.
Generally, the format contains lists of the components of a planar map
followed by each other. For each component we write its associative
geometric information and some topological information in order to be
able to update the \ccc{Dcel} efficiently. The format is detailed
below.

\begin{enumerate}
\item The data begins with a line of three integer values specifying the number of vertices, 
halfedges and faces in the planar map.
\item The vertices list: each component in the vertices list contains the point of its associative vertex.
\item The halfedges list: each halfedge component is written by an index indicating the vertex origin 
of the halfedge, and a curve specifying the halfedge's curve.
\item The faces list: each component in the faces list contains its outer boundary, 
if the face is bounded, and a list of its holes which can be empty in case the face 
has no holes. The format of the outer boundary is the number of halfedges of its connected 
component followed by the indices indicating the halfedges of that component, those indices 
have the same order of the halfedges on the connected component. 
The format of the list of the holes is first the number of holes followed by the connected 
components per each hole, the format of each connected components resembles the format 
of the outer boundary specified above.
\item Lines beginning with '\#' serve as comments and are ignored.
\item The format does not differentiate between spaces and new lines, 
except new lines which belong to commented lines. 
And hence, writing the planar in one single line having no comments is also considered legal. 
If users would like to keep the commented lines, they may write all the 
components between two consecutive commented lines in one single line.

\end{enumerate}

The current format may not be comfortable for a user to read because
of the extensive use of indices. The user can print a planar map in a
verbose format (shorthand for verbose mode format).  The skeleton of 
the verbose format is the same. However, in order for the output to be
clearer for a human reader points and halfedges are explicitly written
rather than being represented by indices. Also the direction of the
halfedges are printed in a more convenient way to read. This verbose
format cannot be scanned by the reading functions of
\ccc{Planar_map_2}.

\ccExample

The example below presents a representation of a planar map containing
one triangle with the coordinates {\em (0,0)}, {\em (1,1)} and {\em
(2,0)}.  The \ccc{Planar_map_2} instance that was used to produce this
example was templated with the \ccStyle{Pm_segment_traits_2}
class, which in turn was templated with the representation class
\ccStyle{Cartesian<leda_rational>}.  The first line specifies that the
planar map has three vertices, six halfedges, and 2 faces (the
triangle and the unbounded face).  The list of vertices each
represented by its associated point follows, as shown in the output
example.  The next list is the one of halfedges, each component is
represented by its index (0,1 or 2) in the vertices list and its
associated segment.  The faces list is presented next. It starts with
the \ccStyle{unbounded face} having one hole which is the triangle,
this connected component specifies that the hole has three halfedges
with the indices 4, 0 and 3. The next face presenting the triangle is
written in the same manner.

\ccIncludeExampleCode{Planar_map/example9.cin}

\subsection*{Example of User Defined I/O Functions}
\label{PM_sec:example10}

The following program demonstrates the usage of I/O functions while users have 
an additional attribute in their planar map.
The attribute chosen here is adding an associative color to each vertex.
First the program extends the \ccc{Dcel} to maintain this attribute.
Second, the program extends the \ccc{Pm_file_writer} class to handle the newly 
defined vertex. 
It simply overrides the functions for writing a vertex to print the color of the vertex as well. 
Finally, the main function defines an empty \ccc{Planar map},  
reads it from the standard input stream, and then set all vertices colors. It then defines an object of its extended writer class and 
parameterize the function \ccc{write_pm} with that object.

\ccIncludeExampleCode{Planar_map/example10.C}

The input of the program is a text file presenting the \ccc{Planar map}:
\ccIncludeExampleCode{Planar_map/example10.cin}

The output is the \ccc{Planar map} written in both formats, non verbose and verbose. In addition the two lists 
(non verbose and verbose) of halfedges are written.
\ccIncludeExampleCode{Planar_map/example10.cout}

More details are given in sections
\ccc{File_header}\lcTex{
   (\ccRefPage{CGAL::File_header})},
\ccc{Pm_file_scanner<Planar_map>}\lcTex{
   (\ccRefPage{CGAL::Pm_file_scanner<Planar_map>})}, 
\ccc{Pm_file_writer<Planar_map>}\lcTex{ 
   (\ccRefPage{CGAL::Pm_file_writer<Planar_map>})} and
\ccc{Pm_drawer<Planar_map>}\lcTex{
   (\ccRefPage{CGAL::Pm_drawer<Planar_map, Window>})}.

\end{ccAdvanced}

\begin{ccAdvanced}
\subsection*{Point Location Strategies}
The \ccClassTemplateName\ class has a point location function
(namely, the \ccc{locate} function that determines which feature of the map 
contains a given query point)
which is also used internally in the \ccc{insert} function.
The planar map users can define which algorithm to use in the
point location queries. This is done with a {\em point location class}
passed to the map in the constructor. The class passed should be derived
from the base class \ccc{Pm_point_location_base} which is a
(\ccc{pure virtual}) base class that defines the interface between the 
algorithm
implemented by the users and the planar map. This follows the 
known {\it Strategy}
pattern ~\cite{ghjv-dpero-95}. The indirection overhead due to the virtual functions is
negligible since the optimal point location algorithm 
(e.g., the one implemented in our default strategy) takes $\Theta(\log n)$ time.  

We have derived three concrete classes for point location strategies,
the {\it default\/} strategy, based on trapezoidal decomposition of the map,
the {\it naive\/} strategy, which goes over all the vertices and halfedges
of the planar map and the {\it walk-along-a-line strategy\/}, which improves
the {\it naive\/} one by ``walking'' only along the zone of the vertical ray emanating
from the query point. All three strategies are classes
that inherit \ccc{Pm_point_location_base<Planar_map>}.
More details are give in sections 
  \ccc{Pm_default_point_location<Planar_map>}\lcTex{
  (\ccRefPage{CGAL::Pm_default_point_location<Planar_map>})}, 
  \ccc{Pm_naive_point_location<Planar_map>}\lcTex{ 
  (\ccRefPage{CGAL::Pm_naive_point_location<Planar_map>})} and
  \ccc{Pm_walk_along_a_line_point_location<Planar_map>}\lcTex{
  (\ccRefPage{CGAL::Pm_walk_along_a_line_point_location<Planar_map>})}.

% Advanced users may wish to write their own point location class.  We
% state that a point location class should inherit the \cgal\/ supplied
% class \ccc{Pm_point_location_base<Planar_map>}.

\paragraph{Trade-off Issues}
The main trade-off among the three strategies implemented, is between
time and storage. Using the naive or walk strategies takes more
time but saves storage space.

Another trade-off depends on the need for point location queries compared
to the need for other functions. If the users do not need point location 
queries, but do need other modifying functions (e.g., \ccc{remove_edge}, 
\ccc{split_edge} and \ccc{merge_edge}) then using the naive or walk strategies
is preferable. Note that using the \ccc{insert} function invokes the
point location query, therefore when using the naive or walk strategies it
is recommended to use the specialized insertion functions : 
\ccc{insert_in_face_interior}, \ccc{insert_from_vertex} and 
\ccc{insert_at_vertices}.
For example, when using the planar map to represent polygons (e.g., when
computing boolean operations on polygons) it might be preferable to use
the walk strategy with the specialized insertion functions. 

There are two modes of the {\it default\/} strategy which enables the user
to choose whether preprocessing should be performed or not (read more in
the section stated above).
There is a trade-off between the those two modes. If
preprocessing is not used, the building of the structure is faster. However,
for some input sequences the structure might be unbalanced and therefore 
queries and updates might take longer, especially, if many removal and split 
operation are performed.
\end{ccAdvanced}

\begin{ccAdvanced}
\subsection*{Bounding box strategies}
The planar map can support infinite objects. In such cases a special utility, namely a \ccc{bounding box} is used. 
Analogous to the point location strategy the users can define which algorithm 
will be used to update and query the map's bounding box. However, if such a
behavior is not expected or wanted the user can ignore the bounding box
or choose the \ccStyle{Pm_unbounding_box}, which is also the default one used by the planar map.

Defining the bounding box algorithm is done with a {\em bounding box} class
passed to the map in the constructor. The class passed should be a model of the
concept \ccc{PlanarMapBoundingBox_2}. The \ccc{bounding box} class should be
derived from the base class \ccStyle{Pm_bounding_box_base<Planar_map>} which is a
(\ccc{pure virtual}) base class that defines the interface between the algorithm
implemented and the planar map according to the requirements of the above concept.
All concrete classes for bounding box strategies, namely the
{\em unbounding box strategy} and the {\em dynamic open strategy} 
stand up to these criteria 
%are models of the concept \ccc{PlanarMapBoundingBox_2}. These classes were
%implemented as classes that inherit from the base class
% \ccc{Pm_bounding_box_base<Planar_map>}.

The following sections list and describe the \ccc{PlanarMapBoundingBox_2}
concept and the supplied bounding box strategies which model this concept:
\ccc{PlanarMapBoundingBox_2}\lcTex{ (\ccRefPage{PlanarMapBoundingBox_2})}, 
\ccc{Pm_unbounding_box<Planar_map>}\lcTex{ (\ccRefPage{CGAL::Pm_unbounding_box<Planar_map>})}
and
\ccc{Pm_dynamic_open_bounding_box<Planar_map>}\lcTex{
(\ccRefPage{CGAL::Pm_dynamic_open_bounding_box<Planar_map>})}.

\paragraph{Trade-off Issues}
The main trade-off issues between the different strategies
implemented are concerned with functionality and time efficiency. 

The unbounding box is efficient in time though compared to 
the others it is limited to bounded curves traits, e.g., circle traits or
line segment traits.
This means that if the users want to work with, say, straight lines the unbounding
box will not work.
To the other extreme, if one chooses the dynamic bounding box and queries a 
point 
outside the bounded area then this might cause the whole bounding box to be 
modified. The requirement may be of a linear time (in the number of curves).
In addition to that, in order to ensure that the bounding box is
large enough, the traits has to supply 
additional functions, namely to model the \ccc{PlanarMapBoundingBoxTraits_2} concept
(which is a refinement of the \ccc{PlanarMapTraits_2} concept).
%, namely to model the planar ray shoot traits concept.

%% add \ccInheritFrom{Pm_bounding_box_base<Planar_map>} to each
%% of these classes
\end{ccAdvanced}

\subsection*{Accelerating Work with Planar Maps}

This section presents some tips on how to tune
\ccc{CGAL::Planar_map_2<Dcel,Traits>} for best performance.

Before the specific tips, it should be reminded that compiling
programs with debug flags turned off and with optimization flags
significantly reduces running time.

\begin{enumerate}
   \item The default point location strategy (i.e. using
   \ccc{trapezoidal decomposition}) is the fastest one when queries
   are concerned. However, since it has to build a search structure it
   might slow down the incremental building process of the map. If it
   is known in advance that there will not be many point location or
   vertical ray shoot queries use another point location strategy
   (such as the \ccc{walk} or \ccc{simple} strategies) which does not
   slow down the building process (no search structure is being
   built).

   \item Prior knowledge of the combinatorial structure of the map can
   be used to accelerate insertion time. The specialized insertion
   functions, i.e \ccc{insert_in_face_interior},
   \ccc{insert_from_vertex} or \ccc{insert_at_vertices} should be used
   according to this information. The insert function performs point
   location queries and then calls one of the other update functions
   and therefore takes more time.  The function
   \ccc{insert_in_face_interior} even takes constant time. The other
   two are linear in the worst case, but should be much faster most of
   the time.

   Insertion of a polygon, which is represented by a list of segments
   along its boundary, into an empty planar map should be done in the
   following way. First, some segment should be inserted using
   \ccc{insert_in_face_interior} with the unbounded face. Then a
   segment with a common end point can be inserted using
   \ccc{insert_from_vertex} and so on with the rest of the segments
   but last. The last segment can be inserted using
   \ccc{insert_at_vertices} since both it endpoints are represented as
   vertices of the map and are known in advanced.
 
   \item If the user has \leda\/ installed it is recommended to use
   the specialized traits classes \ccc{Pm_leda_segment_traits_2}
   or \ccc{Arr_leda_polyline_traits}. These traits classes are much
   faster since they are specialized for \leda\/'s \ccc{rational
   geometric kernel}. Note that these traits classes are models of
   \ccc{PlanarMapTraits_2} since they model its refinement, the
   \ccc{ArrangementTraits_2} concept.

\end{enumerate}

\begin{ccTexOnly}

\subsection*{Concepts}
\ccRefConceptPage{PlanarMapPointLocation_2}\\
\ccRefConceptPage{PlanarMapBoundingBox_2}\\
\ccRefConceptPage{PlanarMapDcel_2}\\
\ccRefConceptPage{PlanarMapDcelVertex_2}\\
\ccRefConceptPage{PlanarMapDcelHalfedge_2}\\
\ccRefConceptPage{PlanarMapTraits_2}\\
\ccRefConceptPage{PlanarMapBoundingBoxTraits_2}

\subsection*{Classes}

\ccRefIdfierPage{CGAL::Planar_map_2<Dcel,Traits>}\\
\ccRefIdfierPage{CGAL::Pm_default_point_location<Planar_map>}\\
\ccRefIdfierPage{CGAL::Pm_naive_point_location<Planar_map>}\\
\ccRefIdfierPage{CGAL::Pm_walk_along_a_line_point_location<Planar_map>}\\
%\ccc{Planar_map_2<Dcel, Traits>::Vertex}\\
%\ccc{Planar_map_2<Dcel, Traits>::Halfedge}\\
%\ccc{Planar_map_2<Dcel, Traits>::Face}\\
\ccRefIdfierPage{CGAL::Pm_unbounding_box<Planar_map>}\\
\ccRefIdfierPage{CGAL::Pm_dynamic_open_bounding_box<Planar_map>}\\
\ccRefIdfierPage{CGAL::Pm_dcel<V,H,F>}\\
\ccRefIdfierPage{CGAL::Pm_segment_traits_2<R>}\\
\ccRefIdfierPage{CGAL::Pm_leda_segment_traits_2}\\
\ccRefIdfierPage{CGAL::Pm_straight_traits_2<R>}\\
\ccRefIdfierPage{CGAL::File_header}\\
\ccRefIdfierPage{CGAL::Pm_file_scanner<Planar_map>}\\
\ccRefIdfierPage{CGAL::Pm_file_writer<Planar_map>}\\
\ccRefIdfierPage{CGAL::Pm_drawer<Planar_map, Window>}\\


\subsection*{Functions}

\ccRefIdfierPage{CGAL::write_pm}\\
\ccRefIdfierPage{CGAL::draw_pm}\\


%\ccc{Topological_map<Dcel>}\\
%\ccc{Tpm_face_base}

\end{ccTexOnly}    



