% +------------------------------------------------------------------------+
% | Reference manual page: Pm_segment_epsilon_traits.tex (Planar_map)
% +------------------------------------------------------------------------+
% | 
% | Package: pm (Planar_map_2)
% | 
% +------------------------------------------------------------------------+

\ccRefPageBegin

%%RefPage: end of header, begin of main body
% +------------------------------------------------------------------------+

\begin {ccRefClass} {Pm_segment_epsilon_traits<R>}
    %\ccDefinition 
    The class
    \ccStyle{Pm_segment_epsilon_traits<R>} is used for finite
    precision (floating-point) arithmetic. 
    This is a temporary Traits
    Class.  (We plan to implement more sophisticated methods to handle
    floating point arithmetic at a later stage.) 
    The class uses a
    predefined $\varepsilon$ (of type \ccStyle{double}) to decide
    whether two values should be considered the same.  Two real values
    $a$ and $b$ are considered equal if $|a - b| < \varepsilon$. Thus
    when $L_{\infty}$ norm is used, two points $p_1$, $p_2$ are
    treated as the same point if $\| p_1 - p_2\|_{\infty} =
    \max(|p_1.x() - p_2.x()|, |p_1.y() - p_2.y()|) < \varepsilon $.
    The value of $\varepsilon$ can be determined through the class's
    constructor. 

\ccInclude{ CGAL/Pm_segment_epsilon_traits.h }

\ccIsModel
    \ccc{PlanarMapTraits_2}
 
%    \ccTypes
    
%    \ccNestedType{Point}{of type \ccStyle{Point_2<R>}}

%    \ccNestedType{X_curve}{of type \ccStyle{Segment_2<R>}}

%    \ccNestedType{Info_face}{of type \ccStyle{int}}
    
%    \ccNestedType{Info_edge}{of type \ccStyle{int}}
    
%    \ccNestedType{Info_vertex}{of type \ccStyle{int}}

%    \ccCreation
%    \ccCreationVariable{T_fp}
     
%    \ccConstructor{Pm_segment_epsilon_traits(double eps = 0.0001 );} 
%    {}
 
\end{ccRefClass} % Pm_segment_epsilon_traits

% +------------------------------------------------------------------------+
%%RefPage: end of main body, begin of footer
\ccRefPageEnd
% EOF
% +------------------------------------------------------------------------+
