% +------------------------------------------------------------------------+
% | Reference manual page: Topological_map_Dcel.tex
% +------------------------------------------------------------------------+
% | 
% | Package: pm (Planar Map)
% | 
% +------------------------------------------------------------------------+

\ccRefPageBegin

%%RefPage: end of header, begin of main body
% +------------------------------------------------------------------------+

\begin{ccRefConcept}{TopologicalMapDcel}

A \ccc{DCEL} (Doubly Connected Edge List) consists of vertices V, edges E, facets F and an incidence relation on them. Each edge is
represented by two halfedges with opposite orientations. 
A \ccc{DCEL} serves as an interface for the \ccStyle{Topological_map<Dcel>} so a model of it must be provided as a template parameter.
The predefined DCEL class
implementation described in the next section can be used 
as a starting point for
building one's own DCEL class. (The naming conventions were chosen to comply 
with those of the \ccStyle{Halfedge_data_structure}.)

A model for the \ccRefName\ concept must provide the following types and operations. 
(In addition to the requirements here, the local types 
\ccHtmlNoLinksFrom{\ccStyle{Vertex},\ccStyle{Halfedge} and \ccStyle{Face}}
must be models of the concepts 
\ccc{TopologicalMapDcelVertex}\lcTex{ 
(\ccRefPage{TopologicalMapDcelVertex})},
\ccc{TopologicalMapDcelHalfedge}\lcTex{
(\ccRefPage{TopologicalMapDcelHalfedge})} and
\ccc{TopologicalMapDcelFace}\lcTex{
(\ccRefPage{TopologicalMapDcelFace})} respectively.)

%\ccRefIdfierPage{Triangulation_traits_2} \\
\ccThree{Topological_map_Decl::Halfedge_iterator}{}{\hspace*{11cm}}
%\ccThreeToTwo

\ccTypes

	\ccNestedType{Vertex}{vertex type.}
	\ccGlue
	\ccNestedType{Halfedge}{halfedge type.}
	\ccGlue
	\ccNestedType{Face}{face type.}
	\ccGlue
	\ccNestedType{Size}{type for size values.}

	%\ccNestedType{Point}{a point associated with a vertex.}
	%\ccNestedType{Curve}{a curve associated with an edge.}

	\ccNestedType{Vertex_iterator}{a bidirectional iterator over the
	       vertices. Its value-type is 
               \ccStyle{Topological_map<Dcel>::Vertex}.}
	\ccGlue
	\ccNestedType{Halfedge_iterator}{a bidirectional iterator over the
	       halfedges. Its value-type is 
               \ccStyle{Topological_map<Dcel>::Halfedge}.}
	\ccGlue
	\ccNestedType{Face_iterator}{a bidirectional iterator over the
	       faces. Its value-type is 
               \ccStyle{Topological_map<Dcel>::Face}.}


\ccCreation
\ccCreationVariable{d}
    
    \ccConstructor{Pm_dcel();} {constructs an
       empty DCEL with one outer face.}

\ccThree{Halfedge_iterator}{}{\hspace*{7.5cm}}
    
%\ccOperations
\ccAccessFunctions
    
     \ccMethod{Size  size_of_vertices() const;}{number of vertices.}
     \ccGlue
     \ccMethod{Size  size_of_halfedges() const;}{number of halfedges (always even).}
     \ccGlue
     \ccMethod{Size  size_of_faces() const;}{number of faces.}

     The following operations have an equivalent \ccc{const} operations
     such as \ccc{Vertex_const_iterator  vertices_begin();} etc.

     \ccMethod{Vertex_iterator vertices_begin();}{ returns the begin-iterator of the vertices in \ccVar{}. }
     \ccGlue
     \ccMethod{Vertex_iterator vertices_end();}{ returns the past-the-end iterator of the vertices in \ccVar{}. }

     \ccMethod{Halfedge_iterator halfedges_begin();}{ returns the begin-iterator of the halfedges in \ccVar{}. }
     \ccGlue
     \ccMethod{Halfedge_iterator halfedges_end();}{ returns the past-the-end iterator of the halfedges in \ccVar{}. }

     \ccMethod{Vertex_iterator faces_begin();}{ returns the begin-iterator of the faces in \ccVar{}. }
     \ccGlue
     \ccMethod{Vertex_iterator faces_end();}{ returns the past-the-end iterator of the faces in \ccVar{}. }

\ccModifiers

     The following operations allocate a new element of the respective type. Halfedges 
are always allocated in pairs of opposite halfedges. The twin pointers are 
automatically set.

     \ccHtmlNoLinksFrom{ % to avoid linkage of Vertex, etc to HDS::Vertex
     \ccMethod{Vertex* new_vertex();}{creates a default vertex.}
     \ccGlue
     \ccMethod{Halfedge* new_edge();}{creates a new pair of 
     opposite halfedges.}
     \ccGlue
     \ccMethod{Face* new_face();}{creates a new face.}

     \ccMethod{void delete_vertex(Vertex* v);}{deletes the vertex $v$.}
     \ccGlue
     \ccMethod{void delete_edge(Halfedge* h);}{deletes the pair of opposite 
     halfedges $h$.}
     \ccGlue
     \ccMethod{void delete_face(Face* f);}{deletes the face $f$.}
     } % ccHtmlNoLinksFrom
\ccHasModels

     \ccc{Pm_dcel<V,H,F>}\lcTex{ (\ccRefPage{CGAL::Pm_dcel<V,H,F>})}\\

\ccSeeAlso
     \ccc{TopologicalMapDcelVertex}\lcTex{ 
     (\ccRefPage{TopologicalMapDcelVertex})}\\
     \ccc{TopologicalMapDcelHalfedge}\lcTex{
     (\ccRefPage{TopologicalMapDcelHalfedge})}\\
     \ccc{TopologicalMapDcelFace}\lcTex{
     (\ccRefPage{TopologicalMapDcelFace})}

\end{ccRefConcept}

\ccRefPageBegin
% +------------------------------------------------------------------------+
%%RefPage: end of main body, begin of footer
\ccRefPageEnd
% EOF
% +------------------------------------------------------------------------+
