% +------------------------------------------------------------------------+
% | Reference manual page: intro.tex
% +------------------------------------------------------------------------+
% | 
% | Package: pm (Planar Map)
% | 
% +------------------------------------------------------------------------+

\clearpage
\section{Topological Map Reference Pages}

% +========================================================================+
%   Introduction
% +========================================================================+

\subsection*{Introduction}
\ccRefLabel{Tpm_ref_intro}

A topological map is a graph which consists of vertices,
edges, faces and an incidence relation on them.
Each edge is represented by two halfedges with opposite orientations.
A face of the topological map is defined by the ordered
circular sequences 
(inner and outer) of halfedges along its boundary.

We consider each edge $e$ to be two-sided, representing it by two
directed {\em halfedges} \lcTex{$\vec{e}$}\lcHtml{$e$} and 
\lcTex{${\rm Twin}(\vec{e})$}\lcHtml{${\rm Twin}({e})$} 
(In other packages the twin halfedge is called $Opposite$).  
A halfedge \lcTex{$\vec{e}$}\lcHtml{$e$} is an ordered pair $(u,v)$ of its endpoints, and
it is directed from $u$, the {\em source}, to $v$, the {\em target} (there 
is no need to store both in each halfedge since 
\lcTex{${\rm Target}(\vec{e}) \equiv {\rm Source}({\rm Twin}(\vec{e}))$.}%
\lcHtml{${\rm Target}({e}) \equiv {\rm Source}({\rm Twin}({e}))$.})
We consider each halfedge to lie on the boundary of a single face.
Each connected component of the boundary of a face is %represented 
defined by a
circular list of halfedges. 

For a face $f$ of a topological map, 
we call each
connected component of the boundary of $f$ a {\em CCB}.
A {\em bounded face} has a
unique CCB that is defined to be
its outer-CCB. An
{\em unbounded} face does not have an outer boundary.
In the topological map we have one unbounded face.

Every maximal set of halfedges that share the same target can be viewed 
as a circular list of halfedges ordered %clockwise  
around their target vertex.

For a topological map, its {\em DCEL} representation consists of a
connected list of halfedges for every CCB of every face in the
subdivision, with additional incidence information that enables us to
traverse the subdivision. %In particular, for 
For each halfedge the DCEL
stores a pointer to its twin halfedge and to the next
halfedge around its incident face (see Figure~\ref{fig:DCEL_ref}). In
addition, for each halfedge the DCEL stores a pointer to the incident
face and the target vertex.
For each face the DCEL stores a pointer to a halfedge representing
its outer-CCB and an iterator over pointers to halfedges representing
its inner-CCBs (traversing over a CCB is thus done with repetitive
calls to the next halfedge pointer).
For each vertex the DCEL stores a pointer to an incident halfedge. 

The DCEL is a low-level container class that stores the objects.
The topological map layer adds high-level functions and protection of
combinatorial validity. Iterators, handles and circulators are also
introduced in this layer (pointers are no longer visible in this layer).

In the following
specifications, we implement the subdivision by a DCEL. Read about the concept
\ccc{TopologicalMapDcel}\lcTex{ (\ccRefPage{TopologicalMapDcel})}
for a specification of the requirements for a DCEL in our implementation.

\begin{figure}
\begin{ccTexOnly}
    \centerline{
       \Ipe{dcel.ipe}
       }
\end{ccTexOnly}
\caption{Source and target vertices, and twin halfedges \label{fig:DCEL_ref}}

\begin{ccHtmlOnly}
<P>
<center><img border=0 src="./dcel.gif" alt=" ">
<!-- <br> Source and target vertices, and twin halfedges -->
</center>
\end{ccHtmlOnly}
\end{figure}

\begin{ccTexOnly}

\section*{Concepts}

\ccRefConceptPage{TopologicalMapDcel}\\
\ccRefConceptPage{TopologicalMapDcelVertex}\\
\ccRefConceptPage{TopologicalMapDcelHalfedge}\\
\ccRefConceptPage{TopologicalMapDcelFace}

\section*{Classes} 

\ccRefIdfierPage{CGAL::Topological_map<Dcel>}\\
\ccRefIdfierPage{CGAL::Topological_map<Dcel>::Vertex}\\
\ccRefIdfierPage{CGAL::Topological_map<Dcel>::Halfedge}\\
\ccRefIdfierPage{CGAL::Topological_map<Dcel>::Face}\\
\ccRefIdfierPage{CGAL::Tpm_vertex_base}\\
\ccRefIdfierPage{CGAL::Tpm_halfedge_base}\\
\ccRefIdfierPage{CGAL::Tpm_face_base}

\end{ccTexOnly}    

% EOF
% +------------------------------------------------------------------------+
