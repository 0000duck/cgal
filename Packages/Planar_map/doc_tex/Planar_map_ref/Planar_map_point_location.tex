% +------------------------------------------------------------------------+
% | Reference manual page: Planar_map_point_location.tex (Planar_map)
% +------------------------------------------------------------------------+
% | 
% | Package: pm (Planar_map_2)
% | 
% +------------------------------------------------------------------------+

\ccRefPageBegin

%%RefPage: end of header, begin of main body
% +------------------------------------------------------------------------+

% enabling the advanced brackets before reference name
% simply putting \begin{ccAdvanced} will not do, since there will be
% a page break because of the ccRef...
\renewcommand{\ccRefPageBegin}{\begin{ccAdvanced}}
\renewcommand{\ccRefPageEnd}{\end{ccAdvanced}}

\begin{ccRefConcept}{PlanarMapPointLocation_2}
%{Pm_point_location_base<Planar_map>}
\label{PL_sec:base}
%\begin{ccAdvanced}
\ccDefinition
\ccRefName\ is the point location strategy used by \ccc{Planar_map_2<Dcel, Traits>}.

In order for the advanced user to write a model of a~\ccRefName\ the user's class should inherit the \cgal\/ supplied class 
\ccc{Pm_point_location_base<Planar_map>}.
This class is an abstract base class for the different point location strategies including the supplied ones.

\ccc{Planar_map_2<Dcel, Traits>} is the
type that aggregates the strategy. The return values of the queries
are given in terms that the \ccc{Planar_map_2<Dcel, Traits>} handles.

\ccInclude{CGAL/Pm_point_location_base.h}

\ccThree{typedef typename Planar_map::Halfedge_handle }{}{\hspace*{6cm}}

\ccTypes
\ccTypedef{typedef typename Planar_map::Traits Traits;}{}
\ccTypedef{typedef typename Planar_map::Locate_type Locate_type;}{}
\ccTypedef{typedef typename Planar_map::Halfedge_handle Halfedge_handle;}{}

\ccTypedef{typedef typename Traits::X_monotone_curve_2 X_monotone_curve_2;}{}
\ccTypedef{typedef typename Traits::Point_2 Point_2;}{}

\ccThree{virtual Halfedge_handle}{}{\hspace*{10cm}}

\ccOperations
\ccCreationVariable{pl}

\ccMethod{virtual void init(Planar_map& pmp, Traits& tr);}{initializes
the strategy; will be called by the planar map directly after its 
construction, with \ccc{*this} and the traits instantiation inside the 
planar map, as its parameters.}

%\ccHeading{Query Functions}
\ccQueryFunctions

%\ccMethod{virtual Halfedge_handle locate(const Point_2& p,
%   Locate_type& lt) const;}{}
\ccMethod{virtual Halfedge_handle locate(const Point_2& p, Locate_type& lt);}
{point location query. A \ccc{const} version is also required.}

%\ccMethod{virtual Halfedge_handle vertical_ray_shoot(
%const Point_2& p, Locate_type& lt, bool up) const;}{}
\ccMethod{virtual Halfedge_handle vertical_ray_shoot(
const Point_2& p, Locate_type& lt, bool up);}
{vertical ray shooting query. A \ccc{const} version is also required.}

The non const versions may update the internal structure of the Point location.
%maybe add is vertex in??

%\ccHeading{Updating Functions}
\ccModifiers

\ccMethod{virtual void insert(Halfedge_handle h,const X_monotone_curve_2& cv);}
{updates the internal structure of the strategy after an insertion in the 
planar map. The planar
map will call this function after the insertion has taken place. \ccc{h} is
the handle which is returned by the map after insertion, \ccc{cv} is
the curve inserted into the map.}

\ccMethod{virtual void split_edge(const X_monotone_curve_2 &cv, Halfedge_handle e1,
   Halfedge_handle e2,const X_monotone_curve_2& cv1, const X_monotone_curve_2& cv2);}{updates the 
internal structure of the strategy 
after a \ccc{split_edge} operation in the planar map. The planar
map will call this function after the splitting has taken place. \ccc{e1}
and \ccc{e2} are the two edges created after the split. \ccc{cv} is 
the original curve that was in the map before the splitting,
\ccc{cv1} and \ccc{cv2} are the curves in \ccc{e1} and \ccc{e2} (respectively.)
after the splitting has taken place.}

\ccMethod{virtual void merge_edge(const X_monotone_curve_2 &cv1, const X_monotone_curve_2 &cv2,
                        Halfedge_handle e,const X_monotone_curve_2& cv );}{updates 
the internal structure of the strategy 
after a \ccc{merge_edge} operation in the planar map. The planar
map will call this function after the merging has taken place. \ccc{e}
is the merged edge. \ccc{cv1} and \ccc{cv2} are 
the original curves that were in the map before the merging and
\ccc{cv} is the curve in \ccc{e} after the merging.}

\ccMethod{virtual void remove_edge(Halfedge_handle e);}{updates the internal
structure of the strategy during a \ccc{remove_edge} operation in the 
planar map.The planar map will call this function {\em before} the removal 
has taken place (so the internal structure can be updated before the
\ccHtmlNoLinksFrom{\ccc{Halfedge}} has been removed from the planar map itself.)} 

\ccHasModels
   \ccc{Pm_trapezoid_dag_point_location<Planar_map>}\\
   \ccc{Pm_naive_point_location<Planar_map>}\\
   \ccc{Pm_walk_along_a_line_point_location<Planar_map>}

%%\ccHasModels
%\end{ccAdvanced} % point location strategies
\end{ccRefConcept}
\renewcommand{\ccRefPageBegin}{}
\renewcommand{\ccRefPageEnd}{}

% +------------------------------------------------------------------------+
%%RefPage: end of main body, begin of footer
\ccRefPageEnd
% EOF
% +------------------------------------------------------------------------+
