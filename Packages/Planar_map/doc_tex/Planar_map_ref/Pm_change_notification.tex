% +------------------------------------------------------------------------+
% | Reference manual page: Pm_change_notification.tex
% +------------------------------------------------------------------------+
% |
% | Package: pm (Planar_map_2)
% |
% +------------------------------------------------------------------------+

\ccRefPageBegin

%%RefPage: end of header, begin of main body
% +------------------------------------------------------------------------+
% +========================================================================+
%   Pm_change_notification
% +========================================================================+
\begin{ccRefConcept}{PlanarMapChangeNotification_2}

\ccDefinition
   An instance of \ccRefName\ can be passed to the modification
   methods of a planar map with intersections object. An insertion
   of an intersecting curve to a planar map may add several
   halfedges and modify several features of the map. The relevant
   function of the change notification instance is called after
   each such modification.

\ccTypes
\ccNestedType{Planar_map}{the planar map with intersections 
                          in which the object is being used.}
\ccNestedType{Traits}{the traits class of \ccc{Planar_map}.}

\ccOperations

\ccCreationVariable{pmwx_cn}

\ccMethod{ void add_edge(const  Traits::X_curve& cv,
                         Planar_map::Halfedge_handle e,
                         bool original_direction, bool overlap);}
  {called each time a pair of twin halfedges is added to the planar map.
  \ccc{cv} is the curve of the added halfedges.
  \ccc{e} is one of the new halfedges.
  \ccc{original_direction} is true iff \ccc{e} is directed the same as the
  original curve inserted into the map.
  \ccc{overlap} is true iff the added halfedges overlap halfedges already in the
  map.}

\ccMethod{ void split_edge(
               Planar_map::Halfedge_handle orig_edge,
               Planar_map::Halfedge_handle new_edge,
              const  Traits::X_curve& c1,
              const  Traits::X_curve& c2);}
  {called each time a pair of twin halfedges is split into two
  pairs. The split operation creates a new pair of twin halfedges
  (\ccc{new_edge} and its twin) and modifies the original
  pair (\ccc{orig_edge} an its twin).
  \ccc{c1} and \ccc{c2} are the curves after the split.
  }

\ccMethod{ void split_face(
               Planar_map::Face_handle orig_face,
               Planar_map::Face_handle new_face);}
  {called each time a face of the planar map is being split by
  the inserted curve into two faces: \ccc{orig_face} is the
  original face and \ccc{new_face} is the new face.
  }

\ccMethod{ void add_hole(
             Planar_map::Face_handle in_face,
             Planar_map::Halfedge_handle new_hole);}
  {called each time a hole is added to a face \ccc{in_face}.
  \ccc{new_hole} is a halfedge on the new hole.
  }

\ccMethod{ bool have_support_curve();}
  {returns true if \ccc{edge_support_curve} function can return the
  supporting curve (used by the arrangement class).
  }

\ccMethod{ const  Traits::X_curve&
  edge_support_curve( Planar_map::Halfedge_handle edge);}
  {returns the supporting curve of a \ccc{edge}, if it exists
  (used by the arrangement class).
  }

\ccHasModels

  The package does not supply models of
  \ccc{PlanarMapChangeNotification_2}. The
  arrangement class has an internal implementation that is used for
  the updating of its hierarchy structure.

\end{ccRefConcept}

% +------------------------------------------------------------------------+
%%RefPage: end of main body, begin of footer
\ccRefPageEnd
% EOF
% +------------------------------------------------------------------------+
