% +------------------------------------------------------------------------+
% | Reference manual page: Pm_bounding_box_traits.tex (Planar_map)
% +------------------------------------------------------------------------+
% | 
% | Package: pm (Planar_map_2)
% | 
% +------------------------------------------------------------------------+

\ccRefPageBegin

%%RefPage: end of header, begin of main body
% +------------------------------------------------------------------------+
\renewcommand{\ccRefPageBegin}{\begin{ccAdvanced}}
\renewcommand{\ccRefPageEnd}{\end{ccAdvanced}}

%Pm_file_scanner
\begin{ccRefClass}{Pm_file_scanner<Planar_map>}

\ccCreationVariable{pm_file_scanner}

\ccDefinition
The \ccRefName\ class is inherited from the \ccc{File_header} class. 
This class contains functions for reading the \ccc{Planar map} components from a given input stream. 
Users have to make sure that methods for reading \ccStyle{Point_2} and \ccStyle{X\_curve_2} defined in  
the \ccc{Planar map traits} class are provided when using the input stream they choose to work with.
This class is templated with \ccc{Planar map}, and hence the user can work with \ccRefName\ with its own \ccc{Planar map}. 
 
%The following requirement catalog lists the primitives, i.e., types, member functions
%etc., that must be defined for any \ccc{Planar map} file scanner class that can be used to parameterize the reading function of \ccc{Planar map}. 

\ccTypes

\ccTwo{Pm_file_scanner<Planar_map>::D_halfedge__bla}{}

\ccNestedType{Traits}{Representation of a bounded curve.}
\ccGlue
\ccNestedType{Point}{A type to hold the location of a vertex in the plane.}
\ccGlue
\ccNestedType{X_curve}{A type to hold an $x$-monotone curve in the plane.}
\ccGlue
\ccNestedType{D_vertex}{The \ccc{Dcel} vertex type.}
\ccGlue
\ccNestedType{D_halfedge}{The \ccc{Dcel} halfedge type.}
\ccGlue
\ccNestedType{D_face}{The \ccc{Dcel} face type.}

\ccCreation
    
\ccConstructor{Pm_file_scanner (istream& in)} 
{A constructor for a given input stream.}
\ccConstructor{Pm_file_scanner (istream& in, const File_header& header)} 
{A constructor for a given input stream and File header.}

\ccOperations
\ccMethod{void scan_pm_vhf_sizes ();}{scans the number of vertices, 
halfedges and faces of the \ccc{Planar map} from the input stream.}
\ccMethod{void scan_index (size_t& index);}{scans one index from the input stream.}
\ccMethod{istream& in ();}{returns the input stream the \ccc{Planar map} is scanned from.} 

\ccMethod{void scan_vertex (D_vertex* v);}{scans a vertex of the \ccc{Dcel} from the input stream.}

\ccMethod{X_curve scan_halfedge (D_halfedge* h);}
{scans an halfedge of the \ccc{Dcel} from the input stream. 
This function also returns the scanned curve. 
The reason for returning that value is because the \ccc{Arrangement} class  
does not update its halfedge with the curve.}

\ccMethod{void scan_face (D_face* f);}{scans a face of the \ccc{Dcel} from the input stream.}

\end{ccRefClass} % Pm_file_scanner

\renewcommand{\ccRefPageBegin}{}
\renewcommand{\ccRefPageEnd}{}

% +------------------------------------------------------------------------+
%%RefPage: end of main body, begin of footer
\ccRefPageEnd
% EOF
% +------------------------------------------------------------------------+











%%% Local Variables: 
%%% mode: latex
%%% TeX-master: "draw_pm"
%%% End: 
