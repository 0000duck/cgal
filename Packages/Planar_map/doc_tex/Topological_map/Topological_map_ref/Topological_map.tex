% +------------------------------------------------------------------------+
% | Reference manual page: tpm_ref.tex (Topological Map)
% +------------------------------------------------------------------------+
% | 
% | Package: pm (Planar Map)
% | 
% +------------------------------------------------------------------------+

\ccRefPageBegin

%%RefPage: end of header, begin of main body
% +------------------------------------------------------------------------+

%%% \section{Topological Map Reference Pages}

% @@@ Should we have this label?
\label{TPM_sec:tpm}

\ccThree{Topological_map<Dcel>::Halfedge}{}{\hspace*{7cm}}
\ccThreeToTwo

\begin{ccRefClass}{Topological_map<Dcel>} 
    
\ccDefinition An object $tpm$ of the class
    \ccClassTemplateName\ is a graph that
    consists of vertices V, edges E, faces F and an incidence relation on them.
    Each edge is represented by two halfedges with opposite orientations.
    A {\em face} of the topological map is defined by the circular 
    sequence of halfedges along its boundary. The representation used 
    for the topological map is a DCEL.
    The available default DCEL class will be described
    later.

\ccInclude{CGAL/Topological_map.h}

\ccTypes

    \ccNestedType{Vertex}{represents a vertex of the topological map.}
    \ccGlue
    \ccNestedType{Halfedge}{represents a halfedge of the topological map.}
    \ccGlue
    \ccNestedType{Face}{represents a face of the topological map.}

%    \ccNestedType{Point}{a type associated with vertex.}
%    \ccNestedType{Curve}{a type associated with halfedge.}

    The following handles, iterators and circulators have appropriate
    constant\footnote{This means that for every \ccc{iterator} or
       \ccc{circulator} there is a \ccc{const_iterator} and 
       \ccc{const_circulator}
       counterpart. See ~\cite{ms-strg-96} for a discussion of constant versus
       mutable iterator types.} counterparts.
    The mutable types are assignable to their
    constant counterparts. Both circulators are assignable to the 
    \ccStyle{Halfedge_iterator}. The iterators are assignable to the
    respective handle types. Wherever the handles appear in function parameter
    lists, the appropriate iterator can be used as well.

    
    \ccNestedType{Vertex_handle}{handle to vertex.}
    \ccGlue
    \ccNestedType{Halfedge_handle}{handle to halfedge.}
    \ccGlue
    \ccNestedType{Face_handle}{handle to face.}
    \ccGlue
    \ccNestedType{Vertex_iterator}{a bidirectional iterator over the
       vertices of the topological map. Its value-type is
       \ccStyle{Topological_map<Dcel>::Vertex}.}
    \ccGlue
    \ccNestedType{Halfedge_iterator}{a bidirectional iterator over the
       halfedges of the topological map.  Its value-type is 
       \ccStyle{Topological_map<Dcel>::Halfedge}.}
    \ccGlue    
    \ccNestedType{Face_iterator}{a bidirectional iterator over the
       faces of the topological map.  Its value-type is 
       \ccStyle{Topological_map<Dcel>::Face}.}
    
    \ccGlue
    \ccNestedType{Ccb_halfedge_circulator}{a forward circulator over the
       halfedges of a CCB (connected component of the boundary).  Its
       value-type is \ccStyle{Topological_map<Dcel>::Halfedge}.}
    \ccGlue    
    \ccNestedType{Halfedge_around_vertex_circulator}{a forward circulator over
       the halfedges that have the vertex as their target.  
       %The halfedges are traversed in their clockwise order around the vertex. 
       Its value-type is \ccStyle{Topological_map<Dcel>::Halfedge}.}
    \ccGlue    
    \ccNestedType{Holes_iterator}{a bidirectional iterator to
       traverse all the holes ( i.e., inner CCBs) of a face 
       (\ccStyle{Holes_iterator++}{} is the next hole in the face).
       Its value type is
       \ccStyle{Ccb_halfedge_circulator}.}
    
\ccCreation
    \ccCreationVariable{tpm}
    
    \ccConstructor{Topological_map<Dcel>();} {constructs an
       empty map - containing one unbounded face.}

\ccThree{Halfedge_iterator}{}{\hspace*{11cm}}
    
\ccAccessFunctions

    \ccMethod
    {Face_handle  unbounded_face() const;}
    {returns the unbounded face of \ccVar .}
  
    \ccMethod
    { Face_iterator faces_begin(); }
    { returns the begin-iterator of the faces in \ccVar .}
    
    \ccMethod
    { Face_iterator faces_end(); }
    { returns the past-the-end iterator of the faces in \ccVar .}
    
    \ccMethod
    { Halfedge_iterator halfedges_begin(); }
    { returns the begin-iterator of the halfedges in \ccVar .}
 
    \ccMethod
    { Halfedge_iterator halfedges_end(); }
    { returns the past-the-end iterator of the halfedges in \ccVar .}
    
    \ccMethod
    { Vertex_iterator vertices_begin(); }
    { returns the begin-iterator of the vertices in \ccVar .}
    
    \ccMethod
    { Vertex_iterator vertices_end(); }
    { returns the past-the-end iterator of the vertices in \ccVar .}

\ccPredicates
    
    \ccMethod { bool is_valid(); } { checks the validity of all the
       features of \ccVar . Check for each vertex \ccStyle{v}
       whether all incident halfedges (by performing
       \ccStyle{next()->twin()}) have \ccStyle{v} as their
       target. 
       Check that all the edges on the boundary
       of each face \ccStyle{f} point to \ccStyle{f} as their
       face.} 

\ccQueryFunctions

    The following methods return the number of map components, faces,
    halfedges or vertives, respectively.

    \ccMethod {Size number_of_faces();} {}
    \ccGlue
    \ccMethod {Size number_of_halfedges();} {}
    \ccGlue
    \ccMethod {Size number_of_vertices();} {}

\ccModifiers

\ccMethod{Halfedge_handle insert_at_vertices(Halfedge_handle previous1, Halfedge_handle previous2);}
     {inserts a new edge
        between the vertices \ccStyle{v1} and \ccStyle{v2} which are the target
        of \ccStyle{previous1} and \ccStyle{previous2} respectively. The return value
        is the new halfedge that has \ccStyle{v1} as its source.
        If a new face is created it will be, by definition, the face incident
        to the halfedge returned --- this means that the order of the vertices
        on the new face will be $v_0...,v1,v2,...v_0$.
        This sets a requirement on the users (e.g,
        \ccStyle{Planar_map_2}) to call the function with the arguments
        in the right order (i.e., the order that will give the desired face).
        Insertion at vertices may require updating of the holes  
        (since a 
        new face can be created that contains some of the holes of the old 
        face). To update the holes the function 
        \ccStyle{move_hole()} should be called by the user.
        \ccPrecond{ \ccc{previous1} and \ccc{previous2} have the same incident
           face.}
        \ccPrecond{ \ccc{previous1} and \ccc{previous2} do not have the same
           target vertex (i.e., no loop).}}

    \ccMethod{Halfedge_handle insert_from_vertex(Halfedge_handle previous);}
    {inserts a new edge for
             which one endpoint, \ccStyle{v1} that is the target of
             \ccStyle{previous}, is already in the map.
             %\ccStyle{cv} will be the curve in the new edge and
             %\ccStyle{pt} will be the point in the new vertex. 
             %If \ccStyle{source} is \ccStyle{true} (respectively.
             %\ccStyle{false}) then the source (respectively. target) of
             %\ccStyle{cv} is \ccStyle{v1}. 
             The return value is the new halfedge
             that has \ccStyle{v1} as its source.
%        \ccPrecond{\ccStyle{Dcel::Halfedge::Curve} has member functions 
%           \ccStyle{source()} and  \ccStyle{target()} which are used to define
%           the point on the new vertex.} 
             }

     \ccMethod{Halfedge_handle insert_in_face_interior(Face_handle f);}
     {inserts an edge as a new inner component of
       \ccStyle{f}. The return value is the new halfedge.
%       which has  \ccStyle{ps} in its source vertex and \ccStyle{pt} in its target vertex.
%        \ccPrecond{\ccStyle{Dcel::Halfedge::Curve} has member functions 
%           \ccStyle{source()} and  \ccStyle{target()} which are used to define
%           the points on the new vertices. }
       }

%maybe add a function move_holes(first,last,first?) or something
     \ccMethod{bool move_hole(Halfedge_handle e, Face_handle f1, Face_handle f2);}
     {moves the hole represented by the halfedge \ccStyle{e} from
        the face \ccStyle{f1} to \ccStyle{f2}. Returns false if
        \ccStyle{e} does not represent a hole in \ccStyle{f1},
        and true otherwise.}


    \ccMethod{Halfedge_handle split_edge (Halfedge_handle e);}
    {splits the edge \ccStyle{e} into \ccStyle{e1} and  \ccStyle{e2} , and
     adds a new vertex incident both \ccc{e1} and \ccc{e2}. 
     The return value
     is the halfedge (\ccc{e1} or \ccc{e2}) with the same source vertex 
     that \ccStyle{e} had.
     }

  \ccMethod{Halfedge_handle merge_edge (Halfedge_handle e1, Halfedge_handle e2);}
  {merges the two consecutive edges \ccStyle{e1} and \ccStyle{e2} into one 
     edge. The return value is the halfedge with the same source vertex
     that \ccc{e1} had and the same target \ccc{e2} had.
     This operation is the inverse of the \ccStyle{split_edge}
     operation.
     \ccPrecond{ \ccStyle{e1} and \ccStyle{e2} are consecutive halfedges 
        (i.e., \ccStyle{e1.target()==e2.source()}).}
     \ccPrecond{There is no other edge attached to the vertex between 
        \ccStyle{e1} and \ccStyle{e2} (i.e., its degree is exactly 2).}}
  
  \ccMethod{Face_handle remove_edge (Halfedge_handle e);}
  {removes the edge referenced by \ccStyle{e} from \ccVar . 
   If the operation causes two faces to merge, the merged face is returned.
   Otherwise, the face in which the edge was incident is returned.
   The removal of an edge might create a new hole inside the face (e.g.,
   when removing an edge in the interior of an ``antenna''). In this
   case by definition \ccc{e} points at the new hole created (i.e., the path
   from \ccc{e} to its twin goes over the new hole) and its twin points
   at the outer CCB (i.e., the path from \ccc{e->twin()} to \ccc{e}
   passes over the outer CCB). This sets a requirement on the users to
   define which halfedge (\ccc{e} or \ccc{e->twin()}) should be
   given as the parameter.} 

\end{ccRefClass}

% +------------------------------------------------------------------------+
%%RefPage: end of main body, begin of footer
\ccRefPageEnd
% EOF
% +------------------------------------------------------------------------+
