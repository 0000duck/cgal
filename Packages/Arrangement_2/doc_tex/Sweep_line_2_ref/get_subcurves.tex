\ccRefPageBegin

\begin{ccRefFunction}{get_subcurves}

\ccInclude{CGAL/Sweep_line_2_algorithms.h}

\ccFunction{template <class Traits, class InputIterator, class OutputIterator>
OutputIterator get_subcurves (InputIterator curves_begin,
                              InputIterator curves_end,
                              OutputIterator subcurves,
                              Traits& traits,
                              bool multiple_overlaps = false);}
{given a range of curves, compute all $x$-monotone subcurves that are pairwise
 disjoint in their interior, as induced by the input curves.
 If the flag \ccc{multiple_overlaps} is \ccc{true}, then a subcurve that
 represents an overlap of $k$ input curves is reported $k$ times; otherwise,
 each subcurve is reported only once. The \ccc{Traits} type must be a model
 of the \ccc{ArrangementTraits_2} concept, such that the value-type of
 \ccc{InputIterator} is \ccc{Traits::Curve_2}, and the value-type of
 \ccc{OutputIterator} is \ccc{Traits::X_monotone_curve_2}.}

\end{ccRefFunction}

\ccRefPageEnd

