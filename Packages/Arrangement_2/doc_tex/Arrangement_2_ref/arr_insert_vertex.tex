% +------------------------------------------------------------------------+
% | Reference manual page: arr_insert_vertex.tex
% +------------------------------------------------------------------------+
% | 
% | Package: Arrangement_2
% | 
% +------------------------------------------------------------------------+

\ccRefPageBegin

\begin{ccRefFunction}{insert_point}

\ccInclude{CGAL/Arrangement_2.h}

\ccFunction{template<class Traits, class Dcel, class PointLocation>
            typename Arrangement_2<Traits,Dcel>::Vertex_handle
            insert_point (Arrangement_2<Traits,Dcel>& arr,
                           const typename Traits::Point_2& p,
                           const PointLocation& pl);}
   {Inserts the point \ccc{p} as an arrangement vertex in \ccc{arr}.
    The function uses the point-location object \ccc{pl} (a model of the
    \ccc{ArrangementPointLocation_2} concept) to locate \ccc{p} in \ccc{arr}.
    If it conincide with an existing vertex, we are done; if it lies on an
    edge, the edge is split at \ccc{p}; otherwise, \ccc{p} is contained inside
    a face, and is inserted as an isolated vertex inside this face.
    At any case, the function returns a handle for the vertex associated
    with \ccc{p}. The \ccc{Triats} parameter is a model of the refined
    \ccc{ArrangementXMonotoneTraits_2}, such that it supports split operations.
    \ccPrecond{\ccc{pl} is attached to the given arrangement \ccc{arr}.}}

\ccFunction{template<class Traits, class Dcel>
            typename Arrangement_2<Traits,Dcel>::Vertex_handle
            insert_point (Arrangement_2<Traits,Dcel>& arr,
                           const typename Traits::Point_2& p);}
   {Inserts the point \ccc{p} as an arrangement vertex in \ccc{arr}.
    This function operated just like the function above, but it uses the
    default ``walk along line'' point-location strategy, so users need not
    provide a point-location object.} 

\end{ccRefFunction}

\ccRefPageEnd
