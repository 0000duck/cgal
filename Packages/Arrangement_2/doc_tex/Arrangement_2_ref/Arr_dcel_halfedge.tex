% +------------------------------------------------------------------------+
% | Reference manual page: Arr_dcel_halfedge.tex
% +------------------------------------------------------------------------+
% | 
% | Package: Arrangement_2
% | 
% +------------------------------------------------------------------------+

\ccRefPageBegin

\begin{ccRefConcept}{ArrangementDcelHalfedge}

\ccDefinition

A halfedge record in a \dcel\ data structure. Two halfedges with opposite
directions always form an edge (a halfedge pair). The halfedges form together
chains, defining the boundaries of connected components, such that all
halfedges along a chain have the same incident face.

An edge is always associated with a curve, but the halfedge records only
store a pointer to the associated curve, and the actual curve objects
are stored elsewhere. Two opposite halfedges are always associated with
the same curve.

\ccCreationVariable{e}

\ccTypes
%=======

\ccNestedType{Vertex}{the corresponding \dcel\ halfedge type.}
\ccGlue
\ccNestedType{Face}{the corresponding \dcel\ face type.}

\ccNestedType{X_monotone_curve}{the curve type associated with the edge.} 

\ccCreation
%==========

\ccConstructor{Arr_dcel_halfedge();}
   {default constructor.}

\ccMethod{void assign (const Self& other);}
   {assigns \ccVar{} with the contents of the \ccc{other} halfedge.}

\ccHtmlNoLinksFrom{   % to avoid linkage of Vertex, etc to HDS::Vertex

\ccAccessFunctions
%=================

All functions below also have \ccc{const} counterparts, returning non-mutable
pointers or references:

  \ccMethod{Halfedge* opposite();}
    {returns the twin halfedge.}

  \ccMethod{Comparison_result direction() const;}
    {returns \ccc{SMALLER} if \ccVar{}'s source vertex is lexicographically
     smaller than it target, and \ccc{LARGER} if it is lexicographically
     larger than the target.}

  \ccMethod{Halfedge* prev();}
    {returns the previous halfedge along the chain.}

  \ccMethod{Halfedge* next();}
    {returns the next halfedge along the chain.}

  \ccMethod{Vertex* vertex();}
    {returns the target vertex.}

  \ccMethod{Face* face();}
    {returns the incident face.}

  \ccMethod{X_monotone_curve& curve();}
    {returns the associated curve.}

\ccModifiers
%===========

  \ccMethod{void set_opposite(Halfedge* opp);}
    {sets the opposite halfedge.}

  \ccMethod{void set_direction (Comparison_result dir);}
    {sets the lexicographical order between \ccVar{}'s source and target
     vertices to be \ccc{dir}. The direction of the opposite halfedge is
     also set to the opposite direction.
     \ccPrecond{\ccc{dir} is either \ccc{SMALLER} or \ccc{LARGER}
                (and not \ccc{EQUAL}).}}

  \ccMethod{void set_prev(Halfedge* prev);}
    {sets the previous halfedge of \ccVar{} along the chain,
     and updates the cross-pointer \ccc{prev->next()}.}

  \ccMethod{void set_next(Halfedge* next);}
    {sets the next halfedge of \ccVar{} along the chain,
     and updates the cross-pointer \ccc{next->prev()}.}

  \ccMethod{void set_vertex(Vertex* v);}
    {sets the target vertex.}

  \ccMethod{void set_face(Face* f);}
    {sets the incident face.}

  \ccMethod{void set_curve (X_monotone_curve* c);}
    {sets the associated curve of \ccVar{} and its opposite halfedge.}

} % ccHtmlNoLinksFrom

\ccSeeAlso

     \ccc{ArrangementDcel}\lcTex{ 
     (\ccRefPage{ArrangementDcel})}\\
     \ccc{ArrangementDcelVertex}\lcTex{
     (\ccRefPage{ArrangementDcelVertex})}\\
     \ccc{ArrangementDcelFace}\lcTex{
     (\ccRefPage{ArrangementDcelFace})}

\end{ccRefConcept}  

\ccRefPageEnd
