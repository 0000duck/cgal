\ccRefPageBegin
\begin{ccRefClass}{Arr_non_caching_segment_traits_2<Kernel>}
    
\ccDefinition 
  The traits class \ccStyle{Arr_non_caching_segment_traits_2<Kernel>} is a
  model of the \ccc{ArrangementTraits_2} concept. It is provided as part of
  the package and can be used as reference for resolving ambiguities. 
  It is templated with a CGAL-Kernel model, and it is derived from it. This
  class is a thin layer above the parameterized kernel. It inherits the kernel
  types and functionality and it complements it with the necessary functors that
  are not provided by the kernel. It's implementation is simple, yet may lead to
  a cascaded representation of intersection points with exponentially long
  bit-length. The \ccStyle{Arr_segment_traits_2} traits class avoids this
  cascading problem. The kernel is parameterized with a number type, which
  should be exact to avoid robustness problems, although other number types
  could be used at the user's own risk. The \ccc{Quotient<MP_Float>} number
  type represents floating-point numbers with an unbounded mantissa, but with a
  bounded exponent. In some cases the exponent may overflow when used with
  \ccStyle{Arr_non_caching_segment_traits_2<Kernel>}. Examples of exact
  kernels are \ccStyle{Cartesian<Gmpq>} and \ccStyle{Homogeneous<leda_integer>}.
    
\ccInclude{CGAL/Arr_segment_traits_2.h}
 
\ccIsModel
    \ccc{ArrangementTraits_2}

\end{ccRefClass}
\ccRefPageEnd
