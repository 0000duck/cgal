% +------------------------------------------------------------------------+
% | Reference manual page: ArrangementVerticalRayShoot_2.tex
% +------------------------------------------------------------------------+
% | 
% | Package: Arrangement_2
% | 
% +------------------------------------------------------------------------+

\ccRefPageBegin

\begin{ccRefConcept}{ArrangementVerticalRayShoot_2}

\ccDefinition
%============

A model of the \ccRefName\ concept can be attached to an \ccc{Arrangement_2}
instance and answer vertical ray-shooting queries on this arrangement.
Namely, given a \ccc{Arrangement_2::Point_2} object, representing a point in
the plane, it returns the arrangement feature (edge or vertex) that lies
strictly above it (or below it). By ``strictly'' we mean that if the
query point lies on an arrangement edge (or on an arrangement vertex) this
edge will {\em not} be the query result, but the feature lying above or
below it (an exception to this rule is the degenerate situation where the
query point lies in the interior of a vertical edge). Note that it may happen
that the query point lies above the upper envelope (or below the lower
envelope) of the arrangement, so that the vertical ray emanating from it
may go to infinity without hitting any arrangement feature on its way. In this
case we return the unbounded face as the feature lying above (or below) the
query point.

\ccTypes
%=======

\ccNestedType{Arrangement_2}{the associated arrangement type.}

\ccNestedType{Point_2}{equivalent to \ccc{Arrangement_2::Point_2}.}

\ccCreation
\ccCreationVariable{rs}
%======================

\ccConstructor{ArrangementVerticalRayShoot_2();}{default constructor.}

\ccConstructor{ArrangementVerticalRayShoot_2 (const Arrangement_2& arr);}
  {constructs a ray-shooting object \ccVar{} attached to the given
   arrangement \ccc{arr}.}

\ccQueryFunctions
%================

\ccMethod{Object ray_shoot_up (const Point_2& q) const;}
  {locates the arrangement feature that is first hit by an upward-facing
   vertical ray emanating from the query point \ccc{q},
   and returns a handle for this feature.
   The function returns an \ccc{Object} instance that may be a wrapper for
   one of the following types:
   \begin{itemize}
   \item \ccc{Arrangement_2::Halfedge_const_handle}, in case the vertical
         ray hits an arrangement edge;
   \item \ccc{Arrangement_2::Vertex_const_handle}, in case the vertical
         ray hits an arrangement vertex.
   \item \ccc{Arrangement_2::Face_const_handle} for the unbounded arrangement
         face in case \ccc{q} lies above the upper envelope of the arrangement.
   \end{itemize}
   \ccPrecond{\ccVar{} has been attached to a valid arrangement instance.}}

\ccMethod{Object ray_shoot_down (const Point_2& q) const;}
  {locates the arrangement feature that is first hit by a downward-facing
   vertical ray emanating from the query point \ccc{q},
   and returns a handle for this feature.
   The function returns an \ccc{Object} instance that may be a wrapper for
   one of the following types:
   \begin{itemize}
   \item \ccc{Arrangement_2::Halfedge_const_handle}, in case the vertical
         ray hits an arrangement edge;
   \item \ccc{Arrangement_2::Vertex_const_handle}, in case the vertical
         ray hits an arrangement vertex.
   \item \ccc{Arrangement_2::Face_const_handle} for the unbounded arrangement
         face in case \ccc{q} lies below the lower envelope of the arrangement.
   \end{itemize}
   \ccPrecond{\ccVar{} has been attached to a valid arrangement instance.}}

\ccOperations
%============

\ccMethod{void attach (const Arrangement_2& arr);}
  {attaches \ccVar{} to the given arrangement \ccc{arr}.}

\ccMethod{void detach ();}
  {detaches \ccVar{} from the arrangement it is currently attached to.}

\ccHasModels
%===========

\ccc{Arr_naive_point_location<Arrangement>}\\
\ccc{Arr_walk_along_a_line_point_location<Arrangement>} \\
\ccc{Arr_trapezoid_ric_point_location<Arrangement>}\\
\ccc{Arr_landmarks_point_location<Arrangement,Generator>}\\

\end{ccRefConcept}

\ccRefPageEnd
