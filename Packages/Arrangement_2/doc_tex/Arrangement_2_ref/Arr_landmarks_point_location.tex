% +------------------------------------------------------------------------+
% | Reference manual page: Arr_landmarks_point_location.tex
% +------------------------------------------------------------------------+
% | 
% | Package: Arrangement_2
% | 
% +------------------------------------------------------------------------+

\ccRefPageBegin

\begin{ccRefClass}{Arr_landmarks_point_location<Arrangement,Generator>}
\label{arr_ref:lm_pl}

The \ccRefName\ class uses a set of ``landmark'' points whose location
in the arrangement is known. Given a query point, it uses a \kdtree\ to
find the nearest landmark and then traverses the straight line
segment connecting this landmark to the query point.

There are various strategies to select the landmark set in the
arrangement, where the strategy is determined by the
\ccc{Generator} template parameter. The following landmark-generator
classes are available:
\begin{description}
\item[\ccc{Arr_landmarks_vertices_generator} ---]
The set of arrangement vertices is used as a landmarks set.

\item[\ccc{Arr_middle_edges_landmarks_generator} ---]
The midpoint of each arrangement edge is computed, and the resulting
set of points is used as the landmarks set. This generator can be applied
only for arrangements of line segments.

\item[\ccc{Arr_grid_landmarks_generator} ---]
A set of $n$ landmarks are chosen on a 
$\lceil \sqrt{n} \rceil \times \lceil \sqrt{n} \rceil$ 
grid that covers the bounding box of the arrangement.

\item[\ccc{Arr_halton_landmarks_generator} ---]
$n$ Halton points in the bounding box of the arrangement are chosen
as the landmarks set.

\item[\ccc{Arr_random_landmarks_generator} ---]
$n$ random points in the bounding box of the arrangement are chosen
as the landmarks set.
\end{description}
The \ccc{Arr_landmarks_vertices_generator} class is the default generator
and used if no \ccc{Generator} parameter is specified.

It is recommended to use the landmarks point-location strategy
when the application frequently issues point-location queries on a
rather static arrangement that only seldom changes.

\ccInclude{CGAL/Arr_landmarks_point_location.h}

\ccIsModel
  \ccc{ArrangementPointLocation_2} \\
  \ccc{ArrangementVerticalRayShoot_2}

\end{ccRefClass}

\ccRefPageEnd
