% +------------------------------------------------------------------------+
% | Reference manual page: Arr_text_formatter.tex
% +------------------------------------------------------------------------+
% | 
% | Package: Arrangement_2
% | 
% +------------------------------------------------------------------------+

\ccRefPageBegin

\begin{ccRefClass}{Arr_text_formatter<Arrangement>}

\ccDefinition
%============

\ccRefName\ defines the format of an arrangement in an input or output stream
(typically a file stream), thus enabling reading and writing an \ccc{Arrangement}
instance using a simple text format. The arrangement is assumed to store no auxiliary
data with its \dcel\ records (and if there are such records they will not be written
or read by the formatter).

The \ccRefName\ class assumes that the nested \ccc{Point_2} and the \ccc{Curve_2} types
defined by the \ccc{Arrangement} template-parameter can both be written to an input
stream using the \ccc{<<} operator and read from an input stream using the \ccc{>>}
operator.

\ccInclude{CGAL/IO/Arr_text_formatter.h}

\ccIsModel
    \ccc{ArrangementInputFormatter} \\
    \ccc{ArrangementOutputFormatter}

\ccSeeAlso
    \ccc{read}\lcTex{
     (\ccRefPage{CGAL::read})} \\
    \ccc{write}\lcTex{
     (\ccRefPage{CGAL::write})}

\end{ccRefClass}

\ccRefPageEnd
