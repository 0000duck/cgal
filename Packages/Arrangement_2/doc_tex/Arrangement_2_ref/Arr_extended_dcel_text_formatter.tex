% +------------------------------------------------------------------------+
% | Reference manual page: Arr_extended_dcel_text_formatter.tex
% +------------------------------------------------------------------------+
% | 
% | Package: Arrangement_2
% | 
% +------------------------------------------------------------------------+

\ccRefPageBegin

\begin{ccRefClass}{Arr_extended_dcel_text_formatter<Arrangement>}

\ccDefinition
%============

\ccRefName\ defines the format of an arrangement in an input or output stream
(typically a file stream), thus enabling reading and writing an \ccc{Arrangement}
instance using a simple text format. The \ccc{Arrangement} class should be
instantiated with a \dcel\ class which in turn instatiates the
\ccc{Arr_extended_dcel} template with the \ccc{VertexData}, \ccc{HalfedgeData} and
\ccc{FaceData} types.
The formatter supports reading and writing the data objects attached to the
arrangement vertices, halfedges and faces.

The \ccRefName\ class assumes that the nested \ccc{Point_2} and the \ccc{Curve_2} types
defined by the \ccc{Arrangement} template-parameter, as well as the \ccc{VertexData},
\ccc{HalfedgeData} and \ccc{FaceData} types, can all be written to an input stream using
the \ccc{<<} operator and read from an input stream using the \ccc{>>} operator.

\ccInclude{CGAL/IO/Arr_text_formatter.h}

\ccIsModel
    \ccc{ArrangementInputFormatter} \\
    \ccc{ArrangementOutputFormatter}

\ccSeeAlso
    \ccc{read}\lcTex{
     (\ccRefPage{CGAL::read})} \\
    \ccc{write}\lcTex{
     (\ccRefPage{CGAL::write})} \\
    \ccc{Arr_extended_dcel<Traits,VData,HData,FData,V,H,F>}\lcTex{
     (\ccRefPage{CGAL::Arr_extended_dcel<Traits,VData,HData,FData,V,H,F>})}

\end{ccRefClass}

\ccRefPageEnd
