% +------------------------------------------------------------------------+
% | Reference manual page: ArrangementBasicTraits.tex
% +------------------------------------------------------------------------+
% | 
% | Package: Arrangement_2
% | 
% +------------------------------------------------------------------------+

\ccRefPageBegin

\begin{ccRefConcept}{ArrangementBasicTraits_2}

\ccDefinition

The basic arrangement-traits concept defines the minimal set of geometric
predicates needed for the constrcution and maintenance of instances of the
\ccc{Arrangement_2} class, as well as performing simple queries (such as
point-location queries) on such arrangements.

A model of this concept must define nested \ccc{Point_2} and
\ccc{X_monotone_curve_2} types, which represent planar points and
$x$-monotone curves (a vertical segment is also considered to be
{\sl weakly} $x$-monotone), respectively. The $x$-monotone curves are assumed
to be pairwise disjoint in their interiors, so they do not intersect, and
their endpoints are representable as \ccc{Point_2} objects.

\ccTypes
%=======

\ccNestedType{Point_2}{represents a point on the plane.}

\ccNestedType{X_monotone_curve_2}{represents a planar (weakly) $x$-monotone
                                  curve.}

\ccHeading{Tags}
%===============

\ccNestedType{Has_left_category}{indicates whether the nested functor
                                 \ccc{Compare_at_x_left_2} is provided.}

\ccHeading{Functor Types}
%========================

\ccThree{Compare_y_at_x_2~~~}{}{\hspace*{12cm}}
\ccThreeToTwo

\ccNestedType{Compare_x_2}
{provides the operator~: \\
 \ccc{Comparison_result operator() (Point_2 p1, Point_2 p2)} \\
 which returns \ccc{SMALLER, EQUAL} or \ccc{LARGER} according to the
 $x$-ordering of points \ccc{p1} and \ccc{p2}.}

\ccNestedType{Compare_xy_2}
{provides the operator~: \\
 \ccc{Comparison_result operator() (Point_2 p1, Point_2 p2)} \\
 which returns \ccc{SMALLER}, \ccc{EQUAL}, or \ccc{LARGER} according to the
 lexicographic $xy$-order of the points \ccc{p1} and \ccc{p2}.}

\ccNestedType{Construct_min_vertex_2}
{provides the operator~: \\
 \ccc{Point_2 operator() (X_monotone_curve_2 c)} \\
 which returns the lexicographically smaller (left) endpoint of \ccc{c}.}

\ccNestedType{Construct_max_vertex_2}
{provides the operator~: \\
 \ccc{Point_2 operator() (X_monotone_curve_2 c)} \\
 which returns the lexicographically larger (right) endpoint of \ccc{c}.}

\ccNestedType{Is_vertical_2}
{provides the operator~: \\
 \ccc{bool operator() (X_monotone_curve_2 c)} \\
 which determines whether \ccc{c} is a vertical segment.}

\ccNestedType{Compare_y_at_x_2}
{provides the operator~: \\
 \ccc{Comparison_result operator() (Point_2 p, X_monotone_curve_2 c)} \\
 which compares the $y$-coordinates of \ccc{p} and the vertical projection
 of \ccc{p} on \ccc{c}, and returns \ccc{SMALLER, EQUAL} or \ccc{LARGER}
 according to the result.}

\ccNestedType{Compare_y_at_x_left_2}
{provides the operator~: \\
 \ccc{Comparison_result operator() (X_monotone_curve_2 c1,
                                    X_monotone_curve_2 c2, Point_2 p)} \\
 which accepts two $x$-monotone curves \ccc{c1} and \ccc{c2} that have
 a common right endpoint \ccc{p}, and returns \ccc{SMALLER, EQUAL} or
 \ccc{LARGER} according to the relative position of the two curves
 immediately to the left of $p$. Note that in case one of the $x$-monotone
 curves is a vertical segment (emanating downward from \ccc{p}), it is always
 considered to be {\sl below} the other curve.}

\ccNestedType{Compare_y_at_x_right_2}
{provides the operator~: \\
 \ccc{Comparison_result operator() (X_monotone_curve_2 c1,
                                    X_monotone_curve_2 c2, Point_2 p)} \\
 which accepts two $x$-monotone curves \ccc{c1} and \ccc{c2} that have
 a common left endpoint \ccc{p}, and returns \ccc{SMALLER, EQUAL} or
 \ccc{LARGER} according to the relative position of the two curves
 immediately to the right of $p$. Note that in case one of the $x$-monotone
 curves is a vertical segment emanating upward from \ccc{p}, it is always
 considered to be {\sl above} the other curve.}

\ccNestedType{Equal_2}
{provides the operators~: \\
 \ccc{bool operator() (Point_2 p1, Point_2 p2)} \\
 which determines whether \ccc{p1} and \ccc{p2} are geometrically
 equivalent; and~: \\
 \ccc{bool operator() (X_monotone_curve_2 c1,
                       X_monotone_curve_2 c2)} \\
 which determines whether \ccc{c1} and \ccc{c2} are geometrically equivalent
 (have the same graph).}

\ccCreation
\ccCreationVariable{traits}
%==========================

\ccThree{Construct_x_monotone_curve_2~~~}{}{\hspace*{7cm}}
\ccThreeToTwo

\ccConstructor{ArrangementBasicTraits_2();}{default constructor.}
\ccGlue
\ccConstructor{ArrangementBasicTraits_2(ArrangementBasicTraits_2 other);}
{copy constructor}
\ccGlue
\ccMethod{ArrangementBasicTraits_2  operator=(other);}{assignment operator.}

\ccHeading{Accessing Functor Objects}
%====================================

\ccMethod{Compare_x_2 compare_x_2_object() const;} {}
\ccGlue
\ccMethod{Compare_xy_2 compare_xy_2_object() const;}{}
\ccGlue
\ccMethod{Construct_min_vertex_2 construct_min_vertex_2_object() const;}{}
\ccGlue
\ccMethod{Construct_max_vertex_2 construct_max_vertex_2_object() const;}{}
\ccGlue
\ccMethod{Is_vertical_2 is_vertical_2_object() const;} {}
\ccGlue
\ccMethod{Compare_y_at_x_2 compare_y_at_x_2_object() const;} {}
\ccGlue
\ccMethod{Compare_y_at_x_left_2 compare_y_at_x_left_2_object() const;} {}
\ccGlue
\ccMethod{Compare_y_at_x_right_2 compare_y_at_x_right_2_object() const;} {}
\ccGlue
\ccMethod{Equal_2 equal_2_object() const;} {}

\ccHasModels
%===========

\ccc{CGAL::Arr_segment_traits_2<Kernel>}\\
\ccc{CGAL::Arr_non_caching_segment_basic_traits_2<Kernel>}\\
\ccc{CGAL::Arr_non_caching_segment_traits_2<Kernel>}\\
\ccc{CGAL::Arr_polyine_traits_2<SegmentTraits>}\\
\ccc{CGAL::Arr_circle_segment_traits_2<Kernel>} \\
\ccc{CGAL::Arr_conic_traits_2<RatKernel,AlgKernel,NtTraits>}\\
\ccc{CGAL::Arr_rational_arc_traits_2<AlgKernel,NtTraits>}\\
\ccc{CGAL::Arr_curve_data_traits_2<Tr,XData,Mrg,CData,Cnv>}\\
\ccc{CGAL::Arr_consolidated_curve_data_traits_2<Traits,Data>}

\end{ccRefConcept}

\ccRefPageEnd
