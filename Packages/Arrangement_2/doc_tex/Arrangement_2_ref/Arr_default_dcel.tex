% +------------------------------------------------------------------------+
% | Reference manual page: Arr_default_dcel.tex
% +------------------------------------------------------------------------+
% | 
% | Package: Arrangement_2
% | 
% +------------------------------------------------------------------------+

\ccRefPageBegin

\begin{ccRefClass}{Arr_default_dcel<Traits>}

\ccDefinition
%============

The default \dcel\ class used by the \ccc{Arrangement_2} class-template
is parameterized by a traits class, which is a model of the
\ccc{ArrangementBasicTraits_2} concept. It simply uses the nested
\ccc{Traits::Point_2} and \ccc{Traits::X_monotone_curve_2} to instantiate
the base vertex and halfedge types, respectively. Thus, the default \dcel\
records store no other information but the topological incidence relations
and the geometric data attached to vertices and edges.

\ccInclude{CGAL/Arr_default_dcel.h}

\ccIsModel
    \ccc{ArrangementDcel}

\ccInheritsFrom
    \ccc{Arr_dcel_base<Arr_vertex_base<typename Traits_::Point_2>,} \\
    ~~~~~~~~~~ \ccc{Arr_halfedge_base<typename Traits_::X_monotone_curve_2>,}\\
    ~~~~~~~~~~ \ccc{Arr_face_base>}

\ccTypes

\ccNestedType{template <class T> rebind}
   {allows the rebinding of the \dcel\ with a different traits class \ccc{T}.}

\ccSeeAlso

\ccc{Arr_dcel_base<V,H,F>}\lcTex{
     (\ccRefPage{Arr_dcel_base<V,H,F>})}

\end{ccRefClass}

\ccRefPageEnd
