% +------------------------------------------------------------------------+
% | Reference manual page: arr_read.tex
% +------------------------------------------------------------------------+
% | 
% | Package: Arrangement_2
% | 
% +------------------------------------------------------------------------+

\ccRefPageBegin

\begin{ccRefFunction}{read}

\ccInclude{CGAL/IO/Arr_iostream.h}

\ccFunction{template<class Traits, class Dcel, class Formatter>
            std::istream& read (Arrangement_2<Traits,Dcel>& arr,
                                std::istream& is,
                                Formatter& formatter);}
   {Reads the arrangement \ccc{arr} from the given output stream using a specific
    input format. \ccc{formatter}, which must be a model of the
    \ccc{ArrangementInputFormatter}, defines the input format.}

\ccFunction{template<class Traits, class Dcel>
            std::istream& operator>> (std::ostream& is,
                                      Arrangement_2<Traits,Dcel>& arr);}
   {Reads the arrangement \ccc{arr} from the given input stream using the
    output format defined by the \ccc{Arr_text_formatter} class --- that is,
    only the basic geometric and topological features of the arrangement are
    read and no auxiliary data is attached to the \dcel\ features.}

\end{ccRefFunction}

\ccRefPageEnd
