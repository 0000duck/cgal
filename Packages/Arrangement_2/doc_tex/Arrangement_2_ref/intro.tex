% ===============================================================
\section*{Introduction}
\label{arr_ref_sec:intro}
% =======================
Given a set $\mathcal{C}$ of planar curves, the {\em arrangement}
${\mathcal A}({\mathcal C})$ is the subdivision of the plane induced
by the curves in $\mathcal{C}$ into maximally connected cells. The cells
can be $0$-dimensional ({\em vertices}), $1$-dimensional ({\em edges})
or $2$-dimensional ({\em faces}).

The class \ccc{Arrangement_2<Traits,Dcel>} encapsulates a data structure
that maintains arrangements of arbitrary bounded planar curves. It comes
with a variety of algorithms that operate on planar arrangement,
such as point-location queries and overlay computations, which are
implemented as peripheral classes or as global functions.

\begin{ccTexOnly}

\subsection*{Concepts}

\ccRefConceptPage{ArrangementDcel}\\
\ccRefConceptPage{ArrangementDcelVertex}\\
\ccRefConceptPage{ArrangementDcelHalfedge}\\
\ccRefConceptPage{ArrangementDcelFace}\\
~\\
\ccRefConceptPage{ArrangementBasicTraits_2}\\
\ccRefConceptPage{ArrangementLandmarkTraits_2}\\
\ccRefConceptPage{ArrangementXMonotoneTraits_2}\\
\ccRefConceptPage{ArrangementTraits_2}\\
~\\
\ccRefConceptPage{ArrangementPointLocation_2}\\
\ccRefConceptPage{ArrangementVerticalRayShoot_2}\\


\subsection*{Classes}

\ccRefIdfierPage{CGAL::Arrangement_2<Traits,Dcel>}\\
~\\
\ccRefIdfierPage{CGAL::Arrangement_2<Traits,Dcel>::Vertex}\\
\ccRefIdfierPage{CGAL::Arrangement_2<Traits,Dcel>::Halfedge}\\
\ccRefIdfierPage{CGAL::Arrangement_2<Traits,Dcel>::Face}\\
\ccRefIdfierPage{CGAL::Arr_dcel_base<V,H,F>}\\
\ccRefIdfierPage{CGAL::Arr_default_dcel<Traits>}\\
~\\
\ccRefIdfierPage{CGAL::Arr_segment_traits_2<Kernel>}\\
\ccRefIdfierPage{CGAL::Arr_non_caching_segment_traits_2<Kernel>}\\
\ccRefIdfierPage{CGAL::Arr_polyline_traits_2<Segment_traits_2>}\\
\ccRefIdfierPage{CGAL::Arr_conic_traits_2<Rat_kernel,Alg_kernel,Nt_traits>}\\
\ccRefIdfierPage{CGAL::Arr_rational_arc_traits_2<Alg_kernel,Nt_traits>}\\
~\\
\ccRefIdfierPage{CGAL::Arr_naive_point_location<Arrangement>}\\
\ccRefIdfierPage{CGAL::Arr_walk_along_a_line_point_location<Arrangement>}\\
\ccRefIdfierPage{CGAL::Arr_trapezoid_ric_point_location<Arrangement>}\\
\ccRefIdfierPage{CGAL::Arr_landmark_point_location<Arrangement,Generator>}\\


\subsection*{Functions}

\ccRefIdfierPage{CGAL::insert}\\
\ccRefIdfierPage{CGAL::insert_x_monotone}\\
\ccRefIdfierPage{CGAL::insert_non_intersecting}\\
\ccRefIdfierPage{CGAL::insert_vertex}\\
\ccRefIdfierPage{CGAL::remove}\\
\ccRefIdfierPage{CGAL::locate}\\
\ccRefIdfierPage{CGAL::overlay}\\

\end{ccTexOnly}    
