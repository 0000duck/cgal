% +------------------------------------------------------------------------+
% | Reference manual page: arr_remove.tex
% +------------------------------------------------------------------------+
% | 
% | Package: Arrangement_2
% | 
% +------------------------------------------------------------------------+

\ccRefPageBegin

\begin{ccRefFunction}{remove_vertex}

\ccInclude{CGAL/Arrangement_2.h}

\ccFunction{template <class Traits, class Dcel>
            bool remove_vertex (Arrangement_2<Traits,Dcel>& arr,
                    typename Arrangement_2<Traits,Dcel>::Vertex_handle v);}
   {Tries to removed the vertex \ccc{v} from the given arrangement \ccc{arr}.
    The vertex can be removed if it is either an isolated vertex and has no
    incident edge, or if it is a {\sl redundant} vertex --- that is, it has
    exactly two incident edges whose associated curves can be merged to form
    a single $x$-monotone curve. The assumption is that \ccc{Traits} is a
    model of the refines \ccc{ArrangementXMonotoneTraits_2}, such that it
    supports merge operations on $x$-monotone curves.
    The function returns whether it succeeded in deleting \ccc{v} from the
    arrangement.}

\end{ccRefFunction}

\ccRefPageEnd
