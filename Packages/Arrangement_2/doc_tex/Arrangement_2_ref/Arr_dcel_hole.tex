% +------------------------------------------------------------------------+
% | Reference manual page: Arr_dcel_hole.tex
% +------------------------------------------------------------------------+
% | 
% | Package: Arrangement_2
% | 
% +------------------------------------------------------------------------+

\ccRefPageBegin

\begin{ccRefConcept}{ArrangementDcelHole}

\ccDefinition

A hole record in a \dcel\ data structure, which stores the face that contains
the hole in its interior, along with an iterator for the hole in the holes'
container of this face.

\ccCreationVariable{ho}

\ccTypes
%=======

\ccNestedType{Face}{the corresponding \dcel\ face type.}

\ccTypedef{typedef Face::Hole_iterator Hole_iterator;} {}

\ccCreation
%==========

\ccConstructor{Arr_dcel_hole();}
   {default constructor.}

\ccHtmlNoLinksFrom{   % to avoid linkage of Vertex, etc to HDS::Vertex

\ccAccessFunctions
%=================

All functions below also have \ccc{const} counterparts, returning non-mutable
pointers or iterators:

\ccMethod{Face* face ();}
    {returns the incident face, which contains \ccVar{} in its interior.}

\ccMethod{Hole_iterator iterator();}
    {returns an iterator for the hole.}

\ccModifiers
%===========

  \ccMethod{void set_face (Face* f);}
    {sets the incident face.}

  \ccMethod{void set_iterator (Hole_iterator it);}
    {sets the hole iterator.}

} % ccHtmlNoLinksFrom

\ccSeeAlso

\ccc{ArrangementDcel}\lcTex{ 
     (\ccRefPage{ArrangementDcel})}\\
\ccc{ArrangementDcelFace}\lcTex{
     (\ccRefPage{ArrangementDcelFace})}

\end{ccRefConcept}  

\ccRefPageEnd
