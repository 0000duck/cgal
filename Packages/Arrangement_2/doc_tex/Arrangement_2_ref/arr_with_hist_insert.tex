% +------------------------------------------------------------------------+
% | Reference manual page: arr_with_hist_insert.tex
% +------------------------------------------------------------------------+
% | 
% | Package: Arrangement_2
% | 
% +------------------------------------------------------------------------+

\ccRefPageBegin

\begin{ccRefFunction}{insert_curve}

\ccInclude{CGAL/Arrangement_with_history_2.h}

\ccThree{Orientation~~}{}{\hspace*{8.5cm}}

\ccFunction{template<class Traits, class Dcel, class PointLocation>
            typename Arrangement_with_history_2<Traits,Dcel>::Curve_handle
            insert_curve (Arrangement_with_history_2<Traits,Dcel>& arr,
                    const typename Traits::Curve_2& c,
                    const PointLocation& pl = walk_pl);}
   {Inserts the given curve \ccc{c} into the arrangement \ccc{arr}, and returns
    a handle to the inserted curve.
    The point-location object \ccc{pl}, which should be a model of the
    \ccc{ArrangementPointLocation_2} concept, is used for answering
    point-location queries during the insertion process.
    By default, the function uses the ``walk along line'' point-location
    strategy --- namely an instance of the class
    \ccc{Arr_walk_along_line_point_location<Arrangement_with_history_2<Traits,Dcel> >}.
    \ccPrecond{If provided, \ccc{pl} is attached to the given arrangement
    \ccc{arr}.}}

%\ccFunction{template<class Traits, class Dcel>
%            typename Arrangement_with_history_2<Traits,Dcel>::Curve_handle
%            insert_curve (Arrangement_with_history_2<Traits,Dcel>& arr,
%                    const typename Traits::Curve_2& c);}
%   {Inserts the given curve \ccc{c} into the arrangement \ccc{arr}, and
%    returns a handle to the inserted curve.
%   The function operates as the function listed above, but it uses the default
%    ``walk along line'' point-location strategy, so users need not provide a
%    point-location object.}

\end{ccRefFunction}

\ccRefPageEnd
