\ccRefPageBegin
\begin{ccRefClass}{Arr_polyline_traits_2<Segment_traits>}

\ccDefinition

  The traits class \ccRefName\ is a model of the \ccc{ArrangementTraits_2}
  concept. It handles piecewise linear curves, commonly referred to as
  polylines. Each polyline is a chain of segments, where each two neighboring
  segments in the chain share a common endpoint. The traits class exploits the
  functionality of the \ccc{Segment_traits} template parameter to handle the
  segments that comprise the polyline curve.

  The class instantiated for the template parameter \ccc{Segment_traits} must
  be a model of the \ccc{ArrangementTraits_2} concept that handles line
  segments (e.g., \ccc{Arr_segment_traits_2<Kernel>} or 
  \ccc{Arr_segment_cached_traits_2<Kernel>}).

  The number type used by the injected segment traits should be exact to avoid
  robustness problems, although other number types could be used at the user's
  own risk.

\ccInclude{CGAL/Arr_polyline_traits_2.h}

\ccIsModel
  \ccc{ArrangementTraits_2} \lcTex{(\ccRefPage{ArrangementTraits_2}).}

\subsection*{Class \ccc{Arr_polyline_traits_2<Segment_traits>::Curve_2}}

The \ccc{Curve_2} class nested within the polyline traits is used to
represent (general) piecewise linear curves and support their construction
from any range of points. The copy and default constructor as well as 
the assignment operator are provided for polyline curves. In addition, 
an \ccc{operator<<} for the curves is defined for standard output streams.

\begin{ccClass}{Arr_polyline_traits_2<Segment_traits>::Curve_2}

\ccTypes

  \ccNestedType{const_iterator}{A reversible iterator for traversing the
    points that comprise the polyline curves. Supports the increment 
    (\ccc{operator++}), decrement (\ccc{operator--}), dereference 
    (\ccc{operator*}) and equality (\ccc{operator==} and \ccc{operator!=}) 
    operators.}

  \ccCreation
  \ccCreationVariable{cv}

  \ccStyle{template <class Iterator>} \\
  \ccConstructor{Curve_2 (const Iterator& begin, const Iterator& end);}
  {constructs a polyline defined by the given range of points.
   The \ccc{Iterator} type must be \ccc{Segment_traits::Point_2}.
   If the range contains $n$ points labeled $(p_{0},p_{1},\ldots,p_{n-1})$,
   the generated polyline consists of $n-1$ segments, where the $k$th segment 
   is defined by the endpoints $[p_{k-1},p_{k}]$. The first point in the 
   range is considered as the source point of the polyline while the last 
   point is considered as its target.
   \ccPrecond{There are at least two points in the range}.}

\ccOperations

\ccMethod{const_iterator begin() const;}
  {returns a \ccc{const_iterator} pointing at the source point of the 
   polyline.}

\ccMethod{const_iterator end() const;}
  {returns a \ccc{const_iterator} pointing after the end of the polyline.}

\ccMethod{const_iterator rbegin() const;}
  {returns a \ccc{const_iterator} pointing at the target point of the 
   polyline.}

\ccMethod{const_iterator rend() const;}
  {returns a \ccc{const_iterator} pointing before the beginning of the 
   polyline.}

\ccMethod{int points() const;}
  {returns the number of points that comprise the polyline. Note that if there
   are $n$ points in the polyline, it is comprised of $n-1$ segments.}

\end{ccClass}

\subsection*{Class \ccc{Arr_polyline_traits_2<Segment_traits>::X_monotone_curve_2}}

The \ccc{X_monotone_curve_2} class nested within the polyline traits is used
to represent $X$-monotone piecewise linear curves. It inherits from the
\ccc{Curve_2} type. The chain of segments that comprise the $X$-monotone
curve are ordered lexigoraphically.

\begin{ccClass}{Arr_polyline_traits_2<Segment_traits>::X_monotone_curve_2}
\end{ccClass}

\end{ccRefClass}
\ccRefPageEnd
