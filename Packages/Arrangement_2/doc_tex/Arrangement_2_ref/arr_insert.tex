% +------------------------------------------------------------------------+
% | Reference manual page: arr_insert.tex
% +------------------------------------------------------------------------+
% | 
% | Package: Arrangement_2
% | 
% +------------------------------------------------------------------------+

\ccRefPageBegin

\begin{ccRefFunction}{insert_curve}

\ccInclude{CGAL/Arrangement_2.h}

\ccFunction{template<class Traits, class Dcel, class PointLocation>
            void insert_curve (Arrangement_2<Traits,Dcel>& arr,
                               const typename Traits::Curve_2& c,
                               const PointLocation& pl = walk_pl);}
   {Inserts the given curve \ccc{c} into the arrangement \ccc{arr}, where
    no restrictions are made on the nature of the inserted curve.
    The \ccc{Traits} parameter must be a model of the
    \ccc{ArrangementTraits_2}
    concept. That is, it should define the \ccc{Curve_2} type and support
    its subdivision into $x$-monotone subcurves (and perhaps isolated points).
    Each subcurve is in turn inserted into the arrangement by locating its
    left endpoint and computing its zone until reaching the right endpoint.
    The point-location object \ccc{pl}, which must be a model of the
    \ccc{ArrangementPointLocation_2} concept, is used for answering
    point-location queries during the insertion process.
    By default, the function uses the ``walk along line'' point-location
    strategy --- namely an instance of the class
    \ccc{Arr_walk_along_line_point_location<Arrangement_2<Traits,Dcel> >}.
    \ccPrecond{If provided, \ccc{pl} must be attached to the given arrangement
    \ccc{arr}.}}

%\ccFunction{template<class Traits, class Dcel>
%            void insert_curve (Arrangement_2<Traits,Dcel>& arr,
%                               const typename Traits::Curve_2& c);}
%   {Inserts the given curve \ccc{c} into the arrangement \ccc{arr}, where
%    no restrictions are made on the nature of the inserted curve.
%   The function operates as the function listed above, but it uses the default
%    ``walk along line'' point-location strategy, so users need not provide a
%    point-location object.}

\begin{ccHtmlOnly}<p>\end{ccHtmlOnly}
\ccInclude{CGAL/Arrangement_with_history_2.h}

\ccThree{Orientation~~}{}{\hspace*{8.5cm}}

\ccFunction{template<class Traits, class Dcel, class PointLocation>
            typename Arrangement_with_history_2<Traits,Dcel>::Curve_handle
            insert_curve (Arrangement_with_history_2<Traits,Dcel>& arr,
                    const typename Traits::Curve_2& c,
                    const PointLocation& pl = walk_pl);}
   {Inserts the given curve \ccc{c} into the arrangement \ccc{arr}, and returns
    a handle to the inserted curve.
    The point-location object \ccc{pl}, which should be a model of the
    \ccc{ArrangementPointLocation_2} concept, is used for answering
    point-location queries during the insertion process.
    By default, the function uses the ``walk along line'' point-location
    strategy --- namely an instance of the class
    \ccc{Arr_walk_along_line_point_location<Arrangement_with_history_2<Traits,Dcel> >}.
    \ccPrecond{If provided, \ccc{pl} is attached to the given arrangement
    \ccc{arr}.}}

%\ccFunction{template<class Traits, class Dcel>
%            typename Arrangement_with_history_2<Traits,Dcel>::Curve_handle
%            insert_curve (Arrangement_with_history_2<Traits,Dcel>& arr,
%                    const typename Traits::Curve_2& c);}
%   {Inserts the given curve \ccc{c} into the arrangement \ccc{arr}, and
%    returns a handle to the inserted curve.
%   The function operates as the function listed above, but it uses the default
%    ``walk along line'' point-location strategy, so users need not provide a
%    point-location object.}

\end{ccRefFunction}

\ccRefPageEnd
