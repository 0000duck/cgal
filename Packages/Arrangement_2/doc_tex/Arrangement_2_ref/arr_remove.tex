% +------------------------------------------------------------------------+
% | Reference manual page: arr_remove.tex
% +------------------------------------------------------------------------+
% | 
% | Package: Arrangement_2
% | 
% +------------------------------------------------------------------------+

\ccRefPageBegin

\begin{ccRefFunction}{remove}

\ccInclude{CGAL/Arrangement_2.h}

\ccFunction{template <class Traits, class Dcel>
            bool remove_vertex (Arrangement_2<Traits,Dcel>& arr,
                    typename Arrangement_2<Traits,Dcel>::Vertex_handle v);}
   {Tries to removed the vertex \ccc{v} from the given arrangement \ccc{arr}.
    The vertex can be removed if it is either an isolated vertex and has no
    incident edge, or if it is a {\sl redundant} vertex --- that is, it has
    exactly two incident edges whose associated curves can be merged to form
    a single $x$-monotone curve. The assumption is that \ccc{Traits} is a
    model of the refines \ccc{ArrangementXMonotoneTraits_2}, such that it
    supports merge operations on $x$-monotone curves.
    The function returns whether it succeeded in deleting \ccc{v} from the
    arrangement.}

\ccFunction{template<class Traits, class Dcel>
            typename Arrangement_2<Traits,Dcel>::Face_handle
            remove_edge (Arrangement_2<Traits,Dcel>& arr,
                     typename Arrangement_2<Traits,Dcel>::Halfedge_handle e);}
   {Removes an edge (given by \ccc{e}, one of the twin halfedges that form it)
    from the arrangement \ccc{arr}. Using this function is equivalent to
    invoking \ccc{arr.remove_edge (e, true, true)} --- that is, to delete the
    edge and remove its end-vertices, if they become isolated. However, this
    free function requires that \ccc{Traits} is a model of the refines
    \ccc{ArrangementXMonotoneTraits_2}, such that it supports merge operations
    on $x$-monotone curves. Thus, if one of the end-vertices of \ccc{e} becomes
    redundant after \ccc{e} is removed (see above), it is removed and its
    incident edges are merged.
    If the edge-removal operation causes two faces to merge, the merged face
    is returned. Otherwise, the face to which the edge was incident is
    returned.}

\end{ccRefFunction}

\ccRefPageEnd
