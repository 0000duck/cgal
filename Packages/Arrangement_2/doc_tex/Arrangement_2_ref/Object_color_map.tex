% +------------------------------------------------------------------------+
% | Reference manual page: Object_color_map.tex
% +------------------------------------------------------------------------+
% | 
% | Package: Arrangement_2
% | 
% +------------------------------------------------------------------------+

\ccRefPageBegin

\begin{ccRefClass}{Object_color_map<Handle>}
    
\ccDefinition
%============

The class implements the BOOST ReadWritePropertyMap concept for CGAL handles 
to return and set a value of type \ccc{boost::default_color_type} from the 
object refferenced by the handle. The map just return and sets the color 
directly from the refferenced object.

The arrangement template has one parameter:
\begin{itemize}
\item The \ccc{Handle} template-parameter should be instantiated with
a class that is a handle. This parameter is the handle to the object that has
the color.
\end{itemize}

The object refferenced by the handle should have these two functions:
\begin{itemize}
\item data() --- to return a value of type \ccc{boost::default_color_type} that 
is the color of the object.
\item \ccc{set_data(boost::default_color_type)} --- to set the color of the 
object.
\end{itemize}

This object is mainly used combined with the \ccc{Arr_face_graph_adator}
in different BOOST algorithms, such as BFS and DFS.

\ccInclude{CGAL/Object_color_map.h}

\ccCreation
\ccCreationVariable{arr}
%=======================
    
\ccConstructor{Object_color_map();}
    {default constructor.}
    
\ccAccessFunctions
%=================

\ccMethod{boost::default_color_type get(Handle handle) const;}
    {returns the color of the object refferenced by the given handle.}
    
\ccModifiers
%===========

\ccMethod{void put(Handle& handle, const boost::default_color_type &color);}
   {sets the given color to be the color of the object refferenced by the 
handle.}
    
\ccSeeAlso
    \ccc{Arr_face_graph_adator}\lcTex{ 
      (\ccRefPage{Arr_face_graph_adator})}

\end{ccRefClass}

\ccRefPageEnd
