% +------------------------------------------------------------------------+
% | Reference manual page: arr_overlay.tex
% +------------------------------------------------------------------------+
% | 
% | Package: Arrangement_2
% | 
% +------------------------------------------------------------------------+

\ccRefPageBegin

\begin{ccRefFunction}{overlay}

\ccInclude{CGAL/Arr_overlay.h}

\ccFunction{template<class Traits, class Dcel1, class Dcel2,
                     class RedDcel, class OverlayTraits>
            void overlay (const Arrangement_2<Traits,Dcel1>& arr1,
	                  const Arrangement_2<Traits,Dcel2>& arr2,
                          Arrangement_2<Traits,ResDcel>& res,
                          OverlayTraits ovl_tr);}
   {Computes the overlay of two input arrangement instances \ccc{arr1} and
    \ccc{arr2} and sets the output arrangement \ccc{res} to represent the
    overlaid arrangement. All three arrangements are instantiated using the
    same geometric traits class, but may be represented using different
    \dcel\ classes. In order to properly construct the overlaid \dcel\ that
    represents \ccc{res}, the function uses the overlay-traits object
    \ccc{ovl_tr}, which should be a model of the \ccc{OverlayTraits} concept,
    which is able to construct records of the \ccc{ResDcel} class on the
    basis of the \ccc{Dcel1} and \ccc{Dcel2} records that induce them.
    \ccPrecond{\ccc{res} does not refer to either \ccc{arr1} or \ccc{arr2}
    (that is, ``self overlay'' is not supported).}}

\begin{ccHtmlOnly}<p>\end{ccHtmlOnly}
\ccInclude{CGAL/Arrangement_with_history_2.h}

\ccThree{Orientation~~}{}{\hspace*{8.5cm}}

\ccFunction{template<class Traits, class Dcel1, class Dcel2,
                     class RedDcel, class OverlayTraits>
            void overlay (const Arrangement_with_history_2<Traits,Dcel1>& arr1,
	                  const Arrangement_with_history_2<Traits,Dcel2>& arr2,
                          Arrangement_2<Traits,ResDcel>& res,
                          OverlayTraits ovl_tr);}
   {Computes the overlay of two input arrangement-with-history instances
    \ccc{arr1} and \ccc{arr2} and sets the output arrangement \ccc{res}
    to represent the overlaid arrangement. All three arrangements are
    instantiated using the same geometric traits class, but may be represented
    using different \dcel\ classes. In order to properly construct the
    overlaid \dcel\ that represents \ccc{res}, the function uses the
    overlay-traits object \ccc{ovl_tr}, which should be a model of the
    \ccc{OverlayTraits} concept, which is able to construct records of the
    \ccc{ResDcel} class on the basis of the \ccc{Dcel1} and \ccc{Dcel2}
    records that induce them. The function also constructs a consolidated set
    of curves that induce \ccc{res}.}

\ccSeeAlso
    \ccc{OverlayTraits}\lcTex{(\ccRefPage{OverlayTraits})}

\end{ccRefFunction}

\ccRefPageEnd
