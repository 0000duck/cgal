\begin{ccRefClass}{Segment_tree_k<Traits>}

\ccDefinition


An object of the class  \ccClassName\ is a $k$-dimensional segment tree
that can store k-dimensional intervals  of type \ccc{Interval}. 
The class allows to perform
window queries, enclosing queries, and inverse range queries on the keys.  The class \ccClassName\ is parameterized with
a segment tree traits class \ccc{Traits} that defines, among other things, 
the type of the \ccc{Interval}. 
In order to perform an inverse range query, a range query of
$\epsilon$ width has to be performed. We prefered not to offer an
extra function for this sort of query, since the inverse range
query is a special case of the range query. Furthermore, offering
an inverse range query in the segment tree class implies offering this
function also in the range tree class and having an extra item in
the traits class that accesses the inverse range query point.


\cgal\ provides traits class implementations that allow to use 
the segment tree with point classes from the \cgal\ kernel as keys.
These classes are \ccc{CGAL::Range_segment_tree_traits_set_2<R>},
\ccc{CGAL::Range_segment_tree_traits_set_3<R>}, 
\ccc{CGAL::Segment_tree_traits_map_2<R>} and
\ccc{CGAL::Segment_tree_traits_map_3<R>}.  The concept
RangeSegmentTreeTraits\_d defines the requirements that segment tree traits
classes must fulfill. This allows the advanced user to develop further
segment tree traits classes.

\ccInclude{CGAL/Segment_tree_k.h}


\ccTypes

\ccNestedType{Traits}{the type of the segment tree traits class.}
\ccTypedef{typedef Traits::Key Key;}{}
\ccTypedef{typedef Traits::Interval Interval;}{}

\ccCreationVariable{S}


\ccCreation

\ccConstructor{Segment_tree_k ();}
{Introduces an empty segment tree \ccVar.}

\ccConstructor{template < class ForwardIterator >
               Segment_tree_k (ForwardIterator first,
                                     ForwardIterator last);}%
{Introduces a segment tree \ccVar\ and initializes it with the data
in the range \ccc{[first, last)}.
\ccPrecond \ccc{value_type(first) == Traits::Interval}.}

\ccOperations

\ccMethod{template < class ForwardIterator >
          void 
          make_tree(ForwardIterator first,
                    ForwardIterator last);}%
{Introduces a segment tree \ccVar\ and initializes it with the data
in the range \ccc{[first, last)}. This function can only be applied
once on an empty segment tree.
\ccPrecond \ccc{value_type(first) == Traits::Interval}.}


\ccMethod{template < class OutputIterator >
          OutputIterator
          window_query(Interval window,
                       OutputIterator out);}%
{writes all intervals that have non empty intersection with interval \ccc{window} to the container
where \ccc{out} points to, and returns an output iterator that points
to the last location the function wrote to.
\ccPrecond  \ccc{value_type(out) == Traits::Interval}.}

\ccMethod{template < class OutputIterator >
          OutputIterator
          enclosing_query(Interval window,
                       OutputIterator out);}%
{writes all intervals that enclose in the interval \ccc{window} to the container
where \ccc{out} points to, and returns an output iterator that points
to the last location the function wrote to.
\ccPrecond  \ccc{value_type(out) == Traits::Interval}.}


\ccExample

This example illustrates the use of the predefined segment tree
on 3-dimensional interval data (with no value associated). After
the definition of the traits type and tree type, some intervals
are constructed and the tree is build according to the
intervals. Then, a window query is performed and the query
elements are given out.

\begin{verbatim}
#include <CGAL/Cartesian.h>
#include <CGAL/Segment_tree_k.h>
#include <CGAL/Range_segment_tree_traits.h>

typedef CGAL::Cartesian<int> K;
typedef CGAL::Range_segment_tree_set_traits_3<K> Traits;
typedef CGAL::Segment_tree_3<Traits> Segment_tree_3_type;

int main()
{
  typedef Traits::Interval Interval;
  typedef Traits::Key Key;
  std::list<Interval> InputList, OutputList;

  InputList.push_back(Interval(Key(1,5,7), Key(2,7,9)));
  InputList.push_back(Interval(Key(2,7,6), Key(3,8,9)));
  InputList.push_back(Interval(Key(6,9,5), Key(9,13,8)));
  InputList.push_back(Interval(Key(1,3,4), Key(3,9,8)));
 
  Segment_tree_3_type Segment_tree_3(InputList.begin(),InputList.end());

  Interval a(Key(3,6,5), Key(7,12,8));
  Segment_tree_3.window_query(a,std::back_inserter(OutputList));
  std::list<Interval>::iterator j = OutputList1.begin();
  std::cout << "\n window_query (3,6,5),(7,12,8) \n";
  while(j!=OutputList.end()){
    std::cout << (*j).first.x() << "," << (*j).first.y() << ",";
    std::cout << (*j).first.z() <<", " << (*j).second.x() << ",";
    std::cout << (*j).second.y() << "," << (*j).second.z() << std::endl; 
    j++;
  }
}
\end{verbatim}

\end{ccRefClass}
