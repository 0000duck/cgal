\begin{ccRefClass} {Segment_tree_d<Data, Window, Traits>}

\ccTypes
\ccNestedType{Data}{container \ccStyle{Data}.}
\ccNestedType{Window}{container \ccStyle{Window}.}
\ccNestedType{Traits}{class \ccStyle{Traits}.}

\ccCreation
\ccCreationVariable{s}
%\renewcommand{\ccAlternateThreeColumn}{\ccFalse}
\ccInclude{CGAL/Segment_tree_d.h}\\
\ccConstructor{Segment_tree_d<Data, Window,
  Traits> s(Tree_base<Data, Window> sublayer_tree)}%
{A segment tree is defined, such that the subtree of  each
  vertex is of the   same type prototype
\ccStyle{sublayer\_tree} is.\\
We assume that the dimension of the tree is $d$. This means, that
\ccStyle{ sublayer_tree} is a  prototype of a $d-1$-dimensional
tree. All data items of the $d$-dimensional segment tree
  have container type \ccStyle{Data}. The query window of the
  tree has container type
  \ccStyle{Window}. \ccStyle{Traits}
 provides access to the corresponding data slots of container
 \ccStyle{Data} and  \ccStyle{Window} for the $d$-th
 dimension. The traits  class \ccStyle{Traits}
 must at least provide all functions and type definitions
 described, for example, in the reference page for
 \ccc{tree_point_traits}.
The template class
 described there is fully generic and should fulfill the most
 requirements one can have.
In order
  to generate a one-dimensional segment tree instantiate %
  \ccStyle{Tree\_anchor<Data, Window> sublayer\_tree} %
  with the same template parameters \ccStyle{Data} and
  \ccStyle{Window} \ccStyle{Segment\_tree\_d} is defined. In
  order to construct a two-dimensional segment tree, create
  \ccStyle{Segment\_tree\_d} with
   a one-dimensional \ccStyle{Segment\_tree\_d} with the
  corresponding \ccStyle{Traits} of the first dimension.\\
  \ccStyle{Precondition: Traits::Data==Data and
    Traits::Window==Window.}}

\ccOperations
%\renewcommand{\ccAlternateThreeColumn}{\ccTrue}

\ccMethod{bool make_tree(In_it first, In_it last);}
{The tree is constructed according to the data items in the
  sequence between the element pointed by iterator \ccStyle{first} and
  iterator \ccStyle{last}.}
\ccPrecond{This function can only be called once. If it is the
first call the tree is build and \ccStyle{true} is
returned. Otherwise, nothing is done but a  \ccStyle{CGAL
  warning} is given and \ccStyle{false}
returned.}



\ccMethod{OutputIterator window_query(Window win, OutputIterator result);}
{ \ccStyle{
      win}$=[a_1,b_1),\ldots , [a_d,b_d)$, $a_i,b_i\in T_i$, $1\le
    i\le d$. All elements that
  intersect the associated $d$-dimensional interval of
\ccStyle{win} are placed in the
associated sequence container of  \ccStyle{OutputIterator}  and
returns an
output iterator that points
to the last location the function wrote to.
In order to perform an inverse range query, a range query of
$\epsilon$ width has to be performed.
}

\ccMethod{OutputIterator enclosing_query(Window win, OutputIterator result);}%
{ All elements that
  enclose the associated $d$-dimensional interval of
\ccStyle{win} are placed in the
associated sequence container of  \ccStyle{OutputIterator}  and returns an output iterator that points
to the last location the function wrote to.}

\ccMethod{bool is_valid();}{The tree structure is checked. For each
  vertex either the
  sublayer tree is a tree anchor, or it stores a (possibly empty)
  list of data items. In the first case, the sublayer tree of the
  vertex is checked on being valid. In the second case, each data
  item is checked weather it contains the associated interval of
  the vertex and does not contain the associated interval of the
  parent vertex or not. True is returned if the tree structure is valid,
  false otherwise.}

{\bf Protected Operations}

\ccMethod{bool is_inside(Window win,
  Data object);}{returns true, if the
  interval of \ccStyle{object} is contained in the
  interval of \ccStyle{win}. False otherwise.}


\ccMethod{bool is_anchor();}{returns false.}

\ccImplementation

A $d$-dimensional segment tree is constructed in ${O}(n\log n^d)$ time.
An inverse range query is performed in time ${O}(k+{\log}^d n )$, where $k$
is the number of reported intervals.
The tree uses ${O}(n\log n^d)$ storage.
\end{ccRefClass}
