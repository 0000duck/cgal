% +------------------------------------------------------------------------+
% | Reference manual page: ch_melkman.tex
% +------------------------------------------------------------------------+
% | 09.05.2001   Susan Hert and Stefan Schirra
% | Package: Convex_hull_2
% | 
% +------------------------------------------------------------------------+


\begin{ccRefFunction}{ch_melkman}  %% add template arg's if necessary
\ccIndexSubitemBegin{convex hull, 2D}{of polyline or polygon}
\ccIndexSubitemBegin{convex hull, 2D}{Melkman algorithm}

\ccDefinition
  
The function \ccRefName\ computes the counterclockwise sequence of
extreme points of a sequence of points that forms a simple polyline or polygon.

\ccInclude{CGAL/ch_melkman.C}

\ccFunction{template <class InputIterator, class OutputIterator>
            OutputIterator
            ch_melkman( InputIterator first, InputIterator last,  
                        OutputIterator result, 
                        const Traits& ch_traits = Default_traits);}
            {generates the counterclockwise sequence of extreme points
            of the points in the range [\ccc{first}, \ccc{beyond}). 
            The resulting sequence is placed starting at
            position \ccc{result}, and the past-the-end iterator for
            the resulting sequence is returned.
            \ccPrecond %\ccIndexSubitem[C]{ch_melkman}{preconditions}
            The source range [\ccc{first},\ccc{beyond}) corresponds 
            to a simple polyline. 
            [\ccc{first},\ccc{beyond}) does not contain \ccc{result}}.

The default traits class \ccc{Default_traits} is the kernel in which the
type \ccc{InputIterator::value_type} is defined.


\ccHeading{Requirements}
\begin{enumerate}
   \item    \ccc{InputIterator::value_type} and \ccc{OutputIterator::value_type}
            are equivalent to \ccc{Traits::Point_2}.
   \item    \ccc{Traits} contains the following subset of types from
            the concept ConvexHullTraits\_2 and their corresponding member
            %\ccIndexMainItem[c]{ConvexHullTraits_2}
            functions that return instances of these types:
            \begin{itemize}
                \item \ccc{Traits::Point_2},
                \item \ccc{Traits::Less_xy_2}, 
                \item \ccc{Traits::Leftturn_2}.
            \end{itemize}
\end{enumerate}

\ccSeeAlso

\ccRefIdfierPage{CGAL::ch_akl_toussaint} \\
\ccRefIdfierPage{CGAL::ch_bykat} \\
\ccRefIdfierPage{CGAL::ch_eddy} \\
\ccRefIdfierPage{CGAL::ch_graham_andrew} \\
\ccRefIdfierPage{CGAL::ch_jarvis} \\
\ccRefIdfierPage{CGAL::ch_melkman} \\
\ccRefIdfierPage{CGAL::convex_hull_2} 

\ccImplementation

It uses an implementation of Melkman's algorithm \cite{m-olcch-87}. Running
time of this is linear.

\ccIndexSubitemEnd{convex hull, 2D}{of polyline or polygon}
\ccIndexSubitemEnd{convex hull, 2D}{Melkman algorithm}
\end{ccRefFunction}

% +------------------------------------------------------------------------+
%%RefPage: end of main body, begin of footer
% EOF
% +------------------------------------------------------------------------+

