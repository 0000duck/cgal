% +------------------------------------------------------------------------+
% | Reference manual page: upper_hull_points_2.tex
% +------------------------------------------------------------------------+
% | 09.05.2001   Susan Hert and Stefan Schirra
% | Package: Convex_hull_2
% | 
% +------------------------------------------------------------------------+


\begin{ccRefFunction}{upper_hull_points_2}  %% add template arg's if necessary
\ccIndexMainItemBegin{upper hull, 2D}

\ccDefinition
  
The function \ccRefName\ generates the counterclockwise sequence of extreme
points on the upper hull of a given set of input points.

\ccInclude{CGAL/convex_hull_2.h}

\ccFunction{template <class InputIterator, class OutputIterator>
            OutputIterator
            upper_hull_points_2(InputIterator first, InputIterator beyond,
                                OutputIterator  result,
                                const Traits & ch_traits = Default_traits );}
           {generates the counterclockwise sequence of extreme points
            on the upper hull of the points in the range [\ccc{first},
            \ccc{beyond}). The resulting sequence is placed starting at
            position \ccc{result}, and the past-the-end iterator for
            the resulting sequence is returned.
            The sequence starts with the rightmost point,
            the leftmost point is not included.
            If there is only one extreme point ({\it i.e.}, the leftmost and
            rightmost point are equal), the extreme point is not reported.
            \ccPrecond% \ccIndexSubitem[C]{upper_hull_points_2}{preconditions}
            The source range [\ccc{first},\ccc{beyond}) does not contain
            \ccc{result}.}


The default traits class \ccc{Default_traits} is the kernel in which the
type \ccc{InputIterator::value_type} is defined.

The different treatment by \ccc{CGAL::lower_hull_points_2} of the case that 
all points are equal ensures that concatenation of lower and upper hull 
points gives the sequence of extreme points.

\ccHeading{Requirements}
\begin{enumerate}
   \item    \ccc{InputIterator::value_type} and \ccc{OutputIterator::value_type}
            are equilvalent to \ccc{Traits::Point_2}.
   \item    \ccc{Traits} contains the following subset of types from
            the concept ConvexHullTraits\_2 and their corresponding member
            %\ccIndexMainItem[c]{ConvexHullTraits_2}
            functions that return instances of these types:
            \begin{itemize}
                \item \ccc{Traits::Point_2},
                \item \ccc{Traits::Less_xy_2}, 
                \item \ccc{Traits::Less_yx_2},
                \item \ccc{Traits::Left_of_line_2}, 
                \item \ccc{Traits::Leftturn_2}.
            \end{itemize}
\end{enumerate}

\ccSeeAlso

\ccRefIdfierPage{CGAL::ch_graham_andrew} \\
\ccRefIdfierPage{CGAL::ch_graham_andrew_scan} \\
\ccRefIdfierPage{CGAL::lower_hull_points_2} 

\ccImplementation

This function uses Andrew's
variant of Graham's scan algorithm \cite{a-aeach-79,m-mdscg-84}.  The algorithm
has worst-case running time of  $O(n \log n)$ for $n$ input points.

\ccIndexMainItemEnd{upper hull, 2D}
\end{ccRefFunction}

% +------------------------------------------------------------------------+
%%RefPage: end of main body, begin of footer
% EOF
% +------------------------------------------------------------------------+

