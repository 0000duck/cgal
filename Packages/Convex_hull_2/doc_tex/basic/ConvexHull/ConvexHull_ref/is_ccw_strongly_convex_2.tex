% +------------------------------------------------------------------------+
% | Reference manual page: is_ccw_strongly_convex_2.tex
% +------------------------------------------------------------------------+
% | 09.05.2001   Susan Hert and Stefan Schirra
% | Package: Convex_hull_2
% | 
% +------------------------------------------------------------------------+


\begin{ccRefFunction}{is_ccw_strongly_convex_2}  
\ccIndexMainItemBegin{convexity checking, 2D}
\ccIndexSubitemBegin{polygon}{strongly convex}
\ccIndexSubitemBegin{strongly convex}{polygon}

\ccDefinition
  
The function \ccRefName\ determines if a given sequence of points defines
a counterclockwise-oriented, stongly convex polygon.  The algorithm requires
$O(n)$ time for a set of $n$ input points.

\ccInclude{CGAL/convexity_check_2.h}

\ccFunction{template <class ForwardIterator, class Traits>
            bool
            is_ccw_strongly_convex_2( 
                               ForwardIterator first,
                               ForwardIterator beyond,
                               const Traits & ch_traits = Default_traits);}
           {returns \ccc{true}, iff the point elements in 
            [\ccc{first},\ccc{beyond})
            form a counterclockwise-oriented strongly convex polygon.
           }


The default traits class \ccc{Default_traits} is the \cgal\ 
\ccc{Kernel_traits_2} class,
with the representation type determined by \ccc{ForwardIterator::value_type}.

\ccHeading{Requirements}
\ccc{Traits} contains the following subset of types from
the concept ConvexHullTraits\_2 and their corresponding member
%\ccIndexMainItem[c]{ConvexHullTraits_2}
functions that return instances of these types:
\begin{itemize}
   \item \ccc{Traits::Less_xy_2}, 
   \item \ccc{Traits::Leftturn_2}.
\end{itemize}


\ccSeeAlso

\ccRefIdfierPage{CGAL::is_cw_strongly_convex_2} \\
\ccRefIdfierPage{CGAL::is_strongly_convex_3} 

\ccIndexMainItemEnd{convexity checking, 2D}
\ccIndexSubitemEnd{polygon}{strongly convex}
\ccIndexSubitemEnd{strongly convex}{polygon}
\end{ccRefFunction}

% +------------------------------------------------------------------------+
%%RefPage: end of main body, begin of footer
% EOF
% +------------------------------------------------------------------------+

