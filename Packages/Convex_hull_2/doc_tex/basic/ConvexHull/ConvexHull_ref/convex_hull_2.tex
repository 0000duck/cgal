% +------------------------------------------------------------------------+
% | Reference manual page: convex_hull_2.tex
% +------------------------------------------------------------------------+
% | 09.05.2001   Susan Hert and Stefan Schirra
% | Package: Convex_hull_2
% | 
% +------------------------------------------------------------------------+


\begin{ccRefFunction}{convex_hull_2}  %% add template arg's if necessary
\ccIndexMainItemBegin{convex hull, 2D}

\ccDefinition
  
The function \ccRefName\ generates the counterclockwise sequence of extreme
points from a given set of input points.  One of two algorithms is used,
depending on the type of iterator used to specify the input points.  For
input iterators, the algorithm used is that of Bykat \cite{b-chfsp-78}, which 
has a worst-case running time of $O(n h)$, where $n$ is the number of input 
points and $h$ is the number of extreme points.  For all other types of 
iterators, the $O(n \log n)$ algorithm of of Akl and Toussaint 
\cite{at-fcha-78} is used.

\ccInclude{CGAL/convex_hull_2.h}

\ccFunction{template <class InputIterator, class OutputIterator>
            OutputIterator
            convex_hull_2(InputIterator first, InputIterator beyond,
                          OutputIterator  result,
                          const Traits & ch_traits = Default_traits );}
            {generates the counterclockwise sequence of extreme points
            of the points in the range [\ccc{first},\ccc{beyond}).
            The resulting sequence is placed starting at position
            \ccc{result}, and the past-the-end iterator for the resulting
            sequence is returned. It is not specified at which point the
            cyclic sequence of extreme points is cut into a linear sequence.
            \ccPrecond %\ccIndexSubitem[C]{convex_hull_2}{preconditions}
            The source range [\ccc{first},\ccc{beyond}) does not contain
            \ccc{result}.}

The default traits class \ccc{Default_traits} is the kernel in which
the type \ccc{InputIterator::value_type} is defined. 

\ccHeading{Requirements}
\begin{enumerate}
   \item    \ccc{InputIterator::value_type} and \ccc{OutputIterator::value_type}
            should be \ccc{Traits::Point_2}.
   \item    \ccc{Traits} contains the following subset of types from
            the concept ConvexHullTraits\_2 and their corresponding member
            %\ccIndexMainItem[c]{ConvexHullTraits_2}
            functions that return instances of these types:
            \begin{itemize}
                \item \ccc{Traits::Point_2},
                \item \ccc{Traits::Less_xy_2}, 
                \item \ccc{Traits::Less_yx_2},
                \item \ccc{Traits::Left_of_line_2}, 
                \item \ccc{Traits::Leftturn_2}.
            \end{itemize}
\end{enumerate}


\ccSeeAlso

\ccRefIdfierPage{CGAL::ch_akl_toussaint} \\
\ccRefIdfierPage{CGAL::ch_bykat} \\
\ccRefIdfierPage{CGAL::ch_eddy} \\
\ccRefIdfierPage{CGAL::ch_graham_andrew} \\
\ccRefIdfierPage{CGAL::ch_jarvis} \\
\ccRefIdfierPage{CGAL::ch_melkman} 

\ccExample
In the following example we use the \stl-compliant interface of
\ccc{CGAL::Polygon_2} to construct the convex hull polygon from the
sequence of extreme points. Point data are read from standard input, the
convex hull polygon is shown in a \cgal\ window.
Remember, that when no traits class is specified for the function
\ccc{convex_hull_2}, the kernel from which the input points come is
used as the default traits class.

\ccIncludeExampleCode{Convex_hull_2/convex_hull_2_demo.C}

\ccIndexMainItemEnd{convex hull, 2D}
\end{ccRefFunction}

% +------------------------------------------------------------------------+
%%RefPage: end of main body, begin of footer
% EOF
% +------------------------------------------------------------------------+

