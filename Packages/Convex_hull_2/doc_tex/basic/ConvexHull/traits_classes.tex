\section{Traits Classes}

Each of the functions used to compute convex hulls or extreme points
is paramterized by a traits class, which specifies the types and geometric
primitives to be used in the computation.  For two dimensions, there are 
several implementations of traits classes provided in the library.  The class
\ccc{Convex_hull_traits_2<R>}\ccIndexMainItem[C]{Convex_hull_traits_2}
corresponds to the default traits class that provides the types and 
predicates presented in the 2-dimensional \cgal\ kernel in which the input 
points lie.  The class 
\ccc{Convex_hull_constructive_traits<R>}\ccIndexMainItem[C]{Convex_hull_constructive_traits}
is a second traits class based on \cgal\ primitives but differs from
\ccc{Convex_hull_traits_2} in that some of its primitives reuse
intermediate results to speed up computation.
There are also two traits classes defined for the two geometry kernels
provided in \leda\ \cite{mn-l-99} 
(\ccc{Convex_hull_leda_traits_2}\ccIndexMainItem[C]{Convex_hull_leda_traits_2}
and 
\ccc{Convex_hull_rat_leda_traits_2}\ccIndexMainItem[C]{Convex_hull_rat_leda_traits_2}), which provide an easy means of comparison with these kernels.
In addition, there are three projective traits classes 
(\ccc{Convex_hull_projective_xy_traits_2}, 
\ccc{Convex_hull_projective_xz_traits_2}, and 
\ccc{Convex_hull_projective_yz_traits_2}), %
\ccIndexMainItem[C]{Convex_hull_projective_xy_traits_2}%
\ccIndexMainItem[C]{Convex_hull_projective_xz_traits_2}%
\ccIndexMainItem[C]{Convex_hull_projective_yz_traits_2}%
which may be used to compute the convex hull of a set of three-dimensional
points projected into each of the three coordinate planes.

The function \ccc{convex_hull_3} also has a traits class in its interface
and the class \ccc{Convex_hull_traits_3} serves as the default here.

% file doc_tex/basic/ConvexHull/traits_classes_d.tex

In $d$ dimensions, the convex hull class \ccc{Convex_hull_d} is
parameterized by a traits class that provides $d$-dimensional data
types and predicates.  The class \ccc{Convex_hull_d_traits_3} adapts
any low-dimensional standard kernel model \textit{e.g.},
\ccc{Homogeneous<RT>} or \ccc{Cartesian<FT>} for use with
\ccc{Convex_hull_d}, where the dimension is fixed to three.


