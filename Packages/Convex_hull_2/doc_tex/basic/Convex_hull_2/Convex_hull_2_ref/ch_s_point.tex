% +------------------------------------------------------------------------+
% | Reference manual page: ch_s_point.tex
% +------------------------------------------------------------------------+
% | 09.05.2001   Susan Hert and Stefan Schirra
% | Package: Convex_hull_2
% | 
% +------------------------------------------------------------------------+


\begin{ccRefFunction}{ch_s_point}  %% add template arg's if necessary
\ccIndexSubitemBegin{extreme points, 2D}{in coordinate directions}

\ccDefinition
  
The function \ccRefName\ finds a points in a given set  
of input points with minimal $y$ coordinates.

\ccInclude{CGAL/ch_selected_extreme_points_2.h}

\ccFunction{template <class ForwardIterator>
            void
            ch_s_point( ForwardIterator first, ForwardIterator beyond,
                        ForwardIterator& s,
                        const Traits & ch_traits = Default_traits);}
           {traverses the range [\ccc{first},\ccc{beyond}).
            After execution, the value of
            \ccc{s} is an iterator in the range such that \ccc{*s} $\le_{yx}$
            \ccc{*it} for all iterators \ccc{it} in the range.}


The default traits class \ccc{Default_traits} is the kernel in which the
type \ccc{ForwardIteartor::value_type} is defined.

\ccHeading{Requirements}
\ccc{Traits} defines the type \ccc{Traits::Less_yx_2} as specified in
the concept ConvexHullTraits\_2 and the corresponding member
function that returns an instance of this type.


\ccSeeAlso

\ccRefIdfierPage{CGAL::ch_e_point} \\
\ccRefIdfierPage{CGAL::ch_nswe_point} \\
\ccRefIdfierPage{CGAL::ch_n_point} \\
\ccRefIdfierPage{CGAL::ch_ns_point} \\
\ccRefIdfierPage{CGAL::ch_w_point} \\
\ccRefIdfierPage{CGAL::ch_we_point} 

\ccIndexSubitemEnd{extreme points, 2D}{in coordinate directions}
\end{ccRefFunction}

% +------------------------------------------------------------------------+
%%RefPage: end of main body, begin of footer
% EOF
% +------------------------------------------------------------------------+

