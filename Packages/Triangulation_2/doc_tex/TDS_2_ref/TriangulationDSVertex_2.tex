% +------------------------------------------------------------------------+
% | Reference manual page: TDS_2::Vertex.tex
% +------------------------------------------------------------------------+
% | 07.04.2000   Author
% | Package: Package
% | 
\RCSdef{\RCSTdsVertexRev}{$Revision$}
\RCSdefDate{\RCSTdsVertexDate}{$Date$}
% |
%%RefPage: end of header, begin of main body
% +------------------------------------------------------------------------+


\begin{ccRefConcept}{TriangulationDataStructure_2::Vertex}

%% \ccHtmlCrossLink{}     %% add further rules for cross referencing links
%% \ccHtmlIndexC[concept]{} %% add further index entries

\ccDefinition
  
The concept \ccRefName\ describes the type used by a
\ccc{TriangulationDataStructure_2} to store the vertices\lcTex{, see 
\ccRefPage{TriangulationDataStructure_2}}.

Some of the requirements listed below are of geometric nature
and are \textit{optional}
when using the triangulation data structure class alone.
They became required when the triangulation data structure is plugged
into a triangulation.


\ccTypes
The class \ccRefName\ defines the same types as 
the triangulation data structure 
except the iterators.

\ccTwo{TriangulationDataStructure_2::Vertex:: Point }{}
\ccNestedType{Point}{\textit{Optional for the triangulation data
structure used alone.}} 

\ccCreationVariable{v}  %% choose variable name

\ccCreation

In order to obtain new vertices or destruct unused vertices, the user must
call the \ccc{create_vertex()} and \ccc{delete_vertex()} methods of the
triangulation data structure.

%\ccConstructor{TriangulationDataStructure_2::Vertex();}{default
%constructor.
% The geometric information is
% initialized by the default constructor of class \ccc{Point}.
%The pointer to the incident face is initialized with \ccc{NULL}.}

%\ccConstructor{TriangulationDataStructure_2::Vertex(const Point&
%p);}
%{introduces a vertex \ccVar, and initializes the geometric information.
%The pointer to the incident face is initialized with \ccc{NULL}.}

%\ccConstructor{TriangulationDataStructure_2::Vertex(const Point& p,
%                      Face_handle f);}
%{introduces a vertex \ccVar, and initializes the geometric information and 
%the pointer to the incident face.}

\ccHeading{Access Functions}
\ccThree{Face_handle}{v.set_face(Face_handle f);}{}
\ccMethod{Point point() const;}
{returns  the geometric information of \ccVar.}
\ccGlue
\ccMethod{Face_handle face() const;}
{returns a face of the triangulation having  \ccVar\ as vertex.}

\begin{ccAdvanced}
\ccHeading{Setting}

\ccMethod{void set_point(const Point& p);}
{sets the geometric information to \ccc{p}.}
\ccGlue
\ccMethod{void set_face(Face_handle f);}
{sets the incident face to \ccc{f}.}

\end{ccAdvanced}

\ccHeading{Checking}

\ccMethod{bool is_valid(bool verbose = false) const;}
{Checks the validity of the vertex. Must check that its incident face
has this vertex. The validity of the base vertex is also checked.\\
When \ccc{verbose} is set to \ccc{true}, messages are printed to give
a precise indication on the kind of invalidity encountered.}

\ccHeading{Miscellaneous}

\ccMethod{int ccw(int i) const;}
{Returns $i+1$ modulo 3.\ccPrecond $0\leq i \leq 2$.}
\ccGlue
\ccMethod{int cw(int i) const;}
{Returns $i+2$ modulo 3.\ccPrecond $0\leq i \leq 2$.}
\ccGlue
\ccMethod{size_type degree() const;}
{Returns the degree of \ccVar\ in the triangulation.}


\ccHasModels
\ccc{CGAL::Triangulation_ds_vertex_2<Vb,Fb>}


\ccSeeAlso
\ccc{TriangulationDataStructure_2} \\
\ccc{TriangulationDataStructure_2::Face} \\

\end{ccRefConcept}

% +------------------------------------------------------------------------+
%%RefPage: end of main body, begin of footer
% EOF
% +------------------------------------------------------------------------+

