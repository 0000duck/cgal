\newcommand{\triangulationcomment}[1]{}

\chapter{Triangulations} \label{I1_Chapter_Triangulations}

\section{Introduction}

A simplicial complex is a set ${C}$ of simplices which satisfy
two conditions:
\begin{enumerate}
\item Any face of a simplex in ${  C}$ is also a simplex in ${  C}$
\item Two simplices in ${  C}$ either do not intersect
or their intersection is  a simplex of smaller dimension
which is their common face of maximal dimension.
\end{enumerate}
The dimension of a complex ${  C}$ is the maximal dimension 
of the simplices
of ${  C}$. A complex of dimension $d$ is said to be pure
if any simplex in ${  C}$ is a face of a $d$-dimensional
simplex of ${  C}$.
The domain of a complex ${  C}$ is the union of the simplices
in ${  C}$. A complex is said to be connected if its domain is connected.

A 
triangulation is a 2-dimensional simplicial complex which is pure
connected and without singularities. Thus a triangulation
can be viewed as a collection of triangular faces,
such that two faces either have an empty intersection or share an edge or a vertex.
The absence of singularities means that each edge belongs to 
at most two triangles and that each vertex belongs  to a set  of faces
whose union forms a topological disk.

Each face of a triangulation can be given an orientation
(clockwise or counterclockwise) which in turn induces an orientation
on the edges incident to that face. The orientation of two adjacent
faces are said to be consistent if they induce
opposite orientations on their common incident edge.
A triangulation is said to be orientable if 
the orientation of each face can be chosen in such a way
that all pairs of incident faces have consistent orientations. 


In this chapter we present a framework to represent
orientable triangulations which  may  be embedded 
in a plane or in a higher dimensional space.
Examples of such triangulations 
are triangulations of a simple polygons in the plane, the
Delaunay triangulations of points in the plane, or triangulated
irregular networks ({\sc Tin}s) which are used as terrain model in
{\sc Gis}.  
Any such  triangulation can be described as a set of  vertices and
triangular faces,  with incidence and adjacence relations:
two facets or two vertices are said to be {\em adjacent}
 (or neighboring) if they are incident to the same edge,
a facet and a vertex are said to be incident if they are incident to the same
edge.

 On top of this  abstract  view of a triangulation come different
geometric layers which define  the geometric information
associated to a vertex (e.g.,  two-dimensional points for triangulations
in the plane, or three-dimensional points for {\sc Tin}s) or to a face
and which define the functionality of the triangulation. 
For example, the insertion of a point into a Delaunay
triangulation or into  {\sc Tin} requires different update steps.


\subsection*{Design Rationale}
Because a triangulation is merely a set of
triangular faces with constant size complexity,
triangulations are not implemented
as a layer on top of a planar map.
\cgal\ uses a proper internal
representation of triangulations based on faces and vertices
rather than on edges. Such a  representation
allows to save storage space and results in faster
algorithms~ \cite{bdty-tc-00}.

Thus the basic elements of the representation are vertices and faces.
Each triangular face gives access to its three incident vertices 
and to its three adjacent faces. 
Each vertex gives access to one of its incident faces
and through that face to the circular list of its incident faces.
The edges are not explicitely represented, they are only represented 
through the adjacencies relations of two faces.


\begin{figure}
\begin{ccTexOnly}
\begin{center}
\input{three_levels.ltex}
\end{center}
\end{ccTexOnly}
\caption{The three layer design of triangulations.
\label{I1_Fig_three_levels}}
\begin{ccHtmlOnly}
<CENTER>
<img border=0 src=three_levels.gif align=center alt="Three_levels">
</CENTER>
\end{ccHtmlOnly}
\end{figure}



The triangulations in \cgal\ are represented
by a three layer structure  analog to the design used for polyhedral
surfaces, see Figure~\ref{I1_Fig_three_levels}.
In the bottom layer, the base classes for vertices and faces
store some 
geometric informations such as the coordinate of vertices 
and any other attribute (such as color, constraint edges etc.)
needed by the application.
The base classes handle
incidence and adjacency relations in term of \ccc{void*} pointers.
The use of \ccc{void*} pointers in the bottom layer allows 
to easily change one of the base
class, to deal with an extra attribute like a color for example,
without having to change the rest of the structure. 
The advantages of strong type 
checking is reestablished in the next layer 
where the \ccc{triangulation data structure} can be thought 
of as a container for faces and vertices which can take care
of all the combinatorial aspects of the triangulation.
The {triangulation data structure}
implements
adjacency and incidence relations with type safe pointer operations
and maintains the combinatorial integrity of the triangulation.
For that purpose, the {triangulation data structure} defines its own face and vertex
classes which are derived
 from the corresponding 
base classes
so that geometric and additional information on vertices and faces 
are simply inherited.
  At last, at the top layer, the \ccc{triangulation class}
implements the user interface with the triangulation. 
This classes offer to the user
the high level functionalities that can be expected from a triangulation:
insertion  or removal of a vertex, traversal of the faces,
enumeration of the vertices,
traversal of the   faces incident to a given vertex, location of a given point etc.
The {triangulation class} defines its own 
vertex and face classes
derived from the corresponding class of the {triangulation data structure}.
The vertices and faces of the {triangulation class}
are only accessed through high levels concepts such as 
handles, iterators, or circulators,
according to the required functionalities of the access.
The top layer {triangulation class}
is responsible for the geometric embedding of the triangulation
and comes in different flavors according to the kind of triangulation represented:
planar triangulation of a set of points, Delaunay or regular or constrained
 triangulations etc.

The triangulation classes of \cgal\ depends on two template parameters.
The first template parameter stands for
 a geometric traits class which is assumed to provide
the geometric objects (points, segments and triangles) 
forming  the triangulation and the geometric predicates on those objects.
The second template parameter stands for a model
of  triangulation data
structure acting as a container for faces and vertices
while  taking care of the combinatorial aspects of the triangulation. 
The triangulation data structure class is itself a template
class parametrized by the base classes for faces and vertices.


\subsection*{Organization of this chapter}
Section~\ref{I1_Sect_Triangulation_2} introduces the basic triangulation class of \cgal ,
 \ccc{Triangulation_2<Traits, Tds>},  and gives some examples for a simple use of this class.
 The \ccc{Triangulation_2<Traits, Tds>} class  is merely designed to represent 
a triangulation for a set of points in the plane.
The next section~\ref{I1_Sect_Geom_traits}
describes the requirements for the geometric
traits class and presents some default implementations
for this traits class offered in \cgal. 
Section~\ref{I1_Sect_Tds}  presents the requirements for the 
{triangulation data structure} class,
 and the {triangulation data structure} classes
provided by \cgal. 
Section~\ref{I1_Sect_Base_classes} describes the requirements
for the base classes \ccc{Vertex_base} and \ccc{Face_base}  
together with the defaults provided for these classes.
The remaining sections of this chapter introduce
several classes derived from the basic triangulation class
and designed to handle more specific triangulations.
Section~\ref{I1_Sect_Delaunay} present a class to 
maintain the Delaunay triangulation of a set of points in the plane.
 Section~\ref{I1_Sect_Regular}
describe a class to maintain regular triangulations.
While the Delaunay triangulation of a set of points
is the dual of the Voronoi
diagram of these points,
 regular triangulations are dual to weighted points power diagrams
and  appear as a generalization of
Delaunay triangulations. Section~\ref{I1_Sect_Constrained} 
describes a class to handle a constrained triangulation,
that is a triangulation in which certain edges are enforced.
Constrained triangulations allow in particular to deal with the 
triangulations of a planar polygons.
At last section~\ref{I1_Sect_Hierarchy} describes the
Delaunay hierarchy, a hierarchical data
structure which may be used to improve the efficiency
of point location queries.

 

\section{Triangulation of points in the plane }
\label{I1_Sect_Triangulation_2}

The basic triangulation class 
 of \cgal\ is primarily designed to represent the  triangulations
 of a set of points ${  A}$ in the plane.
Such a triangulation has as its vertices the points of ${  A}$
and its domain covers the convex hull of ${  A}$.
It can be viewed as a planar partition of the plane
whoses bounded faces are triangular. and cover
the convex hull of ${  A}$. The single unbounded face of this partition
is the complementary of the convex hull of ${  A}$. 
In many applications, such as Kirkpatrick's hierarchy
or incremental Delaunay construction, it is convenient to
deal only with triangular faces. Therefore, we add to the
triangulation
a fictitious vertex, called the \ccc{infinite vertex}
and we make each  convex hull edge incident 
to an \ccc{infinite} 
face having as third vertex  the \ccc{infinite vertex}.
 In that way, each edge is incident to exactly two faces
and special cases at the
boundary of the convex hull are simpler to deal with.



\begin{figure}
\begin{ccTexOnly}
\begin{center} \IpeScale{50} \Ipe{infinite_vertex.ipe} \end{center}
\end{ccTexOnly}
\caption{The infinite vertex.
\label{I1_Fig_infinite_vertex}}
\begin{ccHtmlOnly}
<CENTER>
<img border=0 src=infinite_vertex.gif align=center alt="Vertices at
infinity">
</CENTER>
\end{ccHtmlOnly}
\end{figure}


The class \ccc{Triangulation_2<Traits,Tds>}
implements this point of view
and therefore considers  the triangulation of the set of points 
as a set of  triangular,  finite and
infinite faces. 
Although it is convenient to draw a triangulation as in
figure~\ref{I1_Fig_infinite_vertex}, note that
the \ccc{infinite vertex} has no significant
coordinates and that no geometric predicate can be applied on it
or on an infinite face.

A triangulation is a collection of vertices and faces that
are linked together through incidence and adjacency relations.
Each face give access to its three incident vertices and to
its 
three adjacent faces. Each vertex give access to one of its  incident
faces. 

The three vertices of a face are indexed with 0, 1 and 2
in counterclockwise order. The neighbor of a face are also 
indexed with 0,1,2 in such a way that the neighbor indexed by $i$
is opposite to the vertex with the same index.

The edges of a face $f$ are implicitely represented
through the adjacency relations between $f$ and its neifghbors.
Each edge has two implicit representation, one for each of its
incident face.

Many of the classes in the triangulation package
offer  two functions \ccStyle{int cw(int i)} and 
\ccStyle{int ccw(int i)} 
which given the index of a vertex in a face
compute the index of the next vertex  of the same face
in clockwise
or counterclockwise order.
 Thus, for example the neighbor 
\ccc{neighbor(cw(i))} is
 the
neighbor of \ccc{f}  which is next to \ccc{neighbor(i)} turning clockwise
around \ccc{f}. The face \ccc{neighbor(cw(i))}
is also the first face encountered after \ccc{f} when
turning clockwise around vertex \ccc{i}
of~\ccc{f}. See Figure~\ref{I1_Fig_neighbors}.



 \begin{figure}
\begin{ccTexOnly}
    \begin{center}
     \input{neighbors.ltex}
    \end{center}
\end{ccTexOnly} 
    \caption{Vertices and neighbors.
    \label{I1_Fig_neighbors}}
  \begin{ccHtmlOnly}
<CENTER>
<img border=0 src=neighbors.gif align=center alt="Neighbors">
</CENTER>
\end{ccHtmlOnly} 
\end{figure}



\smallskip
A triangulation is valid from the combinatorial point of view if the
following is true.
\\
{\bf (a)} Two adjacent faces have neighbor pointers to each 
other and they have two vertices in common. 
The faces have a
coherent orientation, that is, they index their common  vertices 
in opposite order.
\\
{\bf (b)} All faces that are incident to a vertex \ccc{v} 
must be linked with neighbor pointers. Vertex \ccc{v} points to an 
arbitrary incident face. 

Furthermore, it is said to be geometrically valid iff
\\
{\bf (c)} Any face has its vertices indexed according to
counterclockwise order.

\smallskip
\begin{figure}
\begin{ccTexOnly}
\begin{center} \IpeScale{70}\Ipe{valid1.ipe} \end{center}
\end{ccTexOnly} 
\caption{Validity test.
\label{I1_Fig_valid}}

\begin{ccHtmlOnly}
<CENTER>
<img border=0 src=valid1.gif align=center alt="Validity">
</CENTER>
\end{ccHtmlOnly} 
\end{figure}



\subsection{The class \protect \ccc{Triangulation_2<Traits,Tds}}

The triangulations are represented in \cgal through the class
\ccc{Triangulation_2<Traits,Tds}
which  expects a model of  {geometric traits} class
as first template argument and a model of {triangulation data structure}
as second argument. The requirements  and defaults for these classes 
are described in the next sections \ref{I1_Sect_Geom_traits}
and \ref{I1_Sect_Tds}.


The vertices and faces of the triangulations are accessed through 
\ccc{handles}\footnote{ A handle is a type which supports the two
dereference operators \ccc{operator*} and \ccc{operator->}.}, 
\ccc{iterators} and \ccc{circulators}. 
Handles are used whenever the accessed element( vertex or face)
is not part of a sequence,
iterators and circulators
allow to traverse (completely or partially) the triangulation.
 In the following, we called {\it infinite} any face or edge 
incident  to the infinite vertex and the infinite vertex itself.
 Any other feature (face, edge or vertex) of the triangulation is said 
to be {\it finite}.
Some iterators (the \ccc{All} iterators ) allows to visit all finite or 
infinite feature while others (the \ccc{Finite} iterators) visit only
all finite features. 
Circulators are used to visit the vertices or faces 
incident to a given vertex. 
Circulators visit infinite features as well as finite 
ones. Circulators around the infinite vertex
 allow to visit the vertices on the convex hull and
the infinite edges and faces.
The triangulation provide also a circluator type to visit all the
faces
traversed by a given line.
The iterators and circulators
are all bidirectional and non mutable.
The circulators and iterators are assignable to the 
corresponding handle types. Whenever a handle appear in the parameter 
list of a function, an appropriate iterator or circulator can be used as well.
The edges of the triangulation can also be visited through iterators
and circulators,
the edge circulators and iterators
are also bidirectional and non mutable.

The triangulation class provides method to test
the finite or infinite caracter of any feature,
and also methods to test the presence in the triangulation
of a particular feature (edge or face).



The triangulation class    provides methods to locate
a given point with respect to a triangulation. The location of a point 
is described  by the locate type \ccc{lt}, a face handle \ccc{f} and
an index \ccc{li}. The locate type is : \\
\ccc{VERTEX} if the point coincides with a vertex \ccc{v} of the triangulation
 In this case the face handle \ccc{f} points to a face incident to \ccc{v}
and the index \ccc{li} is the index of \ccc{v}  in \ccc{f}.\\
\ccc{EDGE} if the point lies on an edge \ccc{e}. In this case, the face
handle \ccc{f}
points to a face incident to \ccc{e} and \ccc{li}  is the
index of \ccc{e} in \ccc{f}. \\
\ccc{FACE} if the point lies in the relative interior of a bounded
face \ccc{f}. In this case the face handle point to \ccc{f} and the
index \ccc{li}
has no meaning. \\
\ccc{OUSIDE_CONVEX_HULL} if the point lies within the affine hull of
the triangulation but outside its convex hull. In this case, 
the face handle \ccc{f} points to any infinite face which is a {\em
witness } of the
query point location. For a two dimensional triangulation, this means
that the supporting line of the finite edge of \ccc{f}  separates
the query point from the triangulation. For a one-dimensional
triangulation
it means that the 
finite vertex of \ccc{f}  separates
the query point from the triangulation. \\
\ccc{OUTSIDE_AFFINE_HULL} if the point lies outside the affine hull
(in the case of a degenerate lower dimensional triangulation). In this 
case, the face
handle \ccc{f} and  the index \ccc{li} have no meaning.



The triangulation class also provides
methods to locate a point with respect to
a given  finite face of the triangulation or with respect to the
circumcircle of such a feace.
The face or its circimcircle are assumed to be counterclockwisely
oriented.



The triangulation can be modified by several fonctions~:
insertion of a point, removal of a vertex,
flipping of an edge adjacent to two faces the union of which forms
 a convex body (see figure~\ref{I1_fig_flip_bis}). 



\begin{figure}
\begin{ccTexOnly}
\begin{center} %\IpeScale{70} \Ipe{Flip.ipe} \end{center}
\input{flip.ltex}
\end{center}
\end{ccTexOnly} 
\caption{Flip.
\label{I1_fig_flip_bis}}

\begin{ccHtmlOnly}
<CENTER>
<img border=0 src=Flip.gif align=center alt="Flip">
</CENTER>
\end{ccHtmlOnly} 
\end{figure}











\ccExample

The following code fragment creates a  triangulation of 2D points
for the  usual Euclidean metric. The points are read from {\tt cin},
inserted in the triangulation 
and finally points on the convex hull are written to {\tt cout}. 


\ccIncludeExampleCode{Triangulation/triangulation_prog1.C}
\begin{cprog}
/* triangulation_prog1.C */
/* --------------------- */
#include <CGAL/basic.h>
#include <iostream>
#include <CGAL/Cartesian.h>
#include <CGAL/Point_2.h>
#include <CGAL/Triangulation_2.h>
#include <CGAL/Triangulation_euclidean_traits_2.h>

using namespace CGAL;

typedef Cartesian<double> Rp;
typedef Triangulation_euclidean_traits_2<Rp> Traits;
typedef Triangulation_vertex_base_2<Traits> Vb;
typedef Triangulation_face_base_2<Traits>  Fb;
typedef Triangulation_default_data_structure_2<Traits,Vb,Fb> Tds;
typedef Triangulation_2<Traits, Tds> Triangulation;
typedef Triangulation::Vertex_circulator Vertex_circulator;
typedef Traits::Point   Point;

int main() {
  Triangulation t;
    
  Point p;
  while (std::cin >> p){
    t.insert(p);
  }
  
  Vertex_circulator vc = t.incident_vertices(t.infinite_vertex()),
    done(vc);
  if (vc != 0) {
    do{
      std::cout << vc->point() << std::endl;
    }while(++vc != done);
  }
}
\end{cprog} 

\ccHeading{Implementation}

Locate is implemented by a line walk from a vertex of the face given
as optional parameter (or from a finite vertex of
\ccStyle{infinite_face()} if no optional parameter is given). It takes
time \ccTexHtml{$O(n)$}{O(n)} in the worst case, but only \ccTexHtml{$O(\sqrt{n})$}{O(sqrt(n))}
on average if the vertices are distributed uniformly at random.

Insertion of a point is done by locating a face that contains the
point, and then splitting this face.
If the point falls outside the convex hull, the triangulation
 is restored by flips.  Apart from the location, insertion takes a time 
time \ccTexHtml{$O(1)$}{O(1)}. This bound is only an amortized bound
for points located outside the convex hull.

Removal of a vertex is done by removing all adjacent triangles, and
retriangulating the hole. Removal takes time \ccTexHtml{$O(d^2)$}{O(d^2)} in the worst
case, if \ccTexHtml{$d$}{d} is the degree of the removed vertex,
which is \ccTexHtml{$O(1)$}{O(1)} for a random vertex.

The face, edge, and vertex iterators on finite features
are derived from their counterparts visiting all (finite and infinite)
features which are themselves derived from the corresponding iterators
of the triangulation data structure.


\section{ The Geometric Traits Class}
\label{I1_Sect_Geom_traits}

The first template parameter of the triangulation classes of \cgal\ 
is the geometric traits class. The traits class 
is required to provide
the geometric objects (points, segments and triangles)
building up the triangulation
together with the geometric predicates on those objects.
The required predicate are: \\
- comparison of two points \ccc{x} or \ccc{y} coordinates \\
- orientation tests providing \ccc{CGAL_orientation}
  corresponding to the order type of three given point.




\subsection{Predefined Geometric Traits Classes}

The \cgal\  kernel classes \ccc{Homogeneous} and \ccc{Cartesian}
can be used as traits class for the class \ccc{Triangulation_2<Traits, Tds>}

The \cgal\  library provides also some predefined geometric traits classes
for the triangulations (using the kernel geometric objects and predicates).
These classes are themselves templated with a representation class.

First, the \cgal\  kernel classes \ccc{Homogeneous} and \ccc{Cartesian}
can be used as traits class for the class \ccc{Triangulation_2<Traits, Tds>}

The traits class \ccc{Triangulation_euclidean_traits_2<R>}
is designed to deal with ordinary  two dimensional points.

The class \ccc{Triangulation_euclidean_traits_xy_3<R>} 
is a geometric traits class allowing to
build a two dimensional triangulation of the projections
on the $xy$ plane  of three dimensional points.
This is the usual case when dealing with GIS terrains.
Instead of really projecting the 3D points and
maintaining a mapping between each point and its projection
 (which costs space and is error prone)
this class supplies geometric predicates that ignore the
\ccc{z}-coordinate of the points.

\cgal\ provides also predefined geometric traits class
\ccc{Triangulation_euclidean_traits_yz_3<R>} and
\ccc{Triangulation_euclidean_traits_zx_3<R>} to
deal with projections on the
 \ccc{xz}- or  the \ccc{yz}-plane,
respectively.

\subsection{Example}
The following code creates a  valid triangulation traits class 
for a triangulation of 2D points in usual Euclidean space
and use it to define a triangulation class.

\begin{cprog}
typedef Cartesian<double> Rp;
typedef Triangulation_euclidean_traits_2<Rp> Traits;
typedef Triangulation_vertex_base_2<Traits> Vb;
typedef Triangulation_face_base_2<Traits> Fb;
typedef Triangulation_default_data_structure_2<Traits,Vb,Fb > Tds;
typedef Triangulation_2<Traits,Tds> Triangulation;
\end{cprog}



\section{The Triangulation Data Structure}
\label{I1_Sect_Tds}

The second template parameter of the basic triangulation class
\ccc{Triangulation_2<Traits,Tds>} is a triangulation data structure class.
This class can be seen has a container for the 
faces and vertices maintaining incidence and adjacency relations.
The triangulation data structure class 
 is responsible for  the combinatorial integrity of the triangulation
(i. e. proper incidence and adjacency relations among vertices
and faces) while 
allowing  to perform combinatorial modifications
such has 
insertion of a new vertex in a face, or in an edge,
suppression of a vertex of degree three,  flip of two edges.
The term combinatorial means that those operation are purely topological
and do not depend on the geometric embedding.



The triangulation data structure is itself a template class
parametrized by the base classes \ccc{Vertex_base} and \ccc{Face_base},
and derives from thoses base classes its own 
vertex and face classes. This design  allows to restore at the 
triangulation data structure
level 
the strong type checking which does not exists at the base classes levels.


\subsection{Requirements for a Triangulation Data Structure Class}

The triangulation data structure 
is required to provide :
\begin{itemize}
\item
iterators to visit all the vertices, edges and faces
of the triangulation data structure,
\item
circulators to visit all the vertices, edges and faces
incident to a given vertex
\end{itemize}


The triangulation data structure is responsible 
for the creation and removal of faces and vertices 
(memory management).
It provides function that gives the number of faces, edges and
vertices
of the triangulation.

All the purely combinatorial transformation of the triangulation
are performed at the triangulation data structure level.
Here we call \ccc{combinatorial}
 a transformation 
of the triangulation which does not involve any knowledge
on the geometric embedding. Thus the flip of two
adjacent given faces, the addition  of a new vertex splitting a given face,
or a given edge, the removal of a vertex that has degree three 
are combinatorial operations.

\begin{figure}
\begin{ccTexOnly}
\begin{center} %\IpeScale{70} \Ipe{Flip.ipe} \end{center}
\input{flip.ltex}
\end{center}
\end{ccTexOnly} 
\caption{Flip. \label{I1_fig_flip_bis}}

\begin{ccHtmlOnly}
<CENTER>
<img border=0 src=Flip.gif align=center alt="Flip">
</CENTER>
\end{ccHtmlOnly} 
\end{figure}



\begin{figure}
\begin{ccTexOnly}
%\begin{center} \IpeScale{70} \Ipe{Three.ipe} \end{center}
\begin{center} \documentclass{article}
\documentclass{article}
\documentclass{article}
\input{tmp.inputs}
\pagestyle{empty}
\begin{document}
\input{tmp.pstex_t}
\end{document}

\pagestyle{empty}
\begin{document}
\documentclass{article}
\input{tmp.inputs}
\pagestyle{empty}
\begin{document}
\input{tmp.pstex_t}
\end{document}

\end{document}

\pagestyle{empty}
\begin{document}
  \begin{tabular}{ccc}
    \input{insert1.pstex_t} &
    \input{insert2.pstex_t} &
    \input{insert3.pstex_t}\\
  {\small (a)} & {\small (b)} & {\small (c)}\\
  \end{tabular}
\end{document}
 \end{center}
\caption{Insertion}
\end{ccTexOnly} 

\begin{ccHtmlOnly}
<CENTER>
<img border=0 src=Three.gif align=center alt="Insertion">
</CENTER>
\end{ccHtmlOnly} 
\end{figure}



\subsection{Models of Triangulation Data Structure}
\cgal offers currently two models of triangulation data structures.
The first one \ccc{Triangulation_default_data_structure_2<Tds_gt,Vb,Fb
>}
is highly economic with respect to memory space but its use is
restricted
to planar embedded triangulation. The second one
\ccc{Triangulation_data_structure_using_list_2<Vb,Fb>} can be used for 
the representation of any  orientable triangulations.

\subsubsection{The Default Triangulation Data Structure}
\cgal\ proposes the class \ccc{Triangulation_default_data_structure_2<Tds_gt,Vb,Fb >}
as a default for the triangulation data structure class of a triangulation.
This class has two template parameters \ccc{Vb} anf \ccc{Fb}
which have 
to be models for respectively the
\ccc{Vertex_base} and \ccc{Face_base} concepts whose requirements are described in next 
section~\ref{I1_Sect_Base_classes}. 

In addition, the class \ccClassTemplateName\ has a first template parameter
which is a geometric traits class. This may be surprising because
the triangulation data structure is supposed to deal only with the combinatorial
aspect of the triangulation and not with any geometric embedding
The reason for that is the following.
The class \ccClassTemplateName\ does not use any additional data structure
such as a list or a vector to act as a container for faces and vertices.
The iterators which allows to visit all faces and vertices of the
triangulation
data structure
is implemented using  an implicit tree structure over the faces
as described by
 de Berg, van Oostrum, and Overmars, 
in Proc.\ 12th Annual Symp.\ on Comput.\ Geom.,
1996, pages C5--C6. This tree structure is  based on the planar
geometric embedding
the triangulation. Each face 
 can find its parent 
and its children using only simple comparisons on the
coordinates of the points embedding its vertices.
Thus the tree structure may remain implicit 
and does not require any additional memory. 

The requirements concerning the geometric traits \ccc{Tds_gt} of
\ccc{Triangulation_default_data_structure_2}
 are very light and form a subset of the requirements needed
for the geometric traits of the triangulations. 
This class is required  to provide a type \ccc{Point}
and the coordinate comparison functions \ccc{compare_x(Point p0, Point p1)} and
\ccc{compare_y(Point p0, Point p1)}
described in section~\ref{I1_Sect_Geom_traits}. The point type
defined by the geometric traits class of the triangulation data structure
has to be the same 
as the point type defined by the geometric traits of the triangulation.
This is achieved if the same model is used for both traits classes
which is recommended but not compulsory.



\subsubsection{A triangulation data structure using list}
The class \ccc{Triangulation_data_structure_using_list_2<Vb,Fb>}
can be used as a triangulation data structure for any
orientable triangulation. It uses aa \stl list to store the
full dimensional faces of the triangulation.



\section{The Base Classes \protect \ccc{Vertex_base} and  \protect \ccc{Face_base} }
\label{I1_Sect_Base_classes}

\subsection{Requirements for the Base Classes}
At the bottom layer,  
a vertex stores a point and a  provides access to one of its incident
face through a
\ccc{void *} pointer.


At the bottom layer, a 
face stores \ccc{void *} pointers
 to its three vertices  and to its three neighboring faces.
The vertices and neighbors are indexed 0,1 and 2 in counterclockwise
order around the face. Neighbor $i$ lies opposite to vertex $i$.



\subsection{The Default Base Classes}
\cgal\ provides two default base classes 
\ccc{Triangulation_face_base_2<Traits>} and \ccc{Triangulation_vertex_base_2<Traits>}
which model respectively 
the \ccc{Vertex_base} and the \ccc{Face_base} concept.
Both of them are templated by a geometric traits class.
Using for this traits class, the geometric traits class used for the triangulation class
is strongly recommended. 
It ensures that the point type defined by \ccc{Triangulation_vertex_base_2<Traits>}
is the same as the point type defined the  geometric traits class of
the triangulation.

These default base classes can be used directly or can serve as a base to derive
other base classes with some additional attribute (a color for example)
tuned for a specific application.


\ccExample
The following example derives a new \ccc{Face_base} class from the default
one and add a color to the faces of the triangulation. 
The face of the triangulation data structure
and the face of the triangulation will inherit the new data member 
and its functionality.
Any kind of additional fonctionality can thus be added to faces or vertices of a triangulation 
as long as this functionality  does not involve additional pointers to vertices or faces
(because the base classes use only void* pointer and have no knowledge
of the vertex or face types.).

\ccIncludeExampleCode{Triangulation/colored_face.C}



\section{Delaunay Triangulations}
\label{I1_Sect_Delaunay}


\subsection{The class \protect \ccc{Delaunay_triangulation_2<Traits,Tds>}}
The class \ccc{Delaunay_triangulation_2<Traits,Tds>} is designed to represent
the Delaunay triangulation of a set of points in a plane.
A  Delaunay triangulation of a set of points
is a triangulation of the sets of points that fulfills
the following {\em empty circle property} 
(also called {\em Delaunay property}): the circumscribing
circle of each
 face does not contain any vertex.
For a point set with no case of cocircularity 
of more than three points,
the Delaunay triangulation is unique, it is  the dual
of the Voronoi diagram of the points.


A Delaunay triangulation is a special triangulation of a set of points.
So it is natural to derive  the class \ccc{Delaunay_triangulation_2<Traits,Tds>}
from the basic class \ccc{Triangulation_2<Traits,Tds>}.
The template parameters \ccc{Traits} and \ccc{Tds} stand 
respectively
for a model of geometric traits and for a model of triangulation data structure.
The requirements for the triangulation data structure
are those described previously in section~\ref{I1_Sect_Tds}
and the same models can be used to instantiate the
triangulation data structure of either a \ccc{Triangulation_2<Traits,Tds>}
or a \ccc{Delaunay_triangulation_2<Traits,Tds>}. On the contrary, because the
concept of Delaunay triangulation relies on the notions of 
empty circles and of distance, 
the geometric traits has to provide
the \ccc{side_of_oriented_circle} predicate
(which was only optional in the traits of plain triangulation),
and also
a Distance class. The additional requirements
to be fulfilled by the geometric traits class
of a \ccc{Delaunay_triangulation_2<Traits,Tds>} are described in 
subsection~\ref{I1_Sect_Delaunay_geom_traits}. 
Changing the Distance class and the \ccc{side_of_oriented_circle} predicate 
allows to build Delaunay triangulations for different metrics
such that $L_1$ or $L_{\infty}$ or any metric defined by a
convex object. However, the user of an exotic metric
must be carefull that the constructed triangulation 
has to be a triangulation of the convex hull
which means that convex hull edges have to be Delaunay edges.
This is granted for any smooth convex metric (like $L_2$)
and can be ensured for other metrics (like  $L_{\infty}$)
by the addition to the point set of well chosen sentinel points.


The class \ccc{Delaunay_triangulation_2<Traits,Tds>}
inherits all the types of the \ccc{Triangulation_2<Traits>}.
and define additionnal types to represent the dual Voronoi diagram.
\ccTypedef{typedef Traits::Line Line;}{}
\ccGlue
\ccTypedef{typedef Traits::Direction Direction;}{}
\ccGlue
\ccTypedef{typedef Traits::Ray Ray;}{}

The class \ccc{Delaunay_triangulation_2<Traits,Tds>}
overwrite the insertion and removal  functions
to maintain the Delaunay property.
It also has a member function to answer nearest vertex queries
and member function to construct the elements of the dual Voronoi diagram.

\ccExample
The following code fragment creates a Delaunay triangulation with 
the usual Euclidean metric for the vertical projection of a 
terrain model. The points have elevation, that is they are 3D points
and the predicates which are defined in the Delaunay triangulation 
traits class forget about the $z$-coordinate of these points. 
\ccIncludeExampleCode{Triangulation/terrain.C}
%\begin{cprog}
%/*  terrain.C      */
%/*  -------------- */
%#include <CGAL/Homogeneous.h>
%#include <CGAL/leda_integer.h>
%#include <CGAL/Triangulation_euclidean_traits_xy_3.h>
%#include <CGAL/Delaunay_triangulation_2.h>

%using namespace CGAL;

%typedef Homogeneous<leda_integer>  Rp;
%typedef Triangulation_euclidean_traits_xy_3<Rp>  Terrain;
%typedef Triangulation_vertex_base_2<Traits> Vb;
%typedef Triangulation_face_base_2<Traits> Fb;
%typedef Triangulation_default_data_structure_2<Traits,Vb,Fb > Tds;
%typedef Delaunay_triangulation_2<Terrain, Tds> Delaunay;

%int main()
%{
%    Delaunay dt(Terrain());

%    Point_3<Rp> p;
%    while(std::cin >> p){
%        dt.insert(p);
%    }
%    return 1;    
%}
%\end{cprog} 


\ccHeading{Implementation}

Insertion is implemented by inserting in the triangulation, then
performing a sequence of Delaunay flips. The number of flips is \ccTexHtml{$O(d)$}{O(d)}
if the new vertex is of degree \ccTexHtml{$d$}{d} in the new triangulation. For
points distributed uniformly at random, insertion takes time \ccTexHtml{$O(1)$}{O(1)} on
average.

Removal calls the removal in the triangulation and then retriangulates
the hole in such a way that  the Delaunay criterion is satisfied. Removal of a
vertex of degree \ccTexHtml{$d$}{d} takes time \ccTexHtml{$O(d^2)$}{O(d^2)},
which is \ccTexHtml{$O(1)$}{O(1)} for a random
vertex in the triangulation.

Nearest neighbor is found in time \ccTexHtml{$O(n)$}{O(n)} in the
worst case, but in time \ccTexHtml{$O(1)$}{O(1)}
for vertices distributed uniformly at random  and any query point. 



\subsection{The Geometric Traits of a Delaunay Triangulation}
\label{I1_Sect_Delaunay_geom_traits}

\subsubsection{Requirements}
In addition to the requirements described in section~\ref{I1_Sect_Geom_traits}
for the geometrics traits of any triangulation,
the geometrics traits of a Delaunay triangulation
has to  provide a function object type \ccc{Side_of_oriented_circle_2} 
for the basic \ccc{in_circle} test used to maintain the
empty circle property.
The \ccc{in_circle} test is the test 
that actually defines the triangulation.
The additional types \ccc{Line_2}, \ccc{Direction_2} and
and \ccc{Ray_2} and the constructor objects
\ccc{Construct_direction_of_line2}, \ccc{Construct_ray_2}
\ccc{Construct_circumcenter_2}, \ccc{Construct_bisector_2},
\ccc{Construct_midpoint}
are used to build the dual Voronoi diagram
and are required only if the dual functions are called.
The additional \ccc{Less_distance_to_point_2} is only required
if \ccc{nearest_vertex(..)} query are issued.




\subsubsection{Predefined Geometric Traits Class}
The class \ccc{Triangulation_euclidean_traits_2<R>}
introduced in section~\ref{I1_Sect_Geom_traits} is 
 designed to be 
the geometric traits class of a Delaunay triangulation.
It implements the usual Euclidean metric
for the  two dimensional points \ccc{Points_2<R>}.
Three traits classes are provided to deal with
the Delaunay triangulation of two dimensional points which are
the \ccc{xy}, \ccc{yz} or \ccc{zx} projections of three dimensional points:\\
\ccc{Triangulation_euclidean_traits_xy_3<R>},\\
\ccc{Triangulation_euclidean_traits_yz_3<R>}, and\\
\ccc{Triangulation_euclidean_traits_zx_3<R>} \\
The requirements for the duality functions and nearest vertex
queries are not yet satisfied by
these last three classes.


\section{Regular triangulations}
\label{I1_Sect_Regular}
 
\subsection{Definition}
Let ${  PW} = \{(p_i, w_i), i = 1, \ldots , n \}$ be a set of 
weighted points where each $p_i$ is a point and each $w_i$
is a scalar called the weight of point $p_i$.
Alternatively, each weighted point $(p_i, w_i)$ can be regarded
as a two dimensional sphere with center $p_i$ and radius $r_i=\sqrt{w_i}$.

The power diagram of the set ${  PW}$ is a planar partition 
such that each cell corresponds to sphere $(p_i, w_i)$ of ${  PW}$
and is the locus of points  $p$ whose power with respect to $(p_i, w_i)$
is less than its power with respect to any other sphere $(p_j, w_j)$
in ${  PW}$. 
The dual of this diagram is a triangulation 
whose domain covers the convex hull of the set 
${  P}= \{ p_i, i = 1, \ldots , n \}$ of center points
and whose vertices are a subset of ${  P}$.
Such a triangulation is called a regular triangulation.
The  three points $p_i, p_j$ and $p_k$ of ${  P}$
form a triangle in the regular triangulation of ${  PW}$
iff there is a point $p$ of the plane whose
powers with respect to $(p_i, w_i)$, $(p_j, w_j)$
and $(p_k, w_k)$ are equal and less than the power of $p$
with respect to any other sphere in  ${  PW}$.

Let us defined the power product of two weighted points
$(p_i, w_i)$ and $(p_j, w_j)$ as:
\[\Pi(p_i, w_i,p_j, w_j) = p_ip_j ^2 - w_i  - w_j  .\]
$\Pi(p_i, w_i,p_j, 0)$ is simply the power of point $p_j$
with respect to the sphere $(p_i, w_i)$, and two weighted points 
are said to be orthogonal if their power product is null.
The power circle of three weighted points
 $(p_i, w_i)$, $(p_j, w_j)$
and $(p_k, w_k)$ is defined as the unique circle
$(\pi, \omega)$  orthogonal to
 $(p_i, w_i)$, $(p_j, w_j)$
and $(p_k, w_k)$.

The regular triangulation of the sets ${  PW}$
satisfies the following {\em regular property} (which just reduces to the 
Delaunay property when all the weights are null):
a triangle $p_ip_jp_k$ of the regular triangulation
of ${  PW}$ is such that the power product of any weighted point
 $(p_l, w_l)$ of ${  PW}$ with the power circle of
 $(p_i, w_i)$, $(p_j, w_j)$ is $(p_k, w_k)$ is positive or null.
We call  power test of the weighted point $(p_l, w_l)$ with respect
to the face  $p_ip_jp_k$, the predicates which amount to compute
the sign of 
the power product of $(p_l, w_l)$ with respect to
the power circle of
 $(p_i, w_i)$, $(p_j, w_j)$ is $(p_k, w_k)$,
which is given by the following
determinant
\[\left| \begin{array}{cccc}
1    & 1                       & 1                       & 1 \\
x_i & x_j & x_k & x_l \\
y_i & y_j & y_k & y_l \\
x_i ^2 + y_i ^2 -w_i & x_j ^2 + y_j ^2 - w_j & x_k ^2 + y_k ^2 - w_k &
x_l ^2 + y_l ^2 -w_l  
\end{array}
\right|
\]

A pair of neighboring faces $p_ip_jp_k$
and $p_ip_jp_l$ is said to be locally regular
(with respect to  the weights in ${  PW}$)
if the power test of $(p_l,w_l)$ with respect to
$p_ip_jp_k$ is positive.
A classical  result of computational geometry
establishes that a triangulation of the convex hull of ${  P}$
such that any pair of neighboring faces is regular with respect
to ${  PW}$, is a
 regular triangulation of ${  PW}$.

Alternatively, the regular triangulation
of the weighted points set ${  PW}$
can be obtained as the projection
on the two dimensional plane of the convex hull of the set of three
dimensional points 
${  P'}= \{ (p_i,p_i ^2 - w_i ), i = 1, \ldots , n \}$.

\subsection{ The  Class \protect \ccc{Regular_triangulation_2<Traits, Tds>}}
The \cgal\  class \ccc{Regular_triangulation_2<Traits, Tds>}
 is designed to maintain the
regular triangulation of a set of weighted points.
The template parameters \ccc{Traits}  and \ccc{Tds} stand respectively
 for a geometric traits class and a triangulation data structure class.
Any triangulation data structure that fulfills the requirements of 
section~\ref{I1_Sect_Tds} can be used for a regular triangulation.
The geometric traits class must provide a weighted point type
and a power test on these weighted points. The requirements and defaults
for the geometric traits and the power point type are listed below.



The class \ccc{Regular_triangulation_2<Traits, Tds>}
derives from the class \ccc{Triangulation_2<Traits, Tds>}.
The
functions \ccc{insert} and 
\ccc{remove} are overwritten to maintain the regular
property.
The vertices of the regular triangulation
of a set of weighted points ${  PW}$ form only a subset
of the set of center points of ${   PW}$.
Therefore the  insertion of a weighted point  in a regular triangulation
does not necessarily imply the creation of a new vertex.
If the new inserted point does not appear as a vertex in the
regular triangulation, it is said to be hidden 
by the face in which the corresponding center point is located.
Such a weighted point is stored in a list attached to the hidding face,
to be used for later tentative of insertions when future removal
of some points implies the destruction of the hidding face.




%\ccHeading{Duality}

%\ccMethod{Point dual(const Face_handle &f) const;}
%{Returns the center of the power circle corresponding to face \ccc{f}.
%\ccPrecond \ccc{f} is not infinite}

%\ccMethod{Object dual(const Edge &e) const;}
%{If both incident faces are finite, returns a segment whose endpoints are the
%duals of each incident face. If only one incident face is finite, returns a
%ray whose endpoint is the dual of the finite incident face and supported by
%the line which is the bisector of the edge's endpoints. If both incident faces
%are infinite, returns the line which is the bisector of the edge's endpoints
%otherwise.  }

%\ccMethod{Object dual(const Edge_circulator& ec) const;}
%{Idem}

%\ccMethod{Object dual(const Edge_iterator& ei) const;}
%{Idem}

%*************************

\ccExample
The following code fragment creates a regular triangulation 
of a set of weighted points.
\ccIncludeExampleCode{Triangulation/regular.C}

%\begin{cprog}
%/*     regular.C */
%/*-------------- */
%#include <CGAL/basic.h>
%#include <iostream>
%#include <CGAL/Cartesian.h>
%#include <CGAL/Regular_triangulation_euclidean_traits_2.h>
%#include <CGAL/Regular_triangulation_2.h>

%using namespace CGAL;

%typedef Cartesian<double> Rp;
%typedef double W;
%typedef Regular_triangulation_euclidean_traits_2<Rp,W>  Traits;
%typedef Triangulation_vertex_base_2<Traits> Vb;
%typedef Regular_triangulation_face_base_2<Traits> Fb;
%typedef Triangulation_default_data_structure_2<Traits,Vb,Fb > Tds;
%typedef Regular_triangulation_2<Traits, Tds> Regular_triangulation;

%int main()
%{
%   Regular_triangulation rt;

%   Traits::Weighted_point wp;
%   while(std::cin >> wp){
%     std::cout << wp << std::endl;
%     rt.insert(wp);
%   }
%   rt.is_valid();
%   return 1;	
%}

%\end{cprog} 

\subsection{The Face Type of a Regular Triangulation}

\subsection{Requirements for the Base Face}
The regular triangulation of a set of weighted point does not
necessarily
have one vertex for each of the input points. Some of the input
weigthed points have no cell in the dual power diagrams
and therefore do not correspond to a vertex of the regular
triangulation.
Those weighted point are said to be {\it hidden} points.
A  point which is hidden at a given time may appear later as a vertex of
the regular triangulation upon removal on some other weighted point.
Therefore, hidden points have to be stored somewhere.
A hidden point can appear as vertex of the triangulation
only when the
two dimensional face where its point component is located
(the face which hides it)
is removed. Therefore we decided to store each hidden point
in the face which  hides it and the nested face type of a 
regular triangulation is assumed to include a list of hidden 
weighted points. This list of weighted point is in fact included
in the base face of a regular triangulation.

Therefore the base face type of a regular triangulation
is required to provide access to a list of weigthed points,
designed to store the points hidden by the face.


\subsubsection{A Default Base Face .}
\cgal\ provides the templated class 
\ccc{Regular_triangulation_face_base_2<Traits>}
which derives from \ccc{Triangulation_face_base_2<Traits>}
and can be used as a default base class for faces of regular triangulations.



\subsection{The Geometric Traits of a Regular Triangulation}

\subsubsection{Requirements}

The geometrics traits class of a regular triangulation
must fulfill the requirements of section~\ref{I1_Sect_Geom_traits}
and additionally provide  
\ccc{Weitghed_point} type (which derives from the type \ccc{Point} 
and add a weight data member )
and a two types \ccc{Power_test_2} and \ccc{Power_test_degenerated_2}
for the power tests.


\subsubsection{A Predefined Geometric Traits Class}
\cgal\ provides the predefined geometric traits class \\
\ccc{Regular_triangulation_euclidean_traits_2<Rep,Weight>}.
This traits class is templated by a representation class \ccc{Rep}
and a weight type \ccc{Weight}.
This class inherits from
\ccc{Triangulation_euclidean_traits_2 <Rep >}
and uses a \ccc{Weighted_point} type
derived from the type \ccc{Point} of
\ccc{Triangulation_euclidean_traits_2 < R >}.



\subsection{Weighted Points}

\subsubsection{requirements for a weighted-point}
The weighted point type  required by a regular triangulation
is required to provide access to the underlying  point and to its weight.

\subsubsection{Default Class for a Weighted Point}
\cgal\ provides the class \ccc{Weighted_point<Point,Weight>}
as a default type for a  weighted two dimensional point.
This default type has two template parameters \ccc{Point}
and  \ccc{Weight} which have to be instantiated respectively 
with a point type and a weight type.
The class \ccc{Weighted_point<Point,Weight>}
 inherits from \ccc{Point}.




\section{Constrained Triangulations}
\label{I1_Sect_Constrained}


\subsection{Definition}
A constrained triangulation is a triangulation of a set of points
which has to include among its edges 
a given set of segments joining the points. The corresponding 
edges are called {\em constrained edges}. 

The set of points defining the vertices of the triangulation
includes the set of constrained edges endpoints. It may include other points
(considered as null length constrained edges) as well.  The set of
constrained edges forms a set of segments which do not intersect
except possibly at their endpoints.  Any number of constrained edges
are allowed to share the same endpoint.  Vertical constrained edges or
constrained edges with null length are allowed.


\begin{ccTexOnly}
\begin{center} \IpeScale{50} \Ipe{constraints.ipe} \end{center}
\end{ccTexOnly}
 
\begin{ccHtmlOnly}
<CENTER>
<img border=0 src=constraints.gif align=center alt="A set of
constraints and its constrained triangulation">
</CENTER>
\end{ccHtmlOnly}

\subsection{ The Class \protect \ccc{Constrained_triangulation_2<Traits,Tds>}}
A constrained triangulation is represented in the CGAL library as an
object of the class \ccc{Constrained_triangulation_2<Traits,Tds>}.
As usual the template parameters \ccc{Traits} and \ccc{Tds}
stand respectively for a geometric traits class and
a triangulation data structure class. There is no additional
requirements for the geometric traits and the triangulation data
structure of a constrained triangulation.
Models used to instantiate these classes are simply required
to fulfill repectively the requirements 
of section~\ref{I1_Sect_Geom_traits} and ~\ref{I1_Sect_Tds}.

The class \ccc{Constrained_triangulation_2<Traits,Tds>}
inherits from \ccc{Triangulation_2<Traits,Tds>}.
It defines an additionnal type \ccc{Constraint;}
to represent the constraints. A
constraint is represented as a pair of points.

A  constrained triangulation can be created
from a
list of constrained edges. Those constrained edges are assumed to have no
intersection other than endpoints. Any number of constrained edges are
allowed to share the same endpoint.  Vertical constrained edges or
constrained edges with null length are allowed.

The class \ccc{Constrained_triangulation_2<Traits,Tds>}
overrides the insertion and removal of a point to take care of the
information about contrained edges. The class also allows inline
insertion of a new constraint, given by its two endpoints
or the removal of a constraint.


\ccHeading{Implementation}

The constructors build the triangulation using a sweeping line
algorithm. The complexity of this algorithm is $O(n\log n)$ if $n$
endpoints are present. The sweep structure is an \stl\ map.
 The insertion of a constrained edge runs in time
proportionnal to the number of triangles intersected by this edge.



\subsection{The Face Type of a Constrained Triangulation}
\label{I1_Sect_Constrained_face}
 The information about constrained edges is store in the 
faces of the triangulation. Thus the nested \ccc{Face}
type of a constrained triangulation offers
additonnal functionalities to deal with this information.
This additional functionalities related to the constraints
are requirements which have to be fulfilled
by the base face  a constrained triangulation
in addition to the functionalities required in section~\ref{I1_Sect_Base_classes}

\subsubsection{Requirements for the Base Face}
The base face of a constrained triangulation is required
to have a fonction members providing access to the status
(constrained ou unconstrained) of each of its edges.
has function mem

\subsubsection{A predefined Base Face}
Of course  \cgal\ provides a default \ccc{Face_base} class
for the constrained triangulation. The class
\ccc{Constrained_triangulation_face_base_2<Traits>}
simply derived from 
\ccc{Triangulation_face_base_2<Traits>}. 
%and override the
%functions \ccc{reorient()}, \ccc{ccw_permute()} and 
%\ccc{cw_permute()}.


\subsection{A Constrained Triangulation Class for Animation Purposes
\label{I1_Subsec_Constrained_Demo_2}}

\cgal provides 
a class  \ccc{Constrained_triangulation_demo_2<Traits,Tds>}
specially designed to show an animation of the sweep algorithm 
that builds a 
constrained triangulation from a list of constraints.



\section{Constrained Delaunay Triangulations}
\label{I1_Sect_Constrained_Delaunay_Triangulations}

\subsection{Definition}
A constrained Delaunay triangulation is a triangulation with
constrained edges which tries to be as much Delaunay as possible.
As constrained edges are not necessarily Delaunay edges,
the triangles of a constrained Delaunay triangulation do not
necessarily fulfill the empty circle property
but they fulfill a weaker constrained empty circle property.
 To state this property,
it is convenient to think of  constrained
edges as blocking the view. Then, a triangulation is 
constrained Delaunay if
 the circumscribing circle
of any of its triangular faces includes in its interior 
no vertex  that is visible
from the interior of the triangle.

\subsection{The Class 
\protect \ccc{Constrained_Delaunay_triangulation_2<Traits,Tds>}}

The \cgal\ class
\ccc{Constrained_Delaunay_triangulation_2<Traits,Tds>}
is designed to represent
constrained Delaunay triangulations.

The class is templated by a geometric traits class \ccc{Traits}
and a triangulation data structure \ccc{Tds}.
There are no special requirements for the triangulation data
structure
of a constrained Delaunay triangulations and the requirements
for this class are those described
in section~\ref{I1_Sect_Tds}. The geometric traits 
of a constrained Delaunay triangulation is required
to provide the \ccc{side_of_oriented_circle} test as the geometric traits
of a Delaunay triangulation and the requirements for this traits
are described in section~\ref{I1_Sect_Delaunay_geom_traits}. 

A constrained Delaunay triangulation is not a Delaunay
triangulation but it is a constrained triangulation.
Therefore the class \ccClassTemplateName\ derives from
the class \ccc{Constrained_triangulation_2<Traits,Tds>}.
Also, information about the status (constrained or not)
of the edges of the triangulation has to be stored
in the face class
 and the requirements for the base face class
of a constrained Delaunay triangulation are 
identical to those described in 
section~\ref{I1_Sect_Constrained_face} for the face base class
of a constrained  triangulation.

The constrained Delaunay triangulation
has member functions to override the 
insertion and removal of a point or of a constraint.
Each of those member function takes care
to  restore
 the Delaunay constrained
property.

