% +------------------------------------------------------------------------+
% | Reference manual page: Constrained_triangulation_2.tex
% +------------------------------------------------------------------------+
% | 12.04.2000   Author
% | Package: Package
% | 
\RCSdef{\RCSConstrainedtriangulationRev}{$Revision$}
\RCSdefDate{\RCSConstrainedtriangulationDate}{$Date$}
% |
%%RefPage: end of header, begin of main body
% +------------------------------------------------------------------------+


\begin{ccRefClass}{Constrained_triangulation_2<Traits,Tds>}  %% add template arg's if necessary

%% \ccHtmlCrossLink{}     %% add further rules for cross referencing links
%% \ccHtmlIndexC[class]{} %% add further index entries

\ccDefinition
  
A constrained triangulation is a triangulation of a set of points
which has to include among its edges 
a given set of segments joining the points. The corresponding 
edges are called {\em constrained edges}. 

The set of points defining the vertices of the triangulation
includes the set of constrained edges endpoints. It may include other points
(considered as null length constrained edges) as well.  The set of
constrained edges forms a set of segments which do not intersect
except possibly at their endpoints.  Any number of constrained edges
are allowed to share the same endpoint.  Vertical constrained edges or
constrained edges with null length are allowed.

\begin{ccTexOnly}
\begin{center} \IpeScale{50} \Ipe{constraints.ipe} \end{center}
\end{ccTexOnly}

\begin{ccHtmlOnly}
<CENTER>
<img border=0 src=constraints.gif align=center alt="A set of
constraints and its constrained triangulation">
</CENTER>
\end{ccHtmlOnly}

The class \ccRefName\ of the CGAL library
implements constrained triangulations.
The template parameters \ccc{Traits} and \ccc{Tds}
are to be instantiated with models of respectively
\ccc{TriangulationTraits_2}  and
\ccc{TriangulationDdataStructure_2}.
 The information about constrained edges is store in the 
faces of the triangulation. Thus the nested \ccc{Face}
type of a constrained triangulation offers
additonnal functionalities to deal with this information.
Within the design of a triangulation data structure
models templated by the base face class and the base vertex class,
these additional functionalities related to the constraints
are requirements which have to be fulfilled
by the base face.
The base face of a constrained triangulation
has to be a model of the concept
\ccc{ConstrainedTriangulationFaceBase_2}.

\ccInclude{CGAL/Constrained_triangulation_2.h} 
 
\ccInheritsFrom

\ccc{Triangulation_2<Traits,Tds>}

\ccTypes
The only new type defined by
\ccRefName\ is a constraint type: a
constraint is represented as a pair of points.

\ccTypedef{typedef pair<Point,Point> Constraint;}{}



\ccCreation
\ccCreationVariable{ct}  %% choose variable name
The creators of the class build the constrained triangulation from a
list of constrained edges. Constrained edges are assumed to have no
intersection other than endpoints. Any number of constrained edges are
allowed to share the same endpoint.  Vertical constrained edges or
constrained edges with null length are allowed.

\ccConstructor{Constrained_triangulation_2();}{default constructor.}
\ccConstructor{Constrained_triangulation_2(const
Constrained_triangulation_2& ct1)}
 {Copy constructor, all faces and vertices
are duplicated and  the constrained status of edges
is copied. This last feature is not yet implemented.}

\ccConstructor{Constrained_triangulation_2(list<Constrained>& lc,
const Traits& t = Traits());}
{Introduces a constrained triangulation, the constrained edges of which
are the edges of the list \ccc{lc}.}

\ccConstructor{ template<class InputIterator> Constrained_triangulation_2(
        InputIterator first,
                               InputIterator last,
                               const Traits& t=Traits());}
{A templated constructor which introduces and builds
 a constrained triangulation with constrained edges in the range 
$\left[\right.$\ccc{first}, \ccc{last}$\left.\right)$.
\ccPrecond The \ccc{value_type} of \ccc{first} and \ccc{last}
 is \ccc{Constraint}.}

\ccHeading{Queries}
\ccMethod{ bool is_constrained(Edge e);}
{Returns true if edge \ccc{e} is a constrained edge.}
\ccMethod{bool are_there_incident_constraints(Vertex_handle v);}
{Returns true if at least one of the edges incident to vertex \ccc{v}
is constrained.}
\ccMethod{template<class OutputIterator>
  bool are_there_incident_constraints(Vertex_handle v, 
 OutputIterator out) const;}
{Same as above. In addition the constrained edges incident to \ccc{v}
are inserted in the sequence pointed to by the output iterator \ccc{out}.}


\ccHeading{Insertion and removal}

\ccMethod{Vertex_handle insert(Point a);}
{ Inserts point a and restore the status (constrained or not) of all
the
touched edges.}

\ccMethod{void insert(Point a, Point b);}
{ Inserts points a and b, and inserts segment ab as a
constraint. Removes the faces crossed by segment ab and creates new
faces instead. If a vertex c lies on segment ab, constraint ab is
replaced by the two constraints ac and cb. Apart from the insertion of
a and b, the algorithm runs in time proportionnal to the number of
removed triangles. 
\ccPrecond The relative interior of segment \ccc{ab} does not
intersect the relative interior of another constrained edge.}

\ccMethod{ void insert(const Vertex_handle & va, const Vertex_handle & vb);}
{ Inserts the line segment \ccc{s} whose endpoints are the vertices 
\ccc{va} and
\ccc{vb}  as a constrained edge \ccc{e}. The triangles intersected by s
are removed and new ones are created. 
\ccPrecond The relative interior of \ccc{s} does not
intersect the relative interior of another constrained edge.  
\ccPrecond  va and vb are distinct vertices of t.  }

\ccMethod{ void insert(const Vertex_handle & va, const Vertex_handle & vb,
	    Face_handle & fr, int & i);}
{Same as above. In addition,  sets the face \ccc{fr} incident to the
egde \ccc{e} 
and on the right of \ccc{e} oriented from  \ccc{va} to \ccc{vb}
and the index \ccc{i}  of the vertex of \ccc{fr} opposite to
\ccc{e}, i.e.  \ccc{e}=\ccc{(fr,i)}.}	    

\ccMethod{void insert(const Vertex_handle & va, const Vertex_handle & vb,
	    Face_handle & fr, int & i, List_edges & new_edges );}
{Same as above. In addition,  the edges that are created are put in
the list \ccc{new\_edges}.} 

\ccMethod{void remove(const Vertex_handle & v);}
{ Removes a vertex v. 
\ccPrecond Vertex \ccc{v}  is not incident to a constrained edge.}

\ccMethod{ void remove_incident_constraints(Vertex_handle  v);}
{Make the edges incident to vertex \ccc{v} unconstrained edges.}

\ccMethod{void remove_constraint(const Face_handle & f, int i);}
{ Edge \ccc{e}=\ccc{(f,i)}=\ccc{(g,j)} is no longer constrained.}

\ccHeading{I/O}

\ccFunction{ostream & operator<<(ostream& os, const CGAL_Constrained_triangulation_2<Traits,Tds> &Ct);}
{Writes the triangulation and, for each face f, and integers i=0,1,2,
write ``C'' or ``N'' depending whether edge 
\ccc{(f,i)} is constrained or not.}

\ccSeeAlso
\ccc{CGAL::Triangulation_2<Traits,Tds>}, \\
\ccc{TriangulationDataStructure_2}, \\
\ccc{TriangulationTraits_2} \\
\ccc{ConstrainedTriangulationFaceBase_2}



\ccHeading{Implementation}

The constructors build the triangulation using a sweeping line
algorithm. The complexity of this algorithm is $O(n\log n)$ if $n$
endpoints are present. The sweep structure is an \stl\ map.
 The insertion of a constrained edge runs in time
proportionnal to the number of triangles intersected by this edge.

There is no need for a special implementation
of the method \ccVar\ccc{.is_valid()}
because the base class function
\ccc{ Triangulation_2<Traits>::is_valid()}
call the face class method
\ccc{Tds::Face::is_valid()} 
which, in the case of a constrained triangulation,
includes a test of the consistency of the
information about constrained edges.

\end{ccRefClass}

% +------------------------------------------------------------------------+
%%RefPage: end of main body, begin of footer
% EOF
% +------------------------------------------------------------------------+

