% +------------------------------------------------------------------------+
% | Reference manual page: Delaunay_triangulation_traits_2.tex
% +------------------------------------------------------------------------+
% | 12.04.2000   Author
% | Package: Package
% | 
\RCSdef{\RCSDelaunaytriangulationtraitsRev}{$Revision$}
\RCSdefDate{\RCSDelaunaytriangulationtraitsDate}{$Date$}
% |
%%RefPage: end of header, begin of main body
% +------------------------------------------------------------------------+


\begin{ccRefConcept}{Delaunay_triangulation_traits_2}

%% \ccHtmlCrossLink{}     %% add further rules for cross referencing links
%% \ccHtmlIndexC[concept]{} %% add further index entries

\ccDefinition
  
In addition to the requirements 
of the concept \ccc{Triangulation_traits_2}
described \ccRefPage{Triangulation_traits_2},
the concept \ccRefName\
has to  provide a predicate type  \ccc{Side_of_oriented_circle_2} 
to implement the \ccc{side_of_oriented_circle} test.
This test test is the basic test used to maintain the
empty circle property and actually defines the triangulation.
The additional types \ccc{Line_2}, \ccc{Direction_2} and
and \ccc{Ray_2} and the constructor objects
\ccc{Construct_direction_2}, \ccc{Construct_ray_2}
\ccc{Construct_circumcenter_2}, \ccc{Construct_bisector_2},
\ccc{Construct_midpoint}
are used to build the dual Voronoi diagram
and are required only if the dual functions are called.
The additional predicate type \ccc{Less_distance_to_point_2} is o
required if calls to 
\ccc{nearest_vertex(..)} are issued.


\ccTypes
\ccThree{Oriented_side}{xxxxxx}{}
\ccThreeToTwo
\ccNestedType{Line_2}{The line type.}
\ccGlue
\ccNestedType{Direction_2}{The direction of a line.}
\ccGlue
\ccNestedType{Ray_2}{The type for ray.}




\ccNestedType{Side_of_oriented_circle_2}{Predicate object. Must
provide the operator
\ccc{Oriented_side operator()(Point p, Point q, Point r, Point s)}
which takes four points $p, q, r, s$ as arguments and returns
\ccc{ON_POSITIVE_SIDE}, \ccc{ON_NEGATIVE_SIDE} or, 
\ccc{ON_ORIENTED_BOUNDARY} according to the position of points \ccc{s}
with respect to the oriented circle through  through $p,q$
and $r$.}
\ccGlue
\ccNestedType{Less_distance_to_point_2} {Predicate type. Must provide
the constructor
\ccc{Less_distance_to_point_2( Point_2 p)}
and the operator
\ccc{bool operator()(Point_2 q, Point_2 r)}
which returns \ccc{true}
iff the distance between q and p is smaller
than the distance between r and p.  This type is only require if
\ccc{nearest_vertex} queries are issued.}

\ccNestedType{Construct_circumcenter_2}{Constructor object. Provides
the operator 
\ccc{ Point_2 operator()(Point_2 p, Point_2 q, Point_2 r)} which returns
the  circumcenter of the three points  \ccc{p, q} anf \ccc{r}.
This type is required only if  functions
relative to the dual Voronoi diagram are called.}
\ccGlue
\ccNestedType{Construct_bisector_2}{Constructor object. Provides
the operator 
\ccc{ Line_2 operator()(Point_2 p, Point_2 q)} which constructs the
bisector line of points \ccc{p} and \ccc{q}.
This type is required only if  functions
relative to the dual Voronoi diagram are called.}
\ccGlue
\ccNestedType{Construct_direction_of_line_2}{Constructor object  to build
the direction of a line. Must provide 
\ccc{Direction_2 operator()(Line_2 l);}.}
\ccGlue
\ccNestedType{Construct_ray_2}{A constructor object  to build
a ray from a point and a direction. Must provide
\ccc{Ray_2 operator() ( Point_2 p, Direction_2 d);}}
   

\ccCreation
\ccCreationVariable{traits}  %% choose variable name

\ccConstructor{Delaunay_triangulation_traits_2();}{default constructor.}
\ccGlue
\ccConstructor{Delaunay_triangulation_traits_2(Delaunay_triangulation_traits_2 
dtt);}{copy constructor}
\ccGlue
\ccMethod{Delaunay_triangulation_traits_2 
operator=(Delaunay_triangulation_traits_2);}{Assignment operator.}


\ccHeading{Access to predicate and constructor objects}
\ccMethod{Side_of_oriented_circle_2
side_of_oriented_circle_2_object();}{}
\ccGlue
\ccMethod{Less_distance_to_point_2
less_distance_to_point_2_object(Point_2 p);}
{This function is required only if
\ccc{nearest_vertex} queries are issued.
It forwards its argument $p$ to the constructor
of \ccc{Less_distance_to_point_2}.}
\ccGlue
\ccMethod{Construct_circumcenter_2 construct_circumcenter_2_object();}
{Required only if  functions
relative to the dual Voronoi diagram are called.}
\ccGlue
\ccMethod{Construct_bisector_2 construct_bisector_2_object();}
{Required only if  functions
relative to the dual Voronoi diagram are called.}
\ccGlue
\ccMethod{Construct_direction_of_line_2 construct_direction_of_line_2_object();} {}
\ccGlue
\ccMethod{Construct_ray_2 construct_ray_2_object();} {}


\ccHasModels
\ccc{CGAL::Cartesian<R>} \\
\ccc{CGAL::Homogeneous<R>} \\
\ccc{CGAL::Triangulation_euclidean_traits_2<Rep>}, \\
\ccc{CGAL::Triangulation_euclidean_traits_xy_3<Rep>}, \\
\ccc{CGAL::Triangulation_euclidean_traits_yz_3<Rep>}, \\
\ccc{CGAL::Triangulation_euclidean_traits_zx_3<Rep>}.

\ccSeeAlso
\ccc{Triangulation_traits_2}

\end{ccRefConcept}

% +------------------------------------------------------------------------+
%%RefPage: end of main body, begin of footer
% EOF
% +------------------------------------------------------------------------+

