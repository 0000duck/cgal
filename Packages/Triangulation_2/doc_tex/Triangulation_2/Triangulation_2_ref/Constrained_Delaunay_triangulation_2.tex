% +------------------------------------------------------------------------+
% | Reference manual page: Constrained_Delaunay_triangulation_2.tex
% +------------------------------------------------------------------------+
% | 12.04.2000   Author
% | Package: Package
% | 
\RCSdef{\RCSConstrainedDelaunaytriangulationRev}{$Revision$}
\RCSdefDate{\RCSConstrainedDelaunaytriangulationDate}{$Date$}
% |
%%RefPage: end of header, begin of main body
% +------------------------------------------------------------------------+


\begin{ccRefClass}{Constrained_Delaunay_triangulation_2<Traits,Tds>}  %% add template arg's if necessary

%% \ccHtmlCrossLink{}     %% add further rules for cross referencing links
%% \ccHtmlIndexC[class]{} %% add further index entries

\ccDefinition
A constrained Delaunay triangulation is a triangulation with
constrained edges which tries to be as much Delaunay as possible.
As constrained edges are not necessarily Delaunay edges,
the triangles of a constrained Delaunay triangulation do not
necessarily fulfill the empty circle property
but they fulfill a weaker constrained empty circle property.
 To state this property,
it is convenient to think of  constrained
edges as blocking the view. Then, a triangulation is 
constrained Delaunay if
 the circumscribing circle
of any of its triangular faces includes in its interior 
no vertex  that is visible
from the interior of the triangle.  
The class \ccRefName\ is designed to represent
constrained Delaunay triangulations.

The template parameters \ccc{Tds} 
has to be instantiate with a model of \ccc{Triangulation_data_structure_2}.
The geometric traits 
of a constrained Delaunay triangulation is required
to provide the \ccc{side_of_oriented_circle} test as the geometric traits
of a Delaunay triangulation and the \ccc{Traits}
parameter has 
to be instantiated with a model
\ccc{Delaunay_triangulation_traits_2}.

A constrained Delaunay triangulation is not a Delaunay
triangulation but it is a constrained triangulation.
Therefore the class \ccRefName\ derives from
the class \ccc{Constrained_triangulation_2<Traits,Tds>}.
Also, information about the status (constrained or not)
of the edges of the triangulation is stored
in the faces. 
Thus the nested \ccc{Face}
type of a constrained triangulation offers
additonnal functionalities to deal with this information.
Within the design of a triangulation data structure
models templated by the base face class and the base vertex class,
these additional functionalities related to the constraints
are requirements which have to be fulfilled
by the base face.
The base face of a constrained triangulation
has to be a model of the concept
\ccc{Constrained_triangulation_face_base_2}.

\ccInclude{Constrained_Delaunay_triangulation_2.h}

\ccInheritsFrom \ccc{Constrained_triangulation_2<Traits,Tds>}

\ccTypes
All types used in this class are inherited from the base class
\ccc{Constrained_triangulation_2<Traits,Tds>}.



\ccCreation
\ccCreationVariable{cdt}  %% choose variable name

\ccConstructor{Constrained_Delaunay_triangulation_2(const Traits &t = Traits())}
{Introduces an empty constrained Delaunay triangulation \ccVar.}

\ccConstructor{Constrained_Delaunay_triangulation_2(const
Constrained_Delaunay_triangulation_2& cdtbis)}
 {Copy constructor, all faces and vertices
are duplicated and  the constrained status of edges
is copied. This last feature is not yet implemented.}

\ccConstructor{Constrained_Delaunay_triangulation_2(list<Constrained>& lc, Traits& t = Traits());}
{Introduces a constrained triangulation, the constrained edges of which
are the edges of the list \ccc{lc}.}


\ccConstructor{ template<class InputIterator> Constrained_triangulation_2(
        InputIterator first,
                               InputIterator last,
                               const Traits& t=Traits());}
{A templated constructor which introduces and builds
 a constrained triangulation with constrained edges in the range 
$\left[\right.$\ccc{first}, \ccc{last}$\left.\right)$.
\ccPrecond The \ccc{value_type} of \ccc{first} and \ccc{last}
 is \ccc{Constraint}.}


\ccHeading{Insertion and Removal}
The following member functions override the corresponding
member of the base class to include a step restoring
 the Delaunay constrained
property after modification of the triangulation.

\ccMethod{ Vertex_handle insert(Point a);}
{ Inserts point a in the triangulation. }

\ccMethod{ void insert(Point a, Point b);}
{ Inserts segment ab as a constrained edge in the triangulation. }

\ccMethod{ void insert(Vertex_handle va, Vertex_handle vb);}
{ Inserts the line segment whose endpoints are the vertices va and vb
as an edge e in the triangulation. }

\ccMethod{void insert(Vertex_handle va, Vertex_handle vb,
		   Face_handle & fr, int & i);}
{Same as the previous procedure. In addition, returns f and i such that
e=(f,i) and f is the face lying to the right of the oriented edge (va,vb). }


\ccMethod{void remove(Vertex_handle & v);}
{ Removes vertex v. }


\ccHeading{Checking}

\ccMethod{bool is_valid();}
{ Checks if the triangulation is valid and if each constrained edge is
a constraint for its two incident faces.}

\begin{ccAdvanced}
\ccHeading{Flips}

\ccMethod{bool is_flipable(Face_handle f, int i);}
{ Determines if edge (f,i) can be flipped. Returns true if
1. edge(f,i) is not constrained and 2. the circle circumscribing f
does not contain the vertex of f->neighbor(i) not on edge(f,i). }

\ccMethod{void flip(Face_handle& f, int i);}
{ Flip f and f->neighbor(i).
\ccPrecond f->is\_constrained(i) == FALSE.}


\ccMethod{void propagating_flip(List_edges & edges);}
{ Makes the triangulation constrained Delaunay by flipping edges. 
List edges contains an
initial list of edges to be flipped. The returned
triangulation is constrained Delaunay 
if the list edges contains all edges of the
input triangulation that need to be flipped (plus possibly others). }
\end{ccAdvanced}


\ccSeeAlso
\ccc{CGAL::Constrained_triangulation_2<Traits,Tds>} \\
\ccc{Triangulation_data_structure_2} \\
\ccc{Delaunay_triangulation_traits_2} \\
\ccc{Constrained_triangulation_face_base_2}

\ccExample



%% \ccIncludeExampleCode{examples/Package/Constrained_Delaunay_triangulation_2_prog.C}

\end{ccRefClass}

% +------------------------------------------------------------------------+
%%RefPage: end of main body, begin of footer
% EOF
% +------------------------------------------------------------------------+

