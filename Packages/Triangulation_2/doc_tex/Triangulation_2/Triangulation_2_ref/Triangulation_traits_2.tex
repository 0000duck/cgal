% +------------------------------------------------------------------------+
% | Reference manual page: Triangulation_traits_2.tex
% +------------------------------------------------------------------------+
% | 06.04.2000   Author
% | Package: Package
% | 
\RCSdef{\RCSTriangulationtraitsRev}{$Revision$}
\RCSdefDate{\RCSTriangulationtraitsDate}{$Date$}
% |
%%RefPage: end of header, begin of main body
% +------------------------------------------------------------------------+


\begin{ccRefConcept}{Triangulation_traits_2}

%% \ccHtmlCrossLink{}     %% add further rules for cross referencing links
%% \ccHtmlIndexC[concept]{} %% add further index entries

\ccDefinition
  
The concept \ccRefName\ describes 
 the set of requirements to be
fulfilled
by any class used to instantiate the first template parameter of the class 
\ccc{Triangulation_2<Traits,Tds>}.
This concept provides the types of the geometric primitives used in the
triangulation and the some function object types
for the required predicates on those primitives.

\ccTypes
\ccNestedType{Point_2}{The  type must provide
a copy constructor and assignment operator.}
\ccGlue
\ccNestedType{Segment_2}{The segment  type. Must provide a constructor that takes
two points as arguments.}
\ccGlue
\ccNestedType{Triangle_2}{The triangle type. Must provide a constructor
that takes three points as arguments.}

\ccNestedType{Construct_segment_2} {A constructor object for
\ccc{Segment_2}.Must provide \ccc{Segment_2 operator()(Point_2 p,Point_2
                       q)}, which constructs a  segment from two points.}
\ccGlue
\ccNestedType{Construct_triangle_2} {A constructor object for
\ccc{Triangle_2}.Must provide \ccc{Triangle_2 operator()(Point_2 p,Point_2
                       q,Point_2 r )}, which constructs a triangle from three points.}

\ccNestedType{Compare_x_2}{Predicate object. Must provide
the operator 
\ccc{Comparison_result operator()(Point p, Point q)}
which returns 
\ccc{SMALLER, EQUAL} or \ccc{ LARGER}
 according ding to the
$x$-ordering of points \ccc{p} and \ccc{q}.}
\ccGlue
\ccNestedType{Compare_y_2}{Predicate object. Must proveide
the operator 
\ccc{Comparison_result operator()(Point p, Point q)}
which returns the
(\ccc{SMALLER, EQUAL} or \ccc{ LARGER})
according to the
$y$-ordering of points \ccc{p} and \ccc{q}.}
\ccGlue
\ccNestedType{Orientation_2}{Predicate object. Must provide
the  operator
\ccc{Orientation operator()(Point p, Point q, Point r)}
which returns
\ccc{LEFTTURN} (resp. \ccc{RIGHTTURN}) is $r$ is on the left 
(resp. on the right) of the oriented line \ccc{pq}
and \ccc{COLLINEAR} is the three points are colinear.}
\ccGlue
\ccNestedType{Side_of_oriented_circle_2}{Predicate object. Must
provide the operator
\ccc{Oriented_side operator()(Point p, Point q, Point r, Point s)}
which takes four points $p, q, r, s$ as arguments and returns
\ccc{ON_POSITIVE_SIDE}, \ccc{ON_NEGATIVE_SIDE} or, 
\ccc{ON_ORIENTED_BOUNDARY} according to the position of points \ccc{s}
with respect to the oriented circle through  through $p,q$
and $r$.
 This type is required only if the function 
\ccc{side_of_oriented_circle(Face_handle f, const Point & p)} is
called.}


\ccNestedType{Construct_circumcenter_2}{Constructor object. Provides
the operator 
\ccc{ Point operator()(Point p, Point q, Point r)} which returns
the  circumcenter of the three points  \ccc{p, q} anf \ccc{r}.
This type is required only if  the function 
\ccc{Point circumcenter(Face_handle  f)}is called.}

\ccCreation
\ccCreationVariable{traits}  %% choose variable name
Only a default constructor, copy constructor
 and an assignement operator are required. 
Note that further constructors
can be provided. 

\ccConstructor{Triangulation_traits_2();}{default constructor.}
\ccGlue
\ccConstructor{Triangulation_traits_2(Triangulation_traits_2);}
{Copy constructor}
\ccMethod{Triangulation_traits_2 operator=(Triangulation_traits_2 gtr);}
{Assignment operator.}

\ccHeading{Predicate functions}
The following functions give access to the  predicate 
and constructor objects.
\ccThree{Construct_segment_2}{gt.compare_x(Point p0, Point p1)x}{}
\ccMethod{Construct_segment_2 construct_segment_2_object();}{}
\ccGlue
\ccMethod{Construct_triangle_2 construct_triangle_2_object();}{}
\ccGlue
\ccMethod{Comparison_x_2 compare_x_2_object();}{}                               \ccGlue
\ccMethod{Comparison_y_2 compare_y_2_object();}{}
\ccGlue
\ccMethod{Orientation_2  orientation_2_object();}{}
\ccGlue
\ccMethod{Side_of_oriented_circle_2
side_of_oriented_circle_2_object();}
{Required only
if \ccc{side_of_oriented_circle(Face_handle f, const Point & p)} is
called.}
\ccGlue
\ccMethod{Construct_circumcenter_2 construct_circumcenter_2_object();}
{Required only
if \ccc{Point circumcenter(Face_handle  f)}is called.}



\ccHasModels
\ccc{Triangulation_euclidean_traits_2<R>}
\ccc{Triangulation_euclidean_traits_xy_3<R>}
\ccc{Triangulation_euclidean_traits_yz_3<R>}
\ccc{Triangulation_euclidean_traits_zx_3<R>}

\ccSeeAlso
\ccc{Triangulation_2<Traits,Tds>}



\end{ccRefConcept}

% +------------------------------------------------------------------------+
%%RefPage: end of main body, begin of footer
% EOF
% +------------------------------------------------------------------------+

