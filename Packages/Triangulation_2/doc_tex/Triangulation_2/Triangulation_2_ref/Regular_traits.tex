% +------------------------------------------------------------------------+
% | Reference manual page: Regular_triangulation_traits_2.tex
% +------------------------------------------------------------------------+
% | 12.04.2000   Author
% | Package: Package
% | 
\RCSdef{\RCSRegulartriangulationtraitsRev}{$Revision$}
\RCSdefDate{\RCSRegulartriangulationtraitsDate}{$Date$}
% |
%%RefPage: end of header, begin of main body
% +------------------------------------------------------------------------+


\begin{ccRefConcept}{Regular_triangulation_traits_2}

%% \ccHtmlCrossLink{}     %% add further rules for cross referencing links
%% \ccHtmlIndexC[concept]{} %% add further index entries

\ccDefinition
  
The concept \ccRefName\ fulfill the requirements of
concept \ccc{Triangulation_traits_2} 
described \ccRefPage{Triangulation_traits_2}
and additionally provides  a type
\ccc{Weitghed_point}  (which derives from the type \ccc{Point_2} 
and add a weight data member )
and the power tests for weighted points
and two function objects type for generic and degenerate
power\_test predicates.
%The traits must also provide the types 
%\ccc{Ray} and \ccc{Line} to output the dual power diagram,



\ccTypes
\ccNestedType{Bare_point}{Another name for the \ccc{Point_2} type.}
\ccGlue    
\ccNestedType{Weighted_point}{Is required to derive from \ccc{Point_2}
and to be a model of the \ccc{Weighted_point} concept.}
%\ccGlue
%\ccNestedType{Line_2}{Only required if the dual power diagram is asked.}
%\ccGlue
%\ccNestedType{Ray_2}{Only required if the dual power diagram is asked.}
%\ccGlue

\ccNestedType{Power_test_2}{A predicate object type. Must provide
the operator 
\ccc{Oriented_side operator() ( Weighted_point p,
			     Weighted_point q,
			     Weighted_point r,
			     Weighted_point s) }
which is the power test for points \ccc{p}, \ccc{q}, \ccc{r} and 
\ccc{s}. \ccPrecond{the bare points corresponding to 
\ccc{p}, \ccc{q}, \ccc{r} are not collinear}.}
\ccGlue
\ccNestedType{Power_test_degenerated_2}{A predicate object type. Must provide
the operator 
\ccc{Oriented_side operator() ( Weighted_point p,
			     Weighted_point q,
			     Weighted_point r) }
which is the degenerated power test for collinear points 
\ccc{p}, \ccc{q}, \ccc{r}.
\ccPrecond{the bare points corresponding to 
\ccc{p}, \ccc{q}, \ccc{r} are collinear and \ccc{p != q}.}}


 

\ccCreation
\ccCreationVariable{traits}  %% choose variable name

\ccConstructor{Regular_triangulation_traits_2();}{default constructor.}
\ccConstructor{ Regular_triangulation_traits_2 ( 
    const  Regular_triangulatio_traits_2& );} { copy constructor.}
\ccMethod{Regular_triangulation_euclidean_traits_2& operator=
  (const  Regular_triangulation_euclidean_traits_2& );} {assignement operator}


\ccHeading{Access to predicate objects}
\ccMethod{Power_test_2 power_test_2_object();}{}
\ccGlue
\ccMethod{Power_test_degenerated_2
  power_test_degenerated_2_object();}{}

%\ccOperations
%\ccMethod{Rt power(Weighted_point wp, Weighted_point wq) const;}{Returns the power product
%of \ccc{wp} \ and \ccc{wq}. This function is NOT YET required nor used.}


\ccHasModels

\ccc{CGAL::Regular_triangulation_traits_2<Rep>}


\end{ccRefConcept}

% +------------------------------------------------------------------------+
%%RefPage: end of main body, begin of footer
% EOF
% +------------------------------------------------------------------------+

