% +------------------------------------------------------------------------+
% | Reference manual page: Triangulation_data_structure_2.tex
% +------------------------------------------------------------------------+
% | 07.04.2000   Author
% | Package: Package
% | 
\RCSdef{\RCSTriangulationdatastructureRev}{$Revision$}
\RCSdefDate{\RCSTriangulationdatastructureDate}{$Date$}
% |
%%RefPage: end of header, begin of main body
% +------------------------------------------------------------------------+

\begin{ccRefConcept}{TriangulationDataStructure_2}

%% \ccHtmlCrossLink{}     %% add further rules for cross referencing links
%% \ccHtmlIndexC[concept]{} %% add further index entries
\ccCreationVariable{tds}
\ccDefinition
  
The concept \ccRefName\ describes the requirements  for
the second template parameter of the basic triangulation class
\ccc{Triangulation_2<Traits,Tds>}.
The concept can be seen as a container for the 
faces and vertices maintaining incidence and adjacency relations.
The triangulation data structure
 is responsible for  the combinatorial integrity of the triangulation
(i. e. proper incidence and adjacency relations among vertices
and faces) while 
allowing  to perform combinatorial modifications
such has 
insertion of a new vertex in a face, or in an edge,
suppression of a vertex of degree three,  flip of two edges.
The term combinatorial means that those operation are purely topological
and do not depend on the geometric embedding.


\ccTypes
%\ccTypedef{typedef  Vb Vertex_base;}{the \ccc{Vertex_base}}
%\ccGlue
%\ccTypedef{typedef  Fb Face_base;}{the \ccc{Face_base}}
%\ccGlue
\ccNestedType{Vertex} {}
\ccGlue
\ccNestedType{Face}{}
\ccGlue
\ccTypedef{typedef std::pair<Face*,int> Edge;}{ Edge type.}
\ccGlue
\ccTypedef{typedef std::list<Edge> List_edges;}{ Lists of edges.}

To ensure all the \textbf{flexibility} of class \ccc{Triangulation_2}
described in \ccRefPage{CGAL::Triangulation_2<Traits,Tds>}
and in Chapter~\ref{I1_Chapter_Triangulations}, a model of 
\ccRefName derives its \ccc{Vertex} and \ccc{Face} types
from two template parameters \ccc{Vb} and \ccc{Fb}.
The parameters \ccc{Vb} and \ccc{Fb}
have to be instantiated respectively
by models of \ccc{TriangulationVertexBase_2} and 
\ccc{TriangulationFaceBase_2}.




The following circulators allow to visit all the vertices, edges and faces
incident to a given vertex. They are all bidirectional and non mutable.

\ccNestedType{Face_circulator}{}
\ccGlue
\ccNestedType{Edge_circulator}{}
\ccGlue
\ccNestedType{Vertex_circulator}{}
 




\ccCreation
\ccCreationVariable{tds}  %% choose variable name

\ccConstructor{TriangulationDataStructure_2();}{default constructor.}

\ccConstructor{TriangulationDataStructure_2( const Tds& tds1)}
{Copy constructor. All the vertices and faces are duplicated.}

\ccMethod{TriangulationDataStructure_2<Vb,Fb> operator=(TriangulationDataStructure_2<Vb,Fb> tds1);}
{Assignation. All the vertices and faces are duplicated.}

\ccMethod{void swap(TriangulationDataStructure_2<Vb,Fb> tds1);}
{Swaps \ccVar\ and \ccc{tds1}. Should be preferred to \ccVar=\ccc{tds1} or \ccVar(\ccc{tds1})
when tds1 is deleted after that.}
\ccMethod{void clear();}{Deletes all faces and all finite vertices.}

\ccFunction{void ~Tds<Vb,Fb> ();}
{Destructor. All vertices and faces are deleted.}


\ccAccessFunctions
\ccThree{void}{tds.set_number_of_vertices(int n)}{}
\ccMethod{int dimension() const;}
{The dimension of the triangulation.}
\ccGlue
\ccMethod{int number_of_vertices() const;}
{The number of vertices in the data structure.}
\ccGlue
\ccMethod{int number_of_faces() const ;}
{The number of two dimensional faces in the data structure.}
\ccGlue
\ccMethod{int number_of_edges() const;}
{The number of edges  in the triangulation data structure.}
\ccGlue
\ccMethod{int number_of_full_dim_faces() const;}
{The number of full dimensional faces, 
i.e. faces of dimension equal to the dimension
of the triangulation. This is the actual
number of faces stored in the triangulation data structure.}

\ccHeading{Setting}
\ccMethod{void set_number_of_vertices(int n);}{}
\ccGlue
\ccMethod{void set_dimension (int n);}{}

\ccHeading{Queries}
\ccMethod{bool is_vertex(const Vertex* v) const;}{returns true if
\ccc{v} is a vertex of \ccVar.}
\ccGlue
\ccMethod{bool is_edge(const Vertex* va, const Vertex* vb) const;}{returns true if
\ccc{va vb} is an edge of \ccVar.}
\ccGlue
\ccMethod{bool is_edge(const Vertex* va, const Vertex* vb, Face* &fr,
int &i) const;}{as previous. In addition, if true is returned
\ccc{fr} and \ccc{i} are set such that the pair \ccc{(fr,i)}
is the descrition 
of  the ordered edge \ccc{va vb}.}
\ccGlue
\ccMethod{bool is_face(const Vertex* v1, const Vertex* v2, const
Vertex* v3) const;}{\ccc{true} if there is a face having \ccc{v1}, \ccc{v2} and\ccc{v3} 
as vertices.}
\ccGlue
\ccMethod{bool is_face(const Vertex* v1, const Vertex* v2, const Vertex* v3,
      Face* &fr) const;}{as above. In addition, if \ccc{true} is returned, fr is a pointer
to the face with  \ccc{v1}, \ccc{v2} and \ccc{v3} 
as vertices.}

\ccHeading{Modifiers}
The following modifier member functions  guarantee
the combinatorial validity of the resulting triangulation.

\ccThree{void}{tds.remove_second(Vertex* v)x}{}
\ccMethod{void flip(Face* f, int i);}{exchanges the edge incident to
\ccc{f} and \ccc{f->neighbor(i)} with the other
diagonal of the quadrilateral formed by \ccc{f} and  \ccc{f->neighbor(i)}.}


\begin{figure}
\begin{ccTexOnly}
\begin{center} %\IpeScale{70} \Ipe{Flip.ipe} \end{center}
\input{flip.ltex}
\end{center}
\end{ccTexOnly} 
\caption{Flip.
\label{I1_fig_flip}}

\begin{ccHtmlOnly}
<CENTER>
<img border=0 src=Flip.gif align=center alt="Flip">
</CENTER>
\end{ccHtmlOnly} 
\end{figure}


\ccMethod{Vertex* insert_first();} {creates the first 
vertex and returns a pointer to it.}
\ccGlue
\ccMethod{Vertex* insert_second();} {creates the second 
vertex and returns a pointer to it.}
\ccMethod{Vertex* insert_in_edge(Face* f, int i);} {adds a
vertex \ccc{v} splitting 
edge \ccc{i} of face \ccc{f}. Return a  pointer to \ccc{v}.}
\ccGlue
\ccMethod{Vertex* insert_in_face(Face* f);} {adds a vertex
\ccc{v} splitting  face
\ccc{f} in three. Face \ccc{f} is modified,
two new faces are created. Return a  pointer to \ccc{v} }
\ccGlue
\ccMethod{Vertex* insert_dim_up(Vertex* w, bool orient=true);} {adds
a vertex \ccc{v}, increasing by one the dimension of the triangulation.
Vertex \ccc{v} and the existing vertex \ccc{w} are linked to all 
the vertices of the triangulation. 
The boolean \ccc{orient} decides the final orientation of all 
faces. A pointer to vertex \ccc{v} is returned.
}

\begin{figure}
\begin{ccTexOnly}
%\begin{center} \IpeScale{70} \Ipe{Three.ipe} \end{center}
\begin{center} \documentclass{article}
\documentclass{article}
\documentclass{article}
\input{tmp.inputs}
\pagestyle{empty}
\begin{document}
\input{tmp.pstex_t}
\end{document}

\pagestyle{empty}
\begin{document}
\documentclass{article}
\input{tmp.inputs}
\pagestyle{empty}
\begin{document}
\input{tmp.pstex_t}
\end{document}

\end{document}

\pagestyle{empty}
\begin{document}
  \begin{tabular}{ccc}
    \input{insert1.pstex_t} &
    \input{insert2.pstex_t} &
    \input{insert3.pstex_t}\\
  {\small (a)} & {\small (b)} & {\small (c)}\\
  \end{tabular}
\end{document}
 \end{center}
\caption{Insertion}
\end{ccTexOnly} 

\begin{ccHtmlOnly}
<CENTER>
<img border=0 src=Three.gif align=center alt="Insertion">
</CENTER>
\end{ccHtmlOnly} 
\end{figure}



\ccMethod{void remove_degree_3(Vertex* v, Face *f=NULL);}
{removes a vertex of degree 3. Two of the incident faces are destroyed,
the third one is modified.
If parameter \ccc{f}  is specified, it has to be a face incident to \ccc{v}
and will be the modified face.
\ccPrecond{Vertex
\ccc{v} is a finite vertex with degree 3
and, if specified, face \ccc{f} is incident to \ccc{v}.}}



\ccMethod{void remove_second(Vertex* v);}{removes the before last
vertex.}
\ccGlue
\ccMethod{void remove_first(Vertex* v);}{removes the last vertex.}
\ccGlue
\ccMethod{void remove_dim_down(Vertex* v);}
{removes vertex \ccc{v} incident to all other vertices
and  decreases by one the dimension of the triangulation.
\ccPrecond{if the dimension is 2, the number of vertices is more than
3,
if the dimension is 1, the number of vertices is 2.} }


\begin{ccAdvanced}
The following modifiers are required for convenience of the advanced
user.
They do not guarantee the combinatorial validity 
of the resulting triangulation.

\ccMethod{ template< class EdgeIt>
   Vertex* star_hole(EdgeIt edge_begin,EdgeIt edge_end);}
{creates a new vertex \ccc{v} and use it to star the hole 
whose boundary is described  by the sequence of edges \ccc{[edge_begin, 
edge_end[}. Returns a pointer to the  vertex.}
\ccGlue 
\ccMethod{   template< class EdgeIt, class FaceIt>
   Vertex* star_hole(EdgeIt edge_begin, 
		    EdgeIt edge_end,
 		    FaceIt face_begin,
		    FaceIt face_end);}
{ same as above, except that, to build the new faces, the  algorithm 
first recycles faces in the sequence \ccc{[face_begin, 
face_end[} and create new ones when the sequence is exhausted.}
\ccGlue
\ccMethod{   template< class EdgeIt>
 void  star_hole(Vertex* v, EdgeIt edge_begin,  EdgeIt edge_end);}
{uses vertex v to  star the hole 
whose boundary is described  by the sequence of edges\ccc{[edge_begin, 
edge_end[}.
}
\ccGlue
\ccMethod{   template< class EdgeIt, class FaceIt>
   void  star_hole(Vertex* v,
 		  EdgeIt edge_begin, 
 		  EdgeIt edge_end,
 		  FaceIt face_begin,
 		  FaceIt face_end);}
{same as above, recycling faces in the sequence  \ccc{[face_begin, 
face_end[ . } }

%\ccMethod{Vertex* star_hole(List_edges& hole);}
%{create a new vertex \ccc{v} and use it to star the hole \ccc{hole}.}
%\ccGlue
%\ccMethod{void star_hole(Vertex* v, List_edges& hole);}
%{stars the hole \ccc{hole} using vertex \ccc{v}.}
\ccGlue
\ccMethod{void make_hole(Vertex* v, List_edges& hole);}
{removes the vertex v, and store in \ccc{hole} the list of edges
on the boundary of the hole.}

\ccMethod{Vertex* create_vertex();}{adds a new vertex.}
\ccGlue
\ccMethod{Face* create_face(Face* f1, int i1, Face* f2, int i2, Face*
f3, int i3);}{adds a face which is the neighbor \ccc{i1} of \ccc{f1}, 
\ccc{i2} of \ccc{f2} and \ccc{i3} of \ccc{f3}.}
\ccGlue
\ccMethod{Face* create_face(Face* f1, int i1, Face* f2, int i2);}
{adds a face which is the neighbor \ccc{i1} of \ccc{f1}, 
and the neighbor \ccc{i2} of \ccc{f2}.}
\ccGlue
\ccMethod{Face* create_face(Face* f1, int i1, Vertex* v);}
{adds a face which is the neighbor \ccc{i1} of \ccc{f1},
and has \ccc{v} as vertex.}
\ccGlue
\ccMethod{ Face* create_face(Vertex* v1, Vertex* v2, Vertex* v3);}
{adds a face with vertices \ccc{v1}, \ccc{v2} and \ccc{v3}.}
\ccGlue
\ccMethod{Face* create_face(Vertex* v1, Vertex* v2, Vertex* v3,
		    Face* f1, Face* f2, Face* f3);}
{adds a face with vertices \ccc{v1}, \ccc{v2} and \ccc{v3},
and neighbors \ccc{f1}, \ccc{f2}, \ccc{f3}.}
\ccGlue
\ccMethod{Face* create_face();}
{adds a face whose vertices and neighbors are set to NULL.}
 \ccGlue
\ccMethod{void  delete_face(Face*);}{deletes a face.}
\ccGlue
\ccMethod{void  delete_vertex(Vertex*);}{deletes a vertex.}
\end{ccAdvanced}

\ccHeading{Traversing the triangulation}
\ccThree{Vertex_iterator}{tds.number_of_faces()x}{}
\ccMethod{Face_iterator faces_begin() const;}{visits all faces}
\ccGlue
\ccMethod{Face_iterator faces_end() const;}{}
\ccGlue
\ccMethod{Vertex_iterator vertices_begin() const;}{visits all vertices}
\ccGlue
\ccMethod{Vertex_iterator vertices_end() const;}{}
\ccGlue
\ccMethod{Edge_iterator edges_begin() const;}{visits all edges}
\ccGlue
\ccMethod{ Edge_iterator edges_end() const;}{}

\ccHeading{Miscelleanous}

\ccMethod{int ccw(int i) const;}{returns $i+1$ modulo 3.\ccPrecond $0\leq i \leq 2$.}
\ccGlue
\ccMethod{int cw(int i) const;}
{returns $i+2$ modulo 3.\ccPrecond $0\leq i \leq 2$.}
\ccGlue
\ccMethod{bool is_valid();}{checks the combinatorial validity of the
triangulation: call the \ccc{is_valid()} member function for each vertex and 
each face, checks the number of vertices and the Euler relation
between numbers of vertices, faces and edges.}


\ccHasModels
\ccc{CGAL::Triangulation_default_data_structure_2<Tds_traits,Vb,Fb>}, \\
\ccc{CGAL::Triangulation_data_structure_using_list_2<Vb,Fb>}.

\ccSeeAlso
\ccc{TriangulationDataStructure_2::Face} \\
\ccc{TriangulationDataStructure_2::Vertex} \\
\ccc{CGAL::Triangulation_2<Traits,Tds>}

\end{ccRefConcept}

% +------------------------------------------------------------------------+
%%RefPage: end of main body, begin of footer
% EOF
% +------------------------------------------------------------------------+

