% +------------------------------------------------------------------------+
% | Reference manual page: Constrained_triangulation_face_base_2.tex
% +------------------------------------------------------------------------+
% | 12.04.2000   Author
% | Package: Package
% | 
\RCSdef{\RCSConstrainedtriangulationfacebaseRev}{$Revision$}
\RCSdefDate{\RCSConstrainedtriangulationfacebaseDate}{$Date$}
% |
%%RefPage: end of header, begin of main body
% +------------------------------------------------------------------------+


\begin{ccRefClass}{Constrained_triangulation_face_base_2<Traits>}  %% add template arg's if necessary

%% \ccHtmlCrossLink{}     %% add further rules for cross referencing links
%% \ccHtmlIndexC[class]{} %% add further index entries

\ccDefinition
  
The class \ccRefName\ is the default model for the concept
\ccc{Constrained_triangulation_face_base_2}.
It derives from \ccc{Triangulation_face_base_2<Traits>},
add three boolean to deal with information about
constrained edges and override the
functions \ccc{reorient()}, \ccc{ccw_permute()} and 
\ccc{cw_permute()}.

\ccInclude{Constrained_triangulation_face_base_2.h}

\ccIsModel

\ccc{Constrained_triangulation_face_base_2}


\ccSeeAlso
\ccc{CGAL::Constrained_triangulation_2<Traits,Tds>} \\
\ccc{Triangulation_face_base_2}


\ccExample
See the example of \ccRefPage{CGAL::Constrained_triangulation_2<Traits,Tds>}.




\end{ccRefClass}

% +------------------------------------------------------------------------+
%%RefPage: end of main body, begin of footer
% EOF
% +------------------------------------------------------------------------+

