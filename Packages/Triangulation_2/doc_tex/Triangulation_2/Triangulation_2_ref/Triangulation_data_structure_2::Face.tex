% +------------------------------------------------------------------------+
% | Reference manual page: Triangulation_data_structure_2::Face.tex
% +------------------------------------------------------------------------+
% | 07.04.2000   Author
% | Package: Package
% | 
\RCSdef{\RCSTriangulationdatastructure::FaceRev}{$Revision$}
\RCSdefDate{\RCSTriangulationdatastructure::FaceDate}{$Date$}
% |
%%RefPage: end of header, begin of main body
% +------------------------------------------------------------------------+


\begin{ccRefConcept}{Triangulation_data_structure_2::Face}

%% \ccHtmlCrossLink{}     %% add further rules for cross referencing links
%% \ccHtmlIndexC[concept]{} %% add further index entries

\ccDefinition
  
The concept \ccRefName\  describes the Face class of a 
\ccc{Triangulation_data_structure_2} 
\ccRefPage{Triangulation_data_structure_2}.
A \ccRefName\ stores three pointers to its three vertices
and three pointers to its three neighbors.
The vertices are indexed 0,1, and 2 in counterclockwise order.
The  neighbor indexed $i$ lies
opposite to vertex i.

To ensure all the \textbf{flexibility} of class \ccc{Triangulation_2}
described in \ccRefPage{CGAL::Triangulation_2<Traits,Tds>}
and in Chapter~\ref{I1_Chapter_Triangulations}, a model of 
\ccRefName derives from a template parameter
\ccc{Fb} to be instantiated with a model of 
\ccc{Triangulation_face_base_2}.

\ccTypes
The class \ccRefName\ defines the same types as 
the triangulation data structure 
except the iterators and the circulators.

\ccCreationVariable{f}  %% choose variable name
\begin{ccAdvanced}
\ccCreation

For user defined triangulation algorithms, faces need to
be explicitly constructed and linked to their neighbors.

\ccConstructor{Triangulation_data_structure_2::Face();}{default
constructor,
initializes all vertices and neighbors 
 with \ccc{NULL}.}

\ccConstructor{Triangulation_data_structure_2::Face(Vertex* v0,
                                           Vertex* v1,
                                           Vertex* v2);}
{introduces a face \ccVar, and initializes the vertices. The 
neighbors are initialized with \ccc{NULL}.}

\ccConstructor{Triangulation_data_structure_2::Face(Vertex* v0,
                    Vertex* v1,
                    Vertex* v2,
                    Face* n0,
                    Face* n1,
                    Face* n2);}
{introduces a face \ccVar, and initializes the vertices and the neighbors.}


\ccHeading{Setting}

\ccThree{Vertex*}{f.has_vertex( Vertex* v, int& i)x}{}
\ccMethod{void set_vertex(int i, Vertex* v);}
{sets vertex \ccc{i} to be \ccc{v}.
\ccPrecond $0\leq i \leq 2$.
}
\ccGlue
\ccMethod{void set_neighbor(int i, Face* n);}
{sets neighbor \ccc{i} to be \ccc{n}.
\ccPrecond $0\leq i \leq 2$.
}
\ccGlue
\ccMethod{ void set_vertices();}{sets the vertices pointers to \ccc{NULL}.}
\ccGlue
\ccMethod{void set_vertices(Vertex* v0,
		    Vertex* v1,
		    Vertex* v2);}{sets the vertices pointers.}
\ccGlue
\ccMethod{void set_neighbors();} {sets the neighbors pointers to \ccc{NULL}.}
\ccGlue
\ccMethod{void set_neighbors(Face* n0,
		     Face* n1,
		     Face* n2);}{sets the neighbors pointers.}

\end{ccAdvanced}


\ccHeading{Vertex Access Functions}

\ccMethod{Vertex* vertex(int i) const;}
{returns  the vertex \ccc{i} of \ccVar.
\ccPrecond $0\leq i \leq 2$.}
\ccGlue
\ccMethod{int index(Vertex* v) const;}
{returns the index of vertex \ccc{v} in \ccVar. \ccPrecond \ccc{v} is
a vertex of \ccVar}
\ccGlue
\ccMethod{bool has_vertex(Vertex* v) const;}
{returns \ccc{true} if  \ccc{v} is a vertex of \ccVar.}
\ccGlue
\ccMethod{bool has_vertex(Vertex* v, int& i) const;}
{returns \ccc{true} if  \ccc{v} is a vertex of \ccVar, and
 computes the index \ccc{i} of \ccc{v} in \ccc{f}.}


\ccHeading{Neighbor Access Functions}

The neighbor with index \ccc{i} is the neighbor which is opposite
to the vertex with index \ccc{i}.


\ccMethod{Face* neighbor(int i) const;}
{returns  the neighbor \ccc{i} of \ccVar. 
\ccPrecond $0\leq i \leq 2$.
}
\ccGlue
\ccMethod{int index(Face* n) const;}
{returns the index of face \ccc{n}.
\ccPrecond \ccc{n} is a neighbor of \ccVar.}
\ccGlue
\ccMethod{bool has_neighbor(Face* n) const;}
{returns \ccc{true} if \ccc{n} is a neighbor of \ccVar.}
\ccGlue
\ccMethod{bool has_neighbor(Face* n, int& i) const;}
{returns \ccc{true} if \ccc{n} is a neighbor of \ccVar,  and
compute the index \ccc{i} of \ccc{n}.}

\ccHeading{Other Access Functions}
\ccMethod{int  mirror_index(int i) const;}
{index of \ccVar\ as a neighbor of \ccVar.\ccc{neighbor(i)}}
\ccGlue
\ccMethod{Vertex* mirror_vertex(int i) const;}
{vertex of \ccVar.\ccc{neighbor(i)}} opposite to  \ccVar.


\begin{ccAdvanced}
\ccHeading{Checking}

\ccMethod{bool is_valid() const;}
{returns \ccc{true} if  the function 
\ccc{is_valid()} of the base class
returns \ccc{true} and if, for each index $i$, $0 \le i < 3$,
face $f$ is a neighbor of its neighboring face \ccc{neighbor(i)}
and shares with this neighbor the  vertices \ccc{cw(i)} and \ccc{ccw(i)}
in correct reverse order.}

\end{ccAdvanced}

\ccHeading{Miscellaneous} 

\ccMethod{int ccw(int i) const;}
{Returns $i+1$ modulo 3.\ccPrecond $0\leq i \leq 2$.}

\ccMethod{int cw(int i) const;}
{Returns $i+2$ modulo 3.\ccPrecond $0\leq i \leq 2$.}



\ccHasModels
\ccc{CGAL::Triangulation_ds_Face_2<Vb,Fb>}


\ccSeeAlso
\ccc{Triangulation_data_structure_2}
\ccc{Triangulation_data_structure_2::Vertex},
\ccc{Triangulation_2::Face_base}.


\end{ccRefConcept}

% +------------------------------------------------------------------------+
%%RefPage: end of main body, begin of footer
% EOF
% +------------------------------------------------------------------------+

