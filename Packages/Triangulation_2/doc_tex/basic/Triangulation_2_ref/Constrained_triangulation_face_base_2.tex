% +------------------------------------------------------------------------+
% | Reference manual page: Constrained_triangulation_face_base_2.tex
% +------------------------------------------------------------------------+
% | 12.04.2000   Author
% | Package: Package
% | 
\RCSdef{\RCSConstrainedtriangulationfacebaseRev}{$Revision$}
\RCSdefDate{\RCSConstrainedtriangulationfacebaseDate}{$Date$}
% |
%%RefPage: end of header, begin of main body
% +------------------------------------------------------------------------+


\begin{ccRefClass}{Constrained_triangulation_face_base_2<Traits,Fb>}  %% add template arg's if necessary

%% \ccHtmlCrossLink{}     %% add further rules for cross referencing links
%% \ccHtmlIndexC[class]{} %% add further index entries
\ccCreationVariable{ctfb} 

\ccDefinition
 The class \ccRefName\ is the default model for the concept
\ccc{ConstrainedTriangulationFaceBase_2} to be used as base face class
of constrained triangulation.

\ccInclude{CGAL/Constrained_triangulation_face_base_2.h}

\ccIsModel
\ccc{ConstrainedTriangulationFaceBase_2}

\ccParameters
The first template parameter is a geometric traits.

The second template parameter has to be a model
of the concept \ccc{TriangulationFaceBase_2}.
Its default is \ccc{CGAL::Triangulation_face_base_2<Traits>}

\ccInheritsFrom
The class \ccRefName\  derives from its parameter \ccc{Fb}.
and add three boolean to deal with information about
constrained edges.

The member functions \ccc{cw(int i)}, \ccc{ccw(int i)}
and \ccc{reorient} are overloaded  to update
information about constrained edges.

\ccSeeAlso
\ccc{TriangulationFaceBase_2} \\
\ccc{ConstrainedTriangulationFaceBase_2}\\
\ccc{CGAL::Constrained_triangulation_2<Traits,Tds>} \\
\ccc{CGAL::Triangulation_face_base_2<Traits>}


\end{ccRefClass}

% +------------------------------------------------------------------------+
%%RefPage: end of main body, begin of footer
% EOF
% +------------------------------------------------------------------------+

