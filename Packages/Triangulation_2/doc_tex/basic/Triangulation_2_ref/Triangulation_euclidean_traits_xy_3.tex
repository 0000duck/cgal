% +------------------------------------------------------------------------+
% | Reference manual page: Triangulation_euclidean_traits_xy_3.tex
% +------------------------------------------------------------------------+
% | 06.04.2000   Author
% | Package: Package
% | 
\RCSdef{\RCSTriangulationeuclideantraitsxyRev}{$Revision$}
\RCSdefDate{\RCSTriangulationeuclideantraitsxyDate}{$Date$}
% |
%%RefPage: end of header, begin of main body
% +------------------------------------------------------------------------+


\begin{ccRefClass}{Triangulation_euclidean_traits_xy_3<R>}  %% add template arg's if necessary

%% \ccHtmlCrossLink{}     %% add further rules for cross referencing links
%% \ccHtmlIndexC[class]{} %% add further index entries
\ccCreationVariable{traits}
\ccDefinition
  
The class \ccRefName\ is a geometric traits class allowing to
build a two dimensional triangulation of the projections
on the $xy$ plane  of three dimensional points.
This is the usual case when dealing with GIS terrains.
Instead of really projecting the 3D points and
maintaining a mapping between each point and its projection
 (which costs space and is error prone)
this class supplies geometric predicates that ignore the
\ccc{z}-coordinate of the points.

\ccInclude{Triangulation_euclidean_traits_xy_3.h}

\ccIsModel
\ccc{TriangulationTraits_2} \\
\ccc{DelaunayTriangulationTraits_2} 


\ccTypes
\ccTypedef{typedef Point_3<R>   Point_2;}{}
\ccGlue
\ccTypedef{typedef Segment_3<R>   Segment_2;}{}
\ccGlue
\ccTypedef{typedef Triangle_3<R>   Triangle_2;}{}

The following predicates and constructor types are provided

\ccNestedType{Construct_segment_2}
\ccGlue
\ccNestedType{Construct_triangle_2}
\ccGlue
\ccNestedType{Compare_x_2}{}
\ccGlue
\ccNestedType{Compare_y_2}{}
\ccGlue
\ccNestedType{Orientation_2}{}
\ccGlue
\ccNestedType{Side_of_oriented_circle_2}{}

\ccHeading{Access to predicate objects}
The following access functions are provided

\ccMethod{Construct_segment_2 construct_segment_2_object();}
\ccGlue
\ccMethod{Construct_triangle_2 construct_triangle_2_object();}
\ccGlue
\ccMethod{Comparison_x_2 compare_x_2_object();}{}                               \ccGlue
\ccMethod{Comparison_y_2 compare_y_2_object();}{}
\ccGlue
\ccMethod{Orientation_2  orientation_2_object();}{}
\ccGlue
\ccMethod{Side_of_oriented_circle_2
side_of_oriented_circle_2_object();}{}

\ccSeeAlso
\ccc{TriangulationTraits_2} \\
\ccc{DelaunayTriangulationTraits_2} \\
\ccc{CGAL::Triangulation_2<Traits,Tds>}\\
\ccc{CGAL::Delaunay_triangulation_2<Traits,Tds>} 


\ccExample
The following code builds a triangulated terrain
using a Delaunay triangulation with the traits class for 
terrains.
%\ccIncludeExampleCode{../../../examples/Triangulation_2/terrain.C} 
%\ccIncludeExampleCode{examples/Triangulation_2/terrain.C}
\ccIncludeExampleCode{Triangulation_2/terrain.C}
%\lcTex{
%\ccIncludeExampleCode{Triangulation_2/terrain.C}
%}
%\lcHtml{
%\ccIncludeExampleCode{../../../examples/Triangulation_2/terrain.C}
%}


\cgal\ provides also predefined geometric traits class
\ccc{Triangulation_euclidean_traits_yz_3<R>} and
\ccc{Triangulation_euclidean_traits_zx_3<R>} to
deal with projections on the
 \ccc{xz}- or  the \ccc{yz}-plane,
respectively.

\ccInclude{CGAL/Triangulation_euclidean_traits_xz_3.h}\\
\ccInclude{CGAL/Triangulation_euclidean_traits_yz_3.h}


\end{ccRefClass}

% +------------------------------------------------------------------------+
%%RefPage: end of main body, begin of footer
% EOF
% +------------------------------------------------------------------------+

