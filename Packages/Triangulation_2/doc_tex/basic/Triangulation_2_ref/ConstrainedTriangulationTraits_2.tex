% +------------------------------------------------------------------------+
% | Reference manual page: ConstrainedTriangulationTraits_2.tex
% +------------------------------------------------------------------------+
% | 05.02.2002   Mariette Yvinec
% | Package: Triangulation_2
% | 
\RCSdef{\RCSConstrainedTriangulationTraitsRev}{$Revision$}
\RCSdefDate{\RCSConstrainedTriangulationTraitsDate}{$Date$}
% |
%%RefPage: end of header, begin of main body
% +------------------------------------------------------------------------+


\begin{ccRefConcept}{ConstrainedTriangulationTraits_2}

%% \ccHtmlCrossLink{}     %% add further rules for cross referencing links
%% \ccHtmlIndexC[concept]{} %% add further index entries

\ccDefinition
  
The concept \ccRefName\ defines the requirements for the geometric
traits class of a constrained triangulation
( \ccc{Constrained_Triangulation_2<Traits,Tds,Itag>})
that supports intersections of input constraints (i. e.
when the template parameter \ccc{Itag} is instantiated
by one of the tag classes \ccc{Exact_intersections_tag} or
\ccc{Exact_predicates_tag}). This concept refines the concept
\ccc{TriangulationTraits_2}, adding requirements for function objects
to compute the intersection points of two constraints.
When \ccc{Exact_predicates_tag} is used, the 
traits class is
also required to provide  additionnal types
to compute the squared distance between a point and a line


\ccRefines
\ccc{TriangulationTraits_2} 


\ccTypes

\ccNestedType{Intersect_2}{A function object whose operator()
computes the intersection of two segments : \\
\ccc{Object_2 operator()(Segment_2 s1, Segment_2 s2);}
Returns the intersection of \ccc{s1} and \ccc{s2}.}

The geometric traits of a  constrained triangulation instantiated with
the intersection tag \ccc{Exact_predicates_tag}
is also required to provide the following types~:
\ccNestedType{Line_2}{The line type.}
\ccGlue
\ccNestedType{Construct_line_2} {A function object whose operator()
constructs a line from two points
\ccc{Line_2 operator()(Point_2 p1, Point_2 p2)}.}
\ccGlue
\ccNestedType{Compute_squared_distance_2}{A function object with an
operator() designed to compute the squared distance between
a line  and a point. 
\ccc{RT operator()(Point_2 p1, Line_2);} Return the squared distance
betwen \ccc{p} and \ccc{l}.}


\ccCreation
\ccCreationVariable{traits}  %% choose variable name

\ccConstructor{ConstrainedTriangulationTraits_2();}{default constructor.}
\ccGlue
\ccConstructor{ConstrainedTriangulationTraits_2(
                    const ConstrainedTriangulationTraits_2& traits2);}
{Copy constructor}

\ccMethod{ConstrainedTriangulationTraits_2 operator=
	(const ConstrainedTriangulationTraits_2& traits2);}
{Assignment operator.}


\ccHeading{Access to constructor object}
\ccMethod{Intersect_2  intersect_2_object();}{}
\ccMethod{Construct_line_2 construct_line_2_object();}{required when
the intersection tag is \ccc{Exact_predicates_tag}.}
\ccMethod{Compute_squared_distance_2
compute_squared_distance_2_object();}
{required when
the intersection tag is \ccc{Exact_predicates_tag}.}


\ccHasModels
The kernels of \cgal\ are models for this traits class.



\ccSeeAlso
\ccc{TriangulationTraits_2} \\
\ccc{ConstrainedTriangulationTraits_2}


%\ccExample

%A short example program.
%Instead of a short program fragment, a full running program can be
%included using the 
%\verb|\ccIncludeExampleCode{examples/Package/ConstrainedTriangulationPlusTraits_2_prog.C}| 
%macro. The program example would be part of the source code distribution and
%also part of the automatic test suite.

%\begin{ccExampleCode}
%void your_example_code() {
%}
%\end{ccExampleCode}

%% \ccIncludeExampleCode{examples/Package/ConstrainedTriangulationPlusTraits_2_prog.C}

\end{ccRefConcept}

% +------------------------------------------------------------------------+
%%RefPage: end of main body, begin of footer
% EOF
% +------------------------------------------------------------------------+

