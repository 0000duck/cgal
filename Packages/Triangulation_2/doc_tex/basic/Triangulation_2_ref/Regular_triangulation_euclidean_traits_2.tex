% +------------------------------------------------------------------------+
% | Reference manual page: Regular_triangulation_traits_2.tex
% +------------------------------------------------------------------------+
% | 12.04.2000   Author
% | Package: Package
% | 
\RCSdef{\RCSRegulartriangulationtraitsRev}{$Revision$}
\RCSdefDate{\RCSRegulartriangulationtraitsDate}{$Date$}
% |
%%RefPage: end of header, begin of main body
% +------------------------------------------------------------------------+


\begin{ccRefClass}{Regular_triangulation_euclidean_traits_2<Rep,Weight>}  %% add template arg's if necessary

%% \ccHtmlCrossLink{}     %% add further rules for cross referencing links
%% \ccHtmlIndexC[class]{} %% add further index entries
\ccCreationVariable{traits}
\ccDefinition
  
\cgal\ provides the predefined geometric traits
 \ccRefName\ 
This traits class is templated by a representation class \ccc{Rep}
and a weight type \ccc{Weight}.
This class inherits from
\ccc{Triangulation_euclidean_traits_2 <Rep >}
and uses a \ccc{Weighted_point} type
derived from the type \ccc{Point} of
\ccc{Triangulation_euclidean_traits_2 < Rep >}.

\ccInclude{CGAL/Regular_triangulation_euclidean_traits_2.h}

\ccIsModel
\ccc{RegularTriangulationTraits_2}

\ccTypes
\ccTypedef{typedef Triangulation_euclidean_traits_2<Rep>::Point_2
Point_2;}{}
\ccGlue
\ccTypedef{typedef Point_2  Bare_point;}{}
\ccGlue
\ccTypedef{typedef Weighted_point<Point_2,Weight> Weighted_point;}{}

\ccNestedType{Power_test_2}{}
\ccGlue
\ccNestedType{Power_test_degenerated_2}{}

The following types are required if member functions  relative to the
dual power diagram are called 
\ccNestedType{Line_2}{The line type.}
\ccGlue
\ccNestedType{Ray_2}{The type for ray.}
\ccGlue
\ccNestedType{Construct_weighted_circumcenter_2}{Constructor object. Provides
the operator 
\ccc{ Point_2 operator()(Weigthed_point_2 p, Weigthed_point_2 q, Weigthed_point_2 r)} which returns
the center of the circle orthogonal to the three weighted points  \ccc{p, q} anf \ccc{r}.}
\ccGlue
\ccNestedType{Construct_radical_axis_2}{Constructor object. Provides
the operator 
\ccc{ Line_2 operator()(Weigthed_point_2 p, Weigthed_point_2 q)} which 
construct the radical axis of the two weighted points  \ccc{p} and
\ccc{q}}
\ccGlue
\ccNestedType{Construct_ray_2}{A constructor object  to build
a ray from a point and a line. Must provide
\ccc{Ray_2 operator() ( Point_2 p, Line_2 d);}}

\ccHeading{Access to predicate objects}
\ccMethod{Power_test_2 power_test_2_object();}{}
\ccGlue
\ccMethod{Power_test_degenerated_2
  power_test_degenerated_2_object();}{}



\ccHeading{Access to constructor objects}
These member functions are required if functions  relative to the
dual power diagram are called 
\ccMethod{Construct_weighted_circumcenter_2
  construct_weighted_circumcenter_2_object();}{}
\ccGlue
\ccMethod{Construct_radical_axis_2
  construct_radical_axis_2_object();}
\ccGlue
\ccMethod{Construct_direction_2 construct_2_object();} {}
\ccGlue
\ccMethod{Construct_ray_2 construct_ray_2_object();} {}

\ccSeeAlso
\ccc{RegularTriangulationTraits_2} \\
\ccc{CGAL::Regular_triangulation_2}


\end{ccRefClass}

% +------------------------------------------------------------------------+
%%RefPage: end of main body, begin of footer
% EOF
% +------------------------------------------------------------------------+

