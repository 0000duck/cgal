\newcommand{\triangulationcomment}[1]{}

\chapter{Triangulations} \label{I1_Chapter_Triangulations}

\section{Introduction}

A simplicial complex is a set ${C}$ of simplices which satisfy
two conditions:
\begin{enumerate}
\item Any face of a simplex in ${  C}$ is also a simplex in ${  C}$
\item Two simplices in ${  C}$ either do not intersect
or their intersection is  a simplex of smaller dimension
which is their common face of maximal dimension.
\end{enumerate}
The dimension of a complex ${  C}$ is the maximal dimension 
of the simplices
of ${  C}$. A complex of dimension $d$ is said to be pure
if any simplex in ${  C}$ is a face of a $d$-dimensional
simplex of ${  C}$.
The domain of a complex ${  C}$ is the union of the simplices
in ${  C}$. A complex is said to be connected if its domain is connected.

A 
triangulation is a 2-dimensional simplicial complex which is pure
connected and without singularities. Thus a triangulation
can be viewed as a collection of triangular faces,
such that two faces either have an empty intersection or share an edge or a vertex.
The absence of singularities means that each edge belongs to 
at most two triangles and that each vertex belongs  to a set  of faces
whose union forms a topological disk.

Each face of a triangulation can be given an orientation
(clockwise or counterclockwise) which in turn induces an orientation
on the edges incident to that face. The orientation of two adjacent
faces are said to be consistent if they induce
opposite orientations on their common incident edge.
A triangulation is said to be orientable if 
the orientation of each face can be chosen in such a way
that all pairs of incident faces have consistent orientations. 


In this chapter we present a framework to represent
orientable triangulations which  may  be embedded 
in a plane or in a higher dimensional space.
Examples of such triangulations 
are triangulations of a simple polygons in the plane, the
Delaunay triangulations of points in the plane, or triangulated
irregular networks ({\sc Tin}s) which are used as terrain model in
{\sc Gis}.  
Any such  triangulation can be described as a set of  vertices and
triangular faces,  with incidence and adjacence relations:
two facets or two vertices are said to be {\em adjacent}
 (or neighboring) if they are incident to the same edge,
a facet and a vertex are said to be incident if they are incident to the same
edge.

 On top of this  abstract  view of a triangulation come different
geometric layers which define  the geometric information
associated to a vertex (e.g.,  two-dimensional points for triangulations
in the plane, or three-dimensional points for {\sc Tin}s) or to a face
and which define the functionality of the triangulation. 
For example, the insertion of a point into a Delaunay
triangulation or into  {\sc Tin} requires different update steps.


\subsection*{Design Rationale}
Because a triangulation is merely a set of
triangular faces with constant size complexity,
triangulations are not implemented
as a layer on top of a planar map.
\cgal\ uses a proper internal
representation of triangulations based on faces and vertices
rather than on edges. Such a representation
allows to save storage space and results in faster
algorithms.

Thus the basic elements of the representation are vertices and faces.
Each triangular face gives access to its three incident vertices 
and to its three adjacent faces. 
Each vertex gives access to one of its incident faces
and through that face to the circular list of its incident faces.
The edges are not explicitely represented, they are only represented 
through the adjacencies relations of two faces.


\begin{figure}
\begin{ccTexOnly}
\begin{center}
\input{three_levels.ltex}
\end{center}
\end{ccTexOnly}
\caption{The three layer design of triangulations.
\label{I1_Fig_three_levels}}
\begin{ccHtmlOnly}
<CENTER>
<img border=0 src=three_levels.gif align=center alt="Three_levels">
</CENTER>
\end{ccHtmlOnly}
\end{figure}



The triangulations in \cgal\ are represented
by a three layer structure  analog to the design used for polyhedral
surfaces, see Figure~\ref{I1_Fig_three_levels}.
In the bottom layer, the base classes for vertices and faces
store some 
geometric informations such as the coordinate of vertices 
and any other attribute (such as color, constraint edges etc.)
needed by the application.
The base classes handle
incidence and adjacency relations in term of \ccc{void*} pointers.
The use of \ccc{void*} pointers in the bottom layer allows 
to easily change one of the base
class, to deal with an extra attribute like a color for example,
without having to change the rest of the structure. 
The advantages of strong type 
checking is reestablished in the next layer 
where the \ccc{triangulation data structure} can be thought 
of as a container for faces and vertices which can take care
of all the combinatorial aspects of the triangulation.
The {triangulation data structure}
implements
adjacency and incidence relations with type safe pointer operations
and maintains the combinatorial integrity of the triangulation.
For that purpose, the {triangulation data structure} defines its own face and vertex
classes which are derived
 from the corresponding 
base classes
so that geometric and additional information on vertices and faces 
are simply inherited.
  At last, at the top layer, the \ccc{triangulation class}
implements the user interface with the triangulation. 
This classes offer to the user
the high level functionalities that can be expected from a triangulation:
insertion  or removal of a vertex, traversal of the faces,
enumeration of the vertices,
traversal of the   faces incident to a given vertex, location of a given point etc.
The {triangulation class} defines its own 
vertex and face classes
derived from the corresponding class of the {triangulation data structure}.
The vertices and faces of the {triangulation class}
are only accessed through high levels concepts such as 
handles, iterators, or circulators,
according to the required functionalities of the access.
The top layer {triangulation class}
is responsible for the geometric embedding of the triangulation
and comes in different flavors according to the kind of triangulation represented:
planar triangulation of a set of points, Delaunay or regular or constrained
 triangulations etc.

The triangulation classes of \cgal\ depends on two template parameters.
The first template parameter stands for
 a geometric traits class which is assumed to provide
the geometric objects (points, segments and triangles) 
forming  the triangulation and the geometric predicates on those objects.
The second template parameter stands for a model
of  triangulation data
structure acting as a container for faces and vertices
while  taking care of the combinatorial aspects of the triangulation. 
The triangulation data structure class is itself a template
class parametrized by the base classes for faces and vertices.


\subsection*{Organization of this chapter}
Section~\ref{I1_Sect_Triangulation_2} introduces the basic triangulation class of \cgal ,
 \ccc{Triangulation_2<Traits, Tds>}, its local \ccc{Vertex} and \ccc{Face}
classes and gives some examples for a simple use of this class.
 The \ccc{Triangulation_2<Traits, Tds>} class  is merely designed to represent 
a triangulation for a set of points in the plane.
The next section~\ref{I1_Sect_Geom_traits}
describes the requirements for the geometric
traits class and presents some default implementations
for this traits class offered in \cgal. 
Section~\ref{I1_Sect_Tds}  presents the requirements for the 
{triangulation data structure} class,
 its local \ccc{Vertex} and \ccc{Face}
classes and the default {triangulation data structure} class
provided by \cgal. 
Section~\ref{I1_Sect_Base_classes} describes the requirements
for the base classes \ccc{Vertex_base} and \ccc{Face_base}  
together with the defaults provided for these classes.
The remaining sections of this chapter introduce
several classes derived from the basic triangulation class
and designed to handle more specific triangulations.
Section~\ref{I1_Sect_Delaunay} present a class to 
maintain the Delaunay triangulation of a set of points in the plane.
 Section~\ref{I1_Sect_Regular}
describe a class to maintain regular triangulations.
While the Delaunay triangulation of a set of points
is the dual of the Voronoi
diagram of these points,
 regular triangulations are dual to weighted points power diagrams
and  appear as a generalization of
Delaunay triangulations. Section~\ref{I1_Sect_Constrained} 
describes a class to handle a constrained triangulation,
that is a triangulation in which certain edges are enforced.
Constrained triangulations allow in particular to deal with the 
triangulations of a planar polygons.

 

\section{Triangulation of points in the plane }
\label{I1_Sect_Triangulation_2}

The basic triangulation class 
 of \cgal\ is primarily designed to represent the  triangulations
 of a set of points ${  A}$ in the plane.
Such a triangulation has as its vertices the points of ${  A}$
and its domain covers the convex hull of ${  A}$.
It can be viewed as a planar partition of the plane
whoses bounded faces are triangular. and cover
the convex hull of ${  A}$. The single unbounded face of this partition
is the complementary of the convex hull of ${  A}$. 
In many applications, such as Kirkpatrick's hierarchy
or incremental Delaunay construction, it is convenient to
deal only with triangular faces. Therefore, we add to the
triangulation
a fictitious vertex, called the \ccc{infinite vertex}
and we make each  convex hull edge incident 
to an \ccc{infinite} 
face having as third vertex  the \ccc{infinite vertex}.
 In that way, each edge is incident to exactly two faces
and special cases at the
boundary of the convex hull are simpler to deal with.



\begin{figure}
\begin{ccTexOnly}
\begin{center} \IpeScale{50} \Ipe{infinite_vertex.ipe} \end{center}
\end{ccTexOnly}
\caption{The infinite vertex.
\label{I1_Fig_infinite_vertex}}
\begin{ccHtmlOnly}
<CENTER>
<img border=0 src=infinite_vertex.gif align=center alt="Vertices at
infinity">
</CENTER>
\end{ccHtmlOnly}
\end{figure}


The class \ccc{Triangulation_2<Traits,Tds>}
implements this point of view
and therefore considers  the triangulation of the set of points 
as a set of  triangular,  finite and
infinite faces. 
Although it is convenient to draw a triangulation as in
figure~\ref{I1_Fig_infinite_vertex}, note that
the \ccc{infinite vertex} has no significant
coordinates and that no geometric predicate can be applied on it
or on an infinite face.

A triangulation is a collection of vertices and faces that
are linked together through incidence and adjacency relations.
Each face give access to its three incident vertices and to
its 
three adjacent faces. Each vertex give access to one of its  incident
faces. 

The three vertices of a face are indexed with 0, 1 and 2
in counterclockwise order. The neighbor of a face are also 
indexed with 0,1,2 in such a way that the neighbor indexed by $i$
is opposite to the vertex with the same index.

Many of the classes in the triangulation package
offer  two functions \ccStyle{int cw(int i)} and 
\ccStyle{int ccw(int i)} 
which given the index of a vertex in a face
compute the index of the next vertex  of the same face
in clockwise
or counterclockwise order.
 Thus, for example the neighbor 
\ccc{neighbor(cw(i))} is
 the
neighbor of \ccc{f}  which is next to \ccc{neighbor(i)} turning clockwise
around \ccc{f}. The face \ccc{neighbor(cw(i))}
is also the first face encountered after \ccc{f} when
turning clockwise around vertex \ccc{i}
of~\ccc{f}.



 \begin{figure}
\begin{ccTexOnly}
    \begin{center}
     \input{neighbors.ltex}
    \end{center}
\end{ccTexOnly} 
    \caption{Vertices and neighbors.
    \label{I1_Fig_neighbors}}
  \begin{ccHtmlOnly}
<CENTER>
<img border=0 src=neighbors.gif align=center alt="Neighbors">
</CENTER>
\end{ccHtmlOnly} 
\end{figure}



\smallskip
A triangulation is valid from the combinatorial point of view if the
following is true.
\\
{\bf (a)} Two adjacent faces have neighbor pointers to each 
other and they have two vertices in common. 
The faces have a
coherent orientation, that is, they index their common  vertices 
in opposite order.
\\
{\bf (b)} All faces that are incident to a vertex \ccc{v} 
must be linked with neighbor pointers. Vertex \ccc{v} points to an 
arbitrary incident face. 

Furthermore, it is said to be geometrically valid iff
\\
{\bf (c)} Any face has its vertices indexed according to
counterclockwise order.

\smallskip
\begin{figure}
\begin{ccTexOnly}
\begin{center} \IpeScale{70}\Ipe{valid1.ipe} \end{center}
\end{ccTexOnly} 
\caption{Validity test.
\label{I1_Fig_valid}}

\begin{ccHtmlOnly}
<CENTER>
<img border=0 src=valid1.gif align=center alt="Validity">
</CENTER>
\end{ccHtmlOnly} 
\end{figure}


\begin{ccClassTemplate}{Triangulation_2<Traits,Tds>}
\subsection{The Triangulation Class \protect \ccClassTemplateName} 

The \ccClassTemplateName\  expects a model of  {geometric traits} class
as first template argument and a model of {triangulation data structure}
as second argument. The requirements  and defaults for these classes 
are described in the next sections \ref{I1_Sect_Geom_traits}
and \ref{I1_Sect_Tds}.

\ccInclude{CGAL/Triangulation_2.h}

\ccInheritsFrom{\ccc{Triangulation_cw_ccw_2}}
This class provides the function \ccc{cw(i)} et \ccc{ccw(i)}.

\ccTypes
\ccThree{typedef Traits::Triangle}{}{the triangulation data structure type.}
\ccThreeToTwo

\ccTypedef{typedef Traits Geom_traits;}{the geometric traits type.}
\ccGlue
\ccTypedef{typedef Tds Triangulation_data_structure;}{the triangulation data structure type.}

\ccTypedef{typedef Traits::Point Point;}{the point type}
\ccGlue
\ccTypedef{typedef Traits::Segment Segment;}{the segment type}
\ccGlue
\ccTypedef{typedef Traits::Triangle Triangle;}{the triangle type}

The following types gives the classes representing the vertices and faces
 of the triangulation and the type corresponding to the edges.
Recall that the edges are not explicitly represented and thus there is no corresponding class.
The functionalities of the vertex and face classes
 are described in sections~\ref{I1_Sect_Triangulation_Vertex}
and~\ref{I1_Sect_Triangulation_Face}.

\ccNestedType{Vertex}{the vertex type.}
\ccGlue
\ccNestedType{Face}{the face type.}
\ccGlue
\ccThree{typedef pair<Face_handle, int>}{}{the triangulation data structure type}
\ccTypedef{typedef pair<Face_handle, int> Edge;} {the edge type.
The \ccc{Edge(f,i)} is edge common to faces \ccc{f} and 
\ccc{f.neighbor(i)}. It is also the edge joining the vertices
\ccc{vertex(cw(i))} and \ccc{vertex(ccw(i))} of \ccc{f}.}

\ccThree{typedef Traits::Triangle}{}{the triangulation data structure type}
\ccThreeToTwo
The vertices and faces of the triangulations are accessed through 
\ccc{handles}\footnote{ A handle is a type which supports the two
dereference operators \ccc{operator*} and \ccc{operator->}.}, 
\ccc{iterators} and \ccc{circulators}. 
The iterators and circulators
are all bidirectional and non mutable.
The circulators and iterators are assignable to the 
corresponding handle types. Whenever a handle appear in the parameter 
list of a function, an appropriate iterator or circulator can be used as well.
The edges of the triangulation can also be visited through iterators
and circulators,
the edge circulators and iterators
are also bidirectional and non mutable.
In the following, we called {\it infinite} any face or edge 
incident  to the infinite vertex and the infinite vertex itself.
 Any other feature (face, edge or vertex) of the triangulation is said 
to be {\it finite}.
Some iterators (the \ccc{All} iterators ) allows to visit finite or 
infinite feature while others (the \ccc{Finite} iterators) visit only
finite features. Circulators visit infinite features as well as finite 
ones.

\ccNestedType{Vertex_handle}{handle to a vertex}
\ccGlue
\ccNestedType{Face_handle}{handle to a facet}


\ccNestedType{All_faces_iterator}{iterator over all faces.}
\ccGlue
\ccNestedType{All_edges_iterator
}{iterator over all edges.}
\ccGlue
\ccNestedType{All_vertices_iterator}{iterator over all vertices.}

\ccNestedType{Finite_faces_iterator}{iterator over finite faces.}
\ccGlue
\ccNestedType{Finite_edges_iterator
}{iterator over finite edges.}
\ccGlue
\ccNestedType{Finite_vertices_iterator}{iterator over finite vertices.}


\ccNestedType{Line_face_circulator}{circulator over faces intersected by a line.}


\ccNestedType{Face_circulator} 
{circulator over faces incident to a given vertex.}
\ccGlue
\ccNestedType{Edge_circulator}
{circulator over edges incident to a given vertex.}
\ccGlue
\ccNestedType{Vertex_circulator}
{circulator over vertices incident to a given vertex.}

The triangulation class also defines the following enum type to specify
which case occurs when locating a point in the triangulation.

\ccEnum{enum Locate_type {VERTEX=0, EDGE, FACE, OUTSIDE_CONVEX_HULL,
OUTSIDE_AFFINE_HULL};}{The locate type is \ccc{OUTSIDE_CONVEX_HULL} when the point
is  outside the convex hull but in the affine hull of the current triangulation. \\
The locate type is \ccc{OUTSIDE_AFFINE_HULL} 
when the point is outside the affine hull
of the current triangulation.}

\ccCreation
\ccCreationVariable{t}
\ccThree{Triangulation_2<Traits,Tds>}{t = tr}{}
\ccThreeToTwo

\ccConstructor{Triangulation_2<Traits,Tds>(
                   const Geom_traits& gt = Geom_traits() );}
{Introduces an empty triangulation \ccVar.}


\ccConstructor{Triangulation_2<Traits,Tds>(
                   const Triangulation_2<Traits,Tds>& tr);}
{Copy constructor. All the vertices and faces are duplicated.
 After the copy, \ccVar\ and \ccc{tr}
refer to different triangulations~: 
 if \ccc{tr} is modified, \ccVar\ is not. }

\ccMethod{Triangulation_2<Traits,Tds> operator=(const Triangulation_2<Traits,Tds>& tr);}
{Assignement. All the vertices and faces are duplicated.
 After the assignement, \ccVar\ and \ccc{tr}
refer to different triangulations~: 
 if \ccc{tr} is modified, \ccVar\ is not.}

\ccThree{Vertex_handle}{t.number_of_vertices()x}{}
\ccMethod{void swap(Triangulation_2<Traits,Tds>& tr);}
{The triangulations \ccc{tr} and \ccVar\ are swapped.
\ccVar. \ccc{swap(tr)} should be preferred to \ccc{t} = \ccc{tr} or to
\ccc{t(tr)} if \ccc{tr} is deleted after that.}

\ccMethod{void clear();}{Deletes all faces and finite vertices
resulting
 in an
empty triangulation.}

\ccFunction{void ~Triangulation_2<Traits,Tds> ();}
{Destructor. All vertices and faces are deleted.}


\ccAccessFunctions
\ccMethod{const Geom_traits& geom_traits() const;}
{Returns a const reference to the triangulation traits object.}
\ccGlue
\ccMethod{int dimension() const;}
{Returns the dimension of the convex hull.}
\ccGlue
\ccMethod{int number_of_vertices() const;}
{Returns the number of finite vertices.}
\ccGlue
\ccMethod{int number_of_faces() const;}
{Returns the number of finite faces.}

\ccMethod{Face_handle infinite_face() const;}
{a  face incident to the \ccc{infinite_vertex}.}
\ccGlue
\ccMethod{Vertex_handle
          infinite_vertex();}
{the \ccc{infinite_vertex}.}
\ccGlue
\ccMethod{Vertex_handle finite_vertex() const;}
{a vertex distinct from  the \ccc{infinite_vertex}.}

\ccHeading{Test triangulation features}

The class \ccClassTemplateName\ provides method to test
the finite or infinite caracter of any feature,
and also methods to test the presence in the triangulation
of a particular feature (edge or face).

\ccThree{bool }{t.is_infinite( Face_handle f, int i)x}{}
\ccMethod{bool
          is_infinite(Vertex_handle v) const;}
{\ccc{true}, iff \ccc{v} is the \ccc{infinite_vertex}.}
\ccGlue
\ccMethod{bool
          is_infinite(Face_handle f) const;}
{\ccc{true}, iff face \ccc{f} is infinite.}
\ccGlue
\ccMethod{bool is_infinite(Face_handle f, int i) const;}
{\ccc{true}, iff edge \ccc{(f,i)} is infinite.}
\ccGlue
\ccMethod{bool
          is_infinite(Edge e) const;}
{\ccc{true} iff edge \ccc{e} is infinite.}
\ccGlue
\ccMethod{bool
          is_infinite(Edge_circulator ec) const;}
{\ccc{true} iff edge \ccc{*ec} is infinite.}
\ccGlue
\ccMethod{bool
          is_infinite(Edge_iterator ei) const;}
{\ccc{true} iff edge \ccc{*ei} is infinite.}


\ccMethod{bool is_edge(Vertex_handle va, Vertex_handle vb);}
{\ccc{true} if there is an edge having \ccc{va} and \ccc{vb} as
vertices.}
\ccGlue
\ccMethod{bool is_edge(Vertex_handle va, Vertex_handle vb, Face_handle& fr,
	       int & i);}
{ as above. In addition, if \ccc{true} is returned,  the edge with
vertices \ccc{va} and \ccc{vb} is the edge \ccc{e=(fr,i)} where
\ccc{fr} is a handle to the face incident to \ccc{e} and 
on the right side of  \ccc{e} oriented from \ccc{va} to \ccc{vb}.}
\ccGlue
\ccMethod{bool includes_edge(Vertex_handle va, Vertex_handle & vb,
		     Face_handle& fr, int & i);}
{\ccc{true} if the line segment from \ccc{va} to \ccc{vb} includes
an edge \ccc{e} incident to \ccc{va}. If \ccc{true}, \ccc{vb} becomes
the other vertex of \ccc{e}, \ccc{e} is the edge \ccc{(fr,i)} where
\ccc{fr} is a handle to the face incident to \ccc{e} and 
on the right side \ccc{e} oriented from \ccc{va} to \ccc{vb}.}
\ccGlue
\ccMethod{bool is_face(Vertex_handle v1, Vertex_handle v2, Vertex_handle v3);}
{\ccc{true} if there is a face having \ccc{v1}, \ccc{v2} and a\ccc{v3} 
as vertices. This function is NOT YET implemented.}
\ccGlue
\ccMethod{bool is_face(Vertex_handle v1, Vertex_handle v2, Vertex_handle v3,
      Face_handle &fr);}
{as above. In addition, if \ccc{true} is returned, fr is a handle
to the face with  \ccc{v1}, \ccc{v2} and \ccc{v3} 
as vertices. This function is NOT YET implemented.} 


\ccHeading{Queries}

The class \ccClassTemplateName\  provides methods to locate
a given point with respect to a triangulation. The location of a point 
is described  by the locate type \ccc{lt}, a face handle \ccc{f} and
an index \ccc{li}. The locate type is : \\
\ccc{VERTEX} if the point coincides with a vertex \ccc{v} of the triangulation
 In this case the face handle \ccc{f} points to a face incident to \ccc{v}
and the index \ccc{li} is the index of \ccc{v}  in \ccc{f}.\\
\ccc{EDGE} if the point lies on an edge \ccc{e}. In this case, the face
handle \ccc{f}
points to a face incident to \ccc{e} and \ccc{li}  is the
index of \ccc{e} in \ccc{f}. \\
\ccc{FACE} if the point lies in the relative interior of a bounded
face \ccc{f}. In this case the face handle point to \ccc{f} and the
index \ccc{li}
has no meaning. \\
\ccc{OUSIDE_CONVEX_HULL} if the point lies within the affine hull of
the triangulation but outside its convex hull. In this case, 
the face handle \ccc{f} points to any infinite face which is a {\em
witness } of the
query point location. For a two dimensional triangulation, this means
that the supporting line of the finite edge of \ccc{f}  separates
the query point from the triangulation. For a one-dimensional
triangulation
it means that the 
finite vertex of \ccc{f}  separates
the query point from the triangulation. \\
\ccc{OUTSIDE_AFFINE_HULL} if the point lies outside the affine hull
(in the case of a degenerate lower dimensional triangulation). In this 
case, the face
handle \ccc{f} and  the index \ccc{li} have no meaning.



\ccThree{Oriented_side }{t.is_infinite( Face_handle f, int i)x}{}
\ccMethod{Face_handle
          locate(const Point& query) const;}{}


\ccMethod{Face_handle
          locate(const Point& query,
                 Locate_type& lt,
                 int& li) const;}
{Parameters \ccc{lt} and \ccc{li} are set respectively
to the locate type and the index describing the location.}
  

The triangulation class also provides
methods to locate a point with respect to
a given  finite face of the triangulation or with respect to the
circumcircle of such a feace.
The face or its circimcircle are assumed to be counterclockwisely
oriented.

\ccMethod{Oriented_side
           oriented_side(Face_handle f,
                         const Point& p) const;}
{}

\ccMethod{Oriented_side
 side_of_oriented_circle(Face_handle f, const Point & p);}
{}

\ccHeading{Insertion, Removal and other Modifiers}

The following operations are guaranteed to lead to a valid triangulation 
when they are applied on a valid triangulation.

\ccThree{Vertex_handle }{t.flip( Face_handle f, int i)}{}
\ccMethod{void flip(Face_handle f, int i);}{Exchanges the edge incident to
\ccc{f} and \ccc{f->neighbor(i)} with the other
diagonal of the quadrilateral formed by \ccc{f} and  \ccc{f->neighbor(i)}.
\ccPrecond {The faces \ccc{f} and \ccc{f->neighbor(i)} are finite faces
and their union form a convex quadrilateral.}}



\ccMethod{Vertex_handle insert(const Point& p);}{returns a handle to
the created \ccc{Vertex}.}
                         
\ccMethod{Vertex_handle push_back(const Point& p);}
{Equivalent to \ccc{insert(p)}.}

\ccMethod{template < class InputIterator >
          int
          insert(InputIterator first, InputIterator last);}
{Inserts the points in the range
 $\left[\right.$\ccc{first}, \ccc{last}$\left.\right)$.
 Returns the number of inserted points.
 \ccPrecond The \ccc{value_type} of \ccc{first} and \ccc{last}
 is \ccc{Point}.}

\ccMethod{void    remove(Vertex_handle v);}
{Removes the vertex from the triangulation.}


\begin{figure}
\begin{ccTexOnly}
\begin{center}
\input{insert1.ltex}
\end{center}
\end{ccTexOnly}
\caption{Insertion of a point on an edge.
\label{I1_Fig_inser1t}}

\begin{ccHtmlOnly}
<CENTER>
<img border=0 src=insert1.gif align=center alt="Insertion in an edge">
</CENTER>
\end{ccHtmlOnly}
\end{figure}




\begin{figure}
\begin{ccTexOnly}
\begin{center}
\input{insert2.ltex}
\end{center}
\end{ccTexOnly}
\caption{Insertion in a face.
\label{I1_Fig_insert2}}

\begin{ccHtmlOnly}
<CENTER>
<img border=0 src=insert2.gif align=center alt="Insertion in a Face">
</CENTER>
\end{ccHtmlOnly}
\end{figure}


\begin{figure}
\begin{ccTexOnly}
\begin{center}
\input{insert3.ltex}
\end{center}
\end{ccTexOnly}
\caption{Insertion outside the convex hull.
\label{I1_Fig_insert3}}

\begin{ccHtmlOnly}
<CENTER>
<img border=0 src=insert3.gif align=center alt="Insertion outside the
convex hull">
</CENTER>
\end{ccHtmlOnly}
\end{figure}

\begin{figure}
\begin{ccTexOnly}
\begin{center}
\input{remove.ltex}
\end{center}
\end{ccTexOnly}
\caption{Removal
\label{I1_Fig_remove}}

\begin{ccHtmlOnly}
<CENTER>
<img border=0 src=remove.gif align=center alt="Remove">
</CENTER>
\end{ccHtmlOnly}
\end{figure}


\ccHeading{Traversal of the Triangulation}


A triangulation can be seen as a container of faces and vertices.
Therefore the triangulation provides several iterators and circulators
that allow to traverse it (completely or partially).



\ccHeading{Face, Edge and Vertex Iterators}

The following iterators allow respectively to visit 
finite faces,  finite edges and  finite vertices
of the triangulation. These iterators are non mutable, bidirectional
and their value types are respectively
\ccc{Face}, \ccc{Edge} and \ccc{Vertex}. 
They are all invalidated by any change in the triangulation.

\ccThree{Finite_vertices_iterator}{t.finite_vertices_begin()x}{}
\ccMethod{Finite_vertices_iterator finite_vertices_begin() const;}{Starts at an arbitrary finite vertex}
\ccGlue
\ccMethod{Finite_vertices_iterator finite_vertices_end() const;}{Past-the-end iterator}

\ccMethod{Finite_edges_iterator finite_edges_begin() const;}{Starts at an arbitrary finite edge}
\ccGlue
\ccMethod{Finite_edges_iterator finite_edges_end() const;}{Past-the-end iterator}

\ccMethod{Finite_faces_iterator finite_faces_begin() const;}{Starts at an arbitrary finite face}
\ccGlue
\ccMethod{Finite_faces_iterator finite_faces_end() const;}{Past-the-end iterator}


The following iterators allow respectively to visit all
(finite or infinite) faces, edges and vertices
of the triangulation. These iterators are non mutable, bidirectional
and their value types are respectively
\ccc{Face}, \ccc{Edge} and \ccc{Vertex}. 
They are all invalidated by any change in the triangulation.


\ccMethod{All_vertices_iterator all_vertices_begin() const;}{Starts at an arbitrary  vertex}
\ccGlue
\ccMethod{All_vertices_iterator all_vertices_end() const;}{Past-the-end iterator}

\ccMethod{All_edges_iterator all_edges_begin() const;}{Starts at an arbitrary edge}
\ccGlue
\ccMethod{All_edges_iterator all_edges_end() const;}{Past-the-end iterator}

\ccMethod{All_faces_iterator all_faces_begin() const;}{Starts at an arbitrary face}
\ccGlue
\ccMethod{All_faces_iterator all_faces_end() const;}{Past-the-end iterator}

\ccThree{Line_face_circulator}{T.line_walk(Point p, }{}
\ccHeading{Line Face Circulator}

The triangulation defines a circulator that allows
to visit all faces that are intersected by a line. 
This circulator is
non-mutable and bidirectional. Its value type is \ccc{Face}.

\ccMethod{Line_face_circulator
          line_walk(const Point& p,
                    const Point& q,
                    Face_handle f = Face_handle()) const;}
{ returns a circulator to visit  all the 
 faces intersected by the line \ccc{pq}.} 

\begin{figure}
\begin{ccTexOnly}
\begin{center}  \Ipe{walk.ipe} \end{center}
\end{ccTexOnly} 
\caption{The line face circulator.
\label{I1_fig_Line_face_circulator}}

\begin{ccHtmlOnly}
<CENTER>
<img border=0 src=walk.gif align=center alt="The Infinite Vertex">
</CENTER>
\end{ccHtmlOnly} 
\end{figure}

A line face circulator is invalidated if the face the circulator refers
to is changed.

\ccThree{Vertex_circulator}{t.number_of_vertices()x}{}
\ccThreeToTwo



\ccHeading{Face, Edge and Vertex Circulators}

The triangulation also provides circulators that allows to visit 
respectively all faces or edges incident to a given vertex
or all vertices adjacent to a given vertex.
These circulator are
non-mutable
and bidirectional.
 The operator \ccc{operator++} moves the circulator
counterclockwise around the vertex while
the \ccc{operator--} moves clockwise.
A face circulator is invalidated by any modification of the face pointed to.
An edge or a vertex circulator are invalidated by any modification
of one of the two faces incident to the edge pointed to.

\ccMethod{Face_circulator incident_faces(Vertex_handle v) const;}{}
\ccGlue
\ccMethod{Edge_circulator incident_edges(Vertex_handle v) const;}{}
\ccGlue
\ccMethod{Vertex_circulator incident_vertices(Vertex_handle v) const;}{}


\ccHeading{Traversal of the Convex Hull}

Applied on the \ccc{infinite_vertex}
the above  functions  allow to visit the vertices on the convex hull and
the infinite edges and faces. Note that a counterclockwise
traversal of the vertices adjacent to the \ccc{infinite_vertex} is
a clockwise traversal of the convex hull.

\ccMethod{Face_circulator incident_faces(t.infinite_vertex()) const;}{}
\ccGlue
\ccMethod{Edge_circulator incident_edges(t.infinite_vertex()) const;}{}
\ccGlue
\ccMethod{Vertex_circulator incident_vertices(t.infinite_vertex() v) ;} {}


\ccHeading{Miscellaneous}

\ccThree{Segment}{t.segment(Face_handle f, int i) }{}
\ccMethod{int ccw(int i) const;}
{Returns $i+1$ modulo 3.\ccPrecond $0\leq i \leq 2$.}
\ccGlue
\ccMethod{int cw(int i) const;}
{Returns $i+2$ modulo 3.\ccPrecond $0\leq i \leq 2$.}
\ccGlue
%\ccMethod{int number_of_faces() const;}
%{Returns the number of finite and infinite faces.
%\ccc{TBC_TO} Returns the number of finite faces. This access number functions
%%requires to count the degree 
%of the \ccc{infinite_vertex} and thus is not a constant time access function.}
%\ccGlue
\ccMethod{Triangle
          triangle(Face_handle f) const;}
{Returns the triangle formed by the three vertices of \ccc{f}.
 \ccPrecond The face is finite.}
\ccGlue
\ccMethod{Segment
          segment(Face_handle f, int i) const;}
{Returns the line segment formed by the vertices \ccc{ccw(i)}
 and \ccc{cw(i)} of face \ccc{f}.
\ccPrecond $0\leq i \leq 2$. The vertices \ccc{ccw(i)}
 and \ccc{cw(i)} of  \ccc{f}
 are finite.}
\ccGlue
\ccMethod{Segment
          segment(const Edge& e) const;}
{Returns the line segment corresponding to edge \ccc{e}.
\ccPrecond \ccc{e} is a finite edge}
\ccGlue
\ccMethod{Segment
          segment(const Edge_circulator& ec) const;}
{Returns the line segment corresponding to edge \ccc{*ec}.
\ccPrecond \ccc{*ec} is a finite edge.}
\ccGlue
\ccMethod{Segment
          segment(const Edge_iterator& ei) const;}
{Returns the line segment corresponding to edge \ccc{*ei}.
\ccPrecond \ccc{*ei} is a finite edge.}
\ccGlue
\ccMethod{Point circumcenter(Face_handle  f) const;}
{Compute the circumcenter of the face pointed to by f. This function
is available only if the correspoding function is provided in the
geometric traits.}

%\begin{ccAdvanced}

%\ccHeading{Setting}
%\ccMethod{void set_infinite_vertex(const Vertex_handle&  v);}{}
%\ccGlue
%\ccMethod{void set_dimension(int n);}{}
%\ccGlue
%\ccMethod{void set_number_of_vertices(int n);}{}

%\ccHeading{Checking}
%The responsibility of keeping a valid triangulation
%belongs to the users if advanced operations are used.
%Obviously the advanced user, who implements higher levels operations
%may have to make a triangulation invalid at some times. The following
%method is provided to help the debugging.

%\ccMethod{bool
%          is_valid(bool verbose = false, int level = 0) const;}
%{Checks the combinatorial validity of the triangulation and
%also the validity of its geometric embedding.
% This method is  mainly a debugging help
%for the users of advanced features.
%}
%\end{ccAdvanced}



\ccHeading{I/O}


The I/O operators are defined for \ccc{iostream}, and for
the window stream provided by \cgal. The format for the iostream
is an internal format. 

%\ccInclude{CGAL/IO/ostream_2.h}

\ccThree{ostream&x}{ostream& os << T }{}
\ccFunction{ostream& operator<<(ostream& os,
                  const Triangulation_2<Traits,Tds>& T);}
{Inserts the triangulation \ccVar\ into the stream \ccc{os}.
\ccPrecond The insert operator must be defined for \ccc{Point}.}

\ccFunction{istream& operator>>(istream& is,
                  const Triangulation_2<Traits,Tds>& T);}
{Reads a triangulation from stream \ccc{is} and assigns it
to \ccVar. \ccPrecond The extract operator must be defined for \ccc{Point}.}

\ccInclude{CGAL/IO/Window_stream.h}

\ccFunction{Window_stream& operator<<(Window_stream& W,
                         const Triangulation_2<Traits,Tds>& T);}
{Inserts the triangulation \ccVar\ into the window stream \ccc{W}.
The insert operator must be defined for \ccc{Point}
and \ccc{Segment}.}

\ccExample

The following code fragment creates a  triangulation of 2D points
for the  usual Euclidean metric. The points are read from {\tt cin},
inserted in the triangulation 
and finally points on the convex hull are written to {\tt cout}. 
%The
%{\tt while} loop can be replaced by the code in the comment,
%as the triangulation class has an \ccc{insert} member
%function that takes an \stl\ range of points as argument.


\begin{cprog}
/* triangulation_prog1.C */
/* --------------------- */
#include <CGAL/basic.h>
#include <iostream>
#include <CGAL/Cartesian.h>
#include <CGAL/Point_2.h>
#include <CGAL/Triangulation_2.h>
#include <CGAL/Triangulation_euclidean_traits_2.h>

using namespace CGAL;

typedef Cartesian<double> Rp;
typedef Triangulation_euclidean_traits_2<Rp> Traits;
typedef Triangulation_vertex_base_2<Traits> Vb;
typedef Triangulation_face_base_2<Traits>  Fb;
typedef Triangulation_default_data_structure_2<Traits,Vb,Fb> Tds;
typedef Triangulation_2<Traits, Tds> Triangulation;
typedef Triangulation::Vertex_circulator Vertex_circulator;
typedef Traits::Point   Point;

int main() {
  Triangulation t;
    
  Point p;
  while (std::cin >> p){
    t.insert(p);
  }
  
  Vertex_circulator vc = t.incident_vertices(t.infinite_vertex()),
    done(vc);
  if (vc != 0) {
    do{
      std::cout << vc->point() << std::endl;
    }while(++vc != done);
  }
}
\end{cprog} 

\ccHeading{Implementation}

Locate is implemented by a line walk from a vertex of the face given
as optional parameter (or from a finite vertex of
\ccStyle{infinite_face()} if no optional parameter is given). It takes
time \ccTexHtml{$O(n)$}{O(n)} in the worst case, but only \ccTexHtml{$O(\sqrt{n})$}{O(sqrt(n))}
on average if the vertices are distributed uniformly at random.

Insertion of a point is done by locating a face that contains the
point, and then splitting this face.
If the point falls outside the convex hull, the triangulation
 is restored by flips.  Apart from the location, insertion takes a time 
time \ccTexHtml{$O(1)$}{O(1)}. This bound is only an amortized bound
for points located outside the convex hull.

Removal of a vertex is done by removing all adjacent triangles, and
retriangulating the hole. Removal takes time \ccTexHtml{$O(d^2)$}{O(d^2)} in the worst
case, if \ccTexHtml{$d$}{d} is the degree of the removed vertex,
which is \ccTexHtml{$O(1)$}{O(1)} for a random vertex.

The face, edge, and vertex iterators on finite features
are derived from their counterparts visiting all (finite and infinite)
features which are themselves derived from the corresponding iterators
of the triangulation data structure.

\end{ccClassTemplate}


\begin{ccClass}{Vertex}
\subsection{The Vertex Class of a Triangulation}
\label{I1_Sect_Triangulation_Vertex}

\ccThree{Vertex_circulator}{t.number_of_vertices()x}{}
\ccThreeToTwo


\ccCreationVariable{v}

\ccDefinition

The vertex stores a point, gives access to an incident face
and provides circulators to visit all incident faces and edges
and all adjacent vertices.

\ccInheritsFrom{Tds::Vertex}

\ccTypes
The class  {Vertex} defines the same types 
as the class \ccc{Triangulation_2<Traits, Tds>} except for the iterators and the 
\ccc{Line_face_circulator} which are not needed.

%\begin{ccAdvanced}
%\ccCreation
%For triangulation algorithms designed by the user, vertices need to be
%explicitly constructed.  If the modification of the triangulation is
%done through a member of the triangulation class, there is no need for these
%functions.

%\ccConstructor{Vertex()}{Introduces a new vertex. The geometric information  is initialized by the default constructor of the class \ccc{Point}. The pointer to the incident face is initialized to \ccc{NULL}.}

%\ccConstructor{Vertex(const Point& p);}
%{Introduces a variable \ccVar, and initializes the geometric information.
%The pointer to the incident face is initialized with \ccc{NULL}.}

%\ccConstructor{Vertex(const Point& p,
%                      Face_handle f);}
%{Introduces a variable \ccVar, and initializes the geometric information and 
%the pointer to the incident face.}
%\end{ccAdvanced}



\ccHeading{Access Functions}

\ccMethod{Point point() const;}
{returns  the geometric information of \ccVar.}

\ccMethod{Face_handle face() const;}
{returns a face of the triangulation having  \ccVar\ as vertex.}


%\begin{ccAdvanced}
%\ccHeading{Setting}
%\ccMethod{void set_face(Face_handle f); }{sets the incident face to \ccc{f}.}
%\end{ccAdvanced}



\ccHeading{Miscellaneous}

%\ccMethod{int ccw(int i) const;}
%{Returns $i+1$ modulo 3.\ccPrecond $0\leq i \leq 2$.}
%\ccGlue
%\ccMethod{int cw(int i) const;}
%{Returns $i+2$ modulo 3.\ccPrecond $0\leq i \leq 2$.}
%\ccGlue
\ccMethod{int degree() const;}
{Returns the degree of \ccVar\ in the triangulation.}
\ccGlue
\ccMethod{Vertex_handle handle() const;}
{Returns an handle to the vertex.}
\ccGlue
\ccMethod{bool is_valid();} 
{Check the validity of a vertex:
i.e. the pointer to the incident face  and call the \ccc{is_valid()}
function of the base class to perform any geometric test provided by the user
therein.}


\ccHeading{Traversal of the Adjacent Vertices, Incident Edges and Faces.}

Three circulator classes allow to traverse the edges, and faces
incident to a given vertex or  the adjacent vertices.
 These circulators are bidirectional
and their value types are respectively \ccc{Vertex}, \ccc{Edge}
and \ccc{Face}.
The  \ccc{operator++} moves the circulator
counterclockwise around the vertex 
and  the \ccc{operator--} moves the circulator
clockwise. 

%A face circulator is invalidated by any modification of the face it
%points to. An edge circulator is invalidated
%by any modification of one of the two faces that are incident to the edge
%pointed to.  A vertex circulator that turns around vertex \ccc{v}
%and and points  to a vertex \ccc{w}, is invalidated
%by any modification of the faces incident to the edge \ccc{vw}.




\ccMethod{Face_circulator
          incident_faces(Face_handle f=Face_handle()) const;}{}
%{A circulator for the  incident faces 
%that refers to face \ccc{f} or to an arbitrary face
%incident to \ccc{v}  if \ccc{f} is omitted.
%\ccPrecond Face \ccc{f} is incident to vertex \ccc{v}.}
\ccGlue
\ccMethod{Edge_circulator
          incident_edges(Face_handle f=Face_handle()) const;}{}
%{A circulator for the incident edges that refers to an abitrary edge incident
%to \ccc{v} if \ccc{f} is omitted. Otherwise the circulator refers
%to the first edge of \ccc{f}
%incident
%to \ccc{v} in counterclockwise order around \ccc{v}.
%\ccPrecond Face \ccc{f} is incident to vertex \ccc{v}. }
\ccGlue
\ccMethod{Vertex_circulator
          incident_vertices(Face_handle f=Face_handle()) const;}{}



\end{ccClass} 

\begin{ccClass} {Face}
\subsection{The Face  Class of a triangulation}
\label{I1_Sect_Triangulation_Face}
\ccCreationVariable{f}
\def\ccTagRmEigenClassName{\ccFalse}

\ccDefinition
A face of a triangulation gives access to its three
vertices indexed 0, 1, and 2 in counterclockwise order
and to its three adjacent faces, also called neighbors.
The neighbors are indexed in such a way that neighbor $i$ lies
opposite to vertex i. 

\ccInheritsFrom{Tds::Face}

\ccTypes
The class  {Face} defines the same types 
as the class \ccc{Triangulation_2<Traits,Tds>} except the iterators and circulators
which are not needed.


%\begin{ccAdvanced}
%\ccHeading{Creation}

%For user defined triangulation algorithms, faces need to
%be explicitly constructed and linked to their neighbors.

%\ccConstructor{Face();}
%{Introduces a variable \ccVar\ and initializes all vertices and neighbors 
% with \ccc{NULL}.}

%\ccConstructor{Face(Vertex_handle v0,
%                                           Vertex_handle v1,
%                                           Vertex_handle v2);}
%{Introduces a variable \ccVar, and initializes the vertices. The 
%neighbors are initialized with \ccc{NULL}.}

%\ccConstructor{Face(Vertex_handle v0,
%                    Vertex_handle v1,
%                    Vertex_handle v2,
%                     {Face_handle} n0,
%                     {Face_handle} n1,
%                     {Face_handle} n2);}
%{Introduces a variable \ccVar, and initializes the vertices and the neighbors.}



%\ccHeading{Setting}


%\ccThree{Vertex_handle}{f.has_vertex( Vertex_handle v, int& i)x}{}
%\ccMethod{void set_vertex(int i, Vertex_handle v);}
%{Sets vertex \ccc{i} to be \ccc{v}.
%\ccPrecond $0\leq i \leq 2$.
%}
%\ccGlue
%\ccMethod{void set_neighbor(int i,  {Face_handle} n);}
%{Sets neighbor \ccc{i} to be \ccc{n}.
%\ccPrecond $0\leq i \leq 2$.
%}
%\ccGlue
%\ccMethod{void set_vertices();}{Sets vertices to \ccc{NULL}}
%\ccGlue
%\ccMethod{void set_neighbors();}{Sets neighbors to \ccc{NULL}}
%\ccGlue
%\ccMethod{void set_vertices(
%Vertex_handle v0, Vertex_handle v1, Vertex_handle v2);}{}
%\ccGlue
%\ccMethod{void set_neighbors(
% {Face_handle} n0,  {Face_handle} n1,  {Face_handle} n2);}{}
%\end{ccAdvanced} 


\ccHeading{Vertex Access Functions}

\ccMethod{Vertex_handle vertex(int i) const;}
{Returns  the vertex \ccc{i} of \ccVar.
\ccPrecond $0\leq i \leq 2$.}
\ccGlue
\ccMethod{int index(Vertex_handle v) const;}
{Returns the index of vertex \ccc{v} in \ccVar. \ccPrecond \ccc{v} is
a vertex of \ccVar}
\ccGlue
\ccMethod{bool has_vertex(Vertex_handle v) const;}
{Returns \ccc{true} if  \ccc{v} is a vertex of \ccVar.}
\ccGlue
\ccMethod{bool has_vertex(Vertex_handle v, int& i) const;}
{Returns \ccc{true} if  \ccc{v} is a vertex of \ccVar, and
 computes the index \ccc{i} of the vertex.}



\ccHeading{Neighbor Access Functions}

The neighbor with index \ccc{i} is the neighbor which is opposite
to the vertex with index \ccc{i}.


\ccMethod{{Face_handle} neighbor(int i) const;}
{Returns  the neighbor \ccc{i} of \ccVar.
\ccPrecond $0\leq i \leq 2$.
}
\ccGlue
\ccMethod{int index( {Face_handle} n) const;}
{Returns the index of face \ccc{n}.
\ccPrecond \ccc{n} is a neighbor of \ccVar.}
\ccGlue
\ccMethod{bool has_neighbor({Face_handle} n) const;}
{Returns \ccc{true} if \ccc{n} is a neighbor of \ccVar.}
\ccGlue
\ccMethod{bool has_neighbor({Face_handle} n, int& i) const;}
{Returns \ccc{true} if \ccc{n} is a neighbor of \ccVar,  and
compute the index \ccc{i} of the neighbor.}

\ccHeading{Other Access Functions}
\ccMethod{int mirror_index(int i) const;}
{Returns the index of the face \ccc{f} as a neighbor
of its neighbor \ccc{f.neighbor(i)}, or equivalently the index
of the vertex opposite to \ccc{f} in \ccc{f.neighbor(i)}.}
\ccGlue
\ccMethod{Vertex_handle mirror_vertex(int i) const;}
{Returns the vertex of \ccc{f.neighbor(i)}
which is opposite to \ccc{f}.}



\ccHeading{Miscellaneous} 
 
\ccMethod{int ccw(int i) const;}
{Returns $i+1$ modulo 3.\ccPrecond $0\leq i \leq 2$.}
\ccGlue
\ccMethod{int cw(int i) const;}
{Returns $i+2$ modulo 3.\ccPrecond $0\leq i \leq 2$.}
\ccGlue
\ccMethod{{Face_handle} handle() const;}
{Returns a handle to the face.}
\ccGlue
\ccMethod{bool is_valid();} 
{Check the validity of an individual face:
i.e. the pointers to the incident vertices
and neighboring faces
 and call the \ccc{is_valid()}
function of the base class to perform any geometric test provided by the user
therein.}

\end{ccClass}


\section{ The Geometric Traits Class}
\label{I1_Sect_Geom_traits}

The first template parameter of the triangulation classes of \cgal\ 
is the geometric traits class
which provides the geometric objects (points, segments and triangles) building up the triangulation
together with the geometric predicates on those objects.
The first subsection of this section describes the requirements
that the geometric traits class of the basic triangulation class
\ccc{Triangulation_2<Traits, Tds>} has to fulfill. The second subsection
presents some predefined geometric traits classes available in \cgal .


\subsection{Requirements for the Geometric Traits Class}
\begin{ccClass}{Geom_traits}

\ccCreationVariable{gt}

The requirements for the geometric traits class of the basic triangulation class
\ccc{Triangulation_2<Traits, Tds>} are  kept rather light.
Essentially, a geometric traits class \ccClassName\ has to provide some  type
for points, segments and triangles,  
fonction objects  for 
coordinate comparison and  orientation tests, and functions
to access to those predicate objects.


\ccTypes
\ccThree{Comparison_result}{gt.compare_x()}{}
\ccThreeToTwo

\ccNestedType{Point_2}{The  point type must provide
a copy constructor and assignment operator.}
\ccGlue
\ccNestedType{Segment_2}{The segment  type must provide a constructor that takes
two points as argument.}
\ccGlue
\ccNestedType{Triangle_2}{The triangle type must provide a constructor
that takes three points as argument.}

\ccNestedType{Compare_x_2}{A function object whose operator() takes
two points as arguments and returns the 
\ccc{CGAL_comparison_result} correponding to the
ordering of their x-coordinates.}
\ccGlue
\ccNestedType{Compare_y_2}{A function object whose operator() takes
two points as arguments and 
returns  the \ccc{CGAL_comparison_result} correponding to the
ordering of their y-coordinates.}
\ccGlue
\ccNestedType{Orientation_2}{ A function object whose operator() takes
three points as arguments and returns the \ccc{CGAL_orientation}
  corresponding to their order type.}


\ccCreation
Only a default constructor, copy constructor and 
assignement operator are required. 
Note that further constructors
can be provided. 

\ccConstructor{Geom_traits();} {A default constructor.}
\ccGlue
\ccConstructor{Geom_traits( const Geom_traits&);}{Copy constructor}
\ccMethod{Geom_traits operator=(Geom_traits gtr);}{Assignment operator.}

\ccHeading{Predicates functions}
The following functions give access to the  predicate objects.
\ccThree{Comparison_result}{gt.compare_x(Point p0, Point p1)x}{}
\ccMethod{Comparison_x_2 compare_x_2_object();}{}                               \ccGlue
\ccMethod{Comparison_y_2 compare_y_2_object();}{}
\ccGlue
\ccMethod{Orientation_2  orientation_2_object();}{}

\end{ccClass}



\subsection{Predefined Geometric Traits Classes}

The \cgal\  library provides some predefined geometric traits classes
for the triangulations using the kernel geometric objects and predicates.
These classes are themselves templated with a representation class.



\begin{ccClassTemplate} {Triangulation_euclidean_traits_2<R>}
\subsubsection{A Geometric Traits Class for 2D Points}
The traits class \ccc{Triangulation_euclidean_traits_2<R>}
is designed to deal with ordinary  two dimensional points.

\ccInclude{CGAL/Triangulation_euclidean_traits_2.h}

\ccTypes
\ccThree{typedef Triangle_2<R>}{Triangle_2}{}

\ccTypedef{typedef Point_2<R>  Point_2;}{}
\ccGlue
\ccTypedef{typedef Segment_2<R>   Segment_2;}{}
\ccGlue
\ccTypedef{typedef Triangle_2<R>   Triangle_2;}{}
\ccGlue
\ccTypedef{typedef  R::Compare_x_2 Compare_x_2;}{}
\ccGlue
\ccTypedef{typedef  R::Compare_y_2 Compare_y_2;}{}
\ccGlue
\ccTypedef{typedef  R::Orientation_2 Orientation_2;}{}
\end{ccClassTemplate}


\begin{ccClassTemplate} {Triangulation_euclidean_traits_xy_3<R>}
\subsubsection{Geometric Traits Classes for Projections}
The classes \ccc{Triangulation_euclidean_traits_xy_3<R>} 
is a geometric traits class allowing to
build a two dimensional triangulation of the projections
on the $xy$ plane  of three dimensional points.
This is the usual case when dealing with GIS terrains.
Instead of really projecting the 3D points and
maintaining a mapping between each point and its projection
 (which costs space and is error prone)
this class supplies geometric predicates that ignore the
\ccc{z}-coordinate of the points.

\ccInclude{CGAL/Triangulation_euclidean_traits_xy_3.h}

\ccTypes

\ccTypedef{typedef Point_3<R>   Point_2;}{}
\ccGlue
\ccTypedef{typedef Segment_3<R>   Segment_2;}{}
\ccGlue
\ccTypedef{typedef Triangle_3<R>   Triangle_2;}{}
\end{ccClassTemplate}

\cgal\ provides also predefined geometric traits class
\ccc{Triangulation_euclidean_traits_yz_3<R>} and
\ccc{Triangulation_euclidean_traits_zx_3<R>} to
deal with projections on the
 \ccc{xz}- or  the \ccc{yz}-plane,
respectively.

\ccInclude{CGAL/Triangulation_euclidean_traits_xz_3.h}\\
\ccInclude{CGAL/Triangulation_euclidean_traits_yz_3.h}

\section{The Triangulation Data Structure}
\label{I1_Sect_Tds}

The second template parameter of the basic triangulation class
\ccc{Triangulation_2<Traits,Tds>} is a triangulation data structure class.
This class can be seen has a container for the 
faces and vertices maintaining incidence and adjacency relations.
The triangulation data structure class 
 is responsible for  the combinatorial integrity of the triangulation
(i. e. proper incidence and adjacency relations among vertices
and faces) while 
allowing  to perform combinatorial modifications
such has 
insertion of a new vertex in a face, or in an edge,
suppression of a vertex of degree three,  flip of two edges.
The term combinatorial means that those operation are purely topological
and do not depend on the geometric embedding.



The triangulation data structure is itself a template class
parametrized by the base classes \ccc{Vertex_base} and \ccc{Face_base},
and derives from thoses base classes its own 
vertex and face classes. This design  allows to restore at the 
triangulation data structure
level 
the strong type checking which does not exists at the base classes levels.

The following subsections list the requirements for a triangulation data structure
and for its nested vertex and face classes.
The default triangulation data structure provided by \cgal\ is describe next.


\subsection{Requirements for a Triangulation Data Structure Class}



\begin{ccClassTemplate}{Tds<Vb,Fb>}
\ccCreationVariable{tds}

To be a model of  triangulation data structure 
the class \ccClassTemplateName\
is required to  provide the
following types and operations.

\ccThree{typedef   Tds<Vb,Fb>}{Tdsxxxx}{}
\ccTypedef{typedef   Tds<Vb,Fb> Tds;}{self}
\ccGlue
\ccThree{Tds<Vb,Fb>}{tds.swap(Tds tds1)x}{}
\ccThreeToTwo
\ccTypedef{typedef  Vb Vertex_base;}{the \ccc{Vertex_base}}
\ccGlue
\ccTypedef{typedef  Fb Face_base;}{the \ccc{Face_base}}
\ccGlue
\ccNestedType{Vertex} { requirement for this type are described in  section~\ref{I1_Sect_Tds_Vertex}.}
\ccGlue
\ccNestedType{Face}{requirements for this type are described in
described~\ref{I1_Sect_Tds_Face}.}

The following iterators allow to visit all the vertices, edges and faces
of the triangulation data structure. They are all bidirectional and non mutable.

\ccNestedType{Face_iterator}{}
\ccGlue
\ccNestedType{Edge_iterator}{}
\ccGlue
\ccNestedType{Vertex_iterator}{}

The following circulators allow to visit all the vertices, edges and faces
incident to a given vertex. They are all bidirectional and non mutable.

\ccNestedType{Face_circulator}{}
\ccGlue
\ccNestedType{Edge_circulator}{}
\ccGlue
\ccNestedType{Vertex_circulator}{}
  
\ccCreation
\ccConstructor{Tds()}
{A default constructor.}

\ccConstructor{Tds( const Tds& tds1)}
{Copy constructor. All the vertices and faces are duplicated.}

\ccMethod{Tds<Vb,Fb> operator=(const Tds<Vb,Fb>& tds1);}
{Assignation. All the vertices and faces are duplicated.}

\ccMethod{void swap(const Tds& tds1);}
{Swaps \ccVar\ and \ccc{tds1}. Should be preferred to \ccVar=\ccc{tds1} or \ccVar(\ccc{tds1})
when tds1 is deleted after that.}
\ccMethod{void clear();}{Deletes all faces and all finite vertices.}

\ccFunction{void ~Tds<Vb,Fb> ();}
{Destructor. All vertices and faces are deleted.}

\ccAccessFunctions
\ccThree{void}{tds.set_number_of_vertices(int n)}{}
\ccMethod{int dimension() const;}
{The dimension of the triangulation.}
\ccGlue
\ccMethod{int number_of_vertices() const;}
{The number of vertices in the data structure.}
\ccGlue
\ccMethod{int number_of_faces() const ;}
{The number of two dimensional faces in the data structure.}
\ccGlue
\ccMethod{int number_of_edges() const;}
{The number of edges  in the triangulation data structure.}
\ccGlue
\ccMethod{int number_of_full_dim_faces() const;}
{The number of full dimensional faces, 
i.e. faces of dimension equal to the dimension
of the triangulation. This is the actual
number of faces stored in the triangulation data structure.}

\ccHeading{Setting}
\ccMethod{void set_number_of_vertices(int n);}{}
\ccGlue
\ccMethod{void set_dimension (int n);}{}


\ccHeading{Modifiers}
The following modifier member functions  guarantee
the combinatorial validity of the resulting triangulation.

\ccThree{Vertex*}{tds.insert_in_edge(Fce* f, int i)x}{}
\ccMethod{void flip(Face* f, int i);}{exchanges the edge incident to
\ccc{f} and \ccc{f->neighbor(i)} with the other
diagonal of the quadrilateral formed by \ccc{f} and  \ccc{f->neighbor(i)}.}


\begin{figure}
\begin{ccTexOnly}
\begin{center} %\IpeScale{70} \Ipe{Flip.ipe} \end{center}
\input{flip.ltex}
\end{center}
\end{ccTexOnly} 
\caption{Flip.
\label{I1_fig_flip_bis}}

\begin{ccHtmlOnly}
<CENTER>
<img border=0 src=Flip.gif align=center alt="Flip">
</CENTER>
\end{ccHtmlOnly} 
\end{figure}


\ccMethod{Vertex* insert_first();} {creates the first 
vertex.}
\ccGlue
\ccMethod{Vertex* insert_second();} {creates the second 
vertex.}
\ccMethod{Vertex* insert_in_edge(Face* f, int i);} {adds a
vertex \ccc{v} splitting 
edge \ccc{i} of face \ccc{f}.}
\ccGlue
\ccMethod{Vertex* insert_in_face(Face* f);} {adds a vertex
\ccc{v} splitting  face
\ccc{f} in three }
\ccGlue
\ccMethod{Vertex* insert_dim_up(Vertex* w, bool orient=true);} {adds
a vertex \ccc{v}, increasing by one the dimension of the triangulation.
}

\begin{figure}
\begin{ccTexOnly}
%\begin{center} \IpeScale{70} \Ipe{Three.ipe} \end{center}
\begin{center} \documentclass{article}
\documentclass{article}
\documentclass{article}
\input{tmp.inputs}
\pagestyle{empty}
\begin{document}
\input{tmp.pstex_t}
\end{document}

\pagestyle{empty}
\begin{document}
\documentclass{article}
\input{tmp.inputs}
\pagestyle{empty}
\begin{document}
\input{tmp.pstex_t}
\end{document}

\end{document}

\pagestyle{empty}
\begin{document}
  \begin{tabular}{ccc}
    \input{insert1.pstex_t} &
    \input{insert2.pstex_t} &
    \input{insert3.pstex_t}\\
  {\small (a)} & {\small (b)} & {\small (c)}\\
  \end{tabular}
\end{document}
 \end{center}
\caption{Insertion}
\end{ccTexOnly} 

\begin{ccHtmlOnly}
<CENTER>
<img border=0 src=Three.gif align=center alt="Insertion">
</CENTER>
\end{ccHtmlOnly} 
\end{figure}


\ccThree {void }{remove_degree_3(Vertex* v); }{}
\ccMethod{void remove_degree_3(Vertex* v, Face *f=NULL);}
{removes a vertex of degree 3.}

\ccMethod{void remove_second(Vertex* v);}{removes the before last
vertex.}
\ccGlue
\ccMethod{void remove_first(Vertex* v);}{removes the last vertex.}
\ccGlue
\ccMethod{void remove_dim_down(Vertex* v);}
{removes vertex \ccc{v} incident to all other vertices
and  decreases by one the dimension of the triangulation.
}

%\begin{ccAdvanced}
%The following modifiers are required for convenience of the advanced
%user.
%They do not guarantee the combinatorial validity 
%of the resulting triangulation.
%\ccMethod{Face* create_face(Face* f1, int i1, Face* f2, int i2, Face*
%f3, int i3);}{adds a face which is the neighbor \ccc{i1} of \ccc{f1}, 
%\ccc{i2} of ccc{f2} and \ccc{i3} of \ccc{f3}.}
%\ccGlue
%\ccMethod{Face* create_face(Face* f1, int i1, Face* f2, int i2);}
%{adds a face which is the neighbor \ccc{i1} of \ccc{f1}, 
%and the neighbor \ccc{i2} of ccc{f2}.}
%\ccGlue
%\ccMethod{Face* create_face(Face* f1, int i1, Vertex* v);}
%{adds a face which is the neighbor \ccc{i1} of \ccc{f1},
%and has \ccc{v} as vertex.}
%\ccGlue
%\ccMethod{ Face* create_face(Vertex* v1, Vertex* v2, Vertex* v3);}
%{adds a face with vertices \ccc{v1}, \ccc{v2} and \ccc{v3}.}
%\ccGlue
%\ccMethod{Face* create_face(Vertex* v1, Vertex* v2, Vertex* v3,
%		    Face* f1, Face* f2, Face* f3);}
%{adds a face with vertices \ccc{v1}, \ccc{v2} and \ccc{v3},
%and neighbors \ccc{f1}, \ccc{f2}, \ccc{f3}.}
%\ccGlue
%\ccMethod{Face* create_face();}
%{adds a face whose vertices and neighbors are set to NULL.}
% \ccGlue
%\ccMethod{void  delete_face(Face*);}{deletes a face.}
%\end{ccAdvanced}


\ccHeading{Traversing the triangulation}
\ccThree{Vertex_iterator}{tds.number_of_faces()x}{}
\ccMethod{Face_iterator faces_begin() const;}{visits all faces}
\ccGlue
\ccMethod{Face_iterator faces_end() const;}{}
\ccGlue
\ccMethod{Vertex_iterator vertices_begin() const;}{visits all vertices}
\ccGlue
\ccMethod{Vertex_iterator vertices_end() const;}{}
\ccGlue
\ccMethod{Edge_iterator edges_begin() const;}{visits all edges}
\ccGlue
\ccMethod{ Edge_iterator edges_end() const;}{}



\ccHeading{Miscelleanous}

\ccMethod{int ccw(int i) const;}{returns $i+1$ modulo 3.\ccPrecond $0\leq i \leq 2$.}
\ccGlue
\ccMethod{int cw(int i) const;}
{returns $i+2$ modulo 3.\ccPrecond $0\leq i \leq 2$.}
\ccGlue
\ccMethod{bool is_valid();}{checks the combinatorial validity of the
triangulation: call the \ccc{is_valid()} member function for each vertex and 
each face, checks the number of vertices and the Euler relation
between numbers of vertices, faces and edges.}
\end{ccClassTemplate}

\subsection{Requirements for the Vertex Class of a Triangulation Data Strucure}
\label{I1_Sect_Tds_Vertex}
\begin{ccClass}{Tds::Vertex}
\ccCreationVariable{v}

\ccThree{Vertex_circulator}{tds.infinite_vertex()cc}{}
\ccThreeToTwo

The vertex stores a point and a pointer to an incident face.

\ccInheritsFrom \ccc{Tds::Vertex_base}

\ccTypes
The class \ccClassName\
defines the same types as the triangulation data structure class \ccc{Tds}
except the iterators.
%\begin{ccAdvanced}

%\ccCreation

%\ccConstructor{Vertex();}
%{introduces a vertex \ccVar. The geometric information is
% initialized by the default constructor of class \ccc{Point}.
%The pointer to the incident face is initialized with \ccc{NULL}.}

%\ccConstructor{Vertex(const Point& p);}
%{introduces a vertex \ccVar, and initializes the geometric information.
%The pointer to the incident face is initialized with \ccc{NULL}.}

%\ccConstructor{Vertex(const Point& p,
%                      Face* f);}
%{introduces a vertex \ccVar, and initializes the geometric information and 
%the pointer to the incident face.}


%\ccHeading{Setting}

%\ccMethod{void set_point(const Point& p);}
%{sets the geometric information to \ccc{p}.}
%\ccGlue
%\ccMethod{void set_face(Face* f);}
%{sets the incident face to \ccc{f}.}


%\end{ccAdvanced}


\ccHeading{Access Functions}

\ccMethod{Point point() const;}
{returns  the geometric information of \ccVar.}
\ccGlue
\ccMethod{Face* face() const;}
{returns a face of the triangulation having  \ccVar\ as vertex.}

\ccHeading{Traversal of the Adjacent Vertices, Incident Edges and Faces.}
Three circulator classes allow to traverse the edges, and faces
incident to a given vertex and the adjacent vertices.
Note that infinite as well as finite incident edges and faces are visited.
 These circulators are bidirectional
and their value types are respectively \ccc{Vertex}, \ccc{Edge}
and \ccc{Face}.
The  \ccc{operator++} moves the circulator
counterclockwise around the vertex 
and  the \ccc{operator--} moves the circulator
clockwise. 

%A face circulator is invalidated by any modification of the face it
%points to. An edge circulator is invalidated
%by any modification of anyone of the two faces incident to the edge
%pointed to.  A vertex circulator that turns around vertex \ccc{v}
%and that has as value a pointer to vertex \ccc{w}, is invalidated
%by any modification of anyone of the two faces incident to \ccc{v}
%and \ccc{w}.

\ccMethod{Vertex_circulator
          incident_vertices(Face* f=NULL) const;}{} 
\ccGlue
\ccMethod{Edge_circulator
          incident_edges(Face* f=NULL) const;}{}
\ccGlue
\ccMethod{Face_circulator
          incident_faces(Face* f=NULL) const;}{}

\ccHeading{Miscellaneous}

\ccMethod{int ccw(int i) const;}
{Returns $i+1$ modulo 3.\ccPrecond $0\leq i \leq 2$.}
\ccGlue
\ccMethod{int cw(int i) const;}
{Returns $i+2$ modulo 3.\ccPrecond $0\leq i \leq 2$.}
\ccGlue
\ccMethod{int degree() const;}
{Returns the degree of \ccVar\ in the triangulation.}

\end{ccClass}



\subsection{Requirements for the Face  Class of a Triangulation Data Structure}
\label{I1_Sect_Tds_Face}
\begin{ccClass}{Tds::Face}
\ccCreationVariable{f}

\ccThree{Vertex*}{t.number_of_vertices(xxxxx)x}{}
\ccThreeToTwo

The face class \ccClassName\ of a triangulation data structure
stores three pointers to its three vertices
and three pointers to its three neighbors.
The vertices are indexed 0,1, and 2 in counterclockwise order.
The  neighbor indexed $i$ lies
opposite to vertex i. 

\ccInheritsFrom \ccc{Tds::Face_base}

\ccTypes
The class \ccClassName\ defines the same types as 
the triangulation data structure 
except the iterators and circulators


%\begin{ccAdvanced}
%\ccHeading{Creation}
%For user defined triangulation algorithms, faces need to
%be explicitly constructed and linked to their neighbors.

%\ccConstructor{Face();}
%{introduces a face \ccVar\ and initializes all vertices and neighbors 
% with \ccc{NULL}.}

%\ccConstructor{Face(Vertex* v0,
%                                           Vertex* v1,
%                                           Vertex* v2);}
%{introduces a face \ccVar, and initializes the vertices. The 
%neighbors are initialized with \ccc{NULL}.}

%\ccConstructor{Face(Vertex* v0,
%                    Vertex* v1,
%                    Vertex* v2,
%                    Face* n0,
%                    Face* n1,
%                    Face* n2);}
%{introduces a face \ccVar, and initializes the vertices and the neighbors.}


%\ccHeading{Setting}

%\ccThree{Vertex*}{f.has_vertex( Vertex* v, int& i)x}{}
%\ccMethod{void set_vertex(int i, Vertex* v);}
%{sets vertex \ccc{i} to be \ccc{v}.
%\ccPrecond $0\leq i \leq 2$.
%}
%\ccGlue
%\ccMethod{void set_neighbor(int i, Face* n);}
%{sets neighbor \ccc{i} to be \ccc{n}.
%\ccPrecond $0\leq i \leq 2$.
%}
%\ccGlue
%\ccMethod{ void set_vertices();}{sets the vertices pointers to \ccc{NULL}.}
%\ccGlue
%\ccMethod{void set_vertices(Vertex* v0,
%		    Vertex* v1,
%		    Vertex* v2);}{sets the vertices pointers.}
%\ccGlue
%\ccMethod{void set_neighbors();} {sets the neighbors pointers to \ccc{NULL}.}
%\ccGlue
%\ccMethod{void set_neighbors(Face* n0,
%		     Face* n1,
%		     Face* n2);}{sets the neighbors pointers.}

%\end{ccAdvanced}

\ccHeading{Vertex Access Functions}

\ccMethod{Vertex* vertex(int i) const;}
{returns  the vertex \ccc{i} of \ccVar.
\ccPrecond $0\leq i \leq 2$.}
\ccGlue
\ccMethod{int index(Vertex* v) const;}
{returns the index of vertex \ccc{v} in \ccVar. \ccPrecond \ccc{v} is
a vertex of \ccVar}
\ccGlue
\ccMethod{bool has_vertex(Vertex* v) const;}
{returns \ccc{true} if  \ccc{v} is a vertex of \ccVar.}
\ccGlue
\ccMethod{bool has_vertex(Vertex* v, int& i) const;}
{returns \ccc{true} if  \ccc{v} is a vertex of \ccVar, and
 computes the index \ccc{i} of \ccc{v} in \ccc{f}.}



\ccHeading{Neighbor Access Functions}

The neighbor with index \ccc{i} is the neighbor which is opposite
to the vertex with index \ccc{i}.


\ccMethod{Face* neighbor(int i) const;}
{returns  the neighbor \ccc{i} of \ccVar. 
\ccPrecond $0\leq i \leq 2$.
}
\ccGlue
\ccMethod{int index(Face* n) const;}
{returns the index of face \ccc{n}.
\ccPrecond \ccc{n} is a neighbor of \ccVar.}
\ccGlue
\ccMethod{bool has_neighbor(Face* n) const;}
{returns \ccc{true} if \ccc{n} is a neighbor of \ccVar.}
\ccGlue
\ccMethod{bool has_neighbor(Face* n, int& i) const;}
{returns \ccc{true} if \ccc{n} is a neighbor of \ccVar,  and
compute the index \ccc{i} of \ccc{n}.}

\ccHeading{Other Access Functions}
\ccMethod{int  mirror_index(int i) const;}
{index of \ccVar\ as a neighbor of \ccVar.\ccc{neighbor(i)}}
\ccGlue
\ccMethod{Vertex* mirror_vertex(int i) const;}
{vertex of \ccVar.\ccc{neighbor(i)} opposite to  \ccVar.}


%\begin{ccAdvanced}
%\ccHeading{Checking}

%\end{ccAdvanced}


\ccHeading{Miscellaneous} 

\ccMethod{bool is_valid() const;}
{returns \ccc{true} if  the function 
\ccc{is_valid()} of the base class
returns \ccc{true} and if, for each index $i$, $0 \le i < 3$,
face $f$ is a neighbor of its neighboring face \ccc{neighbor(i)}
and shares with this neighbor the  vertices \ccc{cw(i)} and \ccc{ccw(i)}
in correct reverse order.}

\ccMethod{int ccw(int i) const;}
{Returns $i+1$ modulo 3.}
\ccGlue
\ccMethod{int cw(int i) const;}
{Returns $i+2$ modulo 3.}

\end{ccClass}

\subsection{Models of Triangulation Data Structure}
\cgal offers currently two models of triangulation data structures.
The first one \ccc{Triangulation_default_data_structure_2<Tds_gt,Vb,Fb
>}
is highly economic with respect to memory space but its use is
restricted
to planar embedded triangulation. The second one
\ccc{Triangulation_data_structure_using_list_2<Vb,Fb>} can be used for 
the representation of any  orientable triangulations.

\subsubsection{The Default Triangulation Data Structure}
\begin{ccClassTemplate}{Triangulation_default_data_structure_2<Tds_gt,Vb,Fb>}
\cgal\ proposes the class \ccc{Triangulation_default_data_structure_2<Tds_gt,Vb,Fb >}
as a default for the triangulation data structure class of a triangulation.
As required this class has two template parameters \ccc{Vb} anf \ccc{Fb}
which have 
to be models for respectively the
\ccc{Vertex_base} and \ccc{Face_base} concepts whose requirements are described in next 
section~\ref{I1_Sect_Base_classes}. 

In addition, the class \ccClassTemplateName\ has a first template parameter
which is a geometric traits class. This may be surprising because
the triangulation data structure is supposed to deal only with the combinatorial
aspect of the triangulation and not with any geometric embedding
The reason for that is the following.
The class \ccClassTemplateName\ does not use any additional data structure
such as a list or a vector to act as a container for faces and vertices.
The iterators which allows to visit all faces and vertices of the
triangulation
data structure
is implemented using  an implicit tree structure over the faces
as described by
 de Berg, van Oostrum, and Overmars, 
in Proc.\ 12th Annual Symp.\ on Comput.\ Geom.,
1996, pages C5--C6. This tree structure is  based on the planar
geometric embedding
the triangulation. Each face 
 can find its parent 
and its children using only simple comparisons on the
coordinates of the points embedding its vertices.
Thus the tree structure may remain implicit 
and does not require any additional memory. 

The requirements concerning the geometric traits \ccc{Tds_gt} of
\ccc{Triangulation_default_data_structure_2}
 are very light and form a subset of the requirements needed
for the geometric traits of the triangulations. 
This class is required  to provide a type \ccc{Point}
and the coordinate comparison functions \ccc{compare_x(Point p0, Point p1)} and
\ccc{compare_y(Point p0, Point p1)}
described in section~\ref{I1_Sect_Geom_traits}. The point type
defined by the geometric traits class of the triangulation data structure
has to be the same 
as the point type defined by the geometric traits of the triangulation.
This is achieved if the same model is used for both traits classes
which is recommended but not compulsory.


\ccInclude{CGAL/Triangulation_default_data_structure_2.h}
\end{ccClassTemplate}

\subsubsection{A triangulation data structure using list}
The class \ccc{Triangulation_data_structure_using_list_2<Vb,Fb>}
can be used as a triangulation data structure for any
orientable triangulation. It uses aa \stl list to store the
full dimensional faces of the triangulation.
\ccInclude{CGAL/Triangulation_data_structure_using_list_2.h}


\section{The Base Classes \protect \ccc{Vertex_base} and  \protect \ccc{Face_base} }
\label{I1_Sect_Base_classes}

\subsection{Requirements for the \protect \ccc{Vertex_base} Class}
\begin{ccClass} {Vertex_base}
\ccCreationVariable{v}

\ccThree{Point}{f.has_neighbor(const void* v. int& i)xx}{}
\ccThreeToTwo


At the bottom layer,  
a vertex stores a point and a  provides access to one of its incident
face through a
\ccc{void *} pointer.

\ccTypes
\ccNestedType{Point}{must be the same as the point type \ccc{Traits::Point}
defined by the geometric traits class of the triangulation.}

\ccCreation
\ccConstructor{Vertex_base();}{default constructor.}
\ccGlue
\ccConstructor{Vertex_base(Point p);}{constructs a vertex embedded in point \ccc{p}.}
\ccGlue
\ccConstructor{Vertex_base(Point p,
                     void* f);}{constructs a vertex embedded in point \ccc{p}
and pointing on face \ccc{f}.}


\ccAccessFunctions
\ccMethod{Point point() const;}
{returns  the point.}
\ccGlue
\ccMethod{void* face() const;}{ a pointer to an incident face.}

\ccHeading{Setting}
\ccMethod{void set_point(Point p);}
{sets the point.}
\ccGlue
\ccMethod{void set_face(void* f);}
{sets the incident face.}

\ccHeading{Checking}
\ccMethod{bool is_valid() const;}{ performs any required geometrical test on a vertex.}
\end{ccClass}


\subsection{Requirements for the \protect \ccc{Face_base} Class}
\begin{ccClass} {Face_base}

\ccCreationVariable{f}

\ccDefinition
At the bottom layer, a 
face stores \ccc{void *} pointers
 to its three vertices  and to its three neighboring faces.
The vertices and neighbors are indexed 0,1 and 2 in counterclockwise
order around the face. Neighbor $i$ lies opposite to vertex $i$.


\ccCreation
\ccConstructor{F();}{default constructor}
\ccGlue
\ccConstructor{F( void* v0, void* v1, void* v2);}{neighbors are initialized to
\ccc{NULL}.}
\ccGlue
\ccConstructor{Face(void* v0, void* v1, void*  v2, void* n0, void* n1, void* n2);} {initializes the vertices with \ccc{ v0,v1, v2} and the neighbors with
\ccc{n0, n1, n2}.}
                    

                                           
\ccAccessFunctions
\ccMethod{ int dimension();}{returns the dimension.}
\ccMethod{void* vertex(int i) const;}
{ \ccPrecond $0\leq i \leq 2$.}
\ccGlue
\ccMethod{bool has_vertex(const void* v);}{true if \ccc{v} is a vertex
of \ccc{f}.}
\ccGlue
\ccMethod{bool has_vertex(const void* v, int& i) const;}
{as above and sets \ccc{i} to the index of \ccc{v}}
\ccGlue
\ccMethod{int vertex_index(const void* v) const;}{the index of \ccc{v} 
in \ccc{f}.}
\ccGlue
\ccMethod{void* neighbor(int i) const;}{\ccPrecond $0\leq i \leq 2$.}
\ccGlue
\ccMethod{bool has_neighbor(void* n);}{true if \ccc{n} is a neighbor
of \ccc{f}.}
\ccGlue
\ccMethod{bool has_neigbor(const void* n, int& i) const;}
{as above and sets i to the index of \ccc{n}.}
\ccGlue
\ccMethod{int face_index(const void* n) const;}
{returns the index of neighbor \ccc{n}.}

\ccHeading{Setting}
\ccMethod{void set_vertex(int i, void* v);}{\ccPrecond $0\leq i \leq 2$.}
\ccGlue
\ccMethod{ void set_vertices();}{sets the vertices pointers to \ccc{NULL}.}
\ccGlue
\ccMethod{void set_vertices(void* v0,
		    void* v1,
		    void* v2);}{sets the vertices pointers.}
\ccGlue
\ccMethod{void set_neighbor(int i, void* n);}{\ccPrecond $0\leq i \leq 2$.} 
\ccGlue
\ccMethod{void set_neighbors();} {sets the neighbors pointers to \ccc{NULL}.}
\ccGlue
\ccMethod{void set_neighbors(void* n0,
		     void* n1,
		     void* n2);}{sets the neighbors pointers.}
\ccMethod{void reorient();}
{Changes the orientation of \ccVar by exchanging \ccc{vertex(0)}
with \ccc{vertex(1)} and \ccc{neighbor(0)} with \ccc{neighbor(1)}.}
\ccMethod{void ccw_permute();}
{preforms a counterclockwise permutation of the
 vertices and neighbors of \ccVar.}
\ccMethod{ void cw_permute();}
{preforms a clockwise permutation of the
 vertices and neighbors of \ccVar.}

\ccHeading{Checking}
\ccMethod{bool is_valid() const;}{ To perform any required geometrical test on a face.}
\end{ccClass}

\subsection{The Default Base Classes}
\cgal\ provides two default base classes 
\ccc{Triangulation_face_base_2<Traits>} and \ccc{Triangulation_vertex_base_2<Traits>}
which model respectively 
the \ccc{Vertex_base} and the \ccc{Face_base} concept.
Both of them are templated by a geometric traits class.
Using for this traits class, the geometric traits class used for the triangulation class
is strongly recommended. 
It ensures that the point type defined by \ccc{Triangulation_vertex_base_2<Traits>}
is the same as the point type defined the  geometric traits class of
the triangulation.

These default base classes can be used directly or can serve as a base to derive
other base classes with some additional attribute (a color for example)
tuned for a specific application.

\ccInclude{CGAL/Triangulation_face_base_2.h} \\
\ccInclude{CGAL/Triangulation_vertex_base_2.h}

\subsection{Examples}
\ccExample
The following code creates a  valid triangulation traits class 
for a triangulation of 2D points in usual Euclidean space
and use it to define a triangulation class.

\begin{cprog}

typedef Cartesian<double> Rp;
typedef Triangulation_euclidean_traits_2<Rp> Traits;
typedef Triangulation_vertex_base_2<Traits> Vb;
typedef Triangulation_face_base_2<Traits> Fb;
typedef Triangulation_default_data_structure_2<Traits,Vb,Fb > Tds;
typedef Triangulation_2<Traits,Tds> Triangulation;

\end{cprog}

\ccExample
The following example derives a new \ccc{Face_base} class from the default
one and add a color to the faces of the triangulation. 
The face of the triangulation data structure
and the face of the triangulation will inherit the new data member 
and its functionality.
Any kind of additional fonctionality can thus be added to faces or vertices of a triangulation 
as long as this functionality  does not involve additional pointers to vertices or faces
(because the base classes use only void* pointer and have no knowledge
of the vertex or face types.).


%\begin{cprog}
\begin{ccExampleCode}
/*  colored_face.C   */     
/*  ---------------- */
#include <CGAL/basic.h>
#include <CGAL/Cartesian.h>
#include <CGAL/IO/Color.h>
#include <CGAL/Triangulation_euclidean_traits_2.h>
#include <CGAL/Triangulation_default_data_structure_2.h>
#include <CGAL/Triangulation_2.h>

using namespace CGAL;
using namespace std;

/* A facet with a color member variable. */
template < class Traits >
class My_face_base : public Triangulation_face_base_2<Traits>
{
public:
  Color color;
  My_face_base() :
    Triangulation_face_base_2<Traits>() {}
  My_face_base(void* v0, void* v1, void* v2) : 
    Triangulation_face_base_2<Traits>(v0,v1,v2) {}
  My_face_base(void* v0, void* v1, void* v2, void* n0, void* n1, void* n2) : 
    Triangulation_face_base_2<Traits>(v0,v1,v2,n0,n1,n2) {}
};

typedef Cartesian<double> Rp;
typedef Triangulation_euclidean_traits_2<Rp> Traits;
typedef Triangulation_vertex_base_2<Traits> Vb;
typedef My_face_base<Traits> Fb;
typedef Triangulation_default_data_structure_2<Traits,Vb,Fb > Tds;
typedef Triangulation_2<Traits,Tds> Triangulation;
typedef Triangulation::Face_handle Face_handle;
typedef Triangulation::Face_iterator Face_iterator;
typedef Triangulation::Vertex_handle Vertex_handle;
typedef Point_2<Rp>  Point;

int main() {
  Triangulation t;
  Point p;

  while (std::cin >> p){
    t.insert(p);
  }
  Face_iterator fc = t.faces_begin();
  while (fc != t.faces_end()) {
    fc->color = BLUE;
    ++fc;
  }
  return 1;
}
\end{ccExampleCode}
%\end{cprog}


\section{Delaunay Triangulations}
\label{I1_Sect_Delaunay}

\subsection{The Class \protect \ccc{Delaunay_triangulation_2<Traits,Tds>}}
\begin{ccClassTemplate} {Delaunay_triangulation_2<Traits,Tds>}
\ccDefinition

The class \ccClassTemplateName\ is designed to represent
the Delaunay triangulation of a set of points in a plane.
A  Delaunay triangulation of a set of points
is a triangulation of the sets of points that fulfills
the following {\em empty circle property} 
(also called {\em Delaunay property}): the circumscribing
circle of each
 face does not contain any vertex.
For a point set with no case of cocircularity 
of more than three points,
the Delaunay triangulation is unique, it is  the dual
of the Voronoi diagram of the points.


A Delaunay triangulation is a special triangulation of a set of points.
So it is natural to derive  the class \ccClassTemplateName\
from the basic class \ccc{Triangulation_2<Traits,Tds>}.
The template parameters \ccc{Traits} and \ccc{Tds} stand 
respectively
for a model of geometric traits and for a model of triangulation data structure.
The requirements for the triangulation data structure
are those described previously in section~\ref{I1_Sect_Tds}
and the same models can be used to instantiate the
triangulation data structure of either a \ccc{Triangulation_2<Traits,Tds>}
or a \ccClassTemplateName. On the contrary, because the
concept of Delaunay triangulation relies on the notions of 
empty circles and of distance, 
the geometric traits has to provide
the \ccc{side_of_oriented_circle} predicate
(which was only optional in the traits of plain triangulation),
and also
a Distance class. The additional requirements
to be fulfilled by the geometric traits class
of a \ccClassTemplateName\ are described in 
subsection~\ref{I1_Sect_Delaunay_geom_traits}. 
Changing the Distance class and the \ccc{side_of_oriented_circle} predicate 
allows to build Delaunay triangulations for different metrics
such that $L_1$ or $L_{\infty}$ or any metric defined by a
convex object. However, the user of an exotic metric
must be carefull that the constructed triangulation 
has to be a triangulation of the convex hull
which means that convex hull edges have to be Delaunay edges.
This is granted for any smooth convex metric (like $L_2$)
and can be ensured for other metrics (like  $L_{\infty}$)
by the addition to the point set of well chosen sentinel points.



\ccInheritsFrom

\ccc{Triangulation_2<Traits,Tds>}

\ccTypes
\ccThree{typedef Traits::Direction}{Direction}{}
\ccThreeToTwo

Inherits all the types of the \ccc{Triangulation_2<Traits>}.
In addition to the types inherited from \ccc{Triangulation_2<Traits,Tds>}
the class \ccClassTemplateName\
defines a type for the distance object function, and 
some types to
represent the dual Voronoi diagram.

\ccTypedef{typedef Traits::Distance Distance;}{}

\ccTypedef{typedef Traits::Line Line;}{}
\ccGlue
\ccTypedef{typedef Traits::Direction Direction;}{}
\ccGlue
\ccTypedef{typedef Traits::Ray Ray;}{}

\ccCreation
\ccCreationVariable{dt}
\ccThree{Vertex_handle}{T.push_back(const Point &p);}{}
\ccThreeToTwo

\ccConstructor{Delaunay_triangulation_2(const Traits& gt = Traits());}
{introduces an empty Delaunay triangulation \ccVar.}

\ccConstructor{Delaunay_triangulation_2(
	       const Delaunay_triangulation_2<Traits,Tds> &tr);}
{Copy constructor. All the vertices and faces are duplicated.}


\ccHeading{Insertion and Removal}

The following insertion and removal  functions overwrite
the functions inherited from the class
\ccc{Triangulation_2<Traits,Tds>} to maintain the Delaunay property.




\ccMethod{Vertex_handle insert(const Point& p);}{returns a handle to the created
vertex}
%{Inserts  point \ccc{p}.
%If point \ccc{p} coincides with an already existing vertex, this 
%vertex is returned and the triangulation is not updated.
%Optional parameter \ccc{f} is used to initialize the location of \ccc{p}.
%}

%\ccMethod{Vertex_handle insert(const Point& p, Locate_type& lt,
%Face_handle loc, int li );}
%{ inerts a point \ccc{p}, the location of which is supposed to be
%given by \ccc{(lt,loc,li)}, see the description of member function
%\ccc{locate} in class \ccc{Triangulation_2<Traits,Tds>}.}

\ccMethod{Vertex_handle push_back(const Point& p);}
{Equivalent to \ccc{insert(p)}.}

\ccMethod{template < class InputIterator >
          int
          insert(InputIterator first, InputIterator last);}
{inserts the points in the range
 $\left[\right.$\ccc{first}, \ccc{last}$\left.\right)$.
 Returns the number of inserted points.
 \ccPrecond The \ccc{value_type} of \ccc{first} and \ccc{last}
 is \ccc{Point}.}



\ccMethod{void    remove(Vertex_handle v);}
{removes the vertex from the triangulation.}


\ccHeading{Queries}

\ccMethod{Vertex_handle
        nearest_vertex(const Point& p, Face_handle f=Face_handle());}
{returns any nearest vertex of \ccc{p}. The implemented function
begins with a location step and
\ccc{f} may be used to initialize the location.}


\ccHeading{Duality}
The following functions help to build the dual Vorono\"i diagrams.
Note that since the dual of a egde of the triangulation can be either
a segment or a ray or a line, the return type is a \cgal \ccc{Object}.

\ccMethod{Point dual(const Face_handle &f) const;}
{Returns the center of the circle circumscribed to face \ccc{f}.
\ccPrecond \ccc{f} is not infinite}

\ccMethod{Object dual(const Edge &e) const;}
{returns a segment, a ray or a line supported by the bisector of the
endpoints of \ccc{e}.
If faces incident to \ccc{e} are both finite, a segment whose endpoints are the
duals of each incident face is returned. If only one incident face is
finite, a
ray whose endpoint is the dual of the finite incident face is returned.
Otherwise both incident faces
are infinite and the bisector line is returned.}

\ccMethod{Object dual(const Edge_circulator& ec) const;}
{Idem}

\ccMethod{Object dual(const Edge_iterator& ei) const;}
{Idem}



%\ccHeading{Geometric Predicates}
%\ccThree{Oriented_side}{side_of_}{}
%\ccThreeToTwo

%\ccMethod{Oriented_side
%  side_of_oriented_circle(Face_handle f,
%                          const Point& p) const;}
%{Returns the side of \ccc{p} with respect to the circle circumscribing
% the triangle associated with \ccc{f}}


%\begin{ccAdvanced}
%\ccHeading{Miscellaneous}
%\ccThree{Vertex_handle}{T.push_back(const Point &p);}{}

%The checking function \ccc{is_valid()} is also overwritten
%to additionally test the empty circle property.

%\ccMethod{bool is_valid(bool verbose = false, int level = 0) const;}
%{ Tests the validity of the triangulation as a \ccc{Triangulation_2}
%and additionally test the Delaunay property. This method is
% mainly  useful for debugging Delaunay triangulation algorithms designed by
% the user.}
%\end{ccAdvanced}

\ccHeading{I/O}
The I/O operators for \ccc{iostream} and for
the window stream are simply those defined for the base class
\ccc{Triangulation_2<Traits, Tds>}.
In addition, there is a template member function to ouput
the dual Voronoi diagram on a stream

\ccMethod{
template < class Stream>
  Stream& draw_dual(Stream & ps);}
{output the dual voronoi diagram to stream ps.}

\ccExample

The following code fragment creates a Delaunay triangulation with 
the usual Euclidean metric for the vertical projection of a 
terrain model. The points have elevation, that is they are 3D points
and the predicates which are defined in the Delaunay triangulation 
traits class forget about the $z$-coordinate of these points. 

\begin{cprog}
/*  terrain.C      */
/*  -------------- */
#include <CGAL/Homogeneous.h>
#include <CGAL/leda_integer.h>
#include <CGAL/Triangulation_euclidean_traits_xy_3.h>
#include <CGAL/Delaunay_triangulation_2.h>

using namespace CGAL;

typedef Homogeneous<leda_integer>  Rp;
typedef Triangulation_euclidean_traits_xy_3<Rp>  Terrain;
typedef Triangulation_vertex_base_2<Traits> Vb;
typedef Triangulation_face_base_2<Traits> Fb;
typedef Triangulation_default_data_structure_2<Traits,Vb,Fb > Tds;
typedef Delaunay_triangulation_2<Terrain, Tds> Delaunay;

int main()
{
    Delaunay dt(Terrain());

    Point_3<Rp> p;
    while(std::cin >> p){
        dt.insert(p);
    }
    return 1;    
}
\end{cprog} 


\ccHeading{Implementation}

Insertion is implemented by inserting in the triangulation, then
performing a sequence of Delaunay flips. The number of flips is \ccTexHtml{$O(d)$}{O(d)}
if the new vertex is of degree \ccTexHtml{$d$}{d} in the new triangulation. For
points distributed uniformly at random, insertion takes time \ccTexHtml{$O(1)$}{O(1)} on
average.

Removal calls the removal in the triangulation and then retriangulates
the hole in such a way that  the Delaunay criterion is satisfied. Removal of a
vertex of degree \ccTexHtml{$d$}{d} takes time \ccTexHtml{$O(d^2)$}{O(d^2)},
which is \ccTexHtml{$O(1)$}{O(1)} for a random
vertex in the triangulation.

Nearest neighbor is found in time \ccTexHtml{$O(n)$}{O(n)} in the
worst case, but in time \ccTexHtml{$O(1)$}{O(1)}
for vertices distributed uniformly at random  and any query point. 

\end{ccClassTemplate} 

\subsection{Requirements for the Geometric Traits of a Delaunay Triangulation}
\label{I1_Sect_Delaunay_geom_traits}

In addition to the requirements described in section~\ref{I1_Sect_Geom_traits}
for the geometrics traits of any triangulation,
the geometrics traits of a Delaunay triangulation
has to  provide a function object type \ccc{Side_of_oriented_circle_2} 
for the basic \ccc{in_circle} test used to maintain the
empty circle property.
The \ccc{in_circle} test is the test 
that actually defines the triangulation.
The additional types \ccc{Line_2} and
and \ccc{Ray_2} and the constructor objects
\ccc{Construct_circumcenter_2}, \ccc{Construct_bisector_2},
\ccc{Construct_midpoint}
are used to build the dual Voronoi diagram
and are required only if the dual functions are called.
The additional \ccc{Less_distance_to_pont_2} is only required
if \ccc{nearest_vertex(..)} query are issued.

\begin{ccClass}{Delaunay_geom_traits}
\ccCreationVariable{gt}

\ccTypes
\ccThree{Oriented_side}{xxxxxx}{}
\ccThreeToTwo
\ccNestedType{Line_2}{}
\ccGlue
\ccNestedType{Ray_2}{}

\ccNestedType{Side_of_oriented_circle_2}{Function object for the
\ccc{in_circle} test.}
\ccNestedType{Less_distance_to_point_2}{ Function object type to
compare the distance of two points to a third one.
Needed only if the \ccc{nearest_neighbor()} 
function is called.}

\ccNestedType{Construct_circumcenter_2}{Constructor object  to build
the circumcenter of three points}
\ccGlue
\ccNestedType{Construct_bisector_2}{Constructor object  to build
the bisector line of two points.}
\ccGlue
\ccNestedType{Construct_midpoint}{constructor object to construct the
midpoint of two points.}


\ccOperations
The following function provide access to the predicates and constructors
type objects.
\ccMethod{Side_of_oriented_circle_2 side_of_oriented_circle_2_object();}{}
\ccGlue
\ccMethod{Less_distance_to_point_2 less_distance_to_point_2_object(const Point_2& p);}{}
\ccMethod{ Construct_circumcenter_2 construct_circumcenter_2_object();}{}
\ccGlue                                        
\ccMethod{Construct_bisector_2 construct_bisector_2_object();} {}
\ccGlue
\ccMethod{Construct midpoint construct_midpoint_object();}{}

\end{ccClass}




\subsubsection{Predefined Geometric Traits Class}
The class \ccc{Triangulation_euclidean_traits_2<R>}
introduced in section~\ref{I1_Sect_Geom_traits} is 
 designed to be 
the geometric traits class of a Delaunay triangulation.
It implements the usual Euclidean metric
for the  two dimensional points \ccc{Points_2<R>}.
Three traits classes are provided to deal with
the Delaunay triangulation of two dimensional points which are
the \ccc{xy}, \ccc{yz} or \ccc{zx} projections of three dimensional points:\\
\ccc{Triangulation_euclidean_traits_xy_3<R>},\\
\ccc{Triangulation_euclidean_traits_yz_3<R>}, and\\
\ccc{Triangulation_euclidean_traits_zx_3<R>} \\
The requirements for the duality functions and nearest vertex
queries are not yet satisfied by
these last three classes.


\section{Regular triangulations}
\label{I1_Sect_Regular}
 
Let ${  PW} = \{(p_i, w_i), i = 1, \ldots , n \}$ be a set of 
weighted points where each $p_i$ is a point and each $w_i$
is a scalar called the weight of point $p_i$.
Alternatively, each weighted point $(p_i, w_i)$ can be regarded
as a two dimensional sphere with center $p_i$ and radius $r_i=\sqrt{w_i}$.

The power diagram of the set ${  PW}$ is a planar partition 
such that each cell corresponds to sphere $(p_i, w_i)$ of ${  PW}$
and is the locus of points  $p$ whose power with respect to $(p_i, w_i)$
is less than its power with respect to any other sphere $(p_j, w_j)$
in ${  PW}$. 
The dual of this diagram is a triangulation 
whose domain covers the convex hull of the set 
${  P}= \{ p_i, i = 1, \ldots , n \}$ of center points
and whose vertices are a subset of ${  P}$.
Such a triangulation is called a regular triangulation.
The  three points $p_i, p_j$ and $p_k$ of ${  P}$
form a triangle in the regular triangulation of ${  PW}$
iff there is a point $p$ of the plane whose
powers with respect to $(p_i, w_i)$, $(p_j, w_j)$
and $(p_k, w_k)$ are equal and less than the power of $p$
with respect to any other sphere in  ${  PW}$.

Let us defined the power product of two weighted points
$(p_i, w_i)$ and $(p_j, w_j)$ as:
\[\Pi(p_i, w_i,p_j, w_j) = p_ip_j ^2 - w_i  - w_j  .\]
$\Pi(p_i, w_i,p_j, 0)$ is simply the power of point $p_j$
with respect to the sphere $(p_i, w_i)$, and two weighted points 
are said to be orthogonal if their power product is null.
The power circle of three weighted points
 $(p_i, w_i)$, $(p_j, w_j)$
and $(p_k, w_k)$ is defined as the unique circle
$(\pi, \omega)$  orthogonal to
 $(p_i, w_i)$, $(p_j, w_j)$
and $(p_k, w_k)$.

The regular triangulation of the sets ${  PW}$
satisfies the following {\em regular property} (which just reduces to the 
Delaunay property when all the weights are null):
a triangle $p_ip_jp_k$ of the regular triangulation
of ${  PW}$ is such that the power product of any weighted point
 $(p_l, w_l)$ of ${  PW}$ with the power circle of
 $(p_i, w_i)$, $(p_j, w_j)$ is $(p_k, w_k)$ is positive or null.
We call  power test of the weighted point $(p_l, w_l)$ with respect
to the face  $p_ip_jp_k$, the predicates which amount to compute
the sign of 
the power product of $(p_l, w_l)$ with respect to
the power circle of
 $(p_i, w_i)$, $(p_j, w_j)$ is $(p_k, w_k)$,
which is given by the following
determinant
\[\left| \begin{array}{cccc}
1    & 1                       & 1                       & 1 \\
x_i & x_j & x_k & x_l \\
y_i & y_j & y_k & y_l \\
x_i ^2 + y_i ^2 -w_i & x_j ^2 + y_j ^2 - w_j & x_k ^2 + y_k ^2 - w_k &
x_l ^2 + y_l ^2 -w_l  
\end{array}
\right|
\]

A pair of neighboring faces $p_ip_jp_k$
and $p_ip_jp_l$ is said to be locally regular
(with respect to  the weights in ${  PW}$)
if the power test of $(p_l,w_l)$ with respect to
$p_ip_jp_k$ is positive.
A classical  result of computational geometry
establishes that a triangulation of the convex hull of ${  P}$
such that any pair of neighboring faces is regular with respect
to ${  PW}$, is a
 regular triangulation of ${  PW}$.

Alternatively, the regular triangulation
of the weighted points set ${  PW}$
can be obtained as the projection
on the two dimensional plane of the convex hull of the set of three
dimensional points 
${  P'}= \{ (p_i,p_i ^2 - w_i ), i = 1, \ldots , n \}$.

\subsection{ The Regular Triangulation  Class}


\begin{ccClassTemplate}{Regular_triangulation_2<Traits, Tds>}
The \cgal\  class \ccClassTemplateName\ is designed to maintain the
regular triangulation of a set of weighted points.
The template parameters \ccc{Traits}  and \ccc{Tds} stand respectively
 for a geometric traits class and a triangulation data structure class.
Any triangulation data structure that fulfills the requirements of 
section~\ref{I1_Sect_Tds} can be used for a regular triangulation.
The geometric traits class must provide a weighted point type
and a power test on these weighted points. The requirements and defaults
for the geometric traits and the power point type are list below.

\ccInclude{CGAL/Regular_triangulation_2.h}

 \ccInheritsFrom

\ccc{Triangulation_2<Traits,Tds>}

The functions \ccc{insert} and 
\ccc{remove} are overwritten to maintain the regular
property
and the checking function \ccc{is_valid()} is also overwritten
to additionally test the local regular property of 
any pair of neighboring faces.

\ccTypes
\ccThree{typedef Traits::Weighted_point}{Weighted_point;}{}
\ccThreeToTwo

\ccTypedef{typedef Traits::Bare_point  Bare_point;}{}
\ccGlue
\ccTypedef{typedef Traits::Weighted_point  Weighted_point;}{}

\ccCreation
\ccCreationVariable{rt}

\ccConstructor{Regular_triangulation_2(const Traits& gt = Traits());}
{Introduces an empty regular triangulation \ccVar.}

\ccConstructor{Regular_triangulation_2(const Regular_triangulation_2 &rt);}
{Copy constructor.}

\ccHeading{Insertion and Removal}


The vertices of the regular triangulation
of a set of weighted points ${  PW}$ form only a subset
of the set of center points of ${   PW}$.
Therefore the  insertion of a weighted point  in a regular triangulation
does not necessarily imply the creation of a new vertex.
If the new inserted point does not appear as a vertex in the
regular triangulation, it is said to be hidden 
by the face in which the corresponding center point is located.
Such a weighted point is stored in a list attached to the hiding face,
to be used for later tentative of insertions when future removal
of some points implies the destruction of the hiding face.


\ccThree{Vertex_handle}{rt.push_back(Point p)x;}{}

\ccMethod{bool insert(const Weighted_point& p);}
{inserts  weighted point \ccc{p}.
returns true if a new vertex is created. 
If a weighted point with the same center point but a 
different weight already exists in the triangulation, 
it is removed and replaced by the new point.
}

%\ccMethod{bool insert(const Weighted_point &p,
%	 	       Locate_type  lt,
%		       Face_handle loc, int li );}
%{insert a weighted point \ccc{p} whose bare-point is assumed to be
%located in  \ccc{lt,loc,li}.}

\ccMethod{Vertex_handle push_back(const Point& p);}
{Equivalent to \ccc{insert(p)}.}

\ccMethod{template < class InputIterator >
          int
          insert(InputIterator first, InputIterator last);}
{inserts the weighted points in the range
 $\left[\right.$\ccc{first}, \ccc{last}$\left.\right)$.
 Returns the number of created vertices.
 \ccPrecond The \ccc{value_type} of \ccc{first} and \ccc{last}
 is \ccc{Weighted_point}.}

\ccMethod{int   remove(Vertex_handle v);}
{removes the vertex from the triangulation and
returns the number of new vertices created by the insertion of previously
hidden points.}


%\ccHeading{Duality}

%\ccMethod{Point dual(const Face_handle &f) const;}
%{Returns the center of the power circle corresponding to face \ccc{f}.
%\ccPrecond \ccc{f} is not infinite}

%\ccMethod{Object dual(const Edge &e) const;}
%{If both incident faces are finite, returns a segment whose endpoints are the
%duals of each incident face. If only one incident face is finite, returns a
%ray whose endpoint is the dual of the finite incident face and supported by
%the line which is the bisector of the edge's endpoints. If both incident faces
%are infinite, returns the line which is the bisector of the edge's endpoints
%otherwise.  }

%\ccMethod{Object dual(const Edge_circulator& ec) const;}
%{Idem}

%\ccMethod{Object dual(const Edge_iterator& ei) const;}
%{Idem}

%*************************

\ccHeading{Geometric Predicates}
\ccThree{Oriented_side}{side_of_}{}
\ccThreeToTwo

\ccMethod{Oriented_side
  power_test(Face_handle f,
                          const Weighted_point& p) const;}
{Returns the power test of  \ccc{p} with respect to the 
power circle  associated with \ccc{f}}



\begin{ccAdvanced}
\ccHeading{Miscellaneous}
\ccThree{Vertex_handle}{T.push_back(const Point &p);}{}

\ccMethod{bool is_valid(bool verbose = false, int level = 0) const;}
{ Tests the validity of the triangulation as a \ccc{Triangulation_2}
and additionally test the regularity of the triangulation. This method is
 mainly  useful for debugging Delaunay triangulation algorithms designed by
 the user.}
\end{ccAdvanced}
\end{ccClassTemplate}

\ccExample

The following code fragment creates a regular triangulation 
of a set of weighted points.


\begin{cprog}
/*     regular.C */
/*-------------- */
#include <CGAL/basic.h>
#include <iostream>
#include <CGAL/Cartesian.h>
#include <CGAL/Regular_triangulation_euclidean_traits_2.h>
#include <CGAL/Regular_triangulation_2.h>

using namespace CGAL;

typedef Cartesian<double> Rp;
typedef double W;
typedef Regular_triangulation_euclidean_traits_2<Rp,W>  Traits;
typedef Triangulation_vertex_base_2<Traits> Vb;
typedef Regular_triangulation_face_base_2<Traits> Fb;
typedef Triangulation_default_data_structure_2<Traits,Vb,Fb > Tds;
typedef Regular_triangulation_2<Traits, Tds> Regular_triangulation;

int main()
{
   Regular_triangulation rt;

   Traits::Weighted_point wp;
   while(std::cin >> wp){
     std::cout << wp << std::endl;
     rt.insert(wp);
   }
   rt.is_valid();
   return 1;	
}

\end{cprog} 

\subsection{The Base Face Type of a Regular Triangulation}
The regular triangulation of a set of weighted point does not
necessarily
have one vertex for each of the input points. Some of the input
weigthed points have no cell in the dual power diagrams
and therefore do not correspond to a vertex of the regular
triangulation.
Those weighted point are said to be {\it hidden} points.
A  point which is hidden at a given time may appear later as a vertex of
the regular triangulation upon removal on some other weighted point.
Therefore, hidden points have to be stored somewhere.
A hidden point can appear as vertex of the triangulation
only when the
two dimensional face where its point component is located
(the face which hides it)
is removed. Therefore we decided to store each hidden point
in the face which  hides it and the nested face type of a 
regular triangulation is assumed to include a list of hidden 
weighted points. This list of weighted point is in fact included
in the base face of a regular triangulation.



\subsubsection{Requirements for the Base Face Class of a Regular Triangulation}
The base face type of a regular triangulation 
has to fulfiils the following requirements
in addition to those of section~\ref{I1_Sect_Base_classes}

\begin{ccClass}{Regular_face_base}
\ccCreationVariable{f}
\ccThree{bool}{f.is_constrained(int i)xxx}{}
\ccThreeToTwo

\ccTypes
\ccNestedType{ Weighted_point}
{This type must be the same
as the weighted point type 
\ccc{Traits::Weigthed_point}
defined by the geometric traits class of the regular triangulation.}
\ccGlue
\ccTypedef{typedef std::list<Weighted_point> Point_list;}
{An std list of hidden points.}

\ccCreation
\ccConstructor{Regular_face_base()}{Default constructor}
\ccGlue
\ccConstructor{Regular_face_base(void* v0, void* v1, void* v2)}{Constructor setting the incident vertices.}
\ccGlue
\ccConstructor{Regular_face_base(void* v0, void* v1, void*  v2, void*
n0, void* n1, void* n2)} {Constructor setting the incident vertices and 
the neighboring faces.}

\ccHeading{Access Functions}
\ccMethod{Point_list& point_list();}
{Returns a reference to the list of weighted points
hidden by the face.}
\end{ccClass}

\subsubsection{A Default Base Face Class for Regular Triangulations.}
\cgal\ provides the templated class 
\ccc{Regular_triangulation_face_base_2<Traits>}
which derives from \ccc{Triangulation_face_base_2<Traits>}
and can be used as a default base class for faces of regular triangulations.

\ccInclude{CGAL/Regular_triangulation_face_base_2.h} 

\subsection{The Geometric Traits class of a Regular Triangulation}

\subsubsection{Requirements}
\begin{ccClass} {Regular_geom_traits}

\ccCreationVariable{gt}

The geometrics traits class of a regular triangulation
must fulfill the requirements of section~\ref{I1_Sect_Geom_traits}
and additionally provide  
\ccc{Weitghed_point} type (which derives from the type \ccc{Point} 
and add a weight data member )
and a two types \ccc{Power_test_2} and \ccc{Power_test_degenerated_2}
for the power tests.
and a  power tests for weighted points.
%The traits must also provide the types 
%\ccc{Ray_2} and \ccc{Line_2} to output the dual power diagram,
%and also eventually a \ccc{Distance} type 
%(analogue to the distance type defined in ~\ref{I1_Sect_Delaunay_geom_traits})
%to answer to lowest power query (localisation in the power diagram).


\ccThree{Comparison_result}{t.compare_x();}{}
\ccThreeToTwo

\ccTypes
\ccTypedef{typedef Point_2 Bare_point;}{}
\ccGlue    
\ccNestedType{Weighted_point}{}

\ccNestedType{Power_test_2}{Function objects for generic power test on four
points;}
\ccGlue
\ccNestedType{Power_test_degenerated_2}{Function objects for
degenerated power test involving three collinear points.}
%\ccGlue
%\ccNestedType{Line}{}
%\ccGlue
%\ccNestedType{Ray}{}
%\ccGlue
%\ccNestedType{Direction}{}
%\ccGlue
%\ccNestedType{Distance} {NOT YET required.}


\ccHeading{Acces to predicate function objects}
\ccMethod
{Power_test_2   power_test_2_object();}{}
\ccMethod
{Power_test_degenerated_2  power_test_degenerated_2_object();}{}







  
\end{ccClass}

\subsubsection{A Predefined Geometric Traits Class}
\cgal\ provides the predefined geometric traits class \\
\ccc{Regular_triangulation_euclidean_traits_2<Rep,Weight>}.
This traits class is templated by a representation class \ccc{Rep}
and a weight type \ccc{Weight}.
This class inherits from
\ccc{Triangulation_euclidean_traits_2 <Rep >}
and uses a \ccc{Weighted_point} type
derived from the type \ccc{Point} of
\ccc{Triangulation_euclidean_traits_2 < R >}.

\begin{ccClassTemplate}{Regular_triangulation_euclidean_traits_2<Rep,Weight>}
\ccInclude{CGAL/Regular_triangulation_euclidean_traits_2.h}

 \ccInheritsFrom

\ccc{Triangulation_euclidean_traits_2<Rep>}

\ccTypes
\ccThree{typedef Triangulation_euclidean_traits_2<Rep>::Point}{Bare_point}{}
\ccTypedef{typedef Triangulation_euclidean_traits_2<Rep>::Point
Point;}{}
\ccGlue
\ccTypedef{typedef Weighted_point<Point,Weight> Weighted_point;}{}

\end{ccClassTemplate}

\subsection{Weighted Points}

\subsubsection{requirements for a weighted-point}
\begin{ccClass}{Weighted_point}
The weighted point type  provided by the geometric traits class
of a regular triangulation has to fulfill
the following requirements


\ccThree{typedef Point::Rt }{wp.power(wq)xx}{}
\ccThreeToTwo
\ccTypes
\ccNestedType{Point}{The point type}
\ccGlue
\ccNestedType{Weight}{The weight type.}
\ccGlue
\ccTypedef{typedef Point::Rt Rt;}{The ring type}

\ccInheritsFrom \ccc{Point}

\ccCreation
\ccCreationVariable{wp}
\ccConstructor{Weighted_point(Point p, Weight w= Weight(0))}{}
\ccGlue
\ccConstructor{Weighted_point(Weighted_point wq)}{copy constructor.}

\ccAccessFunctions
\ccMethod{Point point() const;}{}
\ccGlue
\ccMethod{ Weight weight() const;}{}

\end{ccClass}


\subsubsection{Default Class for a Weighted Point}
\cgal\ provides the class \ccc{Weighted_point<Point,Weight>}
as a default type for a  weighted two dimensional point.
This default type has two template parameters \ccc{Point}
and  \ccc{Weight} which have to be instantiated respectively 
with a point type and a weight type.
The class \ccc{Weighted_point<Point,Weight>}
 inherits from \ccc{Point}.




\section{Constrained Triangulations}
\label{I1_Sect_Constrained}



A constrained triangulation is a triangulation of a set of points
which has to include among its edges 
a given set of segments joining the points. The corresponding 
edges are called {\em constrained edges}. 

The set of points defining the vertices of the triangulation
includes the set of constrained edges endpoints. It may include other points
(considered as null length constrained edges) as well.  The set of
constrained edges forms a set of segments which do not intersect
except possibly at their endpoints.  Any number of constrained edges
are allowed to share the same endpoint.  Vertical constrained edges or
constrained edges with null length are allowed.


\begin{ccTexOnly}
\begin{center} \IpeScale{50} \Ipe{constraints.ipe} \end{center}
\end{ccTexOnly}

\begin{ccHtmlOnly}
<CENTER>
<img border=0 src=constraints.gif align=center alt="A set of
constraints and its constrained triangulation">
</CENTER>
\end{ccHtmlOnly}

\begin{ccClassTemplate} {Constrained_triangulation_2<Traits,Tds>}
\subsection{ The Constrained Triangulation  Class}
A constrained triangulation is represented in the CGAL library as an
object of the class \ccClassTemplateName .
As usual the template parameters \ccc{Traits} and \ccc{Tds}
stand respectively for a geometric traits class and
a triangulation data structure class. There is no additional
requirements for the geometric traits and the triangulation data
structure of a constrained triangulation.
Models used to instantiate these classes are simply required
to fulfill repectively the requirements 
of section~\ref{I1_Sect_Geom_traits} and ~\ref{I1_Sect_Tds}.

\ccInclude{CGAL/Constrained_triangulation_2.h} 
 
\ccInheritsFrom

\ccc{Triangulation_2<Traits,Tds>}

\ccTypes
\ccThree{typedef pair<Point,Point>}{Constraint;}{}
\ccThreeToTwo
The only new type defined by
\ccClassTemplateName\ is a constraint type: a
constraint is represented as a pair of points.

\ccTypedef{typedef pair<Point,Point> Constraint;}{}

\ccCreation
\ccCreationVariable{ct}

The creators of the class build the constrained triangulation from a
list of constrained edges. Constrained edges are assumed to have no
intersection other than endpoints. Any number of constrained edges are
allowed to share the same endpoint.  Vertical constrained edges or
constrained edges with null length are allowed.

\ccConstructor{Constrained_triangulation_2(const Traits &t = Traits())}
{Introduces an empty constrained triangulation \ccVar.}

\ccConstructor{Constrained_triangulation_2(const
Constrained_triangulation_2& ctbis)}
 {Copy constructor, all faces and vertices
are duplicated and  the constrained status of edges
is copied. This last feature is not yet implemented.}

\ccConstructor{Constrained_triangulation_2(list<Constrained>& lc, Traits& t = Traits());}
{Introduces a constrained triangulation, the constrained edges of which
are the edges of the list \ccc{lc}.}


\ccConstructor{ template<class InputIterator> Constrained_triangulation_2(
        InputIterator first,
                               InputIterator last,
                               const Traits& t=Traits());}
{A templated constructor which introduces and builds
 a constrained triangulation with constrained edges in the range 
$\left[\right.$\ccc{first}, \ccc{last}$\left.\right)$.
\ccPrecond The \ccc{value_type} of \ccc{first} and \ccc{last}
 is \ccc{Constraint}.}


\ccHeading{Insertion and removal}

\ccMethod{Vertex_handle insert(Point a);}
{ Inserts point a and restore the status (constrained or not) of all
the
touched edges.}

\ccMethod{void insert(Point a, Point b);}
{ Inserts points a and b, and inserts segment ab as a
constraint. Removes the faces crossed by segment ab and creates new
faces instead. If a vertex c lies on segment ab, constraint ab is
replaced by the two constraints ac and cb. Apart from the insertion of
a and b, the algorithm runs in time proportionnal to the number of
removed triangles. 
\ccPrecond The relative interior of segment \ccc{ab} does not
intersect the relative interior of another constrained edge}

%\ccMethod{ void insert(const Vertex_handle & va, const Vertex_handle & vb);}
%{ Inserts the line segment \ccc{s} whose endpoints are the vertices 
%\ccc{va} and
%\ccc{vb}  as a constraintedge \ccc{e}. The triangles intersected by s
%are removed and new ones are created. 
%\ccPrecond The relative interior of \ccc{s} does not
%intersect the relative interior of another constrained edge.  
%\ccPrecond  va and vb are distinct vertices of t.  }

%\ccMethod{ void insert(const Vertex_handle & va, const Vertex_handle & vb,
%	    Face_handle & fr, int & i);}
%{Same as above. In addition,  sets the face \ccc{fr} incident to the
%egde \ccc{e} 
%and on the right of \ccc{e} oriented from  \ccc{va} to \ccc{vb}
%and the index \ccc{i}  of the vertex of \ccc{fr} opposite to
%\ccc{e}, i.e.  \ccc{e}=\ccc{(fr,i)}.}	    

%\ccMethod{void insert(const Vertex_handle & va, const Vertex_handle & vb,
%	    Face_handle & fr, int & i, List_edges & new_edges,
%	List_vertices & new_vertices);}
%{Same as above. In addition,  the edges that are created are put in
%the list \ccc{new\_edges} and the new vertices
%\(resulting from the intersection of two constraints\) are put in the
%list
%\ccc{new\_vertices}.}

\ccMethod{void remove(const Vertex_handle & v);}
{ Removes a vertex v. All constraints incident to the removed vertex are removed. }

\ccMethod{void remove_constraint(const Face_handle & f, int i);}
{ Edge \ccc{e}=\ccc{(f,i)}=\ccc{(g,j)} is no longer constrained.}

\ccHeading{I/O}

\ccFunction{ostream & operator<<(ostream& os, const CGAL_Constrained_triangulation_2<Traits,Tds> &Ct);}
{Writes the triangulation and, for each face f, and integers i=0,1,2,
write ``C'' or ``N'' depending whether edge 
\ccc{(f,i)} is constrained or not.}


\ccHeading{Implementation}

The constructors build the triangulation using a sweeping line
algorithm. The complexity of this algorithm is $O(n\log n)$ if $n$
endpoints are present. The sweep structure is an \stl\ map.
 The insertion of a constrained edge runs in time
proportionnal to the number of triangles intersected by this edge.

There is no need for a special implementation
of the method \ccVar\ccc{.is_valid()}
because the base class function
\ccc{ Triangulation_2<Traits>::is_valid()}
call the face class method
\ccc{Tds::Face::is_valid()} 
which, in the case of a constrained triangulation,
includes a test of the consistency of the
information about constrained edges.
\end{ccClassTemplate}

\subsection{The Face Type of a Constrained Triangulation}
\label{I1_Sect_Constrained_face}
 The information about constrained edges is store in the 
faces of the triangulation. Thus the nested \ccc{Face}
type of a constrained triangulation offers
additonnal functionalities to deal with this information.
This additional functionalities related to the constraints
are requirements which have to be fulfilled
by the base face  a constrained triangulation
in addition to the functionalities required in section~\ref{I1_Sect_Base_classes}
They are listed below as such.

\begin{ccClass}{Constrained_face_base}
\ccCreationVariable{f}

\ccThree{bool}{f.is_constrained(int i)xxx}{}
\ccThreeToTwo

\ccCreation
\ccConstructor{Constrained_face_base()}{default constructor.}
\ccGlue
\ccConstructor{Constrained_face_base(void* v0, void* v1, void* v2)}{constructor setting the incident vertices.}
\ccGlue
\ccConstructor{Constrained_face_base(void* v0, void* v1, void*  v2, void* n0, void* n1, void* n2)} 
{constructor setting the incident vertices and the neighboring faces.}
\ccGlue
\ccConstructor{Constrained_face_base(void* v0, void* v1, void*  v2,
void* n0, void* n1, void* n2, bool c0, bool c1, bool c2)}
 {constructor setting the incident vertices, the neighboring faces and 
the constrained or constrained status of the edges of the face}
                   
\ccThree{bool}{f.is_constrained(int i)}{}
\ccHeading{Access Functions}
\ccMethod{bool is_constrained(int i);}
{returns true if the edge between \ccVar\ and its neighbor 
\ccVar .\ccc{neighbor(i)} is constrained.
\ccPrecond $0\leq i \leq 2$.}


\ccModifiers
\ccMethod{void  set_constraint(int i, bool b);}
{sets the edge between \ccVar\ and its neighbor \ccVar .\ccc{neighbor(i)}
 as a constrained or unconstrained edge according to \ccc{b}.}

\ccMethod{void set_constraints(bool c0, bool c1, bool c2);}
{sets the status (constrained or unconstrained) of the three
edges of \ccVar.}

\ccMethod{void reorient();}
{Changes the orientation of \ccVar by exchanging \ccc{vertex(0)}
with \ccc{vertex(1)} and \ccc{neighbor(0)} with \ccc{neighbor(1)}
and the corresponding constrained status.}
\ccMethod{void ccw_permute();}
{preforms a counterclockwise permutation of the
 vertices, neighbors and constrained status of  \ccVar.}
\ccMethod{ void cw_permute();}
{preforms a clockwise permutation of the
 vertices and neighbors and constrained status of \ccVar.}


\begin{ccAdvanced}
\ccHeading{Miscelleanous}
\ccMethod{bool is_valid();}
{tests the validity of face  \ccVar\ 
as a face of a plain triangulation
and additionally checks
if the edges of \ccVar\ are consistently marked
as constrained or unconstrained edges
in face \ccVar and its neighbors.}
\end{ccAdvanced}
\end{ccClass}


Of course  \cgal\ provides a default \ccc{Face_base} class
for the constrained triangulation. The class
\ccc{Constrained_triangulation_face_base_2<Traits>}
simply derived from 
\ccc{Triangulation_face_base_2<Traits>} and override the
functions \ccc{reorient()}, \ccc{ccw_permute()} and 
\ccc{cw_permute()}.

\ccInclude{CGAL/Constrained_triangulation_face_base_2.h}

\subsection{A Constrained Triangulation Class for Animation Purposes
\label{I1_Subsec_Constrained_Demo_2}}

\begin{ccClassTemplate}{Constrained_triangulation_demo_2<Traits,Tds>}

\ccDefinition
The class  \ccClassTemplateName\ is intended to provide
an animation of the sweep algorithm that builds a 
constrained triangulation.

\ccInclude{CGAL/Constrained_triangulation_demo_2.h}\\


\ccInheritsFrom
\ccc{Constrained_triangulation_2<Traits,Tds>}

\ccTypes
\ccThree{typedef Window_stream}{Window_stream;}{}
\ccThreeToTwo
\ccTypedef{ typedef Window_stream Window_stream;}{}

\ccCreation
\ccCreationVariable{ctd}

\ccConstructor{Constrained_triangulation_demo_2(Window_stream& W,
                                     list<Constrained>& lc, Traits gt=Traits());}
{Outputs the created faces at each step of
the sweeping line algorithm, providing thus an animation
of the sweeping line algorithm.}
\end{ccClassTemplate}


\section{Constrained Delaunay Triangulations}
\label{I1_Sect_Constrained_Delaunay_Triangulations}

A constrained Delaunay triangulation is a triangulation with
constrained edges which tries to be as much Delaunay as possible.
As constrained edges are not necessarily Delaunay edges,
the triangles of a constrained Delaunay triangulation do not
necessarily fulfill the empty circle property
but they fulfill a weaker constrained empty circle property.
 To state this property,
it is convenient to think of  constrained
edges as blocking the view. Then, a triangulation is 
constrained Delaunay if
 the circumscribing circle
of any of its triangular faces includes in its interior 
no vertex  that is visible
from the interior of the triangle.

\subsection{The Class 
\protect \ccc{Constrained_Delaunay_triangulation_2<Traits,Tds>}}
\begin{ccClassTemplate}{Constrained_Delaunay_triangulation_2<Traits,Tds>}
The \cgal\ class \ccClassTemplateName\ is designed to represent
constrained Delaunay triangulations.

The class is templated by a geometric traits class \ccc{Traits}
and a triangulation data structure \ccc{Tds}.
There are no special requirements for the triangulation data
structure
of a constrained Delaunay triangulations and the requirements
for this class are those described
in section~\ref{I1_Sect_Tds}. The geometric traits 
of a constrained Delaunay triangulation is required
to provide the \ccc{side_of_oriented_circle} test as the geometric traits
of a Delaunay triangulation and the requirements for this traits
are described in section~\ref{I1_Sect_Delaunay_geom_traits}. 

A constrained Delaunay triangulation is not a Delaunay
triangulation but it is a constrained triangulation.
Therefore the class \ccClassTemplateName\ derives from
the class \ccc{Constrained_triangulation_2<Traits,Tds>}.
Also, information about the status (constrained or not)
of the edges of the triangulation has to be stored
in the face class
 and the requirements for the base face class
of a constrained Delaunay triangulation are 
identical to those described in 
section~\ref{I1_Sect_Constrained_face} for the face base class
of a constrained  triangulation.

\ccInclude{CGAL/Constrained_triangulation_2.h} 
 
\ccInheritsFrom \ccc{Constrained_triangulation_2<Traits,Tds>}

\ccTypes
All types used in this class are inherited from the base class
\ccc{Constrained_triangulation_2<Traits,Tds>}.

\ccCreation
\ccCreationVariable{cdt}

\ccConstructor{Constrained_Delaunay_triangulation_2(const Traits &t = Traits())}
{Introduces an empty constrained Delaunay triangulation \ccVar.}

\ccConstructor{Constrained_Delaunay_triangulation_2(const
Constrained_Delaunay_triangulation_2& cdtbis)}
 {Copy constructor, all faces and vertices
are duplicated and  the constrained status of edges
is copied. This last feature is not yet implemented.}

\ccConstructor{Constrained_Delaunay_triangulation_2(list<Constrained>& lc, Traits& t = Traits());}
{Introduces a constrained triangulation, the constrained edges of which
are the edges of the list \ccc{lc}.}


\ccConstructor{ template<class InputIterator> Constrained_triangulation_2(
        InputIterator first,
                               InputIterator last,
                               const Traits& t=Traits());}
{A templated constructor which introduces and builds
 a constrained triangulation with constrained edges in the range 
$\left[\right.$\ccc{first}, \ccc{last}$\left.\right)$.
\ccPrecond The \ccc{value_type} of \ccc{first} and \ccc{last}
 is \ccc{Constraint}.}

%\begin{ccAdvanced}
%\ccHeading{Flips}

%\ccMethod{bool is_flipable(Face_handle f, int i);}
%{ Determines if edge (f,i) can be flipped. Returns true if
%1. edge(f,i) is not constrained and 2. the circle circumscribing f
%does not contain the vertex of f->neighbor(i) not on edge(f,i). }

%\ccMethod{void flip(Face_handle& f, int i);}
%{ Flip f and f->neighbor(i).
%\ccPrecond f->is\_constrained(i) == FALSE.}


%\ccMethod{void propagating_flip(List_edges & edges);}
%{ Makes the triangulation constrained Delaunay by flipping edges. 
%List edges contains an
%initial list of edges to be flipped. The returned
%triangulation is constrained Delaunay 
%if the list edges contains all edges of the
%input triangulation that need to be flipped (plus possibly others). }
%\end{ccAdvanced}

\ccHeading{Insertion and Removal}
The following member functions override the corresponding
member of the base class to include a step restoring
 the Delaunay constrained
property after modification of the triangulation.

\ccThree{Vertex_handle}{insert(Point a, Point b);}{}
\ccMethod{ Vertex_handle insert(Point a);}
{ Inserts point a in the triangulation. }

\ccMethod{ void insert(Point a, Point b);}
{ Inserts segment ab as a constrained edge in the triangulation. }

%\ccMethod{ void insert(Vertex_handle va, Vertex_handle vb);}
%{ Inserts the line segment whose endpoints are the vertices va and vb
%as an edge e in the triangulation. }

%\ccMethod{void insert(Vertex_handle va, Vertex_handle vb,
%		   Face_handle & fr, int & i);}
%{Same as the previous procedure. In addition, returns f and i such that
%e=(f,i) and f is the face lying to the right of the oriented edge (va,vb). }


\ccMethod{void remove(Vertex_handle & v);}
{ Removes vertex v. }


\ccHeading{Checking}

\ccMethod{bool is_valid();}
{ Checks if the triangulation is valid and if each constrained edge is
a constraint for its two incident faces.}

\end{ccClassTemplate}
