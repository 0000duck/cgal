% +------------------------------------------------------------------------+
% | Reference manual page: Regular_triangulation_2.tex
% +------------------------------------------------------------------------+
% | 12.04.2000   Author
% | Package: Package
% | 
\RCSdef{\RCSRegulartriangulationRev}{$Revision$}
\RCSdefDate{\RCSRegulartriangulationDate}{$Date$}
% |
%%RefPage: end of header, begin of main body
% +------------------------------------------------------------------------+


\begin{ccRefClass}{Regular_triangulation_2<Traits,Tds>}  %% add template arg's if necessary

%% \ccHtmlCrossLink{}     %% add further rules for cross referencing links
%% \ccHtmlIndexC[class]{} %% add further index entries

\ccDefinition
  
The class \ccRefName\ 
is designed to maintain the
regular triangulation of a set of weighted points.


Let ${  PW} = \{(p_i, w_i), i = 1, \ldots , n \}$ be a set of 
weighted points where each $p_i$ is a point and each $w_i$
is a scalar called the weight of point $p_i$.
Alternatively, each weighted point $(p_i, w_i)$ can be regarded
as a two dimensional sphere with center $p_i$ and radius $r_i=\sqrt{w_i}$.

The power diagram of the set ${  PW}$ is a planar partition 
such that each cell corresponds to sphere $(p_i, w_i)$ of ${  PW}$
and is the locus of points  $p$ whose power with respect to $(p_i, w_i)$
is less than its power with respect to any other sphere $(p_j, w_j)$
in ${  PW}$. 
The dual of this diagram is a triangulation 
whose domain covers the convex hull of the set 
${  P}= \{ p_i, i = 1, \ldots , n \}$ of center points
and whose vertices are a subset of ${  P}$.
Such a triangulation is called a regular triangulation.
The  three points $p_i, p_j$ and $p_k$ of ${  P}$
form a triangle in the regular triangulation of ${  PW}$
iff there is a point $p$ of the plane whose
powers with respect to $(p_i, w_i)$, $(p_j, w_j)$
and $(p_k, w_k)$ are equal and less than the power of $p$
with respect to any other sphere in  ${  PW}$.

Let us defined the power product of two weighted points
$(p_i, w_i)$ and $(p_j, w_j)$ as:
\[\Pi(p_i, w_i,p_j, w_j) = p_ip_j ^2 - w_i  - w_j  .\]
$\Pi(p_i, w_i,p_j, 0)$ is simply the power of point $p_j$
with respect to the sphere $(p_i, w_i)$, and two weighted points 
are said to be orthogonal if their power product is null.
The power circle of three weighted points
 $(p_i, w_i)$, $(p_j, w_j)$
and $(p_k, w_k)$ is defined as the unique circle
$(\pi, \omega)$  orthogonal to
 $(p_i, w_i)$, $(p_j, w_j)$
and $(p_k, w_k)$.

The regular triangulation of the sets ${  PW}$
satisfies the following {\em regular property} (which just reduces to the 
Delaunay property when all the weights are null):
a triangle $p_ip_jp_k$ of the regular triangulation
of ${  PW}$ is such that the power product of any weighted point
 $(p_l, w_l)$ of ${  PW}$ with the power circle of
 $(p_i, w_i)$, $(p_j, w_j)$ is $(p_k, w_k)$ is positive or null.
We call  power test of the weighted point $(p_l, w_l)$ with respect
to the face  $p_ip_jp_k$, the predicates which amount to compute
the sign of 
the power product of $(p_l, w_l)$ with respect to
the power circle of
 $(p_i, w_i)$, $(p_j, w_j)$ is $(p_k, w_k)$,
which is given by the following
determinant
\[\left| \begin{array}{cccc}
1  &  x_i  &  y_i  &  x_i ^2 + y_i ^2 - w_i  \\
1  &  x_j  &  y_j  &  x_j ^2 + y_j ^2 - w_j  \\
1  &  x_k  &  y_k  &  x_k ^2 + y_k ^2 - w_k  \\
1  &  x_l  &  y_l  &  x_l ^2 + y_l ^2 - w_l
\end{array}
\right|
\]

A pair of neighboring faces $p_ip_jp_k$
and $p_ip_jp_l$ is said to be locally regular
(with respect to  the weights in ${  PW}$)
if the power test of $(p_l,w_l)$ with respect to
$p_ip_jp_k$ is positive.
A classical  result of computational geometry
establishes that a triangulation of the convex hull of ${  P}$
such that any pair of neighboring faces is regular with respect
to ${  PW}$, is a
 regular triangulation of ${  PW}$.

Alternatively, the regular triangulation
of the weighted points set ${  PW}$
can be obtained as the projection
on the two dimensional plane of the convex hull of the set of three
dimensional points 
${  P'}= \{ (p_i,p_i ^2 - w_i ), i = 1, \ldots , n \}$.



 
A Regular triangulation is a special triangulation of a set of points,
so it is natural to derive  the class \ccRefName\
from the basic class \ccc{Triangulation_2<Traits,Tds>}.
The template parameters \ccc{Traits} and \ccc{Tds} 
are to be instantiated with models of respectively
\ccc{RegularTriangulationTraits_2}  and
\ccc{TriangulationDataStructure_2}.


The concept of triangulation data structure
is described in \ccRefPage{TriangulationDataStructure_2}
and the same models can be used to instantiate the
triangulation data structure of either a \ccc{Triangulation_2<Traits,Tds>}
or a \ccRefName.
The  concept of \ccc{RegularTriangulationTraits_2},
described page~\ccRefPage{RegularTriangulationTraits_2}, 
must provide a weighted point type
and a power test on these weighted points. 

\ccInclude{Regular_triangulation_2.h}

\ccInheritsFrom
\ccc{Triangulation_2<Traits,Tds>}

\ccTypes
\ccThree{typedef Traits::Weighted_point}{Weighted_point;}{}
\ccThreeToTwo
\ccTypedef{typedef Traits::Distance Distance;}{}

\ccTypedef{typedef Traits::Line Line;}{}
\ccGlue
\ccTypedef{typedef Traits::Direction Direction;}{}
\ccGlue
\ccTypedef{typedef Traits::Ray Ray;}{}
\ccGlue
\ccTypedef{typedef Traits::Bare_point  Bare_point;}{}
\ccGlue
\ccTypedef{typedef Traits::Weighted_point  Weighted_point;}{}


\ccCreation
\ccCreationVariable{rt}  %% choose variable name

\ccConstructor{Regular_triangulation_2(const Traits& gt = Traits());}
{Introduces an empty regular triangulation \ccVar.}

\ccConstructor{Regular_triangulation_2(const Regular_triangulation_2 &rt);}
{Copy constructor.}


\ccHeading{Insertion and Removal}

The vertices of the regular triangulation
of a set of weighted points ${  PW}$ form only a subset
of the set of center points of ${   PW}$.
Therefore the  insertion of a weighted point  in a regular triangulation
does not necessarily imply the creation of a new vertex.
If the new inserted point does not appear as a vertex in the
regular triangulation, it is said to be hidden 
by the face in which the corresponding center point is located.
Such a weighted point is stored in a list attached to the hiding face,
to be used for later tentative of insertions when future removal
of some points implies the destruction of the hiding face.


\ccThree{Vertex_handle}{T.push_back(const Point &p);}{}

\ccMethod{bool insert(const Weighted_point& p, Face_handle f=Face_handle());}
{inserts  weighted point \ccc{p} and 
returns true if a new vertex is created. 
If a weighted point with the same center point but a 
different weight already exists in the triangulation, 
it is removed and replaced by the new point.
If given the parameter \ccc{f} is used as an hint
for the place to start the location process of point \ccc{p}.}


\ccMethod{bool insert(const Weighted_point &p,
	 	       Locate_type  lt,
		       Face_handle loc, int li );}
{insert a weighted point \ccc{p} whose bare-point is assumed to be
located in  \ccc{lt,loc,li}.}

\ccMethod{Vertex_handle push_back(const Point& p);}
{Equivalent to \ccc{insert(p)}.}

\ccMethod{template < class InputIterator >
          int
          insert(InputIterator first, InputIterator last);}
{inserts the weighted points in the range
 $\left[\right.$\ccc{first}, \ccc{last}$\left.\right)$.
 Returns the number of created vertices.
 \ccPrecond The \ccc{value_type} of \ccc{first} and \ccc{last}
 is \ccc{Weighted_point}.}

\ccMethod{int   remove(Vertex_handle v);}
{removes the vertex from the triangulation and
returns the number of new vertices created by the insertion of previously
hidden points.}

\ccHeading{Duality}
\ccMethod{Point weighted_circumcenter(const Face_handle &f) const;}
{returns the center of the circle orthogonal to the three weighted
points corresponding to the vertices of face \ccc{f}.
\ccPrecond \ccc{f} is not infinite}

\ccMethod{Point dual(const Face_handle &f) const;}
{same as weighted\_circumcenter}

\ccMethod{Object dual(const Edge &e) const;}
{If both incident faces are finite, returns a segment whose endpoints are the
duals of each incident face. If only one incident face is finite, returns a
ray whose endpoint is the dual of the finite incident face and supported by
the line which is the bisector of the edge's endpoints. If both incident faces
are infinite, returns the line which is the bisector of the edge's endpoints
otherwise.  }

\ccMethod{Object dual(const Edge_circulator& ec) const;}
{Idem}

\ccMethod{Object dual(const Edge_iterator& ei) const;}
{Idem}



\ccPredicates
\ccThree{Oriented_side}{side_of_}{}
\ccThreeToTwo

\ccMethod{Oriented_side
  power_test(Face_handle f,
                          const Weighted_point& p) const;}
{Returns the power test of  \ccc{p} with respect to the 
power circle  associated with \ccc{f}}



\begin{ccAdvanced}
\ccHeading{Miscellaneous}
\ccThree{Vertex_handle}{T.push_back(const Point &p);}{}

\ccMethod{bool is_valid(bool verbose = false, int level = 0) const;}
{ Tests the validity of the triangulation as a \ccc{Triangulation_2}
and additionally test the regularity of the triangulation. This method is
 mainly  useful for debugging Delaunay triangulation algorithms designed by
 the user.}
\end{ccAdvanced}




\ccSeeAlso
\ccc{CGAL::Triangulation_2<Traits,Tds>}, \\
\ccc{TriangulationDataStructure_2}, \\
\ccc{RegularTriangulationTraits_2} \\
\ccc{RegularTriangulationFaceBase_2} \\
\ccc{CGAL::Regular_triangulation_face_base_2<Traits>}


%\ccExample
%The following code fragment creates a regular triangulation 
%of a set of weighted points.


%\ccIncludeExampleCode{Triangulation_2/regular.C}


\end{ccRefClass}

% +------------------------------------------------------------------------+
%%RefPage: end of main body, begin of footer
% EOF
% +------------------------------------------------------------------------+

