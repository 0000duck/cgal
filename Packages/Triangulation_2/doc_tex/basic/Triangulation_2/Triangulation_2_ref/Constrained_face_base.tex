% +------------------------------------------------------------------------+
% | Reference manual page: Constrained_face_base.tex
% +------------------------------------------------------------------------+
% | 12.04.2000   Author
% | Package: Package
% | 
\RCSdef{\RCSConstrainedfacebaseRev}{$Revision$}
\RCSdefDate{\RCSConstrainedfacebaseDate}{$Date$}
% |
%%RefPage: end of header, begin of main body
% +------------------------------------------------------------------------+


\begin{ccRefConcept}{Constrained_triangulation_face_base_2_concept}

%% \ccHtmlCrossLink{}     %% add further rules for cross referencing links
%% \ccHtmlIndexC[concept]{} %% add further index entries

\ccDefinition
The concept \ccRefName\ is  an extension of the concept
\ccc{Triangulation_face_base_2_concept} 
providing functionalities to deal with
constraints.  
In a constrained triangulation,
the information about constrained edges is store in the 
faces of the triangulation.
Thus the nested \ccc{Face}
type of a constrained triangulation offers
additonnal functionalities to deal with this information.
Within the design of a triangulation data structure
model templated by the base face class and the base vertex class,
these additional functionalities related to the constraints
are requirements which have to be fulfilled
by the base face of  a constrained triangulation.
The base face of a constrained triangulation
has to be a model of the concept
\ccRefName\ which provides the
required functionalities.

The concept \ccRefName\ is  an extension of 
\ccc{Triangulation_face_base_2_concept}. Only additionnal
requirements are listed below.

\ccTypes
Defines the same types as the \ccc{Triangulation_face_base_2} concept


\ccCreation
\ccCreationVariable{f}  %% choose variable name

\ccConstructor{Constrained_triangulation_face_base_2_concept();}{default constructor.}
\ccGlue
\ccConstructor{Constrained_triangulation_face_base_2_concept(void* v0, void* v1, void* v2)}{constructor setting the incident vertices.}
\ccGlue
\ccConstructor{Constrained_triangulation_face_base_2_concept(void* v0, void* v1, void*  v2, void* n0, void* n1, void* n2)} 
{constructor setting the incident vertices and the neighboring faces.}
\ccGlue
\ccConstructor{Constrained_triangulation_face_base_2_concept(void* v0, void* v1, void*  v2,
void* n0, void* n1, void* n2, bool c0, bool c1, bool c2)}
 {constructor setting the incident vertices, the neighboring faces and 
the status (constrained or not constrained) of the edges of the face}
                   

\ccThree{bool}{f.is_constrained(int i)}{}
\ccHeading{Access Functions}
\ccMethod{bool is_constrained(int i);}
{returns true if the edge between \ccVar\ and its neighbor 
\ccVar .\ccc{neighbor(i)} is constrained.
\ccPrecond $0\leq i \leq 2$.}

\begin{ccAdvanced}
\ccModifiers
\ccMethod{void  set_constraint(int i, bool b);}
{sets the edge between \ccVar\ and its neighbor \ccVar .\ccc{neighbor(i)}
 as a constrained or unconstrained edge according to \ccc{b}.}

\ccMethod{void set_constraints(bool c0, bool c1, bool c2);}
{sets the status (constrained or unconstrained) of the three
edges of \ccVar.}

\ccMethod{void reorient();}
{Changes the orientation of \ccVar\ by exchanging \ccc{vertex(0)}
with \ccc{vertex(1)} and \ccc{neighbor(0)} with \ccc{neighbor(1)}
and the corresponding constrained status.}
\ccMethod{void ccw_permute();}
{performs a counterclockwise permutation of the
 vertices, neighbors and constrained status of  \ccVar.}
\ccMethod{ void cw_permute();}
{performs a clockwise permutation of the
 vertices and neighbors and constrained status of \ccVar.}



\ccHeading{Miscelleanous}
\ccMethod{bool is_valid();}
{tests the validity of face  \ccVar\ 
as a face of a plain triangulation
and additionally checks
if the edges of \ccVar\ are consistently marked
as constrained or unconstrained edges
in face \ccVar and its neighbors.}
\end{ccAdvanced}



\ccHasModels

\ccc{CGAL::Constrained_triangulation_face_base_2<Traits>}


\ccSeeAlso
\ccc{Triangulation_face_base_2_concept} \\
\ccc{CGAL::Constrained_triangulation_2<Traits,Tds>} \\
\ccc{CGAL::Constrained_triangulation_face_base_2<Traits>}


\end{ccRefConcept}

% +------------------------------------------------------------------------+
%%RefPage: end of main body, begin of footer
% EOF
% +------------------------------------------------------------------------+

