% +------------------------------------------------------------------------+
% | Reference manual page: Triangulation_2::Vertex.tex
% +------------------------------------------------------------------------+
% | 29.03.2000   Mariette Yvinec
% | Package: Triangulation
% | 
\RCSdef{\RCSTriangulation::VertexRev}{$Revision$}
\RCSdefDate{\RCSTriangulation::VertexDate}{$Date$}
% |
%%RefPage: end of header, begin of main body
% +------------------------------------------------------------------------+


\begin{ccRefClass}[Triangulation_2<Traits,Tds>::]{Vertex}  %% add template arg's if necessary

%% \ccHtmlCrossLink{}     %% add further rules for cross referencing links
%% \ccHtmlIndexC[class]{} %% add further index entries

\ccDefinition
  
The class \ccRefName\ stores a point, gives access to an incident face
and provides circulators to visit all incident faces and edges
and all adjacent vertices.

\ccInclude{CGAL/Triangulation_2.h}

\ccInheritsFrom{\ccc{Tds::Vertex}}



\ccTypes
The class \ccRefName\ defines the same types 
as the class \ccc{CGAL::Triangulation_2<Traits,Tds>} 
\ccRefPage{CGAL::Triangulation_2<Traits,Tds>}
except for the iterators and the 
\ccc{Line_face_circulator} which are not needed.

\ccCreationVariable{v}  %% choose variable name
\begin{ccAdvanced}
\ccCreation

For triangulation algorithms designed by the user, vertices need to be
explicitly constructed.  If the modification of the triangulation is
done through a member of the triangulation class, there is no need for these
functions.

\ccConstructor{Triangulation_2::Vertex();}{default constructor.
The geometric information  is initialized by the default constructor of the class \ccc{Point}. The pointer to the incident face is initialized to \ccc{NULL}.}

\ccConstructor{Triangulation_2::Vertex(const Point& p);}
{Introduces a vertex \ccVar, and initializes the geometric information
with the point \ccc{p}.
The pointer to the incident face is initialized with \ccc{NULL}.}

\ccConstructor{Triangulation_2::Vertex(const Point& p,
                      Face_handle f);}
{Introduces a vertex v \ccVar, and initializes the geometric information and 
the pointer to the incident face.}
\end{ccAdvanced}

\ccAccessFunctions
\ccMethod{Point point() const;}
{Returns  the geometric information of \ccVar.}

\ccMethod{Face_handle face() const;}
{Returns a face of the triangulation having  \ccVar\ as vertex.}

\begin{ccAdvanced}
\ccHeading{Setting}
\ccMethod{void set_face(Face_handle f); }{Sets the incident face to \ccc{f}.}
\end{ccAdvanced}

\ccHeading{Miscellaneous}
\ccMethod{int degree() const;}
{Returns the degree of \ccVar\ in the triangulation.}
\ccGlue
\ccMethod{Vertex_handle handle() const;}
{Returns an handle to the vertex.}
\ccGlue
\ccMethod{bool is_valid();} 
{Check the validity of a vertex:
i.e. the pointer to the incident face  and call the \ccc{is_valid()}
function of the base class to perform any geometric test provided by the user
therein.}

\ccHeading{Traversal of the Adjacent Vertices, Incident Edges and Faces.}

Three circulator classes allow to traverse the edges, and faces
incident to a given vertex or  the adjacent vertices.
 These circulators are bidirectional
and their value types are respectively \ccc{Vertex}, \ccc{Edge}
and \ccc{Face}.
The  \ccc{operator++} moves the circulator
counterclockwise around the vertex 
and  the \ccc{operator--} moves the circulator
clockwise. 

A face circulator is invalidated by any modification of the face it
points to. An edge circulator is invalidated
by any modification of one of the two faces that are incident to the edge
pointed to.  A vertex circulator that turns around vertex \ccc{v}
and and points  to a vertex \ccc{w}, is invalidated
by any modification of the faces incident to the edge \ccc{vw}.




\ccMethod{Face_circulator
          incident_faces(Face_handle f=Face_handle()) const;}
{A circulator for the  incident faces 
that refers to face \ccc{f} or to an arbitrary face
incident to \ccc{v}  if \ccc{f} is omitted.
\ccPrecond Face \ccc{f} is incident to vertex \ccc{v}.}

\ccMethod{Edge_circulator
          incident_edges(Face_handle f=Face_handle()) const;}
{A circulator for the incident edges that refers to an abitrary edge incident
to \ccc{v} if \ccc{f} is omitted. Otherwise the circulator refers
to the first edge of \ccc{f}
incident
to \ccc{v} in counterclockwise order around \ccc{v}.
\ccPrecond Face \ccc{f} is incident to vertex \ccc{v}. }

\ccMethod{Vertex_circulator
          incident_vertices(Face_handle f=Face_handle()) const;}
{A circulator for the adjacent vertices.
If \ccc{f} is omitted, the circulator begins with
 an abitrary vertex incident
to \ccc{v}, otherwise it begins with the
vertex of face \ccc{f} that, 
in counterclockwise order around \ccc{v},
is the first vertex of \ccc{f} incident to \ccc{v}.
\ccPrecond Face \ccc{f} is incident to vertex \ccc{v}.}


\ccSeeAlso
\ccc{CGAL::Triangulation_cw_ccw_2} \\
\ccc{CGAL::Triangulation_2<Traits,Tds>} \\
\ccc{CGAL::Triangulation_2<Traits,Tds>::Face} \\
\ccc{TriangulationTraits_2} \\
\ccc{TriangulationDataStructure}
\ccc{TriangulationDataStructure::Vertex}
\ccc{TriangulationVertexBase_2}




\end{ccRefClass}

% +------------------------------------------------------------------------+
%%RefPage: end of main body, begin of footer
% EOF
% +------------------------------------------------------------------------+

