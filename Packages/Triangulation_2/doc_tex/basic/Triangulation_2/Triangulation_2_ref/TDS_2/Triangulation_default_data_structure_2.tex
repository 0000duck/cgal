% +------------------------------------------------------------------------+
% | Reference manual page: Triangulation_default_data_structure_2.tex
% +------------------------------------------------------------------------+
% | 07.04.2000   Author
% | Package: Package
% | 
\RCSdef{\RCSTriangulationdefaultdatastructureRev}{$Revision$}
\RCSdefDate{\RCSTriangulationdefaultdatastructureDate}{$Date$}
% |
%%RefPage: end of header, begin of main body
% +------------------------------------------------------------------------+


\begin{ccRefClass}{Triangulation_default_data_structure_2<Traits,Vb,Fb>} 
 %% add template arg's if necessary

%% \ccHtmlCrossLink{}     %% add further rules for cross referencing links
%% \ccHtmlIndexC[class]{} %% add further index entries
\ccCreationVariable{tds}
\ccDefinition
  
The class \ccRefName\ is the default model for the concept
\ccc{TriangulationDataStructure_2}.
The class \ccRefName 
is highly economic with respect to memory space but its use is
restricted
to planar embedded triangulation.

As required this class has two template parameters \ccc{Vb} anf \ccc{Fb}
which have 
to be models for respectively the
\ccc{TriangulationVertexBase_2} and 
the \ccc{TriangulationFaceBase_2} concepts.

In addition, the class \ccClassTemplateName\ has a first template parameter
which is a geometric traits class. This may be surprising because
the triangulation data structure is supposed to deal only with the combinatorial
aspect of the triangulation and not with any geometric embedding
The reason for that is the following.
The class \ccRefName\ does not use any additional data structure
such as a list or a vector to act as a container for faces and vertices.
The iterators which allows to visit all faces and vertices of the
triangulation
data structure
is implemented using  an implicit tree structure over the faces
as described by
 de Berg, van Oostrum, and Overmars, 
in Proc.\ 12th Annual Symp.\ on Comput.\ Geom.,
1996, pages C5--C6. This tree structure is  based on the planar
geometric embedding
the triangulation. Each face 
 can find its parent 
and its children using only simple comparisons on the
coordinates of the points embedding its vertices.
Thus the tree structure may remain implicit 
and does not require any additional memory. 

The requirements concerning the geometric traits \ccc{Traits} of
\ccc{Triangulation_default_data_structure_2}
 are very light and form a subset of the concept
\ccc{TriangulationTraits_2}.
This class is required  to provide a type \ccc{Point}
and the coordinate comparison functions \ccc{compare_x(Point p0, Point p1)} and
\ccc{compare_y(Point p0, Point p1)}. The point type
defined by the geometric traits class of the triangulation data structure
has to be the same 
as the point type defined by the geometric traits of the triangulation.
This is achieved if the same model is used for both traits classes
which is recommended but not compulsory.


\ccInclude{Triangulation_default_data_structure_2.h}

\ccIsModel

\ccc{TriangulationDataStructure_2}


\ccSeeAlso

\ccc{TriangulationDataStructure_2},
\ccc{CGAL::Triangulation_data_structure_using_list_2<Vb,Fb>}.


\end{ccRefClass}

% +------------------------------------------------------------------------+
%%RefPage: end of main body, begin of footer
% EOF
% +------------------------------------------------------------------------+

