% +------------------------------------------------------------------------+
% | Reference manual page: Triangulation_face_base_2.tex
% +------------------------------------------------------------------------+
% | 11.04.2000   Author
% | Package: Package
% | 
\RCSdef{\RCSTriangulationfacebaseRev}{$Revision$}
\RCSdefDate{\RCSTriangulationfacebaseDate}{$Date$}
% |
%%RefPage: end of header, begin of main body
% +------------------------------------------------------------------------+


\begin{ccRefClass}{Triangulation_face_base_2<Traits>}  %% add template arg's if necessary

%% \ccHtmlCrossLink{}     %% add further rules for cross referencing links
%% \ccHtmlIndexC[class]{} %% add further index entries
\ccCreationVariable{fb}
\ccDefinition
  
The class \ccRefName\ is the default  model for the concept
\ccc{Triangulation_face_base_2_concept}.
It is  templated by a geometric traits class.
Using for this traits class, the geometric traits class used for the triangulation class
is strongly recommended. 

These default base class can be used directly or can serve as a base to derive
other base classes with some additional attribute (a color for example)
tuned for a specific application.

\ccInclude{Triangulation_face_base_2.h}

\ccIsModel
\ccc{Triangulation_face_base_2_concept}

\ccSeeAlso
\ccc{Triangulation_face_base_2_concept}

\ccIndexHintAsSubitem[C]{Triangulation_2}{customizing faces or 
vertices}
\ccIndexHintAsSubitem[C]{Triangulation_face_base_2}{customizing}

\ccExample
The following example derives a new \ccc{Face_base} class from the default
one and add a color to the faces of the triangulation. 
The face of the triangulation data structure
and the face of the triangulation will inherit the new data member 
and its functionality.
Any kind of additional fonctionality can thus be added to faces or vertices of a triangulation 
as long as this functionality  does not involve additional pointers to vertices or faces
(because the base classes use only void* pointer and have no knowledge
of the vertex or face types.).


\ccIncludeExampleCode{Triangulation_2/colored_face.C}



\end{ccRefClass}

% +------------------------------------------------------------------------+
%%RefPage: end of main body, begin of footer
% EOF
% +------------------------------------------------------------------------+

