% +------------------------------------------------------------------------+
% | Reference manual page: Triangulation_vertex_base_2.tex
% +------------------------------------------------------------------------+
% | 11.04.2000   Author
% | Package: Package
% | 
\RCSdef{\RCSTriangulationvertexbaseRev}{$Revision$}
\RCSdefDate{\RCSTriangulationvertexbaseDate}{$Date$}
% |
%%RefPage: end of header, begin of main body
% +------------------------------------------------------------------------+


\begin{ccRefClass}{Triangulation_vertex_base_2<Traits>}  %% add template arg's if necessary

%% \ccHtmlCrossLink{}     %% add further rules for cross referencing links
%% \ccHtmlIndexC[class]{} %% add further index entries

\ccDefinition
  
The class \ccRefName\ is the default model for the
\ccc{Triangulation_vertex_base_2} concept.
It is  templated by a geometric traits class which provide the type
\ccc{Point}. It is strongly recommanded to instantiate this
traits class with the model used for the triangulation traits class.
This ensures that the point type defined by \ccRefName
is the same as the point type defined by 
the triangulation.

\ccInclude{Triangulation_vertex_base_2.h}

\ccIsModel

\ccc{Triangulation_vertex_base_2}



\ccSeeAlso

\ccc{Triangulation_vertex_base_2}


\end{ccRefClass}

% +------------------------------------------------------------------------+
%%RefPage: end of main body, begin of footer
% EOF
% +------------------------------------------------------------------------+

