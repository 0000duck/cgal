% +------------------------------------------------------------------------+
% | Reference manual page: Triangulation_2.tex
% +------------------------------------------------------------------------+
% | 29.03.2000   Mariette Yvinec
% | Package: Triangulation
% | 
\RCSdef{\RCSTriangulationRev}{$Revision$}
\RCSdefDate{\RCSTriangulationDate}{$Date$}
% |
%%RefPage: end of header, begin of main body
% +------------------------------------------------------------------------+

\clearpage
\section{Reference pages for Triangulation}

\subsection*{Definition}
A 
triangulation is a 2-dimensional simplicial complex which is pure
connected and without singularities. Thus a triangulation
can be viewed as a collection of triangular faces,
such that two faces either have an empty intersection or share an edge or a vertex.
The absence of singularities means that each edge belongs to 
at most two triangles and that each vertex belongs  to a set  of faces
whose union forms a topological disk.

Each face of a triangulation can be given an orientation
(clockwise or counterclockwise) which in turn induces an orientation
on the edges incident to that face. The orientation of two adjacent
faces are said to be consistent if they induce
opposite orientations on their common incident edge.
A triangulation is said to be orientable if 
the orientation of each face can be chosen in such a way
that all pairs of incident faces have consistent orientations. 

In this chapter we present a framework to represent
orientable triangulations which  may  be embedded 
in a plane or in a higher dimensional space.
Any such  triangulation can be described as a set of  vertices and
triangular faces,  with incidence and adjacence relations:
two facets or two vertices are said to be {\em adjacent}
 (or neighboring) if they are incident to the same edge,
a facet and a vertex are said to be incident if they are incident to the same
edge.

\subsection*{Design Rationale}
Because a triangulation is merely a set of
triangular faces with constant size complexity,
triangulations are not implemented
as a layer on top of a planar map.
\cgal\ uses a proper internal
representation of triangulations based on faces and vertices
rather than on edges. Such a representation
allows to save storage space and results in faster
algorithms.

Thus the basic elements of the representation are vertices and faces.
Each triangular face gives access to its three incident vertices 
and to its three adjacent faces. 
Each vertex gives access to one of its incident faces
and through that face to the circular list of its incident faces.
The edges are not explicitely represented, they are only represented 
through the adjacencies relations of two faces.


\begin{figure}
\begin{ccTexOnly}
\begin{center}
\input{three_levels.ltex}
\end{center}
\end{ccTexOnly}
\caption{The three layer design of triangulations.
\label{Triangulation_ref_Fig_three_levels}}
\begin{ccHtmlOnly}
<CENTER>
<img border=0 src=three_levels.gif align=center alt="Three_levels">
</CENTER>
\end{ccHtmlOnly}
\end{figure}



The triangulations in \cgal\ are represented
by a three layer structure  analog to the design used for polyhedral
surfaces, see Figure~\ref{Triangulation_ref_Fig_three_levels}.
In the bottom layer, the base classes for vertices and faces
store some 
geometric informations such as the coordinate of vertices 
and any other attribute (such as color, constraint edges etc.)
needed by the application.
The base classes handle
incidence and adjacency relations in term of \ccc{void*} pointers.
The use of \ccc{void*} pointers in the bottom layer allows 
to easily change one of the base
class, to deal with an extra attribute like a color for example,
without having to change the rest of the structure. 
The advantages of strong type 
checking is reestablished in the next layer 
where the \ccc{triangulation data structure} can be thought 
of as a container for faces and vertices which can take care
of all the combinatorial aspects of the triangulation.
The {triangulation data structure}
implements
adjacency and incidence relations with type safe pointer operations
and maintains the combinatorial integrity of the triangulation.
For that purpose, the {triangulation data structure} defines its own face and vertex
classes which are derived
 from the corresponding 
base classes
so that geometric and additional information on vertices and faces 
are simply inherited.
  At last, at the top layer, the \ccc{triangulation class}
implements the user interface with the triangulation. 
This classes offer to the user
the high level functionalities that can be expected from a triangulation:
insertion  or removal of a vertex, traversal of the faces,
enumeration of the vertices,
traversal of the   faces incident to a given vertex, location of a given point etc.
The {triangulation class} defines its own 
vertex and face classes
derived from the corresponding class of the {triangulation data structure}.
The vertices and faces of the {triangulation class}
are only accessed through high levels concepts such as 
handles, iterators, or circulators,
according to the required functionalities of the access.
The top layer {triangulation class}
is responsible for the geometric embedding of the triangulation
and comes in different flavors according to the kind of triangulation represented:
planar triangulation of a set of points, Delaunay or regular or constrained
 triangulations etc.

The triangulation classes of \cgal\ depends on two template parameters.
The first template parameter stands for
 a geometric traits class which is assumed to provide
the geometric objects (points, segments and triangles) 
forming  the triangulation and the geometric predicates on those objects.
The second template parameter stands for a model
of  triangulation data
structure acting as a container for faces and vertices
while  taking care of the combinatorial aspects of the triangulation. 
The triangulation data structure class is itself a template
class parametrized by the base classes for faces and vertices.


\subsection*{Concepts}
\ccRefConceptPage{Triangulation_traits_2} \\
\ccRefConceptPage{Delaunay_triangulation_traits_2} \\
\ccRefConceptPage{Regular_triangulation_traits_2} \\

\ccRefConceptPage{TriangulationDataStructure_2}\\
\ccRefConceptPage{TriangulationDataStructure_2::Face}\\
\ccRefConceptPage{TriangulationDataStructure_2::Vertex}\\
\ccRefConceptPage{TriangulationFaceBase_2}\\
\ccRefConceptPage{TriangulationVertexBase_2}\\
\ccRefConceptPage{ConstrainedTriangulationFaceBase_2}\\
\ccRefConceptPage{RegularTriangulationFaceBase_2}\\

\subsection*{Classes}
\ccRefIdfierPage{CGAL::Triangulation_2<Traits,Tds>}\\
\ccRefIdfierPage{CGAL::Delaunay_triangulation_2<Traits,Tds>}\\
\ccRefIdfierPage{CGAL::Regular_triangulation_2<Traits,Tds>}\\
\ccRefIdfierPage{CGAL::Constrained_triangulation_2<Traits,Tds>}\\
\ccRefIdfierPage{CGAL::Constrained_triangulation_demo_2<Traits,Tds>}\\
\ccRefIdfierPage{CGAL::Constrained_Delaunay_triangulation_2<Traits,Tds>}\\
\ccRefIdfierPage{CGAL::Triangulation_hierarchy_2<Tr>} \\

\ccRefIdfierPage{CGAL::Triangulation_euclidean_traits_2<R>}\\
\ccRefIdfierPage{CGAL::Triangulation_euclidean_traits_xy_3<R>}\\
\ccRefIdfierPage{CGAL::Regular_triangulation_euclidean_traits_2<Rep,Weight>}\\

\ccRefIdfierPage{CGAL::Triangulation_data_structure_using_list_2<Vb,Fb>}\\
\ccRefIdfierPage{CGAL::Triangulation_default_data_structure_2<Traits,Vb,Fb>}\\

\ccRefIdfierPage{CGAL::Triangulation_2<Traits,Tds>::Face}\\
\ccRefIdfierPage{CGAL::Triangulation_2<Traits,Tds>::Vertex}\\
\ccRefIdfierPage{CGAL::Triangulation_face_base_2<Traits>}\\
\ccRefIdfierPage{CGAL::Triangulation_vertex_base_2<Traits>}\\
\ccRefIdfierPage{CGAL::Regular_triangulation_face_base_2<Traits>}\\
\ccRefIdfierPage{CGAL::Constrained_triangulation_face_base_2<Traits>}\\

\ccRefIdfierPage{CGAL::Weighted_point<Pt,Wt>}\\
\ccRefIdfierPage{CGAL::Triangulation_cw_ccw_2}\\


\subsection*{Enum}
\ccRefIdfierPage{CGAL::Triangulation_2<Traits,Tds>::Locate_type}



% +------------------------------------------------------------------------+
%%RefPage: end of main body, begin of footer
% EOF
% +------------------------------------------------------------------------+
