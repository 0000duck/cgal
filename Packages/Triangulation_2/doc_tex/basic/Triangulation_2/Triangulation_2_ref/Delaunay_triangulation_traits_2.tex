% +------------------------------------------------------------------------+
% | Reference manual page: Delaunay_triangulation_traits_2.tex
% +------------------------------------------------------------------------+
% | 12.04.2000   Author
% | Package: Package
% | 
\RCSdef{\RCSDelaunaytriangulationtraitsRev}{$Revision$}
\RCSdefDate{\RCSDelaunaytriangulationtraitsDate}{$Date$}
% |
%%RefPage: end of header, begin of main body
% +------------------------------------------------------------------------+


\begin{ccRefConcept}{Delaunay_triangulation_traits_2}

%% \ccHtmlCrossLink{}     %% add further rules for cross referencing links
%% \ccHtmlIndexC[concept]{} %% add further index entries

\ccDefinition
  
The concept \ccRefName\ does this and that.
In addition to the requirements 
of the concept \ccc{Triangulation_traits_2}
described \ccRefPage{Triangulation_traits_2},
the concept \ccRefName\
has to  provide a \ccc{side_of_oriented_circle} test
and some additional types.
The \ccc{side_of_oriented_circle} test is the test used to maintain the
empty circle property and actually defines the triangulation.
The additional types \ccc{Line}, \ccc{Direction}
and \ccc{Ray} are used to describe the dual Voronoi diagram.
The additional \ccc{Distance} type is only used by the
\ccc{nearest_vertex(..)} query and the requirements for this type
are given below.


\ccTypes
\ccNestedType{Line}
\ccGlue
\ccNestedType{Ray}{}
\ccGlue
\ccNestedType{Direction}{}
\ccNestedType{Distance}{Needed only if the \ccc{nearest_neighbor()} 
function is called.}

Each of the three classes \ccc{Line}, \ccc{Direction} and 
\ccc{Ray}  must know the two others.
In addition, the \ccc{Line} class is required to have the two
following member functions:
\begin{ccClass}{Line}

\ccCreationVariable{l}
\ccThree{Direction }{l.direction()x}{}
\ccMethod{Line opposite();}{the same line with opposite orientation.}
\ccGlue
\ccMethod{Direction direction();}{the direction of the line.}
\end{ccClass}
The \ccc{Ray} class must define a \ccc{Point} class which is the same
as \ccc{Traits::Point} 
is required to have the following constructor:

\begin{ccClass}{Ray}
\ccCreationVariable{r}
\ccConstructor{Ray(Traits::Point p, Direction d);}{}
\end{ccClass}


\ccCreation
\ccCreationVariable{traits}  %% choose variable name

\ccConstructor{Delaunay_triangulation_traits_2();}{default constructor.}

\ccPredicates
\ccMethod{Oriented_side side_of_oriented_circle(const Point& p0,
                                                     const Point& p1,
                                                     const Point& p2,
                                                     const Point& test);}
{Gives the \ccc{Oriented_side} of point \ccc{test} with respect to
the oriented circumscribing circle of  points \ccc{p0, p1} and \ccc{p2}.
This answers the \ccc{in_circle}  test if the \ccc{Orientation}
of  Points \ccc{p0, p1} and \ccc{p2} is known,
which is the case for the points associated with the three
vertices of a triangulation face.}


\ccOperations
\ccMethod{Point circumcenter(const Point &p, const Point &q, const
Point &r) const;} { Returns the center point equidistant from \ccc{p}, \ccc{q} and
\ccc{r}. This method is required only if the duality functions are called.}

\ccMethod{Line bisector(const Segment &s);}
{Returns the bisector line of segment s. This method is required only if duality functions
are called.}

\ccHasModels
\ccc{CGAL::Triangulation_euclidean_traits_2<Rep>},
\ccc{CGAL::Triangulation_euclidean_traits_xy_3<Rep>},
\ccc{CGAL::Triangulation_euclidean_traits_yz_3<Rep>},
\ccc{CGAL::Triangulation_euclidean_traits_zx_3<Rep>}.


\end{ccRefConcept}

% +------------------------------------------------------------------------+
%%RefPage: end of main body, begin of footer
% EOF
% +------------------------------------------------------------------------+

