% +------------------------------------------------------------------------+
% | Reference manual page: Triangulation_2::Face.tex
% +------------------------------------------------------------------------+
% | 29.03.2000   Mariette Yvinec
% | Package: Triangulation
% | 
\RCSdef{\RCSTriangulation::FaceRev}{$Revision$}
\RCSdefDate{\RCSTriangulation::FaceDate}{$Date$}
% |
%%RefPage: end of header, begin of main body
% +------------------------------------------------------------------------+


\begin{ccRefClass}{Triangulation_2<Traits,Tds>::Face}  %% add template arg's if necessary

%% \ccHtmlCrossLink{}     %% add further rules for cross referencing links
%% \ccHtmlIndexC[class]{} %% add further index entries

\ccDefinition
  
A face of a triangulation gives access to its three
vertices indexed 0,1, and 2 in counterclockwise order
and to its three adjacent faces, also called neighbors.
The neighbors are indexed in such a way that neighbor $i$ lies
opposite to vertex i. 

\ccInclude{CGAL/Triangulation_2.h}

\ccInheritsFrom{\ccc{Tds::Face}}


\ccTypes
The class  \ccRefName\  defines the same types 
as the class \ccc{Triangulation_2<Traits,Tds>} except the iterators and circulators
which are not needed.

\ccCreationVariable{f}  %% choose variable name
\begin{ccAdvanced}
\ccCreation

For user defined triangulation algorithms, faces need to
be explicitly constructed and linked to their neighbors.

\ccConstructor{Triangulation_2::Face()}
{Introduces a variable \ccVar\ and initializes all vertices and neighbors 
 with \ccc{NULL}.}

\ccConstructor{Triangulation_2::Face(Vertex_handle v0,
                                       Vertex_handle v1,
                                           Vertex_handle v2);}
{Introduces a variable \ccVar, and initializes the vertices. The 
neighbors are initialized with \ccc{NULL}.}

\ccConstructor{Triangulation_2::Face(Vertex_handle v0,
                    Vertex_handle v1,
                    Vertex_handle v2,
                     {Face_handle} n0,
                     {Face_handle} n1,
                     {Face_handle} n2);}
{Introduces a variable \ccVar, and initializes the vertices and the neighbors.}



\ccHeading{Setting}


\ccThree{Vertex_handle}{f.has_vertex( Vertex_handle v, int& i)x}{}
\ccMethod{void set_vertex(int i, Vertex_handle v);}
{Sets vertex \ccc{i} to be \ccc{v}.
\ccPrecond $0\leq i \leq 2$.
}
\ccGlue
\ccMethod{void set_neighbor(int i,  {Face_handle} n);}
{Sets neighbor \ccc{i} to be \ccc{n}.
\ccPrecond $0\leq i \leq 2$.
}
\ccGlue
\ccMethod{void set_vertices();}{Sets vertices to \ccc{NULL}}
\ccGlue
\ccMethod{void set_neighbors();}{Sets neighbors to \ccc{NULL}}
\ccGlue
\ccMethod{void set_vertices(
Vertex_handle v0, Vertex_handle v1, Vertex_handle v2);}{}
\ccGlue
\ccMethod{void set_neighbors(
 {Face_handle} n0,  {Face_handle} n1,  {Face_handle} n2);}{}
\end{ccAdvanced} 


\ccHeading{Vertex Access Functions}

\ccMethod{Vertex_handle vertex(int i) const;}
{Returns  the vertex \ccc{i} of \ccVar.
\ccPrecond $0\leq i \leq 2$.}
\ccGlue
\ccMethod{int index(Vertex_handle v) const;}
{Returns the index of vertex \ccc{v} in \ccVar. \ccPrecond \ccc{v} is
a vertex of \ccVar}
\ccGlue
\ccMethod{bool has_vertex(Vertex_handle v) const;}
{Returns \ccc{true} if  \ccc{v} is a vertex of \ccVar.}
\ccGlue
\ccMethod{bool has_vertex(Vertex_handle v, int& i) const;}
{Returns \ccc{true} if  \ccc{v} is a vertex of \ccVar, and
 computes the index \ccc{i} of the vertex.}



\ccHeading{Neighbor Access Functions}

The neighbor with index \ccc{i} is the neighbor which is opposite
to the vertex with index \ccc{i}.


\ccMethod{{Face_handle} neighbor(int i) const;}
{Returns  the neighbor \ccc{i} of \ccVar.
\ccPrecond $0\leq i \leq 2$.
}
\ccGlue
\ccMethod{int index( {Face_handle} n) const;}
{Returns the index of face \ccc{n}.
\ccPrecond \ccc{n} is a neighbor of \ccVar.}
\ccGlue
\ccMethod{bool has_neighbor({Face_handle} n) const;}
{Returns \ccc{true} if \ccc{n} is a neighbor of \ccVar.}
\ccGlue
\ccMethod{bool has_neighbor({Face_handle} n, int& i) const;}
{Returns \ccc{true} if \ccc{n} is a neighbor of \ccVar,  and
compute the index \ccc{i} of the neighbor.}

\ccHeading{Other Access Functions}
\ccMethod{int mirror_index(int i) const;}
{Returns the index of the face \ccc{f} as a neighbor
of its neighbor \ccc{f.neighbor(i)}, or equivalently the index
of the vertex opposite to \ccc{f} in \ccc{f.neighbor(i)}.}
\ccGlue
\ccMethod{Vertex_handle mirror_vertex(int i) const;}
{Returns the vertex of \ccc{f.neighbor(i)}
which is opposite to \ccc{f}.}



\ccHeading{Miscellaneous} 
 
\ccMethod{{Face_handle} handle() const;}
{Returns a handle to the face.}
\ccGlue
\ccMethod{bool is_valid();} 
{Check the validity of an individual face:
i.e. the pointers to the incident vertices
and neighboring faces
 and call the \ccc{is_valid()}
function of the base class to perform any geometric test provided by the user
therein.}


\ccSeeAlso
\ccc{CGAL::Triangulation_cw_ccw_2} \\
\ccc{CGAL::Triangulation_2<Traits,Tds>} \\
\ccc{CGAL::Triangulation_2<Traits,Tds>::Vertex} \\
\ccc{Triangulation_data_structure} \\
\ccc{Triangulation_data_structure::Face} \\
\ccc{Triangulation_face_base_2} \\
\end{ccRefClass}

% +------------------------------------------------------------------------+
%%RefPage: end of main body, begin of footer
% EOF
% +------------------------------------------------------------------------+

