% +------------------------------------------------------------------------+
% | Reference manual page: Triangulation_data_structure_2::Vertex.tex
% +------------------------------------------------------------------------+
% | 07.04.2000   Author
% | Package: Package
% | 
\RCSdef{\RCSTriangulationdatastructure::VertexRev}{$Revision$}
\RCSdefDate{\RCSTriangulationdatastructure::VertexDate}{$Date$}
% |
%%RefPage: end of header, begin of main body
% +------------------------------------------------------------------------+


\begin{ccRefConcept}{TriangulationDataStructure_2::Vertex}

%% \ccHtmlCrossLink{}     %% add further rules for cross referencing links
%% \ccHtmlIndexC[concept]{} %% add further index entries

\ccDefinition
  
The concept \ccRefName\ describes the vertex class of a
\ccc{Triangulation_data_structure_2}.
The vertex stores a point and a pointer to an incident face.

To ensure all the \textbf{flexibility} of class 
\ccc{CGAL::Triangulation_2<Traits,Tds>}
described in \ccRefPage{CGAL::Triangulation_2<Traits,Tds>}
and in Chapter~\ref{Chapter_2D_Triangulations}, a model of 
\ccRefName derives from a template parameter
\ccc{Vb} to be instantiated with a model of 
\ccc{Triangulation_vertex_base_2}.


\ccTypes
The class \ccRefName\ defines the same types as 
the triangulation data structure 
except the iterators.

\ccCreationVariable{v}  %% choose variable name

\begin{ccAdvanced}
\ccCreation

In order to obtain new vertices or destruct unused vertices, the user must
call the \ccc{create_vertex()} and \ccc{delete_vertex()} methods of the
triangulation data structure.

%\ccConstructor{TriangulationDataStructure_2::Vertex();}{default
%constructor.
% The geometric information is
% initialized by the default constructor of class \ccc{Point}.
%The pointer to the incident face is initialized with \ccc{NULL}.}

%\ccConstructor{TriangulationDataStructure_2::Vertex(const Point&
%p);}
%{introduces a vertex \ccVar, and initializes the geometric information.
%The pointer to the incident face is initialized with \ccc{NULL}.}

%\ccConstructor{TriangulationDataStructure_2::Vertex(const Point& p,
%                      Face_handle f);}
%{introduces a vertex \ccVar, and initializes the geometric information and 
%the pointer to the incident face.}

\ccHeading{Setting}

\ccMethod{void set_point(const Point& p);}
{sets the geometric information to \ccc{p}.}
\ccGlue
\ccMethod{void set_face(Face_handle f);}
{sets the incident face to \ccc{f}.}


\end{ccAdvanced}

\ccHeading{Access Functions}

\ccMethod{Point point() const;}
{returns  the geometric information of \ccVar.}
\ccGlue
\ccMethod{Face_handle face() const;}
{returns a face of the triangulation having  \ccVar\ as vertex.}

\ccHeading{Traversal of the Adjacent Vertices, Incident Edges and
Faces.}
The access to the circulators describing faces, edges and vertices
incident or adjacent to a given vertices from the \ccc{Vertex} class is
deprecated.
Use the corresponding functions  of the \ccc{Triangulation_data_structure_2}
class (see \ccc{TriangulationDataStructure_2} ) instead.


Three circulator classes allow to traverse the edges, and faces
incident to a given vertex and the adjacent vertices.
Note that infinite as well as finite incident edges and faces are visited.
 These circulators are bidirectional
and their value types are respectively \ccc{Vertex}, \ccc{Edge}
and \ccc{Face}.
The  \ccc{operator++} moves the circulator
counterclockwise around the vertex 
and  the \ccc{operator--} moves the circulator
clockwise. 

A face circulator is invalidated by any modification of the face it
points to. An edge circulator is invalidated
by any modification of anyone of the two faces incident to the edge
pointed to.  A vertex circulator that turns around vertex \ccc{v}
and that has as value a pointer to vertex \ccc{w}, is invalidated
by any modification of anyone of the two faces incident to \ccc{v}
and \ccc{w}.

\ccMethod{Vertex_circulator
          incident_vertices(Face_handle f=NULL) const;}{\ccPrecond If the
face \ccc{f} is given, it has to be incident to \ccVar\ 
and the circulators begins with a vertex of \ccc{f} :
the vertex \ccc{f->vertex(ccw(i))} 
if \ccc{i} is the index of \ccVar\  in \ccc{f}.} 
\ccGlue
\ccMethod{Edge_circulator
          incident_edges(Face_handle f=NULL) const;}{\ccPrecond If the
face \ccc{f} is given, it has to be incident to \ccVar\ 
and the circulators begins with an edge of \ccc{f} :
the edge \ccc{(f,cw(i))}
where \ccc{i} is the index of \ccVar\  in \ccc{f}.}
\ccGlue
\ccMethod{Face_circulator
          incident_faces(Face_handle f=NULL) const;}{\ccPrecond If the
face \ccc{f} is given, it has to be incident to \ccVar\
and the crculators begins with the face
\ccc{f}.}

\ccHeading{Miscellaneous}

\ccMethod{int ccw(int i) const;}
{Returns $i+1$ modulo 3.\ccPrecond $0\leq i \leq 2$.}
\ccGlue
\ccMethod{int cw(int i) const;}
{Returns $i+2$ modulo 3.\ccPrecond $0\leq i \leq 2$.}
\ccGlue
\ccMethod{int degree() const;}
{Returns the degree of \ccVar\ in the triangulation.}


\ccHasModels
\ccc{CGAL::Triangulation_ds_vertex_2<Vb,Fb>}


\ccSeeAlso
\ccc{TriangulationDataStructure_2} \\
\ccc{TriangulationDataStructure_2::Face} \\
\ccc{TriangulationVertexBase_2}.



\end{ccRefConcept}

% +------------------------------------------------------------------------+
%%RefPage: end of main body, begin of footer
% EOF
% +------------------------------------------------------------------------+

