% +------------------------------------------------------------------------+
% | Reference manual page: TriangulationDSFaceBase_2.tex
% +------------------------------------------------------------------------+
% | 13.08.2003   Author
% | Package: Package
% | 
\RCSdef{\RCSTriangulationDSFaceBaseRev}{$Revision$}
\RCSdefDate{\RCSTriangulationDSFaceBaseDate}{$Date$}
% |
%%RefPage: end of header, begin of main body
% +------------------------------------------------------------------------+


\begin{ccRefConcept}{TriangulationDSFaceBase_2}

%% \ccHtmlCrossLink{}     %% add further rules for cross referencing links
%% \ccHtmlIndexC[concept]{} %% add further index entries

\ccDefinition
  
The concept \ccRefName\ does this and that.

\ccGeneralizes

ThisConcept \\
ThatConcept

\ccTypes

\ccNestedType{TYPE}{some nested types}

\ccCreation
\ccCreationVariable{a}  %% choose variable name

\ccConstructor{TriangulationDSFaceBase_2();}{default constructor.}

\ccOperations

\ccMethod{void foo();}{some member functions}

\ccHasModels

\ccc{Some_class},
\ccc{Some_other_class}.

\ccSeeAlso

Some\_other\_concept,
\ccc{some_other_function}.

\ccExample

A short example program.
Instead of a short program fragment, a full running program can be
included using the 
\verb|\ccIncludeExampleCode{examples/Package/TriangulationDSFaceBase_2_prog.C}| 
macro. The program example would be part of the source code distribution and
also part of the automatic test suite.

\begin{ccExampleCode}
void your_example_code() {
}
\end{ccExampleCode}

%% \ccIncludeExampleCode{examples/Package/TriangulationDSFaceBase_2_prog.C}

\end{ccRefConcept}

% +------------------------------------------------------------------------+
%%RefPage: end of main body, begin of footer
% EOF
% +------------------------------------------------------------------------+

