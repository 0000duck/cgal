% +------------------------------------------------------------------------+
% | Reference manual page: Triangulation_ds_vertex_base_2.tex
% +------------------------------------------------------------------------+
% | 13.08.2003   Author
% | Package: Package
% | 
\RCSdef{\RCSTriangulationdsvertexbaseRev}{$Revision$}
\RCSdefDate{\RCSTriangulationdsvertexbaseDate}{$Date$}
% |
%%RefPage: end of header, begin of main body
% +------------------------------------------------------------------------+


\begin{ccRefClass}{Triangulation_ds_vertex_base_2<Tds>}  %% add template arg's if necessary

%% \ccHtmlCrossLink{}     %% add further rules for cross referencing links
%% \ccHtmlIndexC[class]{} %% add further index entries

\ccDefinition

The class \ccc{Triangulation_ds_vertex_base_3} can be used as the base vertex
for a \ccc{Triangulation_data_structure_2}, it is a model of the concept
\ccc{TriangulationDSVertexBase_2}.

This base class can be used directly or can serve as a base to derive
other base classes with some additional attributes (a color for
example) tuned for a specific application.

Note that if the \ccc{Triangulation_data_structure_2} 
is used as a parameter of a
geometric triangulation, there are additionnal geometric requirements
to be fulfilled by the vertex base class,
and \ccc{Triangulation_ds_vertex_base_3} cannot be plugged in.


\ccInclude{CGAL/Triangulation_ds_vertex_base_2.h}

\ccIsModel

\ccc{TriangulationDSVertexBase_2}

\ccSeeAlso

\ccc{CGAL::Triangulation_vertex_base_2<Traits,Vb>}\\
\ccc{CGAL::Triangulation_ds_face_base_2<Tds>}
  

\end{ccRefClass}

% +------------------------------------------------------------------------+
%%RefPage: end of main body, begin of footer
% EOF
% +------------------------------------------------------------------------+

