% +------------------------------------------------------------------------+
% | Reference manual page: Regular_face_base.tex
% +------------------------------------------------------------------------+
% | 12.04.2000   Author
% | Package: Package
% | 
\RCSdef{\RCSRegularfacebaseRev}{$Revision$}
\RCSdefDate{\RCSRegularfacebaseDate}{$Date$}
% |
%%RefPage: end of header, begin of main body
% +------------------------------------------------------------------------+


\begin{ccRefConcept}{RegularTriangulationVertexBase_2}

%% \ccHtmlCrossLink{}     %% add further rules for cross referencing links
%% \ccHtmlIndexC[concept]{} %% add further index entries

\ccDefinition
  
The regular triangulation of a set of weighted point does not
necessarily
have one vertex for each of the input points. Some of the input
weigthed points have no cell in the dual power diagrams
and therefore do not correspond to a vertex of the regular
triangulation.
Those weighted point are said to be {\it hidden} points.
A  point which is hidden at a given time may appear later as a vertex of
the regular triangulation upon removal on some other weighted point.
Therefore, hidden points have to be stored somewhere.
A hidden point can appear as vertex of the triangulation
only when the
two dimensional face where its point component is located
(the face which hides it)
is removed. Therefore we decided to store each hidden point
in the face which  hides it and the nested face type of a 
regular triangulation is assumed to include a list of hidden 
weighted points. This list of weighted point is in fact included
in the base face of a regular triangulation.

The concept \ccRefName\ 
has to fulfills the following requirements
in addition to the requirements of \ccc{TriangulationVertexBase_2},
described \ccRefPage{TriangulationVertexBase_2}


\ccCreation
\ccCreationVariable{rvb}  %% choose variable name

\ccConstructor{RegularTriangulationVertexBase_2();}{default constructor.}
\ccGlue
\ccConstructor{RegularTriangulationVertexBase_2(const Point & p, void * f = NULL)}
{constructs a vertex embedded in point \ccc{p} and pointing on face \ccc{f}.}


\ccHeading{Access Functions}
\ccMethod{bool is_hidden();}
{returns \ccc{true}, iff the vertex is hidden.}

\ccMethod{void set_hidden(bool b);}
{Mark the vertex as hidden or as not hidden.}


\ccHasModels
\ccc{CGAL::Regular_triangulation_vertex_base_2}

\ccSeeAlso
\ccc{TriangulationVertexBase_2}
\ccc{CGAL::Regular_triangulation_vertex_base_2}

\end{ccRefConcept}

% +------------------------------------------------------------------------+
%%RefPage: end of main body, begin of footer
% EOF
% +------------------------------------------------------------------------+

