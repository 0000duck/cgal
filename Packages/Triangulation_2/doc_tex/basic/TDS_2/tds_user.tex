\chapter{2D Triangulation Data Structure} 
\label{user_chapter_2D_Triangulation_Data_Structure}
\label{Chapter_2D_Triangulation_Data_Structure}
\minitoc

The second template parameter of a 2D triangulation class
has to be instanciated with a model
of \ccc{TriangulationDataStructure_2}.
Section~\ref{2D_TDS_Concept} describes the concept
of \ccc{TriangulationDataStructure_2}.

These models are described in section~\ref{2D_TDS_Models}.
Section~\ref{2D_TDS_Base_Classes} described the base
vertex and face classes of a triangulation.
At last, section~\ref{2D_TDS_Example} shows 
how the user can plug in the triangulation
data structure his own vertex or face base class.


\section{The Concept}
\label{2D_TDS_Concept}
A model of \ccc{TriangulationDataStructure_2}
can be seen has a container for the 
faces and vertices of the triangulation.
This class is also responsible for the combinatorial
integrity of the triangulation. This means that
the triangulation data strucure 
maintains  proper incidence and adjacency relations among the vertices
and faces of a triangulation while
combinatorial modifications
of the triangulation are performed.
The term  combinatorial modification 
refers to  operations which do not 
involve any knowledge about the geometric embedding of the triangulation.
For example, the  
insertion of a new vertex in a given face, or in a given edge,
the suppression
of a vertex of degree three,  the flip of two edge are 
examples of combinatorial operation performed at the data structure level.


The triangulation data structure 
is required to provide :
\begin{itemize}
\item
iterators to visit all the vertices, edges and faces
of the triangulation,
\item
circulators to visit all the vertices, edges and faces
incident to a given vertex
\end{itemize}


The triangulation data structure is responsible 
for the creation and removal of faces and vertices 
(memory management).
It provides function that gives the number of faces, edges and
vertices
of the triangulation.

The triangulation data structure provides member functions
to perform the following  combinatorial transformation of the triangulation:\\
flip of two adjacent faces, \\
addition  of a new vertex splitting a given face,\\
addition  of a new vertex splitting a given edge,\\
addition of a new vertex raising by one the dimension of a degenerate
lower dimensional triangulation, \\
removal of a vertex incident to three faces, \\
removal of a vertex lowering the dimension of the triangulation.\\


%\begin{figure}
%\begin{ccTexOnly}
%\begin{center} %\IpeScale{70} \Ipe{Flip.ipe} \end{center}
%\input{flip.ltex}
%\end{center}
%\end{ccTexOnly} 
%\caption{Flip. \label{I1_fig_flip_bis}}

%\begin{ccHtmlOnly}
%<CENTER>
%<img border=0 src=Flip.gif align=center alt="Flip">
%</CENTER>
%\end{ccHtmlOnly} 
%\end{figure}



\begin{figure}
\begin{ccTexOnly}
%\begin{center} \IpeScale{70} \Ipe{Three.ipe} \end{center}
\begin{center} \documentclass{article}
\documentclass{article}
\documentclass{article}
\input{tmp.inputs}
\pagestyle{empty}
\begin{document}
\input{tmp.pstex_t}
\end{document}

\pagestyle{empty}
\begin{document}
\documentclass{article}
\input{tmp.inputs}
\pagestyle{empty}
\begin{document}
\input{tmp.pstex_t}
\end{document}

\end{document}

\pagestyle{empty}
\begin{document}
  \begin{tabular}{ccc}
    \input{insert1.pstex_t} &
    \input{insert2.pstex_t} &
    \input{insert3.pstex_t}\\
  {\small (a)} & {\small (b)} & {\small (c)}\\
  \end{tabular}
\end{document}
 \end{center}
\caption{Insertion}
\end{ccTexOnly} 

\begin{ccHtmlOnly}
<CENTER>
<img border=0 src=Three.gif align=center alt="Insertion">
</CENTER>
\end{ccHtmlOnly} 
\end{figure}




\subsection{Software design}
According to the software design described in 
section~\ref{Section_2D_Triangulations_Software_Design},
the models of triangulation data structure
offered by \cgal\ 
 are themselves template classes,
parametrized by the base classes for vertices and faces.
The concepts of \ccc{TriangulationVertexBase_2} and 
\ccc{TriangulationFaceBase_2} are described 
in next section~\ref{2D_TDS_Base_Classes}.
The triangulation data structure 
derives from thoses base classes its own 
vertex and face classes. 
This design  allows the user to plug in the
triangulation data structure
his own base classes tuned for  his application.







\subsection{A triangulation data structure using list}
The class \ccc{Triangulation_data_structure_using_list_2<Vb,Fb>}
can be used as a triangulation data structure for any
orientable triangulation. It uses a \stl\ list to store the
full dimensional faces of the triangulation.



\section{The Base Vertex and Face Classes }
\label{2D_TDS_Base_Classes}

\subsection*{The concepts}
The concepts \ccc{TriangulationVertexBase_2} and 
\ccc{TriangulationFaceBase_2}  described the requirements
for the base vertex and face classes of a two-dimensional
triangulation.

At the bottom layer,  
a vertex is required to provide access to the embedding point
and to one of its incident face
through a \ccc{void *} pointer.

At the bottom layer, a 
face provides access to its three vertices and to its three
neighboring faces through \ccc{void *} pointers.
The vertices and neighbors are indexed 0,1 and 2 in counterclockwise
order around the face. The neighbor indexed \ccc{i}
 lies opposite to vertex with the same index.



\subsection*{The Default Models}
\cgal\ provides the models
\ccc{Triangulation_face_base_2<Traits>} and
\ccc{Triangulation_vertex_base_2<Traits>} for 
respectively 
the \ccc{TriangulationVertexBase_2} and the 
\ccc{TriangulationFaceBase_2} concepts.
Both of them are templated by a geometric traits class.
Using for this traits class, the geometric traits class used for the triangulation class
is strongly recommended. 
It ensures that the point type defined by \ccc{Triangulation_vertex_base_2<Traits>}
is the same as the point type defined the  geometric traits class of
the triangulation.

These default base classes can be used directly or can serve as a base to derive
other base classes with some additional attribute (a color for example)
tuned for a specific application.


\section{Example : Using one's own Base Face}
\label {2D_TDS_Example}

The following example derives a new base face class from the default
one and adds a color to the faces of the triangulation. 
The face of the triangulation data structure
and the face of the triangulation will inherit the new data member 
and its functionality.
Any kind of additional functionality can thus be added to faces or vertices of a triangulation 
as long as this functionality  does not involve additional pointers to vertices or faces
(because the base classes use only void* pointer and have no knowledge
of the vertex or face types.).

\ccExample
\ccIncludeExampleCode{Triangulation_2/colored_face.C}

