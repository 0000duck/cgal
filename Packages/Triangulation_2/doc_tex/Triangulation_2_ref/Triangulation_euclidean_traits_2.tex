% +------------------------------------------------------------------------+
% | Reference manual page: Triangulation_euclidean_traits_2.tex
% +------------------------------------------------------------------------+
% | 06.04.2000   Author
% | Package: Package
% | 
\RCSdef{\RCSTriangulationeuclideantraitsRev}{$Revision$}
\RCSdefDate{\RCSTriangulationeuclideantraitsDate}{$Date$}
% |
%%RefPage: end of header, begin of main body
% +------------------------------------------------------------------------+


\begin{ccRefClass}{Triangulation_euclidean_traits_2<R>}  %% add template arg's if necessary

%% \ccHtmlCrossLink{}     %% add further rules for cross referencing links
%% \ccHtmlIndexC[class]{} %% add further index entries
\ccCreationVariable{traits}
\ccDefinition
  
The class \ccRefName\ is a model for the \ccc{Triangulation_traits_2}
and \ccc{Delaunay_triangulation_traits_2} 
concepts. It uses types defined in \cgal\ kernel
as geometric primitives types.

\ccInclude{Triangulation_euclidean_traits_2.h}

\ccIsModel 

\ccc{TriangulationTraits_2} \\
\ccc{DelaunayTriangulationTraits_2}

\ccTypes
\ccThree{typedef Triangle_2<R>}{Triangle_2}{}
\ccTypedef{typedef CGAL::Point_2<R>  Point_2;}{}
\ccGlue
\ccTypedef{typedef CGAL::Segment_2<R>   Segment_2;}{}
\ccGlue
\ccTypedef{typedef CGAL::Triangle_2<R>   Triangle_2;}{}
\ccGlue
\ccTypedef{typedef CGAL::Line_2<R> Line_2;}{}
\ccGlue
\ccTypedef{typedef CGAL::Direction_2<R> Direction_2;}{}
\ccGlue
\ccTypedef{typedef CGAL::Ray_2<R>  Ray_2;}{}

The following predicates and constructor types are provided

\ccNestedType{Construct_segment_2}
\ccGlue
\ccNestedType{Construct_triangle_2}
\ccGlue
\ccNestedType{Compare_x_2}{}
\ccGlue
\ccNestedType{Compare_y_2}{}
\ccGlue
\ccNestedType{Orientation_2}{}
\ccGlue
\ccNestedType{Side_of_oriented_circle_2}{}
\ccGlue
\ccNestedType{Compare_distance_2}
\ccGlue
\ccNestedType{Construct_circumcenter_2}{}
\ccGlue
\ccNestedType{Construct_bisector_2}{}
\ccGlue
\ccNestedType{Construct_ray_2}{}
\ccGlue
\ccNestedType{Construct_direction_2}{}

\ccHeading{Access to predicate objects}
The following access functions are provided

\ccMethod{Construct_segment_2 construct_segment_2_object();}
\ccGlue
\ccMethod{Construct_triangle_2 construct_triangle_2_object();}
\ccGlue
\ccMethod{Comparison_x_2 compare_x_2_object();}{}                               \ccGlue
\ccMethod{Comparison_y_2 compare_y_2_object();}{}
\ccGlue
\ccMethod{Orientation_2  orientation_2_object();}{}
\ccGlue
\ccMethod{Side_of_oriented_circle_2
          side_of_oriented_circle_2_object();}{}
\ccGlue
\ccMethod{Compare_distance_2 compare_distance_2_object();}{}
\ccGlue
\ccMethod{Construct_circumcenter_2 construct_circumcenter_2_object();}{}
\ccGlue
\ccMethod{Construct_bisector_2 construct_bisector_2_object();}{}
\ccGlue
\ccMethod{Construct_ray_2 construct_ray_2_object();}{}
\ccGlue
\ccMethod{Construct_directon_2 
          construct_direction_2_object();}{} 

\ccSeeAlso
\ccc{TriangulationTraits_2} \\
\ccc{DelaunayTriangulationTraits_2} \\
\ccc{CGAL::Triangulation_2<Traits,Tds>}\\
\ccc{CGAL::Delaunay_triangulation_2<Traits,Tds>} \\
\ccc{CGAL::Triangulation_euclidean_traits_xy_3<R>} \\

\ccExample
The following code creates a  valid triangulation traits class 
for a triangulation of 2D points in usual Euclidean space
and use it to define a triangulation class.

\begin{cprog}

typedef Cartesian<double> Rp;
typedef Triangulation_euclidean_traits_2<Rp> Traits;
typedef Triangulation_vertex_base_2<Traits> Vb;
typedef Triangulation_face_base_2<Traits> Fb;
typedef Triangulation_data_structure_using_list_2<Vb,Fb > Tds;
typedef Triangulation_2<Traits,Tds> Triangulation;

\end{cprog}


\end{ccRefClass}

% +------------------------------------------------------------------------+
%%RefPage: end of main body, begin of footer
% EOF
% +------------------------------------------------------------------------+

