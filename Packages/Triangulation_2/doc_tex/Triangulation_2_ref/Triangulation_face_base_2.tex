% +------------------------------------------------------------------------+
% | Reference manual page: Triangulation_face_base_2.tex
% +------------------------------------------------------------------------+
% | 11.04.2000   Author
% | Package: Package
% | 
\RCSdef{\RCSTriangulationfacebaseRev}{$Revision$}
\RCSdefDate{\RCSTriangulationfacebaseDate}{$Date$}
% |
%%RefPage: end of header, begin of main body
% +------------------------------------------------------------------------+


\begin{ccRefClass}{Triangulation_face_base_2<Traits,Fb>}  %% add template arg's if necessary

%% \ccHtmlCrossLink{}     %% add further rules for cross referencing links
%% \ccHtmlIndexC[class]{} %% add further index entries
\ccCreationVariable{fb}
\ccDefinition
  
The class \ccRefName\ is model for the concept
\ccc{TriangulationFaceBase_2}. It is the default face base class
for basic and Delaunay triangulation.

These default base class can be used directly or can serve as a base to derive
other base classes with some additional attribute (a color for example)
tuned for a specific application.


\ccParameters
The first template parameter of \ccRefName\ is  a geometric traits class.
The geometric traits is actually not used by the class.

The second template parameter 
has to be a  model of the concept \ccc{TriagulationDSFaceBase_2}
and will serve as a base class for \ccRefName\.
The default value for this parameter is
\ccc{Triangulation_ds_face_base_2<>}.


\ccInclude{CGAL/Triangulation_face_base_2.h}

\ccIsModel
\ccc{TriangulationFaceBase_2}

\ccSeeAlso
\ccc{CGAL::Triangulation_ds_face_base_2<Tds>}\\
\ccc{CGAL::Triangulation_vertex_base_2<Traits,Vb> }\\
\ccc{CGAL::Triangulation_2<Traits,Tds>}



\end{ccRefClass}

% +------------------------------------------------------------------------+
%%RefPage: end of main body, begin of footer
% EOF
% +------------------------------------------------------------------------+

