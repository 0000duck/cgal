% +------------------------------------------------------------------------+
% | Reference manual page: Constrained_triangulation_plus_2.tex
% +------------------------------------------------------------------------+
% | 05.02.2002   Mariette Yvinec
% | Package: Triangulation_2
% | 
\RCSdef{\RCSConstrainedtriangulationplusRev}{$Revision$}
\RCSdefDate{\RCSConstrainedtriangulationplusDate}{$Date$}
% |
%%RefPage: end of header, begin of main body
% +------------------------------------------------------------------------+


\begin{ccRefClass}{Constrained_triangulation_plus_2<Tr,Itag>}  %% add template arg's if necessary

%% \ccHtmlCrossLink{}     %% add further rules for cross referencing links
%% \ccHtmlIndexC[class]{} %% add further index entries

The class \ccRefName\ implements a constrained triangulation that 
can deal with intersecting constraints. 
The class allows as input a set of constraint segments which may be
intersecting, overlapping or partially
overlapping segments. 
The class \ccRefName\ constructs a triangulated arrangements 
of these constraints.
A vertex is introduced at each intersection point,
thus refining the input constraints into subconstraints
which are edges of the triangulation.
In addition, the class \ccRefName\ is equipped
with a data structure called  the constraint hierarchy that
 allows to retrieve, 
for each input constraint, the 
sequence of intersection vertices added on this constraint
and for each constrained edge of the
triangulation the set of input constraints that overlap it.

The class has two template parameters.  The first one has to be
instantiated by a constrained or Delaunay constrained triangulation
class. The second parameter is an intersection tag
 which allows to choose different
strategies to deal with constraint intersections. 
\cgal\ provides three valid types for this parameter : \\
the type \ccc{Tag_no_intersection} disallows intersections of
 input constraints,\\
the type \ccc{Tag_exact_intersections} is to be used in conjonction
with an exact aritmetic type \\
the type \ccc{Tag_exact_predicates} is to be used when the triangulation traits
provide exact predicates but approximate constructions of the
intersection points.

When constraint intersection are allowed, the traits class of the
triangulation is required to provide the type
\ccc{Intersect_2},  a function object whose operator()
compute the intersection of two segments.


\ccInclude{CGAL/Constrained_triangulation_plus_2.h}

\ccInheritsFrom
\ccc{Tr}

\ccTypes 
\ccTypedef{typedef Triangulation;}{the triangulation base class.}
\ccGlue
\ccTypedef{typedef Intersection_tag;}{the intersection tag}

\ccTypedef{typedef Constraint_iterator;}{An iterator to visit
all the input  constraints. The order of visit is arbitrary.
The value type of this iterator  is a pair of \ccc{Vertex_handle}
(\ccc{std::pair<Vertex_handle, Vertex_handle>}) corresponding to the
endpoints of the constraint.}
\ccGlue
\ccTypedef{typedef Subconstraint_iterator;}{An iterator 
to visit all the subconstraints  of the triangulation.
The order of visit is arbitrary.
The value type of this iterator is a pair of \ccc{Vertex_handle}
correponding to the vertices of  the
subconstraint.}
\ccGlue
\ccTypedef{typedef Vertices_in_constraint;}{ An iterator on the
vertices of the chain of triangulation edges representing a
constraint.}

\ccTypedef{typedef Context;}{This type is intended to describe
a constraint including a subconstraint and the position of the
subconstraint in this constraint.
It provides three member functions \ccc{first()}, \ccc{last()}
and \ccc{current()} returning 
iterators  on the sequence of vertices of the enclosing constraint.
\ccc{first()} returns a  \ccc{Vertices_in_constraint} iterator pointing 
on the first vertex of the enclosing constraint,
\ccc{last()} returns a  \ccc{Vertices_in_constraint} iterator 
pointing on the last vertex of the enclosing constraint, and 
\ccc{current()} returns a \ccc{Vertices_in_constraint} iterator  pointing
on the first vertex of the subconstraint.}


\ccTypedef{typedef Context_iterator;}{An iterator on the \ccc{Contexts}
corresponding to the different constraint enclosing a given subconstraint.}


\ccCreation
\ccCreationVariable{ctp}

\ccConstructor{Constrained_triangulation_plus_2(const Geom_traits& gt=Geom_traits());}
{Introduces an empty triangulation.}

\ccConstructor{Constrained_triangulation_plus_2(const
Constrained_triangulation_plus_2& ct);}{copy constructor}

\ccConstructor{Constrained_triangulation_plus_2(list<Constrained>& lc,
const Geom_traits& t = Geom_traits());}
{Introduces and builds a constrained triangulation 
from the list of constraints \ccc{lc}.}

\ccConstructor{ template<class InputIterator> 
Constrained_triangulation_plus_2(
        			InputIterator first,
                               	InputIterator last,
                               const Geom_traits& gt= Geom_traits());}
{Introduces and builds a constrained triangulation 
from the constraints in the range
$\left[\right.$\ccc{first}, \ccc{last}$\left.\right)$.
\ccPrecond The \ccc{value_type} of \ccc{first} and \ccc{last}
 is \ccc{Constraint}.
}

\ccFunction{void ~ Constrained_triangulation_plus_2();}
{Destructor. All vertices and faces are deleted.
The constraint hierarchy is deleted.}



\ccHeading{Assignment}

\ccMethod{Constrained_triangulation_plus_2 operator=(const
Constrained_triangulation_plus_2 tr);)}
{Assignement. All the vertices and faces are duplicated.
 The constraint hierarchy is also duplicated.}

\ccMethod{void swap(Constrained_triangulation_plus_2 tr);}
{The triangulations \ccc{tr} and \ccVar\ are swapped.
This operation should be preferred to \ccVar\ = \ccc{tr} or to
\ccc{t(tr)} if \ccc{tr} is deleted after that.}

\ccHeading{Insertion and Removal}
The class \ccRefName\ overwrites the following insertion and removal
member functions for points and constraints.

\ccMethod{Vertex_handle insert(const Point& p, 
Face_handle start = Face_handle() );}
{Inserts point \ccc{p} as a vertex of the triangulation.}

\ccMethod{Vertex_handle insert(const Point& p,
		       Locate_type lt,
		       Face_handle loc, int li );}
{inserts a point \ccc{p} whose location is assumed 
to be given by \ccc{(lt,loc,li)}.}

\ccMethod{Vertex_handle push_back(const Point& p);}
{Equivalent to \ccc{insert(p)}.}

\ccMethod{template < class InputIterator >
          int
          insert(InputIterator first, InputIterator last);}
{Inserts the points in the range
 $\left[\right.$\ccc{first}, \ccc{last}$\left.\right)$.
 Returns the number of inserted points.
 \ccPrecond The \ccc{value_type} of \ccc{first} and \ccc{last}
 is \ccc{Point}.}

\ccMethod{ void insert(Point a, Point b);}
{Inserts the constraint segment ab in the triangulation. }

\ccMethod{ void push_back(const Constraint& c);}
{Inserts the constraints \ccc{c} as above.}

\ccMethod{ void insert(Vertex_handle va, Vertex_handle vb);}
{ Inserts a constraint whose endpoints are the vertices va and vb
in the triangulation. }

\ccMethod{void remove_constraint(Vertex_handle va, Vertex_handle vb);}
{Removes the segment joining \ccc{va} and \ccc{vb}
from the constraints of the triangulation. 
\ccPrecond The segment \ccc{va vb} has to be one of the input
constraint.}


\ccHeading{Queries}
\ccMethod{ Constraint_iterator constraints_begin() const;}
{Returns a \ccc{Constraint_iterator} pointing on the first
constraint.}
\ccGlue
\ccMethod{Constraint_iterator constraints_end()   const;}
{Returns a \ccc{Constraint_iterator} pointing past the last
constraint.}
\ccGlue
\ccMethod{Subconstraint_iterator subconstraints_begin() const;}
{Returns a \ccc{Subconstraint_iterator} pointing on the first
subconstraint.}
\ccGlue
\ccMethod{Subconstraint_iterator subconstraints_end() const;}
{Returns a \ccc{Subconstraint_iterator} pointing past the last
subconstraint.}
\ccGlue
\ccMethod{int number_of_enclosing_constraints(Vertex_handle va, 
				      Vertex_handle vb);}
{Returns the number of constraints enclosing the subconstraint
\ccc{(va,vb).}
\ccPrecond \ccc{(va,vb)} has to be a subconstraint, i. e. an edge
of the triangulation.}
\ccGlue
\ccMethod{Context  context(Vertex_handle va, Vertex_handle vb);}
{Returns the \ccc{Context}  of one enclosing constraint of the
subconstraint  \ccc{(va,vb)}
\ccPrecond \ccc{(va,vb)} has to be a subconstraint.}
\ccGlue
\ccMethod{Context_iterator   contexts_begin(Vertex_handle va, 
				    Vertex_handle vb);}
{Returns an iterator pointing on the first \ccc{Context}
of the sequence of \ccc{Contexts}
corresponding to the constraints enclosing the subconstraint\ccc{(va,vb).}
\ccPrecond \ccc{(va,vb)} has to be a subconstraint.}
\ccGlue
\ccMethod{Context_iterator   contexts_end(Vertex_handle va, 
				  Vertex_handle vb);}
{Returns an iterator past the last  \ccc{Context}
of the sequence of \ccc{Contexts}
corresponding to the constraints enclosing the  \ccc{(va,vb).}}
\ccGlue
\ccMethod{Vertices_in_constraint 
vertices_in_constraint_begin(Vertex_handle va, Vertex_handle vb);}
{Returns an iterator on the first vertex on the constraint
\ccc{(va,vb)}
\ccPrecond \ccc{(va,vb)} has to be a constraint.}
\ccGlue
\ccMethod{Vertices_in_constraint 
vertices_in_constraint_end(Vertex_handle va, ertex_handle vb);}
{Returns an iterator past  the last  vertex on  the constraint
\ccc{(va,vb)}
\ccPrecond \ccc{(va,vb)} has to be a constraint.}


\ccSeeAlso
\ccc{CGAL::Constrained_triangulation_2<Traits,Tds>} \\
\ccc{CGAL::Constrained_Delaunay_triangulation_2<Traits,Tds>} \\
\ccc{ConstrainedTriangulationPlusTraits_2}

%\ccExample

%A short example program.
%Instead of a short program fragment, a full running program can be
%included using the 
%\verb|\ccIncludeExampleCode{examples/Package/Constrained_triangulation_plus_2_prog.C}| 
%macro. The program example would be part of the source code distribution and
%also part of the automatic test suite.

%\begin{ccExampleCode}
%void your_example_code() {
%}
%\end{ccExampleCode}

%% \ccIncludeExampleCode{examples/Package/Constrained_triangulation_plus_2_prog.C}

\end{ccRefClass}

% +------------------------------------------------------------------------+
%%RefPage: end of main body, begin of footer
% EOF
% +------------------------------------------------------------------------+

