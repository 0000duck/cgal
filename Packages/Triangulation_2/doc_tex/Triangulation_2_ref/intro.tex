% +------------------------------------------------------------------------+
% | Reference manual page: Triangulation_2.tex
% +------------------------------------------------------------------------+
% | 29.03.2000   Mariette Yvinec
% | Package: Triangulation
% | 
\RCSdef{\RCSTriangulationRev}{$Revision$}
\RCSdefDate{\RCSTriangulationDate}{$Date$}
% |
%%RefPage: end of header, begin of main body
% +------------------------------------------------------------------------+

%\clearpage
%\section{Reference pages for 2D Triangulations}
\chapter{2D Triangulations}

\subsection*{Definition}
A triangulation is a 2-dimensional simplicial complex which is pure
connected and without singularities. Thus a triangulation
can be viewed as a collection of triangular faces,
such that two faces either have an empty intersection or share an edge or a vertex.

Thus the basic elements of the representation are vertices and faces.
Each triangular face gives access to its three incident vertices 
and to its three adjacent faces. 
Each vertex gives access to one of its incident faces
and through that face to the circular list of its incident faces.
The edges are not explicitely represented, they are only represented 
through the adjacencies relations of two faces.

The triangulation classes of \cgal\ depends on two template parameters.
The first template parameter stands for
 a geometric traits class which is assumed to provide
the geometric objects (points, segments and triangles) 
forming  the triangulation and the geometric predicates on those objects.
The second template parameter stands for a model
of  triangulation data
structure acting as a container for faces and vertices
while  taking care of the combinatorial aspects of the triangulation. 



\subsection*{Concepts}
\ccRefConceptPage{TriangulationTraits_2} \\
\ccRefConceptPage{DelaunayTriangulationTraits_2} \\
\ccRefConceptPage{RegularTriangulationTraits_2} \\

\subsection*{Classes}
\ccRefIdfierPage{CGAL::Triangulation_2<Traits,Tds>}\\
\ccRefIdfierPage{CGAL::Delaunay_triangulation_2<Traits,Tds>}\\
\ccRefIdfierPage{CGAL::Regular_triangulation_2<Traits,Tds>}\\
\ccRefIdfierPage{CGAL::Constrained_triangulation_2<Traits,Tds>}\\
%\ccRefIdfierPage{CGAL::Constrained_triangulation_demo_2<Traits,Tds>}\\
\ccRefIdfierPage{CGAL::Constrained_Delaunay_triangulation_2<Traits,Tds>}\\
\ccRefIdfierPage{CGAL::Triangulation_hierarchy_2<Tr>} \\

\ccRefIdfierPage{CGAL::Triangulation_euclidean_traits_2<R>}\\
\ccRefIdfierPage{CGAL::Triangulation_euclidean_traits_xy_3<R>}\\
\ccRefIdfierPage{CGAL::Regular_triangulation_euclidean_traits_2<Rep,Weight>}\\


\ccRefIdfierPage{CGAL::Triangulation_2<Traits,Tds>::Face}\\
\ccRefIdfierPage{CGAL::Triangulation_2<Traits,Tds>::Vertex}\\

\ccRefIdfierPage{CGAL::Weighted_point<Pt,Wt>}\\
\ccRefIdfierPage{CGAL::Triangulation_cw_ccw_2}\\


\subsection*{Enum}
\ccRefIdfierPage{CGAL::Triangulation_2<Traits,Tds>::Locate_type}

\lcHtml{\subsection*{Links to the Reference Sections}}

% +------------------------------------------------------------------------+
%%RefPage: end of main body, begin of footer
% EOF
% +------------------------------------------------------------------------+
