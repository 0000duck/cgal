% +------------------------------------------------------------------------+
% | Reference manual page: Triangulation_2::Face_base.tex
% +------------------------------------------------------------------------+
% | 11.04.2000   Author
% | Package: Package
% | 
\RCSdef{\RCSTriangulation::FacebaseRev}{$Revision$}
\RCSdefDate{\RCSTriangulation::FacebaseDate}{$Date$}
% |
%%RefPage: end of header, begin of main body
% +------------------------------------------------------------------------+


\begin{ccRefConcept}{TriangulationFaceBase_2}

%% \ccHtmlCrossLink{}     %% add further rules for cross referencing links
%% \ccHtmlIndexC[concept]{} %% add further index entries

\ccDefinition
  
The concept \ccRefName\ describes the requirements for
the base face class of a triangulation data structure
that is itself plugged into a basic triangulation
or a Delaunay triangulation.

This concept refines the concept \ccc{TriangulationDSFaceBase_2}
and could add  geometric information. In fact, 
currently the triangulations of
CGAL do not  store any geometric information in the faces
and, thus this concept is just equal to \ccc{TriangulationDSFaceBase_2}
and only provided for symmetry with the vertex case.


\ccRefines
\ccc{TriangulationDSFaceBase_2}

\ccHasModels
\ccc{CGAL::Triangulation_face_base_2<Traits>}


\ccSeeAlso
\ccc{TriangulationVertexBase_2} \\
\ccc{CGAL::Triangulation_face_base_2<Traits>} \\
\ccc{CGAL::Triangulation_2<Traits,Tds>} \\
\ccc{CGAL::Delaunay_triangulation_2<Traits,Tds>}

\end{ccRefConcept}

% +------------------------------------------------------------------------+
%%RefPage: end of main body, begin of footer
% EOF
% +------------------------------------------------------------------------+

