% begin cgal manual page

\begin{ccRefClass}{Partition<T,A>}\ccCreationVariable{P}

\ccDefinition

An instance \ccc{P} of the data type \ccc{Partition<T,A>}
consists of a finite set of items (\ccc{partition_item}) and a partition
of this set into blocks. Each item has an associated information of
type \ccc{T}.  The type \ccc{A} has to be a model of the allocator concept 
as defined in the C++ standard. 



\ccSetOneOfTwoColumns{5cm}

\ccTypes

\ccNestedType{item}{the item type. 
}

\ccNestedType{allocator}{the allocator. 
}

\ccSetOneOfTwoColumns{4cm}

\ccCreation

\ccConstructor{Partition<T,A>()}{creates an instance \ccc{P} of type \ccc{Partition<T,A>} and initializes 
     it to the empty partition. 
}

\ccSetTwoOfThreeColumns{3cm}{2cm}

\ccOperations

\ccMethod{  void clear();}{makes \ccc{P} the empty partition. 
}

\ccMethod{  partition_item make_block(const T& x);}{returns a new \ccc{partition_item} \ccc{it}, adds
     the block \ccc{{it}} to partition \ccc{P}, and associates \ccc{x} with \ccc{it}. 
}

\ccMethod{  partition_item find(partition_item p) ;}{returns a canonical item of the block that
     contains item $p$, i.e., iff \ccc{P.same_block(p,q)} 
     then \ccc{P.find(p)} and \ccc{P.find(q)} return the same item.\\
     \ccPrecond $p$ is an item in \ccc{P}.  
}

\ccMethod{  size_t size(partition_item p) ;}{returns the size of the block containing $p$. 
}

\ccMethod{  size_t number_of_blocks() ;}{returns the number of blocks of \ccc{P}. 
}

\ccMethod{  bool same_block(partition_item p, partition_item q) ;}{returns true if $p$ and $q$ belong to the same
     block of partition \ccc{P}.\\
     \ccPrecond $p$ and $q$ are items in \ccc{P}. 
}

\ccMethod{  void union_blocks(partition_item p, partition_item q);}{unites the blocks of partition \ccc{P} containing
  items $p$ and $q$.\\
  \ccPrecond $p$ and $q$ are items in \ccc{P}. 
}

\ccMethod{  const T& inf(partition_item p) ;}{returns the information associated with \ccc{it}. 
}

\ccMethod{  void change_inf(partition_item p, const T& x)   ;}{changes the information associates with \ccc{it} to \ccc{x}. 
}

\end{ccRefClass}


