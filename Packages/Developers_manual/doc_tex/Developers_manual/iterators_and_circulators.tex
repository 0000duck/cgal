% =============================================================================
% The CGAL Developers' Manual
% Chapter: Iterators and Circulators (and Handles)
% -----------------------------------------------------------------------------
% file   : iterators_and_circulators.tex
% authors: Mariette Yvinec <yvinec@sophia.inria.fr>
% -----------------------------------------------------------------------------
% $Revision$
% $Date$
% =============================================================================

\chapter{Iterators and Circulators (and Handles)}
\label{chap:iterators_and_circulators}
\ccIndexMainItemBegin{iterators}
\ccIndexMainItemBegin{circulators}
\ccIndexMainItemBegin{\stl} 
\ccChapterRelease{Chapter Version: 1.0} \\
\ccChapterAuthor{Mariette Yvinec ({\tt yvinec@sophia.inria.fr})}


\ccAnchor{http://www.cgal.org/Manual/doc_html/use_of_stl/Chapter_Iterator.html}{Iterators}
\ccIndexMainItemDef{iterators}
are a generalization of pointers that allow a programmer to work 
with different data structures (containers) in a uniform manner. 
An iterator is the glue that allows one to write a single implementation of 
an algorithm that will work for data contained in an array, a list or some 
other container -- even a container that did not yet exist when the algorithm 
was implemented. 

The concept of an iterator is one of the major tools of the genericity
in \stl.
Iterators are used almost everywhere in the \stl\  to
achieve the communication between containers and algorithms.
Iterators are widely used in \cgal\ too.
\cgal\ extends the idea of the iterator, which works for linear data
structures, to circular data structures by defining the concept of
a circulator.  
\ccAnchor{http://www.cgal.org/Manual/doc_html/use_of_stl/Chapter_Circulator_stl.html}{Circulators}
\ccIndexMainItemDef{circulators}
are quite similar to iterators, with the
major difference being the absence of a past-the-end position in a sequence.
Note that circulators are NOT part of the \stl, but of \cgal.

\ccIndexMainItemBegin{handles}
In \cgal, we also define the concept of a handle, 
\ccIndexMainItemDef{handles}
which bahaves roughly
like a pointer to an object without an increment or decrement operation.
More details about handles and their requirements can be found in 
the \ccAnchor{http://www.cgal.org/Manual/doc_html/frameset/fsSupport.html}{\cgal\ Support Library Reference Manual}. 
Section~\ref{sec:handle_vs_it_vs_circ} below discusses when handles
should be used in your code.
\ccIndexMainItemEnd{handles}

The concept of iterators is relatively well described in textbooks such as 
Stroustrup's {\em The C++ Programming Langage} \cite{cgal:s-cpl-97} 
and Austern's {\em Generic Programming and the \stl} \cite{cgal:a-gps-98}
and in the
\ccAnchor{http://www.cgal.org/Manual/doc_html/frameset/fsUseOfSTL.html}{Use of \stl\ and \stl\ Extensions in \cgal\ } manual (which also
discusses the circulator concept).
Thus we will not give a full description of this concept here
but only a few hints about how to use and write iterators (and circulators)
in \cgal.  Developers should consult the above-mentioned references to become
familiar with the iterator and circulator concepts in general and, in
particular, iterator and circulator ranges, dereferencable and past-the-end 
values, mutable and constant iterators and circulators, and the different 
categories (forward, bidirectional, random-access, etc.) of iterators and 
circulators.


\section{Iterator and circulator traits}
\label{sec:iterator_traits}
\ccIndexMainItemBegin{iterator traits}

The algorithms working with iterators and/or circulators often need to refer
to types associated with the iterator or circulator (\eg,
the type of the object referred to by the iterator
or the type of the distance between two circulators).
These types are usually declared in the iterator or circulator classes. 
However, for pointer classes, which can be valid models of the iterator
and circulator concepts,  this not possible. 
Thus iterator traits have been introduced to resolve this problem.
An algorithm using an iterator of the type \ccc{Iter} will find the 
relevant types in an instantiation of a small templated class 
\ccc{iterator_traits}.

There is a  general templated version
of \ccc{iterator_traits} that looks like:
\begin{verbatim}
template <class Iter> 
struct iterator_traits {
  typedef typename Iter::iterator_category  iterator_category ;
  typedef typename Iter::value_type         value_type;
  typedef typename Iter::difference_type    difference_type;
  typedef typename Iter::pointer            pointer;
  typedef typename Iter::reference          reference;
};
\end{verbatim}
and a partial specialization of \ccc{iterator_traits} classes for pointers:
\ccIndexSubitem{iterator traits}{for pointers}
\begin{verbatim}
template <class T> 
struct iterator_traits<T*> {
  typedef random_access_iterator            iterator_category ;
  typedef T                                 value_type;
  typedef ptrdiff_t                         difference_type;
  typedef T*                                pointer;
  typedef T&                                reference;
};
\end{verbatim}

\ccIndexMainItemEnd{iterator traits}


\section{Input and output iterators}
\label{sec:input_and_output_iterators}
\ccIndexSubitemBegin{iterators}{input}
\ccIndexSubitemBegin{iterators}{output}

{\bf Operator * for input and output iterators}
\ccIndexSubsubitemBegin{iterators}{input}{dereferencing}
\ccIndexSubsubitemBegin{iterators}{output}{dereferencing}

The operator * of input and output iterators has a restricted semantics.
Input iterators are designed for input operations, and it is not
required that the value type \ccc{T}  
of  an input iterator \ccc{it} be assignable. 
Thus, while assignments of the type \ccc{t = *it} are the usual way
to get values from the input iterators,
statements like \ccc{*it = ...} are likely to be illegal.
On the other hand, output iterators are designed for write operations,
and  the only legal use of the operator * of an output iterator
\ccc{it} is in the assignement \ccc{ *it = ....}.
The code of a standard copy function of the \stl\ 
provides an example of both of these operations: 

\lcRawHtml{<A NAME="copy_function">}
\begin{verbatim}
template< class InputIterator, class OutputIterator>
OutputIterator copy(InputIterator first, 
                    InputIterator last,
                    OutputIterator result) {
    while (first != last)
    {
       *result = *first;
       ++first; 
       ++result;
    }
    return result;
}
\end{verbatim}
\lcRawHtml{</A>} 

The first two arguments of \ccc{copy} are of type \ccc{InputIterator} 
(meaning any type that fulfills the requirements for an input iteraor)
while the third one is of type \ccc{OutputIterator}. If these types 
were  exchanged, then the statement \ccc{*result = *first;}
might not be valid.
\ccIndexSubsubitemEnd{iterators}{input}{dereferencing}
\ccIndexSubsubitemEnd{iterators}{output}{dereferencing}

{\bf Stream iterators}
\ccIndexSubitemBegin{iterators}{stream}

\stl\  provides a special type of input iterator called \ccc{istream_iterator},
\ccIndexMainItem{\ccFont istream_iterator}
which is designed to be bound to an object of the class \ccc{istream} and 
provides a way to read a sequence of values from the input stream to which 
it is bound.  For example, the following code reads numbers of type 
\ccc{double} from the standard input stream \ccc{cin} and computes their sum.
\begin{verbatim}
istream_iterator<double> it(cin);
istream_iterator<double> end();

double sum=0.0;
while(it != end) {
   sum += *it;
   ++it;
}
cout << sum << endl;
\end{verbatim}


In a similar fashion, \stl\ provides the type \ccc{ostream_iterator}, 
\ccIndexMainItem{\ccFont ostream_iterator}
which is designed to be bound to an object of the class \ccc{ostream}  
and used to output values to the output stream to which it is bound.

\ccIndexSubitem{\ccFont ostream_iterator}{extension}
\cgal\ provides extensions of the classes \ccc{istream_iterator}
and \ccc{ostream_iterator}. The class \ccc{CGAL::Ostream_iterator<T,Stream>} 
is an output iterator adaptator for the stream class \ccc{Stream} and value 
\ccIndexSubitem{adaptor}{output iterator}
type \ccc{T}.  It provides output iterators that can be used to ouput values 
of type \ccc{T} to objects of the class \ccc{Stream}.
For example, the following code fragment inserts a list of segments into
a window stream (\ie, it draws the segments in the window) using the 
standard copy function: 

\begin{verbatim}
typedef CGAL::Cartesian<double>   K;
typedef K::Segment_2              Segment;

std::vector<Segment>       segments;
CGAL::Window_stream        W( 400, 400);

int main (int argc, char** argv)
{
  // initialize segments
  std::copy( segments.begin(),
             segments.end(),
             CGAL::Ostream_iterator< Segment, CGAL::Window_stream>( W));
}
\end{verbatim}


\ccIndexSubitem{\ccFont istream_iterator}{extension}
\ccIndexSubitem{adaptor}{input iterator}
Likewise, the class \ccc{CGAL::Istream_iterator<T,Stream>} is an input
iterator adaptor for the stream class \ccc{Stream} and value type \ccc{T}. 
These adaptors are particularly useful for stream classes that are similar to
but not derived from \ccc{std::istream} and \ccc{std::ostream}. The only 
requirements of the stream classes are that they define \ccc{operator>>}  
(for \ccc{Istream_iterator}) and \ccc{operator<<} 
(for \ccc{Ostream_iterator}).
\ccIndexSubitemEnd{iterators}{stream}

{\bf Insert iterators}
\ccIndexSubitemBegin{iterators}{insert}
\ccIndexSubitem{iterators}{output}

Insert iterators are output iterators that can be used to insert items 
into containers.\ccIndexSubitem{containers}{insertion into}
With regular iterator classes, the code given above
for the \ccAnchor{#copy_function}{\ccc{copy} function} of \stl, 
causes the range \ccc{[first,last)} to be copied into an existing range 
starting with \ccc{result}. No memory allocation is involved and the
existing range is overwritten. With an 
insert iterator supplied as the third argument, the same code will 
cause elements to be inserted into the container with which the output
iterator is associated.  That is, new memory may be allocated for these
inserted elements.

The \stl\ provides three kinds of insert iterators: \ccc{insert_iterator}s,
\ccc{back_insert_iterator}s and \ccc{front_insert_iterator}s.
The \ccc{back_inserter_iterator}s are used to insert elements at the end of 
a container by using the \ccc{push_back} member function of the container.
Similarly, \ccc{front_insert_iterator}s are used to insert elements at the
beginning of a container by using the container's \ccc{push_front}
function.  The general \ccc{insert_iterator} is used to insert elements
at any point in a container, by using the container's \ccc{insert} 
member function and a provided location of the insertion.
\ccIndexSubitem{containers}{insertion into}

For convenience, \stl\  provides the templated
functions (or adaptors) 
\ccc{front_inserter}, 
\ccIndexMainItem{\ccFont front_inserter}
\ccc{back_inserter}
\ccIndexMainItem{\ccFont back_inserter}
\ccIndexSubitem{adaptor}{for insert iterators}
and \ccc{inserter} to get inserters from containers.
\ccIndexMainItem{\ccFont inserter}
\ccIndexSubitem{containers}{insertion adaptors}

\begin{verbatim}
template<class Container, class Iterator> 
insert_iterators<Container> inserter(Container& c, Iterator it); 

template<class Container> 
back_insert_iterators<Container> back_inserter(Container& c); 

template<class Container> 
front_insert_iterators<Container> front_inserter(Container& c); 
\end{verbatim}

Thus, the \ccc{inserter} adaptor can be called for any container that
has an \ccc{insert} member function, and \ccc{back_inserter}
(resp. \ccc{front_inserter}) can be called for any container that has a
\ccc{push_back} (resp. \ccc{push_front}) member function.  Some versions
of \stl\ (in particular, the one of KCC and Borland) also require that 
containers define a \ccc{value_type} and a \ccc{const_reference} type.

The following code will insert 200 copies of the value 7 at the end of
\ccc{vec}.
\begin{verbatim}
void g(vector<int>& vec)
{
    fill_n(std::back_inserter(vec),200,7);
}
\end{verbatim}
and this code will insert the points contained in the vector \ccc{vertices}
into a Delaunay triangulation data structure:

\begin{verbatim}

  typedef CGAL::Cartesian<double>                          K;
  typedef CGAL::Triangulation_euclidean_traits_2<K>        Gt;
  typedef CGAL::Triangulation_vertex_base_2<Gt>            Vb;
  typedef CGAL::Triangulation_face_base_2<Gt>              Fb;
  typedef CGAL::Triangulation_default_data_structure_2<Gt,Vb,Fb>
                                                           Tds;
  typedef CGAL::Delaunay_triangulation_2<Gt,Tds>           DT;

  DT triangulation;

  std::copy( vertices.begin(),
             vertices.end(),
             std::back_inserter( triangulation ));

\end{verbatim}
\ccIndexSubitemEnd{iterators}{insert}
\ccIndexSubitemEnd{iterators}{input}
\ccIndexSubitemEnd{iterators}{output}
\ccIndexMainItemEnd{iterators}
\ccIndexMainItemEnd{circulators}
\ccIndexMainItemEnd{\stl} 

%\section{Handles}
%\label{sec:handles}
%\ccIndexMainItemBegin{handles}
%
%The concept of an iterator is actually a generalization of the concept
%of a handle, which is defined in \cgal\ (but not in \stl).  A handle
%behaves roughly like a pointer to an object without an increment or
%decrement operation. The requirements for handles are detailed in
%the \ccAnchor{http://www.cgal.org/Manual/doc_html/support_lib/contents}{\cgal\ Support Library Reference Manual}. 
%\ccIndexMainItemEnd{handles}

\section{Writing code with and for iterators, circulators, and handles}

Because you should write generic code for \cgal, algorithms that require
a sequence of data for input should be written to take an iterator 
(or circulator) range as input instead of, say, a particular container.
Similarly, algorithms that compute a sequence of data as output should 
place the output data into an output iterator range.  Both of these
points are illustrated by the prototype of the following function that
computes the convex hull of a set of points in two dimensions:

\ccFunction{template <class InputIterator, class OutputIterator>
            OutputIterator
            convex_hull_points_2(InputIterator first, InputIterator beyond,
                                      OutputIterator  result,
                                      const Traits & ch_traits );}
           {generates the counterclockwise sequence of extreme points
            of the points in the range [\ccc{first},\ccc{beyond}).
            The resulting sequence is placed starting at position
            \ccc{result}, and the past-the-end iterator for the resulting
            sequence is returned. It is not specified at which point the
            cyclic sequence of extreme points is cut into a linear sequence.}

\ccIndexSubitem{containers}{writing}
Also, when writing container classes, you should be sure to provide
iterators and/or circulators for the containers and design the interfaces
so they can be used with generic algorithms from the \stl\ and other
\cgal\ algorithm.  Here we give a few more details about how to accomplish
these goals.


\subsection{Handle, iterator, or circulator?}
\label{sec:handle_vs_it_vs_circ}
\ccIndexSubitem{handles}{when to use}
\ccIndexSubitem{iterators}{when to use}
\ccIndexSubitem{circulators}{when to use}

Handles are indirect references that do not move, so whenever you need a
pointer-like reference to a single element of a data structure, and it is
not necessary to iterate (or circulate), use a handle.
In contrast, iterators should be used when you want to move (that is, 
iterate) over a linear sequences of elements.
When the sequence is circular, prefer a circulator over an iterator.

\subsection{Writing functions for iterators AND circulators}
\label{sec:it_and_circ_code}
\ccIndexSubitem{iterators}{writing code for}
\ccIndexSubitem{circulators}{writing code for}

To make your code as generic as possible, you should, where appropriate,
write functions that can accept either a circulator range or an iterator
range to delimit the input values.  Since empty circulator ranges are
represented differently than empty iterator ranges, the following function
is defined in \texttt{<CGAL/circulator.h>} so the test for an empty range 
can be done generically:

\ccSetThreeColumns{bool}{is_empty_range( IC i, IC j);}{}
\ccFunction{template< class IC>
    bool is_empty_range( const IC& i, const IC& j);}{%
    is \ccc{true} if the range [\ccc{i, j}) is empty,
    \ccc{false} otherwise.
    \ccPrecond \ccc{IC} is either a circulator or an iterator
    type. The range [\ccc{i, j}) is valid.
}

One would use this function in conjunction with a \ccc{do-while} loop
as follows:

\begin{verbatim}
if ( ! CGAL::is_empty_range( i, j) )
{
  do
  { 
    // ...
  } while ( ++i != j )
}
\end{verbatim}
  
The following two macros are also defined as a generic means for iterating
over either a linear or circular sequence:

\ccStyle{CGAL_For_all( ic1, ic2)}
\ccIndexMainItem[C]{CGAL_For_all} 

\ccStyle{CGAL_For_all_backwards( ic1, ic2)}
\ccIndexMainItem[C]{CGAL_For_all_backwards}

See the \ccAnchor{http://www.cgal.org/Manual/doc_html/support_lib/Circulator/Chapter_circulator.html}{Circulator documentation} in the
Support Library Reference Manual for more information and examples.

\subsection{Writing an iterator for your container}
\label{sec:class_iterator}
\ccIndexSubitemBegin{iterators}{writing}
\ccIndexSubitemBegin{containers}{iterators for}

Every container class in \cgal\ should strive to be a model for the
\stl\ concept of a container.  As for all concepts, this means that 
certain types and functions are provided, as detailed, for example 
in \cite{cgal:a-gps-98}.  For the purposes of this discussion, the relevant
types are:
\begin{tabbing}
\lcTex{constxiteratorxxxxxxxx\=\kill}
\ccStyle{iterator} \> type of iterator \\
\ccStyle{const_iterator} \> iterator type for container with constant elements
\end{tabbing}

and the relevant functions are:

\ccSetThreeColumns{const_iterator}{begin();}{}
\ccFunction{iterator begin();}{beginning of container}
\ccFunction{const_iterator begin();}{beginning of container with constant elements}
\ccFunction{iterator end();}{past-the-end value for container}
\ccFunction{const_iterator end();}{past-the-end value for container with constant elements} 
Variations on the above names are possible when, for example, the container
contains more than one thing that can be iterated over.  See 
Section~\ref{sec:naming_scheme} for more details about the naming conventions
for iterators and their access functions.
\ccIndexSubitemEnd{containers}{iterators for}
\ccIndexSubitemEnd{iterators}{writing}

\subsection{Writing a circulator for your container}
\label{sec:class_circulator}
\ccIndexSubitemBegin{circulators}{writing}
\ccIndexSubitemBegin{containers}{circulators for}

When a container represents a circular data structure (\ie, one without a
defined beginning or end), one should provide circulators for the data
elements in addition to (or, where appropriate, instead of) the iterators.
This means that the following types should be defined:
\begin{tabbing}
\lcTex{constxcirculatorxxxxxxxx\=\kill}
\ccStyle{circulator} \> type of circulator \\
\ccStyle{const_circulator} \> circulator type for container with constant elements
\end{tabbing}
as well as two access functions, one for each of the two types, with names
that end in the suffix \ccc{_circulator} (Section~\ref{sec:naming_scheme}).
\ccIndexSubitemEnd{containers}{circulators for}
\ccIndexSubitemEnd{circulators}{writing}

\InternalOnly{

\section{Requirements and recommendations}
\label{it_and_circ:it_and_circ_req_and_rec}

\noindent
Requirements:
\begin{itemize}
\item All container classes should provide iterator types and access functions.

\item All container classes that represent circular data structures should
      provide circulator types and access functions.
      \ccIndexSubitem{containers}{types}

\item Take care that decrement of the past-the-end value
      is, in accordance with the standard, a legal operation for a 
      bidirectionnal iterator.
      \ccIndexSubitem{iterators}{bidirectional}
      \ccIndexSubsubitem{iterators}{past-the-end value}{decrement of}
      This can, for example, be used to get the last element of 
      a sequence.
\end{itemize}

\noindent
Recommendations:
\begin{itemize}
\item Be aware that postincrement (respectively, postdecrement) is more
      expensive than preincrement (predecrement) since the iterator or
      circulator value must be copied in the former case.
      \ccIndexSubitem{iterators}{post- vs. preincrement}
      \ccIndexSubitem{circulators}{post- vs. preincrement}

\item Remember that iterators and circulators are intended to be lightweight 
      objects.  That is, copying them should require only constant time.
      \ccIndexMainItemDef{lightweight objects}
      \ccIndexSubitem{iterators}{as lightweight objects}
      \ccIndexSubitem{circulators}{as lightweight objects}

\item When writing a container-like structure, provide \ccc{push_back}, 
      \ccc{push_front}, and \ccc{insert} member functions
      so all insert iterators can be used with your container.
      \ccIndexSubitem{containers}{member functions}
\end{itemize}

}
