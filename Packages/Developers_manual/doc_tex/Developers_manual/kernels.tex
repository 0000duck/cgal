% =============================================================================
% The CGAL Developers' Manual
% Chapter: Geometry Kernels
% -----------------------------------------------------------------------------
% file   : kernels.tex
% authors: Stefan Schirra <stschirr@mpi-sb.mpg.de>
% -----------------------------------------------------------------------------
% $Revision$
% $Date$
% =============================================================================

\chapter{Geometry Kernels} \label{chap:kernels}
\ccChapterAuthor{Stefan Schirra ({\tt stschirr@mpi-sb.mpg.de})}


The layer of geometry kernels\ccIndexMainItem[t]{kernel} provides
basic geometric entities of constant size\footnote{In dimension $d$,
an entity of size $O(d)$ is considered to be of constant size.} and
primitive operations on them. Each entity is provided as both a
stand-alone class, which is parameterized by a kernel class, and as a
type in the kernel class. Each operation in the kernel is provided via
a functor class\footnote{A class which defines a member
\ccc{operator()}.} in the \ccIndexMainItem[t]{function objects} kernel
class and also as either a member function or a global function.
See~\cite{hhkps-aegk-01} for more details about this design.
%%
\ccIndexMainItemDef[t]{kernel traits}% 
\ccIndexSubitem{traits class}{kernel as a}% 
%%
Ideally, if the kernel provides all the primitives required, you can
use any kernel as a traits class directly with your algorithm or data
structure; see also Chapter~\ref{chap:traits_classes}. If you need
primitives not provided by the kernel (yet), please read
Section~\ref{sec:NewKernelTraits} below.

\section{Cartesian and homogeneous representation}
\label{sec:cart_and_hom_representation}

There are two different coordinate representations%
\ccIndexMainItem[t]{coordinate}%
\ccIndexMainItem[t]{representation}
available in the
kernel at present: Cartesian representation and homogeneous
representation. Cartesian representation%
\ccIndexSubitemDef{Cartesian}{representation}%
\ccIndexSubitemDef{coordinate}{Cartesian}%
\ccIndexSubitem{representation}{Cartesian}%
\ccIndexSubitem{kernel traits}{Cartesian}
is the one you are familiar with. 
A \ccAnchor{http://www.geom.uiuc.edu/docs/reference/CRC-formulas/node4.html}{point in the plane} is given by its $x$- and its $y$-coordinates;%
\lcTex{\footnote{%
\path|http://www.geom.uiuc.edu/docs/reference/CRC-formulas/node4.html|}} 
a \ccAnchor{http://www.geom.uiuc.edu/docs/reference/CRC-formulas/node39.html}{point in space} by its $x$-, $y$- and $z$-coordinates.\lcTex{\footnote{%
\path|http://www.geom.uiuc.edu/docs/reference/CRC-formulas/node39.html|}}
Homogeneous representation
\ccIndexSubitemDef{homogeneous}{representation}%
\ccIndexSubitemDef{coordinate}{homogeneous}%
\ccIndexSubitem{representation}{homogeneous}%
\ccIndexSubitem{kernel traits}{homogeneous}
can be seen as a divison-free representation
of Cartesian coordinates. There is an additional coordinate, sometimes called
the {\em homogenizing} coordinate.%
\ccIndexMainItemDef{homogenizing coordinate}
So with homogeneous representation in
2-space\lcTex{\footnote{%
\path|http://www.geom.uiuc.edu/docs/reference/CRC-formulas/node6.html|}}, 
we have \ccAnchor{http://www.geom.uiuc.edu/docs/reference/CRC-formulas/node6.html}{coordinates $(x,y,w)$}, where $w$ is the homogenizing coordinate,
and in 3-space\lcTex{\footnote{%
\path|http://www.geom.uiuc.edu/docs/reference/CRC-formulas/node43.html|}}, 
we have homogeneous \ccAnchor{http://www.geom.uiuc.edu/docs/reference/CRC-formulas/node43.html}{coordinates $(x,y,z,w)$}.
Since \cgal\ uses homogeneous representation for affine geometry (not for
projective geometry), we assume $w \neq 0$.
The Cartesian representation corresponding to $(x_0, x_1, \ldots, x_d, w)$
is $(x_0/w, x_1/w, \ldots, x_d/w)$. Hence, homogeneous representation
is not unique; $(\alpha x,\alpha y,\alpha z,\alpha w)$ is an alternative
representation to $(x,y,z,w)$ for any $\alpha \neq 0$.
Internally, \cgal\ always maintains a non-negative homogenizing coordinate.

With the homogeneous representation, division operations%
\ccIndexMainItem{division} can be avoided.
Homogeneous representation is advantageous over Cartesian representation
whenever systems of linear equations with integral coefficients
are to be solved. By Cramer's rule,
the rational solutions%
\ccIndexMainItem{rational computation} all have the same denominator $D$.
The Cartesian representation would be
$$(N_0/D, N_1/D, \ldots, N_{d-1}/D)$$
while a corresponding less space-consuming homogeneous representation is 
$$(N_0, N_1, \ldots, N_{d-1}, D)$$
For example, computing the Cartesian coordinates $(x,y)$ of the
intersection point  of lines with equations
$a_1 X + b_1 Y + c_1 = 0$ and $a_2 X + b_2 Y + c_2 = 0$ gives
\lcTex{
$$ (x,y) = 
\Bigg(
\frac{\left|\begin{array}{cc}b_1& c_1 \\ b_2 & c_2\end{array}\right|}
     {\left|\begin{array}{cc}a_1& b_1 \\ a_2 & b_2\end{array}\right|},
\frac{-\left|\begin{array}{cc}a_1& c_1 \\ a_2 & c_2\end{array}\right|}
     {\phantom{-}\left|\begin{array}{cc}a_1& b_1 \\ a_2 & b_2\end{array}\right|}
\Bigg)
$$ 
}
\lcRawHtml{
<CENTER>
<IMG BORDER=0 SRC="fig/Cartesian_ipoint.gif" 
  ALIGN=CENTER ALT="Intersection point with Cartesian coordinates"><BR>
</CENTER>
}
while with homogeneous representation we have
\lcTex{
$$ (x,y,w) = 
\Bigg(
{\left|\begin{array}{cc}b_1& c_1 \\ b_2 & c_2\end{array}\right|},
{-\left|\begin{array}{cc}a_1& c_1 \\ a_2 & c_2\end{array}\right|},
{\left|\begin{array}{cc}a_1& b_1 \\ a_2 & b_2\end{array}\right|}\Bigg)
$$
}
\lcRawHtml{
<CENTER>
<IMG BORDER=0 SRC="fig/homogeneous_ipoint.gif"
   ALIGN=CENTER ALT="Intersection point with homogeneous coordinates"><BR>
</CENTER>
}
In general, however, homogeneous representation is not more space-efficient.
Naive conversion from a rational Cartesian represenatation 
$$(N_0/D_0, N_1/D_1, \ldots, N_{d-1}/D_{d-1})$$ to a homogeneous
representation will lead to much bigger numbers, namely,
$$\textstyle
(N_0 \prod D_i /D_0, N_1\prod D_i /D_1, \ldots, N_{d-1}\prod D_i /D_{d-1},
\prod D_i )$$. 

\section{Cartesian versus homogeneous computation}
\label{sec:CartesianHomogeneousComputation}
Homogeneous representation has the disadvantage that predicates become
more complicated. Testing equality of points is more complicated because
the homogeneous representation is not unique. In the orientation predicate,
the sign of a 3x3 determinant must be computed:
\lcTex{
$${\mbox{\small sign}}\left|
\begin{array}{ccc}
{x_1} & {y_1} & w_1 \\
{x_2} & {y_2} & w_2 \\
{x_3} & {y_3} & w_3 \\
\end{array}
\right|
$$
}
\lcRawHtml{
<CENTER>
<IMG BORDER=0 SRC="fig/homogeneous_orientation.gif"
   ALIGN=CENTER ALT="Orientation test with homogeneous coordinates"><BR>
</CENTER>
}
With Cartesian representation, it is essentially a 2x2 determinant only:
\lcTex{
$$
{\mbox{\small sign}}\left|
\begin{array}{ccc}
{x_1} & {y_1} & 1 \\
{x_2} & {y_2} & 1 \\
{x_3} & {y_3} & 1 \\
\end{array}
\right| =
{\mbox{\small sign}}\left|
\begin{array}{ccc}
{x_1 - x_3} & {y_1 - y_3} & 0 \\
{x_2 - x_3} & {y_2 - y_3} & 0 \\
{x_3} & {y_3} & 1 \\
\end{array}
\right|
$$
}
\lcRawHtml{
<CENTER>
<IMG BORDER=0 SRC="fig/Cartesian_orientation.gif"
   ALIGN=CENTER ALT="Orientation test with Cartesian coordinates"><BR>
</CENTER>
}
With homogeneous coordinates all formulas are homogeneous polynomials
\ccIndexSubitem{homogeneous}{polynomial} 
in the coordinates, \ie, all monomials of homogeneous coordinates
have the same total degree.%
\footnote{the sum of the degrees of each variable in the monomial} 
\ccIndexMainItem{total degree}%
For a sign test for some polynomial expression over Cartesian coordinates 
you get a corresponding sign test with an expression over homogeneous 
coordinates by replacing 
each Cartesian coordinate $x_{ij}$ by $hx_{ij}/hw_j$ and then multiplying by
the least common multiple of the denominators, \ie, some multiple of
the $hw_j$. 
Here, $x_{ij}$ denotes the $i$-th Cartesian coordinate,
$hx_{ij}$ denotes the $i$-th homogeneous coordinate, and
$hw_j$ denotes the homogenizing coordinate of point $j$.
Since all $hw_j$ are positive in \cgal, sign is not affected
by this multiplication. 

\section{Available kernels}
At present, there are two homogeneous and two Cartesian kernels,
one with reference counting\ccIndexMainItem{reference counting} 
(Chapter~\ref{chap:reference_counting}) and one without.
The corresponding kernel classes are
\begin{verbatim}
CGAL::Cartesian< FieldNumberType >
CGAL::Homogeneous< RingNumberType >
CGAL::Simple_cartesian< FieldNumberType >
CGAL::Simple_homogeneous< RingNumberType >
\end{verbatim}
\ccIndexMainItem{kernel traits}
\ccIndexSubitem{kernel}{\ccFont Cartesian}%
\ccIndexSubitem{kernel}{\ccFont Homogeneous}%
\ccIndexSubitem{kernel}{\ccFont Simple_cartesian}%
\ccIndexSubitem{kernel}{\ccFont Simple_homogeneous}%
These are all parameterized by a number type, which is used for storing
the coordinates and the arithmetic in the corresponding primitives and
predicates. Actually, parameterization of \ccc{Homogeneous<>} 
involves two number types. While in the internal homogeneous representation,
an integral number type is sufficient, rational numbers must sometimes be
used outside the internal representation, for example, when the squared length 
of a vector is computed. To represent such rational values, there is a second
number type whose default value is {\tt CGAL::Quotient< RingNumberType >}. 
The type of this number type can be accessed as \ccc{Homogeneous<>::FT}.% 
\ccIndexMainItem{\ccFont FT}
\ccIndexSubitem{kernel}{\ccFont FT}
The internally used number type can be accessed as \ccc{Homogeneous<>::RT}.%
\ccIndexMainItem{\ccFont RT}
\ccIndexSubitem{kernel}{\ccFont RT}
For the sake of a uniform interface for both representations, the
Cartesian kernels provide such types as well. For Cartesian representation,
both \ccc{FT} and \ccc{RT} map to same number type used everywhere in the
implementation of the Cartesian kernel.  

Whenever one wants to ensure that predicates are evaluated correctly,
an exact number type is usually necessary. As exact computations are
time consuming, filtering techniques have been developed. The
predicate is first evaluated using efficient but inexact floating
point arithmetic. At the same time an error bound is computed that is
used to judge the quality of the computed solution. If the solution is
not guaranteed to be correct, the predicate is re-evaluated using
exact arithmetic. The classes
\begin{verbatim}
CGAL::Filtered_kernel< CK >
CGAL::Filtered_kernel_adaptor< CK >
\end{verbatim}
are adaptors that add such a filtering mechanism to the predicates of
a given Kernel \ccc{CK}. Note that only predicates are affected by
these adaptors, the constructions are still the same as in the
original kernel \ccc{CK}. 

The difference between \ccc{Filtered_kernel} and
\ccc{Filtered_kernel_adaptor} is that the latter affects the predicate
functors in the kernel only while the former also affecs the behaviour
of predicates that are implemented as global functions.

\section{Kernel design and conventions}
\ccIndexSubitem{kernel}{design}%
\ccIndexSubitem{design}{kernel}%
Each kernel object is provided as both a stand-alone class, which is 
parameterized by a kernel class (\ccc{Geo_object_D<K>}), and as a type 
in the kernel class (\ccc{K::Geo_object_D}).  While the former use may
be more natural for users not interested in the flexibility of the kernel
(and is compatable with the original kernel design \cite{fgkss-dccga-00}), the
latter syntax should be used in all code distributed with the library
as it allows types in the kernel to be easily exchanged and modified.
Similarly, each operation and construction in the kernel is provided via 
a function object class in the \ccIndexMainItem[t]{function objects}
kernel class and also as either a member function or a global function; 
developers should use the function object classes to gain access to the
functionality.  See~\cite{hhkps-aegk-01} for more details about this 
design and how it is accomplished.

The classes for the geometric objects in the kernel have a 
standardized interface.%
\ccIndexSubitem{kernel}{conventions}
\begin{itemize}
\item
All classes have a \ccc{bbox()} member function computing a bounding box.
\ccIndexMainItem[C]{bbox}
\item
All classes have a \ccc{transform(Aff_transformation_d t)} member function
to compute the object transformed by \ccc{t}.%
\ccIndexMainItem[C]{transform}  
\item
Oriented $d-1$ dimensional objects\footnote{Note that the dimension
of an object might depend on its use. A line in the plane has dimension
$d-1$. As a halfspace, it has dimension $d$.} provide member functions
\ccc{has_on_positive_side(Point_d)}%
\ccIndexMainItem[C]{has_on_positive_side}, \ccc{has_on_boundary(Point_d)}%
\ccIndexMainItem[C]{has_on_boundary}, 
and \ccc{has_on_negative_side(Point_d)}%
\ccIndexMainItem[C]{has_on_negative_side}. Furthermore, there is a member
function \ccc{oriented_side(Point_d)}%
\ccIndexMainItem[C]{oriented_side} returning an object of type
\ccc{CGAL::Oriented_side}.\ccIndexMainItem[C]{Oriented_side}
\item
Full-dimensional bounded objects provide member functions
\ccc{has_on_bounded_side(Point_d)}%
\ccIndexMainItem[C]{has_on_bounded_side}, \ccc{has_on_boundary(Point_d)}%
\ccIndexMainItem[C]{has_on_boundary}, 
and \ccc{has_on_unbounded_side(Point_d)}%
\ccIndexMainItem[C]{has_on_unbounded_side}. Furthermore, there is a member
function \ccc{bounded_side(Point_d)}%
\ccIndexMainItem[C]{bounded_side} returning an object of type
\ccc{CGAL::Bounded_side}.\ccIndexMainItem[C]{Bounded_side}
\item
Oriented objects have a member function \ccc{opposite()}%
\ccIndexMainItem[C]{opposite} returning
the same object with opposite orientation.
\end{itemize}


\section{Number-type based predicates}
\ccIndexSubitem{predicate}{number-type based}
For a number of predicates, there are versions that operate on
the coordinates directly, not on the geometric objects. These
number-type based predicates ease re-use with non-\cgal\ types.

\section{Missing functionality}
\label{sec:NewKernelTraits}
\ccIndexSubitem{predicate}{missing} Kernel traits do not provide
redundant functionality. In particular, they do not provide a right
turn predicate, since a left turn predicate exists. A right turn
functor can be created from the left turn functor using the function
\ccc{CGAL::swap_1}.

Whenever you need a predicate that is not present in the current kernel
traits, you should first try to re-use the available predicates (you
might rewrite the code or implement the new predicate using existing ones).
If this is not feasible (especially for efficiency reasons), we have to 
decide on adding the new predicate to the kernel traits. 
If the new predicate is not too special, it will be added. 
Otherwise you cannot use the kernel as a traits class, but have
to use additional traits.%
\ccIndexSubitem{traits class}{additional} 

See Section \ref{sec:CartesianHomogeneousComputation} on how to derive the
homogeneous version of a predicate from the Cartesian version. 


