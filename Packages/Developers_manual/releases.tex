\chapter{Making Releases}
\label{chap:releases}

\section{Internal Releases}
\label{sec:internal_releases}
\ccIndexSubitem{releases}{internal}

Internal releases are currently created twice a week from a script run
at Utrecht.  This script packages together all the latest versions of
submitted packages into a tar file and then sends mail to the people
responsible for running the test suite so they can begin running the
tests (Section~\ref{sec:autotest_cgal}).

\section{Public Releases}
\label{sec:public_releases}
\ccIndexSubitem{releases}{public}

A public release can be created from an internal release by following the
steps detailed in the
\texttt{README} file in the package \texttt{Release} on the CVS server.
\index{CVS server!Release package@\texttt{Release} package}
This file describes how to create the code, the documentation and the
updated web pages for a new release.

Each public release is tagged on the CVS server with a tag in the following
format \texttt{CGAL\_N\_release}, where \texttt{N} is the release number
with all `.'s replaced by `\_'s (\eg, for release 3.0, the tag is
\texttt{CGAL\_3\_0\_release}.  There is also a branch tag created for
each release with the name \texttt{CGAL\_N\_branch}, where \texttt{N} is
the release number as before.  This facilitates the creation of bug fix
releases.

%\section{Bug Fix Releases}
%\label{sec:bug_fix_releases}
%\ccIndexSubitem{releases}{bug fix}
%\ccIndexMainItem{bug fixe}
