%--------------------------------------------------------
% kd-tree.tex
%    Specification of kd-tree,
%
%  by Eyal Flato, Dan Halperin, Sariel Har-Peled
%--------------------------------------------------------
% History
% Version  1.0, 26/4/97  
%    All about kd-trees.
% Adapted by Hans Tangelder using reference pages
%--------------------------------------------------------

%\documentclass[12pt]{book}
%\usepackage{epsfig,epsf}
%\usepackage{cprog}

%\pagestyle{empty}
%\textwidth 15.6cm 
%\textheight 23 cm
%\topmargin -14mm       
%\evensidemargin 3mm 
%\oddsidemargin 3mm
%\headsep 1cm
%\def\keyword#1{{\em #1\/}}

\renewcommand{\Re}{{\rm I\!\hspace{-0.025em} R}}
%\newcommand{\Section}[1]{Section~{\ref{#1}}}
%\newcommand{\Chapter}[1]{Chapter~{\ref{#1}}}
\newcommand{\new}[1]{\marginpar{\sf #1}}
\newcommand{\kdt}{kd-tree}
\newcommand{\kdts}{kd-trees}
\newcommand{\brc}[1]{\left\{ {#1} \right\}}
\def\Ipe#1{\def\IPEfile{#1}\input{#1}}

\parindent0em
\setlength{\parskip}{1ex minus 0.9ex}
\sloppy

%\begin{document}
%\tableofcontents


\clearpage

\section{Kd-Trees} 
\label{KDT_section}

%This is a simple, but nevertheless efficient, data structure for orthogonal
%range searching.  In particular, {\cgal}  {\kdt}  seems to perform better
%than the theoreticly superior range trees, as implemented in \cgal.
%See Section \ref{KDT_sec:appendix}, for comparative results. Of
%course, the {\cgal}  range-tree implementation is more flexible than the
%{\kdt}  implementation by enabling to layer together range-trees and
%segment-trees in the same data structure, see \Chapter{Range:Trees} for the 
%details.

The implementation of the {\kdt}  is independent of the implementation of
the rest of {\cgal}  and can be used separately from \cgal. 

%In \Section{KDT_sec:intro}, we present the {\kdt}  structure and its
%performance. \Section{KDT_sec:class:main} presents the class
%\ccStyle{Kdtree_d<I>}.

The kd-tree class is parameterized with a traits class, that defines
the interface between the class and the geometric primitives used.
For the description of the traits classes provided by {\cgal}
and the description of the formal requirements for a class to be 
a kd-tree traits class please refer to the reference pages.
% {\cgal}  provides ready-made interface classes that are presented in
% Section~\ref{KDT_sec:def:interface}.
% The formal requirements for a class to be a {\kdt} traits class is
% described in Section~\ref{KDT_sec:def:interface}. 


% \section{Introduction}
% \label{KDT_sec:intro}

For a given set $S = \brc{ p_1, \ldots, p_n }$ on $n$ points in
$\Re^d$, it is sometimes useful to be able to answer {\em orthogonal
   range-searching} on $S$; namely, given an axis parallel query box
$B$ in $\Re^d$, one would like to ``quickly'' determine the subset of
points of $S$ lying inside $B$.




%Orthogonal range-searching is being widely used in databases,
%computational-geometry, and vision {\large \bf (references?)}. 

Several data structures were suggested for that problem. 
Foremost among those, at least theoreticly, is the range-tree data structure
with $O(n \log^d{(n)})$ preprocessing time, $O(n\log^{d-1}(n))$ space,
and $O(\log^{d}(n) + k)$ query time, where $k$ is the output size of
the query.
%See \Chapter{Range:Trees} for the the {\cgal}implementation of range-trees.
A theoreticly inferior data structure is the \kdt, which offers $O(n
\log{n})$ preprocessing time, $O(n)$ space, and $O(n^{1-1/d})$ query
time. The {\kdt}  is a binary tree constructed as follows: we compute
the point $p_i$ of $S$ such that its first coordinate is the median
value among $p_1^1, \ldots, p_n^1$, where $p_i^k$ denote the $k$-th
coordinate of the $i$-th point of $S$. Let $S_1$ denote all the points
of $S$ with first coordinate smaller than $p_i$'s first coordinate, and
let $S_2 = S \setminus S_1$.  Let $T_1,T_2$ be the {\kdts}  constructed
recursively for $S_1, S_2$, respectively.  The {\kdt}  of $S$ is simply
the binary tree having $T_1$ as its left subtree and $T_2$ as its
right subtree. We apply this algorithm recursively, splitting the sets
in the $i$-level of the {\kdt}  using the median point in the $k$-th
coordinate, where $k=(i \;\;mod \;\;n) + 1$. See Figure
\ref{KDT_fig:kdtree} for an illustration.

The resulting data structure has linear size, $O(n\log{n})$
preprocessing time, and can answer a query in $O(n^{1-1/d} +k)$ time,
where $k$ is the size of the query output. See \cite{bkos-cgaa-97}.

\begin{ccTexOnly}

\begin{figure}[hb]
    \begin{center}
        \Ipe{kdtree.ipe}
    \end{center}

    \caption{The partition of a set of  points induced by a  
       {\kdt}  of the points}

    \label{KDT_fig:kdtree}
\end{figure}

\end{ccTexOnly}

%for html
\lcHtml{\label{KDT_fig:kdtree}}
\begin{ccHtmlOnly}
<P>
<center><img border=0 src="./kdtree.gif" alt=" "><br>
The partition of a set of  points induced by a kd-tree of the points</center>
\end{ccHtmlOnly}

\section{Reference Pages for Kd-Trees}
\label{KDT_sec:RefPages}

\begin{ccRefClass}{Kdtree_d<Traits>}

\ccThree{KdtreexdxTraitsx}{}{\hspace*{7.6cm}}
\ccTwo{}{\hspace*{7.6cm}}

%\begin{ccClassTemplate}{Kdtree_d<Traits>} 
    \ccDefinition An object $T$ of the class
    \ccStyle{Kdtree_d<Traits>} is the {\kdt}  induced by a set of points
    in $d$-dimensions.

    \ccInclude{ CGAL/kdtree_d.h}

    \ccTypes
    
    \ccNestedType{Box}{represents an axis-parallel box in
       $d$-dimensions. The box might be unbounded.}
    
    \ccTypedef{typedef Traits::Point Point;}{}
    \ccTypedef{typedef list<Point>  List_points;}{}

    \ccCreation
    \ccCreationVariable{kd_tree}
    
    \ccConstructor{Kdtree_d<Traits>( int dim = 2 );}{construct an
       empty {\kdt}   of dimension \ccStyle{dim}.}

    \ccOperations
    
\ccThree{bool}{}{\hspace*{7.6cm}}

    \ccMethod{bool is_valid(bool verbose = false, int level = 0)
       const;} {perform internal consistency checks to verify the
       correctness of the \kdt}
    
    \ccMethod{void build( list<Point> &l );}{construct the 
       {\kdt}  from the points stored in \ccStyle{l}.  
       \ccPrecond{all the points in \ccStyle{l} are of dimension no
          smaller than the dimension of {{\ccVar} } itself.} }
    
    \ccMethod{void search( back_insert_iterator<List_points>
       result, Box & query_box );}{return into \ccStyle{result} all
       the points of the {{\kdt}  } that lie inside \ccStyle{query_box}}
%\end{ccClassTemplate}

\end{ccRefClass}

%\ccHeading{\ccStyle{Kdtree_d<Traits>::Box}}

%\begin{ccClassTemplate}{Kdtree_d<I>::Box} 
\begin{ccRefClass}[Kdtree_d<Traits>::]{Box} 
    \ccDefinition An object $B$ of the class
    \ccStyle{Box} is a $d$-dimensional box (it may
    be unbounded). A $d$-dimensional box is the set defined by the
    Cartesian set $[l_1, r_1) \times [l_2, r_2) \times \cdots \times
    [l_d,r_d)$.

    \ccCreation
    \ccCreationVariable{box}
    
    \ccConstructor{Box( int d );}{Construct a box corresponding to
       the whole $d$-dimensional space}
    
    \ccConstructor{Box( Point left, Right right, int d );}{ Construct
       the axis parallel box in the $d$-dimensional space defined by
       the points \ccStyle{left}, \ccStyle{right}.}

    \ccOperations
    
    \ccMethod{void set_coord_left( int k, Point & left );}{ set
       the left endpoint of the $k$-th dimensional interval of \ccVar\ {}
       to be the $k$-th coordinate of \ccStyle{left}}
   
    \ccMethod{void set_coord_right( int k, Point & right );}{ set
       the right endpoint of the $k$-th dimensional interval of \ccVar\ {}
       to be the $k$-th coordinate of \ccStyle{right}}
    
    \ccMethod{bool is_in( Point pnt );}{return \ccStyle{true}
       if \ccStyle{pnt} lies inside \ccVar.}
    
    \ccMethod{bool is_coord_in_range( int k, Point pnt );}{ return
       \ccStyle{true} if the $k$-th coordinate of \ccStyle{pnt} lies
       inside the $k$-th dimensional interval of \ccVar.}
%\end{ccClassTemplate}
%\end{ccClass}
\end{ccRefClass}

%********************************************************
\begin{ccRefConcept}{Kdtree_d_traits}

The \ccStyle{Kdtree_d}  class is  parameterized with the
interface class \ccStyle{Traits} which defines the abstract interface
between the \ccStyle{Kdtree_d} class and the data (i.e., points). The following
requirement catalog lists the primitives, i.e.,~types, member functions
etc., that must be defined for a class that can be used to
parameterize \kdts. Ready-made implementation are available
by the \ccStyle{Kdtree_d} default traits classes.

\ccThree{static void}{}{\hspace*{7.1cm}}
\ccTwo{}{\hspace*{7.1cm}}


\begin{ccClass} {Traits}

    \ccCreationVariable{t}

    \ccDefinition
    A class \ccClassName\ that satisfies the requirements of an
    interface class for a \ccStyle{Kdtree_d} class must provide the
    following types and operations. 
  
    \ccTypes
    
    \ccNestedType{Point}{A type to hold a input item.}

    \ccOperations
%    \ccSetTwoOfThreeColumns{4cm}{4cm}

    \ccMethod{static Comparison_result compare(
       int k,
       const Point & p0,
       const Point & p1);} {compare the \ccStyle{k}-th coordinate
       of $p0$ and $p1$.  Return \ccStyle{LARGER} if $p0_k >
       p1_k$, \ccStyle{SMALLER} if $p0_k < p1_k$, or else
       \ccStyle{EQUAL}.  }

    \ccMethod{static void  copy_coord( int  k, Point   & dest, 
       const Point & src );}{Copy the $k$-th coordinate of $src$ to
       the $k$-th coordinate of $dest$.}

    \ccMethod{static int  dimension( const Point  & pnt );}
    {return the dimension of \ccStyle{pnt}}
\end{ccClass}
\end{ccRefConcept}

\begin{ccRefClass}{Kdtree_Interface<Point>}

{\cgal}  contains a default implementation for the Kdtree\_d traits
class, that
may be applied to any
standard class of point provided by \cgal.

The default traits class \ccStyle{Kdtree_Interface<Point>} is
templated with a parameter \ccStyle{Point}, which is required to
supply the following methods:

\begin{itemize}
    \item Constructor, and copy constructor.

    \item \ccStyle{int dimension();} - return the dimension of the point.
    
    \item \ccStyle{operator[](int dim);} - an operator for accessing
    the various coordinates of the given point. Used also to copy
    coordinates between points.
\end{itemize}

There are two other default traits classes \ccStyle{Kdtree_Interface_2d<Point>} 
and \ccStyle{Kdtree_Interface_3d<Point>} which should be used when using 
2D and 3D points from the {\cgal} kernel. This is done since the points in the
kernel do not support changing their coordinates through direct access. 

\end{ccRefClass}
%********************************************************

%********************************************************
\begin{ccRefClass}{Kdtree_Interface_2d<Point>}

The default traits class \ccStyle{Kdtree_Interface_2d<Point>} 
should be used when using 2D points from the {\cgal} kernel.
It is
templated with a parameter \ccStyle{Point}, which is required to
supply the following methods:

\begin{itemize}
    \item Constructor, and copy constructor.

    \item \ccStyle{int dimension();} - return the dimension of the point.
    
    \item \ccStyle{operator[](int dim);} - an operator for accessing
    the various coordinates of the given point. Used also to copy
    coordinates between points.
\end{itemize}

\end{ccRefClass}

%********************************************************
\begin{ccRefClass}{Kdtree_Interface_3d<Point>}

The default traits class \ccStyle{Kdtree_Interface_3d<Point>} 
should be used when using 3D points from the {\cgal} kernel.
It is
templated with a parameter \ccStyle{Point}, which is required to
supply the following methods:

\begin{itemize}
    \item Constructor, and copy constructor.

    \item \ccStyle{int dimension();} - return the dimension of the point.
    
    \item \ccStyle{operator[](int dim);} - an operator for accessing
    the various coordinates of the given point. Used also to copy
    coordinates between points.
\end{itemize}

\end{ccRefClass}

%*********************************************************

%-------------------------------------------------------------------------
% Bibiliography 
%-------------------------------------------------------------------------

%\bibliographystyle{alpha}
%\bibliography{geometry}


%\end{document}

%%% Local Variables: 
%%% mode: latex
%%% TeX-master: t
%%% End: 


