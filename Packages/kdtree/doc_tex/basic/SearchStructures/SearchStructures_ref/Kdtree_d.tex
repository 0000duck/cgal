\newcommand{\kdt}{kd-tree}

\begin{ccRefClass}{Kdtree_d<Traits>}

\ccThree{KdtreexdxTraitsx}{}{\hspace*{7.6cm}}
\ccTwo{}{\hspace*{7.6cm}}

%\begin{ccClassTemplate}{Kdtree_d<Traits>} 
    \ccDefinition An object $T$ of the class
    \ccStyle{Kdtree_d<Traits>} is the {\kdt}  induced by a set of points
    in $d$-dimensions.

    \ccInclude{ CGAL/kdtree_d.h}

    \ccTypes
    
    \ccNestedType{Box}{represents an axis-parallel box in
       $d$-dimensions. The box might be unbounded.}
    
    \ccTypedef{typedef Traits::Point Point;}{}
    \ccTypedef{typedef list<Point>  List_points;}{}

    \ccCreation
    \ccCreationVariable{kd_tree}
    
    \ccConstructor{Kdtree_d<Traits>( int dim = 2 );}{construct an
       empty {\kdt}   of dimension \ccStyle{dim}.}

    \ccOperations
    
\ccThree{bool}{}{\hspace*{7.6cm}}

    \ccMethod{bool is_valid(bool verbose = false, int level = 0)
       const;} {perform internal consistency checks to verify the
       correctness of the \kdt}
    
    \ccMethod{void build( list<Point> &l );}{construct the 
       {\kdt}  from the points stored in \ccStyle{l}.  
       \ccPrecond{all the points in \ccStyle{l} are of dimension no
          smaller than the dimension of {{\ccVar} } itself.} }
    
    \ccMethod{void search( back_insert_iterator<List_points>
       result, Box & query_box );}{return into \ccStyle{result} all
       the points of the {{\kdt}  } that lie inside \ccStyle{query_box}}
%\end{ccClassTemplate}

\end{ccRefClass}
