

\subsection{Two Dimensional Delaunay}
\label{sec:delaunay_2_example}

The \ccc{CGAL::KDS::Delaunay_triangulation_2<Traits, Visitor, Triangulation>} class is the simplist
interesting kinetic data structure we provide. 

The source code, included at the end of the subsection, shows the
basic parts necessary to a \ccc{CGAL::KDS::Delaunay_triangulation_2<Traits, Visitor, Triangulation>}.

In order to simplify implementing, debugging and using kinetic data
structures, the common types needed by each kinetic data structure are
grouped into traits classes, called \ccc{SimulationTraits}. The
framework provides a number of choices which work with appropriate
provided kinetic data structures. The provided traits classes are
named {Exact,Inexact}\_simulation\_traits\_{1,2,3}, depending on
whether computations are performed exactly and the dimension
required. The details of what is in the traits classes are described
both in the
\ccc{SimulationTraits} reference pages and later in the user manual. Here we opt for exact computations in two dimensions.

In order to monitor the Delaunay triangulation as the simulation is
run, we use a visitor. This is a small class, which is passed to the
Delaunay triangulation. The triangulation calls methods on the visitor
class whenever things happen. In the case, the visitor,
\ccc{CGAL::KDS::Delaunay_triangulation_event_log_visitor_2} simply makes a record of each event that occurs.

Once the traits class and kinetic Delaunay have been created, we need
to add points to the simulation. To do this, we add them directly to
the
\ccc{MovingObjectTable} using the \ccc{insert} method. 


Now that the triangulation has been set up at points added (or
scheduled for addition), we can run the kinetic data structure. Here
we ask the simulation to process all events and then using the visitor
to print out a record of all events that occur. Note that there are a
finite number of events since eventually all the points are spead far
apart and simply moving outward. If instead, we had added a
\ccc{CGAL::KDS::Enclosing_box_2<Traits>} to the simulation, then 
there would be an infinite number of events as the points repeatedly
bounce off the walls.

For an equivalent example with a graphical interface, see
demo/Kinetic\_data\_structures/Delaunay\_triangulation\_2.C.

\label{fig:delaunay_2_usage_program}
\ccIncludeExampleCode{Kinetic_data_structures/Delaunay_triangulation_2.C}
