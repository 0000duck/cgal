\section{Todo}

\subsection{General thoughts}

\begin{itemize}

\item there might be a bug in Simple_interval_root due to its use of
  approximate_interval_width (which uses doubles and so might never
  get small enough).

\item The other trajectory modification events need to be documented
  (Set\_moving\_point). Hmmm, I don't remember what this means. 

\item There is no 2d regular triangulation. This is not too much work,
  but a bit (since I will need to refactor my 2d delaunay)

\item the FunctionKernel is not documented much (the options are never
  explained)

\item The is some confusion about where a KDS should get a
  FunctionKernel. Currently they mostly get them from the Simulator
  (which directly provides the RootStack). I am not sure that this is
  the right solution. The better solution may just be to have the
  FunctionKernel (and RootStack) be fetched from the SimulationTraits.

\item The FunctionKernel numeric solver I implemented is not as good
  as the one provided by GSL which is shipped on most linux boxes. It
  is GPL, so the user does have to make a decision about using it.
  Currently, there is a traits class that the user can select if they
  want to use it (at the cost of reassembling the SimulationTraits).

\item The 3d visualizion uses Coin. It is currently all in headers so
  there is really no problem with the user only using it if they have
  coin. It is pretty simple to port it to any other 3d viewer if CGAL
  picks one.

\item The whole SimulationTraits is a bit funny since it allows you to
  a) fetch a bunch of kernels and b) fetch pointers to the
  ActiveObjectsTable and simulator. I don't see an obvious way to keep
  its simplicity and have it make more sense.

\item the Qt stuff is broken since I am waiting for the proper person
  to update the makefile


\end{itemize}

%%% Local Variables: 
%%% mode: latex
%%% TeX-master: t
%%% End: 
