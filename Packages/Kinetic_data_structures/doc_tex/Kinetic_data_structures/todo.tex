\section{Todo}

\subsection{General thoughts}

\begin{itemize}


\item The other trajectory modification events need to be documented
  (Set\_moving\_point)

\item There is no 2d regular triangulation. This is not too much work,
  but a bit (since I will need to refactor my 2d delaunay)

\item the FunctionKernel is not documented much (the options are never
  explained)

\item The is some confusion about where a KDS should get a
  FunctionKernel. Currently they mostly get them from the Simulator
  (which directly provides the RootStack). I am not sure that this is
  the right solution. The better solution may just be to have the
  FunctionKernel (and RootStack) be fetched from the SimulationTraits.

\item I use TNT which is currently put in a include/TNT (parallel to
  include/CGAL). Exposing it like this is a bit problematic. I could
  modify the headers to nest TNT in the CGAL namespace and hide it in
  include/CGAL. I can actually just remove it now as I have another
  solver which is better numerically, but slower, and does not need
  TNT.

\item The FunctionKernel numeric solver I implemented is not as good
  as the one provided by GSL which is shipped on most linux boxes. It
  is GPL, so the user does have to make a decision about using it.
  Currently, there is a traits class that the user can select if they
  want to use it (at the cost of reassembling the SimulationTraits).

\item The 3d visualizion uses Coin. It is currently all in headers so
  there is really no problem with the user only using it if they have
  coin. It is pretty simple to port it to any other 3d viewer if CGAL
  picks one.

\item The whole SimulationTraits is a bit funny since it allows you to
  a) fetch a bunch of kernels and b) fetch pointers to the
  movingpointtable and simulator. I don't see an obvious way to keep
  its simplicity and have it make more sense.

\item the bib entries has not been added yet

\item the Qt stuff is broken since no one has figured out how to add
  my files to CGALQt (or has told me how)


\end{itemize}

%%% Local Variables: 
%%% mode: latex
%%% TeX-master: t
%%% End: 
