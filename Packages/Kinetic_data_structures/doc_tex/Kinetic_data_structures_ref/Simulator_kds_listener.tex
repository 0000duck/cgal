% +------------------------------------------------------------------------+
% | Reference manual page: Simulator_kds_listener.tex
% +------------------------------------------------------------------------+
% | 20.03.2005   Author
% | Package: Kinetic_data_structures
% | 
\RCSdef{\RCSSimulatorkdslistenerRev}{$Revision$}
\RCSdefDate{\RCSSimulatorkdslistenerDate}{$Date$}
% |
%%RefPage: end of header, begin of main body
% +------------------------------------------------------------------------+


\begin{ccRefClass}{KDS::Simulator_kds_listener<Simulator_listener, KDS>}  %% add template arg's if necessary

%% \ccHtmlCrossLink{}     %% add further rules for cross referencing links
%% \ccHtmlIndexC[class]{} %% add further index entries

\ccDefinition
  
The class \ccRefName\ acts as a helper class for kinetic data
structures which want to respond to
\ccc{Simulator::Listener::HAS_AUDIT_TIME} notifications. When kinetic
data structures can audit themselves, the \ccRefName\ calls the
\ccc{audit()} method on the \ccc{KDS}.

\ccInclude{CGAL/KDS/Simulator_kds_listener.h}


\ccCreation
\ccCreationVariable{a}  %% choose variable name

\ccConstructor{Simulator_kds_listener(Simulator::Pointer, KDS *kds);}{default constructor.}

\ccSeeAlso

\ccc{CGAL::KDS::Simulator},
\ccc{CGAL::Listener}.

\ccExample

See the \ccc{CGAL::KDS::Simple_kds_base} source code for a simple
example using this class.


\end{ccRefClass}

% +------------------------------------------------------------------------+
%%RefPage: end of main body, begin of footer
% EOF
% +------------------------------------------------------------------------+

