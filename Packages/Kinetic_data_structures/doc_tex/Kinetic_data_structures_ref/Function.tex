% +------------------------------------------------------------------------+
% | Reference manual page: FunctionKernel.tex
% +------------------------------------------------------------------------+
% | 20.03.2005   Author
% | Package: Kinetic_data_structures
% | 
\RCSdef{\RCSFunctionRev}{$Revision$}
\RCSdefDate{\RCSFunctionDate}{$Date$}
% |
%%RefPage: end of header, begin of main body
% +------------------------------------------------------------------------+


\begin{ccRefConcept}{Function}

%% \ccHtmlCrossLink{}     %% add further rules for cross referencing links
%% \ccHtmlIndexC[concept]{} %% add further index entries

\ccDefinition
  
The concept \ccRefName\ encapsulates what the KDS framework needs to know about functions.

\ccTypes

\ccNestedType{NT}{The basic representational number type.}

\ccOperations

\ccMethod{NT operator()(NT t);}{Evaluate the function at the value \ccc{t}. The result returned can be exact or not depending on whether \ccc{NT} is an exact number type or not.}

\ccHasModels
\ccc{POLYNOMIAL::Polynomial<NT>}

\ccSeeAlso

RootEnumerator, FunctionKernel


\end{ccRefConcept}

% +------------------------------------------------------------------------+
%%RefPage: end of main body, begin of footer
% EOF
% +------------------------------------------------------------------------+

