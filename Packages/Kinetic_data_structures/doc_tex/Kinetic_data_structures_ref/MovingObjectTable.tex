% +------------------------------------------------------------------------+
% | Reference manual page: MovingObjectTable.tex
% +------------------------------------------------------------------------+
% | 20.03.2005   Author
% | Package: Kinetic_data_structures
% | 
\RCSdef{\RCSMovingObjectTableRev}{$Revision$}
\RCSdefDate{\RCSMovingObjectTableDate}{$Date$}
% |
%%RefPage: end of header, begin of main body
% +------------------------------------------------------------------------+


\begin{ccRefConcept}{MovingObjectTable}

%% \ccHtmlCrossLink{}     %% add further rules for cross referencing links
%% \ccHtmlIndexC[concept]{} %% add further index entries

\ccDefinition
  
The concept \ccRefName\ implements a table for holding moving objects which provides notifications when objects change their motion.

See \ccc{CGAL::KDS::Notifying_table<MovingObject>} for details of which methods and types the table should provide. 

%\ccTypes

%\ccNestedType{TYPE}{some nested types}

%\ccCreation
%\ccCreationVariable{a}  %% choose variable name

%\ccConstructor{MovingObjectTable();}{default constructor.}

%\ccOperations

%\ccMethod{void foo();}{some member functions}

\ccHasModels

\ccc{CGAL::KDS::Notifying_table<MovingObject>}

%\ccSeeAlso

%Some\_other\_concept,
%\ccc{some_other_function}.


\end{ccRefConcept}

% +------------------------------------------------------------------------+
%%RefPage: end of main body, begin of footer
% EOF
% +------------------------------------------------------------------------+

