% +------------------------------------------------------------------------+
% | Reference manual page: Listener.tex
% +------------------------------------------------------------------------+
% | 21.03.2005   Author
% | Package: Support
% | 
\RCSdef{\RCSListenerRev}{$Revision$}
\RCSdefDate{\RCSListenerDate}{$Date$}
% |
%%RefPage: end of header, begin of main body
% +------------------------------------------------------------------------+


\begin{ccRefClass}{Listener<Interface>}  %% add template arg's if necessary

%% \ccHtmlCrossLink{}     %% add further rules for cross referencing links
%% \ccHtmlIndexC[class]{} %% add further index entries

\ccDefinition
 
The \ccRefName\ class provides the core of the run time notification
system used by the kinetic data structures package. In short,
notifications are handled through proxy objects called listeners. In order to listen for notifications from an object, called the notifier, you make define a small class called a listener proxy, which inherits from the Listener interface defined by the notifier. When constructing your listner poxy, you pass a reference counted pointer to the notifier, which is used to register the proxy for notifications. When a notification occurs, the notifier calls the \ccc{new_notification} method on the proxy, passing the type of the notification. The proxy stores a reference counted pointer to the notifier, ensuring that there are never any dangling pointers in the system. 

The class \ccRefName\ provides base class for listener proxy objects. An notifier should provide class which
uses this base. To use this base class, implement a class, here called
Interface, which defines a type \ccc{Interface::Notification_type} and a
type \ccc{Interface::Notifier_pointer}.

  The \ccc{Notification_type} is generally an enum with one value for each
  type of notification which can be used.

  The \ccc{Notifier_pointer} is the type of a (ref counted) pointer to the
  object providing the notifications. The ref counter pointer must
  provide a nested type \ccc{Pointer} which is the type of a raw pointer.

  The \ccRefName\ maintains a ref counted pointer to the object performing
  notifications. It is registered for notifications on construction
  and unregistered on destruction using the function \ccc{set_listener} on
  the object providing the notifications. The use of ref counted
  pointers means that as long as the notification object exists, the
  object providing the notifications must exist, ensuring that the
  object providing the notifications is not prematurely destroyed.

  These objects cannot be copied since the notifier only support one
  listener.

  Boost provides a similar functionality in the Boost.Signal
  package. However, it is quite a bit more complex (and
  flexible). This complexity add significantly to compile time and
  (although I did not test this directly), I suspect it is much slower
  at runtime due to the overhead of worrying about signal orders and
  not supporting single signals. In addition, it does not get on well
  with Qt due to collisions with the Qt moc keywords.

  There is also the TinyTL library which implements signals. As of
  writing it did not have any easy support for making sure all
  pointers are valid, so it did not seem to offer significant code
  saving over writing my own.

\ccInclude{CGAL/KDS/Listener.h}

\ccTypes

\ccNestedType{Notifier_pointer}{This type is inherited from the \ccc{Interface} template argument. It is a reference counted pointer type for the object providing notifications.}

\ccNestedType{Notification_type}{The type (usually an enum) used to distinguish different types of notifications. This is inherited from the \ccc{Interface} template argument.}

\ccCreation
\ccCreationVariable{l}  %% choose variable name

\ccConstructor{Listener(Notifier_pointer np);}{The \ccRefName\ subscribes to events coming from the notifier and stores a pointer to the notifier.}

\ccOperations

\ccMethod{Notifier_pointer notifier();}{Return a pointer to the notifier.}

\ccMethod{virtual void new_notification(Notification_type);}{This method is pure virtual. A class which wishes to receive events must inherit from this class and implement this method. The method will then be called whenever there is a notification.}

\ccSeeAlso

\ccc{Multi_listener}.

\ccExample


\ccIncludeExampleCode{Kinetic_data_structures/listener.C}


\end{ccRefClass}

% +------------------------------------------------------------------------+
%%RefPage: end of main body, begin of footer
% EOF
% +------------------------------------------------------------------------+

