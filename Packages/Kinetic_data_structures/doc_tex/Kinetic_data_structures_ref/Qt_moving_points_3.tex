% +------------------------------------------------------------------------+
% | Reference manual page: Qt_moving_points_3.tex
% +------------------------------------------------------------------------+
% | 20.03.2005   Author
% | Package: Kinetic_data_structures
% | 
\RCSdef{\RCSQtmovingpointsRev}{$Revision$}
\RCSdefDate{\RCSQtmovingpointsDate}{$Date$}
% |
%%RefPage: end of header, begin of main body
% +------------------------------------------------------------------------+


\begin{ccRefClass}{KDS::Qt_moving_points_3<Traits, QtGui>}  %% add template arg's if necessary

%% \ccHtmlCrossLink{}     %% add further rules for cross referencing links
%% \ccHtmlIndexC[class]{} %% add further index entries

\ccDefinition
  
The class \ccRefName\ displays a set of set of moving points in 3D.

\ccInclude{CGAL/KDS/IO/Qt_moving_points_3.h}


\ccCreation
\ccCreationVariable{a}  %% choose variable name

\ccConstructor{Qt_moving_points_3(QtGui::Pointer, Traits::Moving_point_table::Pointer);}{default constructor.}


\ccSeeAlso

\ccc{KDS::Qt_moving_weighted_points_3},
\ccc{some_other_function}.

\ccExample

\ccIncludeExampleCode{Kinetic_data_structures/Delaunay_3.cc}| 

\end{ccRefClass}

% +------------------------------------------------------------------------+
%%RefPage: end of main body, begin of footer
% EOF
% +------------------------------------------------------------------------+

