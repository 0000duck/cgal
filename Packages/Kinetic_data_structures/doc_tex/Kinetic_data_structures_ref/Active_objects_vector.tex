% +------------------------------------------------------------------------+
% | Reference manual page: Moving_object_table.tex
% +------------------------------------------------------------------------+
% | 20.03.2005   Author
% | Package: Kinetic_data_structures
% | 
\RCSdef{\RCSMovingobjecttableRev}{$Revision$}
\RCSdefDate{\RCSMovingobjecttableDate}{$Date$}
% |
%%RefPage: end of header, begin of main body
% +------------------------------------------------------------------------+


\begin{ccRefClass}{KDS::Notifying_table<MovingObject>}  %% add template arg's if necessary

%% \ccHtmlCrossLink{}     %% add further rules for cross referencing links
%% \ccHtmlIndexC[class]{} %% add further index entries

\ccDefinition
  

MovingObjects are stored in a vector. This means that access is constant
time, but storage is not generally freed. The only way to be sure is
to remove all reference counts for the table or to call \ccc{clear()}.


\ccInclude{CGAL/KDS/Notifying_table.h}

\ccIsModel

NotifyingTable




\end{ccRefClass}

% +------------------------------------------------------------------------+
%%RefPage: end of main body, begin of footer
% EOF
% +------------------------------------------------------------------------+

