% +------------------------------------------------------------------------+
% | Reference manual page: RootEnumerator.tex
% +------------------------------------------------------------------------+
% | 20.03.2005   Author
% | Package: Kinetic_data_structures
% | 
\RCSdef{\RCSRootEnumeratorRev}{$Revision$}
\RCSdefDate{\RCSRootEnumeratorDate}{$Date$}
% |
%%RefPage: end of header, begin of main body
% +------------------------------------------------------------------------+


\begin{ccRefConcept}{RootStack}

%% \ccHtmlCrossLink{}     %% add further rules for cross referencing links
%% \ccHtmlIndexC[concept]{} %% add further index entries

\ccDefinition
  
The concept \ccRefName\ enumerates through roots of a function.

\ccTypes

\ccNestedType{Root}{The root of a function.}
\ccNestedType{Traits}{The traits class for this concept.}

\ccCreation
\ccCreationVariable{re}  %% choose variable name

\ccConstructor{RootStack();}{default constructor.}

\ccConstructor{RootStack(Function f, Root lb, Root ub, Traits tr);}{Construct a \ccRefName\ over the roots of \ccc{f} in the open interval \ccc{lb} to \ccc{ub}.}

\ccOperations

\ccMethod{void pop();}{Advance to the next root. As a precondition, finished() must be false.}

\ccMethod{Root top();}{Return the current root. As a precondition, finished() must be false.}

\ccMethod{bool empty();}{Return true if there are no more roots left.}

%\ccHasModels

%\ccc{KDS::Numeric_root_enumerator}.

\ccSeeAlso

FunctionKernel, Simulator.



\end{ccRefConcept}

% +------------------------------------------------------------------------+
%%RefPage: end of main body, begin of footer
% EOF
% +------------------------------------------------------------------------+

