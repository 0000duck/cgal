% +------------------------------------------------------------------------+
% | Reference manual page: Delaunay_triangulation_3.tex
% +------------------------------------------------------------------------+
% | 20.03.2005   Author
% | Package: Kinetic_data_structures
% | 
\RCSdef{\RCSDelaunaytriangulationRev}{$Revision$}
\RCSdefDate{\RCSDelaunaytriangulationDate}{$Date$}
% |
%%RefPage: end of header, begin of main body
% +------------------------------------------------------------------------+


\begin{ccRefClass}{KDS::Delaunay_triangulation_3<Traits, Visitor, Triangulation>}  %% add template arg's if necessary

%% \ccHtmlCrossLink{}     %% add further rules for cross referencing links
%% \ccHtmlIndexC[class]{} %% add further index entries

\ccDefinition
  
The class \ccRefName\ maintains a Delaunay triangulation on top of the
points contained in a \ccc{ActiveObjectsTable}. It has one main method
of interest. \ccc{triangulation()} which returns the triangulation it
is maintaining. In addition, as an optimisation, you can turn on and
off whether it is currently maintaining its certificates. This allows
a large number of changes to the underlying points to be made at one
time without recomputing the certificates each time a single point
changes.

The class \ccc{KDS::Qt_Delaunay_triangulation_3} displays a kinetic
Delaunay triangulation in three dimensions using the Coin library.

The optional \ccc{Visitor} template argument is a model of
\ccc{DelaunayTriangulationVisitor_3} and can be used to monitor
changes in the kinetic data structure.

The optional \ccc{Triangulation} template argument must be a model of
a static Delaunay triangulation and have
\ccc{CGAL::KDS::Delaunay_triangulation_cell_base_3<Traits>} a the cell
type.

\ccInclude{CGAL/KDS/Delaunay_triangulation_3.h}


\ccTypes

\ccNestedType{Triangulation}{The type of the Delaunay triangulation it maintains. This is a \ccc{CGAL::Delaunay_triangulation} built using the \ccc{InstantaneousKernel} as the kernel.}

\ccNestedType{Listener}{The base class to extend if you wish to subscribe to events. The \ccRefName\ has exactly one event \ccc{TRIANGULATION} which occurs when the triangulation changes.}

\ccCreation
\ccCreationVariable{dt}  %% choose variable name

\ccConstructor{Delaunay_triangulation_3(Simulator::Pointer sp, ActiveObjectsTable::Pointer mot);}{Maintain the Delaunay triangulation of the points in \ccc{mot}.}

\ccOperations

\ccMethod{const Triangulation* triangulation();}{Access the triangulation that is maintained. }

\ccMethod{bool has_certificates();}{This method returns true if the \ccRefName\ is currently maintaining certificates for a Delaunay triangulation. }

\ccMethod{void set_has_certificates(bool tf);}{This method allows you to control whether the triangulation is maintaining certificates.}

\ccSeeAlso

\ccc{KDS::Regular_triangulation_3},
\ccc{KDS::Delaunay_triangulation_2}, 
\ccc{Listener}, 
\ccc{KDS::Qt_Delaunay_triangulation_3}.

\ccExample

\ccIncludeExampleCode{Kinetic_data_structures/Delaunay_triangulation_3.C}



\end{ccRefClass}

% +------------------------------------------------------------------------+
%%RefPage: end of main body, begin of footer
% EOF
% +------------------------------------------------------------------------+

