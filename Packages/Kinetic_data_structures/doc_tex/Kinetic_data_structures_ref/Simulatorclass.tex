% +------------------------------------------------------------------------+
% | Reference manual page: Simulator.tex
% +------------------------------------------------------------------------+
% | 20.03.2005   Daniel Russel
% | Package: Kinetic_data_structures
% | 
\RCSdef{\RCSSimulatorRev}{$Revision$}
\RCSdefDate{\RCSSimulatorDate}{$Date$}
% |
%%RefPage: end of header, begin of main body
% +------------------------------------------------------------------------+


\begin{ccRefClass}{KDS::Simulator<FunctionKernel, EventQueue>}  %% add template arg's if necessary

%% \ccHtmlCrossLink{}     %% add further rules for cross referencing links
%% \ccHtmlIndexC[class]{} %% add further index entries

\ccDefinition
  
The class \ccRefName\ controls kinetic data structures by maintaining
a concept of time and ensuring that events are processed when
necessary. 

In addition, the \ccRefName\ can call on the kinetic data structures
to audit themselves at appropriate times. When the last event
processed and the next to be processed have different times, then
there is a rational value of time at which all kinetic data structures
should be non-degenerate (since there are no events at that time). At
such a time, kinetic data structures can easily verify their
correctness by checking that all the certificate predicates have the
correct value. When exactness checks are enabled, whenever the last
event processed and the next event to be processed have different
times, a
\ccc{KDS::Simulator::Listener::HAS_AUDIT_TIME} notification is made. Kinetic
data structures can listen for that event, and when it is made, they
can call \ccc{KDS::Simulator::audit_time()} to get the time value and
then verify that their structure is correct.

\ccInclude{CGAL/KDS/Simulator.h}

\ccIsModel

InstantaneousKernel



\ccCreation
\ccCreationVariable{sim}  %% choose variable name

\ccConstructor{Simulator(const Time start=Time(0), const Time end= Time::infinity());}{Construct a \ccRefName\ which will process events between times start and end (events outside this window will be discarded).}

\ccExample

\ccRefName\ is used in all of the example programs, such as the sorting example
\ccIncludeExampleCode{Kinetic_data_structures/sort.C}.



\end{ccRefClass}

% +------------------------------------------------------------------------+
%%RefPage: end of main body, begin of footer
% EOF
% +------------------------------------------------------------------------+

