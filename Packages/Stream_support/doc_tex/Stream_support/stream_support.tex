% +------------------------------------------------------------------------+
% | CGAL Reference Manual:  stream_support.tex
% +------------------------------------------------------------------------+
% | Stream Support for CGAL
% |
% | 09.06.1997   Lutz Kettner
% | 
\RCSdef{\streamSupportRev}{$Revision$}
\RCSdefDate{\streamSupportDate}{$Date$}
% +------------------------------------------------------------------------+

\beforecprogskip\medskipamount
\aftercprogskip\medskipamount
\ccParDims

\chapter{Stream Support}
\label{chapterStreamSupport}
\ccChapterRelease{\streamSupportRev. \ \streamSupportDate}\\
\ccChapterAuthor{Lutz Kettner}

% +------------------------------------------------------------------------+
\begin{ccClassTemplate}{Ostream_iterator<T,Stream>}
\section{Stream Iterator}
\ccThree{OutputIterator}{rand}{}

\ccDefinition

The class \ccClassTemplateName\ is an output iterator adaptor for the
output stream class \ccc{Stream} and value type $T$.

\ccInclude{CGAL/IO/Ostream_iterator.h}

\ccCreation
\ccCreationVariable{o}

\ccConstructor{Ostream_iterator( Stream& s);}{creates an output
  iterator \ccVar\ writing to $s$.}

\ccOperations

\ccVar\ fulfills the requirements for an output iterator.

\ccImplementation

The \ccc{operator*()} in class \ccClassTemplateName\ uses a proxy class. 

\end{ccClassTemplate}

% +------------------------------------------------------------------------+
\newpage
 
\begin{ccClassTemplate}{Istream_iterator<T,Stream>}
\section{Stream Iterator}

\ccDefinition

The class \ccClassTemplateName\ is an input iterator adaptor for the
input stream class \ccc{Stream} and value type $T$. It is particularly
useful for classes that are similar but not compatible to \ccc{std::istream}.

\ccInclude{CGAL/IO/Istream_iterator.h}

\ccCreation
\ccCreationVariable{i}

\ccConstructor{Istream_iterator();}{creates an end-of-stream
  iterator \ccVar. This is a past-the-end iterator, and it is useful
  when constructing a range.}

\ccConstructor{Istream_iterator( Stream& s);}{creates an input
  iterator \ccVar\ reading from $s$. When $s$ reaches end of stream,
  this iterator will compare equal to an end-of-stream iterator
  created using the default constructor.}

\ccOperations

\ccVar\ fulfills the requirements for an input iterator.

\ccExample

The following program reads points from a \ccc{Window_stream} until
the right mouse button gets clicked.

\ccIncludeExampleCode{Stream_support_LEDA/Istream_iterator.C}

\end{ccClassTemplate}

% +------------------------------------------------------------------------+
\begin{ccClass}{Verbose_ostream}
\section{Verbose ostream}
\ccThree{Verbose_ostreammm&}{rand}{}

\ccDefinition

The class \ccClassTemplateName\ can be used as an output stream. The
stream output operator \ccc{<<} is defined for any type. The
class \ccClassTemplateName\ stores in an internal state a stream and
whether the output is active or not. If the state is active, the
stream output operator \ccc{<<} uses the internal stream to output
its argument. If the state is inactive, nothing happens.

\ccInclude{CGAL/IO/Verbose_ostream.h}

\ccCreation
\ccCreationVariable{verr}

\ccConstructor{Verbose_ostream( bool active = false, 
  std::ostream& out = std::cerr);}{creates an output stream with state set to
  \ccc{active} that writes to the stream \ccc{out}.}

\ccOperations

\ccMethod{template < class T >
          Verbose_ostream&  operator<<( const T& t);}{}

\ccExample

The class \ccClassTemplateName\ can be conveniently used to implement
for example the \ccc{is_valid()} member function for triangulations or
other complex data structures.

\begin{verbatim}
    bool is_valid( bool verbose = false, int level = 0) {
        Verbose_ostream verr( verbose);
        verr << "Triangulation::is_valid( level = " << level << ')' << endl;
        verr << "    Number of vertices = " << size_of_vertices() << endl;
        // ...
    }
\end{verbatim}

\end{ccClass}

% +--------------------------------------------------------+
% restore default column and paragraph layout
\ccParDims
\beforecprogskip\parskip
\aftercprogskip0pt

% EOF
