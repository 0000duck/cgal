%% Copyright (c) 2005  Foundation for Research and Technology-Hellas (Greece).
%% All rights reserved.
%%
%% This file is part of CGAL (www.cgal.org); you may redistribute it under
%% the terms of the Q Public License version 1.0.
%% See the file LICENSE.QPL distributed with CGAL.
%%
%% Licensees holding a valid commercial license may use this file in
%% accordance with the commercial license agreement provided with the software.
%%
%% This file is provided AS IS with NO WARRANTY OF ANY KIND, INCLUDING THE
%% WARRANTY OF DESIGN, MERCHANTABILITY AND FITNESS FOR A PARTICULAR PURPOSE.
%%
%% $Source$
%% $Revision$ $Date$
%% $Name$
%%
%% Author(s)     : Menelaos Karavelas <mkaravel@tem.uoc.gr>

This chapter describes an adaptor that adapts two-dimensional Delaunay
graphs to the corresponding Voronoi diagrams. We start with a few
definitions and a description of the issues that this adaptor
addresses in Section~\ref{sec:vda2-intro}. The software design
of the Voronoi diagram adaptor package is described in
Section~\ref{sec:vda2-design}. In Section~\ref{sec:vda2-traits} we
discuss the traits required for performing the adaptation, and finally
in Section~\ref{sec:vda2-examples} we present a few examples using
this adaptor.

\section{Introduction}
\label{sec:vda2-intro}

In this chapter when we refer to Voronoi diagrams (or their dual
graphs) we actually refer to instances of abstract Voronoi diagrams \cite{}.
A typical approach when we want to compute abstract Voronoi diagrams is to
actually compute their dual graphs, which we will call {\em Delaunay
graphs}. In general these graphs have faces with three or more
edges. However, when the generators of the Voronoi diagram are in
general position, the dual Delaunay graph is actually a graph with
triangular faces, which is often referred to as a Delaunay
triangulation if it can be embedded with straight line segments. In
fact, in many applications it is desirable to have triangulations, so
even in the cases where the actual Delaunay graph contains
non-triangular faces, these faces are triangulated.

One drawback of computing triangulated Delaunay graphs is that their
dual graphs are not always the original Voronoi diagrams. If the
original generating data contain subsets in degenerate positions, the
Voronoi diagrams, that are duals of triangulated Delaunay graphs,
contain certain degenerate features, namely edges with zero
length or faces with zero area. Another drawback is that users
typically want to view Voronoi diagrams as planar subdivisions,
instead of graphs with triangular faces which is the typical API
provided.

The purpose of this package is to provide an adaptor that
addresses the above-mentioned issues. Firstly, it adapts
triangulated Delaunay graphs to look like planar subdivisions or
arrangements, and secondly it removes the degenerate features that
where artificially introduced by the requirement to produce
triangulated Delaunay graphs.

The adaptor that this package offers is a non-mutable one. Once
the adaptor has been constructed using a Delaunay graph, the Delaunay
graph stored internally cannot be modified any longer, nor its internals
can be accessed in a modifiable way. Although this could be
restrictive in some cases, it allows for optimizations that greatly
enhance the performance of the adaptation. For example, when iterating
over the edges of the dual Delaunay graph to get the dual edges of the
Voronoi diagram we have to test whether the dual Voronoi edge has zero
length or not. Instead of performing this geometric test every time,
which is a test that can be very expensive when we have degenerate or
near degenerate configurations, we can compute the result of this test
once and cash it for future use. This is actually something that we do
in this package.

\section{Software design}
\label{sec:vda2-design}

The \ccc{Voronoi_diagram_adaptor_2<DG,VT>} class is parameterized by
two template parameters. The first one must be a model of the
\ccc{DelaunayGraph_2} concept. It corresponds to the API required by
an object representing a Delaunay graph. All classes of \cgal that
represent duals of Voronoi diagrams are models of this concept,
namely, Delaunay triangulations, regular triangulations, Apollonius
graphs and duals of segment Voronoi diagrams.
%
The second template parameter must be a model of the
\ccc{VoronoiTraits_2} concept. We discuss this concept in detail in
Section~\ref{sec:vda2-traits}.

The \ccc{Voronoi_diagram_adaptor_2<DG,VT>} class has been
intentionally designed to provide an API similar to the arrangements
class in \cgal: Voronoi diagrams are special case of arrangements
after all. The API of the two classes, however, could not be
identical. The reason is that arrangements in \cgal do not yet support
more than one unbounded faces, or equivalently, cannot handle
unbounded curves. On the contrary, a Voronoi diagram defined over at
least two generating sites, has at least two unbounded faces.

On a more practical basis, arrangements in \cgal require as input a
set of curves. In the case of Voronoi diagrams these curves lie on
bisectors of generating sites, and can be bounded arcs, semi-bounded
arcs or the bisectors themselves. Constructing the bisectors can be
a complicated task, yielding a complicated outcome, which becomes even
more complicated when we want to compute their endpoints, which are
the vertices in the Voronoi diagram. Consider, for example, the
Voronoi diagram of segments: if the segments' coordinates are represented
by ring number type, computing the Voronoi bisectors and Voronoi
vertices requires a field number type that also supports square roots
on top of the field operations.

On the other hand, we can fully determine the combinatorial structure
of the Voronoi diagram using the dual Delaunay graph, modulo of course
the degenerate features we discussed above. In other words, in order
to compute the combinatorial representation of the Voronoi diagram we
do not really need to know the bisecting curves or arcs, or the
Voronoi vertices. Furthermore, not only can we determine the
combinatorial structure without computing the Voronoi features, but
can also determine whether a feature is degenerate or not without
computing its geometry. More specifically, determining whether the dual
Voronoi edge of an edge in the Delaunay graph is degenerate (of zero
length) or not can be done without really computing the curve itself.
For example, in the case of the Delaunay triangulation, determining
whether the dual Delaunay edge has zero length amounts to performing
an incircle test.

On a more technical level, the \ccc{Voronoi_diagram_adaptor_2<DG,VT>}
class imitates the representation of the Voronoi diagram (seen as a
planar subdivision) by a DCEL (Doubly Connected Edge List) data
structure. We have vertices (the Voronoi vertices), halfedges
(oriented versions of the Voronoi edges) and faces (the Voronoi
cells). In particular, we can basically perform every operation we can
perform in a standard DCEL data structure:
\begin{itemize}
\item go from a halfedge to its next and previous in the face;
\item go from one face to an adjacent one through a halfedge and its
  opposite halfedge;
\item walk around the boundary of a face;
\item enumerate the halfedges adjacent to a vertex
\item from a halfedge, access the adjacent face;
\item from a face, access an adjacent halfedges;
\item from a halfedges, access its source and target vertex;
\item from a vertex, access an adjacent halfedge.
\end{itemize}
In addition to the above possibilities for traversal, we can also
traverse the following features through iterators:
\begin{itemize}
\item the vertices of the Voronoi diagram;
\item the edges or halfedges of the Voronoi diagram;
\item the faces of the Voronoi diagram;
\item the bounded faces of the Voronoi diagram.
\item the bounded halfedges of the Voronoi diagram.
\item the unbounded faces of the Voronoi diagram;
\item the unbounded halfedges of the Voronoi diagram;
\end{itemize}


\section{The Voronoi traits}
\label{sec:vda2-traits}

The \ccc{VoronoiTraits_2} concept defines the types and
predicates required by the adaptor in order to perform the
adaptation.
%
Primarily, it provides functors for detecting whether the dual Voronoi
edge (resp. face) of an edge (resp. vertex) in the triangulated
Delaunay graph is degenerate, i.e., it has zero length (resp. zero
area). Secondly, it provides types for representing the 
geometric features of the Voronoi diagram, namely its edges and
vertices. From the design point of view, the models of this concept
complement the geometric traits required for computing the
corresponding Delaunay graph.
%
Finally, it defines a tag that indicates whether point
location queries are to be supported by the Voronoi diagram
adaptor. If such queries are to be supported, a functor is
required. Given a query point, this functor should return information
related to how many and which generators of the Voronoi diagram are at
equal and minimal distance from the query point. In particular, if the
query point is closest to a single generator, the vertex handle of the
Delaunay graph corresponding to this generator is returned. If the
query point is closest to exactly two generators, the edge of the
Delaunay graph connecting the two generators is returned. If three (or
more) generators are closest to the query point, the face handle of
the face in the Delaunay graph connecting these generators is
returned. This way of abstracting the point location mechanism allows
for multiple different point location strategies, which are passed to
the Voronoi diagram adaptor through different models of the Voronoi
traits concept.
