%% Copyright (c) 2005  Foundation for Research and Technology-Hellas (Greece).
%% All rights reserved.
%%
%% This file is part of CGAL (www.cgal.org); you may redistribute it under
%% the terms of the Q Public License version 1.0.
%% See the file LICENSE.QPL distributed with CGAL.
%%
%% Licensees holding a valid commercial license may use this file in
%% accordance with the commercial license agreement provided with the software.
%%
%% This file is provided AS IS with NO WARRANTY OF ANY KIND, INCLUDING THE
%% WARRANTY OF DESIGN, MERCHANTABILITY AND FITNESS FOR A PARTICULAR PURPOSE.
%%
%% $Source$
%% $Revision$ $Date$
%% $Name$
%%
%% Author(s)     : Menelaos Karavelas <mkaravel@tem.uoc.gr>

\section{Examples}
\label{sec:vda2-examples}

In this section we present two examples that show how to traverse the
Voronoi diagram or how to perform point location queries.

\subsection{First example}

The following example shows how we can compute the number of connected
components of the Voronoi skeleton (this is the union of Voronoi edges
and vertices), by appropriately traversing the edges and vertices of
the Voronoi diagram using the Voronoi diagram adaptor. The algorithm
applied is a simple DFS algorithm. In the same example, the number of
bisecting arcs, bisecting half-bisectors (semi-unbounded arcs) and
full bisectors are computed.

\ccIncludeExampleCode{Voronoi_diagram_2/connected_components.C}

\subsection{Second example}

The following example shows how we can perform point location using
the Voronoi diagram adaptor.

\ccIncludeExampleCode{Voronoi_diagram_2/point_location.C}
