%% Copyright (c) 2005  Foundation for Research and Technology-Hellas (Greece).
%% All rights reserved.
%%
%% This file is part of CGAL (www.cgal.org); you may redistribute it under
%% the terms of the Q Public License version 1.0.
%% See the file LICENSE.QPL distributed with CGAL.
%%
%% Licensees holding a valid commercial license may use this file in
%% accordance with the commercial license agreement provided with the software.
%%
%% This file is provided AS IS with NO WARRANTY OF ANY KIND, INCLUDING THE
%% WARRANTY OF DESIGN, MERCHANTABILITY AND FITNESS FOR A PARTICULAR PURPOSE.
%%
%% $Source$
%% $Revision$ $Date$
%% $Name$
%%
%% Author(s)     : Menelaos Karavelas <mkaravel@tem.uoc.gr>



\begin{ccRefConcept}{VoronoiTraits_2}

%% \ccHtmlCrossLink{}     %% add further rules for cross referencing links
%% \ccHtmlIndexC[concept]{} %% add further index entries
\ccDefinition

The concept \ccc{VoronoiTraits_2} defines the combinatorial and
geometric predicate requirements for the
\ccc{Voronoi_diagram_adaptor_2<DG,VT>} class. The
\ccc{VoronoiTraits_2} concept essentially defines the requirements on
the predicates that determine whether a feature of the triangulated
Delaunay graph has a degenerate dual feature in the Voronoi diagram
that is the dual of the triangulated Delaunay graph. It also provides
the types for a Voronoi vertex or a Voronoi edge and methods for
constructing these objects.

\ccRefines
\ccc{DefaultConstructible}\\
\ccc{CopyConstructible}\\
\ccc{Assignable}

\ccTypes
\ccTwo{VoronoiTraits_2::Face_degeneracy_tester+}{}
%
\ccNestedType{Point_2}{A type for a point.}
\ccGlue
\ccNestedType{Site_2}{A type for the generators of the Voronoi
  diagram.}
\ccGlue
\ccNestedType{Dual_graph}{A type for the triangulated Delaunay
  graph. The type \ccc{Dual_graph} must be a model of the
  \ccc{DelaunayGraph_2} concept.}
\ccGlue
\ccNestedType{Voronoi_vertex_2}{A type for Voronoi vertices. It must
  be a model of the \ccc{VoronoiVertex_2} concept.}
\ccGlue
\ccNestedType{Voronoi_edge_2}{A type for Voronoi edges. It must
  be a model of the \ccc{VoronoiEdge_2} concept.}
\ccGlue
\ccNestedType{Edge_degeneracy_tester}{A type for the predicate that
  tests whether a Voronoi edge is degenerate, i.e., it has zero
  length. It must be a model of the \ccc{EdgeDegeneracyTester} concept.}
\ccGlue
\ccNestedType{Face_degeneracy_tester}{A type for the predicate that
  tests whether a Voronoi face is degenerate, i.e., it has zero
  area. It must be a model of the \ccc{FaceDegeneracyTester} concept.}
\ccGlue
\ccNestedType{Has_nearest_site_2}{A tag for determining if the Voronoi
  traits can provide a functor for performing nearest site queries.
  This tag is equal to either \ccc{CGAL::Tag_true} (a nearest site
  query functor is available) or \ccc{CGAL::Tag_false} (a nearest site
  query functor is not available).}
\ccGlue
\ccNestedType{Nearest_site_2}{A type for a functor the performs
  nearest site queries. It must be a model of the concept
  \ccc{DelaunayGraphNearestSite_2}. This type is required only if
  \ccc{Has_nearest_site_2} is equal to \ccc{CGAL::Tag_true}.}

\ccThree{typedef typename Dual_graph::Vertex_handle}{Vertex_handle+}{}
\ccThreeToTwo
\ccTypedef{typedef typename Dual_graph::Vertex_handle
  Vertex_handle;}{Handle to the vertices of the Delaunay graph.}
\ccTypedef{typedef Voronoi_edge_2 Curve;}{This type is provided in
  order to comply with the
  \ccc{CGAL::Arrangement_2<Dcel,Traits,Base_node>} API.}


\ccCreationVariable{vt}  %% choose variable name

\ccHeading{Access to objects}
\ccThree{Edge_degeneracy_tester}{vt.edge_degeneracy_tester_object()+}{}
\ccMethod{Edge_degeneracy_tester edge_degeneracy_tester_object();}{}
\ccGlue
\ccMethod{Face_degeneracy_tester face_degeneracy_tester_object();}{}
\ccGlue
\ccMethod{Point_locator point_locator_object();}{This method is
  required only if \ccc{Has_point_locator} is equal to \ccc{CGAL::Tag_true}.}

%
%
% Construction of edges and vertices
%
\ccHeading{Construction of Voronoi vertices and edges}
\ccThree{Voronoi_vertex_2}{make_vertex(Vertex)+}{}
\ccThreeToTwo
\ccMethod{Voronoi_vertex_2 make_vertex(Vertex_handle v1, Vertex_handle
  v2, Vertex_handle v3);}
{Creates the Voronoi vertex defined by the sites corresponding to the
  vertex handles \ccc{v1}, \ccc{v2} and \ccc{v3}.}
\ccGlue
\ccMethod{Voronoi_edge_2 make_edge(Vertex_handle v1, Vertex_handle
  v2);}
{Creates the oriented Voronoi edge that is the bisector of the sites
  corresponding to the vertices \ccc{v1} and \ccc{v2}. The Voronoi
  edge in this case is a curve homeomorphic to a line and, thus, has
  no adjacent Voronoi vertices. The site corresponding to \ccc{v1} is
  on the positive halfspace w.r.t. the bisector.}
\ccGlue
\ccMethod{Voronoi_edge_2 make_edge(Vertex_handle v1, Vertex_handle
  v2, Vertex_handle v3, bool is_src);}
{Creates the oriented Voronoi edge that lies on the bisector of the sites
  corresponding to the vertices \ccc{v1} and \ccc{v2}. The Voronoi
  edge in this case is a curve homeomorphic to a ray and, thus, has
  a single adjacent Voronoi vertex. If \ccc{is_src} is \ccStyle{true},
  the Voronoi vertex is the source of the bisecting ray and, more precisely,
  it is the Voronoi vertex of the sites of \ccc{v2}, \ccc{v1} and
  \ccc{v3}. If \ccc{is_src} is \ccStyle{false}, the Voronoi vertex is
  the target of the bisecting ray and corresponds to the Voronoi
  vertex of the sites of \ccc{v1}, \ccc{v2} and \ccc{v3}. The site
  corresponding to \ccc{v1} is on the positive halfspace w.r.t. the
  supporting bisector.}
\ccGlue
\ccMethod{Voronoi_edge_2 make_edge(Vertex_handle v1, Vertex_handle
  v2, Vertex_handle v3, Vertex_handle v4);}
{Creates the oriented Voronoi edge that lies on the bisector of the sites
  corresponding to the vertices \ccc{v1} and \ccc{v2}. The Voronoi
  edge in this case is a curve homeomorphic to a segment and, thus, has
  two adjacent Voronoi vertices. The source of the bisecting segment
  is the Voronoi vertex of the sites of \ccc{v2}, \ccc{v1} and
  \ccc{v3}, whereas the target of the bisecting segment is the Voronoi
  vertex of the sites of \ccc{v1}, \ccc{v2} and \ccc{v4}. The site
  corresponding to \ccc{v1} is on the positive halfspace w.r.t. the
  supporting bisector.}

\ccHasModels
\ccc{CGAL::Apollonius_graph_Voronoi_traits_2<AG2>}\\
\ccc{CGAL::Delaunay_triangulation_Voronoi_traits_2<DT2>}\\
\ccc{CGAL::Regular_triangulation_Voronoi_traits_2<RT2>}\\
\ccc{CGAL::Segment_Voronoi_diagram_Voronoi_traits_2<SVD2>}\\

\ccSeeAlso
\ccc{DelaunayGraph_2}\\
\ccc{VoronoiVertex_2}\\
\ccc{VoronoiEdge_2}\\
\ccc{EdgeDegeneracyTester}\\
\ccc{FaceDegeneracyTester}\\
\ccc{DelaunayGraphPointLocator_2}\\
\ccc{CGAL::Voronoi_diagram_adaptor_2<DG,VT>}
\end{ccRefConcept}

% +------------------------------------------------------------------------+
%%RefPage: end of main body, begin of footer
% EOF
% +------------------------------------------------------------------------+

