%% Copyright (c) 2005  Foundation for Research and Technology-Hellas (Greece).
%% All rights reserved.
%%
%% This file is part of CGAL (www.cgal.org); you may redistribute it under
%% the terms of the Q Public License version 1.0.
%% See the file LICENSE.QPL distributed with CGAL.
%%
%% Licensees holding a valid commercial license may use this file in
%% accordance with the commercial license agreement provided with the software.
%%
%% This file is provided AS IS with NO WARRANTY OF ANY KIND, INCLUDING THE
%% WARRANTY OF DESIGN, MERCHANTABILITY AND FITNESS FOR A PARTICULAR PURPOSE.
%%
%% $Source$
%% $Revision$ $Date$
%% $Name$
%%
%% Author(s)     : Menelaos Karavelas <mkaravel@tem.uoc.gr>



\begin{ccRefConcept}{VoronoiEdge_2}

%% \ccHtmlCrossLink{}     %% add further rules for cross referencing links
%% \ccHtmlIndexC[concept]{} %% add further index entries
\ccDefinition

The concept \ccc{VoronoiEdge_2} defines the requirements of the
\ccc{Voronoi_edge_2} type defined in the \ccc{VoronoiTraits_2}
concept. Essentially it provides the requirements for a class
representing oriented edges (arcs) in the Voronoi diagram.

\ccRefines
\ccc{DefaultConstructible}\\
\ccc{CopyConstructible}\\
\ccc{Assignable}

\ccTypes
\ccTwo{VoronoiEdge_2::Voronoi_vertex_2+}{}
%
\ccNestedType{Dual_graph}{A type for the Delaunay graph. It must be a
  model of the \ccc{DelaunayGraph_2} concept.}
\ccGlue
\ccNestedType{Point_2}{A type for a point.}
\ccGlue
\ccNestedType{Site_2}{A type for the generators of the Voronoi diagram.}
\ccGlue
\ccNestedType{Voronoi_vertex_2}{A type for a vertex of the Voronoi
  diagram. It must a model of the \ccc{VoronoiVertex_2} concept.}

\ccCreationVariable{e}  %% choose variable name


\ccHeading{Access Methods}
\ccThree{Voronoi_vertex_2}{e.opposite()+}{}
\ccThreeToTwo
%
\ccMethod{Voronoi_vertex_2 source();}{returns the source Voronoi
  vertex of the Voronoi edge.
\ccPrecond{\ccc{has_source()} must be true.}}
\ccGlue
\ccMethod{Voronoi_vertex_2 target();}{returns the target Voronoi
  vertex of the Voronoi edge.
\ccPrecond{\ccc{has_target()} must be true.}}
%
\ccGlue
\ccMethod{VoronoiEdge_2 opposite();}{return the Voronoi edge with the
  opposite orientation.}

A Voronoi edge is defined by four generators, either finite or the
site at infinity. If we consider Voronoi edges locally, we can think
of them as lying on the horizontal axis with orientation from minus to
plus infinity. In this setting the four defining generators can be
thought of as occupying the four major directions \ccStyle{north},
\ccStyle{south}, \ccStyle{west} and \ccStyle{east}, where
\ccStyle{north} means above the horizontal axis (on the positive
side of the axis), \ccStyle{south} means below the horizontal axis (on the
negative side of the axis), \ccStyle{west} means towards minus
infinity and \ccStyle{east} means towards plus infinity. Notice that
then the Voronoi vertices, that is the endpoints of the Voronoi edge,
are defined by the generator triplets $\{$\ccStyle{south},
\ccStyle{north}, \ccStyle{west}$\}$ (this is the source of the Voronoi
edge) and $\{$\ccStyle{north}, \ccStyle{south}, \ccStyle{east}$\}$
(this is the target  of the Voronoi edge). Notice that \ccStyle{north}
and \ccStyle{south} can never correspond to the site at infinity. If
either \ccStyle{west} or \ccStyle{east}, but not both, are the
site an infinity, then the Voronoi edge is essentially a bisecting
ray, whereas if both \ccStyle{west} and \ccStyle{east} are the site at
infinity the Voronoi edge is the entire bisector of \ccStyle{north}
and \ccStyle{south}. The following methods provide access to these
four defining sites:

\ccThree{Site_2}{e.north()+}{}
\ccThreeToTwo
%
\ccMethod{Site_2 north();}{return the defining site to the north of
  the Voronoi edge.}
\ccGlue
\ccMethod{Site_2 south();}{return the defining site to the south of
  the Voronoi edge.}
\ccGlue
\ccMethod{Site_2 west();}{return the defining site to the west of
  the Voronoi edge.
\ccPrecond{\ccc{has_source()} must be \ccStyle{true}.}}
\ccGlue
\ccMethod{Site_2 east();}{return the defining site to the east of
  the Voronoi edge.
\ccPrecond{\ccc{has_target()} must be \ccStyle{true}.}}


\ccHeading{Predicate Methods}

\ccThree{bool}{e.is_bisector()+}{}
\ccThreeToTwo
%
\ccMethod{bool is_bisector();}{returns \ccStyle{true} iff the Voronoi
  edge is an entire bisector.}
\ccGlue
\ccMethod{bool is_segment();}{returns \ccStyle{true} iff the Voronoi
  edge has both a source and a target Voronoi vertex.}
\ccGlue
\ccMethod{bool is_ray();}{returns \ccStyle{true} iff the Voronoi
  edge has either a source and a target Voronoi vertex, but not both;
  in other words it is a bisecting ray.}
\ccGlue
\ccMethod{bool has_source();}{return \ccStyle{true} iff the Voronoi
  edge has a source Voronoi vertex.}
\ccGlue
\ccMethod{bool has_target();}{return \ccStyle{true} iff the Voronoi
  edge has a target Voronoi vertex.}


\ccHasModels
\ccc{CGAL::Apollonius_graph_Voronoi_traits_2<DG>::Voronoi_edge_2}\\
\ccc{CGAL::Delaunay_triangulation_Voronoi_traits_2<DG>::Voronoi_edge_2}\\
\ccc{CGAL::Regular_triangulation_Voronoi_traits_2<DG>::Voronoi_edge_2}\\
\ccc{CGAL::Segment_Voronoi_diagram_Voronoi_traits_2<DG>::Voronoi_edge_2}\\

\ccSeeAlso
\ccc{VoronoiTraits_2}\\
\ccc{VoronoiVertex_2}
\end{ccRefConcept}

% +------------------------------------------------------------------------+
%%RefPage: end of main body, begin of footer
% EOF
% +------------------------------------------------------------------------+

