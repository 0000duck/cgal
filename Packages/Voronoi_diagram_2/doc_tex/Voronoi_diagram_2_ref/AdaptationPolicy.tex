%% Copyright (c) 2005  Foundation for Research and Technology-Hellas (Greece).
%% All rights reserved.
%%
%% This file is part of CGAL (www.cgal.org); you may redistribute it under
%% the terms of the Q Public License version 1.0.
%% See the file LICENSE.QPL distributed with CGAL.
%%
%% Licensees holding a valid commercial license may use this file in
%% accordance with the commercial license agreement provided with the software.
%%
%% This file is provided AS IS with NO WARRANTY OF ANY KIND, INCLUDING THE
%% WARRANTY OF DESIGN, MERCHANTABILITY AND FITNESS FOR A PARTICULAR PURPOSE.
%%
%% $Source$
%% $Revision$ $Date$
%% $Name$
%%
%% Author(s)     : Menelaos Karavelas <mkaravel@tem.uoc.gr>



\begin{ccRefConcept}{AdaptationPolicy_2}

%% \ccHtmlCrossLink{}     %% add further rules for cross referencing links
%% \ccHtmlIndexC[concept]{} %% add further index entries
\ccDefinition

The concept \ccc{AdaptationPolicy_2} defines the requirements on
the predicate functors that determine whether a feature of the
triangulated Delaunay graph should be rejected or not. It also
provides a functor for inserting sites in the Delaunay graph. The last
functor is optional and a tag determines whether it is provided or
not. Note that while the first two functors do not modify the Delaunay
graph they take as an argument, the last ones does.

\ccRefines
\ccc{DefaultConstructible}, \ccc{CopyConstructible}, \ccc{Assignable}

\ccTypes
\ccTwo{AdaptationPolicy_2::Delaunay_graph+}{}
%
\ccNestedType{Site_2}{A type for the sites of the Voronoi diagram.}
\ccGlue
\ccNestedType{Delaunay_graph}{A type for the triangulated Delaunay
  graph. The type \ccc{Delaunay_graph} must be a model of the
  \ccc{DelaunayGraph_2} concept.}
%%
\ccThree{typedef Delaunay_graph::Finite_edges_iterator}
{Finite_edges_iterator+}{}
\ccThreeToTwo
\ccGlue
\ccTypedef{typedef Delaunay_graph::Vertex_handle Vertex_handle;}{}
%{Type for the vertex handles of the Delaunay graph.}
\ccGlue
\ccTypedef{typedef Delaunay_graph::Face_handle Face_handle;}{}
%{Type for the face handles of the Delaunay graph.}
\ccGlue
\ccTypedef{typedef Delaunay_graph::Edge Edge;}{}
%{Type for the edges of the Delaunay graph.}
\ccGlue
\ccTypedef{typedef Delaunay_graph::All_edges_iterator All_edges_iterator;}{}
%{Type for the iterator over all edges of the Delaunay graph.}
\ccGlue
\ccTypedef{typedef Delaunay_graph::Finite_edges_iterator Finite_edges_iterator;}
{}
%{Type for the iterator over the finite edges of the Delaunay graph.}
\ccGlue
\ccTypedef{typedef Delaunay_graph::Edge_circulator Edge_circulator;}{}
%{Type for the edge circulator of the Delaunay graph.}
%%
\ccTwo{AdaptationPolicy_2::Edge_rejector+}{}
\ccGlue
\ccNestedType{Edge_rejector}{A type for the predicate functor that is
  responsible for rejecting an edge of the Delaunay graph. It must be
  model of the concepts \ccc{DefaultConstructible},
  \ccc{CopyConstructible}, \ccc{Assignable}, and \ccc{AdaptableFunctor} 
  (with two arguments). It must provide the following operators:\\
  \hspace*{7mm}\ccc{bool operator()(Delaunay_graph dg, Edge e)}\\
  \hspace*{7mm}\ccc{bool operator()(Delaunay_graph dg, Face_handle f, int i)}\\
  \hspace*{7mm}\ccc{bool operator()(Delaunay_graph dg, Edge_circulator ec)}\\
  \hspace*{7mm}\ccc{bool operator()(Delaunay_graph dg,
    All_edges_iterator eit)}\\
  \hspace*{7mm}\ccc{bool operator()(Delaunay_graph dg,
    Finite_edges_iterator eit)}\\
  The functor returns \ccStyle{true} iff the edge is rejected.}
\ccGlue
\ccNestedType{Face_rejector}{A type for the predicate functor that is
  responsible for rejecting a face of the Delaunay graph. It must be
  model of the concepts \ccc{DefaultConstructible},
  \ccc{CopyConstructible}, \ccc{Assignable}, \ccc{AdaptableFunctor}
  (with two arguments). It must provide the following operator:\\
  \centerline{\ccc{bool operator()(Delaunay graph dg, Vertex_handle v)}}\\
  The functor returns \ccStyle{true} iff the face is rejected.}
\ccGlue
\ccNestedType{Has_inserter}{A tag for determining if the adaptation
  policy class provides a functor for inserting sites in the Delaunay
  graph. This tag is equal to either \ccc{CGAL::Tag_true} (a site
  inserter functor is available) ir \ccc{CGAL::Tag_false} (a site
  inserter functor is not available).}
\ccGlue
\ccNestedType{Site_inserter}{A type for a functor that inserts sites
  in the Delaunay graph. It must be model of the concepts
  \ccc{DefaultConstructible}, \ccc{CopyConstructible}, \ccc{Assignable}, 
  \ccc{AdaptableFunctor} (with two arguments). It must provide the
  following operator\\
  \centerline{\ccc{Vertex_handle operaotr()(Delaunay_graph& dg, Site_2 t)}}\\
  The vertex handle returned either points to the vertex of the
  Delaunay graph corresponding to the site just inserted or the
  default constructed vertex handle. The latter case can happen if the
  site inserted is \textit{hidden}, i.e., it has an empty Voronoi cell.\\
  This type is required only if the \ccc{Has_inserter} tag is equal to
  \ccc{CGAL::Tag_true}.}


\ccCreationVariable{ap}  %% choose variable name

\ccHeading{Access to objects}
\ccThree{Edge_rejector}{ap.edge_rejector_object()+}{}
\ccMethod{Edge_rejector edge_rejector_object();}{}
\ccGlue
\ccMethod{Face_rejector face_rejector_object();}{}
\ccGlue
\ccMethod{Site_inserter site_inserter_object();}{This method is
  required only if \ccc{Has_inserter} is equal to \ccc{CGAL::Tag_true}.}

\ccHeading{Miscellaneous}
The following methods are important when the adaptation policy maintains a
state. This can happen if we have a caching adaptation policy, i.e., when
we cache the results of the edge and face rejectors.

\ccThree{bool}{ap.swap(other)++}{}
%
\ccMethod{void clear();}{Clears the state of the adaptation policy.}
\ccGlue
\ccMethod{void swap(AdaptationPolicy_2 other);}{The adaptation policies
  \ccc{ap} and \ccc{other} are swapped. This method should be
  preferred to \ccStyle{vt=other} or \ccStyle{vt(other)} if
  \ccc{other} is deleted afterwards.}
\ccGlue
\ccMethod{bool is_valid();}{Tests the validity of the adaptation policy.}
\ccMethod{bool is_valid(Delaunay_graph dg);}{Tests the validity of the
  adaptation policy using extra information from the Delaunay graph
  \ccc{dg}.}

%
\ccHasModels
\ccc{CGAL::Identity_policy_2<DG,VT>}\\
\ccc{CGAL::Apollonius_graph_degeneracy_removal_policy_2<AG2>}\\
\ccc{CGAL::Apollonius_graph_caching_degeneracy_removal_policy_2<AG2>}\\
\ccc{CGAL::Delaunay_triangulation_degeneracy_removal_policy_2<DT2>}\\
\ccc{CGAL::Delaunay_triangulation_caching_degeneracy_removal_policy_2<DT2>}\\
\ccc{CGAL::Regular_triangulation_degeneracy_removal_policy_2<RT2>}\\
\ccc{CGAL::Regular_triangulation_caching_degeneracy_removal_policy_2<RT2>}\\
\ccc{CGAL::Segment_Voronoi_diagram_degeneracy_removal_policy_2<SVD2>}\\
\ccc{CGAL::Segment_Voronoi_diagram_caching_degeneracy_removal_policy_2<SVD2>}

\ccSeeAlso
\ccc{DelaunayGraph_2}\\
\ccc{VoronoiTraits_2}\\
\ccc{CGAL::Voronoi_diagram_2<DG,VT>}
\end{ccRefConcept}

% +------------------------------------------------------------------------+
%%RefPage: end of main body, begin of footer
% EOF
% +------------------------------------------------------------------------+

