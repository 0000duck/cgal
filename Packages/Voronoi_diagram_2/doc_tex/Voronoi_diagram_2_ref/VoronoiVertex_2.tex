%% Copyright (c) 2005  University of Crete (Greece).
%% All rights reserved.
%%
%% This file is part of CGAL (www.cgal.org); you may redistribute it under
%% the terms of the Q Public License version 1.0.
%% See the file LICENSE.QPL distributed with CGAL.
%%
%% Licensees holding a valid commercial license may use this file in
%% accordance with the commercial license agreement provided with the software.
%%
%% This file is provided AS IS with NO WARRANTY OF ANY KIND, INCLUDING THE
%% WARRANTY OF DESIGN, MERCHANTABILITY AND FITNESS FOR A PARTICULAR PURPOSE.
%%
%% $Source$
%% $Revision$ $Date$
%% $Name$
%%
%% Author(s)     : Menelaos Karavelas <mkaravel@tem.uoc.gr>



\begin{ccRefConcept}{VoronoiVertex_2}

%% \ccHtmlCrossLink{}     %% add further rules for cross referencing links
%% \ccHtmlIndexC[concept]{} %% add further index entries
\ccDefinition

The concept \ccc{VoronoiVertex_2} defines the requirements of the
\ccc{Voronoi_vertex_2} type defined in the \ccc{VoronoiTraits_2}
concept. Essentially it provides the requirements for a class
representing the Voronoi vertices in the Voronoi diagram.

In addition to the requirements listed below, a model of the
\ccc{VoronoiVertex_2} concept must also be convertible to a
point, or more precisely to an object of the type
\ccc{VoronoiVertex_2::Point_2}.


\ccRefines
\ccc{DefaultConstructible}\\
\ccc{CopyConstructible}\\
\ccc{Assignable}

\ccTypes
\ccTwo{VoronoiVertex_2::Dual_graph+}{}
%
\ccNestedType{Dual_graph}{A type for the Delaunay graph. It must be a
  model of the \ccc{DelaunayGraph_2} concept.}
\ccGlue
\ccNestedType{Point_2}{A type for a point.}
\ccGlue
\ccNestedType{Site_2}{A type for the generators of the Voronoi diagram.}


\ccCreationVariable{v}  %% choose variable name

\ccHeading{Access Methods}

A Voronoi vertex is defined by three generators. The order of the
generators in this triplet is very important since the Voronoi vertex
of the generators $g_1$, $g_2$ and $g_3$ is different from that of the
triplet $g_1$, $g_3$ and $g_2$. For this purpose we provide the
following methods for accessing the generating sites of the Voronoi
vertex:

\ccThree{Site_2}{e.site(unsigned int i)+}{}
\ccThreeToTwo
%
\ccMethod{Site_2 first();}{returns the first generating site of the
  Voronoi vertex.}
\ccGlue
\ccMethod{Site_2 second();}{returns the second generating site of the
  Voronoi vertex.}
\ccGlue
\ccMethod{Site_2 third();}{returns the third generating site of the
  Voronoi vertex.}
\ccGlue
\ccMethod{Site_2 site(unsigned int i);}{returns the $i+1$-th
  generating site of the Voronoi vertex.
  \ccPrecond{\ccc{i} must be smaller or equal to 2.}}


\ccHasModels
\ccc{CGAL::Apollonius_graph_Voronoi_traits_2<DG>::Voronoi_vertex_2}\\
\ccc{CGAL::Delaunay_triangulation_Voronoi_traits_2<DG>::Voronoi_vertex_2}\\
\ccc{CGAL::Regular_triangulation_Voronoi_traits_2<DG>::Voronoi_vertex_2}\\
\ccc{CGAL::Segment_Voronoi_diagram_Voronoi_traits_2<DG>::Voronoi_vertex_2}\\

\ccSeeAlso
\ccc{VoronoiTraits_2}\\
\ccc{VoronoiEdge_2}
\end{ccRefConcept}

% +------------------------------------------------------------------------+
%%RefPage: end of main body, begin of footer
% EOF
% +------------------------------------------------------------------------+

