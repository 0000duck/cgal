%% Copyright (c) 2005  University of Crete (Greece).
%% All rights reserved.
%%
%% This file is part of CGAL (www.cgal.org); you may redistribute it under
%% the terms of the Q Public License version 1.0.
%% See the file LICENSE.QPL distributed with CGAL.
%%
%% Licensees holding a valid commercial license may use this file in
%% accordance with the commercial license agreement provided with the software.
%%
%% This file is provided AS IS with NO WARRANTY OF ANY KIND, INCLUDING THE
%% WARRANTY OF DESIGN, MERCHANTABILITY AND FITNESS FOR A PARTICULAR PURPOSE.
%%
%% $Source$
%% $Revision$ $Date$
%% $Name$
%%
%% Author(s)     : Menelaos Karavelas <mkaravel@tem.uoc.gr>



\begin{ccRefConcept}{VoronoiDiagramHalfedge_2}

%% \ccHtmlCrossLink{}     %% add further rules for cross referencing links
%% \ccHtmlIndexC[concept]{} %% add further index entries
\ccDefinition

The concept \ccc{VoronoiDiagramHalfedge_2} defines the requirements of the
halfedge provided by the Voronoi diagram adaptor. It refines
the \ccc{PlanapMapDcelHalfedge_2} concept (the non-mutable part), by
adding some additional requirements specific to Voronoi diagrams.

\ccRefines
\ccc{DefaultConstructible}\\
\ccc{CopyConstructible}\\
\ccc{Assignable}\\
\ccc{PlanarMapDcelHalfedge_2}

\ccTypes
The \ccc{VoronoiDiagramHalfedge_2} concept introduces the following
types in addition to those of the \ccc{PlanarMapDcelHalfedge_2} concept.

\ccThree{typedef typename Dual_graph::Edge}{Dual_edge;}{}
\ccThreeToTwo
\ccNestedType{Dual_graph}{A type for the Delaunay graph. It must be a
  model of the \ccc{DelaunayGraph_2} concept.}
\ccTypedef{typedef typename Dual_graph::Edge Dual_edge;}{A type for
  the dual edge.}

\ccCreationVariable{e}  %% choose variable name

\ccThree{Dual_edge}{e.has_source()+}{}
\ccThreeToTwo

\ccHeading{Access Methods}
\ccMethod{Dual_edge dual_edge();}{returns the dual edge, i.e., the
  corresponding dual edge in the Delaunay graph.}

\ccHeading{Predicate Methods}

\ccMethod{bool has_source();}{returns \ccStyle{true} iff the halfedge
  corresponds to a bisecting segment or a bisecting ray oriented
  appropriately so that its apex is its source.}
\ccGlue
\ccMethod{bool has_target();}{returns \ccStyle{true} iff the halfedge
  corresponds to a bisecting segment or a bisecting ray oriented
  appropriately so that its apex is its target.}
\ccGlue
\ccMethod{bool is_valid();}{returns \ccStyle{true} if the following
  conditions are met: the halfedge does not have zero length; the
  twin edge of its twin edge is itself; its adjacent face does not
  have zero area; its source and target vertices are valid (provided
  they exist, of course); the previous of its next halfedge is itself
  and the next of its previous halfedge is itself.}

\ccHasModels
\ccc{CGAL::Voronoi_diagram_adaptor_2<DG,VT>::Halfedge}\\

\ccSeeAlso
\ccc{VoronoiDiagramVertex_2}\\
\ccc{VoronoiDiagramFace_2}\\
\ccc{CGAL::Voronoi_diagram_adaptor_2<DG,VT>}
\end{ccRefConcept}

% +------------------------------------------------------------------------+
%%RefPage: end of main body, begin of footer
% EOF
% +------------------------------------------------------------------------+

