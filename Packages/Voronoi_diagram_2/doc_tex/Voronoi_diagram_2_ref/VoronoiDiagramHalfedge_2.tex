%% Copyright (c) 2005  Foundation for Research and Technology-Hellas (Greece).
%% All rights reserved.
%%
%% This file is part of CGAL (www.cgal.org); you may redistribute it under
%% the terms of the Q Public License version 1.0.
%% See the file LICENSE.QPL distributed with CGAL.
%%
%% Licensees holding a valid commercial license may use this file in
%% accordance with the commercial license agreement provided with the software.
%%
%% This file is provided AS IS with NO WARRANTY OF ANY KIND, INCLUDING THE
%% WARRANTY OF DESIGN, MERCHANTABILITY AND FITNESS FOR A PARTICULAR PURPOSE.
%%
%% $Source$
%% $Revision$ $Date$
%% $Name$
%%
%% Author(s)     : Menelaos Karavelas <mkaravel@tem.uoc.gr>



\begin{ccRefConcept}{VoronoiDiagramHalfedge_2}

\ccDefinition
The concept \ccc{VoronoiDiagramHalfedge_2} defines the requirements of the
halfedge provided by the \ccc{Voronoi_diagram_2} class.

\ccRefines
\ccc{DefaultConstructible}, \ccc{CopyConstructible}, \ccc{Assignable},
\ccc{EqualityComparable}, \ccc{LessThanComparable}

\ccTypes
%
\ccTwo{VoronoiDiagramHalfedge_2::Ccb_halfedge_circulator+}{}
\ccNestedType{Vertex}{A type for the vertices of the Voronoi diagram.}
\ccGlue
\ccNestedType{Face}{A type for the faces of the Voronoi diagram.}
\ccGlue
\ccNestedType{Vertex_handle}{Handle for the vertices of the Voronoi diagram.}
\ccGlue
\ccNestedType{Face_handle}{Handle for the faces of the Voronoi diagram.}
\ccGlue
\ccNestedType{Halfedge_handle}{Handle for the halfedges of the Voronoi
  diagram.}
\ccGlue
\ccNestedType{Ccb_halfedge_circulator}{A type for a circulator over
  the halfedges of the connected components of the boundary of a
  Voronoi face. The value type of the circulator is
  \ccc{VoronoiDiagramHalfedge_2}.}
\ccGlue
\ccNestedType{Delaunay_graph}{A type for the Delaunay graph. It must be a
  model of the \ccc{DelaunayGraph_2} concept.}
\ccThree{typedef Delaunay_graph::Vertex_handle}{Delaunay_vertex_handle;}{}
\ccThreeToTwo
\ccGlue
\ccTypedef{typedef Delaunay_graph::Edge Delaunay_edge;}{A type for
  the dual edge in the Delaunay graph.}
\ccGlue
\ccTypedef{typedef Delaunay_graph::Vertex_handle Delaunay_vertex_handle;}
{A type for vertex handles in the Delaunay graph.}

\ccCreationVariable{e}  %% choose variable name

\ccThree{Delaunay_edge}{e.has_source()+}{}
\ccThreeToTwo

\ccHeading{Access Methods}
\ccThree{Ccb_halfedge_circulator}{e.opposite()+}{}
%
\ccMethod{Halfedge_handle opposite();}{Returns the twin halfedge.}
\ccGlue
\ccMethod{Halfedge_handle twin();}{Same as \ccc{e.opposite()}.}
\ccGlue
\ccMethod{Halfedge_handle next();}{Returns the next halfedge in the
  counterclockwise sense around the boundary of the face that \ccc{e}
  is incident to.}
\ccGlue
\ccMethod{Halfedge_handle previous();}{Returns the previous halfedge in the
  counterclockwise sense around the boundary of the adjacent face.}
\ccGlue
\ccMethod{Face_handle face();}{Returns the face that \ccc{e} is
  incident to.}
\ccGlue
\ccMethod{Vertex_handle source();}{Returns the source vertex of
  \ccc{e}.
  \ccPrecond{The source vertex must exist, i.e., \ccc{has_source()}
    must return \ccStyle{true}.}}
\ccGlue
\ccMethod{Vertex_handle target();}{Returns the target vertex of
  \ccc{e}.
  \ccPrecond{The target vertex must exist, i.e., \ccc{has_target()}
    must return \ccStyle{true}.}}
\ccGlue
\ccMethod{Ccb_halfedge_circulator ccb();}{Returns a circulator to
  traverse the halfedges of the boundary of the Voronoi face
  containing \ccc{e}. The circulator is initialized to \ccc{e}.}
\ccGlue
\ccMethod{Delaunay_edge dual();}{Returns the dual edge, i.e., the
  corresponding dual edge in the Delaunay graph.}
\ccGlue
\ccMethod{Delaunay_vertex_handle up();}{Returns a handle to the vertex in
the Delaunay graph corresponding to the defining site above
the Voronoi edge.}
\ccGlue
\ccMethod{Delaunay_vertex_handle down();}{Returns a handle to the vertex
in the Delaunay graph corresponding to the defining site below
the Voronoi edge.}
\ccGlue
\ccMethod{Delaunay_vertex_handle left();}{Returns a handle to the vertex in
the Delaunay graph corresponding to the defining site to the left of
the Voronoi edge.
\ccPrecond{\ccc{has_source()} must be \ccStyle{true}.}}
\ccGlue
\ccMethod{Delaunay_vertex_handle right();}{Returns a handle to the vertex in
the Delaunay graph corresponding to the defining site to the right of
the Voronoi edge.
\ccPrecond{\ccc{has_target()} must be \ccStyle{true}.}}

\ccHeading{Predicate Methods}
\ccThree{bool}{e.is_unbounded()+}{}
%
\ccMethod{bool has_source();}{Returns \ccStyle{true} iff the halfedge
  corresponds to a bisecting segment or a bisecting ray oriented
  appropriately so that its apex is its source.}
\ccGlue
\ccMethod{bool has_target();}{Returns \ccStyle{true} iff the halfedge
  corresponds to a bisecting segment or a bisecting ray oriented
  appropriately so that its apex is its target.}
\ccGlue
\ccMethod{bool is_unbounded();}{Returns \ccStyle{true} iff the source or
  the target of the halfedge does not exist, i.e., if either of
  \ccc{has_source()} or \ccc{has_target()} return \ccStyle{false}.}
\ccGlue
\ccMethod{bool is_bisector();}{Returns \ccStyle{true} iff the Voronoi
  edge is an entire bisector.}
\ccGlue
\ccMethod{bool is_segment();}{Returns \ccStyle{true} iff the Voronoi
  edge has both a source and a target Voronoi vertex.}
\ccGlue
\ccMethod{bool is_ray();}{Returns \ccStyle{true} iff the Voronoi
  edge has either a source or a target Voronoi vertex, but not both;
  in other words it is a bisecting ray.}
\ccGlue
\ccMethod{bool is_valid();}{Returns \ccStyle{true} if the following
  conditions are met: the halfedge does not have zero length; the
  twin edge of its twin edge is itself; its adjacent face does not
  have zero area; its source and target vertices are valid (provided
  they exist, of course); the previous of its next halfedge is itself
  and the next of its previous halfedge is itself.}

\ccHasModels
\ccc{CGAL::Voronoi_diagram_2<DG,VT>::Halfedge}\\

\ccSeeAlso
\ccc{VoronoiDiagramVertex_2}\\
\ccc{VoronoiDiagramFace_2}\\
\ccc{DelaunayGraph_2}\\
\ccc{CGAL::Voronoi_diagram_2<DG,VT>}
\end{ccRefConcept}

% +------------------------------------------------------------------------+
%%RefPage: end of main body, begin of footer
% EOF
% +------------------------------------------------------------------------+

