% =============================================================================
% The CGAL Reference Manual
% Chapter: Geometric Optimisation
% Section: Smallest Enclosing Sphere
% =============================================================================

\begin{ccRefClass}{QPE_full_filtered_pricing<Traits, NT, ET2NT>}
\ccIndexSubitem[t]{sphere}{smallest enclosing sphere of spheres}
\ccIndexSubitem[t]{smallest enclosing}{sphere of spheres}
\ccIndexSubitem[t]{bounding volumes}{smallest enclosing sphere of spheres}
\ccIndexSubitemSeeAlso[t]{circle}{smallest enclosing sphere of spheres}

% -----------------------------------------------------------------------------
\ccDefinition
%The pricing step of the solvers algorithm consists of determining  whether the
%current solution is optimal, if not the pricing step then determines the next
%entering variable.
%The pricing
%strategy is the part of the solvers pricing step that selects a variable among the
%nonbasic variables that qualify as entering variable. A nonbasic variable
%$x_{j}$ qualifies as an entering variable if its associated $\mu_{j}$, which
%coincides with the notion of reduced cost in the linear case, is strictly
%negative.

With \emph{full filtered pricing} the $\mu_{j}$, $j \in N$, where $N$ denotes
the set of currently nonbasic variables, are approximated by
$\tilde{\mu}_{j}$, $j \in N$ with a fast (compared to \ccc{ET})
floating point number type \ccc{NT} at the beginning of the pricing step.
If the exact check of the nonbasic variable with the smallest
approximated $\tilde{\mu}_{j}$ succeeds, that is if $\mu_{j_{N}} < 0$, where 
$j_{N} =\arg\min_{j \in N}\tilde{\mu}_{j}$, then $x_{j_{N}}$ is the
entering variable. Otherwise the nonnegativity for a part of the
nonbasic variables is \emph{certified} by the use of an error bound $b$ on
$\left|\tilde{\mu}_{j} - \mu_{j}\right|$:
if $\tilde{\mu}_{j} \geq b$ then $\mu_{j}$ is known to
be nonnegative. The $\tilde{\mu}_{j}$, $j \in N$ whose exact $\mu_{j}$ values  
cannot be certified to be nonnegative by this \emph{floating point filter}
are then evaluated
using exact arithmetic. The error bound actually consists of two bounds, one
that does not depend on $j \in N$ and one that does depend on $j$. Note that
the bounds themselves can be evaluated with floating point arithmetic. 

Full filtered pricing improves performance for quadratic programs
where the floating point
number type \ccc{NT} is faster than the internal exact number type \ccc{ET}.
The floating point number type \ccc{NT} and a functor providing conversion from
the exact number type \ccc{ET} to the floating point type \ccc{NT} can be
configured by template parameters for the class \ccRefName\ .

\ccInclude{CGAL/QPE_full_filtered_pricing.h}

\ccInheritsFrom
\ccc{QPE__filtered_base<Traits, NT, ET2NT>}

\ccRequirements
\ccIndexRequirements

The class \ccRefName\ expects a model of the concept
\ccc{QPESolverTraits} as its template argument for the first template parameter,
a model of the \cgal \ concept \ccc{Ring number type} for the second template
parameter and a functor type
providing conversion from the exact number type \ccc{Traits::ET} to \ccc{NT}
for the third template parameter. The second template parameter defaults to
\ccc{double} and the third template parameter defaults to a functor providing
conversion from \ccc{Traits::ET} to \ccc{double}. Furthermore the element types
of $A$, $b$, $c$, and $D$ are required to provide implicit lossless conversion
to \ccc{NT}. Currently the number type $NT$ is supported for \ccc{double} only.  


\ccTypes \ccIndexClassTypes


\ccCreation
\ccIndexClassCreation
\ccCreationVariable{ffps}

\ccConstructor{ (ET2NT et2nt = ET2NT());}
{creates a variable of type \ccRefName\ .}


\ccUnchecked

\ccAccessFunctions
\begin{ccIndexMemberFunctions}
\ccIndexMemberFunctionGroup{access}


% -----------------------------------------------------------------------------
%\ccPredicates
%\ccIndexMemberFunctionGroup{predicates}


% -----------------------------------------------------------------------------
\ccModifiers
\ccIndexMemberFunctionGroup{modifiers}

%\ccMemberFunction{int pricing();}{determines the index of the entering
%variable, returns $-1$ if the current solution is optimal.}


% -----------------------------------------------------------------------------
\ccHeading{Validity Check}
\ccIndexMemberFunctionGroup{validity check}

% -----------------------------------------------------------------------------
\ccHeading{Miscellaneous}
\ccIndexMemberFunctionGroup{miscellaneous}

\end{ccIndexMemberFunctions}

% -----------------------------------------------------------------------------
\ccSeeAlso
%
%    \ccRefIdfierPage{CGAL::Min_sphere_d<Traits>}\\
%    \ccRefIdfierPage{CGAL::Min_circle_2<Traits>}\\
% -----------------------------------------------------------------------------

\ccImplementation
\ccIndexImplementation
%\ccIndexSubitem[t]{incremental algorithm}{\ccc{Min_sphere_of_spheres_d}}
%\ccIndexSubitem[t]{farthest-first heuristic}{\ccc{Min_sphere_of_spheres_d}}


\ccExample
In the following example we demonstrate how the problem of finding the
smallest distance between two polytopes, each given as the convex hull of a
set of points in
$d$-dimensional space, can be mapped to a quadratic program. Furthermore,
the explicit configuration of the pricing strategy as well as
the use of the class \ccc{QPE_const_value_iterator<T>} is shown.  

\ccHeading{Polytope Distance}
Consider two point sets $P=\{p_{1}, \ldots, p_{r} \}$ and
$Q=\{q_{1}, \ldots, q_{s} \}$ with $r+s=n$ in $d$-dimensional space. The
\emph{polytope distance} problem consists of minimizing $\|p-q\|$ such that
$p=\sum_{i=1}^{r}\lambda_{i}p_{i}$ and $q=\sum_{i=1}^{s}\mu_{i}q_{i}$. The
$\lambda_{i}$ and $\mu_{i}$ are in $\left[0, 1\right]$ and sum up to one.
This can be written as a quadratic program in $n$ variables and two constraints,
\begin{eqnarray*}
(PD) & minimize & x^{T}C^{T}Cx  \\
     & s.t.     & \sum_{i=1}^{r}x_{i} = 1 \\
     &          & \sum_{i=r+1}^{n}x_{i} = 1 \\
     &          & x \geq 0,
\end{eqnarray*}
where $C=(p_{1}, \ldots , p_{r}, -q_{1}, \ldots, -q_{s})$. Here, $D=C^{T}C$ is an
$n \times n$-matrix, but its rank is only $d$. The example solves the quadratic
program for $d=2$, $r=3$ and $s=3$ with the following $C$-matrix:
\[
C:=
\left(\begin{array}{cccccc}
        2 & 6 & 5 & -7 & -9 & -6 \\
	2 & 1 & 5 & -4 & -7 & -7
      \end{array}
\right).
\]

\ccIncludeVerbatim{Optimisation_ref/QPE_full_filtered_pricing_strategy_example.C}

\end{ccRefClass}

% ===== EOF ===================================================================

