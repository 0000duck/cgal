% =============================================================================
% The CGAL Reference Manual
% Chapter: Geometric Optimisation
% Section: Smallest Enclosing Sphere of Spheres
% -----------------------------------------------------------------------------
% file  : doc_tex/basic/Optimisation/Optimisation_ref/Traits.tex
% author: Kaspar Fischer (fischerk@inf.ethz.ch)
% -----------------------------------------------------------------------------
% $Revision$
% $Date$
% $CGAL_Package$
% =============================================================================

\begin{ccRefConcept}{QPSolverTraits}

% -----------------------------------------------------------------------------
\ccDefinition

A model of concept \ccc{QPSolverTraits} must provide the
following constants, types, predicates and operations.

\ccHasModels
%\ccc{CGAL::Min_sphere_of_spheres_d_traits_2<K,FT,UseSqrt,Algorithm>} \\
%\ccc{CGAL::Min_sphere_of_spheres_d_traits_3<K,FT,UseSqrt,Algorithm>} \\
%\ccc{CGAL::Min_sphere_of_spheres_d_traits_d<K,FT,Dim,UseSqrt,Algorithm>}

\ccConstants 
\ccEnum{enum Row_type {LESS_EQUAL, EQUAL, GREATER_EQUAL};}{constants indicating
inequality and equality constraints.}

\ccTypes

\ccNestedType{ET}{a model for the CGAL concept \ccc{RingNumberType}
that has to support addition, subtraction and multiplication.
Division is only required in cases where the remainder is zero.}

\ccNestedType{A_iterator}{model of the \stl \ concept \ccc{RandomAccessIterator}
referring to an iterator, which in turn refers to the elements
of a column of the matrix $A$.
The \ccc{value_type} of \ccc{A_iterator} is a model of the \stl \ concept
\ccc{RandomAccessIterator} that refers to the elements of a
column of the matrix $A$.
The element type of the matrix $A$ must provide implicit conversions to
\ccc{QPSolverTraits::ET}
}

\ccNestedType{B_iterator}{model of the \stl \ concept
\ccc{RandomAccessIterator} referring to the entries of the vector $b$.
The element type of the vector $b$, that is
\ccc{B_iterator}'s \ccc{value_type},
must provide implicit conversions to \ccc{QPSolverTraits::ET}.}

\ccNestedType{C_iterator}{model of the \stl \ concept
\ccc{RandomAccessIterator} referring to the entries of the vector $c$.
The element type of the vector $c$, that is
\ccc{C_iterator}'s \ccc{value_type},
must provide implicit conversions to \ccc{QPSolverTraits::ET}.}

\ccNestedType{D_iterator}{model of the \stl \ concept \ccc{RandomAccessIterator}
referring to an iterator, which in turn refers to the elements
of a row of the matrix $D$.
The \ccc{value_type} of \ccc{D_iterator} is a model of the \stl \ concept
\ccc{RandomAccessIterator} that refers to the elements of a
row of the matrix $D$.
The element type of the matrix $D$ must provide implicit conversions to
\ccc{QPSolverTraits::ET}
}

\ccNestedType{Row_type}{enumeration type providing the three constants
\ccc{LESS_EQUAL}, \ccc{EQUAL}, \ccc{GREATER_EQUAL} used to describe the
constraint type.}

\ccNestedType{Row_type_iterator}{model of the STL concept
\ccc{RandomAccessIterator}. The value type of \ccc{Row_type_iterator} is
\ccc{QPSolverTraits::Row_type}.}

\ccNestedType{Is_linear}{
Is either \ccc{CGAL::Tag_true} or \ccc{CGAL::Tag_false}, if \ccc{Is_linear}
is of type \ccc{CGAL::Tag_true} the solver assumes $D=0$, otherwise
$D\neq0$ is assumed.}

\ccNestedType{Is_symmetric}{
Is either \ccc{CGAL::Tag_true} or \ccc{CGAL::Tag_false}, if \ccc{Is_symmetric}
is of type \ccc{CGAL::Tag_true} the solver assumes the matrix $D$ to be
symmetric and the number of accesses to entries of $D$ are halfed.}

\ccNestedType{Has_full_row_rank}{
Is either \ccc{CGAL::Tag_true} or \ccc{CGAL::Tag_false}, enhances
performance if it is of type \ccc{CGAL::Tag_true}, requiring the
constraint matrix $A$ to have full row rank. If
\ccc{Has_full_row_rank_and_equalities_only} is of type \ccc{CGAL::Tag_false}
no such requirement must be met by the constraint matrix $A$.}

\ccNestedType{Use_perturbation}{
Is either \ccc{CGAL::Tag_true} or \ccc{CGAL::Tag_false}. If
\ccc{Use_perturbation} is of type \ccc{CGAL::Tag_true}, symbolic perturbation
is used and cycling is avoided, such that the algorithm always terminates.
With symbolic perturbation a small performance penalty is incurred.  Implementation note: \ccc{Use_perturbation} must be \ccc{CGAL::Tag_false}, currently.}

\ccAccessFunctions

\ccCreationVariable{traits}

%\ccMemberFunction{ FT radius(const
%Sphere& s);}{ returns the radius of sphere \ccc{s}.
%\ccPostcond The returned number is
% greater or equal to~$0$.}

%\ccMemberFunction{ Cartesian_const_iterator center_cartesian_begin(const Sphere& s);}{
%  returns an iterator referring to the first of the \ccc{D} Cartesian
%  coordinates of the center of \ccc{s}.}

\end{ccRefConcept}

