% +------------------------------------------------------------------------+
% | Reference manual page: Extended_homogeneous.tex
% +------------------------------------------------------------------------+
% | 26.06.2001 Michael Seel
% | Package:   Nef_2
% | 
% +------------------------------------------------------------------------+

\ccAutoIndexingOff
\begin{ccRefClass}{Extended_homogeneous<RT>}
\ccAutoIndexingOn
\ccIndexTraitsClassBegin{Extended_homogeneous}{}{Nef polyhedron, 2D}

\ccDefinition
  
The class \ccRefName\ serves as a traits class for the class
\ccc{CGAL::Nef_polyhedron_2<T>}.  It uses a polynomial component
representation based on a Euclidean ring number type \ccc{RT}.

\ccInclude{CGAL/Extended_homogeneous.h}

\ccIsModel

\ccRefConceptPage{ExtendedKernelTraits_2}%
\ccIndexSubitem[c]{ExtendedKernelTraits_2}{model} \\

\ccCreation
\ccCreationVariable{traits}  %% choose variable name

\ccConstructor{Extended_homogeneous();}{default constructor.}

\ccHeading{Requirements} 

To make an Euclidean ring number type
\ccc{RT_model} work with this class the number type must support
a gcd computation in namespace \ccc{CGAL::NTS}.  \cgal\ provides
a function template for this, which will be used by default when
your number type is  not one of the built-in number types, one of
the number types distrubuted with \cgal\ or one of the \leda\ 
number types.  


\ccOperations

Fits all operation requirements of the concept.

\ccSeeAlso

\ccRefIdfierPage{CGAL::Extended_cartesian<FT>} \\
\ccRefIdfierPage{CGAL::Filtered_extended_homogeneous<RT>} 


\ccIndexTraitsClassEnd
\ccAutoIndexingOff
\end{ccRefClass}
\ccAutoIndexingOn

% EOF
