% +------------------------------------------------------------------------+
% | Reference manual page: Extended_homogeneous.tex
% +------------------------------------------------------------------------+
% | 26.06.2001 Michael Seel
% | Package:   Nef_2
% | 
% +------------------------------------------------------------------------+

\ccAutoIndexingOff
\begin{ccRefClass}{Extended_homogeneous<RT>}
\ccAutoIndexingOn
\ccIndexTraitsClassBegin{Extended_homogeneous}{}{Nef polyhedron, 2D}

\ccDefinition
  
The class \ccRefName\ serves as a traits class for the traits
parameter \ccc{T} of the class \ccc{CGAL::Nef_polyhedron_2<T>}.  It
uses a polynomial component representation based on an Euclidean ring
number type \ccc{RT}.

\ccInclude{CGAL/Extended_homogeneous.h}

\ccIsModel

\ccRefConceptPage{ExtendedKernelTraits_2}%
\ccIndexSubitem[c]{ExtendedKernelTraits_2}{model} \\

\ccCreation
\ccCreationVariable{traits}  %% choose variable name

\ccConstructor{Extended_homogeneous();}{default constructor.}

\ccHeading{Requirements} 

To make an Euclidean ring number type
\ccc{RT_model} work with this class you have to provide the following
number type traits class in global namespace.

\begin{ccExampleCode}
template <>
struct ring_or_field<RT_model> {
  typedef ring_with_gcd kind;
  static RT_model gcd(const RT_model& a, const RT_model& b) 
  { /* provide gcd operation on RT_model here */ }
};    
\end{ccExampleCode}

\ccOperations

Fits all operation requirements of the concept.

\ccSeeAlso

\ccRefIdfierPage{CGAL::Extended_cartesian<FT>} \\
\ccRefIdfierPage{CGAL::Filtered_extended_homogeneous<RT>} 


\ccIndexTraitsClassEnd
\ccAutoIndexingOff
\end{ccRefClass}
\ccAutoIndexingOn

% EOF
