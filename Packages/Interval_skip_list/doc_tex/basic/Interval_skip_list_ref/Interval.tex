% +------------------------------------------------------------------------+
% | Reference manual page: Interval.tex
% +------------------------------------------------------------------------+
% | 11.04.2000   Author
% | Package: Package
% | 
\RCSdef{\RCSIntervalRev}{$Revision$}
\RCSdefDate{\RCSIntervalDate}{$Date$}
% |
%%RefPage: end of header, begin of main body
% +------------------------------------------------------------------------+


\begin{ccRefConcept}{Interval}

%% \ccHtmlCrossLink{}     %% add further rules for cross referencing links
%% \ccHtmlIndexC[concept]{} %% add further index entries

\ccDefinition
  
The concept \ccRefName\ describes the requirements for the
interval class of a \ccc{Interval_skip_list<Interval>}.



\ccCreation
\ccCreationVariable{in}  %% choose variable name

\ccConstructor{Interval();}{default constructor.}
\ccTypes

\ccNestedType{Value}{The type of infimum and supremum of the interval.}


\ccAccessFunctions
\ccMethod{Value inf() const;}
{returns  the infimum.}
\ccGlue
\ccMethod{Value sup() const;}{returns the supremum.}


\ccMethod{bool contains(const Value& v) const;}
{returns \ccc{true}, iff \ccc{in} contains \ccc{v}.}

\ccMethod{bool contains_interval(const Value& i, const Value& s) const;}
{returns \ccc{true}, iff \ccc{in} contains \ccc{(i,s)}.}

\ccMethod{bool operator==(const Interval& I) const;}
{Equality test.}

\ccMethod{bool operator!=(const Interval& I) const;}
{Unequality test.}

\ccHasModels

\ccc{CGAL::Interval_skip_list_interval<Value>}\\
\ccc{CGAL::Face_interval}


\ccSeeAlso

\ccc{Interval_skip_list} 


\end{ccRefConcept}

% +------------------------------------------------------------------------+
%%RefPage: end of main body, begin of footer
% EOF
% +------------------------------------------------------------------------+

