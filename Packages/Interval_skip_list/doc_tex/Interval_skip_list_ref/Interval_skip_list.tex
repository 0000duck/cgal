% ======================================================================
%
% Copyright (c) 2003 GeometryFactory
%
% This software and related documentation is part of the
% Computational Geometry Algorithms Library (CGAL).
%
% Every use of CGAL requires a license. Licenses come in three kinds:
%
% - For academic research and teaching purposes, permission to use and
%   copy the software and its documentation is hereby granted free of  
%   charge, provided that
%   (1) it is not a component of a commercial product, and
%   (2) this notice appears in all copies of the software and
%       related documentation.
% - Development licenses grant access to the source code of the library 
%   to develop programs. These programs may be sold to other parties as 
%   executable code. To obtain a development license, please contact
%   the GALIA Consortium (at cgal@cs.uu.nl).
% - Commercialization licenses grant access to the source code and the
%   right to sell development licenses. To obtain a commercialization 
%   license, please contact the GALIA Consortium (at cgal@cs.uu.nl).
%
% This software and documentation is provided "as-is" and without
% warranty of any kind. In no event shall the CGAL Consortium be
% liable for any damage of any kind.
%
% The GALIA Consortium consists of Utrecht University (The Netherlands),
% ETH Zurich (Switzerland), Free University of Berlin (Germany),
% INRIA Sophia-Antipolis (France), Martin-Luther-University Halle-Wittenberg
% (Germany), Max-Planck-Institute Saarbrucken (Germany),
% and Tel-Aviv University (Israel).
%
% ----------------------------------------------------------------------
%
% package       : Interval_skip_list
% author(s)     : Andreas Fabri <Andreas.Fabri@geometryfactory.com>
%
% coordinator   : GeometryFactory (<Andreas.Fabri@geometryfactory.com>)
%
% ======================================================================

\RCSdef{\alphashapeRevision}{$Revision$}
\RCSdefDate{\alphashapeDate}{$Date$}

%----------------------------------------------------------------------

\begin{ccRefClass} {Interval_skip_list<Interval>}

\ccDefinition

The class \ccClassTemplateName\ represents .


\ccTypes
\ccSetThreeColumns{Oriented_side}{}{\hspace*{10cm}}
\ccThreeToTwo

\ccTypedef{typedef Interval::Value Value;}{the type of inf and sup of the interval.}

\ccNestedType{const_iterator}{An iterator that allows to enumerate all intervals in the 
interval skip list.}



\ccCreation
\ccCreationVariable{isl}

\ccConstructor{Interval_skip_list();}
{Default constructor.}


\ccConstructor{template < class InputIterator >
		Alpha_shape_2(
			InputIterator first,
			InputIterator last);}
{Constructor that inserts the iterator range \ccc{[first, last)} in the interval skip list. 
\ccPrecond The \ccc{value_type} of \ccc{first} and
\ccc{last} is \ccc{Interval}.}

\ccOperations

\ccMethod{template < class InputIterator >
		int insert(
			InputIterator first,
			InputIterator last);}
{Inserts the iterator range \ccc{[first, last)} in the interval skip list, and returns
the number of inserted intervals.\\ 
\ccPrecond The \ccc{value_type} of \ccc{first} and
\ccc{last} is \ccc{Interval}.}



\ccMethod{(void insert(const Interval& i);}
{inserts interval \ccc{i} in the interval skip list.}

\ccMethod{(void remove(const Interval& i);}
{removes interval \ccc{i} from the interval skip list.}


\ccMethod{template < class OutputIterator >
		int find_intervals(
			const Value& v,
			OutputIterator out);}
{Writes the intervals \ccc{i} with \ccc{i.inf()} $<=$ \ccc{v} $<=$ \ccc{i.sup} in \ccc{out}.\\
\ccPrecond The \ccc{value_type} of \ccc{out} is \ccc{Interval}.}


\ccMethod{void
	clear();}
{Clears the structure.}



\ccMethod{const_iterator begin() const;}
{Returns the .}

\ccMethod{const_iterator end() const;}
{Returns the past the end iterator.}


\ccHeading{I/O}

The output operator is defined for \ccc{std::ostream}

\ccFunction{ostream& operator<<(ostream& os,
                  const Interval_skip_list<Interval>& isl);}
{Inserts the interval skip list \ccc{isl} into the stream \ccc{os}.
\ccPrecond The insert operator must be defined for \ccc{Interval}.}


\ccImplementation
The 
\end{ccRefClass}

