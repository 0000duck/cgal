% ======================================================================
%
% Copyright (c) 2003 GeometryFactory
%
% This software and related documentation is part of the
% Computational Geometry Algorithms Library (CGAL).
%
% Every use of CGAL requires a license. Licenses come in three kinds:
%
% - For academic research and teaching purposes, permission to use and
%   copy the software and its documentation is hereby granted free of  
%   charge, provided that
%   (1) it is not a component of a commercial product, and
%   (2) this notice appears in all copies of the software and
%       related documentation.
% - Development licenses grant access to the source code of the library 
%   to develop programs. These programs may be sold to other parties as 
%   executable code. To obtain a development license, please contact
%   the GALIA Consortium (at cgal@cs.uu.nl).
% - Commercialization licenses grant access to the source code and the
%   right to sell development licenses. To obtain a commercialization 
%   license, please contact the GALIA Consortium (at cgal@cs.uu.nl).
%
% This software and documentation is provided "as-is" and without
% warranty of any kind. In no event shall the CGAL Consortium be
% liable for any damage of any kind.
%
% The GALIA Consortium consists of Utrecht University (The Netherlands),
% ETH Zurich (Switzerland), Free University of Berlin (Germany),
% INRIA Sophia-Antipolis (France), Martin-Luther-University Halle-Wittenberg
% (Germany), Max-Planck-Institute Saarbruecken (Germany),
% and Tel-Aviv University (Israel).
%
% ----------------------------------------------------------------------
%
% package       : Interval_skip_list
% author(s)     : Andreas Fabri <Andreas.Fabri@geometryfactory.com>
%
% coordinator   : GeometryFactory (<Andreas.Fabri@geometryfactory.com>)
%
% ======================================================================

\RCSdef{\FaceIntervalRevision}{$Revision$}
\RCSdefDate{\FaceIntervalDate}{$Date$}

%----------------------------------------------------------------------

\begin{ccRefClass} {Level_interval<FaceHandle>}

\ccDefinition

The class \ccClassTemplateName\ represents intervals for the minimum and
maximum value of the \ccc{z}-coordinate of a face of a triangulation.



\ccInclude{CGAL/Level_interval.h}

\ccSetThreeColumns{Oriented_side}{}{\hspace*{10cm}}
\ccThreeToTwo



\ccRequirements
The \ccc{value_type} of  \ccc{FaceHandle} must be \ccc{Face}, which must have a 
nested type \ccc{Vertex}, which must have a nested type \ccc{Point},
whose \ccc{Kernel_traits<Point>Kernel} must have a nested type \ccc{FT}.
These requirements are fulfilled, if one uses a {\sc Cgal} triangulation
and a {\sc Cgal} kernel.

\ccTypes

\ccTypedef{typedef FT Value;}{The type of the $z$-coordinate of points stored in vertices of faces.} 

\ccCreation
\ccCreationVariable{i}

\ccConstructor{Level_interval();}
{Default constructor.}

\ccConstructor{Level_interval(FaceHandle fh);}
{Constructs the interval with smallest and largest \ccc{z} coordinate of the points
stored in the vertices of the face \ccc{fh} points to.}

\ccOperations


\ccMethod{FaceHandle face_handle();}
{returns the face handle.}

\ccHeading{I/O}

\ccFunction{ostream& operator<<(ostream& os,
                  const Level_interval<FaceHandle>& i);}
{Inserts the interval \ccc{i} into the stream \ccc{os}.
\ccPrecond The output operator for \ccc{*Face_handle} is defined.}

\ccIsModel
\ccc{Interval}

\end{ccRefClass}

