% +------------------------------------------------------------------------+
% | Reference manual page: MarchingTetrahedraTraits_3.tex
% +------------------------------------------------------------------------+
% | 27.09.2005   Nico Kruithof
% | Package: Skin_surface
% | 
\RCSdef{\RCSMarchingtetrahedratraitsRev}{$Revision$}
\RCSdefDate{\RCSMarchingtetrahedratraitsDate}{$Date$}
% +------------------------------------------------------------------------+

\ccRefPageBegin

%%RefPage: end of header, begin of main body
% +------------------------------------------------------------------------+


\begin{ccRefConcept}{MarchingTetrahedraTraits_3}
%\label{pageMarchingTetrahedraTraits}

  Required types and member functions for the \ccRefName\ concept.
  This geometric traits concept is the by the function
  \ccc{CGAL::marching_tetrahedra_3} for the extraction of a mesh from
  a triangulation.

\ccRefines
  
\ccTypes

\ccNestedType{Converter;}{Function object that can convert points defined by
  the \ccc{triangulation} to points defined by the \ccc{HDS}}

\ccNestedType{Vertex_handle;}{handle to the Vertex type of the triangulation}
\ccGlue
\ccNestedType{Cell_handle;}{handle to the Cell type of the triangulation}

\ccNestedType{HDS_rt;}{}
\ccGlue
\ccNestedType{HDS_point;}{the point type defined by the kernel of the
  polyhedron}

\ccCreation
\ccCreationVariable{traits}

\ccConstructor{MarchingTetrahedraTraits_3(HDS_rt iso_value=0);}{}

\ccOperations

\ccMethod{Sign sign(const Vertex_handle vh) const;}{returns whether a
  vertex lies inside or outside the surface. The value
  \ccc{CGAL::POSITIVE} implies outside the surface, all other values
  inside the surface.}
\ccMethod{HDS_point intersection(Cell_handle ch, int i, int j)
  const;}{Returns the point of the polyhedral vertex lying on the edge
  between \ccc{ch->vertex(i)} and \ccc{ch->vertex(j)}.}

\ccHasModels

\ccRefIdfierPage{CGAL::Marching_tetrahedra_traits_skin_surface_3<Triangulation_3, HalfedgeDS, Converter_ >}

%\ccSeeAlso
%\ccRefIdfierPage{Skin_surface_traits_3<Traits>}


\end{ccRefConcept}

% +------------------------------------------------------------------------+
%%RefPage: end of main body, begin of footer
\ccRefPageEnd
% EOF
% +------------------------------------------------------------------------+

