\ccHeading{Restricted Spherical Geometry}

We introduce geometric objects that are part of the spherical surface
$S_2$ and operations on them. We define types \ccc{Sphere_point},
\ccc{Sphere_circle}, \ccc{Sphere_segment}, and \ccc{Sphere_direction}.
\ccc{Sphere_point}s are points on $S_2$, \ccc{Sphere_circle}s are
oriented great circles of $S_2$, \ccc{Sphere_segment}s are oriented
parts of \ccc{Sphere_circles} bounded by a pair of
\ccc{Sphere_point}s, and \ccc{Sphere_direction}s are directions that
are part of great circles. (a direction is usually defined to be a
vector without length, that floats around in its underlying space and
can be used to specify a movement at any point of the underlying
space; in our case we use directions only at points that are part of
the great circle that underlies also the direction.)

Note that we have to consider special geometric properties of the
objects. For example two points that are part of a great circle define
two \ccc{Sphere_segment}s, and two arbitrary \ccc{Sphere_segment}s can
intersect in two points.

If we restrict our geometric objects to a so-called perfect hemisphere
of $S_2$\footnote{A perfect hemisphere of $S_2$ is an open half-sphere
  plus an open half-circle in the boundary of the open half-sphere
  plus one endpoint of the half-circle.} then the restricted objects
behave like in classical geometry, e.g., two points define exactly one
segment, two segments intersect in at most one interior point
(non-degenerately), or three non-cocircular sphere points can be
qualified as being positively or negatively oriented.

% begin cgal manual page

\begin{ccRefClass}{Sphere_point<R>}\ccCreationVariable{p}

\ccDefinition

An object \ccc{p} of type \ccc{Sphere_point<R>} is a point on the
surface of a unit sphere. Such points correspond to the nontrivial
directions in space and similarly to the equivalence classes of all
nontrivial vectors under normalization.

\ccSetOneOfTwoColumns{5cm}

\ccTypes

\ccNestedType{R}{representation class.}

\ccNestedType{RT}{field number type.}

\ccSetOneOfTwoColumns{5cm}

\ccCreation

\ccConstructor{Sphere_point<R>()}{ creates some sphere point.  }

\ccConstructor{Sphere_point<R>(RT x, RT y, RT z)}{ creates a sphere
  point corresponding to the point of intersection of the ray starting
  at the origin in direction $(x,y,z)$ and the surface of $S_2$.  }

\ccSetTwoOfThreeColumns{4cm}{2cm}

\ccOperations

Access to the coordinates is provided by the following operations.
Note that the vector $(x,y,z)$ is not normalized.

\ccMethod{RT x() ;}{ the $x$-coordinate.  }

\ccMethod{RT y() ;}{ the $y$-coordinate.  }

\ccMethod{RT z() ;}{ the $z$-coordinate.  }

\ccMethod{bool operator==(const Sphere_point<R>& q) ;}{Equality.}

\ccMethod{bool operator!=(const Sphere_point<R>& q) ;}{Inequality.}

\ccMethod{Sphere_point<R> antipode() ;}{returns the antipode of \ccc{p}.}

\end{ccRefClass}
% begin cgal manual page

\begin{ccRefClass}[Nef_polyhedron_S2<Traits>::]{Sphere_segment}
\ccCreationVariable{s}

\ccDefinition

An object \ccc{s} of type \ccc{Sphere_segment} is a segment in the
surface of a unit sphere that is part of a great circle trough the
origin. Sphere segments are represented by two sphere points $p$ and
$q$ plus an oriented plane $h$ that contains $p$ and $q$. The plane
determines the sphere segment as follows. Let $c$ be the circle in the
intersection of $h$ and $S_2$. Then $s$ is that part of $c$ that is
swept, when we rotate $p$ into $q$ in counterclockwise rotation around
the normal vector of $h$ as seen from the positive halfspace.

\ccSetOneOfTwoColumns{4cm}

\ccCreation

\ccConstructor{Sphere_segment()}{ creates some sphere segment.  }

\ccConstructor{Sphere_segment(Sphere_point p1, Sphere_point
  p2, bool shorter_arc=true)}{ creates a spherical segment spanning
  the shorter arc from \ccc{p1} to \ccc{p2} if \ccc{shorter_arc ==
    true}. Otherwise the longer arc is created. \ccPrecond \ccc{p1 !=
    p2} and \ccc{p1 != p2.opposite()}.  }

\ccConstructor{Sphere_segment(Sphere_point p1, Sphere_point
  p2, Sphere_circle c)}{ creates a spherical segment spanning the
  arc from \ccc{p1} to \ccc{p2} as part of the oriented circle \ccc{c}
  (\ccc{p1 == p2} or \ccc{p1 == p2.opposite()} are possible.)
  \ccPrecond \ccc{p1} and \ccc{p2} are contained in \ccc{c}.  }

\ccConstructor{Sphere_segment(Sphere_circle c1, Sphere_circle
  c2)}{ creates the spherical segment as part of \ccc{c1} that is part
  of the halfsphere left of the oriented circle \ccc{c2}.  \ccPrecond
  \ccc{c1 != c2} as unoriented circles.  }

\ccSetTwoOfThreeColumns{3.5cm}{2.5cm}

\ccOperations

\ccMethod{const Sphere_point source() ;}{ the source point of
  \ccc{s}.  }

\ccMethod{const Sphere_point target() ;}{ the target point of
  \ccc{s}.  }

\ccMethod{const Sphere_circle sphere_circle() ;}{ the great circle
  supporting \ccc{s}.  }

\ccMethod{Sphere_segment opposite() ;}{ returns the sperical
  segment oriented from \ccc{target()} to \ccc{source()} with the same
  point set as \ccc{s}.  }

\ccMethod{Sphere_segment complement() ;}{ returns the sperical
  segment oriented from \ccc{target()} to \ccc{source()} with the
  point set completing \ccc{s} to a full circle.  }

\ccMethod{int intersection(const Sphere_circle c,
  Sphere_segment s1, Sphere_segment s2) ;}{ returns the number
  of non-trivial connected components of the intersection of \ccc{s}
  and the closed halfsphere left of \ccc{c}.  }

\ccMethod{Sphere_point intersection(const Sphere_segment so)
  ;}{ returns the point of intersection of \ccc{s} and \ccc{so}.
  \ccPrecond \ccc{s} and \ccc{so} do intersect.  }

\ccMethod{void split_halfcircle(Sphere_segment s1,
  Sphere_segment s2) ;}{ splits a halfcircle into two equally
  sized segments.  \ccPrecond \ccc{s} is a halfcircle.  }

\ccMethod{bool is_short() ;}{ a segment is short iff it is shorter
  than a half-circle.  }

\ccMethod{bool is_long() ;}{ a segment is long iff it is longer than a
  half-circle.  }

\ccMethod{bool is_degenerate() ;}{ return true iff \ccc{s} is
  degenerate, \\ i.e. source and target are the same.  }

\ccMethod{bool is_halfcircle() ;}{ return true iff \ccc{s} is a
  perfect half-circle, i.e. \ccc{source().antipode == target()}.  }

\ccMethod{bool has_on(const Sphere_point p) ;}{ return true iff
  \ccc{s} contains \ccc{p}.  }

\ccMethod{bool has_in_relative_interior(const Sphere_point p) ;}{
  return true iff \ccc{s} contains \ccc{p} in its relative interior.
  }

\ccFunction{bool do_intersect_internally(const
  Sphere_segment s1, const Sphere_segment s2, Sphere_point
  p);}{ return true iff \ccc{s1} and \ccc{s2} intersect internally, i.e. the 
  intersection point is not equal to any endpoint of \ccc{s1} and \ccc{s2}.  
  If \ccc{true} the parameter \ccc{p} returns the
  point of intersection.  }

\end{ccRefClass}
% begin cgal manual page

\begin{ccRefClass}[Nef_polyhedron_S2<Traits>::]{Sphere_circle}
\ccCreationVariable{c}
  
\ccDefinition

An object \ccc{c} of type \ccc{Sphere_circle} is an oriented great
circle on the surface of a unit sphere.  Such circles correspond to
the intersection of an oriented plane (that contains the origin) and
the surface of $S_2$. The orientation of the great circle is that of a
counterclockwise walk along the circle as seen from the positive
halfspace of the oriented plane.

\ccSetOneOfTwoColumns{6.5cm}

\ccTypes

\ccNestedType{RT}{ ring type.  }

\ccNestedType{Plane_3}{ plane a \ccc{Sphere_circle} lies in.}

\ccCreation

\ccConstructor{Sphere_circle()}{ creates some great circle.  }

\ccConstructor{Sphere_circle(const Sphere_point& p, 
  const Sphere_point& q)}{ If $p$ and $q$ are
  opposite of each other, then we create the unique great circle on $S_2$ 
  which contains p and q. This circle is oriented such
  that a walk along \ccc{c} meets $p$ just before the shorter segment
  between $p$ and $q$. If $p$ and $q$ are opposite of each other then
  we create any great circle that contains $p$ and $q$.  }

\ccConstructor{Sphere_circle(const Plane_3& h)}{ creates the
  circle corresponding to the plane \ccc{h}.  \ccPrecond{\ccc{h}
  contains the origin.}  }

\ccConstructor{Sphere_circle(const RT& x, const RT& y, const RT& z)}
{ creates the circle orthogonal to the vector $(x,y,z)$.  }

\ccConstructor{Sphere_circle(Sphere_circle c, const Sphere_point&
  p)}{ creates a great circle orthogonal to $c$ that contains $p$.
  \ccPrecond{$p$ is not part of $c$.}  }

\ccSetTwoOfThreeColumns{2.5cm}{3.8cm}

\ccOperations

\ccMethod{Sphere_circle opposite() ;}{ Returns a sphere circle 
  in the oppostie direction of \ccc{c}.  }

\ccMethod{bool has_on(const Sphere_point& p) ;}{ returns true iff
  \ccc{c} contains \ccc{p}.  }

\ccMethod{Plane_3 plane() ;}{ returns the plane supporting \ccc{c}.  }

\ccMethod{Sphere_point orthogonal_pole() ;}{ returns the point that
  is the pole of the hemisphere left of \ccc{c}.  }

\ccHeading{Global functions} 

\ccFunction{bool equal_as_sets(
  const Nef_polyhedron_S2<Traits>::Sphere_circle c1, 
  const Nef_polyhedron_S2<Traits>::Sphere_circle c2) ;}{
  returns true iff \ccc{c1} and \ccc{c2} are equal as unoriented
  circles.  }

\end{ccRefClass}

% begin cgal manual page

\begin{ccRefClass}{Sphere_direction<R>}\ccCreationVariable{d}

\ccDefinition

An object \ccc{c} of type \ccc{Sphere_direction<R>} is a direction on
the surface of the unit sphere.  Such directions can be used to
describe walks that are part of great circles.

\ccSetOneOfTwoColumns{5cm}

\ccTypes

\ccNestedType{R}{representation class.}

\ccNestedType{RT}{ring type.}

\ccSetOneOfTwoColumns{5cm}

\ccCreation

\ccConstructor{Sphere_direction<R>()}{creates some direction.}

\ccConstructor{Sphere_direction<R>(Sphere_circle<R> c)}{ creates the
  direction corresponding to the circle \ccc{c}.  }

\ccConstructor{Sphere_direction<R>(Sphere_point<R> p, Sphere_point<R>
  q)}{ creates a direction that describes the orientation of the great
  circle through $p$ and $q$ (oriented such that the segment $pq$ is
  the shorter one of the two possible ones. \ccPrecond $p$ and $q$ are
  not opposite on $S_2$.  }

\ccSetTwoOfThreeColumns{4cm}{2cm}

\ccOperations

\ccMethod{Sphere_direction<R> opposite() ;}{returns the opposite of
  \ccc{d}.  }

\end{ccRefClass}
% begin cgal manual page

\begin{ccRefClass}{Sphere_triangle<R>}\ccCreationVariable{t}

\ccDefinition

An object \ccc{t} of type \ccc{Sphere_triangle<R>} is a triangle on
the surface of the unit sphere.

\ccSetOneOfTwoColumns{5cm}

\ccTypes

\ccNestedType{R}{ representation class.  }

\ccNestedType{RT}{ ring type.  }

\ccSetOneOfTwoColumns{5cm}

\ccCreation

\ccConstructor{Sphere_triangle<R>()}{ creates some triangle.  }

\ccConstructor{Sphere_triangle<R>(Sphere_point<R> p0, Sphere_point<R>
  p1, Sphere_point<R> p2, Sphere_circle<R> c0, Sphere_circle<R> c1,
  Sphere_circle<R> c2)}{ creates a triangle spanned by the three
  points \ccc{p0}, \ccc{p1}, \ccc{p2}, where the triangle is left of
  the three circles \ccc{c0}, \ccc{c1}, \ccc{c2}. \ccPrecond{$c_i$
  contains $p_i$ and $p_{i+1}$ mod 3.}  }

\ccSetTwoOfThreeColumns{4cm}{2cm}

\ccOperations

\ccMethod{const Sphere_point<R>& point(unsigned i) ;}{ returns the ith
  point of \ccc{t}.  }

\ccMethod{const Sphere_circle<R>& circle(unsigned i) ;}{ returns the
  ith circle of \ccc{t}.  }

\ccMethod{Sphere_triangle<R> opposite() ;}{ returns the opposite of
  \ccc{t}.  }

\end{ccRefClass}


