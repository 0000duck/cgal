% ======================================================================
%
% Copyright (c) 1999 The GALIA Consortium
%
% This software and related documentation is part of the
% Computational Geometry Algorithms Library (CGAL).
%
% Every use of CGAL requires a license. Licenses come in three kinds:
%
% - For academic research and teaching purposes, permission to use and
%   copy the software and its documentation is hereby granted free of  
%   charge, provided that
%   (1) it is not a component of a commercial product, and
%   (2) this notice appears in all copies of the software and
%       related documentation.
% - Development licenses grant access to the source code of the library 
%   to develop programs. These programs may be sold to other parties as 
%   executable code. To obtain a development license, please contact
%   the GALIA Consortium (at cgal@cs.uu.nl).
% - Commercialization licenses grant access to the source code and the
%   right to sell development licenses. To obtain a commercialization 
%   license, please contact the GALIA Consortium (at cgal@cs.uu.nl).
%
% This software and documentation is provided "as-is" and without
% warranty of any kind. In no event shall the CGAL Consortium be
% liable for any damage of any kind.
%
% The GALIA Consortium consists of Utrecht University (The Netherlands),
% ETH Zurich (Switzerland), Free University of Berlin (Germany),
% INRIA Sophia-Antipolis (France), Martin-Luther-University Halle-Wittenberg
% (Germany), Max-Planck-Institute Saarbrucken (Germany),
% and Tel-Aviv University (Israel).
%
% ----------------------------------------------------------------------
%
% package       : Alpha_shapes_3
% author(s)     : Tran Kai Frank DA <Frank.Da@sophia.inria.fr>
%
% coordinator   : INRIA Sophia-Antipolis (<Mariette.Yvinec@sophia.inria.fr>)
%
% ======================================================================

\RCSdef{\alphashapeRevision}{$Revision$}
\RCSdefDate{\alphashapeDate}{$Date$}

%----------------------------------------------------------------------

\chapter{Alpha-Shapes 3D} \label{I1_ChapterAlphashapes3D}

%\ccChapterSubTitle{\alphashapeRevision, \alphashapeDate}

\section{Introduction}

This chapter presents a framework for alpha shapes. The description is based on
the articles \cite{em-tdas-94,e-was-92}. Alpha shapes are
the generalization of the convex hull of a point set. Let $S$ be a finite set of
points in $\R^d$, $d = 2,3$ and $\alpha$ a parameter with $0 \leq \alpha \leq
\infty$. For $\alpha = \infty$, the $\alpha$-shape is the convex hull of $S$. As 
$\alpha$ decreases, the $\alpha$-shape shrinks and develops cavities, as soon as 
a sphere of radius $\sqrt{\alpha}$ can be put inside.
Finally, for $\alpha = 0$, the $\alpha$-shape is the set $S$ itself.

We distinguish two versions of alpha shapes, one is based on the Delaunay
triangulation and the other on its generalization, the regular triangulation,
replacing the natural distance by the power to weighted points. The metric used
determines an underlying triangulation of the alpha shape and thus, the version
computed. 
In one hand, there is the {\em classic alpha shapes}
(cf. \ref{I1_SectClassicAS3D}) associated with the Delaunay triangulations
(cf. \ref{chapter-Triangulation3}), in the other hand, the {\em weighted alpha shapes}
(cf. \ref{I1_SectWeightedAS3D}) associated with the regular triangulations
(cf. \ref{Triangulation3-sec-class-Regulartriangulation}). 

There is a close connection between alpha shapes and the underlying
triangulations. More precisely, the $\alpha$-complex of $S$ is a
subcomplex of this triangulation of $S$, containing the $\alpha$-exposed
$k$-simplices, $0 \leq k \leq d$. A simplex is $\alpha$-exposed, if there is an
open disk (resp.\ ball) of radius $\sqrt{\alpha}$ through the vertices of the
simplex that does not contain any other point of $S$, for the metric used in
the computation of the underlying triangulation.  The corresponding
$\alpha$-shape is defined as the underlying interior space of the
$\alpha$-complex. 

In general, an $\alpha$-complex is a non-connected and non-pure polytope, it
means, that one $k$-simplice, $0 \leq k \leq d-1$ is not necessary adjacent to
a $(k+1)$-simplice.

The $\alpha$-shapes of $S$ form a discrete family, even though they
are defined for all real numbers $\alpha$ with $0 \leq \alpha
\leq \infty$. Thus, we can represent the entire family of $\alpha$-shapes
of $S$ by the underlying triangulation of $S$. In this representation
each $k$-simplex of the underlying triangulation is associated with an
interval that specifies for which values of $\alpha$ the $k$-simplex
belongs to the $\alpha$-shape. Relying on this result, the family of
$\alpha$-shapes can be computed efficiently and relatively
easily. Furthermore, we can select an appropriate $\alpha$-shape from a
finite number of different $\alpha$-shapes and corresponding
$\alpha$-values.

\subsection*{Main Application}

Alpha shapes can be used for shape reconstruction from a dense unorganized set
of data points. Indeed, an  $\alpha$-shape is demarcated by a frontier, which is
a linear approximation of the original contour, under conditions described for
the classic version, in \cite{bb-srmua-97t}.

%----------------------------------------------------------------------

\section{Generic Alpha-Shapes of Points in 3D Space \label{I1_SectAlpha_Shape_3}}

The class \ccc{Alpha_shape_3<Dt>} represents the family of
$\alpha$-shapes of points in 3D Space for {\em all} positive
$\alpha$. It maintains the underlying triangulation \ccc{Dt} which
represents connectivity and order among its faces. Each
$k$-dimensional face of the \ccc{Dt} is associated with
an interval that specifies for which values of $\alpha$ the face
belongs to the $\alpha$-shape. There are links between the intervals
and the $k$-dimensional faces of the triangulation.

The main class of this package is documented in the Reference Manual
\ccRefPage{CGAL::Alpha_shape_3<Dt>} and can be included from~: 

\ccInclude{CGAL/Alpha_shape_3.h}

%----------------------------------------------------------------------

\section{Requirements for the Underlying Triangulation\label{I1_SectDtClass3D}}

For the \ccc{Dt} class, you have to choose between a Delaunay or a regular
triangulation, depending the version of alpha shapes, you want.

The class \ccc{Dt} must be parameterized with a special alpha shape
traits class and a simple triangulation data structure, with no more requirements as
for a simple triangulation class, but parameterized by a special vertex base
class and a special cell base class.

\subsection*{The Alpha-Shape Traits Class (\mbox{\it Gt})\label{I1_ASTraits3D}} 

\subsection{Requirements for the Alpha-Shape Traits Class}

First, this traits class has the same requirements as the 
triangulation traits class, for the \ccc{Dt} triangulation, you opted.

Furthermore, you have also to provide specific methods to compute Alpha-Shapes~:\\
as two constructions to compute the minimum squared radius for a cell, for a
facet and for an edge, and a predicate to certify if a point is in the smallest
circle associated with an facet or not. You can find the complete description in
the Reference Manual \ccRefPage{CGAL::Alpha_shape_traits_3}.

\subsection*{Triangulation Data Structure Class (\mbox{\it Tds}) for an Alpha-Shape}

The underlying triangulation \ccc{Dt} is parameterized with her
triangulation data structure class \ccc{Tds}, that has no additional
requirements. In fact, the requirements are on the \ccc{Tds} parameters.

She requires to be templated by special \ccc{Alpha_shape_vertex_base_3} and
\ccc{Alpha_shape_cell_base_3<Df>}, to store the alpha values associated. Thus, we need to
define the requirements of such \ccc{Alpha_shape_vertex_base_3} and
\ccc{Alpha_shape_cell_base_3<Df>} classes.

\subsection{Requirements for the \protect \ccc{Vertex_base} Class of an
Alpha-Shape\label{I1_SectVertexAS3D}} 

The information about the alpha values associated  are accessible by the 
vertices of the alpha shape. Thus the nested \ccc{Alpha_shape_vertex_base_3}
type of an alpha shape offers additional functionalities to deal with these
methods.  

This additional functionalities related to the alpha shape
are requirements which have to be fulfilled by the base vertex for an alpha
shape. You can find the complete description in the Reference Manual
\ccRefPage{CGAL::Alpha_shape_vertex_base_3}.  

\subsection{Requirements for the \protect \ccc{Cell_base} Class of an Alpha-Shape}

The information about the alpha values, associated to the cell herself and the
incident edges, are accessible by the 
cells of the alpha shape. Thus the nested \ccc{Alpha_shape_cell_base_3<Df>}
type of an alpha shape offers additional functionalities to deal with these
methods.

This additional functionalities related to the alpha shape are requirements
which have to be fulfilled by the base cell  for an alpha shape. You can find
the complete description in the Reference Manual
\ccRefPage{CGAL::Alpha_shape_cell_base_3<Df>}.

%----------------------------------------------------------------------

\section{Classic Version of Alpha-Shapes\label{I1_SectClassicAS3D}}

\subsection*{The Underlying Triangulation~: a Delaunay Triangulation}

For a simple alpha shape, you have to choose a Delaunay triangulation as
underlying triangulation \ccc{Dt} and follow the requirements listed above.

\ccInclude{CGAL/Delaunay_triangulation_3.h}

\subsection*{Predefined Geometric Traits Class (\mbox{\it Gt})}

Of course, \cgal\ provides a default \ccc{Alpha_shape_traits_3} class in this
case, with an implementation of the appropriate predicate and constructions.
The class \ccc{Alpha_shape_euclidean_traits_3<Rp>} simply derived from 
\ccc{Triangulation_euclidean_traits_3<Rp>}.

\ccInclude{CGAL/Alpha_shape_euclidean_traits_3.h}

\subsection*{Predefined \protect \ccc{Vertex_base} Class}

\cgal\ provides a default \ccc{Vertex_base} class
for the Alpha Shape. The class
\ccc{Alpha_shape_vertex_base_3<Gt>} simply derived from 
\ccc{Triangulation_vertex_base_3<Gt>}.

\ccInclude{CGAL/Alpha_shape_vertex_base_3.h}

\subsection*{Predefined \protect \ccc{Cell_base} Class}

\cgal\ provides a default \ccc{Cell_base} class
for the Alpha Shape. The class
\ccc{Alpha_shape_cell_base_3<Gt,Df>} simply derived from 
\ccc{Df}, which can be instantiated with \ccc{Triangulation_cell_base_3<Gt>}.

\ccInclude{CGAL/Triangulation_cell_base_3.h}

\ccInclude{CGAL/Alpha_shape_cell_base_3.h}

\subsection*{To Do a Classic Alpha-Shape}

And last, you have to include~:

\ccInclude{CGAL/Alpha_shape_3.h}

\ccExample
The following code details the different steps to create 
a classic version of alpha shape with the defaults.

\begin{cprog}
typedef CGAL::Cartesian<double> Rp;
typedef CGAL::Alpha_shape_euclidean_traits_3<Rp> Gt;
typedef CGAL::Alpha_shape_vertex_base_3<Gt> Vb;
typedef CGAL::Triangulation_cell_base_3<Gt> Df;
typedef CGAL::Alpha_shape_cell_base_3<Gt,Df> Fb;
typedef CGAL::Triangulation_default_data_structure_3<Gt,Vb,Fb> Tds;
typedef CGAL::Delaunay_triangulation_3<Gt,Tds> Dt;
typedef CGAL::Alpha_shape_3<Dt> Alpha_shape_3;
\end{cprog}

%----------------------------------------------------------------------

\section{Weighted Version of Alpha-Shapes\label{I1_SectWeightedAS3D}}

\subsection*{The Underlying Triangulation~: a Regular Triangulation}

For a weighted alpha shape, you have to choose a regular triangulation as
underlying triangulation \ccc{Dt} and follow the requirements listed above.

\ccInclude{CGAL/Regular_triangulation_3.h}

\subsection*{Predefined Geometric Traits Class (\mbox{\it Gt})}

Of course, \cgal\ provides a default \ccc{Alpha_shape_traits_3} class in this
case, with an implementation of the appropriate predicate and constructions.
The class \ccc{Weighted_alpha_shape_euclidean_traits_3<Rp>} simply derived from 
\ccc{Regular_triangulation_euclidean_traits_3<Rp>}.

\ccInclude{CGAL/Weighted_alpha_shape_euclidean_traits_3.h}

\subsection*{Predefined \protect \ccc{Vertex_base} Class}

\cgal\ provides a default \ccc{Vertex_base} class
for the Alpha Shape. The class
\ccc{Alpha_shape_vertex_base_3<Gt>} simply derived from 
\ccc{Triangulation_vertex_base_3<Gt>}.

\ccInclude{CGAL/Alpha_shape_vertex_base_3.h}

\subsection*{Predefined \protect \ccc{Cell_base} Class}

\cgal\ provides a default \ccc{Cell_base} class
for the Alpha Shape. The class
\ccc{Alpha_shape_cell_base_3<Gt,Df>} simply derived from 
\ccc{Df}, which can be instantiated with \ccc{Regular_triangulation_cell_base_3<Gt>}.

\ccInclude{CGAL/Regular_triangulation_cell_base_3.h}

\ccInclude{CGAL/Alpha_shape_cell_base_3.h}

\subsection*{To Do a Weighted Alpha-Shape}

And last, you have to include~:

\ccInclude{CGAL/Alpha_shape_3.h}

\ccExample
The following code details the different steps to create 
a weighted version of alpha shape with the defaults.

\begin{cprog}
typedef CGAL::Cartesian<double> Rp;
typedef CGAL::Weighted_alpha_shape_euclidean_traits_3<Rp> Gt;
typedef CGAL::Alpha_shape_vertex_base_3<Gt> Vb;
typedef CGAL::Regular_triangulation_cell_base_3<Gt> Rf;
typedef CGAL::Alpha_shape_cell_base_3<Gt,Rf>  Fb;
typedef CGAL::Triangulation_default_data_structure_3<Gt,Vb,Fb> Tds;
typedef CGAL::Regular_triangulation_3<Gt,Tds> Rt;
typedef CGAL::Alpha_shape_3<Rt> Alpha_shape_3;
\end{cprog}
