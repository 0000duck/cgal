% ======================================================================
%
% Copyright (c) 1999 The GALIA Consortium
%
% This software and related documentation is part of the
% Computational Geometry Algorithms Library (CGAL).
%
% Every use of CGAL requires a license. Licenses come in three kinds:
%
% - For academic research and teaching purposes, permission to use and
%   copy the software and its documentation is hereby granted free of  
%   charge, provided that
%   (1) it is not a component of a commercial product, and
%   (2) this notice appears in all copies of the software and
%       related documentation.
% - Development licenses grant access to the source code of the library 
%   to develop programs. These programs may be sold to other parties as 
%   executable code. To obtain a development license, please contact
%   the GALIA Consortium (at cgal@cs.uu.nl).
% - Commercialization licenses grant access to the source code and the
%   right to sell development licenses. To obtain a commercialization 
%   license, please contact the GALIA Consortium (at cgal@cs.uu.nl).
%
% This software and documentation is provided "as-is" and without
% warranty of any kind. In no event shall the CGAL Consortium be
% liable for any damage of any kind.
%
% The GALIA Consortium consists of Utrecht University (The Netherlands),
% ETH Zurich (Switzerland), Free University of Berlin (Germany),
% INRIA Sophia-Antipolis (France), Martin-Luther-University Halle-Wittenberg
% (Germany), Max-Planck-Institute Saarbrucken (Germany),
% and Tel-Aviv University (Israel).
%
% ----------------------------------------------------------------------
%
% package       : Alpha_shapes_3
% author(s)     : Tran Kai Frank DA <Frank.Da@sophia.inria.fr>
%
% coordinator   : INRIA Sophia-Antipolis (<Mariette.Yvinec@sophia.inria.fr>)
%
% ======================================================================

\RCSdef{\alphashapeRevision}{$Revision$}
\RCSdefDate{\alphashapeDate}{$Date$}

%----------------------------------------------------------------------

\clearpage
\section{Reference pages for Alpha Shapes in 3D}

This chapter presents a framework for alpha shapes. The description is based on
the articles \cite{em-tdas-94,e-was-92}. Alpha shapes are
the generalization of the convex hull of a point set. Let $S$ be a finite set of
points in $\R^d$, $d = 2,3$ and $\alpha$ a parameter with $0 \leq \alpha \leq
\infty$. For $\alpha = \infty$, the $\alpha$-shape is the convex hull of $S$. As 
$\alpha$ decreases, the $\alpha$-shape shrinks and develops cavities, as soon as 
a sphere of radius $\sqrt{\alpha}$ can be put inside.
Finally, for $\alpha = 0$, the $\alpha$-shape is the set $S$ itself.

We distinguish two versions of alpha shapes, one is based on the Delaunay
triangulation and the other on its generalization, the regular triangulation,
replacing the natural distance by the power to weighted points. The metric used
determines an underlying triangulation of the alpha shape and thus, the version
computed. 
In one hand, there is the {\em classic alpha shapes}
(cf. \ref{I1_SectClassicAS3D}) associated with the Delaunay triangulations
(cf. \ref{I1_Sect_Delaunay}), in the other hand, the {\em weighted alpha shapes}
(cf. \ref{I1_SectWeightedAS3D}) associated with the regular triangulations
(cf. \ref{I1_Sect_Regular}). 

There is a close connection between alpha shapes and the underlying
triangulations. More precisely, the $\alpha$-complex of $S$ is a
subcomplex of this triangulation of $S$, containing the $\alpha$-exposed
$k$-simplices, $0 \leq k \leq d$. A simplex is $\alpha$-exposed, if there is an
open disk (resp.\ ball) of radius $\sqrt{\alpha}$ through the vertices of the
simplex that does not contain any other point of $S$, for the metric used in
the computation of the underlying triangulation.  The corresponding
$\alpha$-shape is defined as the underlying interior space of the
$\alpha$-complex. 

In general, an $\alpha$-complex is a non-connected and non-pure polytope, it
means, that one $k$-simplice, $0 \leq k \leq d-1$ is not necessary adjacent to
a $(k+1)$-simplice.

The $\alpha$-shapes of $S$ form a discrete family, even though they
are defined for all real numbers $\alpha$ with $0 \leq \alpha
\leq \infty$. Thus, we can represent the entire family of $\alpha$-shapes
of $S$ by the underlying triangulation of $S$. In this representation
each $k$-simplex of the underlying triangulation is associated with an
interval that specifies for which values of $\alpha$ the $k$-simplex
belongs to the $\alpha$-shape. Relying on this result, the family of
$\alpha$-shapes can be computed efficiently and relatively
easily. Furthermore, we can select an appropriate $\alpha$-shape from a
finite number of different $\alpha$-shapes and corresponding
$\alpha$-values.

\subsection*{Classes}

\ccRefIdfierPage{CGAL::Alpha_shape_3<Dt>}\\
\ccRefIdfierPage{CGAL::Alpha_shape_traits_3}\\
\ccRefIdfierPage{CGAL::Alpha_shape_vertex_base_3}\\
\ccRefIdfierPage{CGAL::Alpha_shape_cell_base_3<Df>}\\
