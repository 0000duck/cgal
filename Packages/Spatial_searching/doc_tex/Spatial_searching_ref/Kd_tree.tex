% +------------------------------------------------------------------------+
% | Reference manual page: Kd_tree.tex
% +------------------------------------------------------------------------+
% | 1.07.2001   Johan W.H. Tangelder
% | Package: ASPAS
% | 
\RCSdef{\RCSKdtreeRev}{$Revision$}
\RCSdefDate{\RCSKdtreeDate}{$Date$}
% |
%%RefPage: end of header, begin of main body
% +------------------------------------------------------------------------+


\begin{ccRefClass}{Kd_tree<PointTraits, Splitter, UseExtendedNode>}  %% add template arg's if necessary

%% \ccHtmlCrossLink{}     %% add further rules for cross referencing links
%% \ccHtmlIndexC[class]{} %% add further index entries

\ccDefinition  
The class \ccRefName\ defines a $k$-$d$ tree.

\ccInclude{CGAL/Kd_tree.h}

\ccParameters

Expects for the first template argument a model of the concept
\ccc{PointTraits}, for example \ccc{CGAL::Kd_tree_traits_point_2<Kernel>}.

Excepts for the second template argument a model for the concept \ccc{Splitter}.
It defaults  to \ccc{Sliding_midpoint<PointTraits>}

Expects for the third template argument \ccc{CGAL::Tag_true}, if the 
tree shall be built with extended nodes, and \ccc{CGAL::Tag_false} else.

\ccTypes

\ccTypedef{PointTraits::Point Point;} {Point class.}
\ccTypedef{PointTraits::NT NT;}{Number type.}
\ccNestedType{Splitter;}{Splitter type.}

\ccNestedType{iterator;}{Bidirectional iterator with value type \ccc{Point} that allows
	to enumerate all points in the tree.}

\begin{ccAdvanced}
\ccNestedType{Node_handle;}{A handle with value type \ccc{Kd_tree_node<PointTraits,Splitter>}.}
\ccNestedType{Point_iterator;} {Random access iterator 
with value type \ccc{Point*}.}
\end{ccAdvanced}

\ccCreation
\ccCreationVariable{tree}

\ccConstructor
{template <class InputIterator> Kd_tree(InputIterator first, InputIterator beyond, Splitter s=Splitter());}
{
Constructs a $k$-$d$ tree on the elements from the sequence 
\ccc{first, beyond} using the splitting rule implemented by \ccc{s}. 
The value type of the \ccc{InputIterator} must be \ccc{Point}.
}

% -----------------------------------
\ccOperations

\ccMethod{
template <class OutputIterator, class FuzzyQueryItem>
OutputIterator search(OutputIterator it, FuzzyQueryItem q);}
{Reports the points that are approximately contained by~\ccc{q}.
The types \ccc{FuzzyQueryItem::Point} and \ccc{Point} must be equivalent.} 

\ccMethod{iterator begin();}{Returns an iterator to the first point in the tree.}
\ccMethod{iterator end();}{Returns the corresponding past-the-end iterator.}





\ccMethod{int size();}
{Returns the number of items that are stored in the tree.}

\begin{ccAdvanced}

\ccMethod{Node_handle root();}
{Returns a handle to the root node of the tree.}

\ccMethod{Kd_tree_rectangle<PointTraits>* bounding_box();}{returns a pointer to the bounding box 
of the root node of the tree.}

\ccMethod{std::ostream& statistics(std::ostream& s);}
{
Inserts statistics of the tree into the output stream \ccc{s}.
}

\end{ccAdvanced}

% -----------------------------------

\ccSeeAlso
Tree.
\ccc{CGAL::Kd_tree_node<PointTraits>},\\
\ccc{CGAL::Kd_tree_traits_point_2<Kernel>},\\
\ccc{CGAL::Kd_tree_traits_point_2<Kernel>}.

\end{ccRefClass}

% +------------------------------------------------------------------------+
%%RefPage: end of main body, begin of footer
% EOF
% +------------------------------------------------------------------------+

