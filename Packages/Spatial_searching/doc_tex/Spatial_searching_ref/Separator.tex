% +------------------------------------------------------------------------+
% | Reference manual page: Separator.tex
% +------------------------------------------------------------------------+
% | 1.07.2001   Johan W.H. Tangelder
% | Package: ASPAS
% | 
\RCSdef{\RCSSeparatorRev}{$Revision$}
\RCSdefDate{\RCSSeparatorDate}{$Date$}
% |
%%RefPage: end of header, begin of main body
% +------------------------------------------------------------------------+


\begin{ccRefConcept}{Separator<NT>}

%% \ccHtmlCrossLink{}     %% add further rules for cross referencing links
%% \ccHtmlIndexC[concept]{} %% add further index entries

\begin{ccAdvanced}

\ccDefinition
  
The concept \ccRefName\ defines the requirements for a separator.
A separator is a ($d$-1)-dimensional subspace that separates a $d$-dimensional space into two parts.
One part of space is said to be on the negative side of the separator and the other part of space is 
said to be on the positive side of the separator.

\ccCreation
\ccCreationVariable{s}  %% choose variable name

\ccConstructor{Separator();}{Default constructor.}

{\bf Member functions}
\ccMethod{template <class Item>
bool on_negative_side(Item& i);}
{Returns true if and only if the item value on the negative side of the separator.}


\ccHasModels

\ccc{Plane_separator}.

\ccSeeAlso

\ccc{Plane_separator}.

\end{ccAdvanced}

\end{ccRefConcept}

% +------------------------------------------------------------------------+
%%RefPage: end of main body, begin of footer
% EOF
% +------------------------------------------------------------------------+

