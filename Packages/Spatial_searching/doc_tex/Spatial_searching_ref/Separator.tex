% +------------------------------------------------------------------------+
% | Reference manual page: Separator.tex
% +------------------------------------------------------------------------+
% | 1.07.2001   Johan W.H. Tangelder
% | Package: ASPAS
% | 
\RCSdef{\RCSSeparatorRev}{$Revision$}
\RCSdefDate{\RCSSeparatorDate}{$Date$}
% |
%%RefPage: end of header, begin of main body
% +------------------------------------------------------------------------+


\begin{ccRefConcept}{Separator<NT>}

%% \ccHtmlCrossLink{}     %% add further rules for cross referencing links
%% \ccHtmlIndexC[concept]{} %% add further index entries

\ccDefinition
  
The concept \ccRefName\ defines the requirements for a separator.
A separator is a $d-1$-dimensional subspace that separates a $d$-dimensional space into two parts.

\ccCreation
\ccCreationVariable{s}  %% choose variable name

\ccConstructor{Separator();}{Default constructor.}

{\bf Member functions}
\ccMethod{template <class Item>
bool below(Item& i);}
{Returns true if and only if the item value in the cutting dimension is smaller than the cutting value.}


\ccHasModels

\ccc{Plane_separator}.

\ccSeeAlso

\ccc{Plane_separator}.

\end{ccRefConcept}

% +------------------------------------------------------------------------+
%%RefPage: end of main body, begin of footer
% EOF
% +------------------------------------------------------------------------+

