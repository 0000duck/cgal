% +------------------------------------------------------------------------+
% | Reference manual page: General_standard_search.tex
% +------------------------------------------------------------------------+
% | 1.07.2001   Johan W.H. Tangelder
% | Package: ASPAS
% | 
\RCSdef{\RCSGeneralstandardsearchRev}{$Revision$}
\RCSdefDate{\RCSGeneralstandardsearchDate}{$Date$}
% |
%%RefPage: end of header, begin of main body
% +------------------------------------------------------------------------+


\begin{ccRefClass}{General_standard_search<Traits, QueryItem, Distance>}  %% add template arg's if necessary

%% \ccHtmlCrossLink{}     %% add further rules for cross referencing links
%% \ccHtmlIndexC[class]{} %% add further index entries

\ccDefinition
  
The class \ccRefName\ implements approximate $k$-nearest and $k$-furthest neighbour searching
using standard search on a $k$-$d$ tree using a general distance 
class. It is not required that he $k$-$d$ tree is built with extended nodes.

\ccInclude{CGAL/General_standard_search.h}

\ccParameters

Expects for the parameter \ccc{Traits} an implementation of the concept TreeTraits.
Expects for the parameter \ccc{QueryItem}  an implementation of
of a $d$-dimensional point, for example \ccc{CGAL::Point_d<NT>}.
or an implementation of a spatial object, for
example \ccc{CGAL::Kd_tree_rectangle<NT>} implementing iso-rectangles.
Expects for the parameter \ccc{Distance} an implementation of the
concept GeneralDistance.

\ccTypes

\ccTypedef{Traits::Item Item;}{Item type.}
\ccTypedef{Traits::NT NT;}{Number type.}
\ccTypedef{std::pair<Item*,NT> Item_with_distance;}{Pair of item and distance} 
\ccTypedef{Kd_tree<Traits> Tree;}{$k$-$d$ tree storing the items.}

\ccNestedType{iterator}{Input iterator.}

\ccCreation
\ccCreationVariable{s}  %% choose variable name

\ccConstructor{General_standard_search(Tree& tree, QueryItem& q, Distance& d, int k, NT Eps,
bool Search_nearest=true);}
{Constructor for searching approximately $k$ neighbours of the query item \ccc{q} 
in the items stored in the $k$-$d$ tree \ccc{tree} using a general distance
traits class.} 

\ccHeading{Member Functions}

\ccMethod{template <class OutputIterator>
OutputIterator  the_k_neighbours(OutputIterator res);}
{Returns a sequence of $k$ elements of type \ccc{Item_with_distance}}. 

\ccMethod{void statistics();}
{
Prints statistics of the search process.
}

\ccSeeAlso

\ccc{Orthogonal_standard_search}.

\end{ccRefClass}

% +------------------------------------------------------------------------+
%%RefPage: end of main body, begin of footer
% EOF
% +------------------------------------------------------------------------+

