% +------------------------------------------------------------------------+
% | Reference manual page: Incremental_neighbor_search.tex
% +------------------------------------------------------------------------+
% | 1.07.2001   Johan W.H. Tangelder
% | Package: ASPAS
% |
\RCSdef{\RCSGeneralprioritysearchRev}{$Revision$}
\RCSdefDate{\RCSGeneralprioritysearchDate}{$Date$}
% |
%%RefPage: end of header, begin of main body
% +------------------------------------------------------------------------+


\begin{ccRefClass}{Incremental_neighbor_search<Traits, GeneralDistance, Splitter, SpatialTree>}

%% \ccHtmlCrossLink{}     %% add further rules for cross referencing links
%% \ccHtmlIndexC[class]{} %% add further index entries

\ccDefinition

The class \ccRefName\ implements incremental nearest and furthest neighbor searching
on a tree. The tree may have extended or unextended nodes.

\ccInclude{CGAL/Incremental_neighbor_search.h}

\ccParameters

Expects for the first template argument a model of the concept \ccc{SearchTraits},
for example \ccc{CGAL::Search_traits_2<CGAL::Cartesian<double> >}.

Expects for the second template argument a model of the
concept \ccc{GeneralDistance}. The default type is 
\ccc{CGAL::Euclidean_distance<Traits>}.


Expects for third template argument a model of the concept \ccc{SpatialTree}.
The default type is \ccc{CGAL::Kd_tree<Traits,Splitter,CGAL::Tag_false>}. The 
template argument \ccc{CGAL::Tag_false} makes that the tree is built with unextended nodes.

\ccTypes

\ccTypedef{Traits::Point_d Point_d;}{Point type.}
\ccTypedef{Traits::NT NT;}{Number type.}
\ccTypedef{std::pair<Point_d,NT> Point_with_distance;}{Pair of point and distance. }
\ccNestedType{iterator}{Input iterator with value type \ccc{Point_with_distance}
                        for enumerating approximate neighbors.}

\ccTypedef{GeneralDistance::Query_item Query_item;}{Query item type.}

\ccTypedef{SpatialTree Tree;}{The tree type.}


\ccCreation
\ccCreationVariable{s}  %% choose variable name


\def\ccLongParamLayout{\ccTrue}
\ccConstructor{Incremental_neighbor_search(Tree& tree, QueryItem q, NT eps=NT(0.0),
bool search_nearest=true, GeneralDistance d=GeneralDistance());}
{Constructor for incremental neighbor searching of the query item \ccc{q}
in the points stored \ccc{tree} using a distance \ccc{d} and approximation factor \ccc{eps}. }

\ccOperations

\ccMethod{iterator begin();}{Returns an iterator to the approximate nearest or furthest neighbor.}

\ccMethod{iterator end();}{Past-the-end iterator.}


\begin{ccAdvanced}
\ccMethod{std::ostream& statistics(std::ostream& s);}
{
Inserts statistics of the search process into the output stream \ccc{s}.
}
\end{ccAdvanced}


\ccSeeAlso

\ccc{CGAL::Orthogonal_incremental_neighbor_search<Traits, OrthogonalDistance, Splitter, SpatialTree>}.


\end{ccRefClass}

% +------------------------------------------------------------------------+
%%RefPage: end of main body, begin of footer
% EOF
% +------------------------------------------------------------------------+

