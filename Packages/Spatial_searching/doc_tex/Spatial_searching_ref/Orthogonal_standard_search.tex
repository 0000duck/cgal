% +------------------------------------------------------------------------+
% | Reference manual page: Orthogonal_standard_search.tex
% +------------------------------------------------------------------------+
% | 1.07.2001   Johan W.H. Tangelder
% | Package: ASPAS
% |
\RCSdef{\RCSOrthogonalstandardsearchRev}{$Revision$}
\RCSdefDate{\RCSOrthogonalstandardsearchDate}{$Date$}
% |
%%RefPage: end of header, begin of main body
% +------------------------------------------------------------------------+


\begin{ccRefClass}{Orthogonal_standard_search<PointTraits, OrthogonalDistance, Splitter, SpatialTree>}

%% \ccHtmlCrossLink{}     %% add further rules for cross referencing links
%% \ccHtmlIndexC[class]{} %% add further index entries

\ccDefinition

The class \ccRefName\ implements approximate $k$-nearest and $k$-furthest neighbor searching
using standard search on a  tree using an orthogonal distance
class. 

\ccInclude{CGAL/Orthogonal_standard_search.h}

\ccParameters

Expects for the first template argument an implementation of the concept \ccc{PointTraits},
for example \ccc{CGAL::Kd_tree_traits_2<Kernel>}.

Expects for the second template argument a model of the
concept GeneralDistance. The default type is 
\ccc{CGAL::Euclidean_distance<TreeTraits::Point>}.

Expects for third template argument a model of the concept \ccc{Splitter}.
The default type is \ccc{CGAL::Sliding_midpoint<PointTraits>}.

Expects for fourth template argument an implementation of the concept \ccc{SpatialTree}.
The default type is \ccc{CGAL::Kd_tree<TreeTraits, Splitter, Tag_true>}.  The 
template argument \cccTag_true} makes that the tree is built with extended nodes.

\ccTypes

\ccTypedef{PointTraits::Point Point;}{Point type.}
\ccTypedef{PointTraits::NT NT;}{Number type.}
\ccTypedef{std::pair<Point,NT> Point_with_distance;}{Pair of point and distance.}

\ccTypedef{SpatialTree Tree;}{The tree type.}

\ccCreationVariable{s}  %% choose variable name

\ccOperations

\def\ccLongParamLayout{\ccTrue}
\ccConstructor{Orthogonal_standard_search(SpatialTree tree, Point query, int k=1, NT eps=NT(0.0),
                                          bool search_nearest=true, 
                                          OrthogonalDistance d=OrthogonalDistance());}
{Constructor for searching approximately $k$ neighbors of the query item \ccc{query} 
in the points stored in \ccc{tree} using 
distance \ccc{d} and approximation factor \ccc{eps}.}

\ccMethod{template <class OutputIterator>
OutputIterator  the_k_neighbors(OutputIterator res);}
{Returns a sequence of $k$ elements of type \ccc{Point_with_distance}}.


\begin{ccAdvanced}
\ccMethod{std::ostream& statistics(std::ostream& s);}
{
Inserts statistics of the search process into the output stream \ccc{s}.
}
\end{ccAdvanced}

\ccSeeAlso

\ccc{CGAL::General_standard_search<PointTraits, GeneralDistance, QueryItem, Tree>}.

\end{ccRefClass}

% +------------------------------------------------------------------------+
%%RefPage: end of main body, begin of footer
% EOF
% +------------------------------------------------------------------------+

