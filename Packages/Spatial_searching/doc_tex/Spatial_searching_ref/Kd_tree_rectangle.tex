% +------------------------------------------------------------------------+
% | Reference manual page: Kd_tree_rectangle.tex
% +------------------------------------------------------------------------+
% | 1.07.2001   Johan W.H. Tangelder
% | Package: ASPAS
% | 
\RCSdef{\RCSKdtreerectangleRev}{$Revision$}
\RCSdefDate{\RCSKdtreerectangleDate}{$Date$}
% |
%%RefPage: end of header, begin of main body
% +------------------------------------------------------------------------+


\begin{ccRefClass}{Kd_tree_rectangle<PointTraits>}  %% add template arg's if necessary

\begin{ccAdvanced}

%% \ccHtmlCrossLink{}     %% add further rules for cross referencing links
%% \ccHtmlIndexC[class]{} %% add further index entries

\ccDefinition
  
The class \ccRefName\ implements \ccc{d}-dimensional iso-rectangles and related operations.
E.g. constructors and methods to compute bounding boxes of point sets are provided. 


\ccInclude{CGAL/Kd_tree_rectangle.h}

\ccCreation
\ccCreationVariable{r}  %% choose variable name

\ccTypes

\ccTypedef{PointTraits::NT NT;}{Number type.}


\ccConstructor{Kd_tree_rectangle(int d);}
{Constructs a $d$-dimensional rectangle \ccc{r} with lower bound and upper bound set to zero 
in each dimension.}
 
\ccConstructor{
template <class PointIter>
Kd_tree_rectangle(int d, PointIter begin, PointIter end);}
{Constructs the bounding box of the points in the range [\ccc{begin,end}).}

\ccOperations

\ccMethod{NT min_coord(int i);}
{Returns the lower bound of the rectangle in dimension~\ccc{i}.}

\ccMethod{NT max_coord(int i);}
{Returns the upper bound of the rectangle in dimension~\ccc{i}.}



\ccMethod{void set_upper_bound(int i, const NT& x);}
{Sets upper bound in dimension \ccc{i} to \ccc{x}.}

\ccMethod{void set_lower_bound(int i, const NT& x);}
{Sets lower bound in dimension \ccc{i} to \ccc{x}.}
 
\ccMethod{ template <class PointPointerIter>
void update_from_point_pointers(PointPointerIter begin, PointPointerIter end, bool empty);}
{Updates the rectangle $r$ as the the bounding box of the points accessed by the pointers
 in the range [begin,end). If empty is true the result is an empty bounding box.} 
 
\ccMethod{NT max_span();}
{Returns the maximal span of the rectangle.}

\ccMethod{NT max_span_coord();}
{Returns the smallest coordinate for which the rectangle has its maximal span.}

\ccMethod{int dimension();}
{Returns the dimension of the rectangle.}

\ccMethod{Kd_tree_rectangle<PointTraits>* split(int d, NT value);}
{Splits rectangle in dimension \ccc{d} at coordinate-value \ccc{value} 
 by modifying itself to lower half and returning the upper half.
Note that the \ccc{Kd_tree} datastructure is responsible for the
deletion of the pointers to the instances of \ccc{Kd_tree_rectangle}.} 

\ccHeading{Output Routines}

\ccGlobalFunction{ template<class NT>
std::ostream& operator<<(std::ostream& s, Kd_tree_rectangle<PointTraits>& b);} 
{Prints the rectangle \ccc{b} to the output stream \ccc{s} and returns \ccc{s}.}


\end{ccAdvanced}


\end{ccRefClass}

% +------------------------------------------------------------------------+
%%RefPage: end of main body, begin of footer
% EOF
% +------------------------------------------------------------------------+

