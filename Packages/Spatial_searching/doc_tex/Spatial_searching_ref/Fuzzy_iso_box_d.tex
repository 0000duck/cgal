% +------------------------------------------------------------------------+
% | Reference manual page: Fuzzy_iso_box_d.tex
% +------------------------------------------------------------------------+
% | 1.07.2001   Johan W.H. Tangelder
% | Package: ASPAS
% | 
\RCSdef{\RCSFuzzyisoboxdRev}{$Revision$}
\RCSdefDate{\RCSFuzzyisoboxdDate}{$Date$}
% |
%%RefPage: end of header, begin of main body
% +------------------------------------------------------------------------+


\begin{ccRefClass}{Fuzzy_iso_box_d<SpatialPoint,IsoBox_d>}  %% add template arg's if necessary

%% \ccHtmlCrossLink{}     %% add further rules for cross referencing links
%% \ccHtmlIndexC[class]{} %% add further index entries

\ccDefinition
  
The class \ccRefName\ implements fuzzy $d$-dimensional iso boxes.
A fuzzy iso box with fuziness value $\epsilon$ has
as outer approximation a box  dilated with a $d$-dim square of size $\epsilon$ and
as inner approximation a box eroded by a $d$-dim square of size $\epsilon$. 

\ccInclude{CGAL/Fuzzy_iso_box_d.h}

\ccParameters
Expects for the parameter \ccc{SpatialPoint} an implementation of a $d$-dimensional point,\\
for example \ccc{CGAL::Point_d<R>}.
Expects for the parameter \ccc{IsoBox_d} an implementation of a $d$-dimensional iso box,
for example \ccc{CGAL::Iso_rectangle_2<R>}, \ccc{CGAL::Iso_cuboid_3<R>}, or
\ccc{CGAL::Iso_box_d<R>}.
 
\ccIsModel

FuzzyQueryItem

\ccTypes

\ccTypedef{Kernel_traits<SpatialPoint>::Kernel::FT NT;}{Number type.}

\ccCreation
\ccCreationVariable{b}  %% choose variable name


\ccConstructor{Fuzzy_iso_box_d(SpatialPoint p, SpatialPoint q, NT epsilon=NT(0));}{Constructs a fuzzy iso box
specified by the minimal iso box containing $p$ and $q$ and fuzziness value \ccc{epsilon}.}

\ccOperations

\ccMethod{bool contains(SpatialPoint p);}{test whether $b$ contains $p$.}

\ccMethod{bool inner_range_intersects(Kd_tree_rectangle<NT>* rectangle);}
{test whether the inner box intersects a rectangle
associated with a node of a tree.}

\ccMethod{bool outer_range_is_contained_by(Kd_tree_rectangle<NT>* rectangle);}
{test whether the outer box encloses a rectangle
associated with a node of a tree.}

\ccSeeAlso

FuzzyQueryItem.




\end{ccRefClass}

% +------------------------------------------------------------------------+
%%RefPage: end of main body, begin of footer
% EOF
% +------------------------------------------------------------------------+

