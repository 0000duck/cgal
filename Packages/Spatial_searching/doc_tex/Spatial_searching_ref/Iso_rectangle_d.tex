% +------------------------------------------------------------------------+
% | Reference manual page: Iso_rectangle_d.tex
% +------------------------------------------------------------------------+
% | 1.07.2001   Johan W.H. Tangelder
% | Package: ASPAS
% | 
\RCSdef{\RCSIsorectangledRev}{$Revision$}
\RCSdefDate{\RCSIsorectangledDate}{$Date$}
% |
%%RefPage: end of header, begin of main body
% +------------------------------------------------------------------------+


\begin{ccRefClass}{Iso_rectangle_d<R>}  %% add template arg's if necessary

%% \ccHtmlCrossLink{}     %% add further rules for cross referencing links
%% \ccHtmlIndexC[class]{} %% add further index entries

\ccDefinition
  
The class \ccRefName\ implements \ccc{d}-dimensional iso-rectangles.

\ccInclude{CGAL/Iso_rectangle_d.h}

\ccIsModel

Implements the concept query item.


\ccTypes

\ccTypedef{Item::R::FT NT;}{Number type.}

\ccCreation
\ccCreationVariable{r}  %% choose variable name

\ccConstructor{Iso_rectangle_d();}{default constructor.}

\ccConstructor{Iso_rectangle_d(Point_d<R> p, Point_d<R> q);}
{constructs an iso-oriented rectangle \ccc{d} with diagonal opposite vertices \ccc{p} and \ccc{q}}. 

\ccOperations

\ccMethod{NT min_coord(int i);}
{Returns the lower bound of the rectangle in dimension \ccc{i}.}

\ccMethod{NT max_coord(int i);}
{Returns the upper bound of the rectangle in dimension \ccc{i}.}


\end{ccRefClass}

% +------------------------------------------------------------------------+
%%RefPage: end of main body, begin of footer
% EOF
% +------------------------------------------------------------------------+

