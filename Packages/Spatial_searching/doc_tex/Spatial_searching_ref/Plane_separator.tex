% +------------------------------------------------------------------------+
% | Reference manual page: Plane_separator.tex
% +------------------------------------------------------------------------+
% | 1.07.2001   Johan W.H. Tangelder
% | Package: ASPAS
% | 
\RCSdef{\RCSPlaneseparatorRev}{$Revision$}
\RCSdefDate{\RCSPlaneseparatorDate}{$Date$}
% |
%%RefPage: end of header, begin of main body
% +------------------------------------------------------------------------+


\begin{ccRefClass}{Plane_separator<NT>}  %% add template arg's if necessary

%% \ccHtmlCrossLink{}     %% add further rules for cross referencing links
%% \ccHtmlIndexC[class]{} %% add further index entries

\ccDefinition
  
The class \ccRefName\ implements a plane separator, i.e., a hyperplane that is used to separate two half spaces.
This hyperplane is defined by a cutting dimension $d$ and a cutting value $v$
as $x_d=v$, where $v$ denotes the $d^{th}$ coordinate value.

\ccInclude{CGAL/Plane_separator.h}

\ccIsModel

SpatialSeparator

\ccCreation
\ccCreationVariable{s}  %% choose variable name



\ccConstructor{Plane_separator(int d, NT v);}
{Constructs a separator that separates two half spaces by a hyperplane
defined by $x_d=v$, where $v$ denotes the $d^{th}$ coordinate value.}

\ccConstructor{Plane_separator(Plane_separator<NT> p);}
{Copy constructor.}

\ccOperations

\ccMethod{
void set_cutting_dimension(int d);}
{Sets the cutting dimension to \ccc{d}. }

\ccMethod{
void set_cutting_value(NT v);}
{Sets the cutting value to \ccc{v}. }

\ccMethod{int cutting_dimension();}{Returns the number of the cutting dimension.}

\ccMethod{NT cutting_value();}{Returns the cutting value.}

\ccMethod{template <class SpatialPoint>
bool has_on_negative_side(SpatialPoint p);}
{Returns true if and only if the coordinate of \ccc{p} in the cutting dimension is smaller than the cutting value.}

\ccMethod{ 
Plane_separator<NT> operator=(Plane_separator<NT> p);}
{Assignment operator.}

\begin{ccAdvanced}

{\bf Non-Member functions related to \ccc{Plane_separator<NT>}}

\ccGlueBegin
\ccGlobalFunction{ template<class NT>
std::ostream& operator<<(std::ostream& s, Plane_separator<NT> p);}
{Prints the plane separator \ccc{p} to the output stream \ccc{s} and returns \ccc{s}.}
\ccGlueEnd

\end{ccAdvanced}

\end{ccRefClass}

% +------------------------------------------------------------------------+
%%RefPage: end of main body, begin of footer
% EOF
% +------------------------------------------------------------------------+

