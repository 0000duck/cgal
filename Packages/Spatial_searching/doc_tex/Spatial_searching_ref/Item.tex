% +------------------------------------------------------------------------+
% | Reference manual page: Item.tex
% +------------------------------------------------------------------------+
% | 1.07.2001   Johan W.H. Tangelder
% | Package: ASPAS
% | 
\RCSdef{\RCSItemRev}{$Revision$}
\RCSdefDate{\RCSItemDate}{$Date$}
% |
%%RefPage: end of header, begin of main body
% +------------------------------------------------------------------------+


\begin{ccRefConcept}{Item<R>}

%% \ccHtmlCrossLink{}     %% add further rules for cross referencing links
%% \ccHtmlIndexC[concept]{} %% add further index entries

\ccDefinition
  
An object of the class \ccClassTemplateName\ is a point in $d$-dimensional
Euclidean space $E_d$, where $d$ is arbitrary. 

\ccTypes

\ccTypedef{Item::R::FT NT;}{Number type.}

\ccCreation
\ccCreationVariable{p}  %% choose variable name

\ccConstructor{Item();}{default constructor.}




\ccOperations



\ccMemberFunction{ NT operator [] (int i) const;}
        {returns the i'th Cartesian coordinate of \ccVar, starting
        with 0. \ccPrecond $0\leq i < \ccc{d}$.}

\ccMemberFunction{ int dimension () const;}
        {returns the dimension \ccc{d} of \ccVar.}


\ccHasModels

\ccc{Point_2, Point_3, Point_d}

\ccSeeAlso

GeneralDistance, OrthogonalDistance.

\end{ccRefConcept}

% +------------------------------------------------------------------------+
%%RefPage: end of main body, begin of footer
% EOF
% +------------------------------------------------------------------------+

