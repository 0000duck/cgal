% +------------------------------------------------------------------------+
% | Reference manual page: Fuzzy_sphere.tex
% +------------------------------------------------------------------------+
% | 1.07.2001   Johan W.H. Tangelder
% | Package: ASPAS
% | 
\RCSdef{\RCSFuzzyspheredRev}{$Revision$}
\RCSdefDate{\RCSFuzzyspheredDate}{$Date$}
% |
%%RefPage: end of header, begin of main body
% +------------------------------------------------------------------------+


\begin{ccRefClass}{Fuzzy_sphere<Traits>}  %% add template arg's if necessary

%% \ccHtmlCrossLink{}     %% add further rules for cross referencing links
%% \ccHtmlIndexC[class]{} %% add further index entries

\ccDefinition
  
The class \ccRefName\ implements fuzzy $d$-dimensional spheres.
A fuzzy sphere with radius $r$ and fuzziness value $\epsilon$ has
as outer approximation a sphere with radius $r+\epsilon$ and
as inner approximation a sphere with radius $r-\epsilon$. 

\ccInclude{CGAL/Fuzzy_sphere.h}

\ccParameters


Expects for the template argument a model of the concept
\ccc{SearchTraits}, for example \ccc{CGAL::Cartesian_d<double>}.

\ccIsModel

\ccc{FuzzyQueryItem}

\ccTypes
\ccTypedef{Traits::Point_d Point_d;}{Point type.}
\ccTypedef{Point_d Query_point_d;}{Query point type.}
\ccTypedef{Traits::FT FT;}{Number type.}

\ccCreation
\ccCreationVariable{s}  %% choose variable name



\ccConstructor{Fuzzy_sphere(Query_point_d center, FT radius, FT epsilon=FT(0));}{Constructs a fuzzy sphere
centered at \ccc{center} with radius \ccc{radius} and fuzziness value \ccc{epsilon}.}

\ccOperations

\ccMethod{bool contains(const Point_d& p); const}{test whether $s$ contains $p$.}

\ccMethod{bool inner_range_intersects(const Kd_tree_rectangle<FT>& rectangle); const}
{test whether the inner sphere intersects a rectangle
associated with a node of a tree.}

\ccMethod{bool outer_range_is_contained_by(const Kd_tree_rectangle<FT>& rectangle); const}
{test whether the outer sphere is enclosed by a rectangle
associated with a node of a tree.}

\ccSeeAlso

\ccc{FuzzyQueryItem}

\end{ccRefClass}

% +------------------------------------------------------------------------+
%%RefPage: end of main body, begin of footer
% EOF
% +------------------------------------------------------------------------+

