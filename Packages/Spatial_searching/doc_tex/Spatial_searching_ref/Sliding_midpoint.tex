% +------------------------------------------------------------------------+
% | Reference manual page: Sliding_midpoint.tex
% +------------------------------------------------------------------------+
% | 1.07.2001   Johan W.H. Tangelder
% | Package: ASPAS
% | 
\RCSdef{\RCSSlidingmidpointRev}{$Revision$}
\RCSdefDate{\RCSSlidingmidpointDate}{$Date$}
% |
%%RefPage: end of header, begin of main body
% +------------------------------------------------------------------------+


\begin{ccRefFunctionObjectClass}{Sliding_midpoint<SpatialPoint, PointContainer, Separator>}  %% add template arg's if necessary

%% \ccHtmlCrossLink{}     %% add further rules for cross referencing links
%% \ccHtmlIndexC[class]{} %% add further index entries

\ccDefinition
Implements the {\bf sliding midpoint} splitting rule.


\ccParameters

Expects for the first template argument an implementation for
$d$-dimensional points of
the concept \ccc{SpatialPoint}, 
for example \ccc{CGAL::Point_d<Kernel>}. Expects for the second argument an implementation
for a point container of the concept \ccc{PointContainer}, for example \ccc{CGAL::Point_container<SpatialPoint>}.
Expects for the third template argument an implementation of a $d$-1 dimensional 
separating plane of the concept \ccc{Separator}, for example \ccc{CGAL::Plane_separator<NT>}.
The second template argument has default value \ccc{CGAL::Point_container<SpatialPoint>}.
The third template argument has default value \ccc{CGAL::Plane_separator<Kernel_traits<SpatialPoint>::Kernel::FT>}. 

\ccInclude{CGAL/Splitters.h}

\ccIsModel

Splitter

\ccTypes

\ccTypedef{Kernel_traits<SpatialPoint>::Kernel::FT NT;}{Number type.} 
\ccTypedef{PointContainer Container;}{Point container.}
\ccTypedef{Separator Separator;}{\ccHtmlNoLinksFrom{Separator}.}

\ccCreationVariable{s}  %% choose variable name

\ccOperations

\ccMethod{
void operator()(\ccHtmlNoLinksFrom{Separator}& s, Container& c0, Container& c1, NT Aspect_ratio=NT(3));} 
{Modifies the separator $s$ by the {\bf sliding midpoint} splitting rule, 
and splits \ccc{c0} into \ccc{c0} and \ccc{c1},
using the splitting dimension and the splitting value of the modified separator.
}

\ccSeeAlso

Splitter, PointContainer, Separator.
\end{ccRefFunctionObjectClass}

% +------------------------------------------------------------------------+
%%RefPage: end of main body, begin of footer
% EOF
% +------------------------------------------------------------------------+

