% +------------------------------------------------------------------------+
% | Reference manual page: Kd_tree_traits_point.tex
% +------------------------------------------------------------------------+
% | 1.07.2001   Johan W.H. Tangelder
% | Package: ASPAS
% | 
\RCSdef{\RCSKdtreetraitspointRev}{$Revision$}
\RCSdefDate{\RCSKdtreetraitspointDate}{$Date$}
% |
%%RefPage: end of header, begin of main body
% +------------------------------------------------------------------------+


\begin{ccRefClass}{Kd_tree_traits_point<NT_, Point_, CartesianIterator,ConstructCartesianIterator>}

%% \ccHtmlCrossLink{}     %% add further rules for cross referencing links
%% \ccHtmlIndexC[class]{} %% add further index entries

\ccDefinition
  
The class \ccRefName\ is a wrapper for the geometric types needed by a kd tree.


\ccInclude{CGAL/Kd_tree_traits_point.h}

\ccIsModel

PointTraits.


\ccTypes
\ccTypedef {NT_ NT;}{The number type of the coordinates.}
\ccTypedef {Point_ Point;}{Point type.}
\ccTypedef {CartesianIterator Cartesian_const_iterator;}{An iterator over the coordinates.}
\ccTypedef {ConstructCartesianIterator Construct_Cartesian_const_iterator;}{A functor with
two function operators, which return the begin and past the end iterator for the Cartesian coordinates. 
The functor for begin has as argument a \ccc{Point}. The functor for the past the end iterator, 
has as argument a \ccc{Point} and an \ccc{int}.}


\ccSeeAlso

\ccc{Kd_tree_traits_point_2<Kernel>}\\
\ccc{Kd_tree_traits_point_3<Kernel>}\\
\ccc{Kd_tree_traits_point_d<Kernel>}

\end{ccRefClass}


% +------------------------------------------------------------------------+
%%RefPage: end of main body, begin of footer
% EOF
% +------------------------------------------------------------------------+

