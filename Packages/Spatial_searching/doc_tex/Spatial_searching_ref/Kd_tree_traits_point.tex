% +------------------------------------------------------------------------+
% | Reference manual page: Kd_tree_traits_point.tex
% +------------------------------------------------------------------------+
% | 1.07.2001   Johan W.H. Tangelder
% | Package: ASPAS
% | 
\RCSdef{\RCSKdtreetraitspointRev}{$Revision$}
\RCSdefDate{\RCSKdtreetraitspointDate}{$Date$}
% |
%%RefPage: end of header, begin of main body
% +------------------------------------------------------------------------+


\begin{ccRefClass}{Kd_tree_traits_point<SpatialPoint,Splitter>}  %% add template arg's if necessary

%% \ccHtmlCrossLink{}     %% add further rules for cross referencing links
%% \ccHtmlIndexC[class]{} %% add further index entries

\ccDefinition
  
The class \ccRefName\ defines a model of tree traits class, which is used as
a template class for the construction of a $k$-$d$ tree by the class \ccc{Kd_tree}.
The parameters \ccc{selected_split_rule}, \ccc{aspect_ratio}, \ccc{bucket_size}, and \ccc{use_extended_nodes}  
define the way in which $k$-$d$ tree is constructed.


\ccInclude{CGAL/Kd_tree_traits_point.h}

\ccIsModel

TreeTraits.

\ccParameters

Expects for the parameter \ccc{SpatialPoint} an implementation of a $d$-dimensional point, \\
for example \ccc{CGAL::Point_d<R>}.
Expects for the parameter \ccc{Splitter} an implementation of a splitter, for
example \ccc{CGAL::Sliding_midpoint<SpatialPoint,PointContainer,Separator>}.
Splitter has default argument \ccc{CGAL::Sliding_midpoint<SpatialPoint>}.

\ccTypes

\ccTypedef {SpatialPoint Point;}{Point type.}
\ccTypedef {Point** Point_Iterator;}{Point iterator.}
\ccTypedef{Kernel_traits<Point>::Kernel::FT NT;}{Number type.}
\ccTypedef {Splitter::Container Container;}{Point container.}
\ccTypedef {Splitter::Separator Separator;}{Separator.}

\ccCreation
\ccCreationVariable{tr}  %% choose variable name



\ccConstructor{Kd_tree_traits_point
(int bucket_size=5, NT aspect_ratio=NT(3), bool use_extended_nodes=true);}
{Constructor assigning values to the parameters.}

\ccOperations

\ccMethod{NT aspect_ratio();}{Returns the maximal ratio between the largest and smallest side
of a cell allowed for fair splitting.}

\ccMethod{int bucket_size();}{Returns the selected bucket size of the leaf nodes.}

\ccMethod{bool use_extended_nodes();}{Indicates whether extened internal nodes
should be used.}

\ccMethod{void split(Separator& sep, Container<Point>& c0, Container<Point>& c1);}
{Splits \ccc{c0} into \ccc{c0} and \ccc{c1} and returns the separator.}


\ccSeeAlso

TreeTraits,
\ccc{CGAL::Kd_tree_traits_point<SpatialPoint,Splitter>}.

\end{ccRefClass}


% +------------------------------------------------------------------------+
%%RefPage: end of main body, begin of footer
% EOF
% +------------------------------------------------------------------------+

