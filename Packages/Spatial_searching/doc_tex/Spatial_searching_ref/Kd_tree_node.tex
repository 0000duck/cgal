% +------------------------------------------------------------------------+
% | Reference manual page: Kd_tree_node.tex
% +------------------------------------------------------------------------+
% | 1.07.2001   Johan W.H. Tangelder
% | Package: ASPAS
% | 
\RCSdef{\RCSKdtreenodeRev}{$Revision$}
\RCSdefDate{\RCSKdtreenodeDate}{$Date$}
% |
%%RefPage: end of header, begin of main body
% +------------------------------------------------------------------------+


\begin{ccRefClass}{Kd_tree_node<TreeTraits>}  %% add template arg's if necessary

%% \ccHtmlCrossLink{}     %% add further rules for cross referencing links
%% \ccHtmlIndexC[class]{} %% add further index entries

\ccDefinition
  
The class \ccRefName\ implements a node class for a $k$-$d$ tree. 
A node is either a leaf node, an internal node or an extended internal node.
A leaf node contains one or more points. An internal node contains a pointer
to its lower child and a pointer its upper child and a pointer to its separator.
An extended internal node is an internal node containing the lower and 
upper limit of an exteneded node's rectangle
along the node's cutting dimension.

\ccInclude{CGAL/Kd_tree_node.h}

\ccHeading{Parameters}

Expects for the parameter \ccc{TreeTraits} an implementation
of the \ccc{Treetraits}
concept.

\ccTypes

\ccEnum{ enum Node_type { LEAF, INTERNAL, EXTENDED_INTERNAL};}{Denotes type of node.}

\ccTypedef{Kd_tree<TreeTraits>::Node_handle Node_handle;}{Node handle.}
\ccTypedef{TreeTraits::Point Point;}{Point provided by the Traits class.}
\ccTypedef{TreeTraits::Point_iterator Point_iterator;}{Iterator over points.} 
\ccTypedef{TreeTraits::NT NT;}{Number type.}
\ccTypedef{TreeTraits::Separator Separator;}{Separator type.}

\ccCreation
\ccCreationVariable{n}  %% choose variable name

\ccConstructor{Kd_tree_node(Point_container<Point> c);}
{
Creates a leaf node \ccc{n} that stores the points from $c$.
}

\ccConstructor{Kd_tree_node(Point_container<Point> c, Traits t, bool use_extension);}
{
Creates an internal or extended internal node \ccc{n}, which is the root of a subtree 
storing all the points 
from \ccc{c}. One subset is stored at the lower child and the other subset at the upper child.
For each of the childern either a leaf node or an internal node is created as follows.
If the number of points in the subset does not exceed the \ccc{t.bucket_size()} a leaf node is
created that stores the points. Otherwise an internal node is created, which is the root of
a subtree storing the points from the subset. If and only if \ccc{use_extension} 
is true an extended internal node is created. An extended internal node stores 
the lower and upper limit of an extended node's rectangle. 
}

\ccOperations

\ccMethod{
template <class OutputIterator, class FuzzyQueryItem>
OutputIterator search(OutputIterator it, FuzzyQueryItem q);}
{Reports the points from the subtree of the node, that are approximately contained by q.} 

\ccMethod{
template <class OutputIterator>
OutputIterator tree_points(OutputIterator it);}
{Reports all the points contained by the subtree of the node.} 


\ccMethod{bool is_leaf();}{Indicates whether a node is a leaf node.}
\ccMethod{int  size();}{Returns the number of items stored in a leaf node.}
\ccMethod{Item_iterator begin();}{Returns the iterator to the first item in a leaf node.}
\ccMethod{Item_iterator end();}{Returns the iterator to the correspondig past-the-end
iterator in a leaf node.}
\ccMethod{Node* lower();}{Returns a pointer to the lower child of an internal node.}
\ccMethod{Node* upper();}{Returns a pointer to the upper child of an internal node.}
\ccMethod{Separator* separator();}{Returns a pointer to the separator.}
\ccMethod{NT low_val();}{Returns the lower limit of an exteneded node's rectangle
along the node's cutting dimension.}
\ccMethod{NT high_val();}{Returns the upper limit of an extended node's rectangle
along the node's cutting dimension.}

\end{ccRefClass}

% +------------------------------------------------------------------------+
%%RefPage: end of main body, begin of footer
% EOF
% +------------------------------------------------------------------------+

