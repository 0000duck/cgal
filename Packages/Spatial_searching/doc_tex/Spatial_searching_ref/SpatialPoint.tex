% +------------------------------------------------------------------------+
% | Reference manual page: SpatialPoint.tex
% +------------------------------------------------------------------------+
% | 1.07.2001   Johan W.H. Tangelder
% | Package: ASPAS
% | 
\RCSdef{\RCSSpatialPointRev}{$Revision$}
\RCSdefDate{\RCSSpatialPointDate}{$Date$}
% |
%%RefPage: end of header, begin of main body
% +------------------------------------------------------------------------+


\begin{ccRefConcept}{PointTraits}

%% \ccHtmlCrossLink{}     %% add further rules for cross referencing links
%% \ccHtmlIndexC[concept]{} %% add further index entries

\ccDefinition
  
The concept \ccClassTemplateName\ defines the requirements for a 
$d$-dimensional point type.
For a user defined $d$-dimensional point type a nested representation class
\ccc{SpatialPoint::R} has to be provided or \ccc{CGAL::Kernel_traits} has to be
specialized for the point type. 

\ccTypes

\ccNestedType{Point}{Point type.}
\ccNestedType{NT}{The number type of the \ccHtmlNoLinksFrom{Cartesian} coordinates of type \ccc{Point}.}
\ccNestedType{Cartesian_const_iterator}{An random access iterator type to enumerate the 
\ccHtmlNoLinksFrom{Cartesian} coordinates of a point.}
\ccNestedType{Construct_cartesian_const_iterator}{Functor with operators to construct iterators on the
first and the past-the-end iterator for the\ccHtmlNoLinksFrom{Cartesian} coordinates of a point.}
\ccCreationVariable{p}  %% choose variable name



\ccHasModels

\ccc{CGAL::Kd_tree_traits_point_2<Kernel>}\\
\ccc{CGAL::Kd_tree_traits_3<Kernel>, \\
\ccc{CGAL::Kd_tree_traits_d<Kernel>,\\
\ccc{CGAL::Kd_tree_traits_point<NT,Point,CartesianCoordinateIterator,ConstructCartesianCoordinateIterator>}.


\end{ccRefConcept}

% +------------------------------------------------------------------------+
%%RefPage: end of main body, begin of footer
% EOF
% +------------------------------------------------------------------------+

