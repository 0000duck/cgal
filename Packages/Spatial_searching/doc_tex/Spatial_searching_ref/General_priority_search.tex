% +------------------------------------------------------------------------+
% | Reference manual page: General_priority_search.tex
% +------------------------------------------------------------------------+
% | 1.07.2001   Johan W.H. Tangelder
% | Package: ASPAS
% |
\RCSdef{\RCSGeneralprioritysearchRev}{$Revision$}
\RCSdefDate{\RCSGeneralprioritysearchDate}{$Date$}
% |
%%RefPage: end of header, begin of main body
% +------------------------------------------------------------------------+


\begin{ccRefClass}{General_priority_search<TreeTraits, GeneralDistance, QueryItem, SpatialTree>}  %% add template arg's if necessary

%% \ccHtmlCrossLink{}     %% add further rules for cross referencing links
%% \ccHtmlIndexC[class]{} %% add further index entries

\ccDefinition

The class \ccRefName\ implements incremental nearest and furthest neighbor searching
using priority search on a tree. It is not required that the tree is
built with extended nodes.

\ccInclude{CGAL/General_priority_search.h}

\ccParameters

Expects for the first template argument an implementation of the concept \ccc{TreeTraits},
for example \ccc{CGAL::Kd_tree_traits<Point, Splitter>}.
Expects for the second template argument an implementation of the
concept GeneralDistance, for example \ccc{CGAL::Euclidean_distance<TreeTraits::Point>}.
The second template argument has default argument 
\ccc{CGAL::Euclidean_distance<TreeTraits::Point>}.
Expects for the third template argument an implementation of the concept
\ccc{QueryItem}
by a $d$-dimensional point, for example \ccc{CGAL::Point_d<Kernel>}
or by a spatial object, for
example \ccc{CGAL::Iso_box_d<Kernel>} implementing iso-rectangles.
The third template argument has default value \ccc{TreeTraits::Point}.
Expects for fourth template argument an implementation of the concept SpatialTree,
for example \ccc{CGAL::Kd_tree<TreeTraits>},
The fourth template argument has default value \ccc{CGAL::Kd_tree<TreeTraits>}.


\ccTypes

\ccTypedef{TreeTraits::Point Point;}{Point type.}
\ccTypedef{TreeTraits::NT NT;}{Number type.}
\ccTypedef{std::pair<Point*,NT> Point_with_distance;}{Pair of point and distance.}
\ccNestedType{iterator}{Input iterator for searching approximate neighbors.}

\newpage

\ccCreation
\ccCreationVariable{s}  %% choose variable name


\def\ccLongParamLayout{\ccTrue}
\ccConstructor{General_priority_search(Tree tree, QueryItem q, GeneralDistance d=GeneralDistance(), NT Eps=NT(0.0),
bool Search_nearest=true);}
{Constructor for incremental neighbor searching of the query item \ccc{q}
in the points stored \ccc{tree} using a distance
traits class $d$ and approximation factor \ccc{Eps}.}

\ccOperations

\ccMethod{iterator begin();}{Returns an iterator to the approximate neighbor.}

\ccMethod{iterator end();}{Past-the-end iterator. Denotes that all points tree
have been processed.}


\begin{ccAdvanced}
\ccMethod{std::ostream& statistics(std::ostream& s);}
{
Inserts statistics of the search process into the output stream \ccc{s}.
}
\end{ccAdvanced}


\ccSeeAlso

\ccc{CGAL::Orthogonal_priority_search<TreeTraits, OrthogonalDistance, SpatialTree>}.


\end{ccRefClass}

% +------------------------------------------------------------------------+
%%RefPage: end of main body, begin of footer
% EOF
% +------------------------------------------------------------------------+

