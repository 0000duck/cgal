% +------------------------------------------------------------------------+
% | Reference manual page: General_priority_search.tex
% +------------------------------------------------------------------------+
% | 1.07.2001   Johan W.H. Tangelder
% | Package: ASPAS
% |
\RCSdef{\RCSGeneralprioritysearchRev}{$Revision$}
\RCSdefDate{\RCSGeneralprioritysearchDate}{$Date$}
% |
%%RefPage: end of header, begin of main body
% +------------------------------------------------------------------------+


\begin{ccRefClass}{General_priority_search<Traits,QueryItem,Distance>}  %% add template arg's if necessary

%% \ccHtmlCrossLink{}     %% add further rules for cross referencing links
%% \ccHtmlIndexC[class]{} %% add further index entries

\ccDefinition

The class \ccRefName\ implements approximate $k$-nearest and $k$-furthest neighbour searching
using priority search on a $k$-$d$ tree. It is not required that the $k$-$d$ tree is
built with extended nodes.


\ccInclude{CGAL/General_priority_search.h}

\ccParameters

Expects for the parameter \ccc{Traits} an implementation of the concept TreeTraits.
Expects for the parameter \ccc{QueryItem}  an implementation  
of a $d$-dimensional point, for example \ccc{CGAL::Point_d<R>}
or an implementation of a spatial object, for
example \ccc{CGAL::Iso_rectangle_d<R>} implementing iso-rectangles.
Expects for the parameter \ccc{Distance} an implementation of the
concept GeneralDistance.

\ccTypes

\ccTypedef{Traits::Item Item;}{Item type.}
\ccTypedef{Traits::NT NT;}{Number type.}
\ccTypedef{std::pair<Item*,NT> Item_with_distance;}{Pair of item and distance}
\ccTypedef{Kd_tree<Traits> Tree;}{$k$-$d$ tree storing the items.}

\ccNestedType{iterator}{Input iterator.}

\ccCreation
\ccCreationVariable{s}  %% choose variable name

\ccConstructor{General_priority_search(Tree& tree, QueryItem& q, Distance& d, int k, NT Eps,
bool Search_nearest=true);}
{Constructor for searching approximately $k$ neighbours of the query item \ccc{q}
in the items stored in the $k$-$d$ tree \ccc{tree} using a distance
traits class.}

\ccHeading{Member Functions}

\ccMethod{iterator begin();}{Returns an iterator to the approximate neighbour.}

\ccMethod{iterator end();}{Past-the-end iterator. Denotes that all items in the search tree
have been processed.}

\ccMethod{ostream& statistics();}
{
Inserts statistics of the search process into an output stream.
}


\ccSeeAlso

\ccc{General_priority_search::iterator, Orthogonal_priority_search}.


\end{ccRefClass}

% +------------------------------------------------------------------------+
%%RefPage: end of main body, begin of footer
% EOF
% +------------------------------------------------------------------------+

