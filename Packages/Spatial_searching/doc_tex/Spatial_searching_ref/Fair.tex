% +------------------------------------------------------------------------+
% | Reference manual page: Fair.tex
% +------------------------------------------------------------------------+
% | 1.07.2001   Johan W.H. Tangelder
% | Package: ASPAS
% | 
\RCSdef{\RCSFairRev}{$Revision$}
\RCSdefDate{\RCSFairDate}{$Date$}
% |
%%RefPage: end of header, begin of main body
% +------------------------------------------------------------------------+


\begin{ccRefFunctionObjectClass}{Fair<SpatialPoint,Container,Separator>}  %% add template arg's if necessary

%% \ccHtmlCrossLink{}     %% add further rules for cross referencing links
%% \ccHtmlIndexC[class]{} %% add further index entries

\ccDefinition
Implements the {\bf fair} splitting rule.

\ccParameters

Expects for the parameter \ccc{SpatialPoint} an implementation of a $d$-dimensional point,
for example \ccc{CGAL::Point_d<R>}. Expects for the parameter \ccc{Container} an implementation
of a point container for example \ccc{CGAL::Point_container<SpatialPoint>}.
Expects for the parameter \ccc{Separator} an implementation of a $d$-1 dimensional 
separating plane for example \ccc{CGAL::Plane_separator<NT>}.

\ccc{Container} has default argument \ccc{CGAL::Point_container<SpatialPoint>}.
\ccc{Separator} has default argument \ccc{CGAL::Plane_separator<Kernel_traits<SpatialPoint>::Kernel::FT>}. 

\ccInclude{CGAL/Splitters.h}

\ccIsModel

Splitter

\ccTypes

\ccTypedef{Kernel_traits<SpatialPoint>::Kernel::FT NT;}{Number type.} 
\ccTypedef{Container Container;}{Point container.}
\ccTypedef{Separator Separator;}{Separator.}

\ccCreationVariable{s}  %% choose variable name

\ccOperations

\ccMethod{
void operator()(Separator& s, Container& c0, Container& c1, NT Aspect_ratio=NT(3));} 
{Modifies the separator $s$ by the fair splitting rule, 
and splits \ccc{c0} into \ccc{c0} and \ccc{c1},
using the splitting dimension and the splitting value of the modified separator.
\ccc{Aspect_ratio} is a parameter for the {\bf fair} splitting rule.
}

\ccSeeAlso

Splitter, Container, Separator.

\end{ccRefFunctionObjectClass}

% +------------------------------------------------------------------------+
%%RefPage: end of main body, begin of footer
% EOF
% +------------------------------------------------------------------------+

