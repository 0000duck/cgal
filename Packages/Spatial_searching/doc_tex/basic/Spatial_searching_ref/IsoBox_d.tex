% +------------------------------------------------------------------------+
% | Reference manual page: IsoBox_d.tex
% +------------------------------------------------------------------------+
% | 1.07.2001   Johan W.H. Tangelder
% | Package: ASPAS
% | 
\RCSdef{\RCSIsoBoxdRev}{$Revision$}
\RCSdefDate{\RCSIsoBoxdDate}{$Date$}
% |
%%RefPage: end of header, begin of main body
% +------------------------------------------------------------------------+


\begin{ccRefConcept}{IsoBox_d}

%% \ccHtmlCrossLink{}     %% add further rules for cross referencing links
%% \ccHtmlIndexC[concept]{} %% add further index entries

\ccDefinition
  
The concept \ccRefName\ provides a model for $d$-dimensional iso-boxes.

\ccParameters

\ccc{Point} represents a $d$-dimensional point.

\ccTypes

\ccTypedef{Kernel_traits<Point>::Kernel::FT NT;}{Number type.} 

\ccCreation
\ccCreationVariable{b}  %% choose variable name

\ccConstructor{IsoBox_d(Point p, Point q);}{introduces an iso-oriented box with diagonal opposite points 
\ccc{p} and \ccc{}q.}

\ccOperations

\ccMethod{NT min_coord(int i);}
{Returns the lower bound of the iso-oriented box in dimension \ccc{i}.
}

\ccMethod{NT max_coord(int i);}
{Returns the upper bound of the iso-oriented box in dimension \ccc{i}.}


\ccHasModels

\ccc{CGAL::Iso_rectangle_2<R>}, \ccc{CGAL::Iso_cuboid_3<R>}, 
\ccc{CGAL::Iso_box_d<R>}.

\ccSeeAlso


\end{ccRefConcept}

% +------------------------------------------------------------------------+
%%RefPage: end of main body, begin of footer
% EOF
% +------------------------------------------------------------------------+

