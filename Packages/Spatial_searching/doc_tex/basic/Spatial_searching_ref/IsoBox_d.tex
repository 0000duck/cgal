% +------------------------------------------------------------------------+
% | Reference manual page: IsoBox_d.tex
% +------------------------------------------------------------------------+
% | 1.07.2001   Johan W.H. Tangelder
% | Package: ASPAS
% | 
\RCSdef{\RCSIsoBoxdRev}{$Revision$}
\RCSdefDate{\RCSIsoBoxdDate}{$Date$}
% |
%%RefPage: end of header, begin of main body
% +------------------------------------------------------------------------+


\begin{ccRefConcept}{IsoBox_d}

%% \ccHtmlCrossLink{}     %% add further rules for cross referencing links
%% \ccHtmlIndexC[concept]{} %% add further index entries

\ccDefinition
  
The concept \ccRefName\ defines the requirements for $d$-dimensional iso-boxes.

{\bf Refines}

QueryItem

\ccParameters

\ccc{SpatialPoint} represents a $d$-dimensional point.

\ccCreationVariable{b}  %% choose variable name


\ccOperations

\ccMethod{SpatialPoint min();}
{Returns the smallest vertex of the iso-oriented box.}


\ccMethod{SpatialPoint max();}
{Returns the largest vertex of the iso-oriented box.}


\ccHasModels

\ccc{CGAL::Iso_rectangle_2<Kernel>}, \ccc{CGAL::Iso_cuboid_3<Kernel>}, 
\ccc{CGAL::Iso_box_d<Kernel>}.

\ccSeeAlso

QueryItem, \ccc{CGAL::Iso_rectangle_2<Kernel>}, \ccc{CGAL::Iso_cuboid_3<Kernel>}, 
\ccc{CGAL::Iso_box_d<Kernel>}.


\end{ccRefConcept}

% +------------------------------------------------------------------------+
%%RefPage: end of main body, begin of footer
% EOF
% +------------------------------------------------------------------------+

