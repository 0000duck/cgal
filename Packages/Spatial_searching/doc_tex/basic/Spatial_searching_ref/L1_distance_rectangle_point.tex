% +------------------------------------------------------------------------+
% | Reference manual page: L1_distance_rectangle_point.tex
% +------------------------------------------------------------------------+
% | 1.07.2001   Johan W.H. Tangelder
% | Package: ASPAS
% |
\RCSdef{\RCSLdistancerectanglepointRev}{$Revision$}
\RCSdefDate{\RCSLdistancerectanglepointDate}{$Date$}
% |
%%RefPage: end of header, begin of main body
% +------------------------------------------------------------------------+


\begin{ccRefClass}{L1_distance_rectangle_point<Query_item,Item>}  %% add template arg's if necessary

%% \ccHtmlCrossLink{}     %% add further rules for cross referencing links
%% \ccHtmlIndexC[class]{} %% add further index entries

\ccDefinition

The class \ccRefName\ provides an implementation of the GeneralDistance concept
for the L1 distance between two iso-rectangles and the L1 distance between an iso-rectangle
and a point.

\ccInclude{CGAL/L1_distance_rectangle_point.h}

\ccParameters

Expects for the parameter \ccc{Query_item}
\ccc{item} an implementation of iso-rectangles
for example \ccc{CGAL::Kd_tree_rectangle<NT>}.
Expects for the parameter \ccc{Item} an implementation of the
\ccc{Point} concept, for example \ccc{CGAL::Point_d<NT>}.

\ccIsModel

OrthogonalDistance

\ccTypes

\ccTypedef{Item::R::FT NT;}{Number type.}

\ccCreation
\ccCreationVariable{d}  %% choose variable name

\ccConstructor{L1_distance_rectangle_point();}{Default constructor implementing $l_1$ metric for
two-dimensional iso-rectangles.}

\ccConstructor{L1_distance_rectangle_point(int d);}
{Constructor implementing $l_1$ metric for
$d$-dimensional iso-rectangles.}

\ccHeading{Member Functions}

\ccMethod{NT transformed_distance(NT d);} {Returns $d$.}

\ccMethod{NT inverse_of_transformed_distance(NT d);} {Returns $d$.}

\ccMethod{NT distance(Query_Item& q, Item&p);}{Returns the distance between \ccc{q} and \ccc{p}.}

\ccMethod{NT min_distance_to_rectangle(Query_item& q, Kd_tree_rectangle<NT>& r;);}
{Returns the minimal distance between a point from \ccc{q} and a point from
\ccc{r}}

\ccMethod{NT max_distance_to_rectangle(Query_item& q, Kd_tree_rectangle<NT>& r;);}
{Returns the maximal distance between the iso-rectangle \ccc{q} and
a point from \ccc{r} furthest to \ccc{q}.}


\ccSeeAlso

GeneralDistance.

\end{ccRefClass}

% +------------------------------------------------------------------------+
%%RefPage: end of main body, begin of footer
% EOF
% +------------------------------------------------------------------------+

