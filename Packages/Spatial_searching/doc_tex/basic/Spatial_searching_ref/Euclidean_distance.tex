% +------------------------------------------------------------------------+
% | Reference manual page: Euclidean_distance.tex
% +------------------------------------------------------------------------+
% | 1.07.2001   Johan W.H. Tangelder
% | Package: ASPAS
% | 
\RCSdef{\RCSEuclideandistanceRev}{$Revision$}
\RCSdefDate{\RCSEuclideandistanceDate}{$Date$}
% |
%%RefPage: end of header, begin of main body
% +------------------------------------------------------------------------+


\begin{ccRefClass}{Euclidean_distance<PointTraits>}  %% add template arg's if necessary

%% \ccHtmlCrossLink{}     %% add further rules for cross referencing links
%% \ccHtmlIndexC[class]{} %% add further index entries

\ccDefinition
  
The class \ccRefName\ provides an implementation of the concept \ccc{OrthogonalDistance}, with the
Euclidean distance ($l_2$ metric).
To optimize distance computations squared distances are used.


\ccInclude{CGAL/Euclidean_distance.h}

\ccParameters

Expects for the template argument a model of the concept
\ccc{PointTraits}, for example \ccc{CGAL::Kd_tree_traits_2<Kernel>}.

\ccIsModel

OrthogonalDistance

\ccTypes
\ccTypedef{PointTraits::Point Point;}{Point type.}
\ccTypedef{PointTraits::NT NT;}{Number type.}

\ccCreation
\ccCreationVariable{d}  %% choose variable name

\ccConstructor{Euclidean_distance();}{Default constructor.}

\ccOperations
 
\ccMethod{NT distance(Point q, Point p);}{Returns the squared Euclidean distance between \ccc{q} and \ccc{p}.}

\ccMethod{NT min_distance_to_queryitem(Point q, Kd_tree_rectangle<NT> r;);}
{Returns the squared Euclidean distance between \ccc{q} and
the point on the boundary of \ccc{r} closest to \ccc{q}.}

\ccMethod{NT max_distance_to_queryitem(Point q, Kd_tree_rectangle<NT> r;);}
{Returns the squared Euclidean distance, where $d$ denotes the distance between \ccc{q} and
the point on the boundary of \ccc{r} farthest to \ccc{q}.}

\ccMethod{NT new_distance(NT dist, NT old_off, NT new_off, int cutting_dimension);} {Updates the squared \ccc{dist} incrementally
and returns the updated squared distance.}

\ccMethod{NT transformed_distance(NT d);} {Returns $d^2$.}

\ccMethod{NT inverse_of_transformed_distance(NT d);} {Returns $d^{1/2}$.}

\ccSeeAlso

OrthogonalDistance.
\ccc{CGAL::Weighted_Minkowski_distance<PointTraits>}.

\end{ccRefClass}

% +------------------------------------------------------------------------+
%%RefPage: end of main body, begin of footer
% EOF
% +------------------------------------------------------------------------+

