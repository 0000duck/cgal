% +------------------------------------------------------------------------+
% | Reference manual page: Weighted_Minkowski_distance.tex
% +------------------------------------------------------------------------+
% | 1.07.2001   Johan W.H. Tangelder
% | Package: ASPAS
% | 
\RCSdef{\RCSWeightedMinkowskidistanceRev}{$Revision$}
\RCSdefDate{\RCSWeightedMinkowskidistanceDate}{$Date$}
% |
%%RefPage: end of header, begin of main body
% +------------------------------------------------------------------------+


\begin{ccRefClass}{Weighted_Minkowski_distance<Point>}  %% add template arg's if necessary

%% \ccHtmlCrossLink{}     %% add further rules for cross referencing links
%% \ccHtmlIndexC[class]{} %% add further index entries

\ccDefinition
   
The class \ccRefName\ provides an implementation of the concept OrthogonalDistance, with a weighted Minkowski metric
on $d$-dimensional points
defined by $l_p(w)(r,q)=\sqrt[p] {\sum_{i=1}^{i=d} w_i(r_i-q_i)^p}$ for $0 < p <\infty$ and
defined by $l_{\infty}(w)(r,q)=max \{w_i |r_i-q_i| \mid 1 \leq i \leq d\}$.
For the purpose of the distance computations it is more efficient to compute
the transformed distance ${\sum_{i=1}^{i=d} w_i(r_i-q_i)^p}$ instead of the actual distance.

\ccInclude{CGAL/Weighted_Minkowski_distance.h}

\ccParameters

Expects for the parameter
\ccc{Point} an implementation
of a $d$-dimensional point
for example \ccc{CGAL::Point_d<R>}.

\ccIsModel

OrthogonalDistance

\ccTypes

\ccTypedef{Kernel_traits<Point>::Kernel::FT NT;}{Number type.}
\ccTypedef{std::vector<NT> Weight_vector;}{Vector type used to represent weights.}
 

\ccCreation
\ccCreationVariable{d}  %% choose variable name

\ccConstructor{Weighted_Minkowski_distance(int d);}{Constructor implementing $l_2$ metric for $d$-dimensional points.}

\ccConstructor{Weighted_Minkowski_distance(NT power, int dim, Weight_vector weights);}
{Constructor implementing the $l_{power}(weights)$ metric. $power \leq 0$ denotes the $l_{\infty}(weights)$ metric.}

\ccOperations


\ccMethod{NT distance(Point q, Point r);}{Returns $d^{power}$, 
where $d$ denotes the distance between \ccc{q} and \ccc{r}.}

\ccMethod{NT min_distance_to_queryitem(Point q, Kd_tree_rectangle<NT> r;);}
{Returns $d^{power}$, where $d$ denotes the distance between the query item \ccc{q} and
the point on the boundary of \ccc{r} closest to \ccc{q}.}

\ccMethod{NT max_distance_to_queryitem(Point q, Kd_tree_rectangle<NT> r;);}
{Returns $d^{power}$, where $d$ denotes the distance between the query item \ccc{q} and
the point on the boundary of \ccc{r} farthest to \ccc{q}.}

\ccMethod{NT new_distance(NT dist, NT old_off, NT new_off, int cutting_dimension);} {Updates \ccc{dist} incrementally
and returns the updated distance.}

\ccMethod{NT transformed_distance(NT d);} {Returns for $0 < p <\infty$ $d^p$. Returns for $p=\infty$ $d$.}

\ccMethod{NT inverse_of_transformed_distance(NT d);} {Returns for $0 < p <\infty$ $\sqrt[p]{d}$. Returns for $p=\infty$ $d$.}
 

\ccSeeAlso

OrthogonalDistance,
\ccc{CGAL::Euclidean_distance<Point>}.

\end{ccRefClass}

% +------------------------------------------------------------------------+
%%RefPage: end of main body, begin of footer
% EOF
% +------------------------------------------------------------------------+

