% +------------------------------------------------------------------------+
% | Reference manual page: Kd_tree_rectangle.tex
% +------------------------------------------------------------------------+
% | 1.07.2001   Johan W.H. Tangelder
% | Package: ASPAS
% | 
\RCSdef{\RCSKdtreerectangleRev}{$Revision$}
\RCSdefDate{\RCSKdtreerectangleDate}{$Date$}
% |
%%RefPage: end of header, begin of main body
% +------------------------------------------------------------------------+


\begin{ccRefClass}{Kd_tree_rectangle<Traits>}  %% add template arg's if necessary

\begin{ccAdvanced}

%% \ccHtmlCrossLink{}     %% add further rules for cross referencing links
%% \ccHtmlIndexC[class]{} %% add further index entries

\ccDefinition
  
The class \ccRefName\ implements \ccc{d}-dimensional iso-rectangles and related operations,
e.g., methods to compute bounding boxes of point sets. 


\ccInclude{CGAL/Kd_tree_rectangle.h}


\ccTypes

\ccTypedef{Traits::FT FT;}{Number type.}

\ccCreation
\ccCreationVariable{r}  %% choose variable name

\ccConstructor{Kd_tree_rectangle(int d);}
{Constructs a $d$-dimensional rectangle \ccc{r} with lower bound and upper bound set to zero 
in each dimension.}
 
\ccConstructor{
template <class PointIter>
Kd_tree_rectangle(int d, PointIter begin, PointIter end);}
{Constructs the bounding box of the points in the range [\ccc{begin,end}), where the value
type of \ccc{PointIter} must be \ccc{Traits::Point_d}.}

\ccOperations

\ccMethod{FT min_coord(int i);}
{Returns the lower bound of the rectangle in dimension~\ccc{i}.}

\ccMethod{FT max_coord(int i);}
{Returns the upper bound of the rectangle in dimension~\ccc{i}.}



\ccMethod{void set_upper_bound(int i, const FT& x);}
{Sets upper bound in dimension \ccc{i} to \ccc{x}.}

\ccMethod{void set_lower_bound(int i, const FT& x);}
{Sets lower bound in dimension \ccc{i} to \ccc{x}.}
 
 
\ccMethod{FT max_span();}
{Returns the maximal span of the rectangle.}

\ccMethod{FT max_span_coord();}
{Returns the smallest coordinate for which the rectangle has its maximal span.}

\ccMethod{int dimension();}
{Returns the dimension of the rectangle.}

\ccMethod{void split(Kd_tree_rectangle<Traits>& r, int d, FT value);}
{Splits rectangle in dimension \ccc{d} at coordinate-value \ccc{value} 
 by modifying itself to lower half and by modifying \ccc{r} to upper half.} 

\ccHeading{Output Routines}

\ccFunction{ template<class FT>
std::ostream& operator<<(std::ostream& s, Kd_tree_rectangle<Traits>& r);} 
{Inserts rectangle \ccc{r} in the output stream \ccc{s} and returns~\ccc{s}.}


\end{ccAdvanced}


\end{ccRefClass}

% +------------------------------------------------------------------------+
%%RefPage: end of main body, begin of footer
% EOF
% +------------------------------------------------------------------------+

