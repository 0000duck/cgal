% +------------------------------------------------------------------------+
% | Reference manual page: Kd_tree_rectangle.tex
% +------------------------------------------------------------------------+
% | 1.07.2001   Johan W.H. Tangelder
% | Package: ASPAS
% | 
\RCSdef{\RCSKdtreerectangleRev}{$Revision$}
\RCSdefDate{\RCSKdtreerectangleDate}{$Date$}
% |
%%RefPage: end of header, begin of main body
% +------------------------------------------------------------------------+


\begin{ccRefClass}{Kd_tree_rectangle<NT>}  %% add template arg's if necessary

%% \ccHtmlCrossLink{}     %% add further rules for cross referencing links
%% \ccHtmlIndexC[class]{} %% add further index entries

\ccDefinition
  
The class \ccRefName\ implements \ccc{d}-dimensional iso-rectangles and related operations.
E.g. constructors and methods to compute bounding boxes of point sets are provided. 


\ccInclude{CGAL/Kd_tree_rectangle.h}

\ccCreation
\ccCreationVariable{r}  %% choose variable name

\ccConstructor{Kd_tree_rectangle();}{default constructor.}


\ccConstructor{Kd_tree_rectangle(const int d);}
{Constructs a $d$-dimensional rectangle \ccc{r} with lower bound and upper bound set to zero 
in each dimension.}

\ccConstructor{Kd_tree_rectangle(const Kd_tree_rectangle<NT>& rect);}
{Copy constructor.}
 
\ccConstructor{
template <class PointIter>
Kd_tree_rectangle(const int d, PointIter begin, PointIter end);}
{Constructs the bounding box of the points in the range [\ccc{begin,end}).}


\ccHeading{Member Functions}

\ccMethod{void set_upper_bound(int i, const NT& x);}
{Sets upper bound in dimension \ccc{i} to \ccc{x}.}

\ccMethod{void set_lower_bound(int i, const NT& x);}
{Sets lower bound in dimension \ccc{i} to \ccc{x}.}

\ccMethod{NT lower(int i);}
{Returns the lower bound of the rectangle in dimension \ccc{i}.}

\ccMethod{NT upper(int i);}
{Returns the upper bound of the rectangle in dimension \ccc{i}.}
 
\ccMethod{ template <class PointPointerIter>
void update_from_point_pointers(PointPointerIter begin, PointPointIter end, bool empty);}
{Returns if empty equals false the bounding box of the points in the range [begin,end). 
Otherwise returns an empty box with lower bounds set to 1 and upper bounds set to -1.}

\ccMethod{NT max_span();}
{Returns the maximal span of the rectangle.}

\ccMethod{NT max_span_coord();}
{Returns the smallest coordinate for which the rectangle has its maximal span.}

\ccMethod{std::ostream& print(std::ostream& s);}
{Prints the rectangle to the output stream \ccc{s} and returns \ccc{s}.}

\ccMethod{int dimension();}
{Returns the dimension of the rectangle.}

\ccMethod{Kd_tree_rectangle* split(int d, NT value);}
{Splits rectangle in dimension \ccc{d} at coordinate-value \ccc{value} 
 by modifying itself to lower half and returning the the upper half.}

{\bf Non-Member functions related to \ccc{Kd_tree_rectangle<NT>}}

\ccGlueBegin
\ccGlobalFunction{ template<class NT>
std::ostream& operator<<(std::ostream& s, Kd_tree_rectangle<NT>& b);} 
{Prints the rectangle \ccc{b} to the output stream \ccc{s} and returns \ccc{s}.}
\ccGlueEnd

\ccGlueBegin
\ccGlobalFunction{ template <class NT, class Point>
bool belongs(const Point& p, const Kd_tree_rectangle<NT>& r);} \
{Returns true if and only if \ccc{p} is in \ccc{r} or is on the boundary of \ccc{r}.}
\ccGlueEnd



\end{ccRefClass}

% +------------------------------------------------------------------------+
%%RefPage: end of main body, begin of footer
% EOF
% +------------------------------------------------------------------------+

