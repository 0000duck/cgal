% +------------------------------------------------------------------------+
% | Reference manual page: Orthogonal_incremental_neighbor_search.tex
% +------------------------------------------------------------------------+
% | 1.07.2001   Johan W.H. Tangelder
% | Package: ASPAS
% |
\RCSdef{\RCSOrthogonalincrementalneighborsearchRev}{$Revision$}
\RCSdefDate{\RCSOrthogonalincrementalneighborsearchDate}{$Date$}
% |
%%RefPage: end of header, begin of main body
% +------------------------------------------------------------------------+


\begin{ccRefClass}{Orthogonal_incremental_neighbor_search<PointTraits, OrthogonalDistance, Splitter, SpatialTree>}

%% \ccHtmlCrossLink{}     %% add further rules for cross referencing links
%% \ccHtmlIndexC[class]{} %% add further index entries

\ccDefinition

The class \ccRefName\ implements incremental nearest and furthest neighbor searching
using incremental_neighbor search on a tree.

\ccInclude{CGAL/Orthogonal_incremental_neighbor_search.h}

\ccParameters

Expects for the first template argument a model of the concept \ccc{PointTraits},
for example \ccc{CGAL::Kd_tree_traits_2<Kernel>}.

Expects for the second template argument a model of the
concept GeneralDistance. The default type is 
\ccc{CGAL::Euclidean_distance<PointTraits>}.

Expects for third template argument a model of the concept \ccc{Splitter}.
The default type is \ccc{CGAL::Sliding_midpoint<PointTraits>}.

Expects for fourth template argument a model of the concept \ccc{SpatialTree}.
The default type is \ccc{CGAL::Kd_tree<PointTraits, Splitter, CGAL::Tag_true>}. The 
template argument must be \ccc{CGAL::Tag_true} because orthogonal search needs extended 
kd tree nodes.

\ccTypes

\ccTypedef{PointTraits::Point Point;}{Point type.}
\ccTypedef{PointTraits::NT NT;}{Number type.}
\ccTypedef{std::pair<Pair,NT> Point_with_distance;}{Pair of point and distance.}
\ccNestedType{iterator}{Forward iterator for enumerating approximate neighbors.}

\ccTypedef{SpatialTree Tree;}{The tree type.}
\ccCreation
\ccCreationVariable{s}  %% choose variable name

\def\ccLongParamLayout{\ccTrue}
\ccConstructor{Orthogonal_incremental_neighbor_search(SpatialTree& tree, Point query, NT eps=NT(0.0),
                                          bool search_nearest=true, 
                                          OrthogonalDistance d=OrthogonalDistance());}
{Constructor for incremental neighbor searching of the query item \ccc{query}
in the points stored \ccc{tree} using a distance \ccc{d} and approximation factor \ccc{eps}.}

\ccOperations

\ccMethod{iterator begin();}{Returns an iterator to the approximate neighbor.}

\ccMethod{iterator end();}{Past-the-end iterator.}

\
\begin{ccAdvanced}
\ccMethod{std::ostream& statistics(std::ostream& s);}
{
Inserts statistics of the search process into the output stream \ccc{s}.
}
\end{ccAdvanced}

\ccSeeAlso

\ccc{CGAL::Incremental_neighbor_search<PointTraits, GeneralDistance, SpatialTree>}.

\end{ccRefClass}

% +------------------------------------------------------------------------+
%%RefPage: end of main body, begin of footer
% EOF
% +------------------------------------------------------------------------+

