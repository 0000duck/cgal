% +------------------------------------------------------------------------+
% | Reference manual page: TreeTraits.tex
% +------------------------------------------------------------------------+
% | 1.07.2001   Johan W.H. Tangelder
% | Package: ASPAS
% | 
\RCSdef{\RCSTreeTraitsRev}{$Revision$}
\RCSdefDate{\RCSTreeTraitsDate}{$Date$}
% |
%%RefPage: end of header, begin of main body
% +------------------------------------------------------------------------+


\begin{ccRefConcept}{TreeTraits}

\ccDefinition
  
The concept \ccRefName\ defines the requirements for a tree traits class, which is used as
a template class for the construction of a tree. 
It is assumed that the tree consists of internal nodes
and leaf nodes. There are two types of external nodes. Extended internal nodes support orthogonal range searching.
Unextended internal nodes do not support orthogonal range searching.
Leaf nodes have a bucket size denoting the maximal number of points that can be 
stored at a leaf node.

\ccTypes

\ccNestedType{Point;}{$d$-dimensional point.}
\ccNestedType{Point_iterator;}{Point iterator.}
\ccNestedType{Container;}{Point container.}
\ccNestedType{Separator;}{Separator.}
\ccNestedType{NT;}{Number type.}

\ccCreation
\ccCreationVariable{tr}  %% choose variable name

\ccOperations

\ccMethod{void split(Separator& sep, Container<Point>& c0, Container<Point>& c1);}
{Splits \ccc{c0} into \ccc{c0} and \ccc{c1} and returns the separator}

\ccMethod{int bucket_size();} {Returns the bucket size of the leaf nodes.}

\ccMethod{int use_extended_nodes();} {Indicates whether a tree with extended internal nodes should be used.}
 
\ccHasModels

\ccc{CGAL::Kd_tree_traits_point<SpatialPoint,Splitter>}

\ccSeeAlso

Tree.
\ccc{CGAL::Kd_tree_traits_point<SpatialPoint,Splitter>},
\ccc{CGAL::Kd_tree<Traits>}.

\end{ccRefConcept}


% +------------------------------------------------------------------------+
%%RefPage: end of main body, begin of footer
% EOF
% +------------------------------------------------------------------------+

