% +------------------------------------------------------------------------+
% | Reference manual page: SpatialTree.tex
% +------------------------------------------------------------------------+
% | 1.07.2001   Johan W.H. Tangelder
% | Package: ASPAS
% | 
\RCSdef{\RCSSpatialTreeRev}{$Revision$}
\RCSdefDate{\RCSSpatialTreeDate}{$Date$}
% |
%%RefPage: end of header, begin of main body
% +------------------------------------------------------------------------+


\begin{ccRefConcept}{SpatialTree}

%% \ccHtmlCrossLink{}     %% add further rules for cross referencing links
%% \ccHtmlIndexC[concept]{} %% add further index entries

\ccDefinition
  
The concept \ccRefName\ defines the requirements for a tree supporting
both neighbor searching and approximate range searching. 

\ccTypes

\ccNestedType{SearchTraits}{Search traits.}
\ccNestedType{Point_d}{Point type.}
\ccNestedType{iterator;}{Bidirectional iterator with value type \ccc{Point_d} that allows
	to enumerate all points in the tree.}

\ccNestedType{Node_handle;}{Node handle.}
\ccNestedType{Point_d_iterator;}{Iterator with value type \ccc{Point_d*}.}
\ccNestedType{Splitter}{Splitter.}
\ccNestedType{Distance}{Distance.}

\ccCreation
\ccCreationVariable{tree}  %% choose variable name

\ccConstructor{template <class InputIterator> Tree(InputIterator first, InputIterator beyond, SearchTraits t);}
{
Constructs a tree on the elements from the sequence 
\ccc{[first,beyond)}. 
}

\ccOperations

\ccMethod{
template <class OutputIterator, class FuzzyQueryItem>
OutputIterator search(OutputIterator it, FuzzyQueryItem q);}
{Reports the points that are approximately contained by~\ccc{q}. The value type
of \ccc{OutputIterator} must be \ccc{Point_d}.} 


\ccMethod{iterator begin();}{Returns an iterator to the first point in the tree.}
\ccMethod{iterator end();}{Returns the corresponding past-the-end iterator.}


\ccMethod{Node_handle root();}
{
Returns a handle to the root node of the tree.
}

\ccMethod{const Kd_tree_rectangle<SearchTraits>& bounding_box();}{returns a const 
reference to the bounding box of the root node of the tree.}


\ccMethod{unsigned int size();}
{
Returns the number of items that are stored in the tree.
}

\ccHasModels

\ccc{CGAL::Kd_tree<Traits,Splitter,UseExtendedNode>}.

\end{ccRefConcept}

% +------------------------------------------------------------------------+
%%RefPage: end of main body, begin of footer
% EOF
% +------------------------------------------------------------------------+

