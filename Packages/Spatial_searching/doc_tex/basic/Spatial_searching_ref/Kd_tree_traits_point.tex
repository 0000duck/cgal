% +------------------------------------------------------------------------+
% | Reference manual page: Kd_tree_traits_point.tex
% +------------------------------------------------------------------------+
% | 1.07.2001   Johan W.H. Tangelder
% | Package: ASPAS
% | 
\RCSdef{\RCSKdtreetraitspointRev}{$Revision$}
\RCSdefDate{\RCSKdtreetraitspointDate}{$Date$}
% |
%%RefPage: end of header, begin of main body
% +------------------------------------------------------------------------+


\begin{ccRefClass}{Kd_tree_traits_point<Separator,Item>}  %% add template arg's if necessary

%% \ccHtmlCrossLink{}     %% add further rules for cross referencing links
%% \ccHtmlIndexC[class]{} %% add further index entries

\ccDefinition
  
The class \ccRefName\ defines a model of tree traits class, which is used as
a template class for the construction of a $k$-$d$ tree by the class \ccc{Kd_tree}.
The parameters \ccc{selected_split_rule}, \ccc{aspect_ratio}, \ccc{bucket_size}, and \ccc{use_extended_nodes}  
define the way in which $k$-$d$ tree is constructed.


\ccInclude{CGAL/Kd_tree_traits_point.h}

\ccIsModel

TreeTraits.

\ccParameters

Expects for the parameter \ccc{Separator} an implementation of a separator, for
example \ccc{Plane_separator}.
Expects for the parameter \ccc{Item} an implementation of a $d$-dimensional point, \\
for example \ccc{CGAL::Point_d}.


\ccTypes

\ccTypedef {Item::FT NT;}{Number type.}
\ccTypedef {Item** Item_Iterator;}{Item iterator.}
\ccTypedef {std::list<Item>::iterator InputIterator;}{Input iterator.}
\ccTypedef {std::pair<Item*,NT> Item_with_distance;}{Pair of item and distance.}

\ccCreation
\ccCreationVariable{tr}  %% choose variable name

\ccConstructor{Kd_tree_traits_point();}{Default constructor assigning the default values 
\ccc{bucket_size=1}, \ccc{Default_split_rule=SLIDING_MIDPOINT}, \ccc{aspect_ratio=3.0}, and \ccc{use_extended_nodes=true}
to the parameters.}

\ccConstructor{Kd_tree_traits_point
(int bucket_size=1, Split_rule Default_split_rule=SLIDING_MIDPOINT, NT aspect_ratio=3.0, bool use_extended_nodes=true);}
{Constructor assigning values to the parameters.}
\ccHeading{Member Functions}

\ccMethod{NT aspect_ratio();}{Returns the maximal ratio between the largest and smallest side
of a cell allowed for fair splitting.}

\ccMethod{Split_rule_enumeration::Split_rule selected_split_rule();}{Returns the selected splitting rule.}

\ccMethod{int bucket_size();}{Returns the selected bucket size.}

\ccMethod{bool use_extended_nodes();}{Returns boolean indicating whether extened internal nodes
should be used.}

\ccMethod{const Separator& split(Point_container<Item>& c0, Point_container<Item>& c1);}
{Splits \ccc{c0} into \ccc{c0} and \ccc{c1} and returns the separator.}

\ccSeeAlso

TreeTraits,
\ccc{Split_rule_enumeration},
\ccc{Median_of_max_spread},
\ccc{Fair},
\ccc{Sliding_fair},
\ccc{Sliding_midpoint},
\ccc{Median_of_rectangle},
\ccc{Midpoint_of_max_spread},
\ccc{Midpoint_of_rectangle},
\ccc{Kd_tree}.

\end{ccRefClass}


% +------------------------------------------------------------------------+
%%RefPage: end of main body, begin of footer
% EOF
% +------------------------------------------------------------------------+

