% +------------------------------------------------------------------------+
% | Reference manual page: Ball_d.tex
% +------------------------------------------------------------------------+
% | 1.07.2001   Johan W.H. Tangelder
% | Package: ASPAS
% | 
\RCSdef{\RCSBalldRev}{$Revision$}
\RCSdefDate{\RCSBalldDate}{$Date$}
% |
%%RefPage: end of header, begin of main body
% +------------------------------------------------------------------------+


\begin{ccRefConcept}{Ball_d}

%% \ccHtmlCrossLink{}     %% add further rules for cross referencing links
%% \ccHtmlIndexC[concept]{} %% add further index entries

\ccDefinition
  
The concept \ccRefName\ defines the requirements for $d$-dimensional balls.

{\bf Refines}

QueryItem

\ccParameters

\ccc{SpatialPoint} represents a $d$-dimensional point.

\ccTypes

\ccTypedef{Kernel_traits<SpatialPoint>::Kernel::FT NT;}{Number type.} 


\ccCreationVariable{b}  %% choose variable name



\ccOperations

\ccMethod{SpatialPoint center();}
{Returns the center of the ball.}


\ccMethod{NT squared_radius();}
{Returns the squared radius of the ball.}


\ccHasModels

\ccc{CGAL::Iso_circle_2<Kernel>}, \ccc{CGAL::Iso_sphere_3<Kernel>}, 
\ccc{CGAL::Iso_sphere_d<Kernel>}.

\ccSeeAlso

QueryItem, \ccc{CGAL::Iso_circle_2<Kernel>}, \ccc{CGAL::Iso_sphere_3<Kernel>}, 
\ccc{CGAL::Iso_sphere_d<Kernel>}.

\end{ccRefConcept}

% +------------------------------------------------------------------------+
%%RefPage: end of main body, begin of footer
% EOF
% +------------------------------------------------------------------------+

