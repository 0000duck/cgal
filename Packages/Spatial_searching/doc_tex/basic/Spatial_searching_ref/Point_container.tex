% +------------------------------------------------------------------------+
% | Reference manual page: Point_container.tex
% +------------------------------------------------------------------------+
% | 1.07.2001   Johan W.H. Tangelder
% | Package: ASPAS
% |
\RCSdef{\RCSPointcontainerRev}{$Revision$}
\RCSdefDate{\RCSPointcontainerDate}{$Date$}
% |
%%RefPage: end of header, begin of main body
% +------------------------------------------------------------------------+


\begin{ccRefClass}{Point_container<Item>}  %% add template arg's if necessary

%% \ccHtmlCrossLink{}     %% add further rules for cross referencing links
%% \ccHtmlIndexC[class]{} %% add further index entries

\begin{ccAdvanced}



\ccDefinition
A custom container for items used to build a search tree. Each point container
holds the points from a rectangle associated with a node of a $k$-$d$ tree.
In the remainder of this reference page this rectangle is called the
associated rectangle.
Provides a number of methods
to support the implementation of splitting rules.

\ccInclude{CGAL/Point_container.h}

\ccParameters

Expects for the parameter \ccc{Item} an implementation of 
a $d$-dimensional point, for example \ccc{CGAL::Point_d}.

\ccTypes

\ccTypedef{Item::FT NT;}{Number type.}
\ccTypedef{std::list<Item*> Point_list;}{(Sorted) list of pointers to points.}


%----------------------------------------
\ccCreation
\ccCreationVariable{c}

\ccConstructor{Point_container(const int d, Iter begin, Iter end);}
{
Construct the container of points of dimension \ccc{d}
given by the iterator sequence \ccc{begin}, \ccc{end}.
}

\ccConstructor{Point_container(int d);}
{
Construct an empty container for storing \ccc{d}-dimensional items.
}

\ccHeading{Member Functions}

\ccMethod{void swap(Point_container<Item>& C);}
{Swap the contents of \ccc{c} and \ccc{C}}

\ccMethod{void add_points_from_container(Point_container<Item>& C);}
{Add the contents of \ccc{C} to \ccc{c}}

\ccMethod{void recompute_tight_bounding_box();}
{Recompute the bounding box of the points in the container.}

\ccMethod{template <class Separator>
void split_container(Point_container<Item>& C, Separator* sep, bool sliding=false);}
{Given an empty container \ccc{C} with the same dimension as \ccc{c}, splits \ccc{c} into
\ccc{c} and \ccc{C} using the separator \ccc{sep}. If sliding is true the splitting is adapted
such that each container contains at least on point.}

\ccMethod{Point_list::iterator begin();}
{
Return the iterator to the first item.
}

\ccMethod{Point_list::iterator end();}
{
Return the iterator to the correspondiing past-the-end iterator.
}

\ccMethod{int dimension() const;}
{
Return the dimension.
}

\ccMethod{int built_coordinate();}
{
Return coordinate for which the pointer list is built.
}

\ccMethod{int max_span_coord();}
{
Return coordinate where the associated rectangle has maximal span.
}

\ccMethod{int max_tight_span_coord();}
{
Return coordinate where the point coordinates have maximal span.
}

\ccMethod{NT max_span_lower();}
{
Return lower value of the interval corresponding to
\ccc{max_span_coord()}.
}


\ccMethod{NT max_tight_span_lower();}
{
Return lower value of the interval corresponding to
\ccc{max_tight_span_coord()}. That is, the smallest
\ccc{max_tight_span_coord()}-th coordinate of the points in
\ccc{c}.
}


\ccMethod{NT max_span_upper();}
{
Return upper value of the interval corresponding to
\ccc{max_span_coord()}.
}

\ccMethod{NT max_span_upper_without_dim(int d);}
{
Return upper value of the interval over all dimensions
without taking dimension \ccc{d} into account.
}

\ccMethod{NT max_tight_span_upper();}
{
Return upper value of the interval corresponding to
\ccc{max_tight_span_coord()}.
}

\ccMethod{NT  max_spread();}
{
Return the size of the interval corresponding to \ccc{max_span_coord()}.
}

\ccMethod{NT max_tight_spread();}
{
Return the size of the interval corresponding to \ccc{max_tight_span_coord()}.
}

\ccMethod{NT median(int split_coord);}
{
Return the median value of the points stored in the container for
dimension \ccc{split_coord}.
}

\ccMethod{Kd_tree_rectangle<NT>& bounding_box();}
{Return the associated rectangle.}

\ccMethod{Kd_tree_rectangle<NT>& tight_bounding_box();}
{Return the bounding box of the items in associated rectangle.}

\ccMethod{int max_tight_span_coord_balanced(NT Aspect_ratio);}
{Return the dimension with the maximal point spread, for which after fair splitting
the ratio of the length of the longest side and the smallest side of the bounding box of
the items in associated rectangle,
does not exceed \ccc{Aspect_ratio}.}

\ccMethod{NT balanced_fair(int d, NT Aspect_ratio);}
{Return the splitting value for fair splitting.}

\ccMethod{NT balanced_sliding_fair(int d, NT Aspect_ratio);}
{Return the splitting value for sliding fair splitting.}

\ccMethod{int size();}
{
Return the number of points stored.
}

\ccMethod{bool empty();}
{
Return true is no points are present, false otherwise.
}



% -----------------------------------

{\bf Non-Member functions related to \ccc{Point_container<Item>}}

\ccGlueBegin
\ccGlobalFunction{ template<class Item>
std::ostream& operator<<(std::ostream& s, Point_container<Item>& c);}
{Prints the points container \ccc{c} to the output stream \ccc{s} and returns \ccc{s}.}
\ccGlueEnd

\end{ccAdvanced}

\end{ccRefClass}

% +------------------------------------------------------------------------+
%%RefPage: end of main body, begin of footer
% EOF
% +------------------------------------------------------------------------+

