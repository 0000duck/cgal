% +------------------------------------------------------------------------+
% | Reference manual page: Manhattan_distance_rectangle_point.tex
% +------------------------------------------------------------------------+
% | 1.07.2001   Johan W.H. Tangelder
% | Package: ASPAS
% | 
\RCSdef{\RCSManhattandistancerectanglepointRev}{$Revision$}
\RCSdefDate{\RCSManhattandistancerectanglepointDate}{$Date$}
% |
%%RefPage: end of header, begin of main body
% +------------------------------------------------------------------------+


\begin{ccRefClass}{Manhattan_distance_rectangle_point<QueryItem,Point>}  %% add template arg's if necessary

%% \ccHtmlCrossLink{}     %% add further rules for cross referencing links
%% \ccHtmlIndexC[class]{} %% add further index entries

\ccDefinition
  
The class \ccRefName\ provides an implementation of \\
the GeneralDistance concept
for the Manhattan distance ($l_1$ metric)
between a $d$-dimensional iso-rectangle and a $d$-dimensional point 
and the Manhattan distance between a $d$-dimensional iso-rectangle
and a $d$-dimensional iso-rectangle defined as a $k$-$d$ tree rectangle.

\ccInclude{CGAL/Manhattan_distance_rectangle_point.h}

\ccParameters

Expects for the parameter \ccc{QueryItem}
\ccc{item} an implementation of iso-rectangles
for example \ccc{CGAL::Iso_rectangle_d<R>}.
Expects for the parameter \ccc{Item} an implementation
of a $d$-dimensional point, for example \ccc{CGAL::Point_d<R>}.

\ccIsModel

GeneralDistance

\ccTypes

\ccTypedef{Kernel_traits<Point>::Kernel::FT NT;}{Number type.}

\ccCreation
\ccCreationVariable{d}  %% choose variable name

\ccConstructor{Manhattan_distance_rectangle_point();}{Default constructor implementing Manhattan distance for
two-dimensional iso-rectangles.}

\ccConstructor{Manhattan_distance_rectangle_point(int d);}
{Constructor implementing Manhattan distance for
$d$-dimensional iso-rectangles.}

\ccOperations

\ccMethod{NT transformed_distance(NT d);} {Returns $d$.}

\ccMethod{NT inverse_of_transformed_distance(NT d);} {Returns $d$.}

\ccMethod{NT distance(QueryItem q, Point p);}{Returns the distance between \ccc{q} and \ccc{p}.}

\ccMethod{NT min_distance_to_queryitem(QueryItem q, Kd_tree_rectangle<NT> r;);}
{Returns the minimal distance between a point from \ccc{q} and a point from
\ccc{r}}

\ccMethod{NT max_distance_to_queryitem(QueryItem q, Kd_tree_rectangle<NT> r;);}
{Returns the maximal distance between the iso-rectangle \ccc{q} and
a point from \ccc{r} furthest to \ccc{q}.}


\ccSeeAlso

GeneralDistance.

\end{ccRefClass}

% +------------------------------------------------------------------------+
%%RefPage: end of main body, begin of footer
% EOF
% +------------------------------------------------------------------------+

