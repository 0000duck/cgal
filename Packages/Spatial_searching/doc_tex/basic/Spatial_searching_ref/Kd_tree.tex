% +------------------------------------------------------------------------+
% | Reference manual page: Kd_tree.tex
% +------------------------------------------------------------------------+
% | 1.07.2001   Johan W.H. Tangelder
% | Package: ASPAS
% | 
\RCSdef{\RCSKdtreeRev}{$Revision$}
\RCSdefDate{\RCSKdtreeDate}{$Date$}
% |
%%RefPage: end of header, begin of main body
% +------------------------------------------------------------------------+


\begin{ccRefClass}{Kd_tree<TreeTraits>}  %% add template arg's if necessary

%% \ccHtmlCrossLink{}     %% add further rules for cross referencing links
%% \ccHtmlIndexC[class]{} %% add further index entries

\ccDefinition  
The class \ccRefName\ defines a $k$-$d$ tree.

\ccInclude{CGAL/Kd_tree.h}

\ccParameters

Expects for the template argument an implementation for a tree traits class
of the concept \ccc{TreeTraits},
for example \ccc{CGAL::Kd_tree_traits_point<Point,Splitter>}.

\ccTypes

\ccTypedef{TreeTraits::Point Point;} {Point class.}
\ccTypedef{TreeTraits::Container Point_container;} {Point container.}
\ccTypedef{TreeTraits::NT NT;}{Number type.}
\ccTypedef{Kd_tree_node<TreeTraits> Node;}{Node type.}
\ccTypedef{Kd_tree<TreeTraits> Tree;}{Tree type.}
\ccTypedef{Compact_container<Node>::iterator Node_handle;}{Node handle.}
\ccTypedef{std::vector<Point*>::iterator Point_iterator;} {Iterator over points.}
\ccTypedef{std::list<Point>::iterator input_iterator;}{Input iterator.}

\ccCreation
\ccCreationVariable{tree}

\ccConstructor
{Kd_tree(input_iterator first, input_iterator beyond, traits t=TreeTraits());}
{
Constructs a $k$-$d$ tree on the elements from the sequence 
\ccc{first, beyond} using the splitting rule implemented by \ccc{t}. 
}

% -----------------------------------
\ccOperations

\ccMethod{
template <class OutputIterator, class FuzzyQueryItem>
OutputIterator search(OutputIterator it, FuzzyQueryItem q);}
{Reports the points that are approximately contained by \ccc{q}.} 

\ccMethod{
template <class OutputIterator>
OutputIterator report_all_points(OutputIterator it);}
{Reports all the points contained by the $k$-$d$ tree.} 

\ccMethod{TreeTraits traits();}
{
Returns an instance of the traits class.
}
\ccMethod{Node_handle root();}
{
Returns a handle to the root node of the tree.
}

\ccMethod{Kd_tree_rectangle<NT>* bounding_box();}{returns a pointer to the bounding box 
of the root node of the tree.}


\ccMethod{int item_number();}
{
Returns the number of items that are stored in the tree.
}

\begin{ccAdvanced}

\ccMethod{std::ostream& statistics(std::ostream& s);}
{
Inserts statistics of the tree into the output stream \ccc{s}.
}

\end{ccAdvanced}

% -----------------------------------

\ccSeeAlso
Tree.
\ccc{CGAL::Kd_tree_node<TreeTraits>},
\ccc{CGAL::Kd_tree_traits_point<Point,Splitter>}.

\end{ccRefClass}

% +------------------------------------------------------------------------+
%%RefPage: end of main body, begin of footer
% EOF
% +------------------------------------------------------------------------+

