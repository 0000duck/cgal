% +------------------------------------------------------------------------+
% | Reference manual page: Kd_tree.tex
% +------------------------------------------------------------------------+
% | 1.07.2001   Johan W.H. Tangelder
% | Package: ASPAS
% | 
\RCSdef{\RCSKdtreeRev}{$Revision$}
\RCSdefDate{\RCSKdtreeDate}{$Date$}
% |
%%RefPage: end of header, begin of main body
% +------------------------------------------------------------------------+


\begin{ccRefClass}{Kd_tree<Traits>}  %% add template arg's if necessary

%% \ccHtmlCrossLink{}     %% add further rules for cross referencing links
%% \ccHtmlIndexC[class]{} %% add further index entries

\ccDefinition
  
The class \ccRefName\ defines a $k$-$d$ tree.

\ccInclude{CGAL/Kd_tree.h}

\ccHeading{Parameters}

Expects for the parameter \ccc{Traits} an implementation of the \ccc{Tree_traits}
concept.

\ccTypes

\ccTypedef{Traits::InputIterator InputIterator;}{Input iterator over items.}
\ccTypedef{Traits::Item Item;} {Items class.}
\ccTypedef{Traits::NT NT;}{Number type.}
\ccTypedef{Kd_tree_node<Traits> Node;}{Node type.}

\ccCreation
\ccCreationVariable{tree}

\ccConstructor{template <class InputIterator>
Kd_tree(InputIterator first, InputIterator beyond, traits t=Traits());}
{
Construct a $k$-$d$ tree on the elements from the sequence 
\ccc{frist, beyond} using the splitting rule defined by \ccc{Traits()}. 
}

% -----------------------------------
\ccHeading{Member Functions}

\ccMethod{Traits traits();}
{
Returns an instance of the traits class.
}
\ccMethod{Node* root();}
{
Returns a pointer to the root node of the tree.
}

\ccMethod{Kd_tree_rectangle<NT>* bounding_box();}{returns a pointer to the bounding box 
of the root node of the tree.}


\ccMethod{int item_number();}
{
Returns the number of items that are stored in the tree.
}

\ccMethod{void statistics();}
{
Prints statistics of the tree.
}



% -----------------------------------

\ccSeeAlso

\ccc{Kd_tree_node},
\ccc{Search_tree_traits}.

\end{ccRefClass}

% +------------------------------------------------------------------------+
%%RefPage: end of main body, begin of footer
% EOF
% +------------------------------------------------------------------------+

