% +------------------------------------------------------------------------+
% | Reference manual page: FuzzyQueryItem.tex
% +------------------------------------------------------------------------+
% | 1.07.2001   Johan W.H. Tangelder
% | Package: ASPAS
% | 
\RCSdef{\RCSFuzzyQueryItemRev}{$Revision$}
\RCSdefDate{\RCSFuzzyQueryItemDate}{$Date$}
% |
%%RefPage: end of header, begin of main body
% +------------------------------------------------------------------------+


\begin{ccRefConcept}{FuzzyQueryItem}

%% \ccHtmlCrossLink{}     %% add further rules for cross referencing links
%% \ccHtmlIndexC[concept]{} %% add further index entries

\ccDefinition
  
The concept \ccRefName\ provides a model for fuzzy $d$-dimensional spatial objects.

\ccParameters

\ccc{SpatialPoint} represents a $d$-dimensional point.

\ccTypes

\ccTypedef{Kernel_traits<SpatialPoint>::Kernel::FT NT;}{Number type.} 

\ccCreationVariable{q}  %% choose variable name



\ccOperations

\ccMethod{bool contains(const SpatialPoint p); const}{test whether $q$ contains $p$.}

\ccMethod{bool inner_range_intersects(const Kd_tree_rectangle<NT>* rectangle); const}
{test whether the inner approximation of the spatial object intersects a rectangle
associated with a node of a tree.}

\ccMethod{bool outer_range_is_contained_by(const Kd_tree_rectangle<NT>* rectangle); const}
{test whether the outer approximation of the spatial object encloses a rectangle
associated with a node of a tree.}


\ccHasModels

\ccc{CGAL::Fuzzy_sphere_d<SpatialPoint>},
\ccc{CGAL::Fuzzy_iso_box_d<SpatialPoint,Iso_box_d>}.

\ccSeeAlso

\ccc{CGAL::Fuzzy_sphere_d<SpatialPoint>},
\ccc{CGAL::Fuzzy_iso_box_d<SpatialPoint,Iso_box_d>}.


\end{ccRefConcept}

% +------------------------------------------------------------------------+
%%RefPage: end of main body, begin of footer
% EOF
% +------------------------------------------------------------------------+

