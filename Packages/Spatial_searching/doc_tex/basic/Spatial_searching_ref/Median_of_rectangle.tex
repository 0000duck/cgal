% +------------------------------------------------------------------------+
% | Reference manual page: Median_of_rectangle.tex
% +------------------------------------------------------------------------+
% | 1.07.2001   Johan W.H. Tangelder
% | Package: ASPAS
% | 
\RCSdef{\RCSMedianofrectangleRev}{$Revision$}
\RCSdefDate{\RCSMedianofrectangleDate}{$Date$}
% |
%%RefPage: end of header, begin of main body
% +------------------------------------------------------------------------+

\begin{ccRefFunctionObjectClass}{Median_of_rectangle<PointTraits, PointContainer, SpatialSeparator>}

\ccDefinition
Implements the {\bf median of rectangle} splitting rule.

\ccParameters

Expects for the first template argument a model of
the concept \ccc{PointTraits}, for example
the type \ccc{CGAL::Kd_tree_traits_point_3< Cartesian<double> >}.

Expects for the second argument a model of the concept \ccc{PointContainer}. It has as default value
the type \ccc{CGAL::Point_container<PointTraits::Point>}.

Expects for the third template argument a model of the concept \ccc{Separator}. It has as default value
the type, \ccc{CGAL::Plane_separator<PointTraits::NT>}.

\ccInclude{CGAL/Splitters.h}

\ccIsModel

Splitter

\ccTypes

\ccTypedef{PointTraits::NT NT;}{Number type.} 
\ccTypedef{PointContainer Container;}{Point container.}
\ccTypedef{SpatialSeparator Separator;}{Separator.}

\ccCreationVariable{s}  %% choose variable name

\ccOperations

\ccMethod{
void operator()(Separator& s, Container& c0, Container& c1);} 
{Modifies the separator $s$ by the {\bf median of rectangle} splitting rule, 
and splits \ccc{c0} into \ccc{c0} and \ccc{c1},
using the splitting dimension and the splitting value of the modified separator.
}

\ccSeeAlso

Splitter, PointContainer, SpatialSeparator.
\end{ccRefFunctionObjectClass}



% +------------------------------------------------------------------------+
%%RefPage: end of main body, begin of footer
% EOF
% +------------------------------------------------------------------------+

