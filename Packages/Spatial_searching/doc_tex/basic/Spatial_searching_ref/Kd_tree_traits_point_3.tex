% +------------------------------------------------------------------------+
% | Reference manual page: Kd_tree_traits_point.tex
% +------------------------------------------------------------------------+
% | 1.07.2001   Johan W.H. Tangelder
% | Package: ASPAS
% | 
\RCSdef{\RCSKdtreetraitspoint2Rev}{$Revision$}
\RCSdefDate{\RCSKdtreetraitspoint2Date}{$Date$}
% |
%%RefPage: end of header, begin of main body
% +------------------------------------------------------------------------+


\begin{ccRefClass}{Kd_tree_traits_point_3<Kernel>}

%% \ccHtmlCrossLink{}     %% add further rules for cross referencing links
%% \ccHtmlIndexC[class]{} %% add further index entries

\ccDefinition
  
The class \ccRefName\ defines a model of tree traits class, which is used as
a template parameter of the class \ccc{Kd_tree}.

\ccc{Kernel} must be a \cgal\ kernel.


\ccInclude{CGAL/Kd_tree_traits_point_3.h}

\ccParameters
Expects for the template argument a model of the concept \ccc{Kernel},
for example \ccc{CGAL::Cartesian<double>} or \ccc{CGAL::Simple_cartesian<CGAL::Gmp_q>}.

\ccIsModel

PointTraits.

\ccTypes

\ccTypedef {Kernel::Point_3 Point;}{Point type.}
\ccTypedef {Point** Point_iterator;}{Point iterator.}
\ccTypedef {Kernel::Cartesian_const_iterator_3 Cartesian_const_iterator;}{An iterator over the coordinates.}
\ccTypedef {Kernel::Construct_cartesian_const_iterator_3 Construct_cartesian_const_iterator;}{.}
\ccTypedef{Kernel::FT NT;}{Number type.}


\ccSeeAlso

\ccc{Kd_tree_traits_point_2}\\
\ccc{Kd_tree_traits_point_d}\\
\ccc{Kd_tree_traits_point<Point,CartesianConstIterator,ConstructCartesianConstIterator}

\end{ccRefClass}


% +------------------------------------------------------------------------+
%%RefPage: end of main body, begin of footer
% EOF
% +------------------------------------------------------------------------+

