% +------------------------------------------------------------------------+
% | Reference manual page: Orthogonal_standard_search.tex
% +------------------------------------------------------------------------+
% | 1.07.2001   Johan W.H. Tangelder
% | Package: ASPAS
% |
\RCSdef{\RCSOrthogonalstandardsearchRev}{$Revision$}
\RCSdefDate{\RCSOrthogonalstandardsearchDate}{$Date$}
% |
%%RefPage: end of header, begin of main body
% +------------------------------------------------------------------------+


\begin{ccRefClass}{Orthogonal_standard_search<TreeTraits,Distance,Tree>}  %% add template arg's if necessary

%% \ccHtmlCrossLink{}     %% add further rules for cross referencing links
%% \ccHtmlIndexC[class]{} %% add further index entries

\ccDefinition

The class \ccRefName\ implements approximate $k$-nearest and $k$-furthest neighbor searching
using standard search on a  tree using an orthogonal distance
class. It is required that tree is built with extended nodes.

\ccInclude{CGAL/Orthogonal_standard_search.h}

\ccParameters

Expects for the parameter \ccc{TreeTraits} an implementation of the concept TreeTraits,
for example \ccc{CGAL::Kd_tree_traits<Point,Splitter>}.
Expects for the parameter \ccc{Distance} an implementation of the
concept OrthogonalDistance. \ccc{Distance} has default argument 
\ccc{CGAL::Euclidean_distance<TreeTraits::Point>}.
Expects for the parameter \ccc{Tree} an implementation of the concept Tree.
\ccc{Tree} has default argument \ccc{CGAL::Kd_tree<TreeTraits>}.




\ccTypes

\ccTypedef{TreeTraits::Point Point;}{Point type.}
\ccTypedef{TreeTraits::NT NT;}{Number type.}
\ccTypedef{std::pair<Point*,NT> Point_with_distance;}{Pair of points and distance}

\ccCreationVariable{s}  %% choose variable name

\ccOperations

\ccConstructor{Orthogonal_standard_search(Tree tree, QueryItem q, Distance d=Distance(), int k=1, NT Eps=NT(0.0),
bool Search_nearest=true);}
{Constructor for searching approximately $k$ neighbors of the query item \ccc{q} 
in the points stored in \ccc{tree} using 
distance
traits class $d$ and approximation factor \ccc{Eps}.}

\ccMethod{template <class OutputIterator>
OutputIterator  the_k_neighbors(OutputIterator res);}
{Returns a sequence of $k$ elements of type \ccc{Point_with_distance}}.


\begin{ccAdvanced}
\ccMethod{std::ostream& statistics(std::ostream& s);}
{
Inserts statistics of the search process into the output stream \ccc{s}.
}
\end{ccAdvanced}

\ccSeeAlso

\ccc{CGAL::General_standard_search<TreeTraits, Distance, QueryItem, Tree>}.

\end{ccRefClass}

% +------------------------------------------------------------------------+
%%RefPage: end of main body, begin of footer
% EOF
% +------------------------------------------------------------------------+

