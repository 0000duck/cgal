% +------------------------------------------------------------------------+
% | Reference manual page: Point.tex
% +------------------------------------------------------------------------+
% | 1.07.2001   Johan W.H. Tangelder
% | Package: ASPAS
% | 
\RCSdef{\RCSPointRev}{$Revision$}
\RCSdefDate{\RCSPointDate}{$Date$}
% |
%%RefPage: end of header, begin of main body
% +------------------------------------------------------------------------+


\begin{ccRefConcept}{Point}

%% \ccHtmlCrossLink{}     %% add further rules for cross referencing links
%% \ccHtmlIndexC[concept]{} %% add further index entries

\ccDefinition
  
The concept \ccClassTemplateName\ defines the requirements for a 
$d$-dimensional point type.
For a user defined $d$-dimensional point type a nested representation class
Point::R has to be provided or \ccc{CGAL::Kernel_traits} has to be
specialized for the point type. 

\ccCreationVariable{p}  %% choose variable name

\ccOperations

\ccMethod{int dimension();}
{Returns the dimension of a point.}

\ccMethod{ 
FT operator[](int i);}
{Returns the $i^{th}$ coordinate.}

\ccHasModels

\ccc{CGAL::Point_2<R>, CGAL::Point_3<R>, CGAL::Point_d<R>}.

\end{ccRefConcept}

% +------------------------------------------------------------------------+
%%RefPage: end of main body, begin of footer
% EOF
% +------------------------------------------------------------------------+

