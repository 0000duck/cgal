\chapter{Approximate Spatial Searching}

\section{Introduction}

\section{Approximate spatial searching}

To be provided (see paper on distance browsing in spatial databases). 

\section{The $k$-$d$ Tree}

Bentley \cite{b-mbstu-75} introduced the $k$-$d$ tree as a generalization of the binary
search tree in higher dimensions. $k$-$d$ trees hierarchically decompose space into a 
relatively small number of rectangles such that no rectangle contains too many input objects. 
For our purposes, a {\it rectangle} in real $d$ dimensional space, $\R^d$, is the product of $d$ closed 
intervals on the coordinate axes.
$k$-$d$ trees are obtained by partitioning point sets in $\R^d$ using ($d$-1)-dimensional hyperplanes. 
Each node in the tree is split into two children by one such separating plane.

Each internal node of the $k$-$d$ tree is associated with a rectangle and a hyperplane orthogonal to
one of the coordinate axis, which splits the rectangle into two parts. These two parts are then associated
with the two child nodes in the tree. The process of partitioning space continues until the number of data 
points in the rectangle falls below some given treshold. The rectangles associated with the leaf nodes
are called {\it buckets}, and they define a subdivision of the space into rectangles.
Data points are only stored in the leaf nodes of the tree, not in the internal nodes.

Friedmann, Bentley and Finkel \cite{fbf-afbml-77} described the standard
search algorithm to find the $k$th nearest neighbour by searching a $k$-$d$ tree recursively.

When encountering a node of the kd-tree the algorithm first visits the child that is closest 
to the query point. On return, if the box containing  the other child lie within 
1/ (1+$\epsilon$) times the distance to the $k$th nearest neighbours so far, then 
the other child is visited recursively.
Priority search \cite{???} visits the nodes in increasing order of distance from 
the queue with help of a priority queue. 
The search stops when the distance of the query point to the nearest nodes 
exceeds the distance to the nearest point found with a factor 1/ (1+$\epsilon$).
Priority search supports next neighbour search, standard search does not.

Splitting rules ...

\section{Getting started}

The first example program illustrates building $k$-$d$ trees using 1000
100-dimensional points. 
The way the $k$-$d$ tree is build is defined by the traits variable \ccc{tr(bucket_size, s, 3.0 true)}.
In the example the \ccc{bucket_size} is set to 10.
Hence, each leaf nodes stores at most 10 points. 
The strategy for splitting the nodes is defined by $s$.
Some rules use a maximal aspect ratio to define
the maximal allowed ratio between the largest and smallest side of a rectangle.
This aspect ratio is set to 3.0. The boolean value true indicates that
extended internal nodes are used.

\ccIncludeExampleCode{Example03.C}

More examples to be provided.
