% begin cgal manual page

\begin{ccRefClass}{Point_d<R>}\ccCreationVariable{p}

\ccDefinition


An instance of data type \ccc{Point_d<R>} is a point of Euclidean space in
dimension $d$. A point $p = (p_0,\ldots,p_{ d - 1 })$ in
$d$-dimensional space can be represented by homogeneous coordinates
$(h_0,h_1,\ldots,h_d)$ of number type \ccc{RT} such that $p_i = h_i/h_d$,
which is of type \ccc{FT}. The homogenizing coordinate $h_d$ is positive.

We call $p_i$, $0 \leq i < d$ the $i$-th Cartesian coordinate and
$h_i$, $0 \le i \le d$, the $i$-th homogeneous coordinate. We call $d$
the dimension of the point. 



\ccSetOneOfTwoColumns{4cm}

\ccTypes

\ccNestedType{RT}{the ring type. 
}

\ccNestedType{FT}{the field type. 
}

\ccNestedType{LA}{the linear algebra layer. 
}

\ccNestedType{Cartesian_const_iterator}{a read-only iterator for the cartesian coordinates. 
}

\ccNestedType{Homogeneous_const_iterator}{a read-only iterator for the homogeneous coordinates. 
}

\ccSetOneOfTwoColumns{4cm}

\ccCreation

\ccConstructor{Point_d<R>(int d = 0)}{introduces a variable \ccc{p} of type \ccc{Point_d<R>} in 
$d$-dimensional space. 
}

\ccConstructor{Point_d<R>(int d, Origin)}{introduces a variable \ccc{p} of type \ccc{Point_d<R>} in 
$d$-dimensional space, initialized to the origin. 
}

\ccConstructor{template <class InputIterator>
Point_d<R>(int d, InputIterator first, InputIterator last)}{introduces a variable \ccc{p} of type \ccc{Point_d<R>} in
dimension \ccc{d}.  If \ccc{size [first,last) == d} this creates a point with
Cartesian coordinates \ccc{set [first,last)}. If \ccc{size [first,last) == 
p+1} the range specifies the homogeneous coordinates $\ccc{H = set 
[first,last)} = (\pm h_0, \pm h_1, \ldots, \pm h_d)$ where the sign
chosen is the sign of $h_d$.  \ccPrecond \ccc{d} is nonnegative,
\ccc{[first,last)} has \ccc{d} or \ccc{d+1} elements where the last has to be
non-zero, and the value type of \ccc{InputIterator} is \ccc{RT}. 
}

\ccConstructor{template <class InputIterator>
Point_d<R>(int d, InputIterator first, InputIterator last, RT D)}{introduces a variable \ccc{p} of type \ccc{Point_d<R>} in
dimension \ccc{d} initialized to the point with homogeneous coordinates as
defined by \ccc{H = set [first,last)} and \ccc{D}: $(\pm \ccc{H[0]}, \pm\ccc{H[1]},
\ldots, \pm\ccc{H[d-1]}, \pm\ccc{D})$. The sign chosen is the sign of
$D$. \ccPrecond \ccc{D} is non-zero, the iterator range defines a $d$-tuple
of \ccc{RT}, and the value type of \ccc{InputIterator} is \ccc{RT}.  
}

\ccConstructor{Point_d<R>(RT x, RT y, RT w = 1)}{introduces a variable \ccc{p} of type \ccc{Point_d<R>} in 
$2$-dimensional space. 
}

\ccConstructor{Point_d<R>(RT x, RT y, RT z, RT w)}{introduces a variable \ccc{p} of type \ccc{Point_d<R>} in 
$3$-dimensional space. 
}

\ccSetTwoOfThreeColumns{4cm}{3cm}

\ccOperations

\ccMethod{int dimension() ;}{returns the dimension of \ccc{p}.  
}

\ccMethod{FT cartesian(int i) ;}{returns the $i$-th Cartesian coordinate of \ccc{p}. 
   \ccPrecond $0 \leq i < d$. 
}

\ccMethod{FT operator[](int i) ;}{returns the $i$-th Cartesian coordinate of \ccc{p}.
   \ccPrecond $0 \leq i < d$. 
}

\ccMethod{RT homogeneous(int i) ;}{returns the $i$-th homogeneous coordinate of \ccc{p}.
   \ccPrecond $0 \leq i \leq d$. 
}

\ccMethod{Cartesian_const_iterator cartesian_begin() ;}{returns an iterator pointing to the zeroth Cartesian coordinate 
$p_0$ of \ccc{p}.  
}

\ccMethod{Cartesian_const_iterator cartesian_end() ;}{returns an iterator pointing beyond the last Cartesian coordinate 
of \ccc{p}.  
}

\ccMethod{Homogeneous_const_iterator homogeneous_begin() ;}{returns an iterator pointing to the zeroth homogeneous coordinate 
$h_0$ of \ccc{p}.  
}

\ccMethod{Homogeneous_const_iterator homogeneous_end() ;}{returns an iterator pointing beyond the last homogeneous coordinate 
of \ccc{p}.  
}

\ccMethod{Point_d<R> transform(const Aff_transformation_d<R>& t) ;}{returns $t(p)$.  
}

\ccHeading{Arithmetic Operators, Tests and IO} 



\ccMethod{ Vector_d<R> operator-(const Origin& o) ;}{returns the vector $p-O$. 
}

\ccMethod{Vector_d<R> operator-(const Point_d<R>& q) ;}{returns $p - q$. \ccPrecond \ccc{p.dimension() == q.dimension()}. 
}

\ccMethod{Point_d<R> operator+(const Vector_d<R>& v) ;}{returns $p + v$. \ccPrecond \ccc{p.dimension() == v.dimension()}. 
}

\ccMethod{Point_d<R> operator-(const Vector_d<R>& v) ;}{returns $p - v$. \ccPrecond \ccc{p.dimension() == v.dimension()}. 
}

\ccMethod{Point_d<R>& operator+=(const Vector_d<R>& v);}{adds \ccc{v} to \ccc{p}.\\
\ccPrecond \ccc{p.dimension() == v.dimension()}.  
}

\ccMethod{Point_d<R>& operator-=(const Vector_d<R>& v);}{subtracts \ccc{v} from \ccc{p}.\\
\ccPrecond \ccc{p.dimension() == v.dimension()}.  
}

\ccMethod{bool operator==(const Origin&) ;}{returns true if \ccc{p} is the origin.  
}

\ccHeading{Downward compatibility}
We provide operations of the lower dimensional interface \ccc{x()}, \ccc{y()},
\ccc{z()}, \ccc{hx()}, \ccc{hy()}, \ccc{hz()}, \ccc{hw()}. 



\ccImplementation


Points are implemented by arrays of \ccc{RT} items.  All operations like
creation, initialization, tests, point - vector arithmetic, input and
output on a point $p$ take time $O(\ccc{p.dimension()})$. \ccc{dimension()},
coordinate access and conversions take constant time.  The space
requirement for points is $O(\ccc{p.dimension()})$. 



\end{ccRefClass}


