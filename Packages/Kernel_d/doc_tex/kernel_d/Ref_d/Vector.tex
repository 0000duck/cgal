\begin{ccRefConcept}{Vector}
\ccCreationVariable{v}

\ccDefinition

An instance of data type \ccc{Vector} is a vector of variables of
number type \ccc{NT}.  Together with the type \ccc{Matrix} it realizes
the basic operations of linear algebra.

\ccSetOneOfTwoColumns{5.5cm}

\ccTypes

\ccNestedType{NT}{
  the number type of the components.}

\ccNestedType{iterator}{
  the iterator type for accessing components.}

\ccNestedType{const_iterator}{
 the const iterator type for accessing components.}

\ccNestedType{Initialize}{
  a tag class for homogeneous initialization.}

\ccCreationVariable{v}

\ccSetOneOfTwoColumns{3cm}

\ccCreation

\ccConstructor{Vector(int d = 0)}{ creates an instance \ccc{v} of type
  \ccc{Vector}.  \ccc{v} is initialized to a vector of dimension $d$.
  }

\ccConstructor{Vector(int d, Initialize, NT x)}{ creates an instance
  \ccc{v} of type \ccc{Vector}.  \ccc{v} is initialized to a vector of
  dimension $d$ with entries \ccc{x}.  }

\ccConstructor{template <class Forward_iterator>
  Vector(Forward_iterator first, Forward_iterator last)}{ creates an
  instance \ccc{v} of type \ccc{Vector}; \ccc{v} is initialized to the
  vector with entries \ccc{tuple [first,last)}. \ccRequire
  \ccc{Forward_iterator} has value type \ccc{NT}. }

\ccConstructor{Vector(NT a, NT b)}{ creates an instance \ccc{v} of
  type \ccc{Vector}.  \ccc{v} is initialized to the two-dimensional
  vector $(a,b)$.  }

\ccConstructor{Vector(NT a, NT b, NT c)}{
creates an instance \ccc{v} of type \ccc{Vector}. 
\ccc{v} is initialized to the three-dimensional vector 
$(a,b,c)$. 
}

\ccConstructor{Vector(NT a, NT b, NT c, NT d)}{
creates an instance \ccc{v} of type \ccc{Vector}; 
\ccc{v} is initialized to the four-dimensional vector 
$(a,b,c,d)$. 
}

\ccSetTwoOfThreeColumns{3cm}{3cm}

\ccOperations

\ccMethod{int dimension() ;}{returns the dimension of \ccc{v}. }

\ccMethod{bool is_zero() ;}{returns true iff \ccc{v} is the zero
  vector.}


\ccMethod{NT& operator[](int i) ;}{returns $i$-th component of \ccc{v}.
  \ccPrecond $0\le i \le \ccc{v.dimension()-1}$.  }


\ccMethod{ Vector& operator+=(const Vector& v1);}{Addition plus
  assignment. \ccPrecond \ccc{v.dimension() == v1.dimension()}.  }


\ccMethod{ Vector& operator-=(const Vector& v1);}{Subtraction plus
  assignment. \ccPrecond \ccc{v.dimension() == v1.dimension()}.  }

\ccMethod{ Vector& operator*=(const NT& s);}{Scalar multiplication
  plus assignment.  }

\ccMethod{ Vector& operator/=(const NT& s);}{Scalar division plus
  assignment.  }


\ccMethod{Vector  operator+(const Vector& v1) ;}{Addition. \ccPrecond
  \ccc{v.dimension() == v1.dimension()}.  }


\ccMethod{Vector  operator-(const Vector& v1) ;}{Subtraction. \ccPrecond
  \ccc{v.dimension() = v1.dimension()}.  }

\ccMethod{NT operator*(const Vector& v1) ;}{Inner Product. \ccPrecond
  \ccc{v.dimension() = v1.dimension()}.  }

\ccMethod{Vector  operator-() ;}{Negation.}

Component access is also possible via \ccc{iterator} and
\ccc{const_iterator} by the standard iterator range access operations
\ccc{begin()}, \ccc{end()}.


\ccFunction{Vector operator*(const NT& r, const Vector& v) ;}{
  Componentwise multiplication with number $r$.}

\ccFunction{Vector operator*(const Vector& v, const NT& r) ;}{
  Componentwise multiplication with number $r$.}

\end{ccRefConcept}


