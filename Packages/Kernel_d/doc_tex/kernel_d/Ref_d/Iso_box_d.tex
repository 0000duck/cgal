\begin{ccRefClass} {Iso_box_d<Kernel>}

\ccDefinition  An object $b$ of the data type \ccRefName\ is an
iso-box in the Euclidean space $\E^d$ with edges parallel to the 
axes of the coordinate system.
 
\ccCreation
\ccCreationVariable{b}

\ccConstructor{Iso_box_d(const Point_d<Kernel> &p, 
                            const Point_d<Kernel> &q);}
            {introduces an iso-oriented iso-box \ccVar\ with diagonal
             opposite vertices $p$ and $q$. 
}

\ccHidden \ccConstructor{Iso_box_d(const Iso_box_d<Kernel> &b2);}
            {copy constructor.}

\ccOperations

\ccMethod{bool operator==(const Iso_box_d<Kernel> &b2) const;}
       {Test for equality: two iso-oriented cuboid are equal, iff their
        lower left and their upper right vertices are equal.}

\ccMethod{bool operator!=(const Iso_box_d<Kernel> &b2) const;}
       {Test for inequality.}


\ccMethod{Point_3<Kernel> min() const;}
       {returns the smallest vertex of \ccVar.}


\ccMethod{Point_3<Kernel> max() const;}
       {returns the largest vertex of \ccVar.}


\ccMethod{ Kernel::FT  min_coord(int i) const;}
         {returns $i$-th \ccHtmlNoLinksFrom{Cartesian} coordinate of
          the smallest vertex of \ccVar. 
%          (\ccc{min_coord(0) == xmin()}; \ccc{min_coord{2} == zmin()})
          \ccPrecond $0 \leq i \leq d$.}

\ccMethod{ Kernel::FT  max_coord(int i) const;}
         {returns $i$-th \ccHtmlNoLinksFrom{Cartesian} coordinate of
          the largest vertex of \ccVar. 
%          (\ccc{max_coord(0) == xmax()}; \ccc{max_coord{2} == zmax()})
          \ccPrecond $0 \leq i \leq d$.}

\ccPredicates

\ccMethod{bool is_degenerate() const;}
       {%the iso-oriented cuboid 
        \ccVar\ is degenerate, if all vertices
        are collinear.}

\ccMethod{Bounded_side bounded_side(const Point_3<Kernel> &p) const;}
       {returns either \ccStyle{ON_UNBOUNDED_SIDE},
        \ccStyle{ON_BOUNDED_SIDE}, or the constant
        \ccStyle{ON_BOUNDARY}, 
        depending on where point $p$ is.}

\ccMethod{bool has_on_boundary(const Point_3<Kernel> &p) const;}
       {}
\ccGlue
\ccMethod{bool has_on_bounded_side(const Point_3<Kernel> &p) const;}
       {}
\ccGlue
\ccMethod{bool has_on_unbounded_side(const Point_3<Kernel> &p) const;}
       {}

\ccHeading{Miscellaneous}

\ccMethod{Kernel::FT volume() const;}
       {returns the volume of \ccVar. }

\ccSeeAlso
Kernel::IsoBox\_d. \\

\end{ccRefClass} 
