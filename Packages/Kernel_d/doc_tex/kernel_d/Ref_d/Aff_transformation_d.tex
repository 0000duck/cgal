% begin cgal manual page

\begin{ccRefClass}{Aff_transformation_d<R>}\ccCreationVariable{t}

\ccDefinition

An instance of the data type \ccc{Aff_transformation_d<R>} is an
affine transformation of $d$-dimensional space. It is specified by a
square matrix $M$ of dimension $d + 1$. All entries in the last row of
\ccc{M} except the diagonal entry must be zero; the diagonal entry
must be non-zero.  A point $p$ with homogeneous coordinates $(p[0],
\ldots, p[d])$ can be transformed into the point \ccc{p.transform(A)}
= $Mp$, where \ccc{A} is an affine transformation created from \ccc{M}
by the constructors below.

\ccSetOneOfTwoColumns{4cm}

\ccTypes

\ccNestedType{RT}{the ring type.  }

\ccNestedType{FT}{the field type.  }

\ccNestedType{LA}{the linear algebra layer.  }

\ccNestedType{Matrix}{the matrix type.  }

\ccSetOneOfTwoColumns{3cm}

\ccCreation

\ccConstructor{Aff_transformation_d<R>()}{introduces some
transformation.}

\ccConstructor{Aff_transformation_d<R>(int d,
  Identity_transformation)}{introduces the identity transformation in
  $d$-dimensional space. }

\ccConstructor{Aff_transformation_d<R>(Matrix M)}{introduces the
  transformation of $d$-space specified by matrix $M$. \ccPrecond
  \ccc{M} is a square matrix of dimension $d + 1$ where entries in the
  last row of \ccc{M} except the diagonal entry must be zero; the
  diagonal entry must be non-zero.}

\ccConstructor{template <typename Forward_iterator>
  Aff_transformation_d<R>(Scaling, Forward_iterator start,
  Forward_iterator end)}{introduces the transformation of $d$-space
  specified by a diagonal matrix with entries \ccc{set [start,end)} on
  the diagonal (a scaling of the space). \ccPrecond \ccc{set
    [start,end)} is a vector of dimension $d+1$.  }

\ccConstructor{Aff_transformation_d<R>(Translation, Vector_d<R>
  v)}{introduces the translation by vector $v$.  }

\ccConstructor{Aff_transformation_d<R>(int d, Scaling, RT num, RT
  den)}{returns a scaling by a scale factor \ccc{num/den}.  \ccPrecond
  \ccc{den !=0 }.}

\ccConstructor{Aff_transformation_d<R>(int d, Rotation, RT sin_num, RT
  cos_num, RT den, int e1 = 0, int e2 = 1)}{returns a planar rotation
  with sine and cosine values \ccc{sin_num/den} and \ccc{cos_num/den}
  in the plane spanned by the base vectors $b_{e1}$ and $b_{e2}$ in
  $d$-space. Thus the default use delivers a planar rotation in the
  $x$-$y$ plane. \ccPrecond $\ccc{sin_num}^2 + \ccc{cos_num}^2 =
  \ccc{den}^2$ and $0 \leq e_1 < e_2 < d$.  \ccPrecond \ccc{den != 0}
  }

\ccConstructor{Aff_transformation_d<R>(int d, Rotation, Direction_d<R>
  dir, RT num, RT den, int e1 = 0, int e2 = 1)}{returns a planar
  rotation within a two-dimensional linear subspace. The subspace is
  spanned by the base vectors $b_{e1}$ and $b_{e2}$ in $d$-space.  The
  rotation parameters are given by the $2$-dimensional direction
  \ccc{dir}, such that the difference between the sines and cosines of
  the rotation given by \ccc{dir} and the approximated rotation are at
  most \ccc{num/den} each.\\ \ccPrecond \ccc{dir.dimension()==2},
  \ccc{!dir.is_degenerate()} and \ccc{num < den} is positive, \ccc{den
    != 0}, $0 \leq e_1 < e_2 < d$. }

\ccSetTwoOfThreeColumns{5cm}{3cm}

\ccOperations

\ccMethod{int dimension() ;}{the dimension of the underlying space }

\ccMethod{const Matrix& matrix() ;}{returns the transformation matrix
  }

\ccMethod{Aff_transformation_d<R> inverse() ;}{returns the inverse
  transformation.  \ccPrecond \ccc{t.matrix()} is invertible.  }

\ccMethod{Aff_transformation_d<R> operator*(const
  Aff_transformation_d<R>& s) ;}{composition of transformations. Note
  that transformations are not necessarily commutative. \ccc{t*s} is
  the transformation which transforms first by \ccc{t} and then by
  \ccc{s}.  }

\ccImplementation


Affine Transformations are implemented by matrices of number type
\ccc{RT} as a handle type. All operations like creation,
initialization, input and output on a transformation $t$ take time
$O(\ccc{t.dimension()}^2)$. \ccc{dimension()} takes constant time.
The operations for inversion and composition have the cubic costs of
the used matrix operations. The space requirement is
$O(\ccc{t.dimension()}^2)$.



\end{ccRefClass}


