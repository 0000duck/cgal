\begin{ccRefConcept}{Kernel_d}

The concept of a {\em kernel} is defined by a set of requirements on
the provision of certain types and access member functions to create 
objects of these types. The types are function object classes to be used
within the algorithms and data structures in the basic library of \cgal. 
This allows you to use any model of a kernel as a traits class in 
the \cgal\ algorithms and data structures, unless they require types 
beyond those provided by a kernel. 

\ccc{Kernel_d} subsumes the concept of a {\em $d$-dimensional kernel}.

A kernel provides types, construction objects, and generalized
predicates.  The former replace constructors of the kernel classes and
constructive procedures in the kernel. There are also function objects
replacing operators, especially for equality testing.

\ccCreationVariable{kernel}

\ccTypes

\ccNestedType{FT}{a number type that is a model for \ccc{FieldNumberType}}
\ccGlue
\ccNestedType{RT}{a number type that is a model for \ccc{RingNumberType}}

\ccHeading{Geometric Objects}

\ccNestedType{Point_d}{}
\ccGlue
\ccNestedType{Vector_d}{}
\ccGlue
\ccNestedType{Direction_d}{}
\ccGlue
\ccNestedType{Hyperplane_d}{}
\ccGlue
\ccNestedType{Line_d}{}
\ccGlue
\ccNestedType{Ray_d}{}
\ccGlue
\ccNestedType{Segment_d}{}
\ccGlue
\ccNestedType{Sphere_d}{}
\ccGlue
\ccNestedType{Aff_transformation_d}{}

\ccHeading{Constructors}

\ccNestedType{Construct_point_d}{}
\ccGlue
\ccNestedType{Construct_vector_d}{}
\ccGlue
\ccNestedType{Construct_direction_d}{}
\ccGlue
\ccNestedType{Construct_hyperplane_d}{}
\ccGlue
\ccNestedType{Construct_segment_d}{}
\ccGlue
\ccNestedType{Construct_line_d}{}
\ccGlue
\ccNestedType{Construct_ray_d}{}
\ccGlue
\ccNestedType{Construct_sphere_d}{}
\ccGlue
\ccNestedType{Construct_aff_transformation_d}{}
\ccGlue

\ccHeading{Generalized Predicates}

\ccNestedType{Affine_rank_d}{}
\ccGlue
\ccNestedType{Affinely_independent_d}{}
\ccGlue
\ccNestedType{Barycentric_coordinates_d}{}
\ccGlue
\ccNestedType{Center_of_sphere_d}{}
\ccGlue
\ccNestedType{Compare_lexicographically_d}{}
\ccGlue
\ccNestedType{Component_accessor_d}{}
\ccGlue
\ccNestedType{Contained_in_affine_hull_d}{}
\ccGlue
\ccNestedType{Contained_in_linear_hull_d}{}
\ccGlue
\ccNestedType{Contained_in_simplex_d}{}
\ccGlue
\ccNestedType{Equal_d}{}
\ccGlue
\ccNestedType{Has_on_positive_side_d}{}
\ccGlue
\ccNestedType{Intersect_d}{}
\ccGlue
\ccNestedType{Less_lexicographically_d}{}
\ccGlue
\ccNestedType{Less_or_equal_lexicographically_d}{}
\ccGlue
\ccNestedType{Lift_to_paraboloid_d}{}
\ccGlue
\ccNestedType{Linear_base_d}{}
\ccGlue
\ccNestedType{Linear_rank_d}{}
\ccGlue
\ccNestedType{Linearly_independent_d}{}
\ccGlue
\ccNestedType{Midpoint_d}{}
\ccGlue
\ccNestedType{Orientation_d}{}
\ccGlue
\ccNestedType{Oriented_side_d}{}
\ccGlue
\ccNestedType{Orthogonal_vector_d}{}
\ccGlue
\ccNestedType{Point_of_sphere_d}{}
\ccGlue
\ccNestedType{Point_to_vector_d}{}
\ccGlue
\ccNestedType{Position_on_line_d}{}
\ccGlue
\ccNestedType{Project_along_d_axis_d}{}
\ccGlue
\ccNestedType{Side_of_bounded_sphere_d}{}
\ccGlue
\ccNestedType{Side_of_oriented_sphere_d}{}
\ccGlue
\ccNestedType{Squared_distance_d}{}
\ccGlue
\ccNestedType{Value_at_d}{}
\ccGlue
\ccNestedType{Vector_to_point_d}{}
\ccGlue

\ccOperations

The following member functions return function objects of the types
listed above. The name of the access function is the name of the type
returned with an \ccc{_object} suffix and no capital letter at the
beginning. We only give two examples to show the scheme. For the
functors \ccc{Construct_point_d} and \ccc{Orientation_d} the
corresponding functions are:

\ccTexHtml{\ccSetThreeColumns{Kernel::Construct_point_d}{}{%
\hspace*{4.5cm}}}{}
\setlength{\parskip}{0pt}
 
\ccMemberFunction{Kernel::Construct_point_d construct_point_d_object() const;}{}
\ccGlue
\ccMemberFunction{Kernel::Orientation_d orientation_d_object() const;}{}

\ccHasModels

\ccc{Cartesian_d<FieldNumberType>}, \ccc{Homogeneous_d<RingNumberType>} 

\end{ccRefConcept} 
