\section{Introduction}

A subset $S \subseteq \R^d$ is convex if for any two points $p$ and $q$
in the set the line segment with endpoints $p$ and $q$ is contained
in $S$. The convex hull\ccIndexMainItemDef{convex hull} of a set $S$
is the smallest convex set containing
$S$. The convex hull of a set of points $P$ is a convex
polytope with vertices in $P$. A point in $P$ is an extreme point
(with respect to $P$)\ccIndexMainItemDef{extreme point} if it is a vertex of
the convex hull of $P$.  A set of points is said to be strongly convex %
\ccIndexMainItemDef{strongly convex} if it consist of only extreme points.

This chapter describes the class provided in \cgal\ for producing convex 
hull in arbitrary dimensions. There is an intimate relationship between 
the Delaunay triangulation of a point set $S$ and the convex hull of 
\ccc{lift(S)}: The nearest site
Delaunay triangulation is the projection of the lower hull and the
furthest site Delaunay triangulation is the upper hull.  Here we also 
describe the companion class to the convex hull class that computes 
nearest and furthest site Delaunay triangulations.

