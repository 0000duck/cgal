% begin cgal manual page

\begin{ccRefClass}{Ray_d<R>}\ccCreationVariable{r}

\ccDefinition


An instance of data type \ccc{Ray_d} is a ray in $d$-dimensional
Euclidian space. It starts in a point called the source of \ccc{r} and
it goes to infinity. 



\ccSetOneOfTwoColumns{4cm}

\ccTypes

\ccNestedType{R}{the representation type. 
}

\ccNestedType{RT}{the ring type. 
}

\ccNestedType{FT}{the field type. 
}

\ccNestedType{LA}{the linear algebra layer. 
}

\ccSetOneOfTwoColumns{3cm}

\ccCreation

\ccConstructor{Ray_d<R>(int d = 0)}{introduces some ray in $d$-dimensional space  
}

\ccConstructor{Ray_d<R>(Point_d<R> p, Point_d<R> q)}{introduces a ray through \ccc{p} and \ccc{q} and starting at \ccc{p}.
\ccPrecond $p$ and $q$ are distinct and have the same dimension.  
}

\ccConstructor{Ray_d<R>(Point_d<R> p, Direction_d<R> dir)}{introduces a ray starting in \ccc{p} with direction \ccc{dir}.
\ccPrecond \ccc{p} and \ccc{dir} have the same dimension and \ccc{dir} is not
trivial. 
}

\ccConstructor{Ray_d<R>(Segment_d<R> s)}{introduces a ray through \ccc{s.source()} and \ccc{s.target()} and 
starting at \ccc{s.source()}. \ccPrecond $s$ is not trivial.  
}

\ccSetTwoOfThreeColumns{3cm}{3cm}

\ccOperations

\ccMethod{int dimension() ;}{returns the dimension of the underlying space. 
}

\ccMethod{Point_d<R> source() ;}{returns the source point of \ccc{r}.  
}

\ccMethod{Point_d<R> point(int i) ;}{returns a point on \ccc{r}. \ccc{point(0)} is the source.
\ccc{point(i)}, with $i>0$, is different from the source. \ccPrecond $i
\geq 0$. 
}

\ccMethod{Direction_d<R> direction() ;}{returns the direction of \ccc{r}.  
}

\ccMethod{ Line_d<R> supporting_line() ;}{returns the supporting line of \ccc{r}. 
}

\ccMethod{Ray_d<R> opposite() ;}{returns the ray with direction opposite to \ccc{r}
and starting in \ccc{source}. 
}

\ccMethod{Ray_d<R> transform(const Aff_transformation_d<R>& t) ;}{returns $t(l)$.  
}

\ccMethod{Ray_d<R> operator+(const Vector_d<R>& v) ;}{returns \ccc{r+v}, i.e., \ccc{r} translated by vector $v$. 
}

\ccMethod{bool has_on(const Point_d<R>& p) ;}{A point is on \ccc{r}, iff it is equal to the source of \ccc{r}, or if it is
in the interior of \ccc{r}. 
}

\ccHeading{Non-Member Functions} 



\ccFunction{bool parallel(const Ray_d<R>& r1, const Ray_d<R>& r2) ;}{returns true if the unoriented supporting lines of \ccc{r1} and \ccc{r2}
are parallel and false otherwise.  
}

\ccImplementation


Rays are implemented by a pair of points as an item type.  All
operations like creation, initialization, tests, direction
calculation, input and output on a ray $r$ take time
$O(\ccc{r.dimension()})$. \ccc{dimension()}, coordinate and point access, and
identity test take constant time. The space requirement is
$O(\ccc{r.dimension()})$. 



\end{ccRefClass}


