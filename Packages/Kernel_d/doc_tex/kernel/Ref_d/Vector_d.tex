% begin cgal manual page

\begin{ccRefClass}{Vector_d<R>}\ccCreationVariable{v}

\ccDefinition


An instance of data type \ccc{Vector_d<R>} is a vector of Euclidean space in
dimension $d$. A vector $r = (r_0,\ldots,r_{ d - 1})$ can be represented
in homogeneous coordinates $(h_0,\ldots,h_d)$ of number type \ccc{RT},
such that $r_i = h_i/h_d$ which is of type \ccc{FT}. We call the
$r_i$'s the Cartesian coordinates of the vector. The homogenizing
coordinate $h_d$ is positive.

This data type is meant for use in computational geometry. It realizes
free vectors as opposed to position vectors (type \ccc{Point_d}). The
main difference between position vectors and free vectors is their
behavior under affine transformations, e.g., free vectors are
invariant under translations. 



\ccSetOneOfTwoColumns{4cm}

\ccTypes

\ccNestedType{RT}{the ring type. 
}

\ccNestedType{FT}{the field type. 
}

\ccNestedType{LA}{the linear algebra layer. 
}

\ccNestedType{Cartesian_const_iterator}{a read-only iterator for the Cartesian coordinates. 
}

\ccNestedType{Homogeneous_const_iterator}{a read-only iterator for the homogeneous coordinates. 
}

\ccNestedType{Base_vector}{construction tag. 
}

\ccSetOneOfTwoColumns{4cm}

\ccCreation

\ccConstructor{Vector_d<R>(int d = 0)}{introduces a variable \ccc{v} of type \ccc{Vector_d<R>} in 
$d$-dimensional space. 
}

\ccConstructor{Vector_d<R>(int d, Null_vector NULL_VECTOR)}{introduces the zero vector \ccc{v} of type \ccc{Vector_d<R>} in 
$d$-dimensional space. 
}

\ccConstructor{template <class InputIterator>
Vector_d<R>(int d, InputIterator first, InputIterator last)}{introduces a variable \ccc{v} of type \ccc{Vector_d<R>} in dimension \ccc{d}.
If \ccc{size [first,last) == d} this creates a vector with Cartesian coordinates 
\ccc{set [first,last)}. If \ccc{size [first,last) == p+1} the range specifies the
homogeneous coordinates $\ccc{H = set [first,last)} = (\pm h_0, \pm h_1, \ldots,
\pm h_d)$ where the sign chosen is the sign of $h_d$.
\ccPrecond \ccc{d} is nonnegative, \ccc{[first,last)} has \ccc{d} or \ccc{d+1} elements where the
last has to be non-zero, and the value type of \ccc{InputIterator} is \ccc{RT}. 
}

\ccConstructor{template <class InputIterator>
Vector_d<R>(int d, InputIterator first, InputIterator last, RT D)}{introduces a variable \ccc{v} of type \ccc{Vector_d<R>}
in dimension \ccc{d} initialized to the vector with homogeneous 
coordinates as defined by \ccc{H = set [first,last)} and \ccc{D}:
$(\pm \ccc{H[0]}, \pm\ccc{H[1]}, \ldots, \pm\ccc{H[d-1]}, \pm\ccc{D})$. The sign chosen 
is the sign of $D$. \ccPrecond \ccc{D} is non-zero, the iterator range defines 
a $d$-tuple of \ccc{RT}, and the value type of \ccc{InputIterator} is \ccc{RT}.  
}

\ccConstructor{Vector_d<R>(Base_vector, int d, int i)}{returns a variable \ccc{v} of type \ccc{Vector_d<R>} initialized  
to the $i$-th base vector of dimension $d$.  
}

\ccConstructor{Vector_d<R>(RT x, RT y, RT w = 1)}{introduces a variable \ccc{v} of type \ccc{Vector_d<R>} in 
$2$-dimensional space.  
}

\ccConstructor{Vector_d<R>(RT x, RT y, RT z, RT w)}{introduces a variable \ccc{v} of type \ccc{Vector_d<R>} in 
$3$-dimensional space.  
}

\ccSetTwoOfThreeColumns{5cm}{3cm}

\ccOperations

\ccMethod{int dimension() ;}{returns the dimension of \ccc{v}.  
}

\ccMethod{FT cartesian(int i) ;}{returns the $i$-th Cartesian coordinate of \ccc{v}. 
   \ccPrecond $0 \leq i < d$. 
}

\ccMethod{FT operator[](int i) ;}{returns the $i$-th Cartesian coordinate of \ccc{v}.
   \ccPrecond $0 \leq i < d$. 
}

\ccMethod{RT homogeneous(int i) ;}{returns the $i$-th homogeneous coordinate of \ccc{v}. 
   \ccPrecond $0 \leq i \leq d$. 
}

\ccMethod{FT squared_length() ;}{returns the square of the length of \ccc{v}.  
}

\ccMethod{Cartesian_const_iterator cartesian_begin() ;}{returns an iterator pointing to the zeroth Cartesian coordinate 
of \ccc{v}.  
}

\ccMethod{Cartesian_const_iterator cartesian_end() ;}{returns an iterator pointing beyond the last Cartesian coordinate 
of \ccc{v}.  
}

\ccMethod{Homogeneous_const_iterator homogeneous_begin() ;}{returns an iterator pointing to the zeroth homogeneous coordinate 
of \ccc{v}.  
}

\ccMethod{Homogeneous_const_iterator homogeneous_end() ;}{returns an iterator pointing beyond the last homogeneous
coordinate of \ccc{v}.  
}

\ccMethod{ Direction_d<R> direction() ;}{returns the direction of \ccc{v}.  
}

\ccMethod{Vector_d<R> transform(const Aff_transformation_d<R>& t) ;}{returns $t(v)$.  
}

\ccHeading{Arithmetic Operators, Tests and IO} 



\ccMethod{Vector_d<R>& operator*=(const RT& n)  ;}{multiplies all Cartesian coordinates by \ccc{n}. 
}

\ccMethod{Vector_d<R>& operator*=(const FT& r)  ;}{multiplies all Cartesian coordinates by \ccc{r}. 
}

\ccMethod{Vector_d<R> operator/(const RT& n) ;}{returns the vector with Cartesian coordinates 
$v_i/n, 0 \leq i < d$. 
}

\ccMethod{Vector_d<R> operator/(const FT& r) ;}{returns the vector with Cartesian coordinates 
$v_i/r, 0 \leq i < d$. 
}

\ccMethod{Vector_d<R>& operator/=(const RT& n)  ;}{divides all Cartesian coordinates by \ccc{n}. 
}

\ccMethod{Vector_d<R>& operator/=(const FT& r)  ;}{divides all Cartesian coordinates by \ccc{r}. 
}

\ccMethod{FT  operator* (const Vector_d<R>& w) ;}{inner product, i.e., $\sum_{ 0 \le i < d } v_i w_i$, 
where $v_i$ and $w_i$ are the Cartesian coordinates of $v$ and $w$ 
respectively.  
}

\ccMethod{Vector_d<R> operator+(const Vector_d<R>& w) ;}{returns the vector with Cartesian coordinates 
$v_i+w_i, 0 \leq i < d$. 
}

\ccMethod{Vector_d<R>& operator+=(const Vector_d<R>& w)  ;}{addition plus assignment. 
}

\ccMethod{Vector_d<R> operator-(const Vector_d<R>& w) ;}{returns the vector with Cartesian coordinates 
$v_i-w_i, 0 \leq i < d$. 
}

\ccMethod{Vector_d<R>& operator-=(const Vector_d<R>& w)  ;}{subtraction plus assignment. 
}

\ccMethod{Vector_d<R> operator-() ;}{returns the vector in opposite direction. 
}

\ccMethod{bool  is_zero() ;}{returns true if \ccc{v} is the zero vector.  
}

\ccHeading{Downward compatibility}
We provide all operations of the lower dimensional interface \ccc{x()}, \ccc{y()},
\ccc{z()}, \ccc{hx()}, \ccc{hy()}, \ccc{hz()}, \ccc{hw()}. 



\ccFunction{template <class RT, class LA>
template <class RT, class LA> Vector_d<R> operator*(const RT& n, const Vector_d<R>& v)  ;}{returns the vector with Cartesian coordinates $n v_i$. 
}

\ccFunction{template <class RT, class LA>
template <class RT, class LA> Vector_d<R> operator*(const FT& r, const Vector_d<R>& v) ;}{returns the vector with Cartesian coordinates 
$r v_i, 0 \leq i < d$. 
}

\ccImplementation


Vectors are implemented by arrays of variables of type \ccc{RT}.  All
operations like creation, initialization, tests, vector arithmetic,
input and output on a vector $v$ take time $O(\ccc{v.dimension()})$. 
coordinate access, \ccc{dimension()} and conversions
take constant time.  The space requirement of a vector is
$O(\ccc{v.dimension()})$. 



\end{ccRefClass}


