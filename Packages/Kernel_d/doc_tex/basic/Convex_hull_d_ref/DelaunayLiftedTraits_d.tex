\begin{ccRefConcept}{DelaunayLiftedTraits_d}

\ccDefinition Requirements of the second traits class to be used with the
class \ccc{Delaunay_d}.

\ccTypes
\ccAutoIndexingOff
\ccNestedType{Point_d}{the dD point type on which the Delaunay algorithm 
operates}
\ccGlue
\ccNestedType{Hyperplane_d}{a dD plane}
\ccGlue
\ccNestedType{Vector_d}{a dD vector}
\ccGlue
\ccNestedType{Ray_d}{a dD ray}
\ccGlue
\ccNestedType{RT}{a arithmetic ring type}

\ccNestedType{Construct_vector_d}{Function object type that provides
\ccc{Vector_d operator()(int d, CGAL::Null_vector)}, which constructs
and returns the null vector.}

\ccNestedType{Construct_hyperplane_d}{Function object type that
provides \ccc{Hyperplane_d operator()(ForwardIterator first,
ForwardIterator last, Point_d p, CGAL::Oriented_side side)}, which
constructs and returns a hyperplane passing through the points in
\ccc{tuple[first,last)} and oriented such that \ccc{p} is on the side
\ccc{side} of the returned hyperplane. When
\ccc{side==ON_ORIENTED_BOUNDARY} then any hyperplane containing the
tuple is returned.}

\ccNestedType{Vector_to_point_d}{Function object type that provides
\ccc{Point_d operator()(Vector_d v)}, which constructs and
returns the point defined by $0+v$.}

\ccNestedType{Point_to_vector_d}{Function object type that provides
\ccc{Vector_d operator()(Point_d v)}, which constructs and returns the
vector defined by $p-0$.}

\ccNestedType{Orientation_d}{Function object type that provides
\ccc{Orientation operator()(ForwardIterator first,
ForwardIterator last)}, which determines the orientation of the
points \ccc{tuple[first,last)}.}

\ccNestedType{Orthogonal_vector_d}{Function object type that provides
\ccc{Vector_d operator()(Hyperplane_d h)}, which constructs and
returns a vector orthogonal to \ccc{h} and pointing from the boundary
into its positive halfspace.}

\ccNestedType{Oriented_side_d}{Predicate object type that provides
\ccc{Oriented_side operator()(Hyperplane_d h, Point_d p)}, which
determines the oriented side of \ccc{p} with respect to \ccc{h}.}

\ccNestedType{Has_on_positive_side_d}{Predicate object type that
provides \ccc{bool operator()(Hyperplane_d h, Point_d p)}, which
return true iff \ccc{p} lies in the positive halfspace determined by
\ccc{h}.}

\ccNestedType{Affinely_independent_d}{Predicate object type that provides
\ccc{bool operator()(ForwardIterator first, ForwardIterator last)}, which 
determines if the points \ccc{tuple[first,last)} are affinely independent.}

\ccNestedType{Contained_in_simplex_d}{Predicate object type that
provides \ccc{bool operator()(ForwardIterator first, ForwardIterator
last, Point_d p)}, which determines if \ccc{p} is contained in 
the closed simplex defined by the points in \ccc{tuple[first,last)}.}

\ccNestedType{Contained_in_affined_hull_d}{Predicate object type that
provides \ccc{bool operator()(ForwardIterator first, ForwardIterator
last, Point_d p)}, which determines if \ccc{p} is contained in 
the affine hull of the points in \ccc{tuple[first,last)}.}

\ccNestedType{Intersect_d}{Predicate object type that provides
\ccc{Object operator()(Ray_d r, Hyperplane_d h)}, which determines if
\ccc{r} and \ccc{h} intersect and returns the corresponding
polymorphic object.}

The previous requirements are all identical to the concept
\ccc{ConvexHullTraits_d}. The Delaunay class adds the following
requirements.

\ccNestedType{Project_along_d_axis_d}{Predicate object type that
provides \ccc{DelaunayTraits_d::Point_d operator()(Point_d p)}, which
determines the $d-1$-dimensional point from the $d$-dimensional point
$p$ while discarding the last coordinate.}

\ccNestedType{Lift_to_paraboloid_d}{Predicate object type that
provides \ccc{Point_d operator()(DelaunayTraits_d::Point_d p)}, which
determines the $d$-dimensional point from the $d-1$-dimensional point
$p$ while lifting it to the paraboloid of revolution.}

\ccNestedType{Component_accessor_d}{Predicate object type that
provides \ccc{RT homogeneous(Vector_d v,int i)} and \ccc{int
dimension(Vector_d v)}, where the former determines the $i$th
coordinate of $v$ and the latter the dimension of $v$.}

\ccCreation
\ccCreationVariable{traits}

A default constructor and copy constructor is required.

\ccOperations

For each of the above function and predicate object types,
\ccc{Func_obj_type}, a function must exist with the name
\ccc{func_obj_type_object} that creates an instance of the function or
predicate object type.  For example:

\ccMethod{Construct_vector_d construct_vector_d_object();}{}

\ccAutoIndexingOn

\ccHasModels
% cross reference between manuals not currently possible
%\ccRefIdfierPage{CGAL::Cartesian_d<FT,LA>}\\
%\ccRefIdfierPage{CGAL::Homogeneous_d<RT,LA>} \\
\ccc{CGAL::Cartesian_d<FT,LA>}\\
\ccc{CGAL::Homogeneous_d<RT,LA>}

%\ccSeeAlso

\end{ccRefConcept}
