\begin{ccRefConcept}{AlgebraicKernel_2_2::PolynomialCircle_2_2}

\ccDefinition

The \ccc{PolynomialCircle_2_2} represents bivariate polynomials of degree up
to~2 capable of storing equations of circles, whose center has
coordinates of type \ccc{FT}, a model of \ccc{FieldNumberType}, as 
well as the square of the radius. 
\footnote{Of course the name looks bad. suggestions welcome}

\ccTypes

\ccc{RT} is supposed to be a model of \ccc{RingNumberType}, given 
by \ccc{Rational_traits<FT>}.\footnote{again...}

\ccCreation
\ccCreationVariable{p}

\footnote{to be done}

\ccAccessFunctions

\ccMethod{const FT & a();}{\ccc{x}-coordinate of the center of the circle.}
\ccGlue
\ccMethod{const FT & b();}{\ccc{y}-coordinate of the center of the circle.}
\ccGlue
\ccMethod{const FT & r_sq();}{Square radius of the center of the circle.}

\ccOperations

\ccFunction{bool operator ==(const PolynomialCircle_2_2 & p,
	const PolynomialCircle_2_2 & q);}{}

\ccHasModels

\footnote{to be done}

\ccSeeAlso

\ccRefIdfierPage{AlgebraicKernel_2_2::RootOf_2}\\
\ccRefIdfierPage{AlgebraicKernel_2_2}

\end{ccRefConcept}

\begin{ccRefConcept}{Algebraic_kernel_4_2::Polynomial_2_2}

\footnote{the indices look strange but there is indeed a logic: solve on two polynomials-2-2 gives roots-of-4...}

\ccDefinition

The \ccc{Polynomial_2_2} represents bivariate polynomials of degree up
to~2 whose coefficients are of type \ccc{RT}, a model of \ccc{RingNumberType}. 

\end{ccRefConcept}
