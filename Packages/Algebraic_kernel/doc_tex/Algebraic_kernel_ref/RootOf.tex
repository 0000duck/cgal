\begin{ccRefConcept}{AlgebraicKernel_2_2::RootOf_2}

\ccDefinition

The \ccc{RootOf_2} concept represents algebraic numbers of
degree up to~2 over a \ccc{RingNumberType} denoted as \ccc{RT}.

\ccc{FT} denotes the \ccc{FieldNumberType} constructed from \ccc{RT}.
\footnote{see \ccc{make_rational}, we need something to get FT from RT in a clean way.}

\ccCreation

\footnote{How to get RT...?}
\footnote{\ccc{make_root_of} to be removed from concept? solve instead} 

\ccFunction{RootOf_2 make_root_of_2(RT a, RT b, RT c, int i);}{Returns the \ccc{i}th root of equation $aX^2+bX+c=0$.}\footnote{numbering of roots from 0 or 1? to be checked}
\ccGlue
\ccFunction{RootOf_2 make_root_of_2(FT a, FT b, FT c, int i);}{Returns the \ccc{i}th root of equation $aX^2+bX+c=0$.}

\ccOperations

The comparison operators \ccc{==, !=, <, >, <=, >=} need to be
provided to compare elements of types \ccc{RootOf_2, RT} and \ccc{FT}. 

In addition, the following operations must be provided:

\def\ccTagRmEigenClassName{\ccFalse}

\ccFunction{RootOf_2 operator+(const RT &a, const RootOf_2 &r);}{}
\ccGlue
\ccFunction{RootOf_2 operator+(const FT &a, const RootOf_2 &r);}{}
\ccGlue
\ccFunction{RootOf_2 operator+(const RootOf_2 &r, const RT &a);}{}
\ccGlue
\ccFunction{RootOf_2 operator+(const RootOf_2 &r, const FT &a);}{}

and similarly for operators -, * and /.

\ccFunction{RootOf_2 square(const RootOf_2 & r);}{}

\def\ccTagRmEigenClassName{\ccTrue}

\ccHasModels

\ccc{double}, \ccc{Root_of_2<RT>}, etc \footnote{to be precised}

\ccSeeAlso

\ccRefIdfierPage{CGAL::Root_of_2<RT>}\\
\ccRefIdfierPage{CGAL::Root_of_traits_2<RT>}\\
\ccRefIdfierPage{AlgebraicKernel_2_2::PolynomialForCircles_2_2}\\
\ccRefIdfierPage{AlgebraicKernel_2_2}

\end{ccRefConcept}
