\chapter{Algebraic Kernel}

\textbf{Submission - Monique - with Sylvain's help...}

\begin{ccAdvanced}
As in Curved-kernel, I use the ``Advanced'' environment in this
document to distinguish between my current submission to the CGAL
editorial board and plans for the future, related to ACS. The
``Advanced'' parts will disappear if/when this is released.
\end{ccAdvanced}

%%%%%%%%%%%%%%%%%%%%%%%%%%%%%%%%%%%%%%%%%%%%%%%%%%%%%%%%%
\section*{Concepts}

\footnote{Same big question as for curved-kernel. Should the kernel here
be described as a concept or a class?}

\ccRefConceptPage{AlgebraicKernel_2_2}\\
\ccRefConceptPage{AlgebraicKernel_2_2::Polynomial_1_2}\\
\ccRefConceptPage{AlgebraicKernel_2_2::PolynomialCircle_2_2}\\
\ccRefConceptPage{AlgebraicKernel_2_2::RootOf_2}\\
\ccRefConceptPage{AlgebraicKernel_2_2::RootCircle_2_2}

\begin{ccAdvanced}
\ccRefConceptPage{AlgebraicKernel_4_2}\\
\ccRefConceptPage{AlgebraicKernel_4_2::RootOf_4}\\
\ccRefConceptPage{Algebraic_kernel_4_2::Polynomial_2_2}

The indices look strange but there is indeed a logic: solve on two
Polynomial-2-2's gives Root-of-4's...

General remark about the suffix \ccc{_d_v}: \ccc{_d} stands 
for the degree of the polynomials and the algebraic numbers, and
\ccc{_v} stands for the number of variables, which is analogous to the
dimension for CGAL geometric objects. I had already mentioned this in an
earlier version of this document (presented at the CGAL meeting at
INRIA in march 05), and it is consistent with what Menelaos proposed
later (CGAL meeting in Pisa in june 05) for a hierarchy of algebraic kernels.
\end{ccAdvanced}

	\subsubsection*{Functors} 

\ccRefConceptPage{AlgebraicKernel_2_2::ConstructPolynomialCircle_2_2}

\ccRefConceptPage{AlgebraicKernel_2_2::CompareX}\\
\ccRefConceptPage{AlgebraicKernel_2_2::CompareY}\\
\ccRefConceptPage{AlgebraicKernel_2_2::CompareXY}

\ccRefConceptPage{AlgebraicKernel_2_2::SignAt}

\ccRefConceptPage{AlgebraicKernel_2_2::Solve}

\ccRefConceptPage{AlgebraicKernel_2_2::XCriticalPoints}\\
\ccRefConceptPage{AlgebraicKernel_2_2::YCriticalPoints}


\begin{ccAdvanced}
\ccRefConceptPage{AlgebraicKernel_4_2::ConstructPolynomial_2_2}

\ccRefConceptPage{AlgebraicKernel_4_2::Solve}

etc
\end{ccAdvanced}

%%%%%%%%%%%%%%%%%%%%%%%%%%%%%%%%%%%%%%%%%%%%%%%%%%%%%%%%%
\section*{Classes}

\footnote{how does the linking work when concepts and classes have the same name...? example \ccc{Polynomial_2_2}}

\ccRefIdfierPage{CGAL::Algebraic_kernel_2_2<RT>}

\ccRefIdfierPage{CGAL::Polynomial_1_2<RT>}\\
\ccRefIdfierPage{CGAL::Polynomial_for_circles_2_2<FT>}

\ccRefIdfierPage{CGAL::Root_of_2<RT>}\\
\ccRefIdfierPage{CGAL::Root_for_circles_2_2<FT>}

\ccRefIdfierPage{CGAL::Root_of_traits_2<RT>}

\begin{ccAdvanced}
\ccRefIdfierPage{CGAL::Root_of_4<RT>}

\ccRefIdfierPage{CGAL::Root_of_traits_4<RT>}
\end{ccAdvanced}

%%%%%%%%%%%%%%%%%%%%%%%%%%%%%%%%%%%%%%%%%%%%%%%%%%%%%%%%%
\section*{Functions}

	\subsubsection*{Predicates}
\ccRefIdfierPage{CGAL::compare_x}\\
\ccRefIdfierPage{CGAL::compare_y}\\
\ccRefIdfierPage{CGAL::compare_xy}

	\subsubsection*{Constructions}

\ccRefIdfierPage{CGAL::make_root_of_2}

\ccRefIdfierPage{CGAL::solve}

\begin{ccAdvanced}
\ccRefIdfierPage{CGAL::make_root_of_4}
\end{ccAdvanced}
