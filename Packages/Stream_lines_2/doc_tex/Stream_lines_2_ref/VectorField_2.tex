% +------------------------------------------------------------------------+
% | Reference manual page: Field_2.tex
% +------------------------------------------------------------------------+
% | 08.12.2000   Author
% | Package: Stream_lines_2
% | 
\RCSdef{\RCSVectorFieldRev}{$Revision$}
\RCSdefDate{\RCSVectorFieldDate}{$Date$}
% |
%%RefPage: end of header, begin of main body
% +------------------------------------------------------------------------+


\begin{ccRefConcept}{VectorField_2}

%% \ccHtmlCrossLink{}     %% add further rules for cross referencing links
%% \ccHtmlIndexC[concept]{} %% add further index entries

\ccDefinition
  
The concept \ccRefName\ describes the set of requirements to be
fulfilled by any class used to instantiate the first template
parameter of the class \ccc{Stream_lines_2<VectorField_2,Integrator_2>}.
This concept provides the types of the geometric primitives used in
the placement of streamlines and some functions for answering
different queries.

\ccTypes
\ccNestedType{Vector_field_2}{The vector field type.}
\ccGlue
\ccNestedType{Geom_traits}{The traits class.}
\ccGlue
\ccNestedType{FT}{The scalar type.}
\ccGlue
\ccNestedType{Point_2}{The  point type.}
\ccGlue
\ccNestedType{Vector_2}{The vector  type.}
\ccGlue


\ccCreation
\ccCreationVariable{vfield}  %% choose variable name
\ccConstructor{VectorField_2();}{Any constructor has to allow the user to fill the vector values (i.e. assign a vector to each position in the domain).}
% \ccGlue

\ccHeading{Query Functions}

\ccMethod{CGAL::Iso_rectangle_2<Geom_traits> bbox();}{return
the bounding box of the whole domain.}
\ccGlue
\ccMethod{std::pair<Vector_2,FT> get_field(Point_2 p);}
{returns the vector field value and the local density.
\ccPrecond \ccc{is_in_domain(p)} must be true.}
\ccGlue
\ccMethod{bool is_in_domain(Point_2 p);}{returns true if the point p
is inside the domain boundaries, false otherwise.}
\ccGlue
\ccMethod{FT get_integration_step(Point_2 p);}
{returns the integration step at the point p (i.e. the distance between \ccc{p} and the next point in the polyli ne.).
\ccPrecond \ccc{is_in_domain(p)} must be true.}
\ccGlue

\ccHasModels
\ccc{CGAL::Regular_grid_2<StreamLinesTraits_2>} \\
\ccc{CGAL::Triangular_field_2<StreamLinesTraits_2>} \\

\end{ccRefConcept}

% +------------------------------------------------------------------------+
%%RefPage: end of main body, begin of footer
% EOF
% +------------------------------------------------------------------------+

