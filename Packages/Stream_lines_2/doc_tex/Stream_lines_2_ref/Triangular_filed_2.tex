%% +------------------------------------------------------------------------+
% | Reference manual page: Triangular_field_2.tex
% +------------------------------------------------------------------------+
% | 07.12.2004 Abdelkrim Mebarki
% | Package: Stream_lines_2
% | 
\RCSdef{\RCSTriangularfieldRev}{$Revision$}
\RCSdefDate{\RCSTriangularfieldDate}{$Date$}
% |
%%RefPage: end of header, begin of main body
% +------------------------------------------------------------------------+


\begin{ccRefClass}{Triangular_field_2<StreamLinesTraits_2>}  %% add template arg's if necessary

%% \ccHtmlCrossLink{}     %% add further rules for cross referencing links
%% \ccHtmlIndexC[class]{} %% add further index entries
\ccCreationVariable{triangular_field_2}
\ccDefinition
  
The class \ccRefName\ can be used to instantiate the field class of
the streamline placement.  The templated parameter \ccc{StreamLinesTraits_2} has to be
instantiated by a model of the concept \ccc{StreamLinesTraits_2}.\\
This class provides an empty 2D vector field defined on a triangulated domain.

\ccCreation
\ccCreationVariable{triangular_field_2}  %% choose variable name


\ccThree{Triangular_field_2<StreamLinesTraits>}{Triangular_field_2 triangular_field_2}{}
\ccThreeToTwo
\ccConstructor{Triangular_field_2( InputIterator1 first_point, InputIterator1 last_point, InputIterator2 first_vector)}
{		Defines the points in the range  
		$\left[\right.$\ccc{first}, \ccc{last}$\left.\right)$ 
		as the sample points of the grid, with the corresponding number of vectors started at \ccc{first_vector}.
\ccPrecond The \ccc{value_type} of \ccc{InputIterator1} is \ccc{Point}.
\ccPrecond The \ccc{value_type} of \ccc{InputIterator2} is \ccc{Vector}.}

\ccIsModel 

\ccc{VectorField_2} \\


\ccSeeAlso
\ccc{Regular_grid_2<StreamLinesTraits_2>} \\
\end{ccRefClass}

% +------------------------------------------------------------------------+
%%RefPage: end of main body, begin of footer
% EOF
% +------------------------------------------------------------------------+

