% +------------------------------------------------------------------------+
% | Reference manual page: Integrator_2.tex
% +------------------------------------------------------------------------+
% | 06.04.2000   Abdelkrim MEBARKI
% | Package: Stream_line_2
% | 
\RCSdef{\RCSIntegratorRev}{$Revision$}
\RCSdefDate{\RCSIntegratorDate}{$Date$}
% |
%%RefPage: end of header, begin of main body
% +------------------------------------------------------------------------+


\begin{ccRefFunctionObjectConcept}{Integrator_2}

%% \ccHtmlCrossLink{}     %% add further rules for cross referencing links
%% \ccHtmlIndexC[concept]{} %% add further index entries

\ccDefinition
  
The concept \ccRefName\ describes the set of requirements to be
fulfilled by any function object used to instantiate the second
template parameter of the class
\ccc{Stream_lines_2<VectorField_2,Integrator_2>}.  This concept
provides the operation that integrates a new point from a given point
with a predefined step, and according to a specified vector.

\ccTypes
\ccNestedType{FT}{The scalar type.}
\ccGlue
\ccNestedType{Point_2}{The  point type.}
\ccGlue
\ccNestedType{Vector_2}{The vector type.}
\ccGlue
\ccNestedType{Vector_field_2}{The vector field type.}
\ccGlue

\ccCreation
\ccCreationVariable{integ}  %% choose variable name
\ccConstructor{Integrator_2();}{only a default constructor is needed.}
\ccGlue

\ccOperations
{}
The following operations return the newly integrated point.
\ccMemberFunction{Point_2 operator()(Point_2 p, Vector_field_2 vector_field_2);}
{\ccPrecond \ccc{vector_field_2.is_in_domain(p)} must be true.}
\ccGlue
\ccMemberFunction{Point_2 operator()(Point_2 p, Vector_field_2 vector_field_2, FT integration_step);}
{\ccPrecond \ccc{vector_field_2.is_in_domain(p)} must be true.}
\ccGlue

\ccHasModels
\ccc{CGAL::Euler_integrator_2<VectorField_2>} \\
\ccc{CGAL::Runge_kutta_integrator_2<VectorField_2>} \\
\end{ccRefFunctionObjectConcept}

% +------------------------------------------------------------------------+
%%RefPage: end of main body, begin of footer
% EOF
% +------------------------------------------------------------------------+

