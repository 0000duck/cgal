% +------------------------------------------------------------------------+
% | Reference manual page: Stream_lines_2.tex
% +------------------------------------------------------------------------+
% | 29.03.2000   Abdelkrim MEBARKI
% | Package: Stream_lines_2
% | 
\RCSdef{\RCSStreamlinesRev}{$Revision$}
\RCSdefDate{\RCSStreamlinesDate}{$Date$}
% |
%%RefPage: end of header, begin of main body
% +------------------------------------------------------------------------+


\ccModifierCrossRefOff
\begin{ccRefClass}{Stream_lines_2<VectorField_2,Integrator_2>}  %% add template arg's if necessary

%% \ccHtmlCrossLink{}     %% add further rules for cross referencing links
%% \ccHtmlIndexC[class]{} %% add further index entries

\ccDefinition
  
The class \ccRefName\ is designed to handle a placement of streamlines
in a 2D domain according to a bidimensional vector field.\\

The class \ccRefName\ creates a placement of streamlines according to
a specified density and gives access to those streamlines via two
iterators over a container of iterators that provide access to the
streamline points.

\ccParameters
The class \ccRefName\ has two template parameters. The first parameter
\ccc{VectorField_2} has to be instantiated by a model of the concept
\ccc{VectorField_2}. The second parameter is the function object
\ccc{Integrator_2}, and has to be instantiated by a model of the concept
\ccc{Integrator_2}.

\ccTypes
\ccThree{typedef VectorField_2:::Point_iterator_x}{xxx_xxxx_xxx;}{}
\ccTypedef{typedef VectorField_2::Geom_traits Geom_traits;}{the traits class.}
\ccGlue
\ccTypedef{typedef VectorField_2::FT FT;}{the scalar type.}
\ccGlue
\ccTypedef{typedef VectorField_2::Point_2 Point_2;}{the point type.}
\ccGlue
\ccTypedef{typedef VectorField_2::Vector_2 Vector_2;}{the vector type.}

The class \ccRefName\ provides also two types for handling
streamlines:

\ccNestedType{Point_iterator_2;}{iterator of points with value type \ccc{Point_2}.}
\ccGlue
\ccNestedType{Stream_line_iterator_2;}{an iterator to visit the streamlines with value type \ccc{std::pair<Point_iterator_2, Point_iterator_2>}.}
\ccGlue



\ccCreation
\ccCreationVariable{streamlines_2}  %% choose variable name


\ccThree{Stream_lines_2<>}{t = Streamlines__2 Streamlines}{}
\ccThreeToTwo
\ccConstructor{Stream_lines_2(VectorField_2 vector_field_2, Integrator_2 integrator_2, FT
separating_distance, FT saturation_ratio);}
{Generates a streamline placement \ccVar.}

\ccHeading{Modifiers}

\ccMethod{void change_separating_distance(FT new_value);}{Modify the
separating distance.}
\ccGlue

\ccMethod{void change_saturation_ratio(FT new_value);}{Modify the
saturation ratio.}
\ccGlue

\ccMethod{void update();}{Update the placement after changing the
separating distance or the saturation ratio.}
\ccGlue

\ccHeading{Streamline iterators}

The following iterators allow to visit all the streamlines generated
by the constructor or the update function.

\ccThree{Stream_line_iterator}{streamline.begin()x}{}
\ccMethod{Stream_line_iterator begin() const;}{Starts at the first streamline}
\ccGlue
\ccMethod{Stream_line_iterator end() const;}{Past-the-end iterator}

\end{ccRefClass}
\ccModifierCrossRefOn

% +------------------------------------------------------------------------+
%%RefPage: end of main body, begin of footer
% EOF
% +------------------------------------------------------------------------+

