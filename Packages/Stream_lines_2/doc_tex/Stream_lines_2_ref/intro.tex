% +------------------------------------------------------------------------+
% | Reference manual page: Stream_lines_2.tex
% +------------------------------------------------------------------------+
% | 08.12.2004   Abdelkrim Mebarki
% | Package: Stream_lines_2
% | 
\RCSdef{\RCSStreamlinesRev}{$Revision$}
\RCSdefDate{\RCSStreamlinesDate}{$Date$}
% |
%%RefPage: end of header, begin of main body
% +------------------------------------------------------------------------+

%\clearpage
%\section{Reference pages for 2D placement of streamlines}
\chapter{2D Placement of Streamlines}
\label{ref_chapter_2D_Streamlines}

\ccChapterAuthor{Abdelkrim Mebarki}

\subsection*{Definitions}

Vector and direction fields are commonly used for modeling physical
phenomena, where a direction and magnitude, namely a vector is assigned to
each point inside a domain.\\

A streamline is a curve everywhere tangent to the field. It can be
considered as the path traced by an imaginary massless particle
dropped into a steady fluid flow described by the field.\\

A streamline is represented as a polyline iteratively elongated by
bidirectional numerical integration started from a seed point, until
it comes close to another streamline, hits the domain boundary, or
reaches a critical point.\\

The \ccc{Stream_lines_2} class consists of saturating the domain with
a set of tangential streamlines in accordance with a specified
density.\\

Streamlines are represented as containers of points, manipulated by an
iterator range of points, and the whole placement is accessible via an
iterator range of streamlines.\\

The main class in this package, the \ccc{Stream_lines_2} class of \cgal\ depends on two template
parameters.  The first template parameter stands for a class which
represents both the vector field and the visualisation domain with
operations on them, and should be instantiated by a model of the concept
\ccc{VectorField_2}.  The second template parameter stands for a
function object that ensures the numerical integration used to
construct the streamlines, and should be instantiated by a model of the concept
\ccc{Integrator_2}.

\subsection*{Concepts}
\ccRefConceptPage{StreamLinesTraits_2} \\
\ccRefConceptPage{Integrator_2} \\
\ccRefConceptPage{VectorField_2} \\

\subsection*{Classes}


\ccRefIdfierPage{CGAL::Stream_lines_2<VectorField_2,Integrator_2>}\\
\ccRefIdfierPage{CGAL::Euler_integrator_2<VectorField_2>}\\
\ccRefIdfierPage{CGAL::Runge_kutta_integrator_2<VectorField_2>}\\
\ccRefIdfierPage{CGAL::Regular_grid_2<StreamLinesTraits_2>}\\
\ccRefIdfierPage{CGAL::Triangular_field_2<StreamLinesTraits_2>}\\


% +------------------------------------------------------------------------+
%%RefPage: end of main body, begin of footer
% EOF
% +------------------------------------------------------------------------+
