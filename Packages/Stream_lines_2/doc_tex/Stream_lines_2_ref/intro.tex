% +------------------------------------------------------------------------+
% | Reference manual page: Stream_lines_2.tex
% +------------------------------------------------------------------------+
% | 08.12.2004   Abdelkrim Mebarki
% | Package: Stream_lines_2
% | 
\RCSdef{\RCSStreamlinesRev}{$Revision$}
\RCSdefDate{\RCSStreamlinesDate}{$Date$}
% |
%%RefPage: end of header, begin of main body
% +------------------------------------------------------------------------+

%\clearpage
%\section{Reference pages for 2D placement of streamlines}
\chapter{Placement of streamlines in 2D field}
\label{ref_chapter_2D_Streamlines}

\subsection*{Definitions}
Vector and direction fields are commonly used for modeling physical
phenomena, where a direction and magnitude, or a vector is assigned to
each point inside a domain.\\ A streamline is a curve everywhere
tangent to the field. It can be considered as the path traced by an
imaginary masless particle dropped into a steady fluid flow described
by the field.\\ A streamline is represented as a polyline iteratively
elongated by bidirectional numerical integration started from a seed
point, until it comes close to another streamline, hits the domain
boundary, or reaches a critical point.\\ The Streamlines class consists
of saturating the domain with a set of tangential streamlines in
accordance with a specified density.\\ Streamlines are represented as
containers of points, manipulated by an iterator range of points, and the
whole placement is accessible via an iterator range of streamlines.\\ The
streamlines class of \cgal\ depends on three
template parameters.  The first template parameter stands for a class
that provide different geometric primitives manipulated by the
algorithm. The second one stands for a class that represent both the
vector field and the visualisation domain with operations on them.
The third template parameter stands for a function object that assure
the integration over the streamlines.

\subsection*{Concepts}
\ccRefConceptPage{StreamLinesTraits_2} \\
\ccRefConceptPage{Integrator_2} \\
\ccRefConceptPage{VectorField_2} \\

\subsection*{Classes}


\ccRefIdfierPage{CGAL::Stream_lines_2<VectorField_2,Integrator_2>}\\
\ccRefIdfierPage{CGAL::Euler_integrator_2<VectorField_2>}\\
\ccRefIdfierPage{CGAL::Runge_kutta_integrator_2<VectorField_2>}\\
\ccRefIdfierPage{CGAL::Regular_grid_2<StreamLinesTraits_2>}\\
\ccRefIdfierPage{CGAL::Triangular_field_2<StreamLinesTraits_2>}\\


% +------------------------------------------------------------------------+
%%RefPage: end of main body, begin of footer
% EOF
% +------------------------------------------------------------------------+
