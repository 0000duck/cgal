% +------------------------------------------------------------------------+
% | CGAL Reference Manual:  main.tex
% +------------------------------------------------------------------------+
% | snap rounding of line segments
% |
% | 9.4.00   Eli Packer
% | 
\RCSdef{\snapRoundingRev}{$Revision$}
\RCSdefDate{\snapRoundingDate}{$Date$}
% +------------------------------------------------------------------------+

\chapter{Snap Rounding - Reference Manual}
\label{chapterSnapRoundibgRef}
\ccChapterRelease\snapRoundingRev. \ \snapRoundingDate\\
\ccChapterAuthor{Eli Packer}


\section*{Summary}

Snap rounding is a well known method for converting
arbitrary-precision arrangements of segments into a fixed-precision
representation. It is classified as a finite precision approximation 
technique. Iterated Snap Rounding is a modification of Snap Rounding in
which each vertex is at leat half-the-width-of-a-pixel away from any
non-incident edge.
This package provides supports both methods.

Given a finite collection $\S$ of segments in the plane, the
arrangement of $\S$ denoted $\A(\S)$ is the subdivision of the plane
into vertices, edges, and faces induced by $\S$. %\cite{arrg-surveys}.
A {\it vertex\/} of the arrangement is either a segment endpoint or
the intersection of two segments. Given an arrangement of segments
whose vertices are represented with arbitrary-precision coordinates,
snap rounding (SR, for short) proceeds as follows.  We tile the plane
with a grid of unit squares, {\it pixels}, each centered at a point
with integer coordinates. A pixel is {\it hot\/} if it contains a
vertex of the arrangement. Each vertex of the arrangement is replaced
by the center of the hot pixel containing it and each edge $e$ is
replaced by the polygonal chain through the centers of the hot pixels
met by $e$, in the same order as they are met by $e$.

In a snap-rounded arrangement, the distance between a vertex and
a non-incident edge can be extremely small compared with the width of a
pixel in the grid used for rounding. Iterated Snap Rounding is a modification
of Snap Rounding which makes the a vertex and a non-incident edge well separated
(the distance between each is at least half-the-width-of-a-pixel).
However, the guaranteed quality of the approximation in Iterated Snap Rounding
degrades.

The traits used here must support arbitrary-precision number type as this is a
basic requirement of Snap Rounding.

%\begin{ccRefClass}{Isr<NT>}

\subsection*{Definition}
\ccInclude{CGAL/Snap_rounding_2.h}

\ccParameters
The template declaration of the \ccc{Snap_rounding_2} function has one parameter:

\ccCreation
    \ccCreationVariable{Snap_rounding_2}

\ccConstructor{snap_rounding(typename std::list<Segment_2>::const_iterator first,typename std::list<Segment_2>::const_iterator beyond,NT pixel_size,bool do_isr = true,bool use_integer_mode = true, int number_of_kd_trees = 1);}{
The first two parameters denote the first and after-the-last iterators of a list of Segment\_2 which are the input of Snap Rounding.
The third parameter denotes the pixel size. The forth parameter determines whether to apply Iterated Snap Rounding
or Snap Rounding. The fifth parameter determines the output representation. If the
parameter value is {\ccc true}, then the output is represented in an integer
grid just as defined by the Snap Rounding algorithm. Note that the number type of
the representation remains {\ccc NT}. Otherwise, the plane is tiled
with unit pixels with size {\ccc pixel\_size} and each vertice of the output is
represented by the coordinates of the cenetr of the hot pixel that contains it.
The sixth parameter is desribed below.
}

\ccPrecond{\ccc{pixel_size} must have a positive value.}
\ccPrecond{\ccc{number_of_kd_trees} must be a positive integer.}

\begin{ccAdvanced}
The sixth parameter denotes the number of KD trees to use (see \cite{isr-2002} for details).
The default, $1$, is ussually the best one to use.
\end{ccAdvanced}

%\end{ccRefClass}

% +--------------------------------------------------------+

% EOF
