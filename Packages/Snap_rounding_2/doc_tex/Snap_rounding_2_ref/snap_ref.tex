% +------------------------------------------------------------------------+
% | CGAL Reference Manual:  main.tex
% +------------------------------------------------------------------------+
% | snap rounding of line segments
% |
% | 9.4.00   Eli Packer
% | 
\RCSdef{\snapRoundingRev}{$Revision$}
\RCSdefDate{\snapRoundingDate}{$Date$}
% +------------------------------------------------------------------------+

\ccRefPageBegin

%\chapter{Snap Rounding - Reference Manual}
%\label{chapterSnapRoundibgRef}
%\ccChapterRelease\snapRoundingRev. \ \snapRoundingDate\\
%\ccChapterAuthor{Eli Packer}

\begin{ccRefFunction}{snap_rounding_2}

\ccDefinition

Snap Rounding (SR, for short) is a well known method for converting
arbitrary-precision arrangements of segments into a fixed-precision
representation \cite{gght-srlse-97, gm-rad-98, h-psifp-99}. In
the study of robust geometric computing, it can be classified as a
finite precision approximation technique. Iterated Snap Rounding (ISR,
for short) is a modification of SR in which each vertex is at least
half-the-width-of-a-pixel away from any non-incident edge
\cite{cgal:hp-isr-02}. This package supports both methods. Algorithmic
details and experimental results are given in \cite{cgal:hp-isr-02}.

Given a finite collection $\S$ of segments in the plane, the
arrangement of $\S$ denoted $\A(\S)$ is the subdivision of the plane
into vertices, edges, and faces induced by $\S$. %\cite{arrg-surveys}.
A {\it vertex\/} of the arrangement is either a segment endpoint or
the intersection of two segments. Given an arrangement of segments
whose vertices are represented with arbitrary-precision coordinates,
SR proceeds as follows.  We tile the plane
with a grid of unit squares, {\it pixels}, each centered at a point
with integer coordinates. A pixel is {\it hot\/} if it contains a
vertex of the arrangement. Each vertex of the arrangement is replaced
by the center of the hot pixel containing it and each edge $e$ is
replaced by the polygonal chain through the centers of the hot pixels
met by $e$, in the same order as they are met by $e$.

In a snap-rounded arrangement, the distance between a vertex and
a non-incident edge can be extremely small compared with the width of a
pixel in the grid used for rounding. ISR is a modification of SR which
makes a vertex and a non-incident edge well separated (the distance
between each is at least half-the-width-of-a-pixel). However, the
guaranteed quality of the approximation in ISR degrades. For more
details on ISR see \cite{cgal:hp-isr-02}.

The traits used here must support arbitrary-precision number type as this is a
basic requirement of SR.

\ccInclude{CGAL/Snap_rounding_2.h}

\ccGlobalFunction{
  template < class Traits, class InputIterator, class OutputContainer >
  void
  snap_rounding_2(
  InputIterator begin,
  InputIterator end,
  OutputContainer& output_container,
  typename Traits::FT pixel_size,
  bool do_isr = true,
  bool int_output = true,
  unsigned int number_of_kd_trees = 1);}

The first two parameters denote the first and after-the-last iterators of the input segments.
The third parameter is a reference to a container of the output polylines. Since a polyline
is composed of a sequence of points, a polyline is a container itself.
The forth parameter denotes the pixel size. The fifth parameter determines whether to apply
ISR or SR. The sixth parameter determines the output representation.
If the parameter value is {\it true}, then the output is represented in an integer
grid where one unit is equal to pixel size. Otherwise, the plane is tiled
with unit pixels with size {\it pixel\_size} and each vertex of the output is
represented by the coordinates of the center of the hot pixel that contains it;
These coordinates are not necessarily integers.
The seventh (and last) parameter is described next.

\begin{ccAdvanced}

 A basic query used in the algorithm is to report the hot pixels of
 size $w$ that a certain segment $s$ intersects.  An alternative way to
 do the same is to query the hot pixels' centers contained in a
 Minkowski sum of $s$ with a pixel of width $w$ centered at the origin;
 we denote this Minkowski sum by $M(s)$. Since efficiently implementing
 this kind of query is difficult, we use an orthogonal range-search
 structure instead. We query with the bounding box $B(M(s))$ of $M(s)$
 in a two-dimensional kd-tree which stores the centers of hot
 pixels. Since $B(M(s))$ in general is larger than $M(s)$, we still
 need to filter out the hot pixels which do not intersect $s$.

 While this approach is easy to implement with CGAL, it may incur
 considerable overhead since the area of $B(M(s))$ may be much larger
 than the area of $M(s)$, possibly resulting in many redundant hot pixels
 to filter out. Our heuristic solution, which we describe next, is to
 use a cluster of kd-trees rather than just one.  The cluster includes
 several kd-trees, each has the plane, and hence the centers of hot
 pixels, rotated by a different angle in the first quadrant of the
 plane; for our purpose, a rotation by angles outside this quadrant
 is symmetric to a rotation by an angle in the first quadrant.

 Given a parameter $c$, the angles of rotation are $(i - 1)
 \frac{\pi}{2c}, i=1,\ldots,c$, and we construct a kd-tree
 corresponding to each of these angles. Then for a query segment $s$,
 we choose the kd-tree for which the area of $B(M(s))$ is the smallest,
 in order to (potentially) get less hot pixels to filter out.  Since
 constructing many kd-trees may be costly, our algorithm avoids
 building a kd-tree which it expects to be queried a relatively small
 number of times (we estimate this number in advance).  How many
 kd-trees should be used?  It is difficult to provide a simple
 answer for that.  There are inputs for which the time to build more
 than one kd-tree is far greater than the time saved by having to
 filter out less hot pixels (sparse arrangements demonstrate this
 behavior), and there are inputs which benefit from using several
 kd-trees. Thus, the user can control the number of kd-trees
 with the parameter {\it number\_of\_kd\_trees}. Typically, but not
 always, one kd-tree (the default) is sufficient.

\end{ccAdvanced}

\ccPrecond{{\it pixel\_size} must have a positive value and {\it number\_of\_kd\_trees} must be a positive integer.}

\end{ccRefFunction}

% EOF
