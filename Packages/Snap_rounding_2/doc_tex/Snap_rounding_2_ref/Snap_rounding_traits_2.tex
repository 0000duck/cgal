% +------------------------------------------------------------------------+
% | Reference manual page: Snap_rounding_2_traits.tex (Snap_rounding_2)
% +------------------------------------------------------------------------+
% | 
% | Package: sr (Snap_rounding_2)
% | 
% +------------------------------------------------------------------------+

\ccRefPageBegin

%%RefPage: end of header, begin of main body
% +------------------------------------------------------------------------+

\begin {ccRefClass} {Snap_rounding_traits_2<Kernel>}
    
%\ccDefinition 

The class \ccStyle{Snap_rounding_traits_2<Kernel>} is a model of the
\ccc{SnapRoundingTraits_2} concept, and is the only traits class supplied
with the package.
This class should be instantiated with an exact geometric kernel that conforms
to the \cgal\ kernel-concept, such as the
\ccStyle{Cartesian<gmpq>} kernel.
% although other inexact kernels could be used at the user's own risk.

This geometric kernel must provide an (arbitrary-precision) rational number type
(\ccc{FT}), \ccc{Point_2}, \ccc{Segment_2} and \ccc{Iso_rectangle_2}.
It should be possible to cast numbers of the number type \ccc{FT} to 
double-precision representation. That is, the function 
\ccc{CGAL::to_double(FT)} must be supported.

The \ccc{CGAL::to_double()} function is used to implement the operation that
rounds the coordinates of a point to a center of a pixel. This operation is one
of the traits-concept requirement. The coordinates are converted to double, 
rounded down to the nearest grid point, and finally adjusted to lie on a center
of a pixel. Notice that if
\ccc{CGAL::to_double()} returns the closet value, then when it rounds up a given
coordinate, the resulting ISR, may be imprecise, and the distance between some
vertex and a non-incident edge can be slightly less than the guaranteed
half-the width-of-a-pixel.

% The traits class inherits from the
% \ccStyle{Arr_segment_cached_traits_2<Kernel>}
% class, so that the \ccStyle{X_monotone_curve_2} type is defined for it.

\ccInclude{CGAL/Snap_rounding_traits_2.h}
 
\ccIsModel
    \ccc{SnapRoundingTraits_2}\\
 
\end{ccRefClass} % Snap_rounding_2_traits
% +------------------------------------------------------------------------+
%%RefPage: end of main body, begin of footer
\ccRefPageEnd
% EOF
% +------------------------------------------------------------------------+
