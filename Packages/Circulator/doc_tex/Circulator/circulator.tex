% +------------------------------------------------------------------------+
% | CGAL Reference Manual:  circulator.tex
% +------------------------------------------------------------------------+
% | Requirements for circulators in analogy to STL iterators.
% | Adaptors between circulators and iterators.
% | Proposal for CGAL.
% |
% | 11.10.1996   Lutz Kettner
% | 
\RCSdef{\circRev}{$Revision$}
\RCSdefDate{\circDate}{$Date$}
% +------------------------------------------------------------------------+

\gdef\lciIfHtmlClassLinks{\lcFalse}
\gdef\lciIfHtmlRefLinks{\lcFalse}
\gdef\lciIfHtmlLinks{\lcFalse}

\section{Circulators}
%\label{chapterCirculators}

%\ccChapterRelease{\circRev. \ \circDate}\\
%\ccChapterAuthor{Lutz Kettner}

An introduction to the concept of circulators is given here. A couple
of adaptors are presented that convert between iterators and
circulators. Some useful functions for circulators follow. This
chapter concludes with a discussion of the design decisions taken. For
the full description of the circulator requirements, the provided base
classes, the circulator tags, and the support for generic algorithms
that work for iterators as well as for circulators please refer to the
reference pages. Note that circulators are not part of \stl, but of \cgal.

% +-----------------------------------------------------+
\subsection{Introduction}
\label{sectionIntroduction}
\label{sectionCirculatorWarning}

The concept of iterators in \stl\ is tailored for linear
sequences~\cite{cgal:ansi-is14882-98,cgal:ms-strg-96}.  In contrast, circular
sequences occur naturally in many combinatorial and geometric
structures. Examples are polyhedral surfaces and planar maps, where
the edges emanating from a vertex or the edges around a facet form a
circular sequence.

Since circular sequences do not allow for efficient iterators, we have
introduced the new concept of {\em circulators}. They share most of
the requirements of iterators, while the main difference is the lack
of a past-the-end position in the sequence. Appropriate adaptors are
provided between iterators and circulators to integrate circulators
smoothly into the framework of \stl. An example of a generic {\tt
  contains} function illustrates the use of circulators.  As usual for
circular structures, a {\tt do}-{\tt while} loop is preferable, such
that for the specific input, {\tt c == d}, all elements in the
sequence are reached.

\begin{ccExampleCode}
template <class Circulator, class T>
bool contains( Circulator c, Circulator d, const T& value) {
    if (c != 0) {
        do {
            if (*c == value)
                return true;
        } while (++c != d);
    }
    return false;
}
\end{ccExampleCode}

\noindent
Three circulator categories are defined: forward, bidirectional and
random-access circulators. Given a circulator {\tt c}, the operation
{\tt *c} denotes the item the circulator refers to. The operation {\tt
  ++c} advances the circulator by one item and {\tt --c} steps a
bidirectional circulator one item backwards. For random-access
circulators {\tt c+n} advances the circulator {\tt n} steps.  Two
circulators can be compared for equality.

Circulators have a different notion of reachability and ranges than
iterators. A circulator {\tt d} is called {\em reachable\/} from a
circulator {\tt c} if {\tt c} can be made equal to {\tt d} with
finitely many applications of the operator {\tt ++}. Due to the
circularity of the sequence this is always true if both circulators
refer to items of the same sequence. In particular, {\tt c} is always
reachable from {\tt c}. Given two circulators {\tt c} and {\tt d}, the
range {\tt [c,d)} denotes all circulators obtained by starting with
{\tt c} and advancing {\tt c} until {\tt d} is reached, but does not
include {\tt d}, for {\tt d} $\neq$ {\tt c}.  So far it is the same
range definition as for iterators.  The difference lies in the use of
{\tt [c,c)} to denote all items in the circular sequence, whereas for
an iterator {\tt i} the range {\tt [i,i)} denotes the empty range.  As
long as {\tt c != d} the range {\tt[c,d)} behaves like an iterator
range and could be used in \stl\ algorithms. For circulators however,
an additional test {\tt c == NULL} is required that returns true if
and only if the circular sequence is empty.  As for
\CC, we recommend the use of 0 instead of {\tt NULL}.

Besides the conceptual cleanness, the main reason for inventing a new
concept with a similar intent as iterators is efficiency. An iterator
is supposed to be a light-weight object -- merely a pointer and a
single indirection to advance the iterator.  Although iterators could
be written for circular sequences, we do not know of an efficient
solution. The missing past-the-end situation in circular sequences can
be solved with an arbitrary sentinel in the cyclic order, but this
would destroy the natural symmetry in the structure (which is in
itself a bad idea) and additional bookkeeping in the items and
checking in the iterator advance method reduces efficiency.  Another
solution may use more bookkeeping in the iterator, e.g.~with a start
item, a current item, and a kind of winding-number that is zero for
the {\tt begin()}-iterator and one for the past-the-end
situation\footnote{This is currently implemented as the
  adaptor class which provides a pair of iterators for a given
  circulator.}.  We have introduced the concept of circulators
that allows light-weight implementations and the \cgal\ support
library provides adaptor classes that convert between iterators and
circulators (with the corresponding penalty in efficiency), so as to
integrate this new concept into the framework of \stl.

A serious design problem is the slight change of the semantic for
circulator ranges as compared to iterator ranges. Since this semantic
is defined by the intuitive operators {\tt ++} and {\tt ==}, which we
would like to keep for circulators as well, circulator ranges can be
used in \stl\ algorithms. This is in itself a useful feature, if there
would not be the definition of a full range $\left[c, c\right)$ that
an \stl\ algorithm will treat as an empty range. However, the
likelihood of a mistake may be overestimated, since for a container
{\tt C} supporting circulators there is no {\tt end()} member
function, and an expression such as {\tt std::sort( C.begin(),
  C.end())} will fail.  It is easy to distinguish iterators and
circulators at compile time, which allows for generic algorithms
supporting both as arguments.  It is also possible to protect
algorithms against inappropriate arguments using the same technique,
see the reference pages for circulators, specifically the
\ccc{Assert_iterator} and \ccc{is_empty_range} functions.

{\bf Warning:} Please note that the definition of a range is different
from that of iterators. An interface of a data structure must declare
whether it works with iterators, circulators, or both. \stl\ 
algorithms always specify only iterators in their interfaces. A range
$\left[c, d\right)$ of circulators used in an interface for iterators
will work as expected as long as $c\; !\!= d$. A range $\left[c,
  c\right)$ will be interpreted as the empty range like for iterators,
which is different than the full range that it should denote for
circulators.

% +---------------------------------------------+
\begin{ccHtmlClassFile}{Circulator.html}{Requirements for Circulators}
\ccHtmlNoClassLinks
\begin{ccClass}{Circulator}
\subsection{Forward Circulator}

A class \ccClassName\ that satisfies the requirements of a forward
circulator with the value type \ccStyle{T}, supports the following
operations. See the reference pages for the full set of requirements.
Note that the stated return values are not required, only a return
value that is convertible to the stated type is required.  As for \CC,
we recommend the use of 0 instead of {\tt NULL}.

\ccTypes
\ccSetTwoColumns{Circulator:: iterator_category}{}

\ccNestedType{value_type}{the value type \ccStyle{T}.}
\ccGlue
\ccNestedType{reference}{either reference or const reference to \ccStyle{T}.}
\ccGlue
\ccNestedType{pointer}{either pointer or const pointer to \ccStyle{T}.}
\ccGlue
\ccNestedType{size_type}{unsigned integral type that can hold the size
  of the sequence.}
\ccGlue
\ccNestedType{difference_type}{signed integral type that can hold the
  distance between two circulators.}
\ccGlue
\ccNestedType{iterator_category}{circulator category 
    \ccc{Forward_circulator_tag}.}

\ccCreation
\ccCreationVariable{c}
%\ccSetThreeColumns{Circulator&}{int n + Circulator d}{}
\ccSetThreeColumns{difference_type}{difference_type n += d}{}
\ccPropagateThreeToTwoColumns

\ccConstructor{Circulator();}{a circulator equal to \ccc{NULL} denoting
                              an empty sequence.}
\ccGlue
\ccConstructor{Circulator(const Circulator& d);}{a circulator equal to $d$.}

\ccOperations

\ccMethod{Circulator& operator=(const Circulator &d);}
{Assignment.}
\ccGlue
\ccMethod{bool operator==(NULL) const;}
{Test for emptiness.}
\ccGlue
\ccMethod{bool operator!=(NULL) const;}
{Test for non-emptiness, i.e.~ \ccStyle{!(c == NULL)}.}
\ccGlue
\ccMethod{bool operator==(const Circulator &d) const;}
{$c$ is equal to $d$ if they refer to the same item.}
\ccGlue
\ccMethod{bool operator!=(const Circulator &d) const;}
{Test for inequality, i.e.~\ccStyle{!(c == d)}.}

\ccMethod{reference operator*();}
{Returns the value of the circulator.
 If \ccClassName\ is mutable \ccStyle{*c = t} is valid.
 \ccPrecond \ccVar\ is dereferenceable.}

\ccMethod{pointer operator->();}
{Returns a pointer to the value of the circulator.
 \ccPrecond \ccVar\ is dereferenceable.}

\ccMethod{Circulator& operator++();}
{Prefix increment operation.
\ccPrecond \ccVar\ is dereferenceable. 
\ccPostcond \ccVar\ is dereferenceable.}

\ccMethod{const Circulator& operator++(int);}
{Postfix increment operation. The result is the same as that of: 
 \ccStyle{Circulator tmp = c; ++c; return tmp;}~.}

\end{ccClass} 


% +---------------------------------------------+
\ccHtmlNoClassLinks
\ccHtmlNoClassIndex
\begin{ccClass}{Circulator}
\subsection{Bidirectional Circulator}

A class \ccClassName\ that satisfies the requirements of a bidirectional
circulator with the value type \ccStyle{T}, supports the following operations
in addition to the operations supported by a forward circulator.

\ccTypes
\ccSetTwoColumns{Circulator:: iterator_category}{}

\ccNestedType{iterator_category}{circulator category 
    \ccc{Bidirectional_circulator_tag}.}

\ccSetThreeColumns{difference_type}{difference_type n += d}{}
\ccPropagateThreeToTwoColumns
\ccCreationVariable{c}
\ccOperations

\ccMethod{Circulator& operator--();}
{Prefix decrement operation.
 \ccPrecond \ccVar\ is dereferenceable. \ccPostcond\ccVar\ is dereferenceable.}

\ccMethod{const Circulator& operator--(int);}
{Postfix decrement operation. The result is the same as that of 
 \ccStyle{Circulator tmp = c; --c; return tmp;}~.}

\end{ccClass} 


% +---------------------------------------------+
\ccHtmlNoClassLinks
\ccHtmlNoClassIndex
\begin{ccClass}{Circulator}
\subsection{Random Access Circulator}
\label{sectionRandomAccessCirculatorRequ}

A class \ccClassName\ that satisfies the requirements of a random
access Circulator for the value type \ccStyle{T}, supports the
following operations in addition to the operations supported by a
bidirectional Circulator.  In contrast to random access iterators, 
no comparison operators are available for
random access circulators.

\ccTypes
\ccSetTwoColumns{Circulator:: iterator_category}{}

\ccNestedType{iterator_category}{circulator category 
    \ccc{Random_access_circulator_tag}.}

\ccSetThreeColumns{difference_type}{difference_type n += d}{}
\ccPropagateThreeToTwoColumns
\ccCreationVariable{c}
\ccOperations

\ccMethod{Circulator& operator+=(difference_type n);}
{The result is the same as if the prefix increment operation 
 was applied $n$ times, but it is computed in constant time.}

\ccMethod{Circulator operator+(difference_type n);}
{Same as above, but returns a new circulator.}

\def\ccTagRmEigenClassName{\ccTrue}
\ccFunction{Circulator operator+(difference_type n, Circulator c);}
{Same as above.}
\def\ccTagRmEigenClassName{\ccFalse}

\ccMethod{Circulator& operator-=(difference_type n);}
{The result is the same as if the prefix decrement operation 
 was applied $n$ times, but it is computed in constant time.}

\ccMethod{Circulator operator-(difference_type n);}
{Same as above, but returns a new circulator.}


\ccMethod{reference operator[](difference_type n);}
{Returns \ccStyle{*(c + n)}.}

\ccMethod{difference_type operator-(const Circulator& d) const;}
{returns the difference between the two circulators. The value will be
  in the interval $\left[ 1-s , s-1 \right]$ if $s$ is the size of the
  total sequence. The difference for a fixed circulator \ccVar\ (or
  $d$) with all other circulators $d$ (or \ccVar) is a consistent
  ordering of the elements in the data structure. There has to be a
  minimal circulator $d_{\mbox{\footnotesize min}}$ for which the
  difference \ccVar $- d_{\mbox{\footnotesize min}}$ to all other
  circulators \ccVar\ is non-negative.}

\ccMethod{Circulator min_circulator() const;}
{Returns the minimal circulator $c_{\mbox{\footnotesize min}}$
  in constant time.} 

\end{ccClass} 
\end{ccHtmlClassFile} 

% +-----------------------------------------------------+
\subsection{Adaptors Between Iterators and Circulators}
\label{sectionCirculatorAdaptor}

Algorithms working on iterator ranges can not be applied to circulator
ranges in full generality, only to subranges (see the warning in
Section~\ref{sectionCirculatorWarning}). The following adaptors
convert circulators to iterators and vice versa (with the unavoidable
space and time penalty) to reestablish this generality.

\ccInclude{CGAL/circulator.h}

\begin{tabbing}
  \ccc{Container_from_circulator}\ \ \ \=
      container-like class with iterators built from a circulator\\
  \ccc{Circulator_from_iterator}  \>
      circulator over a range of two iterators\\
  \ccc{Circulator_from_container} \>
      circulator for a container
\end{tabbing}

The following example applies the generic {\tt std::reverse()} algorithm 
from \stl\ to a sequence given by a bidirectional circulator {\tt c}.
It uses the \ccc{Container_from_circulator} adaptor.

\begin{ccExampleCode}
Circulator c;  // c must be at least bidirectional.
CGAL::Container_from_circulator<Circulator> container(c);
std::reverse( container.begin(), container.end());
\end{ccExampleCode}

Another example defines a circulator \ccc{c} for a vector of
\ccc{int}'s. However, since there are no elements in the vector, the
circulator denotes an empty sequence.  If there were elements in the
vector, the circulator would implement a random access modulus the
size of the sequence.

\begin{ccExampleCode}
std::vector<int> v;
typedef  CGAL::Circulator_from_iterator< 
             std::vector<int>::iterator > Circulator;
Circulator c( v.begin(), v.end());
\end{ccExampleCode}

% +-----------------------------------------------------+
\subsection{Functions on Circulators}
\label{sectionCirculatorFunctions}

A few functions deal with circulators and circulator ranges. The type
\ccc{C} denotes a circulator. The type \ccc{IC} denotes either a circulator
or an iterator. More on algorithms that work with circulators as well with
iterators can be found in the reference pages.

\ccInclude{CGAL/circulator.h}

\begin{tabbing}
  \ccTexHtml{\ccc{Forward_container_from_circulator}\ \ \ \= \kill}{}
  \ccc{circulator_size(C c)} \>
      size of the sequence reachable by \ccc{c}\\
  \ccc{circulator_distance(C c, C d)} \>
      number of elements in the range $\left[c, d\right)$ \\
  \ccc{iterator_distance(IC ic1, IC ic2)} \>
      number of elements in the range $\left[\ccc{ic2}, \ccc{ic1}\right)$ \\
  \ccc{is_empty_range( IC ic1, IC ic2)} \>
      test the range $\left[\ccc{ic2}, \ccc{ic1}\right)$ for emptiness
\end{tabbing}


% +-----------------------------------------------------+
% EOF
