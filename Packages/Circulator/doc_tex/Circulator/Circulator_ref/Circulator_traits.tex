% +------------------------------------------------------------------------+
% | CGAL Reference Manual: Circulator_traits
% +------------------------------------------------------------------------+
% | Requirements for circulators in analogy to STL iterators.
% | Adaptors between circulators and iterators.
% | Proposal for CGAL.
% |
% | 11.10.1996   Lutz Kettner
% | 
% | \RCSdef{\circRev}{$Revision$}
% | \RCSdefDate{\circDate}{$Date$}
% +------------------------------------------------------------------------+

% +-----------------------------------------------------+
\begin{ccRefClass}{Circulator_traits<C>}

\ccDefinition

The circulator traits class distinguishes between circulators and
iterators. It defines a local type \ccc{category} that is identical to the
type \ccc{Circulator_tag} if the iterator category of the argument
\ccc{C} is a circulator category. Otherwise it is identical to the type
\ccc{Iterator_tag}.

The local type \ccc{iterator_category} gives the corresponding
iterator category for circulators, i.e.~one of
\ccc{forward_iterator_tag}, \ccc{bidirectional_iterator_tag}, or
\ccc{random_access_iterator_tag}.

The local type \ccc{circulator_category} gives the corresponding
circulator category for iterators, i.e.~one of
\ccc{Forward_circulator_tag}, \ccc{Bidirectional_circulator_tag}, or
\ccc{Random_access_circulator_tag}.

\ccInclude{CGAL/circulator.h}

\ccTypes

\ccTwo{Circulator_traits<C>:: \ \ circulator_category}{}

\ccNestedType{category}{either \ccc{Iterator_tag} or 
  \ccc{Circulator_tag}.}
\ccNestedType{iterator_category}{
  corresponding iterator category for circulators.}
\ccNestedType{circulator_category}{
  corresponding circulator category for iterator}

\ccExample

A generic function \ccc{bar} that distinguishes between a call with a 
circulator range and a call with an iterator range:

\begin{ccExampleCode}
template <class I>
void bar( I i, I j, CGAL::Iterator_tag) {
    CGAL::Assert_iterator(i);
    // This function is called for iterator ranges [i,j).
}
template <class C>
void bar( C c, C d, CGAL::Circulator_tag) {
    CGAL::Assert_circulator(c);
    // This function is called for circulator ranges [c,d).
}
template <class IC>
void bar( IC i, IC j) {  // calls the correct function
    return bar( i, j, typename CGAL::Circulator_traits<IC>::category());
}
\end{ccExampleCode}

\end{ccRefClass}
% +-----------------------------------------------------+

% EOF
