% +------------------------------------------------------------------------+
% | CGAL Reference Manual: Forward_circulator_from_container.tex
% +------------------------------------------------------------------------+
% | Requirements for circulators in analogy to STL iterators.
% | Adaptors between circulators and iterators.
% | Proposal for CGAL.
% |
% | 11.10.1996   Lutz Kettner
% | 
% | \RCSdef{\circRev}{$Revision$}
% | \RCSdefDate{\circDate}{$Date$}
% +------------------------------------------------------------------------+

% +-----------------------------------------------------+
\begin{ccRefClass}{Circulator_from_container<C>}
\label{pageCircFromContainerRef}

\ccHtmlCrossLink{Const_circulator_from_container}
\ccHtmlCrossLink{Const_circulator_from_container<C>}
\ccHtmlIndexC[class]{Const_circulator_from_container<C>}

\ccDefinition
  
The adaptor \ccClassTemplateName\ provides a circulator for an \stl\
container $C$ of equal category as the iterator provided by the container. 
The iterator must be at least of the forward iterator
category. The corresponding non-mutable circulator is called
\ccc{Const_circulator_from_container<C>}.

The container type \ccc{C} is supposed to conform to the \stl\ 
requirements for container (i.e.~to have a \ccStyle{begin()} and an
\ccStyle{end()} iterator as well as the local types
\ccStyle{reference}, \ccStyle{const_reference}, \ccStyle{value_type},
\ccStyle{size_type}, and \ccStyle{difference_type}).

\ccInclude{CGAL/circulator.h}

\ccTypes
\ccSetThreeColumns{typedef container::const_reference}{const_reference;}{}

All types required for circulators are provided.

\ccCreation
\ccCreationVariable{c}

\ccConstructor{Circulator_from_container();}{%
  a circulator \ccVar\ on an empty sequence.}

\ccConstructor{Circulator_from_container(C* container);}{%
  a circulator \ccVar\ initialized to refer to the first element in
  \ccStyle{container}, i.e.~\ccStyle{container.begin()}.
  The circulator \ccVar\ refers to an empty sequence if the
  \ccStyle{container} is empty.
}

\ccConstructor{Circulator_from_container(C* container, C::iterator i);}{%
  a circulator \ccVar\ initialized to refer to the element \ccStyle{*i} in
  \ccStyle{container}. \ccPrecond \ccStyle{*i} is dereferenceable and refers
  to \ccStyle{container}.
}

\ccOperations

The adaptor conforms to the requirements of the corresponding
circulator category. An additional member function
\ccc{current_iterator()} returns the current iterator pointing to
the same position as the circulator does.

\ccSeeAlso

\ccc{Container_from_circulator},
\ccc{Circulator_from_iterator},
\ccc{Circulator}.

\ccExample

The following program composes two adaptors -- from a container to a
circulator and back to an iterator. It applies an \stl\ sort algorithm
on a \stl\ vector with three elements. The resulting vector will be
{\tt [2 5 9]} as it is checked by the assertions. The program is
part of the \cgal\ distribution.

%\ccIncludeExampleCode{examples/Circulator/circulator_prog2.C}
\ccIncludeExampleCode{Circulator/circulator_prog2.C}

\end{ccRefClass}

% +-----------------------------------------------------+
% EOF
