% +------------------------------------------------------------------------+
% | CGAL Reference Manual: For_all.tex
% +------------------------------------------------------------------------+
% | Requirements for circulators in analogy to STL iterators.
% | Adaptors between circulators and iterators.
% | Proposal for CGAL.
% |
% | 11.10.1996   Lutz Kettner
% | 
% | \RCSdef{\circRev}{$Revision$}
% | \RCSdefDate{\circDate}{$Date$}
% +------------------------------------------------------------------------+

% +-----------------------------------------------------+
\begin{ccRefMacro}{CGAL_For_all}

\ccHtmlCrossLink{CGAL_For_all_backwards}
\ccHtmlIndexC[macro]{CGAL_For_all_backwards}

\ccDefinition

In order to write algorithms that work with iterator ranges as well as
with circulator ranges we have to consider the difference of
representing an empty range. For iterators this is the range $[i,i)$,
while for circulators it would be \ccc{c == NULL}, the empty sequence test.
The function \ccc{is_empty_range} provides the necessary generic test
which accepts an iterator range or a circulator range and says whether
the range is empty or not.

\ccInclude{CGAL/circulator.h}

A macro \ccc{CGAL_For_all( i, j)} simplifies the writing of such simple
loops as the one in the example of the function \ccc{is_empty_range}.
\ccc{i} and \ccc{j} can be either iterators or circulators. The macro
loops through the range [\ccc{i, j}). It increments \ccc{i} until it
reaches \ccc{j}. The implementation looks like:

\ccc{CGAL_For_all(i,j)}  \ \ \ccTexHtml{$\equiv$}{:=} \ \
\begin{minipage}[t]{0.74\textwidth}
\begin{verbatim}
for ( bool _circ_loop_flag = ! ::CGAL::is_empty_range(i,j);
      _circ_loop_flag;
      _circ_loop_flag = ((++i) != (j)) 
)
\end{verbatim}
\end{minipage}%

Note that the macro behaves like a \ccc{for}-loop. It can be used with
a single statement or with a statement block.  For bidirectional
iterators or circulators,  a backwards loop macro
\ccc{CGAL_For_all_backwards( i, j)} exists that decrements \ccc{j} until
it reaches \ccc{i}.


\ccSeeAlso

\ccc{iterator_distance},
\ccc{is_empty_range},
\ccc{Circulator_tag},
\ccc{Circulator_traits},\\
\ccc{Assert_circulator_or_iterator},
\ccc{Circulator}.


\end{ccRefMacro}

% +-----------------------------------------------------+
% EOF
