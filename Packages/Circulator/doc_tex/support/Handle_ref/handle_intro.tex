% +------------------------------------------------------------------------+
% | Reference manual page: Handle_ref.tex
% +------------------------------------------------------------------------+
% | 10 04 2002   Mariette Yvinec
% | Package: Handle
% | 
\RCSdef{\RCSTriangulationRev}{$Revision$}
\RCSdefDate{\RCSTriangulationDate}{$Date$}
% |
%%RefPage: end of header, begin of main body
% +------------------------------------------------------------------------+

%\clearpage
%\section{Reference pages for Handles}
%\chapter{Handles}

Most data structures in \cgal\ use the concept of \ccc{Handle} in their user
interface to refer to the elements they store.  This concept describes what is
sometimes called a trivial iterator.  A \ccc{Handle} is akeen to a pointer to
an object providing the dereference operator \ccc{operator*()} and member
access \ccc{operator->()} but no increment or decrement operators like
iterators.  A \ccc{Handle} is intended to be used whenever the referenced
object is not part of a logical sequence.
