% +------------------------------------------------------------------------+
% | CGAL Reference Manual: Circulator.tex
% +------------------------------------------------------------------------+
% | Requirements for circulators in analogy to STL iterators.
% | Adaptors between circulators and iterators.
% | Proposal for CGAL.
% |
% | 11.10.1996   Lutz Kettner
% | 
% | \RCSdef{\circRev}{$Revision$}
% | \RCSdefDate{\circDate}{$Date$}
% +------------------------------------------------------------------------+

% +-----------------------------------------------------+
\begin{ccRefConcept}{Circulator}

\ccDefinition

\ccHtmlCrossLink{Forward_circulator}
\ccHtmlCrossLink{Bidirectional_circulator}
\ccHtmlCrossLink{Random_access_circulator}
\ccHtmlIndexC[concept]{Forward_circulator}
\ccHtmlIndexC[concept]{Bidirectional_circulator}
\ccHtmlIndexC[concept]{Random_access_circulator}

Note: This specification is a revised version based on the \CC\ 
Standard~\cite{cgal:ansi-is14882-98}, which is available now. In particular,
iterator traits are now assumed and required.

Iterators in the \stl\ were tailored for linear sequences. The
specialization for circular data structures leads to slightly
different requirements which we will summarize in the {\em
  circulators\/} concept. The main difference is that a circular data
structure has no natural past-the-end value. As a consequence, a
container supporting circulators will not have an {\tt end()}-member
function. The semantic of a circulator range differs from the semantic
of an iterator range. For a circulator $c$ the range $\left[c,
  c\right)$ denotes the sequence of all elements in the data
structure. For iterators, this range defines the empty sequence. A
separate test for an empty sequence has been added to the circulator
requirements: A comparison $c ==$ {\tt NULL} for a circulator $c$ is
true for an empty sequence. As for \CC, we recommend the use of 0
instead of {\tt NULL}.

Similar to \stl\ iterators, we distinguish between forward,
bidirectional, and random access circulators\footnote{Input
  circulators are a contradiction, since any circulator is supposed to
  return once to itself.  Output circulators are not supported since
  they would be indistinguishable from output iterators.}. Most
requirements for circulators are equal to those for iterators. We
present the changes, please refer to \cite[chapter 18]{cgal:ms-strg-96} or
\cite{cgal:ansi-is14882-98} for the iterator requirements.

{\bf Past-the-end value:} There is no past-the-end value for circulators.

{\bf Singular values:} There are no singular values for
 circulators\footnote{Since circulators must be implemented as classes
  anyway, there is no need to allow singular values for them. An
  un-initalized circulator does not have a singular value, but is
  supposed to refer to an empty sequence.}


{\bf Empty sequence:} The comparison $c ==$ {\tt NULL} (or $c == 0$)
for a circulator $c$ is {\tt true} if $c$ denotes an empty sequence,
and {\tt false} otherwise.

{\bf Dereferenceable values:} A circulator that does not denote an
empty sequence is dereferenceable.

{\bf Reachability:} Each dereferenceable circulator can reach itself
with a finite and non-empty sequence of applications of \ccc{operator++}.

{\bf Ranges:} For any circulator $c$ the range $\left[c, c\right)$ is
a valid range. If the circulator refers to an empty sequence, the range
$\left[c, c\right)$ denotes the empty range. Otherwise the circulator
is dereferenceable and the range $\left[c, c\right)$ denotes the
sequence of all elements in the data structure. {\em Remark:} When a
circulator is used in a place of an iterator, as, for example, with an \stl\ 
algorithm, it will work as expected with the only exception that,
in STL algorithms,
the range $\left[c, c\right)$ denotes always the empty range.
This is
not a requirement, but a consequence of the requirements
stated here and the fact that the \stl\ requirements for iterator
ranges are based on the {\tt operator++} and the {\tt operator==},
which we use for circulators as well. In principle, we face here the 
difference between a {\tt while} loop and a {\tt do-while} loop.

{\bf Types:} For a circulator of type $C$ the following local types
are required: 

\begin{tabbing}
    \ccTexHtml{{\tt C::iterator\_category}\ \ \ \ \ \= \kill}{}
    {\tt C::value\_type}        \>  value type the circulator refers to.\\
    {\tt C::reference}          \>  reference type used for the return type
                                    of {\tt C::operator*()}.\\
    {\tt C::pointer}            \>  pointer type used for the return type of 
                                    {\tt  C::operator->()}.\\
    {\tt C::size\_type}         \>  unsigned integral type that can hold 
                                    the size of a sequence\\
    {\tt C::difference\_type}   \>  signed integral type that can hold 
                                    the distance between two circulators.\\
    {\tt C::iterator\_category} \>  circulator category.
\end{tabbing}

\subsection*{Forward Circulators}

In the following, we assume that {\tt a} and {\tt b} are circulators
of type {\tt C}, {\tt r} is of type {\tt C\&} (is assignable), and
{\tt T} denotes the value type of {\tt C}.  Let {\tt D} be the
distance type of {\tt C}.  As for \CC, we recommend the use of 0
instead of {\tt NULL}.

\begin{tabbing}
    \ccTexHtml{{\tt C::iterator\_category}\ \ \ \ \ \= \kill}{}
    {\tt C()}         \> a circulator equal to \ccc{NULL} denoting an 
                         empty sequence.\\
    {\tt a == NULL}   \> Returns {\tt true} if {\tt a} denotes an empty
                         sequence, {\tt false} otherwise. \\
                      \> For simplicity, {\tt NULL == a} is not required. The\\
                      \> behavior for comparisons with pointer-like 
                         values different than {\tt NULL} \\
                      \> is undefined. A runtime assertion is 
                         recommended.\\
    {\tt a != NULL}   \> Returns {\tt !(a == NULL)}. \\
    {\tt ++r}         \> Like for forward iterators, but a dereferenceable
                         circulator {\tt r} will always\\
                      \> be dereferenceable after {\tt ++r} (no
                         past-the-end value). {\em Precondition:} {\tt
                           r} \\
                      \> does not denote an empty sequence.\\
    {\tt r++}         \> Same as for {\tt ++r}.\\
    {\tt C::iterator\_category} \>  circulator category {\tt 
                                      CBP\_Forward\_circulator\_tag}.
\end{tabbing}

\subsection*{Bidirectional Circulators}

The same requirements as for the forward circulators hold for
bidirectional iterators with the following change of the iterator
category:

\begin{tabbing}
    \ccTexHtml{{\tt C::iterator\_category}\ \ \ \ \ \= \kill}{}
    {\tt C::iterator\_category} \>  circulator category {\tt 
                                CBP\_Bidirectional\_circulator\_tag}.
\end{tabbing}


\subsection*{Random Access Circulators}
\label{sectionMinCircleRequ}

The same requirements as for the bidirectional circulators hold for
random access iterators with the following changes and extensions. 

The idea of random access extends naturally to circulators using
equivalence classes modulus the length of the sequence. With this in
mind, the additional requirements for random access iterators hold
also for random access circulators. The only exception is that the
random access iterator is required to provide a total order on the
sequence, which a circulator cannot provide\footnote{One might define
  an order by splitting the circle at a fixed point, e.g.~the start
  circulator provided from the data structure. This is what the
  adaptor to iterators will do. Nonetheless, we do not require this
  for circulators.}.

The difference of two circulators is not unique as for iterators. A
reasonable requirement demands that the result is in a certain range
$[1-\mbox{{\it size, size}}-1]$, where {\it size} is the size of the
sequence, and that whenever a circulator {\tt a} is fixed that
the differences with all other circulators of the sequence form a
consistent ordering.

For the adaptor to iterators a minimal circulator
$d_{\mbox{\footnotesize min}}$ is required for which the difference $c
- d_{\mbox{\footnotesize min}}$ to all other circulators $c$ is non
negative.

\begin{tabbing}
    \ccTexHtml{{\tt C::iterator\_category}\ \ \ \ \ \= \kill}{}
    {\tt b - a}   \> limited range and consistent ordering 
                       as explained above.\\
    {\tt a.min\_circulator()}   \>  returns the minimal circulator from the 
                       range $[a,a)$.\\
    {\tt C::iterator\_category} \>  circulator category {\tt 
                                CBP\_Random\_access\_circulator\_tag}.
\end{tabbing}

\subsection*{Const Circulators}

As with iterators, we distinguish between circulators and const
circulators. The expression {\tt *a = t} with {\tt t} of type {\tt T}
is valid for mutable circulators. It is invalid for const circulators.

\subsection*{Circulators in Container Classes}

For a container {\tt x} of type {\tt X} that supports circulators
{\tt c} the following naming convention is recommended:

\begin{tabbing}
    \ccTexHtml{{\tt X::Const\_circulator}\ \ \ \ \= \kill}{}
    {\tt X::Circulator}  \> the type of the mutable circulator.\\
    {\tt X::Const\_circulator}  \> the type of the const circulator.\\
    {\tt c = x.begin()}         \> the start circulator of the sequence.
        It is of type {\tt X::Circulator} for a \\
      \> mutable container or {\tt X::Const\_circulator} for
         a const container. There \\ 
      \> must not be an {\tt end()} member function.
\end{tabbing}

If a container will support iterators and circulators, the member
function {\tt circulator\_begin()} is proposed. However, the support
of iterators and circulators simultaneously is not recommended, since
it would lead to fat interfaces. The natural choice should be
supported, the other concept will be available through adaptors.


\subsection*{Example}


A generic {\tt contains} function accepts a range of circulators and a
value. It returns \ccc{true} if the value is contained in the sequence
of items denoted by the range of circulators.  As usual for circular
structures, a {\tt do}-{\tt while} loop is preferable, such that for
the specific input, {\tt c == d}, all elements in the sequence are
reached. Note that the example simplifies if the sequence is known to
be non-empty, which is for example the common case in polyhedral
surfaces where vertices and facets have at least one incident edge.

\begin{ccExampleCode}
template <class Circulator, class T>
bool contains( Circulator c, Circulator d, const T& value) {
    if (c != 0) {
        do {
            if (*c == value)
                return true;
        } while (++c != d);
    }
    return false;
}
\end{ccExampleCode}

\end{ccRefConcept}

% +-----------------------------------------------------+
% EOF
