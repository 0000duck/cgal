% +------------------------------------------------------------------------+
% | CGAL Reference Manual: Forward_circulator_from_iterator.tex
% +------------------------------------------------------------------------+
% | Requirements for circulators in analogy to STL iterators.
% | Adaptors between circulators and iterators.
% | Proposal for CGAL.
% |
% | 11.10.1996   Lutz Kettner
% | 
% | \RCSdef{\circRev}{$Revision$}
% | \RCSdefDate{\circDate}{$Date$}
% +------------------------------------------------------------------------+

% +-----------------------------------------------------+
\begin{ccRefClass}{Circulator_from_iterator<I>}
\label{pageCircFromIterRef}

\ccDefinition
  
The adaptor \ccClassTemplateName\ converts two iterators of type
\ccc{I}, a begin and a past-the-end value, to a circulator of equal
category. The iterator must be at least of the forward iterator
category. The circulator will be mutable or non-mutable according to
the iterator. Iterators provide no \ccc{size_type}. This adapter
assumes \ccc{std::size_t} instead.


\ccInclude{CGAL/circulator.h}

\ccTypes
\ccSetThreeColumns{typedef const T&}{const_reference;}{}

\ccUnchecked\ccTypedef{typedef I         iterator;}{}

In addition all types required for circulators are provided.

\ccCreation
\ccCreationVariable{c}

\ccConstructor{Circulator_from_iterator();}{%
  a circulator \ccVar\ on an empty sequence.}

\ccConstructor{Circulator_from_iterator(const I& begin,
  const I& end, const I& cur = begin);}{%
  a circulator \ccVar\ initialized to refer to the element
  \ccStyle{*cur} in  a range {\tt [}\ccStyle{begin}{\tt
    ,}\ccStyle{end}{\tt )}.
  The circulator \ccVar\ refers to a empty sequence
  if \ccStyle{begin==end}.
}

\ccConstructor{Circulator_from_iterator(
  const Circulator_from_iterator<I,T,Size,Dist>& d,
  const I& cur);}{%
  a copy of circulator $d$ referring to the element \ccc{*cur}.
  The circulator \ccVar\ refers to a empty sequence
  if $d$ does so.
}

\ccOperations

The adaptor conforms to the requirements of the respective circulator
category. An additional member function \ccc{current_iterator()}
returns the current iterator pointing to the same position as the
circulator does.

\ccSeeAlso

\ccc{Container_from_circulator},
\ccc{Circulator_from_container},
\ccc{Circulator}.

\ccExample

The following program composes two adaptors -- from an iterator to a
circulator and back to an iterator. It applies an \stl\ sort algorithm
on a \stl\ vector containing three elements. The resulting vector will
be {\tt [2 5 9]} as it is checked by the assertions. The program is
part of the \cgal\ distribution.

%\ccIncludeExampleCode{examples/Circulator/circulator_prog1.C}
\ccIncludeExampleCode{Circulator/circulator_prog1.C}

Another example usage for this adaptor is a  random access circulator
over the built-in C arrays. Given an array of type {\tt T*}  with a
begin pointer {\tt b} and a past-the-end pointer {\tt e} the adaptor
\ccStyle{Circulator_from_iterator<T*> c(b,e)} is a random access circulator 
\ccStyle{c} over this array.

\end{ccRefClass}

% +-----------------------------------------------------+
% EOF
