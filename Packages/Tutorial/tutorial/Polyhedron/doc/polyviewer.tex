% viewer

% show how to 
% iterate
% circulate
% enrich primitives

This tutorial implements an interactive basic polyhedron viewer based
on the \cgalpoly\ with the default configuration.  This viewer
demonstrates basic functionalities of a \cgalpoly . We show the mesh
traversal based on the \italic{iterators} and the \italic{circulators}
during the assembly of facet polygons for basic OpenGL rendering. The
viewer is then extended by customizing the \poly\ with extra
attributes and functionalities. This enriched polyhedron supports
facet and vertex normals for rendering, an axis-aligned bounding box
of the polyhedron, and provides geometry items specialized with
algorithmic flags. A number of rendering modes are available to the
user depending on the choices of lighting, shading and edge
superimposing. The superimposition of the control mesh on the
subdivision surfaces is implemented for the quad-triangle scheme with
a boolean flag of the halfedge item, this flag being automatically
propagated to the subdivided edges during subdivision (\figurename\
\ref{fig:quad-triangle}).

% basic combinatorial algorithms

The tutorial demonstrates basic combinatorial algorithms on the
connectivity of the polyhedron by counting the number of connected
components and boundaries, and deducing the combinatorial genus of the
active polyhedron.

% OBJ format import/export

Although \cgal already features OFF file import and export, we show
how to import a polyhedron file in the OBJ format based on the
\italic{modifier callback mechanism} and the \italic{incremental
builder}. The OBJ file exporting is rather simply based on mesh
traversal.

% camera

The camera and transformation states are automatically adjusted when a
new polyhedron is loaded so as to originally view the model in all. A
function snapshots the camera and transformation states for the sake
of comparing two models with the same viewpoint.

% goodies 

The viewer also features a raster output of the current client image
to the clipboard, as well as a vectorial output to a postscript
file. Note however that the latter functionality is not based on the
painter algorithm and performs instead a simple z-sorting of the
polygons based on each facet barycenter and the current viewpoint.
