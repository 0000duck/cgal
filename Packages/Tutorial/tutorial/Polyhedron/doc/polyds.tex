A \cgalpoly\ represents polyhedron surfaces in three dimensions
consisting of vertices, edges, facets and incidence relations among them.
%\footnote{\href{http://www.cgal.org/Manual/doc_html/basic_lib/
%Polyhedron_ref/Class_Polyhedron_3.html}{http://www.cgal.org}}.
The incidence relation (or connectivity) of the \poly\ is
internaly represented by the \cgalhds . 
The \poly\ only maintains the combinatorial integrity of the 
polyhedral surface (using Euler operations) and does not 
consider any geometry information of the primitives. 
As all CGAL geometric entities, geometry information of \poly\ 
are templated by the kernel.
%\footnote{\href{http://www.CGAL.org/Manual/doc_html/frameset/
%fsKernel.html}{CGAL kernel}}.

% ------------------------------------------------------------------------
% Usage the default polyhedron
\subsubsection{Declaration}
\input poly_decl

% ------------------------------------------------------------------------
% specialize a polyhedron (enriched polyhedron?)
\subsubsection{Specialized Polyhedron}
TODO: Enriched polyhedron
%\input poly_enrich

% ------------------------------------------------------------------------
\subsubsection{Polyhedron traversal}
%\subsubsection{Iteration and Circulation}
\input poly_travel
