The \emph{C}ombinatorial \emph{S}ubdivision \emph{L}ibrary 
(CSL) is designed based on the policy-based design 
\cite{Alexandrescu:2001:MCD}.
The policy-based design assembles a class
(called \emph{host}) with complex behavior out of many 
small behaviors (called \emph{policies}).
Each policy defines an interface for a
specific behavior. CSL proposes a 
generic subdivision solution as a \emph{refinement function}
parameterized with the \emph{geometry smoothing rules}.
Subdivisions in CSL are build as proper combinations of the
refinement functions and the geometry policy classes.
The refinement function refines the control mesh,
maintains the correspondence between the control mesh and refined
mesh, and applies the smoothing stencils provided by the policy
class. For example, Catmull-Clark subdivision~\cite{cc} is structured
\begin{figure}[h]
    \centering{\includegraphics[width=7.0cm]{figs/sqrt3}}
    \caption{$\sqrt{3}$ subdivision of the mannequin mesh.}
    \label{fig:sqrt3}
    \vspace{0.5cm}
\end{figure}
as a quadralization function parameterized with the Catmull-Clark
smoothing rules.


\begin{lstlisting}
void CatmullClark_subdivision(Polyhedron& p) {    
  quadralize_polyhedron
                <CatmullClark_rule<Polyhedron>>(p);  
}
class CatmullClark_rule {
public:
  void facet_rule(  Facet_handle  facet, Point& pt);
  void edge_rule(Halfedge_handle   edge, Point& pt);
  void vertex_rule(Vertex_handle vertex, Point& pt);
};
\end{lstlisting}
\noindent The \CodeFmt{quadralize\_polyhedron<>()} 
is the host function refining the input mesh
and the \CodeFmt{CatmullClark\_rule} is the policy 
class applying the Catmull-Clark stencils.
The refinement functions are implemented based on the
Euler operations or the modifier callback mechanism.
The refinement functions also maintain the 
correspondence with the stencil, i.e., the submesh 
centered around the given facet, edge, or
vertex, and the smoothing point.
The smoothing point is calculated by calling the 
policies, e.g., the \CodeFmt{facet\_rule()}, the 
\CodeFmt{edge\_rule()}, and the \CodeFmt{vertex\_rule()} 
respectively. Inside a policy, applying the 
stencil is simplified to the mesh traversal of a 
1-ring neighborhood which can be done with the 
circulators. Following example illustrates  
the policy of the facet-stencil in Catmull-Clark 
subdivision.
\begin{lstlisting}
void facet_rule(Facet_handle facet, Point& point) {
  Halfedge_around_facet_circulator hcir 
                            = facet->facet_begin();
  Vector vec = hcir->vertex()->point() - ORIGIN;
  ++hcir;
  do {
    vec = vec + hcir->vertex()->point();
  } while (++hcir != facet->facet_begin());
  point = ORIGIN + vec/circulator_size(hcir);
}
\end{lstlisting}

This policy-based approach offers a convenient way to
specialize a subdivision with the template smoothing rules.
CSL currently supports Catmull-Clark, 
Loop, Doo-Sabin, $\sqrt{3}$ and Quad-Triangle
subdivisions. %(\figurename\ \ref{fig:subzoo}).
Though demonstrated with a specific enriched \poly\ in our 
polyhedron viewer, CSL accepts any polyhedron mesh specialized 
from the \poly\ with the \CodeFmt{Point} type defined in the vertex.  
