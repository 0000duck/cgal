\documentclass[letter,twoside,10pt]{article}
\usepackage{tutorial}
\usepackage[OT1]{fontenc}

% ------------------------------------------------------------------------
\usepackage{graphicx}
\usepackage{subfigure}
\usepackage{epsfig}
\usepackage{psfrag}
\usepackage{}
% used to write c++ code/algorithms
\usepackage{listings}
\usepackage{fancyvrb}

%\psdraft

% hyperref stuff
\usepackage{hyperref}
\hypersetup{
  pdftitle={Getting started with CGAL Polyhedron},
  pdfauthor={INRIA Geometrica},
  pdfsubject={A tutorial for CGAL},
  pdfkeywords={},
  pdfpagemode=UseThumbs,
  baseurl={http://www.cgal.org},
  colorlinks=true,
  linkcolor=black,
  anchorcolor=black,
  citecolor=black,
  filecolor=black,
  menucolor=black,
  pagecolor=black,
  urlcolor=blue,
  bookmarksopen=false,}
% end hyperref stuff

\lstset{language=C++, basicstyle=\scriptsize}

\graphicspath{{figs/}}
\def\figurename{Figure}
\def\tablename{Tableau}
\newcommand{\italic}[1]{\emph{#1}} 

% ------------------------------------------------------------------------
\newcommand\IL{{\itshape left}}
\newcommand\IR{{\itshape right}}
\newcommand\IM{{\itshape middle}}
\newcommand\IT{{\itshape top}}
\newcommand\IB{{\itshape bottom}}

% ------------------------------------------------------------------------
\newcommand{\CodeFmt}[1]{{\small\texttt{#1}}}

\def\kernel{\CodeFmt{Kernel}}

\def\cgalpoly{\CodeFmt{CGAL::Polyhedron\_3}}
\def\poly{\CodeFmt{Polyhedron\_3}}
\def\polytrait{\CodeFmt{PolyhedronTraits\_3}}
\def\polyitem{\CodeFmt{PolyhedronItems\_3}}
\def\polybuilder{\CodeFmt{Polyhedron\_incremental\_builder\_3}}

\def\cgalhds{\CodeFmt{CGAL::HalfedgeDS}}
\def\hds{\CodeFmt{HalfedgeDS}}
\def\hdsitem{\CodeFmt{PolyhedronItems}}

% L.K. -------------------------------------------------------------------
\newcommand{\CC}{C\raise.08ex\hbox{\texttt{++}}}
\newcommand{\openmesh}{\textsc{OpenMesh}}
\newcommand{\opensg}{\textsc{OpenSG}}
\newcommand{\cgal}{\textsc{Cgal}}
\newcommand{\stl}{\textsc{Stl}}


% =========================================================================
\begin{document}

% TITLE
% ------------------------------------------------------------------------
\date{}
\title{{\LARGE {\sffamily\bfseries Getting started with CGAL
Polyhedron}}\\ the example of subdivision surfaces}
\author{\small
\sffamily Le-Jeng Shiue\footnote{SurfLab, University of Florida}
\and \small
\sffamily Pierre Alliez\footnote{GEOMETRICA, INRIA Sophia-Antipolis}
\and \small
\sffamily Radu Ursu\footnote{GEOMETRICA, INRIA Sophia-Antipolis}}
\maketitle

\thispagestyle{empty}

% ABSTRACT
% ------------------------------------------------------------------------
\abstract{
%% This document is a tutorial on how to get
%% started with the halfedge data structure provided by CGAL, the
%% Computational Geometry Algorithm Library. Assuming the reader to be
%% familiar with the C++ template mechanisms and the key concepts of the
%% STL (Standard Template Library), we describe three different
%% approaches with increasing level of sophistication for implementing
%% mesh subdivision schemes. The simplest approach uses simple Euler
%% operators to implement the $\sqrt{3}$ subdivision scheme applicable to
%% triangle meshes. A second approach overloads the incremental builder
%% already provided by CGAL to implement the quad-triangle subdivision
%% scheme applicable to polygon meshes. The third approach is generic and
%% offers a convenient way to design its own subdivision scheme through
%% the definition of rule templates. Catmull-Clark, Loop and Doo-Sabin
%% schemes are illustrated using the latter approach. Two companion
%% applications, one developed on Windows with MS .NET, MFC and OpenGL,
%% and the other developed for both Linux and Windows with Qt and OpenGL,
%% implement the subdivision schemes listed above, as well as several
%% functionalities for interaction, visualization and raster/vectorial
%% output.
}

\vskip 3mm

\noindent {\bf Keywords:}
                 CGAL library,
                 tutorial,
                 halfedge data structure, 
                 polygon surface mesh,
                 subdivision surfaces,
                 quad-triangle,
                 $\sqrt{3}$,
                 Loop,
                 Doo-Sabin,
                 Catmull-Clark,
                 OpenGL.

% INTRODUCTION                 
% ------------------------------------------------------------------------
\section{Introduction}
%\input intro

% PREREQUISITES
% ------------------------------------------------------------------------
\section{Prerequisites}
%\input prereq 

% ------------------------------------------------------------------------
\section{Polyhedron Data Structure: Fundamentals}
\input polyhedron 

% ------------------------------------------------------------------------
\section{Polyhedron Data Structure: Rendering and Manipulation}

% rendering controls of the polyhedron (wire-frame, surface, normals>generation)
\subsection{Rendering}
%\input ??

% user manipulations of the polyhedron (arc ball?) 
\subsection{Manipulation} 
%\input ??


% ------------------------------------------------------------------------
\section{Design and Implemenation of Subdivisions}

\subsection{Subdivision Surfaces}
\input subdivision

% connectivity ops: specific polyhedron algorithms (sqrt3 subdivisions) 
\subsection{$\sqrt{3}$-Subdivision using Euler Operators}
%\input sqrt3

% ------------------------------------------------------------------------
% inc builder: specific polyhedron algorithms (qt subdivisions) 
\subsection{Quad-triangle Subdivision using Incremental Builder}
\label{sec:subdivision_builder}
%\input qt

% ------------------------------------------------------------------------
% templated rules: a generic framework for subdivisions 
\subsection{Subdivision using a rule template}
\label{sec:subdivision_rule}
%\input subtempl

% APPLICATION DEMO
% ------------------------------------------------------------------------
\section{Application demo}
%\input demo


% ------------------------------------------------------------------------
% REFERENCES
\bibliographystyle{alpha}
\bibliography{tutorial}


\end{document}
