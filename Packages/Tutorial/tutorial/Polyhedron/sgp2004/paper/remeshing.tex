\label{sec:remeshing}

% few words on remeshing (why it became a key ingredient of mesh
% processing).

The process of replacing an arbitrary mesh by a faithful one with
improved sampling, regularity and triangle quality is called
remeshing. Most meshes, be they modeled or acquired by a scanning
device, can this way be considerably improved in order to ease the
visualization as well as the processing (see \cite{ss-dgp-01}). CGAL
has proven successful at easing the combination of several components
used for remeshing. We now give some details over the software
components used for isotropic~\cite{acdi-isr-03} as well as for
anisotropic remeshing~\cite{acdld-apr-03}.

\subsection{Isotropic Remeshing}

% isotropic remeshing, what for ?

Isotropic remeshing leads to well-shaped triangles, and thus
high-quality meshes for applications where the compactness of the mesh
elements is critical.

% main idea

The main idea of the remeshing technique described in
\cite{acdi-isr-03} is to cast the problem into the
one of building a 2D centroidal Voronoi diagram over a set of newly
sampled points. The key is to first parameterize the original mesh to
obtain a bijective mapping and minimize the distortion due to the
flattening process. The sampling and meshing stages are then
considerably simpler on the planar parameter space (see
Fig.\ref{fig:isotropic}).

% figure isotropic remeshing
\begin{figure}
  \centering
  \epsfig{file=figs/isotropic.eps, width=\linewidth}
  \caption{Isotropic remeshing.}
  \label{fig:isotropic}
\end{figure}


% CGAL components

Two key data structures from CGAL have been used to assemble the
algorithm:

\begin{itemize}

\item 
The \emph{Polyhedron} has been used to represent the input mesh, after
having enriched its primitives to store $u,v$ parameter coordinates
per halfedge (hence per corner) in addition to the 3D coordinates per
vertex.

\item 
After initial sampling the \emph{2D constrained Delaunay
triangulation} has been used together with Lloyd's clustering
algorithm to build a weighted centroidal Voronoi diagram over the new
sample points in parameter space. The fast point location in a
triangulation provided by CGAL has been used to determine the final 3D
coordinates of the sample points.

\end{itemize}

The CGAL kernel is used as template parameter in these two data
structures as well as in all algorithms involved in our remeshing
technique (parameterization, error diffusion for sampling, Lloyd's
clustering for sample placement and point location). The kernel thus
nicely unifies all components of the algorithm and provides us with
genericity for all simple geometric primitives and predicates. An
important feature provided by CGAL is the \emph{filtered}
kernel~\cite{bbp-iayed-01,p-iaeia-99}, which gives robustness required
for complex models by mixing exact and floating point arithmetic in a
transparent manner for the programmer.

\subsection{Anisotropic Remeshing}

% anisotropic remeshing, what for ?

Beside quality of the mesh elements, remeshing techniques are also
concerned by approximation efficiency. When accurate representation is
needed, a strategic sizing, alignment and aspect ratio of the mesh
elements is crucial. This is a consequence of the natural anisotropic
nature of generic surfaces.

% main idea

The main idea of the anisotropic remeshing technique described in
\cite{acdld-apr-03} consists of tracing on anisotropic areas two 
orthogonal sets of streamlines in a conformal parameter space to align
mesh edges along principal curvature lines and therefore respect the
local symmetries. A careful control of the density of streamlines
allows us to improve the mesh efficiency so as to match the optimality
conditions for the $\Ltwo$ metric in the limit. On spherical areas the
algorithm simply switches to isotropic point sampling strategy. This
remeshing technique produces quad-dominant meshes as illustrated by
Fig.\ref{fig:anisotropic}.

% figure anisotropic remeshing
\begin{figure}
  \centering
  \epsfig{file=figs/anisotropic.eps, width=\linewidth}
  \caption{Anisotropic remeshing.}
  \label{fig:anisotropic}
\end{figure}

As for the isotropic remeshing technique described above, the
\emph{Polyhedron} has been used to represent the input mesh. Its
primitives have been enriched so as to store two $u,v$ parameter
coordinates per halfedge, and both 3D and 2D curvature tensors per
vertex, or per halfedge on feature vertices.

Extracting the new quad-dominant mesh is another matter since it has
to be deduced from intersection, then straightening of streamlines.
In anisotropic regions, we traced lines of curvature using polyline
approximations while we used regular sample points on spherical
regions. The \italic{vertices} will therefore be the intersections of
curvature lines, and the isotropic samples that we spread. While the
isotropic samples do not require any specific treatment, computing the
line intersection has to be performed.  In order to perform these
intersections quickly, as well as to prepare the straightening stage,
we make use of the 2D constrained Delaunay triangulation (CDT) from
CGAL, in parameter space. The key idea is to embed the quad-dominant
polygon mesh within a triangulation through the constrained edges. We
enter all the little segments defining the streamlines sequentially as
constraints.  Finally, the isotropic samples are added as vertices in
the CDT. The vertices, intersection of streamlines, have
\italic{automatically} been added to the CDT since two intersecting
edge constraints will generate a vertex insertion. The
edge-straightening stage is applied by repeatedly eliminating vertices
from the triangulation while adding new constrained edges until only
intersection and isotropic vertices remain (see
Fig.\ref{fig:trimming}).

% figure trimming
\begin{figure}
  \centering
  \epsfig{file=figs/trimming.eps, width=\linewidth}
  \caption{Remeshing phase: a dome-like shape is sampled with lines of
  curvatures. All the curvature line segments (red/blue) and the
  feature edges (green) are added as constraints in a CDT in parameter
  space. The CDT creates a dense triangulation; a rapid vertex
  decimation (A,B) then suppresses most small edges, and leaves only
  few vertices, defining a coarse polygonal mesh. Adding constraint
  edges to the umbilic (center) point takes care of the near-spherical
  cap.}
  \label{fig:trimming}
\end{figure}


