\label{sec:remeshing}

% few words on remeshing (why it became a key ingredient of mesh
% processing).

The process of replacing an arbitrary mesh by a faithful one with
improved sampling, regularity and triangle quality is called
remeshing. Most meshes, be they modeled or acquired by a scanning
device, can this way be considerably improved in order to ease the
visualization as well as the processing (see \cite{ss-dgp-01}). CGAL
has proven successful at easing the combination of several components
used for remeshing. We now give some details over the components used
for isotropic~\cite{acdi-isr-03} as well as for anisotropic
remeshing~\cite{acdld-apr-03}.

\subsection{Isotropic Remeshing}

% isotropic remeshing, what is that ?

Isotropic remeshing leads to well-shaped triangles, and thus
high-quality meshes when the notion of quality is related to the shape
of the triangles. Such meshes are important for applications where the
quality of the mesh elements is critical.

% main idea

The main idea of the remeshing technique described in
\cite{acdi-isr-03} is to cast the problem into the
one of building a 2D centroidal Voronoi diagram over a set of newly
sampled points. The key is to first parameterize the original mesh to
obtain a bijective mapping and minimize the distortion due to the
flattening process. The sampling and meshing stages are then
considerably simpler on the planar parameter space (see
Fig.\ref{fig:isotropic}).

% figure isotropic remeshing
\begin{figure}
  \centering
  \epsfig{file=figs/isotropic.eps, width=\linewidth}
  \caption{Isotropic remeshing.}
  \label{fig:isotropic}
\end{figure}


% CGAL components

Two data structures from CGAL have been used to assembly the
algorithm:

\begin{itemize}

\item 
The \emph{Polyhedron} has been used to represent the input mesh, after
having enriched its primitives to store $u,v$ parameter coordinates
per halfedge in addition to the 3D coordinates per vertex.

\item 
After initial sampling the \emph{constrained Delaunay triangulation}
has been used together with Lloyd's clustering algorithm to build a
weighted centroidal Voronoi diagram over the newly sampled points in
parameter space. The fast point location in a triangulation provided
by CGAL has been used to determine the 3D coordinates of the sample
points.

\end{itemize}

The CGAL kernel is used to template these two data structures as well
as all algorithms involved in our remeshing technique
(parameterization, error diffusion for sampling, Lloyd's clustering
for sample placement and point location), which nicely unifies all
components of the algorithm and provides us genericity for all
geometric entities and predications. Notice that the filtered kernel
provided by CGAL (add ref) provides us with the robustness required
for complex models by mixing exact and approximate arithmetic in a
transparent manner for the programmer.

\subsection{Anisotropic Remeshing}

Building blocks taken from CGAL:
- Polyhedron (enriched primitives to store curvature tensors per
vertex or per corner, plus uv coordinates)
- 2D constrained Delaunay triangulation

Algorithms:
- Conformal parameterization
- Curvature tensor estimator
- Placement of streamlines
- A bunch of algorithms which I cannot list here

