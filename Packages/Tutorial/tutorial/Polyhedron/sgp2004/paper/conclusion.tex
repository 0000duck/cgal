Coming back to our claims in the introduction, we gave ample examples 
that the \cgal\ polyhedron is indeed a fast, storage efficient,
highly flexible and customizable solution for a mesh data
structure. We also believe that our examples are readable, easy to
understand, and easy to modify. 

The performance comparison with \openmesh\ was brief, and opens new
questions, for example, about the right set of Euler operations to
support certain algorithm classes efficiently. Such sets could be cast
into generic concepts, and algorithms could be written for such
concept, while different mesh data structures, for example \cgal\'s and
\openmesh's, provide the right interface.
