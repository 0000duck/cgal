% polyhedron.tex
% ------------------------------------------------------------------------

The design of \cgalpoly\ \cite{k-ugpdd-99} consists of four components:

\paragraph*{A Geometric Traits Concept} is defined as for all \cgal\
data structures that describes the types for geometric primitives,
predicates and constructions. Relevant for us is the point type
defined here. A \cgal\ geometric kernel is a valid model for this
geomtric traits concept.

\paragraph*{Items}  are collectively the  vertices, halfedges, and
facets. They contain the pointers referring to incident items (see
Fig.~\ref{fig:halfedge}) and the additional attributes including the
geometry, such as points in vertices. Mandatory for \cgalpoly\ are the
\CodeFmt{next}, \CodeFmt{opposite}, and \CodeFmt{vertex} incidence
pointers, and a three-dimensional point in vertices. All other
incidences, also the back pointers from a vertex or facet to an
incident halfedge, are optional, but if they are provided, they will
be automatically maintained in the other levels of this design. By
default all incidences are provided plus a plane equation in facets.

In fact, the design does not use pointers, but the abstraction of
\emph{handles}, a.k.a. the \emph{trivial iterator\/} concept that
requires the dereference operator, but no traversal of a sequence.
These handle types are available to the user when they extend items
with additional attributes and references, e.g., material properties
for rendering or links in multi-resolution representations.

\paragraph*{The Halfedge Data Structure} is a container class managing
the storage for items. Two solutions are provided; one stores items in
a list that allows for efficient insertion and removal, and one
stores items in an array that allows random access and more compact
storage, but is less flexible in removing and inserting items, e.g.,
inserting items beyond the currently reserved capacity causes the
array to reallocate and update all invalidated handles.

The type dependency between the different item types is not hardcoded
on the items level. Instead, the halfedge data structure defines the
iterator and handle types and instantiates the corresponding items
with these types. This gives great flexibility in extending and
exchanging individual item types. The internal library
realization of this design is a brain twister, but since the user is
not exposed to it we skip details and refer to~\cite{k-ugpdd-99}.

\paragraph*{CGAL::Polyhedron\_3} is the top-level interface class that
assembles the other pieces in a convenient user-friendly interface.
It offers iterators for all items and defines \emph{circulators} for
the circular sequence of halfedges around a vertex and for halfedges
around a facet. Circulators are a new concept in \cgal\ that
corresponds closely to iterators, but is tailored for circular
sequences.  The main difference to iterators is the lack of a
past-the-end position that leads to the typical ideom of a
do-while-loop instead of the usual while-loop for iterators.  Let us
write a small function rendering a polyhedral surface with convex
facets in OpenGL with double coordinates in non-reference counted
points and making a bold, but informed guess about its memory layout.
%
\begin{lstlisting}
typedef CGAL::Polyhedron_3< CGAL::Simple_cartesian<double> > 
                                                 Polyhedron;
void render( const Polyhedron& P) {
  Polyhedron::Facet_const_iterator i;
  for ( i = P.facets_begin(); i != P.facets_end(); ++i) {
    glBegin( GL_POLYGON);
    Polyhedron::Halfedge_around_facet_const_circulator c = 
        i->facet_begin();
    do {
      glVertex3dv( & (c->vertex()->point().x()));
    } while ( ++c != i->facet_begin());
    glEnd();
  }
}
\end{lstlisting}%
%
The sequential access is conventional, but the connectivity
modifications needs new ideas. The polyhedron protects the
combinatorial integrity of the underlying halfedge data structure.
Instead of updating individual handles in the items, it offers an
interface with Euler operations, so called because they preserve the
Euler-Poincar\'e invariance of a mesh. Examples of Euler operations
are the well-known edge collapse, split facet, join facets, etc.
In~\cite{Maentylae88} it was shown that a proper set of Euler
operators does not restrict the class of meshes one can create.

Nevertheless, Euler operations can be cumbersome for certain tasks,
especially if a mesh operation needs to work with intermediate invalid
states of the halfedge data structure. The \cgalpoly\ provides a
backdoor access with the \CodeFmt{CGAL::Modifier} class to the
underlying halfedge data structure, but assures that, when this access
finishes, the integrity of the data structure is checked.

The prime example for this backdoor use is an alternative way of
describing meshes in the indexed-facet-set format common in file
formats: First, points are defined with their coordinates, then facets
are defined as sequence of indices to the points on their boundary. An
efficient implementation of such a construction is provided with the
incremental builder class in \cgal. Examples for the different ways of
modifying the mesh are given the following sections.









