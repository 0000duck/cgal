TODO: to be written by lutz (probably a short extracts from lutz's papers)

The CGAL library is a joint effort between nine European
institutes~\cite{fgkss-dccga-00}. The goal of CGAL is to make
available to users in industry and academia some efficient solutions
to basic geometric problems developed in the area of computational
geometry in a C++ software library.

A polyhedron consists of \emph{topology primitives}, 
such as vertices and facets, and \emph{geometry attributes}, 
such as point positions and normals. 
In graphics modeling, rendering attributes, 
such as colors and texture coordinates, or algorithmic 
attributes are also part of a polyhedron mesh.
\cgalpoly\ relies on the halfedge data 
structure \cite{hds} to provide the connectivity of 
topology primitives of the represented polyhedron mesh. 
The connectivity describes the incidences between 
primitives. \poly\ supports generic attributes 
by the templated primitives in the concept of the polyhedron
items.

% ------------------------------------------------------------------------
A \cgalpoly\ represents polyhedron surfaces in three dimensions
consisting of vertices, edges, facets and incidence relations among them.
The \emph{incidence relation} (or connectivity) of the \poly\ is
internally represented by the \cgalhds . 
The \poly\ only maintains the combinatorial integrity of the 
polyhedral surface. The geometry primitives are the abstract 
entities. A \poly\ is specialized by the
template parameters of the kernel and primitives.

TODO: design of the polyhedron (polyhedron(hds(kernel)))

TODO: a specialized polyhedron demos the power and flexibility.

% ------------------------------------------------------------------------
TODO:modifications on a polyhedron that maintain the topology invariance:
known invariant ops (euler ops) and postcondition of the modifier. Unlike
fundamental data structures in STL, polyhedron need to support two
set of the entity operations: connectivity operations and geometry 
(including general attributes) operations. Similar to 
entities in a vector or list, the geometry access is
abstracted into the concepts of the iterator and circulator.
But connectivity modifications need special object functions
to ensure the topology integrity.

TODO: what kind of ops are needed to support different geometry algorithms?

\cgalpoly\ provides a set of operators for combinatorial 
modifications that maintain the combinatorial integrity
of the polyhedron mesh. Most of them are categorized as Euler operations
that assure the Euler-Poincar\'e equality.
These atomic operators can split/join of two facets, two vertices 
or two loops. They can also create/erase a center vertex or a facet.

In addition to these modification operators, 
the \polybuilder\ warrants the direct accesses and modifications
of the internal representation, i.e. the halfedge data structure, 
of the \poly\ in a controlled environment. In CGAL, the 
controlled environment is defined by a callback mechanism that
manage the interaction between the polyhedron and a modifier. 
A modifier is a function object allowed accesses 
of the internal representation when it is delegated by the 
polyhedron. A \polybuilder\ is then created to build  
the connectivity of the halfedge data structure.
The combinatorial integrity of the polyhedron is verified
after the return of the modifier. The \polybuilder\ and the
modifier are particularly useful for implementing file reader for 
common file formats.

TODO: why support modifier in addition to euler ops? euler ops are
more generic and \emph{standard} than modifier. If the connectivity operation
is fully implemented with only these \emph{standard} ops,
the data structure can be general from the connectivity operation.
