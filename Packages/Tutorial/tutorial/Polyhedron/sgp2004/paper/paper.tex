%\documentclass[letter,twoside,10pt]{article}
%\usepackage{tutorial}
%\usepackage[OT1]{fontenc}

\documentclass{egpubl}
\usepackage{SGP2004}

% --- for  Annual CONFERENCE
% \ConferenceSubmission % uncomment for Conference submission
% \ConferencePaper      % uncomment for (final) Conference Paper
% \STAR                 % uncomment for STAR contribution
% \Tutorial             % uncomment for Tutorial contribution
% \ShortPresentation    % uncomment for (final) Short Conference Presentation
%
% --- for  CGF Journal
% \JournalSubmission    % uncomment for submission to Computer Graphics Forum
% \JournalPaper         % uncomment for final version of Journal Paper
%
% --- for  EG Workshop Proceedings
\WsSubmission    % uncomment for submission to EG Workshop
% \WsPaper         % uncomment for final version of EG Workshop contribution
%
\electronicVersion % uncomment if producing the printed version

% ------------------------------------------------------------------------
\ifpdf \usepackage[pdftex]{graphicx} \pdfcompresslevel=9
\else \usepackage[dvips]{graphicx} \fi
%\usepackage{graphicx}
\usepackage{subfigure}
\usepackage{epsfig}
\usepackage{psfrag}
\usepackage{}
% used to write c++ code/algorithms
\usepackage{listings}
\usepackage{fancyvrb}

\PrintedOrElectronic

% prepare for electronic version of your document
\usepackage{t1enc,dfadobe}


%\psdraft

% hyperref stuff
\usepackage{hyperref}
\hypersetup{
  pdftitle={Getting started with CGAL Polyhedron},
  pdfauthor={INRIA Geometrica},
  pdfsubject={A tutorial for CGAL},
  pdfkeywords={},
  pdfpagemode=UseThumbs,
  baseurl={http://www.cgal.org},
  colorlinks=true,
  linkcolor=black,
  anchorcolor=black,
  citecolor=black,
  filecolor=black,
  menucolor=black,
  pagecolor=black,
  urlcolor=blue,
  bookmarksopen=false,}
% end hyperref stuff

\lstset{language=C++, basicstyle=\scriptsize}

\graphicspath{{figs/}}
\def\figurename{Figure}
\def\tablename{Tableau}
\newcommand{\italic}[1]{\emph{#1}} 

% ------------------------------------------------------------------------
\newcommand\IL{{\itshape left}}
\newcommand\IR{{\itshape right}}
\newcommand\IM{{\itshape middle}}
\newcommand\IT{{\itshape top}}
\newcommand\IB{{\itshape bottom}}

% ------------------------------------------------------------------------
\def\kernel{\lstinline!Kernel!}

\def\cgalpoly{\lstinline!CGAL::Polyhedron_3!}
\def\poly{\lstinline!Polyhedron_3!}
\def\polytrait{\lstinline!PolyhedronTraits_3!}
\def\polyitem{\lstinline!PolyhedronItems_3!}
\def\polybuilder{\lstinline!Polyhedron_incremental_builder_3!}

\def\cgalhds{\lstinline!CGAL::HalfedgeDS!}
\def\hds{\lstinline!HalfedgeDS!}
\def\hdsitem{\lstinline!PolyhedronItems!}


% ANDY -------------------------------------------------------------------
\def\tr{topology refinement}
\def\gm{geometry modification}

% L.K. -------------------------------------------------------------------
\newcommand{\CC}{C\raise.08ex\hbox{\texttt{++}}}


% TITLE
% ------------------------------------------------------------------------
\title{Algorithms on Meshes \\ 
       based on the CGAL Polyhedron}

% for anonymous conference submission please enter your SUBMISSION ID
% instead of the author's name (and leave the affiliation blank) !!

% we should submit with N.N.

\author[N.N.]{N.N.}

% \author[Le-Jeng Shiue, Pierre Alliez, Radu Ursu, and Lutz Kettner]
%        {Le-Jeng Shiue\thanks{CISE, University of Florida},
%         Pierre Alliez\thanks{GEOMETRICA, INRIA Sophia-Antipolis},
%         Radu Ursu\thanks{GEOMETRICA, INRIA Sophia-Antipolis}, and
%         Lutz Kettner\thanks{MPII, Saarbr\"ucken}
%        }
%  ------------------------------------------------------------------------

% if the Editors-in-Chief have given you the data, you may uncomment
% the following five lines and insert it here
%
% \volume{23}   % the volume in which the issue will be published;
% \issue{2}     % the issue number of the publication
% \pStartPage{201}      % set starting page


%-------------------------------------------------------------------------
\begin{document}

\maketitle

\begin{abstract}

% pierre: I suggest to keep the abstract without any formatting
% nor macros (ascii only)

The Computational Geometry Algorithms Library CGAL is an efficient
modern C++ library following the generic programming paradigm. CGAL
contains a flexible data structure for meshes in graphics, the
Polyhedron. The software designs are presented for the Polyhedron in
this paper. The flexibility of the Polyhedron is moreover evaluated
through the implementation of several geometry algorithms.  We first
implement a subdivision solution based on the generality of the
Polyhedron. The solution, decoupling the geometry rules from the
refinement, grants users the flexible control of the geometry rules.
Remeshing techniques based on a combination of Polyhedron and Delaunay
triangulation then demonstrate the versatility of the unified
framework provided by CGAL.  Last, several additional functionalities
such as minimum enclosing ball, convex hull, self intersection and
boolean operations are demonstrated on large meshes.  Extensible
algorithm models based on the robust and efficient Polyhedron and
geometry components provided by CGAL are aimed at speeding up the
research, and therefore benefit the geometry processing community.

\begin{classification} % according to http://www.acm.org/class/1998/
\CCScat{I.3.5}{Computer Graphics}{Computational Geometry and Object Modeling}
\end{classification}

\end{abstract}


\section{Introduction}
\input intro

\subsection{Related Work and Background}
\input prereq 

\section{CGAL Polyhedron Meshes}
\input polyhedron 

\section{Algorithm Zoo}

  \subsection{Subdivisions}
  \input subdivision

  \subsection{Remeshing}
  \input remeshing

  \subsection{Geometric Computing}
  \input gc

\section{Conclusions}
\input conclusion

% references
\bibliographystyle{eg-alpha}
\bibliography{paper}

\end{document}
