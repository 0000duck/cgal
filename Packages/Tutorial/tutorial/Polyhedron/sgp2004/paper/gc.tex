% geometric computing

There are geometric algorithms available in \cgal, not directly
processing a mesh, but that can be helpful in the mesh processing
context, for example, a fast self intersection test, the smallest
enclosing sphere, or the minimum width of a point set.

The self intersection test is based on the general algorithm for fast
box intersections~\cite{cgal:ze-fsbi-02}, applied to the bounding
boxes of individual facets, i.e. triangles, as a filtering step. The
triangles of  intersecting boxes are then checked in detail, i.e., if
they share common edge, they do not intersect, if they share a common
vertex, the may intersect or not depending on the opposite edge, and
otherwise the intersection test for triangles in the \cgal\ geometric
kernel is used to decide the intersection. A geometric kernel with
exact predicates is sufficient for this algorithm. 

if they are not adjacent in the data structur
boxes that are reported as are then 

\paragraph{Smallest enclosing spheres} are commonly used as bounding
volumes, for example, to speed up intersection tests.

All 7 digits after the decimal are identical between the exact and the
double implementation.

% \begin{tabular}{l|ll}
%   \textbf{smallest enclosing sphere} & \texttt{double} & exact
%     \texttt{qmpq} (non opt.)\\\hline
%   Bunny     & 0.02 &  14 \\
%   Lion vase & 0.15 & 396 \\
%   David     & 0.13 & 215 \\
%   Raptor    & 0.35  & 589
% \end{tabular}


% models used for benchmarking
\begin{figure}
  \centering
  \epsfig{file=figs/models.eps, width=\linewidth}
  \caption{Models used for benchmarking. 
           Complexity (in triangles):
           Bunny: 69,451,
           Lion vase: 400k,
           David (simplified version): 700k,
           Raptor: 2M.}
  \label{fig:models}
\end{figure}

\paragraph{Convex hulls} are commonly used as bounding
volumes, for example, to speed up intersection tests.

\noindent\hspace*{-3mm}%
{\small
\begin{tabular}{l|ccccc}
  & \multicolumn{2}{c}{\textbf{min. sphere}}
  & \textbf{convex} & \textbf{min.} & \textbf{self inter-}\\
  & \CodeFmt{double} & \CodeFmt{gmpq}
  & \textbf{hull} & \textbf{width}  & \textbf{section}\\\hline
  Bunny     &  0.02 & \hspace*{1ex}14 & 
                 \hspace*{1ex}3.5 & 111 & \hspace*{1ex}3.2\\
  Lion vase\hspace*{-16mm} & 0.15 & 396 & 13.1 & 276 & 22.9 \\
  David     & 0.13 & 215 & 20.3 & 112 & 41.6 \\
  Raptor    & 0.35 & 589 & 45.5 & 123 & 92.3
\end{tabular}
}

% \begin{tabular}{l|l}
%   & \textbf{self intersection} \\\hline
%   Bunny     & 3.2 \\
%   Lion vase & 22.9 \\
%   David     & 41.6 \\
%   Raptor    & 92.3
% \end{tabular}

