
\section{Tutorial}

In the directories \ccc{demo/Qt_widget/basic/tutorial*} we provide some examples that illustrate the widget, layers, and the standard toolbar.

\subsection*{Tutorial 1}

In this tutorial you can see how you can use \ccStyle{Qt\_widget} like
a stream, for the output of \cgal\ objects.  Of course I recommend to read
the tutorial from Trolltech, that is the original Qt tutorial, but I
think that you can pass this tutorials without having strong skills of \qt\
programming. Anyway, the code that belongs to Qt is explained in tutorials.

The following is a typical way of how to create a window using
\qt\ and \ccStyle{Qt\_widget.}
\begin{ccExampleCode}
#include <CGAL/IO/Qt_widget.h>
#include <qapplication.h>

int main( int argc, char **argv )
{
    QApplication app( argc, argv );
    CGAL::Qt_widget * W = new CGAL::Qt_widget();
    app.setMainWidget( W );
    W.resize(400, 400);
    W.set_window(0, 400, 0, 400);
    W.show();

    return app.exec();
}
\end{ccExampleCode}
You always have to include the header:
\begin{ccExampleCode}
#include <qapplication.h>
\end{ccExampleCode}

The entry point for a typical Qt application is the function main. In
this function you should define an application object of Qt:
\begin{ccExampleCode}
QApplication app( argc, argv );
\end{ccExampleCode}
You will run the Qt application with the line:
\begin{ccExampleCode}
return app.exec();
\end{ccExampleCode}
To use \ccStyle{Qt\_widget} you need an instance and tell the
application to use that instance:
\begin{ccExampleCode}
CGAL::Qt_widget *W = new CGAL::Qt_widget();
app.setMainWidget(W);
\end{ccExampleCode}
To resize and set the scaling factor of the window you will use:
\begin{ccExampleCode}
W->resize(400, 400);
W->set_window(0, 400, 0, 400);
\end{ccExampleCode}
At the end you need to show the window when the initialization has been done:
\begin{ccExampleCode}
W->show();
\end{ccExampleCode}

All the drawing code should be put between \ccStyle{Qt\_Widget}'s lock() and
unlock() functions. See the manual reference pages of
\ccStyle{Qt\_widget}. Doing like this, the window will be updated only
once, when \ccStyle{Qt\_widget} finds the last unlock(). This way you
can avoid the window flickering.

As you will notice, this tutorial has some limitations. If you try to
resize the window you'll see that what you have been painted will
disappear. This is not a very pleasant thing but you'll see in the
next tutorial how you can solve this problem.

Applications following this approach are only useful when you quickly
want to see how the output of a computation looks like.

\subsection*{Tutorial 2}

In this tutorial you can see a different method to draw on the window,
and you'll pass over the limitations of the previous example.

This tutorials insert a point in a Delaunay triangulation every time
you press the mouse button over the widget, and calls \ccc{redraw()}. This
means that you'll see the results of your insertions immediately. The
advantage of this approach is that when you resize the window, the
painting will not disappear.

As you see the main entry point is the same as in the previous
tutorial, but instead of using an instance of \ccStyle{Qt\_widget}
class, you'll use an instance of a class derived from \ccStyle{Qt\_widget}.
In this sense, we create a new class \ccc{My\_window} as a child for
\ccStyle{Qt\_widget}. The resize function has been moved into the
constructor of \ccc{My\_window}.

We define a triangulation as a global variable:
\begin{ccExampleCode}
typedef CGAL::Cartesian<double>             Rep;
typedef CGAL::Point_2<Rep>                  Point;
typedef CGAL::Delaunay_triangulation_2<Rep> Delaunay;

Delaunay dt;
\end{ccExampleCode}
The private slot \ccc{redraw_win()} is used to output the
triangulation on the screen. This slot is triggered by the
\ccc{custom\_redraw()} signal emitted by \ccc{Qt\_widget} at the end of 
\ccc{redraw()}. The connection between this signal and our slot is
made in the constructor of \ccc{My\_window}.
\begin{ccExampleCode}
private slots:
  void redraw_win()
  {
    *this << dt;
  }
\end{ccExampleCode}

This slot is triggered every time the \ccc{redraw()}
public member of \ccc{Qt\_widget} is called. This way the window
redraws the painting from this slot in the same time with the other
paintings. You will see in the next tutorial how to use a layer. All
the layers are drawn in this redraw() method of \ccc{Qt\_widget}, and
at the end \ccc{custom\_redraw()} is emitted.

Another private member catches the mouse press event from the Window
system (X11/Windows). The code put in this function will be executed when
you press the mouse over the widget.
\begin{ccExampleCode}
private:
  void mousePressEvent(QMouseEvent *e)
  {
    Qt_widget::mousePressEvent(e);
    dt.insert(Point(x_real(e->x()), y_real(e->y())));
    redraw();
  }
\end{ccExampleCode}
As you see, the code inserts a point in the triangulation, and calls
redraw(). You can comment redraw() to see what happens: You'll see
the results only when you'll resize the window.
\begin{ccAdvanced}
Of course if you want to trigger the mouse press event in the base
class you have to put this in the code of the function:
\begin{ccExampleCode}
Qt_widget::mousePressEvent(e);
\end{ccExampleCode}
It is the \ccc{Qt\_widget} that forward events to the attached
layers. In order to forward the events, it should receive them
first. So, if you are not using layers, and you do not need the
\ccc{Qt\_widget} to handle this event, you should not use this line.
You can experience more this feature in tutorial 5 were it is used the
standard toolbar. Try to remove there this line to see what
happens. The widget will not receive the events and neither the
standard toolbar.
\end{ccAdvanced}

\subsection*{Tutorial 3}

We change the previous tutorial to use layers for drawing. We derive
our own class from \ccc{Qt\_widget\_layer}.

In this example the layer is:
\begin{ccExampleCode}
class My_Layer : public CGAL::Qt_widget_layer{
  void draw(){
    *widget << dt;
  }
};
\end{ccExampleCode}
As you see you have to provide a \ccc{draw()} function in your layer, in
order to create output on the screen. The \ccStyle{Qt\_widget} will call this
\ccc{draw()} function for every attached and active layer. A layer is active
in the moment of attaching by default.

In the code of the constructor of \ccc{My\_window} the layer is attached to
the widget:
\begin{ccExampleCode}
attach(&v);
\end{ccExampleCode}
Because the layer is attached, the triangulation will appear on the
screen every time you call redraw(), as the widget redraws all the
layers. This means that in this tutorial every time you press the mouse
button, the triangulation will be redrawn.

Try to detach the layer to see what happens. Or try to deactivate
the layer. The triangulation will not be shown anymore.

\subsection*{Tutorial 4}

The fourth tutorial shows how to create a more complex application
using the \qt\ class \ccStyle{QMainWindow}. With this class you can create a
{\sc Mdi} (Multiple Document Interface) application as well as a {\sc
Sdi} (Single Document Interface).

For drawing, the tutorial use the same layer as for the previous one
but this time we use another class \ccStyle{QMainWindow}, as the main frame of
the application. We describe here how it is done.

The entry point is the same:
\begin{ccExampleCode}
int main( int argc, char **argv )
{
    QApplication app( argc, argv );
    My_window W(400,400);
    app.setMainWidget( &W );
    W.show();
    W.setCaption("Using QMainWindow QT class");
    return app.exec();
}
\end{ccExampleCode}
This time you see that \ccc{W} is no longer an instance of
\ccStyle{Qt\_widget} but is an instance of
\ccStyle{QMainWindow}. \ccStyle{QMainWindow} is also a widget but
provides functionalities like you can use a toolbar, a status bar ... in other
words you can create a complex application. To use an instance of
\ccStyle{Qt\_widget} you have to say in the constructor:
\begin{ccExampleCode}
setCentralWidget(widget);
\end{ccExampleCode}
where widget is an instance of \ccStyle{Qt\_widget}. As you see it is
also declared in \ccStyle{My\_window}.

There is one more thing, at the constructor you see:
\begin{ccExampleCode}
widget = new My_widget(this);
\end{ccExampleCode}
Also the constructor of \ccStyle{My\_widget} is adapted:
\begin{ccExampleCode}
My_widget(QMainWindow* c) : CGAL::Qt_widget(c) {};
\end{ccExampleCode}
This lines tells the application that \ccStyle{My\_window} is a parent for
\ccStyle{My\_widget}. Try to comment this lines to see what happens. Two
windows will appear, one for \ccStyle{My\_window} and one for \ccStyle{My\_widget}.

In the constructor it is \ccc{widget->attach(\&v)}; This way, the layer is
attached by \ccc{My\_widget}. The other part of the code does the same
thing as the previous tutorials: insert a new point in a Delaunay
triangulation and draw the triangulation every time you click on the window.

\subsection*{Tutorial 5}

The fifth tutorial includes the standard toolbar in the
application. The application has the same functionality but as you
see you can now zoom in, zoom out and translate, using the tools from
the standard toolbar.

The layer is still there, doing nothing but drawing the triangulation.

The class \ccc{My\_widget} is like in the previous example an instance of
\ccStyle{Qt\_widget}, that you need in the class \ccc{My\_window}.

The only difference between this example and the previous one is the
use of standard toolbar. It is declared as private in \ccc{My\_window} class:
\begin{ccExampleCode}
CGAL::Qt_widget_standard_toolbar *stoolbar;
\end{ccExampleCode}
To use it, in the constructor of \ccc{My\_window}, it is added:
\begin{ccExampleCode}
stoolbar = new CGAL::Qt_widget_standard_toolbar(widget, this, ``Standard toolbar'');
\end{ccExampleCode}
In this tutorial you can play a little bit with the standard toolbar
but you will see probably something that is not quite pleasant. If you
select the hand, and try to use it, you'll see that when you click to
select the first point, also a new \ccc{CGAL::Point\_2} is inserted in the
triangulation. This is due to the fact that the mousePressEvent
implemented in My\_window it is called every time you click on the
widget, even if a tool from the standard toolbar is active.

You will see in the next tutorial a solution to this problem.

\begin{ccAdvanced}
In tutorial 2 we have been talked about this line:
\begin{ccExampleCode}
Qt_widget::mousePressEvent(e);
\end{ccExampleCode}
You can try to remove it now to see what happens. The \ccc{Standard Toolbar}
will not receive any event because \ccc{Qt\_widget} will not receive
any event.
\end{ccAdvanced}

\subsection*{Tutorial 6}

The sixth tutorial uses for the first time a layer to build \cgal\
objects. It is declared the class \ccc{My\_input_layer} derived from
\ccStyle{Qt\_widget\_layer}, that is used to create a \cgal\ point
every time you click on the widget.
\begin{ccExampleCode}
class My_input_layer : public CGAL::Qt_widget_layer{
public:
  My_input_layer(){};
private:
  void mousePressEvent(QMouseEvent *e)
  {
    if(e->button() == Qt::LeftButton)
    {
      double x=static_cast<double>(widget->x_real(e->x()));
      double y=static_cast<double>(widget->y_real(e->y()));
      widget->new_object(CGAL::make_object(Point(x, y)));
    }
  }
};
\end{ccExampleCode}
This class has a member function \ccc{mousePressEvent(QMouseEvent *e)} that
is called by \ccStyle{Qt\_widget} if the tool is attached.

In \ccc{My\_window} class an instance of \ccc{My\_input_layer} is created:
\begin{ccExampleCode}
My_input_layer t;
\end{ccExampleCode}
In the constructor of \ccc{My\_window} we attach the layer:
\begin{ccExampleCode}
widget->attach(&t);
\end{ccExampleCode}

To receive the object that the layer creates, in the constructor of
\ccc{My\_window}, we connect:
\begin{ccExampleCode}
connect (widget, SIGNAL(new_cgal_object(CGAL::Object)), 
         this, SLOT(get_object(CGAL::Object)));
\end{ccExampleCode}
In \ccc{My\_window} class the private slot
\ccc{get\_object(CGAL::Object obj)} is declared. This is what you have
to do to receive \cgal\ objects from a layer. The layer calls the new\_object()
member function from \ccStyle{Qt\_widget}, that emits a signal
\ccc{new\_cgal\_object(CGAL::Object)}. To receive this signal, in the
constructor of \ccc{My\_window} a connect is declared:
\begin{ccExampleCode}
connect (widget, SIGNAL(new_cgal_object(CGAL::Object)), 
         this, SLOT(get_object(CGAL::Object)));
\end{ccExampleCode}

This connect tells the application that every time
\ccStyle{Qt\_widget} emits the signal new\_cgal\_object, the function
get\_object is called, having as parameter a \ccc{CGAL::Object}. It could be
anything, and to verify what object have been received you have to use
\ccc{CGAL::assign}.
\begin{ccExampleCode}
  Point p;
  if(CGAL::assign(p, obj)) {
      dt.insert(p);
      widget->redraw();
  }
\end{ccExampleCode}
You can insert new points in the triangulation by clicking on the
widget. When you use the standard toolbar, your already attached layers
will be put on waiting state, untill you finish working with it. For
example if you click on the hand tool and you use it, then you click
on the arrow to detach it, your attached layer will receive the
focus, being brought to life.

You can attach as many layers you like on the \ccc{Qt\_widget}. All
the layers have a order of drawing and of calling the events. The
\ccc{Qt\_widget} dispatch the events to layers in the order they have
been attached.

\subsection*{Tutorial 7}

This tutorial is very similar to the previous one. Insert new points
in a Delaunay triangulation when you click the mouse on the widget,
using a layer, and also let you use the standard toolbar.

The difference is that this example is using a generic layer, developed
by \cgal. This layer creates new \cgal\ points every time you click the
mouse. The coordinates of the point are the coordinates of the real
world. This means that the mouse coordinates are transformed using the
current scales to the real world coordinates.

The generic layers are documented in the manual. They are templatized
by a kernel of \cgal. In this tutorial the generic tool
\ccStyle{CGAL::Qt\_widget\_get\_point} it is templatized by
\ccc{Cartesian<double>}.

First comes the include statement:
\begin{ccExampleCode}
#include <CGAL/IO/Qt_widget_get_point.h>
\end{ccExampleCode}
In the class \ccc{My\_window}, it is declared as a private member:
\begin{ccExampleCode}
CGAL::Qt_widget_get_point<Rep> get_point;
\end{ccExampleCode}
In the constructor of \ccc{My\_window} we attach this generic layer:
\begin{ccExampleCode}
widget->attach(&get_point);
\end{ccExampleCode}

The connect is still there in the constructor, the good news is that
no matter how many tools you use, you will have to connect only once
the {\sc Signal} with your {\sc Slot}.

\subsection*{Tutorial 8}

In this tutorial you learn how to use a button to activate and
deactivate a \ccStyle{Qt\_widget\_layer}. As in the previous tutorial
we use the generic layer \ccStyle{Qt\_widget\_get\_point} and a
toolbar button that will control the process of activating and deactivating.

To use a toolbar, here is what you have to do in the constructor of My\_window:
\begin{ccExampleCode}
QToolBar  *layers_toolbar;
layers_toolbar = new QToolBar("Layers", this, QMainWindow::Top, true, "Layers");
addToolBar(layers_toolbar, Top, false);
\end{ccExampleCode}
To add a button in the toolbar you have to:
declare the button in \ccc{My\_window}:
\begin{ccExampleCode}
QToolButton *get_point_button;     //the toolbar button
\end{ccExampleCode}
add the button in the toolbar:
\begin{ccExampleCode}
get_point_button = new QToolButton(tools_toolbar, ``Get Point'');
get_point_button->setPixmap(QPixmap( (const char**)point_xpm ));
\end{ccExampleCode}
To make the button a toggle button:
\begin{ccExampleCode}
get_point_but->setToggleButton(TRUE);
\end{ccExampleCode}
The connection between the button and the layer is done in the
following way, using the \ccc{stateChanged(int)} public slot of \ccc{Qt\_widget\_layer}:
\begin{ccExampleCode}
connect (get_point_button, SIGNAL(stateChanged(int)),
         &get_point, SLOT(stateChanged(int)));
\end{ccExampleCode}
\begin{ccAdvanced}
This public slot was desinged to be used only with \qt\/ buttons. The
integer value sent as a parameter represents the state of a \qt\/
button. Please see the documentation of \qt\ if you want to know more
about buttons and how to use them properly.
\end{ccAdvanced}
