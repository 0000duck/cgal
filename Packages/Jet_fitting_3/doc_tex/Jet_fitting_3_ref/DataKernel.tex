% +------------------------------------------------------------------------+
% | Reference manual page: DataKernel.tex
% +------------------------------------------------------------------------+
% | 09.02.2006   Marc Pouget and Fr�d�ric Cazals
% | Package: Jet_fitting_3
% | 
\RCSdef{\RCSDataKernelRev}{$Revision$}
\RCSdefDate{\RCSDataKernelDate}{$Date$}
% |
%%RefPage: end of header, begin of main body
% +------------------------------------------------------------------------+


\begin{ccRefConcept}{DataKernel}

%% \ccHtmlCrossLink{}     %% add further rules for cross referencing links
%% \ccHtmlIndexC[concept]{} %% add further index entries

\ccDefinition
  
The concept \ccRefName\ describes the set of requirements to be
fulfilled by any class used to instantiate the template
parameter of the class
\ccc{Monge_rep<DataKernel>} and the first template parameter of the
class
\ccc{Monge_via_jet_fitting<DataKernel,LocalKernel,LinAlgTraits>}.

 

\ccRequirements
% +------------------------------------------------------------------
Casts must be available between types of this concept \ccRefName\ and
the concept \ccc{LocalKernel} which are used together in the class 
\ccc{Monge_via_jet_fitting<DataKernel,LocalKernel,LinAlgTraits>}.
 
%\ccGeneralizes

%ThisConcept \\
%ThatConcept

\ccTypes
% +------------------------------------------------------------------
\ccNestedType{FT}{The scalar type.}
\ccGlue
\ccNestedType{Point_3}{The  point type.}
\ccGlue
\ccNestedType{Vector_3}{The vector  type.}



%\ccCreation
%\ccCreationVariable{a}  %% choose variable name

%\ccConstructor{DataKernel();}{default constructor.}

\ccOperations
% +------------------------------------------------------------------
Only constructors and classical access to coordinates x(), y(), z()
are needed. 

\ccHasModels
% +------------------------------------------------------------------
\ccc{CGAL::Cartesian<FieldNumberType>}

\ccSeeAlso
% +------------------------------------------------------------------
The \ccc{LocalKernel} concept.

%\ccExample

%A short example program.
%Instead of a short program fragment, a full running program can be
%included using the 
%\verb|\ccIncludeExampleCode{Jet_fitting_3/DataKernel.C}| 
%macro. The program example would be part of the source code distribution and
%also part of the automatic test suite.

%\begin{ccExampleCode}
%void your_example_code() {
%}
%\end{ccExampleCode}

%% \ccIncludeExampleCode{Jet_fitting_3/DataKernel.C}

\end{ccRefConcept}

% +------------------------------------------------------------------------+
%%RefPage: end of main body, begin of footer
% EOF
% +------------------------------------------------------------------------+

