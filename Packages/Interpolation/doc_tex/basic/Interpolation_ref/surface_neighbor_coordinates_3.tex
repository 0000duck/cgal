% +------------------------------------------------------------------------+
% | Reference manual page: surface_neighbor_coordinates_3.tex
% +------------------------------------------------------------------------+
% | 
% | August 2003  Julia Floetotto
% | Package:   Interpolation
% | 
% |
% +------------------------------------------------------------------------+

%\renewcommand{\ccRefPageBegin}{\begin{ccAdvanced}}
%\renewcommand{\ccRefPageEnd}{\end{ccAdvanced}}
\begin{ccRefFunction}{surface_neighbor_coordinates_3}  %% add template arg's if necessary
\ccIndexSubitemBegin{interpolation}{surface_neighbor_coordinates_3}

\ccDefinition
  
The function \ccRefName\ computes natural neighbor coordinates for
surface points associated to a finite set of sample points issued from
the surface. The coordinates are computed from the intersection of the
Voronoi cell of the query point \ccc{p} with the tangent plane to the
surface at \ccc{p}. If the sampling is sufficiently dense, the
coordinate systems meets the properties described in the manual pages
and in \cite{bf-lcss-02},\cite{prisme-these-flototto}.  The query
point \ccc{p} needs to lie inside the convex hull of the projection of
the sample points onto the tangent plane at \ccc{p}.

\ccInclude{CGAL/surface_neighbor_coordinates_3.h}

\ccFunction{template <class OutputIterator, class InputIterator, class
  Kernel> std::pair< OutputIterator, typename Kernel::FT >
  surface_neighbor_coordinates_3(InputIterator first, InputIterator
  beyond, const typename Kernel::Point_3& p, const typename
  Kernel::Vector_3& normal, OutputIterator out, const Kernel& K);}{
  The sample points $\mathcal{P}$ are provided in the range
  $\left[\right.$\ccc{first}, \ccc{beyond}$\left.\right)$.
  \ccc{InputIterator::value_type} is the point type
  \ccc{Kernel::Point_3}. The tangent plane is defined by the point
  \ccc{p} and the vector \ccc{normal}.  The
  parameter \ccc{K} determines the kernel type. \\
  The natural neighbor coordinates for \ccc{p} are computed in the
  power diagram that results from the intersection of the $3D$ Voronoi
  diagram of $\mathcal{P}$ with the tangent plane. The sequence of
  point-coordinate pairs that is computed by the function is placed
  starting at \ccc{out}. The function returns a pair with an iterator
  that is placed past-the-end of the resulting sequence of
  point-coordinate pairs and the normalization factor of the
  coordinates.
  \ccPrecond %
  \ccc{p} lies in the convex hull of the projection of the points
  $\mathcal{P}$ onto the tangent plane.}


\ccFunction{template <class OutputIterator, class InputIterator, class
  ITraits> std::pair< OutputIterator, typename ITraits::FT >
  surface_neighbor_coordinates_3(InputIterator first, InputIterator beyond,
  const typename ITraits::Point_2& p,OutputIterator out, 
  const ITraits& traits);}{ the same as above only that the traits
  class must be instantiated by the user. \ccc{ITraits} must be a
  model of \ccc{VoronoiIntersection2Traits3}.}

The next function allows to filter some potential neighbors of the
query point \ccc{p} from $\mathcal{P}$ via its three-dimensioal
Delaunay triangulation. All surface neighbors of \ccc{p} are
necessarily neighbors in the Delaunay triangulation of $\mathcal{P}
\cup \{p\}$.  \ccFunction{template < class Dt, class OutputIterator >
  std::pair< OutputIterator, typename Dt::Geom_traits::FT >
  surface_neighbor_coordinates_3(const Dt& dt, const typename
  Dt::Geom_traits::Point_2& p, const typename
  Dt::Geom_traits::Vector_3& normal, OutputIterator out, typename
  Dt::Face_handle start = typename Dt::Face_handle(NULL));} {computes
  the surface neighbor coordinates with respect to the points that are
  vertices of the Delaunay triangulation \ccc{dt}. The type \ccc{Dt}
  must be equivalent to \ccc{Delaunay_triangulation_3<Gt, Tds>}. The
  optional parameter \ccc{start} is used for the used as a starting
  place for the search of the conflict zone. It may be the result of
  the call \ccc{dt.locate(p)}. This function instantiates the template
  parameter \ccc{ITraits} to be
  \ccc{Voronoi_intersection_2_traits_3<Dt::Geom_traits>}.}


\ccFunction{template < class Dt, class OutputIterator,
  class ITraits> std::pair< OutputIterator, typename
  Dt::Geom_traits::FT > surface_neighbor_coordinates_3(const Dt& dt,
  const typename Dt::Geom_traits::Point_2& p, OutputIterator out,
  const ITraits& traits, typename Dt::Face_handle start = typename
  Dt::Face_handle(NULL));}
%
{The same as above only that the geometric traits class must be
  instantiated by the user with an instance of a model of
  \ccc{VoronoiIntersection2Traits3}.}


\ccHeading{Requirements}
\begin{enumerate}
\item \ccc{Dt} is equivalent to the class
  \ccc{Delaunay_triangulation_3}.
\item \ccc{OutputIterator::value_type} is equivalent to
  \ccc{std::pair<Dt::Point_3, Dt::Geom_traits::FT>}, i.e.\ a pair
  asscociating a point and its natural neighbor coordinate.
\item  \ccc{ITraits} is quivalent to the class \ccc{Voronoi_intersection_2_traits_3<K>}.
\end{enumerate}

\ccSeeAlso
\ccRefIdfierPage{CGAL::linear_interpolation} \\
\ccRefIdfierPage{CGAL::sibson_c1_interpolation}\\ 
\ccRefIdfierPage{CGAL::farin_c1_interpolation}\\ 
\ccRefIdfierPage{CGAL::natural_neighbor_coordinates_3} \\
\ccRefIdfierPage{CGAL::regular_neighbor_coordinates_2} 

\ccImplementation This function computes the areas stolen from the
Voronoi cells of points in \ccc{dt} by the insertion of \ccc{p}. The
total area of the Voronoi cell of \ccc{p} is also computed and
returned by the function.

\ccIndexSubitemEnd{Interpolation}{surface_neighbor_coordinates_3}
\end{ccRefFunction}
%\renewcommand{\ccRefPageBegin}{}
%\renewcommand{\ccRefPageEnd}{}

% +------------------------------------------------------------------------+
% RefPage: end of main body, begin of footer
% EOF
% +------------------------------------------------------------------------+

