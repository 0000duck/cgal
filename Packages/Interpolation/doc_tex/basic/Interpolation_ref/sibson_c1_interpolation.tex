% +------------------------------------------------------------------------+
% | Reference manual page: sibson_c1_interpolation.tex
% +------------------------------------------------------------------------+
% | 
% | August 2003  Julia Floetotto
% | Package:   Interpolation
% | 
% |
% +------------------------------------------------------------------------+

%\renewcommand{\ccRefPageBegin}{\begin{ccAdvanced}}
%\renewcommand{\ccRefPageEnd}{\end{ccAdvanced}}
\begin{ccRefFunction}{sibson_c1_interpolation}  %% add template arg's if necessary
\ccIndexMainItemBegin{sibson_c1_interpolation}

\ccDefinition
  
The function \ccRefName\ interpolates the function values and the
gradients that are provided following the method described in \cite{s-bdnni-81}.


\ccInclude{CGAL/interpolation_functions.h}

\ccFunction{template < class ForwardIterator, class Functor, class
  GradFunctor, class Traits> typename Functor::result_type
  sibson_c1_interpolation(ForwardIterator first, ForwardIterator
  beyond, const typename
  std::iterator_traits<ForwardIterator>::value_type::second_type&
  norm, const typename
  std::iterator_traits<ForwardIterator>::value_type:: first_type& p,
  Functor f, GradFunctor grad_f,const Traits& geom_traits);} {
  \ccc{ForwardIterator::value_type} is a pair associating a point to a
  (non-normalized) barycentric coordinate. \ccc{norm} is the
  normalization factor. The range $\left[\right.$ \ccc{first},
  \ccc{beyond}$\left.\right)$ contains the barycentric coordinates for
  the query point \ccc{p}. \ccc{f} allows to access the value of the
  interpolated function given a point and \ccc{grad_f} allows to
  access the function gradient at a given point. This function
  generates the interpolated function value at the point \ccc{p} using
  Sibson's $Z^1$ interpolant \cite{s-bdnni-81}.}

\ccHeading{Requirements}
\begin{enumerate}
\item \ccc{ForwardIterator::value_type} is a pair of point/coordinate
  value, thus \ccc{ForwardIterator::value_type::first_type} is
  equivalent to a point and
  \ccc{ForwardIterator::value_type::second_type} is a
  number type. 
\item \ccc{Functor::argument_type} must be equivalent to
  \ccc{ForwardIterator::value_type::first_type} and
  \ccc{Functor::result_type} is the function value type. It must
  provide a multiplication and addition operation with the type
  \ccc{ForwardIterator::value_type::second_type}.
\item \ccc{GradFunctor::argument_type} must be equivalent to
  \ccc{ForwardIterator::value_type::first_type} and
  \ccc{Functor::result_type} is the type of the gradient of the
  function value. It must provide a multiplication operation with the
  return type of \ccc{Construct_vector} defined in the traits
  class \ccc{Traits}.
\item  \ccc{Traits} is a model of the concept 
  \ccc{InterpolationTraits}\ccIndexMainItem[c]{InterpolationTraits}.
  %Only the following members of this traits class are used:
  %\begin{itemize}
  %\item \ccc{Construct_vector}
  %\item \ccc{Construct_scaled_vector}
  %\item \ccc{Compute_squared_distance}
  %\item \ccc{FT}
  %\item \ccc{construct_vector_object}
  %\item \ccc{construct_scaled_vector_object} 
  %\item \ccc{compute_squared_distance_object}
  %\end{itemize}
\item The number type \ccc{FT} provided by \ccc{Traits} must support
  the square root operation \ccc{sqrt()}.
\end{enumerate}

\ccFunction{template < class ForwardIterator, class Functor, class GradFunctor, class Traits>
  typename Functor::result_type 
  Sibson_c1_interpolation_square(ForwardIterator first, ForwardIterator beyond,
  const typename std::iterator_traits<ForwardIterator>::value_type::second_type&
  norm, Functor f, GradFunctor grad_f,const Traits& geom_traits);}
{The same as above except that no square root operation is needed for \ccc{FT}.}

\ccSeeAlso
\ccRefIdfierPage{CGAL::sibson_gradient_fitting} \\
\ccRefIdfierPage{CGAL::linear_interpolation} \\
\ccRefIdfierPage{CGAL::farin_c1_interpolation} \\
\ccRefIdfierPage{CGAL::quadratic_interpolation} \\
\ccRefIdfierPage{CGAL::Interpolation_traits_2<K>} \\
\ccRefIdfierPage{CGAL::natural_neighbor_coordinates_2}\\
\ccRefIdfierPage{CGAL::regular_neighbor_coordinates_2} \\
\ccRefIdfierPage{CGAL::surface_neighbor_coordinates_2_3}\\
%\ccImplementation

\ccIndexSubitemEnd{Interpolation}{sibson_c1_interpolation}
\end{ccRefFunction}
%\renewcommand{\ccRefPageBegin}{}
%\renewcommand{\ccRefPageEnd}{}

% +------------------------------------------------------------------------+
%%RefPage: end of main body, begin of footer
% EOF
% +------------------------------------------------------------------------+

