% +------------------------------------------------------------------------+
% | Reference manual page: regular_neighbor_coordinates_2.tex
% +------------------------------------------------------------------------+
% | 
% | August 2003  Julia Floetotto
% | Package:   Interpolation
% | 
% |
% +------------------------------------------------------------------------+

%\renewcommand{\ccRefPageBegin}{\begin{ccAdvanced}}
%\renewcommand{\ccRefPageEnd}{\end{ccAdvanced}}
\begin{ccRefFunction}{regular_neighbor_coordinates_2}  %% add template arg's if necessary
\ccIndexSubitemBegin{interpolation}{regular_neighbor_coordinates_2}

\ccDefinition
  
The function \ccRefName\ computes regular neighbor coordinates, also
called Sibson's coordinates, for $2D$ points provided a two-dimensional
triangulation and a query point inside the convex hull of the vertices
of the triangulation.

\ccInclude{CGAL/regular_neighbor_coordinates_2.h}


\ccFunction{template <class Rt, class OutputIterator, class Traits>
  std::pair< OutputIterator, typename Traits::FT >
  regular_neighbor_coordinates_2(const Rt& rt, const typename
  Traits::Weighted_point& p, OutputIterator out, const Traits& traits,
  typename Rt::Face_handle start = typename Rt::Face_handle(NULL));} {
  computes the regular neighbor coordinates for \ccc{p} with respect
  to the points in the two-dimensional regular triangulation \ccc{rt}.
  The template class \ccc{Rt} should be of type
  \ccc{Regular_triangulation_2<Traits, Tds>}.  The value type of the
  \ccc{OutputIterator} is a pair of \ccc{Rt::Weighted_point} and the
  coordinate value of type \ccc{Rt::Geom_traits::FT}. The sequence of
  point-coordinate pairs that is computed by the function is placed
  starting at \ccc{out}. The function returns a pair with an iterator
  that is placed past-the-end of the resulting sequence of
  point-coordinate pairs and the normalization factor of the
  coordinates.
  \ccPrecond %
  \ccc{p} lies in the convex hull of the
  points in \ccc{rt}.
  }  

\ccFunction{template < class Rt, class OutputIterator > std::pair<
  OutputIterator, typename Rt::Geom_traits::FT >
  regular_neighbor_coordinates_2(const Rt& rt, const typename
  Rt::Geom_traits::Weighted_point& p, OutputIterator out, typename
  Rt::Face_handle start = typename Rt::Face_handle(NULL));} 
%
{The same as above. This function instantiates the template parameter
  \ccc{Traits} to be \ccc{Rt::Geom_traits}.}


\ccFunction{template <class Rt, class OutputIterator, class Traits,
  class EdgeIterator, class VertexIterator > std::pair<
  OutputIterator, typename Traits::FT >
  regular_neighbor_coordinates_2(const Rt& rt, const typename
  Traits::Weighted_point& p, OutputIterator out, EdgeIterator
  hole_begin, EdgeIterator hole_end, VertexIterator
  hidden_vertices_begin, VertexIterator hidden_vertices_end, const
  Traits& traits);} { The same as above.  \ccc{hole_begin} and
  \ccc{hole_end} determines the iterator range over the boundary edges
  of the conflict zone of \ccc{p} in the triangulation.
  \ccc{hidden_vertices_begin} and \ccc{hidden_vertices_end} determines
  the iterator range over the hidden vertices of the conflict zone of
  \ccc{p} in the triangulation. It is the result of the function
  \ccc{T.get_boundary_of_conflicts(p,std::back_inserter(hole),
    std::back_inserter(hidden_vertices),
    start)}, see \ccc{Regular_triangulation_2<Traits, Tds>}.}


\ccFunction{template <class Rt, class OutputIterator, class Traits>
  std::pair< OutputIterator, typename Rt::Geom_traits::FT >
  regular_neighbor_coordinates_2(const Rt& rt, typename
  Rt::Vertex_handle vh, OutputIterator out);}{This function computes
  the regular neighbor coordinates of the point contained in the
  vertex referenced by \ccc{vh}, \ccc{vh->point()}, with respect to
  \ccc{Rt} excluding \ccc{vh->point()}. The same as above for the
  remaining parameters.}  

\ccFunction{template <class Rt, class OutputIterator> 
  std::pair< OutputIterator, typename Rt::Geom_traits::FT > 
  regular_neighbor_coordinates_2(const Rt& rt,
  typename Rt::Vertex_handle vh, OutputIterator out);}{The template
  parameter \ccc{Traits} of the function above is instantiated to
  \ccc{Rt::Geom_traits}.}

\ccHeading{Requirements}
\begin{enumerate}
\item \ccc{Rt} are equivalent to the class
  \ccc{Regular_triangulation_2<Traits, Tds>}.
\item \ccc{OutputIterator::value_type} is equivalent to
  \ccc{std::pair<Rt::Weighted_point, Rt::Geom_traits::FT>}, i.e.\ a pair
  asscociating a point and its regular neighbor coordinate.
\item \ccc{Traits} provides the number type \ccc{FT} which is a model
  for \ccc{FieldNumberType} and it must meet the requirements for the
  traits class of the \ccc{polygon_area_2} function. A model of this
  traits class is \ccc{Regular_neighbor_coordinate_traits_2<K,
    Weight>}.
\end{enumerate}

\ccImplementation This function computes the areas stolen from the
Voronoi cells of points in \ccc{rt} by the insertion of \ccc{p}. The
total area of the Voronoi cell of \ccc{p} is also computed and
returned by the function.

\ccSeeAlso
\ccRefIdfierPage{CGAL::natural_neighbor_coordinates_2} \\
\ccRefIdfierPage{CGAL::Regular_neighbor_coordinates_traits_2<R, Weight>}

\ccIndexSubitemEnd{Interpolation}{regular_neighbor_coordinates_2}
\end{ccRefFunction}
%\renewcommand{\ccRefPageBegin}{}
%\renewcommand{\ccRefPageEnd}{}

% +------------------------------------------------------------------------+
% RefPage: end of main body, begin of footer
% EOF
% +------------------------------------------------------------------------+

