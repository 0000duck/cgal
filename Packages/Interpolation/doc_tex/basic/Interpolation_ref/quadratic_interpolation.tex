% +------------------------------------------------------------------------+
% | Reference manual page: quadratic_interpolation.tex
% +------------------------------------------------------------------------+
% | 
% | August 2003  Julia Floetotto
% | Package:   Interpolation
% | 
% |
% +------------------------------------------------------------------------+

%\renewcommand{\ccRefPageBegin}{\begin{ccAdvanced}}
%\renewcommand{\ccRefPageEnd}{\end{ccAdvanced}}
\begin{ccRefFunction}{quadratic_interpolation}  %% add template arg's if necessary
\ccIndexMainItemBegin{quadratic_interpolation}

\ccDefinition
  
The function \ccRefName\ interpolates the function values and first degree 
functions defined from the function gradients. Both, function values and 
gradients, must be provided by functors.

\ccInclude{CGAL/interpolation_functions.h}

\ccParameters 
See \ccc{sibson_c1_interpolation}.

\ccFunction{template < class ForwardIterator, class Functor, class
  GradFunctor, class Traits> typename Functor::result_type
  quadratic_interpolation(ForwardIterator first, ForwardIterator
  beyond, const typename std::iterator_traits<ForwardIterator>::
  value_type::second_type& norm, Functor function_value, GradFunctor
  function_gradient,const Traits& traits);} {
 % \ccc{ForwardIterator::value_type} is a pair associating a point to a
 % (non-normalized) barycentric coordinate.  \ccc{norm} is the
 % normalization factor.  \ccc{function_value} allows to access a value
 % given a point and \ccc{function_gradient} allows to access a
 % function gradient given a point.  
  This function generates the
  interpolated function value as the weighted sum of the values plus a
  linear term in the gradient for each point of the point/coordinate
  pairs in the range $\left[\right.$ \ccc{first},
  \ccc{beyond}$\left.\right)$.  See also
  \ccc{sibson_c1_interpolation}. \ccPrecond{\ccc{norm} $\neq 0$ 
    \ccc{function_value(p).second == true} for all points
    \ccc{p} of the point/coordinate pairs in the range
    $\left[\right.$\ccc{first}, \ccc{beyond}$\left.\right)$.}}


\ccHeading{Requirements} Same requirements as for
\ccc{sibson_c1_interpolation} only that \ccc{Traits::FT} does not need
to provide the square root operation.

\ccSeeAlso
\ccRefConceptPage{InterpolationTraits}\\
\ccRefConceptPage{GradientFittingTraits}\\
\ccRefIdfierPage{CGAL::Data_access<Map>}\\
\ccRefIdfierPage{CGAL::sibson_gradient_fitting} \\
\ccRefIdfierPage{CGAL::linear_interpolation} \\
%\ccRefIdfierPage{CGAL::farin_c1_interpolation} \\
%\ccRefIdfierPage{CGAL::quadratic_interpolation} \\
\ccRefIdfierPage{CGAL::Interpolation_traits_2<K>} \\
\ccRefIdfierPage{CGAL::Interpolation_gradient_fitting_traits_2<K>} \\
\ccRefIdfierPage{CGAL::natural_neighbor_coordinates_2}\\
\ccRefIdfierPage{CGAL::regular_neighbor_coordinates_2} \\
\ccRefIdfierPage{CGAL::surface_neighbor_coordinates_3}\\
%\ccImplementation

\ccIndexSubitemEnd{Interpolation}{quadratic_interpolation}
\end{ccRefFunction}
%\renewcommand{\ccRefPageBegin}{}
%\renewcommand{\ccRefPageEnd}{}

% +------------------------------------------------------------------------+
%%RefPage: end of main body, begin of footer
% EOF
% +------------------------------------------------------------------------+

