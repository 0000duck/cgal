\section{Interpolation methods}\label{sec:interpolation}
\subsection{Introduction}
\subsubsection{Linear precision interpolation}

Sibson \cite{s-bdnni-81} defines a very simple interpolant that
re-produces linear functions exactly. The interpolation of
$\Phi(\mathbf{x})$ is given as the linear combination of the neighbors' function
values weighted by the coordinates:
\begin{displaymath}
  Z^0(\mathbf{x}) = \sum_i  \lambda_i(\mathbf{x}) z_i.
\end{displaymath}
Indeed, if $z_i=a + \mathbf{b}^t \mathbf{p_i}$ for all natural
neighbors of $\mathbf{x}$, we have
\[  Z^0(\mathbf{x}) = \sum_i  \lambda_i(\mathbf{x}) (a + \mathbf{b}^t\mathbf{p_i}) = a+\mathbf{b}^t \mathbf{x}\]
by the barycentric coordinate property. The first example in
Subsection~\ref{subsec:interpol_examples} shows how the function is
called.

\subsubsection{Sibson's $C^1$ continuous interpolant}
In \cite{s-bdnni-81}, Sibson describes a second interpolation method
that is $C^1$ continuous with gradient $\mathbf{g_i}$ at
$\mathbf{p_i}$. It
re-produces spherical quadrics of the form $\Phi(\mathbf{x}) =a +
\mathbf{b}^t \mathbf{x} +\gamma\ \mathbf{x}^t\mathbf{x}$ exactly. The
proof relies on the barycentric coordinate property of the natural
neighbor coordinates and assumes that the gradient of $\Phi$ at the
data points is known or approximated from the function values as
described in \cite{s-bdnni-81} (see Section \ref{s:gradient_fitting}). 

Sibson's $Z^1$ interpolant is a combination of the linear interpolant
$Z^0$ and an interpolant $\xi$. The interpolant $\xi$ is the weighted
sum of the first degree functions
$$\xi_i(\mathbf{x}) = z_i
+\mathbf{g_i}^t(\mathbf{x}-\mathbf{p_i}),\qquad \xi(\mathbf{x})= \frac{\sum_i \frac{\lambda_i(\mathbf{x})}
  {\|\mathbf{x}-\mathbf{p_i}\|}\xi_i(\mathbf{x}) }{\sum_i
  \frac{\lambda_i(\mathbf{x})}{\|\mathbf{x}-\mathbf{p_i}\|}}.$$
Sibson observed that the combination of $Z^0$ and $\xi$ reconstructs exactly
a spherical quadric if they are mixed as follows:
$$
Z^1(\mathbf{x}) = \frac{\alpha(\mathbf{x}) Z^0(\mathbf{x}) +
  \beta(\mathbf{x}) \xi(\mathbf{x})}{\alpha(\mathbf{x}) +
  \beta(\mathbf{x})} \textrm{ where } \alpha(\mathbf{x}) =
\frac{\sum_i \lambda_i(\mathbf{x}) \frac{\|\mathbf{x} -
    \mathbf{p_i}\|^2}{f(\|\mathbf{x} - \mathbf{p_i}\|)}}{\sum_i
  \frac{\lambda_i(\mathbf{x})} {f(\|\mathbf{x} - \mathbf{p_i}\|)}}
\textrm{ and } \beta(\mathbf{x})= \sum_i \lambda_i(\mathbf{x})
\|\mathbf{x} - \mathbf{p_i}\|^2,$$
where in Sibson's original work,
$f(\|\mathbf{x} - \mathbf{p_i}\|) = \|\mathbf{x} - \mathbf{p_i}\|$.


\cgal\ contains a second implementation with $f(\|\mathbf{x} -
\mathbf{p_i}\|) = \|\mathbf{x} - \mathbf{p_i}\|^2$ which is less
demanding on the number type because it avoids the square-root
computation needed to compute the distance $\|\mathbf{x} -
\mathbf{p_i}\|$. The theoretical guarantees are the same (see
\cite{prisme-these-flototto}), simply, the smaller the slope of $f$ around the origin,
the faster the interpolant approaches $\xi_i$ as $\mathbf{x}
\rightarrow \mathbf{p_i}$.

\subsubsection{Farin's $C^1$ continuous interpolant}

Farin \cite{f-sodt-90} extended Sibson's work and realizes a $C^1$
continuous interpolant by embedding natural neighbor coordinates in
the Bernstein-B\'ezier representation of a cubic simplex. If the
gradient of $\Phi$ at the data points is known, this interpolant
reproduces quadratic functions exactly. The function gradient can be
approximated from the function values by Sibson's method
\cite{s-bdnni-81} (see Section \ref{s:gradient_fitting}) which is exact only
for spherical quadrics.

\subsubsection{Quadratic precision interpolants}


Knowing the gradient $\mathbf{g_i}$, we formulate a very simple
interpolant that reproduces exactly quadratic functions. This
interpolant is not $C^1$ continuous in general.  It is defined as
follows:
\begin{displaymath}
  I^1(\mathbf{x}) = \sum_i \lambda_i(\mathbf{x}) 
  (z_i + \frac{1}{2} \mathbf{g_i}^t (\mathbf{x} - \mathbf{p_i}))  
\end{displaymath} 



\subsection{Gradient fitting} \label{s:gradient_fitting}
Sibson describes a method to approximate the gradient of the function
$f$ from the function values on the data sites. For the data point
$\mathbf{p_i}$, we determine
$$\mathbf{g_i} = \min_\mathbf{g} \sum_j
\frac{\lambda_j(\mathbf{p_i})}{\|\mathbf{p_i} - \mathbf{p_j}\|^2}
\left( z_j - (z_i + \mathbf{g}^t (\mathbf{p_j} -\mathbf{p_i})) \right),
$$
where $\lambda_j(\mathbf{p_i})$ is the natural neighbor coordinate
of $\mathbf{p_i}$ with respect to $\mathbf{p_i}$ associated to
$\mathcal{P} \setminus \{\mathbf{p_i}\}$. This method works only for
points inside the convex hull of the data points because, for a point
$\mathbf{p_i}$ on the convex hull, $\lambda_j(\mathbf{p_i})$ is not
defined. For spherical quadrics, the result is exact.

\cgal\ provides functions to approximate the gradients of all data
points that are inside the convex hull. There is one function for each
type of coordinate computation.
%\begin{ccAdvanced}
%\end{ccAdvanced}

%\subsection{Implementation}

\subsection{Examples}\label{subsec:interpol_examples}
\subsubsection{Linear interpolation method}
\ccIncludeExampleCode{../../examples/Interpolation/linear_interpolation_2.C}

\subsubsection{Sibson's $C^1$ interpolation scheme with gradient estimation}
\ccIncludeExampleCode{../../examples/Interpolation/sibson_interpolation_2.C}

An additional example compares numerically the errors of the different 
interpolation functions with respect to a known function. 
It is distributed in examples directory.

%end of file