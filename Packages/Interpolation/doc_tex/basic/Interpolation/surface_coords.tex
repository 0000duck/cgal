\section{Surface natural neighbor coordinates and surface neighbors}\label{sec:surface}

This section introduces the functions to compute natural neighbor
coordinates and surface neighbors associated to a set of sample points
issued from a surface $\mathcal{S}$ and given a query point
$\mathbf{x}$ on $\mathcal{S}$. We suppose that $\mathcal{S}$ is a
closed and compact surface of $\mathbb{R}^3$, and let $\mathcal{P}=
\{\mathbf{p_1}, \ldots,\mathbf{p_n}\}$ be an $\epsilon$-sample of
$\mathcal{S}$ (refer to Amenta and Bern \cite{ab-srvf-99}). The
concepts are based on the definition of Boissonnat and Fl\"ototto
\cite{bf-lcss-02}, \cite{prisme-these-flototto}.  Both references
contain a thorough description of the requirements and the
mathematical properties.

\subsection{Introduction}

Two observations lead to the definition of surface neighbors and
surface neighbor coordinates: First, it is clear that the tangent
plane $\mathcal{T}_x$ of the surface $\mathcal{S}$ at the point
$\mathbf{x} \in \mathcal{S}$ approximates $\mathcal{S}$ in the
neighborhood of $\mathbf{x}$. It has been shown in \cite{bf-lcss-02}
that, if the surface $\mathcal{S}$ is well sampled with respect to the
curvature and the local thickness of $\mathcal{S}$, i.e.\ it is an $\epsilon$-sample, the intersection
of the tangent plane $\mathcal{T}_x$ with the Voronoi cell of
$\mathbf{x}$ in the Voronoi diagram of $\mathcal{P} \cup
\{\mathbf{x}\}$ has a small diameter.  Consequently, inside this
Voronoi cell, the tangent plane $\mathcal{T}_x$ is a reasonable
approximation of $\mathcal{S}$. Furthermore, the second observation
allows to compute this intersection diagram easily: one can show using
Pythagoras' theorem that the intersection of a three-dimensional
Voronoi diagram with a plane $\mathcal{H}$ is a two-dimensional power
diagram. The points defining the power diagram are the projections of
the points in $\mathcal{P}$ onto $\mathcal{H}$, each point weighted
with its negative square distance to $\mathcal{H}$. Algorithms for the
computation of power diagrams via the dual regular triangulation are
well known and for example provided by \cgal\ in the class
\ccc{Regular_triangulation_2<Gt, Tds>}.

\subsection{Implementation}
\subsubsection{Voronoi intersection diagrams}

In \cgal, the regular triangulation dual to the intersection of a $3D$
Voronoi diagram with a plane $\mathcal{H}$ can be computed by
instantiating the \ccc{Regular_triangulation_2<Gt, Tds>} class with
the traits class \ccc{Voronoi_intersection_2_traits_3<K>}. This traits
class contains a point and a vector as class member which define the
plane $\mathcal{H}$. All predicates and constructions used by
\ccc{Regular_triangulation_2<Gt, Tds>} are replaced by the
corresponding operators on three-dimensional points. For example, the
power test predicate (which takes three weighted $2D$ points
$p'$, $q'$, $r'$ of the regular triangulation and tests the power
distance of a fourth point $t'$ respect to the power circle orthogonal
to $p$, $q$, $r$) is replaced by a
\ccc{Side_of_plane_centered_sphere_2_3} predicate that tests the
position of a $3D$ point $t$ with respect to the sphere centered on
the plane $\mathcal{H}$ passing through the $3D$ points $p$, $q$, $r$.
This approach allows to avoid the explicit constructions of the
projected points and the weights which are very prone to rounding
errors.

\subsubsection{Natural neighbor coordinates on surfaces}

The computation of natural neighbor coordinates on surfaces is based
upon the computation of regular neighbor coordinates with respect to
the regular triangulation that is dual to ${\rm Vor}(\mathcal{P}) \cap
\mathcal{T}_x$, the intersection of $\mathcal{T}_x$ and the Voronoi
diagram of $\mathcal{P}$, via the function
\ccc{regular_neighbor_coordinates_2}.

Of course, we might introduce all data points $\mathcal{P}$ into this
regular triangulation. However, this is not necessary because we are
only interested in the cell of $\mathbf{x}$. It is sufficient to
guarantee that all surface neighbors of the query point $\mathbf{x}$
are among the input points that are passed as argument to the
function. The sample points $\mathcal{P}$ can be filtered for example
by distance, e.g.\ using range search or $k$-nearest neighbor queries,
or with the help of the $3D$ Delaunay triangulation since the surface
neighbors are necessarily a subset of the natural neighbors of the
query point in this triangulation. \cgal\ provides a function that
encapsulates the filtering based on the $3D$ Delaunay triangulation.
For input points filtered by distance, functions are provided that
indicate whether or not points that lie outside the input range (i.e.\ 
points that are further from $\mathbf{x}$ than the furthest input
point) can still influence the result.  This allows to iteratively
enlarge the set of input points until the range is sufficient to
certify the result.

\subsubsection{Surface neighbors}

The surface neighbors of the query point are its neighbors in the
regular triangulation that is dual to ${\rm Vor}(\mathcal{P}) \cap
\mathcal{T}_x$, the intersection of $\mathcal{T}_x$ and the Voronoi
diagram of $\mathcal{P}$. As for surface neighbor coordinates, this
regular triangulation is computed and the same kind of filtering of
the data points as well as the certification described above is
provided.


\subsection{Examples}
\ccIncludeExampleCode{../../examples/Interpolation/surface_neighbor_coordinates_3.C}

