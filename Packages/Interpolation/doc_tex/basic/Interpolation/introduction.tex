\ccChapterAuthor{Julia Fl{\"o}totto}

This chapter describes \cgal's interpolation package which implements
natural neighbor coordinate functions as well as different
methods for scattered data interpolation most of which are based on
natural neighbor coordinates. The functions for computing natural neighbor 
coordinates in Euclidean space are described in 
Section~\ref{sec:coordinates}, 
the functions concerning the coordinate and neighbor 
computation on surfaces are discussed in Section~\ref{sec:surface}. 
In Section~\ref{sec:interpolation}, we describe the different interpolation 
functions.   

Scattered data interpolation solves the following problem: given
measures of a function on a set of discrete data points, the task is
to interpolate this function on an arbitrary query point.
More formally, let $\mathcal{P}=\{\mathbf{p_1},\ldots ,\mathbf{p_n}\}$ be a set of
$n$ points in $\mathbb{R}^2$ or $\mathbb{R}^3$ and $\Phi$ be a scalar
function defined inside the convex hull of $\mathcal{P}$. We assume that
the function values are known at the points of $\mathcal{P}$, i.e. to
each $\mathbf{p_i} \in \mathcal{P}$, we associate $z_i =
\Phi(\mathbf{p_i})$. Sometimes, the gradient of $\Phi$ is also known
at $\mathbf{p_i}$. It is denoted $\mathbf{g_i}= \nabla
\Phi(\mathbf{p_i})$. The interpolation is carried out for an arbitrary query point
$\mathbf{x}$. Except for interpolation on surfaces, $\mathbf{x}$ must lie 
inside the convex hull of $\mathcal{P}$.
