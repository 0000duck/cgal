\section{Natural neighbor coordinates}
\subsection{Introduction}
Natural neighbor interpolation has been introduced by Sibson
\cite{s-bdnni-81} to interpolate multivariate scattered data.  Given
the set of data points $\mathcal{P}$, the natural neighbor coordinates
associated to $\mathcal{P}$ are defined from the Voronoi diagram of
$\mathcal{P}$.  When simulating the insertion of the query point
$\mathbf{x}$ into the Voronoi diagram of $\mathcal{P}$, the potential
Voronoi cell of $\mathbf{x}$ ``steals'' some parts from the existing
cells. 
\begin{figure}[ht!]
\begin{center}
  \IpeScale{90} \Ipe{Interpolation/nn_coords.ipe}
\end{center}
\caption{$2D$ example: $\mathbf{x}$ has five natural neighbors $\mathbf{p_1},\ldots , \mathbf{p_5}$. 
  The natural neighbor coordinate $\lambda_3(\mathbf{x})$ is the ratio
  of the area of the pink polygon, $\pi_3(\mathbf{x})$, over the area
  of the total highlighted zone.
  \label{fig:nn_coords}}
\end{figure}
Let $\pi(\mathbf{x})$ denote the volume of the potential Voronoi cell
of $\mathbf{x}$ and $\pi_i(\mathbf{x})$ denote the volume of the
sub-cell that would be stolen by the cell of $\mathbf{x}$ from the
cell of $\mathbf{p_i}$.  The natural neighbor coordinate of
$\mathbf{x}$ with respect to the data point $\mathbf{p_i}$ is defined
by
$$
\lambda_i(\mathbf{x}) =
\frac{\pi_i(\mathbf{x})}{\pi(\mathbf{x})}.$$
A two-dimensional example
is depicted in Figure \ref{fig:nn_coords}.


Various papers (\cite{s-vidt-80}, \cite{f-sodt-90},
\cite{bp-plcbdt-93}, \cite{b-scaps-97},\cite{hs-vbihc-00}) show that
the natural neighbor coordinates have the following properties:
  \begin{itemize}
  \item[(i)] $\mathbf{x} = \sum_{i=1}^n \lambda_i(\mathbf{x}) \mathbf{p_i}$
    (barycentric coordinate property).
  \item[(ii)] For any $i \leq n, \lambda_i(\mathbf{p_j})= \delta_{ij}$, where
    $\delta_{ij}$ is the Kronecker symbol.
  \item[(iii)] $\sum_{i=1}^n \lambda_i(\mathbf{x}) = 1$ (partition of unity
    property).
  \end{itemize}
  Furthermore, Piper \cite{bp-plcbdt-93} shows that the coordinate
  functions are continuous in the convex hull of $\mathcal{P}$ and
  continuously differentiable except on the data points $\mathcal{P}$.
  \medskip
  
  The interpolation package of \cgal\ provides funtions to compute
  natural neighbor coordinates for $2D$ and $3D$ points with respect
  to Voronoi diagrams as well as with respect to power diagrams (only
  $2D$), i.e.\ for weighted points. Refer to the reference pages
  \ccc{natural_neighbor_coordinates_2},
  \ccc{natural_neighbor_coordinates_3} and
  \ccc{regular_neighbor_coordinates_2}.

In addition, the package provides funtions to compute natural neighbor
coordinates on well sampled point set surfaces. See the reference page
\ccc{surface_neighbor_coordinates_2_3}.

\subsection{Implementation}
Given a Delaunay triangulation or a Regular triangulation, the
vertices in conflict with the query point are determined. The areas of
conflict zone are computed by triangulating the Voronoi sub-cells. The
normalization factor is also returned. If the query point is already
located and/or the boundary edges of the conflict zone are already
determined, alternative functions allow to avoid the re-computation.

\subsection{Example for regular neighbor coordinate computation}
The signature of all coordinate computation functions is roughly the
same. 
\ccIncludeExampleCode{../../examples/Interpolation/nn_coordinates_2.C}

For regular neighbor coordinates, it is sufficient to replace the name
of the function and the type of triangulation passed as parameter. A
special traits class is needed.
\ccIncludeExampleCode{../../examples/Interpolation/rn_coordinates_2.C}
For surface neighbor coordinates, the surface normal at the query
point must be provided.
%
