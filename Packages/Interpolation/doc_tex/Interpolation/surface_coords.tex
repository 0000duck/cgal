\section{Surface natural neighbor coordinates and surface neighbors on
a point set surface}\label{sec:surface}

This section introduces the functions to compute natural neighbor
coordinates and surface neighbors associated to a set of sample points
issued from a surface $\mathcal{S}$ and given a query point
$\mathbf{x}$ on $\mathcal{S}$. We suppose that $\mathcal{S}$ is a
closed and compact surface of $\mathbb{R}^3$, and let $\mathcal{P}=
\{\mathbf{p_1}, \ldots,\mathbf{p_n}\}$ be an $\epsilon$-sample of
$\mathcal{S}$ (refer to Amenta and Bern \cite{ab-srvf-99}). The
concepts are based on the definition of Boissonnat and Fl\"ototto
\cite{bf-lcss-02}, \cite{prisme-these-flototto}.  Both references
contain a thorough description of the requirements and the
mathematical properties.

\subsection{Introduction}

Two observations lead to the definition of surface neighbors and
surface neighbor coordinates: First, it is clear that the tangent
plane $\mathcal{T}_x$ of the surface $\mathcal{S}$ at the point
$\mathbf{x} \in \mathcal{S}$ approximates $\mathcal{S}$ in the
neighborhood of $\mathbf{x}$. It has been shown in \cite{bf-lcss-02}
that, if the surface $\mathcal{S}$ is well sampled with respect to the
curvature and the local thickness of $\mathcal{S}$, the intersection
of the tangent plane $\mathcal{T}_x$ with the Voronoi cell of
$\mathbf{x}$ in the Voronoi diagram of $\mathcal{P} \cup
\{\mathbf{x}\}$ has a small diameter.  Consequently, inside this
Voronoi cell, the tangent plane $\mathcal{T}_x$ is a reasonable
approximation of $\mathcal{S}$. Furthermore, the second observation
allows to compute this intersection diagram easily: one can show using
Pythagoras' theorem that the intersection of a three-dimensional
Voronoi diagram with a plane $\mathcal{H}$ is a two-dimensional power
diagram. The points defining the power diagram are the projections of
the points in $\mathcal{P}$ onto $\mathcal{H}$ each point weighted
with its negative square distance to $\mathcal{H}$. Algorithms for the
computation of power diagrams via the dual regular triangulation are
well known and for example provided by \cgal\ in the class
\ccc{Regular_triangulation_2<Gt, Tds>}.


\subsubsection{Voronoi intersection diagrams}



%We say that $\mathcal{H}$ is in {\em general position} if the
%intersection of an edge (resp.\ a face) of the power diagram with
%$\mathcal{H}$ is either empty or it is a vertex (resp.\ an edge), and
%if $\mathcal{H}$ does not contain any vertex of the power diagram.
%Notice, that a small perturbation of $\mathcal{H}$ always removes a
%degenerate position.  Without loss of generality, we suppose the
%general position of $\mathcal{H}$ in order to ensure that the general
%position assumption for the resulting $k$-dimensional power diagram is
%satisfied.

In \cgal, the regular triangulation dual to the intersection of a $3D$
Voronoi diagram with a plane $\mathcal{H}$ can be computed by
instantiating the \ccc{Regular_triangulation_2<Gt, Tds>} class with
the traits class \ccc{Voronoi_intersection_2_traits_3}. This traits
class contains a point and a vector as class member which define the
plane $\mathcal{H}$. All predicates and constructions used by the
triangulation algorithm are replaced by the corresponding operators on
three-dimensional points. For example, the power test predicate (which
takes three weighted points $p$, $q$, $r$ of the regular triangulation
and tests the power distance of a fourth point $t$ respect to the
power circle orthogonal to $p$, $q$, $r$) is replaced by a
\ccc{Side_of_plane_centered_sphere_2_3} predicate that tests the
position of a $3D$ point $t$ with respect to the sphere centered on
the plane $\mathcal{H}$ passing through the three-dimensional points
$p$, $q$, $r$.  This approach allows to avoid the explicite
constructions of the projected points and the weights which is very
prone to rounding errors.

\subsection{Natural neighbor coordinates on surfaces}

The computation of natural neighbor coordinates on surfaces is based
upon the computation of regular neighbor coordinates via the function
\ccc{regular_neighbor_coordinates_2} with respect to the regular
triangulation dual to the intersection diagram of $\mathcal{T}_x$ and
the Voronoi diagram of $\mathcal{P}$.


\subsection{Implementation}

\subsection{Examples}
