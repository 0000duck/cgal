% +------------------------------------------------------------------------+
% | Reference manual page: natural_neighbor_coordinates_2.tex
% +------------------------------------------------------------------------+
% | 
% | August 2003  Julia Floetotto
% | Package:   Interpolation
% | 
% |
% +------------------------------------------------------------------------+

%\renewcommand{\ccRefPageBegin}{\begin{ccAdvanced}}
%\renewcommand{\ccRefPageEnd}{\end{ccAdvanced}}
\begin{ccRefFunction}{natural_neighbor_coordinates_2}  %% add template arg's if necessary
\ccIndexSubitemBegin{Interpolation}{natural_neighbor_coordinates_2}

\ccDefinition
  
The function \ccRefName\ computes natural neighbor coordinates, also
called Sibson's coordinates, for $2D$ points provided a two-dimensional
triangulation and a query point inside the convex hull of the vertices
of the triangulation.

\ccInclude{CGAL/natural_neighbor_coordinates_2.h}


\ccFunction{template < class Dt, class OutputIterator > CGAL::Triple<
  OutputIterator, typename Dt::Geom_traits::FT, bool >
  natural_neighbor_coordinates_2(const Dt& dt, const typename
  Dt::Geom_traits::Point_2& p, OutputIterator out, typename
  Dt::Face_handle start = typename Dt::Face_handle());} { computes the
  natural neighbor coordinates for \ccc{p} with respect to the points
  in the two-dimensional Delaunay triangulation \ccc{dt}.  The
  template class \ccc{Dt} should be of type
  \ccc{Delaunay_triangulation_2<Traits, Tds>}.  The value type of the
  \ccc{OutputIterator} is a pair of \ccc{Dt::Point_2} and the
  coordinate value of type \ccc{Dt::Geom_traits::FT}. The sequence of
  point/coordinate pairs that is computed by the function is placed
  starting at \ccc{out}. The function returns a triple with an
  iterator that is placed past-the-end of the resulting sequence of
  point/coordinate pairs, the normalization factor of the coordinates
  and a boolean value which is set to true iff the coordinate
  computation was successful, i.e. if \ccc{p} lies inside the
  convex hull of the points in \ccc{dt}.  }

\ccFunction{template <class Dt, class OutputIterator,
  class EdgeIterator > CGAL::Triple< OutputIterator, typename Dt::Geom_traits::FT, 
  bool > natural_neighbor_coordinates_2(const Dt& dt, const typename
  Dt::Geom_traits::Point_2& p, OutputIterator out, EdgeIterator hole_begin,
  EdgeIterator hole_end);} { The same as above.
  \ccc{hole_begin} and \ccc{hole_end} determines the iterator range
  over the boundary edges of the conflict zone of \ccc{p} in the
  triangulation. It is the result of the function
  \ccc{T.get_boundary_of_conflicts(p,std::back_inserter(hole),
    start)}, see \ccc{Delaunay_triangulation_2<Traits, Tds>}.}

\ccFunction{template <class Dt, class OutputIterator> 
  CGAL::Triple< OutputIterator, typename Dt::Geom_traits::FT, bool > 
  natural_neighbor_coordinates_2(const Dt& dt,
  typename Dt::Vertex_handle vh, OutputIterator out);}{This function computes
  the natural neighbor coordinates of the point \ccc{vh->point()} with respect to
  the vertices of \ccc{dt} excluding \ccc{vh->point()}. The same as above for the
  remaining parameters.}

\ccHeading{Requirements}
\begin{enumerate}
\item \ccc{Dt} are equivalent to the class
  \ccc{Delaunay_triangulation_2<Traits, Tds>}.
\item The traits class \ccc{Traits} of \ccc{Dt} is a model of the
  concept \ccc{DelaunayTriangulationTraits_2}.
                                %\ccIndexMainItem[c]{DelaunayTriangulationTraits_2}.
  Only the following members of this traits class are used:
  \begin{itemize}
  \item \ccc{Construct_circumcenter_2}
  \item \ccc{FT}
  \item \ccc{Point_2}
  \item \ccc{construct_circumcenter_2_object}
  \item[] Additionally, \ccc{Traits} must meet the requirements for
    the traits class of the \ccc{polygon_area_2} function.
  \end{itemize}
\item \ccc{OutputIterator::value_type} is equivalent to
  \ccc{std::pair<Dt::Point_2, Dt::Geom_traits::FT>}, i.e.\ a pair
  associating a point and its natural neighbor coordinate.
\end{enumerate}

\ccSeeAlso
\ccRefIdfierPage{CGAL::linear_interpolation} \\
\ccRefIdfierPage{CGAL::sibson_c1_interpolation}\\ 
\ccRefIdfierPage{CGAL::surface_neighbor_coordinates_3} \\
\ccRefIdfierPage{CGAL::regular_neighbor_coordinates_2} 

\ccImplementation This function computes the area of the sub-cells
stolen from the Voronoi cells of the points in \ccc{dt} when inserting
\ccc{p}. The total area of the Voronoi cell of \ccc{p} is also
computed and returned by the function. If \ccc{p} lies outside the
convex hull, the coordinate values cannot be computed and the third
value of the result triple is set to \ccc{false}.

\ccIndexSubitemEnd{Interpolation}{natural_neighbor_coordinates_2}
\end{ccRefFunction}
