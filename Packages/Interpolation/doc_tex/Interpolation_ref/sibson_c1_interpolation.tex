% +------------------------------------------------------------------------+
% | Reference manual page: sibson_c1_interpolation.tex
% +------------------------------------------------------------------------+
% | 
% | August 2003  Julia Floetotto
% | Package:   Interpolation
% | 
% |
% +------------------------------------------------------------------------+

%\renewcommand{\ccRefPageBegin}{\begin{ccAdvanced}}
%\renewcommand{\ccRefPageEnd}{\end{ccAdvanced}}
\begin{ccRefFunction}{sibson_c1_interpolation}  %% add template arg's if necessary
\ccIndexMainItemBegin{sibson_c1_interpolation}

\ccDefinition
  
The function \ccRefName\ interpolates the function values and the
gradients that are provided following the method described in \cite{s-bdnni-81}.

\ccParameters The template parameter \ccc{Traits} is to be
instantiated with a model of \ccc{InterpolationTraits}.
\ccc{ForwardIterator::value_type} is a pair associating a point to a
(non-normalized) barycentric coordinate.  \ccc{norm} is the
normalization factor. The range $\left[\right.$
\ccc{first},\ccc{beyond}$\left.\right)$ contains the barycentric
coordinates for the query point \ccc{p}. The functor
\ccc{function_value} allows to access the value of the interpolated
function given a point. \ccc{function_gradient} allows to access the
function gradient given a point.

\ccInclude{CGAL/interpolation_functions.h}

\ccFunction{template < class ForwardIterator, class Functor, class
  GradFunctor, class Traits> std::pair< typename Functor::result_type,
  bool> sibson_c1_interpolation(ForwardIterator first, ForwardIterator
  beyond, const typename
  std::iterator_traits<ForwardIterator>::value_type::second_type&
  norm, const typename
  std::iterator_traits<ForwardIterator>::value_type:: first_type& p,
  Functor function_value, GradFunctor function_gradient,const Traits&
  traits);} { This function generates the interpolated function
  value at the point \ccc{p} using Sibson's $Z^1$ interpolant
  \cite{s-bdnni-81}.\\
  If the functor \ccc{function_gradient} cannot supply the gradient of
  a point, the function returns a pair where the boolean is set to
  \ccc{false}.  If the interpolation was successful, the pair contains
  the interpolated function value as first and \ccc{true} as second
  value.  \ccPrecond{\ccc{norm} $\neq 0$.
    \ccc{function_value(p).second == true} for all points \ccc{p} of
    the point/coordinate pairs in the range
    $\left[\right.$\ccc{first}, \ccc{beyond}$\left.\right)$.}  }

\ccHeading{Requirements}
\begin{enumerate}
\item  \ccc{Traits} is a model of the concept 
  \ccc{InterpolationTraits}\ccIndexMainItem[c]{InterpolationTraits}.
\item \ccc{ForwardIterator::value_type} is a point/coordinate pair.
  Precisely \ccc{ForwardIterator::value_type::first_type} is
  equivalent to \ccc{Traits::Point_d} and
  \ccc{ForwardIterator::value_type::second_type} is equivalent to
  \ccc{Traits::FT}.
\item \ccc{Functor::argument_type} must be equivalent to
  \ccc{Traits::Point_d} and \ccc{Functor::result_type} is a pair of
  the function value type and a boolean. The function value type must
  provide a multiplication and addition operation with the type
  \ccc{Traits::FT}.
\item \ccc{GradFunctor::argument_type} must be equivalent to
  \ccc{Traits::Point_d} and \ccc{Functor::result_type} is a pair of
  the type of the gradient of the function value and a boolean. The
  function gradient type must provide a multiplication operation with
  \ccc{Traits::Vector_d}.
\item A model of the functor types \ccc{Functor} (resp.\ 
  \ccc{GradFunctor}) is provided by the struct \ccc{Data_access}. It
  must be instantiated accordingly with an associative container
  (e.g.\ \stl\ \ccc{std::map}) having the point type as \ccc{key_type}
  and the function value type (resp.\ function gradient type) as
  \ccc{mapped_type}.
\item The number type \ccc{FT} provided by \ccc{Traits} must support
  the square root operation \ccc{sqrt()}.
\end{enumerate}

\ccFunction{template < class ForwardIterator, class Functor, class
  GradFunctor, class Traits> typename Functor::result_type
  sibson_c1_interpolation_square(ForwardIterator first,
  ForwardIterator beyond, const typename
  std::iterator_traits<ForwardIterator>::value_type::second_type&
  norm, Functor function_value, GradFunctor function_gradient,const
  Traits& traits);} {The same as above except that no square root
  operation is needed for \ccc{FT}.}

\ccSeeAlso 
\ccRefConceptPage{InterpolationTraits}\\
\ccRefConceptPage{GradientFittingTraits}\\
\ccRefIdfierPage{CGAL::Data_access<Map>}\\
\ccRefIdfierPage{CGAL::sibson_gradient_fitting} \\
\ccRefIdfierPage{CGAL::linear_interpolation} \\
%\ccRefIdfierPage{CGAL::farin_c1_interpolation} \\
%\ccRefIdfierPage{CGAL::quadratic_interpolation} \\
\ccRefIdfierPage{CGAL::Interpolation_traits_2<K>} \\
\ccRefIdfierPage{CGAL::Interpolation_gradient_fitting_traits_2<K>} \\
\ccRefIdfierPage{CGAL::natural_neighbor_coordinates_2}\\
\ccRefIdfierPage{CGAL::regular_neighbor_coordinates_2} \\
\ccRefIdfierPage{CGAL::surface_neighbor_coordinates_3}\\
%\ccImplementation

\ccIndexSubitemEnd{Interpolation}{sibson_c1_interpolation}
\end{ccRefFunction}
%\renewcommand{\ccRefPageBegin}{}
%\renewcommand{\ccRefPageEnd}{}

% +------------------------------------------------------------------------+
%%RefPage: end of main body, begin of footer
% EOF
% +------------------------------------------------------------------------+

