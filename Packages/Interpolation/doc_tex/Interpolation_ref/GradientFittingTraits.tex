% +------------------------------------------------------------------------+
% | Reference manual page: GradientFittingTraits.tex
% +------------------------------------------------------------------------+
% | 06.04.2000   Author
% | Package: Package
% | 
\RCSdef{\RCSGradientFittingTraitsRev}{$Revision$}
\RCSdefDate{\RCSGradientFittingTraitsDate}{$Date$}
% |
%%RefPage: end of header, begin of main body
% +------------------------------------------------------------------------+


\begin{ccRefConcept}{GradientFittingTraits}

%% \ccHtmlCrossLink{}     %% add further rules for cross referencing links
%% \ccHtmlIndexC[concept]{} %% add further index entries

\ccDefinition
  
The function \ccc{sibson_gradient_fitting} is parameterized by a
traits class that defines the primitives used by the algorithm.   The
concept \ccRefName\ defines this common set of requirements.


\ccTypes
\ccNestedType{FT}{The number type must follow the model
  \ccc{FieldNumberType}.}  
\ccNestedType{Point_d}{The point type on
  which the function is defined and interpolated.}
\ccNestedType{Vector_d}{The corresponding vector type.}
\ccNestedType{Aff_transformation_d}{defines a
  matrix type.
  Must provide the following member functions~: \\
  \ccc{Aff_transformation tr.inverse ()} which gives the inverse
  transformation, and \\
  \ccc{Aff_transformation tr.transform( Vector v)} which returns the
  multiplication of \ccc{tr} with \ccc{v}.
  }
%%% FUNCTION TYPES:
%VECTOR:
\ccNestedType{Construct_vector_d} {A constructor object for
  \ccc{Vector_d}.
  Provides~: \\
  \ccc{Vector_d operator() (Point_d a, Point_d b)} which produces the
  vector \ccc{b -
    a} and \\
  \ccc{Vector_d operator() (Null_vector NULL_VECTOR)} which introduces
  the null vector.}
\ccNestedType{Construct_scaled_vector_d}{Constructor object for
  \ccc{Vector_d}.
  Provides~: \\
  \ccc{Vector_d operator() (Vector_d v,FT scale)} which produces the
  vector \ccc{v} scaled by a factor \ccc{scale}.}
%AFF_TRANSFORMATION:
\ccNestedType{Construct_null_matrix_d}{Constructor object for
  \ccc{Aff_transformation_d}. Provides~: \\
  \ccc{Aff_transformation_d operator()()} which introduces an affine
  transformation whose matrix has only zero entries.}
\ccNestedType{Construct_scaling_matrix_d}{Constructor object for
  \ccc{Aff_transformation_d}. Provides~: \\
  \ccc{Aff_transformation_d operator()(FT scale)} which introduces a
  scaling by a scale factor \ccc{scale}.}
\ccNestedType{Construct_sum_matrix_d}{Constructor object for
  \ccc{Aff_transformation_d}. Provides~: \\
  \ccc{Aff_transformation_d operator()(Aff_transformation_d tr1,
    Aff_transformation_d tr2)} which returns the sum of the two matrices
  representing \ccc{tr1} and \ccc{tr2}.}
\ccNestedType{Construct_outer_product_d} {Constructor object for
  \ccc{Aff_transformation_d}. Provides~: \\
  \ccc{Aff_transformation_d operator()(Vector v)} which returns the
  outerproduct, i.e.~the quadratic matrix \ccc{v}$^t$\ccc{v}.}


\ccCreation
\ccCreationVariable{traits}  %%
\ccConstructor{GradientFittingTraits();}{default constructor.}


\ccOperations The following functions that create instances of the
above constructor object types must exist.
\ccMethod{Construct_vector_d construct_vector_d_object();}{} \ccGlue
\ccMethod{Construct_scaled_vector_d
  construct_scaled_vector_d_object();}{} \ccGlue
\ccMethod{Construct_null_matrix_d construct_null_matrix_d_object();}{}
\ccGlue \ccMethod{Construct_sum_matrix_d
  construct_sum_matrix_d_object();}{} \ccGlue
\ccMethod{Construct_outer_product_d
  construct_outer_product_d_object();}{}

\ccHasModels
\ccc{CGAL::Interpolation_gradient_fitting_traits_2<K>} \\

\ccSeeAlso
\ccRefConceptPage{InterpolationTraits}\\
\ccRefIdfierPage{CGAL::Interpolation_traits_2<K>} \\
\ccRefIdfierPage{CGAL::sibson_gradient_fitting} \\
\ccRefIdfierPage{CGAL::sibson_c1_interpolation} \\
\ccRefIdfierPage{CGAL::farin_c1_interpolation} \\
\ccRefIdfierPage{CGAL::quadratic_interpolation} \\
%\ccRefIdfierPage{CGAL::natural_neighbor_coordinates_2}\\
%\ccRefIdfierPage{CGAL::regular_neighbor_coordinates_2} \\
%\ccRefIdfierPage{CGAL::surface_neighbor_coordinates_3}\\
\end{ccRefConcept}

% +------------------------------------------------------------------------+
%%RefPage: end of main body, begin of footer
% EOF
% +------------------------------------------------------------------------+

