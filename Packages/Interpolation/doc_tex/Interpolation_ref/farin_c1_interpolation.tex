% +------------------------------------------------------------------------+
% | Reference manual page: farin_c1_interpolation.tex
% +------------------------------------------------------------------------+
% | 
% | August 2003  Julia Floetotto
% | Package:   Interpolation
% | 
% |
% +------------------------------------------------------------------------+

%\renewcommand{\ccRefPageBegin}{\begin{ccAdvanced}}
%\renewcommand{\ccRefPageEnd}{\end{ccAdvanced}}
\begin{ccRefFunction}{farin_c1_interpolation}  %% add template arg's if necessary
\ccIndexMainItemBegin{farin_c1_interpolation}

\ccDefinition
  
The function \ccRefName\ interpolates the function values and the
gradients that are provided by functors using the method described in  \cite{f-sodt-90}.

\ccInclude{CGAL/interpolation_functions.h}

\ccParameters \ccc{RandomAccessIterator::value_type} is a pair
associating a point to a (non-normalized) barycentric coordinate.  See
\ccc{sibson_c1_interpolation} for the other parameters.

\ccFunction{template < class RandomAccessIterator, class Functor,
  class GradFunctor, class Traits> typename Functor::result_type
  farin_c1_interpolation(RandomAccessIterator first,
  RandomAccessIterator beyond, const typename
  std::iterator_traits<RandomAccessIterator>::value_type::second_type&
  norm, Functor function_value, GradFunctor function_gradient, const
  Traits& traits);} {generates the interpolated function value
  computed by Farin's interpolant \cite{f-sodt-90}.  See also
  \ccc{sibson_c1_interpolation}.  \ccPrecond{\ccc{norm} $\neq 0$.
    \ccc{function_value(p).second == true} for all points \ccc{p} of
    the point/coordinate pairs in the range
    $\left[\right.$\ccc{first}, \ccc{beyond}$\left.\right)$.}
  \ccPrecond{The range $\left[\right.$ \ccc{first},
    \ccc{beyond}$\left.\right)$ contains either one or more than three
    elements.  }  }

\ccHeading{Requirements}

Same requirements as for \ccc{sibson_c1_interpolation} only the
iterator must provide random access and \ccc{Traits::FT} does not need
to provide the square root operation.

\ccSeeAlso
\ccRefIdfierPage{CGAL::Data_access<Map>}\\
\ccRefIdfierPage{CGAL::linear_interpolation} \\
\ccRefIdfierPage{CGAL::sibson_c1_interpolation} \\
\ccRefIdfierPage{CGAL::sibson_gradient_interpolation} \\
\ccRefIdfierPage{CGAL::quadratic_interpolation} \\
\ccRefIdfierPage{CGAL::Interpolation_traits_2} \\
\ccRefIdfierPage{CGAL::natural_neighbor_coordinates_2}\\
\ccRefIdfierPage{CGAL::regular_neighbor_coordinates_2} \\
\ccRefIdfierPage{CGAL::surface_neighbor_coordinates_3}\\
%\ccImplementation

\ccIndexSubitemEnd{Interpolation}{farin_c1_interpolation}
\end{ccRefFunction}
%\renewcommand{\ccRefPageBegin}{}
%\renewcommand{\ccRefPageEnd}{}

% +------------------------------------------------------------------------+
%%RefPage: end of main body, begin of footer
% EOF
% +------------------------------------------------------------------------+

