% +------------------------------------------------------------------------+
% | Reference manual page: Voronoi_intersection_2_traits_3.tex
% +------------------------------------------------------------------------+
% | 06.04.2000   Author
% | Package: Package
% | 
\RCSdef{\RCSInterpolationtraitsRev}{$Revision$}
\RCSdefDate{\RCSInterpolationtraitsDate}{$Date$}
% |
%%RefPage: end of header, begin of main body
% +------------------------------------------------------------------------+


\begin{ccRefClass}{Voronoi_intersection_2_traits_3<K>}
  


  %% \ccHtmlCrossLink{}     %% add further rules for cross referencing links
  %% \ccHtmlIndexC[class]{} %% add further index entries
  \ccCreationVariable{traits} 

  \ccDefinition \ccRefName\ is a model for the concept
  \ccc{RegularTriangulationTraits_2} and
  \ccc{InterpolationTraits}.  It can be used to instantiate the
  geometric traits class of a two-dimensional regular triangulation.
  A three-dimensional plane is defined by a point and a vector that
  are member of the traits class. The triangulation is defined on $3D$
  points. It is the regular triangulation of the input points
  projected onto the plane and weighted with the negative squared
  distance of the input point to the plane. It can be shown that it is
  dual to the power diagram obtained by intersecting the
  three-dimensional Voronoi diagram of the input points with the
  plane. All predicates and constructions used in the computation of
  the regular triangulation are formulated on the three dimensional
  points without explicitely constructing the projected points and the
  weights which reduces the arithmetic demands. The traits class is
  templated by a kernel class \ccc{K}.
  
  \ccInclude{CGAL/Voronoi_intersection_2_traits_3.h}

  \ccIsModel 
  \ccRefConceptPage{RegularTriangulationTraits_2}
  \ccIndexSubitem[c]{RegularTriangulationTraits_2}{model} \\
  
  \ccTypes
  \ccThreeToTwo 
  \ccTypedef{typedef K::RT                              Weight;}{}
  \ccTypedef{typedef K::FT                              FT;}{}
  \ccGlue
  \ccTypedef{typedef K::Point_3                         Point_2;}{}
  \ccGlue
  \ccTypedef{typedef K::Segment_3                       Segment_2;}{}
  \ccGlue
  \ccTypedef{typedef K::Triangle_3                      Triangle_2;}{}
  \ccGlue
  \ccTypedef{typedef K::Line_3                           Line_2;}{}
  \ccGlue
  \ccTypedef{typedef Rep::Ray_3                         Ray_2;}{}
  \ccGlue
  \ccTypedef{typedef Rep::Direction_3                 Direction_2;}{}
  \ccGlue
  \ccTypedef{typedef Rep::Vector_3                    Vector_2;}{}
  \ccGlue 
  \ccTypedef{typedef Rep::Construct_triangle_3       Construct_triangle_2;}{}
  \ccGlue
  \ccTypedef{typedef Rep::Construct_ray_3             Construct_ray_2;}{}
  \ccGlue
  \ccTypedef{typedef Rep::Construct_direction_3       Construct_direction_2;}{}
  \ccGlue
  \ccTypedef{typedef Rep::Compare_distance_3          Compare_distance_2;}{}
  \ccTypedef{Compute_area_3<Rep>     Compute_area_2;} 
  {An instance of this function object class computes the square 
    root of the result of \ccc{Rep::Compute_squared_area_3}. 
    If the number type \ccc{FT} does not suppport the square root 
    operation, the result is cast to \ccc{double} 
    before computing the square root.}
  \ccTypedef{ typedef Orientation_with_normal_plane_2_3<Rep>          Orientation_2;}{}
  \ccGlue
  \ccTypedef{typedef Side_of_plane_centered_sphere_2_3<Point_2>       Power_test_2;}{}
  \ccGlue
  \ccTypedef{typedef Side_of_plane_centered_sphere_degenerated_2_3<Point_2>  
    Power_test_degenerated_2;}{}
  \ccGlue
  \ccTypedef{ typedef Construct_plane_centered_circumcenter_3<Point_2> 
    Construct_weighted_circumcenter_2;}{}
  \ccGlue
  \ccTypedef{ typedef Compare_first_projection_3<Point_2>              Compare_x_2;}{}
  \ccGlue
  \ccTypedef{ typedef Compare_second_projection_3<Point_2>             Compare_y_2;}{}
  
  
\ccOperations
\ccAutoIndexingOff


\ccSeeAlso
\ccRefConceptPage{RegularTriangulationTraits_2}\\
\ccRefIdfierPage{CGAL::Regular_triangulation_2<Gt, Tds>}\\
\ccRefIdfierPage{CGAL::regular_neighbor_coordinates_2}\\
\ccRefIdfierPage{CGAL::surface_neighbor_coordinates_3}\\
\end{ccRefClass}

% +------------------------------------------------------------------------+
%%RefPage: end of main body, begin of footer
% EOF
% +------------------------------------------------------------------------+

