% +------------------------------------------------------------------------+
% | Reference manual page: Data_access.tex
% +------------------------------------------------------------------------+
% | 
% | August 2003  Julia Floetotto
% | Package:   Interpolation
% | 
% |
% +------------------------------------------------------------------------+
\begin{ccRefClass}{Data_access<Map>}  %% add template arg's if necessary
\ccIndexMainItemBegin{Data_access<Map>}

\ccDefinition
  
The struct \ccRefName\ implements a functor that allows to retrieve
data from an associative container. The functor keeps a reference to
the container. Given an instance of the container's key type, it
returns a pair of the container's value type and a boolean indicating
whether the retrieval was successful.\\

This class can be used to provide the function\_values and
function\_gradients of the interpolation functions.

\ccInclude{CGAL/interpolation_functions.h} 

\ccParameters The class
\ccRefName\ has the container type \ccc{Map} as template parameter.

\ccTypes 
\ccTypedef{typedef Map::mapped_type Data_type;}{}
\ccGlue
\ccTypedef{typedef Map::key_type Key_type;}{}
\ccCreation
\ccCreationVariable{data_access}  %% choose variable name
\ccConstructor{Data_acces(const Map& map);}
{Introduces a \ccc{Data_access} to the container \ccc{map}.}


\ccMethod{std::pair< Data_type, bool> operator()(const Key_type& p);} {If
  there is an entry for \ccc{p} in the container \ccc{map}, then the
  pair of \ccc{map.find(p)} and \ccc{true} is returned. Otherwise, the
  boolean value of the pair is \ccc{false}.}

\ccIndexSubitemEnd{Interpolation}{Data_access}
\end{ccRefClass}
%\renewcommand{\ccRefPageBegin}{}
%\renewcommand{\ccRefPageEnd}{}

% +------------------------------------------------------------------------+
%%RefPage: end of main body, begin of footer
% EOF
% +------------------------------------------------------------------------+

