% +------------------------------------------------------------------------+
% | Reference manual page: surface_neighbors_3.tex
% +------------------------------------------------------------------------+
% | 
% | August 2003  Julia Floetotto
% | Package:   Interpolation
% | 
% |
% +------------------------------------------------------------------------+

%\renewcommand{\ccRefPageBegin}{\begin{ccAdvanced}}
%\renewcommand{\ccRefPageEnd}{\end{ccAdvanced}}
\begin{ccRefFunction}{surface_neighbors_3}  %% add template arg's if necessary
\ccIndexSubitemBegin{interpolation}{surface_neighbors_3}

\ccDefinition

Given a set of sample points issued from a surface and a query point
\ccc{p}, the function \ccRefName\ computes the neighbors of \ccc{p} on
the surface within the sample points. If the sampling is sufficiently
dense, the neighbors are provably close to the point \ccc{p} on the
surface (cf.\ the manual pages and
\cite{bf-lcss-02},\cite{prisme-these-flototto}). They are defined to
be the neighbors of \ccc{p} in the regular triangulation dual
to the power diagram which is equivalent to the intersection of the
Voronoi cell of the query point \ccc{p} with the tangent plane to the
surface at \ccc{p}.

\ccInclude{CGAL/surface_neighbors_3.h}

\ccFunction{template <class OutputIterator, class InputIterator, class
  Kernel> OutputIterator surface_neighbors_3(InputIterator first,
  InputIterator beyond, const typename Kernel::Point_3& p, const
  typename Kernel::Vector_3& normal, OutputIterator out, const Kernel&
  K);}{ The sample points $\mathcal{P}$ are provided in the range
  $\left[\right.$\ccc{first}, \ccc{beyond}$\left.\right)$.
  \ccc{InputIterator::value_type} is the point type
  \ccc{Kernel::Point_3}. The tangent plane is defined by the point
  \ccc{p} and the vector \ccc{normal}.  The
  parameter \ccc{K} determines the kernel type that will instantiate the template parameter of \ccc{Voronoi_intersection_2_traits_3<K>}. \\
  The surface neighbors of \ccc{p} are computed which are the
  neighbors of \ccc{p} in the regular triangulation that is dual to
  the intersection of the $3D$ Voronoi diagram of $\mathcal{P}$ with
  the tangent plane. The point sequence that is computed by the
  function is placed starting at \ccc{out}. The function returns an
  iterator that is placed past-the-end of the resulting point
  sequence.}


\ccFunction{template <class OutputIterator, class InputIterator, class
  ITraits> OutputIterator
  surface_neighbors_3(InputIterator first, InputIterator beyond,
  const typename ITraits::Point_2& p,OutputIterator out, 
  const ITraits& traits);}{ the same as above only that the traits
  class must be instantiated by the user. \ccc{ITraits} must be equivalent
  to \ccc{Voronoi_intersection_2_traits_3<K>}.}


The next functions return, in addition, a boolean value that certifies
whether or not, the Voronoi cell of \ccc{p} can be affected by points
that lie outside the input range, i.e.\ outside the ball centered on \ccc{p} passing through the
furthest sample point from \ccc{p} in the range
$\left[\right.$\ccc{first}, \ccc{beyond}$\left.\right)$. If the sample
points are collected by a $k$-nearest neighbor or a range search
query, this permits to verify that a large enough neighborhood has
been considered. 

\ccFunction{template <class OutputIterator, class InputIterator, class
  Kernel> std::pair< OutputIterator, bool >
  surface_neighbors_certified_3(InputIterator first,
  InputIterator beyond, const typename Kernel::Point_3& p, const
  typename Kernel::Vector_3& normal, OutputIterator out, const Kernel&
  K);}{Similar to the first function. The additional third return
  value is \ccc{true} if the furthest point in the range
  $\left[\right.$\ccc{first}, \ccc{beyond}$\left.\right)$ is further
  away from \ccc{p} than twice the distance from \ccc{p} to the
  furthest vertex of the intersection of the Voronoi cell of \ccc{p}
  with the tangent plane defined be \ccc{(p,normal)}. It is
  \ccc{false} otherwise.}
\ccFunction{template <class OutputIterator, class InputIterator, class
  Kernel> std::pair< OutputIterator, bool >
  surface_neighbors_certified_3(InputIterator first,
  InputIterator beyond, const typename Kernel::Point_2& p, const
  typename Kernel::FT& max_distance, OutputIterator out, const
  Kernel& kernel);} { The same as above except that this function
  takes the maximal distance from \ccc{p} to the points in the range
  $\left[\right.$\ccc{first}, \ccc{beyond}$\left.\right)$ as additional
  parameter.}

\ccFunction{template <class OutputIterator, class InputIterator, class
  ITraits> std::pair< OutputIterator, bool >
  surface_neighbors_certified_3(InputIterator first, InputIterator
  beyond, const typename ITraits::Point_2& p, OutputIterator out,
  const ITraits& traits);}{The same as above only that the traits
  class must be instantiated by the user.  \ccc{ITraits} must be
  equivalent to \ccc{Voronoi_intersection_2_traits_3<K>}. There is no
  parameter \ccc{max_distance}.}

\ccFunction{template <class OutputIterator, class InputIterator, class
  ITraits> std::pair< OutputIterator, bool >
  surface_neighbors_certified_3(InputIterator first,
  InputIterator beyond, const typename ITraits::Point_2& p, const
  typename ITraits::FT& max_distance, OutputIterator out, const
  ITraits& traits);} { The same as above with the parameter
  \ccc{max_distance}.}

The next function allows to filter some potential neighbors of the
query point \ccc{p} from $\mathcal{P}$ via its three-dimensional
Delaunay triangulation. All surface neighbors of \ccc{p} are
necessarily neighbors in the Delaunay triangulation of $\mathcal{P}
\cup \{p\}$.  

\ccFunction{template < class Dt, class OutputIterator >
  OutputIterator
  surface_neighbors_3(const Dt& dt, const typename
  Dt::Geom_traits::Point_2& p, const typename
  Dt::Geom_traits::Vector_3& normal, OutputIterator out, typename
  Dt::Face_handle start = typename Dt::Face_handle(NULL));} {computes
  the surface neighbor coordinates with respect to the points that are
  vertices of the Delaunay triangulation \ccc{dt}. The type \ccc{Dt}
  must be equivalent to \ccc{Delaunay_triangulation_3<Gt, Tds>}. The
  optional parameter \ccc{start} is used for the used as a starting
  place for the search of the conflict zone. It may be the result of
  the call \ccc{dt.locate(p)}. This function instantiates the template
  parameter \ccc{ITraits} to be
  \ccc{Voronoi_intersection_2_traits_3<Dt::Geom_traits>}.}


\ccFunction{template < class Dt, class OutputIterator,
  class ITraits> 
  OutputIterator surface_neighbors_3(const Dt& dt,
  const typename Dt::Geom_traits::Point_2& p, OutputIterator out,
  const ITraits& traits, typename Dt::Face_handle start = typename
  Dt::Face_handle(NULL));}
%
{The same as above only that the parameter \ccc{traits} instantiates
  the geometric traits class. Its type \ccc{ITraits} must be
  equivalent to \ccc{Voronoi_intersection_2_traits_3<K>}.}



\ccHeading{Requirements}
\begin{enumerate}
\item \ccc{Dt} is equivalent to the class
  \ccc{Delaunay_triangulation_3}.
\item \ccc{OutputIterator::value_type} is equivalent to
  \ccc{Dt::Point_3}, i.e.\ a point type.
\item  \ccc{ITraits} is equivalent to the class \ccc{Voronoi_intersection_2_traits_3<K>}.
\end{enumerate}

\ccSeeAlso
\ccRefIdfierPage{CGAL::Voronoi_intersection_2_traits_3<K>}\\
\ccRefIdfierPage{CGAL::surface_neighbor_coordinates_2} \\

\ccImplementation This function computes the regular triangulation of
the sample points and the point \ccc{p} using a traits class
equivalent to \ccc{Voronoi_intersection_2_traits_3<K>}. The functions
which certify the result, need to compute the Voronoi vertices of the
cell of \ccc{p} in this diagram.

\ccIndexSubitemEnd{Interpolation}{surface_neighbors_3}
\end{ccRefFunction}
%\renewcommand{\ccRefPageBegin}{}
%\renewcommand{\ccRefPageEnd}{}

% +------------------------------------------------------------------------+
% RefPage: end of main body, begin of footer
% EOF
% +------------------------------------------------------------------------+

