% +------------------------------------------------------------------------+
% | Reference manual page: Interpolation_traits.tex
% +------------------------------------------------------------------------+
% | 06.04.2000   Author
% | Package: Package
% | 
\RCSdef{\RCSInterpolationtraitsRev}{$Revision$}
\RCSdefDate{\RCSInterpolationtraitsDate}{$Date$}
% |
%%RefPage: end of header, begin of main body
% +------------------------------------------------------------------------+


\begin{ccRefConcept}{InterpolationTraits}

%% \ccHtmlCrossLink{}     %% add further rules for cross referencing links
%% \ccHtmlIndexC[concept]{} %% add further index entries

\ccDefinition
  
Most interpolation functions are parameterized by a traits class that
defines the primitives used in the algorithms.  The concept
\ccRefName\ defines this common set of requirements.  


\ccTypes \ccNestedType{FT}{The number type.}
\ccNestedType{Construct_vector} {A constructor object for
  \ccc{Vector}.
  Provides~: \\
  \ccc{Vector operator() (Point a, Point b)} which produces the vector
  \ccc{b -
    a} and \\
  \ccc{Vector operator() (Null_vector NULL_VECTOR)} which introduces
  the null vector.}
\ccNestedType{Construct_scaled_vector}{Constructor object
  \ccc{Vector}.
  Provides~: \\
  \ccc{Vector operator() (Vector v,FT scale)} which produces the
  vector \ccc{v} scaled by a factor \ccc{scale}.}
\ccNestedType{Compute_squared_distance} {Constructor
  object. Provides the operator:\\
  \ccc{FT operator() (Point a, Point b)} returning the squared
  distance between \ccc{a} and \ccc{b}.}

\ccCreation
\ccCreationVariable{traits}  %% choose variable name
A default constructor and a copy constructor
 and an assignement operator are required. 
%Note that further constructors
%can be provided. 
%%\ccThree{InterpolationTraits}{traits=gtrxx  }{}
%\ccConstructor{InterpolationTraits();}{default constructor.}
%\ccGlue
%\ccConstructor{InterpolationTraits(InterpolationTraits gtr);}
%{Copy constructor}
%\ccMethod{InterpolationTraits operator=(InterpolationTraits gtr);}
%{Assignment operator.}

\ccOperations
\ccHeading{Predicate functions}
The following functions that create instances of the above predicate object
types must exist.

\ccMethod{Construct_vector construct_vector_object();}{}
\ccGlue
\ccMethod{Construct_scaled_vector construct_scaled_vector_object();}{}
\ccGlue
\ccMethod{Compute_squared_distance compute_squared_distance_object();}{}

\ccHasModels
\ccc{CGAL::Interpolation_traits_2<K>} \\

\ccSeeAlso
\ccRefIdfierPage{CGAL::linear_interpolation} \\
\ccRefIdfierPage{CGAL::sibson_c1_interpolation} \\
\ccRefIdfierPage{CGAL::sibson_gradient_fitting} \\
\ccRefIdfierPage{CGAL::farin_c1_interpolation} \\
\ccRefIdfierPage{CGAL::quadratic_interpolation} \\
\ccRefIdfierPage{CGAL::natural_neighbor_coordinates_2}\\
\ccRefIdfierPage{CGAL::regular_neighbor_coordinates_2} \\
\ccRefIdfierPage{CGAL::surface_neighbor_coordinates_2_3}\\
\end{ccRefConcept}

% +------------------------------------------------------------------------+
%%RefPage: end of main body, begin of footer
% EOF
% +------------------------------------------------------------------------+

