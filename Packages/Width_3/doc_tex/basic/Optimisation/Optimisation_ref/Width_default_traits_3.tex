% =============================================================================
% The CGAL Reference Manual
% Chapter: Geometric Optimisation
% Class  : CGAL::Width_default_traits_3<Traits>
% -----------------------------------------------------------------------------
% file  : doc_tex/basic/Optimisation/..._ref/Width_default_traits_3.tex
% author: Thomas Herrmann, Sven Sch�nherr <herrmann@ifor.math.ethz.ch>
% -----------------------------------------------------------------------------
% $CGAL_Chapter: Geometric Optimisation $
% $CGAL_Package: Width_3 $
% $Revision$
% $Date$
% =============================================================================

\ccAutoIndexingOff

\begin{ccRefClass}{Width_default_traits_3<R>}
\ccIndexTraitsClassBegin{Width_default_traits_3}{Width_3}{}

\ccAutoIndexingOn

% -----------------------------------------------------------------------------
\ccDefinition

The class \ccRefName\ is a traits class for \ccc{Width_3<Traits>}
using the three-dimensional \cgal\ kernel.
 
\ccInclude{CGAL/Width_default_traits_3.h}

% -----------------------------------------------------------------------------
\ccRequirements

The template parameter \ccc{R} is a model for \ccc{Kernel}

% -----------------------------------------------------------------------------
\ccIsModel

\ccRefConceptPage{Width_traits_3}

% -----------------------------------------------------------------------------
\ccTypes

\ccSaveThreeColumns
\ccSetTwoColumns{Width_default_traits_3<R>:: ChullTraits;}{}

\ccGlueBegin
\ccNestedType{Point}{typedef to \ccc{CGAL::Point_3<R>}.}
\ccNestedType{Plane}{typedef to \ccc{CGAL::Plane_3<R>}.}
\ccNestedType{Vector}{typedef to \ccc{CGAL::Vector_3<R>}.}
\ccNestedType{RT}{typedef to \ccc{R::RT}.}
\ccNestedType{ChullTraits}{typedef to \ccc{chull_traits_3<R>}.}
\ccGlueEnd

% -----------------------------------------------------------------------------
\ccCreation
\ccCreationVariable{traits}

\ccConstructor{ Width_default_traits_3( );}{default constructor.}

% -----------------------------------------------------------------------------
\ccOperations

\ccSetThreeColumns{Vector}{get_point_coefficients();}{}
\ccGlueBegin
\ccMemberFunction{ RT get_hx(const Point& p) const;}{returns the
  homogeneous $x$-coordinate of point $p$.}  

\ccMemberFunction{ RT get_hy(const Point& p) const;}{returns the
  homogeneous $y$-coordinate of point $p$.}  

\ccMemberFunction{ RT get_hz(const Point& p) const;}{returns the
  homogeneous $z$-coordinate of point $p$.}  

\ccMemberFunction{ RT get_hw(const Point& p) const;}{returns the
  homogenizing coordinate of point $p$.}  

\ccMemberFunction{ void get_point_coordinates(const Point& p, RT& px,
  RT& py, RT& pz, RT& ph) const;}{returns all homogeneous coordinates
  of point $p$ at once.}
\ccGlueEnd

\ccGlueBegin
\ccMemberFunction{ RT get_a(const Plane& f) const;}{returns the first
  coefficient of plane $f$.}
\ccMemberFunction{ RT get_b(const Plane& f) const;}{returns the second
  coefficient of plane $f$.}
\ccMemberFunction{ RT get_c(const Plane& f) const;}{returns the third
  coefficient of plane $f$.}
\ccMemberFunction{ RT get_d(const Plane& f) const;}{returns the fourth
  coefficient of plane $f$.}
\ccMemberFunction{ void get_plane_coefficients(const Plane& f, 
  RT& a, RT& b, RT& c, RT& d)
  const;}{returns all four plane coefficients of $f$ at once.}
\ccGlueEnd

\ccGlueBegin
\ccMemberFunction{ Point make_point(const RT& hx, const RT& hy, const RT& hz, 
  const RT& hw) const;}{returns a point of type
  \ccc{Point} with homogeneous coordinates $hx$, $hy$, $hz$ and $hw$.}
\ccMemberFunction{ Plane make_plane(const RT& a, const RT& b, const
  RT& c, const RT& d) const;}{returns a plane of type \ccc{Plane}
  whose coefficients are $a$, $b$, $c$ and $d$.}
\ccMemberFunction{ Vector make_vector(const RT& a, const RT& b, const
  RT& c) const;}{returns a vector of type \ccc{Vector} with the four 
  coefficients $a$, $b$, $c$ and 1.}
\ccGlueEnd

\ccGlueBegin
\ccMemberFunction{ void inverse_normal(Vector& normal);}{inverts the
  vector \ccc{normal}, \ie \ccc{normal} becomes \ccc{-normal}.}
\ccMemberFunction{ void opposite_plane(Plane& pl);}{computes the
  opposite plane of \ccc{pl}, \ie the orientation of plane \ccc{pl} is
  inverted.}
\ccMemberFunction{ Vector orthogonal_vector(Plane& pl);} {returns the
  vector, that is orthogonal to the plane \ccc{pl} and that points to
  the positive side of \ccc{pl}.}
\ccGlueEnd

\ccRestoreThreeColumns

% -----------------------------------------------------------------------------
\ccSeeAlso

\ccRefIdfierPage{CGAL::Width_3<Traits>}\\[1ex]
\ccRefConceptPage{Width_traits_3}

% -----------------------------------------------------------------------------

\ccAutoIndexingOff

\ccIndexTraitsClassEnd
\end{ccRefClass}

\ccAutoIndexingOn

% ===== EOF ===================================================================
