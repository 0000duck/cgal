% =============================================================================
% The CGAL Reference Manual
% Chapter: Geometric Optimisation
% Class  : CGAL::Width_3<Traits>
% -----------------------------------------------------------------------------
% file  : doc_tex/basic/Optimisation/Optimisation_ref/Width_3.tex
% author: Thomas Herrmann, Sven Sch�nherr <herrmann@ifor.math.ethz.ch>
% -----------------------------------------------------------------------------
% $CGAL_Chapter: Geometric Optimisation $
% $CGAL_Package: Width_3 $
% $Revision$
% $Date$
% =============================================================================

\begin{ccRefClass}{Width_3<Traits>}

\ccIndexSubitem       [t]{width}{of 3D point set}
\ccIndexSubitem       [t]{point set}{3D width of}
\ccIndexSubitemSeeAlso[t]{polyhedron}{width of 3D point set}
\ccIndexSubitem       [t]{optimal distances}{width of 3D point set}

\ccSaveThreeColumns

% -----------------------------------------------------------------------------
\ccDefinition

Given a set of points $\mathcal{S}=\left\{p_1,\ldots , p_n\right\}$ in
$\R^3$. The width of $\mathcal{S}$, denoted as $\mathcal{W(S)}$, is defined
as the minimum distance between two parallel planes of support of
$\mathit{conv(\mathcal{S})}$; where $\mathit{conv(\mathcal{S})}$ denotes
the convex hull of $\mathcal{S}$. The width in direction $\mathbf{d}$,
denoted as $\mathcal{W}_d\mathcal{(S)}$, is the distance between two
parallel planes of support of $\mathit{conv(\mathcal{S})}$, which are
orthogonal to $\mathbf{d}$.

Subject to the applications of the width algorithm, several objects
might be interesting: 
\begin{enumerate}
\item The two parallel planes of support such that the distance
  between them is as small as possible. These planes are called
  width-planes in further considerations.
\item The width $\mathcal{W(S)}$, i.e., the distance between the
  width-planes.
\item The direction $\mathbf{d}_{opt}$ such that
  $\mathcal{W(S)}=\mathcal{W}_{d_{opt}}\mathcal{(S)}$
\end{enumerate}

\emph{Note:} There might be several optimal build directions. Hence
neither the width-planes nor the direction $\mathbf{d}_{opt}$ are
unique -- only the width is.

\ccInclude{CGAL/Width_3.h}

% -----------------------------------------------------------------------------
\ccHeading{Requirements}
\ccIndexRequirements

The template parameter \ccc{Traits} is a model for \ccc{WidthTraits_3}.
We provide the model \ccc{Width_default_traits_3<Kernel>} based on a
three-dimensional \cgal~kernel.

% -----------------------------------------------------------------------------
\ccTypes
\ccIndexClassTypes

% All the requirements in this section were redundant with the 
% description of WidthTraits_3. Removed. L.K.
%
% According to the two different possiblities to create the width (by an
% iterator-range or by a polyhedron), there are slightly different
% requirements to the types. Note: If the width of a polyhedron is computed,
% then there is no need for a traits class for the convex hull algorithm. For
% any additional requirement to the types and functions of the traits class,
% especially according to the polyhedron and its structure, see concept
% \ccc{WidthTraits_3}. A default traits class \ccc{Width_default_traits_3}
% is provided. For the requirements of the convex hull traits class the
% reader is referred to the reference page for the concept ConvexHullTraits\_3.
%
% \paragraph{Creation with Iterator-Range}\hfill\newline

\ccSetThreeColumns{typedef typename Traits::ChullTraits}{CHullTraits;}{}
\ccThreeToTwo

\ccGlueBegin 

\ccNestedType{Traits}{traits class.}
\ccTypedef{typedef typename Traits::Point_3      Point_3;}{point type.}
\ccTypedef{typedef typename Traits::Plane_3      Plane_3;}{plane type.}
\ccTypedef{typedef typename Traits::Vector_3     Vector_3;}{vector type.}
\ccTypedef{typedef typename Traits::RT           RT;}{algebraic ring type.}
\ccTypedef{typedef typename Traits::ChullTraits  ChullTraits;}{traits
                                    class for the 3D convex hull algorithm.}
\ccGlueEnd

% \paragraph{Creation with a Polyhedron}\hfill\newline
% \ccSetThreeColumns{typedef Traits::Polyhedron}{Polyhedron;}{}
% \ccNestedType{Traits}{typedef to \ccc{Traits}.}
% \ccNestedType{Point_3}{typedef to \ccc{Traits::Point_3}. It is assumed that \ccc{InputPolyhedron::Point_3} equals \ccc{Traits::Point_3}.}
% \ccNestedType{Plane_3}{typedef to \ccc{Traits::Plane_3}. It is assumed that \ccc{Polyhedron::Plane_3} equals \ccc{Traits::Plane_3}.}
% \ccNestedType{Vector_3}{typedef to \ccc{Traits::Vector_3}. It is assumed that \ccc{Plane_3::Normal} equals \ccc{Traits::Vector_3}.}
% \ccNestedType{RT}{typedef to \ccc{Traits::RT}. It is assumed that the underlying number type of \ccc{Point_3, Plane_3} and \ccc{Vector_3} is \ccc{RT}.}
% \ccGlueEnd

% -----------------------------------------------------------------------------
\ccCreation
\ccIndexClassCreation
\ccCreationVariable{width}
\ccSetThreeColumns{Vector_3}{get_build_direction ( );}{}
\ccThreeToTwo

% Again, redundant. The constructors say it also. Removed. L.K.
% A \ccClassTemplateName\ object can be created from an arbitrary point set
% $\mathcal{S}$ or from a polyhedron $\mathcal{P}$. In the latter case the
% given polyhedron is assumed to be convex.

\ccConstructor{ template < class InputIterator > 
    Width_3( InputIterator first, InputIterator beyond);}{
    creates a variable \ccVar\ initialized to the width of $\mathcal{S}$ -- 
    with $\mathcal{S}$ being the set of points in the range 
    [\ccc{first},\ccc{beyond}).  
    \ccCommentHeading{Requirement} The value type of \ccc{InputIterator} 
    is \ccc{Point_3}.}

\ccConstructor{ template < class Polyhedron >
    Width_3( Polyhedron& P);}{creates a variable \ccVar\ initialized to 
    the width of the polyhedron $P$. Note that the vertex point coordinates
    are altered!
    \ccPrecond $P$ is a convex polyhedron.
    \ccCommentHeading{Requirement} \ccc{Polyhedron} is a 
    \ccc{CGAL::Polyhedron_3} with facets supporting plane equations
    where \ccc{Polyhedron::Point_3} $\equiv$ \ccc{Point_3} and
    \ccc{Polyhedron::Plane_3} $\equiv$ \ccc{Plane_3}.}

% -----------------------------------------------------------------------------
\ccAccessFunctions
\begin{ccIndexMemberFunctions}
\ccIndexMemberFunctionGroup{access}

\ccMemberFunction{ void get_squared_width ( RT& width_num, RT&
  width_denom );}{ returns the squared width. For the reason of exact
  computation not the width itself is stored, but the \emph{squared}
  width as a fraction: The numerator in \ccc{width_num} and the
  denominator in \ccc{width_denom}. The width of the point set
  $\mathcal{S}$ is
  $\sqrt{\frac{\ccc{width\_num}}{\ccc{width\_denom}}}$.}

\ccMemberFunction{ void get_width_planes ( Plane_3& e1, Plane_3& e2 );}{
  The planes \ccc{e1} and \ccc{e2} are the two parallel supporting
  planes, which distance is minimal (among all such planes).}

\ccMemberFunction{ void get_width_coefficients ( RT& A,RT& B,RT& C,
  RT& D, RT& K );}{The returned coefficients \ccc{A,B,C,D,K} have the
  property that width-plane \ccc{e1} is given by the equation
  $Ax+By+Cz+D=0$ and width-plane \ccc{e2} by $Ax+By+Cz+K=0$.}  

\ccMemberFunction{ Vector_3 get_build_direction ( );}{ returns a
  direction $\mathbf{d}_{opt}$ such that the width-planes \ccc{e1} and
  \ccc{e2} are perpendicular to $\mathbf{d}_{opt}$. The width of the
  point set is minimal in this direction.}

% naming convention: variable names are lowercase, so no AllDir. L.K.
\ccMemberFunction{ void get_all_build_directions ( std::vector<Vector_3>&
  dir );}{ All the build directions are stored in the vector
  \ccc{dir}. It might happen that a certain body has several
  different build directions, but it is also possible to have only one
  build direction.}  

\ccMemberFunction{ int get_number_of_optimal_solutions( );}{ returns
  the number of optimal solutions, i.e., the number of optimal build
  directions.}

\end{ccIndexMemberFunctions}

% -----------------------------------------------------------------------------
\ccSeeAlso

\ccRefIdfierPage{CGAL::Width_default_traits_3<K>}\\[1ex]
\ccRefConceptPage{WidthTraits_3}

% -----------------------------------------------------------------------------
\ccImplementation
\ccIndexImplementation

Since the width of the point set $\mathcal{S}$ and the width of the convex
hull of $\mathcal{S}$ ($\mathit{conv(\mathcal{S})}$) is the same, the
algorithm uses the 3D convex hull algorithm \cgal\ provides.

The width-algorithm is not incremental and therefore inserting and
erasing points cause not an `automatic' update of the width. Instead
you have to run the width-algorithm again even if the point set is
extended by only one new point.

\begin{ccAdvanced}

\paragraph{Large Numbers.}

Because there is no need for dividing values during the algorithm, the
numbers can get really huge (all the computations are made using a lot
of multiplications).  Therefore it is strongly recommended to use a
number type that can handle numbers of arbitrary length (e.g.,
\ccc{leda_integer} in combination with the homogeneous representation
of the points). But these large numbers have a disadvantage:
Operations on them are slower as greater the number gets. Therefore it
is possible to shorten the numbers by using the compiler flag
\textsc{-Dsimplify}. For using this option it is required that
the underlying number type provides the `modulo' operation. 

\paragraph{Information Output during the Computations.} 

If during the algorithm the program should output some information
(e.g., during the debugging phase) you can turn on the output
information by giving the compiler flag \textsc{debug}. In the file
\texttt{width\_assertions.h} you can turn on/off the output of some
functions and additional informations by changing the defined values
from 0 (no output) to 1 (output available). But then it is required
that the $<\!<$-operator has to been overloaded for \ccc{Point_3},
\ccc{Plane_3}, \ccc{Vector_3} and \ccc{RT}.

\end{ccAdvanced}

% -----------------------------------------------------------------------------
\ccExample
\ccIndexSubitem[C]{Width_3}{example}

\ccIncludeExampleCode{Optimisation_ref/test_width_simplex.C}

% -----------------------------------------------------------------------------

\ccRestoreThreeColumns

\end{ccRefClass}

% ===== EOF ===================================================================
