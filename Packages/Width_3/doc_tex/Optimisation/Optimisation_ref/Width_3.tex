% =============================================================================
% The CGAL Reference Manual
% Chapter: Geometric Optimisation
% Class  : CGAL::Width_3<Traits>
% -----------------------------------------------------------------------------
% file  : doc_tex/basic/Optimisation/Optimisation_ref/Width_3.tex
% author: Thomas Herrmann, Sven Sch�nherr <herrmann@ifor.math.ethz.ch>
% -----------------------------------------------------------------------------
% $CGAL_Chapter: Geometric Optimisation $
% $CGAL_Package: Width_3 $
% $Revision$
% $Date$
% =============================================================================

\begin{ccRefClass}{Width_3<Traits>}

\ccIndexSubitem       [t]{width}{of 3D point set}
\ccIndexSubitem       [t]{point set}{3D width of}
\ccIndexSubitemSeeAlso[t]{polyhedron}{width of 3D point set}
\ccIndexSubitem       [t]{optimal distances}{width of 3D point set}

\ccSaveThreeColumns

% -----------------------------------------------------------------------------
\ccDefinition

Given a set of points $\mathcal{S}=\left\{p_1,\ldots , p_n\right\}$ in
$\R^3$. The width of $\mathcal{S}$, denoted as $\mathcal{W(S)}$, is defined
as the minimum distance between two parallel planes of support of
$\mathit{conv(\mathcal{S})}$; where $\mathit{conv(\mathcal{S})}$ denotes
the convex hull of $\mathcal{S}$. The width in direction $\mathbf{d}$,
denoted as $\mathcal{W}_d\mathcal{(S)}$, is the distance between two
parallel planes of support of $\mathit{conv(\mathcal{S})}$, which are
orthogonal to $\mathbf{d}$.

Subject to the applications of the width algorithm, several objects
might be interesting: 
\begin{enumerate}
\item The two parallel planes of support such that the distance
  between them is as small as possible. These planes are called
  width-planes in further considerations.
\item The width $\mathcal{W(S)}$, \ie the distance between the
  width-planes.
\item The direction $\mathbf{d}_{opt}$ such that
  $\mathcal{W(S)}=\mathcal{W}_{d_{opt}}\mathcal{(S)}$
\end{enumerate}

\emph{Note:} There might be several optimal build directions. Hence
neither the width-planes nor the direction $\mathbf{d}_{opt}$ are
unique -- only the width is.

\ccInclude{CGAL/Width_3.h}

% -----------------------------------------------------------------------------
\ccRequirements
\ccIndexRequirements

The template parameter \ccc{Traits} is a model for \ccc{Width_traits_3}.

We provide the model \ccc{Width_default_traits_3} based on the
three-dimensional \cgal~kernel, respectively.

% -----------------------------------------------------------------------------
\ccTypes
\ccIndexClassTypes

According to the two different possiblities to create the width (by an
iterator-range or by a polyhedron), there are slightly different
requirements to the types. Note: If the width of a polyhedron is computed,
then there is no need for a traits class for the convex hull algorithm. For
any additional requirement to the types and functions of the traits class,
especially according to the polyhedron and its structure, see concept
\ccc{Width_traits_3}. A default traits class \ccc{Width_default_traits_3}
is provided. For the requirements of the convex hull traits class the
reader is referred to the chapters about
the 3D convex hull%~\ref{sec:ConvHull_3}
and to the LEP dd-geokernel documentation.

\paragraph{Creation with Iterator-Range}\hfill\newline
\ccSetThreeColumns{typedef Traits::ChullTraits}{CHullTraits;}{}

\ccGlueBegin 
\ccNestedType{Point}{typedef to \ccc{Traits::Point}.}
\ccNestedType{Plane}{typedef to \ccc{Traits::Plane}.}
\ccNestedType{Vector}{typedef to \ccc{Traits::Vector}.}
\ccNestedType{RT}{typedef to \ccc{Traits::RT}. It is assumed
  that the underlying number type of \ccc{Point, Plane} and
  \ccc{Vector} is \ccc{RT}.}
\ccNestedType{ChullTraits}{typedef to \ccc{Traits::ChullTraits}. The traits
  class for the 3D convex hull algorithm. It has to define a model for the
  subconcepts of the \ccc{chull}-traits class concept as required in the
  LEP dd-geokernel, including the requirements for the
  \ccc{d3_surface_map()} output routine.}
\ccGlueEnd

\paragraph{Creation with a Polyhedron}\hfill\newline
\ccSetThreeColumns{typedef Traits::Polyhedron}{Polyhedron;}{}

\ccGlueBegin 
\ccNestedType{Traits}{typedef to \ccc{Traits}.}
\ccNestedType{Point}{typedef to \ccc{Traits::Point}. It is assumed that \ccc{InputPolyhedron::Point} equals \ccc{Traits::Point}.}
\ccNestedType{Plane}{typedef to \ccc{Traits::Plane}. It is assumed that \ccc{Polyhedron::Plane} equals \ccc{Traits::Plane}.}
\ccNestedType{Vector}{typedef to \ccc{Traits::Vector}. It is assumed that \ccc{Plane::Normal} equals \ccc{Traits::Vector}.}
\ccNestedType{RT}{typedef to \ccc{Traits::RT}. It is assumed that the underlying number type of \ccc{Point, Plane} and \ccc{Vector} is \ccc{RT}.}
\ccGlueEnd

% -----------------------------------------------------------------------------
\ccCreation
\ccIndexClassCreation
\ccCreationVariable{width}

A \ccClassTemplateName\ object can be created from an arbitrary point set
$\mathcal{S}$ or from a polyhedron $\mathcal{P}$. In the latter case the
given polyhedron is assumed to be convex.

\ccConstructor{ template < class InputIterator > width_3(
  const InputIterator first, const InputIterator beyond);}{creates a
  variable \ccVar\ of type \ccClassTemplateName.  It is initialized to
  the width of $\mathcal{S}$ -- with $\mathcal{S}$ being the set of
  points in the range [\ccc{first},\ccc{beyond}).  \ccPrecond The
  value type of \ccc{first} and \ccc{beyond} is \ccc{Point}.}
\ccConstructor{ template < class InputPolyhedron > width_3(
  InputPolyhedron& Poly);}{creates a variable \ccVar\ of type
  \ccClassTemplateName.  It is initialized to the width of
  $\mathcal{P}$ -- with $\mathcal{P}$ being the polyhedron to compute
  the width of. It is assumed for the whole algorithm, that the
  polyhedron is convex, \ie the convex hull won't be computed anymore.
  \ccPrecond \ccc{Poly} has to satisfy several additional the
  restrictions (see concept \ccc{Width_traits_3}) and has to be convex.}

% -----------------------------------------------------------------------------
\ccAccessFunctions
\begin{ccIndexMemberFunctions}
\ccIndexMemberFunctionGroup{access}

\ccSetThreeColumns{Vector}{get_build_direction ( );}{}

\ccMemberFunction{ void get_squared_width ( RT& width_num, RT&
  width_denom );}{ returns the squared width. For the reason of exact
  computation not the width itself is stored, but the \emph{squared}
  width as a fraction: The numerator in \ccc{width_num} and the
  denominator in \ccc{width_denom}. The width of the point set
  $\mathcal{S}$ is
  $\sqrt{\frac{\ccc{width\_num}}{\ccc{width\_denom}}}$.}

\ccGlueBegin 
\ccMemberFunction{ void get_width_planes ( Plane& e1, Plane& e2 );}{
  The planes \ccc{e1} and \ccc{e2} are the two parallel supporting
  planes, which distance is minimal (among all such planes).}

\ccMemberFunction{ void get_width_coefficients ( RT& A,RT& B,RT& C,
  RT& D, RT& K );}{The returned coefficients \ccc{A,B,C,D,K} have the
  property that width-plane \ccc{e1} is given by the equation
  $Ax+By+Cz+D=0$ and width-plane \ccc{e2} by $Ax+By+Cz+K=0$.}  
\ccGlueEnd

\ccGlueBegin 
\ccMemberFunction{ Vector get_build_direction ( );}{ returns a
  direction $\mathbf{d}_{opt}$ such that the width-planes \ccc{e1} and
  \ccc{e2} are perpendicular to $\mathbf{d}_{opt}$. The width of the
  point set is minimal in this direction.}
\ccMemberFunction{ void get_all_build_directions ( vector<Direction>&
  AllDir );}{ All the build directions are stored in the vector
  \ccc{AllDir}. It might happen that a certain body has several
  different build directions, but it is also possible to have only one
  build direction.}  
\ccGlueEnd

\ccMemberFunction{ int get_number_of_optimal_solutions( );}{ returns
  the number of optimal solutions, \ie the number of optimal build
  directions.}

\end{ccIndexMemberFunctions}

% -----------------------------------------------------------------------------
\ccSeeAlso

\ccRefIdfierPage{CGAL::Width_default_traits_3<R>}\\[1ex]
\ccRefConceptPage{Width_traits_3}

% -----------------------------------------------------------------------------
\ccImplementation
\ccIndexImplementation

Since the width of the point set $\mathcal{S}$ and the width of the convex
hull of $\mathcal{S}$ ($\mathit{conv(\mathcal{S})}$) is the same, the
algorithm uses the 3D convex hull algorithm \cgal\ provides.  But that
means you need to have \leda\ installed to compute the width of a set of
points. On the other hand if you want to compute the width of a given
\emph{convex} polyhedron, then \leda\ is not required. In either cases
datastructures from the \stl\ have been used.

The width-algorithm is not incremental and therefore inserting and
erasing points cause not an `automatic' update of the width. Instead
you have to rerun the width-algorithm even if the point set is
extended by only one new point.

\begin{ccAdvanced}

\paragraph{Large Numbers.}

Because there is no need for dividing values during the algorithm, the
numbers can get really huge (all the computations are made using a lot
of multiplications).  Therefore it is strongly recommended to use a
number type that can handle numbers of arbitrary length (\eg
\ccc{leda_integer} in combination with the homogeneous representation
of the points). But these large numbers have a disadvantage:
Operations on them are slower as greater the number gets. Therefore it
is possible to shorten the numbers by using the compiler flag
\textsc{-Dsimplify}. For using this option it is required that
the underlying number type provides the `modulo' operation. 

\paragraph{Information Output during the Computations.} 

If during the algorithm the program should output some information
(\eg during the debugging phase) you can turn on the output
information by giving the compiler flag \textsc{debug}. In the file
\texttt{width\_assertions.h} you can turn on/off the output of some
functions and additional informations by changing the defined values
from 0 (no output) to 1 (output available). But then it is required
that the $<\!<$-operator has to been overloaded for \ccc{Point},
\ccc{Plane}, \ccc{Vector} and \ccc{RT}.

\end{ccAdvanced}

% -----------------------------------------------------------------------------
\ccExample
\ccIndexSubitem[C]{Width_3}{example}

\ccIncludeExampleCode{Optimisation_ref/test_width_simplex.C}

% -----------------------------------------------------------------------------

\ccRestoreThreeColumns

\end{ccRefClass}

% ===== EOF ===================================================================
