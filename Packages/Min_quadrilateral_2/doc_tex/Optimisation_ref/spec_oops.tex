%% ==============================================================
%% Specification: Minimum enclosing quadrilaterals
%% --------------------------------------------------------------
%% file  : spec_oops.awi
%% author: Michael Hoffmann
%% $Id$
%% ==============================================================

\cgalColumnLayout

\begin{ccRefFunction}{min_rectangle_2}
  \ccIndexSubitem[t]{smallest enclosing}{rectangle}
  \ccIndexSubitem[t]{rectangle}{smallest enclosing}
  
  \ccDefinition The function computes a minimum area enclosing
  rectangle $R(P)$ (not necessarily axis-parallel) of a given convex
  point set $P$.  Note that $R(P)$ is not unique in general. The
  restriction to convex sets is not really a restriction, since any
  enclosing rectangle -- as a convex set -- contains the convex hull
  of $P$.

  \ccInclude{CGAL/min_quadrilateral_2.h}

  \def\ccLongParamLayout{\ccTrue} 
  
  \ccGlobalFunction{
    template < class ForwardIterator, class OutputIterator, class Traits >
    OutputIterator
    min_rectangle_2(
    ForwardIterator points_begin,
    ForwardIterator points_end,
    OutputIterator o,
    Traits& t = Default_traits);}
  
  computes a minimum area enclosing rectangle of the point set
  described by [\ccc{points_begin}, \ccc{points_end}), writes its
  vertices (counterclockwise) to \ccc{o} and returns the past-the-end
  iterator
  of this sequence.\\
  If the input range is empty, \ccc{o} remains unchanged.\\
  If the input range consists of one element only,
  exactly this point is written to \ccc{o}.
  
  \ccPrecond The points denoted by the range [\ccc{points_begin},
  \ccc{points_end}) form the boundary of a convex polygon $P$ in
  counterclockwise orientation.
  
  The geometric types and operations to be used for the computation
  are specified by the traits class parameter \ccc{t}. The parameter
  can be omitted if \ccc{ForwardIterator} refers to a point type from
  the 2D-Kernel. In this case, a default traits class
  (\ccc{Min_quadrilateral_default_traits_2<R>}) is used.
  
  \ccRequire
  \begin{enumerate}
  \item If \ccc{Traits} is specified, it is a model for
    \ccc{MinQuadrilateralTraits_2} and the value type \ccc{VT} of
    \ccc{ForwardIterator} is \ccc{Traits::Point_2}.  Otherwise
    \ccc{VT} is \ccc{CGAL::Point_2<R>} for some representation class
    \ccc{R}.
  \item \ccc{OutputIterator} accepts \ccc{VT} as value type.
  \end{enumerate}
  
  \ccSeeAlso
  \ccRefIdfierPage{CGAL::min_parallelogram_2}\\
  \ccRefIdfierPage{CGAL::min_strip_2}\\
  \ccRefConceptPage{MinQuadrilateralTraits_2}\\
  \ccRefIdfierPage{CGAL::Min_quadrilateral_default_traits_2<R>}
  
  \ccImplementation We use a rotating caliper
  \ccIndexMainItem[t]{rotating caliper} algorithm \cite{t-sgprc-83}
  with worst case running time linear in the number of input points.
  
  \ccExample The following code generates a random convex polygon
  \ccc{P} with 20 vertices and computes the minimum enclosing
  rectangle of \ccc{P}.

  \ccIncludeVerbatim{Optimisation_ref/minimum_enclosing_rectangle_2_example_noheader.C}

\end{ccRefFunction}
    
\begin{ccRefFunction}{min_parallelogram_2}
  \ccIndexSubitem[t]{smallest enclosing}{parallelogram}
  \ccIndexSubitem[t]{parallelogram}{smallest enclosing}
  
  \ccDefinition The function computes a minimum area enclosing
  parallelogram $A(P)$ of a given convex point set $P$.  Note that
  $A(P)$ is not unique in general.  The restriction to convex sets is
  not really a restriction, since any enclosing parallelogram -- as a
  convex set -- contains the convex hull of $P$.

  \ccInclude{CGAL/min_quadrilateral_2.h}

  \def\ccLongParamLayout{\ccTrue} 
  
  \ccGlobalFunction{
    template < class ForwardIterator, class OutputIterator, class Traits >
    OutputIterator
    min_parallelogram_2(
    ForwardIterator points_begin,
    ForwardIterator points_end,
    OutputIterator o,
    Traits& t = Default_traits);}
  
  computes a minimum area enclosing parallelogram of the point set
  described by [\ccc{points_begin}, \ccc{points_end}), writes its
  vertices (counterclockwise) to \ccc{o} and returns the past-the-end
  iterator of this sequence.
  If the input range is empty, \ccc{o} remains unchanged.\\
  If the input range consists of one element only,
  exactly this point is written to \ccc{o}.
  
  \ccPrecond The points denoted by the range [\ccc{points_begin},
  \ccc{points_end}) form the boundary of a convex polygon $P$ in
  counterclockwise orientation.

  The geometric types and operations to be used for the computation
  are specified by the traits class parameter \ccc{t}. The parameter
  can be omitted if \ccc{ForwardIterator} refers to a point type from
  the 2D-Kernel. In this case, a default traits class
  (\ccc{Min_quadrilateral_default_traits_2<R>}) is used.
  
  \ccRequire
  \begin{enumerate}
  \item If \ccc{Traits} is specified, it is a model for
    \ccc{MinQuadrilateralTraits_2} and the value type \ccc{VT} of
    \ccc{ForwardIterator} is \ccc{Traits::Point_2}.  Otherwise
    \ccc{VT} is \ccc{CGAL::Point_2<R>} for some representation class
    \ccc{R}.
  \item \ccc{OutputIterator} accepts \ccc{VT} as value type.
  \end{enumerate}
  
  \ccSeeAlso
  \ccRefIdfierPage{CGAL::min_rectangle_2}\\
  \ccRefIdfierPage{CGAL::min_strip_2}\\
  \ccRefConceptPage{MinQuadrilateralTraits_2}\\
  \ccRefIdfierPage{CGAL::Min_quadrilateral_default_traits_2<R>}
  
  \ccImplementation We use a rotating caliper
  \ccIndexMainItem[t]{rotating caliper} algorithm
  \cite{stvwe-mepa-95,v-fmep-90} with worst case running time linear
  in the number of input points.
  
  \ccExample The following code generates a random convex polygon
  \ccc{P} with 20 vertices and computes the minimum enclosing
  parallelogram of \ccc{P}.

  \ccIncludeVerbatim{Optimisation_ref/minimum_enclosing_parallelogram_2_example_noheader.C}

\end{ccRefFunction}

\begin{ccRefFunction}{min_strip_2}
  \ccIndexSubitem[t]{smallest enclosing}{strip}
  \ccIndexSubitem[t]{strip}{smallest enclosing}
  
  \ccDefinition The function computes a minimum width enclosing strip
  $S(P)$ of a given convex point set $P$. A strip is the closed region
  bounded by two parallel lines in the plane. Note that $S(P)$ is not
  unique in general.  The restriction to convex sets is not really a
  restriction, since any enclosing strip -- as a convex set --
  contains the convex hull of $P$.

  \ccInclude{CGAL/min_quadrilateral_2.h}

  \def\ccLongParamLayout{\ccTrue} 
  
  \ccGlobalFunction{
    template < class ForwardIterator, class OutputIterator, class Traits >
    OutputIterator
    min_strip_2(
    ForwardIterator points_begin,
    ForwardIterator points_end,
    OutputIterator o,
    Traits& t = Default_traits);}
  
  computes a minimum enclosing strip of the point set described by
  [\ccc{points_begin}, \ccc{points_end}), writes its bounding lines to
  \ccc{o} and returns the past-the-end iterator of this sequence.\\
  If the input range is empty or consists of one element only, \ccc{o}
  remains unchanged.
  
  \ccPrecond The points denoted by the range [\ccc{points_begin},
  \ccc{points_end}) form the boundary of a convex polygon $P$ in
  counterclockwise orientation.

  The geometric types and operations to be used for the computation
  are specified by the traits class parameter \ccc{t}. The parameter
  can be omitted if \ccc{ForwardIterator} refers to a point type from
  the 2D-Kernel. In this case, a default traits class
  (\ccc{Min_quadrilateral_default_traits_2<R>}) is used.
  
  \ccRequire
  \begin{enumerate}
  \item If \ccc{Traits} is specified, it is a model for
    \ccc{MinQuadrilateralTraits_2} and the value type \ccc{VT} of
    \ccc{ForwardIterator} is \ccc{Traits::Point_2}.  Otherwise
    \ccc{VT} is \ccc{CGAL::Point_2<R>} for some representation class
    \ccc{R}.
  \item \ccc{OutputIterator} accepts \ccc{VT} as value type.
  \end{enumerate}
  
  \ccSeeAlso
  \ccRefIdfierPage{CGAL::min_rectangle_2}\\
  \ccRefIdfierPage{CGAL::min_parallelogram_2}\\
  \ccRefConceptPage{MinQuadrilateralTraits_2}\\
  \ccRefIdfierPage{CGAL::Min_quadrilateral_default_traits_2<R>}
  
  \ccImplementation We use a rotating caliper
  \ccIndexMainItem[t]{rotating caliper} algorithm \cite{t-sgprc-83}
  with worst case running time linear in the number of input points.
    
  \ccExample The following code generates a random convex polygon
  \ccc{P} with 20 vertices and computes the minimum enclosing
  strip of \ccc{P}.

  \ccIncludeVerbatim{Optimisation_ref/minimum_enclosing_strip_2_example_noheader.C}

\end{ccRefFunction}
    
\begin{ccRefClass}{Min_quadrilateral_default_traits_2<R>}
  \ccCreationVariable{t}\ccTagFullDeclarations
  
  \ccDefinition The class \ccRefName\ is a traits class for the
  functions \ccc{min_rectangle_2}, \ccc{min_parallelogram_2} and
  \ccc{min_strip_2} using the two-dimensional \cgal\ kernel.
  
  \ccRequirements The template parameter \ccc{R} is a model for
  \ccc{Kernel}.
  
  \ccInclude{CGAL/Min_quadrilateral_traits_2.h}
  
  \ccIsModel
  \ccRefConceptPage{MinQuadrilateralTraits_2}
  
  \ccTypes 
  
  \ccTwo{Minq_traits::Rotate_direction_by_multiple_of_pi_22}{}
  
  \ccNestedType{Point_2}{typedef to \ccc{R::Point_2}}
  
  \ccNestedType{Direction_2}{typedef to \ccc{R::Direction_2}}
  
  \ccNestedType{Line_2}{typedef to \ccc{R::Line_2}}
  
  \ccNestedType{Rectangle_2}{internal type.}
  
  \ccNestedType{Parallelogram_2}{internal type.}
  
  \ccNestedType{Strip_2}{internal type.}
  
  \ccNestedType{Equal_2}{AdaptableBinaryFunction class\\\ccc{op}:
    \ccc{Point_2} $\times$ \ccc{Point_2} $\rightarrow$ \ccc{bool}.\\
    Returns true, iff the two points are equal.}
  
  \ccNestedType{Less_x_2}{AdaptableBinaryFunction class \\\ccc{op}:
    \ccc{Point_2} $\times$ \ccc{Point_2} $\rightarrow$ \ccc{bool}.\\
    \ccc{op(p,q)} returns true, iff the $x$-coordinate of \ccc{p} is
    smaller than the $x$-coordinate of \ccc{q}.}
  
  \ccNestedType{Less_y_2}{AdaptableBinaryFunction class \\\ccc{op}:
    \ccc{Point_2} $\times$ \ccc{Point_2} $\rightarrow$ \ccc{bool}.\\
    \ccc{op(p,q)} returns true, iff the $y$-coordinate of \ccc{p} is
    smaller than the $y$-coordinate of \ccc{q}.}
  
  \ccNestedType{Greater_x_2}{AdaptableBinaryFunction class
    \\\ccc{op}:
    \ccc{Point_2} $\times$ \ccc{Point_2} $\rightarrow$ \ccc{bool}.\\
    \ccc{op(p,q)} returns true, iff the $x$-coordinate of \ccc{p} is
    greater than the $x$-coordinate of \ccc{q}.}
  
  \ccNestedType{Greater_y_2}{AdaptableBinaryFunction class
    \\\ccc{op}:
    \ccc{Point_2} $\times$ \ccc{Point_2} $\rightarrow$ \ccc{bool}.\\
    \ccc{op(p,q)} returns true, iff the $y$-coordinate of \ccc{p} is
    greater than the $y$-coordinate of \ccc{q}.}
  
  \ccNestedType{Right_of_implicit_line_2}{Function class \\\ccc{op}:
    \ccc{Point_2} $\times$ \ccc{Point_2} $\times$ \ccc{Direction_2}
    $\rightarrow$ \ccc{bool}.\\ \ccc{op(p1,p2,d)} returns true, iff
    the \ccc{p1} is strictly to the right of the oriented line
    through \ccc{p2} with direction \ccc{d}.}
  
  \ccNestedType{Less_rotate_ccw_2}{AdaptableBinaryFunction class
    \\\ccc{op}: \ccc{Direction_2} $\times$ \ccc{Direction_2}
    $\rightarrow$ \ccc{bool}.\\ \ccc{op(d1,d2)} returns true, iff
    the slope of \ccc{d1} is less than the slope of \ccc{d2}.}
  
  \ccNestedType{Area_less_rectangle_2}{AdaptableBinaryFunction class
    \\\ccc{op}: \ccc{Rectangle_2} $\times$ \ccc{Rectangle_2}
    $\rightarrow$ \ccc{bool}.\\ \ccc{op(r1,r2)} returns true, iff
    the area of $r1$ is strictly less than the area of $r2$.}
  
  \ccNestedType{Area_less_parallelogram_2}{AdaptableBinaryFunction
    class \\\ccc{op}: \ccc{Parallelogram_2} $\times$
    \ccc{Parallelogram_2} $\rightarrow$ \ccc{bool}.\\ 
    \ccc{op(p1,p2)} returns true, iff the area of $p1$ is strictly
    less than the area of $p2$.}
  
  \ccNestedType{Width_less_strip_2}{AdaptableBinaryFunction class
    \\\ccc{op}: \ccc{Strip_2} $\times$ \ccc{Strip_2} $\rightarrow$
    \ccc{bool}.\\ \ccc{op(s1,s2)} returns true, iff the width of
    $s1$ is strictly less than the width of $s2$.}
  
  \ccNestedType{Rotate_direction_by_multiple_of_pi_2}{
    AdaptableBinaryFunction class \\\ccc{op}: \ccc{Direction_2}
    $\times$ \ccc{int} $\rightarrow$ \ccc{Direction_2}.\\ For a
    direction $d$ and $i,\, 0 \le i < 4$ \ccc{op(d,i)} returns the
    direction that results from clockwise rotating $d$ by $i \cdot
    \pi$.}
  
  \ccNestedType{Construct_direction_2}{AdaptableBinaryFunction class
    \\\ccc{op}: \ccc{Point_2} $\times$ \ccc{Point_2} $\rightarrow$
    \ccc{Direction_2}.\\ \ccc{op(p,q)} returns the direction of the
    vector from $p$ to $q$.}
  
  \ccNestedType{Construct_rectangle_2}{Function class \\\ccc{op}:
    \ccc{Point_2} $\times$ \ccc{Direction_2} $\times$ \ccc{Point_2}
    $\times$ \ccc{Point_2} $\times$ \ccc{Point_2} $\rightarrow$
    \ccc{Rectangle_2}.\\ If the points
    \ccc{p1},\,\ccc{p2},\,\ccc{p3},\,\ccc{p4} form the boundary of a
    convex polygon (oriented counterclockwise),
    \ccc{op(p1,d,p2,p3,p4)} returns the rectangle with one of the
    points on each side and one sides parallel to \ccc{d}.}
  
  \ccNestedType{Construct_parallelogram_2}{Function class
    \\\ccc{op}: \ccc{Point_2} $\times$ \ccc{Direction_2} $\times$
    \ccc{Point_2} $\times$ \ccc{Direction_2} $\times$ \ccc{Point_2}
    $\times$ \ccc{Point_2} $\rightarrow$ \ccc{Rectangle_2}.\\ If the
    points \ccc{p1},\,\ccc{p2},\,\ccc{p3},\,\ccc{p4} form the
    boundary of a convex polygon (oriented counterclockwise),
    \ccc{op(p1,d1,p2,d2,p3,p4)} returns the parallelogram with one
    of the points on each side and one side parallel to each of
    \ccc{d1} and \ccc{d2}.}
  
  \ccNestedType{Construct_strip_2}{Function class \\\ccc{op}:
    \ccc{Point_2} $\times$ \ccc{Direction_2} $\times$ \ccc{Point_2}
    $\rightarrow$ \ccc{Strip_2}.\\ \ccc{op(p1,d,p2)} returns the
    strip bounded by the lines through \ccc{p1} resp. \ccc{p2} with
    direction \ccc{d}.}
  
  \ccOperations
  
  \ccThree{OutputIterator}{OutputIterator}{}
  
  \ccMemberFunction{template < class OutputIterator > OutputIterator
    copy_rectangle_vertices_2(const Rectangle_2& r, OutputIterator
    o) const;}{copies the four vertices of \ccc{r} in
    counterclockwise order to \ccc{o}.}
  
  \ccMemberFunction{template < class OutputIterator > OutputIterator
    copy_parallelogram_vertices_2(const Parallelogram_2& p,
    OutputIterator o) const;}{copies the four vertices of \ccc{p} in
    counterclockwise order to \ccc{o}.}
  
  \ccMemberFunction{template < class OutputIterator > OutputIterator
    copy_strip_lines_2(const Strip_2& s, OutputIterator o)
    const;}{copies the two lines bounding \ccc{s} to \ccc{o}.}

  The following functions just return the corresponding function
  object.
  
  \ccThree{Rotate_direction_by_multiple_of_pi_2}{
    rotate_direction_by_multiple_of_pi_2_object();}{}
  
  \ccMemberFunction{Equal_2 equal_2_object() const;}{}
  \ccGlue\ccMemberFunction{Less_x_2 less_x_2_object() const;}{}
  \ccGlue\ccMemberFunction{Less_y_2 less_y_2_object() const;}{}
  \ccGlue\ccMemberFunction{Greater_x_2 greater_x_2_object() const;}{}
  \ccGlue\ccMemberFunction{Greater_y_2 greater_y_2_object() const;}{}
  
  \ccGlue\ccMemberFunction{Right_of_implicit_line_2
    right_of_implicit_line_2_object() const;}{}
  \ccGlue\ccMemberFunction{Less_rotate_ccw_2
    less_rotate_ccw_2_object() const;}{}
  \ccGlue\ccMemberFunction{Area_less_rectangle_2
    area_less_rectangle_2_object() const;}{}
  \ccGlue\ccMemberFunction{Area_less_parallelogram_2
    area_less_parallelogram_2_object() const;}{}
  \ccGlue\ccMemberFunction{Width_less_strip_2
    width_less_strip_2_object() const;}{}
  \ccGlue\ccMemberFunction{Construct_direction_2
    construct_direction_2_object() const;}{}
  \ccGlue\ccMemberFunction{Construct_rectangle_2
    construct_rectangle_2_object() const;}{}
  \ccGlue\ccMemberFunction{Construct_parallelogram_2
    construct_parallelogram_2_object() const;}{}
  \ccGlue\ccMemberFunction{Construct_strip_2
    construct_strip_2_object() const;}{}
  \ccGlue\ccMemberFunction{Rotate_direction_by_multiple_of_pi_2
    rotate_direction_by_multiple_of_pi_2_object() const;}{}
  
  \ccSeeAlso
  \ccRefIdfierPage{CGAL::min_rectangle_2}\\
  \ccRefIdfierPage{CGAL::min_parallelogram_2}\\
  \ccRefIdfierPage{CGAL::min_strip_2}
  
\end{ccRefClass}

\begin{ccRefConcept}{MinQuadrilateralTraits_2}
  \ccCreationVariable{t}\ccTagFullDeclarations
  
  \ccDefinition The concept \ccRefName\ defines types and operations
  needed to compute minimum enclosing quadrilaterals of a planar point
  set using the functions \ccc{min_rectangle_2},
  \ccc{min_parallelogram_2} and \ccc{min_strip_2}.
  
  \ccTypes 
  
  \ccTwo{Minq_traits::Rotate_direction_by_multiple_of_pi_22}{}

  \ccNestedType{Point_2}{type for representing points.}
  
  \ccNestedType{Direction_2}{type for representing directions.}
  
  \ccNestedType{Line_2}{type for representing lines.}
  
  \ccNestedType{Rectangle_2}{type for representing (not necessarily
    axis-parallel) rectangles.}
  
  \ccNestedType{Parallelogram_2}{type for representing
    parallelograms.}
  
  \ccNestedType{Strip_2}{type for representing strips, that is the
    closed region bounded by two parallel lines.}
  
  \ccNestedType{Equal_2}{AdaptableBinaryFunction class\\\ccc{op}:
    \ccc{Point_2} $\times$ \ccc{Point_2} $\rightarrow$ \ccc{bool}.\\
    Returns true, iff the two points are equal.}
  
  \ccNestedType{Less_x_2}{AdaptableBinaryFunction class \\\ccc{op}:
    \ccc{Point_2} $\times$ \ccc{Point_2} $\rightarrow$ \ccc{bool}.\\
    \ccc{op(p,q)} returns true, iff the $x$-coordinate of \ccc{p} is
    smaller than the $x$-coordinate of \ccc{q}.}
  
  \ccNestedType{Less_y_2}{AdaptableBinaryFunction class \\\ccc{op}:
    \ccc{Point_2} $\times$ \ccc{Point_2} $\rightarrow$ \ccc{bool}.\\
    \ccc{op(p,q)} returns true, iff the $y$-coordinate of \ccc{p} is
    smaller than the $y$-coordinate of \ccc{q}.}
  
  \ccNestedType{Greater_x_2}{AdaptableBinaryFunction class
    \\\ccc{op}:
    \ccc{Point_2} $\times$ \ccc{Point_2} $\rightarrow$ \ccc{bool}.\\
    \ccc{op(p,q)} returns true, iff the $x$-coordinate of \ccc{p} is
    greater than the $x$-coordinate of \ccc{q}.}
  
  \ccNestedType{Greater_y_2}{AdaptableBinaryFunction class
    \\\ccc{op}:
    \ccc{Point_2} $\times$ \ccc{Point_2} $\rightarrow$ \ccc{bool}.\\
    \ccc{op(p,q)} returns true, iff the $y$-coordinate of \ccc{p} is
    greater than the $y$-coordinate of \ccc{q}.}
  
  \ccNestedType{Right_of_implicit_line_2}{Function class \\\ccc{op}:
    \ccc{Point_2} $\times$ \ccc{Point_2} $\times$ \ccc{Direction_2}
    $\rightarrow$ \ccc{bool}.\\ \ccc{op(p1,p2,d)} returns true, iff
    the \ccc{p1} is strictly to the right of the oriented line
    through \ccc{p2} with direction \ccc{d}.}
  
  \ccNestedType{Less_rotate_ccw_2}{AdaptableBinaryFunction class
    \\\ccc{op}: \ccc{Direction_2} $\times$ \ccc{Direction_2}
    $\rightarrow$ \ccc{bool}.\\ \ccc{op(d1,d2)} returns true, iff
    the slope of \ccc{d1} is less than the slope of \ccc{d2}.}
  
  \ccNestedType{Area_less_rectangle_2}{AdaptableBinaryFunction class
    \\\ccc{op}: \ccc{Rectangle_2} $\times$ \ccc{Rectangle_2}
    $\rightarrow$ \ccc{bool}.\\ \ccc{op(r1,r2)} returns true, iff
    the area of $r1$ is strictly less than the area of $r2$.}
  
  \ccNestedType{Area_less_parallelogram_2}{AdaptableBinaryFunction
    class \\\ccc{op}: \ccc{Parallelogram_2} $\times$
    \ccc{Parallelogram_2} $\rightarrow$ \ccc{bool}.\\ 
    \ccc{op(p1,p2)} returns true, iff the area of $p1$ is strictly
    less than the area of $p2$.}
  
  \ccNestedType{Width_less_strip_2}{AdaptableBinaryFunction class
    \\\ccc{op}: \ccc{Strip_2} $\times$ \ccc{Strip_2} $\rightarrow$
    \ccc{bool}.\\ \ccc{op(s1,s2)} returns true, iff the width of
    $s1$ is strictly less than the width of $s2$.}
  
  \ccNestedType{Rotate_direction_by_multiple_of_pi_2}{
    AdaptableBinaryFunction class \\\ccc{op}: \ccc{Direction_2}
    $\times$ \ccc{int} $\rightarrow$ \ccc{Direction_2}.\\ For a
    direction $d$ and $i,\, 0 \le i < 4$ \ccc{op(d,i)} returns the
    direction that results from clockwise rotating $d$ by $i \cdot
    \pi$.}
  
  \ccNestedType{Construct_direction_2}{AdaptableBinaryFunction class
    \\\ccc{op}: \ccc{Point_2} $\times$ \ccc{Point_2} $\rightarrow$
    \ccc{Direction_2}.\\ \ccc{op(p,q)} returns the direction of the
    vector from $p$ to $q$.}
  
  \ccNestedType{Construct_rectangle_2}{Function class \\\ccc{op}:
    \ccc{Point_2} $\times$ \ccc{Direction_2} $\times$ \ccc{Point_2}
    $\times$ \ccc{Point_2} $\times$ \ccc{Point_2} $\rightarrow$
    \ccc{Rectangle_2}.\\ If the points
    \ccc{p1},\,\ccc{p2},\,\ccc{p3},\,\ccc{p4} form the boundary of a
    convex polygon (oriented counterclockwise),
    \ccc{op(p1,d,p2,p3,p4)} returns the rectangle with one of the
    points on each side and one sides parallel to \ccc{d}.}
  
  \ccNestedType{Construct_parallelogram_2}{Function class
    \\\ccc{op}: \ccc{Point_2} $\times$ \ccc{Direction_2} $\times$
    \ccc{Point_2} $\times$ \ccc{Direction_2} $\times$ \ccc{Point_2}
    $\times$ \ccc{Point_2} $\rightarrow$ \ccc{Rectangle_2}.\\ If the
    points \ccc{p1},\,\ccc{p2},\,\ccc{p3},\,\ccc{p4} form the
    boundary of a convex polygon (oriented counterclockwise),
    \ccc{op(p1,d1,p2,d2,p3,p4)} returns the parallelogram with one
    of the points on each side and one side parallel to each of
    \ccc{d1} and \ccc{d2}.}
  
  \ccNestedType{Construct_strip_2}{Function class \\\ccc{op}:
    \ccc{Point_2} $\times$ \ccc{Direction_2} $\times$ \ccc{Point_2}
    $\rightarrow$ \ccc{Strip_2}.\\ \ccc{op(p1,d,p2)} returns the
    strip bounded by the lines through \ccc{p1} resp. \ccc{p2} with
    direction \ccc{d}.}
  
  \ccOperations
  
  \ccThree{OutputIterator}{OutputIterator}{}
  
  \ccMemberFunction{template < class OutputIterator > OutputIterator
    copy_rectangle_vertices_2(const Rectangle_2& r, OutputIterator
    o) const;}{copies the four vertices of \ccc{r} in
    counterclockwise order to \ccc{o}.}
  
  \ccMemberFunction{template < class OutputIterator > OutputIterator
    copy_parallelogram_vertices_2(const Parallelogram_2& p,
    OutputIterator o) const;}{copies the four vertices of \ccc{p} in
    counterclockwise order to \ccc{o}.}
  
  \ccMemberFunction{template < class OutputIterator > OutputIterator
    copy_strip_lines_2(const Strip_2& s, OutputIterator o)
    const;}{copies the two lines bounding \ccc{s} to \ccc{o}.}

  The following functions just return the corresponding function
  object.
  
  \ccThree{Rotate_direction_by_multiple_of_pi_2}{
    rotate_direction_by_multiple_of_pi_2_object();}{}
  
  \ccMemberFunction{Equal_2 equal_2_object() const;}{}
  \ccGlue\ccMemberFunction{Less_x_2 less_x_2_object() const;}{}
  \ccGlue\ccMemberFunction{Less_y_2 less_y_2_object() const;}{}
  \ccGlue\ccMemberFunction{Greater_x_2 greater_x_2_object() const;}{}
  \ccGlue\ccMemberFunction{Greater_y_2 greater_y_2_object() const;}{}
  
  \ccGlue\ccMemberFunction{Right_of_implicit_line_2
    right_of_implicit_line_2_object() const;}{}
  \ccGlue\ccMemberFunction{Less_rotate_ccw_2
    less_rotate_ccw_2_object() const;}{}
  \ccGlue\ccMemberFunction{Area_less_rectangle_2
    area_less_rectangle_2_object() const;}{}
  \ccGlue\ccMemberFunction{Area_less_parallelogram_2
    area_less_parallelogram_2_object() const;}{}
  \ccGlue\ccMemberFunction{Width_less_strip_2
    width_less_strip_2_object() const;}{}
  \ccGlue\ccMemberFunction{Construct_direction_2
    construct_direction_2_object() const;}{}
  \ccGlue\ccMemberFunction{Construct_rectangle_2
    construct_rectangle_2_object() const;}{}
  \ccGlue\ccMemberFunction{Construct_parallelogram_2
    construct_parallelogram_2_object() const;}{}
  \ccGlue\ccMemberFunction{Construct_strip_2
    construct_strip_2_object() const;}{}
  \ccGlue\ccMemberFunction{Rotate_direction_by_multiple_of_pi_2
    rotate_direction_by_multiple_of_pi_2_object() const;}{}
  
  \ccHasModels
  \ccRefIdfierPage{CGAL::Min_quadrilateral_default_traits_2<R>}

  \ccSeeAlso
  \ccRefIdfierPage{CGAL::min_rectangle_2}\\
  \ccRefIdfierPage{CGAL::min_parallelogram_2}\\
  \ccRefIdfierPage{CGAL::min_strip_2}

\end{ccRefConcept}

%% --------------------------------------------------------------
%% EOF spec_oops.awi
%% --------------------------------------------------------------
