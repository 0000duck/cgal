% +------------------------------------------------------------------------+
% | CBP Reference Manual:  intro.tex
% +------------------------------------------------------------------------+
% | Polyhedron reference manual pages
% |
% | 14.05.1998   Lutz Kettner
\RCSdef{\halfedgeDSRefRev}{$Revision$}
\RCSdefDate{\halfedgeDSRefDate}{$Date$}
% +------------------------------------------------------------------------+

\clearpage
\chapter{Halfedge Data Structure}
%\section{Reference Pages for Halfedge Data Structure}
\label{chapterHalfedgeDSRef}
%% \ccChapterRelease{\halfedgeDSRefRev. \ \halfedgeDSRefDate}\\
%% \ccChapterAuthor{Lutz Kettner}

% +------------------------------------------------------------------------+
\section*{Release Note}

Beginning with \cgal\ R2.3, this package has a new design.  The old
design is still available for backwards compatibility and to support
older compiler, such as MSVC++6.0. However its use is deprecated and
the manual pages are not converted into this new manual
format. Instead, see its old documentation in the manual of 
deprecated packages. The two designs are not interchangeable.

\section*{Summary}

A halfedge data structure (abbreviated as \ccc{HalfedgeDS}, or
\ccc{HDS} for template parameters) is an edge-centered data structure
capable of maintaining incidence informations of vertices, edges and
faces, for example for planar maps or polyhedral surfaces. It is a
combinatorial data structure, geometric interpretation is added by
classes built on top of the halfedge data structure.  These classes
might be more convenient to use than the halfedge data structure
directly, since the halfedge data structure is meant as an
implementation layer.  See for example the \ccc{CGAL::Polyhedron_3}
class in Chapter~\ref{chapterPolyhedronRef}.

The data structure provided here is known as the
FE-structure~\cite{w-ebdss-85}, as
halfedges~\cite{m-ism-88,bfh-mgedm-95} or as the doubly connected edge
list (DCEL)~\cite{bkos-cgaa-97}, although the original reference for
the DCEL~\cite{mp-fitcp-78} describes a related but different data
structure. The halfedge data structure can also be seen as one of the
variants of the quad-edge data structure~\cite{gs-pmgsc-85}. In
general, the quad-edge data can represent non-orientable 2-manifolds,
but the variant here is restricted to orientable 2-manifolds only. An
overview and comparison of these different data structures together
with a thorough description of the design implemented here can be
found in~\cite{k-ugpdd-99}.

\subsection*{Concepts}

\ccRefConceptPage{HalfedgeDS<Traits,Items,Alloc>}\\ 
\ccRefConceptPage{HalfedgeDSItems}\\ 
\ccRefConceptPage{HalfedgeDSVertex}\\ 
\ccRefIdfierPage{HalfedgeDSHalfedge}\\ 
\ccRefIdfierPage{HalfedgeDSFace}

\subsection*{Classes}

\ccRefIdfierPage{CGAL::HalfedgeDS_default<Traits,HalfedgeDSItems,Alloc>}\\ 
\ccRefIdfierPage{CGAL::HalfedgeDS_list<Traits,HalfedgeDSItems,Alloc>}\\
\ccRefIdfierPage{CGAL::HalfedgeDS_vector<Traits,HalfedgeDSItems,Alloc>}

\ccRefIdfierPage{CGAL::HalfedgeDS_min_items}\\
\ccRefIdfierPage{CGAL::HalfedgeDS_items_2}

\ccRefIdfierPage{CGAL::HalfedgeDS_vertex_base<Refs>}\\ 
\ccRefIdfierPage{CGAL::HalfedgeDS_halfedge_base<Refs>}\\ 
\ccRefIdfierPage{CGAL::HalfedgeDS_face_base<Refs>}

\ccRefIdfierPage{CGAL::HalfedgeDS_vertex_min_base<Refs>}\\ 
\ccRefIdfierPage{CGAL::HalfedgeDS_halfedge_min_base<Refs>}\\ 
\ccRefIdfierPage{CGAL::HalfedgeDS_face_min_base<Refs>}

\ccRefIdfierPage{CGAL::HalfedgeDS_items_decorator<HDS>}\\
\ccRefIdfierPage{CGAL::HalfedgeDS_decorator<HDS>}\\
\ccRefIdfierPage{CGAL::HalfedgeDS_const_decorator<HDS>}

\lcHtml{\subsection*{Links to the Reference Sections}}

%% EOF %%
