% begin cgal manual page

\begin{ccRefClass}{Unique_hash_map<Key,Data,UniqueHashFunction>}
\ccCreationVariable{map}

\ccDefinition

An instance \ccVar\ of the parameterized data type \ccRefName\ is an
injective mapping from the set of keys of type \ccc{Key} to the set of
variables of type \ccc{Data}. New keys can be inserted at any time,
however keys cannot be individually deleted.

An object \ccc{hash} of the type \ccc{UniqueHashFunction} returns a
unique integer index \ccc{hash(key)} of type \ccc{std::size_t} for all
objects $key$ stored in \ccVar. The template parameter has as default
the \ccc{Handle_hash_function} that hashes all types of pointers, handles,
iterators, and circulators.

All variables are initialized to \ccc{default_data}, a value
of type \ccc{Data} specified in the definition of \ccVar.

\ccInclude{CGAL/Unique_hash_map.h}

\ccTypes
\ccTwo{Unique_hash_map<Key,Data,UniqueHashFunction>:: Hash_function}{}

\ccNestedType{Key}{the \ccc{Key} type.}
\ccGlue
\ccNestedType{Data}{the \ccc{Data} type.}
\ccGlue
\ccNestedType{Hash_function}{the unique hash function type.}

In compliance with \stl, the types \ccc{key_type}, \ccc{data_type}, and
\ccc{hasher} are defined as well.

\ccCreation
\ccThree{Hash_functi}{mained;}{}
\ccThreeToTwo

\ccConstructor{Unique_hash_map<Key,Data,UniqueHashFunction>( 
               const Data& default = Data(), 
               std::size_t table_size   = 1,
               const Hash_function& fct = Hash_function());}{
    creates an injective function \ccVar\ from \ccc{Key} to the set of unused
    variables of type \ccc{Data}, sets \ccc{default_data} to \ccc{default},
    passes the \ccc{table_size} as argument to the internal implementation,
    and initializes the hash function with \ccc{fct}.}

\ccConstructor{Unique_hash_map<Key,Data,UniqueHashFunction>( 
               Key first1, Key beyond1, Data first2,
               const Data& default = Data(), 
               std::size_t table_size   = 1,
               const Hash_function& fct = Hash_function());}{
    creates an injective function \ccVar\ from \ccc{Key} to the set of unused
    variables of type \ccc{Data}, sets \ccc{default_data} to \ccc{default},
    passes the \ccc{table_size} as argument to the internal implementation,
    initializes the hash function with \ccc{fct}, and inserts all keys 
    from the range \ccc{[first1,beyond1)}. The data variable for each
    inserted \ccc{key} is initialized with the corresponding value from
    the range \ccc{[first2, first2 + (beyond1-first1))}.
    \ccPrecond\ The increment operator must be defined for values
        of type \ccc{Key} and for values of type \ccc{Data}. \ccc{beyond1}
        must be reachable from \ccc{first1} using increments.}


\ccOperations
\ccThree{Hash_function}{map.is_defined( Key key);}{}

\ccMethod{Data default_value() const;}{the current \ccc{default_value}.}
\ccGlue
\ccMethod{Hash_function hash_function() const;}{the current hash function.}

\ccMethod{bool is_defined( const Key& key);}{returns true if $key$ is
    defined in \ccVar. Note that there can be keys defined that have not
    been inserted explicitly. Their variables are initialized to
    \ccc{default_value}.}

%%\ccMethod{element_iterator begin();}{
%%    returns the element iterator pointing to the first element stored.}
%%
%%\ccMethod{element_iterator end();}{
%%    returns the element iterator pointing beyond the last element stored.}

\ccMethod{void clear();}{
    resets \ccVar\ to the injective function \ccVar\ from \ccc{Key} to the
    set of unused variables of type \ccc{Data}. The \ccc{default_data}
    remains unchanged.}

\ccMethod{void clear(const Data& default);}{
    resets \ccVar\ to the injective function \ccVar\ from \ccc{Key} to the
    set of unused variables of type \ccc{Data} and sets \ccc{default_data} 
    to \ccc{default}.}

\ccTagFullDeclarations

\ccMethod{Data& operator[](const Key& key);}{
    returns a reference to the variable \ccVar\ccc{(key)}. If \ccc{key}
    has not been inserted into \ccVar\ before, \ccc{key} is inserted and
    initialized with \ccc{default_value}.}

\ccMethod{const Data& operator[](const Key& key) const;}{
    returns a const reference to the variable \ccVar\ccc{(key)}. If \ccc{key}
    has not been inserted into \ccVar\ before, a const reference to the
    \ccc{default_value} is returned. However, \ccc{key} is not inserted 
    into \ccVar.}

\ccTagDefaults

\ccMethod{Data insert( Key first1, Key beyond1, Data first2);}{
    inserts all keys from the range \ccc{[first1,beyond1)}. 
    The data variable for each inserted \ccc{key} is initilized with the
    corresponding value from the range \ccc{[first2, first2 +
    (beyond1-first1))}. Returns \ccc{first2 + (beyond1-first1)}.
    \ccPrecond\ The increment operator must be defined for values
        of type \ccc{Key} and for values of type \ccc{Data}. \ccc{beyond1}
        must be reachable from \ccc{first1} using increments.}

\vspace*{-5mm}
\ccSeeAlso

\ccRefConceptPage{UniqueHashFunction}\\
\ccRefIdfierPage{CGAL::Handle_hash_function}

\ccImplementation

\ccc{Unique_hash_map} is implemented via a chained hashing scheme. Access
operations \ccVar\ccc{[i]} take expected time $O(1)$. The \ccc{table_size}
parameter passed to chained hashing can be used to avoid unnecessary 
rehashing when set to the number of expected elements in the map.
The design is derived from the \stl\ \ccc{hash_map} and the \leda\ type
\ccc{map}. Its specialization on insertion only and unique hash values 
allow for a more time- and space-efficient implementation, see also
\cite[Chapter 5]{mn-l-99}. This implementation makes also use
of sentinels that lead to defined keys that have not been inserted.

\end{ccRefClass}




