\cleardoublepage
\chapter{Number Type Support}\label{Numbertype}

\cgal\ representation classes are parameterized by number types.  
Depending on the problem and the input data that have to be handled,
one has to make a trade-off between efficiency and accuracy in 
order to select an appropriate number type and representation class.

In homogeneous representation, two number types are involved,
although only one of them appears as a template parameter in
the homogeneous representation class.
This type, for sake of simplicity and readability called ring type is
used for the representation of homogeneous coordinates and all 
internal computations. 
If it is assured that the second operand divides the first one, these 
internal computations are basically division-free.
The ring type is a placeholder for an integer type (or an integral 
domain type) rather than for elements of arbitrary rings. 
The name should remind you that the division operation is not
really needed (with the exception mentioned above) for this number type.
Of course, also more general number types can be used as a ring type 
in a homogeneous representation class. In some computations, e.g.\ accessing 
Cartesian coordinates, divisions cannot be avoided. In these computations a 
second number type, the field type, is used. \cgal\ automatically generates 
this number type as a \ccStyle{Quotient}\ccTexHtml{, 
cf.\ Subsection~\ref{Quotient}}{}. For the Cartesian representation 
there is only one number type that is used for all calculations.

The representation classes provide access to the number types 
involved in the representation, although it is not expected that
such access is needed at this level, since low-level geometric 
operations are wrapped in geometric primitives provided by \cgal.
This access can be useful if appropriate primitives are missing.
In a homogeneous representation class \ccStyle{R} ring type and field
type can be accessed as \ccStyle{R::RT} and \ccStyle{R::FT}, respectively.
The number type used in Cartesian representation is considered as 
ring type and as field type depending on the context.
If can be accessed as \ccStyle{R::RT} and \ccStyle{R::FT}, according
to the use of number types used in the homogeneous counterpart.

\section{Required Functionality of Number Types\label{nt-requirements}}
Number types must fulfill certain requirements, such that they can
be successfully used in \cgal\ code.
This section describes those requirements.
We focus on the syntactical requirements. Of course, number types also
have evident semantic constraints. They should be meaningful in the
sense that they approximate the integers or the rationals 
or some other subfield of the real numbers.

The requirements are described as a class interface of a class \ccStyle{NT}, 
with constructors, methods and the like. 
This is only a matter of presentation.  
In fact \ccc{double} and \ccc{float} also fulfill the requirements.

\begin{ccClass} {NT}
\ccSetTwoColumns{}{\hspace*{8.5cm}}
\ccCreation
\ccCreationVariable{ntvar}

\ccConstructor{NT();}
	    {Declaration of a variable.}


\ccConstructor{NT(const NT &ntval);}
	    {Declaration and initialization.}


\ccConstructor{NT(i)}
	    {Declaration and initialization with a small integer
constant $i$, $0 \leq i \leq 127$. The neutral elements for addition
(zero) and multiplication (one) are needed quite often, but sometimes
other small constants are useful too. The value 127 was chosen such
that even signed 8 bit number types can fulfill this condition.  }

\ccSetThreeColumns{XXXXXX}{}{\hspace*{8.5cm}}
\ccOperations
\ccMethod{NT & operator=(const NT &ntval);}
        {Assignment. 
}


\ccFunction{bool is_valid(const NT &ntval);}
{Not all values of a number type need be valid. The routine
\ccStyle{is_valid} checks this. For example, an expression like
\ccStyle{nt(0)/nt(0)} can result in an invalid number. Routines often
have as a precondition that all numerical values are valid.}

\ccFunction{bool is_finite(const NT &ntval);}
{ When two large doubles are multiplied, the result may not fit in a
  \ccStyle{NT}. 
  Some number types (the standard \ccc{float} and \ccc{double} type
  when they conform to standards) have a way to represent a too big
  value as infinity.  \ccStyle{is_finite} implies
  \ccStyle{is_valid}.}

\ccMethod{bool operator==(const NT &n) const;}
       {}
\ccGlue
\ccMethod{bool operator!=(const NT &n) const;}
       {}
\ccGlue
\ccMethod{bool operator<(const NT &n) const;}
       {}
\ccGlue
\ccMethod{bool operator>(const NT &n) const;}
       {}
\ccGlue
\ccMethod{bool operator<=(const NT &n) const;}
       {}
\ccGlue
\ccMethod{bool operator>=(const NT &n) const;}
       {}
\ccGlue
\ccFunction{NT operator+(const NT &ntval1, const NT &ntval2);}
       {}
\ccGlue
\ccMethod{NT operator+=(const NT &n) const;}
       {}
\ccGlue
\ccFunction{NT operator-(const NT &ntval1, const NT &ntval2);}
       {}
\ccGlue
\ccMethod{NT operator-=(const NT &n) const;}
       {}
\ccGlue
\ccFunction{NT operator*(const NT &ntval1, const NT &ntval2);}
       {}
\ccGlue
\ccMethod{NT operator*=(const NT &n) const;}
       {}
\ccGlue
\ccFunction{NT operator-(const NT &ntval);}
       {}

\ccFunction{double to_double(const NT &ntval);}
         {gives the double value for a number type.
          This is usually an approximation for the real (stored) value.
          It can be used to send numbers to a renderer or to store them 
          in a file.}


\ccFunction{NT operator/(const NT &ntval1, const NT &ntval2);}
       {}
\ccGlue
\ccMethod{NT operator/=(const NT &n) const;}
       {%Division is not required for a ring type.
   Division by zero need not be defined. It may result in a runtime error, an
   invalid value, a valid value or anything else. This basically means that the
   library tests for zero whenever it does a division.}

%If I/O-stream operators are defined for number type \ccc{NT}, 
%the following function should be defined as well:
%
%\ccFunction{io_Operator io_tag(const NT &);}
%        {}


\section{Utility Routines}

The previous section listed all the required functionality.
For the user of a number type it is handy to have a larger set of
operations available.

\subsection{Utility Functions}
\ccSetThreeColumns{Comparison_result}{}{\hspace*{6cm}}

\ccInclude{CGAL/number_utils.h}

\ccFunction{NT min(const NT& ntval1, const NT& ntval2);}
{returns the smaller of the two values.}
\ccGlue
\ccFunction{NT max(const NT& ntval1, const NT& ntval2);}
{returns the larger of the two values.}
\ccGlue
\ccFunction{NT abs(const NT& ntval);}
{returns the absolute value.}
\ccGlue
\ccFunction{NT square(const NT& ntval);}
{returns the square of \ccc{ntval}.}
\ccGlue
\ccFunction{Sign sign(const NT& ntval);}
{returns the sign: \ccc{POSITIVE}, \ccc{ZERO}, or \ccc{NEGATIVE}.}
\ccGlue
\ccFunction{bool is_negative(const NT& ntval);}
{}
\ccGlue
\ccFunction{bool is_positive(const NT& ntval);}
{}
\ccGlue
\ccFunction{bool is_zero(const NT& ntval);}
{}
\ccGlue
\ccFunction{bool is_one(const NT& ntval);}
{}
\ccGlue
\ccFunction{Comparison_result compare(const NT &n1, const NT &n2);}
{returns \ccc{GREATER} iff $n1>n2$, \ccc{EQUAL} iff $n1=n2$, and \ccc{SMALLER} iff $n1<n2$.}

Those routines are implemented using the required operations from the
previous section. They are defined by means of templates, so you do not
have to supply all those operations when you write  a new number type.
But if you have a better implementation for any of them, you can provide a 
corresponding non-template function with the same name for your number types. 
According to the rules for function name overloading, the non-template function
gets preference over the template functions listed above. 

For the number types \ccc{bool}, \ccc{int}, and \ccc{double} there is
the random numbers generator \ccc{Random}.

%\ccInclude{CGAL/basic.h}

%\ccFunction{template <class NT> bool is_even( NT x);}
%    {returns $(i \;\&\; 1) == 0$.}

%\ccFunction{template <class NT> bool is_odd( NT x);}
%    {returns $(i \;\&\; 1) == 1$.}

\end{ccClass} 

\subsection{Utility Function Classes}
In addition to the utility routines listed above, 
there are function object class templates corresponding to these functions.
Note that the function object class corresponding to \ccc{sign} is
named \ccc{Sgn} in order to avoid a conflict with the type 
\ccc{Sign}.

\cgal\ provides the following function object classes:

\ccInclude{CGAL/number_utils_classes.h}

\ccc{class Min<NT>;}\\
\ccc{class Max<NT>;}\\
\ccc{class Abs<NT>;}\\
\ccc{class Sgn<NT>;}\\
\ccc{class Is_negative<NT>;}\\
\ccc{class Is_positive<NT>;}\\
\ccc{class Is_zero<NT>;}\\
\ccc{class Is_one<NT>;}\\
\ccc{class Compare<NT>;}

\section{Built-in Number Types}

The built-in number types \ccc{float} and \ccc{double} have the
required arithmetic and comparison operators. They lack some required
routines though which are automatically included by \cgal.
\ccTexHtml{\footnote{ The functions can be found in the header files 
\ccc{<CGAL/double.h>} and \ccc{<CGAL/float.h>}.}}{}

All built-in number types of \CC\ can represent a discrete (bounded)
subset of the rational numbers only.  We assume that the
floating-point arithmetic of your machine follows {\sc Ieee}
floating-point-standard.  Since the floating-point culture has much
more infrastructural support (hardware, language definition and
compiler) than exact computation, it is very efficient.
Like with all number types with finite precision representation
which are used as approximations to the infinite ranges of 
integers or real numbers, the built-in number types are inherently
potentially inexact.
Be aware of this if you decide to use the efficient built-in 
number types: you have to cope with numerical problems.  
For example, you can compute the intersection point of two lines and 
then check whether this point lies on the two lines. 
%With exact arithmetic, the check will always return \ccc{true}. 
With floating point arithmetic,
roundoff errors may cause the answer of the check to be \ccc{false}. 
With the built-in integer types overflow might occur.

%\newpage
\section{Number Types Provided by \cgal}

%% Currently, \cgal\ has no proper number types of its own.
%% 
%% In the future
%% \cgal\ will provide a number type {\tt CGAL\_Integral\_double} 
%% that stores integer values in variables of the type {\tt double}.  
%% This has three advantages. First, according to {\sc Ieee} 
%% standard the type {\tt
%% double} uses 53 bits for the mantissa of the number (one of them
%% implicitly by the normalization assumption), 1 bit for the sign, and
%% 11 bits for the exponent.  This allows to store integers between
%% $-2^{53}$ and $2^{53}$ which gives us a larger range than what you can
%% get with the built-in type {\tt int}, which uses 32 bit on most platforms.
%% 
%% Secondly, most computers have floating-point units for fast {\tt
%% double} arithmetic. The gain in speed carries through to integers
%% stored in double variables.  Some machines even do not have special
%% hardware for integers anymore.  The floating-point unit is used e.g.\
%% for integer multiplication anyhow.
%% 
%% Finally, if a result of an arithmetic operation exceeds the range
%% $[-2^{53},2^{53}]$, the 
%% {\tt CGAL\_Integral\_double}s overflow nicely, the
%% result approximates the correct value.  This is not true in the {\tt
%% int} arithmetic.  On the other hand computations with {\tt int} are
%% modulo $2^{32}$, which also has some nice effects.  If you have two
%% expressions over the integers involving $+,-,*$ only, that give the
%% same result, a test will report equality even if an overflow
%% occurs. However the test falsely reports equality if the results
%% differ by $2^{32}$.
%% 
%% The main advantage however is, that fast computation of the sign of
%% the determinant of a matrix with entries of type 
%% {\tt CGAL\_Integral\_double}
%% according to \cite{ABDPY} is used in the geometric predicates of
%% \cgal.
%% 
%% There will also be a number type {\tt CGAL\_Filtered\_integer}.  If this
%% number type is used (with homogeneous representation) floating-point
%% filters make exact computation in the geometric predicates of \cgal\
%% more efficient.


\begin{ccRefClass} {Quotient<NT>}
\label{Quotient}
%\subsection{Quotient}

\ccDefinition
An object of the class \ccStyle{Quotient<NT>} is an element of the 
field of quotients of the integral domain type \ccStyle{NT}.
If \ccStyle{NT} behaves like an integer, \ccStyle{Quotient<NT>}
behaves like a rational number. 
{\leda}'s class \ccStyle{rational} (see Section~\ref{leda-nt})
has been the basis for \ccStyle{Quotient<NT>}.
A \ccStyle{Quotient<NT>} \ccStyle{q} is represented as a pair of 
\ccStyle{NT}s, representing numerator and denominator.

\ccc{NT} must be at least model of concept \ccc{IntegralDomainWithoutDivision}.\\
\ccc{NT} must be a model of concept \ccc{RealEmbeddable}. \\


\ccInclude{CGAL/Quotient.h}

\ccIsModel

\ccc{Field}\\
\\
\ccc{RealEmbeddable}

\ccCreation
\ccCreationVariable{q}

\ccConstructor{Quotient();}
             {introduces an uninitialized variable \ccVar.}

\ccHidden \ccConstructor{Quotient(const Quotient<NT> &q);}
 	    {copy constructor.}
\ccGlue
\ccConstructor{template <class T> Quotient<NT>(const T& t);}
{introduces the quotient \ccStyle{t/1}. NT needs to have a constructor from T.}
\ccGlue
\ccConstructor{template <class T> Quotient<NT>(const Quotient<T>& t);}
{introduces the quotient \ccStyle{NT(t.numerator())/NT(t.denominator())}.
NT needs to have a constructor from T.}
\ccGlue
\ccConstructor{Quotient(const NT& n, const NT& d)}
            {introduces the quotient \ccStyle{n/d}.}


\ccOperations

%\ccSetTwoOfThreeColumns{5cm}{4cm}
%SetThreeColumns{std::ostream& }{}{\hspace*{8cm}}

There are two access functions, namely to the
numerator and the denominator of a quotient.
Note that these values are not uniquely defined. 
It is guaranteed that \ccStyle{q.numerator()} and 
\ccStyle{q.denominator()} return values \ccStyle{nt_num} and
\ccStyle{nt_den} such that \ccStyle{q = nt_num/nt_den}, only
if  \ccStyle{q.numerator()} and \ccStyle{q.denominator()} are called
consecutively wrt \ccStyle{q}, i.e.~\ccStyle{q} is not involved in 
any other operation between these calls.

\ccMethod{NT numerator() const;}
       {returns a numerator of \ccStyle{q}.}
\ccGlue
\ccMethod{NT denominator() const;}
       {returns a denominator of \ccStyle{q}.}

\ccHidden \ccMethod{Quotient<NT>& normalize();}
{}

The stream operations are available as well. 
They assume that corresponding stream operators for type \ccc{NT} exist.

\ccFunction{std::ostream& operator<<(std::ostream& out, const Quotient<NT>& q);}
       {writes \ccc{q} to ostream \ccc{out} in format ``{\tt n/d}'', where
       {\tt n}$==$\ccc{q.numerator()} and {\tt d}$==$\ccc{q.denominator()}.}

\ccFunction{std::istream& operator>>(std::istream& in, Quotient<NT>& q);}
       {reads \ccc{q} from istream \ccc{in}. Expected format is
        ``{\tt n/d}'', where {\tt n} and {\tt d} are of type \ccc{NT}.
        A single {\tt n} which is not followed by a {\tt /}\  is also
        accepted and interpreted as {\tt n/1}.}

The following functions are added to fulfill the \cgal\ requirements
on number types.

\ccFunction{double to_double(const Quotient<NT>& q);}
       {returns some double approximation to \ccStyle{q}.}
\ccGlue
\ccFunction{bool  is_valid(const Quotient<NT>& q);}
       {returns true, if numerator and denominator are valid.}
\ccGlue
\ccFunction{bool  is_finite(const Quotient<NT>& q);}
       {returns true, if numerator and denominator are finite.}
\ccGlue
\ccFunction{Quotient<NT>  sqrt(const Quotient<NT>& q);}
       {returns the square root of \ccc{q}.  This is supported if and only if
        \ccc{NT} supports the square root as well.}

\end{ccRefClass} 


% author : Olivier Devillers <Olivier.Devillers@sophia.inria.fr>

%\marginparwidth 1.5cm
%\def\ccTagChapterRelease{\ccTrue}
%\def\ccTagChapterAuthor{\ccTrue}
%
%\chapter{Fixed precision numbers} \label{I1_Chapter_Fixed_precision_nt}
%
%\ccChapterRelease{Revision: , Date: }
%
%\ccChapterAuthor{Olivier Devillers}
%
%\section{Introduction} 

\begin{ccClass}{Fixed_precision_nt}

\subsection{Fixed precision numbers}
\label{I1_Chapter_Fixed_precision_nt}

The class \ccStyle{Fixed_precision_nt} provides 24 bits numbers in 
fixed point representation.
Basically these numbers are integers in the range
$[-2^{24},2^{24}]$ with a multiplying factor $2^b$.
The multiplying factor $2^b$ has to be initialized by the user
before the construction of the first \ccStyle{Fixed_precision_nt}
and is common to all variables.

The interest of such a number type is that geometric predicates
can be overloaded to get exact and very efficient predicates.
The drawback is that any \ccStyle{Fixed_precision_nt} is rounded to 
the nearest multiple of $2^b$, which yields to a very poor arithmetic.
The idea is to not use the arithmetic on \ccStyle{Fixed_precision_nt}
but only the specialized predicates.

Note: you must call \ccc{CGAL::force_ieee_double_precision()} in order for the
\ccStyle{Fixed_precision_nt} to work properly on Intel platforms.  This
initializes the FPU to an IEEE compliant rounding mode which is not the
default.

\ccSetTwoColumns{}{\hspace*{8.5cm}}

\ccCreation
\ccCreationVariable{fvar}

\ccInclude{CGAL/Fixed_precision_nt.h}

\ccConstructor{Fixed_precision_nt();}
            {Declaration.}

\ccConstructor{Fixed_precision_nt(double d);}
            {Initialization of a variable. The variable is rounded to the
                nearest legal fixed number (i.e. a multiple of $2^b=$
                \ccStyle{Fixed_precision_nt::unit_value()}}


\ccConstructor{Fixed_precision_nt(const Fixed_precision_nt &fval);}
            {Declaration and initialization.}


\ccConstructor{Fixed_precision_nt(int i)}
            {Declaration and initialization with an integer.}

\ccSetThreeColumns{XXXXXX}{}{\hspace*{8.5cm}}
\ccOperations


\ccMethod{Fixed_precision_nt & operator=(const Fixed_precision_nt &fval);}
        {Assignment. 
}

\ccFunction{bool is_valid(const Fixed_precision_nt &fval);}
{In case of overflow or division by 0, numbers becomes invalid.
If the precision is changed by usage of 
\ccStyle{Fixed_precision_nt::init()}, already existing numbers may become invalid
if they are no longer multiple of $2^b$.}

\ccFunction{bool is_finite(const Fixed_precision_nt &fval);}
{ \ccClassName\ do not implement infinite numbers. 
  \ccStyle{is_finite} is identical to \ccStyle{is_valid}.}

The comparison operations $==$, $!=$, $<$, $>$, $<=$, and $>=$ are all
available.

The arithmetic operators $+$, $-$, $*$, $/$, $+=$, $-=$, $*=$ and $/=$
are all available. The result of the computation is rounded to the
nearest legal \ccClassName. Overflow is possible, and even probable in case
of multiplication or division. \ccClassName\ are designed to use
specialized predicates, not to use arithmetic.


\ccFunction{double to_double(const Fixed_precision_nt &fval);}
         {casts to \ccStyle{double}.}


%\section{Parametrization routines}
%\ccHeading{Parameterization routines}

\ccHeading{Precision initialization}
As mentioned before, the \ccClassName\ numbers takes their values in an
interval $[-2^{24+b},2^{24+b}]$ of multiples of $2^b$,
this number $b$ as to be defined before any
 use of \ccStyle{Fixed_precision_nt}.

\ccSetThreeColumns{Comparison_result}{}{\hspace*{6cm}}

\ccFunction{static bool init(float B);}
{$B$ is an upper bound on the data, $b$ is the smallest integer such that 
$|B|\leq 2^b$. The result of the function is false if initialization was
already done, in that case already existing \ccClassName\ may become invalid.}

\ccFunction{static float unit_value();}
{returns $2^b$.}

\ccFunction{static float upper_bound();}
{returns $2^{24+b}$.}

%\subsection{Perturbation scheme}
\ccHeading{Perturbation scheme}

\ccClassName\ implements perturbation scheme as described by
Alliez, Devillers and Snoeyink \cite{ads-rdppw-98}.
The perturbation mode can be activated or deactivated for different kinds
of perturbations. The default mode is no perturbation.

\ccFunction{static void perturb_incircle();}
{Activate. \ccStyle{side_of_oriented_circle} predicate
 of 4 cocircular points answers degenerate only if
the 4 points are colinear.}

\ccFunction{static void unperturb_incircle();}
{Deactivate}

\ccFunction{static bool is_perturbed_incircle();}
{returns current mode}

\ccFunction{static void perturb_insphere();}
{Activate. \ccStyle{side_of_oriented_sphere} predicate
 of 5 cospherical points answers degenerate only if
the 5 points are coplanar.}

\ccFunction{static void unperturb_insphere();}
{Deactivate}

\ccFunction{static bool is_perturbed_insphere();}
{returns current mode}

%\section{Geometric predicates}
\ccHeading{Geometric predicates}

Through overloading mechanisms, functions such that
\ccStyle{orientation} for 
\ccStyle{Point_2<Cartesian< Fixed_precision_nt> >}
will correctly call the function below.

\ccSetThreeColumns{Oriented_side}{}{\hspace*{6cm}}

%\subsection{Two dimensional}

\ccFunction{Orientation orientationC2(
  Fixed_precision_nt x0, Fixed_precision_nt y0,
  Fixed_precision_nt x1, Fixed_precision_nt y1,
  Fixed_precision_nt x2, Fixed_precision_nt y2);}{}
\ccGlue
\ccFunction{Oriented_side side_of_oriented_circleC2 (
  Fixed_precision_nt x0, Fixed_precision_nt y0,
  Fixed_precision_nt x1, Fixed_precision_nt y1,
  Fixed_precision_nt x2, Fixed_precision_nt y2,
  Fixed_precision_nt x3, Fixed_precision_nt y3);}{Perturbation mode can be activated.}

%\subsection{Three dimensional}

\ccFunction{Orientation orientationC3(  
  Fixed_precision_nt x0, Fixed_precision_nt y0, Fixed_precision_nt z0,
  Fixed_precision_nt x1, Fixed_precision_nt y1, Fixed_precision_nt z1,
  Fixed_precision_nt x2, Fixed_precision_nt y2, Fixed_precision_nt z2,
  Fixed_precision_nt x3, Fixed_precision_nt y3, Fixed_precision_nt z3);}{}
%
%\ccSetThreeColumns{x}{}{\hspace*{3.5cm}}
\ccGlue
\ccFunction{Oriented_side side_of_oriented_sphereC3 (
  Fixed_precision_nt x0, Fixed_precision_nt y0, Fixed_precision_nt z0,
  Fixed_precision_nt x1, Fixed_precision_nt y1, Fixed_precision_nt z1,
  Fixed_precision_nt x2, Fixed_precision_nt y2, Fixed_precision_nt z2,
  Fixed_precision_nt x3, Fixed_precision_nt y3, Fixed_precision_nt z3,
  Fixed_precision_nt x4, Fixed_precision_nt y4, Fixed_precision_nt z4);}
{Perturbation mode can be activated.}

\end{ccClass} 


% Interval Arithmetic documentation.
%   $Revision$
%   $Date$

\begin{ccClass} {Interval_nt}
\ccSubsection{Interval Arithmetic}
\label{interval}
% \ccChapterAuthor{Sylvain Pion}

This section describes briefly what interval arithmetic is, its implementation
in {\cgal}, and its possible use by geometric programs.
The main reason for having interval arithmetic in {\cgal} is its integration
into the filtered robust and fast predicates scheme, but we also provide a
number type so that you can use it separately if you find any use for it
(such as interval analysis, or to represent data with tolerance).

% This control is
% provided via the comparison operators.  When there is an unsafe comparison
% between intervals, either a warning or an exception (see below) is raised.
% Otherwise, it means that if you do the same computation with doubles, your
% program will have the correct output (with the exact same input data).

The purpose of interval arithmetic is to provide an efficient way to bound
the roundoff errors made by floating point computations.
You can choose the behaviour of your program depending on these errors; that
is what is done for the filtered robust predicates (see Section~\ref{filter}).
You can find more theoretical information on this topic in
\cite{bbp-iayea-98}.
% \cite{bbp-iayed-98scg}. % worked here, but changed by Remco.

\ccDefinition
Interval arithmetic is a large concept and we will only consider here a 
simple arithmetic based on intervals whose bounds are {\it double}s.
So each variable is an interval representing any value inside the interval.
All arithmetic operations (+, -, $*$, $/$, $\sqrt{}$, \ccc{square()},
\ccc{min()}, \ccc{max()} and \ccc{abs()}) on intervals preserve the inclusion.
This property can be expressed by the following formula ($x$ and $y$ are
reals, $X$ and $Y$ are intervals, $\mathcal{OP}$ is an arithmetic operation):

$$
\forall\ x \in X, \forall\ y \in Y, (x\ \mathcal{OP}\ y)
\in (X\ \mathcal{OP}\ Y)
$$

For example, if the final result of a sequence of arithmetic operations is
an interval that does not contain zero, then you can safely determine its sign.

\ccInclude{CGAL/Interval_arithmetic.h}

\ccCreation

\ccConstructor{Interval_nt(double d);}
{introduces the interval [\ccc{d};\ccc{d}].}

\ccConstructor{Interval_nt(double i, double s);}
{introduces the interval [\ccc{i};\ccc{s}].}

\ccSetThreeColumns{Interval_nt}{}{\hspace*{8.5cm}}
\ccFunction{Interval_nt convert_to<Interval_nt>(NT);}
{Several such functions provide a cast from the following numerical types
to an \ccc{Interval_nt} containing the value.  The following types are
supported: \ccc{leda_real}, \ccc{leda_rational}, \ccc{leda_integer},
\ccc{leda_bigfloat}, \ccc{Gmpz}, \ccc{Fixed_precision_nt}, all built-in
types that fit exactly in a \ccc{double} (which excludes \ccc{long double}
and 64 bits integers, which must be treated separately), and all
\ccc{Quotient<RT>} where RT is a type listed above.
The user can add such functions for his own number types, see the files
\ccc{CGAL/Interval_arithmetic/IA_*.h} for examples.}


\ccOperations
\ccSetThreeColumns{Interval_nt}{}{\hspace*{8cm}}

All functions required by a class to be considered as a {\cgal} number type
(see~\ref{Numbertype}) are present, as well as the utility functions,
sometimes with a particular semantic which is described below.  There are also
a few additional functions.

\ccCreationVariable{I}

\ccMethod{Interval_nt operator/(Interval_nt J);} {returns
[$-\infty$;$+\infty$] when the denominator contains 0.}

\ccFunction{Interval_nt sqrt(Interval_nt I);} {returns
[0;$\sqrt{upper\_bound(I)}$] when only the lower bound is negative (expectable
case with roundoff errors), and is unspecified when the upper bound also is
negative (unexpected case).}

\ccFunction{double to_double(Interval_nt I);} {returns the
middle of the interval, as a double approximation of the interval.}

\ccMethod{double inf();} {returns the lower bound of the interval.}
\ccMethod{double sup();} {returns the upper bound of the interval.}
\ccMethod{bool is_point();} {returns whether both bounds are equal.}
\ccMethod{bool is_same(Interval_nt J);} {returns whether both intervals have
the same bounds.}
\ccMethod{bool overlap(Interval_nt J);} {returns whether both intervals have a
non empty intersection.}

The two following operators can be used for interval analysis, they are not
directly useful for \ccc{Interval_nt} as a number type.

\ccMethod{Interval_nt operator||(Interval_nt J);} {returns the smallest
interval containing the two intervals.}
\ccMethod{Interval_nt operator&&(Interval_nt J);} {returns the biggest
interval contained in the two intervals.  The result is unspecified if
the two intervals don't intersect.}


The comparison operators ($<$, $>$, $<=$, $>=$, $==$, $!=$, \ccc{sign()}
and \ccc{compare()}) have the following semantic: it is the intuitive
one when for all couples of values in both intervals, the comparison
is identical (case of non-overlapping intervals).  This can be expressed
by the following formula ($x$ and $y$ are reals, $X$ and $Y$ are
intervals, $\mathcal{OP}$ is a comparison operator):

$$
\left(\forall x \in X, \forall y \in Y, (x\ \mathcal{OP}\ y) = true\right)
\Rightarrow (X\ \mathcal{OP}\ Y) = true
$$
and
$$
\left(\forall x \in X, \forall y \in Y, (x\ \mathcal{OP}\ y) = false\right)
\Rightarrow (X\ \mathcal{OP}\ Y) =false
$$

Otherwise, the comparison is not safe, and we first increment the counter
\ccc{Interval_nt_advanced::number_of_failures}, and then throw the exception
\ccc{Interval_nt_advanced::unsafe_comparison}.

\ccImplementation

\ccc{Interval_nt} derives from \ccc{Interval_nt_advanced}, 
see Section~\ref{interval-adv}.  Nearly all operations on
\ccc{Interval_nt} need the rounding mode to be set to 'round to nearest'
(which is the default),
and leave it in this state if they have to modify it internally.

\end{ccClass}

\begin{ccAdvanced}
\samepage
\begin{ccClass} {Interval_nt_advanced}
\ccSubsection {Advanced Interval Arithmetic}
\label{interval-adv}

\ccc{Interval_nt} derives from \ccc{Interval_nt_advanced}.
The advanced class allows you to make faster computations, but you need to set
the rounding mode of the FPU to 'round to infinity' (see below for how to do
that) before doing any computation with this number type, and each function
(arithmetic operators and conversion functions)
leaves the rounding mode in this state if it needs to modify it internally.
See below for details.

% The other noticeable difference is that it throws the exception
% \ccc{Interval_nt_advanced::unsafe_comparison} when you make unsafe
% comparisons.  This mechanism is used internally by the filtered predicate
% scheme described in the section~/ref{filter}.

The function \ccc{convert_to<Interval_nt>()} is replaced by the
function \ccc{convert_to<Interval_nt_advanced>()} and has the same
requirements concerning the rounding mode as other functions.

\ccInclude{CGAL/Interval_arithmetic.h}

% FPU rounding modes~:

We provide the following functions to change the rounding mode:

\ccEnum{enum FPU_CW_t {FPU_cw_near, FPU_cw_zero, FPU_cw_up, FPU_cw_down};}
       {The different selectors for the rounding mode (respectively to the
       nearest, to zero, to plus infinity, and to minus infinity).}
\ccFunction{void FPU_set_cw (FPU_CW_t R);}
       {To set the rounding mode to \ccc{R}.}
\ccFunction{FPU_CW_t FPU_get_cw (void);}
       {To get the current rounding mode.}
\ccFunction{FPU_CW_t FPU_get_and_set_cw (FPU_CW_t R);}
       {To set the rounding mode to \ccc{R} and get the old one. }

\ccExample

The correct way to protect an area of code that uses operations on the class
\ccc{Interval_nt_advanced} is the following:

\begin{verbatim}
  FPU_CW_t backup = FPU_get_cw();
  FPU_set_cw(FPU_cw_up);
  ... // The code to be protected.
  FPU_set_cw(backup);
\end{verbatim}

\ccImplementation

The basic idea is to use the directed rounding modes specified by the 
{\it IEEE 754} standard, which are implemented by almost all processors 
nowadays.
It states that you have the possibility, concerning the basic floating point
operations ($+,-,*,/,\sqrt{}$) to specify the rounding mode of each operation
instead of using the default, which is set to 'round to the nearest'.
This feature allows us to compute easily on \ccc{intervals}.  For example, to
add the two intervals [a.i;a.s] and [b.i;b.s], compute $c.i=a.i+b.i$ rounded
towards minus infinity, and $c.s=a.s+b.s$ rounded towards plus infinity, and
the result is the interval [c.i;c.s].  This method can be extended easily to
the other operations.

The problem is that we have to change the rounding mode very often, and the
functions of the C library doing this operation are slow and not portable.
That's why assembly versions are used as often as possible.
Another trick is to store the opposite of the lower bound, instead of the
lower bound itself, which allows us to never change the rounding mode inside
simple operations.  Therefore, all basic operations, which are in the class 
\ccc{Interval_nt_advanced} assume that the rounding mode is set to 
'round to infinity', and everything works with this correctly set.  
The class \ccc{Interval_nt} takes care of this, but is a bit slower.

So, if the user needs the speed of \ccc{Interval_nt_advanced}, he must
take care of setting the rounding mode to 'round to infinity' before each
operations on this number type.  And if other operations might be affected by
this, he must take care to reset it to 'round to the nearest' before them.

% Note also that NaNs are not handled, so be careful with that
% (especially if you `divide by zero').

Platform support:\\
This part of {\cgal} must be explicitly ported to each non yet supported
platform.  For Intel, Sparc and Mips, only the GNU compilers use assembly
code.  The MipsPro compiler uses slower C code instead, and does not
support the construction \ccc{convert_to<>()}, but a workaround
\ccc{convert_from_to(target, source)}.

\end{ccClass}

\end{ccAdvanced}


\begin{ccClass} {Filtered_exact<CT, ET, Cache_t = No_Filter_Cache>}
\ccSubsection{Robust filtered predicates}
\label{filter}

The class \ccc{Filtered_exact<CT,ET>} is a wrapper type for the number
type \ccc{CT}, with the difference
that all predicates are specialized such that they are guaranteed to be exact.
Speed is achieved via a filtering scheme using interval arithmetic (see
Section~\ref{interval-adv}).  Here are the necessary requirements:

\begin{itemize}
\item \ccc{CT} is the construction and storage type.  The only data member of
      the class \ccc{Filtered_exact<CT,ET>} is the \ccc{value} of type
      \ccc{CT}.
      All arithmetic operations called
      outside the predicates will be executed with this number type.  You can
      disable this feature by compiling with the flag
      \ccc{-DCGAL_DENY_INEXACT_OPERATIONS_ON_FILTER} (it allows the programmer
      to spot the inexact operations that should be incorporated in the
      predicates).
      The arithmetic operations called from the predicates are always computed
      exactly.
\item The \ccc{ET} type must be able to compute exactly the operations involved
      in the predicates called.
\item A \ccc{convert_to<Interval_nt_advanced>(CT)} function must
      be provided, that returns an interval containing the value of the
      argument of type \ccc{CT}, see Section~\ref{interval}.
\item A \ccc{convert_to<ET>(CT)} function must also be provided, that
      returns a number of type \ccc{ET} representing exactly the argument of
      type \ccc{CT}.  It's a conversion function that is used for the exact
      computation, when the filter fails.
      This conversion has to be done exactly to ensure robustness.
\end{itemize}

The following member functions are used to access the numerical value for the
different number types:

\ccMethod{CT value();} {returns the wrapped value.}
\ccMethod{ET exact();} {returns the converted value to \ccc{ET}.}
\ccMethod{Interval_nt_advanced interval();} {returns the converted value
                                             to \ccc{Interval_nt_advanced}.}

For efficiency reasons, the conversion of the \ccc{value} to an
\ccc{Interval_nt_advanced} may be cached using \ccc{Filter_Cache} as optionnal
third template parameter of the class.  This is only wishful if \ccc{CT} is a
complicated type for which the conversion to an interval takes a somewhat long
time, and is asked for the same variable several times.  You might want to
experiment with this feature, depending on your needs.

\ccExample

You might use at the beginning of your program a \ccc{typedef} as follows:

\begin{verbatim}
    #include<CGAL/Arithmetic_filter.h>
    #include<CGAL/leda_real.h>
    #include<CGAL/double.h>
    typedef Filtered_exact<double, leda_real> NT;
\end{verbatim}

Or if you are sure that the predicates involved do not use divisions nor
square roots:

\begin{verbatim}
    #include<CGAL/Arithmetic_filter.h>
    #include<CGAL/Gmpz.h>
    #include<CGAL/int.h>
    typedef Filtered_exact<int, Gmpz> NT;
\end{verbatim}

And if you know that the double variables contain integer values, you can
use:

\begin{verbatim}
    #include<CGAL/Arithmetic_filter.h>
    #include<CGAL/Gmpz.h>
    #include<CGAL/double.h>
    typedef Filtered_exact<double, Gmpz> NT;
\end{verbatim}

As a general rule, we advise the use of
\ccc{Filtered_exact<double, leda_real>}.

If you need exact computations also outside the predicates, you can use the
following:

\begin{verbatim}
    #include<CGAL/Arithmetic_filter.h>
    #include<CGAL/lera_real.h>
    typedef Filtered_exact<leda_real, leda_real, Filter_Cache> NT;
\end{verbatim}


\ccImplementation

The template definition of the low level predicates of {\cgal} are overloaded
for the type \ccc{Filtered_exact<CT,ET>}.  It is a partial specialisation,
which implies that this is not supported by the compilers that do not support
this C++ feature.  A workaround is in place that allows to use
one such filtered number type at once, by explicitely specializing the
predicates for it.  To do so, just define the macros \ccc{CGAL_IA_CT} and
\ccc{CGAL_IA_ET} to the corresponding number types before any inclusion of
{\cgal} files, as well as \ccc{CGAL_IA_CACHE}.  For example~:

\begin{verbatim} 
#define CGAL_IA_CT double
#define CGAL_IA_ET leda_real
#define CGAL_IA_CACHE No_Filter_Cache
\end{verbatim}

For each predicate file, the overloaded code is generated automatically by a
\ccc{PERL} script
(\ccc{examples/Interval_arithmetic/filtered_predicate_converter}) that you can
use for your own predicates (see \ccc{examples/Interval_arithmetic/README}).
This script parses the template declaration of the functions and generates the
overloaded code the following way:
\begin{itemize}
\item convert the entries to intervals using
    \ccc{convert_to<Interval_nt_advanced>(CT)},
    using the \ccc{interval()} member function,
\item call the original template function with the type
    \ccc{Interval_nt_advanced},
\item if no exception is thrown, return the value,
\item if an exception is thrown (the filter failed), convert the original
    entries using \ccc{convert_to<ET>(CT)}, using the \ccc{exact()} member
    function,
\item and call the original template function with the type \ccc{ET}.
\end{itemize}

\ccExample

The low level template predicates of {\cgal} are in files named
\ccc{CGAL/predicates_on_ftC2.h} (resp. ftC3), the script is used
to produce the files \ccc{CGAL/Arithmetic_filter/predicates_on_ftC2.h}
(resp. ftC3).

For the moment, only the cartesian predicates of the kernel are supported,
as well as the power tests used by the regular triangulation.

\end{ccClass}


\section{Number Types Provided by {\sc Cln}}
\label{CLN}

% \ccChapterAuthor{Sylvain Pion}

\cgal\ defines the functions needed to use the number types provided by
{\sc Cln}, the Class Library for Numbers~\cite{cln}.
To use {\sc Cln} with \cgal, just install \cgal\ with {\sc Cln} support,
and include the corresponding {\sc Cln} file but with the prefix
\ccc{CGAL/CLN/}.

\ccInclude{CGAL/CLN/cl_integer.h}

The class \ccc{cl_I}~: exact multiprecision integers.

\ccInclude{CGAL/CLN/cl_rational.h}

The class \ccc{cl_RA}~: exact multiprecision rationals.

And similarly for the 10 ordinary number type classes provided by {\sc Cln}~:
\ccc{cl_SF, cl_FF, cl_DF, cl_LF, cl_I, cl_RA, cl_F, cl_R, cl_N, cl_number}.

For doing IOs, you need to include, for example~:

\ccInclude{cl_integer_io.h}

See the {\sc Cln} documentation for additionnal details.


\section{Number Types Provided by \leda}
\label{leda-nt}

\leda\ provides number types that can be used for exact computation 
with both Cartesian and homogeneous representations.  If you are using
homogeneous representation with the built-in integer types
\ccc{short}, \ccc{int}, and \ccc{long} as ring type, exactness of
computations can be guaranteed only if your input data come from a
sufficiently small integral range and the depth of the computations is
sufficiently small.  \leda\ provides the number type \ccc{leda_integer} for
integers of arbitrary length. (Of course the length is
somehow bounded by the resources of your computer.)  It can be used as
ring type in homogeneous representation and leads to exact
computation as long as all intermediate results are rational.  For the
same kind of problems Cartesian representation with number type
\ccc{leda_rational} leads to exact computation as well.
The number type \ccc{leda_bigfloat} in \leda\ is a variable precision
floating-point type. Rounding mode and precision (i.e.\ mantissa length) of
\ccc{leda_bigfloat} can be set. 

The most sophisticated number type in \leda\ is the number type called
\ccc{leda_real}. Like in Pascal, where the name \ccc{real} is used for
floating-point numbers, the name \ccc{leda_real} does not describe the
number type precisely, but intentionally.  
\ccc{leda_reals} are a subset of real algebraic
numbers.  Any integer is \ccc{leda_real} and \ccc{leda_real}s are closed under
the operations $+,-,*,/$ and $k$-th root computation. 
%\ccTexHtml{$\sqrt[k]{\ }$}{k-th root computation}. 
\ccc{leda_real}s guarantee that
all comparisons between expressions involving \ccc{leda_real}s produce the
exact result.

In the include files \ccc{<CGAL/leda_integer.h>}, \ccc{<CGAL/leda_rational.h>}, 
\ccc{<CGAL/leda_bigfloat.h>}, and \ccc{<CGAL/leda_real.h>}, 
the \leda\ types \ccc{leda_integer}, \ccc{leda_rational},
\ccc{leda_bigfloat}, 
and \ccc{leda_real} are made conform to the requirements presented in
Section \ref{nt-requirements}. 
Also, in these files the \leda\ number types are included.
For more details on the number types of \leda\ we refer to the \leda\ 
manual~\cite{mnsu-lum}.

\begin{ccRefClass} {Gmpz}
\label{Gmpz}

\ccDefinition

An object of the class \ccStyle{Gmpz} is an arbitrary precision integer 
based on the {\sc Gnu} Multiple Precision Arithmetic Library. 

\ccInclude{CGAL/Gmpz.h}

\ccIsModel
\ccc{EuclideanRing}\\
\ccc{RealEmbeddable}

\ccCreation
\ccCreationVariable{z}

\ccTwo{xxxxxxxxxxxxxxxxxxxxx}{}

\ccConstructor{Gmpz();}
             {creates an uninitialized multiple precision integer \ccVar.}\ccGlue

\ccHidden \ccConstructor{Gmpz(const Gmpz& q);}
            {copy constructor.}

\ccConstructor{Gmpz(int i)}
            {creates a multiple-precision integer initialized with
             \ccStyle{i}.}\ccGlue

\ccConstructor{Gmpz(double d)}
            {creates a multiple-precision integer initialized with
             the integral part of \ccStyle{d}.}\ccGlue

\ccOperations

\ccThree{xxxxxxxxxxxxxx}{xxxxxxxxxxxxxxxxxxxxxxx}{}

\ccMethod{Gmpz & operator++();}{prefix increment}\ccGlue
\ccMethod{Gmpz   operator++(int);}{postfix increment}\ccGlue
\ccMethod{Gmpz & operator--();}{prefix decrement}\ccGlue
\ccMethod{Gmpz   operator--(int);}{postfix decrement}

\ccFunction{Gmpz operator>>(const Gmpz& a, long i);}{Rightshift by i, where $i>=0$.}\ccGlue
\ccFunction{Gmpz operator<<(const Gmpz& a, long i);}{Leftshift by i, where $i>=0$.}\ccGlue
\ccMethod{Gmpz & operator>>=(const long& i);}{Rightshift by i, where $i>=0$.}\ccGlue
\ccMethod{Gmpz & operator<<=(const long& i);}{Leftshift by i, where $i>=0$.}


\ccFunction{Gmpz operator&(const Gmpz& a, const Gmpz& b);}{Bitwise AND.}\ccGlue
\ccFunction{Gmpz operator|(const Gmpz& a, const Gmpz& b);}{Bitwise IOR.}\ccGlue
\ccFunction{Gmpz operator^(const Gmpz& a, const Gmpz& b);}{Bitwise XOR.}\ccGlue
\ccMethod{Gmpz & operator&=(const Gmpz& b);}{Bitwise AND.}\ccGlue
\ccMethod{Gmpz & operator|=(const Gmpz& b);}{Bitwise IOR.}\ccGlue
\ccMethod{Gmpz & operator^=(const Gmpz& b);}{Bitwise XOR.}


\ccFunction{std::ostream& operator<<(std::ostream& out, const Gmpz& z);}
{writes \ccc{z} to the ostream \ccc{out}.}\ccGlue

\ccFunction{std::istream& operator>>(std::istream& in, Gmpz& z);}
{reads an integer from \ccc{in}, then converts it to a \ccc{Gmpz}.}

\ccImplementation
\ccc{Gmpz}s are reference counted.

\end{ccRefClass} 
\ccDefGlobalScope{CGAL::}



\section{User-supplied Number Types}

You can also use your own number type with the \cgal\ representation
classes, e.g.\  the {\sc BigNum} package \cite{svh-bpepa-89}.
Depending on the arithmetic operations carried out by the algorithms
that you are going to use the number types must fulfill the
requirements from Section \ref{nt-requirements}. 
