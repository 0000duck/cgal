% $Revision$
% $Date$
% author : Sylvain Pion

\begin{ccRefClass} {Number_type_checker<NT1,NT2,Comparator>}

\ccDefinition
\ccc{Number_type_checker} is a number type whose instances store two numbers
of types \ccc{NT1} and \ccc{NT2}.  It forwards all arithmetic operations to
them, and calls the binary predicate \ccc{Comparator} to check the equality of
the instances after each modification, as well as for each comparison.

This is a debugging tool which is useful when dealing with number types.

\ccParameters

\ccc{NT1} and \ccc{NT2} must be models of some number type concept.
\ccc{Comparator} has to be a model of a binary predicate taking \ccc{NT1}
as first argument, and \ccc{NT2} as second.  The \ccc{Comparator} parameter
has a default value which is a functor calling \ccc{operator==} between
the two arguments.


\ccInclude{CGAL/Number_type_checker.h}

\ccIsModel
SqrtFieldNumberType

\ccCreation
\ccCreationVariable{c}

\ccConstructor{Number_type_checker();}
{introduces an uninitialized variable \ccVar.}
\ccGlue
\ccConstructor{Number_type_checker(int i)}
{introduces the integral value i.}
\ccGlue
\ccConstructor{Number_type_checker(double d)}
{introduces the floating point value d.}
\ccGlue
\ccConstructor{Number_type_checker(const NT1 &n1, const NT2 &n2)}
{introduces a variable storing the pair \ccc{n1, n2}.}

\ccOperations

Some operations have a particular behavior documented here.

\ccMethod{const NT1 & n1() const;}
{returns a const reference to the object of type \ccc{NT1}.}
\ccGlue
\ccMethod{const NT2 & n2() const;}
{returns a const reference to the object of type \ccc{NT2}.}
\ccGlue
\ccMethod{NT1 & n1();}
{returns a reference to the object of type \ccc{NT1}.}
\ccGlue
\ccMethod{NT2 & n2();}
{returns a reference to the object of type \ccc{NT2}.}
\ccGlue
\ccMethod{bool is_valid() const;}
{calls the \ccc{Comparator} binary predicate on the two stored objects
 and returns its result.}

\ccFunction{double to_double(const Number_type_checker& c);}
{returns \ccc{to_double(c.n1())}.  No particular check is made here.}

\ccFunction{std::pair<>to_interval(const Number_type_checker& c);}
{returns \ccc{to_interval(c.n1())}.  No particular check is made here.}

\ccFunction{std::ostream& operator<<(std::ostream& out, const Number_type_checker& c);}
{writes \ccc{c.n1()} to the ostream \ccc{out}.}

\ccFunction{std::istream& operator>>(std::istream& in, Number_type_checker& c);}
{reads an \ccc{NT1} from \ccc{in}, then converts it to an \ccc{NT2},
 so a conversion from \ccc{NT1} to \ccc{NT2} is required here.}

\end{ccRefClass} 
