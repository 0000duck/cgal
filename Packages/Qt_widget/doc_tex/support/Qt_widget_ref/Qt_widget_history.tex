% +------------------------------------------------------------------------+
% | CGAL Reference Manual: Reference manual for Qt_widget.tex
% +------------------------------------------------------------------------+
% |
% | 02.04.2003  Radu Ursu
% | 
% | \RCSdef{\qtwidgetRev}{$Revision$}
% | \RCSdefDate{\qtwidgetDate}{$Date$}
% +------------------------------------------------------------------------+

% +-----------------------------------------------------+
\begin{ccRefClass}{Qt_widget_history}

\ccDefinition

This class provides basic functionality to manipulate intervals of
\ccc{Qt_widget} class. \ccc{Qt_widget} knows about the current visible
area that is specified through an interval. The default one is (-1, 1, -1,
1). This class is mostly used by the \ccc{Qt_widget_standard_toolbar}
to provide the backward and the forward navigation through the different
intervals.

\ccInclude{CGAL/IO/Qt_widget_history.h}
\ccGlue

\ccInheritsFrom
\ccc{QObject}

\ccCreation
\ccCreationVariable{history}
\ccSetTwoColumns{Qt_widget_history}{}
\ccConstructor{Qt_widget_history(Qt_widget* parent, const char *name
= 0 );}{\ccc{parent} is the parent widget that has to be a
\ccc{Qt_widget} and \ccc{name} is the name of the history. This will
create a history object that can deal with the \ccc{Qt_widget}
visualisation intervals. Each time the visible area will change, it's
interval will be stored in the history list.}


\ccHeading{public slots:}

\ccMethod{void save();}{This saves the widget visible area in the 
history list.}
\ccMethod{void backward();}{This goes back in the list of saved intervals.}
\ccMethod{void forward();}{This goes forward in the list of saved
intervals.}
\ccMethod{void clear();}{This will clear the entire history list.}



\end{ccRefClass}

% +-----------------------------------------------------+
% EOF
