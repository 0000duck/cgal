% +------------------------------------------------------------------------+
% | CGAL Reference Manual: Reference manual for Qt_widget.tex
% +------------------------------------------------------------------------+
% |
% | 03.01.2001	Radu Ursu
% | 
% | \RCSdef{\qtwidgetRev}{$Revision$}
% | \RCSdefDate{\qtwidgetDate}{$Date$}
% +------------------------------------------------------------------------+

% +-----------------------------------------------------+
\begin{ccRefClass}{Qt_widget_get_circle<T>}

\ccDefinition
An object of type \ccRefName\ creates a \cgal\ circle in this way: one 
left click on the mouse will be the center of the circle and the
second point will be considered at the coordinates where the left
mouse button is pressed for the second time, the distance
between those 2 representing the radius of the new circle.

\ccInclude{CGAL/IO/Qt_widget_get_circle.h}

\ccParameters

The full template declaration of \ccc{Qt_widget_get_circle} states one parameter:

\begin{tabbing}
\ccc{template <} \=\ccc{class T >}\\
	\ccc{class Qt_widget_get_circle;}
\end{tabbing}

If T is one of the \cgal\ kernels you don't need additional types. If
not, the parameter T has to provide this types:

\ccTypes
\ccTypedef{typedef T::Point_2 Point_2;}{This should be a Point type}
\ccTypedef{typedef T::Circle_2 Circle_2;}{This should be a Circle type}
\ccTypedef{typedef T::FT FT;}{This should be a Field type}

\ccInheritsFrom
\ccc{Qt_widget_layer}

\ccGlue

\ccCreation
\ccCreationVariable{getcircle}
\ccSetTwoColumns{Qt_widget_get_circle}{}

\ccConstructor{Qt_widget_get_circle();}{The default constructor.}

\end{ccRefClass}

% +-----------------------------------------------------+
% EOF








