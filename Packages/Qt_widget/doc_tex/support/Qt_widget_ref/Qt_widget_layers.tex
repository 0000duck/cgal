% +------------------------------------------------------------------------+
% | CGAL Reference Manual: Reference manual for Qt_widget.tex
% +------------------------------------------------------------------------+
% |
% | 03.01.2001  Radu Ursu
% | 
% | \RCSdef{\qtwidgetRev}{$Revision$}
% | \RCSdefDate{\qtwidgetDate}{$Date$}
% +------------------------------------------------------------------------+

% +-----------------------------------------------------+
\begin{ccRefClass}{Qt_widget_layer}

\ccDefinition
The \ccc{Qt\_widget\_layer} serves as a base class for layers. Layers
are classes that draw on the drawing area of a \ccc{Qt_widget} and
receive events from it. You can attach and detach layers on the
\ccc{Qt\_widget}. All the attached layers will receive the events if
they are active. The same for drawing, all the active layers that are
attached will be drawn.

\ccInclude{CGAL/IO/Qt_widget_layer.h}
\ccGlue

\ccCreation
\ccCreationVariable{layer}
\ccSetTwoColumns{Qt_widget_layer}{}

\ccConstructor{Qt_widget_layer();}{The default constructor.}
\ccSetThreeColumns{const_iterator}{container.begin() const;}{}

\ccMethod{virtual void mousePressEvent(QMouseEvent *);}{}
\ccGlue
\ccMethod{virtual void mouseReleaseEvent(QMouseEvent *);}{}
\ccGlue
\ccMethod{virtual void wheelEvent(QMouseEvent *);}{}
\ccGlue
\ccMethod{virtual void mouseDoubleClickEvent(QMouseEvent *);}{}
\ccGlue
\ccMethod{virtual void mouseMoveEvent(QMouseEvent *);}{}
\ccGlue
\ccMethod{virtual void keyPressEvent(QKeyEvent *);}{}
\ccGlue
\ccMethod{virtual void keyReleaseEvent(QMouseEvent *);}{}
\ccGlue
\ccMethod{virtual void enterEvent(QEvent *);}{}
\ccGlue
\ccMethod{virtual void leaveEvent(QEvent *);}{}
\ccGlue
\ccMethod{virtual bool event(QEvent *);}{These
virtual functions can be overloaded in the derived class. They are
called by the \ccc{Qt\_widget} to which the layer is attached.}

\ccMethod{bool is_active();}{Returns \ccc{true} if this layer is active.}

\ccHeading{Public Slots}

\ccMethod{virtual void draw();}{This virtual member function
 must be overloaded. This is where the drawing code goes.}
\ccMethod{void stateChanged(int s);}{This slot is provided to change
the layer's state from activated to deactivated and reverse if it is
triggered. The layer is activated if \ccc{s} is 2, or it is
deactivated if \ccc{s} is 0. These values match with the signal
\ccc{stateChanged(int)} in the \ccc{QButton} widget.}
\ccMethod{bool activate();}{Activate and return \ccc{true} if it was not active.}
\ccMethod{bool deactivate();}{Deactivate and return \ccc{true} if it was active.}

\ccHeading{Protected}
\ccVariable{Qt_widget *widget;}{The widget a layer is attached to or 0 
otherwise.}
\ccMethod{virtual void activating();}{You should overload this
function if you want to have initializing code for your layer. This
function is called every time the layer is activated.}
\ccMethod{virtual void deactivating();}{You should overload this
function if you want to write clean up code for your
layer. This function is called every time the layer is deactivated.}

\ccHeading{Signals}
\ccMethod{void activated(Qt_widget_layer *l);}{This signal is emitted
every time this layer is activated.}
\ccMethod{void deactivated(Qt_widget_layer *l);}{This signal is emitted
every time this layer is deactivated.}



\ccExample
The following example of a layer draws the points of a triangulation in green.

\begin{ccExampleCode}
#include <CGAL/IO/Qt_widget_layer.h>

namespace CGAL {

template <class T>
class Qt_layer_show_points : public Qt_widget_layer {
public:
  typedef typename T::Point           Point;
  typedef typename T::Segment         Segment;
  typedef typename T::Vertex          Vertex;
  typedef typename T::Vertex_iterator	Vertex_iterator;

  Qt_layer_show_points(T &t) : tr(t){};

  void draw()
  {  
    Vertex_iterator it = tr.vertices_begin(), 
                beyond = tr.vertices_end();
    *widget << CGAL::GREEN << CGAL::PointSize (3) 
            << CGAL::PointStyle (CGAL::DISC);    
    while(it != beyond) {      
      *widget << (*it).point();
      ++it;
    }
  };
private:
  T &tr;
  
};//end class 

} // namespace CGAL
\end{ccExampleCode}

\end{ccRefClass}
% +-----------------------------------------------------+
% EOF








