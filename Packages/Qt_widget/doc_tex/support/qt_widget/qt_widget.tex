% +------------------------------------------------------------------------+
% | CGAL Reference Manual:  qt_widget.tex
% +------------------------------------------------------------------------+
% | Using Qt_widget to visualize CGAL Objects
% | 
% |
% |
% | 12.12.2001	Radu Ursu
% | 
\RCSdef{\qtwidgetRev}{$Revision$}
\RCSdefDate{\qtwidgetDate}{$Date$}
% +------------------------------------------------------------------------+

\newcommand{\qt}{{\sc QT}}	%QT abbreviation

\gdef\lciIfHtmlClassLinks{\lcFalse}
\gdef\lciIfHtmlRefLinks{\lcFalse}
\gdef\lciIfHtmlLinks{\lcFalse}

\chapter{Qt\_widget}
\label{chapterQtwidget}

\ccChapterRelease{\qtwidgetRev. \ \qtwidgetDate}\\
\ccChapterAuthor{Radu Ursu}

\qt\ is a GUI development toolkit\footnote{http://www.trolltech.com} that 
allows to develop cross-platform applications. In this chapter we describe a 
widget that allows to visualize CGAL objects.

% +-----------------------------------------------------+
\section{Introduction}
\label{sectionIntroduction}

The QWidget class is the base class of all user interface objects. The widget
 is the atom of the user interface: It receives mouse, keyboard and other 
events from the window system, and paints a representation of itself on the 
screen. Every widget is rectangular, and they are sorted in a Z-order. A 
widget is clipped by its parent and by the widgets in front of it.

Qt\_widget is a class derived from QWidget in order to receive all the events
from the window system.

Qt\_widget use as a backbuffer for drawing a pixmap
defined inside the class, that is an object of type QPixmap. The QPixmap class
is an off-screen pixel-based paint device. One common use of the QPixmap class
is to enable smooth updating of widgets.
The QPainter class paints on paint devices. There is an object of type 
QPainter defined in Qt\_widget that uses as a paint device the QPixmap object.

Qt\_widget provides a scale system. You can use scales for your objects as 
well as you can draw the objects with one scale and look at the objects with 
different scales. Also there is a possibility to tell the widget that you
want that the visible area should be mapped to a certain interval, and the 
widget adjust the scales according to that.

The Qt\_widget provides output operators for all \cgal\ objects. There are 
operators defined for outputing : points, segments, rays, circles, triangles, 
polygons, and all type of triangulations. Also some operators are defined to 
set Qt\_widget's properties, like BackgroundColor, FillColor, LineWidth, 
PointSize.

For handling input we adopted the event driven design of Qt. Now there is a 
concept of tool, and also some tools are already implemented, for input points,
 segments, line, polygons, simple polygons. Also you can use this concept to 
define some specialised tools that let you edit your already defined \cgal\ objects.

On top of Qt\_widget it is defined a concept of View 

% +-----------------------------------------------------+
% EOF
