% +------------------------------------------------------------------------+
% | CGAL Reference Manual:  qt_widget.tex
% +------------------------------------------------------------------------+
% | Using Qt_widget to visualize CGAL Objects
% | 
% |
% |
% | 12.12.2001	Radu Ursu
% | 
\RCSdef{\qtwidgetRev}{$Revision$}
\RCSdefDate{\qtwidgetDate}{$Date$}
% +------------------------------------------------------------------------+

\newcommand{\qt}{{\sc QT}}	%QT abbreviation

\gdef\lciIfHtmlClassLinks{\lcFalse}
\gdef\lciIfHtmlRefLinks{\lcFalse}
\gdef\lciIfHtmlLinks{\lcFalse}

\chapter{Qt\_widget}
\label{chapterQtwidget}

\ccChapterRelease{\qtwidgetRev. \ \qtwidgetDate}\\
\ccChapterAuthor{Radu Ursu}

You can visualize \cgal\ objects using all kind of viewers, that knows how to
 paint geometry primitives like points, lines, circles, polygons. For this
kind of viewers, your code has to convert the primitives from \cgal\ to each 
viewer style of dealing with this primitives. There was a lot of work to 
create a generic way to use \cgal\ with \leda. As \leda\ became a commercial 
product, \cgal\ had to move to other product, to satisfy his academic 
comunity.

\qt\ GUI development toolkit\footnote{http://www.trolltech.com} is used to 
develop cross-platform applications that will run on Windows, Linux, Unix, 
Macintosh. As \qt\ is a powerfull product and has multiplatform support, 
\cgal\ invested some resurces to create a 2D--visualization tool able to 
deal directly with \cgal\ objects.

% +-----------------------------------------------------+
\section{Introduction}
\label{sectionIntroduction}

The QWidget class is the base class of all user interface objects. The widget
 is the atom of the user interface: It receives mouse, keyboard and other 
events from the window system, and paints a reprezentation of itself on the 
screen. Every widget is rectangular, and they are sorted in a Z-order. A 
widget is clipped by its parent and by the widgets in front of it.

Qt\_widget is a class derived from QWidget in order to receive all the events
from the window system.

Qt\_widget use as a backbuffer for drawing a pixmap
defined inside the class, that is an object of type QPixmap. The QPixmap class
is an off-screen pixel-based paint device. One common use of the QPixmap class
is to enable smooth updating of widgets.
The QPainter class paints on paint devices. There is an object of type 
QPainter defined in Qt\_widget that uses as a paint device the QPixmap object.

Qt\_widget provides a scale system. You can use scales for your objects as 
well as you can draw the objects with one scale and look at the objects with 
different scales. Also there is a possibility to tell the widget that you
want that the visible area should be mapped to a certain interval, and the 
widget adjust the scales according to that.

The main goal of Qt\_widget was to create the possibility to output \cgal\ 
objects. There are operators defined for outputing : points, segments, 
rays, circles, triangles, polygons, and all type of triangulations. Also some
operators are defined to set Qt\_widget's properties, like BackgroundColor,
FillColor, LineWidth, PointSize.

As the main goal of Qt\_widget was fulfiled, it was a challange to create some
generic mechanism for providing input. Now there is a concept of tool, and
 also some tools are already implemented, for input points, segments, line, 
polygons, simple polygons. Also you can use this concept to define some 
specialised tools that let you edit your already defined \cgal\ objects.

On top of Qt\_widget it is defined a concept of View 

% +-----------------------------------------------------+
% EOF
