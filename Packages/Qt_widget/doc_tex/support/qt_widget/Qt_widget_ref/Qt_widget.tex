% +------------------------------------------------------------------------+
% | CGAL Reference Manual: Reference manual for Qt_widget.tex
% +------------------------------------------------------------------------+
% |
% | 13.12.2001	Radu Ursu
% | 
% | \RCSdef{\qtwidgetRev}{$Revision$}
% | \RCSdefDate{\qtwidgetDate}{$Date$}
% +------------------------------------------------------------------------+

% +-----------------------------------------------------+
%\section{Qt\_widget Reference}
%\label{Qt\_widget Reference}

\begin{ccRefClass}{Qt_widget}

\ccDefinition

The widget Qt\_Widget is a class that bla bla bla.

\ccInclude{CGAL/Qt_widget.h}

\ccCreation
\ccCreationVariable{win}
\ccSetTwoColumns{Qt_widget}{}

\ccConstructor{Qt\_widget(QWidget *parent = 0, const char *name = 0);}{}

\ccOperations
\ccSetThreeColumns{const_iterator}{container.begin() const;}{}

%scales
\ccHeading{Scales}
\ccMethod{void set_window(double x_min, double x_max, double y_min, double 
y_max);}{Map the widget coordinates to the interval defined by the rectangle 
with given coordinates. This method should be called before \ccc{show()} from 
QWidget.}
\ccMethod{void set_x_scale(double xscale);}{Set the current X scale.}
\ccMethod{void set_y_scale(double yscale);}{Set the current Y scale.}
\ccMethod{void zoom_in(double ratio);}{Multiply X and Y scales by \ccc{ratio}.}
\ccMethod{void zoom_out(double ratio);}{Multiply X and Y scales by \ccc{ratio}.}
\ccMethod{void move_center(double distx, double disty);}{Move the center of the
 widget with \ccc{distx} on the X axe and with \ccc{disty} on Y axe.}
\ccMethod{void set_center(double x, double y);}{Set the center of the widget to
 O(x, y)}

\ccHeading{Painter}
%painter
\ccMethod{void clear();}{Clear the screen. The properties remain the same 
after calling this member function.}

%properties
\ccHeading{Properties}
\ccMethod{void setColor(Qcolor c);}{Set the current pen color of the widget to 
be c.}
\ccMethod{void setBackgroundColor(QColor c);}{Set the current background color 
to be c.}
\ccMethod{void setFillColor(QColor c);}{Set the current fill color of the 
widget to be c.}
\ccMethod{void setFilled(bool f);}{Set the status of the widget to true or false
 concerning filling the objects: polygons, circles, rectangles \ldots}
\ccMethod{void setLineWidth(uint i);}{Set the current line width of the widget.}
\ccMethod{void setPointSize(uint i);}{Set the current point size of the widget.}
\ccMethod{void setPointStyle(PointStyle s);}{Set the current point style of the
 widget to s. PointStyle is a enumeration declared in Qt\_widget.}
\ccMethod{void setRasterOp(RasterOp r);}{Set the current raster operation.}

%scenes
\ccHeading{Scenes}
\ccMethod{void add_scene(Qt_scene* s);}{Add the scene s in the list of scenes.
The last added will be on top of the screen.}
\ccMethod{void remove_scene(Qt_scene* s);}{Remove the scene s from the list.}
\ccMethod{void redraw();}{Redraw the scenes situated in the list. Before 
redraw the scenes clears the screen. Also calls \ccc{widget\_repainted()} for 
current tool.}

\ccAccessFunctions
%Properties
\ccHeading{Properties}
\ccMethod{QColor color() const;}{Returns the current pen color. The color 
returned is a Qt class.}
\ccMethod{QColor backgroundColor() const;}{Returns the current widget background
color. The color returned is a Qt class.}
\ccMethod{QColor fillColor() const;}{Returns the current color used for filling 
the objects. The color returned is a Qt class.}
\ccMethod{PointStyle pointStyle();}{Returns the current point style. PointStyle
 is a enumeration declared in Qt\_widget.}
\ccMethod{uint pointSize();}{Returns the current point size.}
\ccMethod{uint lineWidth();}{Returns the current line width.}
\ccMethod{RasterOp rasterOp();}{Return the current raster operation.}

%Painter
\ccHeading{Painter}
\ccMethod{QPixmap& get_pixmap();}{Returns the current pixmap.}
\ccMethod{QPainter& painter();}{Returns the current painter.}

%Coordinates
\ccHeading{Coordinates}
\ccMethod{double x_min() const;}{Returns the left X coordinate  of the widget.}
\ccMethod{double y_min() const;}{Returns the lower Y coordinate  of the widget.}
\ccMethod{double x_max() const;}{Returns the right X coordinate of the widget.}
\ccMethod{double y_max() const;}{Returns the upper Y coordinate of the widget.}

\end{ccRefClass}

% +-----------------------------------------------------+
% EOF







