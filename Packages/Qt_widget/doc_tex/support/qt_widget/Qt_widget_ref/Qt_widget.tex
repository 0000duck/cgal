% +------------------------------------------------------------------------+
% | CGAL Reference Manual: Reference manual for Qt_widget.tex
% +------------------------------------------------------------------------+
% |
% | 13.12.2001	Radu Ursu
% | 
% | \RCSdef{\qtwidgetRev}{$Revision$}
% | \RCSdefDate{\qtwidgetDate}{$Date$}
% +------------------------------------------------------------------------+

% +-----------------------------------------------------+
\section{Qt\_widget Reference}
\label{Widget Reference}

\begin{ccRefClass}{Qt_widget}

\ccDefinition

The widget Qt\_Widget is a class that is designed to help \cgal\ users easily 
visualize \cgal\ objects and for advanced users to interact with them. This 
widget is designed only for 2D \cgal\ objects.



\ccInclude{CGAL/Qt_widget.h}
\ccGlue

\ccCreation
\ccCreationVariable{win}
\ccSetTwoColumns{Qt_widget}{}

\ccConstructor{Qt\_widget(QWidget *parent = 0, const char *name = 0);}{}

\ccOperations
\ccSetThreeColumns{const_iterator}{container.begin() const;}{}

%scales
%\ccHeading{Scales}
\ccMethod{void set_window(double x_min, double x_max, double y_min, double 
y_max);}{Map the widget coordinates to the interval defined by the rectangle 
with given coordinates. This method should be called before \ccc{show()} from 
QWidget.}
\ccMethod{void set_x_scale(double xscale);}{Set the current X scale.}
\ccMethod{void set_y_scale(double yscale);}{Set the current Y scale.}
\ccMethod{void zoom_in(double ratio);}{Multiply X and Y scales by \ccc{ratio}.}
\ccMethod{void zoom_out(double ratio);}{Multiply X and Y scales by \ccc{ratio}.}
\ccMethod{void move_center(double distx, double disty);}{Move the center of the
 widget with \ccc{distx} on the X axe and with \ccc{disty} on Y axe.}
\ccMethod{void set_center(double x, double y);}{Set the center of the widget to
 O(x, y)}

%\ccHeading{Painter}
%painter
\ccMethod{void clear();}{Clear the screen. The properties remain the same 
after calling this member function.}
\ccMethod{void lock();}{Locks the widget, keeping the widget away from being 
refreshed. If you lock the widget you should verify that you unlock it 
somewhere else. The number of \ccc{lock()} calls should be the same with the 
number of \ccc{unlock()} calls.}
%\ccMethod(void unlock();){Unlocks the widget, and call do\_paint().}
\ccMethod{void do_paint();}{Refresh the widget calling 
\ccc{paintEvent(QPaintEvent *e)} for the Qt\_widget.}

%properties
%\ccHeading{Properties}
\ccMethod{void setColor(Qcolor c);}{Set the current pen color of the widget to 
be c.}
\ccMethod{void setBackgroundColor(QColor c);}{Set the current background color 
to be c.}
\ccMethod{void setFillColor(QColor c);}{Set the current fill color of the 
widget to be c.}
\ccMethod{void setFilled(bool f);}{Set the status of the widget to true or false
 concerning filling the objects: polygons, circles, rectangles \ldots}
\ccMethod{void setLineWidth(uint i);}{Set the current line width of the widget.}
\ccMethod{void setPointSize(uint i);}{Set the current point size of the widget.}
\ccMethod{void setPointStyle(PointStyle s);}{Set the current point style of the
 widget to s. PointStyle is a enumeration declared in Qt\_widget.}
\ccMethod{void setRasterOp(RasterOp r);}{Set the current raster operation.}

%scenes
%\ccHeading{Scenes}
\ccMethod{void add_scene(Qt_scene* s);}{Add the scene s in the list of scenes.
The last added will be on top of the screen.}
\ccMethod{void remove_scene(Qt_scene* s);}{Remove the scene s from the list.}
\ccMethod{void redraw();}{Redraw the scenes situated in the list. Before 
redraw the scenes clears the screen. Also calls \ccc{widget\_repainted()} for 
current tool.}

\ccAccessFunctions
%Properties
%\ccHeading{Properties}
\ccMethod{QColor color() const;}{Returns the current pen color. The color 
returned is a Qt class.}
\ccMethod{QColor backgroundColor() const;}{Returns the current widget background
color. The color returned is a Qt class.}
\ccMethod{QColor fillColor() const;}{Returns the current color used for filling 
the objects. The color returned is a Qt class.}
\ccMethod{PointStyle pointStyle();}{Returns the current point style. PointStyle
 is a enumeration declared in Qt\_widget.}
\ccMethod{uint pointSize();}{Returns the current point size.}
\ccMethod{uint lineWidth();}{Returns the current line width.}
\ccMethod{RasterOp rasterOp();}{Return the current raster operation.}

%Painter
%\ccHeading{Painter}
\ccMethod{QPixmap& get_pixmap();}{Returns the current pixmap.}
\ccMethod{QPainter& painter();}{Returns the current painter.}

%Coordinates
%\ccHeading{Coordinates}
\ccMethod{double x_min() const;}{Returns the left X coordinate  of the widget.}
\ccMethod{double y_min() const;}{Returns the lower Y coordinate  of the widget.}
\ccMethod{double x_max() const;}{Returns the right X coordinate of the widget.}
\ccMethod{double y_max() const;}{Returns the upper Y coordinate of the widget.}

% +------------------------------------------------------------------------+
% | CGAL Reference Manual: Reference manual for Qt_widget.tex
% +------------------------------------------------------------------------+
% |
% | 13.12.2001	Radu Ursu
% | 
% | \RCSdef{\qtwidgetRev}{$Revision$}
% | \RCSdefDate{\qtwidgetDate}{$Date$}
% +------------------------------------------------------------------------+

% +-----------------------------------------------------+

\ccHeading{Operators for Output}

The output operator is defined for all geometric classes in the \cgal\ kernel.

\ccFunction{template<class R> Qt_widget& operator<<(Qt_widget& win, const
Point_2<R>& p);}{}

\ccFunction{template<class R> Qt_widget& operator<<(Qt_widget& win, const
Segment_2<R>& s);}{}

\ccFunction{template<class R> Qt_widget& operator<<(Qt_widget& win, const
Line_2<R>& l);}{}

\ccFunction{template<class R> Qt_widget& operator<<(Qt_widget& win, const
Ray_2<R>& r);}{}

\ccFunction{template<class R> Qt_widget& operator<<(Qt_widget& win, const
Triangle_2<R>& t);}{}

\ccFunction{template<class R>Qt_widget& operator<<(Qt_widget& win, const 
Iso_rectangle_2<R>& r);}{}

\ccFunction{template<class R> Qt_widget& operator<<(Qt_widget& win, const
Circle_2<R>& c);}{}

To use the other operators described you have to include the right header.

\ccInclude{CGAL/IO/Qt_widget_Polygon_2.h}
\ccGlue

\ccFunction{template<class Tr, class Co> Qt_widget& operator<<(Qt_widget& win, 
const Polygon_2<Tr,Co>& pol);}{}

\ccInclude{CGAL/IO/Qt_widget_Min_ellipse_2.h}
\ccGlue

\ccFunction{template<class Traits_> Qt_widget& operator<<( Qt_widget
&win, const Min_ellipse_2<Traits_>& min_ellipse);}{}

\ccInclude{CGAL/IO/Qt_widget_Optimisation_ellipse_2.h}
\ccGlue

\ccFunction{template<class Traits_> Qt_widget& operator<<( Qt_widget
&win, const Optimisation_ellipse_2<Traits_>& oe);}{}

\ccInclude{CGAL/IO/Qt_widget_Conic_2.h}
\ccGlue

\ccFunction{template<class R> Qt_widget& operator<<( Qt_widget& win,
const Conic_2<R>& c);}{}

\ccInclude{CGAL/IO/Qt_widget_Triangulation_2.h}
\ccGlue

\ccFunction{template <class Gt, class Tds> Qt_widget&
operator<<(Qt_widget& win, const Triangulation_2<Gt, Tds>& t);}{}

\ccInclude{CGAL/IO/Qt_widget_Delaunay_triangulation_2.h}
\ccGlue

\ccFunction{template<class Gt, class Tds> Qt_widget& operator<<(Qt_widget& win,
 const Delaunay_triangulation_2<R>& dt);}{}

\ccInclude{CGAL/IO/Qt_widget_Constrained_triangulation_2.h>}
\ccGlue

\ccFunction{template<class Gt, class Tds> Qt_widget& operator<<(Qt_widget& win,
 const Constrained_triangulation_2<Gt, Tds>& t);}{}

\ccInclude{CGAL/IO/Qt_widget_Regular_triangulation_2.h>}
\ccGlue

\ccFunction{template<class Gt, class Tds> Qt_widget& operator<<(Qt_widget& win,
 const Regular_triangulation_2<Gt, Tds>& t);}{}


% +-----------------------------------------------------+
% EOF








% +------------------------------------------------------------------------+
% | CGAL Reference Manual: Reference manual for Qt_widget.tex
% +------------------------------------------------------------------------+
% |
% | 20.12.2001	Radu Ursu
% | 
% | \RCSdef{\qtwidgetRev}{$Revision$}
% | \RCSdefDate{\qtwidgetDate}{$Date$}
% +------------------------------------------------------------------------+

% +-----------------------------------------------------+

\ccHeading{Manipulators for Qt\_widget}


A \ccc{manipulator} is an object which can be inserted in the \ccc{Qt_widget}
, via the operator \ccc{<<}, to change the context for further
drawing.

Here, we simply document the use of these operators which is all the
user needs to know to modify the state of a stream.

\ccFunction{Qt_widget& operator<<(Qt_widget& win, BackgroundColor(const
Color& c));}{Sets the color used for background color.}
\ccGlue

\ccFunction{Qt_widget& operator<<(Qt_widget& win, FillColor(const
Color& c));}{Sets the color used for filling the objects.}
\ccGlue

\ccFunction{Qt_widget& operator<<(Qt_widget& win, LineWidth(const
unsigned int i));}{Sets the width of the line for drawing objects.}
\ccGlue

\ccFunction{Qt_widget& operator<<(Qt_widget& win, PointSize(const
unsigned int i));}{Sets the size of the points.}
\ccGlue

\ccFunction{Qt_widget& operator<<(Qt_widget& win, noFill());}{Sets the state 
of Qt\_widget concerning filling the objects to be false.}
\ccGlue

\ccFunction{Qt_widget& operator<<(Qt_widget& win, const Color& c);}{Sets the 
color used as the Qt\_widget fillColor.}
\ccGlue

\ccFunction{Qt_widget& operator<<(Qt_widget& win, const PointStyle& ps);}{Sets
 the point style for \ccc{Qt_widget}.}
\ccGlue


% +-----------------------------------------------------+
% EOF









\end{ccRefClass}

\ccExample
We could imagine a typical scenario to use Qt\_widget, like the example shown 
bellow:
\begin{ccExampleCode}
#include <CGAL/Cartesian.h>
#include <CGAL/Point_2.h>
#include <CGAL/Segment_2.h>
#include <CGAL/Line_2.h>
#include <CGAL/Ray_2.h>
#include <CGAL/Circle_2.h>
#include <CGAL/Triangle_2.h>
#include <CGAL/Iso_rectangle_2.h>
#include <CGAL/Bbox_2.h>
#include <CGAL/Polygon_2.h>

#include <qapplication.h>
#include <CGAL/IO/Qt_Widget.h>

typedef CGAL::Cartesian<int> Rep;
typedef CGAL::Point_2<Rep> Point;
typedef CGAL::Circle_2<Rep> Circle;
typedef CGAL::Segment_2<Rep> Segment;
typedef CGAL::Line_2<Rep> Line;
typedef CGAL::Ray_2<Rep> Ray;
typedef CGAL::Triangle_2<Rep> Triangle;
typedef CGAL::Iso_rectangle_2<Rep> Rectangle;
typedef CGAL::Bbox_2 BBox;
typedef std::list<Point> Container;
typedef CGAL::Polygon_2<Rep,Container> Polygon;

int main( int argc, char **argv )
{
    QApplication app( argc, argv );
    using namespace CGAL;
    CGAL::Qt_widget W;
    app.setMainWidget( &W );
    W.resize(600, 600);
    W.set_window(0, 600, 0, 600);
    W.show();
    //painting something on the screen
    W.lock();
    W << BackgroundColor(ORANGE) << RED <<
	  LineWidth(3) << PointSize(3) << PointStyle(DISC);
    W << Segment(Point(10,20),Point(300,400));
    W << LineWidth(5) << GREEN << FillColor(BLACK) <<
      Circle(Point(400,400),50*50);
    W << LineWidth(1) << noFill << Circle(Point(300,300),300*300);
    W << BLUE << LineWidth(2);
    W << Segment(Point(200,200),Point(400,400));
    W << Segment(Point(200,400),Point(400,200));
    W.setFilled(TRUE);
    W << RED << Triangle(Point(150,300),
				   Point(150,350),
				   Point(100,325));
    W << FillColor(RED) << Rectangle(Point(320,220),
					       Point(350,240));
    W << DEEPBLUE << BBox(100,80,260,140);
    Polygon p;
    p.push_back(Point(300,30));
    p.push_back(Point(400,30));
    p.push_back(Point(500,130));
    p.push_back(Point(400,180));
    p.push_back(Point(300,130));
    W << p;
    W << Ray(Point(200,400), Point(180,430))
      << Ray(Point(200,400), Point(180,370));
    
    W.unlock();

    return app.exec();
}
\end{ccExampleCode}

In the given example, that is found in demo/basic/first.C, it is created an 
object of type Qt\_widget and then it's used the operators for output and the
manipulators to show some of the widget's functionality.

% +-----------------------------------------------------+
% EOF







