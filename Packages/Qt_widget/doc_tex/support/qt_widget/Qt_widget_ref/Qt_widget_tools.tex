% +------------------------------------------------------------------------+
% | CGAL Reference Manual: Reference manual for Qt_widget.tex
% +------------------------------------------------------------------------+
% |
% | 03.01.2001	Radu Ursu
% | 
% | \RCSdef{\qtwidgetRev}{$Revision$}
% | \RCSdefDate{\qtwidgetDate}{$Date$}
% +------------------------------------------------------------------------+

% +-----------------------------------------------------+
\section{Qt\_widget\_tools}
\label{Qt_widget_tools}


\ccDefinition

A tool for Qt\_widget is a class that receives events from Qt\_widget and can 
provide some functionalities like input objects for Qt\_widget and modify 
already created objects. Also some basic operations can be provided like tools
for zoomIn, zoomOut, translate.

There is a base class for tools called Qt\_widget\_tool, all the tools should 
be derived from this base class.

\begin{ccExampleCode}
namespace CGAL {
class Qt_widget_zoomrect : public Qt_widget_tool
{
public:
	....
protected:
	....
private:
	....
}
\end{ccExampleCode}

In order to attach a tool to Qt\_widget all you have to do is to declare and 
define the tool and then use the \(>>\) operator. The Qt\_widget and the tool
mechanism is designed to use only one tool at one moment of time. You can't
use two tools at the same time. When a tool is attached, both the tool and the
Qt\_widget will receive the events from the Window System.

\begin{ccExampleCode}
CGAL::Qt_widget_zoomrect	zoomrectbut;
CGAL::Qt_widget			*widget;

*widget >> zoomrectbut;
\end{ccExampleCode}
The events are declared in the base class as virtual functions. Your tool class
has to overload this functions in order to receive the events from the main 
widget. Only the tool that is attached at the moment will receive the events
from the widget.

\begin{ccExampleCode}
namespace CGAL {
class Qt_widget_zoomrect : public Qt_widget_tool
{
.....
private:
  void mousePressEvent(QMouseEvent *e)
  {
    if(e->button() == CGAL_QT_WINDOW_ZOOMRECT_BUTTON)
    {
      if (!on_first)
      {
        first_x = e->x();
        first_y = e->y();
        on_first = TRUE;
      }
    }
  };
......
}
\end{ccExampleCode}

% +-----------------------------------------------------+
% EOF







