% +------------------------------------------------------------------------+
% | CGAL Reference Manual: Reference manual for Qt_widget.tex
% +------------------------------------------------------------------------+
% |
% | 03.01.2001	Radu Ursu
% | 
% | \RCSdef{\qtwidgetRev}{$Revision$}
% | \RCSdefDate{\qtwidgetDate}{$Date$}
% +------------------------------------------------------------------------+

% +-----------------------------------------------------+
\section{Qt\_widget\_tools}
\label{Qt_widget_tools}


\ccDefinition

A tool for Qt\_widget is a class that receives events from Qt\_widget and can 
provide some functionalities like input objects for Qt\_widget and modify 
already created objects. Also some basic operations can be provided like tools
for zoomIn, zoomOut, translate.

There is a base class for tools called Qt\_widget\_tool, all the tools should 
be derived from this base class.
\begin{ccExampleCode}
namespace CGAL {
class Qt_widget_zoomrect : public Qt_widget_tool
{
public:
	....
protected:
	....
private:
	....
}//endclass
}//endnamespace
\end{ccExampleCode}
In order to attach a tool to Qt\_widget all you have to do is to declare and 
define the tool and then use the \(>>\) operator. The Qt\_widget and the tool
mechanism is designed to use only one tool at one moment of time. You can't
use two tools at the same time. When a tool is attached, both the tool and the
Qt\_widget will receive the events from the Window System.
\begin{ccExampleCode}
CGAL::Qt_widget_zoomrect	zoomrectbut;
CGAL::Qt_widget			*widget;

*widget >> zoomrectbut;
\end{ccExampleCode}
The events are declared in the base class as virtual functions. Your tool class
has to overload this functions in order to receive the events from the main 
widget. Only the tool that is attached at the moment will receive the events
from the widget.
\begin{ccExampleCode}
namespace CGAL {
class Qt_widget_zoomrect : public Qt_widget_tool
{
.....
private:
  void mousePressEvent(QMouseEvent *e)
  {
    if(e->button() == CGAL_QT_WINDOW_ZOOMRECT_BUTTON)
    {
      if (!on_first)
      {
        first_x = e->x();
        first_y = e->y();
        on_first = TRUE;
      }
    }
  };
......
}//endclass
}//endnamespace
\end{ccExampleCode}

\begin{ccRefClass}{Qt_widget_tool}

\ccInclude{CGAL/Qt_widget_tool.h}
\ccGlue

\ccCreation
\ccCreationVariable{tool}
\ccSetTwoColumns{Qt_widget_tool}{}

\ccConstructor{Qt_widget_tool();}{The default constructor.}
\ccOperations
\ccSetThreeColumns{const_iterator}{container.begin() const;}{}

\ccMethod{bool is_attached() const;}{Return true if this tool is currently 
attached to the Qt\_widget.}
\ccMethod{virtual void mousePressEvent(QMouseEvent *);}{}
\ccGlue
\ccMethod{virtual void mouseReleaseEvent(QMouseEvent *);}{}
\ccGlue
\ccMethod{virtual void wheelEvent(QMouseEvent *);}{}
\ccGlue
\ccMethod{virtual void mouseDoubleClickEvent(QMouseEvent *);}{}
\ccGlue
\ccMethod{virtual void mouseMoveEvent(QMouseEvent *);}{}
\ccGlue
\ccMethod{virtual void keyPressEvent(QKeyEvent *);}{}
\ccGlue
\ccMethod{virtual void keyReleaseEvent(QKeyEvent *);}{}
\ccGlue
\ccMethod{virtual void enterEvent(QEvent *);}{}
\ccGlue
\ccMethod{virtual void leaveEvent(QEvent *);}{This virtual functions should be
overloaded in the child tool classes and declared as private members. This 
functions are called by the Qt\_widget every time a event is received by the 
widget.}
\ccHeading{protected}
\ccMethod{virtual void attaching();}{}
\ccGlue
\ccMethod{virtual void detaching();}{You have to overload this functions to
provide your tool with initialization and destruction code.}
\ccHeading{signals}
\ccMethod{void redraw();}{Emited when a tool needs that Qt\_widget refresh the
screen. For example after a tool modified an object, the screen should reflect
the change.}
\ccHeading{public slots}
\ccMethod{virtual void widget_repainted();}{Called every time by Qt\_widget
when the widget have been repainted.}
\end{ccRefClass}
\ccExample
An example of a tool that is working as a magnifying glass.
\begin{ccExampleCode}
#ifndef CGAL_QT_WINDOW_ZOOM_H
#define CGAL_QT_WINDOW_ZOOM_H

#include <CGAL/IO/Qt_Widget.h>
#include <CGAL/IO/Qt_Widget_tool.h>
#include <qcolor.h>

namespace CGAL {

class Qt_widget_zoom : public Qt_widget_tool
{
  //  Q_OBJECT
private:
	int									    x2, y2;
	bool circle_already_drawn;
	bool old_mouse_tracking;

public:
  Qt_widget_zoom() : circle_already_drawn(false) {};

  void draw_circle(int x,int y)
  {
    const int
    d=100, // diameter
    r=50;  // radius (should be d/2 :-)

    RasterOp oldRasterOp = widget->rasterOp();	//save the initial raster mode
    widget->setRasterOp(XorROP);
    QColor oldColor=widget->color(); // save the initial color
    widget->setColor(Qt::green);
    widget->painter().drawEllipse(x-r, y-r, d, d);
    widget->setColor(oldColor);
    widget->setRasterOp(oldRasterOp);
    widget->do_paint();
  };

  void leaveEvent(QEvent*)
  {
    if(circle_already_drawn)
      draw_circle(x2,y2);
    circle_already_drawn=false;
  };

  void widget_repainted(){
    circle_already_drawn=false;
  };

  void mousePressEvent(QMouseEvent *e)
  {
    const double ratios=2.0;
    double
      x=widget->x_real(e->x()),
      y=widget->y_real(e->y());
		
    if(e->button() == Qt::LeftButton)
      widget->zoom_in(ratios, x, y);
    if(e->button() == Qt::RightButton)
      widget->zoom_out(ratios, x, y);
		
    widget->redraw();
    emit(redraw()); 
  };

  void mouseMoveEvent(QMouseEvent *e)
  {
    int
      x = e->x(),
      y = e->y();

    widget->lock();
    draw_circle(x,y); // draw the new circle
    if(circle_already_drawn) // erase the old one if needed
        draw_circle(x2,y2);
    widget->unlock();
		
    //save the last coordinates to redraw the screen
    x2 = x;
    y2 = y;
    circle_already_drawn=true;
  };

  void attaching()
  {
    old_mouse_tracking=widget->hasMouseTracking();
    widget->setMouseTracking(TRUE);
    oldcursor = widget->cursor();
    widget->setCursor(crossCursor);
    circle_already_drawn=false;
  };

  void detaching()
  {
    if(circle_already_drawn)
      draw_circle(x2,y2); // erase the circle if needed
    widget->setCursor(oldcursor);
    widget->setMouseTracking(old_mouse_tracking);
  };
};//end class 

} // namespace CGAL

#endif // CGAL_QT_WINDOW_ZOOM_H
\end{ccExampleCode}

\ccHeading{Creating \cgal\ objects}
In order to create \cgal\ objects, your tool has to call the public member 
function new\_object(CGAL::Object) from Qt\_widget. Qt\_widget will emit the 
signal new\_cgal\_object(CGAL::Object). To be more clear, all the objects 
created with tools, will be catched by Qt\_widget and send to user through the
 signal new\_cgal\_object(CGAL::Object). This way, the user should write only 
a connect like this:

connect(Qt\_Widget,SIGNAL(new\_cgal\_object(CGAL::Object)),this,SLOT(receive\_new\_object(CGAL::Object)));

Also in the user class the code should look like this:
\begin{ccExampleCode}
private slots:
void receive_new_object(CGAL::Object obj)
{
    Point p;
    Segment s;
    Line l;
    Polygon_2 poly;
    if (CGAL::assign(p, obj)) {
       ...
    } else if(CGAL::assign(s,obj)) {
       ...      
    } else if(CGAL::assign(l, obj)) {
       ...
    } else if(CGAL::assign(poly, obj)) {
       ...
    }
    ...
};
\end{ccExampleCode}

An example of a tool that create \cgal\ points when the user click the left mouse button over the widget:
\begin{ccExampleCode}
namespace CGAL {

template <class R>
class Qt_widget_get_point : public Qt_widget_tool
{
  //  Q_OBJECT
public:
  typedef typename R::Point_2	Point;
  typedef typename R::FT	FT;
  
  Qt_widget_get_point(const QCursor c=QCursor(Qt::crossCursor)) :
    cursor(c) {};
  
private:
  void mousePressEvent(QMouseEvent *e)
  {
    if(e->button() == CGAL_QT_WIDGET_GET_POINT_BUTTON)
    {
      FT
        x=static_cast<FT>(widget->x_real(e->x())),
        y=static_cast<FT>(widget->y_real(e->y()));
      widget->new_object(make_object(Point(x, y)));
    }
  };
  void attaching()
  {
    oldcursor = widget->cursor();
    widget->setCursor(cursor);
  };
  
  void detaching()
  {
    widget->setCursor(oldcursor);
  };

  QCursor cursor;
};

} // namespace CGAL
\end{ccExampleCode}
% +-----------------------------------------------------+
% EOF








