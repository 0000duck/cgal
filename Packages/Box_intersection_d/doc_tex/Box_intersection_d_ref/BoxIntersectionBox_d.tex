% +------------------------------------------------------------------------+
% | Reference manual page: BoxIntersectionBox_d.tex
% +------------------------------------------------------------------------+


\begin{ccRefConcept}{BoxIntersectionBox_d}
\ccCreationVariable{box}
\ccThree{const int}{BoxIntersectionBox_d::dimension();}{}\ccThreeToTwo
\ccDefinition

The \ccRefName\ concept is used in the context of the intersection
algorithms for sequences of iso-oriented boxes. These algorithms come
with a default traits class that assumes that the boxes are a model of
this \ccRefName\ concept. This concept defines the access functions to
the dimension, the \ccc{id}-number, and the boundaries of the box.

\ccRefines

\ccc{Assignable}.

\ccHasModels

\ccRefIdfierPage{CGAL::Box_intersection_d::Box_d<NT,int N,IdPolicy>}\\
\ccRefIdfierPage{CGAL::Box_intersection_d::Box_with_handle_d<NT, int N, Handle, IdPolicy>}

\ccTypes

\ccNestedType{NT}{number type to represent the box
  boundaries. Supported are the builtin types \ccc{int}, \ccc{unsigned
  int}, \ccc{float}, and \ccc{double}.} 

\ccNestedType{ID}{type for the box \ccc{id}-number,
  must be a model of the \ccc{LessThanComparable} concept.}

\ccAccessFunctions

\ccFunction{const int BoxIntersectionBox_d::dimension();}{%
  returns the number of dimensions.}

\ccMethod{ID id() const;}{%
  returns the unique \ccc{id}-number for the \ccc{box}.}

\ccMethod{NT min_coord( int dim) const;}{%
  returns the lower boundary value of the \ccc{box} in dimension
  \ccc{dim}, $0 \leq $\ccc{dim}$ < $\ccc{dimension()}.}

\ccMethod{NT max_coord( int dim) const;}{%
  returns the upper boundary value of the \ccc{box} in dimension
  \ccc{dim}, $0 \leq $\ccc{dim}$ < $\ccc{dimension()}.}

\ccSeeAlso

\ccRefIdfierPage{CGAL::box_intersection_d}\\
\ccRefIdfierPage{CGAL::box_self_intersection_d}\\
\ccRefIdfierPage{CGAL::box_intersection_all_pairs_d}

\ccRefIdfierPage{CGAL::Box_intersection_d::Box_traits_d<Box>}\\
\ccRefConceptPage{BoxIntersectionTraits_d}


\end{ccRefConcept}
