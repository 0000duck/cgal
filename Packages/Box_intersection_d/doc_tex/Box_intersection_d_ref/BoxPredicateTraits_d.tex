% +------------------------------------------------------------------------+
% | Reference manual page: BoxPredicateTraits_d.tex
% +------------------------------------------------------------------------+
% +------------------------------------------------------------------------+

\ccRefPageBegin
\begin{ccRefConcept}{BoxPredicateTraits_d}
The box predicate traits class specifies predicates between boxes. It propagates number type, box type, dimension and all other fields from the box adapter to the actual implementation of bounding box intersection. Formally, it is a subtype of BoxTraits\_d, and as such must provide all fields that are required in BoxTraits\_d. Usually, this traits class need not be changed or user provided. A default implementation is provided. (reference)

\ccTypes
\ccNestedType{Box} {box handle type. Bounding box objects are passed to functions and predicates using exactly this type.}
\ccNestedType{NT}  {Number type of the box vector components.}

\ccNestedType{Compare}{Function object that compares two boxes with respect to the predicate \ccc{is_lo_less_lo.}}
\ccNestedType{Lo_less}{Function object that compares the starting point of a box in a certain dimension against a fixed value. Or, more compactly in math-like notation: \newline \ccc{Lo_less(val,dim)=fn box->get_lo(box,dim) < val}}
\ccNestedType{Hi_greater}{Function object: \newline \ccc{Hi_greater(val,dim)=fn box->get_hi(box,dim) > val} }
\ccNestedType{Spanning}{Function object: \newline \ccc{Spanning(lo,hi,dim)=fn box->get_lo(box,dim) < lo && get_hi(box,dim) > hi}}


\ccOperations
For all methods that require a dimension parameter, \ccc{dim < get_dim()} must hold. Otherwise, results are undefined.

\ccMethod
  {bool Traits::is_lo_less_lo(Box a, Box b, unsigned int dim);}
  {interval starting point of Box a is less or equal to b's, regarding dimension dim. If equal, the id of a and b are compared. If the id of a is less, return true.}
\ccMethod
  {bool Traits::is_lo_less_hi(Box a, Box b, unsigned int dim);}
  {interval starting point of a is then than end point of b, regarding dimension dim s.t. the intervals in dimension dim could possibly overlap.}
\ccMethod
  {bool Traits::does_intersect (Box a, Box b, unsigned int dim);}
  {intervals of box a and box b intersect in dimension dim.}
\ccMethod
  {bool Traits::contains_lo_point(Box a, Box b, unsigned int dim);}
  {interval of box a contains interval starting point of b in dimension dim, s.t. the intervals in dimension dim could possibly overlap.}

\ccHasModels
\ccRefIdfierPage{CGAL::Box_predicate_traits_d<BoxTraits_d, bool closed>}

\end{ccRefConcept}

% +------------------------------------------------------------------------+
%%RefPage: end of main body, begin of footer
\ccRefPageEnd
% EOF
% +------------------------------------------------------------------------+
