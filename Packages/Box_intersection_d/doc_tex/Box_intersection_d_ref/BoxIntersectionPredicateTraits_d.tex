% +------------------------------------------------------------------------+
% | Reference manual page: BoxIntersectionPredicateTraits_d.tex
% +------------------------------------------------------------------------+

\begin{ccRefConcept}{BoxIntersectionPredicateTraits_d}
\begin{ccAdvanced}
  
This very fine-grained \ccRefName\ concept allows the hand-crafted
implementation of the individual interval comparison tests. This is
only for experts and might only be useful in the context of special
number types or exploitation of special assembler instructions!  The
generic default implementation
\ccc{CGAL::Box_intersection_d::Predicate_traits_d<BoxTraits_d,
  bool closed>} is likely to suffice for all practical purposes.

\ccRefines

\ccRefConceptPage{BoxIntersectionTraits_d}

\ccTypes
\ccTwo{BoxIntersectionPredicateTraits_d::MHi_greater}{}

\ccNestedType{Box}{box type. Since we support in our algorithms
  passing the boxes by value as well as passing them as pointers, we
  have as box type either \ccc{const B&}, \ccc{B*}, or \ccc{const B*}
  respectively, if \ccc{B} is the actually class type storing the box.
  The box objects are passed in parameter lists of functions and
  predicates using exactly this type.} 

\ccNestedType{NT}{number type to represent the box
  boundaries. Supported are the builtin types \ccc{int}, \ccc{unsigned
  int}, \ccc{float}, and \ccc{double}.} 

\ccNestedType{ID}{type for the \ccc{id}-number,
  model of the \ccc{LessThanComparable} concept.}

\ccNestedType{Compare}{Function object that compares two boxes with
  respect to the predicate \ccc{is_lo_less_lo.}}

\ccNestedType{Lo_less}{Function object that compares the starting
  point of a box in a certain dimension against a fixed value. Or,
  more compactly in math-like notation:
  \ccc{Lo_less(val,dim)=fn box->get_lo(box,dim) < val}} 

\ccNestedType{Hi_greater}{Function object: \ccc{Hi_greater(val,dim)=fn
    box->get_hi(box,dim) > val}} 

\ccNestedType{Spanning}{Function object: \ccc{Spanning(lo,hi,dim)=fn
    box->get_lo(box,dim) < lo && get_hi(box,dim) > hi}}


\ccOperations
\ccThree{bool.}{BoxIntersectionPred}{}
For all methods that require a dimension parameter, \ccc{dim <
  dimension()} must hold.

\ccFunction{bool BoxIntersectionPredicateTraits_d::is_lo_less_lo(Box
      a, Box b, int dim);} 
  {the interval starting point of box $a$ is less or equal to box
      $b$'s, regarding dimension \ccc{dim}. If equal, the
      \ccc{id}-number of $a$ and $b$ is compared. If the
      \ccc{id}-number of $a$ is less, return \ccc{true}.} 

\ccFunction{bool BoxIntersectionPredicateTraits_d::is_lo_less_hi(Box
      a, Box b, int dim);} 
  {the interval starting point of box $a$ is less than interval end point of
      box $b$, regarding dimension \ccc{dim}, i.e., the intervals in
      dimension \ccc{dim} could possibly overlap.}

\ccFunction{bool BoxIntersectionPredicateTraits_d::does_intersect (Box
      a, Box b, int dim);} 
  {intervals of box $a$ and box $b$ intersect in dimension \ccc{dim}.}

\ccFunction{bool
  BoxIntersectionPredicateTraits_d::contains_lo_point(Box a, Box b, int dim);} 
  {interval of box $a$ contains interval starting point of box $b$ in
  dimension \ccc{dim}, i.e., the intervals in dimension \ccc{dim}
  could possibly overlap.} 

\ccHasModels

\ccRefIdfierPage{CGAL::Box_intersection_d::Predicate_traits_d<BoxTraits,bool closed>} 

\ccSeeAlso

\ccRefIdfierPage{CGAL::box_intersection_custom_predicates_d}\\
\ccRefIdfierPage{CGAL::box_intersection_all_pairs_custom_predicates_d}


\end{ccAdvanced}
\end{ccRefConcept}
