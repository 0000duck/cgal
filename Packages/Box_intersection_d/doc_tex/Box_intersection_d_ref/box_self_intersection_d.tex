% +------------------------------------------------------------------------+
% | Reference manual page: box_self_intersection_d.tex
% +------------------------------------------------------------------------+
% | 19.05.2004   Lutz Kettner, Andreas Meyer
% | Package: Package
% | 
\RCSdef{\RCSboxintersectiondRev}{$Revision$}
\RCSdefDate{\RCSboxintersectiondDate}{$Date$}
% |
%%RefPage: end of header, begin of main body
% +------------------------------------------------------------------------+


\begin{ccRefFunction}{box_self_intersection_d}
\ccIndexSubitemDef{self-intersection}{iso-oriented boxes}
\ccIndexSubitem{intersection}{self-intersection}

\ccDefinition
  
The function \ccRefName\ computes the pairwise intersecting boxes
in a sequence of iso-oriented boxes in arbitrary dimension.
The sequence of boxes is given with as a random-access iterator
range and will be reordered in the course of the algorithm. For each
intersecting pair of boxes a \ccc{callback} function object is called
with the two intersecting boxes as argument; the first argument is a
box from the sequence, the second argument is a copy of a box from the
sequence.  The performance of the algorithm can be tuned with a
\ccc{cutoff} parameter, see the implementation section of the
\ccc{CGAL::box_intersection_d} function on
page~\ccRefPage{CGAL::box_intersection_d}. 

The algorithm creates a second copy of the boxes and reorders the
boxes in the course of the algorithm. Now, depending on the size of a
box it can be faster to copy the boxes, or to work with pointers to
boxes and copy only pointers. We offer automatic support for both
options. We call the \ccc{value_type} of the iterator ranges the
\emph{box type}, which is either the type of our boxes, or a pointer
type to our type of boxes.

A $d$-dimensional iso-oriented box is defined as the Cartesian product
of $d$ intervals. We call the box \emph{half-open} if the $d$
intervals $\{ [lo_i,hi_i) | 0 \leq i < d\}$ are half-open intervals,
and we call the box \emph{closed} if the $d$ intervals $\{ [lo_i,hi_i]
| 0 \leq i < d\}$ are closed intervals. Note that closed boxes support
zero-width boxes and they can intersect at their boundaries, while
non-empty half-open boxes always have a positive volume and they only
intersect iff their interiors overlap.  The distinction between closed
or half-open boxes does not require a different representation of
boxes, just a different interpretation when comparing boxes, which is
selected with the \ccc{topology} parameter and its two values,
\ccc{CGAL::Box_intersection_d::HALF_OPEN} and
\ccc{CGAL::Box_intersection_d::CLOSED}.

In addition, a box has an \ccc{id}-number that is distinct for
all boxes. It is needed in order to have a pair of intersecting boxes
to be reported only once in the algorithm. This is necessary if boxes
of identical geometry appear several times in the input sequences.
For this self-intersection function a second copy of the box sequence
is created. The copying has to preserve the \ccc{id}-number of boxes.
Note that this implies that the address of the box might not be
sufficient for the \ccc{id}-number if boxes are copied by value. Boxes of
equal \ccc{id}-number are not reported as intersecting pairs since they are
always intersecting trivially.


The algorithm uses a traits class of the \ccc{BoxIntersectionTraits_d}
concept to access the boxes. A default traits class is provided that
works under the assumption that the box type is a model of the
\ccc{BoxIntersectionBox_d} concept, or a pointer to a model of that
concept (see above for the value versus pointer nature of the box type).

\ccInclude{CGAL/box_intersection_d.h}

\ccThree{void}{box_intersection_custom_}{}
%\def\ccLongParamLayout{\ccTrue}
\ccFunction{template< class RandomAccessIterator, class Callback >
void box_intersection_d(
    RandomAccessIterator begin, RandomAccessIterator end,
    Callback& callback,
    std::ptrdiff_t cutoff = 10,
    Box_intersection_d::Topology topology = Box_intersection_d::CLOSED);
}{%
    Invokation of box intersection with default box traits
    \ccc{CGAL::Box_intersection_d::Box_traits_d<Box_type>}, where
    \ccc{Box_type} corresponds to the iterator value type  of
    \ccc{RandomAccessIterator}.}
 
\ccFunction{template< class RandomAccessIterator,
            class Callback, class BoxTraits >
void box_intersection_d(
    RandomAccessIterator begin, RandomAccessIterator end,
    Callback& callback,
    BoxTraits box_traits,
    std::ptrdiff_t cutoff = 10,
    Box_intersection_d::Topology topology = Box_intersection_d::CLOSED);
}{Invocation with custom box traits.}


\ccRequirements

\begin{itemize}
  \item
    \ccc{RandomAccessIterator} must be a mutable random-access
    iterator. We call its value type \ccc{Box_type} in the following.
  \item
    \ccc{Callback} must be of the \ccc{BinaryFunction} concept. 
    The \ccc{Box_type} must be convertible to both argument types. The
    return type is not used and can be \ccc{void}.
  \item
    The \ccc{Box_type} must be a model of the \ccc{Assignable} concept.
  \item
    In addition, if the default box traits is used the \ccc{Box_type} must be 
    a class type \ccc{T}, or a pointer to a class type \ccc{T}, where
    \ccc{T} must be a model of the \ccc{BoxIntersectionBox_d} concept.
    In both cases, the default box traits specializes to a suitable
    implementation.
  \item
    \ccc{BoxTraits} must be of the \ccc{BoxIntersectionTraits_d} concept.
\end{itemize}

\ccSeeAlso

\ccRefIdfierPage{CGAL::box_intersection_d}\\
\ccRefIdfierPage{CGAL::box_intersection_all_pairs_d}\\
\ccRefIdfierPage{CGAL::box_intersection_custom_predicates_d}

\ccRefIdfierPage{CGAL::Box_intersection_d::Box_traits_d<Box>}\\
\ccRefConceptPage{BoxIntersectionBox_d}\\
\ccRefConceptPage{BoxIntersectionTraits_d}

\ccImplementation

See the implementation section of the \ccc{CGAL::box_intersection_d}
function on page~\ccRefPage{CGAL::box_intersection_d}.

\ccExample

We create a sequence of 100 three-dimensional random boxes, compute
the pairwise intersecting boxes, and report them on \ccc{cout}.

\ccIncludeExampleCode{Box_intersection_d/complete.C}

\end{ccRefFunction}

% +------------------------------------------------------------------------+
%%RefPage: end of main body, begin of footer
% EOF
% +------------------------------------------------------------------------+

