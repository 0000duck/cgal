% +------------------------------------------------------------------------+
% | Reference manual page: box_intersection_all_pairs_d.tex
% +------------------------------------------------------------------------+
% | 19.05.2004   Lutz Kettner, Andreas Meyer
% | Package: Package
% | 
\RCSdef{\RCSboxintersectiondRev}{$Revision$}
\RCSdefDate{\RCSboxintersectiondDate}{$Date$}
% |
%%RefPage: end of header, begin of main body
% +------------------------------------------------------------------------+


\begin{ccRefFunction}{box_intersection_all_pairs_d}
\ccHtmlCrossLink{CGAL::box_intersection_all_pairs_custom_predicates_d}
\ccRefLabel{CGAL::box_intersection_all_pairs_custom_predicates_d}
\ccIndexSubitem{intersection}{iso-oriented boxes}
\ccIndexSubitemDef{intersection}{all pairs}

\ccDefinition
  
The function \ccRefName\ computes the pairwise intersecting boxes
between two sequences of iso-oriented boxes in arbitrary dimension.
The sequences of boxes are given with two random-access iterator
ranges. The sequences are not modified. For each intersecting pair of
boxes a \ccc{callback} function object is called with the two
intersecting boxes as argument; the first argument is a box from the
first sequence, the second argument a box from the second sequence.

The algorithm is interface compatible with the \ccc{CGAL::box_intersection_d}
function on page~\ccRefPage{CGAL::box_intersection_d}. Therefore, we
call the \ccc{value_type} of the iterator ranges the \emph{box type},
which is either the type of our boxes, or a pointer type to our type
of boxes. 

A $d$-dimensional iso-oriented box is defined as the Cartesian product
of $d$ intervals. We call the box \emph{half-open} if the $d$
intervals $\{ [lo_i,hi_i) | 0 \leq i < d\}$ are half-open intervals,
and we call the box \emph{closed} if the $d$ intervals $\{ [lo_i,hi_i]
| 0 \leq i < d\}$ are closed intervals. Note that closed boxes support
zero-width boxes and they can intersect at their boundaries, while
non-empty half-open boxes always have a positive volume and they only
intersect iff their interiors overlap.  The distinction between closed
or half-open boxes does not require a different representation of
boxes, just a different interpretation when comparing boxes, which is
selected with the \ccc{topology} parameter and its two values,
\ccc{CGAL::Box_intersection_d::HALF_OPEN} and
\ccc{CGAL::Box_intersection_d::CLOSED}.

In addition, a box has an \ccc{id}-number that is distinct for
all boxes. It is needed in order to have a pair of intersecting boxes
to be reported only once in the algorithm. This is necessary if boxes
of identical geometry appear several times in the input sequences.

The algorithm uses a traits class of the \ccc{BoxIntersectionTraits_d}
concept to access the boxes. A default traits class is provided that
works under the assumption that the box type is a model of the
\ccc{BoxIntersectionBox_d} concept, or a pointer to a model of that
concept (see above for the value versus pointer nature of the box type).

An important special application of this algorithm is the test for
self-intersections where the second box sequence is an identical copy
of the first sequence including the preserved \ccc{id}-number. Note
that this implies that the address of the box might not be sufficient
for the \ccc{id}-number if boxes are copied by value. Boxes of equal
\ccc{id}-number are not reported as intersecting pairs since they are
always intersecting trivially.

\ccInclude{CGAL/box_intersection_d.h}

\ccThree{void}{box_inter}{}
%\def\ccLongParamLayout{\ccTrue}
\ccFunction{template< class RandomAccessIterator1, 
            class RandomAccessIterator2, 
            class Callback >
void box_intersection_all_pairs_d(
    RandomAccessIterator1 begin1, RandomAccessIterator1 end1,
    RandomAccessIterator2 begin2, RandomAccessIterator2 end2,
    Callback callback,
    Box_intersection_d::Topology topology = Box_intersection_d::CLOSED);
}{%
    Invocation of box intersection with default box traits
    \ccc{CGAL::Box_intersection_d::Box_traits_d<Box_type>}, where
    \ccc{Box_type} corresponds to the iterator value type  of
    \ccc{RandomAccessIterator1}.}
 
\ccFunction{template< class RandomAccessIterator1,
            class RandomAccessIterator2,
            class Callback, class BoxTraits >
void box_intersection_all_pairs_d(
    RandomAccessIterator1 begin1, RandomAccessIterator1 end1,
    RandomAccessIterator2 begin2, RandomAccessIterator2 end2,
    Callback callback,
    BoxTraits box_traits,
    Box_intersection_d::Topology topology = Box_intersection_d::CLOSED);
}{Invocation with custom box traits.}

\begin{ccAdvanced}
\def\ccLongParamLayout{\ccTrue}
\ccFunction{template< class RandomAccessIterator1,
                  class RandomAccessIterator2,
                  class Callback, class BoxPredicateTraits >
void box_intersection_all_pairs_custom_predicates_d(
    RandomAccessIterator1 begin1, RandomAccessIterator1 end1,
    RandomAccessIterator2 begin2, RandomAccessIterator2 end2,
    Callback callback,
    BoxPredicateTraits traits);
}{%
    This version of the algorithm is interface compatible with the
    \ccc{CGAL::box_intersection_custom_predicates_d} function on 
    \ccRefPage{CGAL::box_intersection_d}. Therefore,
    this version of the box-intersection algorithms uses a very
    fine-grained predicate traits class that allows the hand-crafted
    implementation of the individual interval comparison tests. This is
    only for experts and might only be useful in the context of special
    number types or exploitation of special assembler instructions!
    The details of half-open or closed boxes and the \ccc{id}-value of a
    box are encoded in this predicate traits.
}
\def\ccLongParamLayout{\ccFalse}
\end{ccAdvanced}

\ccRequirements

\begin{itemize}
  \item
    \ccc{RandomAccessIterator1}, and \ldots \ccc{2}, must be
    random-access iterators and both value types must be
    the same. We call this value type \ccc{Box_type} in the following.
  \item
    \ccc{Callback} must be of the \ccc{BinaryFunction} concept. 
    The \ccc{Box_type} must be convertible to both argument types. The
    return type is not used and can be \ccc{void}.
  \item
    The \ccc{Box_type} must be a model of the \ccc{Assignable} concept.
  \item
    In addition, if the default box traits is used the \ccc{Box_type} must be 
    a class type \ccc{T}, or a pointer to a class type \ccc{T}, where
    \ccc{T} must be a model of the \ccc{BoxIntersectionBox_d} concept.
    In both cases, the default box traits specializes to a suitable
    implementation.
  \item
    \ccc{BoxTraits} must be of the \ccc{BoxIntersectionTraits_d} concept.
  \item
    \ccc{BoxPredicateTraits} must be of the
    \ccc{BoxIntersectionPredicateTraits_d} concept.
\end{itemize}

\ccSeeAlso

\ccRefIdfierPage{CGAL::box_intersection_d}\\
\ccRefIdfierPage{CGAL::box_self_intersection_d}

\ccRefIdfierPage{CGAL::Box_intersection_d::Box_traits_d<Box>}\\
\ccRefConceptPage{BoxIntersectionBox_d}\\
\ccRefConceptPage{BoxIntersectionTraits_d}

\begin{ccAdvanced}

\ccRefIdfierPage{CGAL::Box_intersection_d::Predicate_traits_d<BoxTraits,bool closed>}\\
\ccRefConceptPage{BoxIntersectionPredicateTraits_d}

\end{ccAdvanced}


\ccImplementation

The algorithm is trivially testing all pairs and runs therefore in time
$O(nm)$ where $n$ is the size of the first sequence and $m$ is the
size of the second sequence.

\end{ccRefFunction}

% +------------------------------------------------------------------------+
%%RefPage: end of main body, begin of footer
% EOF
% +------------------------------------------------------------------------+

