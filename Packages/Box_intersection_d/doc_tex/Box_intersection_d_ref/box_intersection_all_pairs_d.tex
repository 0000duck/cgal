% +------------------------------------------------------------------------+
% | Reference manual page: box_intersection_all_pairs_d.tex
% +------------------------------------------------------------------------+
% | 19.05.2004   Lutz Kettner, Andreas Meyer
% | Package: Package
% | 
\RCSdef{\RCSboxintersectiondRev}{$Revision$}
\RCSdefDate{\RCSboxintersectiondDate}{$Date$}
% |
%%RefPage: end of header, begin of main body
% +------------------------------------------------------------------------+


\begin{ccRefFunction}{box_intersection_all_pairs_d}
%\ccHtmlCrossLink{CGAL::box_intersection_all_pairs_custom_predicates_d}
%\ccRefLabel{CGAL::box_intersection_all_pairs_custom_predicates_d}
\ccIndexSubitem{intersection}{iso-oriented boxes}
\ccIndexSubitemDef{intersection}{all pairs}

\ccDefinition
  
The function \ccRefName\ computes the pairwise intersecting boxes
between two sequences of iso-oriented boxes in arbitrary dimension.
It does so by comparing all possible pairs of boxes and is thus
inferior to the fast \ccc{CGAL::box_intersection_d} algorithm on page
\pageref{ccRef_CGAL::box_intersection_d}.

The sequences of boxes are given with two forward iterator ranges. The
sequences are not modified. For each intersecting pair of boxes a
\ccc{callback} function object is called with the two intersecting
boxes as argument; the first argument is a box from the first
sequence, the second argument a box from the second sequence.

The algorithm is interface compatible with the
\ccc{CGAL::box_intersection_d} function. Similarly, we call the
\ccc{value_type} of the iterators the \emph{box handle}, which is
either our box type or a pointer type to our box type.

A $d$-dimensional iso-oriented box is defined as the Cartesian product
of $d$ intervals. We call the box \emph{half-open} if the $d$
intervals $\{ [lo_i,hi_i) \,|\, 0 \leq i < d\}$ are half-open intervals,
and we call the box \emph{closed} if the $d$ intervals $\{ [lo_i,hi_i]
\,|\, 0 \leq i < d\}$ are closed intervals. Note that closed boxes support
zero-width boxes and they can intersect at their boundaries, while
non-empty half-open boxes always have a positive volume and they only
intersect iff their interiors overlap.  The distinction between closed
or half-open boxes does not require a different representation of
boxes, just a different interpretation when comparing boxes, which is
selected with the \ccc{topology} parameter and its two values,
\ccc{CGAL::Box_intersection_d::HALF_OPEN} and
\ccc{CGAL::Box_intersection_d::CLOSED}.

In addition, a box has an unique \ccc{id}-number. Boxes with equal
\ccc{id}-number are not reported since they obviously intersect trivially.

The algorithm uses a traits class of the \ccc{BoxIntersectionTraits_d}
concept to access the boxes. A default traits class is provided that
assumes that the box type is a model of the \ccc{BoxIntersectionBox_d}
concept and that the box handle, i.e., the iterators value type, is
identical to the box type or a pointer to the box type.

An important special application of this algorithm is the test for
self-intersections where the second box sequence is an identical copy
of the first sequence including the preserved \ccc{id}-number. We
offer a specialized implementation
\ccc{CGAL::box_self_intersection_all_pairs} for this application.

\ccInclude{CGAL/box_intersection_d.h}

\ccThree{void}{box_inter}{}
%\def\ccLongParamLayout{\ccTrue}
\ccFunction{template< class ForwardIterator1, 
            class ForwardIterator2, 
            class Callback >
void box_intersection_all_pairs_d(
    ForwardIterator1 begin1, ForwardIterator1 end1,
    ForwardIterator2 begin2, ForwardIterator2 end2,
    Callback callback,
    Box_intersection_d::Topology topology = Box_intersection_d::CLOSED);
}{%
    Invocation of box intersection with default box traits
    \ccc{CGAL::Box_intersection_d::Box_traits_d<Box_handle>}, where
    \ccc{Box_handle} corresponds to the iterator value type  of
    \ccc{ForwardIterator1}.}
 
\ccFunction{template< class ForwardIterator1,
            class ForwardIterator2,
            class Callback, class BoxTraits >
void box_intersection_all_pairs_d(
    ForwardIterator1 begin1, ForwardIterator1 end1,
    ForwardIterator2 begin2, ForwardIterator2 end2,
    Callback callback,
    BoxTraits box_traits,
    Box_intersection_d::Topology topology = Box_intersection_d::CLOSED);
}{Invocation with custom box traits.}

%\begin{ccAdvanced}
%\def\ccLongParamLayout{\ccTrue}
%\ccFunction{template< class RandomAccessIterator1,
%                  class RandomAccessIterator2,
%                  class Callback, class BoxPredicateTraits >
%void box_intersection_all_pairs_custom_predicates_d(
%    RandomAccessIterator1 begin1, RandomAccessIterator1 end1,
%    RandomAccessIterator2 begin2, RandomAccessIterator2 end2,
%    Callback callback,
%    BoxPredicateTraits traits);
%}{%
%    This version of the algorithm is interface compatible with the
%    \ccc{CGAL::box_intersection_custom_predicates_d} function on page
%    \pageref{ccRef_CGAL::box_intersection_d}. Therefore,
%    this version of the box-intersection algorithms uses a very
%    fine-grained predicate traits class that allows the hand-crafted
%    implementation of the individual interval comparison tests. This is
%    only for experts and might only be useful in the context of special
%    number types or exploitation of special assembler instructions!
%    The details of half-open or closed boxes and the \ccc{id}-value of a
%    box are encoded in this predicate traits.
%}
%\def\ccLongParamLayout{\ccFalse}
%\end{ccAdvanced}

\ccRequirements

\begin{itemize}
  \item
    \ccc{ForwardIterator1}, and \ldots \ccc{2}, must be
    forward iterators and both value types must be
    the same. We call this value type \ccc{Box_handle} in the following.
  \item
    \ccc{Callback} must be of the \ccc{BinaryFunction} concept. 
    The \ccc{Box_handle} must be convertible to both argument types. The
    return type is not used and can be \ccc{void}.
  \item
    The \ccc{Box_handle} must be a model of the \ccc{Assignable} concept.
  \item
    In addition, if the default box traits is used the \ccc{Box_handle} must 
    be a class type \ccc{T} or a pointer to a class type \ccc{T}, where
    \ccc{T} must be a model of the \ccc{BoxIntersectionBox_d} concept.
    In both cases, the default box traits specializes to a suitable
    implementation.
  \item
    \ccc{BoxTraits} must be of the \ccc{BoxIntersectionTraits_d} concept.
%  \item
%    \ccc{BoxPredicateTraits} must be of the
%    \ccc{BoxIntersectionPredicateTraits_d} concept.
\end{itemize}

\ccSeeAlso

\ccRefIdfierPage{CGAL::box_intersection_d}\\
\ccRefIdfierPage{CGAL::box_self_intersection_d}\\
\ccRefIdfierPage{CGAL::box_self_intersection_all_pairs_d}

\ccRefIdfierPage{CGAL::Box_intersection_d::Box_traits_d<BoxHandle>}\\
\ccRefConceptPage{BoxIntersectionBox_d}\\
\ccRefConceptPage{BoxIntersectionTraits_d}

%\begin{ccAdvanced}
%
%\ccRefIdfierPage{CGAL::Box_intersection_d::Predicate_traits_d<BoxTraits,bool closed>}\\
%\ccRefConceptPage{BoxIntersectionPredicateTraits_d}
%
%\end{ccAdvanced}


\ccImplementation

The algorithm is trivially testing all pairs and runs therefore in time
$O(nm)$ where $n$ is the size of the first sequence and $m$ is the
size of the second sequence.

\end{ccRefFunction}

% +------------------------------------------------------------------------+
%%RefPage: end of main body, begin of footer
% EOF
% +------------------------------------------------------------------------+

