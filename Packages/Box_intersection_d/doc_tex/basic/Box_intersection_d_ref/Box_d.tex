% +------------------------------------------------------------------------+
% | Reference manual page: Box_d.tex
% +------------------------------------------------------------------------+


\ccRefPageBegin
\begin{ccRefClass}{Box_d<NT,unsigned int N>}

\ccDefinition
Default generic bounding box implementation. Subtypes of this class can be
used as a template parameter to \ccc{BoxTraits_d}.

\ccInclude{CGAL/Box_intersection_d/box_traits.h}
\ccTypes
\ccNestedType{NT}  {Number type of the box vector components.}

\ccCreation
\ccCreationVariable{box}

\ccConstructor{Box_d();}{Default constructor. Vector components remain uninitialized.}
\ccConstructor{Box_d(bool complete);}{Initializes to the complete or the empty space. If empty, all interval starting(end) points will be set to positive(negative) infinity.}
\ccConstructor{Box_d(NT lo[N], NT hi[N]);}{Explicitly initializes box intervals.}

\ccOperations
For all methods that require a dimension parameter, obviously \ccc{dim < get_dim()} must hold. Otherwise, results are undefined.

\ccMethod{void init (bool complete = false);}{see constructors.}
\ccMethod{void extend(NT point[N]);}{\ccPostcond box contains point.}

\ccMethod{NT get_lo(unsigned int dim) const;}{returns interval starting point in dimension dim.}
\ccMethod{NT get_hi(unsigned int dim) const;}{returns interval end point in dimension dim.}
\ccMethod{static T get_id(Box b);} {returns a unique box id. T must be less-than comparable.}
\ccMethod{static NT get_dim();}{returns N.}

\end{ccRefClass}


\ccRefPageEnd
