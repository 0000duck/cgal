% +------------------------------------------------------------------------+
% | Reference manual page: BoxTraits_d.tex
% +------------------------------------------------------------------------+
% +------------------------------------------------------------------------+

\ccRefPageBegin
\begin{ccRefConcept}{BoxTraits_d}
Bounding box access is exclusively done using a box traits class, which serves as an adapter. This adapter hides the specific interface a box may have.

\ccTypes
\ccNestedType{Box} {box handle type. Bounding box objects are passed to functions and predicates using exactly this type. Note: this may be different from the box class. For example, a handle can be \ccc{const Box&} or \ccc{Box*}.}
\ccNestedType{NT}  {Number type of the box vector components.}



\ccOperations
For all methods that require a dimension parameter, \ccc{dim < get_dim()} must hold. Otherwise, results are undefined.

\ccMethod{NT Traits::get_lo(Box b, unsigned int dim);} {returns low interval point of box in dimension index dim, starting at zero.}
\ccMethod{NT Traits::get_hi(Box b, unsigned int dim);} {returns high interval point of box in dimension index dim, starting at zero.}
\ccMethod{T Traits::get_id(Box b);} {returns a unique box id. T must be less-than comparable.}
\ccMethod{unsigned int Traits::get_dim();} {returns number of dimensions, starting at one.}

\ccHasModels
\ccRefIdfierPage{CGAL::Box_traits_d<Box>}

\end{ccRefConcept}

% +------------------------------------------------------------------------+
%%RefPage: end of main body, begin of footer
\ccRefPageEnd
% EOF
% +------------------------------------------------------------------------+
