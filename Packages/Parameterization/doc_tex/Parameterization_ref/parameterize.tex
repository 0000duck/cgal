% +------------------------------------------------------------------------+
% | Reference manual page: parameterize.tex
% +------------------------------------------------------------------------+
% | 23.08.2005   Author
% | Package: Parameterization
% | 
\RCSdef{\RCSparameterizeRev}{$Revision$}
\RCSdefDate{\RCSparameterizeDate}{$Date$}
% |
%%RefPage: end of header, begin of main body
% +------------------------------------------------------------------------+


\begin{ccRefFunction}{parameterize}  %% add template arg's if necessary

%% \ccHtmlCrossLink{}     %% add further rules for cross referencing links
%% \ccHtmlIndexC[function]{} %% add further index entries

\ccDefinition
  
The function \ccRefName\ generates the counterclockwise sequence of extreme
points from a given set of input points.


\ccInclude{CGAL/parameterize.h}

\ccFunction{template <class InputIterator, class OutputIterator, class Traits>
            OutputIterator
            parameterize( InputIterator  first,
                              InputIterator  beyond,
                              OutputIterator result,
                              const Traits & ch_traits = Default_traits);}
            {generates the counterclockwise sequence of extreme points
            of the points in the range [\ccc{first},\ccc{beyond}).
            The resulting sequence is placed starting at position
            \ccc{result}, and the past-the-end iterator for the resulting
            sequence is returned. It is not specified at which point the
            cyclic sequence of extreme points is cut into a linear sequence.
            \ccPrecond %\ccIndexSubitem[C]{parameterize}{preconditions}
            The source range [\ccc{first},\ccc{beyond}) does not contain
            \ccc{result}.}

The default traits class \ccc{Default_traits} is the kernel in which the
type \ccc{InputIterator::value_type} is defined.

\ccHeading{Requirements}
\begin{enumerate}
   \item    \ccc{InputIterator::value_type} and 
            \ccc{OutputIterator::value_type}
            are equivalent to \ccc{Traits::Point_2}.
   \item    \ccc{Traits} defines the following subset of types from
            the concept ConvexHullTraits\_2 and their corresponding member
            %\ccIndexMainItem[c]{MeshAdaptor_3}
            functions that return instances of these types:
            \begin{itemize}
                \item \ccc{Traits::Point_2},
                \item \ccc{Traits::Less_xy_2}, 
                \item \ccc{Traits::Left_turn_2},
		\item \ccc{Traits::Equal_2}.
            \end{itemize}
\end{enumerate}

\ccSeeAlso

\ccRefIdfierPage{CGAL::ch_akl_toussaint} \\
\ccRefIdfierPage{CGAL::ch_bykat} \\
\ccRefIdfierPage{CGAL::ch_eddy} \\
\ccRefIdfierPage{CGAL::ch_graham_andrew_scan} \\
\ccRefIdfierPage{CGAL::ch_jarvis} \\
\ccRefIdfierPage{CGAL::ch_melkman} \\
\ccRefIdfierPage{CGAL::parameterization} \\
\ccRefIdfierPage{CGAL::lower_hull_points_2} \\
\ccRefIdfierPage{CGAL::upper_hull_points_2} 

\ccImplementation
This function implements Andrew's variant of the Graham
scan algorithm \cite{a-aeach-79} and follows the presenation of Mehlhorn
\cite{m-mdscg-84}. This algorithm requires $O(n \log n)$ time 
in the worst case for $n$ input points.  

\end{ccRefFunction}

% +------------------------------------------------------------------------+
%%RefPage: end of main body, begin of footer
% EOF
% +------------------------------------------------------------------------+

