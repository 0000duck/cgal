% +------------------------------------------------------------------------+
% | Reference manual page: Vector.tex
% +------------------------------------------------------------------------+
% | 21.09.2005   Laurent Saboret, Pierre Alliez
% | Package: Parameterization
% | 
\RCSdef{\RCSVectorRev}{$Revision$}
\RCSdefDate{\RCSVectorDate}{$Date$}
% |
%%RefPage: end of header, begin of main body
% +------------------------------------------------------------------------+


\begin{ccRefConcept}[SparseLinearAlgebra_d::]{Vector}

%% \ccHtmlCrossLink{}     %% add further rules for cross referencing links
%% \ccHtmlIndexC[concept]{} %% add further index entries


\ccDefinition

% The section below is automatically generated. Do not edit!
%START-AUTO(\ccDefinition)

Concept SparseLinearAlgebraTraits\_d::Vector is a concept of a vector that can be multiplied by a sparse matrix.

%END-AUTO(\ccDefinition)

\ccRefines

% The section below is automatically generated. Do not edit!
%START-AUTO(\ccRefines)

This is a sub-concept of LinearAlgebraTraits\_d::Vector.

%END-AUTO(\ccRefines)


\ccTypes

% The section below is automatically generated. Do not edit!
%START-AUTO(\ccTypes)

\ccNestedType{NT}
{
}

%END-AUTO(\ccTypes)


\ccConstants

% The section below is automatically generated. Do not edit!
%START-AUTO(\ccConstants)

None.

%END-AUTO(\ccConstants)


\ccCreation
\ccCreationVariable{v}  %% variable name for \ccMethod

% The section below is automatically generated. Do not edit!
%START-AUTO(\ccCreation)

\ccConstructor{Vector (int rows);}
{
Create a vector initialized with zeros.
}
\ccConstructor{Vector (const Vector & toCopy);}
{
Copy constructor.
}

%END-AUTO(\ccCreation)


\ccOperations

% The section below is automatically generated. Do not edit!
%START-AUTO(\ccOperations)

\ccMethod{Vector& operator= (const Vector & toCopy);}
{
operator =()
}
\ccMethod{int dimension () const;}
{
Return the vector's number of coefficients.
}
\ccMethod{NT operator[] (int row) const;}
{
Read/write access to 1 vector coefficient.
Precondition: 0 $<$= row $<$ dimension().
}
\ccMethod{NT& operator[] (int row);}
{
}

%END-AUTO(\ccOperations)


\ccHasModels

% The section below is automatically generated. Do not edit!
%START-AUTO(\ccHasModels)

\begin{itemize}
\item Taucs\_vector\item OpenNL::FullVector \end{itemize}

%END-AUTO(\ccHasModels)


\ccSeeAlso

Some\_other\_concept,
\ccc{some_other_function}.


\ccExample

A short example program.
Instead of a short program fragment, a full running program can be
included using the
\verb|\ccIncludeExampleCode{Parameterization/Vector.C}|
macro. The program example would be part of the source code distribution and
also part of the automatic test suite.

\begin{ccExampleCode}
void your_example_code() {
}
\end{ccExampleCode}

%% \ccIncludeExampleCode{Parameterization/Vector.C}


\end{ccRefConcept}

% +------------------------------------------------------------------------+
%%RefPage: end of main body, begin of footer
% EOF
% +------------------------------------------------------------------------+

