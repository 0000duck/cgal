% +------------------------------------------------------------------------+
% | Reference manual page: Parametizer_traits_3.tex
% +------------------------------------------------------------------------+
% | 21.09.2005   Laurent Saboret, Pierre Alliez
% | Package: Parameterization
% |
\RCSdef{\RCSParametizertraitsRev}{$Revision$}
\RCSdefDate{\RCSParametizertraitsDate}{$Date$}
% |
%%RefPage: end of header, begin of main body
% +------------------------------------------------------------------------+


\begin{ccRefClass}{Parametizer_traits_3}  %% add template arg's if necessary

%% \ccHtmlCrossLink{}     %% add further rules for cross referencing links
%% \ccHtmlIndexC[class]{} %% add further index entries


\ccDefinition

% The section below is automatically generated. Do not edit!
%START-AUTO(\ccDefinition)

Class Parametizer\_traits\_3 is the base class of all parameterization methods. This class is a pure virtual class, thus cannot be instanciated.

This class doesn't do much. Its main goal is to ensure that suclasses will be proper models of the ParametizerTraits\_3 concept:\begin{itemize}
\item Parametizer\_traits\_3 defines the Error\_code list of errors detected by this package\item Parametizer\_traits\_3 declares a pure virtual method parameterize()\end{itemize}

%END-AUTO(\ccDefinition)

\ccInclude{CGAL/Parametizer_traits_3.h}


\ccIsModel

% The section below is automatically generated. Do not edit!
%START-AUTO(\ccIsModel)

Model of the ParametizerTraits\_3 concept (although you cannot instanciate this class).

%END-AUTO(\ccIsModel)


\ccHeading{Design pattern}

% The section below is automatically generated. Do not edit!
%START-AUTO(\ccHeading\{Design pattern\})

None.

%END-AUTO(\ccHeading\{Design pattern\})


\ccParameters

The full template declaration is:

% The section below is automatically generated. Do not edit!
%START-AUTO(\ccParameters)

template$<$
class MeshAdaptor\_3$>$ 
class Parametizer\_traits\_3;

%END-AUTO(\ccParameters)


\ccTypes

% The section below is automatically generated. Do not edit!
%START-AUTO(\ccTypes)

\ccNestedType{Adaptor}
{
Export MeshAdaptor\_3 template parameter.
}
\ccNestedType{NT}
{
Number type to represent coordinates.
}
\ccNestedType{Point_2}
{
2D point that represents (u,v) coordinates computed by parameterization methods. Usual methods are expected.
}
\ccNestedType{Point_3}
{
3D point that represents vertices coordinates. Usual methods are expected.
}
\ccNestedType{Vector_2}
{
2D vector. Usual methods are expected.
}
\ccNestedType{Vector_3}
{
3D vector. Usual methods are expected.
}
\ccNestedType{Facet}
{
Opaque type representing a facet of the 3D mesh. No methods are expected.
}
\ccNestedType{Facet_handle}
{
Handle to a facet. Model of the Handle concept.
}
\ccNestedType{Facet_const_handle}
{
}
\ccNestedType{Facet_iterator}
{
Iterator over all mesh facets. Model of the ForwardIterator concept.
}
\ccNestedType{Facet_const_iterator}
{
}
\ccNestedType{Vertex}
{
Opaque type representing a vertex of the 3D mesh. No methods are expected.
}
\ccNestedType{Vertex_handle}
{
Handle to a vertex. Model of the Handle concept.
}
\ccNestedType{Vertex_const_handle}
{
}
\ccNestedType{Vertex_iterator}
{
Iterator over all vertices of a mesh. Model of the ForwardIterator concept.
}
\ccNestedType{Vertex_const_iterator}
{
}
\ccNestedType{Border_vertex_iterator}
{
Iterator over vertices of the mesh "main boundary". Model of the ForwardIterator concept.
}
\ccNestedType{Border_vertex_const_iterator}
{
}
\ccNestedType{Vertex_around_facet_circulator}
{
Counter-clockwise circulator over a facet's vertices. Model of the BidirectionalCirculator concept.
}
\ccNestedType{Vertex_around_facet_const_circulator}
{
}
\ccNestedType{Vertex_around_vertex_circulator}
{
Clockwise circulator over the vertices incident to a vertex. Model of the BidirectionalCirculator concept.
}
\ccNestedType{Vertex_around_vertex_const_circulator}
{
}

%END-AUTO(\ccTypes)


\ccConstants

% The section below is automatically generated. Do not edit!
%START-AUTO(\ccConstants)

\ccEnum{enum Error_code {...}}
{
The list of errors detected by this package.
}
\begin{description}
\item[Enumerator: ]
\begin{description}
\item[OK
]\item[ERROR\_EMPTY\_MESH
]input mesh is empty \item[ERROR\_NON\_TRIANGULAR\_MESH
]input mesh is not triangular \item[ERROR\_NO\_SURFACE\_MESH
]input mesh is not a surface \item[ERROR\_INVALID\_BOUNDARY
]parameterization requires a convex border \item[ERROR\_BAD\_MATRIX\_CONDITIONING
]result is mathematically unstable \item[ERROR\_CANNOT\_SOLVE\_LINEAR\_SYSTEM
]cannot solve linear system \item[ERROR\_NO\_1\_TO\_1\_MAPPING
]parameterization does not ensure 1 to 1 mapping \item[ERROR\_NOT\_ENOUGH\_MEMORY
]it's time to buy some RAM :-) \item[ERROR\_WRONG\_PARAMETER
]a method received an unexpected parameter \end{description}
\end{description}

%END-AUTO(\ccConstants)


\ccCreation
\ccCreationVariable{p}  % variable name for \ccMethod

\ccConstructor{Parametizer_traits_3();}{default constructor.}

% The section below is automatically generated. Do not edit!
%START-AUTO(\ccCreation)



%END-AUTO(\ccCreation)


\ccOperations

% The section below is automatically generated. Do not edit!
%START-AUTO(\ccOperations)

\ccMethod{virtual Error_code parameterize (Adaptor * mesh);}
{
Compute a 1 to 1 mapping from a 3D surface 'mesh' to a piece of the 2D space. The mapping is linear by pieces (linear in each triangle). The result is the (u,v) pair image of each vertex of the 3D surface.
Preconditions:\begin{itemize}
\item 'mesh' must be a surface with 1 connected component\item 'mesh' must be a triangular mesh \end{itemize}
}
\ccMethod{double cotangent (const Point_3 & P, const Point_3 & Q, const Point_3 & R);}
{
return cotangent of (P,Q,R) corner (ie cotan of QP,QR angle)
}
\ccMethod{double tangent (const Point_3 & P, const Point_3 & Q, const Point_3 & R);}
{
return tangent of (P,Q,R) corner (ie tangent of QP,QR angle)
}
\ccMethod{static double compute_angle_rad (const Point_3 & P, const Point_3 & Q, const Point_3 & R);}
{
return angle (in radians) of of (P,Q,R) corner (ie QP,QR angle)
}

%END-AUTO(\ccOperations)


\ccSeeAlso

\ccRefIdfierPage{CGAL::Fixed_border_parametizer_3<Traits>}  \\
\ccRefIdfierPage{CGAL::Barycentric_mapping_parametizer_3<Traits>}  \\
\ccRefIdfierPage{CGAL::Discrete_authalic_parametizer_3<Traits>}  \\
\ccRefIdfierPage{CGAL::Discrete_conformal_map_parametizer_3<Traits>}  \\
\ccRefIdfierPage{CGAL::LSCM_parametizer_3<Traits>}  \\
\ccRefIdfierPage{CGAL::Mean_value_coordinates_parametizer_3<Traits>}  \\


\end{ccRefClass}

% +------------------------------------------------------------------------+
%%RefPage: end of main body, begin of footer
% EOF
% +------------------------------------------------------------------------+

