% +------------------------------------------------------------------------+
% | Reference manual page: Mesh_adaptor_patch_3.tex
% +------------------------------------------------------------------------+
% | 21.09.2005   Laurent Saboret, Pierre Alliez
% | Package: Parameterization
% |
\RCSdef{\RCSMeshadaptorpatchRev}{$Revision$}
\RCSdefDate{\RCSMeshadaptorpatchDate}{$Date$}
% |
%%RefPage: end of header, begin of main body
% +------------------------------------------------------------------------+


\begin{ccRefClass}{Mesh_adaptor_patch_3}  %% add template arg's if necessary

%% \ccHtmlCrossLink{}     %% add further rules for cross referencing links
%% \ccHtmlIndexC[class]{} %% add further index entries


\ccDefinition

% The section below is automatically generated. Do not edit!
%START-AUTO(\ccDefinition)

Mesh\_adaptor\_patch\_3 is an decorator class to virtually "cut" a patch in a PatchableMeshAdaptor\_3 3D surface. Only the patch is exported, making the 3D surface look like a topological disk.

The input mesh can be of any genus, but it has to come with a "seam" that describes the boundary of a topological disc. This boundary may be an actual border of the mesh or a virtual border.

%END-AUTO(\ccDefinition)

\ccInclude{CGAL/Mesh_adaptor_patch_3.h}


\ccIsModel

% The section below is automatically generated. Do not edit!
%START-AUTO(\ccIsModel)

Model of MeshAdaptor\_3 concept, whose purpose is to allow the parameterization package to access meshes on an uniform manner.

%END-AUTO(\ccIsModel)


\ccHeading{Design pattern}

% The section below is automatically generated. Do not edit!
%START-AUTO(\ccHeading\{Design pattern\})

Mesh\_adaptor\_patch\_3 is a Decorator (see [GOF95]): it changes the behavior of a PatchableMeshAdaptor\_3 3D surface while keeping its MeshAdaptor\_3 interface.

%END-AUTO(\ccHeading\{Design pattern\})


\ccParameters

The full template declaration is:

% The section below is automatically generated. Do not edit!
%START-AUTO(\ccParameters)

template$<$
class PatchableMeshAdaptor\_3$>$
class Mesh\_adaptor\_patch\_3;

%END-AUTO(\ccParameters)


\ccTypes

The following mutable handles, iterators, and circulators have appropriate
non-mutable counterparts, i.e. \ccc{const_handle},
\ccc{const_iterator}, and \ccc{const_circulator}. The mutable types are
assignable to their non-mutable counterparts.  Both circulators are
assignable to the \ccc{Vertex_iterator}. The iterators are
assignable to the respective handle types. Wherever the handles appear
in function parameter lists, the corresponding iterators can be used as
well.

% The section below is automatically generated. Do not edit!
%START-AUTO(\ccTypes)

\ccNestedType{Adaptor}
{
Export template parameter.
}
\ccNestedType{NT}
{
Number type to represent coordinates.
}
\ccNestedType{Point_2}
{
2D point that represents (u,v) coordinates computed by parameterization methods. Usual methods are expected.
}
\ccNestedType{Point_3}
{
3D point that represents vertices coordinates. Usual methods are expected.
}
\ccNestedType{Vector_2}
{
2D vector. Usual methods are expected.
}
\ccNestedType{Vector_3}
{
3D vector. Usual methods are expected.
}
\ccNestedType{Facet}
{
Opaque type representing a facet of the 3D mesh. No methods are expected.
}
\ccNestedType{Facet_handle}
{
Handle to a facet. Model of the Handle concept.
}
\ccNestedType{Facet_const_handle}
{
}
\ccNestedType{Facet_iterator}
{
Iterator over all mesh facets. Model of the ForwardIterator concept.
}
\ccNestedType{Facet_const_iterator}
{
}
\ccNestedType{Vertex}
{
Opaque type representing a vertex of the 3D mesh. No methods are expected.
}
\ccNestedType{Vertex_handle}
{
Handle to a vertex. Model of the Handle concept.
}
\ccNestedType{Vertex_const_handle}
{
}
\ccNestedType{Vertex_iterator}
{
Iterator over all vertices of a mesh. Model of the ForwardIterator concept.
}
\ccNestedType{Vertex_const_iterator}
{
}
\ccNestedType{Border_vertex_iterator}
{
Iterator over vertices of the mesh "main boundary". Model of the ForwardIterator concept.
}
\ccNestedType{Border_vertex_const_iterator}
{
}
\ccNestedType{Vertex_around_facet_circulator}
{
Counter-clockwise circulator over a facet's vertices Model of the BidirectionalCirculator concept.
}
\ccNestedType{Vertex_around_facet_const_circulator}
{
}
\ccNestedType{Vertex_around_vertex_circulator}
{
Clockwise circulator over the vertices incident to a vertex Model of the BidirectionalCirculator concept.
}
\ccNestedType{Vertex_around_vertex_const_circulator}
{
}

%END-AUTO(\ccTypes)


\ccConstants

% The section below is automatically generated. Do not edit!
%START-AUTO(\ccConstants)

None.

%END-AUTO(\ccConstants)


\ccCreation
\ccCreationVariable{mesh}  %% define variable name used by \ccMethod below

% The section below is automatically generated. Do not edit!
%START-AUTO(\ccCreation)

\ccConstructor{Mesh_adaptor_patch_3 (Adaptor * mesh, InputIterator first_seam_vertex, InputIterator end_seam_vertex);}
{
Create an decorator for an existing PatchableMeshAdaptor\_3 mesh The input mesh can be of any genus, but it has to come with a "seam" that describes the boundary of a topological disc. This boundary may be an actual border of the mesh or a virtual border.
Preconditions:\begin{itemize}
\item first\_seam\_vertex -$>$ end\_seam\_vertex defines the outer seam, ie Mesh\_adaptor\_patch\_3 will export the "right" of the seam\item the "seam" is given as a container of Adaptor::Vertex\_handle elements. \end{itemize}
}

%END-AUTO(\ccCreation)


\ccOperations

The following methods returning a mutable handle, iterator, or circulator have appropriate
non-mutable counterpart methods, i.e. \ccc{const}, returning a \ccc{const_handle},
\ccc{const_iterator}, or \ccc{const_circulator}.

% The section below is automatically generated. Do not edit!
%START-AUTO(\ccOperations)

\ccMethod{Adaptor* get_decorated_mesh ();}
{
Get the decorated mesh Using this method is NOT recommended
}
\ccMethod{const Adaptor* get_decorated_mesh () const;}
{
}
\ccMethod{Vertex_iterator mesh_vertices_begin ();}
{
Get iterator over first vertex of mesh.
}
\ccMethod{Vertex_const_iterator mesh_vertices_begin () const;}
{
}
\ccMethod{Vertex_iterator mesh_vertices_end ();}
{
Get iterator over past-the-end vertex of mesh.
}
\ccMethod{Vertex_const_iterator mesh_vertices_end () const;}
{
}
\ccMethod{int count_mesh_vertices () const;}
{
Count the number of vertices of the mesh.
}
\ccMethod{void index_mesh_vertices ();}
{
Index vertices of the mesh from 0 to count\_mesh\_vertices()-1.
}
\ccMethod{Border_vertex_iterator mesh_main_border_vertices_begin ();}
{
Get iterator over first vertex of mesh's main border (aka "seam").
}
\ccMethod{Border_vertex_const_iterator mesh_main_border_vertices_begin () const;}
{
}
\ccMethod{Border_vertex_iterator mesh_main_border_vertices_end ();}
{
Get iterator over past-the-end vertex of mesh's main border (aka "seam").
}
\ccMethod{Border_vertex_const_iterator mesh_main_border_vertices_end () const;}
{
}
\ccMethod{std::list<Vertex_handle> get_boundary (Vertex_handle seed_vertex);}
{
Return the boundary containing seed\_vertex Return an empty list if not found
}
\ccMethod{Facet_iterator mesh_facets_begin ();}
{
Get iterator over first facet of mesh.
}
\ccMethod{Facet_const_iterator mesh_facets_begin () const;}
{
}
\ccMethod{Facet_iterator mesh_facets_end ();}
{
Get iterator over past-the-end facet of mesh.
}
\ccMethod{Facet_const_iterator mesh_facets_end () const;}
{
}
\ccMethod{int count_mesh_facets () const;}
{
Count the number of facets of the mesh.
}
\ccMethod{bool is_mesh_triangular () const;}
{
Return true of all mesh's facets are triangles.
}
\ccMethod{int count_mesh_halfedges () const;}
{
Count the number of halfedges of the mesh.
}
\ccMethod{Vertex_around_facet_circulator facet_vertices_begin (Facet_handle facet);}
{
Get circulator over facet's vertices.
}
\ccMethod{Vertex_around_facet_const_circulator facet_vertices_begin (Facet_const_handle facet) const;}
{
}
\ccMethod{int count_facet_vertices (Facet_const_handle facet) const;}
{
Count the number of vertices of a facet.
}
\ccMethod{Point_3 get_vertex_position (Vertex_const_handle vertex) const;}
{
Get the 3D position of a vertex.
}
\ccMethod{Point_2 get_vertex_uv (Vertex_const_handle vertex) const;}
{
Get/set the 2D position (u/v pair) of a vertex. Default value is undefined.
}
\ccMethod{void set_vertex_uv (Vertex_handle vertex, const Point_2 & uv);}
{
}
\ccMethod{bool is_vertex_parameterized (Vertex_const_handle vertex) const;}
{
Get/set "is parameterized" field of vertex. Default value is undefined.
}
\ccMethod{void set_vertex_parameterized (Vertex_handle vertex, bool parameterized);}
{
}
\ccMethod{int get_vertex_index (Vertex_const_handle vertex) const;}
{
Get/set vertex index. Default value is undefined.
}
\ccMethod{void set_vertex_index (Vertex_handle vertex, int index);}
{
}
\ccMethod{int get_vertex_tag (Vertex_const_handle vertex) const;}
{
Get/set vertex' all purpose tag. Default value is undefined.
}
\ccMethod{void set_vertex_tag (Vertex_handle vertex, int tag);}
{
}
\ccMethod{bool is_vertex_on_border (Vertex_const_handle vertex) const;}
{
Return true if a vertex belongs to ANY mesh's boundary.
}
\ccMethod{bool is_vertex_on_main_border (Vertex_const_handle vertex) const;}
{
Return true if a vertex belongs to the UNIQUE mesh's main border set by the constructor
}
\ccMethod{Vertex_around_vertex_circulator vertices_around_vertex_begin (Vertex_handle vertex, Vertex_handle start_position = NULL);}
{
Get circulator over the vertices incident to 'vertex' 'start\_position' defines the optional initial position of the circulator
}
\ccMethod{Vertex_around_vertex_const_circulator vertices_around_vertex_begin (Vertex_const_handle vertex, Vertex_const_handle start_position = NULL) const;}
{
}

%END-AUTO(\ccOperations)


\ccSeeAlso

\ccRefIdfierPage{CGAL::Mesh_adaptor_polyhedron_3}  \\


\ccExample

This example program instantiates a polyhedron and a polyhedron adaptor. If the mesh has no border, it
virtually "cuts" the mesh and stores the result as a Mesh\_adaptor\_patch\_3 decorator.
Eventually, it parameterizes the polyhedron using the default algorithm
(Floater's mean value coordinates).

\ccIncludeExampleCode{Parameterization/Polyhedron_parameterization5.C}


\end{ccRefClass}

% +------------------------------------------------------------------------+
%%RefPage: end of main body, begin of footer
% EOF
% +------------------------------------------------------------------------+

