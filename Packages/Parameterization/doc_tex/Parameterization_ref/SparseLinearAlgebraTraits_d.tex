% +------------------------------------------------------------------------+
% | Reference manual page: SparseLinearAlgebraTraits_d.tex
% +------------------------------------------------------------------------+
% | 21.09.2005   Laurent Saboret, Pierre Alliez
% | Package: Parameterization
% | 
\RCSdef{\RCSSparseLinearAlgebraTraitsdRev}{$Revision$}
\RCSdefDate{\RCSSparseLinearAlgebraTraitsdDate}{$Date$}
% |
%%RefPage: end of header, begin of main body
% +------------------------------------------------------------------------+


\begin{ccRefConcept}{SparseLinearAlgebraTraits_d}

%% \ccHtmlCrossLink{}     %% add further rules for cross referencing links
%% \ccHtmlIndexC[concept]{} %% add further index entries


\ccDefinition

% The section below is automatically generated. Do not edit!
%START-AUTO(\ccDefinition)

The concept SparseLinearAlgebraTraits\_d is used to solve sparse linear systems "A$\ast$X = B".

%END-AUTO(\ccDefinition)


\ccRefines

% The section below is automatically generated. Do not edit!
%START-AUTO(\ccRefines)

This is a sub-concept of LinearAlgebraTraits\_d.

%END-AUTO(\ccRefines)


\ccTypes

% The section below is automatically generated. Do not edit!
%START-AUTO(\ccTypes)

\ccNestedType{Matrix}
{
}
\ccNestedType{Vector}
{
}
\ccNestedType{NT}
{
}

%END-AUTO(\ccTypes)


\ccConstants

% The section below is automatically generated. Do not edit!
%START-AUTO(\ccConstants)

None.

%END-AUTO(\ccConstants)


\ccCreation
\ccCreationVariable{sparse_LA}  % variable name for \ccMethod

% The section below is automatically generated. Do not edit!
%START-AUTO(\ccCreation)

\ccConstructor{SparseLinearAlgebraTraits_d ();}
{
Default constructor.
}

%END-AUTO(\ccCreation)


\ccOperations

% The section below is automatically generated. Do not edit!
%START-AUTO(\ccOperations)

\ccMethod{bool linear_solver (const Matrix & A, const Vector & B, Vector & X, NT & D);}
{
Solve the sparse linear system "A$\ast$X = B". Return true on success. The solution is then (1/D) $\ast$ X.
Preconditions:\begin{itemize}
\item A.row\_dimension() == B.dimension()\item A.column\_dimension() == X.dimension() \end{itemize}
}
\ccMethod{bool is_solvable (const Matrix & A, const Vector & B);}
{
Indicate if the linear system can be solved and if the matrix conditioning is good.
Preconditions:\begin{itemize}
\item A.row\_dimension() == B.dimension() \end{itemize}
}

%END-AUTO(\ccOperations)


\ccHasModels

% The section below is automatically generated. Do not edit!
%START-AUTO(\ccHasModels)
%END-AUTO(\ccHasModels)


\ccSeeAlso

Some\_other\_concept,
\ccc{some_other_function}.


\ccExample

A short example program.
Instead of a short program fragment, a full running program can be
included using the
\verb|\ccIncludeExampleCode{Parameterization/SparseLinearAlgebraTraits_d.C}|
macro. The program example would be part of the source code distribution and
also part of the automatic test suite.

\begin{ccExampleCode}
void your_example_code() {
}
\end{ccExampleCode}

%% \ccIncludeExampleCode{Parameterization/SparseLinearAlgebraTraits_d.C}


\end{ccRefConcept}

% +------------------------------------------------------------------------+
%%RefPage: end of main body, begin of footer
% EOF
% +------------------------------------------------------------------------+

