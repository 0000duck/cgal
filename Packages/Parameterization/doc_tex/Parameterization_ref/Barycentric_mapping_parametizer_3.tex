% +------------------------------------------------------------------------+
% | Reference manual page: Barycentric_mapping_parametizer_3.tex
% +------------------------------------------------------------------------+
% | 23.08.2005   Laurent Saboret, Pierre Alliez
% | Package: Parameterization
% |
\RCSdef{\RCSBarycentricmappingparametizerRev}{$Revision$}
\RCSdefDate{\RCSBarycentricmappingparametizerDate}{$Date$}
% |
%%RefPage: end of header, begin of main body
% +------------------------------------------------------------------------+


\begin{ccRefClass}{Barycentric_mapping_parametizer_3}  %% add template arg's if necessary

%% \ccHtmlCrossLink{}     %% add further rules for cross referencing links
%% \ccHtmlIndexC[class]{} %% add further index entries


\ccDefinition

As all parameterization algorithms of the package, Barycentric\_mapping\_parametizer\_3
class is usually called via the global function parameterize().

% The section below is automatically generated. Do not edit!
%START-AUTO(\ccDefinition)

Class Barycentric\_mapping\_parametizer\_3 implements Tutte's Barycentric Mapping. 1 to 1 mapping is guaranteed if surface's border is mapped to a convex polygon.

The superclass Fixed\_border\_parametizer\_3 implements most of the parameterization algorithm. Barycentric\_mapping\_parametizer\_3:\begin{itemize}
\item provides default BorderParametizer\_3 and SparseLinearAlgebraTraits\_d template parameters that make sense.\item implements compute\_wij() to compute wij = (i,j) coefficient of matrix A for j neighbor vertex of i based on Tutte's barycentric mapping method.\item implements a tailored version of is\_one\_to\_one\_mapping().\end{itemize}

%END-AUTO(\ccDefinition)

\ccInclude{CGAL/Barycentric_mapping_parametizer_3.h}


\ccIsModel

% The section below is automatically generated. Do not edit!
%START-AUTO(\ccIsModel)

Model of the ParametizerTraits\_3 concept.

%END-AUTO(\ccIsModel)


\ccHeading{Design pattern}

% The section below is automatically generated. Do not edit!
%START-AUTO(\ccHeading\{Design pattern\})

None.

%END-AUTO(\ccHeading\{Design pattern\})


\ccParameters

The full template declaration is:

% The section below is automatically generated. Do not edit!
%START-AUTO(\ccParameters)

template$<$
class MeshAdaptor\_3, 
class BorderParametizer\_3 = Circular\_border\_arc\_length\_parametizer\_3$<$MeshAdaptor\_3$>$, 
class SparseLinearAlgebraTraits\_d = OpenNL::DefaultLinearSolverTraits$<$typename MeshAdaptor\_3::NT$>$$>$ 
class Barycentric\_mapping\_parametizer\_3;

%END-AUTO(\ccParameters)


\ccTypes

% The section below is automatically generated. Do not edit!
%START-AUTO(\ccTypes)

\ccNestedType{Solver}
{
}

%END-AUTO(\ccTypes)


\ccConstants

% The section below is automatically generated. Do not edit!
%START-AUTO(\ccConstants)

None.

%END-AUTO(\ccConstants)


\ccCreation
\ccCreationVariable{p}  %% variable name used by \ccMethod below

% The section below is automatically generated. Do not edit!
%START-AUTO(\ccCreation)

\ccConstructor{Barycentric_mapping_parametizer_3 (Border_param border_param = Border_param(), Sparse_LA sparse_la = Sparse_LA());}
{
Constructor.
}
\begin{description}
\item[Parameters: ]
\begin{description}
\item[border\_param]Object that maps the surface's border to 2D space \item[sparse\_la]Traits object to access a sparse linear system \end{description}
\end{description}

%END-AUTO(\ccCreation)


\ccOperations

% The section below is automatically generated. Do not edit!
%START-AUTO(\ccOperations)

\ccMethod{virtual NT compute_wij (const Adaptor & mesh, Vertex_const_handle main_vertex_Vi, Vertex_around_vertex_const_circulator neighbor_vertex_Vj);}
{
compute wij = (i,j) coefficient of matrix A for j neighbor vertex of i
Tutte algorithm is the most simple one: Wij = 1 for j neighbor vertex of i
}
\ccMethod{virtual bool is_one_to_one_mapping (const Adaptor & mesh, const Matrix & A, const Vector & Bu, const Vector & Bv);}
{
Check if 3D -$>$ 2D mapping is 1 to 1.
Theorem: 1 to 1 mapping is guaranteed if all Wij coefficients are $>$ 0 (for j vertex neighbor of i) and if the surface boundary is mapped onto a 2D convex polygon. All Wij coefficients = 1 (for j vertex neighbor of i), thus mapping is guaranteed if the surface boundary is mapped onto a 2D convex polygon.
}

%END-AUTO(\ccOperations)


\ccSeeAlso

\ccRefIdfierPage{CGAL::Parametizer_traits_3<Traits>}  \\
\ccRefIdfierPage{CGAL::Fixed_border_parametizer_3<Traits>}  \\
\ccRefIdfierPage{CGAL::Discrete_authalic_parametizer_3<Traits>}  \\
\ccRefIdfierPage{CGAL::Discrete_conformal_map_parametizer_3<Traits>}  \\
\ccRefIdfierPage{CGAL::LSCM_parametizer_3<Traits>}  \\
\ccRefIdfierPage{CGAL::Mean_value_coordinates_parametizer_3<Traits>}  \\


\ccExample

\begin{ccExampleCode}

// CGAL kernel
typedef CGAL::Cartesian<double>                         Kernel;

// Mesh true type and parameterization adaptor
typedef CGAL::Polyhedron_3<Kernel>                      Polyhedron;
typedef CGAL::Mesh_adaptor_polyhedron_3<Polyhedron>     Mesh_adaptor_polyhedron;

// Tutte's barycentric mapping
typedef CGAL::Barycentric_mapping_parametizer_3<Mesh_adaptor_polyhedron>
                                                        Parametizer;

int main(int argc,char * argv[])
{
    Polyhedron mesh;
    ...

    // The parameterization package needs an adaptor to handle Polyhedron_3 meshes
    Mesh_adaptor_polyhedron mesh_adaptor(&mesh);

    Parametizer::Error_code err = CGAL::parameterize(&mesh_adaptor, Parametizer());
    ...
}

\end{ccExampleCode}


\end{ccRefClass}

% +------------------------------------------------------------------------+
%%RefPage: end of main body, begin of footer
% EOF
% +------------------------------------------------------------------------+

