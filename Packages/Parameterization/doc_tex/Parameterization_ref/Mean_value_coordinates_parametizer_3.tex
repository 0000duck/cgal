% +------------------------------------------------------------------------+
% | Reference manual page: Mean_value_coordinates_parametizer_3.tex
% +------------------------------------------------------------------------+
% | 23.08.2005   Author
% | Package: Parameterization
% | 
\RCSdef{\RCSMeanvaluecoordinatesparametizerRev}{$Revision$}
\RCSdefDate{\RCSMeanvaluecoordinatesparametizerDate}{$Date$}
% |
%%RefPage: end of header, begin of main body
% +------------------------------------------------------------------------+


\begin{ccRefClass}{Mean_value_coordinates_parametizer_3}  %% add template arg's if necessary

%% \ccHtmlCrossLink{}     %% add further rules for cross referencing links
%% \ccHtmlIndexC[class]{} %% add further index entries

\ccDefinition
  

Class Mean\_value\_coordinates\_parametizer\_3 implements Floater's Mean Value Coordinates parameterization. 1 to 1 mapping is guaranteed if surface's border is mapped onto a convex polygon. This is a conformal parameterization, i.e. it attempts to preserve angles.

\ccInclude{Mean_value_coordinates_parametizer_3.h}

\ccIsModel

Model of the ParametizerTraits\_3 concept.



\ccParameters

The full template declaration is:

template$<$
class MeshAdaptor\_3, 
class BorderParametizer\_3 = Circular\_border\_arc\_length\_parametizer\_3$<$MeshAdaptor\_3$>$, 
class SparseLinearAlgebraTraits\_d = OpenNL::DefaultLinearSolverTraits$<$typename MeshAdaptor\_3::NT$>$$>$ 
class Mean\_value\_coordinates\_parametizer\_3;


\ccTypes


\ccNestedType{Adaptor}
{
}
\ccNestedType{Error_code}
{
The various errors detected by this package.
}
\ccNestedType{NT}
{
}
\ccNestedType{Facet_handle}
{
}
\ccNestedType{Facet_const_handle}
{
}
\ccNestedType{Vertex_handle}
{
}
\ccNestedType{Vertex_const_handle}
{
}
\ccNestedType{Point_3}
{
}
\ccNestedType{Point_2}
{
}
\ccNestedType{Vector_3}
{
}
\ccNestedType{Vector_2}
{
}
\ccNestedType{Facet_iterator}
{
}
\ccNestedType{Facet_const_iterator}
{
}
\ccNestedType{Vertex_iterator}
{
}
\ccNestedType{Vertex_const_iterator}
{
}
\ccNestedType{Border_vertex_iterator}
{
}
\ccNestedType{Border_vertex_const_iterator}
{
}
\ccNestedType{Vertex_around_facet_circulator}
{
}
\ccNestedType{Vertex_around_facet_const_circulator}
{
}
\ccNestedType{Vertex_around_vertex_circulator}
{
}
\ccNestedType{Vertex_around_vertex_const_circulator}
{
}
\ccNestedType{Border_param}
{
}
\ccNestedType{Sparse_LA}
{
}
\ccNestedType{Vector}
{
}
\ccNestedType{Matrix}
{
}
\ccNestedType{Solver}
{
}


\ccCreation
\ccCreationVariable{p}  %% choose variable name

\ccConstructor{Mean_value_coordinates_parametizer_3 (Border_param border_param = Border_param(), Sparse_LA sparse_la = Sparse_LA());}
{
Constructor.
}


\ccOperations

\ccMethod{virtual NT compute_wij (const Adaptor & mesh, Vertex_const_handle main_vertex_Vi, Vertex_around_vertex_const_circulator neighbor_vertex_Vj);}
{
compute wij = (i,j) coefficient of matrix A for j neighbor vertex of i
}
\ccMethod{virtual bool is_one_to_one_mapping (const Adaptor & mesh, const Matrix & A, const Vector & Bu, const Vector & Bv);}
{
Check if 3D -$>$ 2D mapping is 1 to 1.
Theorem: 1 to 1 mapping is guaranteed if all Wij coefficients are $>$ 0 (for j vertex neighbor of i) and if the surface boundary is mapped onto a 2D convex polygon. Floater formula above implies that Wij $>$ 0 (for j vertex neighbor of i), thus mapping is guaranteed if the surface boundary is mapped onto a 2D convex polygon
}


\ccSeeAlso

\ccc{Some_other_class},
\ccc{some_other_function}.

\ccExample

A short example program.
Instead of a short program fragment, a full running program can be
included using the 
\verb|\ccIncludeExampleCode{Parameterization/Mean_value_coordinates_parametizer_3.C}| 
macro. The program example would be part of the source code distribution and
also part of the automatic test suite.

\begin{ccExampleCode}
void your_example_code() {
}
\end{ccExampleCode}

%% \ccIncludeExampleCode{Parameterization/Mean_value_coordinates_parametizer_3.C}

\end{ccRefClass}

% +------------------------------------------------------------------------+
%%RefPage: end of main body, begin of footer
% EOF
% +------------------------------------------------------------------------+

