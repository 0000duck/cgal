% +------------------------------------------------------------------------+
% | Reference manual page: Mesh_adaptor_feature_extractor.tex
% +------------------------------------------------------------------------+
% | 21.09.2005   Laurent Saboret, Pierre Alliez
% | Package: Parameterization
% |
\RCSdef{\RCSMeshadaptorfeatureextractorRev}{$Revision$}
\RCSdefDate{\RCSMeshadaptorfeatureextractorDate}{$Date$}
% |
%%RefPage: end of header, begin of main body
% +------------------------------------------------------------------------+


\begin{ccRefClass}{Mesh_adaptor_feature_extractor}  %% add template arg's if necessary

%% \ccHtmlCrossLink{}     %% add further rules for cross referencing links
%% \ccHtmlIndexC[class]{} %% add further index entries


\ccDefinition

% The section below is automatically generated. Do not edit!
%START-AUTO(\ccDefinition)

Class Mesh\_adaptor\_feature\_extractor computes features (genus, boundaries, ...) of a 3D surface model of the MeshAdaptor\_3 concept.

%END-AUTO(\ccDefinition)

\ccInclude{CGAL/Mesh_adaptor_feature_extractor.h}


\ccParameters

The full template declaration is:

% The section below is automatically generated. Do not edit!
%START-AUTO(\ccParameters)

template$<$
class MeshAdaptor\_3$>$ 
class Mesh\_adaptor\_feature\_extractor;

%END-AUTO(\ccParameters)


\ccTypes

% The section below is automatically generated. Do not edit!
%START-AUTO(\ccTypes)

\ccNestedType{Adaptor}
{
}
\ccNestedType{NT}
{
}
\ccNestedType{Facet_handle}
{
}
\ccNestedType{Facet_const_handle}
{
}
\ccNestedType{Vertex_handle}
{
}
\ccNestedType{Vertex_const_handle}
{
}
\ccNestedType{Point_3}
{
}
\ccNestedType{Point_2}
{
}
\ccNestedType{Vector_3}
{
}
\ccNestedType{Vector_2}
{
}
\ccNestedType{Facet_iterator}
{
}
\ccNestedType{Facet_const_iterator}
{
}
\ccNestedType{Vertex_iterator}
{
}
\ccNestedType{Vertex_const_iterator}
{
}
\ccNestedType{Border_vertex_iterator}
{
}
\ccNestedType{Border_vertex_const_iterator}
{
}
\ccNestedType{Vertex_around_facet_circulator}
{
}
\ccNestedType{Vertex_around_facet_const_circulator}
{
}
\ccNestedType{Vertex_around_vertex_circulator}
{
}
\ccNestedType{Vertex_around_vertex_const_circulator}
{
}
\ccNestedType{Boundary}
{
Mesh boundary.
}
\ccNestedType{Skeleton}
{
List of all boundaries of a mesh.
}

%END-AUTO(\ccTypes)


\ccConstants

% The section below is automatically generated. Do not edit!
%START-AUTO(\ccConstants)

None.

%END-AUTO(\ccConstants)


\ccCreation
\ccCreationVariable{a}  % choose variable name for \ccMethod

% The section below is automatically generated. Do not edit!
%START-AUTO(\ccCreation)

\ccConstructor{Mesh_adaptor_feature_extractor (Adaptor * mesh);}
{
Constructor
CAUTION: Caller must NOT modify 'mesh' during the Mesh\_adaptor\_feature\_extractor life cycle.
}
\ccMethod{virtual ~Mesh_adaptor_feature_extractor ();}
{
}

%END-AUTO(\ccCreation)


\ccOperations

% The section below is automatically generated. Do not edit!
%START-AUTO(\ccOperations)

\ccMethod{int get_nb_boundaries ();}
{
Get number of boundaries.
}
\ccMethod{const Skeleton& get_boundaries ();}
{
Get extracted boundaries The longest boundary is the first one
}
\ccMethod{const Boundary* get_longest_boundary ();}
{
Get longest boundary.
}
\ccMethod{int get_nb_connex_components ();}
{
Get \# of connected components.
}
\ccMethod{int get_genus ();}
{
Get the genus.
}

%END-AUTO(\ccOperations)


\ccSeeAlso

\ccRefConceptPage{MeshAdaptor_3}  \\


\ccExample

This example program instantiates a polyhedron and a polyhedron adaptor. If the mesh has no border, it
virtually "cuts" the mesh before parameterizing it.
Mesh\_adaptor\_feature\_extractor is used to compute the genus and extract the mesh's boundaries.

\ccIncludeExampleCode{Parameterization/Polyhedron_parameterization5.C}


\end{ccRefClass}

% +------------------------------------------------------------------------+
%%RefPage: end of main body, begin of footer
% EOF
% +------------------------------------------------------------------------+

