% +------------------------------------------------------------------------+
% | Reference manual page: Square_border_uniform_parametizer_3.tex
% +------------------------------------------------------------------------+
% | 21.09.2005   Laurent Saboret, Pierre Alliez
% | Package: Parameterization
% |
\RCSdef{\RCSSquareborderuniformparametizerRev}{$Revision$}
\RCSdefDate{\RCSSquareborderuniformparametizerDate}{$Date$}
% |
%%RefPage: end of header, begin of main body
% +------------------------------------------------------------------------+


\begin{ccRefClass}{Square_border_uniform_parametizer_3}  %% add template arg's if necessary

%% \ccHtmlCrossLink{}     %% add further rules for cross referencing links
%% \ccHtmlIndexC[class]{} %% add further index entries


\ccDefinition

% The section below is automatically generated. Do not edit!
%START-AUTO(\ccDefinition)

This class parameterizes the border of a 3D surface onto a square on an uniform manner: points are equally spaced. Square\_border\_parametizer\_3 implements most of the boudary parameterization algorithm. This class implements only compute\_edge\_length() to compute a segment's length.

%END-AUTO(\ccDefinition)

\ccInclude{CGAL/Square_border_parametizer_3.h}


\ccIsModel

% The section below is automatically generated. Do not edit!
%START-AUTO(\ccIsModel)

Model of the BorderParametizer\_3 concept.

%END-AUTO(\ccIsModel)


\ccHeading{Design pattern}

% The section below is automatically generated. Do not edit!
%START-AUTO(\ccHeading\{Design pattern\})

BorderParametizer\_3 models are Strategies (see [GOF95]): they implement a strategy of boundary parameterization for models of MeshAdaptor\_3

%END-AUTO(\ccHeading\{Design pattern\})


\ccParameters

The full template declaration is:

% The section below is automatically generated. Do not edit!
%START-AUTO(\ccParameters)

template$<$
class MeshAdaptor\_3$>$ 
class Square\_border\_uniform\_parametizer\_3;

%END-AUTO(\ccParameters)


\ccTypes

% The section below is automatically generated. Do not edit!
%START-AUTO(\ccTypes)

None.

%END-AUTO(\ccTypes)


\ccConstants

% The section below is automatically generated. Do not edit!
%START-AUTO(\ccConstants)

None.

%END-AUTO(\ccConstants)


\ccCreation
\ccCreationVariable{bp}  % variable name for \ccMethod

\ccConstructor{Square_border_uniform_parametizer_3();}{default constructor.}

% The section below is automatically generated. Do not edit!
%START-AUTO(\ccCreation)



%END-AUTO(\ccCreation)


\ccOperations

% The section below is automatically generated. Do not edit!
%START-AUTO(\ccOperations)

\ccMethod{virtual double compute_edge_length (const Adaptor & mesh, Vertex_const_handle source, Vertex_const_handle target);}
{
Compute length of an edge.
Uniform boundary parameterization: points are equally spaced.
}

%END-AUTO(\ccOperations)


\ccSeeAlso

\ccRefIdfierPage{CGAL::Circular_border_arc_length_parametizer_3}  \\
\ccRefIdfierPage{CGAL::Circular_border_uniform_parametizer_3}  \\
\ccRefIdfierPage{CGAL::Square_border_arc_length_parametizer_3}  \\
\ccRefIdfierPage{CGAL::Two_vertices_parametizer_3}  \\


\ccExample

\begin{ccExampleCode}

// CGAL kernel
typedef CGAL::Cartesian<double>                         Kernel;

// Mesh true type and parameterization adaptor
typedef CGAL::Polyhedron_3<Kernel>                      Polyhedron;
typedef CGAL::Mesh_adaptor_polyhedron_3<Polyhedron>     Mesh_adaptor_polyhedron;

// Square uniform boundary parameterizer
typedef CGAL::Square_border_uniform_parametizer_3<Mesh_adaptor_polyhedron>
                                                  Border_parametizer;

// Floater's mean value coordinates parametizer
// with Square uniform boundary parameterization
typedef CGAL::Mean_value_coordinates_parametizer_3<Mesh_adaptor_polyhedron,
                                                   Border_parametizer>
                                                        Parametizer;

int main(int argc,char * argv[])
{
    Polyhedron mesh;
    ...

    // The parameterization package needs an adaptor to handle Polyhedron_3 meshes
    Mesh_adaptor_polyhedron mesh_adaptor(&mesh);

    Parametizer::Error_code err = CGAL::parameterize(&mesh_adaptor, Parametizer());
    ...
}

\end{ccExampleCode}


\end{ccRefClass}

% +------------------------------------------------------------------------+
%%RefPage: end of main body, begin of footer
% EOF
% +------------------------------------------------------------------------+

