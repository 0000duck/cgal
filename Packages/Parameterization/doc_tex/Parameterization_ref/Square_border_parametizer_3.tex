% +------------------------------------------------------------------------+
% | Reference manual page: Square_border_parametizer_3.tex
% +------------------------------------------------------------------------+
% | 28.09.2005   Laurent Saboret, Pierre Alliez
% | Package: Parameterization
% | 
\RCSdef{\RCSSquareborderparametizerRev}{$Revision$}
\RCSdefDate{\RCSSquareborderparametizerDate}{$Date$}
% |
%%RefPage: end of header, begin of main body
% +------------------------------------------------------------------------+


\begin{ccRefClass}{Square_border_parametizer_3}  %% add template arg's if necessary

%% \ccHtmlCrossLink{}     %% add further rules for cross referencing links
%% \ccHtmlIndexC[class]{} %% add further index entries


\ccDefinition

% The section below is automatically generated. Do not edit!
%START-AUTO(\ccDefinition)

Base class of strategies that parameterize the border of a 3D surface onto a square.

Implementation note: To simplify the implementation, BorderParametizer\_3 models know only the MeshAdaptor\_3 class. They don't know the parameterization algorithm requirements nor the kind of sparse linear system used.

%END-AUTO(\ccDefinition)

\ccInclude{CGAL/Square_border_parametizer_3.h}


\ccIsModel

% The section below is automatically generated. Do not edit!
%START-AUTO(\ccIsModel)

Model of the BorderParametizer\_3 concept.

%END-AUTO(\ccIsModel)


\ccHeading{Design pattern}

% The section below is automatically generated. Do not edit!
%START-AUTO(\ccHeading{Design pattern})

BorderParametizer\_3 models are Strategies (see [GOF95]): they implement a strategy of boundary parameterization for models of MeshAdaptor\_3

%END-AUTO(\ccHeading{Design pattern})


\ccParameters

The full template declaration is:

% The section below is automatically generated. Do not edit!
%START-AUTO(\ccParameters)

template$<$
class MeshAdaptor\_3$>$ 
class Square\_border\_parametizer\_3;

%END-AUTO(\ccParameters)


\ccTypes

% The section below is automatically generated. Do not edit!
%START-AUTO(\ccTypes)

\ccNestedType{Adaptor}
{
Export MeshAdaptor\_3 template parameter.
}
\ccNestedType{Error_code}
{
The various errors detected by this package.
}
\ccNestedType{NT}
{
Number type to represent coordinates.
}
\ccNestedType{Point_2}
{
2D point that represents (u,v) coordinates computed by parameterization methods. Usual methods are expected.
}
\ccNestedType{Point_3}
{
3D point that represents vertices coordinates. Usual methods are expected.
}
\ccNestedType{Vector_2}
{
2D vector. Usual methods are expected.
}
\ccNestedType{Vector_3}
{
3D vector. Usual methods are expected.
}
\ccNestedType{Facet}
{
Opaque type representing a facet of the 3D mesh. No methods are expected.
}
\ccNestedType{Facet_handle}
{
Handle to a facet. Model of the Handle concept.
}
\ccNestedType{Facet_const_handle}
{
}
\ccNestedType{Facet_iterator}
{
Iterator over all mesh facets. Model of the ForwardIterator concept.
}
\ccNestedType{Facet_const_iterator}
{
}
\ccNestedType{Vertex}
{
Opaque type representing a vertex of the 3D mesh. No methods are expected.
}
\ccNestedType{Vertex_handle}
{
Handle to a vertex. Model of the Handle concept.
}
\ccNestedType{Vertex_const_handle}
{
}
\ccNestedType{Vertex_iterator}
{
Iterator over all vertices of a mesh. Model of the ForwardIterator concept.
}
\ccNestedType{Vertex_const_iterator}
{
}
\ccNestedType{Border_vertex_iterator}
{
Iterator over vertices of the mesh "main boundary". Model of the ForwardIterator concept.
}
\ccNestedType{Border_vertex_const_iterator}
{
}
\ccNestedType{Vertex_around_facet_circulator}
{
Counter-clockwise circulator over a facet's vertices Model of the BidirectionalCirculator concept.
}
\ccNestedType{Vertex_around_facet_const_circulator}
{
}
\ccNestedType{Vertex_around_vertex_circulator}
{
Clockwise circulator over the vertices incident to a vertex Model of the BidirectionalCirculator concept.
}
\ccNestedType{Vertex_around_vertex_const_circulator}
{
}

%END-AUTO(\ccTypes)


\ccConstants

% The section below is automatically generated. Do not edit!
%START-AUTO(\ccConstants)

None.

%END-AUTO(\ccConstants)


\ccCreation
\ccCreationVariable{a}  % choose variable name for \ccMethod

\ccConstructor{Square_border_parametizer_3();}{default constructor.}

% The section below is automatically generated. Do not edit!
%START-AUTO(\ccCreation)

\ccMethod{virtual ~Square_border_parametizer_3 ();}
{
Destructor of base class should be virtual.
}

%END-AUTO(\ccCreation)


\ccOperations

% The section below is automatically generated. Do not edit!
%START-AUTO(\ccOperations)

\ccMethod{Parametizer_traits_3< Adaptor >::Error_code parameterize_border (Adaptor * mesh);}
{
Assign to mesh's border vertices a 2D position (ie a (u,v) pair) on border's shape. Mark them as "parameterized".
}
\ccMethod{bool is_border_convex ();}
{
Indicate if border's shape is convex.
}
\ccMethod{virtual double compute_edge_length (const Adaptor & mesh, Vertex_const_handle source, Vertex_const_handle target);}
{
compute length of an edge
}

%END-AUTO(\ccOperations)


\ccSeeAlso

\ccRefIdfierPage{CGAL::Square_border_arc_length_parametizer_3<Traits>}  \\
\ccRefIdfierPage{CGAL::Square_border_uniform_parametizer_3<Traits>}  \\


\end{ccRefClass}

% +------------------------------------------------------------------------+
%%RefPage: end of main body, begin of footer
% EOF
% +------------------------------------------------------------------------+

