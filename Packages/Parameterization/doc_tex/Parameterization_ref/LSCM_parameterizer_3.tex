% +------------------------------------------------------------------------+
% | Reference manual page: LSCM_parameterizer_3.tex
% +------------------------------------------------------------------------+
% | 21.09.2005   Laurent Saboret, Pierre Alliez
% | Package: Parameterization
% |
\RCSdef{\RCSLSCMparameterizerRev}{$Revision$}
\RCSdefDate{\RCSLSCMparameterizerDate}{$Date$}
% |
%%RefPage: end of header, begin of main body
% +------------------------------------------------------------------------+


\begin{ccRefClass}{LSCM_parameterizer_3}  %% add template arg's if necessary

%% \ccHtmlCrossLink{}     %% add further rules for cross referencing links
%% \ccHtmlIndexC[class]{} %% add further index entries


\ccDefinition

% The section below is automatically generated. Do not edit!
%START-AUTO(\ccDefinition)

The class LSCM\_parameterizer\_3 implements the Least Squares Conformal Maps (LSCM) parameterization \cite{cgal:lprm-lscm-02}.

This is a conformal parameterization, i.e. it attempts to preserve angles.

This is a free border parameterization. No need to map the surface's border onto a convex polygon (only 2 pinned vertices are needed to ensure a unique solution), but 1 to 1 mapping is NOT guaranteed.

As all parameterization algorithms of the package, this class is usually called via the global function parameterize().

\begin{description}
\item[Todo]Add to SparseLinearAlgebraTraits\_d the ability to solve linear systems in the least squares sense, then access to the solver via the traits class interface instead of calls specific to OpenNL.\end{description}

%END-AUTO(\ccDefinition)

\ccInclude{CGAL/LSCM_parameterizer_3.h}


\ccIsModel

% The section below is automatically generated. Do not edit!
%START-AUTO(\ccIsModel)

Model of the ParameterizerTraits\_3 concept.

%END-AUTO(\ccIsModel)


\ccParameters

The full template declaration is:

% The section below is automatically generated. Do not edit!
%START-AUTO(\ccParameters)

template$<$  \\
class ParameterizationMesh\_3,   \\
class BorderParameterizer\_3 = Two\_vertices\_parameterizer\_3$<$ParameterizationMesh\_3$>$,   \\
class SparseLinearAlgebraTraits\_d = OpenNL::SymmetricLinearSolverTraits$<$typename ParameterizationMesh\_3::NT$>$$>$   \\
class LSCM\_parameterizer\_3;

%END-AUTO(\ccParameters)


\ccTypes

% The section below is automatically generated. Do not edit!
%START-AUTO(\ccTypes)

\ccNestedType{Border_param}
{
Export BorderParameterizer\_3 template parameter.
}
\ccGlue
\ccNestedType{Sparse_LA}
{
Export SparseLinearAlgebraTraits\_d template parameter.
}
\ccGlue

%END-AUTO(\ccTypes)


\ccCreation
\ccCreationVariable{param}  %% choose variable name for \ccMethod

% The section below is automatically generated. Do not edit!
%START-AUTO(\ccCreation)

\ccConstructor{LSCM_parameterizer_3 (Border_param border_param = Border_param(), Sparse_LA sparse_la = Sparse_LA());}
{
Constructor.
}
\ccGlue
\begin{description}
\item[Parameters: ]
\begin{description}
\item[border\_param]Object that maps the surface's border to 2D space \item[sparse\_la]Traits object to access a sparse linear system \end{description}
\end{description}
\ccGlue

%END-AUTO(\ccCreation)


\ccOperations

% The section below is automatically generated. Do not edit!
%START-AUTO(\ccOperations)

\ccMethod{Parameterizer_traits_3< Adaptor >::Error_code parameterize (Adaptor * mesh);}
{
Compute a 1 to 1 mapping from a triangular 3D surface 'mesh' to a piece of the 2D space. The mapping is linear by pieces (linear in each triangle). The result is the (u,v) pair image of each vertex of the 3D surface.
Preconditions:\begin{itemize}
\item 'mesh' must be a surface with 1 connected component.\item 'mesh' must be a triangular mesh. \end{itemize}
}
\ccGlue

%END-AUTO(\ccOperations)


\ccSeeAlso

\ccRefIdfierPage{CGAL::Parameterizer_traits_3}  \\
\ccRefIdfierPage{CGAL::Fixed_border_parameterizer_3}  \\
\ccRefIdfierPage{CGAL::Barycentric_mapping_parameterizer_3}  \\
\ccRefIdfierPage{CGAL::Discrete_authalic_parameterizer_3}  \\
\ccRefIdfierPage{CGAL::Discrete_conformal_map_parameterizer_3}  \\
\ccRefIdfierPage{CGAL::Mean_value_coordinates_parameterizer_3}  \\


\ccExample

\begin{ccExampleCode}

// CGAL kernel
typedef CGAL::Cartesian<double>                         Kernel;

// Mesh true type and parameterization adaptor
typedef CGAL::Polyhedron_3<Kernel>                      Polyhedron;
typedef CGAL::Parameterization_polyhedron_adaptor_3<Polyhedron>
                                                        Parameterization_polyhedron_adaptor;

// Least Squares Conformal Maps parameterization
typedef CGAL::LSCM_parameterizer_3<Parameterization_polyhedron_adaptor>
                                                        Parameterizer;

int main(int argc,char * argv[])
{
    Polyhedron mesh;
    ...

    // The parameterization package needs an adaptor to handle Polyhedron_3 meshes
    // The mesh must be a topological disk
    Parameterization_polyhedron_adaptor mesh_adaptor(&mesh);

    Parameterizer::Error_code err = CGAL::parameterize(&mesh_adaptor, Parameterizer());
    ...
}

\end{ccExampleCode}


\end{ccRefClass}

% +------------------------------------------------------------------------+
%%RefPage: end of main body, begin of footer
% EOF
% +------------------------------------------------------------------------+

