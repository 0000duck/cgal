% +------------------------------------------------------------------------+
% | Reference manual page: Circular_border_arc_length_parametizer_3.tex
% +------------------------------------------------------------------------+
% | 23.08.2005   Author
% | Package: Parameterization
% |
\RCSdef{\RCSCircularborderarclengthparametizerRev}{$Revision$}
\RCSdefDate{\RCSCircularborderarclengthparametizerDate}{$Date$}
% |
%%RefPage: end of header, begin of main body
% +------------------------------------------------------------------------+


\begin{ccRefClass}{Circular_border_arc_length_parametizer_3}  %% add template arg's if necessary

%% \ccHtmlCrossLink{}     %% add further rules for cross referencing links
%% \ccHtmlIndexC[class]{} %% add further index entries


\ccDefinition

Circular\_border\_arc\_length\_parametizer\_3 is the default boundary parametizer
for fixed boundary parameterization methods.

% The section below is automatically generated. Do not edit!
%START-AUTO(\ccDefinition)

This class parameterizes the border of a 3D surface onto a circle, with an arc-length parameterization: (u,v) values are proportional to the length of boundary edges.

%END-AUTO(\ccDefinition)

\ccInclude{CGAL/Circular_border_arc_length_parametizer_3.h}


\ccIsModel

% The section below is automatically generated. Do not edit!
%START-AUTO(\ccIsModel)

Model of the BorderParametizer\_3 concept.

%END-AUTO(\ccIsModel)


\ccHeading{Design pattern}

% The section below is automatically generated. Do not edit!
%START-AUTO(\ccHeading\{Design pattern\})

BorderParametizer\_3 models are Strategies (see [GOF95]): they implement a strategy of boundary parameterization for models of MeshAdaptor\_3

%END-AUTO(\ccHeading\{Design pattern\})


\ccParameters

The full template declaration is:

% The section below is automatically generated. Do not edit!
%START-AUTO(\ccParameters)

template$<$
class MeshAdaptor\_3$>$ 
class Circular\_border\_arc\_length\_parametizer\_3;

%END-AUTO(\ccParameters)


\ccTypes

% The section below is automatically generated. Do not edit!
%START-AUTO(\ccTypes)

None.

%END-AUTO(\ccTypes)


\ccConstants

% The section below is automatically generated. Do not edit!
%START-AUTO(\ccConstants)

None.

%END-AUTO(\ccConstants)


\ccCreation
\ccCreationVariable{bp}  %% variable name for \ccMethod below

\ccConstructor{Circular_border_arc_length_parametizer_3();}{default constructor.}

% The section below is automatically generated. Do not edit!
%START-AUTO(\ccCreation)
%END-AUTO(\ccCreation)


\ccOperations

% The section below is automatically generated. Do not edit!
%START-AUTO(\ccOperations)

\ccMethod{virtual double compute_edge_length (const Adaptor & mesh, Vertex_const_handle source, Vertex_const_handle target);}
{
Compute length of an edge.
Arc-length boundary parameterization: (u,v) values are proportional to the length of boundary edges.
}

%END-AUTO(\ccOperations)


\ccSeeAlso

\ccRefIdfierPage{CGAL::Circular_border_uniform_parametizer_3<Traits>}  \\
\ccRefIdfierPage{CGAL::Square_border_arc_length_parametizer_3<Traits>}  \\
\ccRefIdfierPage{CGAL::Square_border_uniform_parametizer_3<Traits>}  \\
\ccRefIdfierPage{CGAL::Two_vertices_parametizer_3<Traits>}  \\


\ccExample

This example program instantiates a polyhedron and a polyhedron adaptor
and parameterizes the polyhedron with Floater's mean value coordinates algorithm
with the boundary mapped onto a circle.

\ccIncludeExampleCode{Parameterization/Polyhedron_parameterization4.C}


\end{ccRefClass}

% +------------------------------------------------------------------------+
%%RefPage: end of main body, begin of footer
% EOF
% +------------------------------------------------------------------------+

