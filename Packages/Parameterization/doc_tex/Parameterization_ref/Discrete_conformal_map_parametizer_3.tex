% +------------------------------------------------------------------------+
% | Reference manual page: Discrete_conformal_map_parametizer_3.tex
% +------------------------------------------------------------------------+
% | 21.09.2005   Laurent Saboret, Pierre Alliez
% | Package: Parameterization
% | 
\RCSdef{\RCSDiscreteconformalmapparametizerRev}{$Revision$}
\RCSdefDate{\RCSDiscreteconformalmapparametizerDate}{$Date$}
% |
%%RefPage: end of header, begin of main body
% +------------------------------------------------------------------------+


\begin{ccRefClass}{Discrete_conformal_map_parametizer_3}  %% add template arg's if necessary

%% \ccHtmlCrossLink{}     %% add further rules for cross referencing links
%% \ccHtmlIndexC[class]{} %% add further index entries


\ccDefinition

% The section below is automatically generated. Do not edit!
%START-AUTO(\ccDefinition)

Class Discrete\_conformal\_map\_parametizer\_3 implements Discrete Conformap Map parameterization. 1 to 1 mapping is guaranteed if surface's border is mapped onto a convex polygon. This is a conformal parameterization, i.e. it attempts to preserve angles.

%END-AUTO(\ccDefinition)

\ccInclude{CGAL/Discrete_conformal_map_parametizer_3.h}


\ccIsModel

% The section below is automatically generated. Do not edit!
%START-AUTO(\ccIsModel)

Model of the ParametizerTraits\_3 concept.

%END-AUTO(\ccIsModel)


\ccHeading{Design pattern}

% The section below is automatically generated. Do not edit!
%START-AUTO(\ccHeading\{Design pattern\})

None.

%END-AUTO(\ccHeading\{Design pattern\})


\ccParameters

The full template declaration is:

% The section below is automatically generated. Do not edit!
%START-AUTO(\ccParameters)

template$<$
class MeshAdaptor\_3, 
class BorderParametizer\_3 = Circular\_border\_arc\_length\_parametizer\_3$<$MeshAdaptor\_3$>$, 
class SparseLinearAlgebraTraits\_d = OpenNL::DefaultLinearSolverTraits$<$typename MeshAdaptor\_3::NT$>$$>$ 
class Discrete\_conformal\_map\_parametizer\_3;

%END-AUTO(\ccParameters)


\ccTypes

% The section below is automatically generated. Do not edit!
%START-AUTO(\ccTypes)

\ccNestedType{Adaptor}
{
}
\ccNestedType{Error_code}
{
The various errors detected by this package.
}
\ccNestedType{NT}
{
}
\ccNestedType{Facet_handle}
{
}
\ccNestedType{Facet_const_handle}
{
}
\ccNestedType{Vertex_handle}
{
}
\ccNestedType{Vertex_const_handle}
{
}
\ccNestedType{Point_3}
{
}
\ccNestedType{Point_2}
{
}
\ccNestedType{Vector_3}
{
}
\ccNestedType{Vector_2}
{
}
\ccNestedType{Facet_iterator}
{
}
\ccNestedType{Facet_const_iterator}
{
}
\ccNestedType{Vertex_iterator}
{
}
\ccNestedType{Vertex_const_iterator}
{
}
\ccNestedType{Border_vertex_iterator}
{
}
\ccNestedType{Border_vertex_const_iterator}
{
}
\ccNestedType{Vertex_around_facet_circulator}
{
}
\ccNestedType{Vertex_around_facet_const_circulator}
{
}
\ccNestedType{Vertex_around_vertex_circulator}
{
}
\ccNestedType{Vertex_around_vertex_const_circulator}
{
}
\ccNestedType{Border_param}
{
}
\ccNestedType{Sparse_LA}
{
}
\ccNestedType{Vector}
{
}
\ccNestedType{Matrix}
{
}
\ccNestedType{Solver}
{
}

%END-AUTO(\ccTypes)


\ccConstants

% The section below is automatically generated. Do not edit!
%START-AUTO(\ccConstants)

None.

%END-AUTO(\ccConstants)


\ccCreation
\ccCreationVariable{a}  % choose variable name for \ccMethod

% The section below is automatically generated. Do not edit!
%START-AUTO(\ccCreation)

\ccConstructor{Discrete_conformal_map_parametizer_3 (Border_param border_param = Border_param(), Sparse_LA sparse_la = Sparse_LA());}
{
Constructor.
}
\begin{description}
\item[Parameters: ]
\begin{description}
\item[border\_param]Object that maps the surface's border to 2D space \item[sparse\_la]Traits object to access a sparse linear system \end{description}
\end{description}

%END-AUTO(\ccCreation)


\ccOperations

% The section below is automatically generated. Do not edit!
%START-AUTO(\ccOperations)

\ccMethod{virtual NT compute_wij (const Adaptor & mesh, Vertex_const_handle main_vertex_Vi, Vertex_around_vertex_const_circulator neighbor_vertex_Vj);}
{
compute wij = (i,j) coefficient of matrix A for j neighbor vertex of i
}

%END-AUTO(\ccOperations)


\ccSeeAlso

\ccc{Some_other_class},
\ccc{some_other_function}.


\ccExample

A short example program.
Instead of a short program fragment, a full running program can be
included using the
\verb|\ccIncludeExampleCode{Parameterization/Discrete_conformal_map_parametizer_3.C}|
macro. The program example would be part of the source code distribution and
also part of the automatic test suite.

\begin{ccExampleCode}
void your_example_code() {
}
\end{ccExampleCode}

%% \ccIncludeExampleCode{Parameterization/Discrete_conformal_map_parametizer_3.C}


\end{ccRefClass}

% +------------------------------------------------------------------------+
%%RefPage: end of main body, begin of footer
% EOF
% +------------------------------------------------------------------------+

