% +------------------------------------------------------------------------+
% | Reference manual page: Two_vertices_parametizer_3.tex
% +------------------------------------------------------------------------+
% | 21.09.2005   Laurent Saboret, Pierre Alliez
% | Package: Parameterization
% |
\RCSdef{\RCSTwoverticesparametizerRev}{$Revision$}
\RCSdefDate{\RCSTwoverticesparametizerDate}{$Date$}
% |
%%RefPage: end of header, begin of main body
% +------------------------------------------------------------------------+


\begin{ccRefClass}{Two_vertices_parametizer_3}  %% add template arg's if necessary

%% \ccHtmlCrossLink{}     %% add further rules for cross referencing links
%% \ccHtmlIndexC[class]{} %% add further index entries


\ccDefinition

Two\_vertices\_parametizer\_3 is the default boundary parametizer
for Least Square Conformal Maps parameterization.

% The section below is automatically generated. Do not edit!
%START-AUTO(\ccDefinition)

Class Two\_vertices\_parametizer\_3 parameterizes 2 extreme vertices of a 3D surface. This kind of border parameterization is used by free border parameterizations.

Implementation note: To simplify the implementation, BorderParametizer\_3 models know only the MeshAdaptor\_3 class. They don't know the parameterization algorithm requirements nor the kind of sparse linear system used.

%END-AUTO(\ccDefinition)

\ccInclude{CGAL/Two_vertices_parametizer_3.h}


\ccIsModel

% The section below is automatically generated. Do not edit!
%START-AUTO(\ccIsModel)

Model of the BorderParametizer\_3 concept.

%END-AUTO(\ccIsModel)


\ccHeading{Design pattern}

% The section below is automatically generated. Do not edit!
%START-AUTO(\ccHeading\{Design pattern\})

BorderParametizer\_3 models are Strategies (see [GOF95]): they implement a strategy of boundary parameterization for models of MeshAdaptor\_3.

%END-AUTO(\ccHeading\{Design pattern\})


\ccParameters

The full template declaration is:

% The section below is automatically generated. Do not edit!
%START-AUTO(\ccParameters)

template$<$
class MeshAdaptor\_3$>$ 
class Two\_vertices\_parametizer\_3;

%END-AUTO(\ccParameters)


\ccTypes

% The section below is automatically generated. Do not edit!
%START-AUTO(\ccTypes)

\ccNestedType{Adaptor}
{
}
\ccNestedType{Error_code}
{
}
\ccNestedType{NT}
{
}
\ccNestedType{Facet_handle}
{
}
\ccNestedType{Facet_const_handle}
{
}
\ccNestedType{Vertex_handle}
{
}
\ccNestedType{Vertex_const_handle}
{
}
\ccNestedType{Point_3}
{
}
\ccNestedType{Point_2}
{
}
\ccNestedType{Vector_3}
{
}
\ccNestedType{Vector_2}
{
}
\ccNestedType{Facet_iterator}
{
}
\ccNestedType{Facet_const_iterator}
{
}
\ccNestedType{Vertex_iterator}
{
}
\ccNestedType{Vertex_const_iterator}
{
}
\ccNestedType{Border_vertex_iterator}
{
}
\ccNestedType{Border_vertex_const_iterator}
{
}
\ccNestedType{Vertex_around_facet_circulator}
{
}
\ccNestedType{Vertex_around_facet_const_circulator}
{
}
\ccNestedType{Vertex_around_vertex_circulator}
{
}
\ccNestedType{Vertex_around_vertex_const_circulator}
{
}

%END-AUTO(\ccTypes)


\ccConstants

% The section below is automatically generated. Do not edit!
%START-AUTO(\ccConstants)

None.

%END-AUTO(\ccConstants)


\ccCreation
\ccCreationVariable{a}  % choose variable name for \ccMethod

\ccConstructor{Two_vertices_parametizer_3();}{default constructor.}

% The section below is automatically generated. Do not edit!
%START-AUTO(\ccCreation)
%END-AUTO(\ccCreation)


\ccOperations

% The section below is automatically generated. Do not edit!
%START-AUTO(\ccOperations)

\ccMethod{Parametizer_traits_3< Adaptor >::Error_code parameterize_border (Adaptor * mesh);}
{
Map 2 extreme vertices of the 3D mesh and mark them as "parameterized".
Map 2 extreme vertices of the 3D mesh and mark them as "parameterized". Return false on error.
}
\ccMethod{bool is_border_convex ();}
{
Indicate if border's shape is convex. Meaningless for free border parameterization algorithms.
}

%END-AUTO(\ccOperations)


\ccSeeAlso

\ccRefIdfierPage{CGAL::Circular_border_arc_length_parametizer_3<Traits>}  \\
\ccRefIdfierPage{CGAL::Circular_border_uniform_parametizer_3<Traits>}  \\
\ccRefIdfierPage{CGAL::Square_border_arc_length_parametizer_3<Traits>}  \\
\ccRefIdfierPage{CGAL::Square_border_uniform_parametizer_3<Traits>}  \\


\ccExample

\begin{ccExampleCode}

// CGAL kernel
typedef CGAL::Cartesian<double>                         Kernel;

// Mesh true type and parameterization adaptor
typedef CGAL::Polyhedron_3<Kernel>                      Polyhedron;
typedef CGAL::Mesh_adaptor_polyhedron_3<Polyhedron>     Mesh_adaptor_polyhedron;

// Least Square Conformal Maps parameterization with
// the default boundary parameterization: Two_vertices_parametizer_3
typedef CGAL::LSCM_parametizer_3<Mesh_adaptor_polyhedron> Parametizer;

int main(int argc,char * argv[])
{
    Polyhedron mesh;
    ...

    // The parameterization package needs an adaptor to handle Polyhedron_3 meshes
    Mesh_adaptor_polyhedron mesh_adaptor(&mesh);

    Parametizer::Error_code err = CGAL::parameterize(&mesh_adaptor, Parametizer());
    ...
}

\end{ccExampleCode}


\end{ccRefClass}

% +------------------------------------------------------------------------+
%%RefPage: end of main body, begin of footer
% EOF
% +------------------------------------------------------------------------+

