\section{Cutting a Mesh}

\subsection{Computing a Cut}

The input mesh can be of any genus and have any number of connected components,
-but- it has to come with a description of a boundary (a list of
vertices) which is the boundary of
a topological disc. If no boundary is given, we assume that it
coincides with the longest boundary already in the input mesh.

This package does not provide algorithms that transform
a closed mesh of arbitrary genus into a topological disk.
The package user is responsible for computing such a cut.

We provide in polyhedron\_ex\_parameterization.C a simple cutting algorithm as example.
Serious users will have to use a more robust method.


\subsection{Applying a Cut}

Parameterization methods in this package support only
triangulated surfaces that are homeomorphic to a
disk (models of MeshAdaptor\_3). This software design simplifies the implementation of new
parameterization methods.

Mesh\_adaptor\_patch\_3 class is responsible of virtually "cutting"
a patch in a PatchableMeshAdaptor\_3 mesh, to make it appear as a topological disk
with a MeshAdaptor\_3 interface.

PatchableMeshAdaptor\_3 inherits from concept MeshAdaptor\_3, thus is a concept for a 3D surface mesh.
PatchableMeshAdaptor\_3 adds the ability to support patches and virtual seams. Patches are a subset of a 3D mesh. Virtual seams are the ability to behave exactly as if the surface was "cut" following a certain path.

The MeshAdaptor\_3 interfaces with both the 2D Triangulation Data Structure enriched
with 3D points (not yet implemented) and the Polyhedron are also models of PatchableMeshAdaptor\_3:

\ccRefIdfierPage{CGAL::Mesh_adaptor_polyhedron_3}  \\


\subsection{Cutting a Mesh Example}

The code below virtually "cuts" a Polyhedron\_3 mesh to make it a topological disk,
then applies the default parameterization:

\begin{ccExampleCode}

// CGAL kernel
typedef CGAL::Cartesian<double>                             Kernel;

// Mesh true type and parameterization adaptors
typedef CGAL::Polyhedron_3<Kernel>                          Polyhedron;
typedef CGAL::Mesh_adaptor_polyhedron_3<Polyhedron>         Mesh_adaptor_polyhedron;
typedef CGAL::Mesh_adaptor_patch_3<Mesh_adaptor_polyhedron> Mesh_patch_polyhedron;

// Parametizers base class for this kind of mesh
typedef CGAL::Parametizer_traits_3<Mesh_patch_polyhedron>   Parametizer;

// Type describing a border or seam as a vertex list
typedef std::list<Mesh_adaptor_polyhedron::Vertex_handle>   Seam;

// If the mesh is a topological disk, extract its longest boundary,
// else compute a very simple cut to make it homeomorphic to a disk.
// Return the border/seam (empty on error)
static Seam cut_mesh(Mesh_adaptor_polyhedron* mesh_adaptor)
{
    // To be implemented by package user
    ...
}

int main(int argc,char * argv[])
{
    Polyhedron mesh;
    ...

    // The parameterization package needs an adaptor to handle Polyhedron_3 meshes
    Mesh_adaptor_polyhedron mesh_adaptor(&mesh);

    // The parameterization methods support only meshes that
    // are topological disks => we need to compute a "cutting" of the mesh
    // that makes it it homeomorphic to a disk
    Seam seam = cut_mesh(&mesh_adaptor);

    // Create adaptor that virtually "cuts" the mesh following the 'seam' path
    Mesh_patch_polyhedron   mesh_patch(&mesh_adaptor,
                                       seam.begin(),
                                       seam.end());

    // Floater's mean value coordinates parameterization
    Parametizer::Error_code err = CGAL::parameterize(&mesh_patch);
    ...
}

\end{ccExampleCode}

See the complete code in Polyhedron\_parameterization5.C example.


