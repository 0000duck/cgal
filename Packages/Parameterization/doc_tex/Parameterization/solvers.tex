\section{Sparse Linear Algebra}

Parameterizing triangle meshe require both efficient representations
of sparse matrices and efficient iterative or direct linear
solvers. We provide links to standard packages ({\sc Taucs}, SuperLU)
and include a separate package devoted to OpenNL sparse linear solver.

\subsection{List of Solvers and Concept}

We provide an interface to several sparse linear solvers, as models
of the \ccc{SparseLinearAlgebraTraits_d} concept:

\begin{itemize}

\item OpenNL \cite{cgal:l-nmdgp-05} is shipped with \cgal. This is the default solver:

\ccc{OpenNL::DefaultLinearSolverTraits}  \\
\ccc{OpenNL::SymmetricLinearSolverTraits}  \\

\item {\sc Taucs} is a state-of-the-art direct solver for sparse symmetric matrices.

\ccc{CGAL::Taucs_solver_traits}  \\
\ccc{CGAL::Taucs_symmetric_solver_traits}  \\

\item SuperLU is a state-of-the-art direct solver for sparse unsymmetric matrices.

(traits class not yet implemented)

\end{itemize}


\subsection{{\sc Taucs} Solver Example}

The examples provided so far use the default sparse linear solver
OpenNL. The following C++ code computes the default parameterization
method (Floater Mean Value Coordinates with a circular boundary), but
specifically instantiates the {\sc Taucs} solver:

\begin{ccExampleCode}

// CGAL kernel
typedef CGAL::Cartesian<double>                         Kernel;

// Mesh true type and parameterization adaptors
typedef CGAL::Polyhedron_3<Kernel>                      Polyhedron;
typedef CGAL::Parameterization_polyhedron_adaptor_3<Polyhedron>
                                                        Parameterization_polyhedron_adaptor;

// Circular border parameterizer (the default)
typedef CGAL::Circular_border_arc_length_parameterizer_3<Parameterization_polyhedron_adaptor>
                                                        Border_parameterizer;
// TAUCS solver
typedef CGAL::Taucs_solver_traits<double>               Solver;

// Floater Mean Value Coordinates parameterizer (circular border)
// with TAUCS solver
typedef CGAL::Mean_value_coordinates_parameterizer_3<Parameterization_polyhedron_adaptor,
                                                   Border_parameterizer,
                                                   Solver>
                                                        Parameterizer;

int main(int argc,char * argv[])
{
    Polyhedron mesh;
    ...

    // The parameterization package needs an adaptor to handle Polyhedron_3 meshes
    // The mesh must be a topological disk
    Parameterization_polyhedron_adaptor mesh_adaptor(&mesh);

    // Floater Mean Value Coordinates parameterizer (circular border)
    // with TAUCS solver
    Parameterizer::Error_code err = CGAL::parameterize(&mesh_adaptor, Parameterizer());
    ...
}

\end{ccExampleCode}

See the complete example in \ccc{Taucs_parameterization.C}.

