%% ============================================================================
%% The CGAL Reference Manual
%% Chapter: STL Extensions - The Reference Part
%% ----------------------------------------------------------------------------
%% file  : doc_tex/STL_Extension_ref/multiset.tex
%% author: Ron Wein
%% ----------------------------------------------------------------------------
%% $CGAL_Chapter: STL_Extension $
%% $Revision$
%% $Date$
%% ============================================================================

%% +=========================================================================+

\begin{ccRefClass}{Multiset<Type,Compare,Allocator>}
\label{class:cgal_multimap}
\ccCreationVariable{s}

\ccDefinition
%============

An instance \ccVar\ of the parametrized data type \ccRefName\ is a
multi-set of elements of type \ccc{Type}, represented as a red-black tree
(see~\cite[Chapter~13]{clrs-ia-01} for an excellent introduction to red-black
trees).
The main difference between \ccRefName\ and \stl's \ccc{multiset} is that
the latter uses a less-than functor with a Boolean return type, while our
\ccRefName\ class is parameterized by a comparison functor \ccc{Compare} that
returns the three-valued \ccc{Comparison_result} (namely it returns either
\ccc{SMALLER}, \ccc{EQUAL}, or \ccc{LARGER}). It is thus possible to maintain
the underlying red-black tree with less invocations of the comparison functor.
This leads to a speedup of about 5\% even if we maintain a set of integers.
When each comparison of two elements of type \ccc{Type} is an expensive
operation (for example, when they are geometric entities represented using
exact arithmetic), the usage of a three-valued comparison functor can lead to
considerable decrease in the running times.

Moreover, \ccRefName\ allows the insertion of an element into the set given
its {\em exact} position, and not just using an insertion hint, as done by 
\ccc{std::multiset}. This can further reduce the running times, as additional
comparison operations can be avoided.

In addition, the \ccRefName\ guarantees that the order of elements sent to the
comparison functor is fixed. For example, if we insert a new element \ccc{x}
into the set (or erase an element from the set), then we always invoke 
\ccc{Compare() (x, y)} (and never \ccc{Compare() (y, x)}), where \ccc{y} is an
element already stored in the set. This behavior, not supported by
\ccc{std::multiset}, is sometimes crucial for designing more efficient
comparison predicates.

\ccRefName\ also allows for look-up of keys whose type may differ from
\ccc{Type}, as long as users supply a comparison functor \ccc{CompareKey},
where \ccc{CompareKey() (key, y)} returns the three-valued
\ccc{Comparison_result} (\ccc{key} is the look-up key and \ccc{y} is an
element of type \ccc{Type}). Indeed, it is very convenient to look-up
equivalent objects in the set given just by their key. We note however that
it is also possible to use a key of type \ccc{Type} and to employ the default
\ccc{Compare} functor for the look-up, as done when using the
\ccc{std::multiset} class.

\begin{ccAdvanced}

Finally, \ccRefName\ introduces the \ccc{catenate()} and \ccc{split()}
functions. The first function operates on \ccVar\ and accepts a second
set \ccVar', such that the maximum element in \ccVar\ is not greater than
the minimal element in \ccVar', and concatenates \ccVar' to \ccVar. The
second function splits \ccVar\ into two sets,one containing all the
elements that are less than a given key, and the other contains all
elements greater than (or equal to) this key.

\end{ccAdvanced}

\ccParameters
%------------

The \ccc{Multiset} class-template has three parameters:
\begin{itemize}
\item \ccc{Type} --- the type of the stored elements.
\item \ccc{Compare} --- the comparison-functor type. This type should provide
    the following opeator for comparing two \ccc{Type} elements, namely: \\
  \ccc{Comparison_result operator() (const Type& t1, const Type& t2) const;}\\
    The \ccc{CGAL::Compare<Type>} functor is used by default. In this case,
    \ccc{Type} must support an equality operator (\ccc{operator==}) and a
    less-than operator (\ccc{operator<}).
\item \ccc{Allocator} --- the allocator type. \\
    \ccc{CGAL_ALLOCATOR} is used by default.
\end{itemize}

\ccInclude{CGAL/Multiset.h}

\ccHeading{Assertions}
%---------------------

The assertion and precondition flags for the \ccc{Multiset} class
use \ccc{MULTISET} in their names (i.e., \ccc{CGAL_MULTISET_NO_ASSERTIONS} and
\ccc{CGAL_MULTISET_NO_PRECONDITIONS}).

\ccTypes
%-------

In compliance with \stl, the types \ccc{value_type} and \ccc{key_type}
(both equivalent to \ccc{Type}), \ccc{reference} and \ccc{const_reference}
(reference to a value-type), \ccc{key_compare} and \ccc{value_compare}
(both equivalent to \ccc{Compare}), \ccc{size_type} and \ccc{difference_type}
are defined as well.

\ccNestedType{iterator}{}
\ccGlue
\ccNestedType{const_iterator}
{bi-directional iterators for the elements stored in the set.}

\ccNestedType{reverse_iterator}{}
\ccGlue
\ccNestedType{const_reverse_iterator}
{reverse bi-directional iterators for the elements stored in the set.}

\ccCreation
%----------

\ccConstructor{Multiset<Type,Compare,Allocator>();}{ 
    creates an an empty set \ccVar\ that uses a default comparison
    functor.}

\ccConstructor{Multiset<Type,Compare,Allocator>( 
               const Compare& comp);}{
    creates an an empty set \ccVar\ that uses the given comparison
    functor \ccc{comp}.}

\ccConstructor{template <class InputIterator>
               Multiset<Type,Compare,Allocator>( 
               InputIterator first, InputIterator last,
               const Compare& comp = Compare());}{
    creates a set \ccVar\ containing all elements in the range
    \ccc{[first, last)}, that uses the comparison
    functor \ccc{comp}.}

\ccConstructor{Multiset<Type,Compare,Allocator>(
	       const Multiset<Type,Compare,Allocator>& other);}{ 
    copy constructor.}

\ccMethod{const Multiset<Type,Compare,Allocator>& operator= (
	  const Multiset<Type,Compare,Allocator>& other);}{ 
    assignment operator.}

\ccMethod{void swap (Multiset<Type,Compare,Allocator>& other);}{
    swaps the contents of \ccVar\ with those of the other set.}

\ccHeading{Access Member Functions}
%----------------------------------

\ccMethod{Compare key_comp() const;}{
    the comparison functor used.}
\ccGlue
\ccMethod{Compare value_comp() const;}{
    the comparison functor used (same as above).
    Both functions have a non-const version that return a reference
    to the comparison functor.}

\ccMethod{bool empty ();}{
    returns \ccc{true} if the set is empty, \ccc{false} otherwise.}

\ccMethod{size_t size ();}{
    returns the number of elements stored in the set.}

\ccMethod{size_t max_size ();}{
    returns the maximal number of elements the set can store
    (same as \ccc{size()}).}

\ccMethod{iterator begin();}{
    returns an iterator pointing to the first element stored in the set
    (a \ccc{const} version is also available).}

\ccMethod{iterator end();}{
    returns an iterator pointing beyond the last element stored in the set
    (a \ccc{const} version is also available).}

\ccMethod{reverse_iterator rbegin();}{
    returns a reverse iterator pointing beyond the last element stored in the
    set (a \ccc{const} version is also available).}

\ccMethod{reverse_iterator rend();}{
    returns a reverse iterator pointing to the first element stored in the set
    (a \ccc{const} version is also available).}

\ccHeading{Comparison Operations}
%--------------------------------

\ccMethod{bool operator== (
	  const Multiset<Type,Compare,Allocator>& other) const;}{ 
    returns \ccc{true} if the sequences of elements in the two sets are
    pairwise equal (using the comparison functor).}

\ccMethod{bool operator< (
	  const Multiset<Type,Compare,Allocator>& other) const;}{ 
    returns \ccc{true} if the element sequence in \ccVar{} is
    lexicographically smaller than the element sequence of \ccc{other}.}

\ccHeading{Insertion Methods}
%----------------------------

\ccMethod{iterator insert (const Type& x);}{
    inserts the element \ccc{x} into the set and returns an iterator pointing
    to the newly inserted element.}

\ccMethod{template <class InputIterator>
          void insert (InputIterator first, InputIterator last);}{
    inserts all elements in the range \ccc{[first, last)} into
    the set.}

\ccMethod{iterator insert (iterator position, const Type& x);}{
    inserts the element \ccc{x} with a given iterator used as a hint for the
    position of the new element. It Returns an iterator pointing to the
    newly inserted element.}

\ccMethod{iterator insert_before (iterator position, const Type& x);}{
    inserts the element \ccc{x} as the predecessor of the element at the given
    position.
    \ccPrecond\ The operation does not violate the set order --- that is,
                \ccc{x} is not greater than the element pointed by
                \ccc{position} and not less than its current predecessor.}

\ccMethod{iterator insert_after (iterator position, const Type& x);}{
    inserts the element \ccc{x} as the successor of the element at the given
    position.
    \ccPrecond\ The operation does not violate the set order --- that is,
                \ccc{x} is not less than the element pointed by
                \ccc{position} and not greater than its current successor.}

\ccHeading{Removal Methods}
%--------------------------

\ccMethod{size_t erase (const Type& x);}{
    erases all elements equivalent to \ccc{x} from the set and returns the
    number of erased elements.}

\ccMethod{void erase (iterator position);}{
    erases the element pointed by \ccc{position}.}

\ccMethod{void clear ();}{
    clears the set (erases all stored elements).}

\ccHeading{Look-up Methods}
%--------------------------

All methods listed in this section can also accept a \ccc{Type} element
as a look-up key. In this case, it is not necessary to supply a \ccc{CompareKey}
functor, as the \ccc{Compare} functor will be used by default.

\ccMethod{template <class Key, class CompareKey>
          iterator find (const Key& key, const CompareKey& comp_key);}{
    searches for the an element equivalent to \ccc{key} in the set. If the
    set contains objects equivalent to \ccc{key}, it returns an iterator
    pointing to the first one. Otherwise, \ccc{end()} is returned (a 
    \ccc{const} version is also available).}

\ccMethod{template <class Key, class CompareKey>
          size_t count (const Key& key, const CompareKey& comp_key) const;}{
    returns the number of elements equivalent to \ccc{key} in the set.}

\ccMethod{template <class Key, class CompareKey>
          iterator lower_bound (const Key& key, const CompareKey& comp_key);}{
    returns an iterator pointing to the first element in the set that is not 
    less than \ccc{key}. If all set elements are less than \ccc{key}, 
    \ccc{end()} is returned (a \ccc{const} version is also available).}

\ccMethod{template <class Key, class CompareKey>
          iterator upper_bound (const Key& key, const CompareKey& comp_key);}{
    returns an iterator pointing to the first element in the set that is 
    greater than \ccc{key}. If no set element is greater than \ccc{key}, 
    \ccc{end()} is returned (a \ccc{const} version is also available).}

\ccMethod{template <class Key, class CompareKey>
          std::pair<iterator,iterator> 
          equal_range (const Key& key, const CompareKey& comp_key);}{
    returns the range of set elements equivalent to the given key, namely
    \ccc{(lower_bound(key), upper_bound(key))} (a \ccc{const} version is 
    also available).}

\ccMethod{template <class Key, class CompareKey>
          std::pair<iterator,bool> 
          find_lower (const Key& key, const CompareKey& comp_key);}{
    returns a pair comprised of \ccc{lower_bound(key)} and a Boolean flag
    indicating whether this iterator points to an element equivalent to
    the given key (a \ccc{const} version is also available).}

\begin{ccAdvanced}

\ccHeading{Special Operations}
%-----------------------------

\ccMethod{void replace (iterator position, const Type& x);}{
    replaces the element stored at the given position with \ccc{x}.
    \ccPrecond\ The operation does not violate the set order --- that is,
                \ccc{x} is not less that \ccc{position}'s predecessor and
                not greater than its successor.}

\ccMethod{void swap (iterator pos1, iterator pos2);}{
    swaps places between the two elements given by \ccc{pos1} and  \ccc{pos2}.
    \ccPrecond\ The operation does not violate the set order --- that is,
                \ccc{pos1} and \ccc{pos2} store equivalent elements.}

\def\ccLongParamLayout{\ccTrue}
\ccMethod{void catenate (Self& s_prime);}{
    concatenates all elements in \ccc{s_prime} into \ccVar\ and clears
    \ccc{s_prime}.
    All iterators to \ccVar\ and to \ccc{s_prime} remain valid.
    \ccPrecond\ The maximal element in \ccVar\ is not greater than the minimal
                element in \ccc{s_prime}.}

\ccMethod{template <class Key, class CompareKey>
          void split (Key key, CompareKey comp_key, Self& s_prime);}{
    splits \ccVar\, such that it contains all elements that are less than
    the given \ccc{key} and such that \ccc{s_prime} contains all other elements.
    \ccPrecond\ \ccc{s_prime} is initially empty.}

\ccMethod{void split (iterator position, Self& s_prime);}{
    splits \ccVar\, such that it contains all set elements in the range
    \ccc{[begin, position)} and such that \ccc{s_prime} contains all elements
    in the range \ccc{[position, end())}.
    \ccPrecond\ \ccc{s_prime} is initially empty.}
\def\ccLongParamLayout{\ccTrue}

\end{ccAdvanced}

\ccImplementation
%================

\ccc{Multiset} uses a proprietary implementation of a red-black tree
data-structure. The red-black tree invariants guarantee that the height of a 
tree containing $n$ elements is $O(\log{n})$ (more precisely, it is bounded by 
$2 \log_{2}{n}$). As a consequence, all methods that accept an element and need
to locate it in the tree (namely \ccc{insert(x)}, \ccc{erase(x)}, 
\ccc{find(x)}, \ccc{count(x)}, \ccc{lower_bound(x)} , \ccc{upper_bound(x)},
 \ccc{find_lower(x)} and  \ccc{equal_range(x)}) take $O(\log{n})$ time and
perform $O(\log{n})$ comparison operations.

On the other hand, the set operations that accept a position iterator (namely
\ccc{insert_before(pos, x)}, \ccc{insert_after(pos, x)} and \ccc{erase(pos)})
are much more efficient as they can be performed at a {\em constant} amortized 
cost (see~\cite{gs-dfbt-78} and~\cite{t-dsna-83} for more details).
More important, these set operations require {\em no} comparison operations.
Therefore, it is highly recommended to maintain the set via iterators
to the stored elements, whenever possible. The function \ccc{insert(pos, x)} 
is safer to use, but it takes amortized $O(\min\{d,\log{n}\})$ time, where $d$
is the distance between the given position and the true position of \ccc{x}. 
In addition, it always performs at least two comparison operations.

\begin{ccAdvanced}

The \ccc{catenate()} and \ccc{split()} functions are also very efficient, and
can be performed in $O(\log{n})$ time, where $n$ is the total number of
elements in the sets, and without performing any comparison operations
(see~\cite{t-dsna-83} for the details).
Note however that the size of two sets resulting from a split operation is
initially unknown, as it is impossible to compute it in less than linear time.
Thus, the first invocation of \ccc{size()} on such a set takes linear time,
and {\em not} constant time.

\end{ccAdvanced}

The design is derived from the \stl\ \ccc{multiset} class-template (see,
e.g,~\cite{cgal:ms-strg-96}), where the main differences between the two
classes are highlighted in the class definition above.

\end{ccRefClass}




