%% =============================================================================
%% The CGAL Reference Manual
%% Chapter: STL Extensions - Introduction
%% -----------------------------------------------------------------------------
%% file  : doc_tex/support/STL_Extension/STL_Extension_ref/intro.tex
%% author: Michael Hoffmann, Lutz Kettner 
%% -----------------------------------------------------------------------------
%% $CGAL_Chapter: STL_Extension $
%% $Revision$
%% $Date$
%% =============================================================================

\chapter{\stl\ Extensions for \cgal}
\label{chapterDataStructures}\label{chapterStlExtensions}

\RCSdef{\stlExtensionRev}{$Revision$}
\RCSdefDate{\stlExtensionDate}{$Date$}
\ccChapterRelease{\stlExtensionRev. \ \stlExtensionDate}\ccTexHtml{\\}{<BR>}
\ccChapterAuthor{Michael Hoffmann}\ccTexHtml{\\}{<BR>}
\ccChapterAuthor{Lutz Kettner}

%% +=========================================================================+

\section*{Introduction}

\cgal\ is designed in the spirit of the generic programming paradigm
to work together with the Standard Template Library (\stl)
\cite{ansi-is14882,a-gps-98}. This chapter documents non-geometric
\stl-like components that are not provided in the \stl\ standard but
in \cgal: a doubly-connected list managing items in place (where
inserted items are not copied), generic algorithms, iterators, functor
adaptors for binding and swapping arguments and for composition,
functors for projection and creation and adaptor classes around
iterators and circulators. See also circulators in
Chapter~\ref{chapterCirculators}.

\subsection*{Doubly-Connected List Managing Items in Place.}

The class \ccStyle{In_place_list<T,&T::next,&T::prev>} manages a
sequence of items in place in a doubly-connected list. Its goals are
the flexible handling of memory management and performance
optimization. The item type is supposed to provide the two necessary
pointers \ccc{&T::next_link} and \ccc{&T::prev_link}. One possibility
to obtain these pointers is to inherit them from the base class
\ccStyle{In_place_list_base<T>}.

The class \ccStyle{In_place_list<T>} is a container and quite similar
to \stl\ containers, with the advantage that it is able to handle the
stored elements by reference instead of copying them. It is possible
to delete an element only knowing its address and no iterator to it.
This simplifies mutually pointered data structures like a halfedge
data structure for planar maps or polyhedral surfaces. The usual
iterators are also available.  Another container with this property of
working with pointers to objects is the \stl\ vector (at least in the
current \stl\ implementations).

\ccRefIdfierPage{CGAL::In_place_list<T,bool>}\\
\ccRefIdfierPage{CGAL::In_place_list_base<T>}

\subsection*{Generic Algorithms.}

\ccRefIdfierPage{CGAL::copy_n}\\
\ccRefIdfierPage{CGAL::min_max_element}\\
\ccRefIdfierPage{CGAL::min_element_if}\\
\ccRefIdfierPage{CGAL::max_element_if}

\subsection*{Iterators and Iterator/Circulator Adaptors.}

\ccRefIdfierPage{CGAL::Emptyset_iterator}\\
\ccRefIdfierPage{CGAL::Oneset_iterator<T>}\\
\ccRefIdfierPage{CGAL::Counting_iterator<Iterator, Value>}\\
\ccRefIdfierPage{CGAL::N_step_adaptor<I,int N,Ref,Ptr,Val,Dst,Ctg>}\\
\ccRefIdfierPage{CGAL::Filter_iterator<Iterator, Predicate>}\\
\ccRefIdfierPage{CGAL::Join_input_iterator_1<Iterator, Creator>}\\
\ccRefIdfierPage{CGAL::Inverse_index<IC>}\\
\ccRefIdfierPage{CGAL::Random_access_adaptor<IC>}\\
\ccRefIdfierPage{CGAL::Random_access_value_adaptor<IC,T>}

\subsection*{Functor Adaptors.}

The standard library contains some adaptors for binding functors, that
is fixing one argument of a functor to a specific value thereby
creating a new functor that takes one arguments less than the original
functor.  Also, though non-standard, some STL implementations (such as
SGI) provide adaptors to compose function objects. Unfortunately,
these bind and compose adaptors are limited to unary and binary
functors only, and these functors must not be overloaded.

Since there are a number of functors in \cgal\ that take more than two
arguments, and since functors may also be overloaded, i.e. accept
several different sets of arguments, we have to define our own
adaptors to be used with \cgal\ functors.

\ccRefIdfierPage{CGAL::swap_1}\\
\ccRefIdfierPage{CGAL::swap_2}\\
\ccRefIdfierPage{CGAL::swap_3}\\
\ccRefIdfierPage{CGAL::swap_4}\\
\ccRefIdfierPage{CGAL::bind_1}\\
\ccRefIdfierPage{CGAL::bind_2}\\
\ccRefIdfierPage{CGAL::bind_3}\\
\ccRefIdfierPage{CGAL::bind_4}\\
\ccRefIdfierPage{CGAL::bind_5}\\
\ccRefIdfierPage{CGAL::compose}\\
\ccRefIdfierPage{CGAL::compose_shared}\\
\ccRefIdfierPage{CGAL::Swap<F,i>}\\
\ccRefIdfierPage{CGAL::Bind<F,A,i>}\\
\ccRefIdfierPage{CGAL::Compose<F0,F1,F2,F3>}\\
\ccRefIdfierPage{CGAL::Compose_shared<F0,F1,F2,F3>}\\
\ccRefConceptPage{AdaptableFunctor}\\
\ccRefIdfierPage{CGAL::Arity_tag<int>}\\
\ccRefIdfierPage{CGAL::Arity_traits<F>}

\subsection*{Projection Function Objects.}

\ccRefIdfierPage{CGAL::Identity<Value>}\\
%%\ccRefIdfierPage{CGAL::Compose<Fct1, Fct2>}\\
\ccRefIdfierPage{CGAL::Dereference<Value>}\\
\ccRefIdfierPage{CGAL::Get_address<Value>}\\
\ccRefIdfierPage{CGAL::Cast_function_object<Arg, Result>}\\
\ccRefIdfierPage{CGAL::Project_vertex<Node>}\\
\ccRefIdfierPage{CGAL::Project_facet<Node>}\\
\ccRefIdfierPage{CGAL::Project_point<Node>}\\
\ccRefIdfierPage{CGAL::Project_normal<Node>}\\
\ccRefIdfierPage{CGAL::Project_plane<Node>}\\
\ccRefIdfierPage{CGAL::Project_next<Node>}\\
\ccRefIdfierPage{CGAL::Project_prev<Node>}\\
\ccRefIdfierPage{CGAL::Project_next_opposite<Node>}\\
\ccRefIdfierPage{CGAL::Project_opposite_prev<Node>}

\subsection*{Creator Function Objects.}

\ccRefIdfierPage{CGAL::Creator_1<Arg, Result>}\\
\ccRefIdfierPage{CGAL::Creator_2<Arg1, Arg2, Result>}\\
\ccRefIdfierPage{CGAL::Creator_3<Arg1, Arg2, Arg3, Result>}\\
\ccRefIdfierPage{CGAL::Creator_4<Arg1, Arg2, Arg3, Arg4, Result>}\\
\ccRefIdfierPage{CGAL::Creator_5<Arg1, Arg2, Arg3, Arg4, Arg5, Result>}\\
\ccRefIdfierPage{CGAL::Creator_uniform_2<Arg, Result>}\\
\ccRefIdfierPage{CGAL::Creator_uniform_3<Arg, Result>}\\
\ccRefIdfierPage{CGAL::Creator_uniform_4<Arg, Result>}\\
\ccRefIdfierPage{CGAL::Creator_uniform_5<Arg, Result>}\\
\ccRefIdfierPage{CGAL::Creator_uniform_6<Arg, Result>}\\
\ccRefIdfierPage{CGAL::Creator_uniform_7<Arg, Result>}\\
\ccRefIdfierPage{CGAL::Creator_uniform_8<Arg, Result>}\\
\ccRefIdfierPage{CGAL::Creator_uniform_9<Arg, Result>}

\subsection*{Utilities.}
\ccRefIdfierPage{CGAL::triple<T1, T2, T3>}\\
\ccRefIdfierPage{CGAL::quadruple<T1, T2, T3, T4>}

%% EOF
