\begin{ccRefFunctionObjectConcept}{Kernel::ConstructLine_3}
A model for this must provide:

\ccCreationVariable{fo}

\ccHidden \ccMemberFunction{Kernel::Line_3 operator()();}
             {introduces an uninitialized variable .}

\ccHidden \ccMemberFunction{Kernel::Line_3 operator()(const Kernel::Line_3 &h);}
            {copy constructor.}

\ccMemberFunction{Kernel::Line_3 operator()(const Kernel::Point_3 &p,
                                            const Kernel::Point_3 &q);}
            {introduces a line  passing through the points $p$ and $q$. 
             Line  is directed from $p$ to $q$.}

\ccMemberFunction{Kernel::Line_3 operator()(const Kernel::Point_3 &p,
                                            const Kernel::Vector_3&v);}
            {introduces a line  passing through point $p$ and
             oriented by $v$.}

\ccMemberFunction{Kernel::Line_3 operator()(const Kernel::Point_3 &p,
                                            const Kernel::Direction_3&d);}
            {introduces a line  passing through point $p$ with 
             direction $d$.}

\ccMemberFunction{Kernel::Line_3 operator()(const Kernel::Segment_3 &s);}
            {returns the line supporting the segment $s$,
            oriented from source to target.}

\ccMemberFunction{Kernel::Line_3 operator()(const Kernel::Ray_3 &r);}
            {returns the line supporting the ray $r$, with the
            same orientation.}

\ccRefines
AdaptableFunctor (with two arguments)

\ccSeeAlso
\ccRefIdfierPage{CGAL::Line_3<Kernel>}  \\

\end{ccRefFunctionObjectConcept}
