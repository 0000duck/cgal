\begin{ccRefFunctionObjectConcept}{Kernel::ConstructVector_2}
A model for this must provide:

\ccCreationVariable{fo}

\ccMemberFunction{Kernel::Vector_2 operator()(const Kernel::Point_2 &a,
                                              const Kernel::Point_2 &b);}
            {introduces the vector $b-a$.}

\ccMemberFunction{Kernel::Vector_2 operator()(const Origin &o,
                                              const Kernel::Point_2 &b);}
            {introduces the vector $b$.}

\ccMemberFunction{Kernel::Vector_2 operator()(const Kernel::Point_2 &a,
                                              const Origin &o);}
            {introduces the vector $-a$.}

\ccMemberFunction{Kernel::Vector_2 operator()(const Kernel::Segment_2 &s);}
            {introduces the vector $s.target()-s.source()$.}

\ccMemberFunction{Kernel::Vector_2 operator()(const Kernel::Ray_2 &r);}
            {introduces a vector having the same direction as $r$.}

\ccMemberFunction{Kernel::Vector_2 operator()(const Kernel::Line_2 &l);}
            {introduces a vector having the same direction as $l$.}

\ccMemberFunction{Kernel::Vector_2 operator()(const Null_vector &NULL_VECTOR);}
            {introduces a null vector .}

\ccHidden\ccMemberFunction{Kernel::Vector_2 operator()(const Kernel::RT &hx, const Kernel::RT &hy, const Kernel::RT &hw = Kernel::RT(1));}
            {introduces a vector  initialized to $(hx/hw,hy/hw)$.
             If the third argument is not explicitly given, it defaults
             to \ccStyle{Kernel::RT(1)}.
             \ccPrecond \ccTexHtml{$hw\neq 0$}{hw not equal to 0} }

\ccRefines
AdaptableFunctor (with two arguments)

\ccSeeAlso
\ccRefIdfierPage{CGAL::Vector_2<Kernel>}\\
\ccRefIdfierPage{Kernel::ConstructScaledVector_2}\\

\end{ccRefFunctionObjectConcept}
