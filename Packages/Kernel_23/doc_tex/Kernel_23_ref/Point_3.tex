\begin{ccRefClass} {Point_3<Kernel>}

\ccDefinition
An object of the class \ccRefName\ is a point in the three-dimensional
Euclidean space $\E^3$. 
%% 
%% \cgal\ defines a symbolic constant
%% \ccStyle{ORIGIN}  which denotes the point at the origin. It can be used
%% wherever a point can be used, with the only exception that you can not
%% access its dimension as it is dimensionless.
%% 

Remember that \ccStyle{Kernel::RT} and \ccStyle{Kernel::FT} denote a
RingNumberType and a FieldNumberType, respectively. For the kernel
model \ccStyle{Cartesian<T>}, the two types are the same. For the
kernel model \ccStyle{Homogeneous<T>}, \ccStyle{Kernel::RT} is equal
to \ccStyle{T}, and \ccStyle{Kernel::FT} is equal to
\ccStyle{Quotient<T>}.

\ccTypes
\ccThree{Cartesian_const_iterator}{Facet }{}
\ccThreeToTwo

\ccNestedType{Cartesian_const_iterator}{An iterator for enumerating the
      \ccHtmlNoLinksFrom{Cartesian} coordinates of a point.}

\ccCreation
\ccCreationVariable{p}


\ccHidden \ccConstructor{Point_3();}
             {introduces an uninitialized variable \ccVar.}

\ccHidden \ccConstructor{Point_3(const Point_3<Kernel> &q);}
            {copy constructor.}

\ccConstructor{Point_3(const Origin &ORIGIN);}
            {introduces a point with \ccHtmlNoLinks{Cartesian} coordinates$(0,0,0)$.}

\ccConstructor{Point_3(const Kernel::RT &hx, const Kernel::RT &hy, const Kernel::RT &hz, const Kernel::RT &hw = RT(1));}
            {introduces a point \ccVar\ initialized to $(hx/hw,hy/hw, hz/hw)$.
             \ccPrecond \ccc{hw} $\neq$ 0.}


\ccOperations
%\ccSetTwoOfThreeColumns{5cm}{4cm}

\ccHidden \ccMethod{Point_3<Kernel> & operator=(const Point_3<Kernel> &q);}
        {Assignment.}

\ccMethod{bool operator==(const Point_3<Kernel> &q) const;}
       {Test for equality: Two points are equal, iff their $x$, $y$ and $z$
        coordinates are equal.}

\ccMethod{bool operator!=(const Point_3<Kernel> &q) const;}
       {Test for inequality.}

There are two sets of coordinate access functions, namely to the
homogeneous and to the \ccHtmlNoLinksFrom{Cartesian} coordinates. They can be used
independently from the chosen kernel model.

\ccMethod{Kernel::RT hx() const;}
       {returns the homogeneous $x$ coordinate.}
\ccGlue
\ccMethod{Kernel::RT hy() const;}
       {returns the homogeneous $y$ coordinate.}
\ccGlue
\ccMethod{Kernel::RT hz() const;}
       {returns the homogeneous $z$ coordinate.}
\ccGlue
\ccMethod{Kernel::RT hw() const;}
       {returns the homogenizing coordinate.}

Note that you do not loose information with the homogeneous
representation, because the FieldNumberType is a quotient.

\ccMethod{Kernel::FT x() const;}
       {returns the \ccHtmlNoLinks{Cartesian} $x$ coordinate, that is $hx/hw$.}
\ccGlue
\ccMethod{Kernel::FT y() const;}
       {returns the \ccHtmlNoLinks{Cartesian} $y$ coordinate, that is $hy/hw$.}
\ccGlue
\ccMethod{Kernel::FT z() const;}
       {returns the \ccHtmlNoLinks{Cartesian} $z$ coordinate, that is $hz/hw$.}

The following operations are for convenience and for compatibility
with code for higher dimensional points. Again they come in a
\ccHtmlNoLinksFrom{Cartesian} and in a homogeneous flavor.

\ccMethod{Kernel::RT homogeneous(int i) const;}
       {returns the i'th homogeneous coordinate of \ccVar, starting with 0.
        \ccPrecond $0\leq i \leq 3$.}

\ccMethod{Kernel::FT cartesian(int i) const;}
       {returns the i'th \ccHtmlNoLinks{Cartesian} coordinate of \ccVar, starting with 0.
        \ccPrecond $0\leq i \leq 2$.}

\ccMethod{Kernel::FT operator[](int i) const;}
       {returns \ccStyle{cartesian(i)}.
        \ccPrecond $0\leq i \leq 2$.}

\ccMethod{Cartesian_const_iterator cartesian_begin() const;}
       {returns an iterator to the \ccHtmlNoLinksFrom{Cartesian} coordinates 
        of \ccVar, starting with the 0th coordinate.}

\ccMethod{Cartesian_const_iterator cartesian_end() const;}
       {returns an off the end iterator to the \ccHtmlNoLinksFrom{Cartesian} 
        coordinates of \ccVar.}

\ccMethod{int dimension() const;}
       {returns the dimension (the constant 3).}

\ccMethod{Bbox_3 bbox() const;}
       {returns a bounding box containing \ccVar.}

\ccMethod{Point_3<Kernel>  transform(const Aff_transformation_3<Kernel> &t) const;}
       {returns the point obtained by applying $t$ on \ccVar.}

\ccHeading{Operators}

The following operations can be applied on points:

\ccFunction{bool operator<(const Point_3<Kernel> &p,
                           const Point_3<Kernel> &q);}
       {returns true iff \ccc{p} is lexicographically smaller than \ccc{q}
       (the lexicographical order being defined on the Cartesian
       coordinates).}

\ccFunction{bool operator>(const Point_3<Kernel> &p,
                           const Point_3<Kernel> &q);}
       {returns true iff \ccc{p} is lexicographically greater than \ccc{q}.}

\ccFunction{bool operator<=(const Point_3<Kernel> &p,
                           const Point_3<Kernel> &q);}
       {returns true iff \ccc{p} is lexicographically smaller or equal to
\ccc{q}.}

\ccFunction{bool operator>=(const Point_3<Kernel> &p,
                           const Point_3<Kernel> &q);}
       {returns true iff \ccc{p} is lexicographically greater or equal to
\ccc{q}.}

\ccFunction{Vector_3<Kernel> operator-(const Point_3<Kernel> &p,
                                  const Point_3<Kernel> &q);}
       {returns the difference vector between \ccStyle{q} and \ccStyle{p}. 
        You can substitute \ccc{ORIGIN} for either \ccc{p} or \ccc{q}, 
        but not for both.}

\ccFunction{Point_3<Kernel> operator+(const Point_3<Kernel> &p,
                                 const Vector_3<Kernel> &v);}
       {returns the point obtained by translating \ccStyle{p} by the 
        vector \ccStyle{v}.}

\ccFunction{Point_3<Kernel> operator-(const Point_3<Kernel> &p,
                                 const Vector_3<Kernel> &v);}
       {returns the point obtained by translating \ccStyle{p} by the 
        vector -\ccStyle{v}.}

\ccSeeAlso

\ccRefConceptPage{Kernel::Point_3}

\end{ccRefClass} 

