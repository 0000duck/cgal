\chapter{Extensible Kernel}

\section{Introduction}

The following manual sections describe how users can plug user defined
geometric classes in existing \cgal\ kernels.

\section{The Anatomy of a Kernel}

\subsection{Introduction}

\cgal\ defines the concept of a geometry kernel. Such a kernel provides types,
construction objects and generalized predicates. Most implementations of CG
algorithms and data structures in the basic library of \cgal\ were done in a way
that classes or functions can be parametrized with a geometric traits class. 

In most cases this geometric traits class must be a model of the \cgal\ geometry
kernel concept (but there are some exceptions).



\subsection{Restricting the Interface}



\ccHtmlLinksOff

\begin{ccExampleCode}
#include <CGAL/Kernel_archetype.h>
\end{ccExampleCode}

\ccHtmlLinksOn


\subsection{Example Program}

The following example shows a program for 

\ccHtmlLinksOff

{\bf MyKernel.C :}

\begin{ccExampleCode}
#include <CGAL/basic.h>
#include <CGAL/convex_hull_2.h>
#include ``./MyKernel.h''
#include <list>

typedef MyKernel<double>           MyK;
typedef CGAL::Filtered_kernel<MyK> K;
typedef K::Point_2                 Point_2;

int main()
{
  std::list<Point_2> input;
  
  Point_2 act;
  input.push_back(act);

  std::list<Point_2> output;

  K  traits;

  CGAL::convex_hull_2(input.begin(), input.end(),
                      std::back_inserter(output), traits);		        
  return 0;
}
\end{ccExampleCode}

\ccHtmlLinksOn


