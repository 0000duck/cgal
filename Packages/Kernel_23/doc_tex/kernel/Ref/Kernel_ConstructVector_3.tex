\begin{ccRefFunctionObjectConcept}{Kernel::ConstructVector_3}
A model for this must provide:

\ccCreationVariable{fo}


\ccHidden\ccMemberFunction{Kernel::Vector_3 operator()();}
             {introduces an uninitialized variable .}

\ccHidden \ccMemberFunction{Kernel::Vector_3 operator()(const Kernel::Vector_3 &w);}
            {copy constructor.}

\ccMemberFunction{Kernel::Vector_3 operator()(const Kernel::Point_3 &a,
                                              const Kernel::Point_3 &b);}
            {introduces the vector $b-a$.}

\ccMemberFunction{Kernel::Vector_3 operator()(const Kernel::Segment_3 &s);}
            {introduces the vector $s.target()-s.source()$.}

\ccMemberFunction{Kernel::Vector_3 operator()(const Kernel::Ray_3 &r);}
            {introduces a vector having the same direction as $r$.}

\ccMemberFunction{Kernel::Vector_3 operator()(const Kernel::Line_3 &l);}
            {introduces a vector having the same direction as $l$.}

\ccMemberFunction{Kernel::Vector_3 operator()(const Null_vector &NULL_VECTOR);}
            {introduces a null vector .}

\ccHidden\ccMemberFunction{Kernel::Vector_3 operator()(const Kernel::RT &hx, const Kernel::RT &hy, const Kernel::FT &hz, const Kernel::RT &hw = Kernel::RT(1));}
            {introduces a vector  initialized to $(hx/hw, hy/hw, hz/hw)$.
             If the third argument is not explicitly given it defaults
             to \ccStyle{Kernel::RT(1)}.}

\ccRefines
AdaptableFunctor (with two arguments)

\ccSeeAlso
\ccRefIdfierPage{CGAL::Vector_3<Kernel>}\\

\end{ccRefFunctionObjectConcept}
