\begin{ccRefClass}{Point_d<R>}
\ccInclude{CGAL/Point_d.h}

\ccDefinition

An object of the class \ccClassTemplateName\ is a point in $d$-dimensional
Euclidean space $E_d$, where $d$ is arbitrary. 


% -----------------------------------------------------------------------------
\ccCreation
\ccCreationVariable{p}

A \ccClassTemplateName\ object can be created from an iterator range. 

\ccConstructor{ template <class InputIterator>
		Point_d (int dim, 
			      InputIterator first,
			      InputIterator last);}
	{if the range \ccc{[first,last)} contains \ccc{dim} elements,
	 this creates a point with Cartesian coordinates as specified by 
	 the range. If \ccc{[first,last)} contains \ccc{dim+1} elements,
	 the range specifies the homogeneous coordinates of \ccVar. 
	 \ccPrecond \ccc{dim} is nonnegative, \ccc{[first,last)} has 
	 \ccc{dim} or \ccc{dim+1} elements, and the value types of first
	 and last are \ccc{R::RT}.}

\emph{Note}: in case your compiler does not support member templates, \cgal\
provides the following specialized constructor.

\ccConstructor {Point_d (int dim, const R::RT* first, 
				       const R::RT* last);}{}

% -----------------------------------------------------------------------------
\ccOperations

\ccMemberFunction{ bool operator == (const Point_d<R>& q) const;}
	{Equality test. Two points are equal if they have the same
	 dimension and agree in all coordinates.}

\ccMemberFunction{ bool operator != (const Point_d<R>& q) const;}
	{Test for inequality.}

\ccMemberFunction{ R::RT homogeneous( int i) const;}
	{returns the i'th homogeneous coordinate of \ccVar, starting
	 with 0. \ccPrecond $0\leq i\leq \ccc{d}$.}

\ccMemberFunction{ R::FT cartesian( int i) const;}
	{returns the i'th Cartesian coordinate of \ccVar, starting
	 with 0. \ccPrecond $0\leq i < \ccc{d}$.}

\ccMemberFunction{ R::FT operator [] (int i) const;}
	{returns the i'th Cartesian coordinate of \ccVar, starting
	with 0. \ccPrecond $0\leq i < \ccc{d}$.}

\ccMemberFunction{ int dimension () const;}
	{returns the dimension \ccc{d} of \ccVar.}

% -----------------------------------------------------------------------------
\end{ccRefClass}% 
