\begin{ccRefConcept}{RingNumberType}

\ccDefinition

The concept \ccRefName\ defines the syntactic requirements a number type must
meet in order to be used in \cgal.  This implies that
\ccc{CGAL::Number_type_traits<RingNumberType>::Has_division} is 
\ccc{CGAL::Tag_false}.

\ccRefines

CopyConstructable, Assignable 


\ccHasModels

\CC\ built-in number types \\
\ccc{CGAL::Filtered_exact<RingNumberType, ET>} \\
\ccc{CGAL::Fixed_precision_nt} \\
\ccc{CGAL::Gmpq} \\
\ccc{CGAL::Gmpz} \\
\ccc{CGAL::Interval_nt} \\
\ccc{CGAL::Interval_nt_advanced} \\
\ccc{CGAL::Lazy_exact_nt<RingNumberType>} \\
\ccc{CGAL::MP_Float} \\
\ccc{CGAL::Quotient<RingNumberType>} \\
\ccc{leda_integer} \\
\ccc{leda_rational} \\
\ccc{leda_bigfloat} \\
\ccc{leda_real} \\

\ccTypes

\ccSetTwoColumns{}{\hspace*{8.5cm}}
\ccCreation
\ccCreationVariable{ntvar}

\ccConstructor{RingNumberType();}
            {Declaration of a variable.}


\ccConstructor{RingNumberType(const RingNumberType &ntval);}
            {Declaration and initialization.}


\ccConstructor{RingNumberType(i)}
            {Declaration and initialization with a small integer
constant $i$, $0 \leq i \leq 127$. The neutral elements for addition
(zero) and multiplication (one) are needed quite often, but sometimes
other small constants are useful too. The value 127 was chosen such
that even signed 8 bit number types can fulfill this condition.  }

\ccSetThreeColumns{RingNumberTypeXXX}{}{\hspace*{8.5cm}}
\ccOperations
\ccMethod{RingNumberType & operator=(const RingNumberType &ntval);}
        {Assignment.  }


\ccFunction{bool is_valid(const RingNumberType &ntval);}
{Not all values of a number type need be valid. The routine
\ccStyle{is_valid} checks this. For example, an expression like
\ccStyle{NT(0)/NT(0)} can result in an invalid number. Routines may
have as a precondition that all numerical values are valid.}

\ccFunction{bool is_finite(const RingNumberType &ntval);}
{ When two large values are multiplied, the result may not fit in a
  \ccStyle{NT}. 
  Some number types (the standard \ccc{float} and \ccc{double} type
  when they conform to standards) have a way to represent a too big
  value as infinity.  \ccStyle{is_finite} implies \ccStyle{is_valid}.}

\ccMethod{bool operator==(const RingNumberType &n) const;}
       {}
\ccGlue
\ccMethod{bool operator!=(const RingNumberType &n) const;}
       {}
\ccGlue
\ccMethod{bool operator<(const RingNumberType &n) const;}
       {}
\ccGlue
\ccMethod{bool operator>(const RingNumberType &n) const;}
       {}
\ccGlue
\ccMethod{bool operator<=(const RingNumberType &n) const;}
       {}
\ccGlue
\ccMethod{bool operator>=(const RingNumberType &n) const;}
       {}
\ccGlue
\ccFunction{RingNumberType operator+(const RingNumberType &ntval1, const RingNumberType &ntval2);}
       {}
\ccGlue
\ccMethod{RingNumberType operator+=(const RingNumberType &n) const;}
       {}
\ccGlue
\ccFunction{RingNumberType operator-(const RingNumberType &ntval1, const RingNumberType &ntval2);}
       {}
\ccGlue
\ccMethod{RingNumberType operator-=(const RingNumberType &n) const;}
       {}
\ccGlue
\ccFunction{RingNumberType operator*(const RingNumberType &ntval1, const RingNumberType &ntval2);}
       {}
\ccGlue
\ccMethod{RingNumberType operator*=(const RingNumberType &n) const;}
       {}
\ccGlue
\ccFunction{RingNumberType operator-(const RingNumberType &ntval);}
       {}

\ccFunction{double to_double(const RingNumberType &ntval);}
         {gives the double value for a number type.
          This is usually an approximation for the real (stored) value.
          It can be used to send numbers to a renderer or to store them 
          in a file.}


\ccSeeAlso
\ccRefConceptPage{EuclideanRingNumberType} \\
\ccRefConceptPage{FieldNumberType} \\
\ccRefIdfierPage{Ring_tag} \\
Support Library Manual

\end{ccRefConcept}
