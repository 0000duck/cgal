\begin{ccRefClass} {Point_2<R>}

\ccDefinition
An object of the class \ccRefName\ is a point in the two-dimensional
Euclidean plane $\E^2$. 

Remember that \ccStyle{R::RT} and \ccStyle{R::FT} denote a ring type
and a field type. For the representation class
\ccStyle{Cartesian<T>} the two types are the same. For the
representation class \ccStyle{Homogeneous<T>} the ring type
\ccStyle{R::RT} is equal to \ccStyle{T}, and the field type \ccStyle{R::FT} is 
equal to \ccStyle{Quotient<T>}.

\ccCreation
\ccCreationVariable{p}


\ccHidden \ccConstructor{Point_2();}
             {introduces an uninitialized variable \ccVar.}

\ccHidden \ccConstructor{Point_2(const Point_2<R> &q);}
 	    {copy constructor.}

\ccConstructor{Point_2(const Origin &ORIGIN);}
            {introduces a variable \ccVar\ with \ccHtmlNoLinksFrom{Cartesian} coordinates
	      $(0,0)$.}

\ccConstructor{Point_2(const R::RT &hx, const R::RT &hy, const R::RT &hw = R::RT(1));}
            {introduces a point \ccVar\ initialized to $(hx/hw,hy/hw)$.
             If the third argument is not explicitly given, it defaults
             to \ccStyle{R::RT(1)}.
	     \ccPrecond \ccc{hw} $\neq$ \ccc{R::RT(0)} }


\ccOperations
%\ccSetTwoOfThreeColumns{5cm}{4cm}

\ccHidden \ccMethod{Point_2<R> & operator=(const Point_2<R> &q);}
        {Assignment.}

\ccMethod{bool operator==(const Point_2<R> &q) const;}
       {Test for equality. Two points are equal, iff their $x$ and $y$ 
        coordinates are equal. The point can be compared with
		\ccc{ORIGIN}.}

\ccMethod{bool operator!=(const Point_2<R> &q) const;}
       {Test for inequality. The point can be compared with \ccc{ORIGIN}.}

There are two sets of coordinate access functions, namely to the
homogeneous and to the \ccHtmlNoLinksFrom{Cartesian} coordinates. They can be used
independently from the chosen representation type \ccStyle{R}.

\ccMethod{R::RT hx() const;}
       {returns the homogeneous $x$ coordinate.}
\ccGlue
\ccMethod{R::RT hy() const;}
       {returns the homogeneous $y$ coordinate.}
\ccGlue
\ccMethod{R::RT hw() const;}
       {returns the homogenizing  coordinate.}

Here come the \ccHtmlNoLinksFrom{Cartesian} access functions. Note that you do not loose
information with the homogeneous representation, because then the field
type is a quotient.

\ccMethod{R::FT x() const;}
       {returns the \ccHtmlNoLinksFrom{Cartesian} $x$ coordinate, that is $hx/hw$.}
\ccGlue
\ccMethod{R::FT y() const;}
       {returns the \ccHtmlNoLinksFrom{Cartesian} $y$ coordinate, that is $hy/hw$.}


The following operations are for convenience and for making this
point class compatible with code for higher dimensional points.
Again they come in a \ccHtmlNoLinksFrom{Cartesian} and homogeneous flavor.

\ccMethod{R::RT homogeneous(int i) const;}
       {returns the i'th homogeneous coordinate of \ccVar, starting with 0.
        \ccPrecond $0\leq i \leq 2$.}

\ccMethod{R::FT cartesian(int i) const;}
       {returns the i'th \ccHtmlNoLinksFrom{Cartesian} coordinate of \ccVar, starting with 0.
        \ccPrecond $0\leq i \leq 1$.}

\ccMethod{R::FT operator[](int i) const;}
       {returns \ccStyle{cartesian(i)}.
        \ccPrecond $0\leq i \leq 1$.}

\ccMethod{int dimension() const;}
       {returns the dimension (the constant 2).}

\ccMethod{Bbox_2 bbox() const;}
       {returns a bounding box containing \ccVar. Note that bounding boxes
        are not parameterized with whatsoever. }

\ccMethod{Point_2<R>  transform(const Aff_transformation_2<R> &t) const;}
       {returns the point obtained by applying $t$ on \ccVar.}


\ccHeading{Operators}

The following operations can be applied on points:

\ccFunction{Vector_2<R> operator-(const Point_2<R> &p,
                                  const Point_2<R> &q);}
       {returns the difference vector between \ccStyle{q} and \ccStyle{p}. 
        You can substitute \ccc{ORIGIN} for either \ccc{p} or \ccc{q},
        but not for both.}

\ccFunction{Point_2<R> operator+(const Point_2<R> &p,
                                 const Vector_2<R> &v);}
       {returns the point obtained by translating \ccStyle{p} by the 
        vector \ccStyle{v}.}

\ccFunction{Point_2<R> operator-(const Point_2<R> &p,
                                 const Vector_2<R> &v);}
       {returns the point obtained by translating \ccStyle{p} by the 
        vector -\ccStyle{v}.}

\ccExample

The following declaration creates two points with \ccHtmlNoLinksFrom{Cartesian} double coordinates.

\begin{cprog}

  Point_2< Cartesian<double> > p, q(1.0, 2.0);
\end{cprog} 

The variable {\tt p} is uninitialized and should first be used on 
the left hand side of an assignment. 
\begin{cprog}

  p = q;

  std::cout << p.x() << "  " << p.y() << std::endl; 
\end{cprog} 

\ccSeeAlso

\ccc{Point_3<R>}, \ccc{Point_d<R>}, \ccc{Vector_2<R>}, \ccc{ORIGIN}, \ccc{operator+}, \ccc{operator-}

\end{ccRefClass} 
