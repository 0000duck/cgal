
\begin{ccRefClass}{Object}

\ccInclude{CGAL/Object.h}
%\ccInclude{CGAL/basic.h}


\ccDefinition  
Some functions can return different types of objects. A typical
\CC\ solution to this problem is to derive all possible return
types from a common base class, to return a pointer to this 
class and to perform a dynamic cast on this pointer. The class
\ccRefName\ provides an abstraction.
An object \ccStyle{obj} of the class \ccRefName\ can
represent an arbitrary class. The only operations it provides is
to make copies and assignments, so that you can put them in lists
or arrays. Note that \ccRefName\ is NOT a common base class for the
elementary classes. Therefore, there is no 
automatic conversion from these classes to \ccRefName. Rather 
this is done with the global function \ccc{make_object}. This 
encapsulation mechanism requires the use of \ccc{assign} to use 
the functionality of the encapsulated class.

\ccCreation
\ccCreationVariable{obj}

\ccConstructor{Object();}
            {introduces an uninitialized variable.}

\ccConstructor{Object(const Object &o);}
            {Copy constructor.}

Objects of type \ccRefName\ are normally created via the global function
\ccc{make_object}.

\ccOperations

\ccMethod{Object &operator=(const Object &o);}
            {Assignment.}

Assignment of an object of type \ccRefName\ to an object of type \ccc{T} 
is done using \ccc{assign}. 

There is also a member function to check whether an object of type \ccRefName\
contains an object.

\ccMethod{bool is_empty();}{returns true, if \ccc{object} does not 
        contain an object.}

\ccExample
In the following example, the object class is used as return value for the 
\ccHtmlNoLinksFrom{intersection} computation, as there are possibly different return values.

\begin{cprog}
{
    Point_2< Cartesian<double> > point;
    Segment_2< Cartesian<double> > segment,  segment_1, segment_2;

    std::cin >> segment_1 >> segment_2;

    Object obj = intersection(segment_1, segment_2);

    if (assign(point, obj)) {
        /* do something with point */
    } else if ((assign(segment, obj)) {
        /* do something with segment*/
    }
\end{cprog}
\ccHtmlLinksOff%
\begin{cprog}
    /*  there was no intersection */
}
\end{cprog}
\ccHtmlLinksOn%

\medskip
The \ccHtmlNoLinksFrom{intersection} routine itself looks roughly as follows:

\begin{cprog}

template < class R >
Object  intersection(Segment_2<R> s1, Segment_2<R> s2)
{
\end{cprog} 
\ccHtmlLinksOff%
\begin{cprog}
    if (/* intersection in a point */ ) {
\end{cprog} 
\ccHtmlLinksOn%
\begin{cprog}
       Point_2<R> p = ... ;
       return make_object(p);
\end{cprog} 
\ccHtmlLinksOff%
\begin{cprog}
    } else if (/* intersection in a segment */ ) {
\end{cprog} 
\ccHtmlLinksOn%
\begin{cprog}
       Segment_2<R> s = ... ;
       return make_object(s);
    }
    return Object();
}
\end{cprog} 
\end{ccRefClass} 
