\begin{ccRefFunctionObjectConcept}{Kernel::ConstructOppositeLine_2}
A model for this must provide:

\ccCreationVariable{fo}

\ccMemberFunction{Kernel::Line_2 operator()(const Kernel::Line_2& l);}
{returns the line representing the same set of points as $l$,
but with opposite direction.}

\ccRefines
AdaptableFunctor (with one argument)

\ccSeeAlso
\ccRefIdfierPage{CGAL::Line_2<Kernel>} \\

\end{ccRefFunctionObjectConcept}
