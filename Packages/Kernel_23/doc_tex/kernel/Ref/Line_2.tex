\begin{ccRefClass} {Line_2<R>}
\ccInclude{CGAL/Line_2.h}

\ccDefinition
An object \ccStyle{l} of the data type \ccRefName\ is a directed
straight line in the two-dimensional Euclidean plane $\E^2$. It is
defined by the set of points with Cartesian coordinates $(x,y)$ 
that satisfy the equation 
\begin{ccTexOnly}
\[ l:\; a\, x +b\, y +c = 0. \]
\end{ccTexOnly}
\begin{ccHtmlOnly}
 l : ax + by + c = 0 
\end{ccHtmlOnly}
 
The line splits $\E^2$ in a {\em positive} and a {\em negative}
side. A point $p$ with Cartesian coordinates 
$(px, py)$ is on the positive side of \ccStyle{l}, iff
\ccTexHtml{$a\, px + b\, py +c > 0$}{a px + b py + c > 0}, it is
on the negative side of \ccStyle{l}, iff 
\ccTexHtml{$a\, px + b\, py +c < 0$}{a px + b py + c < 0}.
The positive side is to the left of \ccc{l}.

\ccCreation
\ccCreationVariable{l}

\ccHidden \ccConstructor{Line_2();}
             {introduces an uninitialized variable \ccVar.}

\ccHidden \ccConstructor{Line_2(const Line_2<R> &h);}
 	    {copy constructor.}

\ccConstructor{Line_2(const R::RT &a, const R::RT &b, const R::RT &c)}
            {introduces a line \ccVar\ with the line equation in Cartesian
	      coordinates $ax +by +c = 0$.}

\ccConstructor{Line_2(const Point_2<R> &p, const Point_2<R> &q);}
            {introduces a line \ccVar\ passing through the points $p$ and $q$. 
             Line \ccVar\ is directed from $p$ to $q$.}

\ccConstructor{Line_2(const Point_2<R> &p, const Direction_2<R>&d)}
            {introduces a line \ccVar\ passing through point $p$ with 
             direction $d$.}

\ccConstructor{Line_2(const Segment_2<R> &s);}
            {introduces a line \ccVar\ supporting the segment $s$,
	    oriented from source to target.}

\ccConstructor{Line_2(const Ray_2<R> &r)}
            {introduces a line \ccVar\ supporting the ray $r$,
	    with same orientation.}

\ccOperations

\ccSetThreeColumns{Direction_2<R> & }{}{\hspace*{7.4cm}}

\ccHidden \ccMethod{Line_2<R> & operator=(const Line_2<R> &h);}
        {Assignment.}

\ccMethod{bool operator==(const Line_2<R> &h) const;}
       {Test for equality: two lines are equal, iff they have a non 
        empty \ccHtmlNoLinksFrom{intersection} and the same direction.}

\ccMethod{bool operator!=(const Line_2<R> &h) const;}
       {Test for inequality.}


\ccMethod{R::RT a() const;}
       {returns the first coefficient of $l$.}
\ccGlue
\ccMethod{R::RT b() const;}
       {returns the second coefficient of $l$.}
\ccGlue
\ccMethod{R::RT c() const;}
       {returns the third coefficient of $l$.}

\ccMethod{Point_2<R> point(int i) const;}
       {returns an arbitrary point on \ccVar. It holds 
        \ccStyle{point(i) == point(j)}, iff \ccStyle{i==j}.
        Furthermore, \ccVar\ is directed from \ccStyle{point(i)}
        to \ccStyle{point(j)}, for all \ccStyle{i} $<$ \ccStyle{j}.}

\ccMethod{Point_2<R>    projection(const Point_2<R> &p) const;}
       {returns the orthogonal projection of $p$ onto \ccVar.}

\ccMethod{R::FT               x_at_y(const R::FT &y) const;}
       {returns the $x$-coordinate of the point at \ccVar\ with
        given $y$-coordinate.
        \ccPrecond \ccVar\ is not horizontal.}

\ccMethod{R::FT               y_at_x(const R::FT &x) const;}
       {returns the $y$-coordinate of the point at \ccVar\ with
        given $x$-coordinate.
        \ccPrecond \ccVar\ is not vertical.}

\ccHeading{Predicates}

\ccMethod{bool is_degenerate() const;}
       {line \ccVar\ is degenerate, if the coefficients \ccStyle{a} and 
        \ccStyle{b} of the line equation are zero.}

\ccMethod{bool is_horizontal() const;}
       {}
\ccGlue
\ccMethod{bool is_vertical() const;}
       {}

\ccMethod{Oriented_side oriented_side(const Point_2<R> &p) const;}
       {returns \ccStyle{ON_ORIENTED_BOUNDARY},
        \ccStyle{ON_NEGATIVE_SIDE}, or the constant
        \ccStyle{ON_POSITIVE_SIDE},
        depending on the position of $p$  relative to the oriented line \ccVar.
        }


For convenience we provide the following boolean functions:

\ccMethod{bool has_on(const Point_2<R> &p) const;}
       {}
\ccGlue
\ccMethod{bool has_on_boundary(const Point_2<R> &p) const;}
       {returns \ccStyle{has_on()}.}
\ccGlue
\ccMethod{bool has_on_positive_side(const Point_2<R> &p) const;}
       {}
\ccGlue
\ccMethod{bool has_on_negative_side(const Point_2<R> &p) const;}
       {}

\ccHeading{Miscellaneous}

\ccMethod{Direction_2<R> direction() const;}
       {returns the direction of \ccVar.}

\ccMethod{Line_2<R>      opposite() const;}
       {returns the line with opposite direction.}

\ccMethod{Line_2<R>      perpendicular(const Point_2<R> &p) const;}
        {returns the line perpendicular to \ccVar\ and passing through $p$,
         where the direction is the direction of \ccVar\ rotated 
         counterclockwise by 90 degrees.}

\ccMethod{Line_2<R>  transform(const Aff_transformation_2<R> &t) const;}
       {returns the line obtained by applying $t$ on a point on \ccVar\ 
        and the direction of \ccVar.}


%\ccImplementation
%Lines are implemented as a line equation. 
%Construction from points or a point and a direction, might lead to 
%loss of precision if the number type is not exact.

\ccExample
Let us first define two Cartesian two-dimensional points in the Euclidean 
plane $\E^2$. Their
dimension and the fact that they are Cartesian is expressed by
the suffix \ccStyle{_2} and the representation type \ccStyle{Cartesian}.

\begin{cprog}

  Point_2< Cartesian<double> >  p(1.0,1.0), q(4.0,7.0);
\end{cprog} 

To define a line $l$ we write:

\begin{cprog}

  Line_2< Cartesian<double> > l(p,q);
\end{cprog} 


\end{ccRefClass} 
