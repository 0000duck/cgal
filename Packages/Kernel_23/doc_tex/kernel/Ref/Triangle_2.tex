\begin{ccRefClass} {Triangle_2<Kernel>}

\ccDefinition  An object $t$ of the class \ccRefName\ is a triangle 
in the two-dimensional Euclidean plane~$\E^2$. 
Triangle  $t$  is oriented, i.e., its boundary has
clockwise or counterclockwise orientation. We call the side to the left
of the boundary the positive side and the side to the right of the
boundary the negative side.

The boundary of a triangle splits the plane in
two open regions, a bounded one and an unbounded one. 

\ccCreation
\ccCreationVariable{t}


\ccHidden \ccConstructor{Triangle_2();}
             {introduces an uninitialized variable \ccVar.}

\ccHidden \ccConstructor{Triangle_2(const Triangle_2<Kernel> &u);}
 	    {copy constructor.}


\ccConstructor{Triangle_2(const Point_2<Kernel> &p, 
	                     const Point_2<Kernel> &q, 
	                     const Point_2<Kernel> &r);}
            {introduces a triangle \ccVar\ with vertices $p$,  $q$ and $r$.}


\ccOperations

\ccHidden \ccMethod{Triangle_2<Kernel> & operator=(const Triangle_2<Kernel> &t2);}
        {Assignment.}

\ccMethod{bool operator==(const Triangle_2<Kernel> &t2) const;}
       {Test for equality: two triangles are equal, iff there exists a 
        cyclic permutation of the vertices of $t2$, such that they are 
        equal to the vertices of \ccVar.}

\ccMethod{bool operator!=(const Triangle_2<Kernel> &t2) const;}
       {Test for inequality.}

\ccMethod{Point_2<Kernel> vertex(int i) const;}
       {returns the i'th vertex modulo 3  of~\ccVar.}

\ccMethod{Point_2<Kernel> operator[](int i) const;}
       {returns \ccStyle{vertex(i)}.}

\ccPredicates

\ccMethod{bool is_degenerate() const;}
       {triangle \ccVar\ is degenerate, if the vertices are collinear.}

\ccMethod{Orientation orientation() const;}
       {returns the orientation of~\ccVar.}

\ccMethod{Oriented_side oriented_side(const Point_2<Kernel> &p) const;}
       {returns 
        \ccStyle{ON_ORIENTED_BOUNDARY}, or
        \ccStyle{POSITIVE_SIDE}, 
        or the constant
        \ccStyle{ON_NEGATIVE_SIDE},
        determined by the position of point $p$.
        \ccPrecond \ccVar\ is not degenerate.}

\ccMethod{Bounded_side bounded_side(const Point_2<Kernel> &p) const;}
       {returns the constant \ccStyle{ON_BOUNDARY},
        \ccStyle{ON_BOUNDED_SIDE}, or else
        \ccStyle{ON_UNBOUNDED_SIDE},
        depending on where point $p$ is.
        \ccPrecond \ccVar\ is not degenerate.}

For convenience we provide the following boolean functions:

\ccMethod{bool has_on_positive_side(const Point_2<Kernel> &p) const;}
       {}
\ccGlue
\ccMethod{bool has_on_negative_side(const Point_2<Kernel> &p) const;}
       {}
\ccGlue
\ccMethod{bool has_on_boundary(const Point_2<Kernel> &p) const;}
       {}
\ccGlue
\ccMethod{bool has_on_bounded_side(const Point_2<Kernel> &p) const;}
       {} 
\ccGlue
\ccMethod{bool has_on_unbounded_side(const Point_2<Kernel> &p) const;}
       {\ccPrecond \ccVar\ is not degenerate.}

\ccHeading{Miscellaneous}

\ccMethod{Triangle_2<Kernel> opposite();}
       {returns a triangle where the boundary is oriented the other
        way round (this flips the positive and the negative side, but
        not the bounded and unbounded side).}

\ccMethod{FT area() const;}
       {returns the signed area of \ccVar.}

\ccMethod{Bbox_2 bbox() const;}
       {returns a bounding box containing \ccVar.}


\ccMethod{Triangle_2<Kernel>  transform(const Aff_transformation_2<Kernel> &at) const;}
       {returns the triangle obtained by applying $at$ on the three
        vertices of \ccVar.}

\ccSeeAlso

\ccRefConceptPage{Kernel::Triangle_2} \\

\end{ccRefClass} 
