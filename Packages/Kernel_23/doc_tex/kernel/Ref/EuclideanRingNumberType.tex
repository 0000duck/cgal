\begin{ccRefConcept}{EuclideanRingNumberType}

The number type supports the operations $+$, $-$ and $*$ as well as a function
$div$, which performs an integer division, the modulus operator $\%$,
that returns the remainder of integer division and the function $gcd$.

\ccRefines
RingNumberType

\ccHasModels
\ccc{int} \\
\ccc{long} \\
\ccc{long long} \\
\ccc{Gmpz} \\
\ccc{leda\_integer} \\
//\ccc{CGAL::Filtered_exact<EuclideanRingNumberType, ET>} \\
//\ccc{CGAL::Lazy_exact_nt<EuclideanRingNumberType>} \\

\ccCreationVariable{ntvar}

\ccOperations
\ccGlue
\ccFunction{EuclideanRingNumberType div(
                    const EuclideanRingNumberType& n1,
                    const EuclideanRingNumberType &n2) const;}
{returns the result of the integer division \ccc{n1}/\ccc{n2}.}
       

\ccGlue
\ccMethod{EuclideanRingNumberType operator%(const EuclideanRingNumberType &n) const;}
       {returns the remainder achieved when dividing \ccVar\ by \ccc{n}.}

\ccGlue
\ccMethod{EuclideanRingNumberType operator%=(const EuclideanRingNumberType &n) const;}
       {computes the remainder achieved when dividing \ccVar\ by \ccc{n} and
        assigns it to \ccVar.}

\ccGlue
\ccFunction{EuclideanRingNumberType gcd(const EuclideanRingNumberType n1, const EuclideanRingNumberType &n2) const;}
       {Computes the greatest common divisor between \ccc{n1} and \ccc{n2}.}


\ccSeeAlso
\ccRefConceptPage{FieldNumberType} \\
\ccRefConceptPage{Kernel} \\
\ccRefConceptPage{R}\\
Support Library Manual

\end{ccRefConcept}
