\chapter{Kernel Representations}
Our object of study is the $d$-dimensional affine Euclidean space. 
Here we are mainly concerned with cases $d=2$ and $d=3$.
Objects in that space are sets of points. A common way to represent 
the points is the use of \ccHtmlNoLinksFrom{Cartesian} coordinates,
which assumes a reference frame (an origin and $d$ orthogonal axes).
In that framework, a point is represented by a $d$-tuple
\ccTexHtml{$(c_0,c_1,\ldots,c_{d-1})$}{<MATH>(c<SUB>0</SUB>,c<SUB>1</SUB>,...,c<SUB>d-1</SUB>)</MATH>},
and so are vectors in the underlying linear space. Each point is 
represented uniquely by such \ccHtmlNoLinksFrom{Cartesian} coordinates.
%An object can then be represented by an equation, that is, it is the 
%set of
%points whose coordinates satisfy the equation. For example, the 
%equation of a
%line is usually given as $ax + by +c = 0$.
%The line can then be represented by three numbers $a$, $b$, $c$.
Another way to represent points is by homogeneous coordinates. In that
framework, a point is represented by a $(d+1)$-tuple
\ccTexHtml{$(h_0,h_1,\ldots,h_d)$}{<MATH>(h<SUB>0</SUB>,h<SUB>1</SUB>,...,h<SUB>d</SUB>)</MATH>}.
Via the formulae
\ccTexHtml{$c_i = h_i/h_d$}{<MATH>c<SUB>i</SUB>=h<SUB>i</SUB>/h<SUB>d</SUB></MATH>},
the corresponding point with \ccHtmlNoLinksFrom{Cartesian} coordinates
\ccTexHtml{$(c_0,c_1,\ldots,c_{d-1})$}{<MATH>(c<SUB>0</SUB>,c<SUB>1</SUB>,...,c<SUB>d-1</SUB>)</MATH>}
can be computed.  Note that homogeneous coordinates are not unique. 
For \ccTexHtml{$\lambda\ne 0$}{<MATH>lambda != 0</MATH>},
the tuples 
\ccTexHtml{$(h_0,h_1,\ldots,h_d)$}{<MATH>(h<SUB>0</SUB>,h<SUB>1</SUB>
,...,h<SUB>d</SUB>)</MATH>}
 and 
\ccTexHtml{$(\lambda\cdot h_0,\lambda\cdot h_1,\ldots,\lambda\cdot 
h_d)$}{<MATH>(lambda  h<SUB>0</SUB>,lambda  h<SUB>1</SUB>,...,lambda  
h<SUB>d</SUB>)</MATH>} represent the
same point.
For a point with \ccHtmlNoLinksFrom{Cartesian} coordinates \ccTexHtml{$(c_0,c_1,\ldots,c_{
d-1})$}{<MATH>(c<SUB>0</SUB>,c<SUB>1</SUB>,...,c<SUB>d-1</SUB>)</MATH>} a
possible homogeneous representation is
\ccTexHtml{$(c_0,c_1,\ldots,c_{d-1},1)$}{<MATH>(c<SUB>0</SUB>,c<SUB>1</SUB>,...,c<SUB>d-1</SUB>,1)</MATH>}.
\ccHtmlNoLinksFrom{Homogeneous} coordinates in fact allow to represent 
objects in a more general space, the projective space
\ccTexHtml{$\P_d$}{<MATH>P<SUB>d</SUB></MATH>}.
In \cgal, we do not compute in projective geometry. Rather, we use 
homogeneous coordinates to avoid division operations,
since the additional coordinate can serve as a common denominator.

\section{Genericity through Parameterization}
Almost all the kernel objects (and the corresponding functions) are
templates with a parameter that allows the user to choose the
representation of the kernel objects. A type that is used as an
argument for this parameter must fulfill certain requirements on 
syntax and semantics. The list of requirements defines an abstract
kernel concept. In \cgal\ such a kernel concept is often also called 
a \emph{representation class} and denoted by \ccc{R}. A representation 
class provides the actual implementations of the kernel objects.
For all kernel objects \ccc{Kernel_object}, the types 
\ccc{CGAL::Kernel_object<R>} and \ccc{R::Kernel_object} are identical.

CGAL offers three families of concrete models for the concept
representation class, two based on the \ccHtmlNoLinksFrom{Cartesian} representation of points
and one based on the homogeneous representation of points. 
The interface of the kernel objects is designed such that it works well
with both \ccHtmlNoLinksFrom{Cartesian} and homogeneous representation, for example, points
in 2D have a constructor with three arguments as well
(the three homogeneous coordinates of the point).
The common interfaces parameterized with a representation class allow
one to develop code independent of the chosen representation. 
We said ``families'' of models, because both families are parameterized too.
A user can choose the number type used to represent the coordinates.

For reasons that will become evident later,
a representation class provides two typenames for number types, 
namely \ccc{R::FT} and \ccc{R::RT}.%
\ccTexHtml{\footnote{The double colon :: is the \CC\ scope operator.}}{}
The type \ccStyle{R::FT} must fulfill the requirements on what is 
called a {\em field type} in \cgal. This roughly means that \ccStyle{R::FT} 
is a type for which  operations $+$, $-$, $*$ and $/$ are defined
with semantics (approximately) corresponding to those of a field in 
a mathematical sense. Note that, strictly speaking, the built-in type 
\ccc{int} does not fullfil the requirements on a field type, since \ccc{int}s
correspond to elements of a ring rather than a field, especially 
operation $/$ is not the inverse of $*$.
The requirements on the type \ccStyle{R::RT} are weaker.
This type must fulfill the requirements on what is called a {\em ring type}
in \cgal. This roughly means that \ccStyle{R::RT} 
is a type for which  operations $+$, $-$, $*$ are defined with semantics
(approximately) corresponding to those of a ring in a mathematical sense. 
A very limited division operation $/$ must be available as well.
It must work for exact (i.e., no remainder) integer divisions only.
Furthermore, both number types should fulfill \cgal's requirements on a
number type. %see  
Note that a ring type is always a field type but not the other way 
round.

\section{Cartesian Kernels}
With \ccc{Cartesian<FieldNumberType>} you can choose \ccHtmlNoLinksFrom{Cartesian} representation 
of coordinates. When you choose \ccHtmlNoLinksFrom{Cartesian} representation you 
have to declare at the same time the type of the coordinates.
A number type used with the \ccc{Cartesian} representation class
should be a {\em field type} as described above.
As mentioned above, the built-in type \ccc{int} is not a field type.
However, for some computations with  \ccHtmlNoLinksFrom{Cartesian} representation, no 
division operation is needed, i.e., 
a {\em ring type} is sufficient in this case.
With \ccc{Cartesian<FieldNumberType>}, both \ccc{Cartesian<FieldNumberType>::FT}
and \ccc{Cartesian<FieldNumberType>::RT} are mapped to number type \ccc{FieldNumberType}.

\ccc{Cartesian<FieldNumberType>} uses reference counting internally to save copying
costs. CGAL also provides \ccc{Simple_cartesian<FieldNumberType>}, a kernel
that uses \ccHtmlNoLinksFrom{Cartesian} representation but no reference counting. 
Debugging is easier with \ccc{Simple_cartesian<FieldNumberType>}, since the coordinates
are stored within the class and hence direct access to the coordinates is
possible.  Depending on the algorithm, it can also be slightly more or less
efficient than \ccc{Cartesian<FieldNumberType>}.
With \ccc{Simple_cartesian<FieldNumberType>}, both \ccc{Simple_cartesian<FieldNumberType>::FT}
and \ccc{Simple_cartesian<FieldNumberType>::RT} are mapped to number type \ccc{FieldNumberType}.

\section{Homogeneous Kernel}
As we mentioned before, homogeneous coordinates permit to avoid 
division operations in numerical computations, since the additional 
coordinate can serve as a common denominator. 
Avoiding divisions can be useful for exact geometric computation. 
With \ccc{Homogeneous<RingNumberType>} you can choose homogeneous representation 
of coordinates with the kernel objects. As for \ccHtmlNoLinksFrom{Cartesian} representation you 
have to declare at the same time the type used to store the homogeneous
coordinates. Since the homogeneous representation allows one to avoid the
divisions, the number type associated with a homogeneous representation class
must be a model for the weaker concept ring type only. However,
some operations provided by this kernel involve division operations, for
example computing squared distances or returning a \ccHtmlNoLinksFrom{Cartesian} coordinate.
To keep the requirements on the number type parameter of \ccc{Homogeneous}
low, the number type \ccStyle{Quotient<RingNumberType>} is used instead.
This number type turns a ring type into a field type. It maintains
numbers as quotients, i.e. a numerator and a denominator. 
Thereby, divisions are circumvented.
With \ccc{Homogeneous<RingNumberType>}, \ccc{Homogeneous<RingNumberType>::FT} is
equal to \ccStyle{Quotient<RingNumberType>} while 
\ccc{Homogeneous<RingNumberType>::RT} is equal to \ccc{RingNumberType}.

\ccc{Homogeneous<RingNumberType>} uses reference counting internally to save
copying costs. CGAL also provides \ccc{Simple_homogeneous<RingNumberType>}, a
kernel that uses \ccHtmlNoLinksFrom{homogeneous} representation but no reference
counting.  Debugging is easier with \ccc{Simple_homogeneous<RingNumberType>},
since the coordinates are stored within the class and hence direct access to
the coordinates is possible.  Depending on the algorithm, it can also be
slightly more or less efficient than \ccc{Homogeneous<RingNumberType>}.
With \ccc{Simple_homogeneous<RingNumberType>},
\ccc{Simple_homogeneous<RingNumberType>::FT} is equal to
\ccStyle{Quotient<RingNumberType>} while
\ccc{Simple_homogeneous<RingNumberType>::RT} is equal to \ccc{RingNumberType}.

\section{Naming conventions}

The use of representation classes not only avoids problems, it
also makes all \cgal\ classes very uniform. They {\bf always} consist of:
\begin{enumerate}
\begin{ccTexOnly}
\itemsep0pt\topsep0pt\partopsep0pt\parskip0pt\parsep0pt
\end{ccTexOnly}

%\item The {\em name space prefix} \ccStyle{}. This avoids name 
%      clashes. It will be dropped as soon as \CC\ compilers support the 
%      concept of name spaces as a feature of the programming language.

\item The {\em capitalized base name} of the geometric object, such as 
      \ccStyle{Point}, \ccStyle{Segment}, \ccStyle{Triangle}.

\item An {\em underscore} followed by the {\em dimension} of the object, 
      for example $\_2$, $\_3$ or $\_d$.

\item A {\em representation class} as parameter, which itself is
      parameterized with a number type, such as \ccStyle{Cartesian<double>} 
      or \ccStyle{Homogeneous<leda_integer>}.
\end{enumerate}

\section{Kernel as a Traits Class}
Algorithms and data structures in the basic library of CGAL are parameterized 
by a traits class that subsumes the objects on which the algorithm or data
structure operates as well as the operations to do so. For most of the 
algorithms and data structures in the basic library you can use a kernel
as a traits class. For some algorithms you even do not have to specify
the kernel; it is detected automatically using the types of the geometric
objects passed to the algorithm. In some other cases, the algorithms 
or data structures needs more than is provided by a kernel. In these
cases, a kernel can not be used as a traits class.

\section{Choosing a Kernel}
If you start with integral \ccHtmlNoLinksFrom{Cartesian} coordinates, many geometric computations
will involve integral numerical values only. Especially, this is true for
geometric computations that evaluate only predicates, which are tantamount to
determinant computations. Examples are triangulation of point sets and 
convex hull computation.
In this case, the \ccHtmlNoLinksFrom{Cartesian} representation is probably the first choice, even
with a ring type. You might use limited precision integer types like
\ccc{int} or \ccc{long}, use \ccc{double} to present your integers (they
have more bits in their mantissa than an \ccc{int} and overflow nicely), or an 
arbitrary precision integer type like the wrapper \ccc{Gmpz} for the
GMP integers or \ccc{leda_integer}. Note, that  unless you use an
arbitrary precision integer type, incorrect results might arise due to
overflow. 

If new points are to be constructed, for example
the \ccHtmlNoLinksFrom{intersection} point of two lines, computation of 
\ccHtmlNoLinksFrom{Cartesian} coordinates usually involves divisions, so
you need to use a field type with \ccHtmlNoLinksFrom{Cartesian} representation
or have to switch to homogeneous representation. 
\ccc{double} is a possible, but imprecise field type.
You can also put any ring type into \ccc{Quotient} to get a
field type and put it into \ccc{Cartesian}, but you better put
the ring type into \ccc{Homogeneous}.
\ccc{leda_rational} and \ccc{leda_real} are valid field types, too.

If it is crucial for you that the computation is reliable, 
the right choice are probably number types that guarantee
exact computation. The number type \ccc{leda_real} guarantees
that all decisions and hence all branchings in a computation
are correct. They also allow you to compute approximations to whatever
precision you need. Furthermore computation with 
\ccc{leda_real} is faster than computation with arbitrary precision
arithmetic. So if you would like to avoid surprises caused by imprecise
computation, this is a good choice. In fact, it is a good choice with
both representations, since divisions slow down the computation of
the reals and hence it might pay-off to avoid them.

Still other people will prefer the built-in
type {\tt double}, because they need speed and can live with
approximate results, or even algorithms that, from time to time,
crash or compute incorrect results due to accumulated rounding errors.
