% +------------------------------------------------------------------------+
% | CGAL Reference Manual:  geowin.tex
% +------------------------------------------------------------------------+

\ccParDims

\chapter{GeoWin}
\label{chapterGeoWin}
\ccChapterAuthor{Matthias B\"asken}


%---------------------------------------------
% Introduction
%---------------------------------------------
\section{Introduction}

The $GeoWin$ data type provides a geometry-oriented environment for the visualization
of algorithms. \\
An instance of the data type $GeoWin$ is an editor that maintains 
a collection of so-called {\em scenes}. 
The scenes are instances of generic data types and maintain an associated {\em container} (a \stl\ or \leda\ list)
of geometric {\em objects}, whose members are displayed according to a set of visual attributes 
(color, line width,line style, etc.). \\
These (visual) attributes are associated with the scenes; but it is for course possible to
assign individual attributes to the geometric objects.
A scene maintained by $GeoWin$ can be {\em visible} or {\em invisible}. 
One of the scenes in the collection can be {\em active}. 
The active scene receives all editing input and can be manipulated
through the interactive interface of $GeoWin$. \\
The interface for the visualization 
of the result and progression of geometric algorithms uses the window data type of \leda. That means that you
must have \leda\ installed if you want to use $GeoWin$.
There are different kinds of scene types available in $GeoWin$ (see the reference manual pages of the
$GeoWin$ class for the details). Some scenes can be edited using the interactive interface of $GeoWin$.
The $GeoWin$ data type can be used for:

\begin{itemize}

\item the construction and display of geometric objects and data structures
\item the visualization of geometric algorithms
\item writing interactive demos for geometric algorithms
\item debugging geometric algorithms.

\end{itemize}

\section{The user interface}

Let us first have a look at the interactive use of $GeoWin$. You should open a $GeoWin$ by starting
one of the demonstration programs. A window with menus File, Edit, Scenes, Window, Options,
Algorithms and Help will pop up with an empty display region and a status line showing 
the coordinates of the position of the mouse and additional information about the edited scene.
In the Help menu you will find, among others, these three menu buttons :

\begin{itemize}
\item {\em About}    - a simple about dialog
\item {\em News}     - tells you what's new in the current release
\item {\em geowin\_buttons}  - teaches you how to use the mouse buttons in $GeoWin$.
\end{itemize}

If you edit a scene of geometric objects, you can 
input a geometric object with the left mouse button or scroll the scene by holding the
left button while moving the mouse. 
By using the middle mouse button you can select an object by
clicking on it or a group of objects by holding the button and moving the mouse (that draws a
box, and all objects intersecting this box will be selected). The right mouse button opens a
menu by clicking on an object. This menu contains the options setup (setup the properties of the
object), select (selection of the object), delete (delete the object), 
object (change the value of the object) and raise (raise the object in its scene). By clicking on
the background with the right mouse button you can access the setup dialog of the whole scene,
where you can setup some parameters of the scene (for instance colors and line parameters). \\
Next we give a short description of the menus in $GeoWin$. Explanations of these menus can also be
found in the other submenus of the help menu.
\\
\\
{\em File} menu:
Here you will find buttons for I/O operations. You can save, load and export (save without header) scenes,
and you can also output the contents of scenes in postscipt format and make screenshots.
In the File menu you will also find the Quit button.
\\
\\
{\em Edit} menu:
A menu, where you find buttons for generating, reading, marking, unmarking, copying, deleting, moving
and rotating objects.
\\
\\
{\em Scenes} menu:
A menu with buttons and submenus for opening, closing, clearing and activating (making a scene the current or
active scene) scenes. This menu also contains a panel for setting the visibility and other options of a scene.
There are also submenus for looking at the textual representation of the contents of the scene.
\\
\\
{\em Window} menu:
A menu providing redraw and zoom (adjusting the window coordinates) operations. Here you can also find a button for
showing 3d output (that is only interesting for scenes supporting this output mode).
\\
\\
{\em Options} menu:
A menu with sub-panels for editing various parameters of $GeoWin$ and for setting the default parameter values for scenes.
\\
\\
{\em Algorithms} menu:
For some scenes (for instance \leda\ point scenes) you can access various built-in algorithms. But for most scenes
the submenus of this menu will be empty.
\\
\\
{\em Help} menu:
In this menu you will find online help for the user interface of $GeoWin$.
\\
\\
{\em Done} :
The Done-button (see the following description of the $GeoWin$ data type).

\section{Global parameters and scene parameters}

After this short discussion of the menus of $GeoWin$ let us have a look at the various parameters and attributes controlling
the behavior of $GeoWin$.
Every scene defines default attributes for its geometric objects. In editable scenes you can change the
attributes for every single object.
First, let's have a look at the default attributes of the scene. Open a scene in $GeoWin$, and then click on the 
background (or use the options submenu in the Scenes menu) to open a panel, where you can set them. You will find the following
options:

\begin{itemize}

\item {\em highlight focus}: If this option is enabled, the object under the mousepointer will be highlighted.
                 (for highlighting $GeoWin$ will use the color of selected boundaries)
		 
\item {\em show/handle def. points?} Some scenes support the movement of a point that helps defining
                 an object (for instance a vertex of a polygon). Using this choice item you can change
		 the behavior of this feature. If you choose $Show$, all the 'defining points' will be
		 shown and movement of these points will be supported. If you choose $Hide$, the
		 'defining points' will not be shown and the movement of these points will not be
		 supported. If you choose $Highlight$, only the 'defining points' of the object under
		 the mousepointer will be shown and movement of these points will be supported.
		 Note again that not all scenes support handling of these 'defining points'.
		 
\item {\em Input objects}: If the scene supports additional input objects, this item will be shown.
                 You can choose one of the input objects or the default input mode (using the 
		 window input operator).
             
\item {\em interior color}: An object that has to be filled will be filled with this color.
             
\item {\em boundary color}: The default color of the boundary of a non selected object in this scene.
                 
\item {\em selected color}: The default color of the boundary of a selected object in this scene.

\item {\em point style}: The point style used in the scene.
                 
\item {\em active/non active line width}: The default line width in active and non active scenes. 
                            Used for painting the boundary of objects.
                
\item {\em line style}: The default line style used in the scene (for painting the boundary). Can be solid, dashed or dotted.
            
\item {\em Debug}: If the scene is a result scene, you can choose here, that before every update
                of this scene the contents of its input scene will be written to a file.
      
\end{itemize}

Using this panel you can set some default attributes for the whole scene. But in editable scenes it is also
possible to change attributes of a single object.
Open a scene and input some objects. Now click with the right mouse button on an object. A little menu appears.
Using the setup button in this menu you get a panel on the screen.
There you can set the following attributes:

\begin{itemize}

\item {\em color of interior}: This color is used to paint the interior of the object.

\item {\em color of boundary}: Boundary color of the object, if it is not selected 

\item {\em line width}: The line width that is used to paint the boundary of the object.

\item {\em line style}: The line style that is used to paint the boundary of the object. 

\item {\em object label}: An attribute defining the label of an object 

\end{itemize}

{\bf Global parameters}

A $GeoWin$ has the window parameters background color, background pixmap and the grid parameters
(grid style, grid dist and show grid).
There are also some additional parameters, that can be set in the panels of the {\em Options} menu.

\begin{itemize}

\item {\em Grid Settings}: Here you can control whether or not a grid should be shown in the
              background of the drawing area of your $GeoWin$ and change grid parameters like grid dist
	      and style.

\item {\em bg texture/bg color}: These parameters control color and texture of the background of 
              your $GeoWin$.

\item {\em Show position}: This parameter controls whether the mouse coordinates should be shown in the status line.
                 
\item {\em Time Outs for Double Click and Dragging, Capture width}: Influences the behavior of the mouse input. 

\item {\em Rational kernel options}: If you are using rational \leda\ types in your $GeoWin$, these options
               are helpful. They let you set the precision (type: int) that is used when converting floating
               point types to rational \leda\ types and let you switch on (1) and off (0) the floating point filter
               used by many functions working with rational \leda\ types. 

\item {\em Measure Time}:  If this flag is true, $GeoWin$ outputs the used time after running a geometric algorithm. 
       
\item {\em 3d Output parameters}: They are only important for scenes supporting 3d output. 

\item {\em Eliminate}: This flag controls whether hidden lines should be eliminated or not. 

\item {\em Solid}: This flag controls whether the 3d output should be painted solid.

\item {\em Speed}: A Parameter that defines the speed of the animation when the three-dimensional output is painted.

\item {\em Scaling parameters}:
Here you can set the {\em x} and {\em y} coordinates of the window display area.

\end{itemize}
After this short discussion of the user interface of $GeoWin$ we'll have a look at a few example programs.

\section{Examples using the $GeoWin$ data type}

In this section two examples will show you how to use $GeoWin$ for writing interactive 
demos for geometric algorithms. You will find these (and a lot of other) demo programs in the
$demo/GeoWin$ directory of your \cgal\ installation. Note again that you must have \leda\ installed
to use $GeoWin$.

In the following examples a few typedefs from $geowin\_support.h$ are used.
They are all in the form

\begin{verbatim}
typedef CGAL::Cartesian<double>  REP;
typedef CGAL::Point_2< REP >     CGALPoint;
typedef std::list<CGALPoint>     CGALPointlist;

typedef CGAL::Circle_2< REP >    CGALCircle;
typedef std::list<CGALCircle>    CGALCirclelist;

typedef CGAL::Line_2< REP >      CGALLine;
typedef std::list<CGALLine>      CGALLinelist;

...

\end{verbatim}

\subsection{Simple demo demonstrating the minimum enclosing circle algorithm}

The following demonstration program first registers lists with \cgal\ 2d points, because such a container
will be used in the editable scene created shortly afterwards.
Then a $GeoWin$ and two geo\_scenes are created.
The first created scene is an editable scene storing \cgal\ 2d Points in a list. The second created scene
is a result scene. The input for this scene is the contents of scene $my\_scene$. For computing the contents
of the result scene $result$ the update object $min\_circ$ is used. $GeoWin$ calls the update function 
of the update object every time the input scene $my\_scene$ changes to update the result scene.
The update function gets the current contents of the input scene (a list of
\cgal\ 2d Points) and
computes the minimum enclosing circle.
After creating the two scenes the visibility flag of the result scene is set and the interactive mode
of $GeoWin$ is started.

\ccHtmlLinksOff

\ccIncludeExampleCode{GeoWin/CGAL_min_circle_2.C}

\ccHtmlLinksOn


\subsection{A Delaunay triangulation demo using redraw and update objects}

The second example is a bit more complex. It uses four user editable input scenes:\\
$my\_scene$ contains points that are the input for an algorithm computing the
Delaunay triangulation.
$line\_scene$ contains lines; they are used for finding triangles of the Delaunay
triangulation intersected by these lines.
$pointloc\_scene$ contains points; they are used for locating triangles of the Delaunay
triangulation.
$p\_scene$ contains points as well; they are used for nearest neighbor searches on the
the Delaunay triangulation.
After the creation of the edit scenes four result scenes are created. The result scene
$sc1$, $res3$ and $res4$ use classes derived from $geowin\_redraw$ and $geowin\_update$ for
updating and redrawing.
After creating the result scenes we tell $GeoWin$ using the $add\_dependence$ method,
that three result scenes should be updated not only when their input scenes change, but also
when other scenes change.

\ccHtmlLinksOff

\ccIncludeExampleCode{CGAL_delaunay_triang_2.C}

\ccHtmlLinksOn
