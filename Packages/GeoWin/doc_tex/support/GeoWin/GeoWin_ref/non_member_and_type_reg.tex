\subsection{Non-Member Functions}

\ccFunction{GeoWin* get_geowin(geo_scene sc);}
{
  returns a pointer to the $GeoWin$ of scene $sc$.
}

\subsection{Registration of types}

The following two function templates can be used to register types for usage
in edit scenes of $GeoWin$. Note that you have to perform this registration before the creation
of the first $GeoWin$.
To use $GeoWin$ with your
own types, the container class storing the geometric objects and your geometric
objects must meet some requirements:
\begin{enumerate}
\item The container must provide the functionality of the \stl\ list (especially 
$CONTAINER\_TYPE::iterator$ and $CONTAINER\_TYPE::value\_type$).
You can use \leda\ lists too.
\item You have to provide a generator function
\begin{verbatim}
void geowin_generate_objects(GeoWin& gw,CONTAINER_TYPE& L);
\end{verbatim}
This generator function can be used to generate input and return it in L.
\item If you don't want to use \leda\ lists, you should provide an info function
\begin{verbatim}
leda_string geowin_info_fcn(const CONTAINER_TYPE& L);
\end{verbatim}
This info function should return a string desribing the container.
\end{enumerate}
For the geometric objects the following requirements must be fulfilled:
\begin{enumerate}
\item default and copy constructor
\begin{verbatim}
TYPE();
TYPE(const TYPE&);
\end{verbatim}
\item operators for writing to an ostream and reading from an istream
\begin{verbatim}
std::ostream& operator<<(std::ostream&, const TYPE&);
std::istream operator>>(std::istream&, TYPE&);
\end{verbatim}
\item operators for writing to/ reading from a \leda\ window
\begin{verbatim}
leda_window& operator<<(leda_window&, const TYPE&);
leda_window& operator>>(leda_window&, TYPE&);
\end{verbatim}
\item an operator for writing to a ps\_file for postscript output
\begin{verbatim}
ps_file& operator<<(ps_file&, const TYPE&);
\end{verbatim}
\item a function for translating the object by a vector (dx,dy):
\begin{verbatim}
void geowin_Translate(TYPE& obj, double dx, double dy);
\end{verbatim}
\item a function for rotating the object  about point(dx,dy) by angle a:
\begin{verbatim}
void geowin_Rotate(TYPE& obj,double dx,double dy,double a);
\end{verbatim}
\item a function for calculating the bounding box of the object:
\begin{verbatim}
void geowin_BoundingBox(const TYPE& obj, double& x1, 
         double& x2, double& y1, double& y2);
\end{verbatim}
The function has to return the minimum and maximum x and y coordinates of the object $obj$.
\item a function for deciding whether the object intersects a box:
\begin{verbatim}
bool geowin_IntersectsBox(const TYPE& obj, double x1,
        double y1, double x2, double y2,bool f)
\end{verbatim}
This function gets the object obj, the coordinates of the box $x1,y1,x2,y2$ and a parameter $f$
(true, if the object is filled, false otherwise). The function should return true if
the object intersects the box $(x1, y1, x2, y2)$, false otherwise.

To register an additional type (that means a container class) for usage in 
edit scenes of $GeoWin$, you
have to call one of the two following functions:
\begin{verbatim}
template<class T> void geowin_init_default_type( T* t, string str);
template<class T> void geowin_init_basic_type(T* t,string str);
\end{verbatim}
str is a name desribing the container and its storage; it must be unique and is used for instance
for saving objects.
The first function will register the container class, build a scene prototype and set functions.
The second function will do the same, but it will not set an info-, translate-, rotate- and generate
function. That means that you have to provide in this case only the remaning functions and operators.
Register your additional types before creating a $GeoWin$.

\end{enumerate}

\ccFunction{void geowin_init_default_type( T* t, leda_string str);}
{ registers a type (container including geometric objects) for usage
  in scenes of $GeoWin$
  and links a scene typename $str$ with the real type. The container can be a \leda\ list 
  or an \stl\ list. Some functions and operators are required (see above).
}

\ccFunction{void geowin_init_basic_type(T* t,leda_string str);}
{ registers a type (container including geometric objects) for usage
  in scenes of $GeoWin$
  and links a scene typename $str$ with the real type. The container can be a \leda\ list
  or an \stl\ list. Some functions and operators are required (see above).
}

For the \cgal\ kernel types (using \stl\ lists as containers)  
2d Point, Line, Ray, Segment, Triangle, Iso Rectangle, Circle,
for the non kernel type Polygon and for the \cgal\ 3d kernel types Point,
Segment, Line, Ray, Triangle and Tetrahedron the required functions and operators
are already provided in $geowin\_support.h$ . For a better understanding of the explained
registration functions see also the demo programs in $demo/GeoWin$ .



