% +------------------------------------------------------------------------+
% | Reference manual page: Map_overlay.tex (Map_overlay)
% +------------------------------------------------------------------------+
% | 
% | Package: ovl (Map_overlay_2)
% | 
% +------------------------------------------------------------------------+

\ccRefPageBegin

%%RefPage: end of header, begin of main body
% +------------------------------------------------------------------------+


% +========================================================================+
%   Map Overlay
% +========================================================================+
\begin{ccRefClass}{Boolean_operations_2<Map_overlay>}
\label{OVL_sec:bops}
    
\ccDefinition
The class \ccClassTemplateName\ provides boolean operations 
on two given subdivisions. 
An object $bops$ of the class \ccClassTemplateName\ is constructed by two input 
subdivisions. 
Their overlay is constructed by the creation of $bops$, in order to compute 
all boolean operations, on the two input subdivisions, available for users.

\ccc{Self} is an abbreviation of the \ccRefName\ type hereafter.

\ccInclude{CGAL/Boolean_operations_2.h}

\ccTypes

  \ccNestedType{Map_overlay}{map overlay type.}
  \ccGlue
  \ccNestedType{Vertex_container}
  {a container of vertices. Serves to store the output vertices.}
  \ccGlue
  \ccNestedType{Halfedges_container}
  {a container of halfedges. Serves to store the output halfedges.}
  \ccGlue
  \ccNestedType{Faces_container}
  {a container of faces. Serves to store the output faces.}

  \ccTypedef{typedef typename Map_overlay::Subdivision  Subdivision;}
  {the subdivision type presenting the overlay and the two input subdivisions.}
  %\ccGlue
  %\ccTypedef{typedef typename Map_overlay::Change_notification  Change_notification;}
  %{the notifier class of the overlay induced by the two input subdivisions.}
  %\ccGlue
  %\ccTypedef{typedef typename Map_overlay::Map_overlay_algorithm  Map_overlay_algorithm;}
  %{the algorithm for constructing the overlay induced by the two input subdivisions.}
  
\ccCreation
  \ccCreationVariable{bops}
    
  \ccConstructor{Boolean_operations_2<Map_overlay>();} {constructs an
     ``empty boolean operations'' object of two ``empty input subdivisions'', 
     each of which contains one unbounded face, which corresponds to the entire plane.}
  
  \ccConstructor{Boolean_operations_2<Map_overlay>(const Subdivision & s1,const Subdivision & s2);}{constructs a boolean operations object out of two planar subdivisions;}
  
  \ccConstructor{Boolean_operations_2<Map_overlay>(const Self & bops);}{copy constructor;}

The \ccClassTemplateName\ defines an assignment operator.

%\ccMethod{const Self& operator=(const Self& bops);}{}

\begin{ccAdvanced}

\ccHeading{Employing Map-Overlay}

A \ccStyle{Boolean_operations_2} object can be constructed out of a 
\ccStyle{Map_overlay_2} object.
In this case, users may construct a \ccStyle{Map_overlay_2} object out of two 
input subdivision, where the a choice of the notifier class, the overlay 
construction algorithm, or the point location strategy can be taken, 
as described in \ccRefPage{CGAL::Map_overlay_2<Subdivision,Notifier>}. 
When constructing a \ccStyle{Boolean_operations_2} object out of two input 
subdivision, as described above, the \ccStyle{Map_overlay_2} object maintained in 
the \ccClassTemplateName\ class, is constructed, when taking the default notifier 
class, overlay construction algorithm and point location.
As described in \ccRefPage{Ovl_map_overlay}, the default choices are not always 
the most efficient. Hence, users may provide their own 
\ccStyle{Map_overlay_2} object, defined under the optimal choices, 
and then, they may construct the \ccStyle{Boolean_operations_2} object defined 
over their \ccStyle{Map_overlay_2} object. 

\ccConstructor{Boolean_operations_2<Map_overlay>(Map_overlay& ovl>);}
{constructs a boolean operations object from $ovl$;}

\end{ccAdvanced}
%\ccThree{Halfedge_handle}{}{\hspace*{11cm}}


The \ccClassTemplateName\ class contains methods which perform
boolean operations over the input subdivisions.
In each such method, we return all vertices, halfedge, and faces 
resulted. 
All these three output parameters are optional 
in our methods, and it is possible for users to call these methods 
only with a subset of the three output parameters.

\ccMethod{void intersection (Faces_container& list_of_faces,
   Halfedges_container& list_of_halfedges, 
   Vertices_container& list_of_vertices);}
{returns all vertices, halfedges and faces in the intersection of the two input subdivisions.}

\ccMethod{void Union (Faces_container& list_of_faces,
   Halfedges_container& list_of_halfedges, 
   Vertices_container& list_of_vertices);}
{returns all vertices, halfedges and faces in the union of the two input subdivisions.}

\ccMethod{void  symmetric_difference (Faces_container& list_of_faces,
   Halfedges_container& list_of_halfedges, 
   Vertices_container& list_of_vertices);}
{returns all vertices, halfedges and faces in the symmetric difference of the two input subdivisions.}

\ccMethod{void  difference (Faces_container& list_of_faces,
   Halfedges_container& list_of_halfedges, 
   Vertices_container& list_of_vertices,
   bool first = true);}
{returns all vertices, halfedges and faces in the difference of the two input subdivisions.
   if $first$ is set to $true$, then the difference is computed as the first input subdivision 
   minus the second one. Otherwise, the difference is computed vise versa}

\ccSeeAlso
\ccc{CGAL::Map_overlay_2}\lcTex{ 
   (\ccRefPage{CGAL::Map_overlay_2<Subdivision,Notifier>})}

\end{ccRefClass}

% FILL IN THE REST OF THE FILES
% +------------------------------------------------------------------------+
%%RefPage: end of main body, begin of footer
\ccRefPageEnd
% EOF
% +------------------------------------------------------------------------+






