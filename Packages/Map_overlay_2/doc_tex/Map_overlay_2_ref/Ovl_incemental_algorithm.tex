% +------------------------------------------------------------------------+
% | Reference manual page: Ovl_incremental_algorithm.tex (Map_overlay)
% +------------------------------------------------------------------------+
% | 
% | Package: ovl (Map_overlay_2)
% | 
% +------------------------------------------------------------------------+

\ccRefPageBegin

%%RefPage: end of header, begin of main body
% +------------------------------------------------------------------------+

%**************************************************************************
\renewcommand{\ccRefPageBegin}{\begin{ccAdvanced}}
\renewcommand{\ccRefPageEnd}{\end{ccAdvanced}}

\begin{ccRefClass}{Map_overlay_incremental<Subdivision,Notifier>}
\label{OVL_sec:incremental}

The \ccRefName\ class implements the incremental algorithm which inserts 
all curves of the two creators one by one in a random order. 
Theoretically, when maintaining the trapezoidal decomposition of 
the overlay constructed so far, the construction of the overlay 
takes randomized expected time $O(N\log{N} + k)$ employing Mulmuley game~\cite{m-cgitr-93},
where $N$ is the number of curves the two creators have in total, 
and $k$ is the number of intersections betweem curves 
from the first creator and curves from the second one. 
However, the usage of the \ccc{Planar Map With Intersections}
package employing trapezoidal decomposition is too slow in practice. 
Hence, we construct the arrangement presenting the overlay 
using the ``walk along a line'' strategy, implying that the construction 
running time is $O(n^2)$.

\ccInclude{CGAL/Map_overlay_incremental.h}

\ccIsModel
  \ccc{MapOverlayAlgorithm_2}

\ccInheritsFrom
  \ccc{Map_overlay_base<Subdivision,Notifier>}

\ccSeeAlso
   Discussion of the different overlay construction strategies in the introduction
of \ccc{Map_overlay} reference pages\lcTex{ (\ccRefPage{Ovl_Ref_intro})}.

\end{ccRefClass}
\renewcommand{\ccRefPageBegin}{}
\renewcommand{\ccRefPageEnd}{}


% +------------------------------------------------------------------------+
%%RefPage: end of main body, begin of footer
\ccRefPageEnd
% EOF
% +------------------------------------------------------------------------+
