% +------------------------------------------------------------------------+
% | Reference manual page: Polygons_do_intersect_2.tex (Map_overlay)
% +------------------------------------------------------------------------+
% | 
% | Package: ovl (Map_overlay_2)
% | 
% +------------------------------------------------------------------------+

\ccRefPageBegin

%%RefPage: end of header, begin of main body
% +------------------------------------------------------------------------+


% +========================================================================+
%   Map Overlay
% +========================================================================+
\begin{ccRefFunctionObjectClass}{Polygons_do_intersect_2<Traits>}
\label{OVL_sec:polygon_do_intersect}
    
\ccDefinition
The class \ccClassTemplateName\ is a function object,
checking whether two given polygons have a non-empty 
intersection.


\ccInclude{CGAL/Polygons_bops_2.h}

\ccMemberFunction{template <class Polygon>
   bool operator()(const Polygon& polygon1,const Polygon& polygon2);}
{Checking whether the two given polygons intersect.
This function returns a boolean flag, 
indicating whether the intersection is non-empty.
}

\ccHeading{Requirements}
\begin{enumerate}
   \item
   \ccc{Traits} is a model to the \ccc{PolygonBopsTraits_2} concept.
   \item
   \ccc{Polygon} is a model of the CGAL polygon.
\end{enumerate}

\ccSeeAlso

\ccc{CGAL::Polygons_intersection_2} \\
\ccc{CGAL::Polygons_union_2} \\
\ccc{CGAL::Polygons_difference_2} \\
\ccc{CGAL::Polygons_symmetric_difference_2} \\

% \ccImplementation

% The implementation uses the container $map$ defined in STL for
% implementing the event queue and the status line. The implementation
% also keeps per each input curve an associated container of all its
% intersection points ordered from left to right, and also keeps per
% each event point all its outcoming curves. The complexity of this
% algorithm is $O(n\log{n} + k\log{n})$ where $n$ is the number of the input
% curves and $k$ is the number of intersection points induced by these
% curves.


\end{ccRefFunctionObjectClass}

% FILL IN THE REST OF THE FILES
% +------------------------------------------------------------------------+
%%RefPage: end of main body, begin of footer
\ccRefPageEnd
% EOF
% +------------------------------------------------------------------------+






