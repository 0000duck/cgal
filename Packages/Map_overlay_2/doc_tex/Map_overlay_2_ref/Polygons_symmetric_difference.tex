% +------------------------------------------------------------------------+
% | Reference manual page:  Polygons_symmetric_difference_2.tex (Map_overlay)
% +------------------------------------------------------------------------+
% | 
% | Package: ovl (Map_overlay_2)
% | 
% +------------------------------------------------------------------------+

\ccRefPageBegin

%%RefPage: end of header, begin of main body
% +------------------------------------------------------------------------+


% +========================================================================+
%   Map Overlay
% +========================================================================+
\begin{ccRefFunctionObjectClass}{Polygons_symmetric_difference_2<Traits>}
\label{OVL_sec:polygon_symmetric_difference}
    
\ccDefinition
The class \ccClassTemplateName\ is a function object,
returning the symmetric intersection of two given polygons.


\ccInclude{CGAL/Polygons_bops_2.h}

\ccMemberFunction{
   template <class Polygon,
   class PolygonOutputIterator,
   class CurvesOutputIterator,
   class PointsOutputIterator>    
   void operator()(
   const Polygon& polygon1,
   const Polygon& polygon2,
   PolygonOutputIterator polygons,
   CurvesOutputIterator curves,
   PointsOutputIterator points);}
{Computes symmetric difference on two input polygons.
This function returns three containers (as output irerators):
The first container is the points in the symmetric difference, 
the second is the subcurves in the symmetric difference,
and the third is all polygons in the symmetric difference.
The three output containers are given as template
output iterators.
}

\ccHeading{Requirements}
\begin{enumerate}
   \item
   \ccc{Traits} is a model to the \ccc{PolygonBopsTraits_2} concept.
   \item
   \ccc{Polygon} is a model of the CGAL polygon.
   \item
   \ccc{PolygonOutputIterator::value_type} is a Polygon.
   \item    
   \ccc{CurvesOutputIterator::value_type} is equivalent to \ccc{Traits::Curve_2}.
   \item
   \ccc{PointsOutputIterator::value_type} is equivalent to \ccc{Traits::Point_2}.
\end{enumerate}

\ccSeeAlso

\ccc{CGAL::Polygons_do_intersect_2} \lcTex{ 
   (\ccRefPage{CGAL::Polygons_do_intersect_2<Traits>})}\\
\ccc{CGAL::Polygons_intersection_2} \lcTex{ 
   (\ccRefPage{CGAL::Polygons_intersection_2<Traits>})}\\
\ccc{CGAL::Polygons_union_2} \lcTex{ 
   (\ccRefPage{CGAL::Polygons_union_2<Traits>})}\\
\ccc{CGAL::Polygons_difference_2} \lcTex{ 
   (\ccRefPage{CGAL::Polygons_difference_2<Traits>})}\\


% \ccImplementation

% The implementation uses the container $map$ defined in STL for
% implementing the event queue and the status line. The implementation
% also keeps per each input curve an associated container of all its
% intersection points ordered from left to right, and also keeps per
% each event point all its outcoming curves. The complexity of this
% algorithm is $O(n\log{n} + k\log{n})$ where $n$ is the number of the input
% curves and $k$ is the number of intersection points induced by these
% curves.


\end{ccRefFunctionObjectClass}

% FILL IN THE REST OF THE FILES
% +------------------------------------------------------------------------+
%%RefPage: end of main body, begin of footer
\ccRefPageEnd
% EOF
% +------------------------------------------------------------------------+






