%% ==============================================================
%% Specification: Rectangular p-Centers
%% --------------------------------------------------------------
%% file  : spec_rectangular_p_centers.awi
%% author: Michael Hoffmann
%% $Id$
%% ==============================================================

\clearpage
\section{Rectangular $p$-Centers}
\label{sec_RectangularPCenters}
\cgalColumnLayout

This section describes a function to compute rectilinear $p$-centers
of a planar point set, i.e. a set of $p$ points such that the maximum
minimal distance between both sets is minimized.

\ccHtmlNoClassToc
\begin{ccHtmlClassFile}{rectangular_p_centers.html}
  {Definition of function \ccc{CGAL_rectangular_p_center_2}}
  \ccTexHtml{}{
    <H3>Rectangular <I>p</I>-Centers</H3>
    
    This section describes a function to compute rectilinear
    <I>p</I>-centers of a planar point set, i.e. a set of <I>p</I>
    points such that the maximum minimal distance between both sets is
    minimized.}
  
  More formally the problem can be defined as follows.
  
  \ccTexHtml{Given a finite set $\mathcal{P}$ of points, compute a
    point set $\mathcal{C}$ with $|\mathcal{C}| \le p$ such that the
    $p$-radius of $\mathcal{P}$,
    $$
    rad_p(\mathcal{P}) := \max_{P \in \mathcal{P}} \min_{Q \in
      \mathcal{C}} || P - Q ||_\infty
    $$
    is minimized. We can interpret $\mathcal{C}$ as the best
    approximation (with respect to the given metric) for $\mathcal{P}$
    with at most $p$ points.}{Given a finite set <IMG WIDTH=12
    HEIGHT=12 ALIGN=BOTTOM ALT="tex2html_wrap_inline17"
    SRC="./MatrixSearch_pcenter1.gif" > of points, compute a point set
    <IMG WIDTH=9 HEIGHT=13 ALIGN=BOTTOM ALT="tex2html_wrap_inline19"
    SRC="./MatrixSearch_pcenter2.gif" > with <IMG WIDTH=46 HEIGHT=24
    ALIGN=MIDDLE ALT="tex2html_wrap_inline21"
    SRC="./MatrixSearch_pcenter3.gif" > such that the <I>p</I>-radius
    of <IMG WIDTH=12 HEIGHT=12 ALIGN=BOTTOM
    ALT="tex2html_wrap_inline17" SRC="./MatrixSearch_pcenter1.gif" > ,
    <P> <IMG WIDTH=358 HEIGHT=24 ALIGN=BOTTOM ALT="displaymath27"
    SRC="./MatrixSearch_pcenter4.gif" > <P> is minimized. We can
    interpret <IMG WIDTH=9 HEIGHT=13 ALIGN=BOTTOM
    ALT="tex2html_wrap_inline19" SRC="./MatrixSearch_pcenter2.gif" >
    as the best approximation (with respect to the given metric) for
    <IMG WIDTH=12 HEIGHT=12 ALIGN=BOTTOM ALT="tex2html_wrap_inline17"
    SRC="./MatrixSearch_pcenter1.gif" > with at most <I>p</I> points.}

\ccInclude{CGAL/rectangular_p_center_2.h}

  \def\ccLongParamLayout{\ccTrue} 

  \ccGlobalFunction{template < class ForwardIterator, class
    OutputIterator, class Traits > OutputIterator
    CGAL_rectangular_p_center_2( ForwardIterator f, ForwardIterator l,
    OutputIterator o, Traits::FT& r, int p, const Traits& t =
    Default_traits);}
  
  computes rectilinear \ccc{p}-centers for the point set described by
  the range [\ccc{f}, \ccc{l}), sets \ccc{r} to the corresponding
  $p$-radius, writes the at most \ccc{p} center points to \ccc{o} and
  returns the past-the-end iterator of this sequence.

  \ccHeading{Precondition}
  \begin{enumerate}
  \item If \ccc{t} is specified explicitly, \ccc{Traits} satisfies the
    requirements stated in section
    \ref{req_RectangularP-centerTraitsClasses},
  \item Value type of \ccc{ForwardIterator} is \ccc{Traits::Point_2}
    or -- if \ccc{t} is not specified explicitly --
    \ccc{CGAL_Point_2<R>} for some representation class \ccc{R},
  \item \ccc{OutputIterator} accepts the value type of
    \ccc{ForwardIterator} as value type,
  \item the range [\ccc{f}, \ccc{l}) is not empty \textit{and}
  \item 2 $\le$ \ccc{p} $\le$ 4.
  \end{enumerate}

  \ccHeading{Note}
  
  On compilers not supporting member function templates, the parameter
  \ccc{ForwardIterator} is fixed to \ccc{vector<Point_2>::iterator}
  and \ccc{OutputIterator} is fixed to
  \ccc{back_insert_iterator<vector<Point_2>>} where \ccc{Point_2} is
  \ccc{Traits::Point_2}.
  
  \ccImplementation The implementation uses sorted matrix search (see
  section \ref{secSortedMatrixSearch}) and fast algorithms for
  piercing rectangles\cite{sw-rpppp-96} which leads to a runtime
  complexity of $\mathcal{O}(n \cdot \log n)$ where $n$ is the number
  of input points.
  
  \ccExample The following code generates a random set of ten points
  and computes its two-centers.

  \ccIncludeVerbatim{rectangular_p_center_2_example_noheader.C}

\end{ccHtmlClassFile}

\begin{ccAdvanced}
  \ccHtmlNoClassToc\ccHtmlNoClassIndex\begin{ccClass}{Rpc_traits}
    \ccCreationVariable{t}\ccTagFullDeclarations
    
    \subsection{Requirements for Rectangular $p$-center Traits Classes}
    \label{req_RectangularP-centerTraitsClasses}
    
    \ccDefinition A class \ccClassName\ has to provide the following
    types in order to qualify as a traits class for
    \ccc{CGAL_rectangular_p_center_2}.
    
    \ccTypes 
    
    \ccNestedType{Iso_rectangle_2}{class used for representing
      axis-parallel rectangles.}
    
    \ccNestedType{Point_2}{class used for representing the input
      points.}
    
    \ccNestedType{FT}{class used for doing computations on point
      coordinates (has to fulfill field-type requirements).}
    
    \ccNestedType{X}{adaptable unary function class: \ccc{Point_2}
      $\rightarrow$ \ccc{FT} computing the $x$-coordinate of a given
      point.}
    
    \ccNestedType{Y}{adaptable unary function class: \ccc{Point_2}
      $\rightarrow$ \ccc{FT} computing the $y$-coordinate of a given
      point.}
    
    \ccNestedType{Xmin}{adaptable unary function class:
      \ccc{Iso_rectangle_2} $\rightarrow$ \ccc{FT} computing the
      minimal $x$-coordinate of a given rectangle.}
    
    \ccNestedType{Xmax}{adaptable unary function class:
      \ccc{Iso_rectangle_2} $\rightarrow$ \ccc{FT} computing the
      maximal $x$-coordinate of a given rectangle.}
    
    \ccNestedType{Ymin}{adaptable unary function class:
      \ccc{Iso_rectangle_2} $\rightarrow$ \ccc{FT} computing the
      minimal $y$-coordinate of a given rectangle.}
    
    \ccNestedType{Ymax}{adaptable unary function class:
      \ccc{Iso_rectangle_2} $\rightarrow$ \ccc{FT} computing the
      maximal $y$-coordinate of a given rectangle.}
    
    \ccNestedType{Build_point}{adaptable binary function class:
      \ccc{FT} $\times$ \ccc{FT} $\rightarrow$ \ccc{Point_2} creating
      a point with the given coordinates.}
    
    \ccNestedType{Build_rectangle}{adaptable unary function class:
      \ccc{Point_2} $\rightarrow$ \ccc{Iso_rectangle_2} creating a
      rectangle containing exactly the given point.}

  \end{ccClass}
\end{ccAdvanced}

%% --------------------------------------------------------------
%% EOF spec_rectangular_p_centers.awi
%% --------------------------------------------------------------
