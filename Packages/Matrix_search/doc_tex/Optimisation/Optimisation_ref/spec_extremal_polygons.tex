%% ==============================================================
%% Specification: Computing Extremal Polygons
%% --------------------------------------------------------------
%% file  : spec_extremal_polygons.awi
%% author: Michael Hoffmann
%% $Id$
%% ==============================================================

\clearpage
\section{Computing Extremal Polygons}
\label{secComputingExtremalPolygons}
\cgalColumnLayout

This section describes several functions to compute a maximal $k$-gon
$P_k$ that can be inscribed into a given convex polygon $P$. The
criterion for maximality can be chosen freely by defining an
appropriate traits class as specified in section
\ref{req_ExtremalPolygonTraits}. For \cgal\ point classes there are
two predefined traits classes to compute a maximum area (see section
\ref{secMaximumAreaInscribedKgon}) resp.  perimeter (see section
\ref{secMaximumPerimeterInscribedKgon}) inscribed $k$-gon.

\ccHtmlNoClassToc
\begin{ccHtmlClassFile}{computing_maximum_area_inscribed_k_gon.html}
  {Function Declaration of \ccc{maximum_area_inscribed_k_gon}}
  \ccHtmlNoClassIndex\ccHtmlNoClassLinks
  %% class wrapper to keep the font at a uniform size:
  \begin{ccClass}{dummy}
    \ccHtmlNoIndex\subsection{Computing a Maximum Area Inscribed $k$-gon}
  \label{secMaximumAreaInscribedKgon}
  \end{ccClass}
  
  This section describes a function to compute a maximal area $k$-gon
  $P_k$ that can be inscribed into a given convex polygon $P$. Note
  that $P_k$ is not unique in general, but it can be chosen in such a
  way that its vertices form a subset of the vertex set of $P$.

  \ccInclude{CGAL/extremal_polygon_2.h}

  \def\ccLongParamLayout{\ccTrue} 
  
  \ccGlobalFunction{
    template < class RandomAccessIC, class OutputIterator >
    OutputIterator
    maximum_area_inscribed_k_gon(
    RandomAccessIC points_begin,
    RandomAccessIC points_end,
    int k,
    OutputIterator o);}
  
  computes a maximum area inscribed $k$-gon of the convex polygon
  described by [\ccc{points_begin}, \ccc{points_end}), writes its
  vertices to \ccc{o} and returns the past-the-end iterator of this
  sequence.
  
  \ccHeading{Precondition}
  \begin{enumerate}
  \item Value type of \ccc{RandomAccessIC} has to be
    \ccc{Point_2<R>} for some representation class \ccc{R}.
  \item \ccc{OutputIterator} accepts the value type of
    \ccc{RandomAccessIC} as value type,
  \item the -- at least three -- points denoted by the range
    [\ccc{points_begin}, \ccc{points_end}) form the boundary of a convex
    polygon (oriented clock-- or counterclockwise) \textit{and}
  \item $k \ge 3$.
  \end{enumerate}

  \ccHeading{Note}
  
  On compilers not supporting member function templates, the parameter
  \ccc{RandomAccessIC} is fixed to \ccc{vector<Point_2>::iterator}
  where \ccc{Point_2} is the value type of \ccc{RandomAccessIC}.
  
  \ccImplementation The implementation uses monotone matrix
  search\cite{akmsw-gamsa-87} and has a worst case running time of $O(k
  \cdot n + n \cdot \log n)$, where $n$ is the number of vertices in
  $P$.

  \ccExample The following code generates a random convex polygon
  \ccc{p} with ten vertices and computes the maximum area inscribed
  five-gon of \ccc{p}.

  \ccIncludeVerbatim{extremal_polygon_2_example_area.C}

\end{ccHtmlClassFile}
    
\ccHtmlNoClassToc
\begin{ccHtmlClassFile}{computing_maximum_perimeter_inscribed_k_gon.html}
  {Function Declaration of \ccc{maximum_perimeter_inscribed_k_gon}}
  \ccHtmlNoClassIndex\ccHtmlNoClassLinks
  %% class wrapper to keep the font at a uniform size:
  \begin{ccClass}{dummy}
    \ccHtmlNoIndex\subsection{Computing a Maximum Perimeter Inscribed
      $k$-gon}
    \label{secMaximumPerimeterInscribedKgon}
  \end{ccClass}
  
  This section describes a function to compute a largest perimeter
  $k$-gon $P_k$ that can be inscribed in a given convex polygon $P$.
  Note that $P_k$ is not unique in general, but we know that its
  vertices form a subset of the vertex set of $P$.

  \ccInclude{CGAL/extremal_polygon_2.h}

  \def\ccLongParamLayout{\ccTrue}
  \ccGlobalFunction{
    template < class RandomAccessIC, class OutputIterator >
    OutputIterator
    maximum_perimeter_inscribed_k_gon(
    RandomAccessIC points_begin,
    RandomAccessIC points_end,
    int k,
    OutputIterator o);}
  
  computes a maximum perimeter inscribed $k$-gon of the convex polygon
  described by [\ccc{points_begin}, \ccc{points_end}), writes its
  vertices to \ccc{o} and returns the past-the-end iterator of this
  sequence.

  \ccHeading{Precondition}
  \begin{enumerate}
  \item Value type of \ccc{RandomAccessIC} has to be
    \ccc{Point_2<R>} for some representation class \ccc{R},
  \item there is a global function \ccc{R::FT sqrt( R::FT)}
    defined that computes the squareroot of a number,
  \item \ccc{OutputIterator} accepts the value type of
    \ccc{RandomAccessIC} as value type,
  \item the -- at least three -- points denoted by the range
    [\ccc{points_begin}, \ccc{points_end}) form the boundary of a
    convex polygon (oriented clock-- or counterclockwise) \textit{and}
  \item $k \ge 2$.
  \end{enumerate}

  \ccTagDefaults

  \ccHeading{Note}
  
  On compilers not supporting member function templates, the parameter
  \ccc{RandomAccessIC} is fixed to \ccc{vector<Point_2>::iterator}
  where \ccc{Point_2} is the value type of \ccc{RandomAccessIC}.
  
  \ccImplementation The implementation uses monotone matrix
  search\cite{akmsw-gamsa-87} and has a worst case running time of $O(k
  \cdot n + n \cdot \log n)$, where $n$ is the number of vertices in
  $P$.

  \ccExample The following code generates a random convex polygon
  \ccc{p} with ten vertices and computes the maximum perimeter inscribed
  five-gon of \ccc{p}.

  \ccIncludeVerbatim{extremal_polygon_2_example_perimeter.C}

\end{ccHtmlClassFile}

\begin{ccAdvanced}
  \ccHtmlNoClassToc
  \begin{ccHtmlClassFile}{computing_general_extremal_polygons.html}
    {Function Declaration of \ccc{extremal_polygon}}
    \ccHtmlNoClassIndex\ccHtmlNoClassLinks
    %% class wrapper to keep the font at a uniform size:
    \begin{ccClass}{dummy}
      \ccHtmlNoIndex\subsection{Computing General Extremal
        Polygons}\label{secGeneralExtremalPolygons}
    \end{ccClass}
    
    This section describes a general function to compute a maximal
    $k$-gon $P_k$ that can be inscribed in a given convex polygon $P$.
    The criterion for maximality and some basic operations have to
    specified in an appropriate traits class as specified in section
    \ref{req_ExtremalPolygonTraits}.
    
    \ccInclude{CGAL/extremal_polygons_2.h}

    \def\ccLongParamLayout{\ccTrue} 
    
    \ccGlobalFunction{
      template < class RandomAccessIC, class OutputIterator, class Traits >
      OutputIterator
      extremal_polygon(
      RandomAccessIC points_begin,
      RandomAccessIC points_end,
      int k,
      OutputIterator o,
      const Traits& t);}
    
    computes a maximal (as specified by \ccc{t}) inscribed $k$-gon of
    the convex polygon described by [\ccc{points_begin},
    \ccc{points_end}), writes its vertices to \ccc{o} and returns the
    past-the-end iterator of this sequence.
    
    \ccHeading{Precondition}
    \begin{enumerate}
    \item \ccc{Traits} has to satisfy the requirements stated in section
      \ref{req_ExtremalPolygonTraits},
    \item Value type of \ccc{RandomAccessIC} must be
      \ccc{Traits::Point_2},
    \item \ccc{OutputIterator} accepts \ccc{Traits::Point_2} as value
      type,
    \item the -- at least three -- points denoted by the range
      [\ccc{points_begin}, \ccc{points_end}) form the boundary of a
      convex polygon (oriented clock-- or counterclockwise) \textit{and}
    \item $k \ge \ccc{t.min_k()}$.
    \end{enumerate}
    
    \ccImplementation The implementation uses monotone matrix
    search\cite{akmsw-gamsa-87} and has a worst case running time of
    $O(k \cdot n + n \cdot \log n)$, where $n$ is the number of vertices
    in $P$.
  \end{ccHtmlClassFile}
  
  \ccHtmlNoClassToc\ccHtmlNoClassIndex\begin{ccClass}{Exp_traits}
    \ccCreationVariable{t}\ccTagFullDeclarations
    
    \subsection{Requirements for Extremal Polygon Traits
      Classes}\label{req_ExtremalPolygonTraits}
    
    \ccDefinition A class \ccClassName\ has to provide the following
    types and operations in order to qualify as a traits class for
    \ccc{extremal_polygon}.
    
    \ccTypes 
    
    \ccNestedType{Point_2}{class used for representing the input
      points.}
    
    \ccNestedType{FT}{class used for doing computations on point
      coordinates (has to fulfill field-type requirements).}
    
    \ccNestedType{Operation}{AdaptableBinaryFunction class \ccc{op}:
      \ccc{Point_2} $\times$ \ccc{Point_2} $\rightarrow$ \ccc{FT}.
      Together with \ccc{init} this operation recursively defines the
      objective function to maximize.  Let $p$ and $q$ be two vertices
      of a polygon $P$ such that $q$ precedes $p$ in the oriented
      vertex chain of $P$ starting with vertex $root$.  Then
      \ccc{op(p,q)} returns the value by which an arbitrary
      sub-polygon of $P$ with vertices from $[root,\, q]$ increases
      when $p$ is added to it. E.g. in the maximum area case this is
      the area of the triangle $(root,\, q,\, p)$.}

    \ccOperations
    
    \ccMemberFunction{int min_k() const;}{returns the minimal $k$ for
      which a maximal $k$-gon can be computed. (e.g. in the maximum
      area case this is three.)}
    
    \ccMemberFunction{FT init( const Point_2& p, const Point_2& q)
      const;}{returns the value of the objective function for a
      polygon consisting of the two points \ccc{p} and \ccc{q}. (e.g.
      in the maximum area case this is \ccc{FT( 0)}.)}
    
    \ccMemberFunction{Operation operation( const Point_2& p)
      const;}{return \ccc{Operation} where \ccc{p} is the fixed $root$
      point.}
    
    \ccMemberFunction{template < class RandomAccessIC, class
      OutputIterator > OutputIterator compute_min_k_gon(
      RandomAccessIC points_begin, RandomAccessIC points_end, FT&
      max_area, OutputIterator o) const;}{writes the points of
      [\ccc{points_begin}, \ccc{points_end}) forming a
      \ccc{min_k()}-gon rooted at \ccc{points_begin[0]} of maximal
      value to o and returns the past-the-end iterator for that
      sequence (== \ccc{o + min_k()}).}
    
    \ccMemberFunction{template < class RandomAccessIC > bool
      is_convex( RandomAccessIC points_begin, RandomAccessIC
      points_end) const;}{returns true, iff the points
      [\ccc{points_begin}, \ccc{points_end}) form a convex chain.}
    
    \ccHeading{Notes}
    \begin{itemize}
    \item \ccClassName\ccc{::is_convex} is only used for precondition
      checking. Therefore it needs not to be specified, in case that
      precondition checking is disabled.
    \item On compilers not supporting member function templates,
      \ccc{RandomAccessIC} is fixed to \ccc{vector<Point_2>::iterator}
      and \ccc{OutputIterator} is fixed to
      \ccc{vector<int>::reverse_iterator}.
      
    \end{itemize}
    
    \ccSeeAlso \ccInclude{CGAL/Extremal_polygon_traits_2.h}
    
    The classes \ccc{Kgon_area_traits<R>} and
    \ccc{Kgon_perimeter_traits<R>} (templatized with a \cgal\ 
    representation class) both fulfill these requirements.
    
  \end{ccClass}
\end{ccAdvanced}

%% --------------------------------------------------------------
%% EOF spec_extremal_polygons.awi
%% --------------------------------------------------------------
