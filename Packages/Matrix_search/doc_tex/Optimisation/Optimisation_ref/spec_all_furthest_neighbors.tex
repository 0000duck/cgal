%% ==============================================================
%% Specification: All Furthest Neighbors
%% --------------------------------------------------------------
%% file  : spec_all_furthest_neighbors.awi
%% author: Michael Hoffmann
%% $Id$
%% ==============================================================

\clearpage
\section{All Furthest Neighbors}
\label{secAllFurthestNeighbors}
\cgalColumnLayout

This section describes a function to compute all furthest neighbors
for the vertices of a convex polygon. 

\ccHtmlNoClassToc
\begin{ccHtmlClassFile}{all_furthest_neighbors.html}
  {Definition of function \ccc{all_furthest_neighbors}}
  \ccTexHtml{}{
    <H3>All Furthest Neighbors</H3>
    
    This section describes a function to compute all furthest
    neighbors for the vertices of a convex polygon $P$, i.e. for each
    vertex $v$ of $P$ a vertex $f_v$ of $P$ such that the distance
    between $v$ and $f_v$ is maximized.}
  
  \ccInclude{CGAL/all_furthest_neighbors_2.h}

  \def\ccLongParamLayout{\ccTrue} 

  \ccGlobalFunction{
    template < class RandomAccessIC, class OutputIterator, class Traits >
    OutputIterator
    all_furthest_neighbors(
    RandomAccessIC points_begin,
    RandomAccessIC points_end,
    OutputIterator o,
    Traits t = Default_traits);}
  
  computes all furthest neighbors for the vertices of the convex
  polygon described by the range [\ccc{points_begin},
  \ccc{points_end}), writes their indices (relative to
  \ccc{points_begin}) to \ccc{o}\footnote{i.e. the furthest neighbor
    of \ccc{points_begin[}i\ccc{]} is \ccc{points_begin[}$i$-th number
    written to \ccc{o}\ccc{]}} and returns the past-the-end iterator
  of this sequence.

  \ccHeading{Precondition}
  \begin{enumerate}
  \item If \ccc{t} is specified explicitly, \ccc{Traits} satisfies the
    requirements stated in section \ref{req_AllFurthestNeighborsTraits},
  \item Value type of \ccc{RandomAccessIC} is \ccc{Traits::Point_2} or
    -- if \ccc{t} is not specified explicitly -- \ccc{Point_2<R>}
    for some representation class \ccc{R},
  \item \ccc{OutputIterator} accepts \ccc{int} as value type
    \textit{and}
  \item the points denoted by the non-empty range [\ccc{points_begin},
    \ccc{points_end}) form the boundary of a convex polygon $P$
    (oriented clock-- or counterclockwise).
  \end{enumerate}
  
  \ccImplementation The implementation uses monotone matrix
  search\cite{akmsw-gamsa-87} (see also section
  \ref{secMonotoneMatrixSearch}). Its runtime complexity is linear in
  the number of vertices of $P$.

  \ccExample The following code generates a random convex polygon
  \ccc{p} with ten vertices, computes all furthest neighbors and writes
  the sequence of their indices (relative to \ccc{points_begin}) to
  \ccc{cout} (e.g. a sequence of \ccc{4788911224} means the furthest
  neighbor of \ccc{points_begin[0]} is \ccc{points_begin[4]}, the
  furthest neighbor of \ccc{points_begin[1]} is \ccc{points_begin[7]}
  etc.).

  \ccIncludeVerbatim{all_furthest_neighbors_2_example_noheader.C}

\end{ccHtmlClassFile}

\begin{ccAdvanced}
  \ccHtmlNoClassToc\ccHtmlNoClassIndex\begin{ccClass}{Afn_traits}
    \ccCreationVariable{t}\ccTagFullDeclarations
    
    \subsection{Requirements for All Furthest Neighbors Traits
      Classes}\label{req_AllFurthestNeighborsTraits}
    
    \ccDefinition A class \ccClassName\ has to provide the following
    types and operations in order to qualify as a traits class for
    \ccc{all_furthest_neighbors}.
    
    \ccTypes 
    
    \ccNestedType{Point_2}{class used for representing the input
      points.}
    
    \ccNestedType{FT}{class used for doing computations on point
      coordinates (has to fulfill field-type requirements).}
    
    \ccNestedType{Distance}{AdaptableBinaryFunction class: \ccc{Point_2}
      $\times$ \ccc{Point_2} $\rightarrow$ \ccc{FT} computing the
      squared Euclidean distance between two points.}

    \ccOperations
    
    \ccMemberFunction{template < class RandomAccessIC > bool is_convex(
      RandomAccessIC points_begin, RandomAccessIC points_end)
      const;}{returns true, iff the points [\ccc{points_begin},
      \ccc{points_end}) form a convex chain.}
    
    \ccHeading{Notes}
    \begin{itemize}
    \item \ccClassName\ccc{::is_convex} is only used for precondition
      checking.
    \item On compilers not supporting member function templates,
      \ccc{RandomAccessIC} is fixed to
      \ccc{vector<Point_2>::iterator}.
    \end{itemize}
  \end{ccClass}
\end{ccAdvanced}

%% --------------------------------------------------------------
%% EOF spec_all_furthest_neighbors.awi
%% --------------------------------------------------------------
