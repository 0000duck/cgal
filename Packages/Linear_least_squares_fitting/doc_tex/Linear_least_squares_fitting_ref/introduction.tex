% +------------------------------------------------------------------------+
% | Reference manual page
% +------------------------------------------------------------------------+
% | March 2005  Pierre Alliez and Sylvain Pion
% | Package: Linear Least Squares Fitting
% | 
% |
% +------------------------------------------------------------------------+

\chapter{Principal Component Analysis}
\label{ref_chapter_pca}

\ccChapterAuthor{Pierre Alliez and Sylvain Pion}

\subsection*{Definition}

Given a point set in $R^d$, linear least squares fitting amounts to
find the linear sub-space of $R^d$ which minimizes the sum of squared
distances from the points to their projection onto this linear
sub-space. This problem is equivalent to search for the linear
sub-space which maximizes the variance of projected points, the latter
being obtained by eigen decomposition of the covariance
matrix. Eigenvectors corresponding to large eigenvalues are the
directions in which the data has strong component, or equivalently
large variance. If eigenvalues are the same there is no preferable
sub-space.\\

This \cgal\ package implements the linear least squares fitting for
several objects of a CGAL 2D or 3D kernel: the best fit 2D line for 2D
point and triangle sets, and the best fit 3D line/plane for point and
triangle sets.
It allows fitting a set of kernel primitive elements
which are stored in an iterator range of a container.
Bounding box, centroid and barycenter computations are also provided.

%\ccHeading{Concepts}
%\ccRefConceptPage{FittingTraits}\\

\ccHeading{Functions}

\ccRefIdfierPage{CGAL::barycenter}\\
\ccRefIdfierPage{CGAL::bounding_box}\\
\ccRefIdfierPage{CGAL::centroid}\\
\ccRefIdfierPage{CGAL::linear_least_squares_fitting}\\
\clearpage

\lcHtml{\ccHeading{Alphabetical Listing of Reference Pages}}
