% +------------------------------------------------------------------------+
% | Reference manual page: bounding_box.tex
% +------------------------------------------------------------------------+
% | 
% | March 2005  Pierre Alliez and Sylvain Pion
% | Package:   Linear Least Squares Fitting
% |
% +------------------------------------------------------------------------+

\begin{ccRefFunction}{bounding_box}  
%% add template arg's if necessary

\ccIndexMainItemBegin{bounding_box}

\ccDefinition
  
The function \ccRefName\ computes the bounding box of a set of 2D or 3D objects.

\ccInclude{CGAL/bounding_box.h}

There is a set of overloaded \ccc{bounding_box} functions for 2D and 3D objects.
The user can also optionally pass an explicit kernel, in case the default,
based on \ccc{Kernel_traits} is not sufficient.
The dimension is also deduced automatically.

\ccFunction{template < typename InputIterator >
            K::Iso_rectangle_2
            bounding_box(InputIterator first, InputIterator beyond);}
{ computes the bounding box of a non-empty set of 2D points.
  \ccc{K} is \ccc{Kernel_traits<std::iterator_traits<InputIterator>::value_type>::Kernel}.  The value type must be \ccc{K::Point_2}.
\ccPrecond{first != beyond.} }

\ccFunction{template < typename InputIterator, typename K >
            K::Iso_rectangle_2
            bounding_box(InputIterator first, InputIterator beyond, const K & k);}
{ computes the bounding box of a non-empty set of 2D points.
  The value type must be \ccc{K::Point_2}.
\ccPrecond{first != beyond.} }

\ccFunction{template < typename InputIterator >
            K::Iso_cuboid_3
            bounding_box(InputIterator first, InputIterator beyond);}
{ computes the bounding box of a non-empty set of 3D points.
  \ccc{K} is \ccc{Kernel_traits<std::iterator_traits<InputIterator>::value_type>::Kernel}.
  The value type must be \ccc{K::Point_3}.
\ccPrecond{first != beyond.} }

\ccFunction{template < typename InputIterator, typename K >
            K::Iso_cuboid_3
            bounding_box(InputIterator first, InputIterator beyond, const K & k);}
{ computes the bounding box of a non-empty set of 3D points.
  The value type must be \ccc{K::Point_3}.
\ccPrecond{first != beyond.} }


\end{ccRefFunction}
