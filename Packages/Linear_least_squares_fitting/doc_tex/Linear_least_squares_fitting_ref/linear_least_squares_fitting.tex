% +------------------------------------------------------------------------+
% | Reference manual page: linear_least_squares_fitting.tex
% +------------------------------------------------------------------------+
% | 
% | March 2005  Pierre Alliez and Sylvain Pion
% | Package:   Linear Least Squares Fitting
% | 
% |
% +------------------------------------------------------------------------+

\begin{ccRefFunction}{linear_least_squares_fitting_2}  %% add template arg's if necessary
\ccIndexMainItemBegin{linear_least_squares_fitting_2}

\ccDefinition
  
The function \ccRefName\ computes the best fitting 2D line of a 2D point set. Best
fit means that it minimizes the sum of squared distances from the points to
their projection onto the line.

\ccInclude{CGAL/linear_least_squares_fitting_2.h}

\ccFunction{template < typename InputIterator, typename K> typename K::FT linear_least_squares_fitting_2(InputIterator first,
                               InputIterator beyond, 
                               typename K::Line_2& line,   
                               typename K::Point_2& c,     
                               const K& k;}
  
  
  {\ccc{InputIterator::value_type} is a 2D point. 
   \ccc{line} is the best fitting line computed. 
   \ccc{c} is the centroid computed.
   \ccc{k} is the kernel (brrr... this should be a traits class).
   It can be omitted (deduced automatically from the value type).
     
  \ccSeeAlso
  }\\

  % \ccIndexMainItemEnd{linear_interpolation}
\end{ccRefFunction}
