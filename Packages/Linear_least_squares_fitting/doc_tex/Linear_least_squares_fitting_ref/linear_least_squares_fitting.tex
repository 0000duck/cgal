% +------------------------------------------------------------------------+
% | Reference manual page: linear_least_squares_fitting.tex
% +------------------------------------------------------------------------+
% | 
% | March 2005  Pierre Alliez and Sylvain Pion
% | Package:   Linear Least Squares Fitting
% | 
% |
% +------------------------------------------------------------------------+

\begin{ccRefFunction}{linear_least_squares_fitting_2}  
%% add template arg's if necessary

\ccIndexMainItemBegin{linear least squares fitting, 2D}

\ccDefinition
  
The function \ccRefName\ computes the best fitting 2D line of a 2D
point set. The best fit line minimizes the sum of squared distances
from the points to their projections onto the line.

\ccInclude{CGAL/linear_least_squares_fitting_2.h}

\ccFunction{template < typename InputIterator, typename K> 
            typename K::FT 
             linear_least_squares_fitting_2(InputIterator first,
                                            InputIterator beyond,
                                            typename K::Line_2 & line,
                                            typename K::Point_2 & centroid,     
                                            const K & k);}
{ computes the best fitting 2D line of a 2D point set in the range
[\ccc{first},\ccc{beyond}). The value returned is a fitting quality
between $0$ and $1$, where $0$ means that the variance is the same
along any line (a horizontal line is output by default), and $1$ means
that the variance is null orthogonally to the best fitting line. }

The class \ccc{K} is the kernel in which the type
\ccc{InputIterator::value_type} is defined. It can be omitted and
deduced automatically from the value type.

\ccHeading{Requirements}

\begin{enumerate}
   \item  \ccc{InputIterator::value_type} is equivalent to \ccc{K::Point_2}.
   \item  \ccc{line} is the best fitting line computed.
   \item  \ccc{centroid} is the centroid computed.
\end{enumerate}

\end{ccRefFunction}
