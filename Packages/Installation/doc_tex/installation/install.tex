%\chapter{Installing \cgal \label{set-up-lab}}

%% macro for GNU
\def\gnu{{\sc Gnu}}
%% macro for g++ 2.7.2
\def\gcc272{\gnu\ {\tt g++} {\rm 2.7.2}}
%% macro for SUNPRO CC
\def\sunprocc{{\sc Sunpro} {\tt CC}}
%% macro for SGI CC
\def\mipsprocc{{\sc Sgi} Mips(Pro) {\tt CC}}
%% macro for CGAL release number
\def\cgalrelease{0.9}
%% macro for CGAL directory
\def\cgaldir{{\tt CGAL-\cgalrelease}}

This document describes how to install \cgal\ on your system.  Besides
that, you will find some information about the makefile structure of
\cgal\ and the support for simultaneously using \cgal,
\leda\footnote{Library of Efficient Datatypes and Algorithms, see
  \path~http://www.mpi-sb.mpg.de/LEDA/index.html~ for information.}
and/or \gnu\ GMP\footnote{\gnu\ Multiple Precision library.}.

\section{Downloading \cgal}

The \cgal\ library can be downloaded in two different ways: using ftp
or using WWW.  If you have a WWW connection, the easiest way to
download \cgal\ is via the \cgal\ homepage:
\begin{quote}
      \path'http://www.cs.ruu.nl/CGAL/'
\end{quote}
and go to the `Software' section.

Just follow the instructions on this page to obtain your copy of the
library. The \cgal\ library can also be downloaded using FTP. The
library can be found at the following location:

\begin{verbatim}
      ftp.cs.ruu.nl
\end{verbatim}

in the directory \texttt{/pub/CGAL}. This directory contains release
\cgalrelease\ of the \cgal\ library. There is also a \texttt{README}
file that contains descriptions of the files in this directory. An
example of an FTP-session is given below.

\begin{verbatim}
      $ ftp ftp.cs.ruu.nl
      Name (ftp.cs.ruu.nl:<your username>): anonymous
      Password: <type your email address here>
      ftp> cd pub/CGAL
      ftp> get README
      ftp> binary
      ftp> get CGAL-0.9.tar.gz
      ftp> quit
\end{verbatim}

\section{Supported platforms}

The current version of \cgal\ has been tested on the following platforms:

\begin{center}
\begin{tabular}{||c||c||}
\hline\hline
\textbf{operating system} & \textbf{compiler} \\ \hline\hline
\multicolumn{1}{|l}{IRIX 5.3} & \multicolumn{1}{|l|}{\mipsprocc\ 4.0} \\ \hline
\multicolumn{1}{|l}{IRIX 6.2} & \multicolumn{1}{|l|}{\mipsprocc\ 7.1} \\ \hline
\multicolumn{1}{|l}{IRIX 5.3} & \multicolumn{1}{|l|}{\gcc272} \\ \hline
\multicolumn{1}{|l}{SUN Solaris 2.5} & \multicolumn{1}{|l|}{\sunprocc\ 4.1} \\ \hline
\multicolumn{1}{|l}{SUN Solaris 2.5} & \multicolumn{1}{|l|}{\sunprocc\ 4.2} \\ \hline
\multicolumn{1}{|l}{SUN Solaris 2.5} & \multicolumn{1}{|l|}{\gcc272} \\ \hline
\end{tabular}
\end{center}

\section{Extracting \cgal}

After you have downloaded the file containing the \cgal\ library, you have to
decompress it. For the zipfile use the command

\begin{verbatim}
      unzip <filename>.zip
\end{verbatim}

and for the gzipped file use the commands

\begin{verbatim}
      gunzip <filename>.tar.gz
      tar xvf <filename>.tar
\end{verbatim}

In both cases the directory \cgaldir\ will be created. This directory
contains the following subdirectories:

\begin{center}
\begin{tabular}{||c||c||}
\hline\hline
\textbf{directory} & \textbf{contents} \\ \hline\hline
\multicolumn{1}{|l}{\texttt{auxiliary}} & \multicolumn{1}{|l|}{packages that can optionally be used with \cgal} \\ \hline
\multicolumn{1}{|l}{\texttt{config}} & \multicolumn{1}{|l|}{configuration files for install script} \\ \hline
\multicolumn{1}{|l}{\texttt{doc\_html}} & \multicolumn{1}{|l|}{documentation (HTML)} \\ \hline
\multicolumn{1}{|l}{\texttt{doc\_ps}} & \multicolumn{1}{|l|}{documentation (Postscript)} \\ \hline
\multicolumn{1}{|l}{\texttt{examples}} & \multicolumn{1}{|l|}{example programs } \\ \hline
\multicolumn{1}{|l}{\texttt{include}} & \multicolumn{1}{|l|}{header files} \\ \hline
\multicolumn{1}{|l}{\texttt{lib}} & \multicolumn{1}{|l|}{(shared) object libraries} \\ \hline
\multicolumn{1}{|l}{\texttt{make}} & \multicolumn{1}{|l|}{files with platform dependent makefile settings} \\ \hline
\multicolumn{1}{|l}{\texttt{src}} & \multicolumn{1}{|l|}{source files} \\ \hline
\end{tabular}
\end{center}

\section{The install script}

The directory \cgaldir\ contains a Bourne shell script called
\texttt{install\_cgal}. The script can be run in two modes: a
menu-driven interactive mode and a non-interactive mode.  Normally you
should use the interactive mode, but in case you run into problems
with it or prefer the non-interactive way for some reason, you can
still use the non-interactive mode.

\subsection{A sample Installation}

In this section we sketch an example installation on a {\sc Sun}
running Solaris 2.5 with the \sunprocc\ 4.2 compiler. This should give
you an overview of the interactive installation procedure. For a
complete description of the different menus and their options refer to
section \ref{sec:interactive-mode}.

\subsubsection{Starting the script}

Go to the \cgaldir\ directory and enter the command
\begin{verbatim}
      ./install_cgal -i
\end{verbatim}

You get a message indicating the \cgal\ version you are going to
install and that you are running the interactive mode. Then it takes
some time while the script locates a number of utility program (such
as {\tt awk}). You will not get informed about this\footnote{If you
  are that curious what happens exactly, you can have a look at the
  file {\tt \cgaldir/install.log}.}, but see some dots written to
screen that indicate progress.

\begin{scriptsize}
\begin{verbatim}
--------------------------------------------------------
  This is the install script for CGAL 0.9
--------------------------------------------------------

starting interactive mode - one moment, please
......

  Choosing compiler SUNPRO 4.2.
  Testing for builtin STL ... no.
.
\end{verbatim}
\end{scriptsize}
  
If there is any compiler installed on your system and on your {\tt
  PATH} that is supported by \cgal, one of these compilers is chosen
and it is checked whether it has a ``builtin'' \stl (see section \ref{sec:stl}),
i.e. if \stl\ 
works without adding any specific compiler flags. Here the chosen
compiler is \sunprocc\ which has no builtin \stl. If there is more
than one compiler installed on your system (and supported by \cgal),
you may later choose to use a different compiler\footnote{This can be
  done from the compiler menu.}.

A menu similar to the following will appear on your screen.

\begin{scriptsize}\label{pic:main-menu}
\begin{verbatim}
  ****************************************************************
  **              CGAL 0.9 Installation Main Menu               **
  **              -------------------------------               **
  **                                                            **
  **   OS:                  sparc_SunOS-5.5                     **
  **   Compiler:            SUNPRO CC 4.2                       **
  **   STL_DIR:             please configure!!!                 **
  **   LEDA:                not supported.                      **
  **   GMP:                 not supported.                      **
  **                                                            **
  **   The setup has not been tested.                           **
  **                                                            **
  **   There are no libs for this os/compiler.                  **
  **                                                            **
  **   <C>  Compiler Menu                                       **
  **   <L>  LEDA Menu                                           **
  **   <G>  GMP Menu                                            **
  **   <T>  Test (and save) setup                               **
  **   <A>  Run all setup tests (no cache)                      **
  **                                                            **
  **   <B>  Build CGAL Libraries                                **
  **                                                            **
  **   <Q>  Back to OS                                          **
  **                                                            **
  **   Your Choice:                                             **
  **                                                            **
  ****************************************************************
\end{verbatim}
\end{scriptsize}

The first lines below the headline contain some kind of status report:
current OS and compiler, are \leda\ and GMP supported etc., in this
case it also tells you that \stl\ still has to be configured.

Moreover you can see that the current setup has not been tested yet
and that there do not exist \cgal\ libraries for this OS/compiler
combination in the \cgal\ lib directory by now. It is always a good
idea to test your configuration before you start building the \cgal\ 
libraries, but before that we first have to setup \stl.

\subsubsection{Setting up \stl}

Please type ``{\tt c}'' to go to the compiler menu where all compiler
specific options can be configured. Then type ``{\tt i}'' to set the
include directory where \stl\ header files reside on your system. The
script only accepts directories that exist and pass a confidence test
(In this case it checks for a file {\tt iterator.h} in the directory.
This file should be present in all \stl\ implementations). Let's
assume, you have a {\sc HP} \stl\ on your system in {\tt
  /pub/local/STL/HP}.

After you set up the \stl\ directory, the script tries to compile some
small example programs to check whether \stl\ works principially and
if it does, which language features are supported by your compiler and
your \stl\ version. This is necessary, since there are many different
\stl\ implementations and \cgal\ (partly) depends on which
implementation is used. A message {\tt ok} always indicates that your
compiler works as it should i.e. a feature is supported resp. a bug is
{\it not} present.\footnote{This is expected to change, once all
  compilers conform to the ANSI (draft) standard.}

\begin{scriptsize}
\begin{verbatim}
  ****************************************************************
  **            CGAL 0.9 Installation Compiler Menu             **
  **            -----------------------------------             **
  **                                                            **
  **   OS:                  sparc_SunOS-5.5                     **
  **   Compiler:            SUNPRO CC 4.2                       **
  **   STL_DIR:             please configure!!!                 **
  **   LEDA:                not supported.                      **
  **   GMP:                 not supported.                      **
  **                                                            **
  **   The setup has not been tested.                           **
  **                                                            **
  **   <C>  Choose compiler                                     **
  **   <I>  STL  include directory                              **
  **        <undefined>                                         **
  **   <F>  Set custom compiler flags                           **
  **       <none>                                               **
  **   <L>  Set custom linker flags                             **
  **       <none>                                               **
  **   <T>  Test (and save) setup                               **
  **                                                            **
  **   <Q>  Back to Main Menu                                   **
  **                                                            **
  **   New STL_DIR: /pub/local/STL/HP                           **
  **                                                            **
  ****************************************************************

  Testing for STL ... ok.
  Testing for SGI STL 3.0 ... no.
  Testing for SGI 6/97 STL ... no.
  Testing for SGI 1996 STL ... no.
  Testing for SGI CC STL ... no.
  Testing for HP STL ... ok.
  ...
  <tests for several compiler features>
  ...
  Saving current setup ... done.
\end{verbatim}
\end{scriptsize}

If the \stl\ test succeeds, the current setup is marked as tested and
saved to the \cgaldir/config directory. Thus, if you run the install
script a second time for this OS/compiler, you won't have to enter the
\stl\ directory again, but it is retrieved from the corresponding
config file instead.

\subsubsection{Building the \cgal\ libraries}

First go back to the main menu with ``{\tt q}`` and then type ``{\tt
  b}'' to start compilation. Building consists of three steps:
\begin{enumerate}
\item writing the include makefile,
\item compiling the static libraries {\it and}
\item compiling the shared libraries.
\end{enumerate}
The include makefile encapsulates the OS-- and compiler-specific
settings and should be included (hence the name) in all makefiles that
compile \cgal\ applications. If anything went ok, the output should
look as follows (if not, you should have a look at the error messages
from compiler or linker).

\begin{scriptsize}
\begin{verbatim}
  ****************************************************************
  **                                                            **
  **                     Compiling CGAL 0.9                     **
  **                     ------------------                     **
  **                                                            **
  ****************************************************************

  OS:           sparc_SunOS-5.5
  COMPILER:     SUNPRO CC 4.2
  STL:          CGAL_STL_HP
  LEDA:         not supported
  GMP:          not supported

  Generating Makefiles ... done.
  Building CGAL_lib ... done.
  Building CGAL_sharedlib ... done.

  ****************************************************************
  **             Please press <ENTER> to continue.              **
  ****************************************************************
\end{verbatim}
\end{scriptsize}

That's all, it's done. Press ``{\tt <ENTER>}'' to return to the main
menu and proceed by installing for a different compiler (go to the
compiler menu and choose ``{\tt c}'' to get a list of supported
compilers detected on your system) or with \leda\ or GMP support (go
to the \leda\ resp.  GMP menu) or simply quit the install script by
typing ``{\tt q}''. When leaving the script, you get a list of
successful builds during the session.

Now it would be a good idea to print and read the `Getting Started
with \cgal' document that can be found in the \texttt{doc\_ps}
directory.

WARNING.  If you want to use the compiler \gcc272, or if you want to
use \leda\ in combination with \cgal, there are some compatibility
issues that need to be addressed first.  See the
sections~\ref{sec:leda} and~\ref{sec:gcc} for more information about
this.

\subsection{The interactive mode}\label{sec:interactive-mode}

To run the install script in the interactive mode, go to the \cgaldir\ 
directory and enter the command
\begin{verbatim}
      ./install_cgal -i
\end{verbatim}

After initialisation during which certain utility programs are located
and your system is searched for compilers supported by \cgal, you get
into the \cgal\ installation {\em main menu} (see page
\pageref{pic:main-menu} for a picture).

>From the main menu you can reach a number of different sub-menus, of
which the most important maybe is the {\em compiler menu}. This is
where you can choose the compiler you want to work with and setup \stl.
The compiler menu is described in section \ref{sec:compiler-menu}.

If you want to use \leda\ or \gnu\ GMP with \cgal, you will have to go
to the {\em leda menu} described in section \ref{sec:leda-menu} resp.
to the {\em gmp menu} described in section \ref{sec:gmp-menu}.

Finally you can build the \cgal\ libraries by typing {\tt b}. However,
it is recommended to run the {\em setup test} which is available in
all menus as option {\tt t} before. The setup test includes an \stl\ 
test, a \leda\ test and a GMP test, but not all
tests are performed always; e.g. the \leda\ test is only done, if you
enabled \leda\ support. The install script keeps track of the tests
passed and only tests again, if you change the setup in a way that
might affect the test result. If you want to redo {\it all} tests, you
have to choose option {\tt a} from the main menu. This also retests
for \leda\ or GMP installation in system directories which otherwise
is only done the first time you enable \leda/GMP support for an
OS/compiler combination.

\subsubsection{Files created during installation}

The install script stores all relevant settings for an OS/compiler
combination in the directory 
\begin{center}
  {\tt \cgaldir/config/$<${\it CGAL-OS-description}$>$}
\end{center}
where $<${\it CGAL-OS-description}$>$ identifies your OS/compiler
combination in a way specified in section \ref{sec:os-compiler-id}.
(Note that these files are only OS/compiler specific, i.e. there are
no different files for \leda-- and/or RTTI support.) This saves you
typing everything again, if you upgrade \cgal\ or another package that
makes recompiling the \cgal\ libraries necessary.

Besides the config files, {\tt install\_cgal} uses several temporary
files during interactive installation. Most of them are removed after
use, but some are not, since it might be helpful to keep some
information about the last run.  You can keep or delete them as you
like, as they are not needed anymore once the script terminated. It
follows a list of these files (all are plain ASCII).
\begin{center}
  \renewcommand{\arraystretch}{1.3}
  \begin{tabular}{|l|l|} \hline
    {\bf filename} & {\bf content}\\\hline\hline
    {\tt install.log} & detailed overall protocol\\\hline
    {\tt install.completed} & list of systems for which \cgal\ libraries 
    have been built\\\hline
    {\tt compile.log} & output of the last compiler call\\\hline
  \end{tabular}
\end{center}

\subsubsection{The Compiler Menu}\label{sec:compiler-menu}

Here is the place to setup the compiler specific options, such as the
compiler to use (if more than one has been detected) and the location of the
\stl\ files.

Some compilers come with their own \stl\ adaption to which we refer as
{\it builtin} and some compilers are not shipped with \stl. In the
latter case you have to supply an {\tt STL\_DIR}, i.e. the path to a
directory where the \stl\ header files are stored (see section
\ref{sec:stl} on where you can obtain an \stl\ implementation and what
problems you might have to consider). Even if your compiler has
builtin \stl, you can still choose to use an external \stl\ by setting
{\tt STL\_DIR} appropriately.

\fbox{
\begin{minipage}{.95\linewidth}
\vspace{2pt}
\centerline{{\large\bf Compiler Menu}}
\vspace{2pt}
\begin{description}
\item[$<$C$>$] Choose the compiler to be used from the list of
  detected compilers.
\item[$<$S$>$] ({\it present if a builtin \stl\ was detected for the
    current compiler)} Determine, if compiler builtin or an external
  \stl\ is used.
\item[$<$I$>$] ({\it present if an external \stl\ is used)} Set the
  include directory for an external \stl.
\item[$<$F$>$] Set custom compiler flags. These are the first flags
  given to the compiler in every call. Under normal circumstances
  there should be no need to set any such flag.
\item[$<$L$>$] Set custom linker flags. These are the first flags
  given to the linker in every call. Under normal circumstances
  there should be no need to set any such flag.
\end{description}
\vspace{2pt}
\end{minipage}
\hfil}

\subsubsection{The \leda\ Menu}\label{sec:leda-menu}

This is the place to setup \leda\ (see also section \ref{sec:leda})
specific options, if you plan to use \leda\ together with \cgal . In
order to enable \leda\ support in \cgal , \leda\ has to be installed on
your system. 

If \leda\ support is enabled the first time, the script tests whether
\leda\ is installed in standard system directories. If this test does
not succeed, you have to supply directories containing the \leda\ header
files ({\tt LEDA\_INCL\_DIR}) and \leda\ libraries ({\tt
  LEDA\_LIB\_DIR}).  Even if the tests are passed, you still have the
option to set these directories differently.\bigskip

\fbox{
\begin{minipage}{.95\linewidth}
\vspace{2pt}
\centerline{{\large\bf \leda\ Menu}}
\vspace{2pt}
\begin{description}
\item[$<$E$>$] Determine, if \leda\ is used in conjuction with \cgal\ or
  not.
\item[$<$I$>$] ({\it present if \leda\ support is enabled)} Set the
  include directory for \leda .
\item[$<$J$>$] ({\it present if \leda\ support is enabled, \leda\ headers
    have been found in a system include directory and {\tt
      LEDA\_INCL\_DIR} has been set)} Use \leda\ header from system
  include directory.
\item[$<$L$>$] ({\it present if \leda\ support is enabled)} Set the
  directory containing the \leda\ libraries.
\item[$<$M$>$] ({\it present if \leda\ support is enabled, \leda\ libs
    have been found in a system lib directory and {\tt LEDA\_LIB\_DIR}
    has been set)} Use \leda\ libraries from system lib directory.
\end{description}
\vspace{2pt}
\end{minipage}
\hfil}

\subsubsection{The GMP Menu}\label{sec:gmp-menu}

This menu is to setup GMP (\gnu\ Muptiple Precision Library) specific
options, if you plan to use GMP together with \cgal . In the {\tt
  auxiliary} directory you can find a GMP distribution, if you do not
already have it installed on your system. This menu contains an option
to install GMP in you \cgal\ directory tree, but of course you can
also install it independently from \cgal .

If GMP support is enabled the first time, the script tests whether GMP
is installed in standard system directories or in the \cgal\ tree. If
this test does not succeed, you have to supply directories containing
the GMP header files ({\tt GMP\_INCL\_DIR}) and GMP libraries ({\tt
  GMP\_LIB\_DIR}).  Even if the tests are passed, you still have the
option to set these directories differently.\bigskip

\fbox{
\begin{minipage}{.95\linewidth}
\vspace{2pt}
\centerline{{\large\bf GMP Menu}}
\vspace{2pt}
\begin{description}
\item[$<$C$>$] Install the GMP distribution shipped with \cgal\ in the
  \cgal\ directory tree. 
\item[$<$G$>$] Determine, if GMP is used in conjuction with \cgal\ or
  not.
\item[$<$I$>$] ({\it present if GMP support is enabled)} Set the
  include directory for GMP.
\item[$<$L$>$] ({\it present if GMP support is enabled)} Set the
  directory containing the GMP libraries.
\item[$<$M$>$] ({\it present if GMP support is enabled, there is a GMP
    installation in system directories or in the \cgal\ tree and {\tt
      GMP\_INCL\_DIR} or {\tt GMP\_LIB\_DIR} have been set)} Use GMP
  installation from system directories / \cgal\ tree.
\end{description}
\vspace{2pt}
\end{minipage}
\hfil}

\subsection{The non-interactive mode \label{sec:non-interactive}}

To run the install script in the non-interactive mode, go to the
\cgaldir\ directory and enter the command
\begin{verbatim}
      ./install_cgal -ni <compiler>
\end{verbatim}
where \texttt{<compiler>} is the C++ compiler executable.\\
You can either specify a full path, e.g. \texttt{/usr/local/bin/g++},
or just the basename, e.g. \texttt{g++}, which means the script
searches your {\tt PATH} for the compiler location. If your compiler
call contains spaces it has to be quoted, e.g. {\tt ./install\_cgal
  -ni "CC -64"}.  The options given this way become part of your
\cgal-OS description (see section \ref{sec:os-compiler-id}) which is
useful e.g. to distinguish between different compilers using the same
frontend (\mipsprocc\ on {\tt IRIX6}).

There are a number of additional command line options to customize
your \cgal\ setup which are discussed below. You should read the
corresponding paragraphs before you continue, especially if one or
more of the following conditions apply to you:
\begin{itemize}
\item your compiler does not have a ``builtin'' \stl\ (section
  \ref{sec:stl-setup}),
\item your compiler is \gcc272\ (section \ref{sec:gcc}),
\item you want to use \leda\ together with \cgal\ (section
  \ref{sec:leda-setup}) {\it or}
\item you want to use \gnu\ GMP together with \cgal\ (section
  \ref{sec:gmp-setup}).
\end{itemize}

Once you started the script, it should give you a message indicating
the \cgal\ version you are going to install and that you are running
the non-interactive mode. Then it proceeds by locating some utility
programs (such as {\tt awk}), determining your OS and compiler
version, displaying the settings you gave via command line and
classifying (see also section \ref{sec:stl}) the \stl\ version used
(which -- of course -- won't work, if you didn't set up \stl\ 
correctly, see section \ref{sec:stl-setup} below). Your compiler is
also checked for a number of bugs resp. support of certain language
features; a message {\tt ok} always indicates that your compiler works
as it should i.e. a feature is supported resp. a bug is {\it not}
present.

Finally the current setup is summarized, system specific directories
for makefiles and libraries are created (if they did not exist before)
and a new include makefile is written into the makefile directory. If
there already exists a makefile for the current OS/compiler
combination, it is backuped and you should get a corresponding
message.

To compile the \cgal\ libraries go now to the \texttt{src} directory.
Then type \texttt{make -f makefile\_lib} to compile the \cgal\ object
library and \texttt{make -f makefile\_sharedlib} to compile the \cgal\ 
shared object library. If you want to make changes to the makefiles
first, see section \ref{sec:makefiles} for an explanation of the
makefile structure of \cgal.

When this is finished it would be a good idea to print and read the
`Getting Started with \cgal' document that can be found in the
\texttt{doc\_ps} directory.

WARNING.  If you want to use the compiler \gcc272, or if you want to
use \leda\ in combination with \cgal, there are some compatibility
issues that need to be addressed first.  See the
sections~\ref{sec:leda} and~\ref{sec:gcc} for more information about
this.

\subsubsection{Setting up \stl}\label{sec:stl-setup}

The install script and the makefiles use the variable {\tt STL\_DIR}
to indicate the STL version that shall be used with \cgal. This
variable should point to the directory where STL header files are
stored. It can be set from command line with option ``{\tt --STL\_DIR
  {\it $<$dir$>$}}''. If you do not set {\tt STL\_DIR} this way, it is
assumed that the compiler has ``builtin'' \stl\ and this is used. If
your compiler does not have builtin \stl, you {\it have to} supply an
{\tt STL\_DIR} to get things running.

\subsubsection{Setting up \leda\ support}\label{sec:leda-setup}

See also section \ref{sec:leda}. By default there is no support for
\leda, but you can change this easily by use of the command line
option ``{\tt -leda}''. If \leda\ is installed in system directories
on your system, you should indicate this by setting the flags ``{\tt
  --leda-sys-incl}'' resp.  ``{\tt --leda-sys-lib}''. If this is not
the case, you have to supply the directories containing the \leda\ 
header files (``{\tt --LEDA\_INCL\_DIR} {\it $<$dir$>$}'') resp. the
\leda\ libraries for your compiler (``{\tt --LEDA\_LIB\_DIR} {\it
  $<$dir$>$}'').

\subsubsection{Setting up support for GMP}\label{sec:gmp-setup}

By default there is no support for GMP, but you can change this easily
by use of the command line option ``{\tt -gmp}''. If GMP is installed
in system directories on your system, you are already done now. If
this is not the case, you have to supply the directories containing
the GMP header files (``{\tt --GMP\_INCL\_DIR} {\it $<$dir$>$}'') and
the GMP library (``{\tt --GMP\_LIB\_DIR} {\it $<$dir$>$}'').

\subsection{Other Options}\label{sec:other-options}

There are some more less important features of the install script
we will summarize here. 

First of all you can get the version number of {\tt cgal\_install}
with option ``{\tt --version}''. Note that all other options are
ignored in this case.

Second there is an option ``{\tt -os {\it $<$compiler$>$}}'' which
allows you to determine your \cgal-OS description (see section
\ref{sec:os-compiler-id}). This is intended for testing purposes and
automatic detection of the correct include makefile (see also section
\ref{sec:makefiles}).

Finally there exists an option ``{\tt --verbose}'' that can be set in
interactive mode as well as in non-interactive mode. When set you get
a detailed summary of error messages occuring during {\it any}
compiler test (determining \stl\ version etc.). Normally you only get
these messages, if a required test (such as the general \stl\ test)
fails, otherwise you are just informed, {\it if} it succeeded or not.
This option is not recommended for general use, but it can be useful
to check why a certain test fails that was expected to be passed.

\section{Identifying OS and Compiler}\label{sec:os-compiler-id}

Since \cgal\ supports several different operating systems and
compilers, this is also reflected in the structure of the \cgal\ 
directory tree. Each OS/compiler combination has its own lib directory
under {\tt \cgaldir/lib}) (and analogously its own include makefile in
{\tt \cgaldir/make}) named as determined by the following scheme.
\begin{center}
  {\it $<$arch$>${\tt \_}$<$os$>${\tt -}$<$os-version$>${\tt \_}$<$comp$>${\tt
    -}$<$comp-version$>$}[{\tt \_}LEDA]
\end{center}

\begin{description}
\item[$<$arch$>$] is the system architecture as defined by ``{\tt
    uname -p}'' or ``{\tt uname -m}'',
\item[$<$os$>$] is the operating system as defined by ``{\tt uname
    -s}'',
\item[$<$os-version$>$] is the operating system version as defined by
  ``{\tt uname -r}'',
\item[$<$comp$>$] is the basename of the compiler executable (if it
  contains spaces, these are replaced by "-") {\it and}
\item[$<$comp-version$>$] is the compiler's version number (which
  unfortunately can not be derived in a uniform manner, since it is
  quite compiler specific).
\end{description}

The suffix {\tt \_}LEDA is appended to indicate \leda\ support. This
destinction is necessary, because the object libraries look (at least
possibly) different in these cases.

We call the resulting string \cgal-OS description. 

Examples are {\tt mips\_IRIX-6.2\_CC-7.1} or {\tt
  sparc\_SunOS-5.5\_g++-2.7.2\_LEDA}. You can use the install
script to get your \cgal-OS description, see section
\ref{sec:other-options}.

\section{The \cgal\ makefile structure \label{sec:makefiles}}

The \cgal\ distribution contains three makefiles:
\begin{itemize}
\item \texttt{\cgaldir/src/makefile\_lib} This makefile is used for
  compiling the \cgal\ object library {\tt libCGAL.a}.
  
\item \texttt{\cgaldir/src/makefile\_sharedlib} This makefile is used
  for compiling the \cgal\ shared object library {\tt libCGAL.so}.
  
\item \texttt{\cgaldir/example/makefile} This makefile is used for
  compiling the \cgal\ example programs.
\end{itemize}

All these makefiles are generic: they can be used for more than one platform.
To achieve this, the first section of each makefile contains an include
statement that looks as follows:

\begin{verbatim}
      CGAL_MAKEFILE = /users/jannes/CGAL-0.9/make/makefile_<CGAL-OS description>
      include $(CGAL_MAKEFILE)
\end{verbatim}

The file \texttt{CGAL\_MAKEFILE} is an include file with platform
dependent makefile settings. The abbreviation {\tt <CGAL-OS
  description>} (see section \ref{sec:os-compiler-id} for details) is
used to identify the operating system and compiler for which the
settings hold. For example, the file
\texttt{makefile\_mips\_IRIX-5.3\_CC-4.0} contains makefile settings
for the IRIX 5.3 operating system and the \mipsprocc\ 4.0
compiler. These include files are generated by the install script and
they are all located in the \texttt{\cgaldir/make} directory. If you
want to compile an application or an object library for a different
platform, the only thing you need to do is select another include file
from this directory.

The three makefiles also contain sections with compiler and/or linker
flags.  You may add your own flags here. For example, you might want
to add the flag \texttt{-DCGAL\_NO\_PRECONDITIONS} to turn off
precondition checking. The flags \texttt{\$(CGAL\_CXXFLAGS)} and
\texttt{\$(CGAL\_LDFLAGS)} should never be removed.

The default extension for \cgal\ source files is \texttt{.C}.  The
last section of the makefiles contains a suffix rule that tells the
compiler how to create a \texttt{.o}-file from a \texttt{.C}-file.  If
you want to use the default rule that is defined by the make utility,
you may want to remove this suffix rule.  However, note that it may
then be necessary to also rename the makefile variables
\texttt{CGAL\_CXX} and \texttt{CXXFLAGS}.

\section{Using \cgal\ and \leda \label{sec:leda}}

The \cgal\ library can be used independently from \leda.  However,
there is support in \cgal\ for using \cgal\ and \leda\ simultaneously.
For example, there are wrappers for the \leda\ number types
\texttt{integer}, \texttt{rational} and \texttt{real}.

Unfortunately, there is a conflict between most \stl\ implementations
and \leda\ 3.4 / 3.5.  The type \texttt{bool} is in \stl\ usually
defined as \texttt{int}, which is prohibited by \leda. If you are
going to use \leda\ and \cgal\ together, you have to resolve this
problem somehow.  There are two approaches: modifying the file
\texttt{bool.h} of \stl\ (which is the easiest solution) or modifying
the file {\tt <LEDA/bool.h>}.  For more information about this see the
file \texttt{\cgaldir/include/CGAL/bool.h}.  In version 3.5 of \leda,
the inclusion of the file {\tt <LEDA/bool.h>} can be prevented by
setting the flag {\tt LEDA\_BOOL\_H}. This flag will be set by \cgal\ 
automatically, so in this case you don't have to do anything.

If you want to use the \leda\ memory management for \cgal\ types, you
have to add the flag \texttt{-DCGAL\_USE\_LEDA\_HANDLE} to the custom
compiler flags during the installation. Depending on how you solved
the {\tt bool} problem, it may also be necessary to add the flag
\texttt{-DCGAL\_USE\_LEDA\_BOOL}.

The makefiles in the \texttt{\cgaldir/make} directory corresponding to
\leda\ can be recognized by the extension \texttt{LEDA}.

\section{Using the \gcc272\ compiler \label{sec:gcc}}

The \cgal\ library uses runtime type information (RTTI) at certain
places.  For compilers that do not support RTTI a workaround was used,
but this workaround was not applicable to \gcc272. As a consequence
the previous version of \cgal\ (0.9) used the RTTI feature from
\gcc272, although there are major problems with it. To overcome these
problems a patch for \gcc272 RTTI was shipped together with \cgal.

Recently a new workaround for RTTI became available that also applies
to \gcc272. Therefore in \cgal\ 1.0 the RTTI from \gcc272 isn't used
anymore, and there is no longer support for the RTTI patch in the
installation script. If you did install the patch for RTTI, and if you
want to continue using the RTTI from \gcc272\ you should set the
following values in the compiler menu during the installation:

\begin{enumerate}
\item Add {\tt -DRTTI\_PATCH -frtti} to the custom compiler flags.
\item Add {\tt -nodefaultlibs -lstdc++ -liberty -lg++ -lgcc -lc} to
  the custom linker flags.
\end{enumerate}

If you did install the patch, but don't want to continue using RTTI
then it shouldn't be necessary to make any changes, since the patch
has no influence on normal use of \gcc272.

\section{The Standard Template Library \label{sec:stl}}

For using the \cgal\ library, the Standard Template Library is
required.  Some compilers (e.g. g++) provide an implementation of
\stl.  There are also free implementations.  For example on
\path~ftp://butler.hpl.hp.com/stl~ you can find the original HP
implementation.  Silicon Graphics also maintains a site,
\path~http://www.sgi.com/Technology/STL/~, where you can find
documentation, references to other \stl\ sources and two
implementations; one for newer compilers and one for older compilers
that support less features.

The install script tries to find out which \stl\ is used.  This is
necessary to work around some limitations and bugs.  In case the right
version is not recognized, you can fill it in in the script or in the
makefiles generated by the script.  The variable
\verb~CGAL_STL_VERSION~ can take the following values:

\begin{center}
\begin{tabular}{|l|l|}
\hline \hline
CGAL\_STL\_GCC                    & \stl\ version that comes with gcc-2.7.2 \\ \hline
CGAL\_STL\_HP                     & Free HP version \\ \hline
CGAL\_STL\_SGI\_3\_0              & SGI STL 3.0 \\ \hline
CGAL\_STL\_SGI\_WWW               & Free new SGI implementation \\ \hline
CGAL\_STL\_SGI\_WWW\_OLD          & Free SGI implementation for older compilers \\ \hline
CGAL\_STL\_SGI\_CC                & SGI implementation that comes with their\\ 
                                  & 7.x compilers \\ \hline
CGAL\_STL\_WITH\_ITERATOR\_TRAITS & \stl\ implementation that relies on\\
                                  & iterator traits\\ \hline
\hline
\end{tabular}
\end{center}

Most \stl\ libraries consist of only header files. An exception is
HP's \stl, which also has source files \verb~random.cpp~ and
\verb~tempbuf.cpp~.  The generated makefiles do {\em not} take care of
this. You have to compile these files yourself and add them to the
custom linker flags during the installation of CGAL, or enter them
manually in the makefiles.

\section{Upgrading a \cgal\ 0.9 installation}

In case you have \cgal\ 0.9 installed on your system, you might like
to reuse your configuration files. Simply use the following commands
to put them into the right place:
\begin{verbatim}
mkdir <NEW_CGAL_DIR>/config/install
cp <OLD_CGAL_DIR>/config/* <NEW_CGAL_DIR>/config/install
\end{verbatim}
where {\tt <OLD\_CGAL\_DIR>} is your \cgal\ 0.9 root directory
(e.g.  {\tt /pub/local/CGAL-0.9}) and {\tt <NEW\_CGAL\_DIR>} is
the root directory of your new \cgal\ installation (e.g. {\tt
  /pub/local/CGAL-1.0}).

It is {\bf strongly} recommended to rerun all tests using the option
{\tt A} from the main menu.

\section{Compiling a \cgal\ application}
A good example for a \cgal\ makefile is the makefile in the
\texttt{\cgaldir/examples} directory. After the installation is
finished, this makefile is ready for use.  Otherwise you may have to
fill in the {\tt CGAL\_MAKEFILE} variable at the beginning of the file
(see also section \ref{sec:makefiles}).

