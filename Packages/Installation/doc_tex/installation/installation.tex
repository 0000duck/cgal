%%
%%  *** CGAL Installation Guide ***
%%
%% file:     installation.tex
%%
%% authors:  Michael Hoffmann, Dima Pasechnik and Wieger Wesselink
%%
%% $Revision$ $Date$
%%

%% macro for GNU
\newcommand{\gnu}{\textsc{Gnu}}

%% macro for GMP
\newcommand{\gmp}{\textsc{Gmp}}

%% macro for Windows-specific installation
\newcommand{\MSInst}{Windows-specific Installation}

%% macro for g++
\newcommand{\Gcc}[1]{\gnu~\texttt{g++}~{\rm #1}}
\newcommand{\gccurl}{\path'http://gcc.gnu.org/'}

%% macro for SUNPRO CC
\newcommand{\sunprocc}[1]{\textsc{Sunpro}~\texttt{CC}~{\rm #1}}
\newcommand{\sunproccurl}{\path'http://www.sun.com/software/Developer-products/cplusplus/'}

%% macro for SGI CC
\newcommand{\mipsprocc}{\textsc{SGI}~Mips(Pro)~\texttt{CC}}
\newcommand{\mipsprourl}{\path'http://www.sgi.com/developers/devtools/languages/mipspro.html'}

%% macro for Borland C++
\newcommand{\bcc}[1]{\textsc{Borland}~\texttt{C++}~{\rm #1}}
\newcommand{\bccurl}{\path'http://www.inprise.com/bcppbuilder/'}

%% macro for Microsoft Visual C++
\newcommand{\msvc}[1]{\textsc{MS}~Visual~\texttt{C++}~{\rm #1}}
\newcommand{\msvcurl}{\path'http://msdn.microsoft.com/visualc/'}

%% macro for Intel C++ Compiler
\newcommand{\icl}[1]{\textsc{Intel}~\texttt{C++}~{\rm #1}}
\newcommand{\iclurl}{\path'http://developer.intel.com/vtune/compilers/cpp/'}

%% macro for Microsoft Windows
\newcommand{\mswin}{\textsc{MS}~Windows}

%% macro for CGAL release number
%% \newcommand{\cgalrelease}{2.3}

%% macro for CGAL directory
\newcommand{\cgaldir}{\texttt{CGAL-\cgalrelease}}

%% macro for YOUR CGAL directory :)
\newcommand{\yourcgaldir}{$<$\textit{insert your \cgaldir\ dir}$>$}

%% macro for CGAL install config directory
\newcommand{\cgalinstconfdir}{\cgaldir\texttt{/config/install}\index{directories!config/install@\texttt{config/install}}}

%% macros for WWW pages
\newcommand{\cgalhomepage}{\path'http://www.cgal.org'}
\newcommand{\cgalwinlibmsvc}{\ccAnchor
{http://www.cgal.org/download/CGALWin/MS_windows/Visual_C++/CGALWin_msvc_native.zip}
{\texttt{http://www.cgal.org/download/CGALWin/MS\_windows/Visual\_C++/CGALWin\_msvc\_native.zip}}
}
\newcommand{\cgalwinlibbcc}{\ccAnchor
{http://www.cgal.org/download/CGALWin/MS_windows/Borland/CGALWin_bcc_native.zip}
{\texttt{http://www.cgal.org/download/CGALWin/MS\_windows/Borland/CGALWin\_bcc\_native.zip}}}

\newcommand{\cgalwinlibpage}{\path'http://www.cgal.org/download/CGALWin/'}
\newcommand{\sgistlpage}{\path'http://www.sgi.com/tech/stl/'}
\newcommand{\stlportpage}{\path'http://www.stlport.org/'}
\newcommand{\ledapage}{\path'http://www.mpi-sb.mpg.de/LEDA'}
\newcommand{\gmppage}{\path'http://www.gnu.org/software/gmp'}
\newcommand{\clnpage}{\path'http://clisp.cons.org/~haible/packages-cln.html'}
\newcommand{\trolltechpage}{\path'http://www.trolltech.com'}
\newcommand{\qtpage}{\path'http://doc.trolltech.com/'}
\newcommand{\qtmocpage}{\path'http://doc.trolltech.com/moc.html'}

%% write index in typewriter type. (c) Susan :-)
\newcommand{\TTindex}[1]{\index{#1@{\tt #1}}}
\newcommand{\TTsubindex}[2]{\index{#1@{\tt #1}!{#2}}}
\newcommand{\TTsubindextwo}[2]{\index{#1!#2@{\tt #2} }}

\lcHtml{\chapter{Installation}}
\section{Introduction}

\cgal\ stands for \textit{Computational Geometry Algorithms Library}.
It is a software library written in \CC, whose development started in
an {\sc Esprit Ltr} project. The goal of \cgal\ is to make the large
body of geometric algorithms developed in the field of computational
geometry available for industrial application.

This document describes how to install \cgal\ on Unix-like systems.
Besides that, you will find some information about the makefile
structure of \cgal\ and the support for using \cgal\ together with
other libraries, such as \leda, the Library of Efficient Datatypes and
Algorithms\footnote{\ledapage}, the \gnu\ Multiple Precision library
GMP\footnote{\gmppage}, and Trolltech's\footnote{\trolltechpage} Qt
toolkit.

\section{Prerequisites}\label{sec:prerequisites}

In order to build the \cgal\ libraries you need a \CC\ compiler.
Currently only a small number of recent compilers on Unix platforms
and \mswin\ are supported. The reason is that most compilers do not
conform to the ISO 14882 standard for \lcTex{\CC\footnote{see e.g.
    \path'http://www.research.att.com/~austern/csc/faq.html\#PartB'
    for information}} \lcHtml{\CC\footnote{see e.g.
    \path'http://www.research.att.com/~austern/csc/faq.html\#PartB'
    for information}} and some of them have so many limitations/bugs
that we could not work around all of them.

\cgaldir\ supports the following compilers/operating systems:

\begin{center}\index{compilers!supported}\index{supported compilers}
  \renewcommand{\arraystretch}{1.3}
  \gdef\lcTabularBorder{2}
  \begin{tabular}{|l|l|} \hline
    \textbf{compiler}        & \textbf{operating system}\\\hline\hline
    \mipsprocc\ 7.3 (n32 and 64) \footnotemark[7] 
    & IRIX 6.5\\\hline
    \Gcc{2.95.3, 3.0, 3.1, 3.2} \footnotemark[9]            
    & IRIX 6.5 / Solaris 2.6+ / Linux 2.x / 
      \mswin\ 95/98/2000/XP/NT4\footnotemark[10]\\\hline
    \sunprocc{5.3} \footnotemark[11]                 
    & Solaris 2.6+\\\hline
    \msvc{7.0} (\textsc{.NET}) \footnotemark[12]           
    & \mswin\ 95/98/2000/XP/NT4\footnotemark[10]\\\hline
    \bcc{5.5.1} \footnotemark[13]                          
    & \mswin\ 95/98/2000/XP/NT4\footnotemark[10]\\\hline
  \end{tabular}
\end{center}
\footnotetext[7]{\mipsprourl}\addtocounter{footnote}{1}
\footnotetext[9]{\gccurl}\addtocounter{footnote}{1}
\footnotetext[10]{with Cygwin
  (\path~http://www.cygwin.com~)}\addtocounter{footnote}{1}
\footnotetext[11]{\sunproccurl}\addtocounter{footnote}{1}
\footnotetext[12]{\msvcurl}\addtocounter{footnote}{1}
\footnotetext[13]{\bccurl}\addtocounter{footnote}{1}
\addtocounter{footnote}{1}

If you are going to install \cgal\ using
Cygwin\footnote{\path~http://www.cygwin.com~}, please read
Section~\ref{sec:cygwin} first.

For the \sunprocc{5.3} compiler, the latest patch set, in particular
Patch 111685-05, is required. You can check with \texttt{CC~-V} which
compiler version you have installed. It should read 
\begin{center}
\begin{verbatim}
  CC: Sun WorkShop 6 update 2 C++ 5.3 Patch 111685-05 2002/02/03
\end{verbatim}
\end{center}
or later.

%%\msvc{5.0} and prior is not supported and there are no plans to
%%support it in future releases.  
%%The \sunprocc{4.2} compiler is not
%%supported anymore, please stay with \cgal-1.2 if you have to use it.
%%Support for \sunprocc{} will be reconsidered as soon as an acceptable
%%degree of standard-conformance is reached. 
Note that \Gcc{2.96/97} are not official \texttt{gcc} releases that
are neither supported by the \texttt{GCC}
team\footnote{\path'http://gcc.gnu.org/gcc-2.96.html'} nor by \cgal.
Please upgrade to \Gcc{3}, if you happen to have this compiler.
%%Support for \icl{4.0} will be considered.

\section{Getting \cgal} \label{sec:gettingcgal}
\index{CGAL@\cgal!getting}\index{getting \cgal} 

The \cgal\ library can be downloaded from the \cgal\ 
homepage:\index{CGAL@\cgal!homepage}
\begin{quote}
      \cgalhomepage
\end{quote}
and go to the `Download' section. Just follow the instructions on this
page to obtain your copy of the library.

After you have downloaded the file containing the \cgal\ library, you
have to decompress it. Use the commands

\begin{verbatim}
      gunzip <filename>.tar.gz
      tar xvf <filename>.tar
\end{verbatim}

Alternatively, your browser might be able to invoke the right
decompression program by itself.

In both cases the directory \cgaldir\ will be created. This directory
contains the following subdirectories:\index{directories!structure}

\footnotetext[14]{\stlportpage}\addtocounter{footnote}{1}
\begin{center}
  \renewcommand{\arraystretch}{1.3}
  \gdef\lcTabularBorder{2}
  \begin{tabular}{|l|l|} \hline
    \textbf{directory} & \textbf{contents}\\\hline\hline
    \texttt{auxiliary} & packages that can optionally be used with \cgal\\\hline
    \texttt{config}    & configuration files for install script\\\hline
    \texttt{demo}      & demo programs (some of them need \leda, geomview
                         or other third-party products)\\\hline
    \texttt{doc\_html} & documentation (HTML)\\\hline
    \texttt{doc\_pdf}  & documentation (PDF)\\\hline
    \texttt{doc\_ps}   & documentation (Postscript)\\\hline
    \texttt{examples}  & example programs\\\hline
    \texttt{include}   & header files\\\hline
    \texttt{lib}       & (shared) object libraries\\\hline
    \texttt{make}      & files with platform dependent makefile settings\\\hline
    \texttt{scripts}   & some useful scripts (e.g. for creating makefiles)\\\hline
    \texttt{src}       & source files\\\hline
    \texttt{stlport}   & 
         a customized version of \texttt{Stlport-3.2.1}\footnotemark[14] 
         that is used with \msvc{6.0}\\\hline
  \end{tabular}
\end{center}

\subsection{Visualization}\label{sec:vis}

The programs in the \texttt{demo} directory provide visual output.
Most of these use the \ccc{CGAL::Window_stream}, which is currently
provided via the \leda\ library.  If you have \leda\ installed, you
need do nothing more than compile \cgal\ with \leda\ support to be
able to use the \ccc{Window_stream} and thus run these demo
programs.\index{visualization!LEDA}

Because \leda\ is no longer available for free for academic use, we
have made arrangements with Algorithmic Solutions Software GmbH, the
distributors of \leda\, to provide a separate
\texttt{CGALWin}\TTsubindextwo{visualization}{CGALWin}\TTindex{CGALWin}
library for those who do not have \leda\ installed. This library is
based on the \leda\ window but is available for download independently
of \leda\ from the \cgal\ download page. It is available in binary
form only and only for academic research and teaching use.

The installation of the \texttt{CGALWin} binary is described in
Section~\ref{sec:cgalwin}. Note that \texttt{CGALWin} should be
installed only after the \cgal\ libraries have been built.

Some demo programs for 3D structures require the geomview program for
visualisation.\index{visualization!geomview} This is available from
\path'http://www.geomview.org' (note that it does not run on \mswin).

\section{Installing \cgal}

The directory \cgaldir\ contains a Bourne shell script called
\texttt{install\_cgal}\TTindex{install\_cgal}
\index{scripts!\texttt{install\_cgal}}. The script can be run in two
modes: a menu-driven interactive mode and a non-interactive mode.
Normally you should use the interactive mode, but in case you run into
problems with it or do not like it for some reason, you can still use
the non-interactive mode.

We first describe a sample installation in section
\ref{sec:sample-inst}. This provides you with an overview on how the
interactive installation works. If you want more detailed information
about specific menus and their options, take a look at section
\ref{sec:interactive-mode}. Finally, for the non-interactive mode
refer to section \ref{sec:non-interactive}.

If you want to use \leda\ together with \cgal, have a look at
section~\ref{sec:leda}.

\section{A Sample Installation}\label{sec:sample-inst}

In this section we sketch an example installation on a \textsc{Sun}
running Solaris 2.6 with the \Gcc{2.95} compiler. For a complete
description of the different menus and their options refer to section
\ref{sec:interactive-mode}.

\subsection{Starting the script}

Go to the \cgaldir\ directory and enter the command
\begin{verbatim}
      ./install_cgal -i
\end{verbatim}

You get a message indicating the \cgal\ version you are going to
install and that you are running the interactive mode. Then it takes
some time while the script locates a number of utility programs. You
will not get informed about this\footnote{If you are that curious what
  happens exactly, have a look at the file
  \texttt{\cgaldir/install.log}.\TTindex{install.log}}, but see some
dots written to the screen indicating progress.

{\ccTexHtml{\scriptsize}{}
\begin{alltt}
--------------------------------------------------------
  This is the install script for CGAL \cgalrelease
--------------------------------------------------------

starting interactive mode - one moment, please
......

  Choosing compiler GNU gcc-2.95.3
  Testing for STL ... ok.
  ...
  <tests for several compiler features>
  ...
  Saving current setup ... done.
.
\end{alltt}}
  
If there is any compiler installed on your system and accessible
through your \texttt{PATH} environment variable that is supported by
\cgal, one of these compilers is chosen and a number of tests are done
to check whether your compiler supports certain language constructs
respectively has specific bugs. There are quite a number of these
tests, so this step may take some time. For each test you should get a
message what particularly is tested at the moment and what the result
is. If there is more than one compiler installed on your system (and
supported by \cgal), you may later choose to use a different compiler
from the compiler menu.

A menu similar to the following will appear on your screen.\index{main
  menu}\index{menus!main}

{\ccTexHtml{\scriptsize}{} \label{pic:main-menu}
\begin{alltt}
  ****************************************************************
  **              CGAL 3.0 Installation Main Menu               **
  **              -------------------------------               **
  **                                                            **
  **   OS:                  sparc_SunOS-5.6                     **
  **   Compiler:            GNU gcc-2.95.3                      **
  **   LEDA:                not supported.                      **
  **   GMP:                 not supported.                      **
  **   GMPXX:               not supported.                      **
  **                                                            **
  **   Compiler is supported by CGAL.                           **
  **   The setup has been tested ok.                            **
  **                                                            **
  **   There are no libs for this os/compiler.                  **
  **                                                            **
  **   <C>  Compiler Menu                                       **
  **   <L>  LEDA Menu                                           **
  **   <G>  GMP Menu                                            **
  **   <K>  Qt Menu                                             **
  **   <T>  Test (and save) setup                               **
  **   <A>  Run all setup tests (no cache)                      **
  **                                                            **
  **   <B>  Build CGAL Libraries                                **
  **                                                            **
  **   <Q>  Back to OS                                          **
  **                                                            **
  **   Your Choice:                                             **
  **                                                            **
  ****************************************************************
\end{alltt}}

The first lines below the headline contain some kind of status report:
current OS and compiler, are \leda, GMP, or GMPXX (GMP's built-in C++
interface) supported etc. Moreover you can see that the current setup
has been tested and that there do not exist \cgal\ libraries for this
OS/compiler combination in the \cgal\ lib directory by now.

The fact that your setup has been tested implies that the current
settings have been saved to a file in the directory \cgalinstconfdir.
Thus, if you run the install script a second time for this
OS/compiler, you will not have to go through the whole config-/test
cycle again, but the configuration will be retrieved from the
corresponding config file instead.

\subsection{Building the \cgal\ libraries}\label{sec:build-the-libs}

We are now ready to build the \cgal\ libraries.  Just type ``{\tt b}''
to start compilation. Building consists of three steps:
\begin{enumerate}
\item writing the include makefile,
\item compiling the static libraries \textit{and}
\item compiling the shared libraries.
\end{enumerate}
The include makefile\index{include makefile} encapsulates the OS-- and
compiler-specific settings and should be included (hence the name) in
all makefiles that compile \cgal\ applications. If everything went ok,
the output should look as follows (if not, you should have a look at
the error messages from compiler or linker).

{\ccTexHtml{\scriptsize}{}
\begin{verbatim}
  ****************************************************************
  **                                                            **
  **                     Compiling CGAL 3.0                     **
  **                     ------------------                     **
  **                                                            **
  ****************************************************************

  OS:           sparc_SunOS-5.6
  COMPILER:     GNU gcc-2.95.3
  LEDA:         not supported
  GMP:          not supported
  GMPXX:        not supported

  Generating Makefiles ... done.
  Building CGAL_lib ... done.
  Building CGAL_sharedlib ... done.

  ****************************************************************
  **             Please press <ENTER> to continue.              **
  ****************************************************************
\end{verbatim}}
{
}  

That's all, it's done. Press ``\texttt{<ENTER>}'' to return to the
main menu and proceed by installing for a different compiler (go to
the compiler menu and choose ``\texttt{c}'' to get a list of supported
compilers detected on your system) or with \leda, GMP, or Qt support
(go to the \leda, GMP, or Qt menu resp.) or simply quit the install
script by typing ``\texttt{q}''. When leaving the script, you get a
list of successful builds during the session. Furthermore, the script
prints the setting of \texttt{CGAL\_MAKEFILE} for the last active
configuration. Remember to set this environment variable before
compiling \cgal\ applications. On bourne shell derivatives, you would
type in our example
\begin{alltt}
      export CGAL_MAKEFILE=\cgaldir/make/makefile_sparc_SunOS-5.6_g++-2.95.3
\end{alltt}
while for \texttt{csh} descendants the syntax is
\begin{alltt}
      setenv CGAL_MAKEFILE \cgaldir/make/makefile_sparc_SunOS-5.6_g++-2.95.3
\end{alltt}
In Section~\ref{sec:makefiles} you can find more information on the
\cgal\ makefile structure, and how to set \texttt{CGAL\_MAKEFILE} when
using \cgal\ on several platforms.

If you want to use the visualization provided by
\ccc{CGAL::Window_stream} (Section~\ref{sec:vis}) and you do not have
the \leda\ libraries installed, or do not want to use them with \cgal, 
you should now proceed to
Section~\ref{sec:cgalwin} and install the \texttt{CGALWin} library.

%% Now it would be a good idea to print and read the document ``Getting
%% Started with \cgal'' that can be found in various formats in the
%% \texttt{doc\_html}, \texttt{doc\_pdf} and \texttt{doc\_ps}
%% directories.

\section{Installing the \texttt{CGALWin} library}\label{sec:cgalwin}
\TTsubindextwo{installation}{CGALWin}\TTsubindextwo{visualization}{CGALWin}
\TTindex{CGALWin}

\textit{Note: This section applies only to those who do \textbf{not}
  have the \leda\ libraries installed on their system, or do not want to
use \leda\ with \cgal.}

The \texttt{CGALWin} library, as described in Section~\ref{sec:vis},
can be shipped in binary format only because of licensing
restrictions. The binaries are available at the \cgal\ download
pages\footnote{\cgalwinlibpage}.

Fetch the file matching your operating system and compiler, uncompress
it, and place it into the corresponding \cgal\ lib directory. This
directory has been created during installation of the \cgal\ library
and it is called
\begin{alltt}
      \cgaldir/lib/<\textit{CGAL-OS-description}>
\end{alltt}
where $<$\textit{CGAL-OS-description}$>$ identifies your OS/compiler
combination in a way specified in section \ref{sec:os-compiler-id}.
When quitting \texttt{install\_cgal}, it prints the \cgal\ lib
directory (\texttt{CGAL\_LIB\_DIR})\TTindex{CGAL\_LIB\_DIR} of the
last active configuration.

If you do not find a binary format that works on your system, contact
us at \texttt{contact@cgal.org}, and we will try to accommodate your
needs.

Continuing the example from the previous section, you would download
the file \texttt{libCGALWin.a.gz} from 
\begin{verbatim}
      Solaris -> Solaris-2.6 -> g++-2.95.3
\end{verbatim}
and uncompress it.
\begin{alltt}
      gunzip libCGALWin.a.gz
\end{alltt}
Finally, you have to place it into the lib directory
\begin{alltt}
      cp libCGALWin.a \cgaldir/lib/sparc_SunOS-5.6_g++-2.95.3
\end{alltt}

\section{The interactive mode}\label{sec:interactive-mode}
\index{interactive installation}\index{installation!interactive}
\TTsubindex{install\_cgal}{interactive mode}

To run the install script in the interactive mode, go to the \cgaldir\ 
directory and enter the command
\begin{verbatim}
      ./install_cgal -i
\end{verbatim}

After initialization during which certain utility programs are located
and your system is searched for compilers supported by \cgal, you get
into the \cgal\ installation \textit{main menu} (see page
\pageref{pic:main-menu} for a picture).

From the main menu you can reach a number of different sub-menus, of
which the most important maybe is the \textit{compiler menu}. This is
where you can choose the compiler you want to work with and set custom
compiler or linker options.  The compiler menu is described in section
\ref{sec:compiler-menu}.

If you want to use \leda, GMP, or Qt with \cgal, you will have to go
to the \textit{leda menu}, \textit{gmp menu}, or \textit{qt menu}.
These are described in section \ref{sec:leda-menu},
\ref{sec:gmp-menu}, \ref{sec:qt-menu}, respectively.

Finally you can build the \cgal\ libraries by typing \texttt{b}.
However, it is recommended to run the \textit{setup test} which is
available in all menus as option \texttt{t} before. The setup test
includes an \stl\ test, a \leda\ test, a GMP test, and a Qt test. But
not all tests are performed always; e.g. the \leda\ test is only done,
if you enabled \leda\ support. The install script keeps track of the
tests passed and only tests again, if you change the setup in a way
that might affect the test result. If you want to redo \textit{all}
tests, you have to choose option ``\texttt{a}'' from the main menu.
This also retests for \leda/GMP/Qt installation in system directories
which otherwise is only done the first time you enable \leda/GMP/Qt
support for an OS/compiler combination.

\subsection{Files created during installation}\label{sec:filescreated}

The install script stores all relevant settings for an OS/compiler
combination in the directory
\begin{center}
  \texttt{\cgalinstconfdir/$<$\textit{CGAL-OS-description}$>$}
\end{center}
where $<$\textit{CGAL-OS-description}$>$ identifies your OS/compiler
combination in a way specified in section \ref{sec:os-compiler-id}.
\footnote{Note that these files are only OS/compiler specific, i.e.
  there are no different files for with and without \leda\ support.}
This saves you typing everything again, if you upgrade
\index{upgrading \cgal}\index{CGAL@\cgal!upgrade} \cgal\ or another
package that makes recompiling the \cgal\ libraries necessary.

Besides the config files, \texttt{install\_cgal} uses several
temporary files during interactive installation. Most of them are
removed after use, but some are not, since it might be helpful to keep
some information about the last run.  You can keep or delete them as
you like, as they are not needed anymore once the script terminated.
A list of these files (all are plain ASCII and reside in \cgaldir)
follows.
\begin{center}\index{files!temporary}\index{logfiles}
  \renewcommand{\arraystretch}{1.3}
  \gdef\lcTabularBorder{2}
  \begin{tabular}{|l|l|} \hline
    \textbf{filename} & \textbf{content}\\\hline\hline
    \texttt{install.log}\TTindex{install.log} & 
    detailed overall protocol\\\hline
    \texttt{install.completed}\TTindex{install.completed} &
    list of systems for which \cgal\ libraries 
    have been built\\\hline
    \texttt{compile.log}\TTindex{compile.log} & 
    output of the last compiler call\\\hline
  \end{tabular}
\end{center}

\subsection{The Compiler Menu}\label{sec:compiler-menu}
\index{compiler menu}\index{menus!compiler}

Here is the place to set the compiler specific options, such as the
compiler to use (if more than one has been detected) and custom
compiler or linker flags.\bigskip

\fbox{
\begin{minipage}{.95\linewidth}
\vspace{2pt}
\centerline{{\large{\bf Compiler Menu}}}
\vspace{2pt}
\begin{description}
\item[$<$C$>$] Choose the compiler to be used from the list of
  detected compilers.  You can also register other compilers, if they
  have not been detected automatically.\index{compilers!choosing}
\item[$<$F$>$] Set custom compiler flags. These are the first flags
  given to the compiler in every call. Under normal circumstances
  there should be no need to set any such
  flag.\index{compilers!setting custom flags}
\item[$<$L$>$] Set custom linker flags. These are the first flags
  given to the linker in every call. Under normal circumstances
  there should be no need to set any such flag.
\end{description}
\vspace{2pt}
\end{minipage}
\hfill}

\subsection{The \leda\ Menu}\label{sec:leda-menu}
\index{LEDA@\leda!menu}\index{menus!\leda}

This is the place to set \leda\ specific options, if you plan to use
\leda\ together with \cgal\ (see also section \ref{sec:leda}). In
order to enable \leda\ support in \cgal, \leda\ has to be installed on
your system.

If \leda\ support is enabled the first time, the script tests whether
\leda\ is installed in standard system directories. If this test does
not succeed, you have to supply directories containing the \leda\ 
header files (\texttt{LEDA\_INCL\_DIR})\TTindex{LEDA\_INCL\_DIR} and
\leda\ libraries ({\tt LEDA\_LIB\_DIR})\TTindex{LEDA\_LIB\_DIR}.  Even
if the tests are passed, you still have the option to set these
directories differently.\bigskip

\index{LEDA@\leda!enable support}
\fbox{
\begin{minipage}{.95\linewidth}
\vspace{2pt}
\centerline{{\large{\bf \leda\ Menu}}}
\vspace{2pt}
\begin{description}
\item[$<$E$>$] Enable/Disable \leda\ support in \cgal.
\item[$<$I$>$] (\textit{present if \leda\ support is enabled)} Set the
  include directory for \leda .
\item[$<$J$>$] (\textit{present if \leda\ support is enabled, \leda\ headers
    have been found in a system include directory and {\tt
      LEDA\_INCL\_DIR} has been set)} Use \leda\ header from system
  include directory.
\item[$<$L$>$] (\textit{present if \leda\ support is enabled)} Set the
  directory containing the \leda\ libraries.
\item[$<$M$>$] (\textit{present if \leda\ support is enabled, \leda\ 
    libs have been found in a system lib directory and
    \texttt{LEDA\_LIB\_DIR} has been set)} Use \leda\ libraries from
  system lib directory.
\end{description}
\vspace{2pt}
\end{minipage}
\hfill}

\subsection{The GMP Menu}\label{sec:gmp-menu}
\index{GMP@\gmp!menu}\index{menus!\gmp}

This menu is to set GMP (\gnu\ Multiple Precision Library) specific
options, if you plan to use GMP together with \cgal . In the {\tt
  auxiliary} directory you can find a GMP distribution, if you do not
already have it installed on your system. This menu contains an option
to install GMP in your \cgal\ directory tree\footnote{This option is
  on \msvc\ just unpacks the corresponding pre-compiled library that
  comes with \cgal\ distribution.}, but of course you can also install
it independently from \cgal .

If GMP support is enabled the first time, the script tests whether GMP
is installed in standard system directories or in the \cgal\ tree. If
this test does not succeed, you have to supply directories containing
the GMP header files (\texttt{GMP\_INCL\_DIR})\TTindex{GMP\_INCL\_DIR}
and GMP libraries ({\tt GMP\_LIB\_DIR})\TTindex{GMP\_LIB\_DIR}.  Even
if the tests are passed, you still have the option to set these
directories differently.\bigskip

If GMP support is enabled, you may additionnally enable support for GMPXX,
GMP's built-in C++ interface, from the GMP menu.  Note that this is not
supported by all C++ compilers, you should check the GMP manual for more
information.  Once you choose to have GMPXX support, then if you install GMP
from the CGAL tree, GMP will be configured with C++ support.

\index{GMP@\gmp!installing}\index{GMP@\gmp!enable support}
\fbox{
\begin{minipage}{.95\linewidth}
\vspace{2pt}
\centerline{{\large{\bf GMP Menu}}}
\vspace{2pt}
\begin{description}
\item[$<$C$>$] Install the GMP distribution shipped with \cgal\ in the
  \cgal\ directory tree.
\item[$<$G$>$] Enable/Disable GMP support in \cgal.
\item[$<$X$>$] Enable/Disable GMPXX support in \cgal.
\item[$<$I$>$] (\textit{present if GMP support is enabled)} Set the
  include directory for GMP.
\item[$<$L$>$] (\textit{present if GMP support is enabled)} Set the
  directory containing the GMP libraries.
\item[$<$M$>$] (\textit{present if GMP support is enabled, there is a GMP
    installation in system directories or in the \cgal\ tree and {\tt
      GMP\_INCL\_DIR} or \texttt{GMP\_LIB\_DIR} have been set)} Use GMP
  installation from system directories / \cgal\ tree.
\end{description}
\vspace{2pt}
\end{minipage}
\hfill}

\subsection{The Qt Menu}\label{sec:qt-menu}
\index{Qt!menu}\index{menus!Qt}

This menu is to set Qt specific options, if you plan to use Qt
together with \cgal . Note that for use with \cgal, Qt version 2.2 (or
later) if required. In order to enable Qt support in \cgal, Qt has to
be installed on your system first.
Unlike for GMP, there is no option to install Qt from the \cgal\ 
installation script.  For information on Qt, please refer to
\begin{quote}
  \qtpage
\end{quote}

\begin{ccAdvanced}
The Qt menu has a basic mode and an advanced one. If your Qt
installation is standard, you shouldn't have to go into the advanced
mode.

The Qt menu starts in basic mode, unless Qt is installed in standard
system directories (because in that case, the advanced mode is quicker).
\end{ccAdvanced}

\subsubsection{Basic mode (to use with a standard Qt installation)}

If Qt support is enabled the first time, the script tests whether the
\texttt{\$QTDIR} environment variable points to a valid Qt directory.
is installed in standard system directories. If that is not the case,
you have to supply the Qt directory containing your Qt
installation. Even if the test passed, you still have the option to
set this directory differently.
If your Qt installation is not standard, you would have to go into the
advanced mode.\bigskip

\fbox{
\begin{minipage}{.95\linewidth}
\vspace{2pt}
\centerline{{\large{\bf Qt basic Menu}}}
\vspace{2pt}
\begin{description}
\item[$<$K$>$] Enable/Disable Qt support in \cgal.\index{Qt!enable
    support}
\item[$<$A$>$] (\textit{present if Qt support is enabled)} Go to the
  advanced mode
\item[$<$D$>$] (\textit{present if Qt support is enabled}) Set the
  Qt directory.
\item[$<$E$>$] (\textit{present if Qt support is enabled,
    \texttt{\$QTDIR} points to a valid Qt directory and {\tt QT\_DIR}
    has been set}) Use \texttt{\$QTDIR} as Qt directory.
\end{description}
\vspace{2pt}
\end{minipage}
\hfill}


\begin{ccAdvanced}
\subsubsection{Advanced mode (to use with a non-standard Qt
  installation) }

In that mode, you have to specify separately directories containing the Qt
header files (\texttt{QT\_INCL\_DIR})\TTindex{QT\_INCL\_DIR} and the
Qt library (\texttt{QT\_LIB\_DIR})\TTindex{QT\_LIB\_DIR}, and the
path to the MOC\footnote{\qtmocpage} executable
(\texttt{QT\_MOC})\TTindex{QT\_MOC} unless they are in system
directories. Even in that case, you still have the option to set them
differently.

\fbox{
\begin{minipage}{.95\linewidth}
\vspace{2pt}
\centerline{{\large{\bf Qt advanced Menu}}}
\vspace{2pt}
\begin{description}
\item[$<$K$>$] Enable/Disable Qt support in \cgal.\index{Qt!enable
    support}
\item[$<$B$>$] (\textit{present if Qt support is enabled)} Go to the
  basic mode
\item[$<$I$>$] (\textit{present if Qt support is enabled}) Set the
  directory containing Qt headers.
\item[$<$J$>$] (\textit{present if Qt support is enabled, Qt headers
    are in system directories and {\tt QT\_INCL\_DIR} has been set})
  Use Qt headers from system include directories.
\item[$<$L$>$] (\textit{present if Qt support is enabled}) Set the
  directory containing Qt library.
\item[$<$M$>$] (\textit{present if Qt support is enabled, Qt library
    is in system directories and {\tt QT\_LIB\_DIR} has been set})
  Use Qt library from system library directories.
\item[$<$O$>$] (\textit{present if Qt support is enabled}) Set the
  path to MOC executable.
\item[$<$P$>$] (\textit{present if Qt support is enabled, MOC is in path and {\tt QT\_MOC} has been set})
  Use Qt MOC executable in path.
\end{description}
\vspace{2pt}
\end{minipage}
\hfill}

\end{ccAdvanced}

\section{The non-interactive mode}\label{sec:non-interactive}
\index{non-interactive
  installation}\index{installation!non-interactive}
\TTsubindex{install\_cgal}{non-interactive mode}

To run the install script in the non-interactive mode, go to the
\cgaldir\ directory and enter the command
\begin{verbatim}
      ./install_cgal -ni <compiler>
\end{verbatim}
where \texttt{<compiler>} is the C++ compiler
executable.\index{compilers!choosing}\\ You can either specify a full
path, e.g. \texttt{/usr/local/bin/g++}, or just the basename, e.g.
\texttt{g++}, which means the script searches your \texttt{PATH} for
the compiler location. If your compiler call contains whitespaces it
has to be quoted, e.g.  \texttt{./install\_cgal -ni "CC -n32"}.  The
options given this way become part of your \cgal-OS
description\index{OS description} (see section
\ref{sec:os-compiler-id}) which is useful e.g. to distinguish between
different compilers using the same frontend (\mipsprocc\ on
\texttt{IRIX6}).

There are a number of additional command line options to customize
your \cgal\ setup which are discussed below. You should read the
corresponding paragraphs before you continue, especially if one or
more of the following conditions apply to you:
\begin{itemize}
\item you want to use \leda\ together with \cgal\ (section
  \ref{sec:leda-setup}),
\item you want to use \gnu\ GMP together with \cgal\ (section
  \ref{sec:gmp-setup}) \textit{or}
\item you want to use \gnu\ GMP's built-in C++ interface together with \cgal\
  (section \ref{sec:gmpxx-setup}) \textit{or}
\item you want to use Qt together with \cgal\ (section
  \ref{sec:qt-setup}).
\end{itemize}

Once you started the script, it should give you a message indicating
the \cgal\ version you are going to install and that you are running
the non-interactive mode. Then it proceeds by locating some utility
programs, determining your OS and compiler version and displaying the
settings you gave via command line. Your compiler is also checked for
a number of bugs resp. support of certain language features; a message
\texttt{ok} always indicates that your compiler works as it should
i.e. a feature is supported resp. a bug is \textit{not} present. On
the other hand \texttt{no} resp.  \texttt{unfortunately} indicate a
lack of support resp. presence of a bug.

Finally the current setup is summarized, system specific directories
for makefiles and libraries are created (if they did not exist before)
and a new include makefile is written into the makefile directory. If
there already exists a makefile for the current OS/compiler
combination, it is backed up and you should get a corresponding
message.

To compile the \cgal\ libraries go now to the \texttt{src} directory.
Then type \texttt{make -f makefile\_lib} to compile the \cgal\ object
library and \texttt{make -f makefile\_sharedlib} to compile the \cgal\ 
shared object library. If you want to make changes to the makefiles
first, see section \ref{sec:makefiles} for an explanation of the
makefile structure of \cgal.

When this is finished it would be a good idea to print and read the
`Getting Started with \cgal'\index{CGAL@\cgal!getting started} document
\texttt{getting\_started.ps} that can be found in the \texttt{doc\_ps}
directory.

\subsection{Setting up \leda\ support}\label{sec:leda-setup}
\index{LEDA@\leda!enable support}

See also section \ref{sec:leda}. By default there is no support for
\leda, but you can change this easily by use of the command line
option ``\texttt{-leda}''. If \leda\ is installed in system
directories on your system, you should indicate this by setting the
flags ``{\tt --leda-sys-incl}'' resp.  ``\texttt{--leda-sys-lib}''. If
this is not the case, you have to supply the directories containing
the \leda\ header files (``\texttt{--LEDA\_INCL\_DIR}
\textit{$<$dir$>$}'')\TTindex{LEDA\_INCL\_DIR} resp. the \leda\ 
libraries for your compiler (``\texttt{--LEDA\_LIB\_DIR} {\it
  $<$dir$>$}''\TTindex{LEDA\_LIB\_DIR}).

\subsection{Setting up support for GMP}\label{sec:gmp-setup}
\index{GMP@\gmp!enable support}

By default there is no support for GMP, but you can change this easily
by use of the command line option ``\texttt{-gmp}''. If GMP is
installed in system directories on your system, you are already done
now. If this is not the case, you have to supply the directories
containing the GMP header files (``\texttt{--GMP\_INCL\_DIR}
\textit{$<$dir$>$}'')\TTindex{GMP\_INCL\_DIR} and the GMP library
(``\texttt{--GMP\_LIB\_DIR}
\textit{$<$dir$>$}'')\TTindex{GMP\_LIB\_DIR}.

\subsection{Setting up support for GMPXX}\label{sec:gmpxx-setup}
\index{GMPXX@\gmpxx!enable support}

By default there is no support for GMPXX, GMP's built-in C++ interface, but
you can change this easily by use of the command line option
``\texttt{-gmpxx}''.  The only requirement for this to work is that GMP
support is enable correctly.

\subsection{Setting up support for Qt}\label{sec:qt-setup}
\index{Qt!enable support}

By default there is no support for Qt, but you can change this easily
by use of the command line option ``\texttt{-qt}''. If Qt is installed
in system directories on your system or the \texttt{\$QTDIR}
environment variable points to the Qt directory, you are already done
now. If this is not the case, you have to supply either the Qt
directory (``\texttt{--QT\_DIR}~\textit{$<$dir$>$}'')\TTindex{QT\_DIR}
if your Qt installation is standard, either directories containing the
Qt header files
(``\texttt{--QT\_INCL\_DIR}~\textit{$<$dir$>$}'')\TTindex{QT\_INCL\_DIR}
and the Qt library
(``\texttt{--QT\_LIB\_DIR}~\textit{$<$dir$>$}'')\TTindex{QT\_LIB\_DIR},
and the path to the MOC\footnote{\qtmocpage} executable
(``\texttt{--QT\_MOC} \textit{$<$exe$>$}'')\TTindex{QT\_MOC} if your
Qt installation is not standard.

\subsection{Setting custom compiler/linker flags}\label{sec:custom-setup}
\index{compilers!setting custom flags}

You can supply custom compiler and linker flags using the options
\mbox{(``\texttt{--CUSTOM\_CXXFLAGS}
  \textit{$<$flags$>$}'')\TTindex{CUSTOM\_CXXFLAGS}} and
\mbox{(``\texttt{--CUSTOM\_LDFLAGS}
  \textit{$<$flags$>$}'')\TTindex{CUSTOM\_LDFLAGS}}.  These are the
first flags given to the compiler/linker in every call.

\textit{Note:} Do not forget to quote your options in case they
contain spaces. Example:
\begin{verbatim}
  ./install_cgal -ni g++ --CUSTOM_CXXFLAGS "-I/my/include -O2"
\end{verbatim}

\subsection{Other Options}\label{sec:other-options}

There are some less important features of the install script we will
summarize here.

First of all, you can get the version number of \texttt{install\_cgal}
with option ``\texttt{--version}''. Note that all other options are
ignored in this case.\TTsubindex{install\_cgal}{version number}

Second there is an option ``\texttt{-os \textit{$<$compiler$>$}}''
where \textit{$<$compiler$>$} is your \CC\ compiler. This allows you
to determine your \cgal-OS description\index{OS description} (see section
\ref{sec:os-compiler-id}). The compiler can either be given by an
absolute path like
\begin{verbatim}
./install_cgal -os /usr/local/gcc-2.95.3/sun/bin/g++
\end{verbatim}
or just by denoting its basename, as long as it is on your path:
\begin{verbatim}
./install_cgal -os CC
\end{verbatim}
The option is intended for testing purposes and automatic detection of
the correct include makefile (see also section \ref{sec:makefiles}).

Finally, there exists an option
``\texttt{--verbose}''\TTsubindex{install\_cgal}{verbose mode} that
can be set in interactive mode as well as in non-interactive mode.
When set you get a detailed summary of error messages occurring during
\textit{any} compiler test (determining \stl\ version etc.). Normally
you only get these messages, if a required test (such as the general
\stl\ test) fails, otherwise you are just informed, \textit{if} it
succeeded or not.  This option is not recommended for general use, but
it can be useful to check why a certain test fails that was expected
to be passed.

\section{Upgrading a previous \cgal\ installation}
\label{sec:upgrade}\index{upgrading \cgal}\index{CGAL@\cgal!upgrade}

In case you already have a previous release of \cgal\ installed on
your system, you might like to reuse your configuration files and GMP
installations. Simply use the following command to copy them into the
right place:\TTsubindex{install\_cgal}{upgrade option}
\begin{verbatim}
./install_cgal --upgrade <OLD_CGAL_DIR>
\end{verbatim}
where \texttt{<OLD\_CGAL\_DIR>} is the root directory of your existing
\cgal\ installation\\ (e.g. \texttt{/pub/local/CGAL-2.4}).  You can
then build all libraries for the actual operating system that existed
in your previous \cgal\ installation with
\TTsubindex{install\_cgal}{rebuild-all option}
\begin{verbatim}
./install_cgal --rebuild-all
\end{verbatim}

If you want to install \cgal\ for more than one operating system in
the same directory structure, you have to run the latter command
(\texttt{rebuild-all}) once on each operating system.

Using \texttt{--build-all} instead of \texttt{--rebuild-all} will save
you the time of the configuration tests, and will only rebuild the libraries.

If you want to install only one configuration on a given operating system,
you can specify its name (the base name of a file in \cgalinstconfdir) with
the option \texttt{--rebuild <config>} or \texttt{--build <config>}.

\textbf{Note} that some compilers that have been supported in previous
\cgal\ releases might not be supported in \cgal-\cgalrelease\ anymore,
see section \ref{sec:prerequisites}. Trying to build
\cgal-\cgalrelease\ with these compilers will most probably fail. You
can solve this problem by deleting the obsolete config files (see
section \ref{sec:filescreated}) from \cgalinstconfdir\ before issuing
the \texttt{rebuild-all} command.

Similarly, you might want to use compilers with \cgal-\cgalrelease\ 
that have not been supported in previous releases. For these compilers
please follow the usual procedure as described in section
\ref{sec:interactive-mode} or \ref{sec:non-interactive}.

\section{Identifying OS and Compiler}\label{sec:os-compiler-id}
\index{OS description}\index{identifying OS and compiler}

Since \cgal\ supports several different operating systems and
compilers, this is also reflected in the structure of the \cgal\ 
directory tree. Each OS/compiler combination has its own lib directory
under \texttt{\cgaldir/lib}) (and analogously its own include
makefile\index{include makefile} in \texttt{\cgaldir/make}) named as
determined by the following scheme.
\begin{center}
  \textit{$<$arch$>$\texttt{\_}$<$os$>$\texttt{-}$<$os-version$>$\texttt{\_}$<$comp$>${\tt
    -}$<$comp-version$>$}[\texttt{\_}LEDA]
\end{center}

\begin{description}
\item[$<$arch$>$] is the system architecture as defined by ``{\tt
    uname -p}'' or ``\texttt{uname -m}'',
\item[$<$os$>$] is the operating system as defined by ``\texttt{uname
    -s}'',
\item[$<$os-version$>$] is the operating system version as defined by
  ``\texttt{uname -r}'',
\item[$<$comp$>$] is the basename of the compiler executable (if it
  contains spaces, these are replaced by "-") \textit{and}
\item[$<$comp-version$>$] is the compiler's version number (which
  unfortunately can not be derived in a uniform manner, since it is
  quite compiler specific).
\end{description}

The suffix \texttt{\_}LEDA is appended to indicate \leda\ support. 

We call the resulting string \cgal-OS description.\\ Examples are
\texttt{mips\_IRIX-6.2\_CC-7.2} or {\tt
  sparc\_SunOS-5.5\_g++-2.95.3\_LEDA}.\\ You can use the install script
to get your \cgal-OS description, see section \ref{sec:other-options}.

\section{The \cgal\ makefile structure}\label{sec:makefiles}
\index{makefile structure}

The \cgal\ distribution contains the following makefiles:
\begin{itemize}
\item \texttt{\cgaldir/src/makefile\_lib} for compiling the \cgal\ 
  object library \texttt{libCGAL.a},
  
\item \texttt{\cgaldir/src/makefile\_sharedlib} for compiling the
  \cgal\ shared object library \texttt{libCGAL.so},
  
\item \texttt{\cgaldir/examples/makefile} as sample makefile
  \textit{and}
  
\item \texttt{\cgaldir/examples/*/makefile} for compiling the \cgal\ 
  example programs.
\end{itemize}

All these makefiles are generic: they can be used for more than one
compiler.  To achieve this, the first section of each makefile
contains an include statement that looks as follows:

\begin{verbatim}
      CGAL_MAKEFILE = /users/jannes/CGAL-3.0/make/makefile_<CGAL-OS description>
      include $(CGAL_MAKEFILE)
\end{verbatim}
%$

The file \texttt{CGAL\_MAKEFILE}\TTindex{CGAL\_MAKEFILE} is an include
file\index{include makefile} with platform dependent makefile
settings. The abbreviation \texttt{<CGAL-OS description>} (see section
\ref{sec:os-compiler-id} for details) is used to identify the
operating system and compiler for which the settings hold. For
example, the file \texttt{makefile\_mips\_IRIX64-6.5\_CC-n32-7.30}
contains makefile settings for the IRIX 6.5 operating system and the
\mipsprocc\ 7.3 compiler.  These include files are automatically
generated by the \texttt{install\_cgal} script and they are all
located in the \texttt{\cgaldir/make} directory.  For convenience, the
\texttt{install\_cgal} script will substitute the include makefile
that was generated most recently.

If you want to compile an application or an object library with a
different compiler, the only thing you need to do is to substitute
another include makefile for the \texttt{CGAL\_MAKEFILE} variable. An
alternative way to do this is to create an environment variable
\texttt{CGAL\_MAKEFILE}. To pass the value of the environment variable
to the makefile you can either comment out the \texttt{CGAL\_MAKEFILE}
line in the makefile or use an appropriate command line option for the
make utility.  A comfortable way to set \texttt{CGAL\_MAKEFILE} is by
using \texttt{install\_cgal~-os} (see section
\ref{sec:other-options}).  E.g. if your compiler is \texttt{g++}, you
would type
\begin{alltt}
CGAL_MAKEFILE=`\yourcgaldir/install_cgal -os g++`
\end{alltt}
in bourne shell resp.
\begin{alltt}
setenv CGAL_MAKEFILE `\yourcgaldir/install_cgal -os g++`
\end{alltt}
in csh derivatives. 

\textit{Tip:} Include the setting of \texttt{CGAL\_MAKEFILE} into your
shell startup script (e.g. \texttt{.}(\texttt{t})\texttt{cshrc} for
(\texttt{t})\texttt{csh} or \texttt{.bashrc} for \texttt{bash}).

All makefiles contain sections with compiler and linker flags.  You
can add your own flags here. For example, you might want to add the
flag \texttt{-DCGAL\_NO\_PRECONDITIONS} to turn off precondition
checking. The flags \texttt{\$(CGAL\_CXXFLAGS)} and
\texttt{\$(CGAL\_LDFLAGS)} should never be removed.

The default extension for \cgal\ source files is \texttt{.C}.  The
last section of the makefiles contains a suffix rule that tells the
compiler how to create a \texttt{.o}-file from a \texttt{.C}-file.  If
you want to use the default rule that is defined by the make utility,
you may want to remove this suffix rule.  However, note that this may
have consequences for the makefile variables \texttt{CGAL\_CXX} and
\texttt{CXXFLAGS}.

\section{Compiling a \cgal\ application}
\index{building applications} 
\index{compiling applications}

The directory \texttt{\cgaldir/examples} contains a small program
(example.C) and a sample makefile with some comments. The
\texttt{CGAL\_MAKEFILE} variable in this makefile (see section
\ref{sec:makefiles}) is automatically substituted by the
\texttt{install\_cgal} script and equals the most recently generated
include makefile in the \texttt{\cgaldir/make} directory. After the
installation of \cgal\ this sample makefile is ready for use. Just
type '\texttt{make example}' to compile the program
\texttt{example.C}. There is a script for conveniently creating
makefiles for \cgal\ applications, see
section~\ref{sec:create_cgal_makefile}.

Furthermore the directories \texttt{\cgaldir/examples} and
\texttt{\cgaldir/demo} contain many subdirectories with non-graphical
and graphical example programs.  In all these directories you will
find a makefile that is ready for use.

\section{Installation on Cygwin}\label{sec:cygwin}
\index{Cygwin!installation on}\index{installation!on Cygwin}

Cygwin is a free Unix-like environment for MS-Windows, distributed by
Cygnus Solutions. For our tests we have used version 1.3.2 and
$B$-20.1.

It consists of a port of a large number of GNU tools, such as bash,
make, gcc, gas, file utilities, etc, as well as tools ensuring an
ability to emulate Unix-like access to resources, for instance mount.
For a comprehensive introduction and details, see
\path~http://www.cygwin.com/~ .

Make sure that the link \texttt{/bin/sh.exe} exists. If not, create
it:
\begin{verbatim}
cd /bin
ln -s bash.exe sh.exe
\end{verbatim}

\subsection{Pathnames}
\index{Cygwin!pathnames}

Cygwin has a UNIX-like way of navigating hard drives, NFS shares, etc.
This is also the way in which directories and pathnames have to given
to the installation script. They are automatically converted to
Win32-style pathnames when given to the compiler or linker.

The main difference is that directories are seperated by slash (``/'')
rather than by backslash (``$\backslash$'').  The other difference is
concerned with specifying drives. One way is to use POSIX-style
pathnames that map Win32-style drives (\texttt{A:}, \texttt{B:}) to
\texttt{//a/\ldots}, \texttt{//b/\ldots} respectively. For instance,
the path
\texttt{D:$\backslash$Mystuff$\backslash$Mydir$\backslash$LEDA}
translates to \texttt{//d/Mystuff/Mydir/LEDA}.

Alternatively, it can be done using the mount utility, that can be
used to establish a map between Win32-style drives and the Unix-like
style. More precisely, it maps the forest of the directories/files on
Win32-drives to a tree with the root that is usually located at the top
level of the boot drive, say \texttt{C:}.  The root location can be
seen by typing \texttt{mount} command without parameters.  For
instance, if \texttt{D:} is mounted on
\texttt{C:$\backslash$ddrive}\footnote{by typing \texttt{mount D:
    /ddrive}} then the path
\texttt{D:$\backslash$Mystuff$\backslash$Mydir$\backslash$LEDA}
translates to \texttt{/ddrive/Mystuff/Mydir/LEDA}.

\paragraph{Upper/lower case and spaces in file names}
Behavour of Cygwin in this regard might be different from the \mswin\ 
behaviour. In particular, using spaces in filenames should better be
avoided.

\paragraph{Links, shortcuts, etc} should be avoided as well.

\subsection{\msvc{}-setup}
\index{Cygwin!setup for \msvc{}}
\index{msvc@\msvc{}!setup on cygwin}

A number of environment variables has to be set (or updated)
in order to use the installation.

\texttt{PATH} should contain \msvc{} command line tools locations.
The environment variables \texttt{INCLUDE} and \texttt{LIB} should
point to the location of \msvc{} header files and to the location of
the \msvc{} libraries, respectively.  The interface for doing this is
different for NT and for Win9*.

\paragraph{\mswin-NT4.0.}
\index{Cygwin!setup on NT4}

One can set the corresponding environment variables using the
usual NT interface\footnote{open MyComputer, press right mouse button,
  select Properties, select Environment, set the relevant variables}.
Alternatively, they can be set in the \texttt{.bashrc} file for the
particular user, or in the system-wide \texttt{bash} customization
file (usually \texttt{/etc/bashrc}).

The result should look roughly as follows, assuming that
\texttt{C:$\backslash$PROGRA$\sim$1$\backslash$MICROS$\sim$2$\backslash$}
is the location of the \msvc{} installation.
\begin{verbatim}
LIB=C:\PROGRA~1\MICROS~2\VC98\LIB
INCLUDE=C:\PROGRA~1\MICROS~2\VC98\INCLUDE
\end{verbatim}
and \texttt{PATH} should contain
\begin{verbatim}
/PROGRA~1/MICROS~2/Common/msdev98/BIN:
/PROGRA~1/MICROS~2/VC98/BIN:/PROGRA~1/MICROS~2/Common/TOOLS:
/PROGRA~1/MICROS~2/Common/TOOLS/WINNT
\end{verbatim}

\paragraph{\mswin-9*.} 
\index{Cygwin!setup on \mswin-9*}

First, the memory for environment variables has to be increased.
Select the Cygwin icon from the Start-menu, press the right mouse
button and choose \textit{Properties}. Go to \textit{Memory}, select
\textit{Initial Environment}, set it to at least 2048 and
\textit{apply} the changes.

Second, edit the file \texttt{cygwin.bat} (or \texttt{cygnus.bat} in
Cygwin~0.9), located in the cygwin main directory and add the line
\begin{verbatim}
call C:\PROGRA~1\MICROS~2\VC98\Bin\MSCVARS32.BAT
\end{verbatim}
where
\nonlinkedpath'C:\PROGRA~1\MICROS~2\'
%\texttt{C:$\backslash$PROGRA$\sim$1$\backslash$MICROS$\sim$2$\backslash$}
has to be customized according to where \msvc{} is installed on your
system. Depending on the version of \msvc{} you might have to replace
\texttt{MSCVARS32.BAT} by \texttt{VCVARS32.BAT}.

\lcTex{\begin{appendix}}

\section{Using \cgal\ and \leda}\label{sec:leda}
\index{LEDA@\leda!support in \cgal}
\cgal\ supports \leda\ in the following ways.

\begin{enumerate}
\item There are support functions defined for the \leda\ number types
  \texttt{big\_float}, \texttt{integer}, \texttt{rational} and
  \texttt{real} (see the files \texttt{<CGAL/leda\_*>}).
\item For all two-dimensional geometric objects there are input/output
  operators for a \texttt{leda\_window}.
\item For all two-dimensional geometric objects there are output
  operators to a \texttt{leda\_ps\_file}.
\item The registration functions needed to interact with a
  \texttt{leda\_geowin} are defined for all geometric objects from the
  \cgal\ kernel.
\item \cgal\ defines the following \leda-related compiler flags:
\begin{itemize}
\item When \leda\ is used, the flags \texttt{CGAL\_USE\_LEDA} and
  \texttt{LEDA\_PREFIX} will be set.
\item When \leda\ is used, the \leda\ handles and memory management
  will be used in \cgal.  This can be turned off by setting the flag
  \texttt{CGAL\_NO\_LEDA\_HANDLE}. In that case, \cgal\ handles will
  be used (see \texttt{<CGAL/Handle.h>}).
\end{itemize}
\end{enumerate}

The include makefiles in the \texttt{\cgaldir/make} directory
corresponding to \leda\ can be recognized by the suffix
``\texttt{\_LEDA}''.

\section{Compiler workarounds}
\index{compilers!workarounds}\index{workaround flags}

In \cgal\ a number of compiler flags is defined, all of them start
with the prefix \texttt{CGAL\_CFG}. These flags are used to work
around compiler bugs and limitations.  For example, the flag
\texttt{CGAL\_CFG\_NO\_MUTABLE} denotes that the compiler does not
know the keyword \texttt{mutable}. 

For each compiler a file \texttt{<CGAL/compiler\_config.h>}
\index{files!\texttt{compiler\_config.h}} is defined, with the correct
settings of all flags. This file is generated automatically by the
\texttt{install\_cgal} script. For this the test programs in the
directory \nonlinkedpath'CGAL-3.0/config/testfiles'
\index{directories!config/testfiles@\texttt{config/testfiles}} are
used.  The file \texttt{<CGAL/compiler\_config.h>} and the test
programs contain a description of the problem, so in case of trouble
with a \texttt{CGAL\_CFG} flag it is a good idea to take a look at it.

The file \texttt{<CGAL/config.h>} \index{files!\texttt{config.h}}
manages all configuration problems of the compiler.  This file
includes the file \texttt{CGAL/compiler\_config.h}.  It is therefore
important that the file \texttt{<CGAL/config.h>} is always included
before any other \cgal\ source file that depends on workaround flags.
In most cases you do not have to do anything special for this, because
many CGAL files already take care of including
\texttt{<CGAL/config.h>}. Nevertheless it would be a good idea to
always start your \cgal\ programs with including
\texttt{<CGAL/config.h>} (or \texttt{<CGAL/basic.h>}, which contains
some more basic \cgal\ definitions).

\section{Compiler Optimizations}\label{sec:compiler-optimisations}
\index{compilers!optimization}
\index{optimization compiler flags}

You may have noticed that we do not set optimizer flags as \texttt{-O}
by default in the include makefiles(see section~\ref{sec:makefiles} for
a description of the makefile structure in \cgal).  The main reason
for not doing this is that compilers run much more stable without. On
the other hand, most if not all \cgal\ programs will run considerably
faster when compiled with optimizations! So if you are going for
performance, you should/have to add \texttt{-O}, \texttt{-O3} or maybe
more specific optimizer flags (please refer to the compiler
documentation for that) to the \texttt{CXXFLAGS} variable in your
application makefile:
\begin{alltt}
#---------------------------------------------------------------------#
#                    compiler flags
#---------------------------------------------------------------------#
# The flag CGAL_CXXFLAGS contains the path to the compiler and is defined
# in the file CGAL_MAKEFILE. You may add your own compiler flags to CXXFLAGS.

CXXFLAGS = $(CGAL_CXXFLAGS) -O
\end{alltt}
%$

\section{Troubleshooting}\label{sec:troubleshooting}
\index{troubleshooting}

This section contains some remarks about known problems and the
solutions we propose. If your problem is not listed here, please have
a look at the \cgal\ homepage:
\begin{alltt}
\cgalhomepage
\end{alltt}
or send an email to \texttt{cgal@cs.uu.nl}.

\subsection{The ``Long-Name-Problem'' on IRIX6}\label{sec:longnameirix}
\index{problems with long names}\index{long name problem}

The system assembler and linker on IRIX6 cannot handle symbols with
more than 4096 characters. But this number can be exceeded where one
starts nesting templates into each other. So if you encounter strange
assembler or linker errors like
\begin{alltt}
as: Error: /var/tmp/ccPBl5vJ.s, line 41289: Truncating token: 
<some ridiculously long token snipped>
\end{alltt}
there is a good chance that you suffer from this ``long-name''
problem.

In contrast to Solaris, using the \gnu\ binutils does not work, since
\texttt{gas} has not been ported to IRIX6 yet. The solution proposed
in the GCC
faq\footnote{\path'http://gcc.gnu.org/cgi-bin/fom.cgi?file=41'} is to
compile with the (experimental) option \texttt{-fsquangle}, that
enables compression of symbol names. This option is experimental and
likely to disappear in GCC3, but there the problem will hopefully be
fixed by a new (standard) ABI.

Citing from the above FAQ:

\begin{alltt}
Note that this option is still under development, and subject to
change. Since it modifies the name mangling mechanism, you'll need to
build libstdc++ and any other C++ libraries with this option enabled.
Furthermore, if this option changes its behavior in the future, you'll
have to rebuild them all again. :-(

This option can be enabled by default by initializing
`flag_do_squangling' with `1' in `gcc/cp/decl2.c' (it is not
initialized by default), then rebuilding GCC and any C++ libraries.
\end{alltt}

\subsection{The ``Long-Name-Problem'' on Solaris}\label{sec:longnamesolaris}
\index{problems with long names}\index{long name problem}

The system assembler and linker on Solaris 2.5 and 2.6 cannot handle
symbols with more than 1024 characters. But this number is quickly
exceeded where one starts nesting templates into each other. So if you
encounter strange assembler or linker errors like 
\begin{alltt}
/usr/ccs/bin/as: "/var/tmp/cc0B5iGc.s", line 24: 
error: can't compute value of an expression involving an external symbol
\end{alltt}
there is a good chance that you suffer from this ``long-name''
problem.

A solution is to install the \gnu -binutils\footnote{see
  \path'http://www.gnu.org/software/binutils/'} and to tell the
compiler that it shall use the \gnu-- instead of the native tools.
From the compiler-menu (described in section \ref{sec:compiler-menu})
you can set the corresponding option through the custom compiler
flags, i.e. for \texttt{gcc} you would add
\begin{alltt}
  -B/my/path/to/gnu/binutils/bin
\end{alltt}
assuming you installed the \gnu -binutils executables in
\texttt{/my/path/to/gnu/binutils/bin}.

If you cannot (or do not want to) install \gnu -binutils, there is a
workaround that lets you compile, link and run your programs, but it
prevents debugging, since the executables have to be stripped. In
short the workaround is to compile with \texttt{-g} and to link with
\texttt{-z nodefs -s} on Solaris, \texttt{-U -s} on IRIX,
respectively.

In order to still have portable makefiles (see
section\ref{sec:makefiles}), we define flags
\texttt{LONG\_NAME\_PROBLEM\_CXXFLAGS} and
\texttt{LONG\_NAME\_PROBLEM\_LDFLAGS} in the include makefiles which
are empty except for the Solaris platform where they are set as stated
above. In order to use these flags, edit your application makefile and
add the flags to \texttt{CXXFLAGS} resp. \texttt{LDFLAGS} as indicated
below.
\begin{alltt}
#---------------------------------------------------------------------#
#                    compiler flags
#---------------------------------------------------------------------#
# The flag CGAL_CXXFLAGS contains the path to the compiler and is defined
# in the file CGAL_MAKEFILE. You may add your own compiler flags to CXXFLAGS.

CXXFLAGS = $(LONG_NAME_PROBLEM_CXXFLAGS) $(CGAL_CXXFLAGS) 

#---------------------------------------------------------------------#
#                    linker flags
#---------------------------------------------------------------------#
# The flag CGAL_LDFLAGS contains common linker flags and is defined
# in the file CGAL_MAKEFILE. You may add your own linker flags to CXXFLAGS.

LDFLAGS = $(LONG_NAME_PROBLEM_LDFLAGS) $(CGAL_LDFLAGS) 
\end{alltt}

\subsection{\leda\ and \stl\ conflicts}\label{subs:ledastlconfl}

If you are using an old version of \leda, the combination of \leda\ 
and \stl\ may give some problems. In order to avoid them, it is highly
recommended to use the latest \leda\ release, since this is what we
test \cgal\ with.

\index{LEDA@\leda!on \bcc}\index{LEDA@\leda!on \msvc}
With \msvc\ or \bcc, \leda\ has to be compiled and used with the
\texttt{LEDA\_STD\_HEADERS} flag set. \cgal\ uses \CC\ standard
conformant headers\footnote{the ones that do not have \texttt{.h}
  suffix}, while \leda\ can also work with the old-style header files;
but mixing the styles is strictly forbidden. Before compiling \leda\,
edit the file \texttt{\$(LEDAROOT)/incl/LEDA/system.h} and uncomment
the \texttt{\#define} in the following fragment there.
\begin{alltt}
// use c++ std headers
//#define LEDA_STD_HEADERS
\end{alltt}

\paragraph{\msvc-specific problems.}

Also, the \leda\ and \cgal\ libraries have to be compiled with the
same options controlling the use of debugging and multithreading.
\footnote{\msvc\ compilation/linking options \texttt{-ML, -MT, -MD,
    -MLD, -MTD, -MDD}}

If a binary release of \leda{} is used, make sure that it is one of
them that uses new-style headers. Namely, among the self-extracting
executables, choose one of these that have the name ending with
\texttt{-std.exe}.

\subsection{\msvc{}-specific \CC\ problems}
\index{msvc@\msvc{}!specific \CC\ problems}

\subsubsection{\msvc{6.0}}

\cgal\ uses a modified drop-in replacement for STL, STLPort, version
3.2.1. Our modifications included
\begin{itemize}
\item modifying the code to enable
the use of \texttt{std::iterator\_traits} template, where possible;
\item providing \texttt{std::vector} class that does not use
a plain C pointer as \texttt{vector\_iterator}.
\end{itemize}
The latter allows using the full power of STL algorithms on
\texttt{std::vector} without the limitation given by the absence
of the support for partial specialization of template parameters.

Always make sure that the subdirectory
\texttt{\$(CGAL\_ROOT)/stlport/} %$
comes first in the list of additional include directories for \msvc.
This is handled automatically if you use \cgal-supplied (or \cgal-generated)
makefiles.

\subsubsection{\msvc{6.0} and \msvc{7.0}
  (\textsc{.NET})}

\paragraph{\texttt{std::iterator\_traits}} is only partially
supported, due to absence of partial specialization support in \msvc.
This means that \texttt{std::iterator\_traits<T>} must be explicitly
instantiated when \texttt{T} is a C pointer. 

\cgal\ provides
these instantiations for most commonly used primitive data types and
\cgal\ kernel types\footnote{\cgal\ \texttt{Point, Segment}, etc.}.
For other types the user must do this himself.  
This can be done by using the macro
\nonlinkedpath'CGAL_DEFINE_ITERATOR_TRAITS_POINTER_SPEC(T)'.
Note that the parameter of this macro cannot contain symbols $<$ and
$>$. Thus the macro cannot be used with types that contain $<$ and $>$
directly. However, one can be done by first define a new type using 
\texttt{typedef} and then applying the macro to this new type.
See e.g. \nonlinkedpath'examples/Getting_started/advanced_hull.C'
for an example of using this macro.

Note that this macro does not have any effect when other than
\msvc{6.0} compiler is used, so it can be safely left in the source code. 
\medskip

\textbf{Other problems.}

Here goes an incomplete list of problems encountered, and
\cgal-specific workarounds, if available.  Compiler error messages are
meant to be hints only, and do not pretend to be complete, as well.
\begin{enumerate}
%
\item Partial specialization of template parameters is not
supported. Do not use it. Error messages say that ``the class has
already been declared'', or ``unrecognizable template
declaration/definition''. See
\nonlinkedpath'config/testfiles/CGAL_CFG_NO_PARTIAL_CLASS_TEMPLATE_SPECIALISATION.C'\\
 for a test program illustrating the problem. \label{msvc::parspec}
%
\item Compiler does not always match the most
specialized instance of a function template correctly. So you get
``error C2667: none of 2 overload have a best conversion''.\\
See \texttt{config/testfiles/CGAL\_CFG\_MATCHING\_BUG\_2.C} \\
for a test program illustrating the problem.\label{msvc::matchbug2}
%
\item Compiler does not support the Koenig
lookup. That is, it does not search in the namespace of the arguments for
the function. See
\texttt{config/testfiles/CGAL\_CFG\_NO\_KOENIG\_LOOKUP.C}. \label{msvc::koenig}
%
\item A scope operator in  parameter types of a member function only
works when the function is defined inline. See
\texttt{config/testfiles/CGAL\_CFG\_NO\_SCOPE\_MEMBER\_FUNCTION\_PARAMETERS.C}. 
\label{msvc::scopememberfunc}
%
\item Passing a dependant type as template parameter cannot be done
with \texttt{typename} keyword on \msvc. Replace \texttt{typename}
by \texttt{CGAL\_TYPENAME\_MSVC\_NULL} in such places.\\ 
See \texttt{config/testfiles/CGAL\_CFG\_TYPENAME\_BUG.C}. \label{msvc::typename}
%
\item Template \texttt{friend} declarations must not contain $<>$.
Replace $<>$  by \texttt{CGAL\_NULL\_TMPL\_ARGS}.\\
See \texttt{config/testfiles/CGAL\_CFG\_NO\_TEMPLATE\_FRIEND\_DISTINCTION}.
\label{msvc::friend}
%
\item Using non-class template parameters often gives internal
  compiler errors.
%
\item Internal compiler errors can sometimes be avoided by increasing
  the amount of memory available to the compiler. Use \texttt{-Zm<number>}
option. In \cgal\ makefiles it is set to \texttt{-Zm900}, meaning
``using 900\% out of the usual memory limit''. \label{msvc::Zm}
%
\item \texttt{[...]/VC98/INCLUDE/xlocnum(268) :
    error C2587: '\_U' :\\
    illegal use of local variable as default parameter} can
  occur\footnote{For instance, in \cgal\ \texttt{Min\_circle}
    package}. The only workaround we know is to redefine the macro
  \texttt{\_VIRTUAL} in \texttt{<xlocnum>}\footnote{Yes, in the \msvc\ 
    header! You need not edit the actual file though. Copy it to
    \cgaldir\texttt{/stlport/} and edit it there. As
    \cgaldir\texttt{/stlport/} is searched by the compiler ahead of
    the other directories, your updates will be used.  DISCLAIMER:
    {\em We do not know if the actions described in this footnote are
      legal in your country. You are on your own here.}}
  to be empty. Search for\\
  \texttt{\#define~\_VIRTUAL~virtual} there and replace it by
  \texttt{\#define~\_VIRTUAL~}. \label{msvc::VIRTUAL}
%
\item Various matching failures for overloaded functions and ctors.
Use dummy parameters.
%
\item Avoid multiple forward declarations.
%
\item If necessary, simplify template parameters by using extra 
\texttt{typedef}s.
\end{enumerate}

\section{Scripts}

\subsection{\texttt{create\_makefile}}\label{sec:create_cgal_makefile}
\TTindex{create\_makefile}\index{scripts!\texttt{create\_makefile}}

The bourne-shell script \texttt{create\_makefile} is contained in the
\cgaldir\texttt{/scripts} directory. It can be used to create
makefiles for compiling \cgal\ applications. Executing
\texttt{create\_makefile} in an application directory creates a
\texttt{makefile} containing rules for every \texttt{*.C} file there.

In order to use this makefile, you have to specify the \cgal\ include
makefile (see section~\ref{sec:makefiles}) to be used. This can be
done be either setting the environment variable
\texttt{CGAL\_MAKEFILE} or by editing the line
\begin{alltt}
  # CGAL_MAKEFILE = ENTER_YOUR_INCLUDE_MAKEFILE_HERE
\end{alltt}
of the created makefile. First remove the ``\texttt{\#}'' at the
beginning of the line and then replace the text after ``\texttt{=}''
by the location of the include makefile. 

Finally type \texttt{make} to compile the application programs.

\subsection{\texttt{use\_cgal\_namespace}}\label{sec:use_cgal_namespace}
\TTindex{use\_cgal\_namespace}\index{scripts!\texttt{use\_cgal\_namespace}}

The perl script \texttt{use\_cgal\_namespace} is contained in the
\cgaldir\texttt{/scripts} directory. It can be used to convert
\cgal\texttt{-1.*} application sourcecode to the \cgal\texttt{-2.*}
format.  Basically, it replaces \texttt{CGAL\_} prefixes by
\texttt{CGAL::} namespace qualifiers. You have to give the files to
convert as arguments, e.g.
\begin{alltt}
  use_cgal_namespace my_great_file1.C *.h
\end{alltt}
The original files are kept with the suffix \texttt{.bck}.

In order to use it, you first have to set the perl path correctly,
i.e. replace \texttt{/net/bin/perl5} in the first line by the path to
perl on your system (try \texttt{which perl}(5), if you do not know).
Alternatively, you can type
\begin{alltt}
  perl -wi.bck -- use_cgal_namespace <FILES>
\end{alltt}

\lcTex{\end{appendix}}

%%
%% EOF
%%


