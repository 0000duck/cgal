%%
%%  *** CGAL Installation Guide ***
%%
%% file:     installation.tex
%%
%% authors:  Michael Hoffmann and Wieger Wesselink
%%
%% $Revision$ $Date$
%%

%% macro for GNU
\newcommand{\gnu}{\textsc{Gnu}}

%% macro for g++
\newcommand{\Gcc}[1]{\gnu~\texttt{g++}~{\rm #1}}

%% macro for egcs
\newcommand{\egcs}[1]{\textsc{Cygnus}~\texttt{egcs}~{\rm #1}}

%% macro for SUNPRO CC
\newcommand{\sunprocc}[1]{\textsc{Sunpro}~\texttt{CC}~{\rm #1}}

%% macro for SGI CC
\newcommand{\mipsprocc}{\textsc{SGI}~Mips(Pro)~\texttt{CC}}

%% macro for Borland C++
\newcommand{\bcc}[1]{\textsc{Borland}~\texttt{C++}~{\rm #1}}

%% macro for Microsoft Visual C++
\newcommand{\msvc}[1]{\textsc{Microsoft}~Visual~\texttt{C++}~{\rm #1}}

%% macro for CGAL release number
\newcommand{\cgalrelease}{1.1}

%% macro for CGAL directory
\newcommand{\cgaldir}{\texttt{CGAL-\cgalrelease}}

%% macro for YOUR CGAL directory :)
\newcommand{\yourcgaldir}{$<$\textit{insert your \cgaldir\ dir}$>$}

%% macro for CGAL install config directory
\newcommand{\cgalinstconfdir}{\cgaldir\texttt{/config/install}}

%% macros for WWW pages
\newcommand{\cgalhomepage}{\path'http://www.cs.uu.nl/CGAL'}
\newcommand{\compilerpage}{\path'http://www.cs.uu.nl/CGAL/Compilers/'}
\newcommand{\faqpage}{\path'http://www.cs.uu.nl/CGAL/FAQ/'}
%%\newcommand{\hpstlpage}{\path'http://www.cs.rpi.edu/~musser/stl.html'}
\newcommand{\sgistlpage}{\path'http://www.sgi.com/Technology/STL/'}
\newcommand{\stlportpage}{\path'http://www.metabyte.com/~fbp/stl/'}

%%
%% titlepage
%%

\pagestyle{empty}

\begin{titlepage}
\mbox{}
\vskip3cm

\ccTexHtml
{\centerline{\IpeScale{25}\Ipe{../Fig/c-simp.ipe}
  \Ipe{../Fig/g-simp.ipe} \Ipe{../Fig/a-simp.ipe} \Ipe{../Fig/l-simp.ipe}}}
 {<p align=center>
   <img border=0 src=Gif/c-simp.gif align=center alt="Logo C">
   <img border=0 src=Gif/g-simp.gif align=center alt="Logo G">
   <img border=0 src=Gif/a-simp.gif align=center alt="Logo A">
   <img border=0 src=Gif/l-simp.gif align=center alt="Logo L">
   <br>
   <img border=0 src=Gif/cgal.gif align=center
    alt="[Computational Geometry Algorithms Library]">
    <hr>}

\vskip2cm
\centerline{ {\Huge\bf CGAL} {\huge\bf Reference Manual } }
\vskip1cm
\centerline{ {\huge d-dimensional Kernel} }

\vskip2cm
\newcommand{\mydate}{July 1998}
\ccTexHtml
{\renewcommand{\mydate}{\ifcase\the\month \or January\or
    February\or March\or April\or May\or June\or July\or August\or
    September\or October\or November\or December\fi~\the\year}}{}
  \centerline{ {\large Release \cgalrelease, \mydate} }

\end{titlepage}






%%
%% table of contents 
%%
\pagenumbering{arabic}

\tableofcontents

%%
%% main body 
%%

\section{Introduction}

\cgal\ is a \textit{Computational Geometry Algorithms Library\/} written
in \CC, which is developed in an {\sc Esprit Ltr} project. The goal is
to make the large body of geometric algorithms developed in the field
of computational geometry available for industrial application.

This document describes how to install \cgal\ on your system.  Besides
that, you will find some information about the makefile structure of
\cgal\ and the support for simultaneously using \cgal\ and \leda, the
Library of Efficient Datatypes and Algorithms\footnote{see
  \path~http://www.mpi-sb.mpg.de/LEDA/index.html~ for information.},
and/or the \gnu\ Multiple Precision library GMP.

\section{Prerequisites \label{sec:prerequisites}}

In order to build the \cgal\ libraries you need a \CC\ compiler.
Currently only a small number of recent compilers on unix platforms
are supported. The reason is that most compilers do not conform to the
upcoming ANSI/ISO standard\footnote{see e.g.
  \path~http://reality.sgi.com/austern/std-c++/faq.html\#PartB~ for
  information} and some of them have so many limitations/bugs that we
could not work around all of them.

More precisely, \cgaldir\ supports the following Unix
compilers/operating systems:

\begin{center}
  \renewcommand{\arraystretch}{1.3}
  \begin{tabular}{|l|l|} \hline
    \textbf{compiler}     & \textbf{operating system}\\\hline\hline
    \mipsprocc\ 7.2 (n32) & IRIX 6.2\\\hline
    \sunprocc{4.2}        & Solaris 2.5\\\hline
    \Gcc{2.8.*}           & IRIX 6.2 / SUN Solaris 2.5 / Linux 2.0\\\hline
    \egcs{1.0.*}          & IRIX 5.3 / IRIX 6.2 / Solaris 2.5 / Linux 2.0\\\hline
  \end{tabular}
\end{center}

There are plans to support PC compilers (\msvc{}) in the future, but
there is no such support at the moment.

%%If your compiler/operating system is not on this list, this not
%%necessarily implies that \cgal\ does not work on your
%%system\footnote{e.g. we would expect \egcs{1.0.*} to work on IRIX 6.4
%%  as well}, it just means we did not test it so far. For some
%%compilers we did the tests, but not all parts of \cgal\ work with
%%them. This includes \mipsprocc\ 4.0, \sunprocc{4.1} and \Gcc{2.7.2}.
%%An overview of known problems is given at the following URL:
%%\begin{quote}
%%      \compilerpage
%%\end{quote}

%%N.B. It is highly recommended to read this information, since some of
%%these older compilers may cause a lot of problems! Especially the
%%support for templates is sometimes very poor. 

%%At the same location you can find some hints about \cgal\ support for
%%PC compilers (\bcc{5.02}\ and \msvc{5.0}), but this in a very
%%premature state at the moment.

%%In case you want to use \leda\ together with \cgal\ (needed e.g. for
%%the graphical demo programs), you should have a look at section
%%\ref{sec:leda} now.

\section{Getting \cgal}

The \cgal\ library can be downloaded in two different ways: using ftp
or using WWW.  If you have a WWW connection, the easiest way to
download \cgal\ is via the \cgal\ homepage:
\begin{quote}
      \cgalhomepage
\end{quote}
and go to the `Software' section.

Just follow the instructions on this page to obtain your copy of the
library. The \cgal\ library can also be downloaded using FTP. The
library can be found at the following location:

\begin{verbatim}
      ftp.cs.uu.nl
\end{verbatim}

in the directory \texttt{/pub/CGAL}. This directory contains release
\cgalrelease\ of the \cgal\ library. There is also a \texttt{README}
file that contains descriptions of the files in this directory. An
example of an FTP-session is given below.

\begin{alltt}
      $ ftp ftp.cs.uu.nl 
      Name (ftp.cs.uu.nl:<your username>): anonymous
      Password: <type your email address here>
      ftp> cd pub/CGAL
      ftp> get README
      ftp> binary
      ftp> get \cgaldir.tar.gz
      ftp> quit
\end{alltt}

After you have downloaded the file containing the \cgal\ library, you
have to decompress it. For the zipfile use the command

\begin{verbatim}
      unzip <filename>.zip
\end{verbatim}

and for the gzipped file use the commands

\begin{verbatim}
      gunzip <filename>.tar.gz
      tar xvf <filename>.tar
\end{verbatim}

%% N.B. On a PC you should use an unzip utility that can deal with long
%% filenames (like WinZip or InfoZip)!

In both cases the directory \cgaldir\ will be created. This directory
contains the following subdirectories:

\begin{center}
  \renewcommand{\arraystretch}{1.3}
  \begin{tabular}{|l|l|} \hline
    \textbf{directory} & \textbf{contents}\\\hline\hline
    \texttt{auxiliary} & packages that can optionally be used with \cgal\\\hline
    \texttt{config}    & configuration files for install script\\\hline
    \texttt{demo}      & demo programs (some of them  need \leda)\\\hline
    \texttt{doc\_html} & documentation (HTML)\\\hline
    \texttt{doc\_pdf}  & documentation (PDF)\\\hline
    \texttt{doc\_ps}   & documentation (Postscript)\\\hline
    \texttt{examples}  & example programs\\\hline
    \texttt{include}   & header files\\\hline
    \texttt{lib}       & (shared) object libraries\\\hline
    \texttt{make}      & files with platform dependent makefile settings\\\hline
    \texttt{src}       & source files\\\hline
  \end{tabular}
\end{center}

\section{Installing \cgal}

The directory \cgaldir\ contains a Bourne shell script called
\texttt{install\_cgal}. The script can be run in two modes: a
menu-driven interactive mode and a non-interactive mode.  Normally you
should use the interactive mode, but in case you run into problems
with it or do not like it for some reason, you can still use the
non-interactive mode.

We first describe a sample installation in section
\ref{sec:sample-inst}. This provides you with an overview on how the
interactive installation works. If you want more detailed information
about specific menus and their options, take a look at section
\ref{sec:interactive-mode}. Finally, for the non-interactive mode
refer to section \ref{sec:non-interactive}.

If you want to use \leda\ in combination with \cgal, there might be
some compatibility issues that need to be addressed first. See
section~\ref{sec:leda} and \compilerpage\ for more information about
this.

\section{A Sample Installation \label{sec:sample-inst}}

In this section we sketch an example installation on a \textsc{Sun}
running Solaris 2.5 with the \sunprocc{4.2} compiler. For a complete
description of the different menus and their options refer to section
\ref{sec:interactive-mode}.

\subsection{Starting the script}

Go to the \cgaldir\ directory and enter the command
\begin{verbatim}
      ./install_cgal -i
\end{verbatim}

You get a message indicating the \cgal\ version you are going to
install and that you are running the interactive mode. Then it takes
some time while the script locates a number of utility programs. You
will not get informed about this\footnote{If you are that curious what
  happens exactly, have a look at the file
  \texttt{\cgaldir/install.log}.}, but see some dots written to the
screen indicating progress.

\begin{scriptsize}
\begin{alltt}
--------------------------------------------------------
  This is the install script for CGAL \cgalrelease
--------------------------------------------------------

starting interactive mode - one moment, please
......

  Choosing compiler SUNPRO 4.2.
  Testing for builtin STL ... no.
.
\end{alltt}
\end{scriptsize}
  
If there is any compiler installed on your system and accessible
through your \texttt{PATH} environment variable that is supported by
\cgal, one of these compilers is chosen and it is checked whether it
has a ``builtin'' \stl, i.e. if \stl\ works without adding any
specific compiler flags.  Here the chosen compiler is \sunprocc{4.2}
that has no builtin \stl.  \footnote{If your compiler has a builtin
  \stl, you will have to wait for a moment while the script tests for
  a number of language features, which cannot be done without \stl.}
If there is more than one compiler installed on your system (and
supported by \cgal), you may later choose to use a different compiler
from the compiler menu.

A menu similar to the following will appear on your screen.

\begin{scriptsize}\label{pic:main-menu}
\begin{alltt}
  ****************************************************************
  **              CGAL 1.1 Installation Main Menu               **
  **              -------------------------------               **
  **                                                            **
  **   OS:                  sparc_SunOS-5.5                     **
  **   Compiler:            SUNPRO CC 4.2                       **
  **   STL_DIR:             please configure!!!                 **
  **   LEDA:                not supported.                      **
  **   GMP:                 not supported.                      **
  **                                                            **
  **   The setup has not been tested.                           **
  **                                                            **
  **   There are no libs for this os/compiler.                  **
  **                                                            **
  **   <C>  Compiler Menu                                       **
  **   <L>  LEDA Menu                                           **
  **   <G>  GMP Menu                                            **
  **   <T>  Test (and save) setup                               **
  **   <A>  Run all setup tests (no cache)                      **
  **                                                            **
  **   <B>  Build CGAL Libraries                                **
  **                                                            **
  **   <Q>  Back to OS                                          **
  **                                                            **
  **   Your Choice:                                             **
  **                                                            **
  ****************************************************************
\end{alltt}
\end{scriptsize}

The first lines below the headline contain some kind of status report:
current OS and compiler, are \leda\ and GMP supported etc., in this
case it also tells you that \stl\ still has to be configured.

Moreover you can see that the current setup has not been tested yet
and that there do not exist \cgal\ libraries for this OS/compiler
combination in the \cgal\ lib directory by now. It is always a good
idea to test your configuration before you start building the \cgal\ 
libraries, but before that we first have to setup \stl.\footnote{This
  is not necessary for compilers with builtin \stl. So you might want
  to skip the next section and go directly to section
  \ref{sec:build-the-libs}.}

\subsection{Setting up \stl \label{sec:set-up-stl}}

Please type ``\texttt{c}'' to go to the compiler menu where all
compiler specific options can be configured. Then type ``\texttt{i}''
and you will be prompted to enter the include directory where \stl\ 
header files reside on your system. The script only accepts
directories that exist and pass a confidence test.\footnote{In this
  case it checks for a file \texttt{iterator.h} in the directory.
  This file should be present in all \stl\ implementations.} Let's
assume, you have a \textsc{HP} \stl\ on your system in {\tt
  /pub/local/STL/HP}.

After you set up the \stl\ directory, the script tries to compile some
small example programs to check whether \stl\ works principally and if
it does, which language features are supported by your compiler and
your \stl\ version. This is necessary, since there are many different
\stl\ implementations and \cgal\ (partly) depends on which
implementation is used. A message \texttt{ok} always indicates that
your compiler works as it should i.e. a feature is supported resp. a
bug is \textit{not} present. On the other hand \texttt{no} resp.
\texttt{unfortunately} indicate a lack of support resp. bug.

\begin{scriptsize}
\begin{verbatim}
  ****************************************************************
  **            CGAL 1.1 Installation Compiler Menu             **
  **            -----------------------------------             **
  **                                                            **
  **   OS:                  sparc_SunOS-5.5                     **
  **   Compiler:            SUNPRO CC 4.2                       **
  **   STL_DIR:             please configure!!!                 **
  **   LEDA:                not supported.                      **
  **   GMP:                 not supported.                      **
  **                                                            **
  **   The setup has not been tested.                           **
  **                                                            **
  **   <C>  Choose compiler                                     **
  **   <I>  STL  include directory                              **
  **        <undefined>                                         **
  **   <F>  Set custom compiler flags                           **
  **       <none>                                               **
  **   <L>  Set custom linker flags                             **
  **       <none>                                               **
  **   <T>  Test (and save) setup                               **
  **                                                            **
  **   <Q>  Back to Main Menu                                   **
  **                                                            **
  **   New STL_DIR: /pub/local/STL/HP                           **
  **                                                            **
  ****************************************************************

  Testing for STL ... ok.
  Testing for SGI STL 3.0 ... no.
  Testing for SGI 6/97 STL ... no.
  Testing for SGI 1996 STL ... no.
  Testing for SGI CC STL ... no.
  Testing for HP STL ... ok.
  ...
  <tests for several compiler features>
  ...
  Saving current setup ... done.
\end{verbatim}
\end{scriptsize}

If the \stl\ test succeeds, the current setup is marked as tested and
the settings are saved to a file in the directory \cgalinstconfdir.
Thus, if you run the install script a second time for this
OS/compiler, you will not have to enter the \stl\ directory again, but
it is retrieved from the corresponding config file instead.

\subsection{Building the \cgal\ libraries \label{sec:build-the-libs}}

Now we are ready to build the \cgal\ libraries.  First go back to the
main menu with ``\texttt{q}`` and then type ``{\tt b}'' to start
compilation. Building consists of three steps:
\begin{enumerate}
\item writing the include makefile,
\item compiling the static libraries \textit{and}
\item compiling the shared libraries.
\end{enumerate}
The include makefile encapsulates the OS-- and compiler-specific
settings and should be included (hence the name) in all makefiles that
compile \cgal\ applications. If everything went ok, the output should
look as follows (if not, you should have a look at the error messages
from compiler or linker).

\begin{scriptsize}
\begin{verbatim}
  ****************************************************************
  **                                                            **
  **                     Compiling CGAL 1.1                     **
  **                     ------------------                     **
  **                                                            **
  ****************************************************************

  OS:           sparc_SunOS-5.5
  COMPILER:     SUNPRO CC 4.2
  STL:          CGAL_STL_HP
  LEDA:         not supported
  GMP:          not supported

  Generating Makefiles ... done.
  Building CGAL_lib ... done.
  Building CGAL_sharedlib ... done.

  ****************************************************************
  **             Please press <ENTER> to continue.              **
  ****************************************************************
\end{verbatim}
\end{scriptsize}

That's all, it's done. Press ``\texttt{<ENTER>}'' to return to the main
menu and proceed by installing for a different compiler (go to the
compiler menu and choose ``\texttt{c}'' to get a list of supported
compilers detected on your system) or with \leda\ or GMP support (go
to the \leda\ resp.  GMP menu) or simply quit the install script by
typing ``\texttt{q}''. When leaving the script, you get a list of
successful builds during the session.

Now it would be a good idea to print and read the document
\texttt{getting\_started.ps} that can be found in the \texttt{doc\_ps}
directory.

\section{The interactive mode}\label{sec:interactive-mode}

To run the install script in the interactive mode, go to the \cgaldir\ 
directory and enter the command
\begin{verbatim}
      ./install_cgal -i
\end{verbatim}

After initialization during which certain utility programs are located
and your system is searched for compilers supported by \cgal, you get
into the \cgal\ installation \textit{main menu} (see page
\pageref{pic:main-menu} for a picture).

From the main menu you can reach a number of different sub-menus, of
which the most important maybe is the \textit{compiler menu}. This is
where you can choose the compiler you want to work with and setup
\stl.  The compiler menu is described in section
\ref{sec:compiler-menu}.

If you want to use \leda\ or \gnu\ GMP with \cgal, you will have to go
to the \textit{leda menu} described in section \ref{sec:leda-menu} resp.
to the \textit{gmp menu} described in section \ref{sec:gmp-menu}.

Finally you can build the \cgal\ libraries by typing \texttt{b}.
However, it is recommended to run the \textit{setup test} which is
available in all menus as option \texttt{t} before. The setup test
includes an \stl\ test, a \leda\ test and a GMP test, but not all
tests are performed always; e.g. the \leda\ test is only done, if you
enabled \leda\ support. The install script keeps track of the tests
passed and only tests again, if you change the setup in a way that
might affect the test result. If you want to redo \textit{all} tests,
you have to choose option ``\texttt{a}'' from the main menu. This also
retests for \leda\ or GMP installation in system directories which
otherwise is only done the first time you enable \leda/GMP support for
an OS/compiler combination.

\subsection{Files created during installation}

The install script stores all relevant settings for an OS/compiler
combination in the directory 
\begin{center}
  \texttt{\cgalinstconfdir/$<$\textit{CGAL-OS-description}$>$}
\end{center}
where $<$\textit{CGAL-OS-description}$>$ identifies your OS/compiler
combination in a way specified in section \ref{sec:os-compiler-id}.
\footnote{Note that these files are only OS/compiler specific, i.e.
  there are no different files for with and without \leda\ support.}
This saves you typing everything again, if you upgrade \cgal\ or
another package that makes recompiling the \cgal\ libraries necessary.

Besides the config files, \texttt{install\_cgal} uses several
temporary files during interactive installation. Most of them are
removed after use, but some are not, since it might be helpful to keep
some information about the last run.  You can keep or delete them as
you like, as they are not needed anymore once the script terminated.
It follows a list of these files (all are plain ASCII and reside in
\cgaldir).
\begin{center}
  \renewcommand{\arraystretch}{1.3}
  \begin{tabular}{|l|l|} \hline
    {\bf filename} & {\bf content}\\\hline\hline
    \texttt{install.log} & detailed overall protocol\\\hline
    \texttt{install.completed} & list of systems for which \cgal\ libraries 
    have been built\\\hline
    \texttt{compile.log} & output of the last compiler call\\\hline
  \end{tabular}
\end{center}

\subsection{The Compiler Menu}\label{sec:compiler-menu}

Here is the place to setup the compiler specific options, such as the
compiler to use (if more than one has been detected) and the location of the
\stl\ files.

Some compilers come with their own \stl\ adaption to which we refer as
\textit{builtin} and some compilers are not shipped with \stl. In the
latter case you have to supply an \texttt{STL\_DIR}, i.e. the path to
a directory where the \stl\ header files are stored.\footnote{See
  section \ref{sec:stl} on where you can obtain an \stl\ 
  implementation and what problems you might have to consider.}  In
this case we speak of an external \stl. Even if your compiler has
builtin \stl, you can still choose to use an external \stl\ by setting
\texttt{STL\_DIR} appropriately.\bigskip

\fbox{
\begin{minipage}{.95\linewidth}
\vspace{2pt}
\centerline{{\large\bf Compiler Menu}}
\vspace{2pt}
\begin{description}
\item[$<$C$>$] Choose the compiler to be used from the list of
  detected compilers.
\item[$<$S$>$] (\textit{present if a builtin \stl\ was detected for the
    current compiler)} Determine, if compiler builtin or an external
  \stl\ is used.
\item[$<$I$>$] (\textit{present if an external \stl\ is used)} Set the
  include directory for an external \stl.
\item[$<$F$>$] Set custom compiler flags. These are the first flags
  given to the compiler in every call. Under normal circumstances
  there should be no need to set any such flag.
\item[$<$L$>$] Set custom linker flags. These are the first flags
  given to the linker in every call. Under normal circumstances
  there should be no need to set any such flag.
\end{description}
\vspace{2pt}
\end{minipage}
\hfil}

\subsection{The \leda\ Menu}\label{sec:leda-menu}

This is the place to setup \leda\ specific options, if you plan to use
\leda\ together with \cgal\ (see also section \ref{sec:leda}). In
order to enable \leda\ support in \cgal, \leda\ has to be installed on
your system.

If \leda\ support is enabled the first time, the script tests whether
\leda\ is installed in standard system directories. If this test does
not succeed, you have to supply directories containing the \leda\ header
files (\texttt{LEDA\_INCL\_DIR}) and \leda\ libraries ({\tt
  LEDA\_LIB\_DIR}).  Even if the tests are passed, you still have the
option to set these directories differently.\bigskip

\fbox{
\begin{minipage}{.95\linewidth}
\vspace{2pt}
\centerline{{\large\bf \leda\ Menu}}
\vspace{2pt}
\begin{description}
\item[$<$E$>$] Enable/Disable \leda\ support in \cgal.
\item[$<$I$>$] (\textit{present if \leda\ support is enabled)} Set the
  include directory for \leda .
\item[$<$J$>$] (\textit{present if \leda\ support is enabled, \leda\ headers
    have been found in a system include directory and {\tt
      LEDA\_INCL\_DIR} has been set)} Use \leda\ header from system
  include directory.
\item[$<$L$>$] (\textit{present if \leda\ support is enabled)} Set the
  directory containing the \leda\ libraries.
\item[$<$M$>$] (\textit{present if \leda\ support is enabled, \leda\ libs
    have been found in a system lib directory and \texttt{LEDA\_LIB\_DIR}
    has been set)} Use \leda\ libraries from system lib directory.
\end{description}
\vspace{2pt}
\end{minipage}
\hfil}

\subsection{The GMP Menu}\label{sec:gmp-menu}

This menu is to setup GMP (\gnu\ Muptiple Precision Library) specific
options, if you plan to use GMP together with \cgal . In the {\tt
  auxiliary} directory you can find a GMP distribution, if you do not
already have it installed on your system. This menu contains an option
to install GMP in you \cgal\ directory tree, but of course you can
also install it independently from \cgal .

If GMP support is enabled the first time, the script tests whether GMP
is installed in standard system directories or in the \cgal\ tree. If
this test does not succeed, you have to supply directories containing
the GMP header files (\texttt{GMP\_INCL\_DIR}) and GMP libraries ({\tt
  GMP\_LIB\_DIR}).  Even if the tests are passed, you still have the
option to set these directories differently.\bigskip

\fbox{
\begin{minipage}{.95\linewidth}
\vspace{2pt}
\centerline{{\large\bf GMP Menu}}
\vspace{2pt}
\begin{description}
\item[$<$C$>$] Install the GMP distribution shipped with \cgal\ in the
  \cgal\ directory tree. 
\item[$<$G$>$] Enable/Disable GMP support in \cgal.
\item[$<$I$>$] (\textit{present if GMP support is enabled)} Set the
  include directory for GMP.
\item[$<$L$>$] (\textit{present if GMP support is enabled)} Set the
  directory containing the GMP libraries.
\item[$<$M$>$] (\textit{present if GMP support is enabled, there is a GMP
    installation in system directories or in the \cgal\ tree and {\tt
      GMP\_INCL\_DIR} or \texttt{GMP\_LIB\_DIR} have been set)} Use GMP
  installation from system directories / \cgal\ tree.
\end{description}
\vspace{2pt}
\end{minipage}
\hfil}

\section{The non-interactive mode \label{sec:non-interactive}}

To run the install script in the non-interactive mode, go to the
\cgaldir\ directory and enter the command
\begin{verbatim}
      ./install_cgal -ni <compiler>
\end{verbatim}
where \texttt{<compiler>} is the C++ compiler executable.\\
You can either specify a full path, e.g. \texttt{/usr/local/bin/g++},
or just the basename, e.g. \texttt{g++}, which means the script
searches your \texttt{PATH} for the compiler location. If your
compiler call contains whitespaces it has to be quoted, e.g.
\texttt{./install\_cgal -ni "CC -n32"}.  The options given this way
become part of your \cgal-OS description (see section
\ref{sec:os-compiler-id}) which is useful e.g. to distinguish between
different compilers using the same frontend (\mipsprocc\ on
\texttt{IRIX6}).

There are a number of additional command line options to customize
your \cgal\ setup which are discussed below. You should read the
corresponding paragraphs before you continue, especially if one or
more of the following conditions apply to you:
\begin{itemize}
\item your compiler does not have a ``builtin'' \stl\ (section
  \ref{sec:stl-setup}),
\item you want to use \leda\ together with \cgal\ (section
  \ref{sec:leda-setup}) \textit{or}
\item you want to use \gnu\ GMP together with \cgal\ (section
  \ref{sec:gmp-setup}).
\end{itemize}

Once you started the script, it should give you a message indicating
the \cgal\ version you are going to install and that you are running
the non-interactive mode. Then it proceeds by locating some utility
programs, determining your OS and compiler version, displaying the
settings you gave via command line and classifying (see also section
\ref{sec:stl}) the \stl\ version used (which -- of course -- will not
work, if you did not set up \stl\ correctly, see section
\ref{sec:stl-setup} below). Your compiler is also checked for a number
of bugs resp. support of certain language features; a message
\texttt{ok} always indicates that your compiler works as it should
i.e. a feature is supported resp. a bug is \textit{not} present. On
the other hand \texttt{no} resp.  \texttt{unfortunately} indicate a
lack of support resp. presence of a bug.

Finally the current setup is summarized, system specific directories
for makefiles and libraries are created (if they did not exist before)
and a new include makefile is written into the makefile directory. If
there already exists a makefile for the current OS/compiler
combination, it is backuped and you should get a corresponding
message.

To compile the \cgal\ libraries go now to the \texttt{src} directory.
Then type \texttt{make -f makefile\_lib} to compile the \cgal\ object
library and \texttt{make -f makefile\_sharedlib} to compile the \cgal\ 
shared object library. If you want to make changes to the makefiles
first, see section \ref{sec:makefiles} for an explanation of the
makefile structure of \cgal.

When this is finished it would be a good idea to print and read the
`Getting Started with \cgal' document \texttt{getting\_started.ps}
that can be found in the \texttt{doc\_ps} directory.

\subsection{Setting up \stl}\label{sec:stl-setup}

The install script and the makefiles use the variable
\texttt{STL\_DIR} to indicate the STL that shall be used with \cgal.
This variable should point to the directory where STL header files are
stored. It can be set from command line with option
``\texttt{--STL\_DIR \textit{$<$dir$>$}}''. If you do not set
\texttt{STL\_DIR} this way, it is assumed that the compiler has
``builtin'' \stl\ and this is used. If your compiler does not have
builtin \stl, you \textit{have to} supply an \texttt{STL\_DIR} to get
things running.

\subsection{Setting up \leda\ support}\label{sec:leda-setup}

See also section \ref{sec:leda}. By default there is no support for
\leda, but you can change this easily by use of the command line
option ``\texttt{-leda}''. If \leda\ is installed in system
directories on your system, you should indicate this by setting the
flags ``{\tt --leda-sys-incl}'' resp.  ``\texttt{--leda-sys-lib}''. If
this is not the case, you have to supply the directories containing
the \leda\ header files (``\texttt{--LEDA\_INCL\_DIR}
\textit{$<$dir$>$}'') resp. the \leda\ libraries for your compiler
(``\texttt{--LEDA\_LIB\_DIR} {\it $<$dir$>$}'').

\subsection{Setting up support for GMP}\label{sec:gmp-setup}

By default there is no support for GMP, but you can change this easily
by use of the command line option ``\texttt{-gmp}''. If GMP is installed
in system directories on your system, you are already done now. If
this is not the case, you have to supply the directories containing
the GMP header files (``\texttt{--GMP\_INCL\_DIR} \textit{$<$dir$>$}'') and
the GMP library (``\texttt{--GMP\_LIB\_DIR} \textit{$<$dir$>$}'').

\subsection{Other Options}\label{sec:other-options}

There are some more less important features of the install script
we will summarize here. 

First of all you can get the version number of \texttt{cgal\_install}
with option ``\texttt{--version}''. Note that all other options are
ignored in this case.

Second there is an option ``\texttt{-os \textit{$<$compiler$>$}}''
where \textit{$<$compiler$>$} is your \CC\ compiler. This allows you
to determine your \cgal-OS description (see section
\ref{sec:os-compiler-id}). The compiler can either be given by an
absolute path like
\begin{verbatim}
./install_cgal -os /usr/local/gcc-2.8.1/sun/bin/g++
\end{verbatim}
or just by denoting its basename, as long as it is on your path:
\begin{verbatim}
./install_cgal -os CC
\end{verbatim}
The option is intended for testing purposes and automatic detection of
the correct include makefile (see also section \ref{sec:makefiles}).

Finally there exists an option ``\texttt{--verbose}'' that can be set
in interactive mode as well as in non-interactive mode. When set you
get a detailed summary of error messages occurring during \textit{any}
compiler test (determining \stl\ version etc.). Normally you only get
these messages, if a required test (such as the general \stl\ test)
fails, otherwise you are just informed, \textit{if} it succeeded or
not.  This option is not recommended for general use, but it can be
useful to check why a certain test fails that was expected to be
passed.

\section{Troubleshooting\label{sec:troubleshooting}}

In case you run into problems with \cgal, you should first have a look
at the \cgal\ homepage:
\begin{alltt}
\cgalhomepage
\end{alltt}
If you cannot find a solution to your problem there, please send email
to \texttt{cgal@cs.uu.nl}.

\section{Identifying OS and Compiler}\label{sec:os-compiler-id}

Since \cgal\ supports several different operating systems and
compilers, this is also reflected in the structure of the \cgal\ 
directory tree. Each OS/compiler combination has its own lib directory
under \texttt{\cgaldir/lib}) (and analogously its own include makefile
in \texttt{\cgaldir/make}) named as determined by the following
scheme.
\begin{center}
  \textit{$<$arch$>$\texttt{\_}$<$os$>$\texttt{-}$<$os-version$>$\texttt{\_}$<$comp$>${\tt
    -}$<$comp-version$>$}[\texttt{\_}LEDA]
\end{center}

\begin{description}
\item[$<$arch$>$] is the system architecture as defined by ``{\tt
    uname -p}'' or ``\texttt{uname -m}'',
\item[$<$os$>$] is the operating system as defined by ``\texttt{uname
    -s}'',
\item[$<$os-version$>$] is the operating system version as defined by
  ``\texttt{uname -r}'',
\item[$<$comp$>$] is the basename of the compiler executable (if it
  contains spaces, these are replaced by "-") \textit{and}
\item[$<$comp-version$>$] is the compiler's version number (which
  unfortunately can not be derived in a uniform manner, since it is
  quite compiler specific).
\end{description}

The suffix \texttt{\_}LEDA is appended to indicate \leda\ support. This
distinction is necessary, because the object libraries look (at least
possibly) different.

We call the resulting string \cgal-OS description. 

Examples are \texttt{mips\_IRIX-6.2\_CC-7.2} or {\tt
  sparc\_SunOS-5.5\_g++-2.8.1\_LEDA}. You can use the install
script to get your \cgal-OS description, see section
\ref{sec:other-options}.

\section{The \cgal\ makefile structure \label{sec:makefiles}}

The \cgal\ distribution contains the following makefiles:
\begin{itemize}
\item \texttt{\cgaldir/src/makefile\_lib} for compiling the \cgal\ 
  object library \texttt{libCGAL.a},
  
\item \texttt{\cgaldir/src/makefile\_sharedlib} for compiling the
  \cgal\ shared object library \texttt{libCGAL.so},
  
\item \texttt{\cgaldir/src/makefile\_geomview} for compiling a library
  for geomview support,
  
\item \texttt{\cgaldir/examples/makefile} as sample makefile
  \textit{and}
  
\item \texttt{\cgaldir/examples/*/makefile} for compiling the \cgal\ 
  example programs.
\end{itemize}

All these makefiles are generic: they can be used for more than one
compiler.  To achieve this, the first section of each makefile
contains an include statement that looks as follows:

\begin{verbatim}
      CGAL_MAKEFILE = /users/jannes/CGAL-1.1/make/makefile_<CGAL-OS description>
      include $(CGAL_MAKEFILE)
\end{verbatim}

The file \texttt{CGAL\_MAKEFILE} is an include file with platform
dependent makefile settings. The abbreviation \texttt{<CGAL-OS
  description>} (see section \ref{sec:os-compiler-id} for details) is
used to identify the operating system and compiler for which the
settings hold. For example, the file
\texttt{makefile\_mips\_IRIX64-6.2\_CC-n32-7.20} contains makefile
settings for the IRIX 6.2 operating system and the \mipsprocc\ 7.2
compiler.  These include files are automatically generated by the
\texttt{install\_cgal} script and they are all located in the
\texttt{\cgaldir/make} directory.  For convenience, the
\texttt{install\_cgal} script will substitute the include makefile
that was generated most recently.

If you want to compile an application or an object library with a
different compiler, the only thing you need to do is to substitute
another include makefile for the \texttt{CGAL\_MAKEFILE} variable. An
alternative way to do this is to create an environment variable
\texttt{CGAL\_MAKEFILE}. To pass the value of the environment variable
to the makefile you can either comment out the \texttt{CGAL\_MAKEFILE}
line in the makefile or use an appropriate command line option for
the make utility.
A comfortable way to set \texttt{CGAL\_MAKEFILE} is by using
\texttt{install\_cgal~-os} (see section \ref{sec:other-options}).
E.g. if your compiler is \texttt{g++}, you would type
\begin{alltt}
CGAL_MAKEFILE=`\yourcgaldir/install_cgal -os g++`
\end{alltt}
in bourne shell resp.
\begin{alltt}
setenv CGAL_MAKEFILE `\yourcgaldir/install_cgal -os g++`
\end{alltt}
in csh derivatives.

All makefiles contain sections with compiler and linker
flags.  You can add your own flags here. For example, you might want
to add the flag \texttt{-DCGAL\_NO\_PRECONDITIONS} to turn off
precondition checking. The flags \texttt{\$(CGAL\_CXXFLAGS)} and
\texttt{\$(CGAL\_LDFLAGS)} should never be removed.

The default extension for \cgal\ source files is \texttt{.C}.  The
last section of the makefiles contains a suffix rule that tells the
compiler how to create a \texttt{.o}-file from a \texttt{.C}-file.  If
you want to use the default rule that is defined by the make utility,
you may want to remove this suffix rule.  However, note that this may
have consequences for the makefile variables \texttt{CGAL\_CXX} and
\texttt{CXXFLAGS}.

\section{Compiling a \cgal\ application}
The directory \texttt{CGAL-1.1/examples} contains a small program
(example.C) and a sample makefile with some comments. The
\texttt{CGAL\_MAKEFILE} variable in this makefile (see section
\ref{sec:makefiles}) is automatically substituted by the
\texttt{install\_cgal} script and equals the most recently generated
include makefile in the \texttt{\cgaldir/make} directory. After the
installation of \cgal\ this sample makefile is ready for use. Just
type '\texttt{make example}' to compile the program
\texttt{example.C}.  You may use this makefile as a blueprint for your
own makefiles.

Furthermore the directories \texttt{\cgaldir/examples} and
\texttt{\cgaldir/demo} contain many subdirectories with non-graphical
and graphical example programs.  In all these directories you will
find a makefile that is ready for use.

\section{Using \cgal\ and \leda \label{sec:leda}}
This section describes how to use \cgal\ and \leda\ simultaneously.

\subsection{Support for \leda}
\cgal\ supports \leda\ in the following ways:

\begin{enumerate}
\item There are support functions defined for the \leda\ number types
  \texttt{big\_float}, \texttt{integer}, \texttt{rational} and
  \texttt{real} (see the files \texttt{<CGAL/leda\_*>}).

\item
\cgal\ defines the following \leda-related compiler flags:
\begin{itemize}
\item When \leda\ is used, the flags \texttt{CGAL\_USE\_LEDA} and
  \texttt{LEDA\_PREFIX} will be set.
\item When \leda\ is used, the \leda\ memory management (\leda\ handles) will be
used for geometric primitives in \cgal. This can be turned of by setting the flag
\texttt{CGAL\_NO\_LEDA\_HANDLE}. In that case \cgal\ memory management will
be used (see \texttt{<CGAL/Handle.h>}).
\end{itemize}
\end{enumerate}

The include makefiles in the \texttt{\cgaldir/make} directory
corresponding to \leda\ can be recognized by the suffix
``\texttt{\_LEDA}''.

\subsection{\leda\ and \stl\ conflicts}

If you are using an old compiler or an old version of \stl, the
combination of \leda\ and \stl\ may give some problems. In order to
avoid them, it is highly recommended to use the latest \leda\ 
release\footnote{At the moment this is \leda\ 3.6.1.}, since this is
what we test \cgal\ with.

\subsubsection{Definition of type bool}

If a compiler does not support the keyword bool, libraries like \leda\ 
and \stl\ provide their own definition of bool. Unfortunately the
\leda\ definition of bool is incompatible with the definition in most
\stl\ implementations. One of them needs to be changed. The easiest
solution is to modify \stl\ as follows:

\begin{enumerate}
\item locate the definition of bool; most of the time it is defined as

\begin{verbatim}
   typedef int bool;
\end{verbatim}

in a file called bool.h or stl\_config.h

\item replace this definition by the following:

\begin{verbatim}
#ifdef CGAL_USE_LEDA
#include <LEDA/bool.h>
#else
   typedef int bool;
#endif
\end{verbatim}
\end{enumerate}

The flag \texttt{CGAL\_USE\_LEDA} ensures that you can continue to use \stl\ wwithout
depending on \leda.

REMARK: If you do not like to modify centrally managed \stl\ files, it
is possible to fake it. This can be done by making changes to a copy
of the \stl\ file bool.h or stl\_config.h. After this you have to make
sure that this copy is found earlier on the include path than the
original file.  It is also possible to modify \leda\ instead of \stl.
However, then you have to modify and recompile (if you have the source
code) \leda.

\subsubsection{'red' and 'black' conflict}

When compiling the graphical demo programs that use \leda\ windows
with \leda\ versions prior to 3.6.1, you may encounter a name clash.
This is because \leda\ defines the colors 'red' and 'black' in the
file \texttt{<LEDA/impl/x\_window.h>}, whereas some \stl\ 
implementations use 'red' and 'black' in the implementation of trees.
One of the libraries needs to be modified. In this case the easiest
solution seems to prefix the \leda\ color constants with something
like \texttt{leda\_}.

\section{The Standard Template Library \label{sec:stl}}

For using the \cgal\ library, the Standard Template Library is
required. Recent compilers provide their own implementation of \stl.
There are also free implementations. Most notable are the
implementation from Silicon Graphics (\sgistlpage, also contains
documentation and links to other \stl\ resources) and a port of this
SGI version to a lot of other compilers (\stlportpage).

The \texttt{install\_cgal} script tries to figure out which \stl\ is
used.  This is necessary for the circulator package, and also to work
around some bugs. In case the right version is not recognized, you
could try to use the non-interactive mode of the
\texttt{install\_cgal} script (see section \ref{sec:non-interactive}).
The compiler flag \verb~CGAL_STL_VERSION~ can take the following
values:

\begin{center}
  \renewcommand{\arraystretch}{1.3}
  \begin{tabular}{|l|l|} \hline
    CGAL\_STL\_GCC                & \stl\ version that comes with gcc-2.7.2 \\ \hline
    CGAL\_STL\_HP                 & Free HP version \\ \hline
    CGAL\_STL\_SGI\_3\_0          & SGI STL 3.0 or newer \\ \hline
    CGAL\_STL\_SGI\_WWW           & SGI STL prior to 3.0 \\ \hline
    CGAL\_STL\_SGI\_WWW\_OLD      & SGI STL for older compilers \\ \hline
    CGAL\_STL\_SGI\_CC            & SGI implementation that comes with their\\ 
    & 7.0/ 7.1 compilers \\ \hline
    CGAL\_STL\_UNKNOWN            & unknown \stl\ version (this will probably not work) \\ \hline
  \end{tabular}
\end{center}

Most \stl\ libraries consist of only header files. An exception is
HP's \stl, which also has source files \verb~random.cpp~ and
\verb~tempbuf.cpp~.  The \cgal\ makefiles do \textit{not} take care of
this. You have to compile these files yourself and add them to the
custom linker flags during the installation of CGAL, or enter them
manually in the makefiles.

\section{Upgrading a \cgal\ 1.0 installation}
In case you have \cgal\ 1.0 installed on your system, you might like
to reuse your configuration files. Simply use the following commands
to put them into the right place:
\begin{verbatim}
cp -r <OLD_CGAL_DIR>/config/install <NEW_CGAL_DIR>/config
\end{verbatim}
where \texttt{<OLD\_CGAL\_DIR>} is your \cgal\ 1.0 root directory
(e.g.  \texttt{/pub/local/CGAL-1.0}) and\\
\texttt{<NEW\_CGAL\_DIR>} is the root directory of your new \cgal\ 
installation (e.g. {\tt /pub/local/CGAL-1.1}).

It is \textbf{strongly} recommended to rerun all tests using the
option \texttt{A} from the main menu.

\section{Compiler workarounds}
In \cgal\ a number of compiler flags is defined, all of them start
with the prefix \texttt{CGAL\_CFG}. These flags are used to work
around compiler bugs and limitations.  For example, the flag
\texttt{CGAL\_CFG\_NO\_BUILTIN\_BOOL} denotes that the compiler does
not know the keyword bool. If this flag is defined, the type bool will
be borrowed from the \stl\ library (currently by including the file
\texttt{<pair.h>}).

For each compiler a file \texttt{<CGAL/compiler\_config.h>} is
defined, with the correct settings of all flags. This file is
generated automatically by the \texttt{install\_cgal} script. For this
the test programs in the directory \texttt{\cgaldir/config/testfiles}
are used. The file \texttt{<CGAL/compiler\_config.h>} and the test
programs contain a description of the problem, so in case of trouble
with a \texttt{CGAL\_CFG} flag it is a good idea to take a look at it.

The file \texttt{<CGAL/config.h>} manages all configuration problems
of the compiler.  This file includes the file
\texttt{CGAL/compiler\_config.h}.  It is therefore important that the
file \texttt{<CGAL/config.h>} is always included before any other
\cgal\ source file that depends on workaround flags. In most cases you
do not have to do anything special for this, because many CGAL files
already take care of including \texttt{<CGAL/config.h>}. Nevertheless
it would be a good idea to always start your \cgal\ programs with
including \texttt{<CGAL/config.h>} (or \texttt{<CGAL/basic.h>}, which
contains some more basic \cgal\ definitions).

