%%
%%  *** CGAL Installation Guide ***
%%
%% file:     installation.tex
%%
%% authors:  Michael Hoffmann and Wieger Wesselink
%%
%% $Revision$ $Date$
%%

%% macro for GNU
\newcommand{\gnu}{\textsc{Gnu}}

%% macro for GNU
\newcommand{\MSInst}{``Windows-specific Installation''}

%% macro for g++
\newcommand{\Gcc}[1]{\gnu~\texttt{g++}~{\rm #1}}

%% macro for egcs
\newcommand{\egcs}[1]{\texttt{Egcs}~{\rm #1}}

%% macro for SUNPRO CC
\newcommand{\sunprocc}[1]{\textsc{Sunpro}~\texttt{CC}~{\rm #1}}

%% macro for SGI CC
\newcommand{\mipsprocc}{\textsc{SGI}~Mips(Pro)~\texttt{CC}}

%% macro for Borland C++
\newcommand{\bcc}[1]{\textsc{Borland}~\texttt{C++}~{\rm #1}}

%% macro for Microsoft Visual C++
\newcommand{\msvc}[1]{\textsc{MS}~Visual~\texttt{C++}~{\rm #1}}

%% macro for Microsoft Windows
\newcommand{\mswin}{\textsc{MS}~Windows}

%% macro for CGAL release number
%% \newcommand{\cgalrelease}{2.1}

%% macro for CGAL directory
\newcommand{\cgaldir}{\texttt{CGAL-\cgalrelease}}

%% macro for YOUR CGAL directory :)
\newcommand{\yourcgaldir}{$<$\textit{insert your \cgaldir\ dir}$>$}

%% macro for CGAL install config directory
\newcommand{\cgalinstconfdir}{\cgaldir\texttt{/config/install}}

%% macros for WWW pages
\newcommand{\cgalhomepage}{\path'http://www.cs.uu.nl/CGAL'}
\newcommand{\compilerpage}{\path'http://www.cs.uu.nl/CGAL/Compilers/'}
\newcommand{\faqpage}{\path'http://www.cs.uu.nl/CGAL/FAQ/'}
%%\newcommand{\hpstlpage}{\path'http://www.cs.rpi.edu/~musser/stl.html'}
\newcommand{\sgistlpage}{\path'http://www.sgi.com/Technology/STL/'}
\newcommand{\stlportpage}{\path'http://www.metabyte.com/~fbp/stl/'}
\newcommand{\ledapage}{\path'http://www.mpi-sb.mpg.de/LEDA'}
\newcommand{\gmppage}{\path'http://www.gnu.org/software/gmp'}
\newcommand{\clnpage}{\path'http://clisp.cons.org/~haible/packages-cln.html'}

\lcHtml{\chapter{Installation}}
\section{Introduction}

\cgal\ is a \textit{Computational Geometry Algorithms Library\/} written
in \CC, which is developed in an {\sc Esprit Ltr} project. The goal is
to make the large body of geometric algorithms developed in the field
of computational geometry available for industrial application.

This document describes how to install \cgal\ on your system.  Besides
that, you will find some information about the makefile structure of
\cgal\ and the support for simultaneously using \cgal\ and \leda, the
Library of Efficient Datatypes and Algorithms\footnote{\ledapage}
Multiple Precision library GMP\footnote{\gmppage} and/or the Class
Library for Numbers (CLN)\footnote{\clnpage}.

\section{Prerequisites}\label{sec:prerequisites}

In order to build the \cgal\ libraries you need a \CC\ compiler.
Currently only a small number of recent compilers on Unix platforms
and \mswin\ are supported. The reason is that most compilers do not
conform to the ISO 14882 standard for \lcTex{\CC\footnote{see e.g.
    \path~http://reality.sgi.com/austern/std-c++/faq.html\#PartB~ for
    information}} \lcHtml{\CC\footnote{see e.g.
    \path~http://reality.sgi.com/austern/std-c++/faq.html#PartB~ for
    information}} and some of them have so many limitations/bugs that
we could not work around all of them.

On \mswin, one has two options for installing: one is specifically
targeted for this OS and is described in \MSInst; the other is using
Cygwin\footnote{see \path~http://sourceware.cygnus.com/cygwin~} and
the generic installation procedure for Unix-like environments
described here. If you are going to install \cgal\ using Cygwin,
please read section \ref{sec:cygwin} first.

More precisely, \cgaldir\ supports the following compilers/operating
systems:

\begin{center}
  \renewcommand{\arraystretch}{1.3}
  \gdef\lcTabularBorder{2}
  \begin{tabular}{|l|l|} \hline
    \textbf{compiler}        & \textbf{operating system}\\\hline\hline
    \mipsprocc\ 7.3 (n32)    & IRIX 6.5\\\hline
    \Gcc{2.95} (and later)   & 
      IRIX 6.5 / Solaris 2.6 / Linux 2.x / 
      \mswin\ 95/98/NT4\footnotemark[6]\\\hline
    \egcs{1.1.2} (and later) & 
      IRIX 6.5 / Solaris 2.6 / Linux 2.x / 
      \mswin\ 95/98/NT4\footnotemark[6]\\\hline
    \msvc{6.0}               & \mswin\ 95/98/NT4\\\hline
  \end{tabular}
\end{center}
\footnotetext[6]{with Cygwin $\beta$20.1}

There are plans to provide full support for \bcc{5.4} in the future,
but there is no such support at the moment. The \sunprocc{4.2}
compiler is not supported anymore, please stay with \cgal-1.2 if you
have to use it.  \Gcc{2.8.1} is not supported anymore, please stay
with \cgal-2.0 if you have to use it.  Support for \sunprocc{5} will
be reconsidered as soon as an acceptable degree of
standard-conformance is reached.

\section{Getting \cgal}

The \cgal\ library can be downloaded in two different ways: using ftp
or using WWW.  If you have a WWW connection, the easiest way to
download \cgal\ is via the \cgal\ homepage:
\begin{quote}
      \cgalhomepage
\end{quote}
and go to the `Software' section.

Just follow the instructions on this page to obtain your copy of the
library. The \cgal\ library can also be downloaded using FTP. The
library can be found at the following location:

\begin{verbatim}
      ftp.cs.uu.nl
\end{verbatim}

in the directory \texttt{/pub/CGAL}. This directory contains release
\cgalrelease\ of the \cgal\ library. There is also a \texttt{README}
file that contains descriptions of the files in this directory. An
example of an FTP-session is given below.

\begin{alltt}
      $ ftp ftp.cs.uu.nl 
      Name (ftp.cs.uu.nl:<your username>): anonymous
      Password: <type your email address here>
      ftp> cd pub/CGAL
      ftp> get README
      ftp> binary
      ftp> get \cgaldir.tar.gz
      ftp> quit
\end{alltt}

After you have downloaded the file containing the \cgal\ library, you
have to decompress it. For the zipfile use the command

\begin{verbatim}
      unzip <filename>.zip
\end{verbatim}

and for the gzipped file use the commands

\begin{verbatim}
      gunzip <filename>.tar.gz
      tar xvf <filename>.tar
\end{verbatim}

%% N.B. On a PC you should use an unzip utility that can deal with long
%% filenames (like WinZip or InfoZip)!

In both cases the directory \cgaldir\ will be created. This directory
contains the following subdirectories:

\begin{center}
  \renewcommand{\arraystretch}{1.3}
  \gdef\lcTabularBorder{2}
  \begin{tabular}{|l|l|} \hline
    \textbf{directory} & \textbf{contents}\\\hline\hline
    \texttt{auxiliary} & packages that can optionally be used with \cgal\\\hline
    \texttt{config}    & configuration files for install script\\\hline
    \texttt{demo}      & demo programs (some of them need \leda, geomview, 
                         OpenGL or other third-party products)\\\hline
    \texttt{doc\_html} & documentation (HTML)\\\hline
    \texttt{doc\_pdf}  & documentation (PDF)\\\hline
    \texttt{doc\_ps}   & documentation (Postscript)\\\hline
    \texttt{examples}  & example programs\\\hline
    \texttt{include}   & header files\\\hline
    \texttt{lib}       & (shared) object libraries\\\hline
    \texttt{make}      & files with platform dependent makefile settings\\\hline
    \texttt{scripts}   & some useful scripts (e.g. for creating makefiles)\\\hline
    \texttt{src}       & source files\\\hline
  \end{tabular}
\end{center}

\section{Installing \cgal}

The directory \cgaldir\ contains a Bourne shell script called
\texttt{install\_cgal}. The script can be run in two modes: a
menu-driven interactive mode and a non-interactive mode.  Normally you
should use the interactive mode, but in case you run into problems
with it or do not like it for some reason, you can still use the
non-interactive mode.

We first describe a sample installation in section
\ref{sec:sample-inst}. This provides you with an overview on how the
interactive installation works. If you want more detailed information
about specific menus and their options, take a look at section
\ref{sec:interactive-mode}. Finally, for the non-interactive mode
refer to section \ref{sec:non-interactive}.

If you want to use \leda\ together with \cgal, have a look at
section~\ref{sec:leda}.

\section{A Sample Installation}\label{sec:sample-inst}

In this section we sketch an example installation on a \textsc{Sun}
running Solaris 2.6 with the \Gcc{2.95} compiler. For a complete
description of the different menus and their options refer to section
\ref{sec:interactive-mode}.

\subsection{Starting the script}

Go to the \cgaldir\ directory and enter the command
\begin{verbatim}
      ./install_cgal -i
\end{verbatim}

You get a message indicating the \cgal\ version you are going to
install and that you are running the interactive mode. Then it takes
some time while the script locates a number of utility programs. You
will not get informed about this\footnote{If you are that curious what
  happens exactly, have a look at the file
  \texttt{\cgaldir/install.log}.}, but see some dots written to the
screen indicating progress.

{\ccTexHtml{\scriptsize}{}
\begin{alltt}
--------------------------------------------------------
  This is the install script for CGAL \cgalrelease
--------------------------------------------------------

starting interactive mode - one moment, please
......

  Choosing compiler GNU gcc-2.95
  Testing for STL ... ok.
  ...
  <tests for several compiler features>
  ...
  Saving current setup ... done.
.
\end{alltt}}
  
If there is any compiler installed on your system and accessible
through your \texttt{PATH} environment variable that is supported by
\cgal, one of these compilers is chosen and a number of tests are done
to check whether your compiler supports certain language constructs
respectively has specific bugs. There are quite a number of these
tests, so this step may take some time. For each test you should get a
message what particularly is tested at the moment and what the result
is. If there is more than one compiler installed on your system (and
supported by \cgal), you may later choose to use a different compiler
from the compiler menu.

A menu similar to the following will appear on your screen.

{\ccTexHtml{\scriptsize}{} \label{pic:main-menu}
\begin{alltt}
  ****************************************************************
  **              CGAL 2.1 Installation Main Menu               **
  **              -------------------------------               **
  **                                                            **
  **   OS:                  sparc_SunOS-5.6                     **
  **   Compiler:            GNU gcc-2.95                        **
  **   LEDA:                not supported.                      **
  **   GMP:                 not supported.                      **
  **   CLN:                 not supported.                      **
  **                                                            **
  **   Compiler is supported by CGAL.                           **
  **   The setup has been tested ok.                            **
  **                                                            **
  **   There are no libs for this os/compiler.                  **
  **                                                            **
  **   <C>  Compiler Menu                                       **
  **   <L>  LEDA Menu                                           **
  **   <G>  GMP Menu                                            **
  **   <M>  CLN Menu                                            **
  **   <T>  Test (and save) setup                               **
  **   <A>  Run all setup tests (no cache)                      **
  **                                                            **
  **   <B>  Build CGAL Libraries                                **
  **                                                            **
  **   <Q>  Back to OS                                          **
  **                                                            **
  **   Your Choice:                                             **
  **                                                            **
  ****************************************************************
\end{alltt}}

The first lines below the headline contain some kind of status report:
current OS and compiler, are \leda, GMP and CLN supported etc.
Moreover you can see that the current setup has been tested and that
there do not exist \cgal\ libraries for this OS/compiler combination
in the \cgal\ lib directory by now.

The fact that your setup has been tested implies that the current
settings have been saved to a file in the directory \cgalinstconfdir.
Thus, if you run the install script a second time for this
OS/compiler, you will not have to go through the whole config-/test
cycle again, but the configuration will be retrieved from the
corresponding config file instead.

\subsection{Building the \cgal\ libraries}\label{sec:build-the-libs}

We are now ready to build the \cgal\ libraries.  Just type ``{\tt b}''
to start compilation. Building consists of three steps:
\begin{enumerate}
\item writing the include makefile,
\item compiling the static libraries \textit{and}
\item compiling the shared libraries.
\end{enumerate}
The include makefile encapsulates the OS-- and compiler-specific
settings and should be included (hence the name) in all makefiles that
compile \cgal\ applications. If everything went ok, the output should
look as follows (if not, you should have a look at the error messages
from compiler or linker).

{\ccTexHtml{\scriptsize}{}
\begin{verbatim}
  ****************************************************************
  **                                                            **
  **                     Compiling CGAL 2.1                     **
  **                     ------------------                     **
  **                                                            **
  ****************************************************************

  OS:           sparc_SunOS-5.6
  COMPILER:     GNU gcc-2.95
  LEDA:         not supported
  GMP:          not supported
  CLN:          not supported

  Generating Makefiles ... done.
  Building CGAL_lib ... done.
  Building CGAL_sharedlib ... done.

  ****************************************************************
  **             Please press <ENTER> to continue.              **
  ****************************************************************
\end{verbatim}}
  
That's all, it's done. Press ``\texttt{<ENTER>}'' to return to the
main menu and proceed by installing for a different compiler (go to
the compiler menu and choose ``\texttt{c}'' to get a list of
supported compilers detected on your system) or with \leda, GMP or
CLN support (go to the \leda, GMP, CLN menu resp.) or simply quit
the install script by typing ``\texttt{q}''. When leaving the
script, you get a list of successful builds during the session.

Now it would be a good idea to print and read the document ``Getting
Started with \cgal'' that can be found in various formats in the
\texttt{doc\_html}, \texttt{doc\_pdf} and \texttt{doc\_ps}
directories.

\section{The interactive mode}\label{sec:interactive-mode}

To run the install script in the interactive mode, go to the \cgaldir\ 
directory and enter the command
\begin{verbatim}
      ./install_cgal -i
\end{verbatim}

After initialization during which certain utility programs are located
and your system is searched for compilers supported by \cgal, you get
into the \cgal\ installation \textit{main menu} (see page
\pageref{pic:main-menu} for a picture).

From the main menu you can reach a number of different sub-menus, of
which the most important maybe is the \textit{compiler menu}. This is
where you can choose the compiler you want to work with and set custom
compiler or linker options.  The compiler menu is described in section
\ref{sec:compiler-menu}.

If you want to use \leda, GMP or CLN with \cgal, you will have to go
to the \textit{leda menu}, \textit{gmp menu} or \textit{cln menu}.
These are described in section \ref{sec:leda-menu},
\ref{sec:gmp-menu}, \ref{sec:cln-menu}, respectively.

Finally you can build the \cgal\ libraries by typing \texttt{b}.
However, it is recommended to run the \textit{setup test} which is
available in all menus as option \texttt{t} before. The setup test
includes an \stl\ test, a \leda\ test, a GMP test and a CLN test. But
not all tests are performed always; e.g. the \leda\ test is only done,
if you enabled \leda\ support. The install script keeps track of the
tests passed and only tests again, if you change the setup in a way
that might affect the test result. If you want to redo \textit{all}
tests, you have to choose option ``\texttt{a}'' from the main menu.
This also retests for \leda/GMP/CLN installation in system directories
which otherwise is only done the first time you enable \leda/GMP/CLN
support for an OS/compiler combination.

\subsection{Files created during installation}

The install script stores all relevant settings for an OS/compiler
combination in the directory
\begin{center}
  \texttt{\cgalinstconfdir/$<$\textit{CGAL-OS-description}$>$}
\end{center}
where $<$\textit{CGAL-OS-description}$>$ identifies your OS/compiler
combination in a way specified in section \ref{sec:os-compiler-id}.
\footnote{Note that these files are only OS/compiler specific, i.e.
  there are no different files for with and without \leda\ support.}
This saves you typing everything again, if you upgrade \cgal\ or
another package that makes recompiling the \cgal\ libraries necessary.

Besides the config files, \texttt{install\_cgal} uses several
temporary files during interactive installation. Most of them are
removed after use, but some are not, since it might be helpful to keep
some information about the last run.  You can keep or delete them as
you like, as they are not needed anymore once the script terminated.
A list of these files (all are plain ASCII and reside in \cgaldir)
follows.
\begin{center}
  \renewcommand{\arraystretch}{1.3}
  \gdef\lcTabularBorder{2}
  \begin{tabular}{|l|l|} \hline
    \textbf{filename} & \textbf{content}\\\hline\hline
    \texttt{install.log} & detailed overall protocol\\\hline
    \texttt{install.completed} & list of systems for which \cgal\ libraries 
    have been built\\\hline
    \texttt{compile.log} & output of the last compiler call\\\hline
  \end{tabular}
\end{center}

\subsection{The Compiler Menu}\label{sec:compiler-menu}

Here is the place to set the compiler specific options, such as the
compiler to use (if more than one has been detected) and custom
compiler or linker flags.\bigskip

\fbox{
\begin{minipage}{.95\linewidth}
\vspace{2pt}
\centerline{{\large{\bf Compiler Menu}}}
\vspace{2pt}
\begin{description}
\item[$<$C$>$] Choose the compiler to be used from the list of
  detected compilers.
\item[$<$F$>$] Set custom compiler flags. These are the first flags
  given to the compiler in every call. Under normal circumstances
  there should be no need to set any such flag.
\item[$<$L$>$] Set custom linker flags. These are the first flags
  given to the linker in every call. Under normal circumstances
  there should be no need to set any such flag.
\end{description}
\vspace{2pt}
\end{minipage}
\hfill}

\subsection{The \leda\ Menu}\label{sec:leda-menu}

This is the place to set \leda\ specific options, if you plan to use
\leda\ together with \cgal\ (see also section \ref{sec:leda}). In
order to enable \leda\ support in \cgal, \leda\ has to be installed on
your system.

If \leda\ support is enabled the first time, the script tests whether
\leda\ is installed in standard system directories. If this test does
not succeed, you have to supply directories containing the \leda\ 
header files (\texttt{LEDA\_INCL\_DIR}) and \leda\ libraries ({\tt
  LEDA\_LIB\_DIR}).  Even if the tests are passed, you still have the
option to set these directories differently.\bigskip

\fbox{
\begin{minipage}{.95\linewidth}
\vspace{2pt}
\centerline{{\large{\bf \leda\ Menu}}}
\vspace{2pt}
\begin{description}
\item[$<$E$>$] Enable/Disable \leda\ support in \cgal.
\item[$<$I$>$] (\textit{present if \leda\ support is enabled)} Set the
  include directory for \leda .
\item[$<$J$>$] (\textit{present if \leda\ support is enabled, \leda\ headers
    have been found in a system include directory and {\tt
      LEDA\_INCL\_DIR} has been set)} Use \leda\ header from system
  include directory.
\item[$<$L$>$] (\textit{present if \leda\ support is enabled)} Set the
  directory containing the \leda\ libraries.
\item[$<$M$>$] (\textit{present if \leda\ support is enabled, \leda\ libs
    have been found in a system lib directory and \texttt{LEDA\_LIB\_DIR}
    has been set)} Use \leda\ libraries from system lib directory.
\end{description}
\vspace{2pt}
\end{minipage}
\hfill}

\subsection{The GMP Menu}\label{sec:gmp-menu}

This menu is to set GMP (\gnu\ Muptiple Precision Library) specific
options, if you plan to use GMP together with \cgal . In the {\tt
  auxiliary} directory you can find a GMP distribution, if you do not
already have it installed on your system. This menu contains an option
to install GMP in your \cgal\ directory tree\footnote{This option is
  not present on \mswin.}, but of course you can also install it
independently from \cgal .

If GMP support is enabled the first time, the script tests whether GMP
is installed in standard system directories or in the \cgal\ tree. If
this test does not succeed, you have to supply directories containing
the GMP header files (\texttt{GMP\_INCL\_DIR}) and GMP libraries ({\tt
  GMP\_LIB\_DIR}).  Even if the tests are passed, you still have the
option to set these directories differently.\bigskip

\fbox{
\begin{minipage}{.95\linewidth}
\vspace{2pt}
\centerline{{\large{\bf GMP Menu}}}
\vspace{2pt}
\begin{description}
\item[$<$C$>$] Install the GMP distribution shipped with \cgal\ in the
  \cgal\ directory tree. 
\item[$<$G$>$] Enable/Disable GMP support in \cgal.
\item[$<$I$>$] (\textit{present if GMP support is enabled)} Set the
  include directory for GMP.
\item[$<$L$>$] (\textit{present if GMP support is enabled)} Set the
  directory containing the GMP libraries.
\item[$<$M$>$] (\textit{present if GMP support is enabled, there is a GMP
    installation in system directories or in the \cgal\ tree and {\tt
      GMP\_INCL\_DIR} or \texttt{GMP\_LIB\_DIR} have been set)} Use GMP
  installation from system directories / \cgal\ tree.
\end{description}
\vspace{2pt}
\end{minipage}
\hfill}

\subsection{The CLN Menu}\label{sec:cln-menu}

This menu is to set CLN (Class Library for Numbers) specific
options, if you plan to use CLN together with \cgal . Note that in
order to enable CLN support in \cgal, CLN has to be installed on your
system first.  Unlike for GMP, there is no option to install CLN from
the \cgal\ installation script. For information on CLN, please refer
to
\begin{quote}
  \clnpage
\end{quote}

If CLN support is enabled the first time, the script tests whether CLN
is installed in standard system directories. If this test does not
succeed, you have to supply directories containing the CLN header
files (\texttt{CLN\_INCL\_DIR}) and CLN libraries ({\tt
  CLN\_LIB\_DIR}). Even if the tests are passed, you still have the
option to set these directories differently.\bigskip

\fbox{
\begin{minipage}{.95\linewidth}
\vspace{2pt}
\centerline{{\large{\bf CLN Menu}}}
\vspace{2pt}
\begin{description}
\item[$<$G$>$] Enable/Disable CLN support in \cgal.
\item[$<$I$>$] (\textit{present if CLN support is enabled)} Set the
  include directory for CLN.
\item[$<$L$>$] (\textit{present if CLN support is enabled)} Set the
  directory containing the CLN libraries.
\item[$<$M$>$] (\textit{present if CLN support is enabled, there is a
    CLN installation in system directories and {\tt CLN\_INCL\_DIR} or
    \texttt{CLN\_LIB\_DIR} have been set)} Use CLN installation from
  system directories.
\end{description}
\vspace{2pt}
\end{minipage}
\hfill}

\section{The non-interactive mode}\label{sec:non-interactive}

To run the install script in the non-interactive mode, go to the
\cgaldir\ directory and enter the command
\begin{verbatim}
      ./install_cgal -ni <compiler>
\end{verbatim}
where \texttt{<compiler>} is the C++ compiler executable.\\
You can either specify a full path, e.g. \texttt{/usr/local/bin/g++},
or just the basename, e.g. \texttt{g++}, which means the script
searches your \texttt{PATH} for the compiler location. If your
compiler call contains whitespaces it has to be quoted, e.g.
\texttt{./install\_cgal -ni "CC -n32"}.  The options given this way
become part of your \cgal-OS description (see section
\ref{sec:os-compiler-id}) which is useful e.g. to distinguish between
different compilers using the same frontend (\mipsprocc\ on
\texttt{IRIX6}).

There are a number of additional command line options to customize
your \cgal\ setup which are discussed below. You should read the
corresponding paragraphs before you continue, especially if one or
more of the following conditions apply to you:
\begin{itemize}
\item you want to use \leda\ together with \cgal\ (section
  \ref{sec:leda-setup}),
\item you want to use \gnu\ GMP together with \cgal\ (section
  \ref{sec:gmp-setup}) \textit{or}
\item you want to use CLN together with \cgal\ (section
  \ref{sec:cln-setup}).
\end{itemize}

Once you started the script, it should give you a message indicating
the \cgal\ version you are going to install and that you are running
the non-interactive mode. Then it proceeds by locating some utility
programs, determining your OS and compiler version and displaying the
settings you gave via command line. Your compiler is also checked for
a number of bugs resp. support of certain language features; a message
\texttt{ok} always indicates that your compiler works as it should
i.e. a feature is supported resp. a bug is \textit{not} present. On
the other hand \texttt{no} resp.  \texttt{unfortunately} indicate a
lack of support resp. presence of a bug.

Finally the current setup is summarized, system specific directories
for makefiles and libraries are created (if they did not exist before)
and a new include makefile is written into the makefile directory. If
there already exists a makefile for the current OS/compiler
combination, it is backed up and you should get a corresponding
message.

To compile the \cgal\ libraries go now to the \texttt{src} directory.
Then type \texttt{make -f makefile\_lib} to compile the \cgal\ object
library and \texttt{make -f makefile\_sharedlib} to compile the \cgal\ 
shared object library. If you want to make changes to the makefiles
first, see section \ref{sec:makefiles} for an explanation of the
makefile structure of \cgal.

When this is finished it would be a good idea to print and read the
`Getting Started with \cgal' document \texttt{getting\_started.ps}
that can be found in the \texttt{doc\_ps} directory.

\subsection{Setting up \leda\ support}\label{sec:leda-setup}

See also section \ref{sec:leda}. By default there is no support for
\leda, but you can change this easily by use of the command line
option ``\texttt{-leda}''. If \leda\ is installed in system
directories on your system, you should indicate this by setting the
flags ``{\tt --leda-sys-incl}'' resp.  ``\texttt{--leda-sys-lib}''. If
this is not the case, you have to supply the directories containing
the \leda\ header files (``\texttt{--LEDA\_INCL\_DIR}
\textit{$<$dir$>$}'') resp. the \leda\ libraries for your compiler
(``\texttt{--LEDA\_LIB\_DIR} {\it $<$dir$>$}'').

\subsection{Setting up support for GMP}\label{sec:gmp-setup}

By default there is no support for GMP, but you can change this easily
by use of the command line option ``\texttt{-gmp}''. If GMP is installed
in system directories on your system, you are already done now. If
this is not the case, you have to supply the directories containing
the GMP header files (``\texttt{--GMP\_INCL\_DIR} \textit{$<$dir$>$}'') and
the GMP library (``\texttt{--GMP\_LIB\_DIR} \textit{$<$dir$>$}'').

\subsection{Setting up support for CLN}\label{sec:cln-setup}

By default there is no support for CLN, but you can change this easily
by use of the command line option ``\texttt{-cln}''. If CLN is
installed in system directories on your system, you are already done
now. If this is not the case, you have to supply the directories
containing the CLN header files (``\texttt{--CLN\_INCL\_DIR}
\textit{$<$dir$>$}'') and the CLN library (``\texttt{--CLN\_LIB\_DIR}
\textit{$<$dir$>$}'').

\subsection{Other Options}\label{sec:other-options}

There are some less important features of the install script we will
summarize here.

First of all you can get the version number of \texttt{cgal\_install}
with option ``\texttt{--version}''. Note that all other options are
ignored in this case.

Second there is an option ``\texttt{-os \textit{$<$compiler$>$}}''
where \textit{$<$compiler$>$} is your \CC\ compiler. This allows you
to determine your \cgal-OS description (see section
\ref{sec:os-compiler-id}). The compiler can either be given by an
absolute path like
\begin{verbatim}
./install_cgal -os /usr/local/gcc-2.95/sun/bin/g++
\end{verbatim}
or just by denoting its basename, as long as it is on your path:
\begin{verbatim}
./install_cgal -os CC
\end{verbatim}
The option is intended for testing purposes and automatic detection of
the correct include makefile (see also section \ref{sec:makefiles}).

Finally there exists an option ``\texttt{--verbose}'' that can be set
in interactive mode as well as in non-interactive mode. When set you
get a detailed summary of error messages occurring during \textit{any}
compiler test (determining \stl\ version etc.). Normally you only get
these messages, if a required test (such as the general \stl\ test)
fails, otherwise you are just informed, \textit{if} it succeeded or
not.  This option is not recommended for general use, but it can be
useful to check why a certain test fails that was expected to be
passed.

\section{Identifying OS and Compiler}\label{sec:os-compiler-id}

Since \cgal\ supports several different operating systems and
compilers, this is also reflected in the structure of the \cgal\ 
directory tree. Each OS/compiler combination has its own lib directory
under \texttt{\cgaldir/lib}) (and analogously its own include makefile
in \texttt{\cgaldir/make}) named as determined by the following
scheme.
\begin{center}
  \textit{$<$arch$>$\texttt{\_}$<$os$>$\texttt{-}$<$os-version$>$\texttt{\_}$<$comp$>${\tt
    -}$<$comp-version$>$}[\texttt{\_}LEDA]
\end{center}

\begin{description}
\item[$<$arch$>$] is the system architecture as defined by ``{\tt
    uname -p}'' or ``\texttt{uname -m}'',
\item[$<$os$>$] is the operating system as defined by ``\texttt{uname
    -s}'',
\item[$<$os-version$>$] is the operating system version as defined by
  ``\texttt{uname -r}'',
\item[$<$comp$>$] is the basename of the compiler executable (if it
  contains spaces, these are replaced by "-") \textit{and}
\item[$<$comp-version$>$] is the compiler's version number (which
  unfortunately can not be derived in a uniform manner, since it is
  quite compiler specific).
\end{description}

The suffix \texttt{\_}LEDA is appended to indicate \leda\ support. 

We call the resulting string \cgal-OS description.\\ Examples are
\texttt{mips\_IRIX-6.2\_CC-7.2} or {\tt
  sparc\_SunOS-5.5\_g++-2.95\_LEDA}.\\ You can use the install script
to get your \cgal-OS description, see section \ref{sec:other-options}.

\section{The \cgal\ makefile structure}\label{sec:makefiles}

The \cgal\ distribution contains the following makefiles:
\begin{itemize}
\item \texttt{\cgaldir/src/makefile\_lib} for compiling the \cgal\ 
  object library \texttt{libCGAL.a},
  
\item \texttt{\cgaldir/src/makefile\_sharedlib} for compiling the
  \cgal\ shared object library \texttt{libCGAL.so},
  
\item \texttt{\cgaldir/src/makefile\_geomview} for compiling a library
  for geomview support,
  
\item \texttt{\cgaldir/examples/makefile} as sample makefile
  \textit{and}
  
\item \texttt{\cgaldir/examples/*/makefile} for compiling the \cgal\ 
  example programs.
\end{itemize}

All these makefiles are generic: they can be used for more than one
compiler.  To achieve this, the first section of each makefile
contains an include statement that looks as follows:

\begin{verbatim}
      CGAL_MAKEFILE = /users/jannes/CGAL-2.1/make/makefile_<CGAL-OS description>
      include $(CGAL_MAKEFILE)
\end{verbatim}

The file \texttt{CGAL\_MAKEFILE} is an include file with platform
dependent makefile settings. The abbreviation \texttt{<CGAL-OS
  description>} (see section \ref{sec:os-compiler-id} for details) is
used to identify the operating system and compiler for which the
settings hold. For example, the file
\texttt{makefile\_mips\_IRIX64-6.5\_CC-n32-7.30} contains makefile
settings for the IRIX 6.5 operating system and the \mipsprocc\ 7.3
compiler.  These include files are automatically generated by the
\texttt{install\_cgal} script and they are all located in the
\texttt{\cgaldir/make} directory.  For convenience, the
\texttt{install\_cgal} script will substitute the include makefile
that was generated most recently.

If you want to compile an application or an object library with a
different compiler, the only thing you need to do is to substitute
another include makefile for the \texttt{CGAL\_MAKEFILE} variable. An
alternative way to do this is to create an environment variable
\texttt{CGAL\_MAKEFILE}. To pass the value of the environment variable
to the makefile you can either comment out the \texttt{CGAL\_MAKEFILE}
line in the makefile or use an appropriate command line option for the
make utility.  A comfortable way to set \texttt{CGAL\_MAKEFILE} is by
using \texttt{install\_cgal~-os} (see section
\ref{sec:other-options}).  E.g. if your compiler is \texttt{g++}, you
would type
\begin{alltt}
CGAL_MAKEFILE=`\yourcgaldir/install_cgal -os g++`
\end{alltt}
in bourne shell resp.
\begin{alltt}
setenv CGAL_MAKEFILE `\yourcgaldir/install_cgal -os g++`
\end{alltt}
in csh derivatives. 

\textit{Tip:} Include the setting of \texttt{CGAL\_MAKEFILE} into your
shell startup script (e.g. \texttt{.}(\texttt{t})\texttt{cshrc} for
(\texttt{t})\texttt{csh} or \texttt{.bashrc} for \texttt{bash}).

All makefiles contain sections with compiler and linker flags.  You
can add your own flags here. For example, you might want to add the
flag \texttt{-DCGAL\_NO\_PRECONDITIONS} to turn off precondition
checking. The flags \texttt{\$(CGAL\_CXXFLAGS)} and
\texttt{\$(CGAL\_LDFLAGS)} should never be removed.

The default extension for \cgal\ source files is \texttt{.C}.  The
last section of the makefiles contains a suffix rule that tells the
compiler how to create a \texttt{.o}-file from a \texttt{.C}-file.  If
you want to use the default rule that is defined by the make utility,
you may want to remove this suffix rule.  However, note that this may
have consequences for the makefile variables \texttt{CGAL\_CXX} and
\texttt{CXXFLAGS}.

\section{Compiling a \cgal\ application}
The directory \texttt{\cgaldir/examples} contains a small program
(example.C) and a sample makefile with some comments. The
\texttt{CGAL\_MAKEFILE} variable in this makefile (see section
\ref{sec:makefiles}) is automatically substituted by the
\texttt{install\_cgal} script and equals the most recently generated
include makefile in the \texttt{\cgaldir/make} directory. After the
installation of \cgal\ this sample makefile is ready for use. Just
type '\texttt{make example}' to compile the program
\texttt{example.C}. There is a script for conveniently creating
makefiles for \cgal\ applications, see
section~\ref{sec:create_cgal_makefile}.

Furthermore the directories \texttt{\cgaldir/examples} and
\texttt{\cgaldir/demo} contain many subdirectories with non-graphical
and graphical example programs.  In all these directories you will
find a makefile that is ready for use.

\section{Using \cgal\ and \leda}\label{sec:leda}
This section describes how to use \cgal\ and \leda\ simultaneously.

\subsection{Support for \leda}
\cgal\ supports \leda\ in the following ways:

\begin{enumerate}
\item There are support functions defined for the \leda\ number types
  \texttt{big\_float}, \texttt{integer}, \texttt{rational} and
  \texttt{real} (see the files \texttt{<CGAL/leda\_*>}).
\item For all two-dimensional geometric objects there are input/output
  operators from/to a \texttt{leda\_window}.
\item \cgal\ defines the following \leda-related compiler flags:
\begin{itemize}
\item When \leda\ is used, the flags \texttt{CGAL\_USE\_LEDA} and
  \texttt{LEDA\_PREFIX} will be set.
\item When \leda\ is used, the \leda\ memory management (\leda\ 
  handles) will be used for geometric primitives in \cgal. This can be
  turned of by setting the flag \texttt{CGAL\_NO\_LEDA\_HANDLE}. In
  that case \cgal\ memory management will be used (see
  \texttt{<CGAL/Handle.h>}).
\end{itemize}
\end{enumerate}

The include makefiles in the \texttt{\cgaldir/make} directory
corresponding to \leda\ can be recognized by the suffix
``\texttt{\_LEDA}''.

\section{Troubleshooting}\label{sec:troubleshooting}

This section contains some remarks about known problems and the
solutions we propose. If your problem is not listed here, please have
a look at the \cgal\ homepage:
\begin{alltt}
\cgalhomepage
\end{alltt}
or send an email to \texttt{cgal@cs.uu.nl}.

\subsection{The ``Long-Name-Problem'' (Solaris only)}

The system assembler and linker on Solaris 2.5 and 2.6 cannot handle
symbols with more than 1024 characters. But this number is quickly
exceeded where one starts nesting templates into each other. So if you
encounter strange assembler or linker errors like 
\begin{alltt}
/usr/ccs/bin/as: "/var/tmp/cc0B5iGc.s", line 24: 
error: can't compute value of an expression involving an external symbol
\end{alltt}
there is a good chance that you suffer from this ``long-name''
problem.

A solution is to install the \gnu -binutils\footnote{see
  \path'http://www.gnu.org/software/binutils/'} and to tell the
compiler that it shall use the \gnu-- instead of the native tools.
From the compiler-menu (described in section \ref{sec:compiler-menu})
you can set the corresponding option through the custom compiler
flags, i.e. for \texttt{gcc}/\texttt{egcs} you would add 
\begin{alltt}
  -B/my/path/to/gnu/binutils/bin
\end{alltt}
assuming you installed the \gnu -binutils executables in
\texttt{/my/path/to/gnu/binutils/bin}.

If you cannot (or do not want to) install \gnu -binutils, there is a
workaround that lets you compile, link and run your programs, but it
prevents debugging, since the executables have to be stripped. In
short the workaround is to compile with \texttt{-g} and to link with
\texttt{-z nodefs -s}. In order to still have portable makefiles (see
section\ref{sec:makefiles}), we define flags
\texttt{LONG\_NAME\_PROBLEM\_CXXFLAGS} and
\texttt{LONG\_NAME\_PROBLEM\_LDFLAGS} in the include makefiles which
are empty except for the Solaris platform where they are set as stated
above. In order to use these flags, edit your application makefile and
add the flags to \texttt{CXXFLAGS} resp. \texttt{LDFLAGS} as indicated
below.
\begin{alltt}
#---------------------------------------------------------------------#
#                    compiler flags
#---------------------------------------------------------------------#
# The flag CGAL_CXXFLAGS contains the path to the compiler and is defined
# in the file CGAL_MAKEFILE. You may add your own compiler flags to CXXFLAGS.

CXXFLAGS = $(LONG_NAME_PROBLEM_CXXFLAGS) $(CGAL_CXXFLAGS) 

#---------------------------------------------------------------------#
#                    linker flags
#---------------------------------------------------------------------#
# The flag CGAL_LDFLAGS contains common linker flags and is defined
# in the file CGAL_MAKEFILE. You may add your own linker flags to CXXFLAGS.

LDFLAGS = $(LONG_NAME_PROBLEM_LDFLAGS) $(CGAL_LDFLAGS) 
\end{alltt}

\subsection{\leda\ and \stl\ conflicts}

If you are using an old version of \leda, the combination of \leda\ 
and \stl\ may give some problems. In order to avoid them, it is highly
recommended to use the latest \leda\ release\footnote{At the moment
  this is \leda\ 4.0.}, since this is what we test \cgal\ with.

\section{Compiler Optimizations}\label{sec:compiler-optimisations}

You may have noticed that we do not set optimizer flags as \texttt{-O}
by default in the include makefiles(see section\ref{sec:makefiles} for
a description of the makefile structure in \cgal).  The main reason
for not doing this is that compilers run much more stable without. On
the other hand, most if not all \cgal\ programs will run considerably
faster when compiled with optimizations! So if you are going for
performance, you should/have to add \texttt{-O}, \texttt{-O3} or maybe
more specific optimizer flags (please refer to the compiler
documentation for that) to the \texttt{CXXFLAGS} variable in your
application makefile:
\begin{alltt}
#---------------------------------------------------------------------#
#                    compiler flags
#---------------------------------------------------------------------#
# The flag CGAL_CXXFLAGS contains the path to the compiler and is defined
# in the file CGAL_MAKEFILE. You may add your own compiler flags to CXXFLAGS.

CXXFLAGS = $(CGAL_CXXFLAGS) -O
\end{alltt}

\section{Upgrading a previous \cgal\ installation}
In case you have \cgal\ 1.*/2.0 installed on your system, you might
like to reuse your configuration files and GMP installations.  Simply
use the following command to copy them into the right place:
\begin{verbatim}
./install_cgal --upgrade <OLD_CGAL_DIR>
\end{verbatim}
where \texttt{<OLD\_CGAL\_DIR>} is the root directory of your existing
\cgal\ installation (e.g. \texttt{/pub/local/CGAL-1.2}).

You can then build all libraries for the actual operating system that
existed in your \cgal\ 1.* installation with 
\begin{verbatim}
./install_cgal --rebuild-all
\end{verbatim}

If you want to install \cgal\ for more than one operating system in
the same directory structure, you have to run the latter command
(\texttt{rebuild-all}) once on each operating system.

\section{Compiler workarounds}
In \cgal\ a number of compiler flags is defined, all of them start
with the prefix \texttt{CGAL\_CFG}. These flags are used to work
around compiler bugs and limitations.  For example, the flag
\texttt{CGAL\_CFG\_NO\_MUTABLE} denotes that the compiler does not
know the keyword \texttt{mutable}. 

For each compiler a file \texttt{<CGAL/compiler\_config.h>} is
defined, with the correct settings of all flags. This file is
generated automatically by the \texttt{install\_cgal} script. For this
the test programs in the directory \texttt{\cgaldir/config/testfiles}
are used. The file \texttt{<CGAL/compiler\_config.h>} and the test
programs contain a description of the problem, so in case of trouble
with a \texttt{CGAL\_CFG} flag it is a good idea to take a look at it.

The file \texttt{<CGAL/config.h>} manages all configuration problems
of the compiler.  This file includes the file
\texttt{CGAL/compiler\_config.h}.  It is therefore important that the
file \texttt{<CGAL/config.h>} is always included before any other
\cgal\ source file that depends on workaround flags. In most cases you
do not have to do anything special for this, because many CGAL files
already take care of including \texttt{<CGAL/config.h>}. Nevertheless
it would be a good idea to always start your \cgal\ programs with
including \texttt{<CGAL/config.h>} (or \texttt{<CGAL/basic.h>}, which
contains some more basic \cgal\ definitions).

\section{Scripts}

\subsection{\texttt{create\_cgal\_makefile}}\label{sec:create_cgal_makefile}

The bourne-shell script \texttt{create\_cgal\_makefile} is contained
in the \cgaldir\texttt{/scripts} directory. It can be used to create
makefiles for compiling \cgal\ applications. Executing
\texttt{create\_cgal\_makefile} in an application directory creates a
\texttt{makefile} containing rules for every \texttt{*.C} file there.

In order to use this makefile, you have to specify the \cgal\ include
makefile (see section~\ref{sec:makefiles}) to be used. This can be
done be either setting the environment variable
\texttt{CGAL\_MAKEFILE} or by editing the line
\begin{alltt}
  # CGAL_MAKEFILE = ENTER_YOUR_INCLUDE_MAKEFILE_HERE
\end{alltt}
of the created makefile. First remove the ``\texttt{\#}'' at the
beginning of the line and then replace the text after ``\texttt{=}''
by the location of the include makefile. 

Finally type \texttt{make} to compile the application programs.

\subsection{\texttt{use\_cgal\_namespace}}\label{sec:use_cgal_namespace}

The perl script \texttt{use\_cgal\_namespace} is contained in the
\cgaldir\texttt{/scripts} directory. It can be used to convert
\cgal\texttt{-1.*} application sourcecode to the \cgal\texttt{-2.*}
format.  Basically, it replaces \texttt{CGAL\_} prefixes by
\texttt{CGAL::} namespace qualifiers. You have to give the files to
convert as arguments, e.g.
\begin{alltt}
  use_cgal_namespace my_great_file1.C *.h
\end{alltt}
The original files are kept with the suffix \texttt{.bck}.

In order to use it, you first have to set the perl path in the first
line correctly, i.e. replace \texttt{/net/bin/perl5} by the path to
perl on your system (try \texttt{which perl}(5), if you do not know).
Alternatively, you can type
\begin{alltt}
  perl -wi.bck -- use_cgal_namespace <FILES>
\end{alltt}

\subsection{\texttt{replace\_headers}}\label{sec:replace_headers}

The perl script \texttt{redirect\_headers} is contained in the
\cgaldir\texttt{/scripts} directory. It can be used to replace
oldstyle include directives for standard headers by their standard ISO
counterparts, e.g.
\begin{alltt}
  #include <algo.h>
\end{alltt}
is replaced by
\begin{alltt}
  #include <algorithm>
\end{alltt}
You have to give the files to
convert as arguments, e.g.
\begin{alltt}
  use_cgal_namespace my_great_file1.C *.h
\end{alltt}
The original files are kept with the suffix \texttt{.hrbck}.

In order to use it, you first have to set the perl path in the first
line correctly, i.e. replace \texttt{/net/bin/perl5} by the path to
perl on your system (try \texttt{which perl}(5), if you do not know).
Alternatively, you can type
\begin{alltt}
  perl -wi.hrbck -- replace_headers <FILES>
\end{alltt}

\section{Installation on Cygwin}\label{sec:cygwin}

Cygwin is a free Unix-like environment for MS-Windows, distributed by
Cygnus Solutions, see \path~http://sourceware.cygnus.com/cygwin/~ For
our tests we have used version $\beta$-20.1.

It consists of a port of a large number of GNU tools, such as bash,
make, gcc (egcs), gas, file utilities, etc, as well as tools ensuring
an ability to emulate Unix-like access to resources, for instance
mount.  For a comprehensive introduction and details, see the URL
above.

\subsection{Pathnames}

Cygwin has a UNIX-like way of navigating hard drives, NFS shares, etc.
This is also the way in which directories and pathnames have to given
to the installation script. They are automatically converted to
Win*-style pathnames when given to the compiler or linker.

The main difference is that directories are seperated by slash (``/'')
rather than by backslash (``$\backslash$'').  The other difference is
concerned with specifying drives. One way is to use POSIX-style
pathnames that map Win*-style drives (\texttt{A:}, \texttt{B:}) to
\texttt{//a/\ldots}, \texttt{//b/\ldots} respectively. For instance,
the path
\texttt{D:$\backslash$Mystuff$\backslash$Mydir$\backslash$LEDA}
translates to \texttt{//d/Mystuff/Mydir/LEDA}.

Alternatively, it can be done using the mount utility, that can be
used to establish a map between Win*-style drives and the Unix-like
style. More precisely, it maps the forest of the directories/files on
Win*-drives to a tree with the root that is usually located at the top
level of the boot drive, say \texttt{C:}.  The root location can be
seen by typing \texttt{mount} command without parameters.  For
instance, if \texttt{D:} is mounted on
\texttt{C:$\backslash$ddrive}\footnote{by typing \texttt{mount D:
    /ddrive}} then the path
\texttt{D:$\backslash$Mystuff$\backslash$Mydir$\backslash$LEDA}
translates to \texttt{/ddrive/Mystuff/Mydir/LEDA}.

\paragraph{Upper/lower case and spaces in file names}
might also be confusing. Behavour of Cygwin in this regard might be
different from the \mswin\ behavour.
In particular, using spaces in filenames should better be avoided.

\paragraph{Links, shortcuts, etc} should be avoided as well.

\subsection{\msvc{6.0}-setup}
A number of environment variables has to be set (or updated)
in order to use the installation.

\texttt{PATH} should contain \msvc{6.0} command line tools locations.
The environment variables \texttt{INCLUDE} and \texttt{LIB} should
point to the location of \msvc{6.0} header files and to the location
of the \msvc{6.0} libraries, respectively.
The interface for doing this is different for NT and for Win9*.

\paragraph{\mswin-NT4.0.}
One can set the corresponding environment variables using the
usual NT interface\footnote{open MyComputer, press right mouse button,
  select Properties, select Environment, set the relevant variables}.
Alternatively, they can be set in \texttt{.bashrc} file for the
particular user.

The result should look roughly as follows, assuming that
\texttt{C:$\backslash$PROGRA$\sim$1$\backslash$MICROS$\sim$2$\backslash$}
is the location of the \msvc{} installation.
\begin{verbatim}
LIB=C:\PROGRA~1\MICROS~2\VC98\LIB
INCLUDE=C:\PROGRA~1\MICROS~2\VC98\INCLUDE
\end{verbatim}
and \texttt{PATH} should contain
\begin{verbatim}
/PROGRA~1/MICROS~2/Common/msdev98/BIN:
/PROGRA~1/MICROS~2/VC98/BIN:/PROGRA~1/MICROS~2/Common/TOOLS:
/PROGRA~1/MICROS~2/Common/TOOLS/WINNT
\end{verbatim}

\paragraph{\mswin-9*.} 
First, the memory for environment variables has to be increased.
Select the Cygwin icon from the Start-menu, press the right mouse
button and choose \textit{Properties}. Go to \textit{Memory}, select
\textit{Initial Environment}, set it to at least 2048 and
\textit{apply} the changes.

Second, edit the file \texttt{cygnus.bat}, located in the cygwin main
directory and add the line
\begin{verbatim}
call C:\PROGRA~1\MICROS~2\VC98\Bin\MSCVARS32.BAT
\end{verbatim}
where
\texttt{C:$\backslash$PROGRA$\sim$1$\backslash$MICROS$\sim$2$\backslash$}
has to be customized according to where \msvc{} is installed on your
system.

