%+-------------------------------------------------------
%| CGAL Manual : pmwx.tex
%+--------------------------------------------------------
%| update log
%|
%| 04 May 2001  - Eyal Flato
%|    Created from arr documentation
%| 15 May 2003  - Efi Fogel
%|    Cleanup
%|  1 July 2003 - Ron Wein
%|    Some more revisions
%--------------------------------------------------------


\section{Introduction}

Given a collection ${\mathcal C}$ of (possibly intersecting and not necessarily
$x$-monotone curves) in the plane, we construct a collection ${\mathcal C}''$ in
two steps, as follows: First, we decompose each curve in ${\mathcal C}$ into
maximal $x$-monotone curves, thus obtaining the collection
${\mathcal C}'$. Second, We decompose each curve in ${\mathcal C}'$ into 
maximal connected pieces not intersecting any other curve in ${\mathcal C}'$.
This way we obtain the collection ${\mathcal C}''$ of $x$-monotone, pairwise 
interior disjoint curves.  Constructing the {\it planar map with 
intersection} of the curves in ${\mathcal C}$ is therefore equivalent to the
construction of the {\it planar map}(see Chapter~\ref{I1_ChapterPlanarMap}) 
induced by the curves in ${\mathcal C}''$.

The \ccc{Planar Map with Intersections} package extends the functionality 
of the \ccc{Planar Map} package by enabling simple insertion of intersecting 
and not necessarily $x$-monotone curves. The 
\ccc{Planar_map_with_intersections_2} class has different insertion functions 
but it uses the same data structures as the \ccc{Planar_map_2} class. 
Therefore, almost any functionality of the planar map is also supported here 
(e.g., traversal of planar map features, point location queries and I/O 
operations are supported but infinite objects are not supported). However, 
to maintain a planar map with intersections one obviously needs to support 
some additional functions in the geometric traits class, handling with 
intersections and $x$-monotonicity.

Note that if one needs to build a planar map of $x$-monotone, pairwise 
interior disjoint curves, then it would be more efficient (in running time) 
and less demanding (in traits class functionality) to use the 
\ccc{Planar_map_2} class instead.

\paragraph{Degeneracies} 
  
Like the \ccc{Planar Map} package (see Chapter~\ref{I1_ChapterPlanarMap}), 
the \ccc{Planar Map with Intersections} package can deal with $x$-degenerate 
input (including vertical segments). However, while in the planar map the input
curves were assumed to be non-intersecting in their interiors, there is no 
such assumption when using planar map with intersections. Furthermore, 
overlapping curves are also supported: If two curves overlap the traits 
intersection function must return the two endpoints of the common part.

%---------------------------------------------------
\subsection{A Simple Program}
\label{ssec:example1}
The simple program listed below demonstrates the construction of an
$X$-shaped planar subdivision out of two intersecting segments.
The coordinates of the halfedges of the constructed subdivision are
printed to standard output.

\ccIncludeExampleCode{Pm_with_intersections/example1.C}

The output of the program looks like this:
\begin{verbatim}
Inserting the segments:
0 0 1 1
0 1 1 0

Edges of the planar map:
0 0 --- 0.5 0.5
0.5 0.5 --- 1 1
0 1 --- 0.5 0.5
1 0 --- 0.5 0.5
\end{verbatim}

\section{Architecture}

The \ccc{Planar_map_with_intersections_2<Planar_map_2<Dcel, Traits> >} class 
is parameterized with the \ccc{Planar_map_2} class it inherits from, which 
is parameterized with the \ccc{Dcel} and the \ccc{Traits} objects. The
\ccc{Dcel} is a data structure that maintains a
{\em doubly-connected edge list} and represents the underlying topology.
The geometric functionality is provided by the \ccc{Traits} class, and is
tailored to handle a specific family of curves. It encapsulates the
number type used and the coordinate representation. This package
contains traits classes that handle various types of curves (e.g.,
segments, polylines, conics, etc.).

Similar to some constructors of the \ccc{Planar_map_2}, 
(see Chapter~\ref{I1_ChapterPlanarMap}), some of the constructors of the
\ccc{Planar_map_with_intersections_2} class allow you to choose between
various point-location strategies. While the default point-location
strategy of the \ccc{Planar_map_2} class is the {\it trapezoid-ric} strategy,
which is based on a trapezoidal decomposition of the map, and requires
constructing and maintening auxliary data structures, the default strategy
of the \ccc{Planar_map_with_intersections_2} class is the
{\it walk-along-a-line\/} strategy, as the overhead of the former
tends to be very large when no restrictions are applied on the input curves.

\subsection{Operations}
The set of operations that can be applied to a planar map with intersection 
is divided into four subsets, namely constructors, modifiers, queries, and
input/output operations. These operations are overviewed in detail in
section~\ref{I1_ChapterPlanarMap}, so next we will just emphasize the 
differences between the two classes.
% RWRW - Give a more exact reference.

\subsubsection{Aggregated Insert}

A \ccc{Planar Map with Intersections} can be built incrementally by
inserting one curve after the other into the map. However, for a large number
of curves that intersect rather sparsely, it can be more efficient to use the
aggregate insertion method, that inserts a set of curves to an empty map at
once by performing the sweep-line algorithm on the set of input curves.

The aggregate insertion method is more efficient in many cases and it also
has less requirements from the traits class, in comparison with the the
incremental insertion function. Namely, the \ccc{curves_compare_y_at_x_left()} and the \ccc{nearest_intersection_to_left()} functions are not required, nor do
the various reflection functions.

\subsubsection{Example of Aggregate Insertion}
\label{sssec:example1_aggrinsrt}
The following example demonstrates the usage of the aggregate insertion
method. It constructs a planar map out of four segments --- 
$(0,0)-(1,1)$ , $(0,1)-(1,0)$ , $(0,0)-(1,0)$ and $(0,1)-(1,1)$
(an hourglass shape), two of them are intersecting in their interior. 
The resulting planar map will contain all the disjoint interior sub-segments 
obtained by the calculation of the sweep line algorithm. For clarity, we printed all the halfedges of the resulting planar map to the standard output.

% RWRW - where are the examples?
%\ccIncludeExampleCode{Sweep_line_2/example1.C}

The output of the program looks like this:

% RWRW - Where are the examples?
%\ccIncludeExampleCode{Sweep_line_2/example1.cout}

\subsubsection{Non Intersecting Insertion Functions}

In some cases the users insert curves to a planar map with intersections
in an incremental manner, but have some knowledge regarding the location
of several curves. In such cases, special insertion functions may be called
in order to speed up the construction of the map:
\begin{itemize}
\item If it is known in advance that the current curve is $x$-monotone and does
not intersect any one of the curves currently in the map in its interior,
it is possible to insert this curve using the \ccc{non_intersecting_insert()}
function. A similar function is also available for a range of $x$-monotone
and interior-disjoint curves, that does not induce any intersection with the
existing curves in the map.
\item Sometimes the exact location of the $x$-monotone curve in the map is known. It may be inserted (1) within the interior of a given face, (2) with one given vertex as one of its endpoints, or (3) between to given vertices. The \ccc{non_intersecting_insert_in_face_interior()}, \ccc{non_intersecting_insert_from_vertex()} and \ccc{non_intersecting_insert_at_vertices()} functions serve for this purpose. For more details regrading these special insertion functions, as well as for an example for their usage, see section~\ref{I1_ChapterPlanarMap}.
% RWRW - refine this reference.
\end{itemize}

\begin{ccAdvanced}
\subsection{Change Notification}
%-----------------------------------------------
  
An insertion of an intersecting curve into a planar map may add
several halfedges and modify several features of the map (i.e.
split halfedges, split faces, etc.). The so-called \ccc{Change
Notification} class provides this kind of flexibility. The modification 
methods accept an additional parameter, a class which is a model of the
\ccc{PlanarMapWithIntersectionsChangeNotification_2} concept. The
change notification includes an associative function for each
modification method.  This function is called after each such
modification.

The change notification class is useful in many cases. For example,
one may add a color (or other extra data) to any halfedge of a
planar map. An insertion of a new curve can split halfedges that
were previously in the map. After such a split the color of the
newly created halfedges should be updated according to the original
color of the split halfedge. One can do this by implementing the
\ccc{split_edge} function of the change notification class. This
function will be called after each split of an halfedge in the map.

\subsubsection{Example of Change Notification}
\label{ssecn:example2}
The following example demonstrates the usage of the change
notification concept during the construction of a
planar map out of three segments --- $(0,1)-(1,0)$, $(0,0)-(1,1)$
and $(0,1)-(1,1)$.
During the insertion we use \ccc{My_notification} instance to
output the internal process of the construction of the planar map.
We also count how many edges are in the map by incrementing a
counter each time an edge is added (\ccc{add_edge}) or split 
(\ccc{split_edge}).

\ccIncludeExampleCode{Pm_with_intersections/example2.C}

The output of the program looks like this:
\begin{verbatim}
inserting 0 1 1 0
add_edge
add_hole
inserting 0 0 1 1
split_edge
add_edge
add_edge
inserting 0 1 1 1
add_edge
split_face
Total number of edges 5
\end{verbatim}

\end{ccAdvanced}

