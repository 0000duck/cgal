%+-------------------------------------------------------
%| CGAL Manual : pmwx.tex
%|
%| Soon to be split into User and Reference Manuals
%+--------------------------------------------------------
%| update log
%|
%| 04 May 2001  - Eyal Flato
%|    Created from arr documentation
%|
%--------------------------------------------------------

%\documentclass[12pt]{book}
%\usepackage{graphics, amssymb,epsfig}
%\usepackage{cprog,cc_manual}
%
%\def\Ipe#1{\def\IPEfile{#1}\input{#1}}
%
%\setlength{\evensidemargin}{.0in}
%\setlength{\oddsidemargin}{-0.3in}
%\setlength{\textwidth}{6.8in}
%\setlength{\textheight}{8in}
%
%\renewcommand{\Re}{{\rm I\!\hspace{-0.025em} R}}
%\newcommand{\normal}[1]{\eta_{#1}}
%\newtheorem{theorem}{Theorem}[section] % section
%\newtheorem{remark}[theorem]{Remark}
%\newtheorem{lemma}[theorem]{Lemma}
%\newenvironment{dfn}{{\vspace*{1ex} \noindent \bf Definition }}{\vspace*{1ex}}
%\newcommand{\bigdef}[2]{\index{#1}\begin{dfn} {\rm #2} \end{dfn}}
%\newenvironment{proof}{{\em Proof:}}{\hfill{\hfill\rule{2mm}{2mm}}}
%
%\newcommand{\comment}[1]{{\sf * #1 *}}
%\newcommand{\ncomment}[1]{\noindent {\sf * #1 * }}
%
%\newtheorem{defn}[theorem]{Definition}
%\newcommand{\intsupplanes}{P}
%\def\C{{\cal C}}
%\def\G{{\cal G}}
%\def\F{{\cal F}}
%\def\I{{\cal I}}
%\def\U{{\cal U}}
%\def\M{{\cal M}}
%\def\eps{{\varepsilon}}
%\def\bd{{\partial}}
%\def\dm{{\cal D}}
%
% Title
%\title{CGAL Arrangement Specifications}
%
%\date{ \today }
%
%\begin{document}
%
%\tableofcontents
%
%\maketitle
\chapter{2D Planar Maps with Intersections} \label{I1_ChapterPmwx}

\section{Introduction}

\paragraph{2D Planar Map with Intersections:} Given a collection $C$ of (possibly
intersecting) $x$-monotone curves in the plane, we construct a
collection $C'$ as follows: We decompose each curve in $C$ into
maximal connected pieces not intersecting any other curve in $C$.
This way we obtain the collection $C'$ of $x$-monotone, pairwise
interior disjoint curves.
Constructing the {\it planar map with intersection} of the curves
in $C$ gives the {\it planar map}(see
Chapter~\ref{I1_ChapterPlanarMap}) induced by the curves in $C'$.

Planar map with intersections extends the functionality of planar
map by enabling simple insertion of intersecting $x$-monotone
curves. The planar map with intersections class has different
insertion functions but it uses the same data structures of the
planar map. Therefore, any functionality of the planar map is also
supported here (e.g., traversal of planar map features, point
location queries). However, The planar map with intersections
needs additional intersection functions in the geometric traits
class. Note that if one needs to build a planar map of
$x$-monotone, pairwise interior disjoint curves, then it would be
more efficient (in running time) and less demanding (in traits
class functionality) to use the planar map class.


\paragraph{Degeneracies} Like the planar map class (see
Chapter~\ref{I1_ChapterPlanarMap}), the planar map with intersections class can deal with
$x$-degenerate input (including vertical segments). However, while in the
planar map the input curves were assumed to be non
intersecting in their interiors, there is no such assumption when using
planar map with intersections. Furthermore, overlapping curves are
supported. If two curves overlap the traits intersection function returns
the two endpoints of the common part.

\paragraph{Sweep line:}
As mentioned in Chapter ~\ref{I1_ChapterPlanarMap}), 
inserting interior intersecting curves, overlap or non $x$-monotone to \ccc{Planar map} 
is available due to the sweep line algorithm.
When using the sweep line technique it is also possible to report only 
the resulting subcurves induced by the input curves.
The sweep line algorithm will use the traits classes defined for \ccc{Planar Maps with Intersections}. 

%******************************************************************************



\section{Example Programs}
%---------------------------------------------------
\subsection{Simple Example of a Segment Planar Map with Intersections}
\label{ssec:example1}
The following example demonstrates the construction of an
$X$-shaped planar subdivision out of two intersecting segments.
We output the coordinates of the halfedges of the constructed
subdivision.

\ccIncludeExampleCode{Pm_with_intersections/example1.C}

The output of the program looks like this:
\begin{verbatim}
Inserting the segments:
0 0 1 1
0 1 1 0

Edges of the planar map:
0 0 --- 0.5 0.5
0.5 0.5 --- 1 1
0 1 --- 0.5 0.5
1 0 --- 0.5 0.5
\end{verbatim}


\subsubsection{Example of Sweep line}
\label{ssec:example1}
The following example demonstrates the usage of the \ccc {Sweep line} algorithm.
It constructs a planar map out of four segments --- $(0,0)-(1,1)$ , $(0,1)-(1,0)$ , 
$(0,0)-(1,0)$ and $(0,1)-(1,1)$ , two of them are intersecting in their interior.
The resulting planar map will contain all the disjoint interior sub segments obtained 
by the calculation of the \ccc {Sweep line} algorithm. For clearness, we printed all the 
halfedges of the resulting planar map to the standard output using the I/O functions for \ccc{Planar map}.

\ccIncludeExampleCode{Sweep_line/example1.C}

The output of the program looks like this:

\ccIncludeExampleCode{Sweep_line/example1.cout}


% EOF ------------------------------------------------------------------------80

