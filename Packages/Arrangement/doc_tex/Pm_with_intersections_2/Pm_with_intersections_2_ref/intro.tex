% +------------------------------------------------------------------------+
% | Reference manual chapter: intro.tex (Pmwx)
% +------------------------------------------------------------------------+
% |
% | Package: arr (Arrangement_2)
% |
% +------------------------------------------------------------------------+

%+---------------------------------------------------------------------------80
%| update log
%|
%| 04 May 2001  - Eyal Flato
%|    Created from arr documentation
%|
%+---------------------------------------------------------------------------80


% +========================================================================+
%   Introduction
% +========================================================================+
\clearpage
\section{Planar Map with Intersections Reference Pages}

\subsection*{Introduction}
\ccRefLabel{pmwx_ref_intro}

  The \ccc{Planar Map with Intersections} package extends the
  functionality of the \ccc{Planar Map} package. Moreover, the class
  \ccc{Planar_map_with_intersections_2} extends the class
  \ccc{Planar_map_2}. The \ccc{Planar Map with Intersections} package
  supports insertion functions that handle non-$x$-monotone,
  intersecting and overlappings curves among the curves of
  the planar map.

%\subsection*{Planar Map with Intersections Traits Class}
%-----------------------------------------------
  Since the planar map with intersections has an additional
  functionality of handling non $x$-monotone and intersecting curves
  it needs additional functionality of the traits class. We describe
  the \ccc{PlanarMapwithIntersectionsTraits_2} concept following in
  this chapter. 

  %Nevertheless, we do not have specific implementation
  %of this concept, and we use models of the \ccc{ArrangementTraits_2}
  %concept, which is a refinement thereof instead (see
  %Chapter~\ref{I1_ChapterArrangement}).

\subsection*{Sweep line:}
  The construction of a planar map can be done using a sweep line
  technique rather than iteratively, inserting one curve after another.
  The sweep line algorithm perfomrs better, but is done statically and
  therefore requires the list of curves in advance.  Due to the sweep
  line technique, inserting curves which are interior intersect,
  overlap or non $x$-monotone is supported. The interface is given in
  chapter ~\ref{I1_ChapterPlanarMap}.


\begin{ccAdvanced}
\subsection*{Change Notification}
%-----------------------------------------------
  An insertion of an intersecting curve into a planar map may add
  several halfedges and modify several features of the map (i.e.,
  split halfedges, split faces). The so-called \ccc{Change
  Notification} provides this flexibility. The modification methods
  accept an additional parameter, a class which is a model of the
  \ccc{PlanarMapWithIntersectionsChangeNotification_2} concept. The
  change notification includes an associative function for each
  modification method.  This function is called after each such
  modification.

  The change notification class is useful in many cases. For example,
  one may add a color (or other extra data) to any halfedge of a
  planar map. An insertion of a new curve can split halfedges that
  were previously in the map. After such a split the color of the
  newly created halfedges should be updated according to the original
  color of the split halfedge. One can do this by implementing the
  \ccc{split_edge} function of the change notification class. This
  function will be called after each split of an halfedge in the map.

\subsubsection{Example of Change Notification}
\label{ssecn:example2}
The following example demonstrates the usage of the change
notification concept during the construction of a
planar map out of three segments --- $(0,1)-(1,0)$, $(0,0)-(1,1)$
and $(0,1)-(1,1)$.
During the insertion we use \ccc{My_notification} instance to
output the internal process of the construction of the planar map.
We also count how many edges are in the map by incrementing a
counter each time an edge is added (\ccc{add_edge})or
split (\ccc{split_edge}).

\ccIncludeExampleCode{Pm_with_intersections/example2.C}

The output of the program looks like this:
\begin{verbatim}
inserting 0 1 1 0
add_edge
add_hole
inserting 0 0 1 1
split_edge
add_edge
add_edge
inserting 0 1 1 1
add_edge
split_face
Total number of edges 5
\end{verbatim}

\end{ccAdvanced}



\begin{ccTexOnly}

\subsection*{Concepts}
\ccRefConceptPage{PlanarMapWithIntersectionsTraits_2}\\
\ccRefConceptPage{PlanarMapWithIntersectionsChangeNotification_2}\\

\subsection*{Classes}
\ccRefIdfierPage{CGAL::Planar_map_with_intersections_2<Planar_map>}%\\


\subsection*{Functions}
\ccRefIdfierPage{CGAL::sweep_to_construct_planar_map}\\
\ccRefIdfierPage{CGAL::sweep_to_produce_planar_map_subcurves}\\

\end{ccTexOnly}












