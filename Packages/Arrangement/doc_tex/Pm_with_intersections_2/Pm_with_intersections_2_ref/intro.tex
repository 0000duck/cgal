% +------------------------------------------------------------------------+
% | Reference manual chapter: intro.tex (Pmwx)
% +------------------------------------------------------------------------+
% |
% | Package: arr (Arrangement_2)
% |
% +------------------------------------------------------------------------+

%+---------------------------------------------------------------------------80
%| update log
%|
%| 04 May 2001  - Eyal Flato
%|    Created from arr documentation
%|
%+---------------------------------------------------------------------------80


% +========================================================================+
%   Introduction
% +========================================================================+
\clearpage
\section{Planar Map with Intersections Reference Pages}

\subsection*{Introduction}
\ccRefLabel{pmwx_ref_intro}

The planar map with intersections class extends the planar map
class. The planar map with intersections adds insertion functions
that handle intersections and overlapping among the curves of the
planar map. The basic types of the planar map (vertex, halfedge,
face, Dcel, etc.) are kept. \ccc{Planar_map_with_intersections_2}
is templated with the planar map class it derives from.

\subsection*{Planar Map with Intersections Traits Class}
%-----------------------------------------------
Since the planar map with intersections has an additional
functionality of intersecting curves it needs additional
functionality of the traits class. We describe the planar map with
intersections traits concept following in this chapter.
Nevertheless, we do not have specific implementation of this
concept, and we use the arrangement traits instead (see
Chapter~\ref{I1_ChapterArrangement}).

\subsection*{Sweep line}
%-------------------------------------------------
As mentioned at chapter ~\ref{I1_ChapterPlanarMap} inserting curves which are interior intersect, 
overlap or non x-monotone to \ccStyle {Planar map} is supported by a \ccc {Sweep line} algorithm. However, 
the \ccStyle{Sweep line} is a part of \ccStyle {Planar Map with Intersections} since 
the traits we use here are taken from that package, due to calcluations as finding 
intersections, overlapping and x-monotony. 
The \ccc{Sweep line} algorithm is templated with the subdivision it updates and hence we can use 
any subdivision containing insertion functions defined as in \ccc{Planar map}, particularly 
we can use \ccStyle {Planar Map with Intersections} as our subdivision. This may be usefull for cases 
in which sweeping curves can be used, but then inserting intersecting curves each at a time is needed.


\subsubsection{Example of Sweep line}
\label{ssec:example1}
The following example demonstrates the usage of the \ccc {Sweep line} algorithm.
It constructs a planar map out of four segments --- $(0,0)-(1,1)$ , $(0,1)-(1,0)$ , 
$(0,0)-(1,0)$ and $(0,1)-(1,1)$ , two of them are intersecting in their interior.
The resulting planar map will contain all the disjoint interior sub segments obtained 
by the calculation of the \ccc {Sweep line} algorithm. For clearness, we wrote the resulting 
planar map to the standard output using the I/O functions for \ccc{Planar map}.

\ccIncludeExampleCode{Sweep_line/example1.C}

The output of the program looks like this:

\ccIncludeExampleCode{Sweep_line/example1.cout}

\label{ssec:example2}
The following example is also a demonstration of the usage of the \ccc {Sweep line} algorithm. 
In this example we shall present the usage when inserting the resulting disjoint interior 
subcurves to a container. 
The input segments are just like before, and the result we are interested with is only 
reporting the sub segments obtained by the \ccc{Sweep line} algorithm.

\ccIncludeExampleCode{Sweep_line/example3.C}

And the output of the program looks like this:

\ccIncludeExampleCode{Sweep_line/example3.cout}



\begin{ccAdvanced}
\subsection*{Change Notification}
%-----------------------------------------------
An insertion of an intersecting curve to a planar map may add
several halfedges and modify several features of the map (i.e.,
split halfedges, split faces). The relevant function of the change
notification instance, that is passed to the modification methods
of a planar map with intersections, is called after each such
modification. The change notification class is useful in many
cases. For example, one may add a color (or other extra data) to
any halfedge of a planar map. An insertion of a new curve can
split halfedges that were previously in the map. After such split
the color of the newly created halfedges should be updated
according to the original color of the split halfedge. One can do
this by implementing the \ccc{split_edge} function of the change
notification class. This function will be called after each split
of an halfedge in the map.


\subsubsection{Example of Change Notification}
\label{ssec:example2}
The following example demonstrates the usage of the change
notification concept during the construction of a
planar map out of three segments --- $(0,1)-(1,0)$, $(0,0)-(1,1)$
and $(0,1)-(1,1)$.
During the insertion we use \ccc{My_notification} instance to
output the internal process of the construction of the planar map.
We also count how many edges are in the map by incrementing a
counter each time an edge is added (\ccc{add_edge})or
split (\ccc{split_edge}).

\ccIncludeExampleCode{Pm_with_intersections/example2.C}

The output of the program looks like this:
\begin{verbatim}
inserting 0 1 1 0
add_edge
add_hole
inserting 0 0 1 1
split_edge
add_edge
add_edge
inserting 0 1 1 1
add_edge
split_face
Total number of edges 5
\end{verbatim}

\end{ccAdvanced}



\begin{ccTexOnly}

\subsection*{Concepts}
\ccRefConceptPage{Planar_Map_with_Intersections_Traits}\\
\ccRefConceptPage{Planar_Map_with_Intersections_Change_Notification}\\

\subsection*{Classes}
\ccRefIdfierPage{CGAL::Planar_map_with_intersections_2<Planar_map>}%\\


\subsection*{Functions}
\ccRefIdfierPage{CGAL::sweep_to_construct_planar_map<Curve_iterator,Planar_map>(Curve_iterator curves_begin, Curve_iterator curves_end, Planar_map &result);}\\
\ccRefIdfierPage{CGAL::sweep_to_produce_planar_map_subcurves<Curve_iterator,Traits,Containe>(Curve_iterator curves_begin, Curve_iterator curves_end, Container &subcurves, bool overlapping = false);}
\end{ccTexOnly}












