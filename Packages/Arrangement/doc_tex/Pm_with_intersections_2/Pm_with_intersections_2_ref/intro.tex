% +------------------------------------------------------------------------+
% | Reference manual chapter: intro.tex (Pmwx)
% +------------------------------------------------------------------------+
% |
% | Package: arr (Arrangement_2)
% |
% +------------------------------------------------------------------------+

%+---------------------------------------------------------------------------80
%| update log
%|
%| 04 May 2001  - Eyal Flato
%|    Created from arr documentation
%|
%+---------------------------------------------------------------------------80


% +========================================================================+
%   Introduction
% +========================================================================+
\clearpage
\section{Planar Map with Intersections Reference Pages}

\subsection*{Introduction}
\ccRefLabel{pmwx_ref_intro}

  The \ccc{Planar Map with Intersections} package extends the
  functionality of the \ccc{Planar Map} package. Morover, the class
  \ccc{Planar_map_with_intersections_2} extends the class
  \ccc{Planar_map_2}. The \ccc{Planar Map with Intersections} package
  supports insertion functions that handle non $x$-monotone curves and
  intersections and overlappings among the curves of the planar
  map. The basic types of the \ccc{Planar Map} package (vertex,
  halfedge, face, Dcel, etc.) are
  kept. \ccc{Planar_map_with_intersections_2} is templated with the
  planar map class it derives from.

\subsection*{Planar Map with Intersections Traits Class}
%-----------------------------------------------
  Since the planar map with intersections has an additional
  functionality of handling non $x$-monotone and intersecting curves
  it needs additional functionality of the traits class. We describe
  the \ccc{PlanarMapwithIntersectionsTraits_2} concept following in
  this chapter. 

  %Nevertheless, we do not have specific implementation
  %of this concept, and we use models of the \ccc{ArrangementTraits_2}
  %concept, which is a refinement thereof instead (see
  %Chapter~\ref{I1_ChapterArrangement}).

\subsection*{Sweep Line}
%-------------------------------------------------
  The \ccc{Planar Map with Intersections} packge contains a sweep line
  utility. The sweep line algorithm can be used to build a planar map
  much quicker than in an incremental way, where each curve is
  inserted after the other. Two global functions are provided. 

  The function \ccc{sweep_to_construct_planar_map} builds a planar map
  as a result of the call. Mind that it is possible to call this
  function also with an instance the \ccc{Planar_map_2} class rather
  than an instance of \ccc{Planar_map_with_intersections_2}. If no
  additional insertion of intersecting or non $x$-monotone curves are
  planned follwing the building of the map, then it would be more
  efficient (in running time) and less demanding (in traits class
  functionality) to use an instance of the former.

  The function \ccc{sweep_to_produce_planar_map_subcurves} does not
  build a planar map but rather collects the $x$-monotone pairwise
  disjoint subcurves in a container. It is possible to tune the
  function so that it will avoid to produce repetitions of overlapping
  subcurves in the container or not.

  The sweep line utility calculates intersections of curves.  The
  \ccc{PlanarMapWithIntersectionTraits_2} concepts requires this
  very functionality.  Thus, the sweep line functions have to use
  traits classes that model the above concept. This is true even for
  the case where an instance of \ccc{Planar_map_2<Dcel, Traits>} is built.

  See the sweep line functions reference pages for implementation
  details (\ccc{sweep_to_construct_planar_map}\lcTex{
  \ccRefPage{CGAL::sweep_to_construct_planar_map}} and
  \ccc{sweep_to_produce_planar_map_subcurves}\lcTex{
  \ccRefPage{CGAL::sweep_to_produce_planar_map_subcurves}}).


\begin{ccAdvanced}
\subsection*{Change Notification}
%-----------------------------------------------
An insertion of an intersecting curve to a planar map may add
several halfedges and modify several features of the map (i.e.,
split halfedges, split faces). The relevant function of the change
notification instance, that is passed to the modification methods
of a planar map with intersections, is called after each such
modification. The change notification class is useful in many
cases. For example, one may add a color (or other extra data) to
any halfedge of a planar map. An insertion of a new curve can
split halfedges that were previously in the map. After such split
the color of the newly created halfedges should be updated
according to the original color of the split halfedge. One can do
this by implementing the \ccc{split_edge} function of the change
notification class. This function will be called after each split
of an halfedge in the map.


\subsubsection{Example of Change Notification}
\label{ssecn:example2}
The following example demonstrates the usage of the change
notification concept during the construction of a
planar map out of three segments --- $(0,1)-(1,0)$, $(0,0)-(1,1)$
and $(0,1)-(1,1)$.
During the insertion we use \ccc{My_notification} instance to
output the internal process of the construction of the planar map.
We also count how many edges are in the map by incrementing a
counter each time an edge is added (\ccc{add_edge})or
split (\ccc{split_edge}).

\ccIncludeExampleCode{Pm_with_intersections/example2.C}

The output of the program looks like this:
\begin{verbatim}
inserting 0 1 1 0
add_edge
add_hole
inserting 0 0 1 1
split_edge
add_edge
add_edge
inserting 0 1 1 1
add_edge
split_face
Total number of edges 5
\end{verbatim}

\end{ccAdvanced}



\begin{ccTexOnly}

\subsection*{Concepts}
\ccRefConceptPage{PlanarMapWithIntersectionsTraits_2}\\
\ccRefConceptPage{PlanarMapWithIntersectionsChangeNotification_2}\\

\subsection*{Classes}
\ccRefIdfierPage{CGAL::Planar_map_with_intersections_2<Planar_map>}%\\


\subsection*{Functions}
\ccRefIdfierPage{CGAL::sweep_to_construct_planar_map}\\
\ccRefIdfierPage{CGAL::sweep_to_produce_planar_map_subcurves}\\

\end{ccTexOnly}












