% +------------------------------------------------------------------------+
% | Reference manual page: Segment_circle_2.tex (Arrangement)
% +------------------------------------------------------------------------+
% | 
% | Package: arr (Segment_circle_2)
% | 
% +------------------------------------------------------------------------+

\ccRefPageBegin

%%RefPage: end of header, begin of main body
% +------------------------------------------------------------------------+
% +========================================================================+
%   Segment_circle_2
% +========================================================================+
\begin{ccRefClass}{Segment_circle_2<NT>} 
    
\ccDefinition A curve \ccStyle{c} of the class
    \ccClassTemplateName\ may be any one of the following: a line segment,
    a full circle or a circular arc.

    This class is designed to be used along with the
    \ccc{Arr_segment_circle_traits<NT>} traits class, and thus mainly
    offers the functionality needed for the traits.  

    \ccInclude{CGAL/Segment_circle_2.h}
    
\ccCreation
    \ccCreationVariable{arc}

    \ccConstructor{Segment_circle_2<NT> ();}
	{creates a default degenerate curve.}

    \ccConstructor{Segment_circle_2<NT> (Segment_2<NT> s);}
	{creates a line segment curve.}

    \ccConstructor{Segment_circle_2<NT> (Circle_2<NT> c);}
	{creates a full circle curve.}

    \ccConstructor{Segment_circle_2<NT> (Circle_2<NT> c, Point_2<NT> source, Point_2<NT> target);}
	{creates a circular arcs based on the given circle and going in a clockwise direction from the source to the taget. Both the source and the target must be on the circle.}


\ccOperations
    
    \ccMethod{Point_2<NT> source();}
	{returns the curve source.}

    \ccMethod{Point_2<NT> target();}
	{returns the curve source.}

    \ccMethod{bool is_segment();}
	{check whether the curve is a segment.}

    \ccMethod{Segment_2<NT> segment();}
	{convers the curve to a segment.
       	\ccPrecond{\ccc{arc} is a segment.}}

    \ccMethod{bool is_circle();}
	{check whether the curve is a circle or a circular arc.}
    
    \ccMethod{Segment_2<NT> segment();}
	{returns the base circle.
       	\ccPrecond{\ccc{arc} is a full circle or a circular arc.}}

\end{ccRefClass}

%%%%%%%%%%%%%%%%%%%%%%%%%%%%%%%%%%%

% +------------------------------------------------------------------------+
%%RefPage: end of main body, begin of footer
\ccRefPageEnd
% EOF
% +------------------------------------------------------------------------+
