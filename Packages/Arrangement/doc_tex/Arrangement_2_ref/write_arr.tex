% +------------------------------------------------------------------------+
% | Reference manual page: write_arr.tex (Arrangement)
% +------------------------------------------------------------------------+
% | 
% | Package: arr (Arrangement)
% | 
% +------------------------------------------------------------------------+

\ccRefPageBegin

%%RefPage: end of header, begin of main body
% +------------------------------------------------------------------------+

%write_arr

\begin{ccRefFunction}{write_arr}

\ccCreationVariable{write_arr}

\ccDefinition

The function \ccRefName\ sends an object of the template parameter
type \ccc{Arrangement} to an output stream. The function is intended
for cases where the writer object is not the default one, i.e. for
user defined writer classes which aim at writing arrangements with
additional information or aim at special format. For a default use
simply use the insertor operator (\ccc{<<}) of the output stream as in
\ccc{cout << arr}.

Reading is done using the \ccc{read} method of
\ccc{Arrangement_2<Dcel,Traits,Base_node>}.

\ccInclude{CGAL/IO/write_arr.h}

\ccFunction{template <Arrangement, Writer> 
            void 
            write_arr(Arrangement arr, 
            Writer& writer, 
            ostream& o); }
         {The function sends an arrangement object to an output
            stream. The function uses the methods defined within the
            \ccStyle{Writer} class, and prints \ccStyle{arr} which is instance of
            \ccc{Arrangement} in the format discussed in
            \ccRefPage{Arr_IO_format}. The printing is done by sending
            the arrangement object to the output stream \ccc{o}. }

\ccHeading{Requirements}
\begin{enumerate}
    \item    \ccc{Arrangement}  
             should be of type \ccc{Arrangement_2<Dcel,Traits,Base_node>}.
    \item    \ccc{Writer}  
             should be of type \ccc{Arr_file_writer<Arrangement>}.
    
   \item    The output stream should support the insertor operator (\ccc{<<}) 
            for the following types
            \begin{itemize}
                \item \ccc{Arrangement::Point_2},
                \item \ccc{Arrangement::Curve_2}, 
            \end{itemize}
\end{enumerate}

\end{ccRefFunction} % writer_arr

% +------------------------------------------------------------------------+
%%RefPage: end of main body, begin of footer
\ccRefPageEnd
% EOF
% +------------------------------------------------------------------------+











%%% Local Variables: 
%%% mode: latex
%%% TeX-master: "Pm_file_scanner"
%%% End: 
