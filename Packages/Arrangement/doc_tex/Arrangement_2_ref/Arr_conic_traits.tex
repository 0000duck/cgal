% +------------------------------------------------------------------------+
% | Reference manual page: Arr_conic_traits.tex (Arrangement)
% +------------------------------------------------------------------------+
% | 
% | Package: arr (Arrangement_2)
% | 
% +------------------------------------------------------------------------+

\ccRefPageBegin

%%RefPage: end of header, begin of main body
% +------------------------------------------------------------------------+
% +========================================================================+
%  Arr_conic_traits_2<Int_kernel, Alg_kernel>
% +========================================================================+
\begin{ccRefClass}{Arr_conic_traits_2<Int_kernel, Alg_kernel>}

\ccDefinition
    The class \ccc{Arr_conic_traits_2<Int_kernel, Alg_kernel>}
    is a model of the \ccc{ArrangementTraits_2} concept and serves as a 
    traits class for finite segments of algebraic curves of degree 2 at most
    (conic curves).
    Both template parameters must be models of the a CGAL Kernel 
    concept: The first, \ccc{Int_kernel}, should be templated with an 
    unbounded integer number-type, which will be used to represent the
    coefficients of the conic curves. The second parameter, \ccc{Alg_kernel},
    should be templated with a number type constructible from an unbounded
    integer, and that in addition to the arithmetic operations, also support
    the square-root operation in an exact manner, as well as the extraction
    of the $k$th-largest root of a given polynomial with integral coefficients.
    This number type is used to represent the coordinates of the arrangement
    vertices. The recommended choice for the two template parameters is
    \ccStyle{Cartesian<CORE::BigInt>} as \ccc{Int_kernel} and
    \ccStyle{Cartesian<CORE::Expr>} as \ccc{Alg_kernel}.

    A general conic curve $C$ is the locus of all points $(x,y)$ satisfying the
    equation $rx^2 + sy^2 + txy + ux + vy + w = 0$, where:
    \begin{itemize}
    \item If $4rs - t^2 > 0$, $C$ is an ellipse. A special case occurs when
          $r = s \neq 0$ and $t = 0$, where $C$ becomes a circle.
    \item If $4rs - t^2 < 0$, $C$ is a hyperbola.
    \item If $4rs - t^2 = 0$, $C$ is a parabola. A special case occurs when
          $r = s = t = 0$, where $C$ becomes a line. 
    \end{itemize}
    In our representation, all conic coefficients (namely $r,s,t,u,v,w$) must
    be integers.

    A bounded conic arc is defined as one of the following:
    \begin{itemize}
    \item A full ellipse (or a circle) $C$.
    \item The supporting conic curve $C$ and two endpoints $s$ and $t$ (the 
    source and target, respectively). In this case one should determine if
    we proceed from $s$ to $t$ in a clockwise or in a counterclockwise
    direction.
    \end{itemize}

    A very useful subset of the set of conic arcs are line segments and 
    circular arcs, since arrangements of circular arcs and line segments have 
    some interesting applications (e.g. offsetting polygons, motion planning 
    for a disc robot, etc.). Circular arcs and line segment are simpler 
    objects and can be dealt with more efficiently than arbitrary arcs. 
    For these reasons, it is possible to construct conic arcs from segments 
    and from circles. Using these constructors is highly recommended: It is 
    more straightforward and also speeds up the arrangement construction.

\ccInclude{CGAL/Arr_conic_traits_2.h}

\ccIsModel
    \ccc{ArrangementTraits_2} \lcTex{(\ccRefPage{ArrangementTraits_2}).}

% The Conic_arc_2 class:
%
\subsection*{Class \ccc{Arr_conic_traits_2<Int_kernel,Alg_kernel>::Curve_2}}

The \ccc{Curve_2} class nested within the conic arcs' traits can represent
arbitrary conic arcs and support their construction in various ways. The copy 
and default constructor as well as the assignment and equality operators are 
provided for conic arcs. In addition, an \ccc{operator<<} for the curves is 
defined for standard output streams and \ccc{Window_stream}.

\begin{ccClass}{Arr_conic_traits_2<Int_kernel,Alg_kernel>::Curve_2}
    \ccCreation
    \ccCreationVariable{cv}

\ccConstructor{Curve_2(const Int_kernel::Segment_2& seg);}
  {constructs an arc from the line segment \ccc{seg}.}

\ccConstructor{Curve_2(const Int_kernel::Circle_2& circ);}
  {constructs a circular arc that corresponds to the full circle \ccc{circ}
   (notice the circle center has integer coordinates and its squared radius
   is also an integer).}

\ccConstructor{Curve_2(const Int_kernel::Circle_2& circ,
                       const Orientation& orient,
                       const Alg_kernel::Point_2& ps,
                       const Alg_kernel::Point_2& pt);}
  {constructs a circular arc supported by the circle \ccc{circ} and with 
   \ccc{ps} and \ccc{pt} as its endpoints, going in the given orientation.
   \ccPrecond{\ccc{ps} and \ccc{pt} both lie on the circle \ccc{circ}}.}

\ccConstructor{Curve_2(const Int_kernel::Point_2& p1,
                       const Int_kernel::Point_2& p2,
	               const Int_kernel::Point_2& p3);}
  {constructs a circular arc going from \ccc{p1}, the source, through \ccc{p2}
   to \ccc{p3}, the target (notice all points have integer coordinates).
   The orientation of the arc is determined automatically.
   \ccPrecond{The three points are not collinear}.}

\ccConstructor{Curve_2(const Int_kernel::FT& r, 
                       const Int_kernel::FT& s, 
		       const Int_kernel::FT& t, 
                       const Int_kernel::FT& u, 
                       const Int_kernel::FT& v, 
                       const Int_kernel::FT& w);}
  {constructs a conic arc that corresponds to the full curve $rx^2 + sy^2 + 
   txy + ux + vy + w = 0$.
   \ccPrecond{The given curve is an ellipse, that is $4rs - t^2 > 0$.}}

\ccConstructor{Curve_2 (const Int_kernel::FT& r, 
                        const Int_kernel::FT& s, 
			const Int_kernel::FT& t, 
                        const Int_kernel::FT& u, 
                        const Int_kernel::FT& v, 
                        const Int_kernel::FT& w,
	                const Orientation& orient,
                        const Alg_kernel::Point_2& ps,
                        const Alg_kernel::Point_2& pt);}
  {constructs a conic arc supported by the curve $rx^2 + sy^2 + txy + ux + 
   vy + w = 0$ with \ccc{ps} as its source and \ccc{pt} as its target, going
   in the given orientation.
   \ccPrecond{\ccc{ps} and \ccc{pt} both satisfy the equation of the supporting
              curve and define a bounded segment from it (e.g. in case of a
              hyperbolic arc, both point should be located on the same branch
              of the hyperbola.}}

\ccConstructor{Curve_2(const Int_kernel::FT& r, 
                       const Int_kernel::FT& s, 
		       const Int_kernel::FT& t, 
                       const Int_kernel::FT& u, 
                       const Int_kernel::FT& v, 
                       const Int_kernel::FT& w,
	               const Int_kernel::Line_2& l);}
  {constructs a conic arc supported by the curve $rx^2 + sy^2 + 
   txy + ux + vy + w = 0$, and whose endpoints are the intersection
   points of the curve and the line $l$. We take the portion of the curve
   that lies in the positive half-plane defined by $l$.
   \ccPrecond{The line $l$ intersects the conic curve at two points.}}

\ccConstructor{Curve_2(const Int_kernel::Point_2& p1,
                       const Int_kernel::Point_2& p2,
                       const Int_kernel::Point_2& p3,
                       const Int_kernel::Point_2& p4,
	               const Int_kernel::Point_2& p5);}
  {constructs a circular arc going from \ccc{p1}, the source, through \ccc{p2},
   \ccc{p3} and \ccc{p4} to \ccc{p5}, the target (notice all points have
   integer coordinates). The orientation of the arc is determined
   automatically.
   \ccPrecond{No three points of the five are not collinear}.}

\ccConstructor{Curve_2 (const Int_kernel::FT& r, 
                        const Int_kernel::FT& s, 
			const Int_kernel::FT& t, 
                        const Int_kernel::FT& u, 
                        const Int_kernel::FT& v, 
                        const Int_kernel::FT& w,
                        const Orientation& orient,
                        const Alg_kernel::Point_2& ps,
                        const Int_kernel::FT& r1, 
                        const Int_kernel::FT& s1, 
			const Int_kernel::FT& t1, 
                        const Int_kernel::FT& u1, 
                        const Int_kernel::FT& v1, 
                        const Int_kernel::FT& w1,
                        const Alg_kernel::Point_2& pt,
                        const Int_kernel::FT& r2, 
                        const Int_kernel::FT& s2, 
			const Int_kernel::FT& t2, 
                        const Int_kernel::FT& u2, 
                        const Int_kernel::FT& v2, 
                        const Int_kernel::FT& w2);}
  {constructs a conic are supported by the curve $rx^2 + sy^2 + txy + ux + 
   vy + w = 0$ with \ccc{ps} as its source and \ccc{pt} as its target, going
   in the given orientation.
   In this case \ccc{ps} and  \ccc{pt} are just approximations of the 
   endpoints, and their exact values are given implicitly, as the 
   intersections of the supporting curve with $r_{1}x^2 + s_{1}y^2 + t_{1}xy +
    u_{1}x + v_{1}y + w_{1} = 0$ and $r_{2}x^2 + s_{2}y^2 + t_{2}xy +
    u_{2}x + v_{2}y + w_{2} = 0$, respectively.
   \ccPrecond{The two curves specifying the endpoints really intersect with
              the supporting conic curve.}}

\ccOperations

\ccMethod{bool is_full_conic() const;}
  {returns whether the arc represents a full conic curve (a full ellipse).}

\ccMethod{const Alg_kernel::Point_2& source() const;}
  {returns the source point of the arc.}

\ccMethod{const Alg_kernel::Point_2& target() const;}
  {returns the target point of the arc.}

\ccMethod{bool is_x_monotone() const;}{returns \ccc{true} if the arc is
   $x$-monotone, \ccc{false} otherwise.}

\ccMethod{bool is_segment() const;}
  {returns whether the arc is a line segment.}

\ccMethod{bool is_circular() const;}
  {returns whether the arc is supported by a circle.} 

\ccInclude{CGAL/IO/Conic_arc_2_Window_stream.h}

\end{ccClass}

\end{ccRefClass} % Arr_conic_traits

% +------------------------------------------------------------------------+
%%RefPage: end of main body, begin of footer
\ccRefPageEnd
% EOF
% +------------------------------------------------------------------------+
