% +------------------------------------------------------------------------+
% | Reference manual page: Arr_segment_circle_traits.tex (Arrangement)
% +------------------------------------------------------------------------+
% | 
% | Package: arr (Arrangement_2)
% | 
% +------------------------------------------------------------------------+

\ccRefPageBegin

%%RefPage: end of header, begin of main body
% +------------------------------------------------------------------------+
% +========================================================================+
%  Arr_segment_circle_traits<NT>
% +========================================================================+
\begin{ccRefClass}{Arr_segment_circle_traits<NT>}

\ccDefinition
    The class \ccc{Arr_segment_circle_traits<NT>} is a traits class
    for line segments and circular arcs. The template parameter \ccc{NT}
    corresponds to the number type used in the inner representation.
    An additional requirement for this number type, beyond the regular
    requirements of a CGAL number type, is that it has a \ccc{sqrt(NT r)}
    function defined for it. In order for the traits to work robustly
    an exact number type such as \ccc{leda_real} should be used.

    The \ccc{Point} type is \ccc{Point_2<Cartesian<NT> >}. The  
    \ccc{Curve} and \ccc{X_curve}
    representation is a \ccc{Segment_circle_2<Cartesian<NT> >}, a class
    based on the \ccc{Conic_2<Cartesian<NT> >}
    with two additional points corresponding to the source and target points:
    in this manner it is possible to conveniently represent line segments,
    full circles and circular arc.
    The copy and default constructor as well as the
    assignment and equality operators are provided for the curves. 
    An \ccc{operator<<} for the curves is defined for
    standard output streams and \ccc{Window_stream}.

\ccInclude{CGAL/Arr_segment_circle_traits.h}

\ccInclude{CGAL/IO/Segment_circle_Window_stream.h}

\ccIsModel
    \ccc{ArrangementTraits_2} \lcTex{(\ccRefPage{ArrangementTraits_2}).}

\end{ccRefClass} % Arr_segment_circle_traits

% +------------------------------------------------------------------------+
%%RefPage: end of main body, begin of footer
\ccRefPageEnd
% EOF
% +------------------------------------------------------------------------+
