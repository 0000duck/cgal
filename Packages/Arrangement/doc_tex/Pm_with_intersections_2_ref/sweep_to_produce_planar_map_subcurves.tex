% +------------------------------------------------------------------------+
% | Reference manual page: Pm_bounding_box_traits.tex (Planar_map)
% +------------------------------------------------------------------------+
% | 
% | Package: pm (Planar_map_2)
% | 
% +------------------------------------------------------------------------+

\ccRefPageBegin

%%RefPage: end of header, begin of main body
% +------------------------------------------------------------------------+

%sweep_to_produce_planar_map_subcurves
\begin{ccRefFunction}{sweep_to_produce_planar_map_subcurves}

\ccCreationVariable{sweep_to_produce_planar_map_subcurves}

\ccDefinition
\ccc{sweep_to_produce_planar_map_subcurves} is a global function defined for reporting 
all the disjoint interior subcurves induced by the input curves by the sweep line technique.
The reported subcurves will be inserted to a container defined by the user.  

\ccInclude{CGAL/Sweep_curves_to_subcurves.h}

\def\ccLongParamLayout{\ccTrue} 

\ccGlobalFunction{
   template <class Curve_iterator, class Traits, class Container>
   void 
   sweep_to_produce_planar_map_subcurves(
   Curve_iterator curves_begin, 
   Curve_iterator curves_end, 
   Traits& traits, 
   Container &subcurves, 
   bool overlapping = false);
}

reports all the disjoint interior subcurves induced by the curves given in the range 
\ccStyle{[curves_begin, curves_end)}. 
The parameter \ccStyle{subcurves} is a container which will evantually hold all the reported subcurves.
\ccc{sweep_to_produce_planar_map_subcurve} supports curves which are interior intersect, overlap or non x-monotone, and hence the user may provide his curves as input without any resriction of the kind mentioned above. 
When providing overlapping curves as input users can make a choice by the 
boolean parameter \ccStyle{overlapping} whether the output they will get will contain each 
overlapping subcurve only once or that the reported subcurves will have each 
overlapping subcurve per each original curve it created of. 
The default choice provides the first possibility.

\ccSeeAlso
\ccc{sweep_to_construct_planar_map} (\ccRefPage{CGAL::sweep_to_construct_planar_map})

\ccImplementation 
The implementation uses the container $map$ defined in STL for implementing 
the event queue and the status line.
The implementation also keeps per each input curve an associated container of all 
its intersection points ordered from left to right, and also keeps per each event point all 
its outcoming curves.   
The complexity of this algorithm is $O(nlogn + k)$ where $n$ is the number of the input curves 
and $k$ is the number of intersection points induced by these curves.


\label{ssec:example3_sweep}
The following example is a demonstration of the usage of the \ccc {Sweep line} algorithm. 
In this example we shall present the usage of the sweep line technique 
when inserting the resulting disjoint interior subcurves to a container. 
The input segments are just like as in ~\ref{ssec:example1_sweep} , 
and the result we are interested with is only reporting the sub segments 
obtained by the \ccc{Sweep line} algorithm.

\ccIncludeExampleCode{Sweep_line/example3.C}

And the output of the program looks like this:

\ccIncludeExampleCode{Sweep_line/example3.cout}

\end{ccRefFunction} % sweep_to_produce_planar_map_subcurves

% +------------------------------------------------------------------------+
%%RefPage: end of main body, begin of footer
\ccRefPageEnd
% EOF
% +------------------------------------------------------------------------+











%%% Local Variables: 
%%% mode: latex
%%% TeX-master: "Pm_file_scanner"
%%% End: 
