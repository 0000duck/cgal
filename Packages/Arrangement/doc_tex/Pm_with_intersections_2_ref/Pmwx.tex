% +------------------------------------------------------------------------+
% | Reference manual page: Pmwx.tex (Planar map with intersections)
% +------------------------------------------------------------------------+
% |
% | Package: arr (Planar_map_with_intersections_2)
% |
% +------------------------------------------------------------------------+

\ccRefPageBegin

%%RefPage: end of header, begin of main body
% +------------------------------------------------------------------------+
% +========================================================================+
%   Planar_map_with_intersections_2
% +========================================================================+
\begin{ccRefClass}{Planar_map_with_intersections_2<Planar_map>}

\ccDefinition An object \ccStyle{pmwx} of the class
    \ccClassTemplateName\ represents the planar
    subdivision induced by a set of curves that are possibly intersecting.
    This class extends the class \ccStyle{Planar_map_2<Dcel,Traits>}
    which handles $x$-monotone pairwise interior disjoint
    curves.  %The class have the following internal typedefs:
    An object of the class \ccClassTemplateName\ is also a planar map and
    therefore the types, constants, and traversal iterators and operations remain
    the same.

    \ccInclude{CGAL/Pm_with_intersections.h}

\ccInheritsFrom
    \ccStyle{CGAL::Planar_map_2<Dcel,Traits>}

    Note that the modifying insertion functions override the
    inherited functions. The insertion functions of \ccRefName\ handle
    general curves, including non-$x$-monotone curves, pairwise overlapping,
    or pairwise intersecting curves, and are slower than their counterparts
    of the \ccStyle{CGAL::Planar_map_2<Dcel,Traits>} base class.
    $X$-monotone curves that are pairwise disjoint in their interiors can be
    inserted efficiently through a subset of insertion functions listed below.
    These functions begin with the prefix \ccc{non_intersecting}.

\ccTypes

\ccNestedType{Pmwx_change_notification}{}

\ccCreation
\ccCreationVariable{pmwx}

\ccConstructor{Planar_map_with_intersections_2<Planar_map>();}
  {creates an empty planar map with the default point location strategy.}

\ccConstructor{Planar_map_with_intersections_2<Planar_map>(Pm_point_location_base<Planar_map> *pl);}
  {creates an empty planar map with \ccc{*pl} as the point location strategy.}

\ccConstructor{Planar_map_with_intersections_2<Planar_map>(const Traits& tr, Pm_point_location_base<Planar_map> *pl);}
  {creates an empty planar map with \ccc{tr} as the traits class and \ccc{*pl} as the point location strategy.}

\ccOperations

\ccMethod{Halfedge_handle insert(const X_monotone_curve_2& cv,
                                 Pmwx_change_notification *en = NULL);}
         {inserts the $x$-monotone curve \ccStyle{cv} into the planar map.
	 Returns the last inserted halfedge whose target points to the
	 target point of \ccStyle{cv}.
	 \ccPrecond{\ccc{cv} is not equivalent to a point.}}

  \ccMethod{template <X_curve_iterator>
            Halfedge_handle insert(const X_curve_iterator & begin, 
                                   const X_curve_iterator & end,
				   Pmwx_change_notification *en = NULL);}
  {inserts the given range of curves into the map in an aggregated manner,
   using the sweep-line algorithm. \ccc{begin} and \ccc{end} are input 
   iterators that point to the first curve and past-the-end curve of the range
   respectively. \ccc{insert()} returns a handle to a new halfedge directed 
   in the same way as the last curve in the range.
   \ccPrecond{no curve in the given range is equivalent to a point.}}

\begin{ccAdvanced}

\ccHeading{Non Intersecting Insertion Functions}

The following functions do not check for intersections. They simply
call the \ccc{Planar_map_2} corresponding insert functions. Therefore,
they are more efficient than their counterpart insert-functions. but should
be used with care ({\em see also} \ref{PM_sec:pm} for details including the 
full list of preconditions).

\ccMethod{Halfedge_handle
          non_intersecting_insert(const X_monotone_curve_2 & cv,
                                  Change_notification * en = NULL);}
         {inserts the curve \ccc{cv} into the
	 map. \ccc{non_intersecting_insert()} returns a handle to a new
	 halfedge directed in the same way as the curve \ccc{cv}.
	 \ccPrecond{\ccc{cv} has no proper intersection with any curve of the
	 map.}}

\ccMethod{template <class X_monotone_curve_2_iterator>
          Halfedge_iterator
          non_intersecting_insert(const X_monotone_curve_2_iterator & begin,
                                  const X_monotone_curve_2_iterator & end,
                                  Change_notification * en = NULL);}
  {iterates through a given range of curves, inserting the curves into
  the map. \ccc{begin} and \ccc{end} are input iterators that point to
  the first curve and past-the-end curve of the range
  respectively. \ccc{non_intersecting_insert()} returns a handle to a
  new halfedge directed in the same way as the last curve in the range.
  \ccPrecond{the curves of the map do not intersect the any of the curves in 
  the given range in its interior, and all curves in the range are disjoint
  in their interior.}}

\ccMethod{Halfedge_handle
        non_intersecting_insert_in_face_interior
                                            (const X_monotone_curve_2 & cv, 
                                             Face_handle f,
                                             Change_notification * en = NULL);}
  {inserts the curve \ccc{cv} as a new inner component of the face
  \ccc{f}. \ccc{non_intersecting_insert_in_face_interior()} returns a
  handle to a new halfedge directed in the same way as \ccc{cv}.
  \ccPrecond{\ccc{cv} has no proper intersection with any curve of the
  map.}}

\ccMethod{Halfedge_handle
        non_intersecting_insert_from_vertex(const X_monotone_curve_2 & cv, 
                                            Halfedge_handle h,
                                            Change_notification * en = NULL);}
  {inserts the curve \ccc{cv} into the map. One endpoint of \ccc{cv}
  is the point of the target vertex, \ccc{v}, of the given halfedge
  \ccc{h}. \ccc{non_intersecting_insert_from_vertex} returns a handle
  to a new halfedge that has  \ccc{v} as its source vertex. The
  returened twin halfedge is inserted immediately after \ccc{h} in the
  circular list of halfedges that share the same target vertex
  \ccc{v}.
  \ccPrecond{\ccc{cv} has no proper intersection with any curve of the
  map.}}

\ccMethod{Halfedge_handle
        non_intersecting_insert_at_vertices(const X_monotone_curve_2 & cv, 
                                            Halfedge_handle h1, 
                                            Halfedge_handle h2,
                                            Change_notification * en = NULL);}
  {inserts the curve \ccc{cv} into the map. The two endpoints of
  \ccc{cv} are held by the two target vertices \ccc{v1} and \ccc{v2}
  of \ccc{h1} and \ccc{h2}
  respectively. \ccc{non_intersecting_insert_at_vertices()} returns a
  handle to a new halfedge, that has \ccc{v1} and \ccc{v2} as its
  source and target vertices respectively. The returened halfedge is
  inserted immediately after \ccc{h1} in the  circular list of
  halfedges that share the same target vertex \ccc{v1}. Its twin
  halfedge is inserted immediately after \ccc{h2} in the circular list
  of halfedges that share the same target vertex \ccc{v2}.
  \ccPrecond{\ccc{cv} has no proper intersection with any curve of the
  map. }
  }

\ccMethod{Halfedge_handle
        non_intersecting_insert_from_vertex(const X_monotone_curve_2 & cv, 
                                            Vertex_handle v,
                                            Change_notification * en = NULL);}
  {inserts the curve \ccc{cv} into the map. One endpoint of \ccc{cv}
  is the point of the given vertex \ccc{v}, that is already in the
  map. \ccc{non_intersecting_insert_from_vertex} returns a handle to a
  new halfedge that has \ccc{v} as its source vertex.
  \ccPrecond{\ccc{cv} has no proper intersection with any curve of the
  map. }
  }
  
\ccMethod{Halfedge_handle
        non_intersecting_insert_at_vertices(const X_monotone_curve_2 & cv, 
                                            Vertex_handle v1, 
                                            Vertex_handle v2,
                                            Change_notification * en = NULL);}
  {inserts the curve \ccc{cv} into the map. The two endpoints of
  \ccc{cv} are held by the two given vertices \ccc{v1} and \ccc{v2}
  respectively, that are already in the
  map. \ccc{non_intersecting_insert_at_vertices()} returns a handle to
  a new halfedge, that has \ccc{v1} and \ccc{v2} as its source and
  target vertices respectively.
  \ccPrecond{\ccc{cv} has no proper intersection with any curve of the
  map. }
  }

\end{ccAdvanced}

\end{ccRefClass}

\ccRefPageEnd
