% +------------------------------------------------------------------------+
% | Reference manual page: Pmwx.tex (Planar map with intersections)
% +------------------------------------------------------------------------+
% |
% | Package: arr (Planar_map_with_intersections_2)
% |
% +------------------------------------------------------------------------+

\ccRefPageBegin

%%RefPage: end of header, begin of main body
% +------------------------------------------------------------------------+
% +========================================================================+
%   Planar_map_with_intersections_2
% +========================================================================+
\begin{ccRefClass}{Planar_map_with_intersections_2<Planar_map>}

\ccDefinition An object \ccStyle{pmwx} of the class
    \ccClassTemplateName\ represents the planar
    subdivision induced by a set of curves that are possibly intersecting.
    This class extends the class \ccStyle{Planar_map_2<Dcel,Traits>}
    which handles $x$-monotone pairwise interior disjoint
    curves.  %The class have the following internal typedefs:
    An object of the class \ccClassTemplateName\ is also a planar map and
    therefore the types, constants, and traversal iterators and operations remain
    the same.

    \ccInclude{CGAL/Pm_with_intersections.h}

\ccInheritsFrom
    \ccStyle{CGAL::Planar_map_2<Dcel,Traits>}

    The fact that the modifying insert functions override the
    inherited functions (the expected behaviour) should be noted. The
    insert functions of \ccRefName\ are designed to handle possibly
    intersecting, non-$x$-monotone or overlapping curves and are
    slower.  For non-intersecting, $x$-monotone curves are intserted
    one can use the set of insertion functions that begin with the
    prefix \ccc{non_intersecting}.
%\comment{iddo: need to be virtual in planar map}.

\ccTypes

    \ccNestedType{Pmwx_change_notification}{}

\ccCreation
    \ccCreationVariable{pmwx}

    \ccConstructor{Planar_map_with_intersections_2<Planar_map>();}
     {creates an empty planar map with the default point location strategy.}

    \ccConstructor{Planar_map_with_intersections_2<Planar_map>(Pm_point_location_base<Planar_map> *pl);}
    {creates an empty planar map with \ccc{*pl} as the point location strategy.}

    \ccConstructor{Planar_map_with_intersections_2<Planar_map>(const Traits& tr, Pm_point_location_base<Planar_map> *pl);}
    {creates an empty planar map with \ccc{tr} as the traits class and \ccc{*pl} as the point location strategy.}

\ccOperations


    \ccMethod{Halfedge_handle insert(const X_curve& cv, Pmwx_change_notification *en = NULL);}
      {inserts the $x$-monotone curve \ccStyle{cv} into the planar map.
      Returns the last inserted halfedge whose target points to the
      target point of \ccStyle{cv}.
       \ccPrecond{\ccc{cv} is not equivalent to a point. }
       }

    \ccMethod{Halfedge_handle insert_from_vertex(const X_curve& cv, Vertex_handle src,
      Pmwx_change_notification *en = NULL);}
      {inserts  the $x$-monotone curve \ccStyle{cv}, for
       which the source vertex, \ccStyle{src}, is already in the
       arrangement.
       Returns the last inserted halfedge whose target points to the
       target point of \ccStyle{cv}.
       This function is more efficient than the \ccc{insert} function,
       since it does not invoke a point location query for the
       endpoint \ccc{src}. Therefore,
       whenever possible, it is recommended to use this one rather
       than \ccc{insert}.
       \ccPrecond{\ccc{src} holds a point which is the same as \ccc{cv's}
       source.}
       \ccPrecond{\ccc{cv} is not equivalent to a point. }
       }

\begin{ccAdvanced}

  \ccMethod{Halfedge_handle non_intersecting_insert(const X_curve&
  cv);}
  {invokes the planar map implementation of \ccc{insert}
  function. Since this function does not check for intersections it is
  more efficient than \ccc{insert}.
  \ccPrecond{\ccc{cv} has no proper
  intersection with any curve of the map. } }

  \ccMethod{Halfedge_handle non_intersecting_insert_from_vertex(const X_curve& cv, Vertex_handle src, bool
  source);}
  {invokes the planar map implementation of
  \ccc{insert_from_vertex} function. Since this function does not
  check for intersections it is more efficient than \ccc{insert_from_vertex}.
       \ccPrecond{\ccc{cv} has no intersection in the interior with any curve of the map. }
       \ccPrecond{\ccc{src} holds a point thatmn                      is the same as \ccc{cv}'s source.}
       }

\end{ccAdvanced}


%    The following operations have the same functionality as their
%    counterparts in the planar map.
%
%    \ccMethod{Vertex_iterator vertices_begin();}{}
%    \ccMethod{Vertex_iterator vertices_end();}{}
%    \ccMethod{Size number_of_vertices();}{}
%    \ccMethod{Halfedge_iterator halfedges_begin();}{}
%    \ccMethod{Halfedge_iterator halfedges_end();}{}
%    \ccMethod{Size number_of_halfedges();}{}
%    \ccMethod{Face_iterator faces_begin();}{}
%    \ccMethod{Face_iterator faces_end();}{}
%    \ccMethod{Size number_of_faces();}{}
%
%    \ccMethod{Halfedge_handle split_edge(Halfedge_handle e, const X_curve& c1,
%   const X_curve& c2);}{}
%    \ccMethod{Halfedge_handle locate(const Point& p, Locate_type& lt);}{}
%    \ccMethod{Halfedge_handle vertical_ray_shoot(const Point& p, Locate_type& lt, bool up);}{}
%
%    \ccMethod{Face_handle unbounded_face();}{}
%
%    \ccMethod{bool is_valid(bool verbose=false);}{}


\end{ccRefClass}


% +------------------------------------------------------------------------+
%%RefPage: end of main body, begin of footer
\ccRefPageEnd
% EOF
% +------------------------------------------------------------------------+
