% +------------------------------------------------------------------------+
% | Reference manual page: Arrangement_2_traits.tex (Arrangement)
% +------------------------------------------------------------------------+
% |
% | Package: arr (Arrangement_2)
% |
% +------------------------------------------------------------------------+

\ccRefPageBegin

%%RefPage: end of header, begin of main body
% +------------------------------------------------------------------------+
% +========================================================================+
%   Arrangement_2_traits
% +========================================================================+
\begin{ccRefConcept}{Arrangement_2_traits}

% TODO (shai) verify this definition is correct
\ccDefinition
   The \ccRefName\ concept is a refinement of the \ccc{Planar_map_with_intersections_traits}
   concept \lcTex{(\ccRefPage{Planar_map_traits})}.
   The requirements of the latter
   are requirements for the former.
   The additional requirements are the following types and methods.
   %The \ccStyle{Point} can be implemented for non-linear curves as the
   %intersection of 2 curves.

\ccTypes
\ccNestedType{Point}{same as the planar map traits' \ccc{Point}.}
\ccNestedType{X_curve}{same as the planar map traits' \ccc{X_curve}.}
\ccNestedType{Curve}{curve type, this type should be {\it syntactically}
identical to the \ccc{X_curve} type (i.e., it should {\em be} the same type).
However, {\it semantically} it differs from the \ccc{X_curve} in that it
is not necessarily $x$-monotone. In the following methods when we
require an \ccc{X_curve} as a parameter, we assume that the curve
is $x$-monotone.}

\ccCreationVariable{arr_traits}
\ccMethod{bool is_x_monotone(const Curve& c);}{returns true if $c$ is
an $x$-monotone curve.}

\ccMethod{void make_x_monotone(const Curve& cv, list<X_curve>& l);}{cuts
\ccc{cv} into $x$-monotone subcurves and stores them in $l$.
The order in which they are inserted into $l$ defines their order in the
hierarchy tree.}

\ccHasModels
  \ccc{Arr_segment_exact_traits<R>}\\
  \ccc{Arr_leda_segment_exact_traits}\\
  \ccc{Arr_circles_real_traits<NT>}\\
  \ccc{Arr_polyline_traits<R, Container>}

\end{ccRefConcept}

% +------------------------------------------------------------------------+
%%RefPage: end of main body, begin of footer
\ccRefPageEnd
% EOF
% +------------------------------------------------------------------------+
