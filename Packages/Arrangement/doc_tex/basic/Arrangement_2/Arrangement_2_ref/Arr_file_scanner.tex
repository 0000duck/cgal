% +------------------------------------------------------------------------+
% | Reference manual page: Arr_file_scanner(Arrangement)
% +------------------------------------------------------------------------+
% | 
% | Package: arr (Arrangement)
% | 
% +------------------------------------------------------------------------+

\ccRefPageBegin

%%RefPage: end of header, begin of main body
% +------------------------------------------------------------------------+

%Arr_file_scanner
\begin{ccRefClass}{Arr_file_scanner<Arrangement>}

\ccCreationVariable{arr_file_scanner}

\ccDefinition
This class contains functions for reading the \ccc{Arrangement} components from a given input stream. 
Users have to make sure that methods for reading \ccStyle{Point} and \ccStyle{Curve} defined in  
the \ccc{Arrangement traits} class are provided when using the input stream they choose to work with.
This class is templated with \ccc{Arrangement}, and hence the user can work with \ccRefName \/ 
with its own \ccc{Arrangement}. 
 
\ccInheritsFrom
\ccc{Pm_file_scanner}

%The following requirement catalog lists the primitives, i.e., types, member functions
%etc., that must be defined for any \ccc{Planar map} file scanner class that can be used to parameterize the reading function of \ccc{Planar map}. 

\ccTypes

\ccNestedType{Traits}{Representation of a bounded curve.}
\ccNestedType{Point}{A type to hold the location of a vertex in the plane.}
\ccNestedType{X_curve}{A type to hold an $x$-monotone curve in the plane.}
\ccNestedType{Curve}{A type to hold a curve in the plane.}

\ccNestedType{Curve_node}{The \ccc{Arrangement} curve node type.}
\ccNestedType{Subcurve_node}{The \ccc{Arrangement} subcurve node type.}
\ccNestedType{Edge_node}{The \ccc{Arrangement} edge node type.}

\ccCreation
    
\ccConstructor{Arr_file_scanner (istream& in)} 
{A constructor for a given input stream.}
\ccConstructor{Arr_file_scanner (istream& in, const File_header& header)} 
{A constructor for a given input stream and File header.}

The operation \ccRefName \/ supports are all the operations \ccc{Pm_file_scanner} supported plus the following:

\ccOperations

\ccMethod{void scan_Curve_node(Curve_node* cn);}
{scans a curve node from the input stream.}

\ccMethod{void scan_Subcurve_node(Subcurve_node* scn);}
{scans a subcurve node from the input stream.}

\ccMethod{void scan_Edge_node(Edge_node* en);}
{scans an edge node from the input stream.}


\end{ccRefClass} % Arr_file_scanner

% +------------------------------------------------------------------------+
%%RefPage: end of main body, begin of footer
\ccRefPageEnd
% EOF
% +------------------------------------------------------------------------+











%%% Local Variables: 
%%% mode: latex
%%% TeX-master: "draw_pm"
%%% End: 
