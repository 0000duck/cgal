% +------------------------------------------------------------------------+
% | Reference manual chapter: intro.tex (Arrangement)
% +------------------------------------------------------------------------+
% | 
% | Package: arr (Arrangement_2)
% | 
% +------------------------------------------------------------------------+

%+---------------------------------------------------------------------------80
%| update log
%|
%| 21 Jun 2000 - Shai Hirsch
%|    Separated from arr.tex (See previous changes in change log there).
%|     
%+---------------------------------------------------------------------------80


% +========================================================================+
%   Introduction
% +========================================================================+
\clearpage
\section{Arrangement Reference Pages}

\subsection*{Introduction}
\ccRefLabel{arr_ref_intro}

The arrangement class holds a planar map and a hierarchy tree.
Vertices, halfedges and faces of the arrangement derive from
those of the planar map (with the additional functionality of
the arrangement),
in the same way that the vertices, halfedges and faces of the topological map
class derive from those of the Dcel class
%in a way similar to what is done between the
%topological map and the Dcel class
(see Chapter\ref{I1_ChapterTopologicalMap}).


The hierarchy tree is implemented using the \ccc{In_place_list}
class (see the chapter on STL Extensions in the Support Library Manual). Every level of a curve
hierarchy is a list of tree nodes.The \ccc{Curve_node} and 
\ccc{Edge_node} of 
the hierarchy derive from \ccc{Subcurve_node}. This enables the
polymorphic structure of the tree. The \ccc{Subcurve_node} is
derived from the \ccc{Base_node} which is a template parameter
of the arrangement. This enables the addition of attributes to the nodes
of the hierarchy tree by adding them inside the \ccc{Base_node}.
%Since the arrangement holds a planar map its traits requirements
%are a superset of the requirements from the planar map traits.


\subsection*{Arrangement Dcel}
The arrangement is parametrized with the interface class \ccc{Dcel}.
The \ccc{Dcel} is a container class that defines the underlying
combinatorial data structure used by the planar map. We define the
concept \ccc{Arrangement_2_Dcel} \lcTex{(\ccRefPage{Arrangement_2_Dcel})}
where the requirements for a \ccc{Dcel} class are defined.

As part of the \ccc{Dcel} we define requirements for its vertex, halfedge and 
face constituents. If we consider these constituents as (informally) 
subconcepts of the \ccc{Dcel} concept then we have the following models for
its constituents. 
%-----------------------------------------------------

The \ccStyle{Arr_2_vertex_base<Point>} is a model for the 
Dcel vertex subconcept,
the \mbox{\ccc{Arr_2_halfedge_base<Base_node>}} is a
model for the Dcel halfedge subconcept
and the \mbox{\ccc{Arr_2_face_base}} is a
model for the Dcel face subconcept. 

\ccInclude{CGAL/Arr_2_bases.h}


The \ccStyle{Arr_2_default_dcel<Traits>} is a model of the Dcel concept
described above. Its template parameter is the traits class.
It is a wrapper for \ccStyle{Pm_dcel} instantiated 
with \mbox{\ccc{Arr_2_vertex_base<Traits::Point>}},\\
\mbox{\ccc{Arr_2_halfedge_base<Arr_base_node<Traits::X_curve> >}} and 
\mbox{\ccStyle{Arr_2_face_base}}.

\ccInclude{CGAL/Arr_2_default_dcel.h}

\subsection*{Models for an Arrangement Traits Class} 
%-----------------------------------------------
We supply a traits class for segments that use the types
and predicates of the CGAL kernel, a traits class for
circle arcs as well as a traits class for polyline curves.
Since the requirements of the arrangement traits
are a superset of the requirements of the planar map traits,
any traits class that works with arrangements can work with
planar maps as well.

The different traits classes are described at the end of this chapter.

\begin{ccTexOnly}

\subsection*{Concepts}
\ccRefConceptPage{Arrangement_base_node}\\
\ccRefConceptPage{Arrangement_2_Dcel}\\
\ccRefConceptPage{Arrangement_2_traits}

\subsection*{Classes}
\ccRefIdfierPage{CGAL::Arrangement_2<Dcel,Traits,Base_node>}\\
\ccRefIdfierPage{CGAL::Arr_base_node<Curve>}\\
\ccRefIdfierPage{CGAL::Arrangement_2<Dcel,Traits,Base_node>::Subcurve_node}\\
\ccRefIdfierPage{CGAL::Arrangement_2<Dcel,Traits,Base_node>::Curve_node}\\
\ccRefIdfierPage{CGAL::Arrangement_2<Dcel,Traits,Base_node>::Edge_node}\\
\ccRefIdfierPage{CGAL::Arrangement_2<Dcel,Traits,Base_node>::Halfedge}\\

\ccRefIdfierPage{CGAL::Arr_segment_exact_traits<R>}\\
\ccRefIdfierPage{CGAL::Arr_leda_segment_exact_traits}\\
\ccRefIdfierPage{CGAL::Arr_circles_real_traits<NT>}\\
\ccRefIdfierPage{CGAL::Arr_polyline_traits<R, Container>}

\end{ccTexOnly}    
