% +------------------------------------------------------------------------+
% | Reference manual page: Arr_leda_segment_exact_traits.tex (Arrangement)
% +------------------------------------------------------------------------+
% | 
% | Package: arr (Arrangement_2)
% | 
% +------------------------------------------------------------------------+

\ccRefPageBegin

%%RefPage: end of header, begin of main body
% +------------------------------------------------------------------------+
% +========================================================================+
%  Arr_leda_segement_exact_traits
% +========================================================================+
\renewcommand{\ccRefPageBegin}{\begin{ccAdvanced}}
\renewcommand{\ccRefPageEnd}{\end{ccAdvanced}}
\begin{ccRefClass}{Arr_leda_segment_exact_traits}

%\begin{ccAdvanced} %--------------------\

\ccDefinition
    We supply a traits class that uses LEDA's rational kernel ---
    \ccRefName.\/ The class handles
    finite line segments in the plane
    and uses the LEDA rational kernel (\ccStyle{Point} is of type 
    \ccStyle{leda_rat_point}, \ccc{Curve} and \ccc{X_curve} are of type
    \ccStyle{leda_rat_segment}). The predicates make use of LEDA's primitive
    predicates (e.g., orientation) that are implemented using floating point
    ``filters'' \cite{fv-sayee-96} which speed up the use of exact 
    computations.
    Normalization of the intersection points is performed which speeds up
    computations since the bitlength of the points stays relatively small.

\ccInclude{ CGAL/Arr_leda_segment_exact_traits.h}

\ccIsModel
    \ccc{Arrangement_2_traits} \lcTex{(\ccRefPage{Arrangement_2_traits}).}

%\end{ccAdvanced} %----------------------/

\end{ccRefClass} 
\renewcommand{\ccRefPageBegin}{}
\renewcommand{\ccRefPageEnd}{}

% +------------------------------------------------------------------------+
%%RefPage: end of main body, begin of footer
\ccRefPageEnd
% EOF
% +------------------------------------------------------------------------+
