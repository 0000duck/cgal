% +------------------------------------------------------------------------+
% | Reference manual page: Arr_segment_exact_traits.tex (Arrangement)
% +------------------------------------------------------------------------+
% | 
% | Package: arr (Arrangement_2)
% | 
% +------------------------------------------------------------------------+

\ccRefPageBegin

%%RefPage: end of header, begin of main body
% +------------------------------------------------------------------------+
% +========================================================================+
%  Arr_segment_traits_2<Kernel>
% +========================================================================+

\begin{ccRefClass}{Arr_segment_traits_2<Kernel>}

\ccDefinition
   The class \ccRefName\ is a traits class for handling arrangements of line
   segments that uses the geometric kernel intensively. 
   The \ccc{Arr_segment_traits_2<Kernel>::Curve_2} type (as well
   as the \ccc{Arr_segment_traits_2<Kernel>::X_monotone_curve_2} type) is
   actually the \ccc{Kernel::Segment_2} type, while the various predicates
   and constructions supported by the traits class actually wrap the
   kernel implementations.

   The \ccc{Kernel} must use a number type that supports exact computations
   with rational numbers, such as \ccc{Quotient<Gmpz>} or \ccc{leda_rational}. 
   Other inexact representations can be used at the user's own risk.

\ccInclude{CGAL/Arr_segment_traits_2.h}

\ccIsModel
    \ccc{ArrangementTraits_2} \lcTex{(\ccRefPage{ArrangementTraits_2}).}


\end{ccRefClass} % Arr_segment_traits_2<Kernel>

% +------------------------------------------------------------------------+
%%RefPage: end of main body, begin of footer
\ccRefPageEnd
% EOF
% +------------------------------------------------------------------------+
