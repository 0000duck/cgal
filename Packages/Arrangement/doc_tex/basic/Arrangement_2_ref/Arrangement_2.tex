% +------------------------------------------------------------------------+
% | Reference manual page: Arrangement_2.tex (Arrangement)
% +------------------------------------------------------------------------+
% | 
% | Package: arr (Arrangement_2)
% | 
% +------------------------------------------------------------------------+

\ccRefPageBegin

%%RefPage: end of header, begin of main body
% +------------------------------------------------------------------------+
% +========================================================================+
%   Arrangement_2
% +========================================================================+
\begin{ccRefClass}{Arrangement_2<Dcel,Traits,Base_node>} 
    
\ccDefinition An object \ccStyle{arr} of the class
    \ccClassTemplateName\ represents the planar
    subdivision induced by a set of curves that are not necessarily 
    $x$-monotone and possibly intersecting. 
    This class extends
    the class \ccStyle{Planar_map_2<Dcel,Traits>}
    which handles $x$-monotone pairwise interior disjoint
    curves.  %The class have the following internal typedefs:  

    This class also provides additional methods to traverse the
    resulting planar-map along curves inserted into the arrangement
    (note that a single curve in the arrangement might correspond to
    several edges of the planar map).

    \ccInclude{CGAL/Arrangement_2.h}

    %\ccInheritsFrom 
    %\ccStyle{Planar_map_2<Dcel,Traits>}
    The base class of \ccStyle{Halfedge} has additional
    properties apart from the ones defined for planar maps. 

    %The modifying functions \ccStyle{insert} 
    %overwrite the inherited functions 
%\comment{iddo: need to be virtual in planar map}.
    
\ccTypes

    \ccNestedType{Curve_node}{represents a curve inserted into the 
       arrangement, which possibly intersects other curves in the arrangement,
       and is not necessarily $x$-monotone.}

    \ccNestedType{Subcurve_node}{represents a subcurve of the curve inserted 
       into the arrangement. It is created by invoking a user defined 
       \ccc{split_function} on the original curve, or on another curve. By
       default a subcurve is created from the original curve by invoking
       the function \ccc{Traits::make_x_monotone()}.}

    \ccNestedType{Edge_node}{represents a subcurve of the curve inserted 
       into the arrangement which is created by an intersection with
       another curve. It holds a curve which is disjoint in its interior
       from the other curves in the arrangement, and holds a cross reference
       to the \ccc{Halfedge} in the planar map representing it.}



    \ccNestedType{Vertex}{represents a vertex of the planar map induced by the
       arrangement.}
    \ccNestedType{Halfedge}{represents a halfedge of the planar map induced by
       the arrangement. It has additional methods, see class definition
       of \ccc{Arrangement_2<Dcel,Traits,Base_node>::Halfedge}
       \lcTex{(\ccRefPage{Arr_Halfedge})}.}
    \ccNestedType{Face}{represents a face of the planar map induced by
       the arrangement.}



    The following handles, iterators and circulators have appropriate
    constant
%\footnote{This means that for every \ccc{iterator} or
%       \ccc{circulator} there is a \ccc{const_iterator} and 
%       \ccc{const_circulator}
%       counterpart. See ~\cite{ms-strg-96} for a discussion of constant versus
%       mutable iterator types.} 
    counterparts.
    The mutable types are assignable to their
    constant counterparts. All circulators are assignable to the 
    \ccStyle{Halfedge_iterator}. The iterators are assignable to the
    respective handle types. Wherever the handles appear in function parameter
    lists, the appropriate iterator can be used as well.


    \ccNestedType{Curve_iterator}{A bidirectional iterator over 
       all Curve nodes of the arrangement. Its value-type is 
       \ccStyle{Curve_node}.}

    \ccNestedType{Subcurve_iterator} {a bidirectional iterator over all
       the subcurves which were created from the same original curve.
       Its value-type is 
       \ccStyle{Subcurve_node}.}
    
    \ccNestedType{Edge_iterator} {a bidirectional iterator over all
       the \ccc{Edge_nodes} which were created from the same original curve.
       Its value-type is 
       \ccStyle{Edge_node}.}




    \ccNestedType{Vertex_handle}{handle to vertex.}
     
    \ccNestedType{Halfedge_handle}{handle to halfedge.}
    
    \ccNestedType{Face_iterator}{handle to face}

    \ccNestedType{Vertex_iterator}{a bidirectional iterator over the
       vertices of the arrangement. Its value-type is
       \ccStyle{Vertex}.}
     
    \ccNestedType{Halfedge_iterator}{a bidirectional iterator over the
       halfedges of the arrangement.  Its value-type is \ccStyle{Halfedge}.}
    
    \ccNestedType{Face_iterator}{a bidirectional iterator over the
       faces of the arrangement.  Its value-type is \ccStyle{Face}.}
    
    \ccNestedType{Ccb_halfedge_circulator}{a forward circulator over the
       edges of a CCB (connected components of the boundary).  Its
       value-type is \ccStyle{Halfedge}.}
    
    \ccNestedType{Halfedge_around_vertex_circulator}{a forward circulator over
       the half-edges which have  the vertex as their source.  
       The half-edges are traversed in
       their clockwise order around the vertex. Its value-type is
       \ccStyle{Halfedge}.}
    
    \ccNestedType{Holes_iterator}{a bidirectional iterator to
       traverse all the holes (i.e., inner CCBs) of a face 
       (\ccStyle{Holes_iterator++} is the next hole in the face).
       Its value type is
       \ccStyle{Ccb_halfedge_circulator}.}

    \ccNestedType{Overlap_circulator}{a bidirectional circulator over
       the overlapping edge nodes that correspond to a single pair of
       halfedges. Its value-type is \ccStyle{Edge_node} and it can be
       cast to an \ccc{Edge_iterator}.}
  
    \ccTypedef{typedef Planar_map_2<Dcel,Traits> Planar_map;}{the planar map
       type of the arrangement.}

    \ccNestedType{Locate_type}{same as the planar map locate type.}

    \ccTypedef{typedef typename Traits::Point Point;}{the point type
       of the arrangement}
    \ccTypedef{typedef typename Traits::X_curve X_curve;}{
       the x-monotone curve type of the arrangmenet}
    \ccTypedef{typedef typename Traits::Curve Curve;}{the curve type
      of the arrangement}
    
\ccCreation
    \ccCreationVariable{arr}

    \ccConstructor{Arrangement_2<Dcel,Traits,Base_node>();}{create an empty arrangement with the default point location strategy.}

    \ccConstructor{Arrangement_2<Dcel,Traits,Base_node>(Pm_point_location_base<Planar_map> *pl);}{create an empty arrangement with \ccc{*pl} as the point location strategy.}

    \ccConstructor{Arrangement_2<Dcel,Traits,Base_node>(const Traits& tr, Pm_point_location_base<Planar_map> *pl);}{create an empty arrangement with \ccc{tr} as the traits class and \ccc{*pl} as the point location strategy.}

    %\ccConstructor{Arrangement_2<Dcel,Traits,Base_node>( list<Traits::X_curve> &l );}
    %{create the arrangement of the curves of \ccStyle{l}}

    %\ccStyle{tempalte <class InputIterator>} \\
    %\ccConstructor{Arrangement_2<Dcel,Traits,Base_node>( InputIterator begin, 
     %                              InputIterator end );} 
     %           {create the arrangement of the curves from a container iterated by
     %           \ccStyle{InputIterator} from  \ccStyle{begin} to \ccStyle{end}
     %           \ccPrecond{the container contains \ccStyle{Traits::X_curve} objects.}       
     %           }
    
\ccOperations

    \ccMethod{void clear();}
     {clears the arrangement.}

     \ccMethod{Curve_iterator insert(const Curve& cv);}{inserts the curve
       \ccStyle{cv} into the arrangement. It uses the
       \ccc{Traits::make_x_monotone}
       function to cut the curve into $x$-monotone curves and inserts
       them in the hierarchy tree. If the arrangement is in {\em update}
       mode (the default mode) then the curves are inserted into the
       planar map.
       \ccPrecond{\ccc{cv} is not equivalent to a point. }
       }
    

    \ccMethod{Curve_iterator insert_from_vertex( const X_curve & cv,
       Vertex_handle src);}{inserts  the curve
       \ccStyle{cv}, for 
       which the source point, \ccStyle{src}, is already in the
       arrangement. 
       The curve is cut into $x$-monotone curves, as explained for 
       the \ccc{insert} function.
       This function is more efficient than the \ccc{insert} function, 
       since it does not invoke a point location query for the 
       endpoint \ccc{src}. Therefore, 
       whenever possible, it is recommended to use this one rather
       than \ccc{insert}. 
       \ccPrecond{\ccc{src} holds a point which is the same as \ccc{cv's} 
       source.}
       \ccPrecond{\ccc{cv} is not equivalent to a point. }
       }

\begin{ccAdvanced}

        %\ccTypedef{typedef typename Traits::SPLIT_FUNC SPLIT_FUNC;}{the
        %   splitting function to be defined by the user.}
        %\comment{This is not implemented yet in this way, currently we
        %   just use a pointer to function.}
        
        \ccMethod{template <class F_iterator>
           Curve_iterator insert(const Curve &cv, 
           F_iterator F_begin, F_iterator F_end);}{insert the curve \ccc{cv}
           into the arrangement. \ccc{F_begin} and \ccc{F_end} are iterators
           that refer to {\it split-function} pointers
           (or pointers to split-function objects).
           The insert function uses the split-functions
           to cut the curve into subcurves and inserts
           them in the hierarchy tree (every split-function 
           creates one level of the hierarchy tree).
           If the arrangement is in \ccc{update}
           mode (the default mode) then the curves are inserted into the
           planar map.}

        The split-function should get as input a curve and a reference to
        a list of curves (where the subcurves will be stored), its return
        value is \ccc{void}. See Section~\ref{ssec:example4}
        and Section~\ref{ssec:example5} for examples
        of user-defined functions.
        The functions are called with the following
        syntax:

        \ccc{(*(*F_begin))(const Curve& c, list<Curve>& l);}

\end{ccAdvanced}

\begin{ccAdvanced}
\ccHeading{Reading Arrangement}

\ccMethod{bool read (istream &in);}
{reading \ccRefName\ from a given input stream.
 The input stream should support the extractor operator (\ccc{>>}) for the \ccStyle{Point} 
 and \ccStyle{Curve} types of \ccc{Arrangement}. } 

\ccStyle{template <class Scanner>}
\ccMethod{bool read (istream &in, Scanner& scanner);}
{reading \ccRefName\ from a given input stream when taking the scanner class as a parameter.
 The input stream should support the extractor operator (\ccc{>>}) for the \ccStyle{Point}
 and \ccStyle{Curve} types of \ccc{Arrangement}. }

\end{ccAdvanced}    

    %\ccMethod{void insert( list<Traits::X_curve> &l );} {insert the
    %   curves of \ccStyle{l} into the arrangement \comment{iddo: use plane sweep?}}

    %\ccStyle{tempalte <class InputIterator>} \\
    %\ccMethod{void insert( InputIterator begin, InputIterator end );} 
    %            {insert the curves from a container iterated by
    %            \ccStyle{InputIterator} from  \ccStyle{begin} to \ccStyle{end}
    %            into the arrangement.
    %            \ccPrecond{the container contains \\ 
    %               \ccStyle{Traits::X_curve} objects.}}

%\comment{iddo: maybe add the function:} 
%\ccMethod{insert_super_curve(list<Traits::X_curve> &l );}{}
%\comment{iddo: which will insert the X_curves with prev/next to the super-curve from which they originated.}

    \ccMethod{void set_update(bool u);}{sets update mode to \ccc{u}. If
       update mode is true then every curve inserted into the
       arrangement is also inserted into the planar map induced by it (i.e.,
       it is intersected with the rest of the curves in the arrangement).
       If update mode is false then the curve is only split into $x$-monotone
       curves and inserted into the hierarchy tree (without the edge level
       which is not constructed). If update mode is changed from false to true
       then the planar map is updated with all the curves inserted into
       the arrangement since the last time update was true.}



%    \ccMethod{Arrangement_2<Dcel,Traits,Base_node>::Curve_iterator 
%curve_node_begin();}
    \ccMethod{Curve_iterator curve_node_begin();}
    { returns the begin iterator of the curves (i.e., the
       curves that were inserted to \ccVar).}

%    \ccMethod{Arrangement_2 <Dcel,Traits,Base_node>::Curve_iterator 
%curve_node_end();}
    \ccMethod{Curve_iterator curve_node_end();}
    { returns the past-the-end iterator of the curves.}

    \ccMethod{void remove_curve(Curve_iterator cit);}
    {removes the curve and all it's subcurves from the arrangement.}


    The following operations have the same functionality as their
    counterparts in the planar map.

    \ccMethod{Vertex_iterator vertices_begin();}{}
    \ccMethod{Vertex_iterator vertices_end();}{}
    \ccMethod{Size number_of_vertices();}{}
    \ccMethod{Halfedge_iterator halfedges_begin();}{}
    \ccMethod{Halfedge_iterator halfedges_end();}{}
    \ccMethod{Size number_of_halfedges();}{}
    \ccMethod{Face_iterator faces_begin();}{}
    \ccMethod{Face_iterator faces_end();}{}
    \ccMethod{Size number_of_faces();}{}

    \ccMethod{Halfedge_handle split_edge(Halfedge_handle e, const X_curve& c1, 
   const X_curve& c2);}{}
    \ccMethod{Halfedge_handle locate(const Point& p, Locate_type& lt);}{}
    \ccMethod{Halfedge_handle vertical_ray_shoot(const Point& p, Locate_type& lt, bool up);}{}

    \ccMethod{Face_handle unbounded_face();}{}

    \ccMethod{bool is_valid(bool verbose=false);}{
      checks the validity of many features of the arrangements. First,
      checks the validity of the arrangement's planar map. Then, checks
      for the correctness of the hierarchy tree. Finally, some geometric
      properties are verified, e.g., each edge's end point is the start
      point of its next edge. if the \ccc{verbose}parameter
      is \ccc{true} then a verbose description of the vallidity check
      is sent to the output.}
    
    The planar map operations: \ccc{merge_edge} and \ccc{remove_edge} are
    not implemented in the arrangement level,
    since they can cause inconsistencies
    between the hierarchy tree and the planar map.
    If it is necessary for the users to perform these operations
    they can still achieve this. For example, if the users need to remove an
    edge they can remove the curve and insert the two subcurves that are
    created after the edge is removed,
    instead of the original curve.

\end{ccRefClass}

%%%%%%%%%%%%%%%%%%%%%%%%%%%%%%%%%%%
%types: Subcurve_node, Curve_node, Edge_node

% +------------------------------------------------------------------------+
%%RefPage: end of main body, begin of footer
\ccRefPageEnd
% EOF
% +------------------------------------------------------------------------+
