% +------------------------------------------------------------------------+
% | Reference manual page: Arr_polyline_traits.tex (Arrangement)
% +------------------------------------------------------------------------+
% | 
% | Package: arr (Arrangement_2)
% | 
% +------------------------------------------------------------------------+

\ccRefPageBegin

%%RefPage: end of header, begin of main body
% +------------------------------------------------------------------------+
% +========================================================================+
%  Arr_polyline_traits<NT>
% +========================================================================+
\begin{ccRefClass}{Arr_polyline_traits_2<Kernel, Container>}

\ccDefinition
    The class \ccRefName\ is a traits class for polyline curves, which are
    piecewise linear curves --- chains of segments, where each two neighboring
    segments in the chain share a common endpoint. The traits class uses
    functionality of the \ccc{Kernel} to handle the segments that comprise the
    polyline curve.

    The template parameter \ccc{Container} corresponds to the container type 
    for holding the polyline curve points. It must be a model of the 
    \ccc{Reversible Container} concept and it must be templated with
    \ccc{Kernel::Point_2}. \ccc{Container} has a default value which is
    \ccc{std::vector<Kernel::Point_2>}. A container that stores $n$ points 
    labeled $p_{0},p_{1},\ldots,p_{n-1}$ represents a curve with $n-1$ 
    segments, where the $k$th segment is defined by the endpoints
    $[p_{k-1},p_{k}]$. The first point in the container is considered as the 
    source point of the polyline while the last point is considered as its 
    target.

    The \ccc{Kernel} must use a number type that supports exact computations
    with rational numbers, such as \ccc{Quotient<Gmpz>} or 
    \ccc{leda_rational}. Other inexact representations can be used at the 
    user's own risk.

\ccInclude{CGAL/Arr_polyline_traits_2.h}

\ccIsModel
    \ccc{ArrangementTraits_2} \lcTex{(\ccRefPage{ArrangementTraits_2}).}

\end{ccRefClass}


% EOF -----------------------------------------------------------------------80

% +------------------------------------------------------------------------+
%%RefPage: end of main body, begin of footer
\ccRefPageEnd
% EOF
% +------------------------------------------------------------------------+
