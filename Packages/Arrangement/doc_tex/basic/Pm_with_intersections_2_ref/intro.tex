% +------------------------------------------------------------------------+
% | Reference manual chapter: intro.tex (Pmwx)
% +------------------------------------------------------------------------+
% |
% | Package: arr (Arrangement_2)
% |
% +------------------------------------------------------------------------+

%+---------------------------------------------------------------------------80
%| update log
%|
%| 04 May 2001  - Eyal Flato
%|    Created from arr documentation
%|
%+---------------------------------------------------------------------------80


% +========================================================================+
%   Introduction
% +========================================================================+
%\clearpage
%\section{Reference Pages for 2D Planar Maps of Intersecting Curves}

\chapter{2D Planar Maps of Intersecting Curves}

\subsection*{Introduction}
\ccRefLabel{pmwx_ref_intro}

The \ccc{Planar Map with Intersections} package extends the
functionality of the \ccc{Planar Map} package. Moreover, the class
\ccc{Planar_map_with_intersections_2} extends the class
\ccc{Planar_map_2}. The \ccc{Planar Map with Intersections} package
supports insertion functions that handle non-$x$-monotone,
intersecting and overlapping curves among the curves of
the planar map.

%\subsection*{Planar Map with Intersections Traits Class}
%-----------------------------------------------
Since the planar map with intersections has an additional
functionality of handling non $x$-monotone and intersecting curves
it needs additional functionality of the traits class. We describe
the \ccc{PlanarMapwithIntersectionsTraits_2} concept following in
this chapter. 

%Nevertheless, we do not have specific implementation
%of this concept, and we use models of the \ccc{ArrangementTraits_2}
%concept, which is a refinement thereof instead (see
%Chapter~\ref{I1_ChapterArrangement}).

% \subsection*{Sweep line:}
%   The construction of a planar map can be done using a sweep line
%   technique rather than iteratively, inserting one curve after another.
%   The sweep line algorithm performs better, but is done statically and
%   therefore requires the list of curves in advance.  Due to the sweep
%   line technique, inserting curves which are interior intersect,
%   overlap or non $x$-monotone is supported. The interface is given in
%   chapter ~\ref{I1_ChapterPlanarMap}.

\begin{ccTexOnly}

\subsection*{Concepts}
\ccRefConceptPage{PlanarMapWithIntersectionsTraits_2}\\
\ccRefConceptPage{PlanarMapWithIntersectionsChangeNotification_2}\\

\subsection*{Classes}
\ccRefIdfierPage{CGAL::Planar_map_with_intersections_2<Planar_map>}%\\


% \subsection*{Functions}
% \ccRefIdfierPage{CGAL::sweep_to_construct_planar_map}\\
% \ccRefIdfierPage{CGAL::sweep_to_produce_planar_map_subcurves}\\

\end{ccTexOnly}












