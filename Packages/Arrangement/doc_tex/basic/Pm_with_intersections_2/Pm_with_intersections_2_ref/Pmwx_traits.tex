% +------------------------------------------------------------------------+
% | Reference manual page: Pmwx_2_traits.tex (Arrangement)
% +------------------------------------------------------------------------+
% |
% | Package: arr (Planar_map_with_intersections_2)
% |
% +------------------------------------------------------------------------+

\ccRefPageBegin

%%RefPage: end of header, begin of main body
% +------------------------------------------------------------------------+
% +========================================================================+
%   Arrangement_2_traits
% +========================================================================+
\begin{ccRefConcept}{Planar_Map_with_Intersections_Traits}

\ccDefinition
   The \ccRefName\ concept is a refinement of the \ccc{Planar Map Traits}
   concept \lcTex{(\ccRefPage{Planar_map_traits})}.
   The requirements of the latter
   are requirements for the former.
   The additional requirements are the following types and methods.
   %The \ccStyle{Point} can be implemented for non-linear curves as the
   %intersection of 2 curves.

\ccTypes
\ccNestedType{Point}{same as the planar map traits' \ccc{Point}.}
\ccNestedType{X_curve}{same as the planar map traits' \ccc{X_curve}.}

\ccCreationVariable{pmwx_traits}

\ccMethod{Point point_reflect_in_x_and_y(const Point& pt);}{returns the
\ccc{Point}
that is the the reflection of \ccc{pt} in the $x$ and $y$ axes. E.g.,
the point $(2,2)$ will be reflected as $(-2,-2)$. }

\ccMethod{X_curve curve_reflect_in_x_and_y(const X_curve& cv);}{returns the \ccc{X_curve}
that is the the reflection of \ccc{cv} in the $x$ and $y$ axes. E.g., the line segment $((2,2),(3,3))$ will be reflected
as $((-2,-2),(-3,-3))$. }

\ccMethod{void curve_split(const X_curve& cv, X_curve& c1, X_curve& c2,
                   const Point& split_pt);}{splits $cv$ at \ccc{split_pt}
and assigns the resulting two curves to \ccc{c1} and \ccc{c2}.
\ccPrecond{\ccc{split_pt} is on \ccc{cv}.}
\ccPrecond{\ccc{split_pt} is not an endpoint of \ccc{cv}.}
}

\ccMethod{bool do_intersect_to_right(const X_curve& c1, const X_curve& c2,
                             const Point& pt);}{returns $true$ if \ccc{c1}
and \ccc{c2} intersect to the right of the point \ccc{pt}.
Intersection to the right of \ccc{pt} is defined as
an intersection which is lexicographically strictly to the right of \ccc{pt}
(i.e., an intersection above or to the right of
\ccc{pt} but {\em not} on \ccc{pt}).}

\ccMethod{bool nearest_intersection_to_right(const X_curve& c1,
                                      const X_curve& c2,
                                      const Point& pt,
                                      Point& p1, Point& p2);}
{finds the
nearest intersection point to the right of \ccc{pt}. Nearest is defined as the
one lexicographically closest to \ccc{pt} not including \ccc{pt} itself.
If the intersection of \ccc{c1} and \ccc{c2} to the
right of \ccc{pt} is an \ccc{X_curve} (i.e., there is an overlapping subcurve
to the right of \ccc{pt}),
then \ccc{p1} will hold the source point of the intersection and
\ccc{p2} will hold its target point. Otherwise \ccc{p1} and \ccc{p2} are
identical. In the case of an overlap we regard the intersection
to the right as the open curve between \ccc{p1} and \ccc{p2}. This means that
if the overlapping subcurve contains \ccc{pt} either \ccc{p1} or \ccc{p2} will
be equal to \ccc{pt}, this is the only case in which this can happen.
If \ccc{c1} and \ccc{c2} do not intersect to the right of \ccc{pt}
the function returns \ccc{false}.
%\ccPrecond{\ccc{do_intersect_to_right(c1,c2,pt) == true}}
}

%The intersection function is defined in such a way to enable dealing with
%degenerate cases when two curves inserted into the arrangement overlap.

\ccMethod{bool curves_overlap(const X_curve& cv1, const X_curve& cv2);}
{returns \ccc{true} if \ccc{cv1} and \ccc{cv2} overlap
in a one-dimensional subcurve (i.e., not in a finite number
of points), otherwise returns
\ccc{false}.}

\ccHasModels
  The package does not supply specific implementations of the planar
  map with intersections traits concept. However, since the
  \ccc{Arrangement Traits} concept is a refinment of the \ccRefName\
  concept, all of the models of the former are models of the latter.

  \ccc{Arr_segment_exact_traits<R>}\\
  \ccc{Arr_leda_segment_exact_traits}\\
  \ccc{Arr_circles_real_traits<NT>}\\
  \ccc{Arr_polyline_traits<R, Container>}

\end{ccRefConcept}

% +------------------------------------------------------------------------+
%%RefPage: end of main body, begin of footer
\ccRefPageEnd
% EOF
% +------------------------------------------------------------------------+
