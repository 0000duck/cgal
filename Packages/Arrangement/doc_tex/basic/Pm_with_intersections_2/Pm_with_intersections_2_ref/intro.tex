% +------------------------------------------------------------------------+
% | Reference manual chapter: intro.tex (Pmwx)
% +------------------------------------------------------------------------+
% |
% | Package: arr (Arrangement_2)
% |
% +------------------------------------------------------------------------+

%+---------------------------------------------------------------------------80
%| update log
%|
%| 04 May 2001  - Eyal Flato
%|    Created from arr documentation
%|
%+---------------------------------------------------------------------------80


% +========================================================================+
%   Introduction
% +========================================================================+
\clearpage
\section{Planar Map with Intersections Reference Pages}

\subsection*{Introduction}
\ccRefLabel{pmwx_ref_intro}

The planar map with intersections class extends the planar map
class. The planar map with intersections adds insertion functions
that handle intersections and overlapping among the curves of the
planar map. The basic types of the planar map (vertex, halfedge,
face, Dcel, etc.) are kept. \ccc{Planar_map_with_intersections_2}
is templated with the planar map class it derives from.

\subsection*{Planar Map with Intersections Traits Class}
%-----------------------------------------------
Since the planar map with intersections has an additional
functionality of intersecting curves it needs additional
functionality of the traits class. We describe the planar map with
intersections traits concept following in this chapter.
Nevertheless, we do not have specific implementation of this
concept, and we use the arrangement traits instead (see
Chapter~\ref{I1_ChapterArrangement}).

\subsection*{Sweep line}
%-------------------------------------------------
The \ccc{Planar Map With Intersections} contains a sweep lines algorithm. 
Given a collection of intersecting (possibly overlapping or non $x$-monotone)
curves the algorithm calculates the intersection points. As a result, the
collection of pairwise interior disjoint subcurves can be produced, or,
alternatively, a subdivion is being built. 
This subdivion can be either of the type \ccc{Planar Map With Intersections} or of the 
type \ccc{Planar map}.However, the \ccStyle{Sweep line} is a part of 
\ccStyle {Planar Map With Intersections} package since 
the traits used here are those defined in that package, due to calcluations as finding 
intersections, overlapping and $x$-monotony. 
Using sweep line technique, and building a \ccc{Planar Map With Intersections} is most 
usefull when users have a primar list of curves, 
but inserting intersecting curves each at a time will be needed afterward.
 
See the sweep line functions reference pages for implementation details.  
\ccc{sweep_to_construct_planar_map}\lcTex{
  (\ccRefPage{CGAL::sweep_to_construct_planar_map})},
 \ccc{sweep_to_produce_planar_map_subcurves}\lcTex{
  (\ccRefPage{CGAL::sweep_to_produce_planar_map_subcurves})},


\begin{ccAdvanced}
\subsection*{Change Notification}
%-----------------------------------------------
An insertion of an intersecting curve to a planar map may add
several halfedges and modify several features of the map (i.e.,
split halfedges, split faces). The relevant function of the change
notification instance, that is passed to the modification methods
of a planar map with intersections, is called after each such
modification. The change notification class is useful in many
cases. For example, one may add a color (or other extra data) to
any halfedge of a planar map. An insertion of a new curve can
split halfedges that were previously in the map. After such split
the color of the newly created halfedges should be updated
according to the original color of the split halfedge. One can do
this by implementing the \ccc{split_edge} function of the change
notification class. This function will be called after each split
of an halfedge in the map.


\subsubsection{Example of Change Notification}
\label{ssec:example2}
The following example demonstrates the usage of the change
notification concept during the construction of a
planar map out of three segments --- $(0,1)-(1,0)$, $(0,0)-(1,1)$
and $(0,1)-(1,1)$.
During the insertion we use \ccc{My_notification} instance to
output the internal process of the construction of the planar map.
We also count how many edges are in the map by incrementing a
counter each time an edge is added (\ccc{add_edge})or
split (\ccc{split_edge}).

\ccIncludeExampleCode{Pm_with_intersections/example2.C}

The output of the program looks like this:
\begin{verbatim}
inserting 0 1 1 0
add_edge
add_hole
inserting 0 0 1 1
split_edge
add_edge
add_edge
inserting 0 1 1 1
add_edge
split_face
Total number of edges 5
\end{verbatim}

\end{ccAdvanced}



\begin{ccTexOnly}

\subsection*{Concepts}
\ccRefConceptPage{Planar_Map_with_Intersections_Traits}\\
\ccRefConceptPage{Planar_Map_with_Intersections_Change_Notification}\\

\subsection*{Classes}
\ccRefIdfierPage{CGAL::Planar_map_with_intersections_2<Planar_map>}%\\


\subsection*{Functions}
\ccRefIdfierPage{CGAL::sweep_to_construct_planar_map<Curve_iterator,Planar_map>(Curve_iterator curves_begin, Curve_iterator curves_end, Planar_map &result);}\\
\ccRefIdfierPage{CGAL::sweep_to_produce_planar_map_subcurves<Curve_iterator,Traits,Containe>(Curve_iterator curves_begin, Curve_iterator curves_end, Container &subcurves, bool overlapping = false);}
\end{ccTexOnly}












