% +------------------------------------------------------------------------+
% | Reference manual page: Arr_polyline_traits.tex (Arrangement)
% +------------------------------------------------------------------------+
% | 
% | Package: arr (Arrangement_2)
% | 
% +------------------------------------------------------------------------+

\ccRefPageBegin

%%RefPage: end of header, begin of main body
% +------------------------------------------------------------------------+
% +========================================================================+
%  Arr_polyline_traits<NT>
% +========================================================================+
\begin{ccRefClass}{Arr_polyline_traits<R, Container>}

\ccDefinition
    The class \ccRefName\ is a traits class for polyline curves.  The
    representation type \ccc{R} is recommended to be an exact type
    such as \ccc{Cartesian<leda_rational>} or \ccc{Homogeneous<Gmpz>}
    or any other exact type, although other inexact representations
    can be used at the user's own risk.  The template parameter
    \ccc{Container} corresponds to the container type for holding the
    polyline curve points. It must be a model of the \ccc{Reversible
    Container} concept and it must be templated with
    \ccc{Point_2<R>}. \ccc{Container} has a default value which is
    \ccc{std::vector<Point_2<R> >}. The first point in the container
    is considered as the source point of the polyline while the last
    point is considered the target one.

%See example ***SHOW WHERE.
\ccInclude{CGAL/Arr_polyline_traits.h}

\ccIsModel
    \ccc{ArrangementTraits_2} \lcTex{(\ccRefPage{ArrangementTraits_2}).}

\end{ccRefClass}


% EOF -----------------------------------------------------------------------80

% +------------------------------------------------------------------------+
%%RefPage: end of main body, begin of footer
\ccRefPageEnd
% EOF
% +------------------------------------------------------------------------+
