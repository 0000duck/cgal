% +------------------------------------------------------------------------+
% | Reference manual page: Arr_circle_real_traits.tex (Arrangement)
% +------------------------------------------------------------------------+
% | 
% | Package: arr (Arrangement_2)
% | 
% +------------------------------------------------------------------------+

\ccRefPageBegin

%%RefPage: end of header, begin of main body
% +------------------------------------------------------------------------+
% +========================================================================+
%  Arr_circles_real_traits<NT>
% +========================================================================+
\begin{ccRefClass}{Arr_circles_real_traits<NT>}

\ccDefinition
    The class \ccc{Arr_circles_real_traits<NT>} is a traits class
    for circle arcs. The template parameter \ccc{NT} corresponds to
    the number type used in the inner representation. An additional requirement
    for this number type, beyond the regular requirements of
    a CGAL number type, is that it has
    a \ccc{sqrt(NT r)} function defined for it. In order for the traits
    to work robustly an exact number type such as \ccc{leda_real} should
    be used.

    The \ccc{Point} type is \ccc{Point_2<Cartesian<NT> >}. The  
    \ccc{Curve} and \ccc{X_curve}
    representation is a \ccc{Circle_2<Cartesian<NT> >}
    with two additional points corresponding to the source and target points.
    The copy and default constructor as well as the
    assignment and equality operators are provided for the curves. 
    An \ccc{operator<<} for the curves is defined for
    %standard output streams and 
    \ccc{Window_stream}.
    We define the nested class \ccc{Circ_curve} which implements these 
    circular curves, which is listed below with additional method and 
    constructors for its purpose.

\ccInclude{CGAL/Arr_circles_real_traits.h}

\ccInclude{CGAL/IO/Arr_circle_traits_Window_stream.h}

\ccIsModel
    \ccc{Arrangement_2_traits} \lcTex{(\ccRefPage{Arrangement_2_traits}).}

\subsection*{Class \ccc{Arr_circles_real_traits<NT>::Circ_curve}}

\begin{ccClass}{Circ_curve}
    \ccCreation
    \ccCreationVariable{cv}

\ccConstructor{Curve(const Circle_2<Cartesian<NT> >& c);}{constructs
   a curve with a CGAL circle (source and target are identical and
   initialized to the leftmost point on the circle).}
\ccConstructor{Curve(const NT& x, const NT& y, const NT& r2);}{constructs a
   curve corresponding to a full circle, oriented counterclockwise, with
   its center at \ccc{Point(x,y)} and \ccc{r2} as its squared radius.}
\ccConstructor{Curve(const NT& x, const NT& y, const NT& r2, const Point& src,
   const Point& trgt);}{constructs a
   curve corresponding to an arc segment of the circle that has
   its center at \ccc{Point(x,y)} and has \ccc{r2} as its squared radius.
   The arc is oriented counterclockwise and
   \ccc{src} and \ccc{trgt} correspond to its source and target
   points.}

\ccOperations

\ccMethod{const Circle_2<Cartesian<NT> >& circle() const;}{returns the circle the arc is on.} 
\ccMethod{const Point& source() const;}{returns the source point of the arc.}
\ccMethod{const Point& target() const;}{returns the target point of the arc.}
\ccMethod{bool is_x_monotone() const;}{returns \ccc{true} if the arc is
   $x$-monotone, \ccc{false} otherwise.}


%\comment{currently implemented for leda\_real only, and there are things to do
%in the window output operator and the make\_x\_monotone function.}
\end{ccClass}

\end{ccRefClass} % Arr_circles_real_traits

% +------------------------------------------------------------------------+
%%RefPage: end of main body, begin of footer
\ccRefPageEnd
% EOF
% +------------------------------------------------------------------------+
