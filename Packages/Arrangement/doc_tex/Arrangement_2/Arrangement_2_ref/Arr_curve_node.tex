% +------------------------------------------------------------------------+
% | Reference manual page: Arr_curve_node.tex (Arrangement)
% +------------------------------------------------------------------------+
% | 
% | Package: arr (Arrangement_2)
% | 
% +------------------------------------------------------------------------+

\ccRefPageBegin

%%RefPage: end of header, begin of main body
% +------------------------------------------------------------------------+
% +========================================================================+
%   Arrangement_2::Curve_node
% +========================================================================+
\begin{ccRefClass}[Arrangement_2<Dcel,Traits,Base_node>::]{Curve_node}

\ccDefinition An object $cn$ of the class
    \ccStyle{Curve_node} is a node holding the curve in the
    hierarchy tree, corresponding to a root of a hierarchy. It
    holds the original curve inserted into the arrangement.

\ccInheritsFrom
    \ccc{Subcurve_node}

    \ccCreationVariable{cn}
    
    In addition to the operations of \ccc{Subcurve_node} the following
    operations are implemented.

\ccOperations

    \ccMethod{bool is_edge_node();}{returns false;}

    \ccMethod{Subcurve_iterator level_begin(unsigned int i);}{returns the begin iterator
       to the $i$'th level of subcurves (for the first subcurve level $i=0$).
       \ccPrecond{ \ccc{i < cn.number_of_sc_levels()} }}

    \ccMethod{Subcurve_iterator level_end(unsigned int i);}{returns the
       past-the-end iterator
       to the $i$'th level of subcurves (for the first subcurve level $i=0$).
       \ccPrecond{ \ccc{i < cn.number_of_sc_levels()}} }
       
    \ccMethod{unsigned int number_of_sc_levels();}{returns the number of subcurve levels
       (not including the edge level and the curve level).}

    \ccMethod{Edge_iterator edges_begin();}{overrides the same function in
       the \ccc{Subcurve_node}, to gain efficiency.}
    \ccMethod{Edge_iterator edges_end();}{overrides the same function in
       the \ccc{Subcurve_node}, to gain efficiency.}


\end{ccRefClass}
% +------------------------------------------------------------------------+
%%RefPage: end of main body, begin of footer
\ccRefPageEnd
% EOF
% +------------------------------------------------------------------------+
