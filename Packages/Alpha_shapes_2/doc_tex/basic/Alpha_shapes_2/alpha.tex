% ======================================================================
%
% Copyright (c) 1999 The GALIA Consortium
%
% This software and related documentation is part of the
% Computational Geometry Algorithms Library (CGAL).
%
% Every use of CGAL requires a license. Licenses come in three kinds:
%
% - For academic research and teaching purposes, permission to use and
%   copy the software and its documentation is hereby granted free of  
%   charge, provided that
%   (1) it is not a component of a commercial product, and
%   (2) this notice appears in all copies of the software and
%       related documentation.
% - Development licenses grant access to the source code of the library 
%   to develop programs. These programs may be sold to other parties as 
%   executable code. To obtain a development license, please contact
%   the GALIA Consortium (at cgal@cs.uu.nl).
% - Commercialization licenses grant access to the source code and the
%   right to sell development licenses. To obtain a commercialization 
%   license, please contact the GALIA Consortium (at cgal@cs.uu.nl).
%
% This software and documentation is provided "as-is" and without
% warranty of any kind. In no event shall the CGAL Consortium be
% liable for any damage of any kind.
%
% The GALIA Consortium consists of Utrecht University (The Netherlands),
% ETH Zurich (Switzerland), Free University of Berlin (Germany),
% INRIA Sophia-Antipolis (France), Martin-Luther-University Halle-Wittenberg
% (Germany), Max-Planck-Institute Saarbrucken (Germany),
% and Tel-Aviv University (Israel).
%
% ----------------------------------------------------------------------
%
% package       : Alpha_shapes_2
% author(s)     : Tran Kai Frank DA <Frank.Da@sophia.inria.fr>
%
% coordinator   : INRIA Sophia-Antipolis (<Mariette.Yvinec@sophia.inria.fr>)
%
% ======================================================================

\RCSdef{\alphashapeRevision}{$Revision$}
\RCSdefDate{\alphashapeDate}{$Date$}

%----------------------------------------------------------------------

\chapter{Alpha-Shapes} \label{I1_ChapterAlphashapes}

\ccChapterSubTitle{\alphashapeRevision, \alphashapeDate}

\section{Introduction}

This chapter presents a framework for alpha shapes. The description is based on
the articles \cite{em-tdas-94,e-was-92}. Alpha shapes are
the generalization of the convex hull of a point set. Let $S$ be a finite set of
points in $\R^d$, $d = 2,3$ and $\alpha$ a parameter with $0 \leq \alpha \leq
\infty$. For $\alpha = \infty$, the $\alpha$-shape is the convex hull of $S$. As 
$\alpha$ decreases, the $\alpha$-shape shrinks and develops cavities, as soon as 
a sphere of radius $\sqrt{\alpha}$ can be put inside.
Finally, for $\alpha = 0$, the $\alpha$-shape is the set $S$ itself.

We distinguish two versions of alpha shapes, one is based on the Delaunay
triangulation and the other on its generalization, the regular triangulation,
replacing the natural distance by the power to weighted points. The metric used
determines an underlying triangulation of the alpha shape and thus, the version
computed. 
In one hand, there is the {\em classic alpha shapes}
(cf. \ref{I1_SectClassicAS}) associated with the Delaunay triangulations
(cf. \ref{I1_Sect_Delaunay}), in the other hand, the {\em weighted alpha shapes}
(cf. \ref{I1_SectWeightedAS}) associated with the regular triangulations
(cf. \ref{I1_Sect_Regular}). 

There is a close connection between alpha shapes and the underlying
triangulations. More precisely, the $\alpha$-complex of $S$ is a
subcomplex of this triangulation of $S$, containing the $\alpha$-exposed
$k$-simplices, $0 \leq k \leq d$. A simplex is $\alpha$-exposed, if there is an
open disk (resp.\ ball) of radius $\sqrt{\alpha}$ through the vertices of the
simplex that does not contain any other point of $S$, for the metric used in
the computation of the underlying triangulation.  The corresponding
$\alpha$-shape is defined as the underlying interior space of the
$\alpha$-complex. 

In general, an $\alpha$-complex is a non-connected and non-pure polytope, it
means, that one $k$-simplice, $0 \leq k \leq d-1$ is not necessary adjacent to
a $(k+1)$-simplice.

The $\alpha$-shapes of $S$ form a discrete family, even though they
are defined for all real numbers $\alpha$ with $0 \leq \alpha
\leq \infty$. Thus, we can represent the entire family of $\alpha$-shapes
of $S$ by the underlying triangulation of $S$. In this representation
each $k$-simplex of the underlying triangulation is associated with an
interval that specifies for which values of $\alpha$ the $k$-simplex
belongs to the $\alpha$-shape. Relying on this result, the family of
$\alpha$-shapes can be computed efficiently and relatively
easily. Furthermore, we can select an appropriate $\alpha$-shape from a
finite number of different $\alpha$-shapes and corresponding
$\alpha$-values.

\subsection*{Main Application}

Alpha shapes can be used for shape reconstruction from a dense unorganized set
of data points. Indeed, an  $\alpha$-shape is demarcated by a frontier, which is
a linear approximation of the original contour, under conditions described for
the classic version, in \cite{bb-srmua-97t}.

%----------------------------------------------------------------------

\section{Generic Alpha-Shapes of Points in a Plane \label{I1_SectAlpha_Shape_2}}

\begin{ccClassTemplate} {Alpha_shape_2<Dt>}

\ccDefinition

The class \ccClassTemplateName\ represents the family of
$\alpha$-shapes of points in a plane for {\em all} positive
$\alpha$. It maintains the underlying triangulation \ccc{Dt} which
represents connectivity and order among its faces. Each
$k$-dimensional face of the \ccc{Dt} is associated with
an interval that specifies for which values of $\alpha$ the face
belongs to the $\alpha$-shape. There are links between the intervals
and the $k$-dimensional faces of the triangulation.

\ccInclude{CGAL/Alpha_shape_2.h}

\ccInheritsFrom

\ccc{Dt}

This class is the underlying triangulation class.

The modifying functions \ccc{insert} and \ccc{remove} will overwrite
the inherited functions. At the moment, only the static version is implemented.

\ccTypes
\ccSetThreeColumns{Oriented_side}{}{\hspace*{10cm}}
\ccThreeToTwo

\ccNestedType{Gt}{the alpha shape traits type.}
it contains a Delaunay triangulation traits class.  For example \ccc{Dt::Point} is a Point class. 

\ccTypedef{typedef Gt::Coord_type Coord_type;}{the number type for computation.}

\ccNestedType{Alpha_iterator}{An iterator that allow to traverse 
the increasing sequence of different $\alpha$-values. The iterator is
bidirectional and non-mutable. Its \ccc{value_type}
is \ccc{Coord_type}}

\ccEnum{enum Classification_type {EXTERIOR, SINGULAR, REGULAR, INTERIOR};}
{Distinguishes the different cases for classifying a $k$-dimensional face
 of the underlying triangulation of the $\alpha$-shape. \\
\ccc{EXTERIOR} if the face does not belong to the $\alpha$-complex.\\
\ccc{SINGULAR} if the face belongs to the boundary of the $\alpha$-shape,
 but is not incident to any 2-dimensional face of the $\alpha$-complex\\
\ccc{REGULAR} if the face belongs to the boundary of the $\alpha$-shape
 and is incident to a 2-dimensional face of the $\alpha$-complex\\
\ccc{INTERIOR} if the face belongs to the $\alpha$-complex, but does
not belong to the boundary of the $\alpha$-shape\\}

\ccEnum{enum Mode {GENERAL, REGULARIZED};}
{ In general, an alpha shape can be disconnected and contain many singular edges 
or vertices. Its regularized version is formed by the set of regular edges
and their vertices}

\ccCreation
\ccCreationVariable{A}

\ccConstructor{Alpha_shape_2(Coord_type alpha = 0,
				  Mode m = GENERAL);}
{Introduces an empty $\alpha$-shape \ccVar\ for a positive $\alpha$-value
 \ccc{alpha}.
\ccPrecond \ccc{alpha}~$\geq~0$.}


\ccConstructor{template < class InputIterator >
		Alpha_shape_2(
			InputIterator first,
			InputIterator last,
                const Coord_type& alpha = 0,
	        Mode m = GENERAL);}
{Initializes the family of alpha-shapes with the points in the range
$\left[\right.$\ccc{first}, \ccc{last}$\left.\right)$ and 
introduces an $\alpha$-shape \ccVar\ for a positive $\alpha$-value
\ccc{alpha}.  
\ccPrecond The \ccc{value_type} of \ccc{first} and
\ccc{last} is \ccc{Point}.\\
\ccc{alpha} $\geq 0$.}

\ccOperations

\ccMethod{template < class InputIterator >
		int make_alpha_shape(
			InputIterator first,
			InputIterator last,
                        const Coord_type& alpha = 0,
	                Mode m = GENERAL);}
{Initialize the family of alpha-shapes with the points in the range
$\left[\right.$\ccc{first}, \ccc{last}$\left.\right)$ and 
introduces an $\alpha$-shape \ccVar\ for a positive $\alpha$-value
\ccc{alpha}. Returns the number of inserted points. \\
If the function is applied to an non-empty family of alpha-shape, it is cleared
before initialization.
\ccPrecond The \ccc{value_type} of \ccc{first} and
\ccc{last} is \ccc{Point}.\\
\ccc{alpha} $\geq 0$.}

\ccMethod{void
	clear();}
{Clears the structure.}

\ccMethod{Coord_type
	set_alpha(const Coord_type& alpha);}
{Sets the $\alpha$-value to \ccc{alpha}.
 Returns the previous $\alpha$-value.
\ccPrecond \ccc{alpha} $\geq 0$.}

\ccMethod{const Coord_type&
	get_alpha(void) const;}
{Returns the current $\alpha$-value.}

\ccMethod{const Coord_type&  get_nth_alpha(int n) const;}
{Returns the $n$-th alpha-value, sorted in an increasing order.
 \ccPrecond \ccc{n} < number of alphas.}

\ccMethod{int number_of_alphas() const;}
{Returns the number of different alpha-values.}

% dynamic version	
% 
% \ccMethod{Vertex_handle insert(const Point& p);}
% {Inserts point \ccc{p} in the alpha shape and returns the
% corresponding vertex of the underlying Delaunay triangulation.\\ If
% point \ccc{p} coincides with an already existing vertex, this
% vertex is returned and the alpha shape remains unchanged.\\ Otherwise,
% the vertex is inserted in the underlying Delaunay triangulation and
% the associated intervals are updated. }
% 
% \ccMethod{void remove(Vertex *v);}
% {Removes the vertex from the underlying Delaunay triangulation. The
% created hole is retriangulated and the associated intervals are
% updated.}
% 

\ccMethod{Mode
	set_mode(Mode m = GENERAL );}
{Sets \ccVar\ to its general or regularized version. 
Returns the previous mode.}

\ccMethod{Mode
	get_mode(void) const;}
{Returns whether \ccVar\ is general or regularized.}

\ccMethod{template  < class OutputIterator >
	  OutputIterator get_alpha_shape_vertices(
					 OutputIterator result);}
{Writes the vertices of the alpha shape \ccVar\ for the current $\alpha$-value
to the container where \ccc{result} refers to. 
The \ccc{value_type} of \ccc{result} is \ccc{Vertex_handle}.
Returns an output iterator which is the end of the constructed range.}

\ccMethod{template  < class OutputIterator >
	  OutputIterator get_alpha_shape_edges(
					 OutputIterator result);}
{Writes the edges 
of the alpha shape \ccVar\ for the current $\alpha$-value
to the container where \ccc{result} refers to. 
The \ccc{value_type} of \ccc{result} is \ccc{pair<Face_handle, int>}.
Returns an output iterator which is the end of the constructed range.}

\ccHeading{Predicates}


\ccMethod{Classification_type
           classify(const Point& p, 
	const Coord_type& alpha = get_alpha()) const;}
{Locates a point  \ccc{p} in the underlying triangulation and Classifies the 
associated k-face with respect to \ccVar.}

\ccMethod{Classification_type
           classify(Face_handle f, const Coord_type& alpha = get_alpha()) const;}
{Classifies the face \ccc{f} of the underlying triangulation with respect to \ccVar.}

\ccMethod{Classification_type
           classify(pair<Face_handle, int> e, const Coord_type& alpha = get_alpha()) const;}
{Classifies the edge \ccc{e} of the underlying triangulation with respect to \ccVar.}

\ccMethod{Classification_type
           classify(Face_handle f, int i, const Coord_type& alpha = get_alpha()) const;}
{Classifies the edge of the face \ccc{f} opposite to the vertex with index
\ccc{i} 
of the underlying triangulation with respect to \ccVar.}	

\ccMethod{Classification_type
           classify(Vertex_handle v, const Coord_type& alpha = get_alpha()) const;}
{Classifies the vertex \ccc{v} of the underlying triangulation with respect to \ccVar.}


\ccHeading{Traversal of the $\alpha$-Values}

\smallskip
The alpha shape class defines an iterator that allows to visit the
sorted sequence of $\alpha$-values. This iterator is
non-mutable and bidirectional. Its value type is
\ccc{Coord_type}.

\ccMethod{Alpha_iterator alpha_begin() const;}
{Returns an iterator that allows to traverse the
sorted sequence of $\alpha$-values of the family of alpha shapes.}

\ccMethod{Alpha_iterator alpha_end() const;}
{Returns the corresponding past-the-end iterator.}

\ccMethod{Alpha_iterator alpha_find(const Coord_type& alpha) const;}
{Returns an iterator pointing to an element with $\alpha$-value
\ccc{alpha}, or the corresponding past-the-end iterator if such 
an element is not found.}

\ccMethod{Alpha_iterator alpha_lower_bound(const Coord_type& alpha) const;}
{Returns an iterator pointing to the first element with
$\alpha$-value not less than \ccc{alpha}.}

\ccMethod{Alpha_iterator alpha_upper_bound(const Coord_type& alpha) const;}
{Returns an iterator pointing to the first element with $\alpha$-value
greater than \ccc{alpha}.}

\ccHeading{Operations}

\ccMethod{int number_solid_components(const Coord_type& alpha = get_alpha()) const;}
{Returns the number of solid components of \ccVar, that is, the number of
components of its 
regularized version.}

\ccMethod{Alpha_iterator find_optimal_alpha(int nb_components) const;}
{Returns an iterator pointing to the first element with $\alpha$-value
such that \ccVar\ satisfies the following two properties:\\
\ccc{nb_components} equals the number of solid components and \\
all data points are either on the boundary or in the interior of the regularized version of \ccVar.\\
If no such value is found, the iterator points to the first element with 
$\alpha$-value such that \ccVar\ satisfies the second property.}

\ccHeading{I/O}

The I/O operators are defined for \ccc{iostream}, and for
the window stream provided by \cgal. The format for the iostream
is an internal format. 

\ccInclude{CGAL/IO/io.h}

\ccFunction{ostream& operator<<(ostream& os,
                  const Alpha_shape_2<Dt>& A);}
{Inserts the alpha shape \ccVar\ for the current $\alpha$-value into the stream \ccc{os}.
\ccPrecond The insert operator must be defined for \ccc{Point}.}

\ccInclude{CGAL/IO/Window_stream.h}

\ccInclude{CGAL/IO/alpha_shapes_2_window_stream.h}

\ccFunction{Window_stream& operator<<(Window_stream& W,
                         const Alpha_shape_2<Dt>& A);}
{Inserts the alpha shape \ccVar\ for the current $\alpha$-value into the window stream \ccc{W}.
\ccPrecond The insert operator must be defined for \ccc{Point} and \ccc{Segment}.}
\end{ccClassTemplate}

\ccImplementation
In the first static version, the set of intervals associated with the
$k$-dimensional faces of the underlying triangulation are
stored only as sorted \ccc{vectors}. By using an interval tree the
alpha-shape could be constructed more efficiently. For the dynamic
version, we need \ccc{multimaps} or dynamic interval trees.

The cross links between the intervals and the $k$-dimensional faces of the
triangulation are actually realized using methods in the $k$-dimensional faces
themselves. By this way, you can decide if you want to store or re-compute any
of these informations.


\ccc{A.alpha find} uses linear search, while 
\ccc{A.alpha lower bound} and \ccc{A.alpha upper bound} 
use binary search.
\ccc{A.number solid components} performs a graph traversal and takes time
linear in the number of faces of the underlying triangulation.
\ccc{A.find optimal alpha} uses binary search and takes time
$O(\mbox{ \em n } \log{\mbox{ \em n } })$, where  $n$ is the number of points.

%----------------------------------------------------------------------

\section{The Underlying Triangulation\label{I1_SectDtClass}}

For the \ccc{Dt} class, you have to choose between a Delaunay or a regular
triangulation, depending the version of alpha shapes, you want.

The class \ccc{Dt} must be parameterized with a special alpha shape
traits class and a simple triangulation data structure, with no more requirements as
for a simple triangulation class, but parameterized by a special vertex base
class and a special face base class.

\section*{The Alpha-Shape Traits Class (\mbox{\it Gt})\label{I1_ASTraits}} 

\subsection{Requirements for the Alpha-Shape Traits Class}

First, this traits class has the same requirements as the 
triangulation traits class, for the \ccc{Dt} triangulation, you opted.
The following requirement catalog lists the primitives that must be defined
additionally.


\begin{ccClass} {Gt}
\subsection*{\protect \ccc{Alpha_shape_traits} (\mbox{\it Gt})}

\ccCreationVariable{t}

\ccDefinition
A class \ccClassName\ that satisfies the requirements of a
\ccc{Alpha_shape_traits} class must provide the following predicate and
operations in addition to the requirements for the underlying triangulation
traits class.

\ccTypes

\ccNestedType{Coord_type}{A type to hold a coordinate type class. 
The type must provide a copy constructor, assignment, comparison
operators, negation, multiplication, division and allow the
declaration and initialization with a small integer constant
(cf. requirements for number types). 
\ccPrecond An obvious choice would be coordinate type of the point class.}

\ccCreation

Only a default constructor is required. Note that further constructors
can be provided. 

\ccConstructor{Gt();}
{A default constructor.}

\ccHeading{Constructions}

\ccMethod{Coord_type squared_radius(const Point& p0,
				const Point& p1, 
				const Point& p2) const;} 
{Returns the squared radius of the circle of the
points \ccc{p0, p1, p2}, associated with the metric 
used by \ccc{Dt}.}

\ccMethod{Coord_type squared_radius(const Point& p0,
                                const Point& p1) const;} 
{Returns the squared radius of smallest  circle of the
points \ccc{p0, p1}, associated with the metric 
used by \ccc{Dt}.}

\ccHeading{Predicate}

\ccMethod{Bounded_side side_of_circle(const Point& p0,
                                const Point& p1,
                                const Point& test) const;}
{Returns the relative position of point \ccc{test} to the smallest circle of
the points \ccc{p0, p1}, using the same metric as \ccc{Dt}.}

\end{ccClass}


\section*{Triangulation Data Structure Class (\mbox{\it Tds}) for an Alpha-Shape}

The class \ccc{Alpha_shape_2<Dt>} parameterized with a
triangulation data structure class \ccc{Tds}, that has exactly the same
requirements as the triangulation data structure class for a simple
triangulation.  

But, she requires to be templated by special \ccc{Alpha_shape_vertex_base} and
\ccc{Alpha_shape_face_base<Df>}, to store the alpha values associated. Thus, we need to
define the requirements of such \ccc{Alpha_shape_vertex_base} and
\ccc{Alpha_shape_face_base<Df>} classes.

\subsection{Requirements for the \protect \ccc{Vertex_base} Class of an
Alpha-Shape\label{I1_SectVertex}} 

The information about the alpha values associated  are accessible by the 
vertices of the alpha shape. Thus the nested \ccc{Alpha_shape_vertex_base}
type of an alpha shape offers additional functionalities to deal with these
methods. 

This additional functionalities related to the alpha shape
are requirements which have to be fulfilled
by the base vertex for an alpha shape,
in addition to the functionalities required for a simple triangulation vertex.
They are listed below as such.

\ccInheritsFrom

\ccc{Triangulation_vertex_base}

\begin{ccClass}{Alpha_shape_vertex_base}

\ccCreationVariable{v}

\ccTypes

\ccNestedType{Coord_type}{A type to hold a coordinate type class. 
The type must provide a copy constructor, assignment, comparison
operators, negation, multiplication, division and allow the
declaration and initialization with a small integer constant
(cf. requirements for number types). An obvious choice would be
coordinate type of the point class.
\ccPrecond Let's us precise that \ccc{Coord_type} has to be the same as the one used by
the \ccc{Gt}, already used by underlying triangulation \ccc{Dt}.}

\ccCreation

\begin{ccAdvanced}

\ccConstructor{Alpha_shape_vertex_base();}{default constructor.}
\ccGlue
\ccConstructor{Alpha_shape_vertex_base(Point p);}{constructor setting
the point associated to.}
\ccGlue
\ccConstructor{Alpha_shape_vertex_base(Point p, void* f);} 
{constructor setting the point  associated to and an incident face.}

\end{ccAdvanced}

\ccHeading{Access Functions}

\ccMethod{std::pair< Coord_type, Coord_type > get_range();}
{returns two alpha values $\alpha_1 \leq \alpha_2$, such as for
$\alpha$ under $\alpha_2$, the vertex is attached but singular, and
for $\alpha$ under $\alpha_1$, the vertex is regular.}


\ccModifiers

\begin{ccAdvanced}

\ccMethod{void set_range(std::pair< Coord_type, Coord_type > I);}
{sets the alpha values $\alpha_1 \leq \alpha_2$, such as for
$\alpha$ under $\alpha_2$, the vertex is attached but singular, and
for $\alpha$ under $\alpha_1$, the vertex is regular.}

\end{ccAdvanced}

\end{ccClass}

\subsection{Requirements for the \protect \ccc{Face_base} Class of an Alpha-Shape}

The information about the alpha values, associated to the face herself and the
incident edges, are accessible by the 
faces of the alpha shape. Thus the nested \ccc{Alpha_shape_face_base<Df>}
type of an alpha shape offers additional functionalities to deal with these
methods.

\ccc{Df} has to be choosed as the face base for the underlying triangulation. 

This additional functionalities related to the alpha shape
are requirements which have to be fulfilled
by the base face  for an alpha shape,
in addition to the functionalities required by an underlying triangulation face \ccc{Df}.
They are listed below as such.

\ccInheritsFrom

\ccc{Df}

\begin{ccClass}{Alpha_shape_face_base<Df>}

\ccCreationVariable{f}
\ccTypes

\ccNestedType{Interval_3}{A container type to get (and put) the three special values 
($\alpha_1, \alpha_2, \alpha_3$) associated with an alpha shape edge.}

\ccNestedType{Coord_type}{A type to hold a coordinate type class. 
The type must provide a copy constructor, assignment, comparison
operators, negation, multiplication, division and allow the
declaration and initialization with a small integer constant
(cf. requirements for number types). An obvious choice would be
coordinate type of the point class.
\ccPrecond Let's us precise that \ccc{Coord_type} has to be the same as the one used by
\ccc{Gt}, already used by the underlying triangulation \ccc{Dt}, and by
her face base \ccc{Df}.} 

\ccCreation

\begin{ccAdvanced}

\ccConstructor{Alpha_shape_face_base();}{default constructor.}
\ccGlue
\ccConstructor{Alpha_shape_face_base(void* v0, void* v1, void* v2);}{constructor setting the incident vertices.}
\ccGlue
\ccConstructor{Alpha_shape_face_base(void* v0, void* v1, void*  v2, void* n0, void* n1, void* n2);} 
{constructor setting the incident vertices and the neighboring faces.}

\end{ccAdvanced}

\ccHeading{Access Functions}
\ccMethod{Interval_3[3] get_ranges();}
{returns a vector \ccc{V}, in which, for each edge $i$, \ccc{V[i]} contains
three alpha values
 $\alpha_1 \leq \alpha_2 \leq \alpha_3$, such as for
$\alpha$ under $\alpha_3$, the edge is attached but singular,
for $\alpha$ under $\alpha_2$, the face is regular, and for $\alpha$
under $\alpha_1$, the edge is interior.}
\ccMethod{Coord_type get_alpha();}
{return the alpha value, under which the alpha shape contains the
face.}

\ccModifiers

\begin{ccAdvanced}

\ccMethod{void set_ranges(Interval_3[3] V);}
{sets the vector \ccc{V}, in which, for each edge $i$, \ccc{V[i]} contains three
alpha values
 $\alpha_1 \leq \alpha_2 \leq \alpha_3$, such as for
$\alpha$ under $\alpha_3$, the edge is attached but singular,
for $\alpha$ under $\alpha_2$, the face is regular, and for $\alpha$
under $\alpha_1$, the edge is interior.}
\ccMethod{void set_alpha(Coord_type A);}
{sets the alpha value, under which the alpha shape contains the
face.}

\end{ccAdvanced}

\end{ccClass}

%----------------------------------------------------------------------

\section{Classic Version of Alpha-Shapes\label{I1_SectClassicAS}}

\subsection*{The Underlying Triangulation~: a Delaunay Triangulation}

For a simple alpha shape, you have to choose a Delaunay triangulation as
underlying triangulation \ccc{Dt} and follow the requirements listed above.

\ccInclude{CGAL/Delaunay_triangulation_2.h}

\subsection*{Predefined Geometric Traits Class (\mbox{\it Gt})}

Of course, \cgal\ provides a default \ccc{Alpha_shape_traits} class in this
case, with an implementation of the appropriate predicate and constructions.
The class \ccc{Alpha_shape_euclidean_traits_2<Rp>} simply derived from 
\ccc{Triangulation_euclidean_traits_2<Rp>}.

\ccInclude{CGAL/Alpha_shape_euclidean_traits_2.h}

\subsection*{Predefined \protect \ccc{Vertex_base} Class}

\cgal\ provides a default \ccc{Vertex_base} class
for the Alpha Shape. The class
\ccc{Alpha_shape_vertex_base_2<Gt>} simply derived from 
\ccc{Triangulation_vertex_base_2<Gt>}.

\ccInclude{CGAL/Alpha_shape_vertex_base_2.h}

\subsection*{Predefined \protect \ccc{Face_base} Class}

\cgal\ provides a default \ccc{Face_base} class
for the Alpha Shape. The class
\ccc{Alpha_shape_face_base_2<Gt,Df>} simply derived from 
\ccc{Df}, which can be instantiated with \ccc{Triangulation_face_base_2<Gt>}.

\ccInclude{CGAL/Triangulation_face_base_2.h}

\ccInclude{CGAL/Alpha_shape_face_base_2.h}

\subsection*{To Do a Classic Alpha-Shape}

\ccExample
The following code details the different steps to create 
a classic version of alpha shape with the defaults.

\begin{cprog}
typedef CGAL::Cartesian<double> Rp;
typedef CGAL::Alpha_shape_euclidean_traits_2<Rp> Gt;
typedef CGAL::Alpha_shape_vertex_base_2<Gt> Vb;
typedef CGAL::Triangulation_face_base_2<Gt> Df;
typedef CGAL::Alpha_shape_face_base_2<Gt,Df> Fb;
typedef CGAL::Triangulation_default_data_structure_2<Gt,Vb,Fb> Tds;
typedef CGAL::Delaunay_triangulation_2<Gt,Tds> Dt;
typedef CGAL::Alpha_shape_2<Dt> Alpha_shape_2;
\end{cprog}

%----------------------------------------------------------------------

\section{Weighted Version of Alpha-Shapes\label{I1_SectWeightedAS}}

\subsection*{The Underlying Triangulation~: a Regular Triangulation}

For a weighted alpha shape, you have to choose a regular triangulation as
underlying triangulation \ccc{Dt} and follow the requirements listed above.

\ccInclude{CGAL/Regular_triangulation_2.h}

\subsection*{Predefined Geometric Traits Class (\mbox{\it Gt})}

Of course, \cgal\ provides a default \ccc{Alpha_shape_traits} class in this
case, with an implementation of the appropriate predicate and constructions.
The class \ccc{Weighted_alpha_shape_euclidean_traits_2<Rp>} simply derived from 
\ccc{Regular_triangulation_euclidean_traits_2<Rp>}.

\ccInclude{CGAL/Weighted_alpha_shape_euclidean_traits_2.h}

\subsection*{Predefined \protect \ccc{Vertex_base} Class}

\cgal\ provides a default \ccc{Vertex_base} class
for the Alpha Shape. The class
\ccc{Alpha_shape_vertex_base_2<Gt>} simply derived from 
\ccc{Triangulation_vertex_base_2<Gt>}.

\ccInclude{CGAL/Alpha_shape_vertex_base_2.h}

\subsection*{Predefined \protect \ccc{Face_base} Class}

\cgal\ provides a default \ccc{Face_base} class
for the Alpha Shape. The class
\ccc{Alpha_shape_face_base_2<Gt,Df>} simply derived from 
\ccc{Df}, which can be instantiated with \ccc{Regular_triangulation_face_base_2<Gt>}.

\ccInclude{CGAL/Regular_triangulation_face_base_2.h}

\ccInclude{CGAL/Alpha_shape_face_base_2.h}

\subsection*{To Do a Weighted Alpha-Shape}

\ccExample
The following code details the different steps to create 
a weighted version of alpha shape with the defaults.

\begin{cprog}
typedef CGAL::Cartesian<double> Rp;
typedef CGAL::Weighted_alpha_shape_euclidean_traits_2<Rp> Gt;
typedef CGAL::Alpha_shape_vertex_base_2<Gt> Vb;
typedef CGAL::Regular_triangulation_face_base_2<Gt> Rf;
typedef CGAL::Alpha_shape_face_base_2<Gt,Rf>  Fb;
typedef CGAL::Triangulation_default_data_structure_2<Gt,Vb,Fb> Tds;
typedef CGAL::Regular_triangulation_2<Gt,Tds> Rt;
typedef CGAL::Alpha_shape_2<Rt> Alpha_shape_2;
\end{cprog}
