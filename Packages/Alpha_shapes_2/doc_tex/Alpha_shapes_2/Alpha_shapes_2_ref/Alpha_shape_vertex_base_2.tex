% ======================================================================
%
% Copyright (c) 1999 The GALIA Consortium
%
% This software and related documentation is part of the
% Computational Geometry Algorithms Library (CGAL).
%
% Every use of CGAL requires a license. Licenses come in three kinds:
%
% - For academic research and teaching purposes, permission to use and
%   copy the software and its documentation is hereby granted free of  
%   charge, provided that
%   (1) it is not a component of a commercial product, and
%   (2) this notice appears in all copies of the software and
%       related documentation.
% - Development licenses grant access to the source code of the library 
%   to develop programs. These programs may be sold to other parties as 
%   executable code. To obtain a development license, please contact
%   the GALIA Consortium (at cgal@cs.uu.nl).
% - Commercialization licenses grant access to the source code and the
%   right to sell development licenses. To obtain a commercialization 
%   license, please contact the GALIA Consortium (at cgal@cs.uu.nl).
%
% This software and documentation is provided "as-is" and without
% warranty of any kind. In no event shall the CGAL Consortium be
% liable for any damage of any kind.
%
% The GALIA Consortium consists of Utrecht University (The Netherlands),
% ETH Zurich (Switzerland), Free University of Berlin (Germany),
% INRIA Sophia-Antipolis (France), Martin-Luther-University Halle-Wittenberg
% (Germany), Max-Planck-Institute Saarbrucken (Germany),
% and Tel-Aviv University (Israel).
%
% ----------------------------------------------------------------------
%
% package       : Alpha_shapes_2
% author(s)     : Tran Kai Frank DA <Frank.Da@sophia.inria.fr>
%
% coordinator   : INRIA Sophia-Antipolis (<Mariette.Yvinec@sophia.inria.fr>)
%
% ======================================================================

\RCSdef{\alphashapeRevision}{$Revision$}
\RCSdefDate{\alphashapeDate}{$Date$}

%----------------------------------------------------------------------

\begin{ccRefClass}{Alpha_shape_vertex_base}

\ccDefinition

In addition to the functionalities required for a simple triangulation vertex.
They are listed below as such.

\ccInheritsFrom

\ccc{Triangulation_vertex_base}

\ccCreationVariable{v}

\ccTypes

\ccNestedType{Coord_type}{A type to hold a coordinate type class. 
The type must provide a copy constructor, assignment, comparison
operators, negation, multiplication, division and allow the
declaration and initialization with a small integer constant
(cf. requirements for number types). An obvious choice would be
coordinate type of the point class.
\ccPrecond Let's us precise that \ccc{Coord_type} has to be the same as the one used by
the \ccc{Gt}, already used by underlying triangulation \ccc{Dt}.}

\ccCreation

\begin{ccAdvanced}

\ccConstructor{Alpha_shape_vertex_base();}{default constructor.}
\ccGlue
\ccConstructor{Alpha_shape_vertex_base(Point p);}{constructor setting
the point associated to.}
\ccGlue
\ccConstructor{Alpha_shape_vertex_base(Point p, void* f);} 
{constructor setting the point  associated to and an incident face.}

\end{ccAdvanced}

\ccHeading{Access Functions}

\ccMethod{std::pair< Coord_type, Coord_type > get_range();}
{returns two alpha values $\alpha_1 \leq \alpha_2$, such as for
$\alpha$ under $\alpha_2$, the vertex is attached but singular, and
for $\alpha$ under $\alpha_1$, the vertex is regular.}


\ccModifiers

\begin{ccAdvanced}

\ccMethod{void set_range(std::pair< Coord_type, Coord_type > I);}
{sets the alpha values $\alpha_1 \leq \alpha_2$, such as for
$\alpha$ under $\alpha_2$, the vertex is attached but singular, and
for $\alpha$ under $\alpha_1$, the vertex is regular.}

\end{ccAdvanced}

\end{ccRefClass}
