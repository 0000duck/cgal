\begin{ccRefConcept}{ConvexHullPolyhedron_3}

\ccDefinition
The requirements of the polyhedron type to be built by the
function \ccc{convex_hull_3}.

\ccTypes
\ccAutoIndexingOff
\ccNestedType{Point}{ type of point stored in a vertex }
\ccGlue
\ccNestedType{Vertex}{ a model of \ccc{Polyhedron_Vertex}}
\ccGlue
\ccNestedType{Halfedge}{ a model of \ccc{Polyhedron_Halfedge}}
\ccGlue
\ccNestedType{Facet}{ a model of \ccc{Polyhedron_Facet}}
\ccGlue
\ccNestedType{Halfedge_data_structure}{halfedge data structure HDS}
\ccGlue
\ccNestedType{Halfedge_handle}{handle to halfedge}
\ccGlue
\ccNestedType{Halfedge_iterator}{iterator for halfedge}
\ccGlue
\ccNestedType{Facet_handle}{handle to facet}
\ccGlue
\ccNestedType{Facet_iterator}{iterator for facet}

\ccCreation
\ccCreationVariable{p}

Only a default constructor is required.

\ccOperations

\ccAutoIndexingOff
\ccMethod{Facet_iterator facets_begin();}%
{iterator over all facets (excluding holes)}

\ccMethod{Facet_iterator facets_end();}{past-the-end iterator}

\ccMethod{Halfedge_iterator border_halfedges_begin();}{halfedge iterator starting with the border edges.}

\ccMethod{
Halfedge_handle make_tetrahedron(Point p1, Point p2, Point p3, Point p4);
}{
adds a new tetrahedron to the polyhedral surface with its
vertices initialized with \ccc{p1}, \ccc{p2}, \ccc{p3} and \ccc{p4}. 
Returns that halfedge
of the tetrahedron which incident vertex is initialized with \ccc{p1}, the
incident vertex of the next halfedge with \ccc{p2}, and the vertex
thereafter with \ccc{p3}. The remaining fourth vertex is initialized with
\ccc{p4}.
}

\ccMethod{void erase_facet(Halfedge_handle h);}%
{removes the incident facet of \ccc{h} 
and changes all halfedges incident to the facet into border edges or removes 
them from the polyhedral surface if they were already border edges.}

\ccMethod{void normalize_border();}{
sorts halfedges such that the non-border edges precedes the
border edges. For each border edge the halfedge iterator will
reference the halfedge incident to the facet right before the
halfedge incident to the hole. }

\ccMethod{Size size_of_border_halfedges();}{ 
number of border halfedges.
(used only for assertion checking) }

\ccMethod{
Halfedge_handle 
add_vertex_and_facet_to_border(Halfedge_handle h, Halfedge_handle g);
}{
creates a new facet within the hole incident to \ccc{h} and \ccc{g} by
connecting the tip of \ccc{g} with the tip of \ccc{h} with two new halfedges
and a new vertex and filling this separated part of the hole with
a new facet, such that the new facet is incident to \ccc{g}. Returns the
halfedge of the new edge that is incident to the new facet and
the new vertex.
}

\ccMethod{
Halfedge_handle
add_facet_to_border(Halfedge_handle h, Halfedge_handle g);}%
{
creates a new facet within the hole incident to \ccc{h} and \ccc{g} by
connecting the tip of \ccc{g} with the tip of \ccc{h} with a new halfedge and
filling this separated part of the hole with a new facet, such that
the new facet is incident to \ccc{g}. Returns the halfedge of the new
edge that is incident to the new facet.
}

\ccMethod{Halfedge_handle fill_hole(Halfedge_handle h);}{
fills a hole with a newly created facet. Makes all border
halfedges of the hole denoted by h incident to the new facet.
Returns \ccc{h}.
}

\ccMethod{void delegate(Modifier_base<HDS>& m);}%
{
calls the \ccc{operator()} of the modifier \ccc{m}. See \ccc{Modifier_base} in 
the Support Library Manual for a description of modifier design and
its usage.
}
\ccAutoIndexingOn

\ccSeeAlso

\ccRefIdfierPage{CGAL::Polyhedron_3}

\end{ccRefConcept}
