\begin{ccRefFunction}{convex_hull_3}
\ccIndexMainItemBegin{convex hull, 3D}
\ccIndexMainItemBegin{quickhull, 3D}
\ccIndexSubitemBegin{convex hull, 3D}{quickhull}

The function \ccRefName\ computes the convex hull of a given set of 
three-dimensional points using the quickhull algorithm of Barnard
\textit{et al.} \cite{bdh-qach-96}.

\ccInclude{CGAL/convex_hull_3.h}

\ccFunction{
template <class InputIterator, class Polyhedron_3, class Traits>
void convex_hull_3(InputIterator first, InputIterator last,
                    Polyhedron_3& polyhedron,
                    const Traits& ch_traits = Default_traits);
}
{
computes the convex hull of the set of points in the range
[\ccc{first}, \ccc{last}).  The result is stored in \ccc{polyhedron}.
\ccPrecond: There are at least four points in the range 
[\ccc{first}, \ccc{last}) not all of which are collinear.
}

\ccHeading{Requirements}
\begin{enumerate}
   \item \ccc{InputIterator::value_type} should be \ccc{Traits::Point_3}.
   \item \ccc{Traits} is a model of the concept ConvexHullTraits\_3
         \ccIndexMainItem[c]{ConvexHullTraits_3}.
         For the purposes of checking the postcondition that the convex hull
         is valid, \ccc{Traits} should also be a model of the concept
         IsStronglyConvexTraits\_3.
        %\ccIndexMainItem[c]{IsStronglyConvexTraits_3}
  \item \ccc{Polyhedron_3} must be a model of ConvexHullPolyhedron\_3.
\end{enumerate}

\ccFunction{
template <class InputIterator, class Polyhedron_3, class Traits>
void convex_hull_3(InputIterator first, InputIterator last,
                    Object& ch_object,
                    const Traits& ch_traits = Default_traits);
}
{
computes the convex hull of the set of points in the range
[\ccc{first}, \ccc{last}).  The result, which may be a point, a segment,
a triangle, or a polyhedron, is stored in \ccc{ch_object}.
}

\ccHeading{Requirements}
In addition to the first two requirements listed above, \ccc{Traits} must
define a type \ccc{Polyhedron_3} that is a model of ConvexHullPolyhedron\_3
for this version of the function. 

The default traits class for both versions of \ccc{convex_hull_3} is 
\ccc{Convex_hull_traits_3<R>},%
with the representation determined by \ccc{InputIterator::value_type}.

\ccSeeAlso

%\ccRefIdfierPage{CGAL::convex_hull_3_from_d} \\
\ccRefIdfierPage{CGAL::ch_eddy}  \\
\ccRefIdfierPage{CGAL::convex_hull_points_2} 

\ccExample

The following program computes the convex hull of a set of 250 random
points chosen from a sphere of radius 100.  It then determines if the resulting
hull is a segment or a polyhedron.  Notice that the traits class is not
necessary in the call to \ccc{convex_hull_3} but is used in the definition
of \ccc{Polyhedron_3}.

\ccIncludeExampleCode{examples/Convex_hull_3/ch_quickhull_3_ex.C}

\ccIndexSubitemEnd{convex hull, 3D}{quickhull}
\ccIndexMainItemEnd{quickhull, 3D}
\ccIndexMainItemEnd{convex hull, 3D}
\end{ccRefFunction}
