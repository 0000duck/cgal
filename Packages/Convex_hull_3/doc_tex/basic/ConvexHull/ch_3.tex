\newpage
\section{3D Convex Hull}
\cgal\ provides the following function template to compute the convex hull polyhedron
for a set of points given by an iterator range.

\ccSetThreeColumns{typedef Oriented_side}{}{\hspace*{8.0cm}}
\ccInclude{CGAL/convex_hull_3.h}

\ccFunction{template <class InputIterator, class Polyhedron>
            void
            convex_hull_3( InputIterator first, InputIterator beyond,
                                Polyhedron& P);}
{Computes the convex hull polyhedron of the points in the range \ccc{[first,beyond)} and
 assigns it to \ccc{P}.
 \ccPrecond Value type of the iterator and the point type of 
 the Polyhedron are \ccc{Point_3<R>} for some representation class \ccc{R}.}


\ccImplementation
The implementation uses parts of the 
\ccAnchor{http://www.mpi-sb.mpg.de/LEDA/www/leps/dd_geokernel.html}{LEP dd\_geokernel}
to compute the convex hull polyhedron, see also \cite{mmnssuz-cbhdc-97}.
In particular, you need to have \ccAnchor{http://www.mpi-sb.mpg.de/LEDA}{LEDA} installed 
to use this \cgal-function.


\ccExample

The following example uses \cgal's 3D random point generator
\ccc{Random_points_in_sphere_3<>}, which generates random points
on a sphere, and \cgal's \stl\ extension function \ccc{copy_n}, which 
copies $n$ elements.

\begin{cprog}
#include <CGAL/Cartesian.h>
#include <CGAL/Point_3.h>
#include <CGAL/point_generators_3.h>
#include <CGAL/copy_n.h>
#include <CGAL/Halfedge_data_structure_polyhedron_default_3.h>
#include <CGAL/Polyhedron_default_traits_3.h>
#include <CGAL/Polyhedron_3.h>
#include <CGAL/IO/Polyhedron_geomview_ostream.h>
#include <CGAL/convex_hull_3.h>
#include <vector>

/* representation class */
typedef CGAL::Cartesian<double>                                     RepCls;

/* define polyhedron type */
typedef CGAL::Halfedge_data_structure_polyhedron_default_3<RepCls>  HDS;
typedef CGAL::Polyhedron_default_traits_3<RepCls>                   PolyTraits;
typedef CGAL::Polyhedron_3< PolyTraits, HDS>                        Polyhedron;

/* define point creator */
typedef CGAL::Point_3<RepCls>                                       Point;
typedef CGAL::Creator_uniform_3<double,Point>                       PointCreator;

using namespace std;

int 
main()
{
  /* generate 250 points randomly on a sphere of radius 100.0 */
  Random_points_in_sphere_3< Point, PointCreator> gen(100.0);

  /* and copy them to a vector */
  vector<Point> V;
  CGAL::copy_n( gen, 250, back_inserter(V) );
  
  /* define polyhedron to hold convex hull */
  Polyhedron P;

  /* compute convex hull */
  CGAL::convex_hull_3( V.begin(), V.end(), P);

  /* visualize it using geomview - wait for mouse click */
  CGAL::Geomview_stream gvs;
  gvs << P;
  Point click;
  gvs >> click;
}
\end{cprog}


