\section{3D Convex Hull}
\begin{ccPackage}{convex hull, 3D}

There are two functions provided for computing convex hulls in three 
dimensions.  The function 
\ccc{convex_hull_3}\ccIndexMainItem[C]{convex_hull_3} provides an 
implementation of the quickhull algorithm \cite{bdh-qach-96} for three 
dimensions\ccIndexMainItem{quickhull, 3D}.  There are two versions of this
function availalbe, one that can be used when it is known that the output
will be a polyhedron (\textit{i.e.}, there are more than three points and
they are not all collinear) and one that handles all degenerate cases
and returns a \ccc{CGAL::Object}, which may be a point, a segment, a
triangle, or a polyhedron.

%The function \ccc{convex_hull_3_from_d}\ccIndexMainItem[C]{convex_hull_3_from_d}
%provides an interface that produces a polyhedron as output and uses the
%d-dimensional algorithm for building this.  This brings with it the power
%to query the hull structure for point inclusion, {\em etc.} 
%\marginpar{Michael, is there something more to say here?}

\end{ccPackage}

