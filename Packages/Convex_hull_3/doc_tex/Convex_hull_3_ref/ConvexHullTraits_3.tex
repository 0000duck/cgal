\begin{ccRefConcept}{ConvexHullTraits_3}

\ccDefinition
Requirements of the traits class to be used with the function
\ccc{convex_hull_3}. 

\ccTypes
\ccAutoIndexingOff
\ccNestedType{Point_3}{The point type on which the convex hull algorithm operates}
\ccGlue
\ccNestedType{Plane_3}{a 3D plane}
\ccGlue
\ccNestedType{Segment_3}{a 3D segment}
\ccGlue
\ccNestedType{Triangle_3}{a 3D triangle}
\ccGlue
\ccNestedType{Vector_3}{a 3D vector}

\ccNestedType{Construct_plane_3}{Function object type that provides
\ccc{Plane_3 operator()(Point_3 p, Point_3 q, Point_3 r)}, which constructs
and returns a plane passing through \ccc{p}, \ccc{q}, and \ccc{r} and oriented
in a positive sense when seen from the positive side of the plane.}

\ccNestedType{Construct_segment_3}{Function object type that provides
\ccc{Segment_3 operator()(Point_3 p, Point_3 q)}, which constructs and
returns the segment with source \ccc{p} and target \ccc{q}.}

\ccNestedType{Construct_triangle_3}{Function object type that provides
\ccc{Triangle_3 operator()(Point_3 p, Point_3 q, Point_3 r)}, which 
constructs and returns the triangle with vertices \ccc{p}, \ccc{q}, and
\ccc{r}}

\ccNestedType{Construct_vector_3}{Function object type that provides
\ccc{Vector_3 operator()(Point_3 p, Point_3 q)}, which constructs and
returns the vector \ccc{q}-\ccc{p}}

\ccNestedType{Collinear_3}{Predicate object type that provides
\ccc{bool operator()(Point_3 p, Point_3 q, Point_3 r)}, which determines
if points \ccc{p}, \ccc{q} and \ccc{r} are collinear or not}

\ccNestedType{Coplanar_3}{Predicate object type that provides
\ccc{bool operator()(Point_3 p, Point_3 q, Point_3 r, Point_3 s)}, which 
determines if points \ccc{p}, \ccc{q}, \ccc{r}, and \ccc{s} are coplanar
or not}

\ccNestedType{Has_on_positive_side_3}{Predicate object type that provides
\ccc{bool operator()(Plane_3 h, Point_3 q)}, which determines of the point
\ccc{q} is on the positive side of the halfspace \ccc{h}}

\ccNestedType{Less_distance_to_point_3}{Predicate object type that provides
a constructor taking a single \ccc{Point_3} object and
\ccc{bool operator()(Point_3 q, Point_3 r)}, which returns true iff the
distance from \ccc{q} to \ccc{p} is smaller than the distance from
\ccc{r} to \ccc{p}, where \ccc{p} is the point passed to the object
at construction.}

\ccNestedType{Less_signed_distance_to_plane_3}{Predicate object type that
provides \ccc{bool operator()(Plane_3 p, Point_3 q, Point_3 r)}, which 
returns true iff the signed distance from \ccc{q} to \ccc{p} is smaller
than the signed distance from \ccc{r} to \ccc{p}}

To handle the degenerate case when all points are coplanar, the following
three types that are default-constructable are necessary:

\ccNestedType{Traits_xy}{A model of ConvexHullTraits\_2 for points projected
                         into the $xy$-plane}
\ccNestedType{Traits_xz}{A model of ConvexHullTraits\_2 for points projected
                         into the $xz$-plane}
\ccNestedType{Traits_yz}{A model of ConvexHullTraits\_2 for points projected
                         into the $yz$-plane}

One also needs the following function object to help choose which of the above
traits classes to use:

\ccNestedType{Max_coordinate_3}{Function object type that provides
\ccc{int operator()(Vector_3 v)}, which returns the index (0, 1, or 2 for
$x$, $y$, or $z$, respectively) of the coordinate of $v$ with maximum absolute 
value.}

These types need not be provided when it is known that the points are
not all coplanar.

\ccCreation
\ccCreationVariable{traits}

Only a copy constructor is required.

\ccConstructor{ConvexHullTraits_3(ConvexHullTraits_3& ch);}{}

\ccOperations

For each of the above function and predicate object types,
\ccc{Func_obj_type}, a function must exist with the name 
\ccc{func_obj_type_object} that creates an instance of the function or 
predicate object type.  For example:

\ccMethod{Construct_plane_3 construct_plane_3_object();}{}

\ccAutoIndexingOn

\ccHasModels
\ccRefIdfierPage{CGAL::Convex_hull_traits_3<R>}

%\ccSeeAlso

\end{ccRefConcept}
