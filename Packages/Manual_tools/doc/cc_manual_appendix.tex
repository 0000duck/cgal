\appendix
\section{Quick Reference Guide}
\newcommand{\Indent}{\hspace*{5pt}}

In this appendix we summarize the various commands that are defined in
{\tt cc\_manual.sty} and provide either an example of the result or a
brief explanation of what the command does.  

\tablehead{
\hline
&  \\
{\large \bf Command} & {\large \bf Example and/or Result} \\ 
& \\ \hline \hline
}
\tabletail{\hline}

\begin{ccClassTemplate}{Gnu<T>}
\ccCreationVariable{g}
\begin{supertabular}{|p{7.4cm}|p{7.4cm}|} 

\verb|\CC| & \CC \Mindex{CC}\\ \hline

\verb|\ccAccessFunctions| 
&  \ccAccessFunctions 
\ccIndexEntry{AccessFunctions}\\ \hline 

& \\
\begin{minipage}{2.0in}
\verb|\begin{ccAdvanced}| \\
\VarText{text} \\
\verb|\end{ccAdvanced}|
\end{minipage} &
\begin{minipage}{3.0in}
\begin{ccAdvanced} 
\VarText{text}
\end{ccAdvanced} 
\end{minipage} 
\Eindex{ccAdvanced} \\ 
& \\ \hline

\verb|\ccAmpersand| 
& \leavevmode\ccAmpersand 
\ccIndexEntry{Ampersand}\\ \hline

\verb|\ccc{|\VarText{text}\verb|}|
&\verb+\ccc{A f_bar(X<T>& x = "%^~#");}+ \\
& \\
&\ccc{A f_bar(X<T>& x = "%^~#");} 
\ccIndexEntry{c}\\ \hline

&\verb|\def\ccTagChapaterAuthor{\ccTrue}| \\
\verb|\ccChapterAuthor{|\VarText{author name}\verb|}| 
&\verb|\ccChapterAuthor{Irene}| \\
& \\
&{\em Irene}  
\ccIndexEntry{ChapterAuthor}\\ \hline

&\verb|\def\ccTagChapterRelease{\ccTrue}| \\
\verb|\ccChapterRelease{|\VarText{release info}\verb|}|
&\verb|\ccChapterRelease{3.4, 4 February 2000}| \\
& \\
&{\em 3.4, 4 February 2000} 
\ccIndexEntry{ChapterRelease}\\ \hline

\verb|\ccChapterSubTitle{|\VarText{subtitle}\verb|}| 
&\verb|\ccChapterSubTitle{subtitle}| \\
& \\
&{\em subtitle} 
\ccIndexEntry{ChapterSubTitle} \\ \hline

\verb|\begin{ccClass}{|\VarText{class\_name}\verb|}| 
&\Indent \verb|\begin{ccClass}{My_class}| ...  \\
             \VarText{class description} 
& sets the variable \verb|\ccClassName| to \ccc{My_class} (formatted using \verb|\ccc|);  sets \verb|\ccPureClassName| to \verb|My_class| (without formatting) \\
        \verb|\end{ccClass}| 
&
\Eindex{ccClass} \\ \hline

\verb|\ccClassName| & 
the name of the current (nontemplated) class formated using
\verb|\ccc|
\ccIndexEntry{ClassName}  \\ \hline


\verb|\begin{ccClassTemplate}{|\VarText{class\_name}\verb|}| 
& \verb|\begin{ccClassTemplate}{Gnu<T>}| ...\\
    \VarText{class description} 
& sets variable \verb|\ccClassTemplateName| to \ccc{Gnu<T>}
(with template parameters) (formatted using \verb|\ccc|);
sets \verb|\ccPureClassTemplateName| to \verb|Gnu<T>| (without formatting) \\
\verb|\end{ccClassTemplate}| & 
\Eindex{ccClassTemplate} \\ \hline

\verb|\ccCloseAngle| & \leavevmode\ccCloseAngle
\ccIndexEntry{CloseAngle} \\ \hline

\verb|\ccCommentHeading{|\VarText{text}\verb|}| 
&\verb|\ccCommentHeading{comment heading}| \\
& \\
&\ccCommentHeading{comment heading}
\ccIndexEntry{CommentHeading} \\ \hline

\verb|\ccClassTemplateName| & 
the name of the current class with its template parameters
formated using \verb|\ccc|
\ccIndexEntry{ClassTemplateName}  \\ \hline

\verb|\ccConstants| & \ccConstants 
\ccIndexEntry{Constants} \\ \hline

\verb|\ccConstructor{|\VarText{header}\verb|}{|\VarText{comment}\verb|}| 
& \verb|\ccConstructor{Gnu(Gnu<T> a);}{comment}| \\
& \\
& \ccConstructor{Gnu(Gnu<T> a);}{comment}
\ccIndexEntry{Constructor} \\ \hline

\verb|\ccCreation| & \ccCreation 
\ccIndexEntry{Creation} \\ \hline

\verb|\ccCreationVariable{|\VarText{var\_name}\verb|}| & defines \VarText{var\_name} as the variable to use when formatting member functions for a class 
\ccIndexEntry{CreationVariable} \\ \hline

\verb|\ccDate| & prints the date of the {\tt cc\_manual.sty} file being used
\ccIndexEntry{Date} \\ \hline

\verb|\ccDeclaration{|\VarText{decl}\verb|}| 
& \VarText{decl} produces no text in the manual unless \verb|\ccMakeAllVisible| 
has been used; used only for the {\tt cc\_check} tool
\ccIndexEntry{Declaration} \\ \hline

\verb|\ccDefGlobalScope{|\VarText{scope\_name}\verb|}| & 
makes \VarText{scope\_name} the text that is prepended to all reference page 
names (except concepts)  (default: empty)
\ccIndexEntry{DefGlobalScope} \\ \hline

\verb|\ccDefinition| & \ccDefinition 
\ccIndexEntry{Definition} \\ \hline

\verb|\renewcommand{\ccEndFont}{|\VarText{chars}\verb|}| 
& \verb|\renewcommand{\ccEndFont}{}| \\
& \VarText{chars} will be appended after typesetting arguments
provided to the \verb|\ccStyle| and \verb|\ccc| commands (default: \verb|\/|) 
\ccIndexEntry{EndFont}\\ \hline

\verb|\ccEnum{|\VarText{enum\_decl}\verb|}{|\VarText{comment}\verb|}| 
&\verb|\ccEnum{ enum E { E1, E2};}{comment}| \\
& \\
&\ccEnum{ enum E { E1, E2};}{comment} 
\ccIndexEntry{Enum} \\ \hline

\verb|\ccExample| & \ccExample 
\ccIndexEntry{Example} \\ \hline


\verb|\begin{ccExampleCode}| & \\
\VarText{source code}        & \VarText{source code} is typeset in \verb|\tt| 
                               with some formatting of comments, {\em etc.}\\
\verb|\end{ccExampleCode}|  &
\Eindex{ccExampleCode}\\ \hline

\verb|\ccFalse| & no result; a boolean variable used for defining values of other commands
\ccIndexEntry{False}\\ \hline

\verb|\renewcommand{\ccFont}{|\VarText{font}\verb|}| 
& \verb|\renewcommand{\ccFont}{\tt}| \\
& redefines the font to use with the \verb|\ccStyle| and \verb|\ccc| commands 
\ccIndexEntry{Font}\\ \hline

\verb|\ccFunction{|\VarText{header}\verb|}{|\VarText{comment}\verb|}| 
& \verb|\ccFunction{int foo( float d);}{comment}| \\
& \\
& \ccc{int}\hspace*{1.0cm}\ccc{foo( float d);}\hspace*{\fill}\ccc{comment}  \\
& 
\ccIndexEntry{Function} \\ \hline

\verb|\ccFunctionTemplate{|\VarText{parameters}\verb|}%| & \\
\Indent\Indent\verb|{|\VarText{decl.}\verb|}{|\VarText{comment}\verb|}| 
& this is obsolete and remains only for backward compatability
\ccIndexEntry{FunctionTemplate}\\ \hline

\verb|\ccGeneralizes| 
& \ccGeneralizes
\ccIndexEntry{Generalizes} \\ \hline

\verb|\ccGlobalEnum{|\VarText{enum declaration}\verb|}| 
&\verb+\ccGlobalEnum{ enum E { E1, E2};}+ \\
&\ccGlobalEnum{ enum E { E1, E2};}
\ccIndexEntry{GlobalEnum}\\ \hline

\verb|\ccGlobalFunction{|\VarText{declaration}\verb|}| 
& \verb+\ccGlobalFunction{ int foo(X x);}+  \\
& \\
& \ccc{int}\hspace*{1.0cm}\ccc{foo(X x);}  \\
& 
\ccIndexEntry{GlobalFunction}\\ \hline

\verb|\ccGlobalFunctionTemplate{|\VarText{templ. params}\verb|}{|\VarText{declaration}\verb|}| 
& this is obsolete and remains only for backward compatability
\ccIndexEntry{GlobalFunctionTemplate}\\ \hline

\verb|\ccGlobalScope| 
& contains the text given as the argument of \verb|\ccDefGlobalScope|
formatted using \verb|\ccc|
\ccIndexEntry{GlobalScope} \\ \hline

\verb|\ccGlobalStruct{|\VarText{declaration}\verb|}| 
&\verb+\ccGlobalStruct{ struct S { int i;};}+ \\
&\ccGlobalStruct{ struct S { int i;};}
\ccIndexEntry{GlobalStruct} \\ \hline

\verb|\ccGlobalTypedef{|\VarText{declaration}\verb|}| 
&\verb+\ccGlobalTypedef{ typedef A Sleep;}+ \\
& \\
&\ccc{typedef A}\hspace*{2.0cm}\ccc{Sleep;}  \\
&
\ccIndexEntry{GlobalTypedef} \\ \hline

\verb|\ccGlobalVariable{|\VarText{declaration}\verb|}| 
&\verb+\ccGlobalVariable{ const int i = 42;}+  \\
& \\
&\ccc{const int}\hspace*{2.0cm}\ccc{i = 42;} \\
&
\ccIndexEntry{GlobalVariable} \\ \hline


\verb|\ccGlue| 
& see \verb|\ccGlueDeclarations|
\ccIndexEntry{Glue} \\ \hline

\verb|\ccGlueBegin| 
& used with \verb|\ccGlueEnd| to shorten vertical spacing between 
declarations that come between the commands
\ccIndexEntry{GlueBegin} \\ \hline


\verb|\ccGlueDeclarations| 
& causes vertical spacing in the following declarations to be shortened
\ccIndexEntry{GlueDeclarations}\\ \hline

\verb|\ccGlueEnd| 
& see \verb|\ccGlueBegin|
\ccIndexEntry{GlueEnd}\\ \hline

\verb|\ccHasModels| 
& \ccHasModels
\ccIndexEntry{HasModels} \\ \hline

\verb|\ccHat| 
& \ccHat
\ccIndexEntry{Hat} \\ \hline

\verb|\ccHeading{|\VarText{heading}\verb|}|
& \verb|\ccHeading{A Heading}| \\
& \ccHeading{A Heading}
\ccIndexEntry{Heading} \\ \hline

\verb|\ccHidden| 
& causes the following declaration not to appear in the manual
\ccIndexEntry{Hidden}\\ \hline

\verb|\begin{ccHtmlClassFile}{|\VarText{file\_name}\verb|}%| & \\
\Indent\Indent\verb|{|\VarText{description}\verb|}| & \\
            \VarText{text} & writes \VarText{text} to \VarText{file\_name}
                             using \VarText{description} in the HTML table of 
                             contents and the anchor in the current chapter \\
\verb|\end{ccHtmlClassFile}| & 
\Eindex{ccHtmlClassFile} \\ \hline


\verb|\ccHtmlCrossLink{|\VarText{C++ identifier}\verb|}| 
& activates the automatic cross linking for \VarText{C++ identifier}
\ccIndexEntry{HtmlCrossLink}\\ \hline

\verb|\ccHtmlIndex[|\VarText{category}\verb|]{|\VarText{index key}\verb|}| 
&\verb+\ccHtmlIndex[Function]{Style guides}+  \\ 
& creates an entry in the HTML index for \VarText{index key} under category
 \VarText{category} (defualt: Class)
\ccIndexEntry{HtmlIndex} \\ \hline

\verb|\ccHtmlIndexC[|\VarText{category}\verb|]{|\VarText{C++\_id}\verb|}|
&\verb+\ccHtmlIndexC{My_point<R>}+ \\
& creates an entry in the HTML index for \VarText{C++\_id} under category
  \VarText{category} (default: Class)
\ccIndexEntry{HtmlIndexC} \\ \hline

\verb|\ccHtmlLinksOff| 
& inserts a comment into the current {\tt .html} file to indicate that all 
  text until the comment generated by the command \verb|\ccHtmlLinksOn| is 
  reached contain no links. 
\ccIndexEntry{HtmlLinksOff}\\ \hline

\verb|\ccHtmlLinksOn| 
& See \verb|\ccHtmlLinksOff|.
\ccIndexEntry{HtmlLinksOn}\\ \hline

\verb|\ccHtmlNoClassFile| 
& deactivates the creation of a separate HTML file for this class
\ccIndexEntry{HtmlNoClassFile}\\ \hline

\verb|\ccHtmlNoClassIndex|
& no HTML index entries for this class
\ccIndexEntry{HtmlNoClassIndex} \\ \hline

\verb|\ccHtmlNoClassLinks| 
& deactivates the cross linking for this class name
\ccIndexEntry{HtmlNoClassLinks}\\ \hline

\verb|\ccHtmlNoClassToc| 
& no entry in the HTML table of contents for this class
\ccIndexEntry{HtmlNoClassToc} \\ \hline

\verb|\ccHtmlNoIndex| 
& no HTML index entry for the following declaration
\ccIndexEntry{HtmlNoIndex}\\ \hline

\verb|\ccHtmlNoLinksFrom{|\VarText{text}\verb|}| 
& no links from (or to) any words or phrases in \VarText{text} will be created 
  in the HTML manual.
\ccIndexEntry{HtmlNoLinksFrom}\\ \hline

\verb|\ccHtmlNoLinks| 
& no cross linking for the following declaration
\ccIndexEntry{HtmlNoLinks}\\ \hline

\verb|\ccHtmlNoRefIndex| 
& no indexing of the identifier of the following reference page
\ccIndexEntry{HtmlNoRefIndex}\\ \hline

\verb|\ccHtmlNoRefLinks| 
& no cross linking for the identifier of the following reference page
\ccIndexEntry{HtmlNoRefLinks}\\ \hline

\verb|\begin{ccHtmlOnly}| & \\
\VarText{HTML text}       & \VarText{HTML text} is ignored by \LaTeX\
                            and included as raw HTML by 
                            {\tt cc\_manual\_to\_html}\\
\verb|\end{ccHtmlOnly}|  &
\Eindex{ccHtmlOnly}\\ \hline

\verb|\ccImplementation| 
& \ccImplementation
\ccIndexEntry{Implementation}\\ \hline


\verb|\ccInclude{|\VarText{include file name}\verb|}| 
& \verb|\ccInclude{CGAL/header.h}|\\
& \\
& \ccInclude{CGAL/header.h}
\ccIndexEntry{Include} \\ \hline

\verb|\ccIncludeExampleCode{|\VarText{file name}\verb|}| 
& file \VarText{file name} will be included and typeset in \\
& \verb|\tt| with font changes for comments and such 
\ccIndexEntry{IncludeExampleCode}\\ \hline

\verb|\ccIncludeVerbatim{|\VarText{file name}\verb|}| 
& file \VarText{file name} will be included and typeset in \\
& \verb|\tt| without any formatting
\ccIndexEntry{IncludeVerbatim}\\ \hline

\verb|\ccInheritsFrom| 
& \ccInheritsFrom 
\ccIndexEntry{InheritsFrom} \\ \hline

\verb|\ccInitConstructorWidths| 
& sets the values of \verb|\ccwFirst| and \verb|\ccwSecond| to
\verb|\ccwConstructorFirst| and \verb|\ccwConstructorSecond|, respectively 
for formatting constructors
\ccIndexEntry{InitConstructorWidths} \\ \hline

\verb|\ccInitFunctionWidths| 
& sets the values of \verb|\ccwFirst| and \verb|\ccwSecond| to
 \verb|\ccwFunctionFirst| and \verb|\ccwFunctionSecond|, respectively 
 for formatting functions
\ccIndexEntry{InitFunctionWidths} \\ \hline

\verb|\ccInitWidths| 
& sets the widths of columns for formatting declarations
\ccIndexEntry{InitWidths} \\ \hline

\verb|\ccIsModel| 
& \ccIsModel
\ccIndexEntry{IsModel} \\ \hline

\verb|\def\ccLongParamLayout{|\VarText{bool}\verb|}|
& \VarText{bool} is either \verb|\ccTrue| or \verb|\ccFalse| (default) \\
& if true, each function parameter will appear on a separate line
\ccIndexEntry{LongParamLayout}\\ \hline


\verb|\ccMakeAllVisible| 
& makes \verb|\ccDelclaration| and things designated as \verb|\ccHidden| 
appear in the manual
\ccIndexEntry{MakeAllVisible} \\ \hline

&\verb+\begin{ccClassTemplate}{Gnu<T>}+ \\
&\verb+\ccCreationVariable{g}+ \\
\verb|\ccMemberFunction{|\VarText{declaration}\verb|}%| 
&\verb+\ccMemberFunction{ int f(X x);}{comment}+ \\
\Indent\Indent\verb|{|\VarText{comment}\verb|}| & \\
& \ccc{int}\hspace*{1.0cm}\ccc{g.gnat(X x);}\hspace*{\fill}comment \\
&
\ccIndexEntry{MemberFunction} \\ \hline

\verb|\ccMethod{|\VarText{declaration}\verb|}{|\VarText{comment}\verb|}| 
& see \verb|\ccMemberFunction|
\ccIndexEntry{Method}\\ \hline

\verb|\ccModifiers| 
& \ccModifiers
\ccIndexEntry{Modifiers} \\ \hline

\verb|\ccMultiplePartsToc| 
& tells the HTML converter to format the table of contents for a manuals 
  with multiple \verb|\part| commands.
\ccIndexEntry{MultiplePartsToc} \\ \hline

&\verb+\begin{ccClass}{Gnu<T>}+ \\
\verb|\ccNestedType{|\VarText{declaration}\verb|}{|\VarText{comment}\verb|}|  
&\verb+\ccNestedType{ Stampede}{comment}+ \\
&\ccNestedType{ Stampede}{comment}
\ccIndexEntry{NestedType} \\ \hline

\verb|\gdef\ccNewRefManualStyle{|\VarText{bool}\verb|}|
& \VarText{bool} is either \verb|\ccTrue| or \verb|\ccFalse| (default)  
if true, each reference manual environment starts a new page and tabs 
will mark the edges of these pages 
\ccIndexEntry{NewRefManualStyle} \\ \hline

\verb|\ccNumberChaptersByPart| 
& After each \verb+\part+ command start numbering chapters at 1 again.
\ccIndexEntry{NumberChaptersByPart} \\ \hline

\verb|\ccOpenAngle| 
& \leavevmode\ccOpenAngle
\ccIndexEntry{OpenAngle} \\ \hline

\verb|\ccOperations| 
& \ccOperations 
\ccIndexEntry{Operations} \\ \hline

%\verb|\begin{ccPackage}| & \\
%\VarText{package description} &  \\
%& defines \verb|\ccIndexPackageName| used for producing the PostScript index \\
%\verb|\end{ccPackage}| &
%\Eindex{ccPackage} \\ \hline

\verb|\ccParDims| 
& restores the original settings of \verb+\parskip+ and \verb+\parindent+
\ccIndexEntry{ParDims} \\ \hline

\verb|\gdef\ccPortability{|\VarText{bool}\verb|}| 
& \VarText{bool} is either \verb|\ccTrue| or \verb|\ccFalse| (default)  \\& if true, old macro names are activated 
\ccIndexEntry{Portability} \\ \hline

\verb|\ccPostcond| 
& \verb|\ccPostcond: this is a postcondition| \\
& \\
& \ccPostcond: this is a postcondition
\ccIndexEntry{Postcond}\\ \hline

\verb|\ccPrecond| 
& \verb|\ccPrecond: this is a precondition| \\
& \\
& \ccPrecond: this is a precondition
\ccIndexEntry{Precond}\\ \hline

\verb|\ccPredicates| 
& \ccPredicates 
\ccIndexEntry{Predicates}\\ \hline

\verb|\ccPropagateThreeToTwoColumns| 
& propagates the current settings of the three-column layout to the two-column  
layout such that the descriptions align properly
\ccIndexEntry{PropagateThreeToTwoColumns}\\ \hline

\verb|\ccPureClassName|
& the value of the current class name without formatting as \CC\ identifier
\ccIndexEntry{PureClassName} \\ \hline

\verb|\ccPureClassTemplateName|
& the value of the current template class name without formatting as \CC\ identifier
\ccIndexEntry{PureClassTemplateName} \\ \hline

\verb|\ccPureGlobalScope| 
& the unformatted text given as the argument of 
 \verb|\ccDefGlobalScope|
\ccIndexEntry{PureGlobalScope} \\ \hline

\verb|\ccPureRefScope| 
& the unformatted text given as the optional argument to the current 
\verb|\ccRef*| environment 
\ccIndexEntry{PureRefScope} \\ \hline

\verb|\ccPureRefName| 
& the unformatted text given as the argument to the current 
\verb|\ccRef*| environmnet
\ccIndexEntry{PureRefName} \\ \hline

\verb|\ccPureVar| 
& the unformatted text supplied as the argument to 
\verb|\ccCreationVariable|
\ccIndexEntry{PureVar} \\ \hline

\verb|\ccQueryFunctions| 
& \ccQueryFunctions
\ccIndexEntry{QueryFunctions} \\ \hline

\verb|\ccRefCategory| 
& the category of the current reference page
\ccIndexEntry{RefCategory} \\ \hline


\verb|\ccRefChapter{|\VarText{chapter title}\verb|}{|\VarText{label}\verb|}| 
& makes a crosslink to the corresponding page of user manual with \VarText{label} and writes \VarText{chapter title}\\ \hline
\ccIndexEntry{RefChapter}

& \verb+\ccDefGlobalScope{CGAL::}+ \\
\verb|\begin{ccRefClass}[|\VarText{scope}\verb|]{|\VarText{class\_name}\verb|}| 
&\verb+\begin{ccRefClass}[Poly]{A_class<C>}+  \\
           \VarText{class description}  &  \\
        \verb|\end{ccRefClass}| &\\
& {\large\bf \ccPrintTokens Class CGAL::Poly::A_class<C>\ccEnd\ccEndFont}  \\
& \\
& defines a label for the class
that can be referred to by \verb|\ccRefPage| and \verb|\ccRefIdfierPage|
using \VarText{class name} as the argument; 
defines \verb|\ccRefCategory| to \verb|Class|; defines \verb|\ccRefScope|
and \verb|\ccPureRefScope| to \VarText{scope} with and without formatting, 
resp.; defines \verb|\ccRefName| and \verb|\ccPureRefName| to 
\VarText{class name} with and without formatting, resp.; defines
\verb|\ccClassTemplateName| and \verb|\ccPureClassTemplateName| to
\VarText{class name} with and without formatting, resp.
\Eindex{ccRefClass}\\ \hline

\verb|\begin{ccRefConcept}[|\VarText{scope}\verb|]%| 
& \verb+\begin{ccRefConcept}{Circulator}+ \\
\Indent\Indent \verb|{|\VarText{concept name}\verb|}|  & \\
           \VarText{concept description} & \\
        \verb|\end{ccRefConcept}| &  {\large\bf Concept Circulator}   \\
& \\
& defines labels and variables as for the {\tt ccRefClass} environment 
\Eindex{ccRefConcept}\\ \hline

\verb|\ccRefConceptPage{|\VarText{concept}\verb|}| 
& prints \VarText{concept} in the default font at the left
margin and ``page $x$'' on the right margin, where $x$ is the page on which
the reference page for \VarText{concept} begins
\ccIndexEntry{RefConceptPage} \\ \hline

& \verb+\ccDefGlobalScope{CGAL::}+ \\
\verb|\begin{ccRefEnum}[|\VarText{scope}\verb|]|\verb|{|\VarText{enum\_name}\verb|}| 
& \verb+\begin{ccRefEnum}{Orientation}+  \\
           \VarText{enum description}  & \\
        \verb|\end{ccRefEnum}|  &\\
& {\large\bf \ccPrintTokens Enum CGAL::Orientation\ccEnd\ccEndFont}   \\
& \\
& defines labels and variables as for the {\tt ccRefClass} environment 
\Eindex{ccRefEnum}\\ \hline

&  \verb+\ccDefGlobalScope{CGAL::}+ \\
\verb|\begin{ccRefFunction}[|\VarText{scope}\verb|]%| 
& \verb+\begin{ccRefFunction}{ch_jarvis}+ \\
\Indent\Indent\verb|{|\VarText{function name}\verb|}|   & \\
           \VarText{function description} &  \\
        \verb|\end{ccRefFunction}|  & \\ 
& {\large\bf \ccPrintTokens Function CGAL::ch_jarvis\ccEnd\ccEndFont}   \\
& \\
& defines labels and variables as for the {\tt ccRefClass} environment 
\Eindex{ccRefFunction}\\ \hline

&  \verb+\ccDefGlobalScope{CGAL::}+ \\
\verb|\begin{ccRefFunctionObjectClass}%|
& \verb+\begin{ccRefFunctionObjectClass}+ \\
\Indent\Indent\verb|[|\VarText{scope}\verb|]{|\VarText{class name}\verb|}|    
& \Indent\Indent\verb+{Less_xy_2<Point_2>}+ \\
           \VarText{function object description} &  \\
        \verb|\end{ccRefFunctionObjectClass}|  & \\ 
& {\large\bf \ccPrintTokens Function Object Class CGAL::Less_xy_2<Point_2>\ccEnd\ccEndFont}   \\
& \\
& defines labels and variables as for the {\tt ccRefClass} environment 
\Eindex{ccRefFunctionObjectClass}\\ \hline


&  \verb+\ccDefGlobalScope{CGAL::}+ \\
\verb|\begin{ccRefFunctionObjectConcept}%|
& \verb+\begin{ccRefFunctionObjectConcept}+ \\
\Indent\Indent\verb|[|\VarText{scope}\verb|]{|\VarText{concept name}\verb|}|  
& \verb+{Less_xy_2}+ \\
           \VarText{concept description} &  \\
        \verb|\end{ccRefFunctionObjectConcept}|  & \\ 
& {\large\bf \ccPrintTokens Function Object Concept Less_xy_2\ccEnd\ccEndFont}   \\
& \\
& defines labels and variables as for the {\tt ccRefClass} environment 
\Eindex{ccRefFunctionObjectClass}\\ \hline

\verb|\ccRefIdfierPage{|\VarText{identifier}\verb|}| 
& prints \VarText{identifier} formatted using \verb|\ccc| at the left
margin and ``page $x$'' on the right margin, where $x$ is the page on which
the reference page for \VarText{identifier} begins
\ccIndexEntry{RefIdfierPage} \\ \hline

\verb|\ccRefines| 
& \ccRefines
\ccIndexEntry{Refines} \\ \hline

\verb|\ccRefLabel{|\VarText{label name}\verb|}| 
& defines a label called \VarText{label name} that can be referred to by
\verb|\ccRefIdfierPage| or \verb|\ccRefPage|
\ccIndexEntry{RefLabel} \\ \hline

&  \verb+\ccDefGlobalScope{CGAL::}+ \\
\verb|\begin{ccRefMacro}[|\VarText{scope}\verb|]{|\VarText{macro}\verb|}|  
& \verb+\begin{ccRefMacro}{For_all}+ \\
           \VarText{macro description} & \\
        \verb|\end{ccRefMacro}| & \\
& {\large\bf \ccPrintTokens Function CGAL::For_all\ccEnd\ccEndFont}  \\
& \\
& defines labels and variables as for the {\tt ccRefClass} environment   
\Eindex{ccRefMacro}\\ \hline

\verb|\ccRefName| 
& contains the text given as the argument to the current \verb|\ccRef*|
environmnet, formatted using \verb|\ccc|
\ccIndexEntry{RefName} \\ \hline

\verb|\ccRefPage{|\VarText{label name}\verb|}| 
& produces the text ``page $x$'', where $x$ is the page on which the label
\VarText{label name} was defined 
\ccIndexEntry{RefPage}\\ \hline

\verb|\renewcommand\ccRefPageBegin{|\VarText{cmd}\verb|}| 
& causes \VarText{cmd} to be executed at the beginning of each
\verb|ccRef*| environment
\ccIndexEntry{RefPageBegin}\\ \hline

\verb|\gdef\ccRefPageBreak{|\VarText{bool}\verb|}| 
& \VarText{bool} is either \verb|\ccTrue| (default) or \verb|\ccFalse| 
 if true, causes a page break at the beginning and end of each 
\verb|ccRef*| environment when \verb|\ccNewRefManualStyle| is also
set to \verb|\ccTrue|;  if false, no page breaks will happen until 
\verb|\ccRefPageBreak| is set to true.
\ccIndexEntry{RefPageBreak}\\ \hline

\verb|\renewcommand\ccRefPageEnd{|\VarText{cmd}\verb|}| 
& causes \VarText{cmd} to be executed at the end of each
\verb|ccRef*| environment
\ccIndexEntry{RefPageEnd}\\ \hline

\verb|\renewcommand\ccRefPageFill{|\VarText{cmd}\verb|}| 
& causes \VarText{cmd} to be executed to fill the space between an identifier
and a page number in the commands \verb|\ccRefIdfierPage| and 
\verb|\ccRefConceptPage|.  Default command is \verb|\dotfill|.
\ccIndexEntry{RefPageFill}\\ \hline

\verb|\gdef\ccRefPageNumAtMargin{|\VarText{bool}\verb|}| 
& If this is set to \verb|\ccTrue| (default), the page numbers produced
  by \verb|\ccRefIdfierPage| and \verb|\ccRefConceptPage| appear at the 
  right margin with the space in between filled using 
  \verb|\ccRefPageFill|.  When set to \verb|\ccFalse|, the page numbers 
  appear directly after the name using the following format ``(pg. \#)''.
\ccIndexEntry{RefPageNumAtMargin}\\ \hline

\verb|\ccRefScope| 
& contains the text given as the optional argument to the current \verb|\ccRef*|
environment formatted using \verb|\ccc|
\ccIndexEntry{RefScope} \\ \hline

&  \verb+\ccDefGlobalScope{CGAL::}+ \\
\verb|\begin{ccRefVariable}[|\VarText{scope}\verb|]%| 
& \verb+\begin{ccRefVariable}{ORIGIN}+ \\
\Indent\Indent\verb|{|\VarText{variable name}\verb|}| & \\
           \VarText{variable description} &  \\
        \verb|\end{ccRefVariable}|  &\\
& {\large\bf \ccPrintTokens Variable CGAL::ORIGIN\ccEnd\ccEndFont}   \\
& \\
& defines labels and variables as for the {\tt ccRefClass} environment   
\Eindex{ccRefVariable}\\ \hline

\verb|\ccRestoreThreeColumns| 
&restores the values of \verb|\ccFunctionFirst| and \verb|\ccFunctionSecond|
saved by the last use of \verb|\ccSaveThreeColumns|.
\ccIndexEntry{RestoreThreeColumns} \\ \hline

\verb|\ccRestoreTwoColumns| 
&restores the value of \verb|\ccConstructorFirst| 
saved by the last use of \verb|\ccSaveTwoColumns|.
\ccIndexEntry{RestoreTwoColumns} \\ \hline

\verb|\ccRequire| & \ccRequire
\ccIndexEntry{Require} \\ \hline

\verb|\ccRequirements| & \ccRequirements
\ccIndexEntry{Requirements} \\ \hline

\verb|\ccRevision| 
&\ccRevision
\ccIndexEntry{Revision} \\ \hline

\verb|\ccSaveThreeColumns| 
&stores the values of \verb|\ccFunctionFirst| and \verb|\ccFunctionSecond|
in temporary variables so they may be restored using 
\verb|\ccRestoreThreeColumns|.
\ccIndexEntry{SaveThreeColumns} \\ \hline

\verb|\ccSaveTwoColumns| 
&stores the value of \verb|\ccConstructorSecond| 
in a temporary variable so it may be restored using 
\verb|\ccRestoreTwoColumns|.
\ccIndexEntry{SaveTwoColumns} \\ \hline

& \verb|\begin{ccClassTemplate}{Gnu<T>}| \\
\verb|\ccSection{|\VarText{section title}\verb|}| 
& \verb|\ccSection{Section}| \\ 
& \\
& {\Large\bf 2 Section (\ccc{Gnu<T>})} 
%& (assuming command used after \verb|\begin{ccClassTemplate}{Gnu<T>}|)
\ccIndexEntry{Section}\\ \hline

\verb|\ccSeeAlso|
& \ccSeeAlso 
\ccIndexEntry{SeeAlso} \\ \hline

\verb|\ccSetOneOfTwoColumns{|\VarText{dimension}\verb|}| 
& sets the first column width in the two-column layout to the dimension given 
 and sets the second to the remaining width based on the value of \verb|\textwidth|
\ccIndexEntry{SetOneOfTwoColumns} \\ \hline

\verb|\ccSetThreeColumns{|\VarText{text1}\verb|}{|\VarText{text2}\verb|}%| &\\
\Indent\Indent\verb|{|\VarText{text3}\verb|}| 
& one of \VarText{text1}, \VarText{text2}, and \VarText{text3} must be  empty;
 sets the column widths for three-column layout based on the size of the 
 text arguments given and the value of \verb|\textwidth| 
\ccIndexEntry{SetThreeColumns}\\ \hline

\verb|\ccSetTwoColumns{|\VarText{text1}\verb|}{|\VarText{text2}\verb|}|
& one of \VarText{text1} and \VarText{text2} must be empty; 
 sets the column widths for two-column layout based on the size 
 of the text argument given and the value of \verb|\textwidth|
\ccIndexEntry{SetTwoColumns} \\ \hline

\verb|\ccSetTwoOfThreeColumns{|\VarText{col1\_dim}\verb|}%| & \\
\Indent\Indent\verb|{|\VarText{col2\_dim.}\verb|}| 
& sets the first two column widths in the three-column layout to the dimensions
given 
 and sets the third to the remaining width based on the value of \verb|\textwidth|
\ccIndexEntry{SetTwoOfThreeColumns}\\ \hline

\verb|\gdef\ccSrcPrefix{|\VarText{old prefix}\verb|}| 
& specifies the identifier prefix that is to be replaced by
  \verb|\ccTargetPrefix| during parsing
\ccIndexEntry{SrcPrefix}\\ \hline

\verb|\ccStruct{|\VarText{struct\_decl}\verb|}{|\VarText{comment}\verb|}| 
&\verb+\ccStruct{ struct S { int i;};}{comment}+ \\
&\ccStruct{ struct S { int i;};}{comment}
\ccIndexEntry{Struct}\\ \hline

\verb|\ccStyle{|\VarText{text}\verb|}| 
& parses \VarText{text} as \CC\ code
\ccIndexEntry{Style}\\ \hline

& \verb|\begin{ccClassTemplate}{Gnu<T>}| \\
\verb|\ccSubsection{|\VarText{subsection name}\verb|}| 
& \verb|\ccSubsection{Subsection}| \\
& \\
& {\bf 2.1 Subsection (\ccc{Gnu<T>})} 
\\ \hline
\verb|\def\ccTagChapterAuthor{|\VarText{bool}\verb|}|
& \VarText{bool} is either \verb|\ccTrue| or \verb|\ccFalse| (default) 
 if true, chapter author names supplied with \verb|\ccChapterAuthor| will be
printed 
\ccIndexEntry{TagChapterAuthor} \\ \hline

\verb|\ccTagChapterRelease{|\VarText{bool}\verb|}|
& \VarText{bool} is either \verb|\ccTrue| or \verb|\ccFalse| (default) 
 if true, chapter release information supplied with \verb|\ccChapterRelease| will be printed 
\ccIndexEntry{TagChapterRelease}\\ \hline

\verb|\ccTagDefaults| 
& reverts to the default settings of the \verb|\ccTag*| variables
\ccIndexEntry{TagDefaults} \\ \hline

\verb|\ccTagFullDeclarations| 
& deactivates the simplifcation for readability of \CC\ declarations done 
during parsing
\ccIndexEntry{TagFullDeclarations}\\ \hline

\verb|\def\ccTagOperatorLayout{|\VarText{bool}\verb|}|
& \VarText{bool} is either \verb|\ccTrue| (default) or \verb|\ccFalse|;
controls whether an operator declaration is formatted as a function 
declaration or using operator notation
\ccIndexEntry{TagOperatorLayout} \\ \hline

\verb|\def\ccTagReplaceInclude{|\VarText{bool}\verb|}|
& \VarText{bool} is either \verb|\ccTrue| or \verb|\ccFalse| (default);
if true, will replace \verb|\ccSrcPrefix| by \verb|\ccTargetPrefix| in
include paths
\ccIndexEntry{TagReplaceInclude} \\ \hline

\verb|\ccTexHtml{|\VarText{\LaTeX\ text}\verb|}%| & \\
\Indent\Indent \verb|{|\VarText{HTML text}\verb|}| 
& \VarText{\LaTeX\ text} will be processed by \LaTeX\ only;  \\
& \VarText{HTML text} will be processed by {\tt cc\_manual\_to\_html} only
\ccIndexEntry{TexHtml} \\ \hline


\verb|\def\ccTagReplacePrefix{|\VarText{bool}\verb|}|
& \VarText{bool} is either \verb|\ccTrue| or \verb|\ccFalse| (default);\\ 
& if true, will replace \verb|\ccSrcPrefix| by \verb|\ccTargetPrefix| for all
identifiers parsed by the macros
\ccIndexEntry{TagReplacePrefix} \\ \hline

\verb|\def\ccTagRmConstRefPair{|\VarText{bool}\verb|}|
& \VarText{bool} is either \verb|\ccTrue| (default) or \verb|\ccFalse|; \\ 
& controls whether {\tt const\ldots\&}-pairs are removed or not. 
\ccIndexEntry{TagRmConstRefPair} \\ \hline

\verb|\def\ccTagRmEigenClassName{|\VarText{bool}\verb|}|
& \VarText{bool} is either \verb|\ccTrue| (default) or \verb|\ccFalse|; \\ 
& controls whether the name of a class within a class environment is removed 
  from function argument lists. 
\ccIndexEntry{TagRmEigenClassName} \\ \hline

\verb|\def\ccTagRmTrailingConst{|\VarText{bool}\verb|}|
& \VarText{bool} is either \verb|\ccTrue| (default) or \verb|\ccFalse|; \\ 
& controls whether the text appearing after the closing parenthesis of a 
function declaration is removed or not
\ccIndexEntry{TagRmTrailingConst} \\ \hline

\verb|\def\ccTagTemplateInline{|\VarText{bool}\verb|}|
& \VarText{bool} is either \verb|\ccTrue| or \verb|\ccFalse| (default); \\ 
& if true, template parameters will appear on the same line as the rest of the
function header
\ccIndexEntry{TagTemplateInline} \\ \hline


\verb|\gdef\ccTargetPrefix{|\VarText{new prefix}\verb|}| 
& \VarText{new prefix} will be substituted for \VarText{old prefix} 
(see \verb|\ccSrcPrefix|) during parsing
\ccIndexEntry{TargetPrefix} \\ \hline

\verb|\begin{ccTexOnly}| & \VarText{\LaTeX\ text} will be processed by \LaTeX\\
\VarText{\LaTeX\ text}   & but ignored by {\tt cc\_manual\_to\_html} program \\
\verb|\end{ccTexOnly}| &
\Eindex{ccTexOnly} \\ \hline

\verb|\ccThree{|\VarText{text1}\verb|}{|\VarText{text2}\verb|}{|\VarText{text3}\verb|}|
& see \verb|\ccSetThreeColumns|
\ccIndexEntry{Three} \\ \hline

\verb|\ccThreeToTwo| 
& see \verb|\ccPropagateThreeToTwoColumns|
\ccIndexEntry{ThreeToTwo} \\ \hline

\verb|\ccTilde| 
& \leavevmode\ccTilde
\ccIndexEntry{Tilde}\\ \hline


\verb|\begin{ccTraitsClass}{|\VarText{traits\_class}\verb|}%| & \\
\Indent\verb|{|\VarText{class list}\verb|}{|\VarText{package list}\verb|}|
& \verb|\begin{ccTraitsClass}{good_traits}%| \\
& \Indent\verb|{class1; class2}{package}| \\
\VarText{class description} & ... \\
\verb|\end{ccTraitsClass}| & \\
& see \verb|ccClass|
\Eindex{ccTraitsClass} \\ \hline

\verb|\begin{ccTraitsClassTemplate}{|\VarText{t\_class}\verb|}%| & \\
\Indent\verb|{|\VarText{class list}\verb|}{|\VarText{package list}\verb|}|
& \verb|\begin{ccTraitsClass}{better_traits<T>}%| \\
& \Indent\verb|{}{p1;p2}| \\
\VarText{class description} & ... \\
\verb|\end{ccTraitsClassTemplate}| & \\
& see \verb|ccClassTemplate|
\Eindex{ccTraitsClassTemplate} \\ \hline

\verb|\ccTrue| 
& no result; boolean value used to set the value for other commands
\ccIndexEntry{True} \\ \hline

\verb|\ccTwo{|\VarText{text1}\verb|}{|\VarText{text2}\verb|}| 
& see \verb|\ccSetTwoColumns|
\ccIndexEntry{Two}\\ \hline

\verb|\ccTypedef{|\VarText{typedef}\verb|}{|\VarText{comment}\verb|}| 
&\verb+\ccTypedef{ typedef A Sleep;}{comment}+   \\
& \\
&\ccc{typedef A}\hspace*{1.0cm}\ccc{Sleep;}\hspace*{\fill}comment \\
& 
\ccIndexEntry{Typedef}\\ \hline

\verb|\ccTypes| 
& \ccTypes
\ccIndexEntry{Types} \\ \hline

\verb|\ccUnderscore| 
& \ccUnderscore
\ccIndexEntry{Underscore} \\ \hline

\verb|\ccUserChapter{|\VarText{chapter title}\verb|}{|\VarText{label}\verb|}| 
& makes a crosslink to the corresponding page of reference manual with \VarText{label} and writes \VarText{chapter title}\\ \hline
\ccIndexEntry{UserChapter}


\verb|\ccVar| 
& contains the text supplied as the argument to \verb|\ccCreationVariable|,
formatted using \verb|\ccc|
\ccIndexEntry{Var} \\ \hline

\verb|\ccVariable{|\VarText{declaration}\verb|}{|\VarText{comment}\verb|}| 
&\verb+\ccVariable{ const int i = 42;}{comment}+  \\
& \\
&\ccc{const int}\hspace*{1.0cm}\ccc{i = 42;}\hspace*{\fill}{comment} \\
&
\ccIndexEntry{Variable}\\ \hline

\verb|\ccwBetween| 
& the padding space between columns
\ccIndexEntry{wBetween}\\ \hline


\verb|\ccwConstructorFirst| 
& the width of the first column used for constructor formatting
\ccIndexEntry{wConstructorFirst} \\ \hline

\verb|\ccwConstructorSecond| 
&  the width of the second column used for constructor formatting
\ccIndexEntry{wConstructorSecond} \\ \hline

\verb|\ccwFirst| 
&  the width of the first column for the current declaration
\ccIndexEntry{wFirst} \\ \hline

\verb|\ccwFunctionFirst| 
&  the width of the first column used for function formatting
\ccIndexEntry{wFunctionFirst} \\ \hline

\verb|\ccwFunctionSecond| 
& the width of the second column used for function formatting
\ccIndexEntry{wFunctionSecond}\\ \hline


\verb|\ccwIndent| 
& width of indentation before the first column 
\ccIndexEntry{wIndent} \\ \hline

\verb|\ccwRightMargin| 
& width of the right margin for declaration formatting
\ccIndexEntry{wRightMargin} \\ \hline

\verb|\ccwSecond| 
& the width of the second column for the current declaration
\ccIndexEntry{wSecond} \\ \hline
\verb|\cgal| 
& \cgal
\Mindex{cgal} \\ \hline

\verb|\E| 
& \E
\Mindex{E} \\ \hline

\verb|\galia| 
& \galia
\Mindex{galia} \\ \hline

\verb|\gcc| 
& \gcc 
\Mindex{gcc} \\ \hline

\verb|\leda| 
& \leda
\Mindex{leda} \\ \hline

\verb|\listofrefpages| 
& produces a file with extension \texttt{.ref} that contains a list of
  all the reference pages.  When \LaTeX\ is run again this list will
  be inserted at the place where the command appears.  Sort the \texttt{.ref}
  file before the second run of \LaTeX\ to get a sorted list of 
  reference pages.
\Mindex{listofrefpages} \\ \hline

\verb|\N| 
& \N
\Mindex{N} \\ \hline

\verb|\nat| 
& \nat
\Mindex{nat} \\ \hline

\verb+\nonlinkedpath|+\VarText{path name}\verb+|+
& The delimeters (here {\verb+|+}) are chosen as for the \verb|\verb| command.  
Formats \VarText{path name} using \verb|\tt| and uses hyphen-less line breaks 
if necessary; identical to \verb|\path| for \LaTeX\ but for HTML creates no
link for \VarText{path name} 
\Mindex{nonlinkedpath}\\ \hline

\verb+\path|+\VarText{path name}\verb+|+
& The delimeters (here {\verb+|+}) are chosen as for the \verb|\verb| command.  
Formats \VarText{path name} using \verb|\tt| and uses hyphen-less line breaks 
if necessary; for HTML creates a link for \VarText{path name} 
\Mindex{path}\\ \hline

\verb|\plageo| (obsolete)
& \plageo
\Mindex{plageoo} \\ \hline

\verb|\protocgal| (obsolete)
&\protocgal
\Mindex{protocgal} \\ \hline

\verb|\Q|
& \Q
\Mindex{Q} \\ \hline

\verb|\R| 
& \R
\Mindex{R} \\ \hline

\verb|\RCSdef{|\VarText{macro}\verb|}{|\VarText{RCS entry}\verb|}| 
&\verb+\RCSdef{\a}{$Revision$}+ \\
&\verb|\a| now has the value of the second argument supplied by RCS
without the \$s
\Mindex{RCSdef} \\ \hline

\verb|\RCSdefDate{|\VarText{macro}\verb|}{|\VarText{RCS date}\verb|}| 
& \verb+\RCSdefDate{\b}{$Date$}+ \\
& \verb|\b| now has the value of the second argument supplied by RCS
without the \$s
\Mindex{RCSdefDate}\\ \hline

\verb|\real| 
&\real
\Mindex{real} \\ \hline
\verb|\Z| 
& \Z
\Mindex{Z}\\ \hline
\end{supertabular} 

\end{ccClassTemplate}
