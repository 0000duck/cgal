% +------------------------------------------------------------------------+
% | Reference manual page: Orientation.tex
% +------------------------------------------------------------------------+
% | 03.09.1999   Lutz Kettner
% | Package: Example
% | 
\RCSdef{\RCSOrientationRev}{$Revision$}
\RCSdefDate{\RCSOrientationDate}{$Date$}
% |
%%RefPage: end of header, begin of main body
% +------------------------------------------------------------------------+


\begin{ccRefEnum}{Orientation}  %% add template arg's if necessary

\ccRefLabel{CGAL::NEGATIVE}  %% add label for table-of-contents in intro.tex.
\ccRefLabel{CGAL::POSITIVE}  %% add label for table-of-contents in intro.tex.
\ccHtmlCrossLink{NEGATIVE}     %% add further rules for cross referencing links
\ccHtmlCrossLink{POSITIVE}     %% add further rules for cross referencing links
\ccHtmlIndexC[enum_tags]{NEGATIVE} %% add further index entries
\ccHtmlIndexC[enum_tags]{POSITIVE} %% add further index entries

\ccDefinition
  
The enum \ccRefName\ does this and that.

\ccInclude{DIR/Orientation.h}

\ccGlobalEnum{enum Orientation { NEGATIVE, POSITIVE };}

\ccIsModel

\ccc{Concept}

\ccSeeAlso

\ccc{Some_other_enum},
\ccc{some_other_function}.

\ccExample

A short example program.
Instead of a short program fragment, a full running program can be
included using the 
\verb|\ccIncludeExampleCode{examples/Example/Orientation_prog.C}| 
macro. The program example would be part of the source code distribution and
also part of the automatic test suite.

\begin{ccExampleCode}
void your_example_code() {
}
\end{ccExampleCode}

%% \ccIncludeExampleCode{examples/Example/Orientation_prog.C}

\end{ccRefEnum}

% +------------------------------------------------------------------------+
%%RefPage: end of main body, begin of footer
% EOF
% +------------------------------------------------------------------------+

