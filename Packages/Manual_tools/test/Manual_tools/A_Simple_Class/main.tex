\begin{ccClass}{Demo_Class1}

Text\footnote{A footnote in a class environment.}.

\ccCreationVariable{p}

\ccStyle{#include< Demo_Class1.h>}
\ccTwo{Demo_Class1 p;X}{}

  \ccConstructor{Demo_Class1();}{introduces
     a variable \ccVar\ initialized to the default. \CC\ code:
     {\tt Demo\_Class1();}. Test
     ccStyle: \ccStyle{Underscore\_within  ccStyle}.}

  \ccConstructor{Demo_Class1( const Demo_Class1 &);}{copy
    constructor. \CC\ code: {\tt Demo\_Class1(const Demo\_Class1 \&);}}

\ccThree{Demo_Class1}{p.x();}{}

  \ccMethod{FT   x()  const;}{Cartesian x-coordinate\protect\footnotemark{}.
    \CC\ code: {\tt FT x() const;}}

  \ccMethod{const FT&   y();}{Cartesian y-coordinate. \CC\ code: {\tt
      const FT\& y();}}

  \ccMethod{Demo_Class1   
             transform( const CGAL_HAff_transformation<FT,RT> &t) const;
             }{ Longish declarations forces the comment to start in
             the next line.}

  \ccMethod{Demo_Class1   
             longish_function_name( 
                        const CGAL_Aff_transformation<FT,RT> &t,
                        const Dummy_Type &q,
                        Long_Type_Name Variable_Also_Long) const;
             }{ Even more longish declarations forces the parameters printed
             one per line. This was the default formatting.}
           
\footnotetext{Another footnote in this class file,
  implemented with mark and text.}

\end{ccClass}

% ----------------------------------------------------------------------
