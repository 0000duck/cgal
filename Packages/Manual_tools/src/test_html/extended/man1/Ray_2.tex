% $Id$

\begin{ccClassTemplate} {CGAL_Ray_2<R>}
\ccSection{2D Ray}

\ccDefinition
An object \ccStyle{r} of the data type \ccClassName\ is a directed
straight ray in the two-dimensional Euclidean plane $\E_2$. It starts
in a  point called the {\em source} of  \ccStyle{r} and it reaches infinity.

\ccCreation
\ccCreationVariable{r}

\ccInclude{CGAL/Ray_2.h}

\ccHidden \ccConstructor{CGAL_Ray_2();}
             {introduces an uninitialized variable \ccVar.}

\ccHidden \ccConstructor{CGAL_Ray_2(const CGAL_Ray_2<R> &s);}
 	    {copy constructor.}

\ccConstructor{CGAL_Ray_2(const CGAL_Point_2<R> &p, const Point_2 &q);}
            {introduces a ray \ccVar\ 
             with source $p$ and passing through point $q$.}

\ccConstructor{CGAL_Ray_2(const CGAL_Point_2<R> &p, const CGAL_Direction_2<R> &d)}
            {introduces a ray \ccVar\ starting at source $p$ with 
             direction $d$.}

\ccOperations
\ccSetTwoOfThreeColumns{5cm}{4cm}

\ccHidden \ccMethod{CGAL_Ray_2<R> &operator=(const CGAL_Ray_2<R> &s);}
        {Assignment.}

\ccMethod{bool operator==(const CGAL_Ray_2<R> &h) const;}
       {Test for equality: two rays are equal, iff they have the same 
        source and the same direction.}

\ccMethod{bool operator!=(const CGAL_Ray_2<R> &h) const;}
       {Test for inequality.}



\ccMethod{CGAL_Point_2<R> source() const;}
       {returns the source of \ccVar.}


\ccMethod{CGAL_Point_2<R> point(int i) const;}
       {returns a point on \ccVar. \ccStyle{point(0)} is the source,
        \ccStyle{point(i)}, with $i>0$, is different from the 
        source. \ccPrecond $i \geq 0$.}

\ccMethod{CGAL_Direction_2<R> direction() const;}
       {returns the direction of \ccVar.}

\ccMethod{CGAL_Line_2<R>      supporting_line() const;}
       {returns the line supporting \ccVar\ which has the same direction.}

\ccMethod{CGAL_Ray_2<R>       opposite() const;}
       {returns the ray with the same source and the opposite direction.}

\ccMethod{bool is_degenerate() const;}
       {ray \ccVar\ is degenerate, if the source and the second defining
        point fall together (that is if the direction is degenerate).}

\ccMethod{bool is_horizontal() const;}
       {}

\ccMethod{bool is_vertical() const;}
       {}

\ccMethod{bool has_on(const CGAL_Point_2<R> &p) const;}
       {A point is on \ccVar, iff it is equal to the source
        of \ccVar, or if it is in the interior of \ccVar.}

\ccMethod{bool collinear_has_on(const CGAL_Point_2<R> &p) const;}
       {checks if point $p$ is on \ccVar. This function is faster
        than function \ccStyle{has_on()}.
        \ccPrecond{$p$ is on the supporting line of \ccVar.}}

\ccMethod{CGAL_Ray_2<R> transform(const CGAL_Aff_transformation_2<R> &t) const;}
       {returns the ray obtained by applying $t$ on the source
        and on the direction of \ccVar.}

\ccImplementation

A ray is stored as a point and a direction.


\end{ccClassTemplate} 

% $Log$
% Revision 1.1  2001/08/27 18:00:36  spion
% Initial revision
%
% Revision 1.2  1996/07/08 12:19:15  fabri
% *** empty log message ***
%
% Revision 1.1  1995/10/19 18:22:12  fabri
% Initial revision
%
