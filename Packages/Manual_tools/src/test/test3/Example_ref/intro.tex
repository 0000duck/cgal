% +------------------------------------------------------------------------+
% | intro.tex
% +------------------------------------------------------------------------+
% | chapter title, some introduction and a kind of table-of-contents
% +------------------------------------------------------------------------+

\chapter{Examples of \CC\ Reference Pages}
\ccChapterRelease{\ccRevision. \ \ccDate}\\
\ccChapterAuthor{Lutz Kettner}

\section*{Summary}

This example chapter belongs to the documentation of the \LaTeX\ style
file {\tt cc\_manual.sty} and the related tools, for example, the
\LaTeX\ to HTML converter. See the related documention for the details.

This example contains a reference page for an item of each category:
concept, class, enum, function, variable and macro. It documents no
particular functionality. It serves only as an example for the various
layouts. The actual \LaTeX\ sources and the README file in the {\tt
  Tools/example/} directory should be studied in parallel to the
formatted output, they contain further comments and descriptions of
the tools used to create these pages. For example, the
table-of-contents below uses not only the \verb+\ccRefIdfierPage+
macro, but needs also a more hand-written layout for the
\verb+is_empty_range+ function with its parameter list.  All items,
except concepts and macros, are given including the global scope.

\subsection*{Concepts}

\ccRefIdfierPage{Circulator}

\subsection*{Classes}

\ccRefIdfierPage{CGAL::Circulator_traits<C>}

\subsection*{Enums}

\ccRefIdfierPage{CGAL::Orientation}\\
\ccRefIdfierPage{CGAL::POSITIVE}\\
\ccRefIdfierPage{CGAL::NEGATIVE}

\subsection*{Functions}

\ccThree{Orientation}{}{page 100}
\ccFunction{bool CGAL::is_empty_range ( IC i, IC j);}
    {\hfill \ccRefPage{CGAL::is_empty_range}}

\subsection*{Variable}

\ccRefIdfierPage{CGAL::ORIGIN}

\subsection*{Macros}

\ccc{For_all(i,j)}\hfill \ccRefPage{For_all}

%% EOF %%
