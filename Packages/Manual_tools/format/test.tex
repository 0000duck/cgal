% ----------------------------------------------------------------------
% Test document for the CGAL manual LaTeX style:
% 
% 16.08.1995   Lutz Kettner
% 29.09.1999   Susan Hert
%
% $Revision$
% $Date$
% ----------------------------------------------------------------------
% The style is compatible with LaTeX2e:
\documentclass[12pt]{article}

\usepackage{latexsym}
\usepackage{amssymb}
\usepackage{makeidx}
\usepackage{latex_converter}
\usepackage{cc_manual}
\usepackage{cc_manual_index}
\usepackage{alltt}
\usepackage{cprog}
\usepackage{path}

\textwidth 15.4cm 
\textheight 24cm
\topmargin -14mm       
\evensidemargin 3mm 
\oddsidemargin 3mm

\parindent0em
\setlength{\parskip}{1ex minus 0.9ex}

\makeindex

\sloppy

% ----------------------------------------------------------------------
\title {Test Suite for the cc\_manual.sty}
\author{Lutz Kettner and Susan Hert}
\date{\ccRevision. \ccDate}

\begin{document}

\maketitle

% ----------------------------------------------------------------------
\section{Introduction}

This document is a test suite for the {\tt cc\_manual.sty}. It is
arbitrary structured and mostly no introductory explanations will be given.
Each section will test a particularly feature. Sections were written
while developing the style or while debugging. Thanks for bug-reports
made so far and for future reports. Please send bug-reports to {\tt
hert@mpi-sb.mpg.de}.


% ----------------------------------------------------------------------
\section{Structuring Macros}

\ccDefinition Here comes the general introductory explanation for a
class etc. The necessary include files.

\ccInclude{CGAL/Vector_2.h}\\
\ccInclude{CGAL/Point_2.h}

\ccAccessFunctions

\ccPrecond Specific conditions can be stated as preconditions in any place.

\ccCreation Constructors are defined and explained here.

% ----------------------------------------------------------------------
\section{ccStyle Character Test}

The formatting macros handle for example the special characters \_,
{\tt <}, {\tt >}, and {\tt \&} as they appear in \CC\ names and the
characters \#, \%, \ccHat, and \ccTilde\ as they appear with
operators. See the following example that is printed using the
\verb+\cc+{\tt Style} macro\footnote{A footnote in the main text.}:

\ccStyle{
    #define %^~$@&$ Ret_Type<Template_Param, X<Y> > fct_foo( const X<Y> &a);}

% ----------------------------------------------------------------------
\begin{ccClass}{Demo_Class1}
\ccSection{A Simple Class}

Text\footnote{A footnote in a class environment.}.

\ccCreationVariable{p}

\ccStyle{#include< Demo_Class1.h>}
\ccTwo{Demo_Class1 p;X}{}

  \ccConstructor{Demo_Class1();}{introduces
     a variable \ccVar\ initialized to the default. \CC\ code:
     {\tt Demo\_Class1();}. Test
     ccStyle: \ccStyle{Underscore\_within  ccStyle}.}

  \ccConstructor{Demo_Class1( const Demo_Class1 &);}{copy
    constructor. \CC\ code: {\tt Demo\_Class1(const Demo\_Class1 \&);}}

\ccThree{Demo_Class1}{p.x();}{}

  \ccMethod{FT   x()  const;}{Cartesian x-coordinate\protect\footnotemark{}.
    \CC\ code: {\tt FT x() const;}}

  \ccMethod{const FT&   y();}{Cartesian y-coordinate. \CC\ code: {\tt
      const FT\& y();}}

  \ccMethod{Demo_Class1   
             transform( const CGAL_HAff_transformation<FT,RT> &t) const;
             }{ Longish declarations forces the comment to start in
             the next line.}

  \ccMethod{Demo_Class1   
             longish_function_name( 
                        const CGAL_Aff_transformation<FT,RT> &t,
                        const Dummy_Type &q,
                        Long_Type_Name Variable_Also_Long) const;
             }{ Even more longish declarations forces the parameters printed
             one per line. This was the default formatting.}
           
\footnotetext{Another footnote in this class file,
  implemented with mark and text.}

\end{ccClass}

% ----------------------------------------------------------------------
\section{Spaces Around Pointer * and Reference \& Symbols}

Reference or pointer parameters can occur on both sides of the
separating space between the return type and the function or method
name. The formatting normalizes them to the left side. This
formatting is not done within the\footnotemark{} parameters (, but maybe in the
future). An example: 

\footnotetext{Another footnote in this main file,
  implemented with mark and text. Contains enough text to have a 
linebreak. Or how about a paragraph?

Here it is. Or special macros and images, like \LaTeX.}%

\ccThree{Demo_Class&&}{p.x();}{}

    \ccFunction{Demo_Class& foo1( int& a, int* b);}{}
    \ccGlue
    \ccFunction{Demo_Class* foo2( int& a, int* b);}{}
    \ccGlue
    \ccFunction{Demo_Class &foo3( int &a, int *b);}{}
    \ccGlue
    \ccFunction{Demo_Class *foo4( int &a, int *b);}{}

% ----------------------------------------------------------------------
\begin{ccClass}{Demo_Class2}
\section{Operator Test}
\ccCreationVariable{p}

Type casting through a conversion operator is the default behavior 
for the formatting routine if the return type before the {\tt 
operator} keyword is empty.

\ccThree{Demo_Class&&}{A< FT>(p);}{}
  \ccMethod{ operator int () const;}{Conversion operator.}
    \ccGlue
  \ccMethod{operator A<FT>() const;}{Conversion operator.}

Sometimes, there is a choice between
implementing an operator as a method or as a function. Both
declarations will produce the same formatting, as demonstrated
with the next two declarations.


  \ccFunction{Demo_Class2
        operator+(Demo_Class2 p, Demo_Class2 q);}{Declaration via function.}
    \ccGlue
  \ccMethod{Demo_Class2
        operator+(Demo_Class2 q);}{Declaration via method.}

One can locally activate that the operator declaration is shown as it
is written without operator formatting, {\tt const ...\&}, classname, or
trailing const declarations for methods removal. This can be done with
\verb+\+\verb+ccTagFullDeclarations+ within a scope of braces {\tt
  \{...\}}.

\ccTagFullDeclarations
  \ccMethod{Demo_Class2
        operator+(const Demo_Class2& q) const;}{Declaration via method.}
\ccTagDefaults

There is some laziness allowed in placing spaces around the operator
characters. See the following examples:

\ccMethod{A
    operator+(Demo_Class2 q);}{\CC\ code: {\tt A operator+(Demo\_Class2 q);}}
    \ccGlue
\ccMethod{A
    operator +(Demo_Class2 q);}{\CC\ code: {\tt A operator +(Demo\_Class2 q);}}
    \ccGlue
\ccMethod{A
    operator+ (Demo_Class2 q);}{\CC\ code: {\tt A operator+ (Demo\_Class2 q);}}
    \ccGlue
\ccMethod{A
    operator + (Demo_Class2 q);}{\CC\ code: 
                                {\tt A operator + (Demo\_Class2 q);}}

The keyword {\tt operator} is reserved, but it can appear as a
substring in another name. See the following examples that this
style can handle such cases:

\ccMethod{A foo_operator(Demo_Class2 q);}{}
    \ccGlue
\ccMethod{A noperator(Demo_Class2 q);}{}
    \ccGlue
\ccMethod{A operatoro(Demo_Class2 q);}{}
    \ccGlue
\ccMethod{A operator_(Demo_Class2 q);}{}
    \ccGlue
\ccMethod{A operator0(Demo_Class2 q);}{}

A problem has occured in detecting the operator keyword if it was
directly preceded by an {\tt \&} or {\tt *} character. It is fixed as
the following example demonstrates:

\ccFunction{Int &operator+=( Int a, Int b);}{}

\end{ccClass}

%----------------------------------------------------------------------
\ccTagDefaults
\begin{ccClass}{ClassA}
\section{The List of All Operators}

  \ccFunction{Ptr_Class
        operator->(ClassA p);}{}
  \ccGlue
  \ccFunction{ClassA
        operator[](ClassA p, int i);}{}

  \ccFunction{ClassA
        operator()(ClassA p);}{}
  \ccGlue
  \ccFunction{ClassA
        operator()(ClassA p, int i);}{}
  \ccGlue
  \ccFunction{ClassA
        operator()(ClassA p, int i, int j);}{}
  \ccGlue
  \ccFunction{ClassA
        operator()(ClassA p, int i, int j, int k);}{}
  \ccGlue
  \ccFunction{ClassA
        operator()(ClassA p,
                   const A& a, B& b, C c, const D& d, ClassA
                   e);}{all number and types of parameters are possible.}

  \ccFunction{ClassA
        operator++(ClassA p);}{}
  \ccGlue
  \ccFunction{ClassA
        operator++(ClassA p, int);}{The postfix incr.\ operator
        has a hidden {\tt int} parameter that the formatting does not show.}

  \ccFunction{ClassA
        operator--(ClassA p);}{}
  \ccGlue
  \ccFunction{ClassA
        operator--(ClassA p, int);}{}

  \ccFunction{ClassA
        operator~(ClassA p);}{}
  \ccGlue
  \ccFunction{ClassA
        operator!(ClassA p);}{}
  \ccGlue
  \ccFunction{ClassA
        operator-(ClassA p);}{}
  \ccGlue
  \ccFunction{ClassA
        operator+(ClassA p);}{}
  \ccGlue
  \ccFunction{ClassA
        operator&(ClassA p);}{}
  \ccGlue
  \ccFunction{ClassA
        operator*(ClassA p);}{}

  \ccMethod{ void*
        operator new( size_t);}{Hidden parameters are not shown. \CC\ code: 
        $\backslash${\tt method\{ void* operator new( size\_t);\}}.}

  \ccMethod{ void
        operator delete( void*, size_t);}{Hidden parameters are not shown.
        \CC\ code: $\backslash${\tt method\{ void operator 
            delete( void*, size\_t);\}}}

  \ccMethod{ void
        operator delete[]( void*, size_t);}{Hidden parameters are not
        shown again. 
        \CC\ code: $\backslash${\tt method\{ void operator
            delete[]( void*, size\_t);\}}}

  \ccFunction{Member_Ptr
        operator->*(ClassA p);}{}

  \ccFunction{ClassA
        operator*(ClassA p, ClassA q);}{}
  \ccGlue
  \ccFunction{ClassA
        operator/(ClassA p, ClassA q);}{}
  \ccGlue
  \ccFunction{ClassA
        operator%(ClassA p, ClassA q);}{}
  \ccGlue
  \ccFunction{ClassA
        operator+(ClassA p, ClassA q);}{}
  \ccGlue
  \ccFunction{ClassA
        operator-(ClassA p, ClassA q);}{}
  \ccGlue
  \ccFunction{ClassA
        operator<<(ClassA p, int i);}{}
  \ccGlue
  \ccFunction{ClassA
        operator>>(ClassA p, int i);}{}

  \ccFunction{ClassA
        operator<(ClassA p, ClassA q);}{}
  \ccGlue
  \ccFunction{ClassA
        operator<=(ClassA p, ClassA q);}{}
  \ccGlue
  \ccFunction{ClassA
        operator>(ClassA p, ClassA q);}{}
  \ccGlue
  \ccFunction{ClassA
        operator>=(ClassA p, ClassA q);}{}
  \ccGlue
  \ccFunction{ClassA
        operator==(ClassA p, ClassA q);}{}
  \ccGlue
  \ccFunction{ClassA
        operator!=(ClassA p, ClassA q);}{}

  \ccFunction{ClassA
        operator&(ClassA p, ClassA q);}{}
  \ccGlue
  \ccFunction{ClassA
        operator^(ClassA p, ClassA q);}{}
  \ccGlue
  \ccFunction{ClassA
        operator|(ClassA p, ClassA q);}{}

  \ccFunction{ClassA
        operator&&(ClassA p, ClassA q);}{}
  \ccGlue
  \ccFunction{ClassA
        operator||(ClassA p, ClassA q);}{}

  \ccFunction{ClassA
        operator=(ClassA p, ClassA q);}{}
  \ccGlue
  \ccFunction{ClassA
        operator*=(ClassA p, ClassA q);}{}
  \ccGlue
  \ccFunction{ClassA
        operator/=(ClassA p, ClassA q);}{}
  \ccGlue
  \ccFunction{ClassA
        operator%=(ClassA p, ClassA q);}{}
  \ccGlue
  \ccFunction{ClassA
        operator+=(ClassA p, ClassA q);}{}
  \ccGlue
  \ccFunction{ClassA
        operator-=(ClassA p, ClassA q);}{}
  \ccGlue
  \ccFunction{ClassA
        operator<<=(ClassA p, ClassA q);}{}
  \ccGlue
  \ccFunction{ClassA
        operator>>=(ClassA p, ClassA q);}{}
  \ccGlue
  \ccFunction{ClassA
        operator&=(ClassA p, ClassA q);}{}
  \ccGlue
  \ccFunction{ClassA
        operator|=(ClassA p, ClassA q);}{}
  \ccGlue
  \ccFunction{ClassA
        operator^=(ClassA p, ClassA q);}{}

\end{ccClass}

% ----------------------------------------------------------------------
\section{Scope Operator in Types}

CGAL will use {\tt typedef}'s and the scope operator to define types.
Here is an example for the scope operator within types.

  \ccSetTwoOfThreeColumns{5.8cm}{1.8cm}
  \ccFunction{Rep_Class::Nested_Class
        foo(Rep_Class::Nested_Class p, Demo_Class q);}{
         Declaration with scope.}

  \ccFunction{Rep_Class :: Nested_Class
        foo(Rep_Class :: Nested_Class p, Demo_Class q);}{
         The same, surrounded by spaces.}
       
The scope operator now also can be used for the function or method name.

  \ccSetTwoOfThreeColumns{5.8cm}{1.8cm}
  \ccFunction{Rep_Class::Nested_Class
        TI::foo(Rep_Class::Nested_Class p, Demo_Class q);}{
         Declaration with scope.}

  \ccFunction{Rep_Class :: Nested_Class
        CBP::foo(Rep_Class :: Nested_Class p, Demo_Class q);}{
         The same, surrounded by spaces.}
       

% ----------------------------------------------------------------------
\begin{ccClassTemplate}{ClassB<FT<RT> >}
\ccSection{Demo Class Template}

\ccCreation
\ccCreationVariable{p}

\ccTwo{ClassB<FT<RT> > p( A a, B *b);X}{}
  \ccConstructor{ ClassB();}{ default.}
  \ccGlue
  \ccConstructor{ ClassB( ClassB<FT<RT> > q);}{ copy.}
  \ccGlue
  \ccConstructor{ ClassB( A a, B *b);}{ arbitrary.}

\ccOperations
\ccThree{ClassB<FT<RT> >}{foo( q);}{}

  \ccMethod{ ClassB foo( ClassB q);}{
      wrong, without template parameters.}

  \ccFunction{ ClassB<FT<RT> > foo( ClassB<FT<RT> > q);}{
      right, with template parameters.}
\end{ccClassTemplate}

% ----------------------------------------------------------------------
\section{Abbreviations}

The math set symbols need the package
{\tt amssymb} and \LaTeXe.

\begin{tabbing}
  \ccTexHtml{dum \= dummyyyyyyyy \= \kill}{}
  \> \verb+\CC+         \> \CC         \\
  \> \verb+\gcc+        \> \gcc        \\
  \> \verb+\nat+        \> \nat        \\
  \> \verb+\real+       \> \real       \\
  \> \verb+\leda+       \> \leda       \\
  \> \verb+\cgal+       \> \cgal       \\
  \> \verb+\protocgal+  \> \protocgal  \\
  \> \verb+\plageo+     \> \plageo     \\ 
  \> \verb+\+\verb+N+        \> \N        \\
  \> \verb+\+\verb+Z+        \> \Z        \\
  \> \verb+\+\verb+R+        \> \R        \\
  \> \verb+\+\verb+E+        \> \E
\end{tabbing}

% ----------------------------------------------------------------------
\section{Enum's}

Enum's are formatted similiar to constructors. Exactly one pair of
matching braces has to be in the declaration.

\ccTwo{enum Short ( A, B, C);XX}{}

\ccEnum{enum Short { A, B, C};}{ Comment.}

\ccEnum{enum Funny_type_name1 { A_couple_of_entries1,
    one_with_initialisation1 = 5, another1 = -3};}{ Comment.}

We can even switch the alternative layout on:

\def\ccLongParamLayout{\ccTrue}

\ccEnum{enum Funny_type_name2 { A_couple_of_entries2,
    one_with_initialisation2 = 5, another2 = -3};}{ Comment.}

\def\ccLongParamLayout{\ccFalse}


% ----------------------------------------------------------------------
\section{Variables, Constants and Typedefs}

\ccSetTwoOfThreeColumns{3.5cm}{3.5cm}

\ccVariable{long int foo4;}{Local variables are possible.}

\ccVariable{long int foo5 = 15;}{Initialisation.}

\ccVariable{const long int foo6 = 15;}{Make a constant.}

\ccTypedef{typedef int integer;}{Simple typedef.}

\ccTypedef{typedef List< int>  Integer_list;}{Typedef including template
    parameters.}



% ----------------------------------------------------------------------
\section{Global Declarations}

\ccGlobalFunction{Intersection_type CGAL_intersection_type(
      Polygon_2< R>, Polygon_2< R>);}

\ccGlobalEnum{enum global_enum { A_couple_of_entries3,
    one_with_initialisation3 = 5, another3 = -3};}

\ccGlobalVariable{int CGAL_global_var;}


% ----------------------------------------------------------------------
\begin{ccClassTemplate}{CGAL_Point_2< R>}
\section{Default Parameters in Function Argument Lists}

The following example demonstrates the new ability to write default
parameters with initializers in parantheses notion of \CC.

\ccCreationVariable{p}
\ccConstructor{CGAL_Point_2(const R::RT &x, const R::RT &y, const R::RT &w = R::RT(1.0));}{blabla}
\end{ccClassTemplate}



% ----------------------------------------------------------------------
\section{Test Indentation and Alternate Formatting}

This example tests the indentation and right margin setting
possibilities. Two long declarations with alternative formatting rules
for the function arguments are used. First the  declaration without
any indentation or margins.

\ccFunction{int a_really_long_function_name( double paramter1, double
  paramter2);}{the default formatting. A bit more text is necessary to
  demonstrate the right margin.}

Now with 10mm indentation and 10mm right margin. Note that the
description is also further indented since all dimensions are
calculated from left.

\ccTagDefaults
\def\ind{\hspace*{7mm}}
\ccTexHtml{\ccwIndent=10mm \ccwRightMargin=10mm}{}

\ccFunction{int a_really_long_function_name( double paramter1, double
  paramter2);}{the default formatting. A bit more text is necessary to
  demonstrate the right margin.}
\renewcommand{\ccLongParamLayout}{\ccTrue}

\ccFunction{int a_really_long_function_name( double paramter1, double
  paramter2);}{the alternative formatting. A bit more text is necessary to
  demonstrate the right margin.}

Show the indentation also for template functions.

\ccFunction{template<class A> int bar(A a);}{A bit more text to
  demonstrate the right margin.}

\renewcommand{\ccLongParamLayout}{\ccFalse}

Check the same alternate indentation with the compatibility mode 
\verb+\renewcommand{\ccAlternateThreeColumn}{\ccFalse}+

\renewcommand{\ccAlternateThreeColumn}{\ccFalse}

\ccFunction{int a_really_long_function_name( double paramter1, double
  paramter2);}{the alternative formatting. A bit more text is necessary to
  demonstrate the right margin.}

Show the indentation also for template functions.

\ccFunction{template<class A> int bar(A a);}{A bit more text to
  demonstrate the right margin.}

\renewcommand{\ccAlternateThreeColumn}{\ccTrue}
\ccTexHtml{\ccwIndent=0mm \ccwRightMargin=0mm}{}

% ----------------------------------------------------------------------
\section{Test the Setting of the Column Widths by Example Texts}


\ccSetThreeColumns{int}{foo( int i, int j);}{}
\ccFunction{int foo( int i, int j);}{returns gnats$(i,j)$.}

\ccSetThreeColumns{int}{}{returns gnats$(i,j)$.}
\ccFunction{int foo( int i, int j);}{returns gnats$(i,j)$.}

\ccSetThreeColumns{}{foo( int i, int j);}{returns gnats$(i,j)$.}
\ccFunction{int foo( int i, int j);}{returns gnats$(i,j)$.}

Test it with the special characters (changed catcodes) for function
and contructor declarations.  


\ccSetThreeColumns{_#_^}{_#_^f*&*&oo( int i);}{}
\ccSetTwoColumns{_#_^ foo( int i, int j);}{}
\begin{ccClass}{Gnu1}
  \ccCreationVariable{g}
  \ccConstructor{Gnu1( double d);}{test.}
  \ccFunction{int foo( Gnats<T> gn);}{blablabla.}
\end{ccClass}

Specify a column layout for a function and propagate it to a contructor.

\ccSetThreeColumns{intM}{foo( Gnats<T> gn);M}{}
\ccPropagateThreeToTwoColumns
\begin{ccClass}{Gnu2}
  \ccCreationVariable{g}
  \ccConstructor{Gnu2( double d);}{test.}
  \ccFunction{int foo( Gnats<T> gn);}{blablabla.}
\end{ccClass}


% ----------------------------------------------------------------------
\section{Check Long Constructor Calls}

\begin{ccClassTemplate}{CBP_Bidirectional_circulator< C, C*
   (C::*next)(), C* (C::*previous)()>} 
\ccCreationVariable{circ}
\ccSetTwoColumns{}{a const circulator \ccVar\ with singular value.}

NOTHING is interesting here.


\ccConstructor{CBP_Bidirectional_circulator();}{%
 a const circulator \ccVar\ with singular value.}

\ccConstructor{CBP_Bidirectional_circulator( const C* ptr);}{a
  const circulator \ccVar\ initialized to point to the element \ccStyle{*ptr}.} 
\end{ccClassTemplate}

% ----------------------------------------------------------------------
\section{Glueing Declarations Together}

\ccFunction{int foo( double x);}{}
\ccGlue
\ccFunction{int bar( double x);}{}

\ccFunction{int foo( double x);}{Bla.}
\ccGlue
\ccFunction{int bar( double x);}{Blubb blubb.}

\ccFunction{int foo_baaaaaaaarrrr( double x);}{%
     Bla bal blabal blabal blabal blab.}
\ccGlue
\ccFunction{int barfoooooooooooooooooo( double x);}{%
    Blubb blubblubb blubblubb blubblubb blubb.}

\ccFunction{int foo_baaaaaaaarrrr( double x);}{%
     Blkj fdkjbjlh flknlkj kjh kjh lkjh a bal blabal blabal blabal blab.}
\ccGlue
\ccFunction{int bar( double x);}{Blubb blubb.}

\ccFunction{int foo_bar( double x);}{%
     Blkj fdkjbjlh flknlkj kjh kjh lkjh a bal blabal blabal blabal blab.}
\ccGlue
\ccFunction{int foo_bar( double x);}{%
     Blkj fdkjbjlh flknlkj kjh kjh lkjh a bal blabal blabal blabal
     Blkj fdkjbjlh flknlkj kjh kjh lkjh a bkuuhd kjwdhwkjdhh blab.}
\ccGlue
\ccFunction{int bar( double x);}{Blubb blubb.}


% ----------------------------------------------------------------------
\section{Glueing Declarations Together, New Style}

%\parskip0pt

\ccGlueBegin
  \ccFunction{int foo( double x);}{}
  \ccFunction{int bar( double x);}{}
\ccGlueEnd

\ccGlueBegin
  \ccFunction{int foo( double x);}{Bla.}
  \ccFunction{int bar( double x);}{Blubb blubb.}
\ccGlueEnd

\ccGlueBegin
  \ccFunction{int foo_baaaaaaaarrrr( double x);}{%
     Bla bal blabal blabal blabal blab.}
  \ccFunction{int barfoooooooooooooooooo( double x);}{%
    Blubb blubblubb blubblubb blubblubb blubb.}
\ccGlueEnd

\ccGlueBegin
  \ccFunction{int foo_baaaaaaaarrrr( double x);}{%
     Blkj fdkjbjlh flknlkj kjh kjh lkjh a bal blabal blabal blabal blab.}
  \ccFunction{int bar( double x);}{Blubb blubb.}
\ccGlueEnd

\ccGlueBegin
  \ccFunction{int foo_bar( double x);}{%
     Blkj fdkjbjlh flknlkj kjh kjh lkjh a bal blabal blabal blabal blab.}
  \ccFunction{int foo_bar( double x);}{%
     Blkj fdkjbjlh flknlkj kjh kjh lkjh a bal blabal blabal blabal
     Blkj fdkjbjlh flknlkj kjh kjh lkjh a bkuuhd kjwdhwkjdhh blab.}
  \ccFunction{int bar( double x);}{Blubb blubb.}
\ccGlueEnd

A bit text thereafter.

A new paragraph to check that \verb+\parskip+ gets restored properly.

% ----------------------------------------------------------------------
\begin{ccClass}{Base::Derived}
\section{Nested Classes}

First try, use the scope operator in the class name.

\ccCreationVariable{a}

\ccConstructor{Base::Derived( double x);}{a constructor.}
\ccFunction{Base::Derived foo( Base::Derived b);}{a function.}
\end{ccClass}

% ----------------------------------------------------------------------
\section{Prefix Substitution}

The example illustrates both typical usages (with a fantasy function
that does not exist in \cgal).

\ccSetThreeColumns{CGAL_Point<R>}{CGAL_f( CGAL_Vector<R> v)}{}

\ccStyle{#include <CGAL/point.h>}
\\
\ccFunction{CGAL_Point<R> CGAL_f( CGAL_Vector<R> v);}{%
    the original \cgal\ prefix.}
\vspace{-\parskip}

\begin{tabbing}
\ccTexHtml{CCimplementationNNNMMMMMMMMMMMMMMMMMMMM \= ImplementationMMMMM \= \kill}{}
  \verb+\def\ccTargetPrefix{}+ \\
  \verb+\renewcommand{\ccTagReplacePrefix}{\ccTrue}+
\end{tabbing}
\def\ccTargetPrefix{}
\renewcommand{\ccTagReplacePrefix}{\ccTrue}
\vspace{-\parskip}

\ccStyle{#include <CGAL/point.h>}
\\
\ccFunction{CGAL_Point<R> CGAL_f( CGAL_Vector<R> v);}{%
    assuming name spaces. {\tt ;-)}}
\vspace{-\parskip}

\begin{tabbing}
\ccTexHtml{CCimplementationNNNMMMMMMMMMMMMMMMMMMMM \= ImplementationMMMMM \= \kill}{}
  \verb+\def\ccTargetPrefix{GTE}+ \\
  \verb+\renewcommand{\ccTagReplaceInclude}{\ccTrue}+
\end{tabbing}
\def\ccTargetPrefix{GTE}
\renewcommand{\ccTagReplaceInclude}{\ccTrue}
\vspace{-\parskip}

\ccStyle{#include <CGAL/point.h>}
\\
\ccFunction{CGAL_Point<R> CGAL_f( CGAL_Vector<R> v);}{%
    make a Porsche from this declaration.}
\vspace{-\parskip}

\begin{tabbing}
\ccTexHtml{CCimplementationNNNMMMMMMMMMMMMMMMMMMMM \= ImplementationMMMMM \= \kill}{}
  \verb+\renewcommand{\ccTagReplacePrefix}{\ccFalse}+ \\
  \verb+\renewcommand{\ccTagReplaceInclude}{\ccFalse}+
\end{tabbing}
\renewcommand{\ccTagReplacePrefix}{\ccFalse}
\renewcommand{\ccTagReplaceInclude}{\ccFalse}

% ----------------------------------------------------------------------
\section{A Constructor for a Nested Class}

The method how it should work:

\begin{ccClass}{CGAL_SegmentTree_2d<T1>::SegTree}

The classname is \ccClassTemplateName.

\ccCreationVariable{t}
\ccConstructor{CGAL_SegmentTree_2d<T1>::SegTree
(S2dList::const_iterator first);}{constructor.}

\ccMemberFunction{CGAL_SegmentTree_2d<T1>::SegTree
    foo (CGAL_SegmentTree_2d<T1>::SegTree same);}{%
    removal of own type in parameterlist of a member function.} 
\end{ccClass}


The hack I have recommended previously.

\ccFunction{CGAL_SegmentTree_2d<T1>::SegTree
t(S2dList::const_iterator first);}{fake constructor.}

Another check with the trouble making operator in a member function.

\begin{ccClass}{ClassC}
\ccCreationVariable{pp}
  \ccMethod{ClassC operator+(ClassC qq);}{%
      Declaration of an operator as a member function.}
\end{ccClass}


\ccAutoIndexingOff
% ----------------------------------------------------------------------
\section{Font Changes within a Declaration}

In an simple \verb+\ccc{}+ macro: \ccc{var_\tt k_th\ccFont _foo8;} and
boldface: \ccc{var_\bf k_th\ccFont _foo8;}

A global variable declaration.

\ccSetThreeColumns{\tt k_th\ccFont _dim\ }{\tt k_th\ccFont _foo8;}{}
\ccVariable{\tt k_th\ccFont _dim \tt k_th\ccFont _foo8;}{%
    a {\tt k\_th}-dimensional variable.}

A constructor and a member function within a class where the classname
also uses font changing macros.

\def\ccTagRmEigenClassName{\ccFalse}

\begin{ccClass}{\tt k_th\ccFont _dim_Class}
  \ccCreationVariable{v}
  \ccSetThreeColumns{\tt k_th\ccFont _dim_retvalue\ }{v.k_th\ccFont
    _dim_foo( \tt k_th\ccFont _dim_param)\ ;}{}
  \ccSetTwoColumns{\tt k_th\ccFont _dim_Class v( \tt k_th\ccFont
    _dim_param) \ ;}{}
  \ccConstructor{\tt k_th\ccFont _dim_Class();}{The default constructor.}
  \ccConstructor{\tt k_th\ccFont _dim_Class( \tt k_th\ccFont
    _dim_param);}{The custom constructor.}  
  \ccMemberFunction{\tt k_th\ccFont _dim_retvalue \tt k_th\ccFont
    _dim_foo( \tt k_th\ccFont _dim_param);}{a function.}
\end{ccClass}

\def\ccLongParamLayout{\ccTrue}

\begin{ccClassTemplate}{CGAL_Segment_tree_\tt k\ccFont <_T1, _C1, _T2,
    _C2,\dots, _T\tt k\ccFont , _C\tt k\ccFont , _V>}

\ccCreationVariable{s}
\ccThree{void}{Dummy}{}

\ccConstructor{CGAL_Segment_tree_\tt k\ccFont <_T1, _C1, _T2,
    _C2,\dots, _T\tt k\ccFont , _C\tt k\ccFont ,
    _V>();}{Creation of a {\tt k}-dimensional Segment
  Tree, where the i-th coordinate has type \ccStyle{_Ti} and a
  comparator of type \ccStyle{_Ci}. Parameter type
\ccStyle{Value} is set to \ccStyle{_V}.}


\ccMethod{void CGAL_Segment_tree_\tt k \ccFont
    window_query(GIS_interval_data_\tt k \ccFont 
    win, back_insert_iterator<GIS_interval_data_list_\tt k\ccFont>  out);}{%
All elements that
  intersect the associated $d$-dimensional half open interval of 
\ccStyle{win} are placed in the list of  \ccStyle{out}.}

\end{ccClassTemplate}
\def\ccLongParamLayout{\ccFalse}
\def\ccTagRmEigenClassName{\ccTrue}

\ccAutoIndexingOn
% ----------------------------------------------------------------------
\begin{ccClass}{ClassD}
\section{Customization Tags for the Style}

\ccThree{ClassD}{pp.operator+( const ClassD& qq) const;}{}
First, a  declaration  with all default substitution rule active.

\ccCreationVariable{pp}

\ccMemberFunction{ClassD operator+(const ClassD& qq) const;}{%
      member function.}

Second, all rules switched off.
\ccTagFullDeclarations

\ccMemberFunction{ClassD operator+(const ClassD& qq) const;}{%
      member function.}

Third, back to the default.
\ccTagDefaults

\ccMemberFunction{ClassD operator+(const ClassD& qq) const;}{%
      member function.}

\end{ccClass}

% ----------------------------------------------------------------------
\begin{ccClass}{Dummy}
\section{Test the removal or inlining of template declarations}
\ccCreationVariable{q}

Inlining of template declarations. The default is to format the
template declaration in an extra line. Here the default with a struct
and a global function.

\ccSetThreeColumns{template <class T> T*}{X}{}
\ccSetTwoColumns{template <class T> struct circulator_base \{\};}{}
\ccStruct{template <class T> 
          struct circulator_base1 {};}{forward.}

\ccConstructor{template <class T> 
          Dummy();}{constructor.}

\ccGlobalFunction{template <class T> 
  T* value_type( const circulator_base<T>&);}

Here with the inlining on.

\def\ccTagTemplateInline{\ccTrue}
\ccStruct{template <class T> 
          struct circulator_base2 {};}{forward.}

\ccConstructor{template <class T> 
          Dummy();}{constructor.}

\ccGlobalFunction{template <class T> 
  T* value_type( const circulator_base<T>&);}
\def\ccTagTemplateInline{\ccFalse}

Here with the removal on.

\def\ccTagRmTemplate{\ccTrue}
\ccStruct{template <class T> 
          struct circulator_base3 {};}{forward.}

\ccConstructor{template <class T> 
          Dummy();}{constructor.}

\ccGlobalFunction{template <class T> 
  T* value_type( const circulator_base<T>&);}
\def\ccTagRmTemplate{\ccFalse}
\end{ccClass}

% ----------------------------------------------------------------------
\begin{ccClassTemplate}{Demo<T>}
\section{A Function Pointer as an Operator Argument}

\ccCreationVariable{out}
\ccThree{Ascii_ostream&}{out << ostream& (*f)(ostream&);}{}
\ccThreeToTwo

\ccMethod{Ascii_ostream& operator<<(ostream& (*f)(ostream&));}
{manipulators in streams.}

\end{ccClassTemplate}

% ----------------------------------------------------------------------
\section{Long Class Template Names in Nested Types}


\begin{ccClassTemplate}{CGAL_tree_point_interface1<P, W, Key,  
    Left, Right, Comp>}
\ccNestedType{Tree_Point}{complies to the container \ccStyle{Point}.}
\end{ccClassTemplate}

If the class does not fit any longer in one line, it wraps around.

\begin{ccClassTemplate}{CGAL_tree_point_interface2<Point, Window, Key,  
         Point_func, Window_left_func, 
         Window_right_func, Compare>}
\ccNestedType{Tree_Point}{complies to the container \ccStyle{Point}.}
\end{ccClassTemplate}

The formatting for long parameter lists put the new type name in an extra line.

\begin{ccClassTemplate}{CGAL_tree_point_interface3<Point, Window, Key,  
         Point_func, Window_left_func, 
         Window_right_func, Compare>}
\def\ccLongParamLayout{\ccTrue}
\ccNestedType{Tree_Point}{complies to the container \ccStyle{Point}.}
\def\ccLongParamLayout{\ccFalse}
\end{ccClassTemplate}

The proposed solution uses a typedef to avoid duplicating the long
template class name all over the specification.

\begin{ccClass}{Point_interface}
  
The following line must be done by hand if it should be
formatted more nicely.

\ccStyle{typedef CGAL_tree_point_interface<Point, Window, Key,  
         Point_func, Window_left_func, 
         Window_right_func, Compare> Point_interface;}
       
Since here the specification is written in terms of the above
\ccStyle{typedef}.

\ccNestedType{Tree_Point}{complies to the container \ccStyle{Point}.}
\end{ccClass}

% ----------------------------------------------------------------------
\section{HTML Conversion}

Small spaces in math mode: $a\:b\;c\!d$.

The \verb+\ccc+ macro within a math environment: $1 + 3 - \ccc{symbol} = 8$.

A simple class I with a short single character name. No cross linking rules

\begin{ccClassTemplate}{I<X>}
\ccCreationVariable{i}
\ccConstructor{I(I<X> j);}{}
\end{ccClassTemplate}

Superscript $x^2$.
Simple fraction ${1 \over 2}$.
Not so simple fraction $\frac{1+2}{2+1}$.
Emptyset $\emptyset$.
Squareroot $\sqrt{1+3}$.
Root of three $\sqrt[3]{1+3}$.

\ccThree{void}{math_foo();}{}
\ccFunction{void math_foo();}{Simple fraction ${1 \over 2}$.}

% ----------------------------------------------------------------------
\section{Switch Off Linking and Indexing}

\ccThree{boolx}{a.link5( LinkB a, B b)}{}
\ccThreeToTwo

\ccHtmlNoClassIndex\ccHtmlNoLinks
\begin{ccClass}{LinkA}
\ccCreationVariable{a}

A class \ccClassName\ with \verb+\ccHtmlNoClassIndex+ and
\verb+\ccHtmlNoLinks+ to suppress cross linking:

  \ccFunction{
     bool link1(const LinkA& a, const B& b);}{%
             a function is still indexed.
         }
  \ccGlue
  \ccMethod{
     bool link2(const LinkA& a, const B& b);}{%
             a method is never indexed.
         }
  \ccGlue
  \ccStruct{struct LocalX {};}{a nested struct is not indexed.}
\end{ccClass}


\begin{ccClass}{LinkB}
\ccCreationVariable{b}

A class \ccClassName\ without \verb+\ccHtmlNoClassIndex+ but the individual
declarations are prefixed with \verb+\ccHtmlNoIndex+ and 
\verb+\ccHtmlNoLinks+:

  \ccHtmlNoIndex\ccHtmlNoLinks
  \ccFunction{
     bool link3(const LinkB& a, const B& b);}{%
             explicitly not indexed.
         }
  \ccGlue
  \ccMethod{
     bool link4(const LinkB& a, const B& b);}{%
             a method is never indexed.
         }
  \ccGlue
  \ccHtmlNoIndex
  \ccStruct{struct LocalY {};}{a nested struct explicitly not indexed.}
\end{ccClass}

The same for a global function and struct.

  \ccHtmlNoIndex\ccHtmlNoLinks
  \ccFunction{
     bool link5(const LinkB& a, const B& b);}{%
             explicitly not indexed.
         }
  \ccGlue
  \ccHtmlNoIndex\ccHtmlNoLinks
  \ccStruct{struct GlobalX {};}{a global struct explicitly not indexed and 
                                not linked.}
  \ccGlue
  \ccStruct{struct GlobalY {};}{a global struct indexed and linked.}


% ----------------------------------------------------------------------
\section{Member Function versus Function for Operators}

\begin{ccClass}{CGAL_Geomview_stream}
\ccCreationVariable{G}

\ccFunction{template <class R>
            CGAL_Geomview_stream&
            operator<<(CGAL_Geomview_stream& G,
                       const CGAL_Point_2<R>& p);}
{Inserts the point \ccc{p} into the stream \ccVar. (global function)}

\ccMethod{template <class R>
          CGAL_Geomview_stream&
          operator<<(const CGAL_Point_2<R>& p);}
{Inserts the point \ccc{p} into the stream \ccVar. (member function)}
\end{ccClass} 

% ----------------------------------------------------------------------
\section{Operators Formatted as Functions}
\begin{ccClass}{FctClass}
\ccCreationVariable{f}

Use default settings.

\ccGlueBegin
\ccMethod{void method() const;}{Normal Const Method.}
\ccMethod{void operator+(FctClass b) const;}{Normal Const Operator.}
\ccMethod{void operator()();}{Empty Fct. Operator.}
\ccMethod{void operator()(int i);}{Fct. Operator.}
\ccMethod{void operator()() const;}{Empty Fct. Operator.}
\ccMethod{void operator()(int i, double d) const;}{Fct. Operator.}
\ccMethod{operator int();}{Conversion Operator.}
\ccGlueEnd

Format full declarations.

\ccTagFullDeclarations

\ccGlueBegin
\ccMethod{void method() const;}{Normal Const Method.}
\ccMethod{void operator+(FctClass b) const;}{Normal Const Operator.}
\ccMethod{void operator()();}{Empty Fct. Operator.}
\ccMethod{void operator()(int i);}{Fct. Operator.}
\ccMethod{void operator()() const;}{Empty Fct. Operator.}
\ccMethod{void operator()(int i, double d) const;}{Fct. Operator.}
\ccMethod{operator int();}{Conversion Operator.}
\ccGlueEnd

\ccTagDefaults
\end{ccClass}

% ----------------------------------------------------------------------
\section{Verbatim Environment}

\begin{verbatim}
int     x;  // Blabal lkjhdfls hf.s //
int     i;  // This is an int.      //
double  d;           //     Double. //
\end{verbatim}

% ----------------------------------------------------------------------
\begin{ccClass}{Face}
\ccSection{Removal of Own Class Name}
\ccCreationVariable{f}

\ccConstructor{Face(Face face);}{}
\ccGlue
\ccConstructor{Face(Face_handle n);}{}
\ccGlue
\ccConstructor{Face(Facehandle n);}{}
\ccGlue
\ccConstructor{Face(handle_Face n);}{}
\ccGlue
\ccConstructor{Face(Face5_handle n);}{}
\ccGlue
\ccConstructor{Face(handle_5Face n);}{}

\end{ccClass}

% ----------------------------------------------------------------------
\section{Table Layout in HTML}

First, right column:

\ccFunction{CGAL_Return<X,Y> foos_bar( int x);}{Blkj fdkjbjlh 
    flknlkj kjh kjh lkjh all .}
\ccFunction{CGAL_Return<X,Y> foos_bar( int x);}{Blkj fdkjbjlh 
    flknlkj kjh kjh lkjh all bak.}
\ccFunction{CGAL_Return<X,Y> foos_bar( int x);}{Blkj fdkjbjlh 
    flknlkj kjh kjh lkjh all bak bal.}
\ccFunction{CGAL_Return<X,Y> foos_bar( int x);}{Blkj fdkjbjlh 
    flknlkj kjh kjh lkjh all bak bal bur.}
\ccFunction{CGAL_Return<X,Y> foos_bar( int x);}{Blkj fdkjbjlh 
    flknlkj kjh kjh lkjh all bak bal bur nic.}

Second, the return type.

\ccFunction{CGAL_Return<X,Y> foos_bar( int x);}{}
\ccFunction{CGAL_Retu_rn<X,Y> foos_bar( int x);}{}
\ccFunction{CGAL_Retu_1rn<X,Y> foos_bar( int x);}{}
\ccFunction{CGAL_Retu_12rn<X,Y> foos_bar( int x);}{}

Third, the function declaration.

\ccFunction{CGAL_Return<X,Y> foos_bar( int x);}{Blkj fdkjbjlh 
    flknlkj kjh kjh lkjh all .}
\ccFunction{CGAL_Return<X,Y> foos_bar( int1 x);}{Blkj fdkjbjlh 
    flknlkj kjh kjh lkjh all .}
\ccFunction{CGAL_Return<X,Y> foos_bar( int_1 x);}{Blkj fdkjbjlh 
    flknlkj kjh kjh lkjh all .}
\ccFunction{CGAL_Return<X,Y> foos_bar( int_18 x);}{Blkj fdkjbjlh 
    flknlkj kjh kjh lkjh all .}
\ccFunction{int foos_bar( double x);}{}

Same with long return type.

\ccFunction{CGAL_ReturnMMMMM<X,Y> foos_bar( int x);}{Blkj fdkjbjlh 
    flknlkj kjh kjh lkjh all .}
\ccFunction{CGAL_ReturnMMMMM<X,Y> foos_bar( int1 x);}{Blkj fdkjbjlh 
    flknlkj kjh kjh lkjh all .}
\ccFunction{CGAL_ReturnMMMMM<X,Y> foos_bar( int_1 x);}{Blkj fdkjbjlh 
    flknlkj kjh kjh lkjh all .}
\ccFunction{CGAL_ReturnMMMMM<X,Y> foos_bar( int_18 x);}{Blkj fdkjbjlh 
    flknlkj kjh kjh lkjh all .}
\ccFunction{CGAL_Retu_12rn<X,Y> foos_bar( double x);}{}

Long comments with long function delaration.

\ccFunction{CGAL_ReturnMMMMM<X,Y> foos_bar( int x);}{Blkj fdkjbjlh 
    flknlkj kjh kjh lkjh all .}
\ccFunction{CGAL_Return<X,Y> foos_bar( double<> x);}{Blkj fdkjbjlh 
    flknlkj kjh kjh lkjh all .}
\ccFunction{CGAL_Return<X,Y> foos_bar( double<> x);}{Blkj fdkjbjlh 
    flknlkj kjh kjh lkjh all bak.}
\ccFunction{CGAL_Return<X,Y> foos_bar( double<> x);}{Blkj fdkjbjlh 
    flknlkj kjh kjh lkjh all bak bal.}
\ccFunction{CGAL_Return<X,Y> foos_bar( double<> x);}{Blkj fdkjbjlh 
    flknlkj kjh kjh lkjh all bak bal bur.}
\ccFunction{CGAL_Return<X,Y> foos_bar( double<> x);}{Blkj fdkjbjlh 
    flknlkj kjh kjh lkjh all bak bal bur nic.}


Long parameter lists:

\ccFunction{CGAL_Return<X,Y> foos_bar( int x);}{Blkj fdkjbjlh 
    flknlkj kjh kjh lkjh all .}
\ccFunction{CGAL_Return<X,Y> foos_bar( int x, double y);}{Blkj fdkjbjlh 
    flknlkj kjh kjh lkjh all .}
\ccFunction{CGAL_Return<X,Y> foos_bar( int x, double y, double y);}{
    Blkj fdkjbjlh flknlkj kjh kjh lkjh all .}
\ccFunction{CGAL_Return<X,Y> foos_bar( int x, double y, double y, double y);}{
    Blkj fdkjbjlh flknlkj kjh kjh lkjh all .}
\ccFunction{CGAL_Return<X,Y> foos_bar( int x, double y, double y, double y, 
    double y);}{Blkj fdkjbjlh flknlkj kjh kjh lkjh all .}
\ccFunction{CGAL_Return<X,Y> foos_bar( int x, double y, double y, double y,
    double y, double y);}{Blkj fdkjbjlh flknlkj kjh kjh lkjh all .}


Long parameter lists with long return type:

\ccFunction{CGAL_Ret_12urn<X,Y> foos_bar( int x);}{Blkj fdkjbjlh 
    flknlkj kjh kjh lkjh all .}
\ccFunction{CGAL_Ret_12urn<X,Y> foos_bar( int x, double y);}{Blkj fdkjbjlh 
    flknlkj kjh kjh lkjh all .}
\ccFunction{CGAL_Ret_12urn<X,Y> foos_bar( int x, double y, double y);}{
    Blkj fdkjbjlh flknlkj kjh kjh lkjh all .}
\ccFunction{CGAL_Ret_12urn<X,Y> foos_bar( int x, double y, double y, 
    double y);}{Blkj fdkjbjlh flknlkj kjh kjh lkjh all .}
\ccFunction{CGAL_Ret_12urn<X,Y> foos_bar( int x, double y, double y, double y, 
    double y);}{Blkj fdkjbjlh flknlkj kjh kjh lkjh all .}
\ccFunction{CGAL_Ret_12urn<X,Y> foos_bar( int x, double y, double y, double y,
    double y, double y);}{Blkj fdkjbjlh flknlkj kjh kjh lkjh all .}


Same, but with alternate parameter layout for long parameter
names. Long parameter lists:

\def\ccLongParamLayout{\ccTrue}

\ccFunction{CGAL_Return<X,Y> foos_bar( int x);}{Blkj fdkjbjlh 
    flknlkj kjh kjh lkjh all .}
\ccFunction{CGAL_Return<X,Y> foos_bar( int x, double y);}{Blkj fdkjbjlh 
    flknlkj kjh kjh lkjh all .}
\ccFunction{CGAL_Return<X,Y> foos_bar( int x, double y, double y);}{
    Blkj fdkjbjlh flknlkj kjh kjh lkjh all .}
\ccFunction{CGAL_Return<X,Y> foos_bar( int x, double y, double y, double y);}{
    Blkj fdkjbjlh flknlkj kjh kjh lkjh all .}
\ccFunction{CGAL_Return<X,Y> foos_bar( int x, double y, double y, double y, 
    double y);}{Blkj fdkjbjlh flknlkj kjh kjh lkjh all .}
\ccFunction{CGAL_Return<X,Y> foos_bar( int x, double y, double y, double y,
    double y, double y);}{Blkj fdkjbjlh flknlkj kjh kjh lkjh all .}


Long parameter lists with long return type:

\ccFunction{CGAL_Ret_12urn<X,Y> foos_bar( int x);}{Blkj fdkjbjlh 
    flknlkj kjh kjh lkjh all .}
\ccFunction{CGAL_Ret_12urn<X,Y> foos_bar( int x, double y);}{Blkj fdkjbjlh 
    flknlkj kjh kjh lkjh all .}
\ccFunction{CGAL_Ret_12urn<X,Y> foos_bar( int x, double y, double y);}{
    Blkj fdkjbjlh flknlkj kjh kjh lkjh all .}
\ccFunction{CGAL_Ret_12urn<X,Y> foos_bar( int x, double y, double y, 
    double y);}{Blkj fdkjbjlh flknlkj kjh kjh lkjh all .}
\ccFunction{CGAL_Ret_12urn<X,Y> foos_bar( int x, double y, double y, double y, 
    double y);}{Blkj fdkjbjlh flknlkj kjh kjh lkjh all .}
\ccFunction{CGAL_Ret_12urn<X,Y> foos_bar( int x, double y, double y, double y,
    double y, double y);}{Blkj fdkjbjlh flknlkj kjh kjh lkjh all .}

\def\ccLongParamLayout{\ccFalse}

Now, checking a class. Note, comment is always in separate line.

\ccHtmlNoClassFile
\begin{ccClass}{IndentClass}
\ccCreationVariable{indent}

\ccConstructor{IndentClass();}{Short comment. Default constr.}

Now one with and one without comment.

\ccConstructor{IndentClass();}{Short comment. Default constr.}
\ccConstructor{IndentClass();}{}

Here with longer comment.

\ccConstructor{IndentClass();}{Short Short Short Short Short Short
     Short comment. Default constr.}
\ccConstructor{IndentClass();}{Short Short Short Short Short Short
     Short Short Short comment. Default constr.}

Here with long parameter lists.

\ccConstructor{IndentClass( double x, double y, double y, double y, 
    double y);}{Short comment. Default constr.}
\ccConstructor{IndentClass( double x, double y, double y, double y, 
    double y);}{Short comment. Default constr.}
\ccConstructor{IndentClass( double x, double y, double y, double y, 
    double y, double y);}{Short comment. Default constr.}

Same, but with alternate parameter layout for long parameter
names. Here with long parameter lists.

\def\ccLongParamLayout{\ccTrue}

\ccConstructor{IndentClass( double x, double y, double y, double y, 
    double y);}{Short comment. Default constr.}
\ccConstructor{IndentClass( double x, double y, double y, double y, 
    double y);}{Short comment. Default constr.}
\ccConstructor{IndentClass( double x, double y, double y, double y, 
    double y, double y);}{Short comment. Default constr.}

\def\ccLongParamLayout{\ccFalse}

Long template parameter list and short function.

\ccFunction{template < class Point, class Vector, class Traits, 
    class Magnificant> void floo();}{thats it.}

\end{ccClass}

% ----------------------------------------------------------------------
\section{Bibliography and citations}

Some \cgal\ citations: \cite{fgkss-ckbgc-96,kw-ceapp-97}, maybe
with a optional comment: \cite[optional comment]{fgkss-ckbgc-96,kw-ceapp-97}.

% ----------------------------------------------------------------------
\section{Test of New HTML Converter with Macro Expansion}

Accents and Umlaute: \"a\ \"o\ \"u\ \"A\ \"O\ \"{U}\ \'e\ \`e\ \^e\ 
  \~n\ \c{c}\ \ss{}\ \ae\ \AE\ \aa\ \AA\ \oe\ \OE\ \o\ \O\ \S\ 
  \P\ \copyright\ \pounds\ \dag\ \ddag\ \l\ \L.


Macros \TeX{} \LaTeX\ \LaTeXe\ to name the game.

Greek alphabet: $\alpha\ \beta\ \gamma\ \delta\ \epsilon\ 
    \varepsilon\ \zeta\ \eta\ \theta\ \vartheta\ \iota\ \kappa\ \lambda\
    \mu\ \nu\ \xi o\ \pi\ \varpi\ \rho\ \varrho\ \sigma\ \varsigma\ 
    \tau\ \upsilon\ \phi\ \varphi\ \chi\ \psi\ \omega$,  
    $\Gamma\ \Delta\ \Theta\ \Lambda\ \Xi\ \Pi\ \Sigma\ \Upsilon\ \Phi\ 
    \Psi\ \Omega$

Miscelleaneous math: $x \cdot y, x \times y$, $\sqrt{2}$, $\sqrt[4]{18}$,
    $\frac{1}{2}$.

Some dots: $M \ldots M \cdot M \cdots M \vdots M \ddots M$.

Calligraphic letters: $\mathcal{A}$ $\mathcal{B}$ $\mathcal{C}$ $\mathcal{D}$
 $\mathcal{E}$ $\mathcal{F}$ $\mathcal{G}$ $\mathcal{H}$ $\mathcal{I}$
 $\mathcal{J}$ $\mathcal{K}$ $\mathcal{L}$ $\mathcal{M}$ $\mathcal{N}$
 $\mathcal{O}$ $\mathcal{P}$ $\mathcal{Q}$ $\mathcal{R}$ $\mathcal{S}$
 $\mathcal{T}$ $\mathcal{U}$ $\mathcal{V}$ $\mathcal{W}$ $\mathcal{X}$
 $\mathcal{Y}$ $\mathcal{Z}$.

Mathbb letters: $\mathbb{A}$ $\mathbb{B}$ $\mathbb{C}$ $\mathbb{D}$
 $\mathbb{E}$ $\mathbb{F}$ $\mathbb{G}$ $\mathbb{H}$ $\mathbb{I}$
 $\mathbb{J}$ $\mathbb{K}$ $\mathbb{L}$ $\mathbb{M}$ $\mathbb{N}$
 $\mathbb{O}$ $\mathbb{P}$ $\mathbb{Q}$ $\mathbb{R}$ $\mathbb{S}$
 $\mathbb{T}$ $\mathbb{U}$ $\mathbb{V}$ $\mathbb{W}$ $\mathbb{X}$
 $\mathbb{Y}$ $\mathbb{Z}$.

Some not-relations: $ \not\in $.

Binary operation symbols: $ \pm \times \div \mp \ast \star \circ \bullet
 \cap \cup \uplus \sqcap \sqcup \vee \wedge \setminus \wr \diamond
 \bigtriangleup \bigtriangledown \triangleleft \triangleright \lhd
 \rhd \unlhd \unrhd \oplus \ominus \otimes \oslash \odot \bigcirc \dagger
 \ddagger \amalg$.

Relation symbols: $ \leq \prec \preceq \ll \subset \subseteq \sqsubset
 \sqsubseteq \in \vdash \geq \succ \succeq \gg \supset \supseteq \sqsupset
 \sqsupseteq \ni \dashv \equiv \sim \simeq \asymp \approx \cong \neq
 \doteq \notin \models \perp \mid \parallel \bowtie \Join \smile \frown
 \propto$.

Arrow symbols: $ \leftarrow \Leftarrow \rightarrow \Rightarrow \leftrightarrow
 \Leftrightarrow \mapsto \hookleftarrow \leftharpoonup \leftharpoondown
 \rightleftharpoons \longleftarrow \Longleftarrow \longrightarrow
 \Longrightarrow \longleftrightarrow \Longleftrightarrow \longmapsto
 \hookrightarrow \rightharpoonup \rightharpoondown \leadsto \uparrow \Uparrow
 \downarrow \Downarrow \updownarrow \Updownarrow \nearrow \searrow \swarrow
 \nwarrow$.

Miscellaneous symbols: $\prime \neg \| \backslash \aleph \hbar \imath \jmath
 \ell \wp \Re \Im \mho \emptyset \nabla \surd \top \bot \angle \forall \exists
 \flat \natural \sharp \partial \infty \Box \Diamond \triangle \clubsuit
 \diamondsuit \heartsuit \spadesuit$.

Variable sized symbols (small): $ \sum \prod \coprod \int \oint \bigcap
 \bigcup \bigsqcup \bigvee \bigwedge \bigodot \bigotimes \bigoplus \biguplus$.

Variable sized symbols (big): \[ \sum \prod \coprod \int \oint \bigcap
 \bigcup \bigsqcup \bigvee \bigwedge \bigodot \bigotimes \bigoplus \biguplus\].

Delimiters: $ \lfloor x \rfloor, \lceil y \rceil, \langle z \rangle$.

Math accents: $\hat{e}, \acute{e}, \grave{e}, \tilde{n}, \ddot{a}$.


\begin{alltt}
Ha
%
\LaTeX{}
%
Ha
\end{alltt}

\ccHtmlCrossLink{FunnyClass}This line should be the target for the 
hyperlinks emanating from the class \ccc{FunnyClass}.

Here another paragraph mentioning the class \ccc{FunnyClass}. Place
it in the index, first by default under class, and then additionally under
nested types.\ccHtmlIndexC{FunnyClass}\ccHtmlIndexC[nested_type]{FunnyClass}
Put some formated entries in the index. \ccHtmlIndex{PlainClass}
\ccHtmlIndex[function]{This is {\em Stuff} related to \ldots.}


My \ccAnchor{http://www.inf.ethz.ch/personal/kettner/}{homepage} 
is at \path|http://www.inf.ethz.ch/personal/kettner/|.

\begin{ccHtmlClassFile}{qqq.html}{Declaration of \ccc{Qaa}}
\begin{ccClass}{Q}
\ccCreationVariable{y}
\ccConstructor{Q();}{default}
\end{ccClass}
\end{ccHtmlClassFile}

\newcommand{\xa}{hello}
\newcommand{\xb}[1]{\xa }
\newcommand{\xc}[1]{\xa}
\newcommand{\xd}{hello }
\newcommand{\xe}[1]{%
#1
}
\newcommand{\xf}[1]{ #1 }
\newcommand{\xg}[1]{#1}
\newcommand{\xh}[1]{\xg}

Test for spaces in macro expansion: all lines should look the same.

Say hello to me, hellome.\\
Say \xb{A} to me, \xb{A}me.\\
Say \xc{A} to me, \xb{A}me.\\
Say \xd    to me, \xb{A}me.\\
Say\xe{ hello}to me, hellome.\\
Say\xe{}hello to me, hellome.\\
Say\xf{hello}to me, hellome.\\
Say hello \xh{} t\xh{}o me, hellome.\\
Say hello\xh{}\ t\xh{}o me, hellome.

Another example, looking slightly different.

Say \ hello to me.\\
Say\xf{}hello to me.


%% ==========================================
\begin{lcHtmlBlock}
{\bf Macro expansion tests}

Should all look like

No no \textbf{Bla} bla no no.\\
%
\newcommand{\macro}{\textbf{Bla} bla}
No no \macro\ no no.\\
%
\def\macroa{\textbf{Bla} bla}
No no \macroa\ no no.\\
%
\newcommand{\macrob} %%

  {\textbf{Bla}  bla}
%
No no \macrob\ no no.\\
%
\newcommand{\macroc}{\textbf{Bla} % hahahha
  bla}
%
No no \macroc\ no no.\\
%
\newcommand{\a}[1]{\textbf{#1} bla}
%
No no \a{Bla} no no.\\
%
\def\aa #1{\textbf{#1} bla}
%
No no \aa{Bla} no no.\\
%
\def\aaa #1#2#3{\textbf{#3#2#1} bla}
%
No no \aaa alB no no.\\
%
\newcommand{\b}[1]{ #1}
\newcommand{\b@om}{\textbf{#1} #2}
%
No no \b[Bla]{bla}\b{no}\ no.\\
%
\newcommand{\bb}[1]{ #1}
\newcommand{\bb@mo}{\textbf{#1} #2}
%
No no \bb{Bla}[bla]\bb{no}\ no.\\
%
\newcommand{\bb@omo}{\textbf{#1} #2#3}
No no \bb[Bla]{bla}[ no]\ no.


%% ==========================================
Should all look like

No no \textbf{Bla} \% \{ \{ \} bla no no.\\
%
\newcommand{\macrod}{\textbf{Bla} \% \{ \{ \} bla}
No no \macrod\ no no.\\
%
\newcommand{\|}{\textbf{Bla} \% \{ \{ \} bla}
%
No no \| no no.

%% ==========================================
Should all look like

Hallo Dummy World.\\
%
\newenvironment{c}{Hallo}{World}
\begin{c} Dummy \end{c}.\\
%
\newenvironment{d}[1]{#1}{World}
\begin{d}{Hallo} Dummy \end{d}.\\
%
\newenvironment{e}[1]{#1}{World}
\newcommand{\begin@e@om}{#1#2}
\begin{e}{Hallo} Dummy \end{e}.\\
\begin{e}[Hal]{lo} Dummy \end{e}.\\
%
\newcommand{\begin@e@moo}{#1#2#3}
\begin{e}{Ha}[l][lo] Dummy \end{e}.\\
%
\newenvironment{ee}{Hallo }{World}
\newcommand{\begin@ee@o}{Ha#1o}
\begin{ee} Dummy \end{ee}.\\
\begin{ee}[ll] Dummy \end{ee}.\\
%
Hallo~Dummy World.

\newtheorem{mytheorem}{Theorem}

\begin{mytheorem}
    Hallo theorem.
\end{mytheorem}

\begin{mytheorem}[Optional]
    Hallo {\em emphasized\/} theorem.
\end{mytheorem}

\paragraph{Paragraph}

blablabla blablabla blablabla blablabla blablabla blablabla blablabla 
blablabla blablabla blablabla blablabla 

\paragraph*{Paragraph}

blablabla blablabla blablabla blablabla blablabla blablabla blablabla 
blablabla blablabla blablabla blablabla 

\paragraph[Bloek]{Paragraph}

blablabla blablabla blablabla blablabla blablabla blablabla blablabla 
blablabla blablabla blablabla blablabla 

\def\qtrue{\ccTrue}
\def\qfalse{\ccFalse}
\def\qauto{\ccAuto}
\newcommand{\qy}[1]{\lciIf{#1}{\lcAsciiToHtml{#1} is true.}
                              {\lcAsciiToHtml{#1} is false.}}

\newcommand{\qz}[1]{\lciIf{#1}[\lcAsciiToHtml{#1} is auto.]
                              {\lcAsciiToHtml{#1} is true.}
                              {\lcAsciiToHtml{#1} is false.}}

\qy\qtrue \\
\qy\qfalse

\qz\qtrue \\
\qz\qfalse\\
\qz\qauto

ccTrueAndFalse \lciIfDefined{\ccTrueAndFalse}{ is defined.}{ is not defined.}\\
ccTrue \lciIfDefined{\ccTrue}{ is defined.}{ is not defined.}

\lciIfEqual{Hallo}{Hallo}{correct}{wrong}\\
\lciIfEqual{<}{<}{correct}{wrong}\\
\lciIfEqual{Hallo}{<}{wrong}{correct}

\ccFunction{int foo( const Bla& ref) const;}{}

\ccGlobalFunction{int foo( const Bla& ref) const;}

\ccClassDeclaration{class Demo;}{huhu}

\ccInclude{CGAL/Vector_2.h}

Stack implemented using macros:

\newcommand{\mystack}{}

Empty stack: stack = \lciPrintStack{\mystack}\\
Test for emptyness: \lciIfEmpty{\mystack}{correct}{failed}.\\
\lciPushStack{\mystack}{first item}
Push an item: stack = \lciPrintStack{\mystack}\\
Test for non-emptyness: \lciIfEmpty{\mystack}{failed}{correct}.\\
Show top: \lciTopStack{\mystack}{\mytop}\mytop\\
\lciPushStack{\mystack}{second item}
Push an item: stack = \lciPrintStack{\mystack}\\
Test for non-emptyness: \lciIfEmpty{\mystack}{failed}{correct}.\\
Show top: \lciTopStack{\mystack}{\mytop}\mytop\\
\lciPushStack{\mystack}{third item}
Push an item: stack = \lciPrintStack{\mystack}\\
\lciPopStack{\mystack}
Pop an item: stack = \lciPrintStack{\mystack}\\
\lciPopStack{\mystack}
Pop an item: stack = \lciPrintStack{\mystack}\\
\lciPopStack{\mystack}
Pop an item: stack = \lciPrintStack{\mystack}\\
Test for emptyness: \lciIfEmpty{\mystack}{correct}{failed}.\\

\end{lcHtmlBlock}

\def\nn #1#2{A#1B#2C}

\begin{alltt}
\nn {D} {E} F
\nn {%} {
E} F

Ha
%
\LaTeX Hallo
%
Ha
\end{alltt}

{\it Hallo {\em Du} da}

{\tt Hallo {\em Du} da}

{\sc Hallo {\em Du} da}

$\mathcal{HALLO\ DU\ AUCH}$

\begin{abstract}
This is an abstract. Blabla Blabla Blabla Blabla
  Blabla Blabla Blabla Blabla Blabla Blabla Blabla Blabla Blabla
  Blabla Blabla.
\end{abstract}

\begin{quote}
  Two paragraphs in a quoted environment. Blabla Blabla Blabla Blabla
  Blabla Blabla Blabla Blabla Blabla Blabla Blabla Blabla Blabla
  Blabla Blabla.
  
  Blabla Blabla Blabla Blabla Blabla Blabla Blabla Blabla Blabla
  Blabla Blabla Blabla Blabla Blabla Blabla.  Blabla Blabla Blabla
  Blabla Blabla Blabla.
\end{quote}

\centerline{1st centerline.}
\centerline{2nd centerline.}

\begin{center}
  Two paragraphs in a quoted environment. Blabla Blabla Blabla Blabla
  Blabla Blabla Blabla Blabla Blabla Blabla Blabla Blabla Blabla
  Blabla Blabla.
  
  Blabla Blabla Blabla Blabla Blabla Blabla Blabla Blabla Blabla
  Blabla Blabla Blabla Blabla Blabla Blabla.  Blabla Blabla Blabla
  Blabla Blabla Blabla.
\end{center}

\paragraph{Paragraph}

  Blabla Blabla Blabla Blabla Blabla Blabla Blabla Blabla Blabla
  Blabla Blabla Blabla Blabla Blabla Blabla.  Blabla Blabla Blabla
  Blabla Blabla Blabla.

\subparagraph{Subparagraph}

  Blabla Blabla Blabla Blabla Blabla Blabla Blabla Blabla Blabla
  Blabla Blabla Blabla Blabla Blabla Blabla.  Blabla Blabla Blabla
  Blabla Blabla Blabla.

\begin{tabular}{llcr}
A & B & C & D
\end{tabular}

\begin{tabular}{llcr}\hline
\multicolumn{2}{l}{HalloLang} & C & D     \\
A & \multicolumn{2}{l}{HalloLang} & D     \\
A & B & \multicolumn{2}{l}{HalloLang}     \\
\multicolumn{4}{l}{HalloLang}             \\
%
\multicolumn{2}{l}{HalloLang} & C & D     \\\hline
A & \multicolumn{2}{l}{HalloLang} & D     \\\hline
A & B & \multicolumn{2}{l}{HalloLang}     \\\hline
\multicolumn{4}{l}{HalloLang}             \\\hline
%
\multicolumn{2}{l}{HalloLang} & C & D     \\\hline\hline
A & \multicolumn{2}{l}{HalloLang} & D     \\\hline\hline
A & B & \multicolumn{2}{l}{HalloLang}     \\\hline\hline
\multicolumn{4}{l}{HalloLang}             \\\hline\hline
%
A & B & C & D                             \\\hline\hline
\end{tabular}

\newcommand{\lcTabularBorder}{2}

\begin{tabular}{llcr}\hline
\multicolumn{2}{l}{HalloLang} & C & D     \\
A & \multicolumn{2}{l}{HalloLang} & D     \\
A & B & \multicolumn{2}{l}{HalloLang}     \\
\multicolumn{4}{l}{HalloLang}             \\
%
\multicolumn{2}{l}{HalloLang} & C & D     \\\hline
A & \multicolumn{2}{l}{HalloLang} & D     \\\hline
A & B & \multicolumn{2}{l}{HalloLang}     \\\hline
\multicolumn{4}{l}{HalloLang}             \\\hline
%
\multicolumn{2}{l}{HalloLang} & C & D     \\\hline\hline
A & \multicolumn{2}{l}{HalloLang} & D     \\\hline\hline
A & B & \multicolumn{2}{l}{HalloLang}     \\\hline\hline
\multicolumn{4}{l}{HalloLang}             \\\hline\hline
%
A & B & C & D                             \\\hline\hline
\end{tabular}

\begin{verbatim}
<Angle Brackets>
\end{verbatim}

\begin{alltt}
<Angle Brackets>
\end{alltt}

\label{section_here}

Please refer to Section~\ref{section_here}.

Hallo {\bf Boldface} abcdef.

Hallo {\it Italics} abcdef.

Hallo {\tt Typewriter} abcdef.

{\it Hallo {\em Du} da}

{\tt Hallo {\em Du} da}

{\sc Hallo {\em Du} da}

Hallo {\large large} abcdef.

Hallo {\huge huge} abcdef.

Hallo {\small small} abcdef.

\def\llca{Hallo}

\def\llcaa{NEIN}

\def\llcb{
\llca%
llca
}
\llcb

\def\llcc #1{\llca #1\llca }
\begin{alltt}
\llcc{\llca }B
\end{alltt}

% ----------------------------------------------------------------------
\begin{lcHtmlBlock}
\chapter{Test Chapter}

This is my chapter body.

A nice class herin.

\begin{ccClass}{NiceClass}
\ccNestedType{NType}{comment.}
\end{ccClass}
\end{lcHtmlBlock}

% ----------------------------------------------------------------------
\section{The ccRef... macros}

\begin{ccRefClass}{Circulator_from_container<C>}

This is a ref class and should produce an index entry without the template
parameter...
\end{ccRefClass}

\begin{ccRefConcept}{Concept}
A reference concept is talked about here...
\end{ccRefConcept}

\begin{ccRefEnum}[Class::]{Enum}
A reference enumeration is talked about here...
\end{ccRefEnum}

\begin{ccRefEnum}{Enum}
A reference enumeration is talked about here...
\end{ccRefEnum}

\begin{ccRefFunction}[Class::]{Function}
A reference function is talked about here...
\end{ccRefFunction}

\begin{ccRefFunction}{Function}
A reference function is talked about here...
\end{ccRefFunction}

\begin{ccRefVariable}[Class::]{Variable}
A reference variable is talked about here...
\end{ccRefVariable}

\begin{ccRefVariable}{Variable}
A reference variable is talked about here...
\end{ccRefVariable}

\begin{ccRefMacro}[Class::]{Macro}
A reference macro is talked about here...
\end{ccRefMacro}

\begin{ccRefMacro}{Macro}
A reference macro is talked about here...
\end{ccRefMacro}


% ----------------------------------------------------------------------
\bibliographystyle{plain}
\bibliography{cgal}

\lcTex{\printindex}

\end{document}

% EOF %




