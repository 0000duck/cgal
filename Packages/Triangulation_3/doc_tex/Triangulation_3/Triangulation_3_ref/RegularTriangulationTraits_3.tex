% +------------------------------------------------------------------------+
% | Reference manual page: Triangulation_Traits_3.tex
% +------------------------------------------------------------------------+
% | 27.3.2000   Monique Teillaud
% | Package: Triangulation3
% | 
\RCSdef{\RCSTriangulationTraitsRev}{$Revision$}
\RCSdefDate{\RCSTriangulationTraitsDate}{$Date$}
% |
%%RefPage: end of header, begin of main body
% +------------------------------------------------------------------------+


\begin{ccRefConcept}{RegularTriangulationTraits_3}

To be used by \protect 
\ccc{Regular_triangulation_3<RegularTriangulationTraits_3, Tds_3>}

\ccDefinition

To be used as the traits class for the regular triangulation provided
by \cgal, a class must provide definitions for the \textit{power
tests},
in addition to the requirements described for the traits class of a 3D 
Delaunay triangulation \ccc{DelaunayTriangulationTraits_3}.

We use here the same notation as in
Section~\ref{Triangulation3-sec-class-Regulartriangulation}. 
To simplify notation, $p$ will often denote in the sequel either the
point $p\in\R^3$ or the weighted point ${p}^{(w)}=(p,w_p)$.

\ccTwo{RegularTriangulationTraits}{}
\ccNestedType{Power_test_3}{Predicate object.
Must provide the following operators:
%
\\
\\
\ccc{Oriented_side operator()( Weighted_point p,
			    Weighted_point q,
			    Weighted_point r,
			    Weighted_point s,
			    Weighted_point t)}
which performs the following:\\
Let ${z(p,q,r,s)}^{(w)}$ be the power sphere of the weighted points 
$(p,q,r,s)$. Returns \\
\ccc{ON_ORIENTED_BOUNDARY} if \ccc{t} is orthogonal to
${z(p,q,r,s)}^{(w)}$,\\ 
\ccc{ON_NEGATIVE_SIDE} if \ccc{t} lies outside the oriented sphere of
center $z(p,q,r,s)$ and radius $\sqrt{ w_{z(p,q,r,s)}^2 + w_t^2 }$
(which is equivalent to $\Pi({t}^{(w)},{z(p,q,r,s)}^{(w)} >0$)),\\
\ccc{ON_POSITIVE_SIDE} if \ccc{t} lies inside this oriented sphere.
\ccPrecond{\ccc{p, q, r, s} are not coplanar.}
%
Note that with this definition, if all the points have a weight equal
to 0, then
\ccc{power_test(p,q,r,s,t)} = \ccc{side_of_oriented_sphere(p,q,r,s,t)}.
%
%
\\
\\
\ccc{Oriented_side operator()( Weighted_point p,
			    Weighted_point q,
			    Weighted_point r,
			    Weighted_point t) }
which has an
 definition analogous to the previous method, for coplanar points,
with the power circle ${z(p,q,r)}^{(w)}$.
\ccPrecond{\ccc{p, q, r} are not collinear and \ccc{p, q, r, t} are
coplanar.}
%
If all the points have a weight equal to 0, then
\ccc{power_test(p,q,r,s,t)} = \ccc{side_of_oriented_circle(p,q,r,s,t)}.
%
\\
\\
\ccc{Oriented_side operator()( Weighted_point p,
			    Weighted_point q,
			    Weighted_point t)}
which is the same for collinear points, where ${z(p,q)}^{(w)}$ is the
power segment of \ccc{p} and \ccc{q}.
\ccPrecond{\ccc{p} and \ccc{q} have different Bare\_points, and
\ccc{p, q, t} are collinear.}
%
If all the points have a weight equal to 0, then
\ccc{power_test(p,q,t)} gives the same answer as the kernel predicate
\ccc{s(p,q).has_on(t)} would give, where  \ccc{s(p,q)} denotes the
segment with endpoints \ccc{p} and \ccc{q}.
}

\ccOperations

The following function gives access to the predicate object:
\ccThree{coplanar_side_of_bounded_circle_3xxx}{gt.compare_x(Point p0, Point p1)x}{}
\ccCreationVariable{traits}  %% choose variable name

\ccMethod{Power_test_3 power_test_3_object();}{}


\ccHasModels

\ccc{CGAL::Regular_triangulation_euclidean_traits_3<R,Weight>}.

\end{ccRefConcept}

% +------------------------------------------------------------------------+
%%RefPage: end of main body, begin of footer
% EOF
% +------------------------------------------------------------------------+

