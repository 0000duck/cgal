% +------------------------------------------------------------------------+
% | Reference manual page: Triangulation_vertex_3.tex
% +------------------------------------------------------------------------+
% | 27.3.2000   Monique Teillaud
% | Package: Triangulation3
% | 
\RCSdef{\RCSTriangulationvertexRev}{$Revision$}
\RCSdefDate{\RCSTriangulationvertexDate}{$Date$}
% |
%%RefPage: end of header, begin of main body
% +------------------------------------------------------------------------+


\begin{ccRefClass}[Triangulation_3<TriangulationTraits_3,Tds_3>::]{Vertex}  %% add template arg's if necessary

%% \ccHtmlCrossLink{}     %% add further rules for cross referencing links
%% \ccHtmlIndexC[class]{} %% add further index entries

\ccDefinition
  
The class \ccRefName\ represents a vertex of a 3D triangulation. 
It stores a point and gives access to an incident face of 
maximal dimension.

\ccInclude{CGAL/Triangulation_vertex_3.h}

\ccInheritsFrom
\ccc{Tds_3::Vertex}

\ccTypes
\ccThree{typedef TriangulationTraits_3::Pointxx}{Pointxx}{}
\ccThreeToTwo
The class \ccClassTemplateName\ defines the following types, also
defined in the class \ccc{Triangulation_3<TriangulationTraits_3,Tds_3>}:

\ccTypedef{typedef TriangulationTraits_3::Point Point;}{point type}
\ccGlue
\ccNestedType{Vertex_handle}{handle to a vertex}
\ccGlue
\ccNestedType{Cell_handle}{handle to a cell}

\ccCreationVariable{v}  %% choose variable name

\begin{ccAdvanced}
\ccCreation
The user needs to construct vertices explicitly only when implementing his
own triangulation algorithms. If he uses the standard ways proposed by 
\cgal\ to construct a triangulation, he does not need the following
constructors. 

\ccConstructor{{Triangulation_vertex_3<TriangulationTraits_3,Tds_3>}();}
{Introduces a new vertex. The geometric information is initialized by
the default constructor of the class \ccc{Point}. A test on the handle
to the incident cell of \ccVar\ for equality with \ccc{NULL} will
answer \ccc{true}.} 

\ccConstructor{{Triangulation_vertex_3<TriangulationTraits_3,Tds_3>}
(const Point & p);}
{Introduces vertex \ccVar, and initializes its geometric information.
A test on the handle
to the incident cell of \ccVar\ for equality with \ccc{NULL} will
answer \ccc{true}.}

\ccConstructor{{Triangulation_vertex_3<TriangulationTraits_3,Tds_3>}
(const Point & p,
                      		Cell_handle c);}
{Introduces vertex \ccVar, and initializes the geometric information and 
the handle to the incident cell.}

\ccConstructor{{Triangulation_vertex_3<TriangulationTraits_3,Tds_3>}
(Cell_handle c);}
{Introduces vertex \ccVar, and initializes the handle to the incident cell.}

\ccConstructor{Triangulation_vertex_3();}{default constructor.}

\ccOperations

\ccHeading{Setting}
\ccMethod{void set_cell(Cell_handle c);}
{Sets \ccVar's cell to \ccc{c}}
\ccMethod{void set_point(const Point & p);}
{Sets \ccVar's point to \ccc{p}}

\end{ccAdvanced}

\ccAccessFunctions

\ccMethod{Point point() const;}
{Returns the geometric information of \ccVar.}

\ccMethod{Cell_handle cell() const;}
{Finds a cell of the triangulation having \ccVar\ as
vertex. Remember that in degenerate dimensions, a cell stores a face
of maximal dimension (Section~\ref{Triangulation3-sec-degen_dim}).}

\ccMethod{Vertex_handle handle() const;}
{Returns a handle to the vertex.}

%\begin{ccAdvanced}
%\ccHeading{Checking}

%\ccMethod{bool is_valid(bool verbose = false) const;} 
%{Calls the \ccc{is_valid()} method of the \ccc{Tds_3} vertex class.\\
%When \ccc{verbose} is set to \ccc{true}, messages are printed to give
%a precise indication on the kind of invalidity encountered.} 
%\end{ccAdvanced} 

\ccSeeAlso

\ccc{CGAL::Triangulation_3<TriangulationTraits_3,Tds_3>::Cell}.

\end{ccRefClass}

% +------------------------------------------------------------------------+
%%RefPage: end of main body, begin of footer
% EOF
% +------------------------------------------------------------------------+

