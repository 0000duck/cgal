% +------------------------------------------------------------------------+
% | Reference manual page: Delaunay_triangulation_3.tex
% +------------------------------------------------------------------------+
% | 27.3.2000   Monique Teillaud
% | Package: Triangulation3
% | 
\RCSdef{\RCSDelaunaytriangulationRev}{$Revision$}
\RCSdefDate{\RCSDelaunaytriangulationDate}{$Date$}
% |
%%RefPage: end of header, begin of main body
% +------------------------------------------------------------------------+


\begin{ccRefClass}{Delaunay_triangulation_3<Triangulation_traits_3,Tds_3>}  %% add template arg's if necessary

%% \ccHtmlCrossLink{}     %% add further rules for cross referencing links
%% \ccHtmlIndexC[class]{} %% add further index entries

\ccDefinition
  
The class \ccRefName\ represents a three-dimensional Delaunay triangulation.

\ccInclude{CGAL/Delaunay_triangulation_3.h}

\ccInheritsFrom{\ccc{Triangulation_3<Triangulation_traits_3,Tds_3>}}

\ccTypes

Inherits the types of \ccc{Triangulation_3<Triangulation_traits_3,Tds_3>}.

\ccCreation
\ccCreationVariable{dt}  %% choose variable name
\ccThree{Vertex_handle}{dt.remove(Pont p)toto}{}

\ccConstructor{Delaunay_triangulation_3();}{default constructor.}


\ccConstructor{Delaunay_triangulation_3<Triangulation_traits_3,Tds_3>()}
{Creates an empty Delaunay triangulation.}

\ccConstructor{Delaunay_triangulation_3<Triangulation_traits_3,Tds_3>
(const Triangulation_traits_3 & traits)}
{Creates an empty Delaunay triangulation with traits class
\ccc{traits}.}

\ccConstructor{Delaunay_triangulation_3<Triangulation_traits_3,Tds_3>
(const Delaunay_triangulation_3<Triangulation_traits_3,Tds_3> & dt1)}
{Copy constructor.}

\ccOperations

\ccHeading{Insertion}

The following methods overload the corresponding methods of
triangulations to ensure the empty sphere property of Delaunay 
triangulations.

\ccMethod{Vertex_handle insert(const Point & p );}
{Inserts point \ccc{p} in the triangulation and returns the corresponding
 vertex. Similar to the insertion in a triangulation, but ensures in
addition the empty sphere property of all the created faces.}

\ccMethod{Vertex_handle insert(const Point & p, Cell_handle start);}
{Same as the previous method, \ccc{start} is used as a starting
place for the search done within the insertion.}

The following method allows one to insert several points. It returns the
number of inserted points. 

\ccMethod{template < class InputIterator >
          int
          insert(InputIterator first, InputIterator last);}
{Inserts the points in the range $\left[\right.$\ccc{first},
\ccc{last}$\left.\right)$. 
\ccPrecond{The \ccc{value_type} of \ccc{first} and \ccc{last} is
\ccc{Point}.}}
%\ccMethod{int insert(list<Point>::const_iterator first,
%	     list<Point>::const_iterator last);}
%{}
%\ccGlue
%\ccMethod{int insert(vector<Point>::const_iterator first,
%	     vector<Point>::const_iterator last);}
%{}
%\ccGlue
%\ccMethod{int insert(istream_iterator<Point, ptrdiff_t> first,
%	     istream_iterator<Point, ptrdiff_t> last);}
%{}
%\ccGlue
%\ccMethod{int insert(Point* first,
%	     Point* last);}
%{}

\ccHeading{Removal}

\ccMethod{void remove(Point p);}
{Removes the vertex associated with \ccc{p}.
\ccPrecond{There is a vertex of the triangulation associated with \ccc{p}.}\\
\textit{not yet implemented}}

\ccHeading{Queries}

\ccMethod{Bounded_side
          side_of_sphere(Cell_handle c, const Point & p) const;}
{Returns a value indicating on which side of the circumscribed sphere
of \ccc{c} the point \ccc{p} lies. More precisely, it returns:\\
- \ccc{ON_BOUNDED_SIDE} if \ccc{p} is inside the sphere. For an infinite
cell this means that \ccc{p} lies strictly either in the half space
limited by its finite facet and not containing any other point of the
triangulation, or in the interior of the disk circumscribing the
\textit{finite} facet. \\ 
- \ccc{ON_BOUNDARY} if p on the boundary of the sphere. For an infinite
cell this means that \ccc{p} lies on the circle circumscribing
the \textit{finite} facet.\\ 
- \ccc{ON_UNBOUNDED_SIDE} if \ccc{p} lies outside the sphere. For an
infinite cell this means that \ccc{p} does not satisfy either of the
two previous conditions. 
\ccPrecond{\ccVar.\ccc{dimension()} $=3$.}}
\ccMethod{Bounded_side
	  side_of_circle(const Facet & f, const Point & p) const;}
{Returns a value indicating on which side of the circumscribed circle
of \ccc{f} the point \ccc{p} lies. More precisely, it returns:\\
- in dimension~3:\\
-- For a finite facet, \ccc{ON_BOUNDARY} if \ccc{p} lies
on the circle, \ccc{ON_UNBOUNDED_SIDE} when it lies in the exterior of
the disk, \ccc{ON_BOUNDED_SIDE} when it lies in its interior.\\
-- For an infinite facet, it considers the plane defined by the finite
facet of the same cell, and does the same as in dimension~2 in this
plane.\\
- in dimension~2:\\
-- For a finite facet, \ccc{ON_BOUNDARY} if \ccc{p} lies
on the circle, \ccc{ON_UNBOUNDED_SIDE} when it lies in the exterior of
the disk, \ccc{ON_BOUNDED_SIDE} when it lies in its interior.\\
-- For an infinite facet, \ccc{ON_BOUNDARY} if the
point lies on the finite edge of \ccc{f} (endpoints included),
\ccc{ON_BOUNDED_SIDE} for a point in the open half plane defined
by \ccc{f} and not containing any other point of the triangulation,
\ccc{ON_UNBOUNDED_SIDE} elsewhere. 
\ccPrecond{\ccVar.\ccc{dimension()} $\geq 2$ and in dimension 3,
\ccc{p} is coplanar with \ccc{f}.}}

\ccMethod{Bounded_side
	  side_of_circle(Cell_handle c, int i, const Point & p);}
{Same as the previous method for facet \ccc{i} of cell \ccc{c}.}

\begin{ccAdvanced}
\ccHeading{Checking}
\ccMethod{bool
          is_valid(bool verbose = false) const;}
{Checks the combinatorial validity of the triangulation and the
validity of its geometric embedding (see
Section~\ref{Triangulation3-sec-Valid}). Also checks that all the
circumscribing spheres (resp. circles in dimension~2) of  cells
(resp. facets in dimension~2) are empty.\\ When \ccc{verbose} is set to
true,  messages describing the first invalidity encountered are
printed.}

\ccMethod{bool
          is_valid(Cell_handle c, bool verbose = false) const;}
{Checks the combinatorial and geometric validity of the cell (see
Section~\ref{Triangulation3-sec-Valid}). Also checks that the
circumscribing sphere (resp. circle in dimension~2) of  cells
(resp. facet in dimension~2) is empty.\\
 When \ccc{verbose} is set to
true, messages are printed to give
a precise indication of the kind of invalidity encountered.}

These methods are  mainly a debugging help for the users of advanced features.
\end{ccAdvanced}

\ccSeeAlso

\ccc{CGAL::Delaunay_hierarchic_triangulation_3<Triangulation_traits_3,Tds_3>}.

%% \ccExample

%% \ccIncludeExampleCode{examples/Triangulation3/Delaunay_triangulation_3_prog.C}

\end{ccRefClass}

% +------------------------------------------------------------------------+
%%RefPage: end of main body, begin of footer
% EOF
% +------------------------------------------------------------------------+

