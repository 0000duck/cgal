% +------------------------------------------------------------------------+
% | Reference manual page: Triangulation_Traits_3.tex
% +------------------------------------------------------------------------+
% | 27.3.2000   Monique Teillaud
% | Package: Triangulation3
% | 
\RCSdef{\RCSTriangulationTraitsRev}{$Revision$}
\RCSdefDate{\RCSTriangulationTraitsDate}{$Date$}
% |
%%RefPage: end of header, begin of main body
% +------------------------------------------------------------------------+


\begin{ccRefConcept}{TriangulationTraits_3}

%% \ccHtmlCrossLink{}     %% add further rules for cross referencing links
%% \ccHtmlIndexC[concept]{} %% add further index entries

To be used by \protect \ccc{Triangulation_3<TriangulationTraits_3, Tds_3>}

\ccDefinition
The concept \ccRefName\ is the first template parameter of the
triangulation class \ccc{Triangulation_3<TriangulationTraits_3,Tds_3>} of \cgal. It
defines the geometric
objects (points, segments, triangles and tetrahedra) forming the
triangulation together with a few geometric predicates on these objects:
equality, coordinate comparison, orientation in space, orientation
in case of coplanar points, and collinearity tests.

\ccTypes
\ccTwo{TriangulationTraits_3::Tetrahedron_3xx}{}

\ccNestedType{Point_3}
{The point type. It must provide assignment and copy constructor.}
\ccGlue
\ccNestedType{Segment_3}
{The  segment type.}
\ccGlue
\ccNestedType{Triangle_3}
{The triangle type.}
\ccGlue
\ccNestedType{Tetrahedron_3}
{The tetrahedron type.}

\ccNestedType{Construct_segment_3} {A constructor object for
\ccc{Segment_3}.  Must provide
\ccc{Segment_3 operator()(Point_3 p, Point_3 q)},
which constructs a  segment from two points.}
\ccGlue
\ccNestedType{Construct_triangle_3} {A constructor object for
\ccc{Triangle_3}.  Must provide
\ccc{Triangle_3 operator()(Point_3 p, Point_3 q, Point_3 r )},
which constructs a triangle from three points.}
\ccGlue
\ccNestedType{Construct_tetrahedron_3} {A constructor object for
\ccc{Tetrahedron_3}.  Must provide
\ccc{Tetrahedron_3 operator()(Point_3 p, Point_3 q, Point_3 r, Point_3 s)},
which constructs a tetrahedron from four points.}

\ccNestedType{Equal_3}{Predicate object. Must provide the operator 
\ccc{bool operator()(Point p, Point q)}
which tests whether points \ccc{p} and \ccc{q} are equal.}
%
\ccGlue
%
\ccNestedType{Compare_x_3}{Predicate object. Must provide the operator 
\ccc{Comparison_result operator()(Point p, Point q)}
which returns \ccc{SMALLER, EQUAL} or \ccc{LARGER}
according to the $x$-ordering of the points \ccc{p} and \ccc{q}.}
%
\ccGlue
%
\ccNestedType{Compare_y_3}{Predicate object. Must provide the operator
\ccc{Comparison_result operator()(Point p, Point q)}
which returns \ccc{SMALLER, EQUAL} or \ccc{LARGER}
according to the $y$-ordering of the points \ccc{p} and \ccc{q}.}
%
\ccGlue
%
\ccNestedType{Compare_z_3}{Predicate object. Must provide the operator 
\ccc{Comparison_result operator()(Point p, Point q)}
which returns \ccc{SMALLER, EQUAL} or \ccc{LARGER}
according to the $z$-ordering of the points \ccc{p} and \ccc{q}.}
%
\ccGlue
%
\ccNestedType{Orientation_3}{Predicate object. Must provide the operator
\ccc{Orientation operator()(Point p, Point q, Point r, Point s)}
which returns POSITIVE, if s lies on the positive side of the oriented plane h
defined by p, q, and r, returns NEGATIVE if s lies on the negative side of h,
and returns COPLANAR if s lies on h.}
%
\ccGlue
%
\ccNestedType{Coplanar_orientation_3}{Predicate object. Must provide
the operator \ccc{Orientation operator()(Point p, Point q, Point r)}
which returns \ccc{COLLINEAR} if the points are collinear.  Otherwise it must
return a consistent orientation for any three points chosen in a same plane.
}

\ccHasModels

The cartesian and homegenous kernels provided by \cgal.

\end{ccRefConcept}

% +------------------------------------------------------------------------+
%%RefPage: end of main body, begin of footer
% EOF
% +------------------------------------------------------------------------+

