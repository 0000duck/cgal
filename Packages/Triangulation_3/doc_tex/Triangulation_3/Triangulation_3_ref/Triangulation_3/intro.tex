% +------------------------------------------------------------------------+
% | Reference manual page: Triangulation/intro.tex
% +------------------------------------------------------------------------+
% | 27.3.2000   Monique Teillaud
% | Package: Triangulation3
% | 
\RCSdef{\RCSTriangulationcellRev}{$Revision$}
\RCSdefDate{\RCSTriangulationcellDate}{$Date$}
% |
%%RefPage: end of header, begin of main body
% +------------------------------------------------------------------------+

\clearpage
\section{Reference pages for 3D-Triangulations}

A three-dimensional triangulation is a three-dimensional simplicial
complex, pure connected and without singularities. (See
\cite{by-ag-98} or Chapter~\ref{I1_Chapter_Triangulations}.) Its
cells ($3$-faces) are such that two cells either do not intersect or
share a common facet ($2$-face), edge ($1$-face) or vertex ($0$-face).

The basic 3D-triangulation class of \cgal\ is primarily designed to
represent the triangulations of a set of points $A$ in $\R^3$.  It can
be viewed as a partition of the convex hull of {$A$} into tetrahedra
whose vertices are the points of {$A$}.  Together with the unbounded
cell having the convex hull boundary as its frontier, the triangulation
forms a partition of $\R^3$.

In order to deal
only with tetrahedra, which is convenient for many applications, the
unbounded cell can be subdivided into tetrahedra by considering that
each convex hull facet is incident to an \ccc{infinite cell} having as
fourth vertex an auxiliary vertex called the \ccc{infinite vertex}.  In
that way, each facet is incident to exactly two cells and special cases
at the boundary of the convex hull are simple to deal with.


A triangulation is a collection of vertices and cells that are linked
together through incidence and adjacency relations. Each cell gives
access to its four incident vertices and to its four adjacent
cells. Each vertex gives access to one of its incident cells.

The four vertices of a cell are indexed with 0, 1, 2 and 3 in positive
orientation, the positive orientation being defined by the orientation
of the underlying Euclidean space $\R^3$. The neighbors of a cell are also
indexed with 0, 1, 2, 3 in such a way that the neighbor indexed by $i$
is opposite to the vertex with the same index. See
Figure~\ref{Triangulation3-fig-orient}.

\subsection*{Concepts}

\ccRefIdfierPage{Triangulation_traits_3}\\
\ccRefIdfierPage{Tds_3}\\
%\ccRefIdfierPage{Tds_3}{Tds_3<Triangulation_vb_3,Triangulation_cb_3>}\\
\ccRefIdfierPage{Weighted_pt}

\subsection*{Classes}

\subsubsection*{Main classes}

\ccRefIdfierPage{CGAL::Triangulation_3<Triangulation_traits_3,Tds_3>}

\ccRefIdfierPage{CGAL::Delaunay_triangulation_3<Triangulation_traits_3,Tds_3>}\\
\ccRefIdfierPage{CGAL::Delaunay_hierarchic_triangulation_3<Triangulation_traits_3,Tds_3>}\\
\ccRefIdfierPage{CGAL::Regular_triangulation_3<Triangulation_traits_3,Tds_3>}

The vertex stores a point and gives access to an incident face of
maximal dimension. A cell of a triangulation gives access to its four
vertices indexed 0, 1, 2, and 3 in positive orientation and to its
four adjacent cells, also called neighbors. The neighbors are indexed
in such a way that neighbor $i$ lies opposite to vertex $i$.

\ccRefIdfierPage{CGAL::Triangulation_vertex_3<Triangulation_traits_3,Tds_3>}\\
\ccRefIdfierPage{CGAL::Triangulation_cell_3<Triangulation_traits_3,Tds_3>}

\subsubsection*{Traits classes}

\ccRefIdfierPage{CGAL::Triangulation_geom_traits_3<R>}\\
\ccRefIdfierPage{CGAL::Regular_triangulation_euclidean_traits_3<R,Weight>}

\subsubsection*{A class of tools}

\ccRefIdfierPage{CGAL::Triangulation_utils_3}

It defines operations on the indices of vertices and neighbors within
a cell of a triangulation.

\subsection*{Enums}

\ccRefIdfierPage{CGAL::Triangulation_3::Locate_type}

%% EOF %%
