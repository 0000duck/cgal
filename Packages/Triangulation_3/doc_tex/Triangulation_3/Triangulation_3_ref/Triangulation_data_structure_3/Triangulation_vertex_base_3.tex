% +------------------------------------------------------------------------+
% | Reference manual page: Triangulation_vertex_base_3.tex
% +------------------------------------------------------------------------+
% | 29.3.2000   Monique Teillaud
% | Package: Triangulation3
% | 
\RCSdef{\RCSTriangulationvertexbaseRev}{$Revision$}
\RCSdefDate{\RCSTriangulationvertexbaseDate}{$Date$}
% |
%%RefPage: end of header, begin of main body
% +------------------------------------------------------------------------+


\begin{ccRefClass}{Triangulation_vertex_base_3<Triangulation_traits_3>}  %% add template arg's if necessary

%% \ccHtmlCrossLink{}     %% add further rules for cross referencing links
%% \ccHtmlIndexC[class]{} %% add further index entries

\ccDefinition

The class \ccRefName\ can store the base vertex for a 3D-triangulation and a 3D-triangulation data structure. It is templated by a
geometric traits class \ccc{Triangulation_traits_3} that provides the geometric
types. The user who uses the geometric layer
(Section~\ref{TDS3-sec-design} and
Chapter~\ref{chapter-Triangulation3}) is strongly advised to use the
same geometric traits class \ccc{Triangulation_traits_3} as the one used for
\ccc{Triangulation_3<Triangulation_traits_3,Tds_3>}. In this way, the point type
defined by the base vertex is the same as the point type defined by
the geometric traits class. The default geometric traits class
provided by \cgal\ is presented in
Section~\ref{Triangulation3-sec-Traits}. 

This base class can be used directly or can serve as a base to derive
other base classes with some additionnal attributes (a color for
example) tuned for a specific application.

With this class, a vertex stores a point, which means that the points
of the user will be duplicated. This allows the user to delete his
points before the deletion of the triangulation.

\ccInclude{CGAL/Triangulation_vertex_base_3.h}

\ccIsModel

\ccc{Triangulation_vb_3}

\ccSeeAlso

\ccc{CGAL::Triangulation_vertex_base_pointer_3<Triangulation_traits_3>},\\
\ccc{CGAL::Triangulation_cell_base_3<Triangulation_traits_3>}.

%% \ccExample

%% \ccIncludeExampleCode{examples/Triangulation3/Triangulation_vertex_base_3_prog.C}

\end{ccRefClass}

% +------------------------------------------------------------------------+
%%RefPage: end of main body, begin of footer
% EOF
% +------------------------------------------------------------------------+

