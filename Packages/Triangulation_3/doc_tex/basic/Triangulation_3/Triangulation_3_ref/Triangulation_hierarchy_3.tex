% +------------------------------------------------------------------------+
% | Reference manual page: Triangulation_hierarchy_3.tex
% +------------------------------------------------------------------------+
% | 
% | Package: Triangulation3
% | 
\RCSdef{\RCSTriangulationhierarchyRev}{$Revision$}
\RCSdefDate{\RCSTriangulationhierarchyDate}{$Date$}
% |
%%RefPage: end of header, begin of main body
% +------------------------------------------------------------------------+


\begin{ccRefClass}{Triangulation_hierarchy_3<Tr>}
%% add template arg's if necessary

\ccDefinition
The class \ccRefName\ implements a triangulation augmented with
a data structure which allows fast point location queries.
As proved in~\cite{d-iirdt-98}, this structure has an optimal behaviour
when it is built for Delaunay triangulations.
However it can be used as well for other triangulations
and the \ccRefName\ class is templated by a parameter
which must be instantiated by one of the \cgal\ triangulation
classes.

\textit{In the current implementation, \ccc{Triangulation_hierarchy_3} 
supports only Delaunay triangulations:}\\
\ccc{Tr} must be defined as\\
\ccc{Delaunay_triangulation_3<TriangulationHierarchyTraits_3,TriangulationDataStructure_3>}\\
and, in addition to the requirements given by the concept
\ccc{TriangulationDataStructure_3}, the triangulation data structure
class must be parametrized by 
\ccc{Triangulation_hierarchy_vertex_base_3<TriangulationVertexBase_3>}
and \ccc{TriangulationCellBase_3}.

\ccInclude{CGAL/Triangulation_hierarchy_3.h}

\ccInheritsFrom

\ccc{Tr}

\ccRefName\ offers exactly the same functionalities as \ccc{Tr}.
Most of them (point location, insertion, removal\ldots) are overloaded to
improve their efficiency by using the hierarchic structure. 

Note that, since the algorithms that are provided are randomized, the
running time of constructing a triangulation with a hierarchy may be
improved when shuffling the data points.

However, the I/0 operations are not overloaded. So, writing a
hierarchy into a file will lose the hierarchic structure and reading
it from the file will result in an ordinary triangulation whose
efficiency will be the same as \ccc{Tr}.

%\ccTypes
%The class \ccRefName\ inherits the types from its base triangulation
%class \ccc{Tr}.

%\ccCreation
%\ccCreationVariable{trh}  %% choose variable name

%\ccConstructor{Triangulation_hierarchy_3(const Traits& gt =
%Traits());}{default constructor.}
%\ccGlue
%\ccConstructor{Triangulation_hierarchy_3(const
%Triangulation_hierarchy_3& trh1);}{Each triangulation in the hierarchy 
%is duplicated.}
%\ccGlue
%\ccMethod{Triangulation_hierarchy_3<Tr> operator=(const Triangulation_hierarchy_3<Tr>& tr);}
%{Assignment. Each triangulation in the hierarchy is duplicated.}

%\ccHeading{Helper functions}
%\ccMethod{void swap(Triangulation_hierarchy_3<Tr>& trh1);}
%{The triangulations \ccc{trh1} and \ccVar\ are swapped.
%\ccVar. \ccc{swap(tr)} should be preferred to \ccc{trh} = \ccc{trh1} or to
%\ccc{trh(trh1)} if \ccc{trh} is deleted after that.}
%\ccGlue
%\ccMethod{void clear();}{Delete all triangulations in the hierarchy.}

%\ccHeading{Insertion and Removal}
%The class \ccRefName\ overwrites the following insertion and removal
%member functions to update the whole data structure at each operation.

%\ccMethod{Vertex_handle insert(const Point& p);}{Inserts  point \ccc{p}.
%If point \ccc{p} coincides with an already existing vertex, this 
%vertex is returned and the triangulation hierarchy is not updated.}
%\ccGlue
%\ccMethod{Vertex_handle push_back(const Point& p);}
%{Equivalent to \ccc{insert(p)}.}
%\ccGlue
%\ccMethod{template < class InputIterator >
%          int
%          insert(InputIterator first, InputIterator last);}
%{inserts the points in the range
% $\left[\right.$\ccc{first}, \ccc{last}$\left.\right)$.
% Returns the number of inserted points.
% \ccPrecond The \ccc{value_type} of \ccc{first} and \ccc{last}
% is \ccc{Point}.}

%\ccMethod{void    remove(Vertex_handle v);}
%{removes the vertex from the triangulation hierarchy.}

%\ccHeading{Point location}
%The following point location member functions are overwritten.
%They take advantage of the hierarchy to achieve an efficient point location.


%\ccMethod{Cell_handle locate(const Point& p) const;}
%{Locates the point \ccc{p} in the triangulation. See the description of the
%\ccc{locate} member functions in the class \ccc{Triangulation_3<Traits,Tds>}. }

%\ccMethod{Cell_handle  locate(const Point& p, Locate_type& lt, int& li, int &lj) const;}
%{Same as above. Additionally, the locate type \ccc{lt} and the indices
%\ccc{li} and \ccc{lj} describing precisely the location are set.}

%\begin{ccAdvanced}
%\ccHeading{Checking}
%\ccMethod{bool is_valid() const;}{The \ccc{is_valid()} member function is
%overwritten to check the validity of the hierarchy, as well as the validity of
%the bottom triangulation.}
%\end{ccAdvanced}

\ccImplementation
The data structure is a hierarchy 
of triangulations. The triangulation at the lowest level is
the original triangulation where operations and point location are to 
be performed.
Then at each succedding level, the data structure
stores a triangulation of a small random sample of the vertices
of the triangulation at the preceeding level. Point location
is done through a top-down nearest neighbor query.
The nearest neighbor query is first
performed naively in the top level triangulation.
Then, at each following level, the nearest neighbor at that level
is found through a linear walk performed from
the nearest neighbor found at the preceeding level.
Because the number of vertices in each triangulation is only a small
fraction of the number of vertices of the preceeding triangulation 
the data structure remains small and achieves fast point location 
queries on real
data. 

\ccSeeAlso
\ccc{CGAL::Triangulation_hierarchy_vertex_base_3<TriangulationVertexBase_3>}\\
\ccc{TriangulationHierarchyTraits_3}

\end{ccRefClass}

% +------------------------------------------------------------------------+
%%RefPage: end of main body, begin of footer
% EOF
% +------------------------------------------------------------------------+
