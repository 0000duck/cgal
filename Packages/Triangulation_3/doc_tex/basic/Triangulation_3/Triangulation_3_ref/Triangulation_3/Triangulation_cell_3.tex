% +------------------------------------------------------------------------+
% | Reference manual page: Triangulation_cell_3.tex
% +------------------------------------------------------------------------+
% | 27.3.2000   Monique Teillaud
% | Package: Triangulation3
% | 
\RCSdef{\RCSTriangulationcellRev}{$Revision$}
\RCSdefDate{\RCSTriangulationcellDate}{$Date$}
% |
%%RefPage: end of header, begin of main body
% +------------------------------------------------------------------------+


\begin{ccRefClass}{Triangulation_cell_3<Triangulation_traits_3,Tds_3>}  %% add template arg's if necessary

%% \ccHtmlCrossLink{}     %% add further rules for cross referencing links
%% \ccHtmlIndexC[class]{} %% add further index entries

\ccDefinition
  
The class \ccRefName\ represents cells of a 3D triangulation. 
A cell gives access to its four vertices indexed 0,
1, 2, and 3 in positive orientation and to its four adjacent cells, also
called neighbors. The neighbors are indexed in such a way that neighbor
$i$ lies opposite to vertex $i$.
In degenerate dimensions, cells are used to store faces of maximal
dimension: (Section~\ref{Triangulation3-sec-degen_dim}).

\ccInclude{CGAL/Triangulation_cell_3.h}

\ccInheritsFrom
\ccc{Tds_3::Cell}

\ccTypes

The class \ccClassTemplateName\ defines the same types as the
\ccc{Triangulation_vertex_3<Triangulation_traits_3,Tds_3>} class.

\begin{ccAdvanced}
\ccCreation
The user needs to construct cells explicitly only when implementing his
own triangulation algorithms. The constructors of a cell do not insert
it into any triangulation. To add a cell to a given triangulation,
the \ccc{add_cell()} method of the triangulation must be used. Note
that a given cell can be inserted into only textit{one} triangulation.

\ccConstructor{Triangulation_cell_3<Triangulation_traits_3,Tds_3>();}
{Introduces a cell \ccVar\ and initializes all its vertices and
neighbors in such a way that tests on the handles for equality with
\ccc{NULL} will answer \ccc{true}.} 

%\ccConstructor{\ccClassName(Tds_3 & tds);}
%{Introduces a variable \ccVar, initializes the triangulation data
%structure related, and initializes all its vertices and
%neighbors as the preceding constructor.}

\ccConstructor{Triangulation_cell_3<Triangulation_traits_3,Tds_3>(
                            Vertex_handle v0,	
    	                    Vertex_handle v1,
                            Vertex_handle v2,
                            Vertex_handle v3);}
{Introduces a cell \ccVar, and initializes its vertices. The 
neighbors are initialized so that tests on the handles for equality with
\ccc{NULL} will answer \ccc{true}.} 

\ccConstructor{Triangulation_cell_3<Triangulation_traits_3,Tds_3>(
                        Vertex_handle v0,
                    	Vertex_handle v1,
                    	Vertex_handle v2,
                    	Vertex_handle v3,
                    	Cell_handle n0,
                    	Cell_handle n1,
                    	Cell_handle n2,
                    	Cell_handle n3);}
{Introduces a variable \ccVar, and initializes its vertices and neighbors.}

\ccOperations

\ccHeading{Setting}
\ccThree{Cell_handle}{tds.set_number_of_vertices()}{}

\ccMethod{void set_vertices();}
{Sets all vertices so that tests on the handles for equality with
\ccc{NULL} will answer \ccc{true}.}
\ccGlue
\ccMethod{void set_vertex(int i, Vertex_handle v);}
{Sets vertex \ccc{i} of \ccVar\ to be \ccc{v}.
\ccPrecond{$i \in \{0,1,2,3\}$}}
\ccGlue
\ccMethod{void set_vertices(
	          Vertex_handle v0, 
                  Vertex_handle v1, 
                  Vertex_handle v2, 
                  Vertex_handle v3);}
{Sets vertices to the given vertices.}

\ccMethod{void set_neighbors();}
{Sets all neighbors so that tests on the handles for equality with
\ccc{NULL} will answer \ccc{true}.}
\ccGlue
\ccMethod{void set_neighbor(int i, Cell_handle n);}
{Sets neighbor \ccc{i} to be \ccc{n}.
\ccPrecond{$i \in \{0,1,2,3\}$.}}
\ccGlue
\ccMethod{void set_neighbors(
                  Cell_handle n0,
                  Cell_handle n1,  
                  Cell_handle n2,  
                  Cell_handle n3);}
{Sets neighbors to the given cells.}
\end{ccAdvanced} 

\ccAccessFunctions
\ccThree{Vertex_handle}{c.has_vertex( Vertex_handle) v)}{}

\ccMethod{Vertex_handle vertex(int i) const;}
{Returns vertex \ccc{i} of \ccVar.
\ccPrecond{$i \in \{0,1,2,3\}$.}}
\ccGlue
\ccMethod{int index(Vertex_handle v) const;}
{Returns the index of vertex \ccc{v} in \ccVar.
\ccPrecond{\ccc{v} is a vertex of \ccVar.}}
\ccGlue
\ccMethod{bool has_vertex(Vertex_handle v) const;}
{Returns \ccc{true} if  \ccc{v} is a vertex of \ccVar.}
\ccGlue
\ccMethod{bool has_vertex(Vertex_handle v, int & i) const;}
{Returns \ccc{true} if \ccc{v} is a vertex of \ccVar, and
computes the index \ccc{i} of the vertex.}

\ccMethod{Cell_handle neighbor(int i) const;}
{Returns neighbor \ccc{i} of \ccVar.
\ccPrecond{$i \in \{0,1,2,3\}$.}}
\ccGlue
\ccMethod{int index(Cell_handle n) const;}
{Returns the index corresponding to neighboring cell \ccc{n}.
\ccPrecond{\ccc{n} is a neighbor of \ccVar.}}
\ccGlue
\ccMethod{bool has_neighbor(Cell_handle n) const;}
{Returns \ccc{true} if \ccc{n} is a neighbor of \ccVar.}
\ccGlue
\ccMethod{bool has_neighbor(Cell_handle n, int & i) const;}
{Returns \ccc{true} if \ccc{n} is a neighbor of \ccVar, and
computes the index \ccc{i} of the neighbor.}

\ccMethod{Cell_handle handle() const;}
{Returns a handle to the cell.}

\ccSeeAlso

%% \ccExample

%% \ccIncludeExampleCode{examples/Triangulation3/Triangulation_cell_3_prog.C}

\end{ccRefClass}

% +------------------------------------------------------------------------+
%%RefPage: end of main body, begin of footer
% EOF
% +------------------------------------------------------------------------+

