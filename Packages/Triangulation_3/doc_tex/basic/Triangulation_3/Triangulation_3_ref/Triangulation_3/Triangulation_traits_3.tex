% +------------------------------------------------------------------------+
% | Reference manual page: Triangulation_Traits_3.tex
% +------------------------------------------------------------------------+
% | 27.3.2000   Monique Teillaud
% | Package: Triangulation3
% | 
\RCSdef{\RCSTriangulationTraitsRev}{$Revision$}
\RCSdefDate{\RCSTriangulationTraitsDate}{$Date$}
% |
%%RefPage: end of header, begin of main body
% +------------------------------------------------------------------------+


\begin{ccRefConcept}{Triangulation_traits_3}

%% \ccHtmlCrossLink{}     %% add further rules for cross referencing links
%% \ccHtmlIndexC[concept]{} %% add further index entries

\subsubsection{To be used by \protect \ccc{Triangulation_3<Triangulation_traits_3, Tds_3>}}

\ccDefinition
The concept \ccRefName\ is the first template parameter of the
triangulation class \ccc{Triangulation_3<Triangulation_traits_3,Tds_3>} of \cgal. It
defines the geometric
objects (points, segments, triangles and tetrahedra) forming the
triangulation together with a few geometric predicates on these objects:
equality, coordinate comparison, orientation in space, orientation
in case of coplanar points, and collinearity tests.

\ccTypes
\ccTwo{Triangulation_traits_3::Tetrahedron}{}

\ccNestedType{Point_3}
{The point type. It must provide assignment and copy constructor.}
\ccGlue
\ccNestedType{Vector_3}
{The vector type. It must be constructible as the difference of two points.}
\ccGlue
\ccNestedType{Segment_3}
{The  segment type.}
\ccGlue
\ccNestedType{Triangle_3}
{The triangle type.}
\ccGlue
\ccNestedType{Tetrahedron_3}
{The tetrahedron type.}

\ccNestedType{Construct_segment_3} {A constructor object for
\ccc{Segment_3}.Must provide \ccc{Segment_3 operator()(Point_3 p,Point_3
                       q)}, which constructs a  segment from two points.}
\ccGlue
\ccNestedType{Construct_triangle_3} {A constructor object for
\ccc{Triangle_3}.Must provide \ccc{Triangle_3 operator()(Point_3 p,Point_3
                       q,Point_3 r )}, which constructs a triangle from three points.}
\ccGlue
\ccNestedType{Construct_tetrahedron_3} {A constructor object for
\ccc{Tetrahedron_3}.Must provide \ccc{Tetrahedron_3 operator()(Point_3 p,Point_3
                       q,Point_3 r, Point_3 s )}, which constructs a tetrahedron from four points.}

\ccNestedType{Equal_3}{Predicate object. Must provide
the operator 
\ccc{Comparison_result operator()(Point p, Point q)}
which tests whether points \ccc{p} and \ccc{q} are equal.}
%
\ccGlue
%
\ccNestedType{Collinear_3}{Predicate object. Must provide
the operator 
\ccc{Comparison_result operator()(Point p, Point q, Point r)}
which tests whether points \ccc{p}, \ccc{q} and \ccc{r} are collinear.}
%
\ccGlue
%
\ccNestedType{Compare_x_3}{Predicate object. Must provide
the operator 
\ccc{Comparison_result operator()(Point p, Point q)}
which returns 
\ccc{SMALLER, EQUAL} or \ccc{ LARGER}
 according ding to the
$x$-ordering of points \ccc{p} and \ccc{q}.}
%
\ccGlue
%
\ccNestedType{Compare_y_3}{Predicate object. Must proveide
the operator 
\ccc{Comparison_result operator()(Point p, Point q)}
which returns the
(\ccc{SMALLER, EQUAL} or \ccc{ LARGER})
according to the
$y$-ordering of points \ccc{p} and \ccc{q}.}
%
\ccGlue
%
\ccNestedType{Compare_z_3}{Predicate object. Must proveide
the operator 
\ccc{Comparison_result operator()(Point p, Point q)}
which returns the
(\ccc{SMALLER, EQUAL} or \ccc{ LARGER})
according to the
$z$-ordering of points \ccc{p} and \ccc{q}.}
%
\ccGlue
%
\ccNestedType{Orientation_3}{Predicate object. Must provide
the  operator
\ccc{Orientation operator()(Point p, Point q, Point r, Point s)}
which returns
\ccc{LEFTTURN} (resp. \ccc{RIGHTTURN}) is $s$ is on the left 
(resp. on the right) of the oriented plane \ccc{pqr}
and \ccc{COPLANAR} is the four points are coplanar.}
%
\ccGlue
%
\ccNestedType{Coplanar_orientation_3}{Predicate object. Must provide
the  operator
\ccc{Orientation operator()(Point q, Point r, Point s, Point test)}
which tests whether \ccc{test} is on the same side of \ccc{(q, r)} as
\ccc{s} in the common plane of the four points. Returns \ccc{COLLINEAR} if
\ccc{test, q, r} are collinear, \ccc{POSITIVE} if \ccc{(q, r, test)}
and \ccc{(q, r, s)} have the same orientation, \ccc{NEGATIVE} if
\ccc{(q, r, test)} and \ccc{(q, r, s)} have opposite orientations. 
\ccPrecond{\ccc{test, q, r, s} are coplanar and \ccc{q, r, s} are not
collinear.}
}


\subsubsection{To be used by \protect 
\ccc{Delaunay_triangulation_3<Triangulation_traits_3, Tds_3>}}

\ccDefinition
In addition to the requirements described before, the geometric traits
class of a Delaunay triangulation must fulfill the following
requirements:

\ccNestedType{Side_of_oriented_sphere_3}{Predicate object. Must provide
the  operator
\ccc{Orientation operator()(Point p, Point q, Point r, Point s, Point test)}
which determines on which side of the oriented sphere circumscribing 
\ccc{p, q, r, s} the point \ccc{test} lies.}
%

%
\ccGlue
%
\ccNestedType{Coplanar_side_of_oriented_circle_3}{Predicate object. Must provide
the  operator
\ccc{Orientation operator()(Point p, Point q, Point r, Point test, Vector v)}
which determines on which side of the oriented circle circumscribing
\ccc{p, q, r} the point \ccc{test} lies. 
\ccPrecond{\ccc{p, q, r, test} are coplanar and \ccc{v} is the normal vector of this plane
 and \ccc{p, q, r} are notcollinear.}}



\subsubsection{To be used by \protect 
\ccc{Delaunay_hierarchic_triangulation_3<Triangulation_traits_3, Tds_3>}}

\textit{Not yet implemented}

\subsubsection{To be used by \protect 
\ccc{Regular_triangulation_3<Triangulation_traits_3, Tds_3>}}

\ccDefinition

To be used as the traits class for the regular triangulation provided
by \cgal, a class must provide definitions for the \textit{power tests}.

We use here the same notation as in
Section~\ref{Triangulation3-sec-class-Regulartriangulation}. 

\ccNestedType{Power_test_3}{Predicate object. Must provide the following
operators:



\ccc{Oriented_side operator()( Weighted_point p,
			    Weighted_point q,
			    Weighted_point r,
			    Weighted_point s,
			    Weighted_point t) ;}
{Let ${z(p,q,r,s)}^{(w)}$ be the power sphere of the weighted points 
$(p,q,r,s)$. Returns \\
\ccc{ON_ORIENTED_BOUNDARY} if \ccc{t} is orthogonal to
${z(p,q,r,s)}^{(w)}$,\\ 
\ccc{ON_NEGATIVE_SIDE} if \ccc{t} lies outside the oriented sphere of
center $z(p,q,r,s)$ and radius $\sqrt{ w_{z(p,q,r,s)}^2 + w_t^2 }$
(which is equivalent to $\Pi({t}^{(w)},{z(p,q,r,s)}^{(w)} >0$)),\\
\ccc{ON_POSITIVE_SIDE} if \ccc{t} lies inside this oriented sphere.
\ccPrecond{\ccc{p, q, r, s} are not coplanar.}}

Note that with this definition, if all the points have a weight equal
to 0, then
\ccc{power_test(p,q,r,s,t)} = \ccc{side_of_oriented_sphere(p,q,r,s,t)}.


\ccc{Oriented_side operator()( Weighted_point p,
			    Weighted_point q,
			    Weighted_point r,
			    Weighted_point t) ;}
{Analogous definition as the previous method, for coplanar points,
with the power circle ${z(p,q,r)}^{(w)}$.
\ccPrecond{\ccc{p, q, r} are not collinear and \ccc{p, q, r, t} are
coplanar.}}

If all the points have a weight equal to 0, then
\ccc{power_test(p,q,r,s,t)} = \ccc{side_of_oriented_circle(p,q,r,s,t)}.

\ccc{Oriented_side operator()( Weighted_point p,
			    Weighted_point q,
			    Weighted_point t) ;}
{Same for collinear points, where ${z(p,q)}^{(w)}$ is the
power segment of \ccc{p} and \ccc{q}.
\ccPrecond{\ccc{p} and \ccc{q} have different Bare\_points, and
\ccc{p, q, t} are collinear.}}

If all the points have a weight equal to 0, then
\ccc{power_test(p,q,t)} gives the same answer as the kernel predicate
\ccc{s(p,q).has_on(t)} would give, where  \ccc{s(p,q)} denotes the
segment with endpoints \ccc{p} and \ccc{q}.}



\ccCreation
\ccCreationVariable{traits}  %% choose variable name
\ccThree{Comparison_result}{compare_x()toto}{}

\ccConstructor{Triangulation_traits_3_3();}{default constructor.}
Only a default constructor is required.

\ccConstructor{Triangulation_traits_3();} 
{A default constructor.}

\ccOperations

%\ccThree{Comparison_result}{traits.orientation_in_plane( Point q, Point r, Point s, Point test)}{} 

\ccHeading{Predicate functions}
The following functions give access to the  predicate 
and constructor objects.
\ccThree{Construct_segment_3}{gt.compare_x(Point p0, Point p1)x}{}
%
\ccMethod{Construct_segment_3 construct_segment_3_object();}{}
\ccGlue
\ccMethod{Construct_triangle_3 construct_triangle_3_object();}{}
\ccGlue
\ccMethod{Construct_tetrahedron_3 construct_tetrahedron_3_object();}{}
\ccGlue
\ccMethod{Comparison_x_3 compare_x_3_object();}{}
\ccGlue
\ccMethod{Comparison_y_3 compare_y_3_object();}{}
\ccGlue
\ccMethod{Comparison_z_3 compare_z_3_object();}{}
\ccGlue
\ccMethod{Orientation_3  orientation_3_object();}{}
\ccGlue
\ccMethod{Collinear_3  collinear_3_object();}{}
\ccGlue
\ccMethod{Coplanar_orientation_3  coplanar_orientation_3_object();}{}
\ccGlue
\ccGlue
\ccMethod{{Side_of_oriented_sphere_3  side_of_oriented_sphere_3_object();}{}
\ccGlue
\ccGlue
\ccMethod{coplanar_side_of_oriented_circle_3  coplanar_side_of_oriented_circle_3_object();}{}
\ccGlue
\ccGlue
\ccMethod{Power_test_3  power_test_3_object();}{}
\ccGlue



\ccHasModels

\ccc{CGAL::Triangulation_geom_traits_3<R>},\\
\ccc{CGAL::Regular_triangulation_euclidean_traits_3<R,Weight>}.


%% \ccExample

%% \ccIncludeExampleCode{examples/Triangulation3/Triangulation_Traits_3_prog.C}

\end{ccRefConcept}

% +------------------------------------------------------------------------+
%%RefPage: end of main body, begin of footer
% EOF
% +------------------------------------------------------------------------+

