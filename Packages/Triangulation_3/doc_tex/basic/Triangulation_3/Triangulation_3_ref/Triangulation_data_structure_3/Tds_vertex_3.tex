% +------------------------------------------------------------------------+
% | Reference manual page: Tds_vertex_3.tex
% +------------------------------------------------------------------------+
% | 29.3.2000   Monique Teillaud
% | Package: Triangulation3
% | 
\RCSdef{\RCSTdsvertexRev}{$Revision$}
\RCSdefDate{\RCSTdsvertexDate}{$Date$}
% |
%%RefPage: end of header, begin of main body
% +------------------------------------------------------------------------+


\begin{ccRefConcept}[TriangulationDataStructure_3::]{Vertex}

%% \ccHtmlCrossLink{}     %% add further rules for cross referencing links
%% \ccHtmlIndexC[concept]{} %% add further index entries

\ccDefinition
  
The concept \ccRefName\ represents the vertex class of a 3D-triangulation data structure. It must define
the types and operations listed in this section. Some of these
requirements are of geometric nature, they are \textit{optional}
when using the triangulation data structure class alone. They become
compulsory when the triangulation data structure is used as a layer
for the geometric triangulation class. (See Section~\ref{TDS3-sec-design}.)

\ccTypes
\ccThree{typedef TriangulationDataStructure_3::Vertex}{Facet }{}
\ccThreeToTwo
\ccNestedType{Point}{\textit{Optional for the triangulation data
structure alone.}} 

The class \ccClassName\ defines types that are the same as some of the 
types defined by the triangulation data structure class \ccc{TriangulationDataStructure}.

\ccTypedef{typedef TriangulationDataStructure_3::Vertex Vertex;}{}
\ccGlue
\ccTypedef{typedef TriangulationDataStructure_3::Cell Cell;}{}

\ccCreation
\ccCreationVariable{v}  %% choose variable name

In order to obtain new vertices or destruct unused vertices, the user must
call the \ccc{create_vertex()} and \ccc{delete_vertex()} methods of the
triangulation data structure.

%\ccConstructor{Vertex();}
%{Introduces a vertex \ccVar. 
%The pointer to the incident cell is initialized with \ccc{NULL}.}

%\ccConstructor{Vertex(const Point & p);}
%{Introduces a vertex \ccVar. Initializes its point with \ccc{p}.
%The pointer to the incident cell is initialized with
%\ccc{NULL}. {\textit{Optional for the triangulation data structure alone.}}}

%\ccConstructor{Vertex(const Point & p, Cell* c);}
%{Introduces a vertex \ccVar, and initializes the point with \ccc{p}
%and sets the pointer to the incident cell to \ccc{c}. {\textit{Optional for the
%triangulation data structure alone.}}}

%\ccConstructor{Vertex(Cell* c);}
%{Introduces a vertex \ccVar, and sets the pointer to the incident cell
%to \ccc{c}.}

%\ccConstructor{Vertex();}{default constructor.}

\ccOperations
\ccThree{Vertex*}{v.set_point()}{}

\ccAccessFunctions

\ccMethod{Cell* cell() const;}
{Returns a cell of the triangulation having \ccVar\ as vertex.}

\ccHeading{Setting}

\ccMethod{void set_cell(Cell* c);}
{Sets the incident cell to \ccc{c}.}

\ccMethod{void set_point(const Point & p);}
{Sets the point to \ccc{p}. {\textit{Optional for the
triangulation data structure alone.}}}

\begin{ccAdvanced}
\ccHeading{Checking}

\ccMethod{bool is_valid(bool verbose = false) const;}
{Checks the validity of the vertex. Must check that its incident cell
has this vertex. The validity of the base vertex is also checked.\\
When \ccc{verbose} is set to \ccc{true}, messages are printed to give
a precise indication on the kind of invalidity encountered.}
\end{ccAdvanced}

%\ccHasModels

%\ccc{CGAL::Triangulation_data_structure_3<Triangulation_vb_3,Triangulation_cb_3>::Vertex}.

\ccSeeAlso

\ccc{TriangulationDataStructure_3::Cell}.

%% \ccExample

%% \ccIncludeExampleCode{examples/Triangulation3/Tds_vertex_3_prog.C}

\end{ccRefConcept}

% +------------------------------------------------------------------------+
%%RefPage: end of main body, begin of footer
% EOF
% +------------------------------------------------------------------------+

