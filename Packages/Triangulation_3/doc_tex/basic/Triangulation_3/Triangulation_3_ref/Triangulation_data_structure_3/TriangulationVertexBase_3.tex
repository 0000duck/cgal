% +------------------------------------------------------------------------+
% | Reference manual page: TriangulationVertexBase_3.tex
% +------------------------------------------------------------------------+
% | 29.3.2000   Monique Teillaud
% | Package: Triangulation3
% | 
\RCSdef{\RCSTriangulationvbRev}{$Revision$}
\RCSdefDate{\RCSTriangulationvbDate}{$Date$}
% |
%%RefPage: end of header, begin of main body
% +------------------------------------------------------------------------+


\begin{ccRefConcept}{TriangulationVertexBase_3}

%% \ccHtmlCrossLink{}     %% add further rules for cross referencing links
%% \ccHtmlIndexC[concept]{} %% add further index entries

\ccDefinition
  
At the bottom level of 3D-triangulations 
(see Sections~\ref{Triangulation3-sec-design} and~\ref{TDS3-sec-design}),
a vertex provides access to
one of its incident cells through a \ccc{void*} pointer. 
It can also \textit{optionally} store a point. The
information about the point is in fact not used by the triangulation
data structure class provided by \cgal\ and is therefore only
\textit{optional} when this triangulation data structure class is used
alone. It becomes compulsory when the 
triangulation data structure is used as a layer for the geometric
triangulation class. 

\ccTypes
\ccNestedType{Point}
{Must be the same as the point type \ccc{Traits::Point_3} defined by the
geometric traits class of the triangulation. {\textit{Optional for the
triangulation data structure alone.}}} 

\ccCreation
\ccCreationVariable{v}  %% choose variable name

\ccConstructor{TriangulationVertexBase_3();}
{Default constructor}
\ccGlue
\ccConstructor{TriangulationVertexBase_3(const Point & p);}
{Constructs a vertex whose geometric emdedding is point
\ccc{p}. {\textit{Optional for the triangulation data structure
alone.}}} 
\ccGlue
\ccConstructor{TriangulationVertexBase_3(const Point & p,
                     		void* c);}
{Constructs a vertex emdedded in point \ccc{p} and pointing to cell
\ccc{c}. {\textit{Optional for the triangulation data structure alone.}}}
\ccGlue
\ccConstructor{TriangulationVertexBase_3(void* c);}
{Constructs a vertex pointing to cell \ccc{c}.}

\ccOperations

\ccAccessFunctions

\ccMethod{Point point() const;}
{Returns the point. {\textit{Optional for the triangulation data
structure alone.}}} 
\ccGlue
\ccMethod{void* cell() const;}
{Returns the pointer to an incident cell}

\ccHeading{Setting}

\ccMethod{void set_point(Point p);}
{Sets the point. {\textit{Optional for the triangulation data
structure alone.}}} 
\ccGlue
\ccMethod{void set_cell(void* c);}
{Sets the incident cell.}

\ccHeading{Checking}
\ccMethod{bool is_valid(bool verbose=false) const;}
{Performs any desired geometric test on a vertex. Checks that the
pointer to an incident cell is not \ccc{NULL}.}

\ccHeading{I/O}

\ccFunction{istream& operator>>
(istream& is, TriangulationVertexBase_3 & v);}
{Inputs the non-combinatorial information given by the vertex: 
the optional point and other possible information.}
%\ccPrecond{The point and the other information have a corresponding
%operator \ccc{>>}. \textit{This precondition is optional for the
%triangulation data structure alone.}}}  

\ccFunction{ostream& operator<< (ostream& os, 
			const TriangulationVertexBase_3 & v);}
{Outputs the non combinatorial operation given by the vertex: the
optional point and other possible information.}
%\ccPrecond{The point and the other information have a corresponding
%operator \ccc{<<}. \textit{This precondition is optional for the
%triangulation data structure alone.}}} 

\ccHasModels

\ccc{CGAL::Triangulation_vertex_base_3<Triangulation_traits_3>},\\
\ccc{CGAL::Triangulation_vertex_base_pointer_3<Triangulation_traits_3>}.

\ccSeeAlso

\ccc{TriangulationCellBase_3}.

%% \ccExample

%% \ccIncludeExampleCode{examples/Triangulation3/Triangulation_vb_prog.C}

\end{ccRefConcept}

% +------------------------------------------------------------------------+
%%RefPage: end of main body, begin of footer
% EOF
% +------------------------------------------------------------------------+

