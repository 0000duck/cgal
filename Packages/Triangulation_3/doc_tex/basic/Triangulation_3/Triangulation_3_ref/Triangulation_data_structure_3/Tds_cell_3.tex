% +------------------------------------------------------------------------+
% | Reference manual page: Tds_cell_3.tex
% +------------------------------------------------------------------------+
% | 29.3.2000   Monique Teillaud
% | Package: Triangulation3
% | 
\RCSdef{\RCSTdscellRev}{$Revision$}
\RCSdefDate{\RCSTdscellDate}{$Date$}
% |
%%RefPage: end of header, begin of main body
% +------------------------------------------------------------------------+


\begin{ccRefConcept}{Tds_cell_3}

%% \ccHtmlCrossLink{}     %% add further rules for cross referencing links
%% \ccHtmlIndexC[concept]{} %% add further index entries

\ccDefinition
  
The concept \ccRefName\ stores
four pointers to its four vertices and four pointers to its four
neighbors The vertices are indexed 0, 1, 2, and 3 in positive order.
The neighbor indexed $i$ lies opposite to vertex \ccc{i}.

In degenerate dimensions, cells are used to store faces of maximal
dimension: in dimension~2, each cell represents only one
facet of index 3, and 3 edges $(0,1)$, $(1,2)$ and $(2,0)$; in
dimension~1, each cell represents one edge $(0,1)$. (See also
Section~\ref{TDS3-sec-degen_dim}.) 

\ccTypes
\ccThree{typedef triple <Cell*, int, int>}{Facet }{}
\ccThreeToTwo
The class \ccClassName\ defines the following types.

\ccTypedef{typedef Tds::Vertex Vertex;}{}
\ccGlue
\ccTypedef{typedef Tds::Cell Cell;}{}

\ccCreation
\ccCreationVariable{c}  %% choose variable name
\ccThree{Tds::Vertex}{v(const Point & p, Cell* c);}{}

The constructors of a cell do not insert it into any triangulation
data structure. To add a cell into a given triangulation
data structure, the \ccc{add_cell()} method of the triangulation
data structure must be used. Note that a given cell can only be
inserted into \textit{one} triangulation data structure.

\ccConstructor{Cell();}
{Introduces a cell \ccVar\ and initializes all vertices and neighbors 
 with \ccc{NULL}.}

%\ccConstructor{\ccClassName(Tds & tds);}
%{Introduces a cell \ccVar\ and inserts it in the triangulation 
%data structure \ccc{tds}.}

\ccConstructor{Cell(Vertex* v0, Vertex* v1, Vertex* v2, Vertex* v3);}
{Introduces a cell \ccVar, and initializes its vertices. The neighbors
are initialized with \ccc{NULL}.} 

\ccConstructor{Cell(Vertex* v0, Vertex* v1, Vertex* v2, Vertex* v3,
                Cell* n0, Cell* n1, Cell* n2, Cell* n3);}
{Introduces a cell \ccVar, and initializes its vertices and neighbors.}

\ccOperations

\ccAccessFunctions

\ccMethod{Vertex* vertex(int i) const;}
{Returns the vertex \ccc{i} of \ccVar.
\ccPrecond{$i \in \{0, 1, 2, 3\}$.}}
\ccGlue
\ccMethod{int index(const Vertex* v) const;}
{Returns the index of vertex \ccc{v} in \ccVar.
\ccPrecond{\ccc{v} is a vertex of \ccVar}.}
\ccGlue
\ccMethod{bool has_vertex(const Vertex* v) const;}
{Returns \ccc{true} if  \ccc{v} is a vertex of \ccVar.}
\ccGlue
\ccMethod{bool has_vertex(const Vertex* v, int & i) const;}
{Returns \ccc{true} if \ccc{v} is a vertex of \ccVar, and
computes its index \ccc{i} in \ccVar.}

\ccMethod{Cell* neighbor(int i) const;}
{Returns  the neighbor \ccc{i} of \ccVar.
\ccPrecond{$i \in \{0, 1, 2, 3\}$.}}
\ccGlue
\ccMethod{int index(const Cell* n) const;}
{Returns the index corresponding to neighboring cell \ccc{n}.
\ccPrecond{\ccc{n} is a neighbor of \ccVar.}}
\ccGlue
\ccMethod{bool has_neighbor(const Cell* n) const;}
{Returns \ccc{true} if \ccc{n} is a neighbor of \ccVar.}
\ccGlue
\ccMethod{bool has_neighbor(const Cell* n, int & i) const;}
{Returns \ccc{true} if \ccc{n} is a neighbor of \ccVar,  and
computes its index \ccc{i} in \ccVar.}

\ccMethod{Vertex* mirror_vertex(int i) const;}
{Returns the vertex of the neighbor of \ccVar that is opposite to \ccVar.
\ccPrecond{$i \in \{0, 1, 2, 3\}$.}}
\ccMethod{int mirror_index(int i) const;}
{Returns the index of \ccVar\ in its $i^{th}$ neighbor.
\ccPrecond{$i \in \{0, 1, 2, 3\}$.}}

\ccHeading{Setting}

\ccMethod{void set_vertex(int i, Vertex* v);}
{Sets vertex \ccc{i} to \ccc{v}.
\ccPrecond{$i \in \{0, 1, 2, 3\}$.}}
\ccGlue
\ccMethod{void set_vertices(Vertex* v0,
			Vertex* v1,
		    	Vertex* v2,
		    	Vertex* v3);}
{Sets the vertex pointers.}

\ccMethod{void set_neighbor(int i, Cell* n);}
{Sets neighbor \ccc{i} to \ccc{n}.
\ccPrecond{$i \in \{0, 1, 2, 3\}$.}}
\ccGlue
\ccMethod{void set_neighbors(Cell* n0,
			     Cell* n1,
			     Cell* n2,
			     Cell* n3);}
{Sets the neighbors pointers.}

\ccHeading{Checking}
\ccMethod{bool is_valid(int dim=3, bool verbose = false) const;}
{\ccc{dim} is the dimension of the triangulation, with default
value~3.\\
In dimension 3 (resp. 2, 1) this function must check that the cell
shares three (resp. two, one) vertices with its neighbors, and these
neighbors have a correct reciprocal neighboring link. Moreover, the
consistency of the orientations of common faces must be checked.\\ 
The validity of all the vertices of the cell must be checked.\\
The validity of the base cell is also checked.\\ 
When \ccc{verbose} is set to \ccc{true}, messages are printed to give
a precise indication of the kind of invalidity encountered.}

\ccHasModels

\ccc{Triangulation_ds_cell_3}.

\ccSeeAlso

\ccc{Tds_vertex_3}.

%% \ccExample

%% \ccIncludeExampleCode{examples/Triangulation3/Tds_cell_3_prog.C}

\end{ccRefConcept}

% +------------------------------------------------------------------------+
%%RefPage: end of main body, begin of footer
% EOF
% +------------------------------------------------------------------------+

