% +------------------------------------------------------------------------+
% | Reference manual page: TriangulationDSVertexBase_3.tex
% +------------------------------------------------------------------------+
% | 29.3.2000   Monique Teillaud
% | Package: Triangulation3
% | 
\RCSdef{\RCSTriangulationvbRev}{$Revision$}
\RCSdefDate{\RCSTriangulationvbDate}{$Date$}
% |
%%RefPage: end of header, begin of main body
% +------------------------------------------------------------------------+


\begin{ccRefConcept}{TriangulationDSVertexBase_3}

\ccDefinition
  
At the bottom level of 3D-triangulations 
(see Sections~\ref{Triangulation3-sec-design} and~\ref{TDS3-sec-design}),
a vertex provides access to one of its incident cells through a handle.

Note that when you use the triangulation data structure as parameter of a
geometric triangulation, the vertex base class has additional geometric
requirements : it has to match the \ccc{TriangulationVertexBase_3} concept.

Since the Triangulation data structure is the class which defines the handle
types, the vertex base class has to be somehow parameterized by the
Triangulation data structure.  But since it is itself parameterized by the cell
and vertex base classes, there is a cycle in the definition of these classes.
In order to break the cycle, the base classes for vertex and cell which are
given as arguments for the Triangulation data structure use \ccc{void} as
Triangulation data structure parameter, and the Triangulation data structure
then uses a {\it rebind}-like mecanism (similar to the one specified in
\ccc{std::allocator}) in order to put itself as parameter to the vertex and
cell classes.  The {\it rebound} base classes so obtained are the classes which
are used as base class for the final vertex and cell classes.

\ccTypes
\ccThree{typedef TriangulationDataStructure_3::Vertex_handle}{}{}
\ccThreeToTwo
The class \ccRefName\ has to define the following types.

\ccNestedType{
  template <typename TDS2>
  struct Rebind_TDS;}
{This nested template class has to define a type \ccc{Other} which is the
{\it rebound} vertex, that is, the one whose \ccc{Triangulation_data_structure}
will be the actually used one.  The \ccc{Other} type will be the real base
class of \ccc{Triangulation_data_structure_3::Vertex}.}

\ccTypedef{typedef TriangulationDataStructure_3 Triangulation_data_structure;}{}
\ccGlue
\ccTypedef{typedef TriangulationDataStructure_3::Vertex Vertex;}{}
\ccGlue
\ccTypedef{typedef TriangulationDataStructure_3::Cell Cell;}{}
\ccGlue
\ccTypedef{typedef TriangulationDataStructure_3::Vertex_handle Vertex_handle;}{}
\ccGlue
\ccTypedef{typedef TriangulationDataStructure_3::Cell_handle Cell_handle;}{}


\ccCreation
\ccCreationVariable{v}  %% choose variable name

\ccThree{Triangulation}{Facetxxxxxxxxxxxxxxxxxxxxxxxxx}{}

\ccConstructor{TriangulationVertexBase_3();}
{Default constructor.}
\ccGlue
\ccConstructor{TriangulationVertexBase_3(Cell_handle c);}
{Constructs a vertex pointing to cell \ccc{c}.}

\ccOperations

\ccAccessFunctions

\ccMethod{Cell_handle cell() const;}
{Returns the pointer to an incident cell}

\ccHeading{Setting}

\ccMethod{void set_cell(Cell_handle c);}
{Sets the incident cell.}

\ccHeading{Checking}
\ccMethod{bool is_valid(bool verbose=false) const;}
{Performs any desired test on a vertex. Checks that the
pointer to an incident cell is not \ccc{NULL}.}

\ccHeading{Various}

\ccMethod{void * for_compact_container() const;}{}
\ccGlue
\ccMethod{void * & for_compact_container();}{}
{ These member functions are required by \ccc{Triangulation_data_structure_3}
  because it uses \ccc{Compact_container} to store its cells.  See the
  documentation of \ccc{Compact_container} for the exact requirements.}

\ccHeading{I/O}

\ccFunction{istream& operator>>
(istream& is, TriangulationDSVertexBase_3 & v);}
{Inputs the non-combinatorial information given by the vertex: 
the optional point and other possible information.}
%\ccPrecond{The point and the other information have a corresponding
%operator \ccc{>>}. \textit{This precondition is optional for the
%triangulation data structure alone.}}}  

\ccFunction{ostream& operator<< (ostream& os, 
			const TriangulationDSVertexBase_3 & v);}
{Outputs the non combinatorial operation given by the vertex: the
optional point and other possible information.}
%\ccPrecond{The point and the other information have a corresponding
%operator \ccc{<<}. \textit{This precondition is optional for the
%triangulation data structure alone.}}} 

\ccHasModels

\ccc{CGAL::Triangulation_ds_vertex_base_3}\\
\ccc{CGAL::Triangulation_vertex_base_3}\\
\ccc{CGAL::Triangulation_hierarchy_vertex_base_3}

\ccSeeAlso

\ccc{TriangulationVertexBase_3}\\
\ccc{TriangulationHierarchyVertexBase_3}\\
\ccc{TriangulationDSCellBase_3}\\
\ccc{TriangulationCellBase_3}

%% \ccExample

%% \ccIncludeExampleCode{examples/Triangulation3/Triangulation_vb_prog.C}

\end{ccRefConcept}
