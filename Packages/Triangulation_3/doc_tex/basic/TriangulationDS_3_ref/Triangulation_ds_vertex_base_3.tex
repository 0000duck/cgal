% +------------------------------------------------------------------------+
% | Reference manual page: Triangulation_vertex_base_3.tex
% +------------------------------------------------------------------------+
% | 29.3.2000   Monique Teillaud
% | Package: Triangulation3
% | 
\RCSdef{\RCSTriangulationvertexbaseRev}{$Revision$}
\RCSdefDate{\RCSTriangulationvertexbaseDate}{$Date$}
% |
%%RefPage: end of header, begin of main body
% +------------------------------------------------------------------------+


\begin{ccRefClass}{Triangulation_ds_vertex_base_3<>}

%% \ccHtmlCrossLink{}     %% add further rules for cross referencing links
%% \ccHtmlIndexC[class]{} %% add further index entries

\ccDefinition

The class \ccRefName\ can be used as the base vertex for a 3D-triangulation
data structure.

Note that if the triangulation data structure is used as a parameter of a
geometric triangulation (Section~\ref{TDS3-sec-design} and
Chapter~\ref{chapter-Triangulation3}), then the vertex base class has to
fulfill additional geometric requirements, i.e. it has to be a model of the
concept \ccc{TriangulationVertexBase_3}.

This base class can be used directly or can serve as a base to derive
other base classes with some additional attributes (a color for
example) tuned for a specific application.

\ccInclude{CGAL/Triangulation_ds_vertex_base_3.h}

\ccIsModel

\ccc{TriangulationDSVertexBase_3}

\ccSeeAlso

\ccc{CGAL::Triangulation_vertex_base_3<TriangulationTraits_3>},\\
\ccc{CGAL::Triangulation_ds_cell_base_3<>}.

%% \ccExample

%% \ccIncludeExampleCode{examples/Triangulation3/Triangulation_vertex_base_3_prog.C}

\end{ccRefClass}

% +------------------------------------------------------------------------+
%%RefPage: end of main body, begin of footer
% EOF
% +------------------------------------------------------------------------+

