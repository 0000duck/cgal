% +------------------------------------------------------------------------+
% | Reference manual page: TriangulationCellBase_3.tex
% +------------------------------------------------------------------------+
% | 29.3.2000   Monique Teillaud
% | Package: Triangulation3
% | 
\RCSdef{\RCSTriangulationcbRev}{$Revision$}
\RCSdefDate{\RCSTriangulationcbDate}{$Date$}
% |
%%RefPage: end of header, begin of main body
% +------------------------------------------------------------------------+


\begin{ccRefConcept}{TriangulationCellBase_3}

%% \ccHtmlCrossLink{}     %% add further rules for cross referencing links
%% \ccHtmlIndexC[concept]{} %% add further index entries

\ccDefinition
  
At the base level 
(see Sections~\ref{Triangulation3-sec-design} and~\ref{TDS3-sec-design}),
a cell stores \ccc{void*} pointers to its four
vertices and to its four neighbor cells. 
The vertices and neighbors are indexed 0, 1, 2 and 3. Neighbor $i$
lies opposite to vertex $i$.

\ccCreation
\ccCreationVariable{c}  %% choose variable name

\ccConstructor{Cell_base();}
{Default constructor}
\ccGlue
\ccConstructor{Cell_base( void* v0, void* v1, void* v2, void* v3);}
{Initializes the vertices with \ccc{v0, v1, v2, v3}. Neighbors are
initialized to \ccc{NULL}.} 
\ccGlue
\ccConstructor{Cell_base(void* v0, void* v1, void* v2, void* v3,
			    void* n0, void* n1, void* n2, void* n3)} 
{Initializes the vertices with \ccc{v0, v1, v2, v3} and the neighbors with
\ccc{n0, n1, n2, n3}.}
                    
\ccOperations

\ccMethod{void init();}
{This member function is called by the triangulation data structure before
using the cell, it can be used by the user to initialize some of the fields he
defines. }

\ccAccessFunctions

\ccMethod{void* vertex(int i) const;}
{Returns the vertex \ccc{i} of \ccVar.
\ccPrecond{$i \in \{0, 1, 2, 3\}$.}}
\ccGlue
\ccMethod{int vertex_index(const void* v) const;}
{Returns the index of \ccc{v}.
\ccPrecond{\ccc{v} is a vertex of \ccVar}}
\ccGlue
\ccMethod{bool has_vertex(const void* v);}
{True iff \ccc{v} is a vertex of \ccVar.}
\ccGlue
\ccMethod{bool has_vertex(const void* v, int & i) const;}
{Returns \ccc{true} if \ccc{v} is a vertex of \ccVar, and
computes its index \ccc{i} in \ccVar.}

\ccMethod{void* neighbor(int i) const;}
{Returns  the neighbor \ccc{i} of \ccVar.
\ccPrecond{$i \in \{0, 1, 2, 3\}$.}}
\ccGlue
\ccMethod{int cell_index(const void* n) const;}
{Returns the index of cell \ccc{n} in \ccVar.
\ccPrecond{\ccc{n} is a neighbor of \ccVar.}}
\ccGlue
\ccMethod{bool has_neighbor(void* n);}
{Returns \ccc{true} if \ccc{n} is a neighbor of \ccVar.}
\ccGlue
\ccMethod{bool has_neighbor(const void* n, int & i) const;}
{Returns \ccc{true} if \ccc{n} is a neighbor of \ccVar,  and
computes its index \ccc{i} in \ccVar.}

\ccHeading{Setting}

\ccMethod{void set_vertex(int i, void* v);}
{Sets vertex \ccc{i} to \ccc{v}.
\ccPrecond{$i \in \{0, 1, 2, 3\}$.}}
\ccGlue
\ccMethod{ void set_vertices();}
{Sets the vertices to \ccc{NULL}.}
\ccGlue
\ccMethod{void set_vertices(void* v0,
		    	    void* v1,
		       	    void* v2,
		    	    void* v3);}
{Sets the vertices.}

\ccMethod{void set_neighbor(int i, void* n);}
{Sets neighbor \ccc{i} to \ccc{n}.
\ccPrecond{$i \in \{0, 1, 2, 3\}$.}}
\ccGlue
\ccMethod{void set_neighbors();} 
{Sets the neighbors to \ccc{NULL}.}
\ccGlue
\ccMethod{void set_neighbors(void* n0,
		     	void* n1,
		     	void* n2,
		     	void* n3);}
{Sets the neighbors.}

\ccHeading{Checking}
\ccMethod{bool is_valid(bool verbose = false) const;}
{Performs any desired geometric test on a cell.\\
When \ccc{verbose} is set to \ccc{true}, messages are printed to give
a precise indication of the kind of invalidity encountered.}

\ccHeading{I/O}

\ccFunction{istream& operator>>
(istream& is, Cell_base & c);}
{Inputs the possible non combinatorial information given by the cell.
\ccPrecond{This information has a corresponding operator \ccc{>>}.}}

\ccFunction{ostream& operator<< (ostream& os, 
			const Cell_base & c);}
{Outputs the possible non combinatorial information given by the cell.
\ccPrecond{This information has a corresponding operator \ccc{>>}.}}

\ccHasModels

\ccc{CGAL::Triangulation_cell_base_3<TriangulationTraits_3>}.

\ccSeeAlso

\ccc{TriangulationVertexBase_3}

%% \ccExample

%% \ccIncludeExampleCode{examples/Triangulation3/Triangulation_cb_3_prog.C}

\end{ccRefConcept}

% +------------------------------------------------------------------------+
%%RefPage: end of main body, begin of footer
% EOF
% +------------------------------------------------------------------------+

