% +------------------------------------------------------------------------+
% | Reference manual page: Delaunay_triangulation_3.tex
% +------------------------------------------------------------------------+
% | 27.3.2000   Monique Teillaud
% | Package: Triangulation3
% | 
\RCSdef{\RCSDelaunaytriangulationRev}{$Revision$}
\RCSdefDate{\RCSDelaunaytriangulationDate}{$Date$}
% |
%%RefPage: end of header, begin of main body
% +------------------------------------------------------------------------+


\begin{ccRefClass}{Delaunay_triangulation_3<DelaunayTriangulationTraits_3,TriangulationDataStructure_3>}  %% add template arg's if necessary

%% \ccHtmlCrossLink{}     %% add further rules for cross referencing links
%% \ccHtmlIndexC[class]{} %% add further index entries

\ccDefinition
  
The class \ccRefName\ represents a three-dimensional Delaunay
triangulation.

The user is advised to use \ccc{Triangulation_hierarchy<Tr>} rather
than the basic Delaunay triangulation. It offers the same
functionalities but is much more efficient for large data sets.

\ccInclude{CGAL/Delaunay_triangulation_3.h}

\ccInheritsFrom{\ccc{Triangulation_3<DelaunayTriangulationTraits_3,TriangulationDataStructure_3>}}

\ccTypes

\ccThree{typedef DelaunayTriangulationTraits_3::Direction_3 Direction;}{}
\ccThreeToTwo
In addition to those inherited, the following types are used for the
construction of the Voronoi diagram:

\ccTypedef{typedef DelaunayTriangulationTraits_3::Line_3 Line;}{}
\ccGlue
\ccTypedef{typedef DelaunayTriangulationTraits_3::Ray_3 Ray;}{}
\ccGlue
\ccTypedef{typedef DelaunayTriangulationTraits_3::Plane_3 Plane;}{}
\ccGlue
\ccTypedef{typedef DelaunayTriangulationTraits_3::Direction_3 Direction;}{}
\ccGlue
\ccTypedef{typedef DelaunayTriangulationTraits_3::Object_3 Object;}{}

\ccCreation
\ccCreationVariable{dt}  %% choose variable name
\ccThree{Vertex_handle}{dt.remove(Point p)toto}{}

\ccConstructor{Delaunay_triangulation_3();}{default constructor.}


\ccConstructor{Delaunay_triangulation_3<DelaunayTriangulationTraits_3,TriangulationDataStructure_3>()}
{Creates an empty Delaunay triangulation.}

\ccConstructor{Delaunay_triangulation_3<DelaunayTriangulationTraits_3,TriangulationDataStructure_3>
(const DelaunayTriangulationTraits_3 & traits)}
{Creates an empty Delaunay triangulation with traits class
\ccc{traits}.}

\ccConstructor{Delaunay_triangulation_3<DelaunayTriangulationTraits_3,TriangulationDataStructure_3>
(const
Delaunay_triangulation_3<DelaunayTriangulationTraits_3,TriangulationDataStructure_3> & dt1)}
{Copy constructor.}

\ccOperations

\ccHeading{Insertion}

The following methods overload the corresponding methods of
triangulations to ensure the empty sphere property of Delaunay 
triangulations.
\ccThree{Vertex_handle}{dt.remove()}{}

\ccMethod{Vertex_handle insert(const Point & p, Cell_handle start = NULL );}
{Inserts point \ccc{p} in the triangulation and returns the corresponding
 vertex. Similar to the insertion in a triangulation, but ensures in
addition the empty sphere property of all the created faces.
The optional argument \ccc{start} is used as a starting place for the search.}

The following method allows one to insert several points. It returns the
number of inserted points. 

\ccMethod{template < class InputIterator >
          int
          insert(InputIterator first, InputIterator last);}
{Inserts the points in the range $\left[\right.$\ccc{first},
\ccc{last}$\left.\right)$. 
\ccPrecond{The \ccc{value_type} of \ccc{first} and \ccc{last} is
\ccc{Point}.}}
%\ccMethod{int insert(list<Point>::const_iterator first,
%	     list<Point>::const_iterator last);}
%{}
%\ccGlue
%\ccMethod{int insert(vector<Point>::const_iterator first,
%	     vector<Point>::const_iterator last);}
%{}
%\ccGlue
%\ccMethod{int insert(istream_iterator<Point, ptrdiff_t> first,
%	     istream_iterator<Point, ptrdiff_t> last);}
%{}
%\ccGlue
%\ccMethod{int insert(Point* first,
%	     Point* last);}
%{}

\ccMethod{Vertex_handle push_back(const Point& p);}
{Equivalent to \ccc{insert(p)}.}

\ccHeading{Removal}

When a vertex \ccc{v} is removed from a triangulation, all the cells
incident to \ccc{v} must be removed, and the polyhedral region
consisting of all the tetrahedra that are incident to \ccc{v} must be
retriangulated. 
So, the problem reduces to triangulating a polyhedral
region, while preserving its boundary, or to compute a
\textit{constrained} triangulation. This is known to be sometimes
impossible: the Sch\"onhardt polyhedron cannot be triangulated
\cite{s-cgehd-98}. 

However, when dealing with Delaunay triangulations, the case of such
polyhedra that cannot be retriangulated cannot happen, so \cgal\
proposes a vertex removal.

\ccMethod{bool remove(Vertex_handle v);}
{Removes the vertex \ccc{v} from the triangulation and returns
\ccc{true}. It returns false if something went wrong, and in this case
the triangulation is not modified; this should not happen, but may
happen in denegerate cases if the predicates are not exact (which can
occur with a \cgal\ kernel if the arithmetic chosen is not good).
\ccPrecond{\ccc{v} is a vertex of the triangulation and it is not the
infinite vertex.}}

\ccHeading{Queries}

\ccMethod{Bounded_side
          side_of_sphere(Cell_handle c, const Point & p) const;}
{Returns a value indicating on which side of the circumscribed sphere
of \ccc{c} the point \ccc{p} lies. More precisely, it returns:\\
- \ccc{ON_BOUNDED_SIDE} if \ccc{p} is inside the sphere. For an infinite
cell this means that \ccc{p} lies strictly either in the half space
limited by its finite facet and not containing any other point of the
triangulation, or in the interior of the disk circumscribing the
\textit{finite} facet. \\ 
- \ccc{ON_BOUNDARY} if p on the boundary of the sphere. For an infinite
cell this means that \ccc{p} lies on the circle circumscribing
the \textit{finite} facet.\\ 
- \ccc{ON_UNBOUNDED_SIDE} if \ccc{p} lies outside the sphere. For an
infinite cell this means that \ccc{p} does not satisfy either of the
two previous conditions. 
\ccPrecond{\ccVar.\ccc{dimension()} $=3$.}}
\ccMethod{Bounded_side
	  side_of_circle(const Facet & f, const Point & p) const;}
{Returns a value indicating on which side of the circumscribed circle
of \ccc{f} the point \ccc{p} lies. More precisely, it returns:\\
- in dimension~3:\\
-- For a finite facet, \ccc{ON_BOUNDARY} if \ccc{p} lies
on the circle, \ccc{ON_UNBOUNDED_SIDE} when it lies in the exterior of
the disk, \ccc{ON_BOUNDED_SIDE} when it lies in its interior.\\
-- For an infinite facet, it considers the plane defined by the finite
facet of the same cell, and does the same as in dimension~2 in this
plane.\\
- in dimension~2:\\
-- For a finite facet, \ccc{ON_BOUNDARY} if \ccc{p} lies
on the circle, \ccc{ON_UNBOUNDED_SIDE} when it lies in the exterior of
the disk, \ccc{ON_BOUNDED_SIDE} when it lies in its interior.\\
-- For an infinite facet, \ccc{ON_BOUNDARY} if the
point lies on the finite edge of \ccc{f} (endpoints included),
\ccc{ON_BOUNDED_SIDE} for a point in the open half plane defined
by \ccc{f} and not containing any other point of the triangulation,
\ccc{ON_UNBOUNDED_SIDE} elsewhere. 
\ccPrecond{\ccVar.\ccc{dimension()} $\geq 2$ and in dimension 3,
\ccc{p} is coplanar with \ccc{f}.}}

\ccMethod{Bounded_side
	  side_of_circle(Cell_handle c, int i, const Point & p);}
{Same as the previous method for facet \ccc{i} of cell \ccc{c}.}

\ccHeading{Voronoi diagram}
\ccIndexMainItem{Voronoi diagram}
\cgal\ offers several functionalities to display the Voronoi diagram of 
a set of points in 3D.

\ccMethod{Point dual(Cell_handle c) const;}
{Returns the circumcenter of the four vertices of c.
\ccPrecond{\ccVar.\ccc{dimension()}$=3$ and \ccc{c} is not infinite.}}

\ccMethod{Object dual(Facet f) const;}
{Returns the dual of facet \ccc{f}, which is \\
in dimension 3: either a segment, if the two cells incident to \ccc{f}  
are finite, or a ray, if one of them is infinite;\\
in dimension 2: a point.
\ccPrecond{\ccVar.\ccc{dimension()} $\geq 2$ and \ccc{f} is not infinite.}}

\ccMethod{Object dual(Cell_handle c, int i) const;}
{same as the previous method for facet \ccc{(c,i)}.}

\ccMethod{template <class Stream> Stream & draw_dual(Stream & os);}
{Sends the set of duals to all the facets of \ccVar\ into \ccc{os}.}


\begin{ccAdvanced}
\ccHeading{Checking}
\ccMethod{bool
          is_valid(bool verbose = false) const;}
{Checks the combinatorial validity of the triangulation and the
validity of its geometric embedding (see
Section~\ref{Triangulation3-sec-intro}). Also checks that all the
circumscribing spheres (resp. circles in dimension~2) of  cells
(resp. facets in dimension~2) are empty.\\ When \ccc{verbose} is set to
true,  messages describing the first invalidity encountered are
printed.}

\ccMethod{bool
          is_valid(Cell_handle c, bool verbose = false) const;}
{Checks the combinatorial and geometric validity of the cell (see
Section~\ref{Triangulation3-sec-intro}). Also checks that the
circumscribing sphere (resp. circle in dimension~2) of  cells
(resp. facet in dimension~2) is empty.\\
 When \ccc{verbose} is set to
true, messages are printed to give
a precise indication of the kind of invalidity encountered.}

These methods are  mainly a debugging help for the users of advanced features.
\end{ccAdvanced}

\ccSeeAlso

\ccc{CGAL::Triangulation_hierarchy_3<Tr>}.

%% \ccExample

%% \ccIncludeExampleCode{examples/Triangulation3/Delaunay_triangulation_3_prog.C}

\end{ccRefClass}

% +------------------------------------------------------------------------+
%%RefPage: end of main body, begin of footer
% EOF
% +------------------------------------------------------------------------+

