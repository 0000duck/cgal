% +------------------------------------------------------------------------+
% | Reference manual page: Triangulation_Traits_3.tex
% +------------------------------------------------------------------------+
% | 27.3.2000   Monique Teillaud
% | Package: Triangulation3
% | 
\RCSdef{\RCSTriangulationTraitsRev}{$Revision$}
\RCSdefDate{\RCSTriangulationTraitsDate}{$Date$}
% |
%%RefPage: end of header, begin of main body
% +------------------------------------------------------------------------+


\begin{ccRefConcept}{DelaunayTriangulationTraits_3}

%% \ccHtmlCrossLink{}     %% add further rules for cross referencing links
%% \ccHtmlIndexC[concept]{} %% add further index entries

To be used by \protect 
\ccc{Delaunay_triangulation_3<DelaunayTriangulationTraits_3, TriangulationDataStructure_3>}

\ccDefinition
In addition to the requirements described for the traits class of a 3D 
triangulation \ccc{TriangulationTraits_3}, 
the geometric traits
class of a Delaunay triangulation must fulfill the following
requirements:

\ccTwo{DelaunayTriangulationTraits_3}{}

\ccNestedType{Side_of_oriented_sphere_3}{Predicate object. Must provide
the operator\\
\ccc{Oriented_side operator()(Point p, Point q, Point r, Point s, Point test)}
which determines on which side of the oriented sphere circumscribing 
\ccc{p, q, r, s} the point \ccc{test} lies.}
%
\ccGlue
%
\ccNestedType{Coplanar_side_of_bounded_circle_3}{Predicate object. Must provide
the operator\\
\ccc{Bounded_side operator()(Point p, Point q, Point r, Point s)}
which determines the bounded side of the circle defined
by \ccc{p}, \ccc{q}, and \ccc{r} on which \ccc{s} lies.
\ccPrecond \ccc{p}, \ccc{q}, \ccc{r}, and \ccc{s} are coplanar and
\ccc{p}, \ccc{q}, and \ccc{r} are not collinear.}

\ccNestedType{Compare_distance_3}{Predicate object.
Must provide the operator\\
\ccc{Comparison_result operator()(Point p, Point q, Point r)}
which compares the distance between \ccc{p} and \ccc{q} to the distance
between \ccc{p} and \ccc{r}.}

\ccTwo{DelaunayTriangulationTraits_3::Direction_3xxx}{}

\ccNestedType{Line_3}{The line type.}
\ccGlue
\ccNestedType{Plane_3}{The plane type.}
\ccGlue
\ccNestedType{Ray_3}{The ray type.}
\ccGlue
\ccNestedType{Direction_3}{The direction type.}
\ccGlue
\ccNestedType{Object_3}{The object type.}

In addition, in order to display a Voronoi diagram, the traits class
must provide the following:

\ccTwo{DelaunayTriangulationTraits_3}{}
\ccNestedType{Construct_circumcenter_3}{A constructor object that
must provide the operator\\
\ccc{Point_3 operator()(Point_3 p, Point_3 q, 
                                         Point_3 r, Point_3 s)}
that contructs the circumcenter of four non coplanar points and the operator\\
\ccc{Point_3 operator()(Point_3 p, Point_3 q, Point_3 r)}
that contructs the circumcenter of three non collinear points.}
\ccGlue
\ccNestedType{Construct_plane_3}{A constructor object
that must provide the operator\\
\ccc{Plane_3 operator()(Point_3 p, Point_3 q, Point_3 r)}
that constructs the plane passing through the non collinear points
\ccc{p}, \ccc{q} and \ccc{r}.}
\ccGlue
\ccNestedType{Construct_perpendicular_line_3}{A constructor object
that must provide the operator\\
\ccc{Line_3 operator()(Plane_3 pl, Point_3 p)}
that constructs the line perpendicular to \ccc{pl} and passing through
\ccc{p}.}
\ccGlue
\ccNestedType{Construct_direction_of_line_3}{A constructor object
that must provide the operator\\
\ccc{Direction_3 operator()(Line_3 l)} that constructs the oriented
direction of \ccc{l}.} 
\ccGlue
\ccNestedType{Construct_ray_3}{A constructor object
that must provide the operator\\
\ccc{Ray_3 operator()(Point_3 p, Direction_3 d)} that constructs the
ray starting at \ccc{p} and with direction \ccc{d}.}
\ccGlue
\ccNestedType{Construct_object_3}{A constructor object
that must provide the operators\\
\ccc{Object_3 operator()(Point_3 p)},\\
\ccc{Object_3 operator()(Segment_3 s)} and\\
\ccc{Object_3 operator()(Ray_3 r)}\\
that construct an object respectively from a point, a segment and a ray.}


\ccOperations

The following functions give access to the predicate and construction objects:

\ccThree{coplanar_side_of_bounded_circle_3xxx}{gt.xxxxxxxxx(Point p0, Point p1)x}{}
\ccCreationVariable{traits}  %% choose variable name

\ccMethod{Side_of_oriented_sphere_3 side_of_oriented_sphere_3_object();}{}
\ccGlue
\ccMethod{Coplanar_side_of_bounded_circle_3 coplanar_side_of_bounded_circle_3_object();}{}
\ccGlue
\ccMethod{Compare_distance_3 compare_distance_3_object();}{}

The following functions must be provided only if the methods of 
\ccc{Delaunay_triangulation_3} returning elements of the Voronoi diagram 
are instantiated:

\ccMethod{Construct_circumcenter_3 construct_circumcenter_3_object();}{}
\ccGlue
\ccMethod{Construct_perpendicular_line_3 construct_perpendicular_line_object();}{}
\ccGlue
\ccMethod{Construct_plane_3 construct_plane_3_object();}{}
\ccGlue
\ccMethod{Construct_direction_of_line_3 construct_direction_of_line_3_object();}{}
\ccGlue
\ccMethod{Construct_ray_3 construct_ray_3_object();}{}
\ccGlue
\ccMethod{Construct_object_3 construct_object_3_object();}{}

\ccHasModels

The Cartesian and homogeneous kernels provided by \cgal.

\end{ccRefConcept}

% +------------------------------------------------------------------------+
%%RefPage: end of main body, begin of footer
% EOF
% +------------------------------------------------------------------------+

