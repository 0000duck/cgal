% +------------------------------------------------------------------------+
% | Reference manual page: Triangulation_data_structure/intro.tex
% +------------------------------------------------------------------------+
% | 27.3.2000   Monique Teillaud
% | Package: Triangulation3
% | 
\RCSdef{\RCSTriangulationcellRev}{$Revision$}
\RCSdefDate{\RCSTriangulationcellDate}{$Date$}
% |
%%RefPage: end of header, begin of main body
% +------------------------------------------------------------------------+
%\clearpage
%\section{Reference pages for 3D-Triangulation data structures}
\chapter{3D Triangulation Data Structures}

The triangulation data structure must be able to represent a
triangulation of a topological sphere $S^d$ of $\R^{d+1}$, for 
$d \in \{-1,0,1,2,3\}$. (See~\ref{TDS3-sec-intro}.)

The vertex class of a 3D-triangulation data structure must define
a number of types and operations. Some of these
requirements are of geometric nature, they are \textit{optional}
when using the triangulation data structure class alone. They become
compulsory when the triangulation data structure is used as a layer
for the geometric triangulation class. (See Section~\ref{TDS3-sec-design}.)

The cell class of a triangulation data structure stores
four pointers to its four vertices and four pointers to its four
neighbors. The vertices are indexed 0, 1, 2, and 3 in positive order.
The neighbor indexed $i$ lies opposite to vertex \ccc{i}.

In degenerate dimensions, cells are used to store faces of maximal
dimension: in dimension~2, each cell represents only one
facet of index 3, and 3 edges $(0,1)$, $(1,2)$ and $(2,0)$; in
dimension~1, each cell represents one edge $(0,1)$. (See 
Section~\ref{TDS3-sec-intro}.) 

\subsection*{Concepts}

%\ccRefIdfierPage{Tds_3<Triangulation_vb_3,Triangulation_cb_3>}
\ccRefConceptPage{TriangulationDataStructure_3}

\ccRefConceptPage{TriangulationDataStructure_3::Vertex} \\
\ccRefConceptPage{TriangulationDataStructure_3::Cell}

\ccRefConceptPage{TriangulationVertexBase_3} \\
\ccRefConceptPage{TriangulationCellBase_3}

\subsection*{Classes}

\ccRefIdfierPage{CGAL::Triangulation_data_structure_3<TriangulationVertexBase_3,TriangulationCellBase_3>}

This class is a model for the concept of the 3D-triangulation data
structure \ccc{TriangulationDataStructure_3} and provides the optional
geometric 
functionalities. It is templated by base classes for vertices and cells.

%\ccRefIdfierPage{CGAL::Triangulation_data_structure_3<Triangulation_vb_3,Triangulation_cb_3>::Vertex}\\
%\ccRefIdfierPage{CGAL::Triangulation_data_structure_3<Triangulation_vb_3,Triangulation_cb_3>::Cell}

\cgal\ provides two base vertex classes and one base cell class:

\ccRefIdfierPage{CGAL::Triangulation_vertex_base_3<TriangulationTraits_3>}\\
\ccRefIdfierPage{CGAL::Triangulation_vertex_base_pointer_3<TriangulationTraits_3>}\\
\ccRefIdfierPage{CGAL::Triangulation_cell_base_3<TriangulationTraits_3>}

