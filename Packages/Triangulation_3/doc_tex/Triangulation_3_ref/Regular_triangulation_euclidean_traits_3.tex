% +------------------------------------------------------------------------+
% | Reference manual page: Regular_triangulation_euclidean_traits_3.tex
% +------------------------------------------------------------------------+
% | 27.3.2000   Monique Teillaud
% | Package: Triangulation3
% | 
\RCSdef{\RCSRegulartriangulationeuclideantraitsRev}{$Revision$}
\RCSdefDate{\RCSRegulartriangulationeuclideantraitsDate}{$Date$}
% |
%%RefPage: end of header, begin of main body
% +------------------------------------------------------------------------+


\begin{ccRefClass}{Regular_triangulation_euclidean_traits_3<R,Weight>}  %% add template arg's if necessary

%% \ccHtmlCrossLink{}     %% add further rules for cross referencing links
%% \ccHtmlIndexC[class]{} %% add further index entries

\ccDefinition
  
The class \ccRefName\ is designed as a default traits class for the
class \ccc{Regular_triangulation_3<RegularTriangulationTraits_3,TriangulationDataStructure_3>}. 
It provides \ccc{Weighted_point_3}, a class for weighted points, which derives
from the three dimensional point class \ccc{R::Point_3}.

The first argument \ccc{R} must be a model of the \ccc{Kernel} concept. 

The second argument \ccc{Weight} of the class
\ccc{Regular_triangulation_euclidean_traits_3<R,Weight>} is in fact
optional: if is it not provided, \ccc{R::RT} will be used.

\ccInclude{CGAL/Regular_triangulation_euclidean_traits_3.h}

\ccIsModel
\ccc{RegularTriangulationTraits_3}

\ccInheritsFrom{\ccc{R}}

\ccTypes
\ccThree{typedef R::Point}{Weighted_point_3;}{}

\ccTypedef{typedef R::Point_3 Bare_point;}
{The type for point $p$ of a weighted point ${p}^{(w)}=(p,w_p)$.}
\ccGlue
\ccTypedef{typedef Weighted_point <Bare_point, Weight> Weighted_point_3;}{}

%% \ccExample

%% \ccIncludeExampleCode{examples/Triangulation3/Regular_triangulation_euclidean_traits_3_prog.C}

\ccSeeAlso

\ccc{CGAL::Regular_triangulation_filtered_traits_3}.

\end{ccRefClass}

% +------------------------------------------------------------------------+
%%RefPage: end of main body, begin of footer
% EOF
% +------------------------------------------------------------------------+

