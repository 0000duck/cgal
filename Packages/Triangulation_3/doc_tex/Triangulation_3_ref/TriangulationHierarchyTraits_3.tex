% +------------------------------------------------------------------------+
% | Reference manual page: Triangulation_Traits_3.tex
% +------------------------------------------------------------------------+
% | 27.3.2000   Monique Teillaud
% | Package: Triangulation3
% | 
\RCSdef{\RCSTriangulationTraitsRev}{$Revision$}
\RCSdefDate{\RCSTriangulationTraitsDate}{$Date$}
% |
%%RefPage: end of header, begin of main body
% +------------------------------------------------------------------------+


\begin{ccRefConcept}{TriangulationHierarchyTraits_3}

%% \ccHtmlCrossLink{}     %% add further rules for cross referencing links
%% \ccHtmlIndexC[concept]{} %% add further index entries

To be used by \protect \ccc{Triangulation_hierarchy_3<Tr>}

\ccDefinition

The triangulation hierarchy uses the traits class
\ccc{Tr::Geom_traits} provided by \ccc{Tr}, which must define the
following additional predicate: 

\ccTwo{TriangulationHierarchyTraits_3}{}
\ccNestedType{Compare_distance_3}{Predicate object.
Must provide the operator\\
\ccc{Comparison_result operator()(Point p, Point q, Point r)}
which compares the distance between \ccc{p} and \ccc{q} to the distance
between \ccc{p} and \ccc{r}.}

\ccOperations

The following function gives access to the predicate object:

\ccThree{coplanar_side_of_bounded_circle_3xxx}{gt.compare_x(Point p0, Point p1)x}{}
\ccCreationVariable{traits}  %% choose variable name

\ccMethod{Compare_distance_3 compare_distance_3_object();}{}

\ccHasModels

The cartesian and homegenous kernels provided by \cgal.

\end{ccRefConcept}

