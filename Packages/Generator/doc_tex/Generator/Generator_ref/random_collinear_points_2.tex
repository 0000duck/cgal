\begin{ccRefFunction}{random_collinear_points_2}

\ccDefinition

\ccInclude{CGAL/point_generators_2.h}

\def\ccLongParamLayout{\ccTrue}
\ccFunction{template <class RandomAccessIterator, class OutputIterator>
    OutputIterator random_collinear_points_2( RandomAccessIterator first,
        RandomAccessIterator last, 
        std::size_t n, OutputIterator first2, Random& rnd = default_random,
        Creator creator = Creator_uniform_2<double,P>);}
{ randomly chooses two points from the range $[\ccc{first},\ccc{last})$,
    creates a random third point on the segment connecting these two
    points, writes it to \ccc{first2}, and repeats this $n$ times, thus
    writing $n$ points to \ccc{first2} that are collinear with points
    in the range $[\ccc{first},\ccc{last})$.
    Three random numbers are needed from \ccc{rnd} for each point.
    Returns the value of \ccc{first2} after inserting the $n$ points.
}
\def\ccLongParamLayout{\ccFalse}

\ccRequirements
\begin{itemize}
  \item  \ccc{Creator} must be a function object accepting two
         \ccc{double} values $x$ and $y$ and returning an initialized point
         \ccc{(x,y)} of type \ccc{P}. Predefined implementations for these
         creators like the default can be found in
         Section~\ref{sectionCreatorFunctionObjects}. 
  \item The \ccc{value_type} of the \ccc{RandomAccessIterator} must be 
        assignable to \ccc{P}.  \ccc{P} is equal to the \ccc{value_type} of the
        \ccc{RandomAccessIterator} when using the default initializer.
  \item The expressions \ccc{to_double((*first).x())} and
        \ccc{to_double((*first).y())} must result in the respective
        coordinate values.
\end{itemize}

\ccSeeAlso

\ccRefIdfierPage{CGAL::perturb_points_2} \\
\ccRefIdfierPage{CGAL::points_on_segment_2} \\
\ccRefIdfierPage{CGAL::points_on_square_grid_2} \\
\ccRefIdfierPage{CGAL::random_selection} \\
\ccc{std::random_shuffle} \\

\end{ccRefFunction}
