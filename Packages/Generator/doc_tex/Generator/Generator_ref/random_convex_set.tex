%% ==============================================================
%% Specification: Random Convex Sets
%% --------------------------------------------------------------
%% file  : rcs_spec.awi
%% author: Michael Hoffmann
%% maintainer: Susan Hert 
%% $Id$
%% ==============================================================

\RCSdef{\RandomConvexSetRev}{$Revision$}
\RCSdefDate{\RandomConvexSetDate}{$Date$}

\begin{ccRefFunction}{random_convex_set_2}
\ccIndexMainItemBegin[c]{random convex set}
\ccIndexSubitemBegin[c]{generator}{convex set}

\ccDefinition

The function \ccRefName\ computes a random convex planar
point set of given size where the points are drawn from a specific
domain.

\ccInclude{CGAL/random_convex_set_2.h}

\def\ccLongParamLayout{\ccTrue} 

\ccFunction{template < class OutputIterator, class PointGenerator,
  class Traits > OutputIterator random_convex_set_2( int n,
  OutputIterator o, const PointGenerator& pg, Traits t =
  Default_traits);}
{
computes a random convex \ccc{n}-gon by writing its vertices (oriented
counterclockwise) to \ccc{o}. The resulting polygon is scaled such
that it fits into the bounding box as specified by \ccc{pg}. Therefore
we cannot easily describe the resulting distribution.
\ccPrecond $n \ge 3$.
}

\ccHeading{Requirements}
\ccIndexSubitem[C]{random_convex_set}{preconditions}
\begin{enumerate}
\item \ccc{PointGenerator} is a model of the concept PointGenerator
\item \ccc{Traits} is a model of the concept RandomConvexSetTraits\_2 
\item \ccc{Point_generator::value_type} is equivalent to 
      \ccc{Traits::Point_2} and \ccc{OutputIterator::value_type}.
\item if \ccc{Traits} is not specified,
  \ccc{Point_generator::value_type} must be \ccc{Point_2<
    R >} for some representation class \ccc{R},
\end{enumerate}


The default traits class \ccc{Default_traits} is 
\ccc{Random_convex_set_traits_2}.
\ccIndexTraitsClassDefault{random_convex_set}.

\ccSeeAlso 
\ccRefIdfierPage{CGAL::Random_points_in_square_2<Point_2, Creator>} \\
\ccRefIdfierPage{CGAL::Random_points_in_disc_2<Point_2, Creator>} \\

\ccImplementation 
The implementation uses the centroid method
described in \cite{s-zkm-96} and has a worst case running time of $O(r
\cdot n + n \cdot \log n)$, where $r$ is the time needed by \ccc{pg}
to generate a random point.
\ccIndexMainItemEnd[c]{random convex set}
\ccIndexSubitemEnd[c]{generator}{convex set}

\ccExample

The following program displays a random convex 500-gon where the
points are drawn uniformly from the unit square centered at the
origin.

\ccIncludeExampleCode{Generator/rcs_demo.C}


\end{ccRefFunction}

%% EOF %%
