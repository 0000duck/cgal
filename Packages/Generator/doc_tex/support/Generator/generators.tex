% +------------------------------------------------------------------------+
% | CGAL Reference Manual:  generators.tex
% +------------------------------------------------------------------------+
% | Random sources and geometric object generators.
% |
% | 09.06.1997   Lutz Kettner
% | 
\RCSdef{\generatorsRev}{$Revision$}
\RCSdefDate{\generatorsDate}{$Date$}
% +------------------------------------------------------------------------+

\beforecprogskip\medskipamount
\aftercprogskip\medskipamount
\ccParDims

\chapter{Random Sources and Geometric Object Generators}
\label{chapterGenerators}
\ccChapterSubTitle{\generatorsRev. \ \generatorsDate}\\
\ccChapterAuthor{Lutz Kettner}\\
\ccChapterAuthor{Sven Sch\"onherr}


A variety of generators for random numbers and geometric objects is
provided in \cgal.  They are useful as synthetic test data sets,
e.g.~for testing algorithms on degenerate object sets and for
performance analysis.

The first section describes the random number source used for random
generators. The second section provides useful generic functions
related to random numbers like \ccc{CGAL_random_selection()}. The
third section documents generators for two-dimensional point sets, the
fourth section for three-dimensional point sets. The fifth section
presents examples using functions from
Section~\ref{sectionGenericFunctions} to generate composed objects
like segments.  Note that the \stl\ algorithm \ccc{random_shuffle} is
useful in this context to achieve random permutations for otherwise
regular generators (e.g.~points on a grid or segment).


% +------------------------------------------------------------------------+
% =============================================================================
% The CGAL Reference Manual
% Chapter: Geometric Object Generators
% Section: Random Numbers Generator
% -----------------------------------------------------------------------------
% file  : doc_tex/support/Generator/Random.tex
% author: Sven Sch�nherr <sven@inf.ethz.ch>
% -----------------------------------------------------------------------------
% $CGAL_Chapter: Geometric Object Generators $
% $CGAL_Package: Random_numbers WIP $
% $Id$
% $Date$
% =============================================================================

\begin{ccRefClass}{Random}
\label{sec:random_numbers_generator}

% -----------------------------------------------------------------------------
\ccDefinition

The class \ccRefName is a random numbers generator. It generates
uniformly distributed random \ccc{bool}s, \ccc{int}s and \ccc{double}s. 
It can be used as the random number generating function object in the 
\stl\ algorithm \ccc{random_shuffle}.

Instances of \ccClassName\ can be seen as input streams. Different
streams are \emph{independent} of each other, i.e.\ the sequence of
numbers from one stream does \emph{not} depend upon how many numbers
were extracted from the other streams. At each time, an instance has 
a \emph{state} that uniquely determines the subsequent numbers being
produced.

It can be very useful, e.g.\ for debugging, to reproduce a sequence of
random numbers. This can be done by either initialising with a fixed
seed, or by using the state functions as described below.

\ccInclude{CGAL/Random.h}

% -----------------------------------------------------------------------------
\ccTypes

\ccUnchecked
\ccNestedType{State}{The State type.}

% -----------------------------------------------------------------------------
\ccCreation
\ccCreationVariable{random}

\ccConstructor{ Random( );}{
        introduces a variable \ccVar\ of type \ccClassTemplateName. The
        seed is chosen ``randomly'', depending on the system time.}

\ccConstructor{ Random( unsigned int seed);}{
        introduces a variable \ccVar\ of type \ccClassTemplateName\
        and initializes its internal state using \ccc{seed}. Equal
        values for \ccc{seed} result in equal sequences of random
        numbers.}

% -----------------------------------------------------------------------------
\ccOperations

\ccMemberFunction{ bool get_bool( );}{
        returns a random \ccc{bool}.}

\ccMemberFunction{ template <int b> int get_bits();}{
        returns a random \ccc{int} value from the interval
        $[\mbox{\ccc{0},\ccc{2^b}})$.  This is supposed to
	be efficient.}

\ccMemberFunction{ int get_int( int lower, int upper);}{
        returns a random \ccc{int} from the interval
        $[\mbox{\ccc{lower},\ccc{upper}})$. }

\ccMemberFunction{ double get_double( double lower = 0.0,
                                      double upper = 1.0);}{
        returns a random \ccc{double} from the interval
        $[\mbox{\ccc{lower},\ccc{upper}})$.}


\ccHeading{Distributions}

The following member functions are a 1-to-1 correspondence to
some distributions from the boost random library.


\ccMemberFunction{ template <typename IntType> IntType uniform_smallint( IntType lower=0, IntType upper=9);}{
        returns a random \ccc{IntType} from the interval
        $[\mbox{\ccc{lower},\ccc{upper}}]$. \ccc{IntType} can be an integral type 
        as \ccc{int}, \ccc{std::ptrdiff_t}, \ccc{std::size_t},etc. {\bf Warning: In contrast to \ccc{get_int} this function may return \ccc{upper}. }  }

\ccMemberFunction{ template <typename IntType> IntType uniform_int( IntType lower=0, IntType upper=9);}{
        returns a random \ccc{IntType} from the interval
        $[\mbox{\ccc{lower},\ccc{upper}}]$. \ccc{IntType} can be an integral type 
        as \ccc{int}, \ccc{std::ptrdiff_t}, \ccc{std::size_t},etc.  {\bf Warning: In contrast to \ccc{get_int} this function may return \ccc{upper}. } }

\ccMemberFunction{ template <typename RealType> Realtype uniform_real( RealType lower = 0.0,
                                      RealType upper = 1.0);}{
        returns a random \ccc{RealType} from the interval
        $[\mbox{\ccc{lower},\ccc{upper}})$. \ccc{RealType} can be \ccc{float}, \ccc{double}, etc.}

\ccMemberFunction{ template <typename RealType> RealType uniform_01();}{
        returns a random \ccc{RealType} from the interval
        $[0,1)$. \ccc{RealType} can be \ccc{float}, \ccc{double}, etc.}



\ccMemberFunction{ template <typename IntType> IntType operator() ( IntType upper);}{
        returns \ccVar\ccc{uniform_int<IntType>( 0, upper-1)}.}

% -----------------------------------------------------------------------------
\ccHeading{Seed and State Functions}

\ccMemberFunction{ unsigned int get_seed() const;}{
        returns the seed used for initialization.}

\ccMemberFunction{ void save_state( State& state) const;}{
        saves the current internal state in \ccc{state}.}

\ccMemberFunction{ void restore_state( State const& state);}{
        restores the internal state from \ccc{state}.}

% -----------------------------------------------------------------------------
\ccHeading{Equality Test}

\ccMemberFunction{ bool  operator == ( Random const& random2) const;}{
        returns \ccc{true}, iff \ccVar\ and \ccc{random2} have equal
        internal states.}

% -----------------------------------------------------------------------------
\ccImplementation

We use the boost random library function \ccc{boost::rand48} to generate the random
numbers.



\ccSeeAlso

\ccRefIdfierPage{CGAL::default_random}\\

\end{ccRefClass}

% ===== EOF ===================================================================


% +------------------------------------------------------------------------+
%\newpage
\section{Support Functions for Generators}
\ccThree{OutputIterator}{rand}{}


\subsection{{\it CGAL\_random\_selection()}}
\label{sectionRandomSelection}

\ccc{CGAL_random_selection} chooses $n$ items at random from a random
access iterator range which is useful to produce degenerate input data
sets with multiple entries of identical items.

\ccInclude{CGAL/random_selection.h}

\ccFunction{template <class RandomAccessIterator, class Size, 
                      class OutputIterator, class Random>
    OutputIterator CGAL_random_selection( RandomAccessIterator first,
        RandomAccessIterator last, 
        Size n, OutputIterator result, Random& rnd = CGAL_random);}
{ chooses a random item from the range $[\ccc{first},\ccc{last})$ and
    writes it to \ccc{result}, each item from the range with equal
    probability, and repeats this $n$ times, thus writing $n$ items to
    \ccc{result}.
    A single random number is needed from \ccc{rnd} for each item.
    Returns the value of \ccc{result} after inserting the $n$ items.
    \ccPrecond \ccc{Random} is a random number generator type as provided 
    by the STL or by \ccc{CGAL_Random}.
}



% +------------------------------------------------------------------------+
\newpage
\section{2D Point Generators}

Two kind of point generators are provided: First, random point
generators and second deterministic point generators. Most random
point generators and a few deterministic point generators are provided
as input iterators.  The input iterators model an infinite sequence of
points. The function \ccc{CGAL_copy_n()} could be used to copy a
finite sequence, see Section~\ref{sectionCopyN}. The iterator adaptor
\ccc{CGAL_Counting_iterator} can be used to create finite iterator
ranges, see Section~\ref{sectionCountingIterator}.
Other generators are provided as functions writing to an output
iterator. Further functions add degeneracies or random perturbations.


% +------------------------------------------------------------------------+
\subsection{Point Generators as Input Iterators}

\ccDefinition

Input iterators are provided for random points uniformly distributed
over a two-dimensional domain (square or disc) or a one-dimensional
domain (boundary of a square, circle, or segment). Another input
iterator generates equally spaced points from a segment.

All iterators are parameterized with the point type \ccc{P} and all
with the exception of \ccc{CGAL_Points_on_segment_2} have a second
template argument \ccc{Creator} which defaults to
\ccc{CGAL_Creator_uniform_2<double,P>}\footnote{%
  For compilers not supporting these kind of default arguments, both
  template arguments must be provided when using these generators.}.
The \ccc{Creator} must be a function object accepting two \ccc{double}
values $x$ and $y$ and returning an initialized point \ccc{(x,y)} of type
\ccc{P}. Predifined implementations for these creators like the
default can be found in Section~\ref{sectionCreatorFunctionObjects}.
They simply assume an appropriate constructor for type \ccc{P}.

All generators know a range within which the coordinates of the
generated points will lie.

\ccInclude{CGAL/point_generators_2.h}

\ccTypes

The generators comply to the requirements of input iterators which
includes local type declarations including \ccc{value_type} which
denotes \ccc{P} here.

\ccCreation


\ccHtmlNoClassFile
\begin{ccClassTemplate}{CGAL_Random_points_in_disc_2<P,Creator>}
\ccCreationVariable{g}
\ccConstructor{CGAL_Random_points_in_disc_2( double r, CGAL_Random& rnd =
  CGAL_random);}{%
  $g$ is an input iterator creating points of type \ccc{P} uniformly
  distributed in the open disc with radius $r$,
  i.e.~$|\ccc{*g}| < r$~. Two random numbers are needed from
  \ccc{rnd} for each point.
} 
\end{ccClassTemplate}

\ccHtmlNoClassFile
\begin{ccClassTemplate}{CGAL_Random_points_on_circle_2<P,Creator>}
\ccCreationVariable{g}
\ccConstructor{CGAL_Random_points_on_circle_2( double r, CGAL_Random& rnd =
  CGAL_random);}{%
  $g$ is an input iterator creating points of type \ccc{P} uniformly
  distributed on the circle with radius $r$,
  i.e.~$|\ccc{*g}| == r$~. A single random number is needed from
  \ccc{rnd} for each point.
} 
\end{ccClassTemplate}

\ccHtmlNoClassFile
\begin{ccClassTemplate}{CGAL_Random_points_in_square_2<P,Creator>}
\ccCreationVariable{g}
\ccConstructor{CGAL_Random_points_in_square_2( double a, CGAL_Random& rnd =
  CGAL_random);}{%
  $g$ is an input iterator creating points of type \ccc{P} uniformly
  distributed in the half-open square with side length $2 a$, centered
  at the origin, i.e.~$\forall p = \ccc{*g}:  -a \le p.x() < a$ and 
  $-a \le p.y() < a$~. 
  Two random numbers are needed from \ccc{rnd} for each point.
} 
\end{ccClassTemplate}

\ccHtmlNoClassFile
\begin{ccClassTemplate}{CGAL_Random_points_on_square_2<P,Creator>}
\ccCreationVariable{g}
\ccConstructor{CGAL_Random_points_on_square_2( double a, CGAL_Random& rnd =
  CGAL_random);}{%
  $g$ is an input iterator creating points of type \ccc{P} uniformly
  distributed on the boundary of the square with side length $2 a$,
  centered at the origin, i.e.~$\forall p = \ccc{*g}:$ one
  coordinate is either $a$ or $-a$ and for the 
  other coordinate $c$ holds $-a \le c < a$~.
  A single random number is needed from \ccc{rnd} for each point.
} 
\end{ccClassTemplate}

\ccHtmlNoClassFile
\begin{ccClassTemplate}{CGAL_Random_points_on_segment_2<P,Creator>}
\ccCreationVariable{g}
\ccConstructor{CGAL_Random_points_on_segment_2( const P& p, const P& q,
  CGAL_Random& rnd = CGAL_random);}{%
  $g$ is an input iterator creating points of type \ccc{P} uniformly
  distributed on the segment from $p$ to $q$ (excluding $q$),
  i.e.~$\ccc{*g} == (1-\lambda)\, p + \lambda q$ where $0 \le \lambda < 1$~.
  A single random number is needed from \ccc{rnd} for each point.
  \ccPrecond The expressions \ccc{CGAL_to_double(p.x())} and
    \ccc{CGAL_to_double(p.y())} must  result in the respective
    \ccc{double} representation of the coordinates and similar for $q$.}

\end{ccClassTemplate}

\ccHtmlNoClassFile
\begin{ccClassTemplate}{CGAL_Points_on_segment_2<P>}
\ccCreationVariable{g}
\ccConstructor{CGAL_Points_on_segment_2( const P& p, const P& q, 
                                         size_t n, size_t i = 0);}{%
  $g$ is an input iterator creating points of type \ccc{P} equally 
  spaced on the segment from $p$ to $q$. $n$ points are placed on the
  segment including $p$ and $q$. The iterator denoted the point $i$
  where $p$ has the index 0 and $q$ the index $n$.
  \ccPrecond The expressions \ccc{CGAL_to_double(p.x())} and
    \ccc{CGAL_to_double(p.y())} must  result in the respective
    \ccc{double} representation of the coordinates and similar for $q$.}


\ccOperations
\ccThree{double}{g.source();}{}

\ccMethod{double range();}{returns the range in which the point
  coordinates lie, i.e.~$\forall x: |x| \leq $\ccc{range()} and
  $\forall y: |y| \leq $\ccc{range()}.}

The generators \ccc{CGAL_Random_points_on_segment_2} and
\ccc{CGAL_Points_on_segment_2} have to additional methods.

\ccMethod{const P& source();}{returns the source point of the segment.}
\ccGlue
\ccMethod{const P& target();}{returns the target point of the segment.}

\end{ccClassTemplate}


% +------------------------------------------------------------------------+
\subsection{Point Generators as Functions}

\ccHeading{Grid Points}
\ccThree{OutputIterator}{rand}{}

Grid points are generated by functions writing to an output iterator.

\def\ccLongParamLayout{\ccTrue}
\ccFunction{template <class OutputIterator, Creator creator>
    OutputIterator
    CGAL_points_on_square_grid_2( double a, size_t n, OutputIterator o,
                                  const P*, Creator creator = 
                                  CGAL_Creator_uniform_2<double,P>);}
{ creates the $n$ first points on the regular $\lceil\sqrt{n}\,\rceil
    \times \lceil  \sqrt{n}\,\rceil$ grid within the square
    $[-a,a]\times [-a,a]$. Returns the value of $o$ after inserting
    the $n$ points. 
    \ccPrecond \ccc{Creator} must be a function object accepting two
    \ccc{double} values $x$ and $y$ and returning an initialized point
    \ccc{(x,y)} of type \ccc{P}. Predifined implementations for these
    creators like the default can be found in
    Section~\ref{sectionCreatorFunctionObjects}. The
    \ccc{OutputIterator} must accept values of type \ccc{P}. If the
    \ccc{OutputIterator} has a \ccc{value_type} the default
    initializer of the \ccc{creator} can be used. \ccc{P} is set to
    the \ccc{value_type} in this case.}
\def\ccLongParamLayout{\ccFalse}


\ccFunction{template <class P, class OutputIterator>
    OutputIterator CGAL_points_on_segment_2( const P& p, const P& q, size_t n,
    OutputIterator o);}
{ creates $n$ points equally spaced on the segment from $p$ to $q$,
    i.e.~$\forall i: 0 \le i < n: o[i] := \frac{n-i-1}{n-1}\, p +
    \frac{i}{n-1}\, q$. Returns the value of $o$ after inserting
    the $n$ points.}

\ccHeading{Random Perturbations}

Degenerate input sets like grid points can be randomly perturbed by a
small amount to produce {\em quasi}-degenerate test sets. This
challenges numerical stability of algorithms using inexact arithmetic and
exact predicates to compute the sign of expressions slightly off from zero.

\ccFunction{template <class ForwardIterator>
    void CGAL_perturb_points_2( ForwardIterator first, ForwardIterator last, 
        double xeps, double yeps = xeps, CGAL_Random& rnd = CGAL_random,
        Creator creator = CGAL_Creator_uniform_2<double,P>);}
{ perturbs the points in the range $[\ccc{first},\ccc{last})$ by
  replacing each point with a random point from the rectangle
  \ccc{xeps} $\times$ \ccc{yeps} centered at the original point.
  Two random numbers are needed from \ccc{rnd} for each point.
  \ccPrecond   \ccc{Creator} must be a function object accepting two
    \ccc{double} values $x$ and $y$ and returning an initialized point
    \ccc{(x,y)} of type \ccc{P}. Predifined implementations for these
    creators like the default can be found in
    Section~\ref{sectionCreatorFunctionObjects}. The \ccc{value_type} of the
    \ccc{ForwardIterator} must be assignable to \ccc{P}.
    \ccc{P} is equal to the \ccc{value_type} of the
    \ccc{ForwardIterator} when using the default initializer.
    The expressions \ccc{CGAL_to_double((*first).x())} and
    \ccc{CGAL_to_double((*first).y())} must result in the respective
    coordinate values.
}

\ccHeading{Adding Degeneracies}

For a given point set certain kinds of degeneracies can be produced
adding new points. The \ccc{CGAL_random_selection()} function is
useful to generate multiple entries of identical points, see
Section~\ref{sectionRandomSelection}. The
\ccc{CGAL_random_collinear_points_2()} function adds collinearities to
a point set.


\def\ccLongParamLayout{\ccTrue}
\ccFunction{template <class RandomAccessIterator, class OutputIterator>
    OutputIterator CGAL_random_collinear_points_2( RandomAccessIterator first,
        RandomAccessIterator last, 
        size_t n, OutputIterator first2, CGAL_Random& rnd = CGAL_random,
        Creator creator = CGAL_Creator_uniform_2<double,P>);}
{ randomly chooses two points from the range $[\ccc{first},\ccc{last})$,
    creates a random third point on the segment connecting this two
    points, writes it to \ccc{first2}, and repeats this $n$ times, thus
    writing $n$ points to \ccc{first2} that are collinear with points
    in the range $[\ccc{first},\ccc{last})$.
    Three random numbers are needed from \ccc{rnd} for each point.
    Returns the value of \ccc{first2} after inserting the $n$ points.
  \ccPrecond  \ccc{Creator} must be a function object accepting two
    \ccc{double} values $x$ and $y$ and returning an initialized point
    \ccc{(x,y)} of type \ccc{P}. Predifined implementations for these
    creators like the default can be found in
    Section~\ref{sectionCreatorFunctionObjects}. The \ccc{value_type} of the
    \ccc{RandomAccessIterator} must be assignable to \ccc{P}.
    \ccc{P} is equal to the \ccc{value_type} of the
    \ccc{RandomAccessIterator} when using the default initializer.
    The expressions \ccc{CGAL_to_double((*first).x())} and
    \ccc{CGAL_to_double((*first).y())} must result in the respective
    coordinate values.
}
\def\ccLongParamLayout{\ccFalse}


\ccExample

We want to generate a test set of 1000 points, where 60\% are chosen
randomly in a small disc, 20\% are from a larger grid, 10\% duplicates
are added, and 10\% collinearities added. A random shuffle removes the
construction order from the test set. See \ccTexHtml{%
Figure~\ref{figurePointGenerator}}{Figure <A HREF="#PointGenerators">
  <IMG SRC="cc_ref_up_arrow.gif" ALT="reference arrow" WIDTH="10"
  HEIGHT="10"></A>} for the example output.

\cprogfile{generators_prog1.C}

\begin{ccTexOnly}
  \begin{figure}
    \noindent
    \hspace*{0.025\textwidth}%
    \begin{minipage}{0.45\textwidth}%
      \epsfig{figure=generators_prog1.ps,width=\textwidth}
      \caption{Output of example program for point generators.}
      \label{figurePointGenerator}
    \end{minipage}%
    \hspace*{0.05\textwidth}%
    \begin{minipage}{0.45\textwidth}%
      \epsfig{figure=generators_prog2.ps,width=\textwidth}%      
      \caption{Output of example program for point generators working
        on integer points.}
      \label{figureIntegerPointGenerator}
    \end{minipage}%
  \end{figure}
\end{ccTexOnly}

\begin{ccHtmlOnly}
  <A NAME="PointGenerators">
  <TABLE><TR><TD ALIGN=LEFT VALIGN=TOP WIDTH=60%>
    <A HREF="./generators_prog1.gif">Figure:</A>
    Output of example program for point generators.
  </TD><TD ALIGN=LEFT VALIGN=TOP WIDTH=5% NOWRAP>
  </TD><TD ALIGN=LEFT VALIGN=TOP WIDTH=35% NOWRAP>
    <A HREF="./generators_prog1.gif">
        <img src="./generators_prog1_small.gif" 
             alt="Point Generator Example Output"></A>
  </TD></TR></TABLE>
\end{ccHtmlOnly}


The second example demonstrates the point generators with integer
points. Arithmetic with \ccc{double}'s is sufficient to produce
regular integer grids. See \ccTexHtml{%
Figure~\ref{figureIntegerPointGenerator}}{Figure 
  <A HREF="#IntegerPointGenerators">
  <IMG SRC="cc_ref_up_arrow.gif" ALT="reference arrow" WIDTH="10"
  HEIGHT="10"></A>}
for the example output.

\cprogfile{generators_prog2.C}

\begin{ccHtmlOnly}
  <A NAME="IntegerPointGenerators">
  <TABLE><TR><TD ALIGN=LEFT VALIGN=TOP WIDTH=60%>
    <A HREF="./generators_prog2.gif">Figure:</A>
        Output of example program for point generators working
        on integer points.
  </TD><TD ALIGN=LEFT VALIGN=TOP WIDTH=5% NOWRAP>
  </TD><TD ALIGN=LEFT VALIGN=TOP WIDTH=35% NOWRAP>
    <A HREF="./generators_prog2.gif">
        <img src="./generators_prog2_small.gif" 
             alt="Integer Point Generator Example Output"></A>
  </TD></TR></TABLE>
\end{ccHtmlOnly}


% +------------------------------------------------------------------------+
\newpage
\section{3D Point Generators}

One kind of point generators is currently provided: Random point
generators implemented as input iterators.  The input iterators model
an infinite sequence of points. The function \ccc{CGAL_copy_n()} could
be used to copy a finite sequence, see Section~\ref{sectionCopyN}. The
iterator adaptor \ccc{CGAL_Counting_iterator} can be used to create
finite iterator ranges, see Section~\ref{sectionCountingIterator}.


% +------------------------------------------------------------------------+
\subsection{Point Generators as Input Iterators}

\ccDefinition

Input iterators are provided for random points uniformly distributed
in a three-dimensional volume (sphere or cube) or a two-dimensional
surface (boundary of a sphere).

All iterators are parameterized with the point type \ccc{P} and a second
template argument \ccc{Creator} which defaults to
\ccc{CGAL_Creator_uniform_3<double,P>}\footnote{%
  For compilers not supporting these kind of default arguments, both
  template arguments must be provided when using these generators.}.
The \ccc{Creator} must be a function object accepting three
\ccc{double} values $x$, $y$ and $z$ and returning an initialized
point \ccc{(x,y,z)} of type \ccc{P}. Predifined implementations for
these creators like the default can be found in
Section~\ref{sectionCreatorFunctionObjects}.  They simply assume an
appropriate constructor for type \ccc{P}.

All generators know a range within which the coordinates of the
generated points will lie.

\ccInclude{CGAL/point_generators_3.h}

\ccTypes

The generators comply to the requirements of input iterators which
includes local type declarations including \ccc{value_type} which
denotes \ccc{P} here.

\ccCreation


\ccHtmlNoClassFile
\begin{ccClassTemplate}{CGAL_Random_points_in_sphere_3<P,Creator>}
\ccCreationVariable{g}
\ccConstructor{CGAL_Random_points_in_sphere_3( double r, CGAL_Random& rnd =
  CGAL_random);}{%
  $g$ is an input iterator creating points of type \ccc{P} uniformly
  distributed in the open sphere with radius $r$,
  i.e.~$|\ccc{*g}| < r$~. 
} 
\end{ccClassTemplate}

\ccHtmlNoClassFile
\begin{ccClassTemplate}{CGAL_Random_points_on_sphere_3<P,Creator>}
\ccCreationVariable{g}
\ccConstructor{CGAL_Random_points_on_sphere_3( double r, CGAL_Random& rnd =
  CGAL_random);}{%
  $g$ is an input iterator creating points of type \ccc{P} uniformly
  distributed on the boundary of a sphere with radius $r$,
  i.e.~$|\ccc{*g}| == r$~. Two random numbers are needed from
  \ccc{rnd} for each point.
} 
\end{ccClassTemplate}

\ccHtmlNoClassFile
\begin{ccClassTemplate}{CGAL_Random_points_in_cube_3<P,Creator>}
\ccCreationVariable{g}
\ccConstructor{CGAL_Random_points_in_cube_3( double a, CGAL_Random& rnd =
  CGAL_random);}{%
  $g$ is an input iterator creating points of type \ccc{P} uniformly
  distributed in the half-open cube with side length $2 a$, centered
  at the origin, i.e.~$\forall p = \ccc{*g}:  -a \le p.x(),p.y(),p.z() < a$~. 
  Three random numbers are needed from \ccc{rnd} for each point.
} 
\end{ccClassTemplate}



% +------------------------------------------------------------------------+
\newpage
\section{Examples Generating Segments}

The following two examples illustrate the use of the generic functions
from Section~\ref{sectionGenericFunctions} like
\ccc{CGAL_Join_input_iterator} to generate composed objects from other
generators -- here two-dimensional segments from two point generators.

We want to generate a test set of 200 segments, where one endpoint is
chosen randomly from a horizontal segment of length 200, and the other
endpoint is chosen randomly from a circle of radius 250. See
\ccTexHtml{Figure~\ref{figureSegmentGenerator}}{Figure <A
  HREF="#SegmentGenerator"> <IMG SRC="cc_ref_up_arrow.gif"
  ALT="reference arrow" WIDTH="10" HEIGHT="10"></A>} for the example
output.

\begin{ccTexOnly}
  \begin{figure}
    \noindent
    \hspace*{0.025\textwidth}%
    \begin{minipage}[t]{0.45\textwidth}%
      \epsfig{figure=Segment_generator_prog1.ps,width=\textwidth}
      \caption{Output of the first example program for the generic generator.}
      \label{figureSegmentGenerator}
    \end{minipage}%
    \hspace*{0.05\textwidth}%
    \begin{minipage}[t]{0.45\textwidth}%
      \epsfig{figure=Segment_generator_prog2.ps,width=\textwidth}
      \caption{Output of the second example program for the generic
        generator without using intermediate storage.}
      \label{figureSegmentGeneratorFan}
    \end{minipage}%
  \end{figure}
\end{ccTexOnly}

\cprogfile{Segment_generator_prog1.C}

\begin{ccHtmlOnly}
  <A NAME="SegmentGenerator">
  <TABLE><TR><TD ALIGN=LEFT VALIGN=TOP WIDTH=60%>
    <A HREF="./Segment_generator_prog1.gif">Figure:</A>
    Output of example program for the generic segment generator.
  </TD><TD ALIGN=LEFT VALIGN=TOP WIDTH=5% NOWRAP>
  </TD><TD ALIGN=LEFT VALIGN=TOP WIDTH=35% NOWRAP>
    <A HREF="./Segment_generator_prog1.gif">
        <img src="./Segment_generator_prog1_small.gif" 
             alt="Segment Generator Example Output"></A>
  </TD></TR></TABLE>
\end{ccHtmlOnly}

The second example generates a regular structure of 100 segments, see 
\ccTexHtml{Figure~\ref{figureSegmentGeneratorFan}}{Figure <A
  HREF="#SegmentGeneratorFan"> <IMG SRC="cc_ref_up_arrow.gif"
  ALT="reference arrow" WIDTH="10" HEIGHT="10"></A>} for the example
output. It uses the \ccc{CGAL_Points_on_segment_2} iterator,
\ccc{CGAL_Join_input_iterator_2} and \ccc{CGAL_Counting_iterator} to
avoid any intermediate storage of the generated objects until they are
used, in this example copied to a windowstream.

\cprogfile{Segment_generator_prog2.C}

\begin{ccHtmlOnly}
  <A NAME="SegmentGeneratorFan">
  <TABLE><TR><TD ALIGN=LEFT VALIGN=TOP WIDTH=60%>
    <A HREF="./Segment_generator_prog2.gif">Figure:</A>
    Output of example program for the generic segment generator using
    precomputed point locations.
  </TD><TD ALIGN=LEFT VALIGN=TOP WIDTH=5% NOWRAP>
  </TD><TD ALIGN=LEFT VALIGN=TOP WIDTH=35% NOWRAP>
    <A HREF="./Segment_generator_prog2.gif">
        <img src="./Segment_generator_prog2_small.gif" 
             alt="Segment Generator Example Output 2"></A>
  </TD></TR></TABLE>
\end{ccHtmlOnly}

% +--------------------------------------------------------+
\beforecprogskip\parskip
\aftercprogskip0pt

% EOF
