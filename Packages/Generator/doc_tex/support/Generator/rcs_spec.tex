%% ==============================================================
%% Specification: Random Convex Sets
%% --------------------------------------------------------------
%% file  : rcs_spec.awi
%% author: Michael Hoffmann
%% maintainer: Susan Hert 
%% $Id$
%% ==============================================================

\RCSdef{\RandomConvexSetRev}{$Revision$}
\RCSdefDate{\RandomConvexSetDate}{$Date$}

\newpage

\ccParDims

\section{Building Random Convex Sets}
\label{sec:building_random_convex_sets}
\ccIndexMainItemBegin[c]{random convex set}
\ccIndexSubitemBegin[c]{generator}{convex set}

This section describes a function to compute a random convex planar
point set of given size where the points are drawn from a specific
domain.

\ccInclude{CGAL/random_convex_set_2.h}

\def\ccLongParamLayout{\ccTrue} 

\ccFunction{template < class OutputIterator, class Point_generator,
  class Traits > OutputIterator random_convex_set_2( int n,
  OutputIterator o, const Point_generator& pg, Traits t =
  Default_traits);}
{
computes a random convex \ccc{n}-gon by writing its vertices (oriented
counterclockwise) to \ccc{o}. The resulting polygon is scaled such
that it fits into the bounding box as specified by \ccc{pg}. Therefore
we cannot easily describe the resulting distribution.
}

\ccHeading{Precondition}
\ccIndexSubitem[C]{random_convex_set}{preconditions}
\begin{enumerate}
\item \ccc{Point_generator} satisfies the requirements stated in
  section \ref{point_generator_req},
\item If \ccc{Traits} is specified, it has to satisfy the
  requirements stated in section \ref{req_random_convex_sets_traits}
  and \ccc{Traits::Point_2} must be the same as
  \ccc{Point_generator::value_type},
\item if \ccc{Traits} is not specified,
  \ccc{Point_generator::value_type} must be \ccc{Point_2<
    R >} for some representation class \ccc{R},
\item \ccc{OutputIterator} accepts
  \ccc{Point_generator::value_type} as value type {\it and}
\item $n \ge 3$.
\end{enumerate}

%The default traits class \ccc{Default_traits} is 
%\ccc{Random_convex_set_traits}.
%\ccIndexTraitsClassDefault{random_convex_set}.

\ccSeeAlso \ccc{Random_points_in_square_2} and
\ccc{Random_points_in_disc_2}.

\ccImplementation The implementation uses the centroid method
described in \cite{s-zkm-96} and has a worst case running time of $O(r
\cdot n + n \cdot \log n)$, where $r$ is the time needed by \ccc{pg}
to generate a random point.
\ccIndexMainItemEnd[c]{random convex set}
\ccIndexSubitemEnd[c]{generator}{convex set}

\ccExample

The following program displays a random convex 500-gon where the
points are drawn uniformly from the unit square centered at the
origin.

\ccIncludeExampleCode{Generator/rcs_manual_demo.C}

\ccTagDefaults

\begin{ccClass}{Point_generator}
    \ccCreationVariable{pg} \ccTagFullDeclarations
    
    \subsection{Requirements for Point Generator
      Classes}\label{point_generator_req}
    \ccModifierCrossRefOff
    \ccIndexMainItem{generator classes, requirements}
    \ccModifierCrossRefOn
    
    \ccDefinition A class \ccClassName\ satisfying input iterator
    requirements has to provide the following additional types and
    operations in order to qualify as a point generator class. The
    point generators described in section \ref{sec:point_generators_2}
    fulfill these requirements.

    \ccTypes\ccIndexClassTypes
    
    \ccNestedType{value_type}{point class.}  
    
    \ccNestedType{FT}{class used for doing computations on point
      coordinates (has to fulfill field type requirements).}

    \ccOperations
    \begin{ccIndexMemberFunctions}
    
    \ccMemberFunction{FT range() const;}{return an absolute bound for
      the coordinates of all generated points.}
    \end{ccIndexMemberFunctions}
    
\end{ccClass}

\newpage
\begin{ccAdvanced}
  \ccAutoIndexingOff
  \ccHtmlNoIndex\ccHtmlNoClassLinks\begin{ccClass}{Random_Convex_Set_Traits}
    \ccCreationVariable{t}
    \ccTagFullDeclarations
    
    \subsection{Requirements for Random Convex Sets Traits
      Classes}\label{req_random_convex_sets_traits}%
      \ccIndexSubitem[C]{random_convex_set}{traits requirements}
    
    \ccDefinition A class \ccClassName\ has to provide the following
    types and operations in order to qualify as a traits class for
    \ccc{random_convex_set_2}.
    
    \ccTypes 
    
    \ccNestedType{Point_2}{point class.}
    \ccNestedType{FT}{class used for doing computations on point and
      vector coordinates (has to fulfill field type requirements).}
    
    \ccNestedType{Sum}{AdaptableBinaryFunction class:
      \ccc{Point_2} $\times$ \ccc{Point_2} $\rightarrow$
      \ccc{Point_2}. It returns the point that results from adding
      the vectors corresponding to both arguments.}
    
    \ccNestedType{Scale}{AdaptableBinaryFunction class:
      \ccc{Point_2} $\times$ \ccc{FT} $\rightarrow$
      \ccc{Point_2}. \ccc{Scale(p,k)} returns the point that
      results from scaling the vector corresponding to \ccc{p} by a
      factor of \ccc{k}.}
    
    \ccNestedType{Max_coordinate}{AdaptableUnaryFunction class:
      \ccc{Point_2} $\rightarrow$ \ccc{FT}. \ccc{Max_coordinate(p)}
      returns the coordinate of \ccc{p} with largest absolute value.}

    \ccNestedType{Angle_less}{AdaptableBinaryFunction class:
      \ccc{Point_2} $\times$ \ccc{Point_2} $\rightarrow$
      \ccc{bool}. It returns \ccc{true}, iff the angle of the
      direction corresponding to the first argument with respect to
      the positive $x$-axis is less than the angle of the direction
      corresponding to the second argument.}

    \ccOperations
    
    \ccMemberFunction{Point_2 origin() const;}{return origin (neutral
      element for the \ccc{Sum} operation).}
      
  \end{ccClass}
  \ccAutoIndexingOn
\end{ccAdvanced}

%% EOF %%
