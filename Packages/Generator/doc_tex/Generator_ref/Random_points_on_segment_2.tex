\begin{ccRefClass}{Random_points_on_segment_2<Point_2, Creator>}

\ccDefinition

The class \ccRefName\ is an input iterator creating points uniformly
distributed on a segment.

\ccIsModel
InputIterator \\
\ccRefConceptPage{PointGenerator}

\ccInclude{CGAL/point_generators_2.h}

\ccTypes

\ccThree{typedef std::input_iterator_tagxxx}{xxxiterator_category;}{}
\ccTypedef{typedef std::input_iterator_tag iterator_category;}{}
\ccGlue
\ccTypedef{typedef Point_2                 value_type;}{}
\ccGlue
\ccTypedef{typedef std::ptrdiff_t          difference_type;}{}
\ccGlue
\ccTypedef{typedef const Point_2*          pointer;}{}
\ccGlue
\ccTypedef{typedef const Point_2&          reference;}{}


\ccOperations

\ccCreationVariable{g}
\ccConstructor{Random_points_on_segment_2( const Point_2& p, const Point_2& q,
  Random& rnd = default_random);}{%
  $g$ is an input iterator creating points of type \ccc{Point_2} uniformly
  distributed on the segment from $p$ to $q$ (excluding $q$),
  i.e.~$\ccc{*g} == (1-\lambda)\, p + \lambda q$ where $0 \le \lambda < 1$~.
  A single random number is needed from \ccc{rnd} for each point.
  \ccRequire The expressions \ccc{to_double(p.x())} and
    \ccc{to_double(p.y())} must  result in the respective
    \ccc{double} representation of the coordinates of $p$,
    and similarly for $q$.}

\ccSeeAlso

\ccRefIdfierPage{CGAL::copy_n} \\
\ccRefIdfierPage{CGAL::Counting_iterator} \\
\ccRefIdfierPage{CGAL::Points_on_segment_2<Point_2>} \\
\ccRefIdfierPage{CGAL::Random_points_in_disc_2<Point_2, Creator>} \\
\ccRefIdfierPage{CGAL::Random_points_in_square_2<Point_2, Creator>} \\
\ccRefIdfierPage{CGAL::Random_points_on_circle_2<Point_2, Creator>} \\
\ccRefIdfierPage{CGAL::Random_points_on_square_2<Point_2, Creator>} \\
\ccc{std::random_shuffle} \\

\end{ccRefClass}
