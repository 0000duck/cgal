\RCSdef{\RCSPolygonwithholesoperatorostreamRev}{$Revision$}
\RCSdefDate{\RCSPolygonwithholesoperatorostreamDate}{$Date$}

\ccHtmlNoClassLinks
\begin{ccRefFunction}{operator>>}
\label{refPolygon_with_holes_operator_rightshift}

\ccDefinition

This operator imports a general polygon or a general polygon with
holes from the input stream \ccc{in}. Only the point coordinates of
the boundary are used. If the stream \ccc{in} does not contain a
permissible general polygon or a general polygon with holes, the
\ccc{ios::badbit} of the input stream \ccc{in} is set and $P$ remains
unchanged.

An ASCII and a binary format exist. The stream detects the
format automatically and can read both.

\ccInclude{CGAL/General_polygon_2.h}

\ccGlobalFunction{template <class ArrTraits>
    istream& operator>>(istream& in,
                        CGAL::General_polygon_2<ArrTraits>& P);}

\ccInclude{CGAL/General_polygon_with_holes_2.h}

\ccGlobalFunction{template <class Polygon>
  istream& operator>>(istream& in,
                      CGAL::General_polygon_with_holes_2<Polygon>& P);}

\ccSeeAlso

\ccRefIdfierPage{CGAL::Polygon_2<Kernel,Container>}\\ 
\ccRefIdfierPage{CGAL::General_polygon_2<ArrTraits>}\\ 
\ccRefIdfierPage{CGAL::General_polygon_with_holes_2<Polygon>}\\ 
  \lcTex{\ccc{operator<<} \dotfill\ page~\pageref{refPolygon_with_holes_operator_leftshift}}%
  \lcRawHtml{<I><A HREF="Function_operator.html">operator&lt;&lt;</A></I>}

% \ccImplementation

\end{ccRefFunction}

