\ccRefPageBegin
\label{ref_bso_join}

\begin{ccRefFunction}{join}

\ccThree{OutputIterator}{join}{}
\ccThreeToTwo

\ccDefinition

\ccInclude{CGAL/IO/Boolean_set_operations.h}

\ccGlobalFunction{
bool join(const Type1 & p1, const Type2 & p2,
	  General_polygon_with_holes_2 & p);}
{Each one of these functions computes the union of two given polygons
\ccc{p1} and \ccc{p2}. If the two given polygons overlap, it returns
\ccc{true}, and places the resulting polygon in \ccc{p}. Otherwise, it
returns \ccc{false}.}

\begin{tabular}{|l|l|}
\hline
\textbf{Arg 1 Type} & \textbf{Arg 2 Type}\\
\hline
\hline
\ccc{Polygon_2} & 
\ccc{Polygon_2}\\
\hline
\ccc{Polygon_2} & 
\ccc{Polygon_with_holes_2}\\
\hline
\ccc{Polygon_with_holes_2} & 
\ccc{Polygon_2}\\
\hline
\ccc{General_polygon_2} & 
\ccc{General_polygon_2}\\
\hline
\ccc{General_polygon_2} & 
\ccc{General_polygon_with_holes_2}\\
\hline
\ccc{General_polygon_with_holes_2} & 
\ccc{General_polygon_2}\\
\hline
\ccc{General_polygon_with_holes_2} & 
\ccc{General_polygon_with_holes_2}\\
\hline
\end{tabular}

\ccGlobalFunction{template <class Kernel, class Container>
bool join(const Polygon_2<Kernel, Container> & p1,
          const Polygon_2<Kernel, Container> & p2,
	  General_polygon_with_holes_2<Polygon_2<Kernel, Container> > & p);}
\ccGlue
\ccGlobalFunction{template <class Kernel, class Container>
bool join(const Polygon_2<Kernel, Container> & p1,
          const General_polygon_with_holes_2<Polygon_2<Kernel,Container> > & p2,
	  General_polygon_with_holes_2<Polygon_2<Kernel, Container> > & p);}
\ccGlue
\ccGlobalFunction{template <class Kernel, class Container>
bool join(const General_polygon_with_holes_2<Polygon_2<Kernel,Container> > & p2,
          const Polygon_2<Kernel, Container> & p1,
	  General_polygon_with_holes_2<Polygon_2<Kernel, Container> > & p);}
\ccGlue
\ccGlobalFunction{template <class Traits>
bool join(const General_polygon_2<Traits> & p1,
          const General_polygon_2<Traits> & p2,
	  General_polygon_with_holes_2<General_polygon_2<Traits> > & p);}
\ccGlue
\ccGlobalFunction{template <class Traits>
bool join(const General_polygon_2<Traits> & p1,
          const General_polygon_with_holes_2<General_polygon_2<Traits> > & p2,
	  General_polygon_with_holes_2<General_polygon_2<Traits> > & p);}
\ccGlue
\ccGlobalFunction{template <class Traits>
bool join(const General_polygon_with_holes_2<General_polygon_2<Traits> > & p2,
          const General_polygon_2<Traits> & p1,
	  General_polygon_with_holes_2<General_polygon_2<Traits> > & p);}
\ccGlue
\ccGlobalFunction{template <class Polygon>
bool join(const General_polygon_with_holes_2<Polygon> & p1,
          const General_polygon_with_holes_2<Polygon> & p2,
	  Traits::Polygon_with_holes_2 & p);}

\ccGlobalFunction{template <class InputIterator, class OutputIterator>
OutputIterator join(InputIterator & begin, InputIterator & end,
                    OutputIterator oi);}
{Computes the union of the general polygons (or general polygons with
holes) in the given range. (The value type of the input iterator is
used to distinguish between the two.) The result, represented by a set
of general poygon with holes, is inserted into an output container
through a given output iterator \ccc{oi}. The output iterator is
returned. The value type of the \ccc{OutputIterator} is
\ccc{General_polygon_with_holes_2}.}

\ccGlobalFunction{template <class InputIterator1, class InputIterator2,
		  class OutputIterator>
OutputIterator join(InputIterator1 & pgn_begin1, InputIterator1 & pgn_end1,
                    InputIterator2 & pgn_begin2, InputIterator2 & pgn_end2,
		    OutputIterator oi);}
{Computes the union of the general polygons and general polygons
with holes in the given two ranges. The result, represented by a set
of general poygon with holes, is inserted into an output container
through a given output iterator \ccc{oi}. The output iterator is
returned. The value type of the \ccc{OutputIterator} is
\ccc{General_polygon_with_holes_2}.}

\ccSeeAlso
\ccRefIdfierPage{CGAL::do_intersect}\\
\ccRefIdfierPage{CGAL::intersection}\\
\ccRefIdfierPage{CGAL::difference}\\
\ccRefIdfierPage{CGAL::symmetric_difference}

\end{ccRefFunction}

\ccRefPageEnd
