\ccRefPageBegin
\label{ref_bso_join}

\begin{ccRefFunction}{join}

\ccThree{OutputIterator}{join}{}
\ccThreeToTwo

\ccDefinition

\ccInclude{CGAL/Boolean_set_operations_2.h}

\ccGlobalFunction{
bool join(const Type1 & p1, const Type2 & p2,
	  General_polygon_with_holes_2 & p);}
{Each one of these functions computes the union of two given polygons
\ccc{p1} and \ccc{p2}. If the two given polygons overlap, it returns
\ccc{true}, and places the resulting polygon in \ccc{p}. Otherwise, it
returns \ccc{false}.}

\begin{ccTexOnly}
\begin{longtable}[c]{|l|l|}
\multicolumn{2}{l}{\sl \ \ }
\endfirsthead
\multicolumn{2}{l}{\sl continued}
\endhead
\hline
\textbf{Arg 1 Type} & \textbf{Arg 2 Type}\\
\hline
\hline
\ccc{Polygon_2} & \ccc{Polygon_2}\\
\hline
\ccc{Polygon_2} & \ccc{Polygon_with_holes_2}\\
\hline
\ccc{Polygon_with_holes_2} & \ccc{Polygon_2}\\
\hline
\ccc{Polygon_with_holes_2} & \ccc{Polygon_with_holes_2}\\
\hline
\ccc{General_polygon_2} & \ccc{General_polygon_2}\\
\hline
\ccc{General_polygon_2} & \ccc{General_polygon_with_holes_2}\\
\hline
\ccc{General_polygon_with_holes_2} & \ccc{General_polygon_2}\\
\hline
\ccc{General_polygon_with_holes_2} & \ccc{General_polygon_with_holes_2}\\
\hline
\end{longtable}
\end{ccTexOnly}

\begin{ccHtmlOnly}
<div align="center">
<table cellpadding=3 border="1">
<tr><th> Arg 1 type</th><th>Arg 2 type</th></tr>
<tr><td valign="center">Polygon_2</td><td valign="center">Polygon_2</td></tr>
<tr><td valign="center">Polygon_2</td><td valign="center">polygon_with_holes_2</td></tr> 
<tr><td valign="center">Polygon_with_holes_2</td><td valign="center">Polygon_2</td></tr>
<tr><td valign="center">Polygon_with_holes_2</td><td valign="center">Polygon_with_holes_2</td></tr>
<tr><td valign="center">General_polygon_2</td><td valign="center">General_polygon_2</td></tr>
<tr><td valign="center">General_polygon_2</td><td valign="center">General_polygon_with_holes_2</td></tr>
<tr><td valign="center">General_polygon_with_holes_2</td><td valign="center">General_polygon_2</td></tr>
<tr><td valign="center">General_polygon_with_holes_2</td><td valign="center">General_polygon_with_holes_2</td></tr>
</table>
</div>
\end{ccHtmlOnly}

\ccGlobalFunction{template <class Kernel, class Container>
bool join(const Polygon_2<Kernel, Container> & p1,
          const Polygon_2<Kernel, Container> & p2,
	  General_polygon_with_holes_2<Polygon_2<Kernel, Container> > & p);}
\ccGlue
\ccGlobalFunction{template <class Kernel, class Container>
bool join(const Polygon_2<Kernel, Container> & p1,
          const Polygon_with_holes_2<Kernel,Container> & p2,
	  General_polygon_with_holes_2<Polygon_2<Kernel, Container> > & p);}
\ccGlue
\ccGlobalFunction{template <class Kernel, class Container>
bool join(const Polygon_with_holes_2<Kernel, Container> & p2,
          const Polygon_2<Kernel, Container> & p1,
	  General_polygon_with_holes_2<Polygon_2<Kernel, Container> > & p);}
\ccGlue
\ccGlobalFunction{template <class Kernel, class Container>
bool join(const Polygon_with_holes_2<Kernel, Container> & p2,
          const Polygon_with_holes_2<Kernel, Container> & p1,
	  General_polygon_with_holes_2<Polygon_2<Kernel, Container> > & p);}
\ccGlue
\ccGlobalFunction{template <class Traits>
bool join(const General_polygon_2<Traits> & p1,
          const General_polygon_2<Traits> & p2,
	  General_polygon_with_holes_2<General_polygon_2<Traits> > & p);}
\ccGlue
\ccGlobalFunction{template <class Traits>
bool join(const General_polygon_2<Traits> & p1,
          const General_polygon_with_holes_2<General_polygon_2<Traits> > & p2,
	  General_polygon_with_holes_2<General_polygon_2<Traits> > & p);}
\ccGlue
\ccGlobalFunction{template <class Traits>
bool join(const General_polygon_with_holes_2<General_polygon_2<Traits> > & p2,
          const General_polygon_2<Traits> & p1,
	  General_polygon_with_holes_2<General_polygon_2<Traits> > & p);}
\ccGlue
\ccGlobalFunction{template <class Polygon>
bool join(const General_polygon_with_holes_2<Polygon> & p1,
          const General_polygon_with_holes_2<Polygon> & p2,
	  Traits::Polygon_with_holes_2 & p);}

\ccGlobalFunction{template <class InputIterator, class OutputIterator>
OutputIterator join(InputIterator begin, InputIterator end,
                    OutputIterator oi);}
{Computes the union of the general polygons (or general polygons with
holes) in the given range. (The value type of the input iterator is
used to distinguish between the two.) The result, represented by a set
of general poygon with holes, is inserted into an output container
through a given output iterator \ccc{oi}. The output iterator is
returned. The value type of the \ccc{OutputIterator} is
\ccc{Traits::Polygon_with_holes_2}.}

\ccGlobalFunction{template <class InputIterator1, class InputIterator2,
		  class OutputIterator>
OutputIterator join(InputIterator1 pgn_begin1, InputIterator1 pgn_end1,
                    InputIterator2 pgn_begin2, InputIterator2 pgn_end2,
		    OutputIterator oi);}
{Computes the union of the general polygons and general polygons
with holes in the given two ranges. The result, represented by a set
of general poygon with holes, is inserted into an output container
through a given output iterator \ccc{oi}. The output iterator is
returned. The value type of the \ccc{OutputIterator} is
\ccc{Traits::Polygon_with_holes_2}.}

\ccSeeAlso
\ccRefIdfierPage{CGAL::do_intersect}\\
\ccRefIdfierPage{CGAL::intersection}\\
\ccRefIdfierPage{CGAL::difference}\\
\ccRefIdfierPage{CGAL::symmetric_difference}

\end{ccRefFunction}

\ccRefPageEnd
