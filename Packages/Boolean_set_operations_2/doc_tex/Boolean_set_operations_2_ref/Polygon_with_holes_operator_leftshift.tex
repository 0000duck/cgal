\RCSdef{\RCSPolygonwithholesoperatoristreamrev}{$Revision$}
\RCSdefDate{\RCSPolygonwithholesoperatoristreamDate}{$Date$}

\ccHtmlNoClassLinks
\begin{ccRefFunction}{operator<<}
\label{refPolygon_with_holes_operator_leftshift}

\ccDefinition

This operator exports a polygon with holes, a general polygon, or a general 
polygon with holes $P$ to the output stream \ccc{out}. The output is in ASCII
format.

An ASCII and a binary format exist. The format can be selected with
the \cgal\ modifiers for streams, \ccc{set_ascii_mode} and
\ccc{set_binary_mode} respectively. The modifier \ccc{set_pretty_mode}
can be used to allow for (a few) structuring comments in the
output. Otherwise, the output would be free of comments.  The default
for writing is ASCII without comments.

\ccInclude{CGAL/Polygon_with_holes_2.h}
\ccGlobalFunction{template <class Kernel, class Container>
    ostream& operator<<(ostream& out, 
                        const CGAL::Polygon_with_holes_2<Kernel,Container>& P);}

The number of points of the outer boundary is exported followed by the 
points themselves in counterclockwise order. Then, the holes, if exist, are
exported. For each hole, the number of points of the outer boundary of the 
hole is exported followed by the points themselves in clockwise order.

\ccInclude{CGAL/General_polygon_2.h}
\ccGlobalFunction{template <class ArrTraits>
    ostream& operator<<(ostream& out, 
                        const CGAL::General_polygon_2<ArrTraits>& P);}
The number of curves of the outer boundary is exported followed by the 
curves themselves in counterclockwise order.

\ccInclude{CGAL/General_polygon_with_holes_2.h}
\ccGlobalFunction{template <class Polygon>
  ostream& operator<<(ostream& out, 
                      const CGAL::General_polygon_with_holes_2<Polygon>& P);}

The number of curves of the outer boundary is exported followed by the 
curves themselves in counterclockwise order. Then, the holes, if exist, are
exported. For each hole, the number of curves of the outer boundary of the 
hole is exported followed by the curves themselves in clockwise order.

\ccSeeAlso

\ccRefIdfierPage{CGAL::Polygon_2<Kernel,Container>}\\ 
\ccRefIdfierPage{CGAL::General_polygon_2<ArrTraits>}\\ 
\ccRefIdfierPage{CGAL::General_polygon_with_holes_2<Polygon>}\\ 
  \lcTex{\ccc{operator>>} \dotfill\ page~\pageref{refPolygon_with_holes_operator_rightshift}}%
  \lcRawHtml{<I><A HREF="Function_operator--.html">operator&gt;&gt;</A></I>}

\end{ccRefFunction}
