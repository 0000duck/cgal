\ccRefPageBegin

\begin{ccRefClass}{General_polygon_set_2<General_polygon,Traits>}

\ccThree{General_polygon_set_2<General_polygon,Traits>}{General_polygon_2}{}
\ccThreeToTwo
    
\ccDefinition
%============
An instance of the \ccClassTemplateName\ class-template represents a
point set in the plane. An \ccc{Arrangement_2} data structure is used
internally to represent the point set. An instance of the
\ccc{General_polygon_set_2} class-template can be constructed from 
\ccc{General_polygon_2} or \ccc{General_polygon_with_holes_2} objects.
The \ccc{General_polygon_2} and \ccc{General_polygon_with_holes_2} must
be defined in the traits class among the other.

The class template provides methods to apply Boolean-set operations on
pairs of \ccc{General_polygon_set_2} objects, or \ccc{General_polygon_2}
(or \ccc{General_polygon_with_holes_2}) directly, and a few other utility
methods. At this point only regularized Boolean-set operations are
implemented. The input and output of these methods consist of one or
more general polygons, some of which may have holes.

The \ccc{Traits} template-parameter should be instantiated with a
model of the \ccc{GeneralPolygonSetTraits} concept. The traits class
defines the types of points, $x$-monotone curves, general polygons,
and general polygons with holes, that is \ccc{Point_2},
\ccc{X_monotone_curve_2}, \ccc{General_polygon_2}, and
\ccc{General_polygon_with_holes_2} respectively. \ccc{Point_2} must
be the type of the endpoints of \ccc{X_monotone_curve_2}, and
\ccc{X_monotone_curve_2} must be the type of the curves that comprise
the boundaries of the general polygons. The traits class supports
geometric operations on the types above.

The traits parameter is optional for certain values of \ccc{General_polygon},
i.e., \ccc{Polygon_2}, \ccc{Triangle_2}, and \ccc{Iso_rectangle_2}.
In these cases the traits is extracted through the utility class
\ccc{Default_general_polygon_set_traits_2<General_polygon>}. For others
instantiated values of \ccc{General_polygon}, it must be specified and its
nested type \ccc{Traits::General_polygon_2} must match the instantiated
value of \ccc{General_polygon}.

\ccTypes
%=======
\ccNestedType{Traits_2}{the traits class in use.}
\ccGlue
\ccNestedType{General_polygon_2}{the general polygon type.}
\ccGlue
\ccNestedType{General_polygon_with_holes_2}{the general polygon
with holes type.} 

\ccNestedType{Arrangement_2}{the arrangement type used internally.}

\ccCreation
\ccCreationVariable{ps}
%=======================

\ccThree{General_polygon_set_2<General_polygon,Traits>}{0pt}{}
\ccThreeToTwo
    
\ccConstructor{General_polygon_set_2<General_polygon,Traits>();} 
  {constructs an empty set of general polygons represented by an
  empty arrangement.}
    
\ccConstructor{General_polygon_set_2<General_polygon,Traits>(const Self & other);}
  {copy constructor.}
        
\ccConstructor{General_polygon_set_2<General_polygon,Traits>(Traits & traits);}
  {constructs an empty set of general polygons that uses the given
  \ccc{traits} instance for performing the geometric operations.}

\ccConstructor{General_polygon_set_2<General_polygon,Traits>(const General_polygon_2 & pgn);}
  {constructs a set of general polygons from the input general polygon
  \ccc{pgn}.}

\ccConstructor{General_polygon_set_2<General_polygon,Traits>(const General_polygon_with_holes_2 & pgn_with_holes);}
  {constructs a set of general polygons from the input general polygon
  with holes \ccc{pgn_with_holes}.}

\ccConstructor{template <class InputIterator>
General_polygon_set_2<General_polygon,Traits>(InputIterator begin,
                                              InputIterator end);}
  {constructs a set of general polygons from the given range of
  instances of \ccc{General_polygon_with_holes_2} or instances of
  \ccc{General_polygon_2}. (The value type of the input iterator is
  used to distinguish between the two.)
  \ccPrecond{All general polygons (resp. general polygons with holes)
  in the given range do not intersect pairwise, except maybe at the
  vertices.}}
  
\ccConstructor{template <class InputIterator1, class InputIterator2>
  General_polygon_set_2<General_polygon,Traits>
  (InputIterator1 pgn_begin, InputIterator1 pgn_end,
   InputIterator2 pgn_with_holes_begin,
   InputIterator2 pgn_with_holes_end);}
  {constructs a set of general polygons from the given range of
  \ccc{General_polygon_2} and the range of
  \ccc{General_polygon_with_holes_2} instances.
  \ccPrecond{All general polygons in the first range, all
  general polygon with holes in the second range do not intersect
  pairwise, except maybe at the vertices.}}

\ccAccessFunctions
% ================
\ccThree{Arrangement_2}{ps.general_polygons(OutputIterator & out~)}{}
\ccThreeToTwo

\ccMethod{template <class OutputIterator>
  OutputIterator & general_polygons_with_holes(OutputIterator & out);}
  {obtains all the general polygons as general polygons with
  holes.}
  
% \ccMethod{template <class OutputIterator1, class OutputIterator2>
%   std::pair<OutputIterator1 &,OutputIterator2 &>
%   general_polygons(OutputIterator1 out1, OutputIterator2 out2);}
%   {obtains all the general polygons. The general polygons (without
%   holes) are inserted to the range given by the first iterator, and
%   those with holes are inserted to the second one.}

\ccMethod{Size number_of_general_polygons_with_holes() const;}
  {returns the total number of general polygons represented by \ccVar.}

% \ccMethod{std::pair<Size,Size> number_of_general_polygons() const;}
%   {computes the number of general polygons without holes and those with holes
%   represented by \ccVar, and returns as a pair in this order.}

\ccMethod{Traits & traits() const;}
  {obtains an instance of the traits. If the traits was passed as a
  parameter to the constructor of \ccVar, it is returned. Otherwise, a
  newly created instance is returned.}

\ccMethod{const Arrangement_2 & arrangement() const;}
  {obtains the arrangement data structure that internally represtns the
  general-polygon set.}

\ccModifiers
% ==========
\ccThree{void}{ps.difference(General_polygon_with_holes_2 & pgn~)}{}
\ccThreeToTwo

\ccMethod{void clear();}
  {clears \ccVar.}

\ccMethod{void insert(General_polygon_2 & pgn);}
  {inserts \ccc{pgn} into \ccVar.
  \ccPrecond{\ccc{pgn} does not intersect with the point set represented by
  \ccVar, except maybe at the vertices.}}

\ccMethod{void insert(General_polygon_with_holes_2 & pgn_with_holes);}
  {inserts \ccc{pgn_with_holes} into \ccVar.
  \ccPrecond{\ccc{pgn_with_holes} does not intersect with the point set
  represented by \ccVar, except maybe at the vertices.}}

\ccMethod{template <class InputIterator>
void insert(InputIterator & begin, InputIterator & end);}
  {inserts the range of general polygons (or general polygons with
  holes) into \ccVar. (The value type of the input iterator is used to
  distinguish between the two.)
  \ccPrecond{All general polygons (resp. general polygons with holes)
  in the given range and the point set represented by \ccVar\ do not
  intersect pairwise, except maybe at the vertices.}}

\ccMethod{template <class InputIterator1, class InputIterator2>
void insert(InputIterator1 & pgn_begin, InputIterator1 & pgn_end,
            InputIterator2 & pgn_with_holes_begin,
	    InputIterator2 & pgn_with_holes_end);}
  {inserts the two ranges of general polygons and general polygons with holes
  into \ccVar.
  \ccPrecond{All general polygons in the first range, all
  general polygon with holes in the second range, and the point set
  represented by \ccVar\ do not intersect pairwise, except maybe at the
  vertices.}}
  
\ccHeading{Univariate Operations}
% ===============================
In the following univariate and bivariate methods the result is placed in
\ccVar\ after it is cleared.

% intersection
% ------------
\ccMethod{void intersection(const General_polygon_set_2 & cps);}
  {computes the intersection of \ccVar\ and \ccc{cps}.}
\ccGlue
\ccMethod{void intersection(const General_polygon_2 & pgn);}
  {computes the intersection of \ccVar\ and \ccc{pgn}.}
\ccGlue
\ccMethod{void intersection(const General_polygon_with_holes_2 & pgn);}
  {computes the intersection of \ccVar\ and \ccc{pgn}.}

\ccMethod{template <class InputIterator>
void intersection(InputIterator & begin, InputIterator & end);}
  {Computes the intersection of the general polygons (or general
  polygons with holes) in the given range. (The value type of the
  input iterator is used to distinguish between the two.)}
\ccGlue
\ccMethod{template <class InputIterator1, class InputIterator2>
void intersection(InputIterator1 & pgn_begin, InputIterator1 & pgn_end,
                  InputIterator2 & pgn_with_holes_begin,
	          InputIterator2 & pgn_with_holes_end);}
{Computes the intersection of the general polygons and general polygons
with holes in the given two ranges.}

% Union
% -----
\ccMethod{void join(const General_polygon_set_2 & cps);}
  {computes the union of \ccVar\ and \ccc{cps}.}
\ccGlue
\ccMethod{void join(const General_polygon_2 & pgn);}
  {computes the union of \ccVar\ and \ccc{pgn}.}
\ccGlue
\ccMethod{void join(const General_polygon_with_holes_2 & pgn);}
  {computes the union of \ccVar\ and \ccc{pgn}.}

\ccMethod{template <class InputIterator>
void join(InputIterator & begin, InputIterator & end);}
  {Computes the union of the general polygons (or general
  polygons with holes) in the given range. (The value type of the
  input iterator is used to distinguish between the two.)}
\ccGlue
\ccMethod{template <class InputIterator1, class InputIterator2>
void join(InputIterator1 & pgn_begin, InputIterator1 & pgn_end,
          InputIterator2 & pgn_with_holes_begin,
	  InputIterator2 & pgn_with_holes_end);}
{Computes the union of the general polygons and general polygons
with holes in the given two ranges.}

% Difference
% ----------
\ccMethod{void difference(const General_polygon_set_2 & cps);}
  {computes the difference between \ccVar\ and \ccc{cps}.}
\ccGlue
\ccMethod{void difference(const General_polygon_2 & pgn);}
  {computes the difference between \ccVar\ and \ccc{pgn}.}
\ccGlue
\ccMethod{void difference(const General_polygon_with_holes_2 & pgn);}
  {computes the difference between \ccVar\ and \ccc{pgn}.}

\ccMethod{void symmetric_difference(const General_polygon_set_2 & cps);}
  {computes the symmetric difference between \ccVar\ and \ccc{cps}.}
\ccGlue
\ccMethod{void symmetric_difference(const General_polygon_2 & pgn);}
  {computes the symmetric difference between \ccVar\ and \ccc{pgn}.}
\ccGlue
\ccMethod{void symmetric_difference(const General_polygon_with_holes_2 & pgn);}
  {computes the symmetric difference between \ccVar\ and \ccc{pgn}.}

\ccHeading{Bivariate Operations}
% ===============================

\ccMethod{void intersection(const General_polygon_set_2 & cps1,
                            const General_polygon_set_2 & cps2);}
  {computes the intersection of \ccc{cps1} and \ccc{cps2}.}

\ccMethod{void join(const General_polygon_set_2 & cps1,
                     const General_polygon_set_2 & cps2);}
  {computes the union of \ccc{cps1} and \ccc{cps2}.}

\ccMethod{void difference(const General_polygon_set_2 & cps1,
                          const General_polygon_set_2 & cps2);}
  {computes the difference between \ccc{cps1} and \ccc{cps2}.}

\ccMethod{void symmetric_difference(const General_polygon_set_2 & cps1,
                                   const General_polygon_set_2 & cps2);}
  {computes the symmetric difference between \ccc{cps1} and \ccc{cps2}.}
  
\ccPredicates
% ===========

\ccThree{OutputIterator\&}{ps.do_intersect(General_polygon_set_2 & cps)}{}
\ccThreeToTwo

\ccMethod{bool do_intersect(const General_polygon_set_2 & cps);}
  {returns \ccc{true} if \ccVar\ and \ccc{cps} intersect in their
  interior, and \ccc{false} otherwise.}
\ccGlue
\ccMethod{bool do_intersect(const General_polygon_2 & pgn);}
  {returns \ccc{true} if \ccVar\ and \ccc{pgn} intersect in their
  interior, and \ccc{false} otherwise.}
\ccGlue
\ccMethod{bool do_intersect(const General_polygon_with_holes_2 & pgn);}
  {returns \ccc{true} if \ccVar\ and \ccc{pgn} intersect in their
  interior, and \ccc{false} otherwise.}
  
\ccMethod{bool is_empty() const;}
  {returns \ccc{true} if \ccVar\ represents an empty set.}

\ccMethod{Bounded_side bounded_side(const Point_2 & q);}
  {returns the symbolic constant \ccc{ON_BOUNDED_SIDE},
  \ccc{ON_BOUNDARY} or \ccc{ON_UNBOUNDED_SIDE}, depending on the
  location of point \ccc{q}.}

% \ccMethod{Object locate(const Point_2 & p);}
%   {returns a general polygon that contains the query point \ccc{p}, if
%   exists. Object can represent either \ccc{General_polygon_2} or
%   \ccc{General_polygon_with_holes_2}, or it can be empty, in case
%   \ccc{p} is not contained in any general polygon.}
% 
\ccMethod{bool locate(const Point_2 & p, General_polygon_with_holes_2 & pgn);}
  {obtains a general polygon with holes that contains the query point
  \ccc{p}, if exists, through \ccc{pgn}, and returns
  \ccc{true}. Otherwise, returns \ccc{flase}.}

\ccSeeAlso
  \ccc{Arrangement_2}\lcTex{(\ccRefPage{Arrangement_2})}\\
  \ccc{ArrangementXMonotoneTraits_2}\lcTex{(\ccRefPage{ArrangementXMonotoneTraits_2})}\\
  \ccc{Nef_2}\lcTex{(\ccRefPage{Nef_2})}

\end{ccRefClass}

\ccRefPageEnd
