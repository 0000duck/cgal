\ccRefPageBegin

\begin{ccRefClass}{General_polygon_set_2<Traits>}

\ccThree{General_polygon_set_2<Traits>}{Polygon_2}{}
\ccThreeToTwo
    
\ccDefinition
%============
An object of the \ccClassTemplateName\ class-template represents a
point set in the plane bounded by $x$ monotone curves. Points in the set
lie on the boundary or on the positive side of the curves. This class 
template provides methods to apply regularized Boolean set-operations and 
few other utility methods. An \ccc{Arrangement_2} data structure is 
internally used to represent the point set. 

The \ccc{Traits} template-parameter should be instantiated with a
model of the concept \ccc{GeneralPolygonSetTraits_2}. The traits class
defines the types of points, $x$-monotone curves, general polygons,
and general polygons with holes, that is \ccc{Traits::Point_2},
\ccc{Traits::X_monotone_curve_2}, \ccc{Traits::Polygon_2}, and 
\ccc{Traits::Polygon_with_holes_2} respectively. \ccc{Traits::Point_2} must 
be the type of the endpoints of \ccc{Traits::X_monotone_curve_2}, and 
\ccc{Traits::X_monotone_curve_2} must be the type of the curves that comprise
the boundaries of the general polygons. The traits class supports geometric 
operations on the types above. We sometimes use the term {\em polygon} instead
of general polygon for simplicity hereafter.

The input and output of the Boolean set-operations methods consist of one 
or more general polygons, some of which may have holes. In particular, 
these methods operate on pairs of objects of type \ccClassTemplateName, or
directly on objects of type \ccc{Traits::Polygon_2} or 
\ccc{Traits::Polygon_with_holes_2}. An object of type 
\ccc{Traits::Polygon_2} is a valid operand, only if it is strictly simple 
and its boundary is oriented counterclockwise. An object of type
\ccc{Traits::Polygon_with_holes_2} is valid, only if its outer boundary
and each one of its holes represnt a simple polygon, its holes are contained 
inside the outer boundary, its outer boundary and the outer boundary of its 
holes are pairwise disjoint, except perhaps at the vertices, its outer boundary
is oriented counterclockwise, and the outer boundary of its holes are 
oriented clockwise.

\ccTypes
%=======
\ccNestedType{Traits_2}{the traits class in use.}
\ccGlue
\ccNestedType{Polygon_2}{the general polygon type.}
\ccGlue
\ccNestedType{Polygon_with_holes_2}{the general polygon with holes type.} 

\ccNestedType{Size}{number of polygons with holes size type.}

\ccNestedType{Arrangement_2}{the arrangement type used internally.}

\ccCreation
\ccCreationVariable{gps}
%=======================

\ccThree{General_polygon_set_2<Traits>(Polygon_2 pgn~)}{0pt}{}
\ccThreeToTwo
    
\ccConstructor{General_polygon_set_2<Traits>();} 
  {constructs an empty set of polygons represented by an empty arrangement.}
    
\ccConstructor{General_polygon_set_2<Traits>(const Self & other);}
  {copy constructor.}
        
\ccConstructor{General_polygon_set_2<Traits>(Traits & traits);}
  {constructs an empty set of polygons that uses the given
  \ccc{traits} instance for performing the geometric operations.}

\ccConstructor{General_polygon_set_2<Traits>(const Polygon_2 & pgn);}
  {constructs a set of polygons that consists of the single polygon \ccc{pgn}.}

\ccConstructor{General_polygon_set_2<Traits>(const Polygon_with_holes_2 & pgn_with_holes);}
  {constructs a set of polygons that consists of the single polygon with 
   holes \ccc{pgn_with_holes}.}

% \ccConstructor{template <class InputIterator>
% General_polygon_set_2<General_polygon,Traits>(InputIterator begin,
%                                               InputIterator end);}
%   {constructs a set of polygons from the given range of
%   instances of \ccc{Polygon_with_holes_2} or instances of
%   \ccc{Polygon_2}. (The value type of the input iterator is
%   used to distinguish between the two.)
%   \ccPrecond{All polygons (resp. polygons with holes)
%   in the given range are pairwise disjoint, except maybe at the
%   vertices.}}
%   
% \ccConstructor{template <class InputIterator1, class InputIterator2>
%   General_polygon_set_2<GeneralPolygon,Traits>
%   (InputIterator1 pgn_begin, InputIterator1 pgn_end,
%    InputIterator2 pgn_with_holes_begin,
%    InputIterator2 pgn_with_holes_end);}
%   {constructs a set of polygons from the given range of
%   \ccc{Polygon_2} and the range of
%   \ccc{Polygon_with_holes_2} instances.
%   \ccPrecond{All polygons in the first range, all
%   polygon with holes in the second range are pairwise disjoint,
%   except maybe at the vertices.}}

\ccAccessFunctions
% ================
\ccThree{Arrangement_2}{gps.number_of_polygons_with_holes()~}{}
\ccThreeToTwo

\ccMethod{template <class OutputIterator>
  OutputIterator polygons_with_holes(OutputIterator out);}
  {obtains the polygons with holes represented by \ccVar.}
  
\ccMethod{Size number_of_polygons_with_holes() const;}
  {returns the total number of general polygons represented by \ccVar.}

\ccMethod{bool is_empty() const;}
  {returns \ccc{true} if \ccVar\ represents an empty set.}

\ccMethod{bool is_plane() const;}
  {returns \ccc{true} if \ccVar\ represents the entire plane.}

\ccMethod{Traits & traits() const;}
  {obtains an instance of the traits. If the traits was passed as a
  parameter to the constructor of \ccVar, it is returned. Otherwise, a
  newly created instance is returned.}

\ccMethod{const Arrangement_2 & arrangement() const;}
  {obtains the arrangement data structure that internally represtns the
  general-polygon set.}

\ccModifiers
% ==========
\ccThree{void}{ps.difference(Polygon_with_holes_2 & pgn)~}{}
\ccThreeToTwo

\ccMethod{void clear();}
  {clears \ccVar.}

\ccMethod{void insert(Polygon_2 & pgn);}
  {inserts \ccc{pgn} into \ccVar.
   \ccPrecond{\ccc{pgn} and the point set represented by \ccVar\ are disjoint.}
   This precondition enbales the use of very efficient insertion methods.
   Use the respective overloaded method that inserts a polygon of type 
   \ccc{Polygon_with_holes_2}, if only a relaxed precondition can be
   guaranteed. If even the relaxed precondition cannot be guaranteed, use
   the \ccc{join} method.}

\ccMethod{void insert(Polygon_with_holes_2 & pgn_with_holes);}
  {inserts \ccc{pgn_with_holes} into \ccVar.
   \ccPrecond{\ccc{pgn_with_holes} does not intersect with the point set
   represented by \ccVar, except maybe at the vertices.}
   If this relaxed precondition cannot be guaranteed, use the \ccc{join} 
   method.}

\ccMethod{template <class InputIterator>
void insert(InputIterator begin, InputIterator end);}
  {inserts the range of polygons (or polygons with holes) into \ccVar. (The
   value type of the input iterator is used to distinguish between the two.)
   \ccPrecond{If the given range contains objects of type 
   \ccc{Polygon_with_holes_2}, then these polygons and the point set 
   represented by \ccVar\ are pairwise disjoint, except maybe at the vertices. 
   If the given range contains objects of type \ccc{Polygon_2}, then these 
   polygons and the point set represented by \ccVar\ are pairwise disjoint 
   without any exceptions.}}

\ccMethod{template <class InputIterator1, class InputIterator2>
void insert(InputIterator1 pgn_begin, InputIterator1 pgn_end,
            InputIterator2 pgn_with_holes_begin,
	    InputIterator2 pgn_with_holes_end);}
  {inserts the two ranges of polygons and polygons with holes into \ccVar.
   \ccPrecond{All polygons in the first range, all polygon with holes in the 
   second range, and the point set represented by \ccVar\ are pairwise 
   disjoint, except maybe at the vertices}}

\ccMethod{void complement();}
  {computes the complement of \ccVar.}

\ccMethod{void complement(const Polygon_set_2 & other);}
  {computes the complement of \ccc{other}. 
   \ccVar\ is overriden by the result.}

% \ccMethod{void complement(const Polygon_2 & pgn);}
%   {computes the complement of \ccc{pgn}.}
%   
% \ccMethod{void complement(const Polygon_with-holes_2 & pgn);}
%   {computes the complement of \ccc{pgn}.}

\ccHeading{Univariate Operations}
% ===============================
In the following univariate and bivariate methods the result is placed in
\ccVar\ after it is cleared.

% intersection
% ------------
\ccMethod{void intersection(const General_polygon_set_2 & other);}
  {computes the intersection of \ccVar\ and \ccc{other}.}
\ccGlue
\ccMethod{void intersection(const Polygon_2 & pgn);}
  {computes the intersection of \ccVar\ and \ccc{pgn}.}
\ccGlue
\ccMethod{void intersection(const Polygon_with_holes_2 & pgn);}
  {computes the intersection of \ccVar\ and \ccc{pgn}.}

\ccMethod{template <class InputIterator>
void intersection(InputIterator begin, InputIterator end);}
  {computes the intersection of a collection of point sets. The collection
   consists of the polygons (or polygons with holes) in the given range and 
   the point set represented by \ccVar. (The value type of the input iterator 
   is used to distinguish between the two options.)}
\ccGlue
\ccMethod{template <class InputIterator1, class InputIterator2>
void intersection(InputIterator1 pgn_begin, InputIterator1 pgn_end,
                  InputIterator2 pgn_with_holes_begin,
	          InputIterator2 pgn_with_holes_end);}
  {computes the intersection of a collection of point sets. The collection
   consists of the polygons and polygons with holes in the given two ranges
   and the point set represented by \ccVar.}

% Union
% -----
\ccMethod{void join(const General_polygon_set_2 & other);}
  {computes the union of \ccVar\ and \ccc{other}.}
\ccGlue
\ccMethod{void join(const Polygon_2 & pgn);}
  {computes the union of \ccVar\ and \ccc{pgn}.}
\ccGlue
\ccMethod{void join(const Polygon_with_holes_2 & pgn);}
  {computes the union of \ccVar\ and \ccc{pgn}.}

\ccMethod{template <class InputIterator>
void join(InputIterator begin, InputIterator end);}
  {computes the union of the polygons (or polygons with holes) in the 
   given range and the point set represented by \ccVar. (The value type 
   of the input iterator is used to distinguish between the two options.)}
\ccGlue
\ccMethod{template <class InputIterator1, class InputIterator2>
void join(InputIterator1 pgn_begin, InputIterator1 pgn_end,
          InputIterator2 pgn_with_holes_begin,
	  InputIterator2 pgn_with_holes_end);}
  {computes the union of the polygons and polygons with holes in the 
   given two ranges and the point set represented by \ccVar.}

% Difference
% ----------
\ccMethod{void difference(const General_polygon_set_2 & other);}
  {computes the difference between \ccVar\ and \ccc{other}.}
\ccGlue
\ccMethod{void difference(const Polygon_2 & pgn);}
  {computes the difference between \ccVar\ and \ccc{pgn}.}
\ccGlue
\ccMethod{void difference(const Polygon_with_holes_2 & pgn);}
  {computes the difference between \ccVar\ and \ccc{pgn}.}

% Symmetric Difference
% --------------------
\ccMethod{void symmetric_difference(const General_polygon_set_2 & other);}
  {computes the symmetric difference between \ccVar\ and \ccc{other}.}
\ccGlue
\ccMethod{void symmetric_difference(const Polygon_2 & pgn);}
  {computes the symmetric difference between \ccVar\ and \ccc{pgn}.}
\ccGlue
\ccMethod{void symmetric_difference(const Polygon_with_holes_2 & pgn);}
  {computes the symmetric difference between \ccVar\ and \ccc{pgn}.}

\ccMethod{template <class InputIterator>
void symmetric_difference(InputIterator begin, InputIterator end);}
  {computes the symmetric difference (xor) of a collection of point sets.
   The collection consists of the polygons (or polygons with holes) in the 
   given range and the point set represented by \ccVar. (The value type of 
   the input iterator is used to distinguish between the two options.)}
\ccGlue
\ccMethod{template <class InputIterator1, class InputIterator2>
void symmetric_difference(InputIterator1 pgn_begin, InputIterator1 pgn_end,
          InputIterator2 pgn_with_holes_begin,
	  InputIterator2 pgn_with_holes_end);}
  {computes the symmetric difference (xor) of a collection of point sets.
   The collection consists of the polygons and polygons with holes in the
   given two ranges and the point set represented by \ccVar.}

\ccHeading{Bivariate Operations}
% ===============================
The following bibariate function replace this with the result.

\ccMethod{void intersection(const General_polygon_set_2 & gps1,
                            const General_polygon_set_2 & gps2);}
  {computes the intersection of \ccc{gps1} and \ccc{gps2}.}

\ccMethod{void join(const General_polygon_set_2 & gps1,
                    const General_polygon_set_2 & gps2);}
  {computes the union of \ccc{gps1} and \ccc{gps2}.}

\ccMethod{void difference(const General_polygon_set_2 & gps1,
                          const General_polygon_set_2 & gps2);}
  {computes the difference between \ccc{gps1} and \ccc{gps2}.}

\ccMethod{void symmetric_difference(const General_polygon_set_2 & gps1,
                                   const General_polygon_set_2 & gps2);}
  {computes the symmetric difference between \ccc{gps1} and \ccc{gps2}.}
  
\ccQueryFunctions
% ===============

\ccThree{OutputIterator\&}{ps.do_intersect(General_polygon_set_2 & other)}{}
\ccThreeToTwo

\ccMethod{bool do_intersect(const General_polygon_set_2 & other);}
  {returns \ccc{true} if \ccVar\ and \ccc{other} intersect in their
  interior, and \ccc{false} otherwise.}
\ccGlue
\ccMethod{bool do_intersect(const Polygon_2 & pgn);}
  {returns \ccc{true} if \ccVar\ and \ccc{pgn} intersect in their
  interior, and \ccc{false} otherwise.}
\ccGlue
\ccMethod{bool do_intersect(const Polygon_with_holes_2 & pgn);}
  {returns \ccc{true} if \ccVar\ and \ccc{pgn} intersect in their
  interior, and \ccc{false} otherwise.}
  
\ccMethod{template <class InputIterator>
void do_intersect(InputIterator begin, InputIterator end);}
  {returns \ccc{true} if the interior of the point sets in a collection 
   intersect, and \ccc{false} otherwise. The collection consists of the 
   polygons (or polygons with holes) in the given range and the point set 
   represented by \ccVar. (The value type of the input iterator is used to 
   distinguish between the two options.)}
\ccGlue
\ccMethod{template <class InputIterator1, class InputIterator2>
void intersection(InputIterator1 pgn_begin, InputIterator1 pgn_end,
                  InputIterator2 pgn_with_holes_begin,
	          InputIterator2 pgn_with_holes_end);}
  {returns \ccc{true} if the interior of the point sets in a collection 
   intersect, and \ccc{false} otherwise. The collection consists of the 
   polygons and polygons with holes in the given two ranges and the point 
   set represented by \ccVar.}

\ccMethod{bool locate(const Point_2 & p, Polygon_with_holes_2 & pgn);}
  {obtains a polygon with holes that contains the query point \ccc{p}, 
   if exists, through \ccc{pgn}, and returns \ccc{true}. 
   Otherwise, returns \ccc{flase}.}

\ccMethod{Oriented_side oriented_side(const Point_2 & q);}
  {returns either the constant \ccc{ON_ORIENTED_BOUNDARY}, 
   \ccc{ON_POSITIVE_SIDE}, or \ccc{ON_NEGATIVE_SIDE}, iff \ccc{p} lies on 
   the boundary, properly on the positive side, or properly on the negative 
   side of \ccVar\ respectively.}

\ccMethod{Oriented_side oriented_side(const General_polygon_set_2 & other);}
  {returns either the constant \ccc{ON_NEGATIVE_SIDE},
   \ccc{ON_ORIENTED_BOUNDARY}, or \ccc{ON_POSITIVE_SIDE}, iff 
   \ccc{other} and \ccVar\ are completely disjoint, in contact, or
   intersect in their interior, respectively.}
\ccGlue
\ccMethod{Oriented_side oriented_side(const Polygon_2 & pgn);}
  {returns either the constant \ccc{ON_NEGATIVE_SIDE},
   \ccc{ON_ORIENTED_BOUNDARY}, or \ccc{ON_POSITIVE_SIDE}, iff 
   \ccc{pgn} and \ccVar\ are completely disjoint, in contact, or
   intersect in their interior, respectively.}
\ccGlue
\ccMethod{Oriented_side oriented_side(const Polygon_with_holes_2 & pgn);}
  {returns either the constant \ccc{ON_NEGATIVE_SIDE},
   \ccc{ON_ORIENTED_BOUNDARY}, or \ccc{ON_POSITIVE_SIDE}, iff 
   \ccc{pgn} and \ccVar\ are completely disjoint, in contact, or
   intersect in their interior, respectively.}

\ccHeading{Miscellaneous}
% =======================

\ccMethod{bool is_valid() const;}
  {returns \ccc{true} if \ccVar\ represents a valid point set.}

\ccSeeAlso
  \ccc{Arrangement_2}\lcTex{(\ccRefPage{Arrangement_2})}\\
  \ccc{ArrangementXMonotoneTraits_2}\lcTex{(\ccRefPage{ArrangementXMonotoneTraits_2})}\\
  \ccc{Nef_2}\lcTex{(\ccRefPage{Nef_2})}

\end{ccRefClass}

\ccRefPageEnd
