\ccSetThreeColumns{Postscript_file_stream}{}{\hspace*{8.5cm}}
\ccSetTwoColumns{}{\hspace*{8.5cm}}

\medskip
\begin{ccClass} {Postscript_file_stream}
\ccSection{PostScript File Stream}

Note: For the functionality described in this section you need \leda.

\ccDefinition 
An object of type \ccClassName\ provides graphical output in PostScript.
\ccClassName\ is derived from the \leda\ class \ccStyle{ps_file}.
Therefore, the functionality of \ccStyle{ps_file} is available on \ccClassName\ as well.
In addition, the class \ccClassName\ provides input and output 
operators for the classes which are defined in the \cgal\ kernel.
The interface of \ccClassName s is very similar to the interface of \ccStyle{leda_window}s. 
To increase similarity, some functions have been added with respect to \ccStyle{ps_file}.
\begin{ccTexOnly}
For further information on \ccStyle{ps_file}, we refer to \cite{mnsu-lum}.
\end{ccTexOnly}
\begin{ccHtmlOnly}
For further information on <i>ps_file</i>, we refer to the 
<A HREF="http://www.mpi-sb.mpg.de/LEDA/MANUAL/MANUAL.html">LEDA manual</A>.
\end{ccHtmlOnly}

\ccInclude{CGAL/IO/Postscript_file_stream.h}

\ccCreation
\ccCreationVariable{PS}

\ccConstructor{Postscript_file_stream();}
{creates a postscript file stream. The name of the output file is
\ccc{CGAL_unnamed.ps} .}

\ccConstructor{Postscript_file_stream(const char* filename);}
{creates a postscript file stream. The name of the output file is
\ccc{filename} .}

\ccConstructor{Postscript_file_stream(float w, float h);}
{creates a postscript file stream that draws the output to a box
of physical size \ccStyle{width/40} $\times$ \ccStyle{height/40} cm.
The name of the output file is \ccc{CGAL_unnamed.ps} .}

\ccConstructor{Postscript_file_stream(float w, float h, const char* filename);}
{creates a postscript file stream that draws the output to a box
of physical size \ccStyle{width/40} $\times$ \ccStyle{height/40} cm.
The name of the output file is \ccc{filename} .}


\ccOperations
The drawing region of a postscript file stream is initialized by a part of the 
object space by

\ccMethod{void init(double x0, double x1, double y0); }
%{sets $xmin$ to $x_0$, $xmax$ to $x_1$, and $ymin$ to $y_0$, 
%             the scaling factor $scaling$ to $w / (xmax-xmin)$, and 
%             $ymax$ to $ymin + h \cdot scaling$. Here $w$ and $h$ are the
%             width and height of the drawing section in pixels. }
{initializes the  part of the object space that is mapped to the physical
drawing box to [\ccc{x0},\ccc{x1}] $\times$ [\ccc{y0},\ccc{y0} + 
\ccc{height}$*$\ccc{width}/(\ccc{x1}-\ccc{x0})], where \ccc{width} and \ccc{height} 
are the width and the height of the drawing box in cm.}

The following functions have been added to give \ccClassName\ the same interface
as \ccc{Window_stream} to allow easily replacement of \ccc{Window_stream} by
\ccClassName. 

\ccMethod{void display();}
{does nothing}

\ccMethod{void display(int x, int y);}
{does nothing}

\ccMethod{int read_mouse(double& x, double& y);}
{returns 1}

\ccMethod{leda_color set_fg_color(leda_color c);}
{class \ccc{set_color(c).}}

See the \ccAnchor{http://www.mpi-sb.mpg.de/LEDA/MANUAL/MANUAL.html}{LEDA Manual}
\cite{mnsu-lum} for further operations on \ccc{ps_file}.

\def\ccTagRmEigenClassName{\ccFalse}
\ccHeading{Output Operators}

\ccSetThreeColumns{Postscript_file_stream& }{}{\hspace*{8.5cm}}

The output operator is defined for all geometric classes in the two-dimensional
\cgal\ kernel.

\ccFunction{Postscript_file_stream& operator<<(Postscript_file_stream& PS, const Class& c);}
       {inserts object \ccStyle{c} into the stream \ccVar.}

\ccFunction{Postscript_file_stream& operator<<(Postscript_file_stream& PS, const Color& c);}
       {changes the foreground color for the next objects that
        will be inserted into \ccVar.}

\ccFunction{void cgalize(Postscript_file_stream& ps);}
{sets a few default values for drawing to ps. In particular, node width is
set to 5 and line width to 1.}

There are no input operators of course.

\ccExample

It is important that the postscript file stream header file gets included
{\em after} the inclusion of the header files of geometric classes
that get inserted to the postscript stream file.

\begin{cprog}

#include <CGAL/Cartesian.h>
#include <CGAL/Segment_2.h>
#include <CGAL/IO/Postscript_file_stream.h>

typedef CGAL::Point_2< CGAL::Cartesian<double> >     Point;
typedef CGAL::Segment_2< CGAL::Cartesian<double> >   Segment;

int main()
{
    Point p(0,1), q(2,2);
    Segment s(p,q);

    CGAL::Postscript_file_stream PS(100,100);
    PS.init(0,10,0);
    CGAL::cgalize( PS);
    PS.display();

    PS << CGAL::RED << s << CGAL::BLACK << p << q ;
    return 0;
}
\end{cprog} 

\end{ccClass} 
