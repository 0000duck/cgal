\ccParDims

\section{Introduction}

For visualizing and debugging your programs the graphical output of geometric 
objects is important. In this chapter we describe functionality useful for 
graphical IO and output in Postscript format.\\
For window output we use the \leda\ window class if \leda\ is present. If you 
install \cgal\ without \leda\ you can use a ported version of the \leda\ window 
class.  This port of the \leda\ window lib can be downloaded as a precompiled 
library for various platforms from the \cgal\ web page. Note that Algorithmic 
Solutions Software GmbH has offered us the possibility to provide the ported 
version of the \leda\ window.\\ For output in Postscript format using the 
\cgal\ Postscript file stream you need \leda\ installed.

% +------------------------------------------------------------------------+
% | Reference manual page: Color.tex
% +------------------------------------------------------------------------+
% | 05.04.2002   Author
% | Package: Package
% | 
\RCSdef{\RCSColorRev}{$Revision$}
\RCSdefDate{\RCSColorDate}{$Date$}
% |
%%RefPage: end of header, begin of main body
% +------------------------------------------------------------------------+


\begin{ccRefClass}{Color}  %% add template arg's if necessary

%% \ccHtmlCrossLink{}     %% add further rules for cross referencing links
%% \ccHtmlIndexC[class]{} %% add further index entries

\ccDefinition
An object of the class \ccRefName\ is a color available
for drawing operations in \cgal\ output streams.
Each color is defined by a triple of integers $(r,g,b)$ with
$0\le r,g,b \le 255$, the so-called {\em rgb-value} of the color.  

\ccInclude{CGAL/IO/Color.h}

%\ccIsModel

%Concept

%\ccTypes

%\ccNestedType{TYPE}{some nested types}

\ccCreation
\ccCreationVariable{c}  %% choose variable name


\ccConstructor{Color();}
             {creates a color with rgb-value $(0,0,0)$, i.e.\ black.}

\ccConstructor{Color(int red, int green, int blue);}
             {creates a color with rgb-value \ccStyle{(red,green,blue)}.}


\ccOperations

\ccHidden \ccMethod{Color & operator=(const Color &q);}
        {Assignment.}

\ccMethod{bool operator==(const Color &q) const;}
       {Test for equality: Two colors are equal, iff their 
        rgb-values are equal.}
\ccGlue
\ccMethod{bool operator!=(const Color &q) const;}
       {Test for inequality.}

\ccMethod{int red() const;}
       {returns the  red component of \ccVar.}
\ccGlue
\ccMethod{int green() const;}
       {returns the  green component of \ccVar.}
\ccGlue
\ccMethod{int blue() const;}
       {returns the blue component of \ccVar.}

\ccConstants

The following constants are predefined:

\ccThree{const Color &}{WHITE = Color(255, 255, 255);}{}

\ccVariable{const Color BLACK  = Color(0, 0, 0);}{Black.}
\ccGlue
\ccVariable{const Color WHITE  = Color(255, 255, 255);}{White.}
\ccGlue
\ccVariable{const Color RED    = Color(255, 0, 0);}{Red.}
\ccGlue
\ccVariable{const Color GREEN  = Color(0, 255, 0);}{Green.}
\ccGlue
\ccVariable{const Color BLUE   = Color(0, 0, 255);}{Blue.}
\ccGlue
\ccVariable{const Color VIOLET = Color(255, 0, 255);}{Violet.}
\ccGlue
\ccVariable{const Color ORANGE = Color(255, 170, 0);}{Orange.}

\ccSeeAlso

\ccc{CGAL::Window_stream } \\
\ccc{CGAL::PostScript_File_Stream}

%\ccExample

%A short example program.
%Instead of a short program fragment, a full running program can be
%included using the 
%\verb|\ccIncludeExampleCode{examples/Package/Color_prog.C}| 
%macro. The program example would be part of the source code distribution and
%also part of the automatic test suite.

%\begin{ccExampleCode}
%void your_example_code() {
%}
%\end{ccExampleCode}

%%% \ccIncludeExampleCode{examples/Package/Color_prog.C}

\end{ccRefClass}

% +------------------------------------------------------------------------+
%%RefPage: end of main body, begin of footer
% EOF
% +------------------------------------------------------------------------+



\section{Window output}

An object of type \ccStyle{CGAL::Window_stream} is a two-dimensional window for
graphical IO. The input and output operations perform a mapping from
object coordinates to window coordinates.
Actually, \ccStyle{CGAL::Window_stream} is identical to the \leda\ class \ccStyle{leda_window},
if \leda\ is used and to the ported version of the \leda\ window class, if
\leda\ is not used.
Therefore, the rich functionality of \ccStyle{leda_window} is available on
\ccStyle{CGAL::Window_stream} as well, for example panels.
In addition, the class \ccStyle{CGAL::Window_stream} provides input and output 
operators for the classes which are defined in the \cgal\ kernel.
\ccStyle{CGAL::Window_stream} should be initialized with a coordinate frame 
(by calling member function
\ccc{init}) and must be displayed (by calling member function \ccc{display}) before
drawing.

\begin{ccTexOnly}
For further information on \ccStyle{leda_window}, we refer to \cite{mnsu-lum}.
\end{ccTexOnly}
\begin{ccHtmlOnly}
For further information on <i>leda_window</i>, we refer to the 
<A HREF="http://www.mpi-sb.mpg.de/LEDA/MANUAL/MANUAL.html">LEDA manual</A>.
\end{ccHtmlOnly}

Input and output operators for \ccStyle{CGAL::Window_stream} are defined for 
the geometric classes in the \cgal\ kernel.
The coordinates of the geometric objects are in object space. Data are
entered with the left mouse button.

It is important that the window stream header file gets included
{\em after} the inclusion of the header files of geometric classes
that get inserted to or extracted from the window stream. See the following 
small example program.

\begin{verbatim}
#include <CGAL/basic.h>
#include <CGAL/Cartesian.h>
#include <CGAL/Segment_2.h>
#include <CGAL/IO/Window_stream.h>

typedef CGAL::Point_2< CGAL::Cartesian<double> >     Point;
typedef CGAL::Segment_2< CGAL::Cartesian<double> >   Segment;

int main()
{
    Point p(0,1), q(2,2);
    Segment s(p,q);

    CGAL::Window_stream W(100,100);
    W.init(0,10,10);
    CGAL::cgalize( W);
    W.display();

    W << CGAL::RED << s << CGAL::BLACK << p << q ;
    W >> s >> p;
    return 0;
}
\end{verbatim} 

3D objects can be used as well with the \ccc{Window_stream}. The stream output
operator \ccc{<<} projects them to the $xy$-plane. The stream input
operator \ccc{>>} sets the $z$ coordinate to zero.
To get these operators you have to 

\ccInclude{CGAL/IO/window_stream_xy_3.h}


\section{Postscript output}

An object of type \ccStyle{Postscript_file_stream} provides graphical output in PostScript
format.
\ccStyle{Postscript_file_stream} is derived from the \leda\ class \ccStyle{ps_file}.
Therefore, the functionality of \ccStyle{ps_file} is available on \ccStyle{Postscript_file_stream}
as well.
In addition, the class \ccStyle{Postscript_file_stream} provides input and output 
operators for the classes which are defined in the \cgal\ kernel.
The interface of \ccStyle{Postscript_file_stream} is very similar to the interface 
of \ccStyle{leda_window}. 
To increase similarity, some functions have been added with respect to \ccStyle{ps_file}.
\begin{ccTexOnly}
For further information on \ccStyle{ps_file}, we refer to \cite{mnsu-lum}.
\end{ccTexOnly}
\begin{ccHtmlOnly}
For further information on <i>ps_file</i>, we refer to the 
<A HREF="http://www.mpi-sb.mpg.de/LEDA/MANUAL/MANUAL.html">LEDA manual</A>.
\end{ccHtmlOnly}

The output operator to \ccStyle{Postscript_file_stream} is defined for all geometric classes 
in the two-dimensional \cgal\ kernel. There are no input operators of course.

It is important that the postscript file stream header file gets included
{\em after} the inclusion of the header files of geometric classes
that get inserted to the postscript stream file. See the following example program.

\begin{verbatim}
#include <CGAL/Cartesian.h>
#include <CGAL/Segment_2.h>
#include <CGAL/IO/Postscript_file_stream.h>

typedef CGAL::Point_2< CGAL::Cartesian<double> >     Point;
typedef CGAL::Segment_2< CGAL::Cartesian<double> >   Segment;

int main()
{
    Point p(0,1), q(2,2);
    Segment s(p,q);

    CGAL::Postscript_file_stream PS(100,100);
    PS.init(0,10,0);
    CGAL::cgalize( PS);
    PS.display();

    PS << CGAL::RED << s << CGAL::BLACK << p << q ;
    return 0;
}
\end{verbatim} 

