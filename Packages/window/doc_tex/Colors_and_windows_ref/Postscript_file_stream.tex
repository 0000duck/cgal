%\ccSetThreeColumns{Postscript_file_stream}{}{\hspace*{8.5cm}}
%\ccSetTwoColumns{}{\hspace*{8.5cm}}

\begin{ccRefClass} {Postscript_file_stream}

\ccInclude{CGAL/IO/Postscript_file_stream.h}

\ccCreationVariable{PS}

\ccDefinition 
An object of type \ccRefName\ provides graphical output in PostScript.
\ccRefName\ is derived from the \leda\ class \ccStyle{ps_file}.
Therefore, the functionality of \ccStyle{ps_file} is available on \ccRefName\ as well.
In addition, the class \ccRefName\ provides input and output 
operators for the classes which are defined in the \cgal\ kernel.
The interface of \ccRefName\ is very similar to the interface of \ccStyle{leda_window}. 
To increase similarity, some functions have been added with respect to \ccStyle{ps_file}.
\begin{ccTexOnly}
For further information on \ccStyle{ps_file}, we refer to \cite{mnsu-lum}.
\end{ccTexOnly}
\begin{ccHtmlOnly}
For further information on <i>ps_file</i>, we refer to the 
<A HREF="http://www.mpi-sb.mpg.de/LEDA/MANUAL/MANUAL.html">LEDA manual</A>.
\end{ccHtmlOnly}


\ccCreation

\ccConstructor{Postscript_file_stream();}
{creates a postscript file stream. The name of the output file is
\ccc{CGAL_unnamed.ps} .}

\ccConstructor{Postscript_file_stream(const char* filename);}
{creates a postscript file stream. The name of the output file is
\ccc{filename} .}

\ccConstructor{Postscript_file_stream(float w, float h);}
{creates a postscript file stream that draws the output to a box
of physical size \ccStyle{width/40} $\times$ \ccStyle{height/40} cm.
The name of the output file is \ccc{CGAL_unnamed.ps} .}

\ccConstructor{Postscript_file_stream(float w, float h, const char* filename);}
{creates a postscript file stream that draws the output to a box
of physical size \ccStyle{width/40} $\times$ \ccStyle{height/40} cm.
The name of the output file is \ccc{filename} .}


\ccOperations
The drawing region of a postscript file stream is initialized by a part of the 
object space by

\ccMethod{void init(double x0, double x1, double y0); }
%{sets $xmin$ to $x_0$, $xmax$ to $x_1$, and $ymin$ to $y_0$, 
%             the scaling factor $scaling$ to $w / (xmax-xmin)$, and 
%             $ymax$ to $ymin + h \cdot scaling$. Here $w$ and $h$ are the
%             width and height of the drawing section in pixels. }
{initializes the  part of the object space that is mapped to the physical
drawing box to [\ccc{x0},\ccc{x1}] $\times$ [\ccc{y0},\ccc{y0} + 
\ccc{height}$*$\ccc{width}/(\ccc{x1}-\ccc{x0})], where \ccc{width} and \ccc{height} 
are the width and the height of the drawing box in cm.}

The following functions have been added to give \ccRefName\ the same interface
as \ccc{Window_stream} to allow easily replacement of \ccc{Window_stream} by
\ccRefName\ . 

\ccMethod{void display();}
{does nothing}

\ccMethod{void display(int x, int y);}
{does nothing}

\ccMethod{int read_mouse(double& x, double& y);}
{returns 1}

\ccMethod{leda_color set_fg_color(leda_color c);}
{like \ccc{set_color(c).}}

See the \ccAnchor{http://www.mpi-sb.mpg.de/LEDA/MANUAL/MANUAL.html}{LEDA Manual}
\cite{mnsu-lum} for further operations on \ccc{ps_file}.

\def\ccTagRmEigenClassName{\ccFalse}
\ccHeading{Output Operators}

%\ccSetThreeColumns{Postscript_file_stream& }{}{\hspace*{8.5cm}}

The output operator is defined for all geometric classes in the two-dimensional
\cgal\ kernel.

\ccFunction{Postscript_file_stream& operator<<(Postscript_file_stream& PS, const Class& c);}
       {inserts object \ccStyle{c} into the stream \ccVar.}

\ccFunction{Postscript_file_stream& operator<<(Postscript_file_stream& PS, const Color& c);}
       {changes the foreground color for the next objects that
        will be inserted into \ccVar.}

\ccFunction{void cgalize(Postscript_file_stream& ps);}
{sets a few default values for drawing to ps. In particular, node width is
set to 5 and line width to 1.}

There are no input operators of course.

\end{ccRefClass} 
