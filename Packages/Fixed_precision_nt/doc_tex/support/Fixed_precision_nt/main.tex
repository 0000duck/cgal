% // ============================================================================
%//
%// Copyright (c) 1998 The CGAL Consortium
%//
%// This software and related documentation is part of an INTERNAL release
%// of the Computational Geometry Algorithms Library (CGAL). It is not
%// intended for general use.
%//
%// ----------------------------------------------------------------------------
%//
%// release       :
%// release_date  :
%//
%// file          : doc_tex/fixed.tex
%// revision      : $Revision$
%//
%// author(s)     : Olivier Devillers <Olivier.Devillers@sophia.inria.fr>
%//
%// coordinator   : Olivier Devillers <Olivier.Devillers@sophia.inria.fr>
%//
%//============================================================================

\marginparwidth 1.5cm
\def\ccTagChapterRelease{\ccTrue}
\def\ccTagChapterAuthor{\ccTrue}

\chapter{Fixed precision numbers} \label{I1_Chapter_Fixed_precision_nt}

\ccChapterRelease{Revision: , Date: }

\ccChapterAuthor{Olivier Devillers}

\section{Introduction} 

The class \ccStyle{Fixed_precision_nt} provides 24 bits number in fixed point representation.
Basically these numbers are integers in the range
$[-2^{24},2^{24}]$ with a multiplying factor $2^b$.

The multiplying factor $2^b$ has to be initialized by the user
before the construction of the first \ccStyle{Fixed_precision_nt}.

The interest of such a number type is that geometric predicates
can be overloaded to get exact and very efficient predicates.

The drawback is that any \ccStyle{Fixed_precision_nt} is rounded to the nearest
multiple of $2^b$, which yields to a very poor arithmetic.
The idea is to not use the arithmetic on \ccStyle{Fixed_precision_nt}
but only the specialized predicates.

\section{Class description}



\begin{ccClass} {Fixed_precision_nt}
\ccSetTwoColumns{}{\hspace*{8.5cm}}
\ccCreation
\ccCreationVariable{fvar}

\ccInclude{CGAL/Fixed_precision_nt.h}

\ccConstructor{Fixed_precision_nt(double);}
            {Initialization of a variable. The variable is rounded to the
                nearest legal fixed number (i.e. a multiple of $2^b=$
                \ccStyle{Fixed_precision_nt::unit_value()}}


\ccConstructor{Fixed_precision_nt();}
            {Declaration.}

\ccConstructor{Fixed_precision_nt(const Fixed_precision_nt &fval);}
            {Declaration and initialization.}


\ccConstructor{Fixed_precision_nt(i)}
            {Declaration and initialization with an integer.}

\ccSetThreeColumns{XXXXXX}{}{\hspace*{8.5cm}}
\ccOperations
\ccMethod{Fixed_precision_nt & operator=(const Fixed_precision_nt &fval);}
        {Assignment. 
}

The comparison operations 
\ccStyle{==, !=, <, >, <=,} and \ccStyle{>=} are all available. 

\ccFunction{bool is_valid(const Fixed_precision_nt &fval);}
{In case of overflow or division by 0, numbers becomes invalid.
If the precision is changed by usage of 
\ccStyle{Fixed_precision_nt::init()}, already existing numbers may became invalid
if they are non longer multiple of $2^b$.}

\ccFunction{bool is_finite(const Fixed_precision_nt &fval);}
{ \ccClassName do not implement infinite numbers. 
  \ccStyle{is_finite} is identical to \ccStyle{is_valid}.}


\ccFunction{Fixed_precision_nt operator+(const Fixed_precision_nt &fval1, const Fixed_precision_nt &fval2);}
       {rounds the result to nearest legal \ccStyle{ Fixed}. Overflow is possible.}

\ccFunction{Fixed_precision_nt operator-(const Fixed_precision_nt &fval1, const Fixed_precision_nt &fval2);}
       {rounds the result to nearest legal \ccStyle{ Fixed}. Overflow is possible.}

\ccFunction{Fixed_precision_nt operator*(const Fixed_precision_nt &fval1, const Fixed_precision_nt &fval2);}
       {rounds the result to nearest legal \ccStyle{ Fixed}. Overflow is possible (and even probable).}

\ccFunction{Fixed_precision_nt operator-(const Fixed_precision_nt &fval);}
       {rounds the result to nearest legal \ccStyle{ Fixed}.}

\ccMethod{Fixed_precision_nt operator+=(const Fixed_precision_nt &f) const;}
       {rounds the result to nearest legal \ccStyle{ Fixed}.}

\ccMethod{Fixed_precision_nt operator-=(const Fixed_precision_nt &f) const;}
       {rounds the result to nearest legal \ccStyle{ Fixed}.}

\ccMethod{Fixed_precision_nt operator*=(const Fixed_precision_nt &f) const;}
       {rounds the result to nearest legal \ccStyle{ Fixed}.}

\ccMethod{Fixed_precision_nt operator/=(const Fixed_precision_nt &f) const;}
       {rounds the result to nearest legal \ccStyle{ Fixed}. Overflow is possible (and even probable).}

\ccFunction{double to_double(const Fixed_precision_nt &fval);}
         {casts to \ccStyle{double}.}


\ccFunction{Fixed_precision_nt operator/(const Fixed_precision_nt &fval1, const Fixed_precision_nt &fval2);}
       {rounds the result to nearest legal \ccStyle{ Fixed}. Overflow is probable.}



\section{Parametrization routines}

\subsection{Precision initialization}
As announced before, the \ccStyle{Fixed_precision_nt} numbers works in an
interval $[-2^{24+b},2^{24+b}]$ of multiples of $2^b$,
this number $b$ as to be defined before all use of \ccStyle{Fixed_precision_nt}.

\ccSetThreeColumns{Comparison_result}{}{\hspace*{6cm}}

\ccFunction{static bool Fixed_precision_nt::init(float B);}
{$B$ is an upper bound on the data, $b$ is the smallest integer such that 
$|B|\leq 2^b$. The result of the function is false if initialization was
already done, in that case already existing \ccClassName\ may became invalid.}

\ccFunction{static float Fixed_precision_nt::unit_value();}
{returns $2^b$.}

\ccFunction{static float Fixed_precision_nt::upper_bound();}
{returns $2^{24+b}$.}

\subsection{Perturbation scheme}

\ccClassName\ implements perturbation scheme as described by
Alliez, Devillers and Snoeyink
\cite{ads-rdppw-98}.
The perturbation mode can be activate or deactivate for different kinds
of perturbations. The default mode is no perturbation.

\ccFunction{static void Fixed_precision_nt::perturb_incircle();}
{Activate. Incircle test of 4 cocircular points answers degenerate only if
the 4 points are colinear.}

\ccFunction{static void Fixed_precision_nt::unperturb_incircle();}
{Desactivate}

\ccFunction{static bool Fixed_precision_nt::is_perturbed_incircle();}
{returns current mode}

\ccFunction{static void Fixed_precision_nt::perturb_insphere();}
{Activate. Insphere test of 5 cospherical points answers degenerate only if
the 5 points are coplanar.}

\ccFunction{static void Fixed_precision_nt::unperturb_insphere();}
{Desactivate}

\ccFunction{static bool Fixed_precision_nt::is_perturbed_insphere();}
{returns current mode}

\section{Geometric predicates}

Through overloading mechanisms, functions such that
\ccStyle{orientation} for 
\ccStyle{Point_2<Cartesian< Fixed_precision_nt> >}
will correctly call the function below.

\ccSetThreeColumns{Oriented_side}{}{\hspace*{6cm}}

\subsection{Two dimensional}

\ccFunction{Orientation orientationC2(
  Fixed_precision_nt x0, Fixed_precision_nt y0,
  Fixed_precision_nt x1, Fixed_precision_nt y1,
  Fixed_precision_nt x2, Fixed_precision_nt y2);}{}

\ccFunction{Oriented_side side_of_oriented_circleC2 (
  Fixed_precision_nt x0, Fixed_precision_nt y0,
  Fixed_precision_nt x1, Fixed_precision_nt y1,
  Fixed_precision_nt x2, Fixed_precision_nt y2,
  Fixed_precision_nt x3, Fixed_precision_nt y3);}{Perturbation mode can be activated.}

\subsection{Three dimensional}

\ccFunction{Orientation orientationC3(  
  Fixed_precision_nt x0, Fixed_precision_nt y0, Fixed_precision_nt z0,
  Fixed_precision_nt x1, Fixed_precision_nt y1, Fixed_precision_nt z1,
  Fixed_precision_nt x2, Fixed_precision_nt y2, Fixed_precision_nt z2,
  Fixed_precision_nt x3, Fixed_precision_nt y3, Fixed_precision_nt z3);}{}

\ccFunction{Oriented_side side_of_oriented_sphereC3 (
  Fixed_precision_nt x0, Fixed_precision_nt y0, Fixed_precision_nt z0,
  Fixed_precision_nt x1, Fixed_precision_nt y1, Fixed_precision_nt z1,
  Fixed_precision_nt x2, Fixed_precision_nt y2, Fixed_precision_nt z2,
  Fixed_precision_nt x3, Fixed_precision_nt y3, Fixed_precision_nt z3,
  Fixed_precision_nt x4, Fixed_precision_nt y4, Fixed_precision_nt z4);}
{Perturbation mode can be activated.}

\end{ccClass} 
