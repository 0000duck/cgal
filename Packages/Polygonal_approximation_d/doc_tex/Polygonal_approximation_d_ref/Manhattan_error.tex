% +------------------------------------------------------------------------+
% | Reference manual page: Manhattan_error.tex
% +------------------------------------------------------------------------+
% | 4.10.2005   Ovidiu Grigore
% | Package: Polygonal_approximation_d
% | 
% +------------------------------------------------------------------------+

\begin{ccRefClass}{Manhattan_error<Kernel, DistCumul, DistM, Dim>}

\ccDefinition
  
The class \ccRefName\ serves as a traits class for the polyline
simplification functions.

The assessment of the approximation error between an
input polyline and the segment that links its ends is based on the
manhattan distance measured between each point of the polyline and the
approximating segment.

Several approaches of this error traits class are available through
template parameter specializations. The \ccc{DistCumul} template parameter
defines the distances cumulating method used in error assessment,
which can be either the maximum value (\ccc{Max_tag}) or the sum of the distances
(\ccc{Sum_tag})
measured between each polyline point and the approximation segment.
The \ccc{DistM} template parameter defines the distances measurement method used
in error assessment and there can be distances measured either to the
supporting line of the approximation segment (\ccc{Line_tag}) or to the
segment itself (\ccc{Segment_tag}). The \ccc{Dim} template parameter defines the working space dimension
and it is either 2 or 3.



\ccInclude{CGAL/Manhattan_error.h}

\ccIsModel

\ccRefConceptPage{Error}


\ccTypes

\ccTypedef{typedef Kernel::FT FT;}{The number type.}

\ccOperations

\ccMemberFunction{template <class InputIterator>
                  FT operator()( InputIterator begin, 
	                         InputIterator beyond) ; }
{This operator computes the error obtained when the polyline, given as the iterator range \ccc{begin}, \ccc{beyond}, 
is approximated by a line segment. The value type of \ccc{InputIterator} must be \ccc{Kernel::Point_2}, for \ccc{Dim=2},
and\ccc{Kernel::Point_3}, for \ccc{Dim=2}.}

\ccMemberFunction{template <class InputIterator>
                  FT operator()( InputIterator begin, 
	                         InputIterator beyond, 
	                         InputIterator &split_pt ); }
{This operator computes the error obtained when the polyline, given as the iterator range \ccc{begin}, \ccc{beyond},
is approximated by a line segment. Through the parameter \ccc{split_pt}, the operator returns the 
splitting point of the input polyline.  The value type of \ccc{InputIterator} must be \ccc{Kernel::Point_2}, for \ccc{Dim=2},
and\ccc{Kernel::Point_3}, for \ccc{Dim=2}.
If the error is computed as the maximum of the distances between each
polyline point and the line segment, the splitting point is the furthest
point of the polyline from the approximation line segment. In the sum of
the distances error assessment the splitting point is the polyline point
that divides the error in two equal parts. }


\ccSeeAlso

\ccRefIdfierPage{Error}  

\end{ccRefClass}

% EOF
