% +------------------------------------------------------------------------+
% | Reference manual page: simplify_polyline_bnp.tex
% +------------------------------------------------------------------------+
% | 03.10.2005   Ovidiu Grigore
% | Package: Polygonal_approximation_d
% | 
% +------------------------------------------------------------------------+


\begin{ccRefFunction}{simplify_polyline_bound_error}

\ccDefinition
  
The function \ccRefName\ simplifies a
sequence of points that forms a simple polyline or polygon. A
function argument specifies the upper bound on the error of the
resulting polyline.

The template parameter \ccc{Method} may be one of
\ccc{Dynamic_programming}, \ccc{Graph_search},
\ccc{Convex_hull_graph_search}, \ccc{Iterative_graph_search}, or
\ccc{Recursive_split}.  The first four methods are optimal, that is
the obtained result is a minimum error approximation of the input
polyline.

The template parameter \ccc{ErrorAssessment} must be \ccc{Local} or \ccc{Global}.
In the local approach the error evaluates the local
distortion between the input polyline and the simplified polyline,
being computed as the maximum value of the local errors measured
between each approximation line segment and its corresponding input
polyline fragment.
In the global approach the error evaluates the
global distortion between the input polyline and the simplified
polyline, being computed as the sum of all local errors measured
between each approximation line segment and its corresponding input
polyline fragment.

The \ccc{Convex_hull_graph_search} method only supports \ccc{Local},
the \ccc{Iterative_graph_search} method olny \ccc{Global}.


\ccInclude{CGAL/simplify_polyline.h}

\ccFunction{template<class Method, class ErrorAssessment, class Error, class InputIterator,class OutputIterator>
OutputIterator simplify_polyline_bound_error(InputIterator begin, 
                                             InputIterator beyond, 
                                             std::size_t& n_pt, 
                                             Error::FT error_bound, 
                                             OutputIterator result);}
            {simplifies the polyline given by an iterator range and writes the simplified
             polyline with maximal error \ccc{error_bound} into \ccc{result}. After the function call 
             the reference parameter \ccc{n_pt} holds the number of points of
             the simplified polyline.}



\ccHeading{Requirements}
\begin{enumerate}
   \item    The value type of \ccc{InputIterator} and \ccc{OutputIterator}
            are equivalent to \ccc{Error::Point}.
   \item    \ccc{Error} must be a model of the concept \ccc{Error}. For
            the method \ccc{RecursiveSplit} it must provide the operator
            which computes a split point.  For the methods \ccc{Convex_hull_graph_search}
            and \ccc{Iterative_graph_search} the traits classes must be
            \ccc{Convex_hull_graph_search_error_2} and \ccc{Iterative_graph_search_error_2},
	    respectively.
\end{enumerate}




\ccSeeAlso

\ccRefIdfierPage{CGAL::simplify_polyline_bound_number_of_points} \\
\ccRefIdfierPage{CGAL::Error} 

\ccImplementation


In the \ccc{Dynamic_programming} method \cite{cgal:gt-dpasrcp-93}...

In the \ccc{Convex_hull_graph_search} method, the maximum Euclidean
distances are measured using the method of Toussaint \cite{t-capcp-85} based on the
on-line convex hull algorithm. With these distances, it is built the
graph \ccc{G(C,error_bound)} that contains only the arcs corresponding to those
segments that approximate the polyline portion with an error less then
the input \ccc{error_bound}. The optimal polygonal approximation that
minimizes the number of the vertices corresponds to the shortest path
linking the starting and the ending nodes in \ccc{G(C,error_bound)}. It can be found
in linear time using the breadth-first search procedure. Using this
method, the solution can be found in $O(n^2 \log n)$ time.

In the \ccc{Iterative_graph_search} method, the Euclidean distances are measured
using the incremental technique introduced by Perez and Vidal
\cite{cgal:pv-opadc-94}. With these distances, it is built the graph \cccG(C,eps)},
containing only the arcs corresponding to those segments that
approximate the polyline portion with an error less then the input
threshold \ccc{error_bound}. The optimal polygonal approximation that minimizes the
number of the vertices corresponds to the shortest path linking the
starting and the ending nodes in \ccc{G(C,eps)} and that cumulates along it an
error less than the input threshold. Using this method, the solution
can be found in $O(n^3)$ time.


\end{ccRefFunction}

% +------------------------------------------------------------------------+
%%RefPage: end of main body, begin of footer
% EOF
% +------------------------------------------------------------------------+

