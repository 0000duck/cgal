% +------------------------------------------------------------------------+
% | Reference manual page: DistTraits_2.tex
% +------------------------------------------------------------------------+
% | 4.10.2005   Ovidiu Grigore
% | Package: Polygonal_apprcximation
% | 
% +------------------------------------------------------------------------+


\begin{ccRefFunctionObjectConcept}{Error}


\ccDefinition
  

The concept \ccRefName\ defines the complete set of primitives required in these
polyline simplification functions.  The specific subset of these primitives 
required by each function is specified with each function.

\ccTypes

\ccSetTwoColumns{Point}{}
\ccNestedType{FT}{The number type.}
\ccTypedef{Point}{The point type. It must offer member functions \ccc{x()} and \ccc{y()}, 
                  if it is a 2 dimensional point, and additionally \ccc{z()} if it is a 3
                  dimensional point.}
\ccCreation
\ccCreationVariable{traits}  %% choose variable name

\ccConstructor{Error_2(Error& t);}{Copy constructor.}


\ccOperations

\ccMemberFunction{template <class InputIterator>
                  FT operator()( InputIterator begin, 
	                         InputIterator beyond) ; }
{This operator computes the error obtained when the polyline, given as the iterator range \ccc{begin}, \ccc{beyond}, 
is approximated by a line segment.}

\ccMemberFunction{template <class InputIterator>
                  FT operator()( InputIterator begin, 
	                         InputIterator beyond, 
	                         InputIterator &split_pt ); }
{This operator computes the error obtained when the polyline, given as the iterator range \ccc{[begin, beyond)},
is approximated by a line segment. Through the parameter \ccc{split_pt}, the operator returns the 
splitting point of the input polyline.}




\ccHasModels

\ccRefIdfierPage{CGAL::Squared_euclidean_error<K, DistCumul, DistM, Dim>}\\ 
\ccRefIdfierPage{CGAL::Bounded_volume_error_2<K>}\\
\ccRefIdfierPage{CGAL::Vertical_error_2<K>} \\
\ccRefIdfierPage{CGAL::Incremental_sum_squared_euclidean_error_2<K>} \\
\ccRefIdfierPage{CGAL::Incremental_sum_squared_vertical_error_2<K>} 

\end{ccRefFunctionObjectConcept}

% +------------------------------------------------------------------------+
%%RefPage: end of main body, begin of footer
% EOF
% +------------------------------------------------------------------------+

