\begin{ccRefClass}{Delaunay_mesher_2<CDT, Criteria>}

This class implements a 2D mesh generator.

\ccParameters
The template parameter \ccc{CDT} should be a model of the concept
\ccc{ConstrainedDelaunayTriangulation_2}, and its face should be a model of
the concept \ccc{MeshFaceBase_2}.

The geometric traits class of the instance of \ccc{CDT} has to be
a model of the concept \ccc{DelaunayMeshTraits_2}.

The template parameter \ccc{Criteria} should be a model of the concept
\ccc{MeshingCriteria_2}. This traits class defines the shape and size
criteria on the triangles of the mesh. \ccc{Criteria::Face_handle} has to
be the same as \ccc{CDT::Face_handle}.

\ccHeading{Using this class}

The constructor of the class \ccRefName{} takes a reference to a \ccc{CDT}
as an argument. A call to the refinement method \ccc{refine_mesh()} will
refine the constrained Delaunay triangulation into a mesh satisfying the
size and shape criteria specified in the traits class. Note that if, during
the life of the \ccRefName\ object, the triangulation is externally
modified, any further call to its member methods may crash. Considere
constructing a new \ccRefName\ object if the triangulation has been
modified.

\ccHeading{Meshing domain}
The domain to be mesh is defined by the constrained edges and a set
of seed points. See the user manual, for a detailed description of this
domain.

\ccInclude{CGAL/Delaunay_mesher_2.h}

\ccTypes 
\ccTypedef{typedef CDT::Geom_traits Geom_traits;}{the geometric traits class.}
\ccNestedType{Seeds_iterator}{const iterator over defined seeds. Its
  value type is \ccc{Geom_traits::Point_2}}

\ccCreation
\ccCreationVariable{mesher}

\ccConstructor{Delaunay_mesher_2(CDT& t, Criteria criteria = Criteria());}
{Create a new mesher, working on \ccc{t}, with meshing criteria
  \ccc{criteria}.}

\ccHeading{Seeds functions}

The following functions are used to define seeds.

\ccMethod{void  clear_seeds ();}{ Sets seeds to the empty set. All
  finite connected components of the constrained triangulation will be 
  refined.}

\ccMethod{template<class InputIterator>
          void set_seeds(InputIterator begin, InputIterator end,
                         const bool mark=false);}                   
                       { Sets seeds to the sequence [\ccc{begin},
                         \ccc{end}]. If \ccc{mark=true}, the mesh domain
                         is the union of the bounded connected
                         components including at least one seed. If
                         \ccc{mark=false}, the domain is the union of
                         the bounded components including no seed. Note
                         that the unbounded component of the plane is
                         never meshed.
                         \ccPrecond The \ccc{value_type} of
                         \ccc{begin} and \ccc{end} is
                         \ccc{Geom_traits::Point_2}.}

These ones give the sequences of seeds.

\ccMethod{Seeds_const_iterator  seeds_begin () const;}
{ Start of the seeds sequence. }
\ccGlue
\ccMethod{Seeds_const_iterator  seeds_end () const;}
{ Past the end of the seeds sequence. }

\ccHeading{Meshing methods}

\ccMethod{void refine_mesh(); }
{ Refines the constrained Delaunay triangulation into a mesh
  satisfying the criteria defined by the traits.
}

\ccMethod{ void set_criteria(Criteria criteria);}
{ Assigns \ccc{criteria} to the criteria traits object. }

\ccMethod{ const Criteria& get_criteria();}
{ Returns a const reference to the criteria traits object. }

\begin{ccAdvanced}
  The function \ccc{set_} scans all faces to recalculate the list of
  \emph{bad faces}, that are faces not conforming to the meshing criteria.
  This function actually has an optional argument that permits to prevent
  this recalculation. The filling of the list of bad faces can then be done
  by a call to \ccc{set_bad_faces}.
  
  \ccMethod{ void set_criteria(Criteria criteria, bool
    recalculate_bad_faces);}
  { Assigns \ccc{criteria} to the criteria traits object. If
    \ccc{recalculate_bad_faces} is \ccc{false}, the list of bad faces is
    let empty and the function \ccc{set_bad_faces} should be called before
    \ccc{refine_mesh}.}

\ccMethod{template <class InputIterator>
  void set_bad_faces(InputIterator begin,
                     InputIterator end);}
                   { This method permits to set the list of bad triangles
                     directly, from the sequence [begin, end], so that the
                     algorithm will not scan the whole set of triangles to
                     find bad ones. To use if there is a non-naive way to
                     find bad triangles.  \ccPrecond The \ccc{value_type}
                     of \ccc{begin} and \ccc{end} is \ccc{Face_handle}.}

\end{ccAdvanced}


\begin{ccAdvanced}
\ccHeading{Step by step operations}

The \ccRefName{} class allows, for debugging or demos, to play the
meshing algorithm step by step, using the following methods.

\ccMethod{ void init(); }
         { This method must be called just before the first
             call to the following step by step refinement method,
             that is when all vertices and constrained edges have been
             inserted into the constrained Delaunay triangulation. It
             must be called again before any subsequent calls to the
             step by step refinement method if new vertices or constrained
             edges have been inserted since the last call.}

\ccMethod{bool is_refinement_done(); }
         {  Tests if the step by step refinement algorithm is done. }

\ccMethod{ bool step_by_step_refine_mesh(); }
         { Applies one step of the algorithm. At any single step of the
         algorithm, 
         at most one point is inserted. The algorithm sometimes does not
         insert a point at some steps. Anyway, internal datas of the
         \ccRefName{} object are modified at each step, until
         \ccc{mesher.is_refinement_done()}. If the algorithm is finished,
         the function returns \ccc{false}. If not, it returns \ccc{true}.}

\end{ccAdvanced}

\end{ccRefClass}

%%% For emacs/AucTeX:
%%% Local Variables: ***
%%% mode:latex ***
%%% TeX-master: "../Mesh_2.tex"  ***
%%% End: ***
