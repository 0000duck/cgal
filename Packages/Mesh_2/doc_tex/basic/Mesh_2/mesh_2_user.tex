\chapter{2D conforming triangulations and meshes}
\label{user_chapter_2D_Meshes}

\minitoc

\section{Introduction}
\label{sec:Mesh_2_introduction}

This package implements Shewchuk's algorithm to construct conforming
triangulations and 2D meshes.

Conforming triangulations will be described in the
section~\ref{sec:Mesh_2_conforming_triangulation} and meshes in the
section~\ref{sec:Mesh_2_meshes}.


%\section{Definitions}
%\label{sec:Mesh_2_definitions}

%\begin{description}

%\item[Conforming triangulation] A conforming triangulation is a refinement
%  of a constrained triangulation, obtained by inserting points (named
%  \emph{Steiner points}) on constrained segments, so that the resulting
%  triangulation is a \emph{Delaunay} or a \emph{Gabriel}
%  triangulation. Delaunay triangulations are defined in the \cgal\ user
%  manual of \ccc{Triangulation_2} package. A Gabriel triangulation of a
%  sets of points and constrained segments is a constrained Delaunay
%  triangulation with the extra following \emph{Gabriel property}: the
%  \emph{diametral circle} of each constrained segments contains no point in
%  its interior. The Gabriel property is stronger that the Delaunay property:
%  a conforming Gabriel triangulation is also a conforming Delaunay
%  triangulation.
%\item[Quality mesh] In this package, a quality mesh is a contrained
%  Delaunay triangulation of a domain containing only triangles whose shapes
%  and sizes satisfy several criterias.
%\end{description}


\section{Conforming triangulations}
\label{sec:Mesh_2_conforming_triangulation}

\subsection{Definition}
\label{sec:Mesh_2_conforming_defition}

A constrained Delaunay triangulation is said to be a \emph{conforming
  Delaunay triangulation} if every constrained edge is a Delaunay
edge, that is appears in the Delaunay triangulation of the set of
vertices. Thus a conforming Delaunay triangulation is a Delaunay
triangulation where some edges are marked as constrained edges.

A constrained Delaunay triangulation is said to be a \emph{conforming
  Gabriel triangulation} if every constrained edge is a Gabriel edge,
meaning that its diametral circle includes no vertex of the
triangulation in its interior. Observe that each Gabriel edge is a
Delaunay edge, and then conforming Gabriel triangulations are
conforming Delaunay triangulations.

Any contrained Delaunay triangulation can be refined into a conforming
Delaunay or conforming Gabriel triangulation by adding vertices,
called \emph{Steiner vertices}, on constrained edges until they are
cut into subconstraints small enough to be Delaunay or Gabriel edges.


\subsection{Building conforming triangulations}
\label{sec:Mesh_2_building_conforming}

Conforming triangulations can be obtained by two global functions:
\ccc{template<class CDT> void make_conforming_Delaunay_2 (CDT& t)} and
\ccc{template<class CDT> void make_conforming_Gabriel_2 (CDT& t)}. The
template parameter \ccc{CDT} must be a \cgal\ constrained triangulation. It
can be for example a \ccc{Contrained_triangulation_2}, or a
\ccc{Constrained_triangulation_plus_2} or a \ccc{Triangulation_hierarchy_2}
templated by a constrained triangulation.

The triangulation \ccc{t} is passed by reference and is made conforming
Delaunay or Gabriel by modifying it. If you want to keep the original
triangulation, please make a copy of it.

In other to test the Delaunay or the Gabriel property, and to construct
Steiner points, the geometric traits class of \ccc{CDT} has to be a model of
\ccc{ConformingDelaunayTriangulationTraits_2}. It can be, for example, any
\cgal\ kernel.

\subsection{Example: making a triangulation conforming Delaunay and then
  conforming Gabriel}
\label{sec:Mesh_2_example_making_conforming}

This example inserts several segments in a contrained triangulation, mades
it conforming Delaunay, and then conforming Gabriel. At each step, the
number of points is printed.

\ccIncludeExampleCode{Mesh_2/conform.C}

\section{Quality meshes}
\label{sec:Mesh_2_meshes}

Quality meshes can be obtained by a global function \ccc{template<class CDT>
 void refine_Delaunay_mesh_2 (CDT &t, typename CDT::Geom_traits gt)}. It can
 also be obtained by using the class \ccc{Delaunay_mesh_2<CDT>}. In both
 cases, the template parameter \ccc{CDT} has to be a \cgal\ constrained
 triangulation. The class \ccc{Delaunay_mesh_2<CDT>} derives from \ccc{CDT}
 and has several member functions to define the domain
 (see~\ref{sec:Mesh_2_domains}) and mesh it.

\subsection{Geometric traits class}
\label{sec:Mesh_2_geomtraits}

The geometric traits class of the template parameter \ccc{CDT} should not
only provide geometric objects and predicates for the triangulation, as
usual, but also define the geometric constraints that the triangles have to
satisfy. That is why it has to be a model of
\ccc{DelaunayMeshTraits_2}.

\subsection{Domains}
\label{sec:Mesh_2_domains}

Domains that can be meshed are \emph{planar straight line graphes}
(PSLG), which are sets of of vertices and segments such that all
endpoints of every segments are in the set and that segments intersect
only at end-points. Such a domain can represented by a \cgal\
constrained triangulation, whose constrained segment are the segments
of the domain.

By default, the domain to be meshed is the whole plane but the connected
component of the infinite vertex. If the constrained triangulation is not
bounded by a polyline of constrained segments, this domain can be empty and
the algorithm will not do anything.

However, if the class \ccc{Delaunay_mesh_2<CDT>} is used, the domain to be
meshed can be defined more precisely, by setting \ccc{seeds}. Seeds are
a set of points, that are not in the set of points of the
triangulation. The member function \ccc{template<class InputIterator> void
  set_seeds(InputIterator b, InputIterator e, bool mark = false)} sets a
list of seeds and a boolean marker. This marker tells if the seeds are in
the domain or not, and the connected component of each seed is marked the
same way. Anyway, the connected component of the infinite vertex is always
marked as the exterior of the domain.

\subsection{Shape and size criteria}

The shape criteria on triangles should be that the circumradius is
lower than a bound $B$ times the shortest edge of the triangle. This is
equivalent to say that the minimum angle of the triangle is greater
than $\arcsin{\frac{1}{2B}}$. If not angles are smaller than $\theta$,
then no angles are greater than $\pi - \theta$. Unfortunalty, the
algorithm will terminate at all time only if $B \ge \sqrt{2}$. This
package cannot, for the moment, insure that angles of meshes are
greater than $20.7$~degres. The size criteria can be any criteria
that tends to prefere small triangles. Both types of criterias are
defined in a nested type \ccc{Is_bad} of the geometric traits class.

\subsection{Garanties on the resulting mesh}

If all angles between contrained segments of the initial triangulation are
greater than $60$~degres, the criterias on angles and size are garanted to
be fulfilled.

If some input incident segments forme an angle smaller than $60$~degres,
these segments formed a \textit{cluster}. Near clusters, the algorithm
cannot garanty the shape criterias. See~\cite{s-mgdsa-00} for details.

\subsection{Example using the global function and shape and size default
  criterias}

The following example inserts several segments in a contrained
triangulation, then mesh it using the global function
\ccc{refine_Delaunay_mesh_2} and the default traits class.

\ccIncludeExampleCode{Mesh_2/mesh_global.C}

\subsection{Example using the class \ccc{Delaunay_mesh_2<CDT>} and several
  criterias}

This example uses the class \ccc{Delaunay_mesh_2<CDT>} in order to refine
the mesh with new criterias, after the first meshing. One can use
\ccc{refine_Delaunay_mesh_2} twice, with different geometric traits class,
but it is less efficient, because some internal structures needed by the
algorithm are calculated twice.

\ccIncludeExampleCode{Mesh_2/mesh_class.C}

%%% For emacs/AucTeX:
%%% Local Variables: ***
%%% mode:latex ***
%%% TeX-master: "user_manual.tex"  ***
%%% End: ***
