\chapter{2D Conforming Triangulations\\ and Meshes}
\label{user_chapter_2D_Meshes}

\minitoc

This package implements Shewchuk's algorithm to construct conforming
triangulations and 2D meshes.  Conforming triangulations will be
described in the section~\ref{sec:Mesh_2_conforming_triangulation} and
meshes in the section~\ref{sec:Mesh_2_meshes}.

\section{Conforming triangulations}
\label{sec:Mesh_2_conforming_triangulation}

\subsection{Definitions}
\label{sec:Mesh_2_conforming_definitions}

A constrained Delaunay triangulation is said to be a \emph{conforming
  Delaunay triangulation} if every constrained edge is a Delaunay
edge, where we call {\em Delaunay edge} an edge that would appear in
the (unconstrained) Delaunay triangulation of the set of vertices.
Because any edge in a constrained Delaunay triangulation is either a
Delaunay edge or a constrained edge, a conforming Delaunay
triangulation is in fact a Delaunay triangulation. The only difference
is that some of the edges are marked as constrained edges.

A constrained Delaunay triangulation is said to be a \emph{conforming
  Gabriel triangulation} if every constrained edge is a Gabriel edge,
meaning that its diametral circle includes no vertex of the
triangulation in its interior. The Gabriel property is stronger that
the Delaunay property and each Gabriel edge is a Delaunay edge. Thus
conforming Gabriel triangulations are conforming Delaunay
triangulations.

Any constrained Delaunay triangulation can be refined into a
conforming Delaunay triangulation or a conforming Gabriel
triangulation by adding vertices, called \emph{Steiner vertices}, on
constrained edges until they are cut into subconstraints small enough
to be Delaunay or Gabriel edges.

\subsection{Building conforming triangulations}
\label{sec:Mesh_2_building_conforming}

Constrained Delaunay triangulations can be refined into
conforming triangulations 
by two global functions: \\
\ccc{template<class CDT> void make_conforming_Delaunay_2 (CDT& t)} \\
\ccc{template<class CDT> void make_conforming_Gabriel_2 (CDT& t)}. 

In both cases, the template parameter \ccc{CDT} must be instantiated
by a constrained Delaunay triangulation class.  Such a class can be
either a plain constrained Delaunay triangulation
(\ccc{Constrained_Delaunay_triangulation_2<Gt, Tds>}) or a derived
class such as \ccc{Constrained_triangulation_plus_2<CDT2>} or
\ccc{Triangulation_hierarchy_2<CDT2>} where \ccc{CDT2} is
\ccc{Constrained_Delaunay_triangulation_2<Gt, Tds>}.

There are some requirements on the geometric traits of the constrained
Delaunay triangulation used to instantiate the parameter \ccc{CDT}.
In case of \ccc{make_conforming_Delaunay_2 (CDT& t)} the geometric
traits has to be a model of the concept
\ccc{ConformingDelaunayTriangulationTraits_2} which refines the traits
\ccc{ConstrainedDelaunayTriangulationTraits_2}.  In case of
\ccc{make_conforming_Gabriel_2 (CDT& t)}, the geometric traits has to
be a model of the concept \ccc{ConformingGabrielTriangulationTraits_2}
which further refines \ccc{ConformingDelaunayTriangulationTraits_2}.

The constrained Delaunay triangulation \ccc{t} is passed by reference
and is refined into a conforming Delaunay triangulation or a
conforming Gabriel triangulation by adding vertices, that is the
triangulation is modified. If you want to keep the original
triangulation, please make a copy of it.

The advanced user can also use the class
\ccc{Conforming_Delaunay_triangulation_2<CDT>} to refine a constrained
Delaunay triangulation into a conforming Delaunay or a conforming
Gabriel triangulation.

\subsection{Example: making a triangulation conforming Delaunay and then
  conforming Gabriel}
\label{sec:Mesh_2_example_making_conforming}

This example inserts several segments in a constrained Delaunay
triangulation, makes it conforming Delaunay, and then conforming
Gabriel. At each step, the number of vertices of the triangulation is
printed.

\ccIncludeExampleCode{Mesh_2/conform.C}

\section{Meshes}
\label{sec:Mesh_2_meshes}


\subsection{Definition}
\label{sec:Mesh_2_meshes_definition}

A mesh is a partition of a given domain into simplices, whose shapes
and sizes satisfy several criteria.

The domain to be mesh is bounded and can include internal constraints
to be respected.  Such a domain is defined by a \emph{planar straight
  line graphs}, PSLG for short, and a set of seed points.

A PSLG is a one dimensional simplicial complex, that
is a set of vertices and segments such that : \\
- the endpoints of any segment in the set are vertices of the set, \\
- two segments in the set are either disjoint or share a vertex of the 
set. \\
The segments of the PSLG described the boundaries and the internal
constraints of the domain.

The PSLG divides the plan into several connected components.  The
domain is usually defined as the union of the bounded connected
components including the seed points.  Conversely, the seed points can
be used to mark the connected components which are outside the domain.
\new{If the set of seeds is empty, the domain is the union of the
  bounded connected components.}

\subsection{Shape and size criteria}
\label{sec:Mesh_2_criteria}

The shape criteria on triangles is a lower bound $B$ on the ratio
between the circumradius and the shortest edge length.  Such a bound
implies a lower bound of $\arcsin{\frac{1}{2B}}$ on the minimum angle
of the triangle and an upper bound of $\pi - 2* \arcsin{\frac{1}{2B}}$
on the maximum angle.  Unfortunately, the termination of the algorithm
is guaranteed only if $B \ge \sqrt{2}$ which corresponds to a lower
bound of $20.7$~degrees on the angles.

The size criteria can be any criteria that tends to prefer small
triangles.  The size bound can be varying over the domain.

Both types of criteria are defined
in a nested type \ccc{Is_bad} of the geometric traits class.

\subsection{The meshing algorithm}

The input to a meshing problem is a PSLG, a set of seeds
describing the domain to be meshed and a set of valid size and shape
criteria.  The algorithm implemented in this package starts with a
constrained Delaunay triangulation of the input PSLG and produces a
mesh using the Delaunay refinement method.

If all angles between incident segments of the input PSLG
are greater than $60$~degrees, and if the bound on the
circumradius/edge ratio is greater than $\sqrt{2}$
the algorithm is guaranteed to end up with a mesh
satisfying the size and shape criteria.

If some input angles are smaller than $60$~degrees, the algorithm will
end up with a mesh in which some triangles near small input angles
violate the criteria.  This is unavoidable. Indeed small angles formed
by input segments cannot be suppressed. Furthermore, it has been
proved (\cite{s-mgdsa-00}), that some domains with small input angles
cannot be meshed with angles even smaller that the small input angles.
Note that if the domain is a polygonal region, the resulting mesh will
satisfy size and shape criteria except for the small input angles.
Note also that though it is not guaranteed, the algorithm may succeed
in achieving meshes with a lower angle bound greater than
$20.7$~degrees.

\subsection{Building meshes}
\label{sec:Mesh_2_building_meshes}

In this package, meshes are obtained from
constrained Delaunay triangulation by calling the global function \\
\ccc{template<class CDT> void refine_Delaunay_mesh_2 (CDT &t, typename
  CDT::Geom_traits gt)}. \\
They can also be obtained by using the class
\ccc{Delaunay_mesh_2<CDT>} that derives from \ccc{CDT}.  In both
cases, the template parameter \ccc{CDT} must be instantiated by a
constrained Delaunay triangulation class.  Such a class can be either
a plain constrained Delaunay triangulation
(\ccc{Constrained_Delaunay_triangulation_2<Gt, Tds>} ) or a derived
class such as \ccc{Constrained_triangulation_plus_2<CDT2>} or
\ccc{Triangulation_hierarchy_2<CDT2>} where \ccc{CDT2} is a
\ccc{Constrained_Delaunay_triangulation_2<Gt, Tds>}.

The geometric traits class of the instance of \ccc{CDT} has to be a
model of the concept \ccc{DelaunayMeshTraits_2}.  \new{This concept}
refines the concept \ccc{ConformingGabrielTriangulationTraits_2}
adding the geometric criteria that the triangles have to satisfy.
\cgal\ provides models for this concept such as:
\begin{itemize}
\item \ccc{Delaunay_mesh_traits_2<K>}, that defines a shape criteria
  that bounds the minimum angle of triangles, 
\item \ccc{Delaunay_mesh_size_traits<K>}, that adds to the previous one a
  bound on the maximum edge length.
\end{itemize}

The class \ccc{Delaunay_mesh_2<CDT>} derives from \ccc{CDT} and has
several member functions to define the domain and mesh it. See examples
and the reference manual for details.  Note that the insertion of
vertices and constraints is not overwritten, thus any insertion will
break the size and shape guarantee of the mesh until a meshing
function is called again.

\subsection{Example using the global function}

\new{The following example inserts several segments in a constrained
  triangulation, then mesh it using the global function
  \ccc{refine_Delaunay_mesh_2}. The size and shape criteria are the
  defaults provided by the traits class
  \ccc{Delaunay_mesh_traits_2<K>}. No seeds are given, meaning that
  the mesh domain covers all the plan except the unbounded component.}

\ccIncludeExampleCode{Mesh_2/mesh_global.C}

\subsection{Example using the class \ccc{Delaunay_mesh_2<CDT>}}

\new{This example uses the class \ccc{Delaunay_mesh_2<CDT>} and calls
  the \ccc{refine_mesh()} member function twice changing the size and
  shape criteria inbetween. In such a case, using twice the global
  function \ccc{refine_Delaunay_mesh_2} would be less efficient,
  because some internal structures needed by the algorithm would be
  calculated twice.}

\ccIncludeExampleCode{Mesh_2/mesh_class.C}

\subsection{Example using seeds}

\new{Like the previous one, this example uses the class
  \ccc{Delaunay_mesh_2<CDT>} but defines a domain by using one seed.
  The size and shape criteria are the defaults provided by the traits
  class \ccc{Delaunay_mesh_traits_2<K>}. }

\ccIncludeExampleCode{Mesh_2/mesh_global_with_seeds.C}

%%% For emacs/AucTeX:
%%% Local Variables: ***
%%% mode:latex ***
%%% TeX-master: "user_manual.tex"  ***
%%% End: ***
