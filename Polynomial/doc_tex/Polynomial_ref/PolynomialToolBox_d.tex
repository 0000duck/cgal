\begin{ccRefConcept}{PolynomialToolBox_d}

\ccDefinition
A model of \ccc{PolynomialToolBox_d} is associated to an type 
\ccc{Polynomial_d}, representing a multivariate polynomial
\footnote{Univariate polynomials are not excluded by this concept.}. 
The number of variables is denoted as the dimension $d$ of the polynomial,
it is arbitrary but fixed for a certain model of this concept.  

\ccRefines
\ccc{PolynomialTraits_d}

%\ccConstants
%\ccTypes

\ccHeading{Functors}

In case a functor is not provided it is set to \ccc{CGAL::Null_functor}.
%,e.g., \ccc{Sign_at} if \ccc{Innermost_coefficient_type} is not \ccc{RealEmbeddable}. 
\ccSetTwoColumns{xxxxxxxxxxxxxxxxxxxxxxxxxxxxxxxxxxxxxxxxxxxxx}{}

\ccNestedType{Univariate_content}{ 
        In case \ccc{PolynomialTraits_d::Coefficient_type} is {\bf not} a model of 
        \ccc{UniqueFactorizationDomain}, this is \ccc{CGAL::Null_type}, otherwise this is 
        a model of \ccc{PolynomialTraits_d::UnivariateContent}.}
%\begin{ccAdvanced}
\ccNestedType{Multivariate_content}{ 
        In case \ccc{PolynomialTraits_d::Innermost_coefficient_type} is {\bf not} 
        a model of \ccc{UniqueFactorizationDomain}, this is \ccc{CGAL::Null_type}, 
        otherwise this is a model of 
        \ccc{PolynomialTraits_d::MultivariateContent}.}
%\end{ccAdvanced}

%Manipulation
\ccNestedType{Shift}{ A model of \ccc{PolynomialTraits_d::Shift}.}\ccGlue
\ccNestedType{Negate}{ A model of \ccc{PolynomialTraits_d::Negate}.}\ccGlue
\ccNestedType{Invert}{ A model of \ccc{PolynomialTraits_d::Invert}.}

\ccNestedType{Translate}{ A model of \ccc{PolynomialTraits_d::Translate}.}\ccGlue
\ccNestedType{Translate_homogeneous}
{ A model of \ccc{PolynomialTraits_d::TranslateHomogeneous}.}

\ccNestedType{Scale}{ A model of \ccc{PolynomialTraits_d::Scale}.}\ccGlue
\ccNestedType{Scale_homogeneous}
{ A model of \ccc{PolynomialTraits_d::ScaleHomogeneous}.}

%\begin{ccAdvanced}
%\ccNestedType{Scale_up}{ A model of \ccc{PolynomialTraits_d::ScaleUp, return $p(a*x)$}.}
%\ccNestedType{Scale_down}{ A model of \ccc{PolynomialTraits_d::ScaleDown, return $b^{degree}*p(x/b)$}.}
%\end{ccAdvanced}

%unary operations
\ccNestedType{Make_square_free}
{ A model of \ccc{PolynomialTraits_d::MakeSquareFree}.}\ccGlue
\ccNestedType{Square_free_factorize}
{ In case \ccc{PolynomialTraits::Polynomial_d} 
is not a model of \ccc{UniqueFactorizationDomain}, this is of type \ccc{CGAL::Null_type},
 otherwise this is a model of \ccc{PolynomialTraits_d::SquareFreeFactorize}.}


%pseudo division
\ccNestedType{Pseudo_division          }
{ A model of \ccc{PolynomialTraits_d::Pseudo_division}.}\ccGlue
\ccNestedType{Pseudo_division_remainder}
{ A model of \ccc{PolynomialTraits_d::Pseudo_division_remainder}.}\ccGlue
\ccNestedType{Pseudo_division_quotient }
{ A model of \ccc{PolynomialTraits_d::Pseudo_division_quotient}.}


%utcf
\ccNestedType{Gcd_up_to_constant_factor}
{ A model of \ccc{PolynomialTraits_d::GcdUpToConstantFactor}.}
\ccGlue
\ccNestedType{Integral_division_up_to_constant_factor}
{ A model of \ccc{PolynomialTraits_d::IntegralDivisionUpToConstantFactor}.}
\ccGlue
\ccNestedType{Content_up_to_constant_factor}
{ A model of \ccc{PolynomialTraits_d::ContentUpToConstantFactor}.}
\ccGlue
\ccNestedType{Square_free_factorize_up_to_constant_factor}
{ A model of \ccc{PolynomialTraits_d::SquareFreeFactorizeUpToConstantFactor}.}

%resultant
\ccNestedType{Resultant}{ A model of \ccc{PolynomialTraits_d::Resultant}.}
\ccNestedType{Polynomial_subresultants}
{ A model of \ccc{PolynomialTraits_d::PolynomialSubresultant}.}
\ccNestedType{Principal_subresultants}
{ A model of \ccc{PolynomialTraits_d::PrincipalSubresultant}.}
\ccNestedType{Sturm_habicht_sequence}
{ A model of \ccc{PolynomialTraits_d::SturmHabichtSequence}.}
\ccNestedType{Sturm_habicht_sequence_with_cofactors}
{ A model of \ccc{PolynomialTraits_d::SturmHabichtSequenceWithCofactors}.}
\ccNestedType{Principal_sturm_habicht_sequence}
{ A model of \ccc{PolynomialTraits_d::PrincipalSturmHabichtSequence}.}
 % end ccIgnore
 
\ccSeeAlso 

\ccRefIdfierPage{Polynomial_d}\\
\ccRefIdfierPage{PolynomialTraits_d}\\

\end{ccRefConcept}
