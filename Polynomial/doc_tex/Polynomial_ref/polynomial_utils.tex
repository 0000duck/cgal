

% GetCoefficient
% GetInnermostCoefficient
% ConstructCoefficientConstIteratorRange
% ConstructInnermostCoefficientConstIteratorRange

% Permute
\begin{ccRefFunction}{permute}
\ccDefinition
For a given \ccc{Polynomial_d} the function \ccRefName\ adapts the 
according functor in \ccc{Polynomial_traits_d<Polynomial_d>}. \\ \\ 
For more details see the concept \ccc{PolynomialTraits_d::Permute}. \\ \\ 
\ccInclude{CGAL/polynomial_utils.h}
\ccFunction{
template <class Polynomial_d, class InputIterator >
Polynomial_traits_d<Polynomial_d>::Permute::result_type
permute(const Polynomial_d& p, InputIterator begin, InputIterator end );
}{
Adapts \ccc{Polynomial_traits_d<Polynomial_d>::Permute}. 
}
\ccSeeAlso
\ccRefConceptPage{Polynomial_d}\\
\ccRefConceptPage{PolynomialTraits_d}\\
\ccRefConceptPage{PolynomialTraits_d::Permute}\\
\end{ccRefFunction}

% Swap
\begin{ccRefFunction}{swap}
\ccDefinition
For a given \ccc{Polynomial_d} the function \ccRefName\ adapts the 
according functor in \ccc{Polynomial_traits_d<Polynomial_d>}. \\ \\ 
For more details see the concept \ccc{PolynomialTraits_d::Swap}. \\ \\ 
\ccInclude{CGAL/polynomial_utils.h}
\ccFunction{
template <class Polynomial_d>
Polynomial_traits_d<Polynomial_d>::Swap::result_type 
swap(const Polynomial_d& p, int i, int j);
}{
Adapts \ccc{Polynomial_traits_d<Polynomial_d>::Swap}. 
}
\ccSeeAlso
\ccRefConceptPage{Polynomial_d}\\
\ccRefConceptPage{PolynomialTraits_d}\\
\ccRefConceptPage{PolynomialTraits_d::Swap}\\
\end{ccRefFunction}

% Move
\begin{ccRefFunction}{move}
\ccDefinition
For a given \ccc{Polynomial_d} the function \ccRefName\ adapts the 
according functor in \ccc{Polynomial_traits_d<Polynomial_d>}.  \\ \\
For more details see the concept \ccc{PolynomialTraits_d::Move}.  \\ \\
\ccInclude{CGAL/polynomial_utils.h}
\ccFunction{
template <class Polynomial_d>
Polynomial_traits_d<Polynomial_d>::Move::result_type 
move(const Polynomial_d& p, int i, int j);
}{
Adapts \ccc{Polynomial_traits_d<Polynomial_d>::Move}. 
}
\ccSeeAlso
\ccRefConceptPage{Polynomial_d}\\
\ccRefConceptPage{PolynomialTraits_d}\\
\ccRefConceptPage{PolynomialTraits_d::Move}\\
\end{ccRefFunction}

% Degree
\begin{ccRefFunction}{degree}
\ccDefinition
For a given \ccc{Polynomial_d} the function \ccRefName\ adapts the 
according functor in \ccc{Polynomial_traits_d<Polynomial_d>}.  \\ \\
For more details see the concept \ccc{PolynomialTraits_d::Degree}.  \\ \\
\ccInclude{CGAL/polynomial_utils.h}
\ccFunction{
template <class Polynomial_d>
Polynomial_traits_d<Polynomial_d>::Degree::result_type 
degree(const Polynomial_d& p, 
int i, 
index = Polynomial_traits_d<Polynomial_d>::d-1);
}{
Adapts \ccc{Polynomial_traits_d<Polynomial_d>::Degree}. 
}
\ccSeeAlso
\ccRefConceptPage{Polynomial_d}\\
\ccRefConceptPage{PolynomialTraits_d}\\
\ccRefConceptPage{PolynomialTraits_d::Degree}\\
\end{ccRefFunction}

% TotalDegree
\begin{ccRefFunction}{total_degree}
\ccDefinition
For a given \ccc{Polynomial_d} the function \ccRefName\ adapts the 
according functor in \ccc{Polynomial_traits_d<Polynomial_d>}. \\ \\ 
For more details see the concept \ccc{PolynomialTraits_d::TotalDegree}. \\ \\ 
\ccInclude{CGAL/polynomial_utils.h}
\ccFunction{
template <class Polynomial_d>
Polynomial_traits_d<Polynomial_d>::Total_degree::result_type 
total_degree(const Polynomial_d& p);
}{
Adapts \ccc{Polynomial_traits_d<Polynomial_d>::Total_degree}. 
}
\ccSeeAlso
\ccRefConceptPage{Polynomial_d}\\
\ccRefConceptPage{PolynomialTraits_d}\\
\ccRefConceptPage{PolynomialTraits_d::TotalDegree}\\
\end{ccRefFunction}

% DegreeVector
\begin{ccRefFunction}{degree_vector}
\ccDefinition
For a given \ccc{Polynomial_d} the function \ccRefName\ adapts the 
according functor in \ccc{Polynomial_traits_d<Polynomial_d>}. \\ \\ 
For more details see the concept \ccc{PolynomialTraits_d::DegreeVector}. \\ \\ 
\ccInclude{CGAL/polynomial_utils.h}
\ccFunction{
template <class Polynomial_d>
Polynomial_traits_d<Polynomial_d>::Degree_vector::result_type 
degree_vector(const Polynomial_d& p);
}{
Adapts \ccc{Polynomial_traits_d<Polynomial_d>::DegreeVector}. 
}
\ccSeeAlso
\ccRefConceptPage{Polynomial_d}\\
\ccRefConceptPage{PolynomialTraits_d}\\
\ccRefConceptPage{PolynomialTraits_d::Degree_vector}\\
\end{ccRefFunction}

% LeadingCoefficient
\begin{ccRefFunction}{leading_coefficient}
\ccDefinition
For a given \ccc{Polynomial_d} the function \ccRefName\ adapts the 
according functor in \ccc{Polynomial_traits_d<Polynomial_d>}. \\ \\ 
For more details see the concept \ccc{PolynomialTraits_d::LeadingCoefficient}. \\ \\ 
\ccInclude{CGAL/polynomial_utils.h}
\ccFunction{
template <class Polynomial_d>
Polynomial_traits_d<Polynomial_d>::Leading_coefficient::result_type 
leading_coefficient(const Polynomial_d& p);
}{
Adapts \ccc{Polynomial_traits_d<Polynomial_d>::Leading_coefficient}. 
}
\ccSeeAlso
\ccRefConceptPage{Polynomial_d}\\
\ccRefConceptPage{PolynomialTraits_d}\\
\ccRefConceptPage{PolynomialTraits_d::LeadingCoefficient}\\
\end{ccRefFunction}

% InnermostLeadingCoefficient
\begin{ccRefFunction}{innermost_leading_coefficient}
\ccDefinition
For a given \ccc{Polynomial_d} the function \ccRefName\ adapts the 
according functor in \ccc{Polynomial_traits_d<Polynomial_d>}. \\ \\ 
For more details see the concept 
\ccc{PolynomialTraits_d::InnermostLeadingCoefficient}. \\ \\ 
\ccInclude{CGAL/polynomial_utils.h}
\ccFunction{
template <class Polynomial_d>
Polynomial_traits_d<Polynomial_d>::Innermost_leading_coefficient::result_type 
innermost_leading_coefficient(const Polynomial_d& p);
}{
Adapts \ccc{Polynomial_traits_d<Polynomial_d>::InnermostLeadingCoefficient}. 
}
\ccSeeAlso
\ccRefConceptPage{Polynomial_d}\\
\ccRefConceptPage{PolynomialTraits_d}\\
\ccRefConceptPage{PolynomialTraits_d::InnermostLeadingCoefficient}\\
\end{ccRefFunction}

% Canonicalize
\begin{ccRefFunction}{canonicalize}
\ccDefinition
For a given \ccc{Polynomial_d} the function \ccRefName\ adapts the 
according functor in \ccc{Polynomial_traits_d<Polynomial_d>}. \\ \\ 
For more details see the concept \ccc{PolynomialTraits_d::Canonicalize}. \\ \\ 
\ccInclude{CGAL/polynomial_utils.h}
\ccFunction{
template <class Polynomial_d>
Polynomial_traits_d<Polynomial_d>::Canonicalize::result_type 
canonicalize(const Polynomial_d& p);
}{
Adapts \ccc{Polynomial_traits_d<Polynomial_d>::Canonicalize}. 
}
\ccSeeAlso
\ccRefConceptPage{Polynomial_d}\\
\ccRefConceptPage{PolynomialTraits_d}\\
\ccRefConceptPage{PolynomialTraits_d::Canonicalize}\\
\end{ccRefFunction}

% Differentiate
\begin{ccRefFunction}{differentiate}
\ccDefinition
For a given \ccc{Polynomial_d} the function \ccRefName\ adapts the 
according functor in \ccc{Polynomial_traits_d<Polynomial_d>}. \\ \\ 
For more details see the concept \ccc{PolynomialTraits_d::Differentiate}. \\ \\ 
\ccInclude{CGAL/polynomial_utils.h}
\ccFunction{
template <class Polynomial_d>
Polynomial_traits_d<Polynomial_d>::Differentiate::result_type 
differentiate(const Polynomial_d& p,
index = Polynomial_traits_d<Polynomial_d>::d-1 );
}{
Adapts \ccc{Polynomial_traits_d<Polynomial_d>::Differentiate}. 
}
\ccSeeAlso
\ccRefConceptPage{Polynomial_d}\\
\ccRefConceptPage{PolynomialTraits_d}\\
\ccRefConceptPage{PolynomialTraits_d::Differentiate}\\
\end{ccRefFunction}

% Evaluate
\begin{ccRefFunction}{evaluate}
\ccDefinition
For a given \ccc{Polynomial_d} the function \ccRefName\ adapts the 
according functor in \ccc{Polynomial_traits_d<Polynomial_d>}. \\ \\ 
For more details see the concept \ccc{PolynomialTraits_d::Evaluate}. \\ \\ 
\ccInclude{CGAL/polynomial_utils.h}
\ccFunction{
template <class Polynomial_d>
Polynomial_traits_d<Polynomial_d>::Evaluate::result_type 
evaluate(const Polynomial_d& p,
        Polynomial_traits_d<Polynomial_d>::Coefficient_type x);
}{
Adapts \ccc{Polynomial_traits_d<Polynomial_d>::Evaluate}. 
}
\ccSeeAlso
\ccRefConceptPage{Polynomial_d}\\
\ccRefConceptPage{PolynomialTraits_d}\\
\ccRefConceptPage{PolynomialTraits_d::Evaluate}\\
\end{ccRefFunction}
% EvaluateHomogeneous
\begin{ccRefFunction}{evaluate_homogeneous}
\ccDefinition
For a given \ccc{Polynomial_d} the function \ccRefName\ adapts the 
according functor in \ccc{Polynomial_traits_d<Polynomial_d>}. \\ \\ 
For more details see the concept \ccc{PolynomialTraits_d::EvaluateHomogeneous}. \\ \\ 
\ccInclude{CGAL/polynomial_utils.h}
\ccFunction{
template <class Polynomial_d>
Polynomial_traits_d<Polynomial_d>::Evaluate_homogeneous::result_type 
evaluate_homogeneous(
const Polynomial_d& p,
Polynomial_traits_d<Polynomial_d>::Coefficient_type u,
Polynomial_traits_d<Polynomial_d>::Coefficient_type v);
}{
Adapts \ccc{Polynomial_traits_d<Polynomial_d>::Evaluate_homogeneous}. 
}
\ccSeeAlso
\ccRefConceptPage{Polynomial_d}\\
\ccRefConceptPage{PolynomialTraits_d}\\
\ccRefConceptPage{PolynomialTraits_d::EvaluateHomogeneous}\\
\end{ccRefFunction}

% Substitute
\begin{ccRefFunction}{substitute}
\ccDefinition
For a given \ccc{Polynomial_d} the function \ccRefName\ adapts the 
according functor in \ccc{Polynomial_traits_d<Polynomial_d>}. \\ \\ 
For more details see the concept \ccc{PolynomialTraits_d::Substitute}. \\ \\ 
\ccInclude{CGAL/polynomial_utils.h}
\ccFunction{
template < class Polynomial_d, class InputIterator >
CGAL::Coercion_traits<
Polynomial_traits_d<Polynomial_d>::Innermost_coefficient,
std::iterator_traits<Input_iterator>::value_type
>::Type
substitute(
const Polynomial_d& p, InputIterator begin, InputIterator end);
}{
Adapts \ccc{Polynomial_traits_d<Polynomial_d>::Substitute}. 
}
\ccSeeAlso
\ccRefConceptPage{Polynomial_d}\\
\ccRefConceptPage{PolynomialTraits_d}\\
\ccRefConceptPage{PolynomialTraits_d::Substitute}\\
\end{ccRefFunction}
% IsZeroAt
\begin{ccRefFunction}{is_zero_at}
\ccDefinition
For a given \ccc{Polynomial_d} the function \ccRefName\ adapts the 
according functor in \ccc{Polynomial_traits_d<Polynomial_d>}. \\ \\ 
For more details see the concept \ccc{PolynomialTraits_d::IsZeroAt}. \\ \\ 
\ccInclude{CGAL/polynomial_utils.h}
\ccFunction{
template < class Polynomial_d, class InputIterator >
Polynomial_traits_d<Polynomial_d>::Is_zero_at::result_type
is_zero_at(
const Polynomial_d& p, InputIterator begin, InputIterator end);
}{
Adapts \ccc{Polynomial_traits_d<Polynomial_d>::Is_zero_at}. 
}
\ccSeeAlso
\ccRefConceptPage{Polynomial_d}\\
\ccRefConceptPage{PolynomialTraits_d}\\
\ccRefConceptPage{PolynomialTraits_d::IsZeroAt}\\
\end{ccRefFunction}
% SignAt
\begin{ccRefFunction}{sign_at}
\ccDefinition
For a given \ccc{Polynomial_d} the function \ccRefName\ adapts the 
according functor in \ccc{Polynomial_traits_d<Polynomial_d>}. \\ \\ 
For more details see the concept \ccc{PolynomialTraits_d::SignAt}. \\ \\ 
\ccInclude{CGAL/polynomial_utils.h}
\ccFunction{
template < class Polynomial_d, class InputIterator >
Polynomial_traits_d<Polynomial_d>::Sign_at::result_type
sign_at(
const Polynomial_d& p, InputIterator begin, InputIterator end);
}{
Adapts \ccc{Polynomial_traits_d<Polynomial_d>::Sign_at}. 
}
\ccSeeAlso
\ccRefConceptPage{Polynomial_d}\\
\ccRefConceptPage{PolynomialTraits_d}\\
\ccRefConceptPage{PolynomialTraits_d::SignAt}\\
\end{ccRefFunction}


% SubstituteHomogeneous
\begin{ccRefFunction}{substitute_homogeneous}
\ccDefinition
For a given \ccc{Polynomial_d} the function \ccRefName\ adapts the 
according functor in \ccc{Polynomial_traits_d<Polynomial_d>}. \\ \\ 
For more details see the concept \ccc{PolynomialTraits_d::SubstituteHomogeneous}. \\ \\ 
\ccInclude{CGAL/polynomial_utils.h}
\ccFunction{
template < class Polynomial_d, class InputIterator >
CGAL::Coercion_traits<
Polynomial_traits_d<Polynomial_d>::Innermost_coefficient,
std::iterator_traits<Input_iterator>::value_type
>::Type
substitute_homogeneous(
const Polynomial_d& p, InputIterator begin, InputIterator end);
}{
Adapts \ccc{Polynomial_traits_d<Polynomial_d>::Substitute_homogeneous}. 
}
\ccSeeAlso
\ccRefConceptPage{Polynomial_d}\\
\ccRefConceptPage{PolynomialTraits_d}\\
\ccRefConceptPage{PolynomialTraits_d::SubstituteHomogeneous}\\
\end{ccRefFunction}
% IsZeroAtHomogeneous
\begin{ccRefFunction}{is_zero_at_homogeneous}
\ccDefinition
For a given \ccc{Polynomial_d} the function \ccRefName\ adapts the 
according functor in \ccc{Polynomial_traits_d<Polynomial_d>}. \\ \\ 
For more details see the concept \ccc{PolynomialTraits_d::IsZeroAtHomogeneous}. \\ \\ 
\ccInclude{CGAL/polynomial_utils.h}
\ccFunction{
template < class Polynomial_d, class InputIterator >
Polynomial_traits_d<Polynomial_d>::Is_zero_at_homogeneous::result_type
is_zero_at_homogeneous(
const Polynomial_d& p, InputIterator begin, InputIterator end);
}{
Adapts \ccc{Polynomial_traits_d<Polynomial_d>::Is_zero_at_homogeneous}. 
}
\ccSeeAlso
\ccRefConceptPage{Polynomial_d}\\
\ccRefConceptPage{PolynomialTraits_d}\\
\ccRefConceptPage{PolynomialTraits_d::IsZeroAtHomogeneous}\\
\end{ccRefFunction}
% SignAtHomogeneous
\begin{ccRefFunction}{sign_at_homogeneous}
\ccDefinition
For a given \ccc{Polynomial_d} the function \ccRefName\ adapts the 
according functor in \ccc{Polynomial_traits_d<Polynomial_d>}. \\ \\ 
For more details see the concept \ccc{PolynomialTraits_d::SignAtHomogeneous}. \\ \\ 
\ccInclude{CGAL/polynomial_utils.h}
\ccFunction{
template < class Polynomial_d, class InputIterator >
Polynomial_traits_d<Polynomial_d>::Sign_at_homogeneous::result_type
sign_at_homogeneous(
const Polynomial_d& p, InputIterator begin, InputIterator end);
}{
Adapts \ccc{Polynomial_traits_d<Polynomial_d>::Sign_at_homogeneous}. 
}
\ccSeeAlso
\ccRefConceptPage{Polynomial_d}\\
\ccRefConceptPage{PolynomialTraits_d}\\
\ccRefConceptPage{PolynomialTraits_d::SignAtHomogeneous}\\
\end{ccRefFunction}

% Compare 
\begin{ccRefFunction}{compare}
\ccDefinition
For a given \ccc{Polynomial_d} the function \ccRefName\ adapts the 
according functor in \ccc{Polynomial_traits_d<Polynomial_d>}. \\ \\ 
For more details see the concept \ccc{PolynomialTraits_d::Compare}. \\ \\ 
\ccInclude{CGAL/polynomial_utils.h}
\ccFunction{
template <class Polynomial_d>
Polynomial_traits_d<Polynomial_d>::Compare::result_type 
compare(const Polynomial_d& p, const Polynomial_d& q);
}{
Adapts \ccc{Polynomial_traits_d<Polynomial_d>::Compare}. 
}
\ccSeeAlso
\ccRefConceptPage{Polynomial_d}\\
\ccRefConceptPage{PolynomialTraits_d}\\
\ccRefConceptPage{PolynomialTraits_d::Compare}\\
\end{ccRefFunction}

% UnivariateContent
\begin{ccRefFunction}{univariate_content}
\ccDefinition
For a given \ccc{Polynomial_d} the function \ccRefName\ adapts the 
according functor in \ccc{Polynomial_traits_d<Polynomial_d>}. \\ \\ 
For more details see the concept \ccc{PolynomialTraits_d::UnivariateContent}. \\ \\ 
\ccInclude{CGAL/polynomial_utils.h}
\ccFunction{
template <class Polynomial_d>
Polynomial_traits_d<Polynomial_d>::Univariate_content::result_type 
univariate_content(const Polynomial_d& p);
}{
Adapts \ccc{Polynomial_traits_d<Polynomial_d>::UnivariateContent}. 
}
\ccSeeAlso
\ccRefConceptPage{Polynomial_d}\\
\ccRefConceptPage{PolynomialTraits_d}\\
\ccRefConceptPage{PolynomialTraits_d::Univariate_Content}\\
\end{ccRefFunction}

% MultivariateContent
\begin{ccRefFunction}{multivariate_content}
\ccDefinition
For a given \ccc{Polynomial_d} the function \ccRefName\ adapts the 
according functor in \ccc{Polynomial_traits_d<Polynomial_d>}. \\ \\ 
For more details see the concept \ccc{PolynomialTraits_d::MultivariateContent}. \\ \\ 
\ccInclude{CGAL/polynomial_utils.h}
\ccFunction{
template <class Polynomial_d>
Polynomial_traits_d<Polynomial_d>::Multivariate_content::result_type 
multivariate_content(const Polynomial_d& p);
}{
Adapts \ccc{Polynomial_traits_d<Polynomial_d>::Multivariate_content}. 
}
\ccSeeAlso
\ccRefConceptPage{Polynomial_d}\\
\ccRefConceptPage{PolynomialTraits_d}\\
\ccRefConceptPage{PolynomialTraits_d::MultivariateContent}\\
\end{ccRefFunction}

% SquareFreeFactorize
\begin{ccRefFunction}{square_free_factorize}
\ccDefinition
For a given \ccc{Polynomial_d} the function \ccRefName\ adapts the 
according functor in \ccc{Polynomial_traits_d<Polynomial_d>}. \\ \\ 
For more details see the concept \ccc{PolynomialTraits_d::SquareFreeFactorize}. \\ \\ 
\ccInclude{CGAL/polynomial_utils.h}
\ccFunction{
template <class Polynomial_d, class OutputIterator > 
OutputIterator
square_free_factorize(
const Polynomial_d& p, 
OutputIterator it,
Polynomial_traits_d<Polynomial>::Innermost_coefficient& a);
}{
Adapts \ccc{Polynomial_traits_d<Polynomial_d>::Square_free_factorize}. 
}
\ccFunction{
template <class Polynomial_d, class OutputIterator > 
OutputIterator
square_free_factorize(const Polynomial_d& p, OutputIterator it);
}{
Adapts \ccc{Polynomial_traits_d<Polynomial_d>::Square_free_factorize}. 
}
\ccSeeAlso
\ccRefConceptPage{Polynomial_d}\\
\ccRefConceptPage{PolynomialTraits_d}\\
\ccRefConceptPage{PolynomialTraits_d::SquareFreeFactorize}\\
\end{ccRefFunction}
% MakeSquareFree
\begin{ccRefFunction}{make_square_free}
\ccDefinition
For a given \ccc{Polynomial_d} the function \ccRefName\ adapts the 
according functor in \ccc{Polynomial_traits_d<Polynomial_d>}. \\ \\ 
For more details see the concept \ccc{PolynomialTraits_d::MakeSquareFree}. \\ \\ 
\ccInclude{CGAL/polynomial_utils.h}
\ccFunction{
template <class Polynomial_d>
Polynomial_traits_d<Polynomial_d>::Make_square_free::result_type 
make_square_free(const Polynomial_d& p);
}{
Adapts \ccc{Polynomial_traits_d<Polynomial_d>::Make_square_free}. 
}
\ccSeeAlso
\ccRefConceptPage{Polynomial_d}\\
\ccRefConceptPage{PolynomialTraits_d}\\
\ccRefConceptPage{PolynomialTraits_d::MakeSquareFree}\\
\end{ccRefFunction}

% PseudoDivision
\begin{ccRefFunction}{pseudo_division}
\ccDefinition
For a given \ccc{Polynomial_d} the function \ccRefName\ adapts the 
according functor in \ccc{Polynomial_traits_d<Polynomial_d>}. \\ \\ 
For more details see the concept \ccc{PolynomialTraits_d::PseudoDivision}. \\ \\ 
\ccInclude{CGAL/polynomial_utils.h}
\ccFunction{
template <class Polynomial_d>
void
pseudo_division(
const Polynomial_d& f, const Polynomial_d& g, 
Polynomial_d& q, Polynomial_d& r, Polynomial_traits_d<Polynomial_d>::Coefficient_type& D );
}{
Adapts \ccc{Polynomial_traits_d<Polynomial_d>::Pseudo_division}. 
}
\ccSeeAlso
\ccRefConceptPage{Polynomial_d}\\
\ccRefConceptPage{PolynomialTraits_d}\\
\ccRefConceptPage{PolynomialTraits_d::PseudoDivision}\\
\end{ccRefFunction}

% PseudoDivisionQuotient
\begin{ccRefFunction}{pseudo_division_quotient}
\ccDefinition
For a given \ccc{Polynomial_d} the function \ccRefName\ adapts the 
according functor in \ccc{Polynomial_traits_d<Polynomial_d>}. \\ \\ 
For more details see the concept \ccc{PolynomialTraits_d::PseudoDivisionQuotient}. \\ \\ 
\ccInclude{CGAL/polynomial_utils.h}
\ccFunction{
template <class Polynomial_d>
Polynomial_traits_d<Polynomial_d>::Pseudo_division_quotient::result_type 
pseudo_division_quotient(const Polynomial_d& p, const Polynomial_d& q);
}{
Adapts \ccc{Polynomial_traits_d<Polynomial_d>::Pseudo_division_quotient}. 
}
\ccSeeAlso
\ccRefConceptPage{Polynomial_d}\\
\ccRefConceptPage{PolynomialTraits_d}\\
\ccRefConceptPage{PolynomialTraits_d::PseudoDivisionQuotient}\\
\end{ccRefFunction}

% PseudoDivisionRemainder
\begin{ccRefFunction}{pseudo_division_remainder}
\ccDefinition
For a given \ccc{Polynomial_d} the function \ccRefName\ adapts the 
according functor in \ccc{Polynomial_traits_d<Polynomial_d>}. \\ \\ 
For more details see the concept \ccc{PolynomialTraits_d::PseudoDivisionRemainder}. \\ \\ 
\ccInclude{CGAL/polynomial_utils.h}
\ccFunction{
template <class Polynomial_d>
Polynomial_traits_d<Polynomial_d>::Pseudo_division_remainder::result_type 
pseudo_division_remainder(const Polynomial_d& p, const Polynomial_d& q);
}{
Adapts \ccc{Polynomial_traits_d<Polynomial_d>::Pseudo_division_remainder}. 
}
\ccSeeAlso
\ccRefConceptPage{Polynomial_d}\\
\ccRefConceptPage{PolynomialTraits_d}\\
\ccRefConceptPage{PolynomialTraits_d::PseudoDivisionRemainder}\\
\end{ccRefFunction}

% GcdUpToConstantFactor
\begin{ccRefFunction}{gcd_up_to_constant_factor}
\ccDefinition
For a given \ccc{Polynomial_d} the function \ccRefName\ adapts the 
according functor in \ccc{Polynomial_traits_d<Polynomial_d>}. \\ \\ 
For more details see the concept \ccc{PolynomialTraits_d::GcdUpToConstantFactor}. \\ \\ 
\ccInclude{CGAL/polynomial_utils.h}
\ccFunction{
template <class Polynomial_d>
Polynomial_traits_d<Polynomial_d>::Gcd_up_to_constant_factor::result_type 
gcd_up_to_constant_factor(const Polynomial_d& p, const Polynomial_d& q);
}{
Adapts \ccc{Polynomial_traits_d<Polynomial_d>::Gcd_up_to_constant_factor}. 
}
\ccSeeAlso
\ccRefConceptPage{Polynomial_d}\\
\ccRefConceptPage{PolynomialTraits_d}\\
\ccRefConceptPage{PolynomialTraits_d::GcdUpToConstantFactor}\\
\end{ccRefFunction}

% IntegralDivisionUpToConstantFactor
\begin{ccRefFunction}{integral_division_up_to_constant_factor}
\ccDefinition
For a given \ccc{Polynomial_d} the function \ccRefName\ adapts the 
according functor in \ccc{Polynomial_traits_d<Polynomial_d>}. \\ \\ 
For more details see the concept \ccc{PolynomialTraits_d::IntegralDivisionUpToConstantFactor}. \\ \\ 
\ccInclude{CGAL/polynomial_utils.h}
\ccFunction{
template <class Polynomial_d>
Polynomial_traits_d<Polynomial_d>::Integral_division_up_to_constant_factor::result_type 
integral_division_up_to_constant_factor(const Polynomial_d& p, const Polynomial_d& q);
}{
Adapts \ccc{Polynomial_traits_d<Polynomial_d>::Integral_division_up_to_constant_factor}. 
}
\ccSeeAlso
\ccRefConceptPage{Polynomial_d}\\
\ccRefConceptPage{PolynomialTraits_d}\\
\ccRefConceptPage{PolynomialTraits_d::IntegralDivisionUpToConstantFactor}\\
\end{ccRefFunction}
% UnivariateContentUpToConstantFactor
\begin{ccRefFunction}{univariate_content_up_to_constant_factor}
\ccDefinition
For a given \ccc{Polynomial_d} the function \ccRefName\ adapts the 
according functor in \ccc{Polynomial_traits_d<Polynomial_d>}. \\ \\ 
For more details see the concept \ccc{PolynomialTraits_d::UnivariateContentUpToConstantFactor}. \\ \\ 
\ccInclude{CGAL/polynomial_utils.h}
\ccFunction{
template <class Polynomial_d>
Polynomial_traits_d<Polynomial_d>::Univariate_content_up_to_constant_factor::result_type 
univariate_content_up_to_constant_factor(const Polynomial_d& p);
}{
Adapts \ccc{Polynomial_traits_d<Polynomial_d>::Univariate_content_up_to_constant_factor}. 
}
\ccSeeAlso
\ccRefConceptPage{Polynomial_d}\\
\ccRefConceptPage{PolynomialTraits_d}\\
\ccRefConceptPage{PolynomialTraits_d::UnivariateContentUpToConstantFactor}\\
\end{ccRefFunction}

% SquareFreeFactorizeUpToConstantFactorUpToConstantFactor
\begin{ccRefFunction}{square_free_factorize_up_to_constant_factor}
\ccDefinition
For a given \ccc{Polynomial_d} the function \ccRefName\ adapts the 
according functor in \ccc{Polynomial_traits_d<Polynomial_d>}. \\ \\ 
For more details see the concept \ccc{PolynomialTraits_d::SquareFreeFactorizeUpToConstantFactor}. \\ \\ 
\ccInclude{CGAL/polynomial_utils.h}
\ccFunction{
template <class Polynomial_d, class OutputIterator > 
OutputIterator
square_free_factorize_up_to_constant_factor(const Polynomial_d& p, OutputIterator it);
}{
Adapts \ccc{Polynomial_traits_d<Polynomial_d>::Square_free_factorize_up_to_constant_factor}. 
}
\ccSeeAlso
\ccRefConceptPage{Polynomial_d}\\
\ccRefConceptPage{PolynomialTraits_d}\\
\ccRefConceptPage{PolynomialTraits_d::SquareFreeFactorizeUpToConstantFactor}\\
\end{ccRefFunction}

% Shift
\begin{ccRefFunction}{shift}
\ccDefinition
For a given \ccc{Polynomial_d} the function \ccRefName\ adapts the 
according functor in \ccc{Polynomial_traits_d<Polynomial_d>}. \\ \\ 
For more details see the concept \ccc{PolynomialTraits_d::Shift}. \\ \\ 
\ccInclude{CGAL/polynomial_utils.h}
\ccFunction{
template <class Polynomial_d>
Polynomial_traits_d<Polynomial_d>::Shift::result_type 
shift(const Polynomial_d& p, int i, int index = Polynomial_traits_d<Polynomial_d>::d-1);
}{
Adapts \ccc{Polynomial_traits_d<Polynomial_d>::Shift}. 
}
\ccSeeAlso
\ccRefConceptPage{Polynomial_d}\\
\ccRefConceptPage{PolynomialTraits_d}\\
\ccRefConceptPage{PolynomialTraits_d::Shift}\\
\end{ccRefFunction}
% Negate
\begin{ccRefFunction}{negate}
\ccDefinition
For a given \ccc{Polynomial_d} the function \ccRefName\ adapts the 
according functor in \ccc{Polynomial_traits_d<Polynomial_d>}. \\ \\ 
For more details see the concept \ccc{PolynomialTraits_d::Negate}. \\ \\ 
\ccInclude{CGAL/polynomial_utils.h}
\ccFunction{
template <class Polynomial_d>
Polynomial_traits_d<Polynomial_d>::Negate::result_type 
negate(const Polynomial_d& p, int index = Polynomial_traits_d<Polynomial_d>::d-1);
}{
Adapts \ccc{Polynomial_traits_d<Polynomial_d>::Negate}. 
}
\ccSeeAlso
\ccRefConceptPage{Polynomial_d}\\
\ccRefConceptPage{PolynomialTraits_d}\\
\ccRefConceptPage{PolynomialTraits_d::Negate}\\
\end{ccRefFunction}
% Invert
\begin{ccRefFunction}{invert}
\ccDefinition
For a given \ccc{Polynomial_d} the function \ccRefName\ adapts the 
according functor in \ccc{Polynomial_traits_d<Polynomial_d>}. \\ \\ 
For more details see the concept \ccc{PolynomialTraits_d::Invert}. \\ \\ 
\ccInclude{CGAL/polynomial_utils.h}
\ccFunction{
template <class Polynomial_d>
Polynomial_traits_d<Polynomial_d>::Invert::result_type 
invert(const Polynomial_d& p, int index = Polynomial_traits_d<Polynomial_d>::d-1);
}{
Adapts \ccc{Polynomial_traits_d<Polynomial_d>::Invert}. 
}
\ccSeeAlso
\ccRefConceptPage{Polynomial_d}\\
\ccRefConceptPage{PolynomialTraits_d}\\
\ccRefConceptPage{PolynomialTraits_d::Invert}\\
\end{ccRefFunction}
% Translate
\begin{ccRefFunction}{translate}
\ccDefinition
For a given \ccc{Polynomial_d} the function \ccRefName\ adapts the 
according functor in \ccc{Polynomial_traits_d<Polynomial_d>}. \\ \\ 
For more details see the concept \ccc{PolynomialTraits_d::Translate}. \\ \\ 
\ccInclude{CGAL/polynomial_utils.h}
\ccFunction{
template <class Polynomial_d>
Polynomial_traits_d<Polynomial_d>::Translate::result_type 
translate(
const Polynomial_d& p, 
const Polynomial_traits_d<Polynomial_d>::Innermost_coefficient_type& a, 
int index = Polynomial_traits_d<Polynomial_d>::d-1);
}{
Adapts \ccc{Polynomial_traits_d<Polynomial_d>::Translate}. 
}
\ccSeeAlso
\ccRefConceptPage{Polynomial_d}\\
\ccRefConceptPage{PolynomialTraits_d}\\
\ccRefConceptPage{PolynomialTraits_d::Translate}\\
\end{ccRefFunction}
% TranslateHomogeneous
\begin{ccRefFunction}{translate_homogeneous}
\ccDefinition
For a given \ccc{Polynomial_d} the function \ccRefName\ adapts the 
according functor in \ccc{Polynomial_traits_d<Polynomial_d>}. \\ \\ 
For more details see the concept \ccc{PolynomialTraits_d::TranslateHomogeneous}. \\ \\ 
\ccInclude{CGAL/polynomial_utils.h}
\ccFunction{
template <class Polynomial_d>
Polynomial_traits_d<Polynomial_d>::Translate_homogeneous::result_type 
translate_homogeneous(
const Polynomial_d& p, 
const Polynomial_traits_d<Polynomial_d>::Innermost_coefficient_type& u, 
const Polynomial_traits_d<Polynomial_d>::Innermost_coefficient_type& v, 
int index = Polynomial_traits_d<Polynomial_d>::d-1);
}{
Adapts \ccc{Polynomial_traits_d<Polynomial_d>::Translate_homogeneous}. 
}
\ccSeeAlso
\ccRefConceptPage{Polynomial_d}\\
\ccRefConceptPage{PolynomialTraits_d}\\
\ccRefConceptPage{PolynomialTraits_d::TranslateHomogeneous}\\
\end{ccRefFunction}

% Scale
\begin{ccRefFunction}{scale}
\ccDefinition
For a given \ccc{Polynomial_d} the function \ccRefName\ adapts the 
according functor in \ccc{Polynomial_traits_d<Polynomial_d>}. \\ \\ 
For more details see the concept \ccc{PolynomialTraits_d::Scale}. \\ \\ 
\ccInclude{CGAL/polynomial_utils.h}
\ccFunction{
template <class Polynomial_d>
Polynomial_traits_d<Polynomial_d>::Scale::result_type 
scale(
const Polynomial_d& p, 
const Polynomial_traits_d<Polynomial_d>::Innermost_coefficient_type& a, 
int index = Polynomial_traits_d<Polynomial_d>::d-1);
}{
Adapts \ccc{Polynomial_traits_d<Polynomial_d>::Scale}. 
}
\ccSeeAlso
\ccRefConceptPage{Polynomial_d}\\
\ccRefConceptPage{PolynomialTraits_d}\\
\ccRefConceptPage{PolynomialTraits_d::Scale}\\
\end{ccRefFunction}
% ScaleHomogeneous
\begin{ccRefFunction}{scale_homogeneous}
\ccDefinition
For a given \ccc{Polynomial_d} the function \ccRefName\ adapts the 
according functor in \ccc{Polynomial_traits_d<Polynomial_d>}. \\ \\ 
For more details see the concept \ccc{PolynomialTraits_d::ScaleHomogeneous}. \\ \\ 
\ccInclude{CGAL/polynomial_utils.h}
\ccFunction{
template <class Polynomial_d>
Polynomial_traits_d<Polynomial_d>::Scale_homogeneous::result_type 
scale_homogeneous(
const Polynomial_d& p, 
const Polynomial_traits_d<Polynomial_d>::Innermost_coefficient_type& u, 
const Polynomial_traits_d<Polynomial_d>::Innermost_coefficient_type& v, 
int index = Polynomial_traits_d<Polynomial_d>::d-1);
}{
Adapts \ccc{Polynomial_traits_d<Polynomial_d>::Scale_homogeneous}. 
}
\ccSeeAlso
\ccRefConceptPage{Polynomial_d}\\
\ccRefConceptPage{PolynomialTraits_d}\\
\ccRefConceptPage{PolynomialTraits_d::ScaleHomogeneous}\\
\end{ccRefFunction}
% Resultant



% Compare 
\begin{ccRefFunction}{resultant}
\ccDefinition
For a given \ccc{Polynomial_d} the function \ccRefName\ adapts the 
according functor in \ccc{Polynomial_traits_d<Polynomial_d>}. \\ \\ 
For more details see the concept \ccc{PolynomialTraits_d::Resultant}. \\ \\ 
\ccInclude{CGAL/polynomial_utils.h}
\ccFunction{
template <class Polynomial_d>
Polynomial_traits_d<Polynomial_d>::Resultant::result_type 
resultant(const Polynomial_d& p, const Polynomial_d& q);
}{
Adapts \ccc{Polynomial_traits_d<Polynomial_d>::Resultant}. 
}
\ccSeeAlso
\ccRefConceptPage{Polynomial_d}\\
\ccRefConceptPage{PolynomialTraits_d}\\
\ccRefConceptPage{PolynomialTraits_d::Resultant}\\
\end{ccRefFunction}

% polynomial_subresultants
\begin{ccRefFunction}{polynomial_subresultants}
\ccDefinition

Computes the polynomial subresultants
of two polynomials, defined as in the concept 
\ccc{Polynomial_traits_d::PolynomialSubresultants}, 
if \ccc{PolynomialTraits_d::Coefficient_type}
is at least a model of \ccc{CGAL::IntegralDomainWithoutDivision}.

The computation method depends on the algebraic category of
\ccc{PolynomialTraits_d::Coefficient_type}.
if it models \ccc{CGAL::IntegralDomain}, 
the subresultants are computed
by a pseudo-division approach~\cite{ducos-optimizations}. 
If they only model a
\ccc{CGAL::IntegralDomainWithoutDivision}, the subresultants
are computed by evaluating the determinants directly, avoiding divisions
completely~\cite{kerber-division}.

\ccOperations
\ccFunction{template<typename Polynomial_traits_d,typename OutputIterator>
        OutputIterator polynomial_subresultants
            (typename Polynomial_traits_d::Polynomial_d   p,
             typename Polynomial_traits_d::Polynomial_d   q,
             OutputIterator out);}
         { computes the polynomial subresultants of $p$ and $q$, 
           with respect to the outermost variable. Each element is of type
           \ccc{PolynomialTraits_d::Polynomial_d}.}
\ccSeeAlso
\ccRefConceptPage{Polynomial_d}\\
\ccRefConceptPage{PolynomialTraits_d}\\
\ccRefConceptPage{PolynomialTraits_d::PolynomialSubresultants}\\
\end{ccRefFunction}

% polynomial_subresultants_with_cofactors
\begin{ccRefFunction}{polynomial_subresultants_with_cofactors}
\ccDefinition

Computes the polynomial subresultants
of two polynomials and their cofactors, defined as in the concept 
\ccc{Polynomial_traits_d::PolynomialSubresultantsWithCofactors}, 
if \ccc{PolynomialTraits_d::Coefficient_type}
is at least a model of \ccc{CGAL::IntegralDomain}.

\ccOperations
\ccFunction{template<typename Polynomial_traits_d,
    typename OutputIterator1, 
    typename OutputIterator2,
    typename OutputIterator3>
    OutputIterator1 polynomial_subresultants_with_cofactors
      (typename Polynomial_traits_d::Polynomial_d p,
       typename Polynomial_traits_d::Polynomial_d q,
       OutputIterator1 sres_out,
       OutputIterator2 coP_out,
       OutputIterator3 coQ_out);}
         { computes the polynomial subresultants of $p$ and $q$, 
           \ccc{sres_out}, with respect to the outermost variable, and
           the cofactors for $P$, \ccc{coP_out} and $Q$, \ccc{coQ_out}. 
           The elements  of each output range are of type
           \ccc{PolynomialTraits_d::Polynomial_d}.}
\ccSeeAlso
\ccRefConceptPage{Polynomial_d}\\
\ccRefConceptPage{PolynomialTraits_d}\\
\ccRefConceptPage{PolynomialTraits_d::PolynomialSubresultantsWithCofactors}\\
\end{ccRefFunction}


% principal_subresultants
\begin{ccRefFunction}{principal_subresultants}
\ccDefinition

Computes the principal subresultants
of two polynomials, defined as in the concept 
\ccc{Polynomial_traits_d::PrincipalSubresultants}, 
if \ccc{PolynomialTraits_d::Coefficient_type}
is at least a model of \ccc{CGAL::IntegralDomainWithoutDivision}.

\ccOperations
\ccFunction{template<typename Polynomial_traits_d,typename OutputIterator>
        OutputIterator principal_subresultants
            (typename Polynomial_traits_d::Polynomial_d   p,
             typename Polynomial_traits_d::Polynomial_d   q,
             OutputIterator out);}
         { computes the principal subresultants of $p$ and $q$, 
           with respect to the outermost variable. Each element is of type
           \ccc{PolynomialTraits_d::Coefficient_type}.}
\ccSeeAlso
\ccRefConceptPage{Polynomial_d}\\
\ccRefConceptPage{PolynomialTraits_d}\\
\ccRefConceptPage{PolynomialTraits_d::PrincipalSubresultants}\\
\end{ccRefFunction}

% sturm_habicht_sequence
\begin{ccRefFunction}{sturm_habicht_sequence}
\ccDefinition

Computes the Sturm-Habicht-polynomials
of a polynomial, defined as in the concept 
\ccc{Polynomial_traits_d::SturmHabichtSequence}, 
if \ccc{PolynomialTraits_d::Coefficient_type}
is at least a model of \ccc{CGAL::IntegralDomainWithoutDivision}.

The computation works simply by calling \ccc{CGAL::polynomial_subresultants},
and adjusting the signs.

\ccOperations
\ccFunction{template<typename Polynomial_traits_d,typename OutputIterator> OutputIterator
    sturm_habicht_sequence(typename Polynomial_traits_d::Polynomial_d f, 
                           OutputIterator out);}
         { computes the Sturm-Habicht-sequence of $f$
           with respect to the outermost variable. Each element is of type
           \ccc{PolynomialTraits_d::Polynomial_d}.}
\ccSeeAlso
\ccRefConceptPage{Polynomial_d}\\
\ccRefConceptPage{PolynomialTraits_d}\\
\ccRefConceptPage{PolynomialTraits_d::SturmHabichtSequence}\\
\ccRefIdfierPage{CGAL::polynomial_subresultants}\\
\end{ccRefFunction}

% sturm_habicht_sequence_with_cofactors
\begin{ccRefFunction}{sturm_habicht_sequence_with_cofactors}
\ccDefinition

Computes the Sturm-Habicht-sequence of a polynomial
and its cofactors, defined as in the concept 
\ccc{Polynomial_traits_d::SturmHabichtSequenceWithCofactors}, 
if \ccc{PolynomialTraits_d::Coefficient_type}
is at least a model of \ccc{CGAL::IntegralDomain}.

\ccOperations
\ccFunction{template<typename Polynomial_traits_d,
    typename OutputIterator1,
    typename OutputIterator2,
    typename OutputIterator3> 
    OutputIterator1
    sturm_habicht_sequence_with_cofactors
      (typename Polynomial_traits_d::Polynomial_d f,
       OutputIterator1 stha_out,
       OutputIterator2 cof_out,
       OutputIterator3 cofx_out);}
         { computes the Sturm-Habicht sequence of $f$
           \ccc{stha_out}, with respect to the outermost variable, and
           the cofactors for $f$, \ccc{cof_out} and $f'$, \ccc{cofx_out}. 
           The elements  of each output range are of type
           \ccc{PolynomialTraits_d::Polynomial_d}.}
\ccSeeAlso
\ccRefConceptPage{Polynomial_d}\\
\ccRefConceptPage{PolynomialTraits_d}\\
\ccRefConceptPage{PolynomialTraits_d::SturmHabichtSequenceWithCofactors}\\
\end{ccRefFunction}


% principal_subresultants
\begin{ccRefFunction}{principal_sturm_habicht_sequence}
\ccDefinition

Computes the principal Sturm-Habicht coefficients
of a polynomial, defined as in the concept 
\ccc{Polynomial_traits_d::PrincipalSturmHabichtSequence}, 
if \ccc{PolynomialTraits_d::Coefficient_type}
is at least a model of \ccc{CGAL::IntegralDomainWithoutDivision}.

\ccOperations
\ccFunction{  template <typename Polynomial_traits_d,typename OutputIterator> inline
    OutputIterator
    principal_sturm_habicht_sequence
      (typename Polynomial_traits_d::Polynomial_d A, 
       OutputIterator out);}
         { computes the principal Sturm-Habicht coefficients of $f$
           with respect to the outermost variable. Each element is of type
           \ccc{PolynomialTraits_d::Coefficient_type}.}
\ccSeeAlso
\ccRefConceptPage{Polynomial_d}\\
\ccRefConceptPage{PolynomialTraits_d}\\
\ccRefConceptPage{PolynomialTraits_d::PrincipalSturmHabichtSequence}\\
\end{ccRefFunction}

% stha_count_number_of_real_roots
\begin{ccRefFunction}{stha_count_number_of_real_roots}
\ccDefinition

Given a range that is interpreted as the principal Sturm-Habicht coefficients
of a polynomial $f$, the function computes
$$m:=\# \{\alpha\in\mathbb{R}\mid f(\alpha)=0\}$$
that is, the number of distinct real roots of $f$.

The value type of the iterator range must be a model of \ccc{RealEmbeddable}.

\ccOperations
\ccFunction{  template<typename InputIterator>
    int stha_count_number_of_real_roots(InputIterator start,InputIterator end);}
         { computes the number of distinct real roots of $f$ whose principal Sturm-Habicht coefficients are passed by the iterator range.}
\ccSeeAlso
\ccRefConceptPage{Polynomial_d}\\
\ccRefConceptPage{PolynomialTraits_d}\\
\ccRefConceptPage{PolynomialTraits_d::PrincipalSturmHabichtSequence}\\
\end{ccRefFunction}
