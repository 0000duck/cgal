\begin{ccRefConcept}{PolynomialTraits_d::PolynomialSubresultants}

\textbf{Note:} This functor is optional!

\ccDefinition

Computes the polynomial subresultant of two polynomials $p$ and $q$ of 
type \ccc{PolynomialTraits_d::Polynomial_d} with respect to outermost variable.
Let 
$p=\sum_{i=0}^{n} p_i t^i$ and 
$q=\sum_{i=0}^{m} q_i t^i$, where $t$
is the outermost variable.
The $i$th subresultant (with $i=0,\ldots,\min\{n,m\}$) is defined by

\begin{ccTexOnly}
\begin{eqnarray*}
\mathrm{Sres}_i(p,q)&=&\det \left(\begin{array}{cccccc}
p_n & \ldots &\ldots& p_{2i-m+2}&t^{m-i-1}p \\
&\ddots&&\vdots&\vdots\\
&p_n&\ldots&p_{i+1}&p\\
q_m & \ldots &\ldots & q_{2i-n+2}&t^{n-i-1}q \\
&\ddots&&\vdots&\vdots\\
&q_m&\ldots&q_{i+1}&q
\end{array}\right)
\end{eqnarray*}
\end{ccTexOnly}

\begin{ccHtmlOnly}
<CENTER>
<IMG BORDER=0 SRC="./subresultant_def.gif" ALIGN=middle ALT="Subresultants">
</CENTER>
\end{ccHtmlOnly}

In the exceptional case that $n=m$, $\mathrm{Sres_n}$ is set to $q$.

The result is written in an output range, starting with the $0$th subresultant
$\mathrm{Sres}_0(p,q)$
(aka as the resultant of $p$ and $q$).

A default implementation for the case that 
\ccc{Polynomial_traits_d::Coefficient_type} 
is a model of \ccc{CGAL::Integral_domain_without_division}
is provided by the function \ccc{CGAL::polynomial_subresultants}.

\ccOperations
\ccMethod{template<typename OutputIterator>
        OutputIterator operator()(Polynomial_d   p,
                                  Polynomial_d   q,
                                  OutputIterator out);}
         { computes the polynomial subresultants of $p$ and $q$, 
           with respect to the outermost variable. Each element is of type
           \ccc{PolynomialTraits_d::Polynomial_d}.}

\ccMethod{template<typename OutputIterator>
        OutputIterator operator()(Polynomial_d   p,
                                  Polynomial_d   q,
                                  OutputIterator out,
                                  int i);}
         { computes the polynomial subresultants of $p$ and $q$, 
           with respect to the variable $x_i$.}

%\ccHasModels

\ccSeeAlso

\ccRefIdfierPage{Polynomial_d}\\
\ccRefIdfierPage{PolynomialTraits_d}\\
\ccRefIdfierPage{CGAL::polynomial_subresultants}\\

\end{ccRefConcept}
