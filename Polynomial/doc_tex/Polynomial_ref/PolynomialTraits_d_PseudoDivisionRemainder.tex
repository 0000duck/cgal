\begin{ccRefConcept}{PolynomialTraits_d::PseudoDivisionRemainder}

\ccDefinition

This \ccc{AdaptableBinaryFunction} computes the remainder of the so called 
{\em pseudo division} of to polynomials $f$ and $g$. 
          
Given $f$ and $g \not 0$ one can compute quotient $q$ and remainder $r$
such that $D \cdot f = g \cdot q + r$ and $degree(r) < degree(g)$,
where $ D = leading\_coefficient(g)^{max(0, degree(f)-degree(g)+1)}$

This functor computes $r$.

\ccRefines 
\ccc{AdaptableBinaryFunction}


\ccTypes


\ccSetThreeColumns{xxxxxxxxxxxxxxxxxxxxxxxxxxxxxxxxxxxxxxxx}{xxx}{}
\ccCreationVariable{fo}

\ccTypedef{typedef PolynomialTraits_d::Polynomial_d   result_type;}{}\ccGlue
\ccTypedef{typedef PolynomialTraits_d::Polynomial_d   first_argument_type;}{}\ccGlue
\ccTypedef{typedef PolynomialTraits_d::Polynomial_d   second_argument_type;}{}

\ccOperations


\ccMethod{result_type operator()(first_argument_type  f,
                                 second_argument_type g);}{ 
        Returns the remainder $r$ of the pseudo division of $f$ and $g$ with
        respect to the outermost variable $x_{d-1}$.}

%\ccHasModels

\ccSeeAlso

\ccRefIdfierPage{Polynomial_d}\\
\ccRefIdfierPage{PolynomialTraits_d}\\
\ccRefIdfierPage{PolynomialTraits_d::PseudoDivision}\\
\ccRefIdfierPage{PolynomialTraits_d::PseudoDivisionRemainder}\\
\ccRefIdfierPage{PolynomialTraits_d::PseudoDivisionQuotient}\\

\end{ccRefConcept}