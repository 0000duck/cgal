\begin{ccRefConcept}{PolynomialTraits_d::PseudoDivision}

\ccDefinition

This \ccc{AdaptableFunctor} computes the so called {\em pseudo division} 
of to polynomials $f$ and $g$. 
     
Given $f$ and $g \neq 0$ this functor computes quotient $q$ and
remainder $r$ such that $D \cdot f = g \cdot q + r$ and $degree(r) < degree(g)$,
where $ D = leading\_coefficient(g)^{max(0, degree(f)-degree(g)+1)}$

This functor is useful if the regular division is not available, 
which is the case if \ccc{PolynomialTraits_d::Coefficient_type} is not a \ccc{Field}. 
Hence in general it is not possible to invert the leading coefficient of $g$. 
Instead $f$ is extended by $D$ allowing integral divisions in the internal 
computation. 


\ccRefines 
\ccc{AdaptableFunctor}

\ccTypes

\ccSetThreeColumns{xxxxxxxxxxxxxxxxxxxxxxxxxxxxxxxxxxxxxxxx}{xxx}{}
\ccCreationVariable{fo}

\ccTypedef{typedef void                               result_type;}{}\ccGlue

\ccOperations

\ccMethod{result_type operator()(PolynomialTraits_d::Polynomial_d    f,
                                 PolynomialTraits_d::Polynomial_d    g,
                                 PolynomialTraits_d::Polynomial_d &  q,
                                 PolynomialTraits_d::Polynomial_d &  r,
                                 PolynomialTraits_d::Coefficient_type  &  D);}{ 
        Computes the pseudo division with respect to the outermost variable 
        $x_{d-1}$.
        }

%\ccHasModels

\ccSeeAlso

\ccRefIdfierPage{Polynomial_d}\\
\ccRefIdfierPage{PolynomialTraits_d}\\
\ccRefIdfierPage{PolynomialTraits_d::PseudoDivision}\\
\ccRefIdfierPage{PolynomialTraits_d::PseudoDivisionRemainder}\\
\ccRefIdfierPage{PolynomialTraits_d::PseudoDivisionQuotient}\\

\end{ccRefConcept}