\begin{ccRefConcept}{PolynomialTraits_d::PrincipalSturmHabichtSequence}

\textbf{Note:} This functor is optional!

\ccDefinition

Computes the principal leading coefficients of the Sturm-Habicht sequence 
of a polynomials $f$ of type \ccc{PolynomialTraits_d::Polynomial_d} 
with respect a certain variable $x_i$.
This means that for the $j$th Sturm-Habicht polynomial, this methods returns
the coefficient of $x_i^j$. 

Note that the degree of the $j$th Sturm-Habicht polynomial is at most $j$,
but the principal coefficient might be zero, thus, this functor does not
necessarily give the leading coefficient of the Sturm-Habicht polynomials

In case that \ccc{PolynomialTraits_d::Coefficient_type} is \ccc{RealEmbeddable}, the function \ccc{CGAL::stha_count_number_of_real_roots} can be used
on the resulting sequence to count the number of distinct real roots of
the polynomial~$f$.

A default implementation for the case that 
\ccc{Polynomial_traits_d::Coefficient_type} 
is a model of \ccc{CGAL::IntegralDomainWithoutDivision}
is provided by the function \ccc{CGAL::principal_sturm_habicht_sequence}.

\ccOperations
\ccMethod{template<typename OutputIterator>
        OutputIterator operator()(Polynomial_d   f,
                                  OutputIterator out);}
         { computes the principal coefficients of the 
           Sturm-Habicht sequence of $f$, 
           with respect to the outermost variable. Each element is of type
           \ccc{PolynomialTraits_d::Coefficient_type}.}

\ccMethod{template<typename OutputIterator>
        OutputIterator operator()(Polynomial_d   f,
                                  OutputIterator out,
                                  int i);}
         { computes the principal coefficients 
           of the Sturm-Habicht sequence of $f$ 
           with respect to the variable $x_i$.}

%\ccHasModels

\ccSeeAlso

\ccRefIdfierPage{Polynomial_d}\\
\ccRefIdfierPage{PolynomialTraits_d}\\
\ccRefIdfierPage{CGAL::principal_sturm_habicht_sequence}\\
\ccRefIdfierPage{CGAL::stha_count_number_of_real_roots}\\

\end{ccRefConcept}
