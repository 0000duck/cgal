\begin{ccRefConcept}{PolynomialTraits_d::SubstituteHomogeneous}
\ccDefinition

This \ccc{Functor} substitutes all variables of a given multivariate 
\ccc{PolynomialTraits_d::Polynomial_d} by the values given in the 
iterator range, where begin refers the the value for the innermost variabel. 
In conrast to \ccc{PolynomialTraits_d::Substitute} is interpreted as a homogeneous 
polynomial. Hence the iterator range is required to be of lenght \ccc{PolynomialTraits_d::d+1}. 

Note that the \ccc{result_type} is the coercion type of the value type of the 
given iterator range and \ccc{PolynomialTraits_d::Innermost_ceofficient}. 
In particular \ccc{std::iterator_traits<Input_iterator>::value_type} must be at least 
\ccc{ExplicitInteroperable} with \ccc{PolynomialTraits_d::Innermost_ceofficient}. 

\ccRefines 

Assignable\\
CopyConstructible\\
DefaultConstructible\\

% \ccTypes
% no public types 

\ccSetThreeColumns{xxxxxxxxxxxxxxxxxxxxxxxxxxxxxxxxxxxxxxxx}{xxx}{}
\ccCreationVariable{substitute_homogeneous}

\ccOperations
\ccMethod{
template<class Input_iterator>        
result_type operator()(PolynomialTraits_d::Polynomial_d  p,
                       Input_iterator begin, Input_iterator end);}{ 
Substitute each variabel of $p$ by the values given in the iterator range, where the $p$
is interpreted as a homogeneous polynomial. 
The begin iterator refers to the innermost variable $x_0$. 
The homogeneous degree is considered as equal to the total degree of $p$.
\ccPrecond The length of the iterator range is \ccc{PolynomialTraits_d::d+1}.} 

        
%\ccHasModels

\ccSeeAlso

\ccRefIdfierPage{Polynomial_d}\\
\ccRefIdfierPage{PolynomialTraits_d}\\
\ccRefIdfierPage{PolynomialTraits_d::Substitute}\\
\ccRefIdfierPage{CoercionTraits}\\

\end{ccRefConcept}
