\begin{ccRefConcept}{PolynomialTraits_d}

\ccDefinition
A model of \ccc{PolynomialTraits_d} is associated to a type 
\ccc{Polynomial_d}. 
The type \ccc{Polynomial_d} represents a multivariate polynomial.
The number of variables is denoted as the dimension $d$ of the polynomial,
it is arbitrary but fixed for a certain model of this concept.  
Note that univariate polynomials are not excluded by this concept. In this case 
$d$ is just set to one.  

\ccc{PolynomialTraits_d} provides two different views on the 
multivariate polynomial. 

\begin{itemize}
\item The recursive view:  In this view, the polynomials is considered as 
an element of $R[x_0,\dots,x_{d-2}][x_{d-1}]$. That is, the polynomial 
is treated as a univariate polynomial over the ring $R[x_0,\dots,x_{d-2}]$. 
\item The symmetric or multivariate view: This view is symmetric 
with respect to all variables,
considering the polynomials as element of $R [x_0,\dots,x_{d-1}]$.
\end{itemize}

Many functors consider the polynomial as a univariate polynomial in one variable.
By default this is the outermost variable $x_{d-1}$. However, in general it 
is possible to select a certain variable. 

\ccRefines

\ccc{AlgebraicStructureTraits}

\ccConstants
 
\ccVariable{static const int d;}{The dimension and the number of variables respectively.}

\ccTypes

\ccNestedType{Polynomial_d}{ Type representing $R[x_0,\dots,x_{d-1}]$.}\ccGlue
\ccNestedType{Coefficient_type }{ Type representing $R[x_0,\dots,x_{d-2}]$.}\ccGlue
\ccNestedType{Innermost_coefficient_type}{ Type representing the base ring $R$.}

%\ccNestedType{Coefficient_const_iterator}
%{ Const iterator used to iterate through all coefficients of the polynomial.}
%\ccNestedType{Innermost_coefficient_const_iterator}
%{ Const iterator used to iterate through all innermost coefficients of the polynomial.}

\ccNestedType{template <typename T, int d> struct Rebind}
{This nested template class has to define a type \ccc{Other} which is a model 
of the concept \ccc{PolynomialTraits_d}, where \ccc{d} is the number of 
variables and \ccc{T} the \ccc{Innermost_coefficient_type}.}

\ccHeading{Functors}

In case a functor is not provided it is set to \ccc{CGAL::Null_functor}.
%,e.g., \ccc{Sign_at} if \ccc{Innermost_coefficient_type} is not \ccc{RealEmbeddable}. 
\ccSetTwoColumns{xxxxxxxxxxxxxxxxxxxxxxxxxxxxxxxxxxxxxxxxxxxxx}{}

\ccNestedType{Construct_polynomial}
        {A model of \ccc{PolynomialTraits_d::ConstructPolynomial}.}
\ccNestedType{Get_coefficient}
        {A model of \ccc{PolynomialTraits_d::GetCoefficient}.}
\ccNestedType{Get_innermost_coefficient}
        {A model of \ccc{PolynomialTraits_d::GetInnermostCoefficient}.}

\ccNestedType{Swap}
        { A model of \ccc{PolynomialTraits_d::Swap}.}
\ccNestedType{Move}
        { A model of \ccc{PolynomialTraits_d::Move}.}


\ccNestedType{Degree}
        { A model of \ccc{PolynomialTraits_d::Degree}.}
\ccNestedType{Total_degree}
        { A model of \ccc{PolynomialTraits_d::TotalDegree}.}
\ccNestedType{Degree_vector}
        { A model of \ccc{PolynomialTraits_d::DegreeVector}.}
\ccNestedType{Leading_coefficient}
        { A model of \ccc{PolynomialTraits_d::LeadingCoefficient}.}
\ccNestedType{Innermost_leading_coefficient}
        {A model of \ccc{PolynomialTraits_d::InnermostLeadingCoefficient}.}


\ccNestedType{Canonicalize}
        { A model of \ccc{PolynomialTraits_d::Canonicalize}.}
\ccNestedType{Differentiate}    
        { A model of \ccc{PolynomialTraits_d::Differentiate}.}


%Evaluation
\ccNestedType{Evaluate}
        { A model of \ccc{PolynomialTraits_d::Evaluate}.}
\ccNestedType{Evaluate_homogeneous}
        { A model of \ccc{PolynomialTraits_d::EvaluateHomogeneous}.}

\ccNestedType{Substitute}
        { A model of \ccc{PolynomialTraits_d::Substitute}.}
\ccNestedType{Substitute_homogeneous}
        { A model of \ccc{PolynomialTraits_d::SubstituteHomogeneous}.}

\ccNestedType{Is_zero_at}
        { A model of \ccc{PolynomialTraits_d::IsZeroAt}.}
\ccNestedType{Is_zero_at_homogeneous}
        { A model of \ccc{PolynomialTraits_d::IsZeroAtHomogeneous}.}

\ccNestedType{Sign_at}{
        A model of \ccc{PolynomialTraits_d::SignAt}.\\
        In case \ccc{Innermost_coefficient_type} is not \ccc{RealEmbeddable} this 
        is \ccc{CGAL::Null_functor}.}
\ccNestedType{Sign_at_homogeneous}{ 
        A model of \ccc{PolynomialTraits_d::SignAtHomogeneous}.\\
        In case \ccc{Innermost_coefficient_type} is not \ccc{RealEmbeddable} this 
        is \ccc{CGAL::Null_functor}.}

\ccNestedType{Compare}{ 
        A model of \ccc{PolynomialTraits_d::Compare}. \\
        In case \ccc{Innermost_coefficient_type} is not \ccc{RealEmbeddable} this 
        is \ccc{CGAL::Null_functor}.}
 
\ccNestedType{Univariate_content}{ 
        In case \ccc{PolynomialTraits_d::Coefficient_type} is {\bf not} a model of 
        \ccc{UniqueFactorizationDomain}, this is \ccc{CGAL::Null_type}, otherwise this is 
        a model of \ccc{PolynomialTraits_d::UnivariateContent}.}
%\begin{ccAdvanced}
\ccNestedType{Multivariate_content}{ 
        In case \ccc{PolynomialTraits_d::Innermost_coefficient_type} is {\bf not} 
        a model of \ccc{UniqueFactorizationDomain}, this is \ccc{CGAL::Null_type}, 
        otherwise this is a model of 
        \ccc{PolynomialTraits_d::MultivariateContent}.}
%\end{ccAdvanced}

%Manipulation
\ccNestedType{Shift}{ A model of \ccc{PolynomialTraits_d::Shift}.}\ccGlue
\ccNestedType{Negate}{ A model of \ccc{PolynomialTraits_d::Negate}.}\ccGlue
\ccNestedType{Invert}{ A model of \ccc{PolynomialTraits_d::Invert}.}

\ccNestedType{Translate}{ A model of \ccc{PolynomialTraits_d::Translate}.}\ccGlue
\ccNestedType{Translate_homogeneous}
{ A model of \ccc{PolynomialTraits_d::TranslateHomogeneous}.}

\ccNestedType{Scale}{ A model of \ccc{PolynomialTraits_d::Scale}.}\ccGlue
\ccNestedType{Scale_homogeneous}
{ A model of \ccc{PolynomialTraits_d::ScaleHomogeneous}.}

%\begin{ccAdvanced}
%\ccNestedType{Scale_up}{ A model of \ccc{PolynomialTraits_d::ScaleUp, return $p(a*x)$}.}
%\ccNestedType{Scale_down}{ A model of \ccc{PolynomialTraits_d::ScaleDown, return $b^{degree}*p(x/b)$}.}
%\end{ccAdvanced}

%unary operations
\ccNestedType{Make_square_free}
{ A model of \ccc{PolynomialTraits_d::MakeSquareFree}.}\ccGlue
\ccNestedType{Square_free_factorize}
{ In case \ccc{PolynomialTraits::Polynomial_d} 
is not a model of \ccc{UniqueFactorizationDomain}, this is of type \ccc{CGAL::Null_type},
 otherwise this is a model of \ccc{PolynomialTraits_d::SquareFreeFactorize}.}


%pseudo division
\ccNestedType{Pseudo_division          }
{ A model of \ccc{PolynomialTraits_d::PseudoDivision}.}\ccGlue
\ccNestedType{Pseudo_division_remainder}
{ A model of \ccc{PolynomialTraits_d::PseudoDivisionRemainder}.}\ccGlue
\ccNestedType{Pseudo_division_quotient }
{ A model of \ccc{PolynomialTraits_d::PseudoDivisionQuotient}.}


%utcf
\ccNestedType{Gcd_up_to_constant_factor}
{ A model of \ccc{PolynomialTraits_d::GcdUpToConstantFactor}.}
\ccGlue
\ccNestedType{Integral_division_up_to_constant_factor}
{ A model of \ccc{PolynomialTraits_d::IntegralDivisionUpToConstantFactor}.}
\ccGlue
\ccNestedType{Content_up_to_constant_factor}
{ A model of \ccc{PolynomialTraits_d::UnivariateContentUpToConstantFactor}.}
\ccGlue
\ccNestedType{Square_free_factorize_up_to_constant_factor}
{ A model of \ccc{PolynomialTraits_d::SquareFreeFactorizeUpToConstantFactor}.}

%resultant
\ccNestedType{Resultant}{ A model of \ccc{PolynomialTraits_d::Resultant}.}

% This was added by Michael Kerber, no review so far
%\ccNestedType{Polynomial_subresultants}
%{ A model of \ccc{PolynomialTraits_d::PolynomialSubresultant}.}
%\ccNestedType{Principal_subresultants}
%{ A model of \ccc{PolynomialTraits_d::PrincipalSubresultant}.}
%\ccNestedType{Sturm_habicht_sequence}
%{ A model of \ccc{PolynomialTraits_d::SturmHabichtSequence}.}
%\ccNestedType{Sturm_habicht_sequence_with_cofactors}
%{ A model of \ccc{PolynomialTraits_d::SturmHabichtSequenceWithCofactors}.}
%\ccNestedType{Principal_sturm_habicht_sequence}
%{ A model of \ccc{PolynomialTraits_d::PrincipalSturmHabichtSequence}.}


\ccSeeAlso 

\ccRefIdfierPage{Polynomial_d}\\

\ccHasModels

\ccRefIdfierPage{CGAL::Polynomial_traits_d<Polynomial_d>}

\end{ccRefConcept}