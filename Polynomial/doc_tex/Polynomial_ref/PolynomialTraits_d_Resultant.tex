\begin{ccRefConcept}{PolynomialTraits_d::Resultant}
\ccDefinition

This \ccc{AdaptableBinaryFunction} computes the {\em resultant} of two polynomials 
$f$ and $g$ of type \ccc{PolynomialTraits_d::Polynomial_d} with respect to a 
certain variable.
 
Note that this functor operates on the polynomial in the univariate view, 
that is, the polynomial is considered as a univariate polynomial in one 
specific variable. 

Let $f$ and $g$ be two univariate polynomials over some commutative ring $A$, 
where 
\[f = f_mx^m + \dots + f_0\] and 
\[g = g_nx^n + \dots + g_0.\] 
The resultant of $f$ and $g$ is defined as the determinant of the {\em Sylvester matrix}: 

\begin{ccTexOnly}
\[ \left( \begin{array}{cccccccc}
f_m & \dots & f_0&&& \\
&f_m & \dots & f_0&&\\
&&\ddots&&\ddots&\\
&&&f_m &\dots& f_0\\  
g_n & \dots & g_0&&& \\
&g_n & \dots & g_0&&\\
&&\ddots&&\ddots&\\
&&&g_n & \dots& g_0\\  
\end{array} \right) \]
\end{ccTexOnly}

\begin{ccHtmlOnly}
<CENTER>
<IMG BORDER=0 SRC="./sylvester_matrix.gif" ALIGN=middle ALT="Sylvester Matrix">
</CENTER>
\end{ccHtmlOnly}

Note that this is a $(n+m)\times(n+m)$ matrix as there are $n$ rows for $f$ 
and $m$ rows that are used for $g$. The blank spaces are supposed to be 
filled with zeros. 


% In the sequel we list the most important properties of the resultant. 
\begin{ccAdvanced}
Let $L$ be the algebraic closure of $A$, and write $f$ and $g$ as 
\[f := f_m \ccProd{i=1}{m}{(x-\alpha_i)},\ \alpha_i \in L \]
and 
\[g :=  g_n \ccProd{j=1}{n}{(x-\beta_j)},\ \beta_i \in L,\] then
the resultant of $f$ and $g$ is (up to leading coefficients) 
the product of all pairwise differences of the roots of $f$ and $g$, namely 
\[res(f,g) = f_m^n g_n^m \ccProd{i=1}{m}{\ccProd{j=1}{n}{(\alpha_i-\beta_j)}}.\] 
In particular, $res(f,g) \neq 0$ iff $f$ and $g$ have a common factor with a 
positive degree in $X$. %, that is, they may still have a common factor in $A$. 
%As a consequence $res(f,f') \neq 0$ iff $f$ is square free with respect to 
%$X$, that is, the content of $f$ may still contain a square.  


\end{ccAdvanced}


There are various ways to compute the resultant. 
Naive options are the computation of the resultant as the determinant of 
the Sylvester Matrix or the Bezout 
Matrix as well as the so called subresultant algorithm, 
which is a variant of the Euclidean Algorithm. 
More sophisticated methods may use modular arithmetic and interpolation. 
For more information we refer to, e.g., \cite{gg-mca-99}. 

\ccRefines 
\ccc{AdaptableBinaryFunction}

\ccTypes
\ccSetThreeColumns{xxxxxxxxxxxxxxxxxxxxxxxxxxxxxxxxxxxxxxxx}{xxx}{}
\ccCreationVariable{fo}
\ccTypedef{typedef PolynomialTraits_d::Coefficient_type   result_type;}{}
\ccGlue
\ccTypedef{typedef PolynomialTraits_d::Polynomial_d   first_argument_type;}{}
\ccGlue
\ccTypedef{typedef PolynomialTraits_d::Polynomial_d   second_argument_type;}{}

\ccOperations
\ccMethod{result_type operator()(first_argument_type   f,
                                 second_argument_type  g);}
         { Computes the resultant of $f$ and $g$, 
           with respect to the outermost variable.}

%\ccMethod{result_type operator()(first_argument_type  f,
%                                 second_argument_type g,
%                                 int  i);}
%         { Computes the resultant of $f$ and $g$,
%           with respect to variable $x_i$. 
%           \ccPrecond $0 \leq i  < d$ 
%         }

%\ccHasModels

\ccSeeAlso

\ccRefIdfierPage{Polynomial_d}\\
\ccRefIdfierPage{PolynomialTraits_d}\\
\ccRefIdfierPage{PolynomialTraits_d::UnivariateContent}\\
\ccRefIdfierPage{PolynomialTraits_d::PolynomialSubresultants}\\
\ccRefIdfierPage{PolynomialTraits_d::PrincipalSubresultants}\\

\end{ccRefConcept}
