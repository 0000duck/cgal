%OPEN: new name for template argument ROOT ?
%But this will cause changes in the interface  i.e. root() should change too.

\begin{ccRefClass}{Polynomial<Coeff>}
\def\ccTagOperatorLayout{\ccFalse}

\ccDefinition

An instance of the data type \ccc{Polynomial} represents a
polynomial $p = a_0 + a_1*x + ...a_i*x^i$ from the ring $Coeff[x]$.
\ccc{Coeff} can itself be an instance of \ccc{Polynomial}, yielding a form of
multivariate polynomials. 

The template argument \ccc{Coeff} must be at 
least a model of \ccc{IntegralDomainWithoutDiv}.
For all operations naturally involving division, an \ccc{IntegralDomain} 
is required. 
%If more than a \c IntegralDomain is required, this is documented.
\ccc{Polynomial} offers a full set of algebraic operators, i.e.
binary  {\tt +}, {\tt  -}, {\tt  *}, {\tt /} as well as  
{\tt +=}, {\tt -=}, {\tt *=}, {\tt  /=}; not only for polynomials
but also for a polynomial and a number of the coefficient type.
(The  {\tt /} operator must only be used for integral divisions, i.e.
those with remainder zero.)
The operations are implemented naively:  {\tt +} and  {\tt -} need a number of \ccc{Coeff}
operations which is linear in the degree while * is quadratic.
%The advanced algebraic operations are implemented with classical
%non-modular methods.
Unary  {\tt +} and {\tt -} and (in)equality {\tt ==}, {\tt !=} are provided as well.

\ccc{Polynomial} is a model of \ccc{LessThanComparable} if \ccc{Coeff} is a 
model of \ccc{LessThanComparable}. In this case \ccc{Polynomial} provides 
comparison operators  {\tt <}, {\tt >}, {\tt <=}, {\tt >=}, where the comparison amounts to 
lexicographic comparison of the coefficient sequence,
with the coefficient of the highest power taking precedence over
those of lower powers.

\ccc{Polynomial} is a model of \ccc{Fraction} if \ccc{Coeff} is a 
model of \ccc{Fraction}. In this case Polynomial may be decomposed into a 
(scalar) denominator and a compound numerator with a simpler coefficient type. 
Often operations can be performed faster on these denominator-free multiples. 
%In case a \ccc{Polynomial} is a \ccc{Fraction} the relevant functionality as well as 
%the numerator and denominator type are provided by \ccc{CGAL::Fraction_traits}. 
%In particular \ccc{CGAL::Fraction_traits} provides a tag \ccc{Is_fraction} that can 
%be used for dispatching.

\ccc{Polynomial} is a model of \ccc{Modularizable} if \ccc{Coeff} is a 
model of \ccc{Modularizable}, where the homomorphic map on the polynomials
is simply defined as the canonical extension of the homomorphic map which is 
defined on the coefficient type.  



\ccImplementation
Inexact and limited-precision types can be used as coefficients,
but at the user's risk. The algorithms implemented were written with
exact number types in mind.

This data type is implemented as a handle type with value semantics
using \ccc{CGAL::Handle_with_policy}, where \ccc{HandlePolicy}
is \ccc{Handle_policy_no_union}. 
An important invariant to be preserved by all methods is that
the coefficient sequence does not contain leading zero coefficients
(where leading means at the high-degree end), with the exception that
the zero polynomial is represented by a single zero coefficient.
%However, this has the disadvantage that each time a new polynomial is 
%created the leading coefficient is tested against zero. 

%This is not true anymore:
%An empty coefficient sequence denotes an undefined value.

\ccIsModel 
\begin{tabular}{ll}
\ccc{Polynomial_d}\\
\ccc{Assignable}\\
\ccc{CopyConstructible}\\
\ccc{DefaultConstructible}\\
\ccc{EqualityComparable}\\
\ccc{ImplicitInteroperable} & with int\\
\ccc{ImplicitInteroperable} & with Coeff\\
\ccc{Fraction}              & if \ccc{Coeff} is model of \ccc{Fraction}\\
\ccc{LessThanComparable}    & if \ccc{Coeff} is model of \ccc{LessThanComparable}\\
\ccc{Modularizable}         & if \ccc{Coeff} is model of \ccc{Modularizable}.
\end{tabular}

\ccCreation

\ccCreationVariable{poly}
\ccConstructor{Polynomial ();} 
{Introduces an variable initialized with 0.}
\ccGlue
\ccConstructor{Polynomial (const Polynomial& x);}
{copy constructor.}
\ccGlue
\ccConstructor{Polynomial (const int &i);}
{Constructor from int.}
\ccGlue
\ccConstructor{Polynomial (const Coeff &x);}
{Constructor from type Coeff.}
\ccGlue
\ccConstructor{ 
template <class Forward_iterator>
Polynomial(Forward_iterator first, Forward_iterator last);}
{Constructor from iterator range with value type Coeff.}

\ccTypes

\ccOperations
%\ccMethod{const NT & a0 () const ;    }{Const access operator for a0} \ccGlue

\ccMethod{const_iterator begin() const;}
{A const random access iterator pointing to the constant coefficient.} \ccGlue
\ccMethod{const_iterator end()   const;}
{A const random access iterator pointing beyond the leading coefficient.}\ccGlue
\ccMethod{int degree() const;}
{The degree of the polynomial in $x$. The degree of the zero polynomial is 0.}\ccGlue
\ccMethod{const NT& operator[](unsigned int i) const;}
{Const access to the coefficient of $x^i$.}\ccGlue
\ccMethod{const NT& lcoeff() const;}
{Const access to the leading coefficient.}\ccGlue


\ccFunction{
std::ostream& operator<<(std::ostream& os, const Polynomial<Coeff> &poly);}
{Writes \ccc{poly} to ostream \ccc{os}. 
The format depends on the \ccc{CGAL::IO::MODE} of \ccc{os}.
In case the mode is \ccc{CGAL::IO::ASCII} the format is $P[d,a_0,\dots,a_d]$,
where $d$ is the degree of the polynomial. 
In case the mode is \ccc{CGAL::IO::PRETTY} the format is human readable.}


\ccFunction{
std::istream& operator>>(std::istream& is, const Polynomial<Coeff> &poly);}
{Reads \ccc{poly} from istream \ccc{is} in format $POLY[d,a_0,\dots,a_d]$,  
the output format in mode \ccc{CGAL::IO::ASCII} }


\end{ccRefClass}
