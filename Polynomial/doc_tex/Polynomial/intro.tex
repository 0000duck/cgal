
\ccChapterAuthor{Michael Hemmer}

\section{Introduction}

This package introduces a concept \ccc{Polynomial_d}, a concept for multivariate 
polynomials in $d$ variables. Though the concept is written for an arbitrary 
number of variables, the number of variables is considered as fixed for a 
particular model of \ccc{Polynomial_d}. 
The concept also allows univariate polynomials.

First of all a model of \ccc{Polynomial_d} is considered as an algebraic 
structure, that is, the usual arithmetic operations like +,-,* etc. are 
provided due to the fact that \ccc{Polynomial_d} refines at least the 
concept \ccc{IntegralDomainWithoutDivision}. 
However, a model of \ccc{Polynomial_d} has to be accompanied by a traits class 
\ccc{CGAL::Polynomial_traits_d<Polynomial_d>} being a model of 
\ccc{PolynomialTraits_d}. 
In principal this traits class provides all further functionalities 
on polynomials.

Given a $d$-variate polynomial over some base ring $R$ there are at least 
two different possible views on such a polynomial.
\begin{itemize}
\item The recursive or univariate view:  
In this view, a polynomial is considered as 
an element of $R[x_0,\dots,x_{d-2}][x_{d-1}]$. That is, the polynomial 
is treated as a univariate polynomial over the ring $R[x_0,\dots,x_{d-2}]$. 
\item The symmetric or multivariate view: 
This view is almost symmetric with respect to all variables. It considers 
considers the polynomials as an element of $R [x_0,\dots,x_{d-1}]$.
\end{itemize}

According to these two different views the traits class provides two 
different coefficients types:
\begin{itemize}
\item \ccc{CGAL::Polynomial_traits_d<Polynomial_d>::Coefficient} 
representing $R[x_0,\dots,x_{d-2}]$.
\item \ccc{CGAL::Polynomial_traits_d<Polynomial_d>::Innermost_coefficient} 
representing the base ring $R$.
\end{itemize}

Another important type which is introduced by this package is 
\ccc{CGAL::Exponent_vector}. 
It is derived from \ccc{std::vector<int>} and used to identify multivariate 
monomials. For instance the exponent vector containing the sequence 
$[3,2,4]$ corresponds to the trivariate monomial $x_0^3x_1^2x_2^4$. 


% ###########################
\section{Constructing a multivariate polynomial}

First of all the concept \ccc{Polynomial_d} requires that the model is 
constructible from int. This is due to the fact that \ccc{Polynomial_d} 
refines \ccc{IntegralDomainWithoutDivision} which in turn refines 
\ccc{FromIntConstructible}. Of course this allows only the construction of 
constant polynomials. 

In general a polynomial is constructed using the functor 
\ccc{CGAL::Polynomial_traits_d<Polynomial_d>::Construct_polynomial} a model of 
\ccc{PolynomialTraits_d::ConstructPolynomial}.  Basically there are two options:
\begin{itemize}
\item The polynomial is constructed from an iterator range with value type 
\ccc{CGAL::Polynomial_traits_d<Polynomial_d>::Coefficient}, 
where the $begin$ iterator refers to the constant term 
(constant with respect to the outermost variable). 
\item The polynomial is constructed from an iterator range with value type 
\ccc{std::pair< CGAL::Exponent_vector, 
CGAL::Polynomial_traits_d<Polynomial_d>::Innermost_coefficient>},
where each pair defines the coefficient for the monomial defined by 
the exponent vector. 
\end{itemize}

However, in some cases it might be more convenient to just construct 
the polynomials representing the different variables and to obtain the 
final polynomial using 
algebraic expressions. The most elegant way to construct a certain variable is 
\ccc{CGAL::Polynomial_traits_d<Polynomial_d>::Shift} being a model of 
\ccc{PolynomialTraits_d::Shift}. 

\subsection{Examples}

The following example illustrates different ways to construct a 
bivariate polynomial:
\ccIncludeExampleCode{Polynomial/construction.cpp}



% ###########################
\section{Coefficient Access}
 
In order to obtain a certain coefficient the traits class provides 
several functors. 
Note that the functors do not allow a write access to the coefficients. 
\begin{itemize}
\item \ccc{PolynomialTraits_d::GetCoefficient}: 
a model of this concept provides access to a coefficient in the univariate view,
that is, it returns elements of $R[x_0,\dots,x_{d-2}]$.
\item \ccc{PolynomialTraits_d::GetInnermostCoefficient}: 
a model of this concept provides access to a coefficient in the multivariate 
view, that is, it returns elements of $R$.
\item \ccc{PolynomialTraits_d::LeadingCoefficient}: 
a model of this concept provides access to the leading coefficient in 
the univariate view. 
\item \ccc{PolynomialTraits_d::InnermostLeadingCoefficient}: 
a model of this concept provides access to the leading coefficient in 
the multivariate view, that is,
it returns the (innermost) coefficient of the leading multivariate monomial. 
See also  \ccc{PolynomialTraits_d::DegreeVector}.          
\end{itemize}

\subsection{Examples}
The following example illustrates the application of the functors 
discussed above:
\ccIncludeExampleCode{Polynomial/coefficient_access.cpp}


\section{Degree, total degree and degree vector}

There are three functors in \ccc{PolynomialTraits_d} 
related to the degree of a polynomial. 
\begin{itemize}
\item \ccc{PolynomialTraits_d::Degree}: 
a model of this concept returns the degree of the polynomial in the univariate 
view. By default this is the degree with respect to the outermost variable, 
but it is also possible to select another variable. 
\item \ccc{PolynomialTraits_d::TotalDegree}: 
a model of this concept returns the so total degree of a polynomial. 
The polynomial is considered as a multivariate polynomial. 
The total degree is the maximum over the sums of the exponents of each 
multivariate monomial. 
\item \ccc{PolynomialTraits_d::DegreeVector}:
a model of this concept returns the exponent vector of the leading monomial,
where the monomial order is lexicographic and starts with the outermost 
variable. See also  \ccc{PolynomialTraits_d::InnermostLeadingCoefficient}. 
\end{itemize}

\subsection{Examples}
The following example illustrates the application of the functors discussed 
above:
\ccIncludeExampleCode{Polynomial/degree.cpp}

\section{Changing the order of variables}

Given for instance a bivariate polynomial it is conceivable that one wants to 
interchange the role of $x$ and $y$. That is one wants to interpret the 
$x$ as $y$ and vice versa.
For such a case the polynomial traits provides \ccc{PolynomialTraits_d::Swap}:\\
Given a polynomial $p$ and to two indices $i$ and $j$, 
the functor returns the polynomial in which $x_i$ is substituted by $x_j$ and 
vice versa, that is, the variables swap their positions.
The order of the other variables remains untouched. 

Another scenario is, that a particular variable should be moved to another 
position, for instance, it should become the outermost variable while the 
relative order of the other variables remains unchanged. 
For such a case the polynomial traits provides \ccc{PolynomialTraits_d::Move}.

\subsection{Examples}
The following example illustrates the application of the functors discussed 
above:
\ccIncludeExampleCode{Polynomial/swap_move.cpp}


\section{GCD and More}

Since the concept \ccc{PolynomialTraits_d} refines the concept 
\ccc{AlgebraicStructureTraits} the polynomial traits provides functors for 
integral division, division with remainder, greatest common divisor, etc.
But note that the algebraic structure of a polynomial depends on the algebraic 
structure of the innermost coefficient, for instance, a gcd is only available 
if and only if the innermost coefficient is a \ccc{Field} or a 
\ccc{UniqueFactorizationDomain}. Hence, we can not provide a $gcd$ if the 
innermost coefficient is just an \ccc{IntegralDomain} since it is simply 
not well defined\footnote{%
An example for such a number type is the template 
\ccc{CGAL::Sqrt_extension<NT,ROOT>} representing an algebraic extension of 
degree 2. This is just an \ccc{IntegralDomain} if NT is not a \ccc{Field}. }.
% 
However, if we would consider the polynomial over the quotient field of the 
integral domain the $gcd$ would be well defined. The only problem is
that the the result can not be represented over the ring since it contains 
denominators.   
Therefore, the \ccc{PolynomialTraits_d} requires functors such as 
\ccc{PolynomialTraits_d::GcdUpToConstantFactor}. 
This functor computes the gcd of two polynomials up to a constant factor (utcf).
That is, it returns the correct gcd for polynomials over the quotient field, 
but multiplied by some constant such that the result is representable with 
coefficients in the ring. 

However, note that these 'utcf' functions are usually a bit faster than their 
counter parts. This is due to the fact that the 'utcf' functions are allowed 
to skip the computation of the correct constant factor.
Note that in many cases the constant factor is in fact not needed.
In particular if the polynomials are supposed to represent some vanishing set, 
that is, an algebraic curve or surface.  

The concepts for the related functors are: 
\begin{itemize}
\item \ccc{AlgebraicStructureTraits::Gcd}\\     
      \ccc{PolynomialTraits_d::GcdUpToConstantFactor}

\item \ccc{AlgebraicStructureTraits::IntegralDivision}\\        
      \ccc{PolynomialTraits_d::IntegralDivisionUpToConstantFactor}

\item \ccc{PolynomialTraits_d::UnivariateContent}
      \ccc{PolynomialTraits_d::UnivariateContentUpToConstantFactor}\\
       
\item \ccc{PolynomialTraits_d::SquareFreeFactorize}\\
      \ccc{PolynomialTraits_d::SquareFreeFactorizeUpToConstantFactor}
\end{itemize}

In case of \ccc{PolynomialTraits_d::MultivariateContent} a counter part does 
not make sense, since it is defined as the gcd of all innermost coefficients. 

There is also no counterpart to \ccc{PolynomialTraits_d::MakeSquareFree}, since 
the result is canonicalized, which is supposed to remove superfluous constant 
factors anyway. (See also \ccc{PolynomialTraits_d::Canonicalize}.)


Another analog functionality is the pseudo division. 
The related functors recoup the usual division with remainder in case the 
Polynomial is not a \ccc{EuclideanRing}. \\
The concepts for the related functors are: 
\begin{itemize}
\item \ccc{AlgebraicStructureTraits::Div_mod}
\item \ccc{PolynomialTraits_d::PseudoDivision}
\item \ccc{AlgebraicStructureTraits::Div}
\item \ccc{PolynomialTraits_d::PseudoDivisionQuotient}
\item \ccc{AlgebraicStructureTraits::Mod}
\item \ccc{PolynomialTraits_d::PseudoDivisionRemainder}
\end{itemize}


\subsection{Examples}
The following example illustrates the application of some functors 
discussed above:
\ccIncludeExampleCode{Polynomial/gcd_up_to_constant_factor.cpp}

\section{Evaluation and Substitution}

Of course, it should also be possible to evaluate a polynomial 
or substitute its variables. We also require a special functor to 
determine whether a polynomial is zero at a given point. 
In case the inner most coefficient is \ccc{RealEmbeddable} the traits 
also must provide a function to compute the sign at a given point. \\
The concepts for the related functors are:
\begin{itemize}
\item \ccc{PolynomialTraits_d::Substitute}
\item \ccc{PolynomialTraits_d::Evaluate}
\item \ccc{PolynomialTraits_d::IsZeroAt}
\item \ccc{PolynomialTraits_d::SignAt}
\end{itemize}

The traits is also required to provide variants of these functors that 
interpret the polynomial as a homogeneous polynomial. 
For instance, these functors can be useful to determine the sign of a 
polynomial at a rational point although the polynomial is defined 
over integer coefficients. \\
For more details have a look at the following concepts:
\begin{itemize}
\item \ccc{PolynomialTraits_d::SubstituteHomogeneous}
\item \ccc{PolynomialTraits_d::EvaluateHomogeneous}
\item \ccc{PolynomialTraits_d::IsZeroAtHomogeneous}
\item \ccc{PolynomialTraits_d::SignAtHomogeneous}
\end{itemize}

Note that substitute allows the substitution of the variables by any type that 
is \ccc{ExplicitInteroperable} with the innermost coefficient type. 
This is a very powerful tool since it allows the substitution of the variables 
by polynomials. However, for some standard manipulations such as translation 
or scaling we require special functors since they are expected to be faster 
than their equivalent implementation using substitution: \\

\begin{itemize}
\item \ccc{PolynomialTraits_d::Shift}
\item \ccc{PolynomialTraits_d::Negate}
\item \ccc{PolynomialTraits_d::Invert}
\item \ccc{PolynomialTraits_d::Translate}
\item \ccc{PolynomialTraits_d::TranslateHomogeneous}
\item \ccc{PolynomialTraits_d::Scale}
\item \ccc{PolynomialTraits_d::ScaleHomogeneous}
\end{itemize}

\subsection{Examples}   
The following example illustrates the application of some functors 
discussed above:
\ccIncludeExampleCode{Polynomial/substitute.cpp}