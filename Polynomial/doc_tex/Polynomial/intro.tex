
\ccChapterAuthor{Michael Hemmer}

\section{Introduction}

This package introduces a concept \ccc{Polynomial_d}, a concept for multivariate 
polynomials in $d$ variables. Though the concept is written for an arbitrary 
number of variables, the number of variables is considered as fixed for a particular
model of \ccc{Polynomial_d}. The concept also allows univariate polynomials, in which case 
the variable $d$ is just one. 

First of all a model of \ccc{Polynomial_d} is considered as an algebraic structure, that 
is, the usual arithmetic operations like +,-,* etc. are provided due to the fact that 
\ccc{Polynomial_d} refines at least the concept \ccc{IntegralDomainWithoutDivision}. 
However, a model of \ccc{Polynomial_d} has to be accompanied by a traits class 
\ccc{CGAL::Polynomial_traits_d<Polynomial_d>} being a model of \ccc{PolynomialTraits_d}. 
In principal this traits class provides all further functionalities on polynomials.

Given a multivariate polynomial over some base ring $R$ in $d$ variables there are at least 
two different possible views on such a polynomial.
\begin{itemize}
\item The recursive or univariate view:  In this view, a polynomial is considered as 
an element of $R[x_0,\dots,x_{d-2}][x_{d-1}]$. That is, the polynomial 
is treated as a univariate polynomial over the ring $R[x_0,\dots,x_{d-2}]$. 
\item The symmetric or multivariate view: This view is symmetric 
with respect to all variables, considering the polynomials as element of $R [x_0,\dots,x_{d-1}]$.
\end{itemize}

According to these two different views the traits class provides two different coefficients types:
\begin{itemize}
\item \ccc{CGAL::Polynomial_traits_d<Polynomial_d>::Coefficient} representing $R[x_0,\dots,x_{d-2}]$.
\item \ccc{CGAL::Polynomial_traits_d<Polynomial_d>::Innermost_coefficient} representing the base ring $R$.
\end{itemize}

Another important type which is introduced by this package is the so called \ccc{CGAL::Exponent_vector}. 
In principal this is just an \ccc{std::vector<int>} which is used to identify monomials. 
For instance the exponent vector containing the sequence $[3,2,4]$ corresponds to the 
trivariate monomial $x_0^3x_1^2x_2^4$. 


% ###########################
\section{Constructing a multivariate polynomial}

First of all the concept \ccc{Polynomial_d} requires that the model is constructible from int. 
This is due to the fact that \ccc{Polynomial_d} refines \ccc{IntegralDomainWithoutDivision} 
which in turn refines \ccc{FromIntConstructible}. Of course this allows only the construction of 
constant polynomials. 

Hence, in general a polynomial is constructed using the functor 
\ccc{CGAL::Polynomial_traits_d<Polynomial_d>::Construct_polynomial} a model of 
\ccc{PolynomialTraits_d::ConstructPolynomial}.  Basically there are two options:
\begin{itemize}
\item The polynomial is constructed from an iterator range with value type 
\ccc{CGAL::Polynomial_traits_d<Polynomial_d>::Coefficient}, where begin refers to 
the constant term (constant with respect to the outermost variable). 
\item The polynomial is constructed from an iterator range with value type 
\ccc{std::pair< CGAL::Exponent_vector, CGAL::Polynomial_traits_d<Polynomial_d>::Innermost_coefficient>},
where each pair defines the coefficient for the monomial defined by the exponent vector. 
\end{itemize}

However, in some cases it might be more convenient to just construct the polynomials 
representing the different variables and to obtain the final polynomial using 
algebraic expressions. The most elegant way to construct a certain variable is 
\ccc{CGAL::Polynomial_traits_d<Polynomial_d>::Shift} being a model of 
\ccc{PolynomialTraits_d::Shift}. 

\subsection{Examples}

The following example illustrates different ways to construct a bivariate polynomial:
\ccIncludeExampleCode{Polynomial/construction.cpp}



% ###########################
\section{Coefficient Access}
 
In order to obtain a certain coefficient the traits class provides several functors. 
Note that the functors do not allow a write access to the coefficients. 
\begin{itemize}
\item \ccc{PolynomialTraits_d::GetCoefficient}: 
a model of this concept provides access to a coefficient in the univariate view. 
\item \ccc{PolynomialTraits_d::GetInnermostCoefficient}: 
a model of this concept provides access to a coefficient in the multivariate view.
\item \ccc{PolynomialTraits_d::LeadingCoefficient}: 
a model of this concept provides access to the leading coefficient in the univariate view. 
\item \ccc{PolynomialTraits_d::InnermostLeadingCoefficient}: 
a model of this concept provides access to the leading coefficient in the multivariate view, that is,
it returns the (innermost) coefficient of the leading multivariate monomial. 
See also  \ccc{PolynomialTraits_d::DegreeVector}.          
\end{itemize}

\subsection{Examples}
The following example illustrates the application of the functors discussed above:
\ccIncludeExampleCode{Polynomial/coefficient_access.cpp}



\section{Degree}

Currently there are three functors in \ccc{PolynomialTraits_d} related to the degree of 
a polynomial. 
\begin{itemize}
\item \ccc{PolynomialTraits_d::Degree}: 
a model of this concept returns the degree of the polynomial in the univariate view. 
By default this is the degree with respect to the outermost variable, 
but it is also possible to select another variable. 
\item \ccc{PolynomialTraits_d::TotalDegree}: 
a model of this concept returns the so total degree of a polynomial. The polynomial
is considered as a multivariate polynomial. The total degree is the maximum over 
the sums of the exponents of each multivariate monomial. 
\item \ccc{PolynomialTraits_d::DegreeVector}:
a model of this concept returns the exponent vector of the leading monomial.  where the 
monomial order is lexicographic and starts with the outermost variable. 
See also  \ccc{PolynomialTraits_d::InnermostLeadingCoefficient}. 
\end{itemize}

\subsection{Examples}
The following example illustrates the application of the functors discussed above:
\ccIncludeExampleCode{Polynomial/degree.cpp}

\section{Changing the order of variables}

Given for instance a bivariate polynomial it may likely happen that one once to 
interchange the role of $x$ and $y$. That is one wants to interpret the $x$ as $y$ 
and vice versa. For such a case the polynomial traits provides \ccc{PolynomialTraits_d::Swap}. 
Given a polynomial $p$ and to two indices $i$ and $j$, the functor returns the polynomial 
in which $x_i$ is substituted by $x_j$ and vice versa, that is, the variables swap their positions.
The order of the other variables remains untouched. 

Another scenario is, that a particular variable should be moved to another position, e.g., it should become the 
outermost variable while the relative order of the other variables remains unchanged. 
For such a case the polynomial traits provides \ccc{PolynomialTraits_d::Move}.


\subsection{Examples}
The following example illustrates the application of the functors discussed above:
\ccIncludeExampleCode{Polynomial/swap_move.cpp}









Many functors consider the polynomial as a univariate polynomial in one variable.
By default this is the outermost variable $x_{d-1}$. However, in general it 
is possible to select a certain variable. 






\section{Short Introduction for the reviewer}


Note that, depending on the coefficient type,  
it is sometimes not possible to provided a certain functor. 
For instance it is not possible to provide  a \ccc{SignAt} if the coefficient type
is not \ccc{RealEmbeddable}. A similar argument holds for functors requiring  
a gcd on the innermost coefficient. However, whenever possible the 
\ccc{PolynomialToolBox_d} provides an alternative functor, which computes the 
desired entity up to a constant factor. For an example see 
\ccc{SquareFreeFactorize} and \ccc{SquareFreeFactorizeUpToConstantFactor}.

The main idea of the concepts is, that the \ccc{PolynomialTraits_d} provides two 
different views on a multivariate polynomial. 

\begin{itemize}
\item 
A recursive view, that sees the polynomial as an element of 
$R[x_0,\dots,x_{d-2}][x_{d-1}]$. In this view, the polynomial is handled as
an univariate polynomial over the ring $R[x_0,\dots,x_{d-2}]$. 
\item 
A symmetric view, which is symmetric with respect to all variables,
seeing the polynomials as element of $R[x_0,\dots,x_{d-1}]$.
\end{itemize}

All functors related to the univariate view are written such that 
outermost variable $x_{d-1}$ is the default variable. For instance, 
\ccc{PolynomialTraits_d::Degree}$()(p)$ returns the degree of $p$ with respect to $x_{d-1}$. 
However, it is also possible to call 
\ccc{PolynomialTraits_d::Degree}$()(p,i)$ returning the 
degree with respect to variable $x_{i}$. The corresponding functor for the 
multivariate view is \ccc{PolynomialTraits_d::TotalDegree} which is 
symmetric in all variables. 