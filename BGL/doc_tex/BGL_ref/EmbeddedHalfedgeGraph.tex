%% Copyright (c) 2004  SciSoft.  All rights reserved.
%%
%% This file is part of CGAL (www.cgal.org); you may redistribute it under
%% the terms of the Q Public License version 1.0.
%% See the file LICENSE.QPL distributed with CGAL.
%%
%% Licensees holding a valid commercial license may use this file in
%% accordance with the commercial license agreement provided with the software.
%%
%% This file is provided AS IS with NO WARRANTY OF ANY KIND, INCLUDING THE
%% WARRANTY OF DESIGN, MERCHANTABILITY AND FITNESS FOR A PARTICULAR PURPOSE.
%%
%% 
%%
%% Author(s)     : Fernando Cacciola <fernando_cacciola@hotmail.com>

\begin{ccRefConcept}{EmbeddedHalfedgeGraph}

%% \ccHtmlCrossLink{}     %% add further rules for cross referencing links
%% \ccHtmlIndexC[concept]{} %% add further index entries

\ccDefinition

The concept \ccRefName\ describes the requirements of a \ccc{HalfedgeGraph}
with a geometric embedding.

\ccHeading{Requirements}

The graph is geometrically embedded, that is, each vertex is associated with
 a generalized point (whether in the plane, the sphere, a surface, or whatever.)
 
\ccRefines
\ccc{HalfedgeGraph}

A model of this graph is required to have the {\em interior property} \ccc{vertex_point} attached to its vertices.

\ccHeading{Associated Types}

\ccNestedType{embedded_graph_traits<EmbeddedHalfedeGraph>::Point}{The type of generalized point representing the geometric embedding of the graph.}


\end{ccRefConcept}

% +------------------------------------------------------------------------+
%%RefPage: end of main body, begin of footer
% EOF
% +------------------------------------------------------------------------+
