%% Copyright (c) 2004  SciSoft.  All rights reserved.
%%
%% This file is part of CGAL (www.cgal.org); you may redistribute it under
%% the terms of the Q Public License version 1.0.
%% See the file LICENSE.QPL distributed with CGAL.
%%
%% Licensees holding a valid commercial license may use this file in
%% accordance with the commercial license agreement provided with the software.
%%
%% This file is provided AS IS with NO WARRANTY OF ANY KIND, INCLUDING THE
%% WARRANTY OF DESIGN, MERCHANTABILITY AND FITNESS FOR A PARTICULAR PURPOSE.
%%
%% 
%%
%% Author(s)     : Fernando Cacciola <fernando_cacciola@hotmail.com>


\begin{ccRefClass}{CGAL::Polyhedron_vertex_point_map<Traits>}

%% add template arg's if necessary

%% \ccHtmlCrossLink{}     %% add further rules for cross referencing links
%% \ccHtmlIndexC[class]{} %% add further index entries
\ccDefinition

The class \ccRefName\ provides a \ccc{BGL} \ccc{PropertyMap} which gives lvalue access to the Point of a \ccc{Polyhedron_3} Vertex.

\ccInclude{CGAL/boost/graph/Polyhedron_BGL_properties.h}

\ccTypes
  \ccTypedef{boost::lvalue_property_map_tag category;}
    {Indicates that this property is mutable (the property is an lvalue).}
  
  \ccTypedef{typename Polyhedron::Point_3 value_type;}
    {The type of the property.}
    
  \ccTypedef{value_type& reference;}
    {The result type of the map operator.}
    
  \ccTypedef{typename boost::graph_traits< Polyhedron_3<Traits> >::vertex_descriptor key_type;}
  {The key type of the property map: A \ccc{Polyhedron_3::Vertex_handle}.}

\ccCreation
\ccCreationVariable{pm}  %% choose variable name

\ccConstructor{CGAL::Polyhedron_vertex_point_map<Traits>( CGAL::Polyhedron_3<Traits>& poly); }
{Initializes an instance storing a pointer to the polyhedron \ccc{poly}.}

\ccOperations

\ccMethod
  {reference operator[]( key_type const& vertex ) const;}
  {Returns vertex->point().}  
    
\ccIsModel
\ccc{ProperyMap}

\end{ccRefClass}

% +------------------------------------------------------------------------+
%%RefPage: end of main body, begin of footer
% EOF
% +------------------------------------------------------------------------+

