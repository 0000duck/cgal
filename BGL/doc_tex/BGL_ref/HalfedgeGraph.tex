%% Copyright (c) 2004  SciSoft.  All rights reserved.
%%
%% This file is part of CGAL (www.cgal.org); you may redistribute it under
%% the terms of the Q Public License version 1.0.
%% See the file LICENSE.QPL distributed with CGAL.
%%
%% Licensees holding a valid commercial license may use this file in
%% accordance with the commercial license agreement provided with the software.
%%
%% This file is provided AS IS with NO WARRANTY OF ANY KIND, INCLUDING THE
%% WARRANTY OF DESIGN, MERCHANTABILITY AND FITNESS FOR A PARTICULAR PURPOSE.
%%
%% 
%%
%% Author(s)     : Fernando Cacciola <fernando_cacciola@hotmail.com>

\begin{ccRefConcept}{HalfedgeGraph}

%% \ccHtmlCrossLink{}     %% add further rules for cross referencing links
%% \ccHtmlIndexC[concept]{} %% add further index entries

\ccDefinition

The concept \ccRefName\ describes the requirements of a
Boost \ccAnchor{http://www.boost.org/libs/graph/doc/graph_concepts.html}{graph}
representing a subset of a \ccc{HalfedgeDS}.

This graph is a refinement of \ccc{DirectedEmbeddedGraph} with 
some added connectivity requirements:
 
\begin{enumerate}
\item For each directed edge $\overrightarrow{st}$
it's opposite edge $\overrightarrow{ts}$ is part of the graph.

\item For any directed edge it's {\em succesor} (rotational and linear)
is part of the graph.

\item For any directed edge it's {\em predeccesor} (rotational and linear) 
is part of the graph.
\end{enumerate}

Furthermore, because they must come in pairs, there is the additional
notion of an {\em undirected} edge for a pair of opposite edges.
The number of undirected edges is exactly half the number of directed edges.

The directed edges are not called \ccc{halfedges} (as in a
\ccc{HalfedgeDS}) because from the point of view of this graph, being
a refinement of a {\sc Bgl} graph, each directed edge is an
edge in itself. In other words, the unqualified term edge
refers to one and only one directed edge and not to a pair. The
pair of opposite edges, seen as a unity, is an {\em undirected edge}.
 
Note that this concept does not imply the existence of two different types
for directed and undirected edges. Instead, if specifies that there is only one type of edge,
hence only one \ccc{edge_descriptor}, which refers to the directed edge.
An undirected edge must be implicitely addressed by a single \ccc{edge_descriptor},
and there is no requirement about which of the directed edges of the pair must
be used to refer to an undirected edge.

All edges are {\em cyclically connected} and each cycle implicitely
corresponds to a \ccc{facet} (or \ccc{face} or \ccc{cell}) of the
underlying data structure (with the exception of the cycle(s)
corresponding to the border(s) of the underlying DS, which might not
have an associated facet).


\ccRefines
\ccc{DirectedEmbeddedGraph}

\ccTypes
  \ccNestedType{undirected_edge_iterator}
  {An iterator that iterates over one and only one of the directed edges
  in each opposing pair. This iterator accesses a directed edge 
  (as represented by an \ccc{edge_descriptor}) and not an undirected edge 
  since there is no type explicitely representing it}{}
  
\ccOperations

Following the {\sc Bgl} design, the following graph operations are defined as free rather than member functions.

  \ccFunction
  {template<class Graph>
  std::pair < typename CGAL::halfedge_graph_traits<Graph const>::undirected_edge_iterator
             ,typename CGAL::halfedge_graph_traits<Graph const>::undirected_edge_iterator
             >   
  undirected_edges(Graph const& g );
  }
  {Returns a pair of iterators representing the range \ccc{[begin,end)} 
  of undirected edges in this graph.}

\ccHasModels
\ccRefIdfierPage{CGAL::halfedge_graph_traits< CGAL::Polyhedron_3<Traits> const > }\\
\ccRefIdfierPage{CGAL::halfedge_graph_traits< CGAL::Polyhedron_3<Traits> > }\\

\end{ccRefConcept}

% +------------------------------------------------------------------------+
%%RefPage: end of main body, begin of footer
% EOF
% +------------------------------------------------------------------------+
