%% Copyright (c) 2004  SciSoft.  All rights reserved.
%%
%% This file is part of CGAL (www.cgal.org); you may redistribute it under
%% the terms of the Q Public License version 1.0.
%% See the file LICENSE.QPL distributed with CGAL.
%%
%% Licensees holding a valid commercial license may use this file in
%% accordance with the commercial license agreement provided with the software.
%%
%% This file is provided AS IS with NO WARRANTY OF ANY KIND, INCLUDING THE
%% WARRANTY OF DESIGN, MERCHANTABILITY AND FITNESS FOR A PARTICULAR PURPOSE.
%%
%% 
%%
%% Author(s)     : Fernando Cacciola <fernando_cacciola@hotmail.com>

\begin{ccRefConcept}{HalfedgeGraph}

%% \ccHtmlCrossLink{}     %% add further rules for cross referencing links
%% \ccHtmlIndexC[concept]{} %% add further index entries

\ccDefinition

The concept \ccRefName\ describes the requirements of a
\ccAnchor{http://www.boost.org/libs/graph/doc/graph_concepts.html}{Boost Graph}
representing a subset of a \ccc{HalfedgeDS}.

This graph is a refinement of \ccc{DirectedEmbeddedGraph} with the
added requirement that for each directed edge $\overrightarrow{st}$
it's opposite edge $\overrightarrow{ts}$ is part of the graph, that is, the
directed edges are necessarily symmetric. Furthermore, because the
symmetry is mandatory, there is the added concept of an {\em
undirected} edge which corresponds to a pair of symmetric edges as a
whole.

For each pair of {\em symmetric directed edges} $\overrightarrow{st}$
and $\overrightarrow{ts}$, there is a correspoding {\em undirected}
edge $\overline{st}$. Therefore, the number of undirected edges is
exactly half the number of directed edges.

{\em Symmetric directed edges} are not called \ccc{halfedges} (as in a
\ccc{HalfedgeDS}) because from the point of view of this graph, being
a refinement of a {\sc Bgl} graph, each symmetric directed edge is a {\em
whole} edge in itself. In other words, the unqualified term edge
refers to one and only one symmetric directed edge and not both. The
pair of opposing edges, seen as a unity, is an {\em undirected edge}.
 
This concept {\em does not} specify the existence of a explicit type representing 
an undirected edge. Instead, if specifies that there is only one type of edge,
hence only one \ccc{edge_descriptor}, which refers to the directed edge.
An undirected edge must be implicitely addressed by one single edge\_descriptor,
and there is no requirement about which of the directed edges of the pair must
be used to refer to an undirected edge.

All edges are {\em cyclically connected} and each cycle implicitely
corresponds to a \ccc{facet} (or \ccc{face} or \ccc{cell}) of the
underlying data structure (with the exception of the cycle(s)
corresponding to the border(s) of the underlying DS, which might not
have an associated facet).


\ccRefines
\ccc{DirectedEmbeddedGraph}

\ccTypes
  \ccNestedType{undirected_edge_iterator}
  {An iterator that iterates over one and only one of the directed edges
  in each opposing pair. This iterator accesses a directed edge 
  (as represented by an \ccc{edge_descriptor}) and not an undirected edge 
  since there is no type explicitely representing it}{}
  
\ccOperations

Following the {\sc Bgl} interface, the following graph operations are defined as free rather than member functions.

  \ccFunction
  {template<class Graph>
  std::pair < typename CGAL::Halfedge_graph_traits<Graph const>::undirected_edge_iterator
             ,typename CGAL::Halfedge_graph_traits<Graph const>::undirected_edge_iterator
             >   
  undirected_edges(Graph const& g );
  }
  {Returns a pair of iterators representing the range [begin,end) 
  of undirected edges in this graph.\\
  (non-const version ommited)
  }

\ccHasModels
\ccRefIdfierPage{CGAL::halfedge_graph_traits< CGAL::Polyhedron_3<Traits> const > }\\
\ccRefIdfierPage{CGAL::halfedge_graph_traits< CGAL::Polyhedron_3<Traits> > }\\

\end{ccRefConcept}

% +------------------------------------------------------------------------+
%%RefPage: end of main body, begin of footer
% EOF
% +------------------------------------------------------------------------+
