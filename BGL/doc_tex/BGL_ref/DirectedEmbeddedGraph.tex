%% Copyright (c) 2004  SciSoft.  All rights reserved.
%%
%% This file is part of CGAL (www.cgal.org); you may redistribute it under
%% the terms of the Q Public License version 1.0.
%% See the file LICENSE.QPL distributed with CGAL.
%%
%% Licensees holding a valid commercial license may use this file in
%% accordance with the commercial license agreement provided with the software.
%%
%% This file is provided AS IS with NO WARRANTY OF ANY KIND, INCLUDING THE
%% WARRANTY OF DESIGN, MERCHANTABILITY AND FITNESS FOR A PARTICULAR PURPOSE.
%%
%% 
%%
%% Author(s)     : Fernando Cacciola <fernando_cacciola@hotmail.com>

\begin{ccRefConcept}{DirectedEmbeddedGraph}

%% \ccHtmlCrossLink{}     %% add further rules for cross referencing links
%% \ccHtmlIndexC[concept]{} %% add further index entries

\ccDefinition

The concept \ccRefName\ describes the requirements of a basic
\ccAnchor{http://www.boost.org/libs/graph/doc/graph_concepts.html}{Boost Graph}
commonly found associated to many geometric data structures.

\ccHeading{Requirements}

All edges are {\em directed}, that is, they are defined as an {\em ordered pair} 
of vertices.

The graph is geometrically embedded, that is, each vertex is associated with
 a generalized point (whether in the plane, the sphere, a surface, etc.)

Vertices and edges are required to have a distinguishable \ccc{null} state 
such that any default constructed vertex or edge descriptor is formally
referring to a non-existing vertex or edge.

Since the graph has a geometric embedding, edges are ordered around a vertex 
and such ordering can be used to define the following edge relations:

\begin{description}

\item[Opposite] 
An edge \ccc{'v: x->y'} is said to be the {\em opposite} of edge \ccc{'u: a->b'} 
iff \ccc{a} $==$ \ccc{y} and \ccc{b} $==$ \ccc{c}.

\item[Rotational Successor] 
An edge \ccc{'v: s->q'} is said to be the {\em rotational successor} 
of edge \ccc{'u: s->p'} iff \ccc{v} is the {\em first} edge found by 
a counter-clockwise rotation of \ccc{u} around the source vertex \ccc{s}.
 
\item[Rotational Predecessor] 
An edge \ccc{'v: s->q'} is said to be the {\em rotational predecessor}
of edge \ccc{'u: s->p'} iff \ccc{v} is the {\em first} edge found by 
a clockwise rotation of \ccc{u} around the source vertex \ccc{s}.

\item[Linear Successor] 
An edge \ccc{'v: b->c'} is said to be the {\em linear successor} of edge 
\ccc{'u:a->b'} iff \ccc{v} is the rotational successor of the opposite
of v.
 
\item[Linear Predecessor] 
An edge \ccc{'v: b->c'} is said to be the {\em linear predecessor} of edge 
\ccc{'u:a->b'} iff \ccc{v} is the opposite of the rotational predecessor
of v.

\end{description}

\ccRefines
\ccAnchor{http://www.boost.org/libs/graph/doc/PropertyGraph.html}{PropertyGraph}\\

\ccTypes
  \ccNestedType{Point}{The type of generalized point representing the geometric embedding of the graph.}{what is this}

\ccOperations

Following the {\sc Bgl} design, the following graph operations are defined as free rather than member functions.

  \ccFunction
  {template<class Graph>
  typename boost::graph_traits<Graph const>::edge_descriptor 
  opposite_edge(typename boost::graph_traits<Graph const>::edge_descriptor e, Graph const& g );
  }
  {Returns the {\em opposite} of \ccc{e}.\\
  If the resulting edge does not belong to \ccc{g} the result is a \ccc{null} (default constructed) edge.\\
  (non-const version ommited)
  }
  
  \ccFunction
  {template<class Graph>
  typename boost::graph_traits<Graph const>::edge_descriptor 
  next_edge_ccw(typename boost::graph_traits<Graph const>::edge_descriptor e, Graph const& g );
  }
  {Returns the {\em rotational successor} of \ccc{e}.\\
  If the resulting edge does not belong to \ccc{g} the result is a \ccc{null} (default constructed) edge.\\
  (non-const version ommited)
  }
  
  \ccFunction
  {template<class Graph>
  typename boost::graph_traits<Graph const>::edge_descriptor 
  next_edge_cw(typename boost::graph_traits<Graph const>::edge_descriptor e, Graph const& g );
  }
  {Returns the {\em rotational predecessor} of \ccc{e}.\\
  If the resulting edge does not belong to \ccc{g} the result is a \ccc{null} (default constructed) edge. }
  
  \ccFunction
  {template<class Graph>
  typename boost::graph_traits<Graph const>::edge_descriptor 
  next_edge(typename boost::graph_traits<Graph const>::edge_descriptor e, Graph const& g );
  }
  {Returns the {\em linear successor} of \ccc{e}.\\
  If the resulting edge does not belong to \ccc{g} the result is a \ccc{null} (default constructed) edge. }
  
  \ccFunction
  {template<class Graph>
  typename boost::graph_traits<Graph const>::edge_descriptor 
  prev_edge(typename boost::graph_traits<Graph const>::edge_descriptor e, Graph const& g );
  }
  {Returns the {\em linear predecessor} of \ccc{e}.\\
  If the resulting edge does not belong to \ccc{g} the result is a \ccc{null} (default constructed) edge. }
  
  \ccFunction
  {template<class Graph>
  typename CGAL::embedded_graph_traits<Graph>::Point const& 
  get_point(typename boost::graph_traits<Graph const>::vertex_descriptor v, Graph const& g );
  }
  {Returns the generalized point embedding the vertex \ccc{v}.\\
  The exact nature of the point depends on the embedding of the graph and is
  specified by the concrete model.}
  
  \ccFunction
  {template<class Graph>
  void set_point(typename boost::graph_traits<Graph>::vertex_descriptor v
                ,Graph& g
                ,typename CGAL::embedded_graph_traits<Graph>::Point p
                );
  }
  {Sets \ccc{p} as the embedding of the vertex \ccc{v}.
  }

\ccHasModels
\ccRefIdfierPage{boost::graph_traits< CGAL::Polyhedron_3<Traits> const > }\\
\ccRefIdfierPage{boost::graph_traits< CGAL::Polyhedron_3<Traits> > }\\
\ccRefIdfierPage{CGAL::embedded_graph_traits< CGAL::Polyhedron_3<Traits> > }

\end{ccRefConcept}

% +------------------------------------------------------------------------+
%%RefPage: end of main body, begin of footer
% EOF
% +------------------------------------------------------------------------+
