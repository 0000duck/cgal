
A continuous curve $C$ in $\reals^2$ is called {\em $x$-monotone} if
every vertical line intersects it at a single point at most. For
example, the circle $x^2 + y^2 = 1$ is {\em not} $xy$-monotone as the
vertical line $x = 0$ intersects it at $(0, -1)$ and at $(0, 1)$;
however, it is possible to split the circle into an upper part and a
lower part, such that both these parts are $x$-monotone.

An $x$-monotone curve can be represented as a univariate function
$y = C(x)$, defined over some continuous range $R_C \subseteq \reals$.
Given a set $\calC = \{ C_1, C_2, \ldots, C_n \}$ of $x$-monotone
curves, their {\em lower envelope} is defined as the point-wise minimum of
all curves. Namely, the lower envelope of the set $\calC$ can be
defined as the following function:
\begin{eqnarray*}
\calL_{\calC} (x) = \min_{1 \leq k \leq n}{\overline{C}_k (x)} \ ,
\end{eqnarray*}
where we define $\overline{C}_k(x) = C_k(x)$ for $x \in R_{C_k}$,
and $\overline{C}_k(x) = \infty$ otherwise.

Similarly, the {\em upper envelope} of $\calC$ is the point-wise maximum of
the $x$-monotone curves in the set:
\begin{eqnarray*}
\calU_{\calC} (x) = \max_{1 \leq k \leq n}{\underline{C}_k (x)} \ ,
\end{eqnarray*}
where in this case $\underline{C}_k(x) = -\infty$ for $x 
\not\in R_{C_k}$.

Given a set of $x$-monotone curves $\calC$, the {\em minimization
diagram} of $\calC$ is a subdivision of the $x$-axis into cells,
such that the identity of the curves that induce the lower envelope
over a specific cell of the subdivision (an edge or a vertex) is the
same. In non-degenerate situation, an edge --- which represents a
continuous interval on the $x$-axis --- is induced by a single
curve (or by no curves at all, if there are no $x$-monotone curves
defined over the interval), and a vertex is induced by a single curve
and corresponds to one of its endpoints, or by two curves and
corresponds to their intersection point.
The {\em maximization diagram} is symmetrically defined for upper envelopes.
In the rest of this chapter, we refer to both these diagrams as
{\em envelope diagrams}.

Lower and upper envelopes can be efficiently computed using a
divide-and-conquer approach. First note that the envelope diagram for
a single $x$-monotone curve $C_k$ is trivial to compute: we project
the boundary of its range of definition $R_{C_k}$ onto the $x$-axis
and label the features it induces accordingly. Given a set
$\hat{\calC}$ of (non necessarily $x$-monotone) curves in $\reals^2$,
we start by subdividing each curve into a finite number of weakly
$x$-monotone curves\footnote{To handle degenerate inputs, we consider
vertical segments as {\em weakly} $x$-monotone.}, obtaining the set
$\calC$. We continue by splitting the set into two disjoint subsets
$\calC_1$ and $\calC_2$, and we compute their envelope diagrams
recursively. We finally have to merge the diagrams, and we do this in
linear time by traversing both diagrams in parallel.

\section{The Envelope Diagram}
\label{env2_sec:env_diag}
%=============================

\begin{figure}[t]
\begin{ccTexOnly}
  \begin{center}
    \epsfig{figure=Envelope_2/fig/min_diag.eps,width=5in,silent=}
  \end{center}
\end{ccTexOnly}
\begin{ccHtmlOnly}
  <p><center>
  <img src="./fig/min_diag.gif" border=0 alt="The minimization diagram">
  </center>
\end{ccHtmlOnly}
\caption{The lower envelope of eight line segments, labeled
$A, \ldots, H$\,, as constructed in \ccc{ex_envelope_segments.cpp}.
The minimization diagram is shown at the bottom, where
each diagram vertex points to the point associated with it, and the
labels of the segment that induce a diagram edge are displayed below
this edge. Note that there exists one edge that represents an overlap
and there are also a few edges that represent empty
intervals.\label{env2_fig:min_diag}}
\end{figure}

\section{Examples}
%=================

The following example demonstrates how to compute and traverse the
minimization diagram of line segments, ase illustrated in
Figure~\ref{env2_fig:min_diag}. We use the curve-data traits
parameterized by the \ccc{Arr_segment_traits_2} class in order to
attach a label (a \ccc{char} in this case) to each input segment.
We use these labels when we print the minimization diagram:

\ccIncludeExampleCode{../examples/Envelope_2/ex_envelope_segments.cpp}

The next example shows how to construct the envelope of unbounded
curves. In this case we use the \ccc{Arr_linear_traits_2} class to
compute the lower envelope of a set of lines. We read a set of
points $\calP = p_1, \ldots, p_n$ from an input file, and compute the
corresponding dual lines $\calP^{*} = p^{*}_1, \ldots, p^{*}_n$, where
the line $p^{*}$ dual to a point $p = (p_x, p_y)$ is given by
$y = p_x x - p_y$. We then compute the convex hull of the point-set
$\calP$, using the fact that the lines that form the
lower envelope of $\calP^{*}$ are dual to the points along the
{\em upper} part of $\calP$'s convex hull, and the lines that form the
upper envelope of $\calP^{*}$ are dual to the points along the
{\em lower} part of the convex hull; see,
e.g.,~\cite[Section~11.4]{bkos-cgaa-00} for more details.
Note that the leftmost edge of the minimization diagram is associated
with the same line as the rightmost edge of the maximization diagram,
and vic-versa. We can therefore skip the rightmost edges of both
diagrams:

\ccIncludeExampleCode{../examples/Envelope_2/ex_convex_hull.cpp}

\begin{figure}[t]
\begin{ccTexOnly}
  \begin{center}
    \epsfig{figure=Envelope_2/fig/ex_circle.eps,silent=}
  \end{center}
\end{ccTexOnly}
\begin{ccHtmlOnly}
  <p><center>
  <img src="./fig/ex_circle.gif" border=0 alt="Envelopes of circles">
  </center>
\end{ccHtmlOnly}
\caption{A set of four circles, as constructed in
\ccc{ex_envelope_circles.cpp}. The lower envelope and the upper
envelope are shown using thick dashed lines.\label{env2_fig:ex_circ}}
\end{figure}

We conclude by an example of envelopes of non-linear curves. 
We use the \ccc{Arr_circle_segment_traits_2} class to construct the
lower and the upper envelopes of a set of four circles, as depicted
in Figure~\ref{env2_fig:ex_circ}. Note that unlike the two previous
examples, here our curves are not $x$-monotone, so we use the functions
that compute envelopes of arbitrary curves:

\ccIncludeExampleCode{../examples/Envelope_2/ex_envelope_circles.cpp}
