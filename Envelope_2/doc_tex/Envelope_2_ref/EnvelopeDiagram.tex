% +------------------------------------------------------------------------+
% | Reference manual page: Envelopediagram_1.tex
% +------------------------------------------------------------------------+
% | 
% | Package: Envelope_2
% | 
% +------------------------------------------------------------------------+

\ccRefPageBegin

\begin{ccRefConcept}{EnvlopeDiagram_1}

\ccDefinition
%============

This concept defines the representation of an evelope diagram of a set
of planar curve. The {\em envelope diagram} is a subdivision of the $x$-axis
into $0$-dimensional cells ({\em vertices}) and $1$-dimensional cells
({\em edges}), such that the idenitity of the curves that induce the lower
envelope (or the upper envelope) over each cell is fixed.

A vertex is an envelope diagram is therefore associated with a point
on the lower (upper) envelope, and corresponds to either a curve endpoint
or to an intersection point of two curves (or more). The vertex is also
associated with a set of $x$-monotone curves that induce the envelope
over this point. Each vertex is incident to two edges, one lying to its
left and the other to its right.

An edge in the envelope diagram represent a continuous portion of the
$x$-axis, and is associated with a (possibly empty) set of curves that
induce the envelope over this portion of the $x$-axis. An edge may be bounded
by two vertices, one to its left and the other to its right. However, the
diagram contains two unbounded edges, its {\em leftmost} edge, representing
the interval $(-\infty, x_l)$, and its {\em rightmost} edge, representing the
interval $(x_r, \infty)$, where $x_l$ and $x_r$ are the $x$-coodinates of
the leftmost and the rightmost vertices in the diagram, respectively.
Note that a diagram may contain no vertices at all, in which case it
comprises a single edge.

Note that any model of the \ccRefName\ concept must define a geometric
traits class, which in turn defines the \ccc{Point_2} and 
\ccc{X_monotone_curve_2} types defined with the diagram features.

\ccTypes
%=======

\ccNestedType{Traits_2}{the geometric traits class.}

\ccTypedef{typedef Traits_2::Point_2 Point_2;}{the point type.}
\ccGlue
\ccTypedef{typedef Traits_2::X_monotone_curve_2 X_monotone_curve_2;}
{the $x$-monotone curve type.}

\ccNestedType{Size}{the size type.}

\ccNestedType{Curve_const_iterator}
{an iterator for the $x$-monotone curves that induce a diagram feature.
 Its value-type is \ccc{X_monotone_curve_2}.}

\ccNestedType{Vertex}{the vertex type, a model of the concept \ccc{EnvelopeDiagramVertex}.}
\ccGlue
\ccNestedType{Edge}{the edge type, a model of the concept \ccc{EnvelopeDiagramEdge}.}

\ccNestedType{Vertex_handle}
{a handle to a diagram vertex.
 The \ccc{Vertex_const_handle} type is also defined.}
\ccGlue
\ccNestedType{Edge_handle}
{a handle to a diagram edge.
 The \ccc{Edge_const_handle} type is also defined.}

\ccCreation
\ccCreationVariable{diag}
%========================

\ccConstructor{EnvelopeDiagram_1();} 
    {constructs an empty diagram containing one unbounded face,
     which corresponds to the entire plane and has no originators.}
    
\ccConstructor{Envelope_diagram_1 (const Self& other);}
    {copy constructor.}
        
\ccAccessFunctions
%=================

\ccMethod{Edge_const_handle leftmost() const;}
    {returns the leftmost edge of the diagram (a non-const version is also available).}
\ccGlue
\ccMethod{Edge_const_handle rightmost() const;}
    {returns the rightmost edge of the diagram (a non-const version is also available).}

\ccModifiers
%===========

\ccMethod{void set_leftmost (Edge_const_handle e);}
    {sets the leftmost edge of the diagram to be \ccc{e}.}
\ccGlue
\ccMethod{void set_rightmost (Edge_const_handle e);}
    {sets the rightmost edge of the diagram to be \ccc{e}.}

\ccMethod{Vertex_handle new_vertex (const Point_2& p);}
    {creates a new diagram vertex, associated with the point \ccc{p}.}
\ccGlue
\ccMethod{Edge_handle new_edge ();}
    {creates a new diagram edge.}
    
\ccMethod{void delete_vertex (Vertex_handle v);}
    {deletes the given vertex \ccc{v}.}
\ccGlue
\ccMethod{void delete_edge (Edge_handle e);}
    {deletes the given edge \ccc{e}.}

\ccSeeAlso
     \ccc{EnvelopeDiagramVertex}\lcTex{ 
     (\ccRefPage{EnvelopeDiagramVertex})}\\
     \ccc{EnvelopeDiagramEdge}\lcTex{
     (\ccRefPage{EnvelopeDiagramEdge})}\\

\end{ccRefConcept}

\ccRefPageEnd

