% +------------------------------------------------------------------------+
% | Reference manual page: EnvDiagVertex.tex
% +------------------------------------------------------------------------+
% | 
% | Package: Envelope_2
% | 
% +------------------------------------------------------------------------+

\ccRefPageBegin

\begin{ccRefConcept}{EnvelopeDiagramVertex}

\ccDefinition
%============

A vertex record in an envelope diagram. It is always associated with a point
on the lower (upper) envelope of a set of curves. A vertex is also
associated with a set of $x$-monotone curves that induce the envelope
over this point. It is incident to two edges, one lying to its
left and the other to its right.

\ccTypes
%=======

\ccNestedType{Edge}{the corresponding diagram-edge type.}

\ccNestedType{Point_2}{the point type associated with the vertex.}
\ccGlue
\ccNestedType{X_monotone_curve_2}{the $x$-monotone curve type.} 

\ccCreation
\ccCreationVariable{v}
%=====================

\ccConstructor{EnvelopeDiagramVertex();} 
    {default constructor.}
    
\ccConstructor{EnvelopeDiagramVertex (const EnvelopeDiagramVertex& other);}
    {copy constructor.}

\ccConstructor{EnvelopeDiagramVertex (const Point_2& p);} 
    {constructs a vertex associated with the given point \ccc{p}.}
        
\ccAccessFunctions
%=================

\ccMethod{const Point_2& point () const;}
    {returns the point associated with \ccVar.}

\ccMethod{Size number_of_curves () const;}
    {returns the number of $x$-monotone curves associated with \ccVar.}

\ccMethod{Curve_const_iterator curves_begin () const;}
    {return an iterator for the first $x$-monotone curve associated with \ccVar.}
\ccGlue
\ccMethod{Curve_const_iterator curves_end () const;}
    {return a past-the-end iterator for the $x$-monotone curves associated with \ccVar.}

\ccMethod{Edge_const_handle left() const;}
    {returns the edge lying to \ccVar's left.}
\ccGlue
\ccMethod{Edge_const_handle right() const;}
    {returns the edge lying to \ccVar's right.}

\ccModifiers
%===========

\ccMethod{void set_point (const Point_2& p);}
    {associates the point \ccc{p} with \ccVar.}

\ccMethod{void clear_curves();}
    {clears the set of curves associated with \ccVar.}
\ccGlue
\ccMethod{void add_curve (const X_monotone_curve_2& cv);}
    {adds the $x$-monotone curve \ccc{cv} to the set of curves associated with \ccVar.}
\ccGlue
\ccMethod{void add_curves (Curve_const_iterator begin, 
                           Curve_const_iterator end);}
    {adds the given range of $x$-monotone curves to the set of curves associated with \ccVar.}

\ccMethod{void set_left (Edge_const_handle e);}
    {sets the edge lying to the left of \ccVar\ to be \ccc{e}.}
\ccGlue
\ccMethod{void set_right (Edge_const_handle e);}
    {sets the edge lying to the right of \ccVar\ to be \ccc{e}.}

\ccSeeAlso
     \ccc{EnvelopeDiagram_1}\lcTex{ 
     (\ccRefPage{EnvelopeDiagram_1})}\\
     \ccc{EnvelopeDiagramEdge}\lcTex{
     (\ccRefPage{EnvelopeDiagramEdge})}\\

\end{ccRefConcept}

\ccRefPageEnd

