% +------------------------------------------------------------------------+
% | Reference manual page: upper_envelope_2.tex
% +------------------------------------------------------------------------+
% | 
% | Package: Envelope_2
% | 
% +------------------------------------------------------------------------+

\ccRefPageBegin

\begin{ccRefFunction}{upper_envelope_2}

\ccInclude{CGAL/envelope_2.h}

\ccFunction{template<class InputIterator, class EnvelopeDiagram>
            void upper_envelope_2 (InputIterator begin, InputIterator end,
                                   EnvelopeDiagram& diag);}
   {Computes the upper envelope of a set of curves in $\mathbb{R}^2$,
    as given by the range \ccc{[begin, end)}. The upper envelope is
    represented using the output maximization diagram \ccc{diag},
    which must be a model of the \ccc{EnvelopeDiagram_1} concept.
    \ccPrecond{The value-type of \ccc{InputIterator} is
               \ccc{EnvelopeDiagram::Traits_2::Curve_2}.}}

\end{ccRefFunction}

\ccRefPageEnd
