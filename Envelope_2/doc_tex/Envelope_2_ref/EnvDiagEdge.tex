% +------------------------------------------------------------------------+
% | Reference manual page: EnvDiagEdge.tex
% +------------------------------------------------------------------------+
% | 
% | Package: Envelope_2
% | 
% +------------------------------------------------------------------------+

\ccRefPageBegin

\begin{ccRefConcept}{EnvelopeDiagramEdge}

\ccDefinition
%============

An edge record in an envelope diagram, which represent a continuous portion
of the $x$-axis. It is associated with a (possibly empty) set of curves that
induce the envelope over this portion of the $x$-axis. Note that all curves
in this set overlap of the interval represented by the edge.

\ccTypes
%=======

\ccNestedType{Vertex}{the corresponding diagram-vertex type.}

\ccNestedType{X_monotone_curve_2}{the $x$-monotone curve type.} 

\ccCreation
\ccCreationVariable{e}
%=====================

\ccConstructor{EnvelopeDiagramEdge();} 
    {default constructor.}
    
\ccConstructor{EnvelopeDiagramEdge (const EnvelopeDiagramEdge& other);}
    {copy constructor.}

\ccAccessFunctions
%=================

\ccMethod{Size number_of_curves () const;}
    {returns the number of $x$-monotone curves associated with \ccVar.}
\ccGlue
\ccMethod{bool is_empty () const;}
    {returns whether \ccVar\ represents an empty interval --- namely, whether the set of $x$-monotone curves associated with it is empty.}

\ccMethod{const X_monotone_curve_2& curve () const;}
    {returns a representative $x$-monotone curve associated with \ccVar.
     \ccPrecond{\ccVar\ does not represent an empty interval.}}

\ccMethod{Curve_const_iterator curves_begin () const;}
    {return an iterator for the first $x$-monotone curve associated with \ccVar.}
\ccGlue
\ccMethod{Curve_const_iterator curves_end () const;}
    {return a past-the-end iterator for the $x$-monotone curves associated with \ccVar.}

\ccMethod{Vertex_const_handle left() const;}
    {returns the vertex lying to \ccVar's left.}
\ccGlue
\ccMethod{Vertex_const_handle right() const;}
    {returns the vertex lying to \ccVar's right.}

\ccModifiers
%===========

\ccMethod{void clear_curves();}
    {clears the set of curves associated with \ccVar.}
\ccGlue
\ccMethod{void add_curve (const X_monotone_curve_2& cv);}
    {adds the $x$-monotone curve \ccc{cv} to the set of curves associated with \ccVar.}
\ccGlue
\ccMethod{void add_curves (Curve_const_iterator begin, 
                           Curve_const_iterator end);}
    {adds the given range of $x$-monotone curves to the set of curves associated with \ccVar.}

\ccMethod{void set_left (Vertex_const_handle v);}
    {sets the vertex lying to the left of \ccVar\ to be \ccc{v}.}
\ccGlue
\ccMethod{void set_right (Vertex_const_handle v);}
    {sets the vertex lying to the right of \ccVar\ to be \ccc{v}.}

\ccSeeAlso
     \ccc{EnvelopeDiagram_1}\lcTex{ 
     (\ccRefPage{EnvelopeDiagram_1})}\\
     \ccc{EnvelopeDiagramVertex}\lcTex{
     (\ccRefPage{EnvelopeDiagramVertex})}\\

\end{ccRefConcept}

\ccRefPageEnd

