% =============================================================================
% The CGAL Developers' Manual
% Chapter: Introduction
% -----------------------------------------------------------------------------
% file   : philosophy.tex
% authors: Stefan Schirra <stschirr@mpi-sb.mpg.de>
% -----------------------------------------------------------------------------
% $Id$
% $Date$
% =============================================================================

\section{Primary design goals\label{sec:design_goals}}
\ccIndexSubitemBegin{design}{goals}
The primary design goals of \cgal\ are \cite{fgkss-dccga-00}:

\subsection*{Correctness}
\ccIndexMainItem{correctness}
A library component is correct if it behaves according to its
specification. 
Basically, correctness is therefore a matter of 
verification that documentation and implementation coincide.
In a modularized program the correctness of a module is determined 
by its own correctness and the correctness of all the modules it depends on.  
Clearly, in order to get correct results, correct algorithms and data 
structures must be used. 

\ccIndexSubitem{correctness}{vs. exactness}
\ccIndexMainItem{exactness}
Exactness should not be confused with correctness in the sense of
reliability. There is nothing wrong with algorithms computing approximate 
solutions instead of exact solutions, as long as their 
behavior is clearly documented and they do behave as specified.
Also, an algorithm handling only non-degenerate cases can be
correct with respect to its specification, although in \cgal\ we would
like to provide algorithms handling degeneracies.\ccIndexMainItem{degeneracies} 

\subsection*{Robustness}
\ccIndexMainItem{robustness}
A design goal particularly relevant for the implementation of
geometric algorithms is robustness.  Many implementations of geometric
algorithms lack robustness because of precision problems; see 
Chapter~\ref{chap:robustness} for a discussion of robustness issues within 
\cgal.

\subsection*{Flexibility}
\ccIndexMainItem{flexibility}
The different needs of the potential application areas demand 
flexibility in the library. Four sub-issues of flexibility can be identified.

{\bf Modularity.}
\ccIndexMainItem{modularity}
A clear structuring of \cgal\ into modules with as few dependencies as
possible helps a user in learning and using \cgal, since the overall
structure can be grasped more easily and the focus can be narrowed to
those modules that are actually of interest. 

{\bf Adaptability.}
\ccIndexMainItem{adaptability}
\cgal\ might be used in an already established environment with
geometric classes and algorithms in which case the modules will 
most probably need adaptation before they can be used. 

{\bf Extensibility.}
\ccIndexMainItem{extensibility}
Not all wishes can be fulfilled with \cgal. Users may want to
extend the library. It should be possible, and in fact desirable, to
integrate new classes and algorithms into \cgal.

{\bf Openness.}
\ccIndexMainItem{openness}
\cgal\ should be open to coexist with other libraries, or better, to
work together with other libraries and programs. The \CC\ 
Standard~\cite{cgal:ansi-is14882-98}
\index{C++ standard@\CC\ standard}
\ccIndexMainItem{\stl}
defines with the \CC\ Standard Library a common
foundation for all \CC\ platforms. 
So it is easy and natural to gain openness by following this standard.
There are important libraries outside the standard, and \cgal\
should be easily adaptable to them as well.

\subsection*{Ease of Use}
\ccIndexMainItem{ease of use}
Many different qualities can contribute to the ease of use of a
library. Which qualities are most important differs according to 
the experience of the user.
The above-mentioned correctness and robustness issues are among
these qualities. Of general importance is the length of time required
before the library becomes useful. Another issue is the number of 
new concepts and
exceptions to general rules that must be learned and remembered.

Ease of use tends to conflict with flexibility, but in many
situations a solution can be found.
The flexibility of \cgal\ should not distract a novice who takes the 
first steps with \cgal.
\ccIndexSubitem{ease of use}{vs. flexibility}
\ccIndexSubitem{flexibility}{vs. ease of use}

{\bf Uniformity.}
\ccIndexMainItem{uniformity}
A uniform look and feel of the design in \cgal\ will help in learning
and memorizing. A concept once learned should be applicable in all
places where one would wish to apply it. 
A function name once learned for a specific
class should not be named differently for another class. 

\index{C++ standard@\CC\ standard} \ccIndexMainItem{\stl} \cgal\ is
based in many places on concepts borrowed from \stl\ (Standard
Template Library) or the other parts of the \CC\ Standard Library. An
example is the use of streams and stream operators in \cgal. Another
example is the use of container classes and algorithms from the
\stl. So these concepts should be used uniformly.

\index{boost} \ccIndexMainItem{boost} During the past few years,
\cgal\ has moved towards using on concepts and ideas from the boost
libraries, as well as providing interfaces towards boost
libraries. These include the boost graph libary and the boost property
maps library.

{\bf Complete and Minimal Interfaces.}
\ccIndexMainItem{completeness}
\ccIndexSubitem{interfaces}{designing}
A goal with similar implications as uniformity is a design
with complete and minimal interfaces, see for example Item 18 
in Ref.~\cite{cgal:m-ec-97}.
An object or module should be complete in its 
functionality, but should
not provide additional decorating functionality. Even if a certain
function might look like it contributes to the ease of use for a certain 
class, in a more global picture it might hinder the understanding of 
similarities and differences among classes, and make it harder to learn 
and memorize.

{\bf Rich and Complete Functionality.}
\ccIndexMainItem{completeness}
\ccIndexMainItem{functionality}
We aim for a useful and rich collection of geometric classes, data
structures and algorithms. \cgal\ is supposed to be a foundation for
algorithmic research in computational geometry and therefore needs a
certain breadth and depth. The standard techniques in the field are
supposed to appear in \cgal. 

Completeness is also related to robustness.
\ccIndexMainItem{completeness}
\ccIndexMainItem{robustness}
We aim for general-purpose
solutions that are, for example, not restricted by assumptions on
general positions. Algorithms in \cgal\ should be able to handle
special cases and degeneracies. 
\ccIndexMainItem{general position}
\ccIndexMainItem{degeneracies}
In those cases where handling of degeneracies turns out to be
inefficient, special variants that are more efficient but less general
should be provided in the library in addition to the general 
algorithms handling all degeneracies. Of course, it needs to be
clearly documented which degeneracies are handled and which are not.

\subsection*{Efficiency}
\ccIndexMainItem{efficiency}
For most geometric algorithms theoretical results for the time and space
complexity are known. Also, the theoretic interest in efficiency for
realistic inputs, as opposed to worst-case situations, is
growing~\cite{v-ffrim-97}.
For practical purposes, insight into the constant factors hidden in the
$O$-notation is necessary, especially if there are several competing
algorithms.
\ccModifierCrossRefOff
\ccIndexMainItem{implementations, multiple}
\ccModifierCrossRefOn
Therefore, different implementations should be supplied if there is 
not one best solution, as, for example, when there is a tradeoff between 
time and space or a more efficient implementation when there are no or few 
degeneracies. 
\ccIndexMainItem{time-space tradeoff}
\ccIndexSubitemEnd{design}{goals}
