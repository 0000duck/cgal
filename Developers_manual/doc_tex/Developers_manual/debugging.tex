% =============================================================================
% The CGAL Developers' Manual
% Chapter: Debugging Tips
% -----------------------------------------------------------------------------
% file   : debugging.tex
% authors: Oren Nechushtan <theoren@math.tau.ac.il>
% -----------------------------------------------------------------------------
% $Id$
% $Date$
% =============================================================================

\chapter{Debugging Tips\label{chap:debugging}}
\ccChapterRelease{Chapter Version: 1.0}
\ccChapterAuthor{Oren Nechushtan ({\tt theoren@math.tau.ac.il })}

\ccIndexMainItemBegin{debugging}

Efficient debugging techniques can become an asset when writing geometric 
libraries such as \cgal. 
This chapter discusses debugging-related issues, like how to use 
the demo as a powerful debugger (Section~\ref{sec:graphical_debugging}), 
why and how to check your geometric predicates 
(Section~\ref{sec:debugging_cross_checker}), and what 
to do in order to evaluate handles and iterators during the debugging phase
(Section~\ref{sec:debugging_handles_and_it}).

\section{Graphical debugging\label{sec:graphical_debugging}}
\ccIndexSubitemBegin{debugging}{graphical}

\cgal\ packages usually provide a graphical demo that demonstrates the 
functionality in the package.  Many times this demo is simply a fancier
version of a program that was used in the early stages of development as a
(graphical) debugging tool.  In many cases, the output of a geometric
algorithm is much easier to interpret in graphical form than numeric
form.  Thus you should use the powerful graphical output capabilities
of \cgal\ (see the \ccAnchor{http://www.cgal.org/Manual/doc_html/frameset/fsSupport.html}{Support Library documentation}) 
to develop 
\begin{itemize}
   \item programs that can be used for debugging the internal workings
         of your package (\ie, things a user may not have access to)
   \item interesting and informative demos that highlight the features and, 
         at the same time, the absence or presence of bugs in your package.  
         Other demo/debugging programs can be found in the \texttt{demo} 
         directory \index{demo directory@{\tt demo} directory}
         of every internal release and \cgal\ installation.
\end{itemize}
\ccIndexSubitemEnd{debugging}{graphical}

\section{Cross-checkers\label{sec:debugging_cross_checker}}
\ccIndexSubitemBegin{debugging}{with cross-checkers}
\ccIndexMainItemBegin{cross-checker}

A cross-checker is a powerful means to allow for efficient
maintenance of your code.  A cross-checker for a given concept is a
model of that concept \ccIndexMainItemDef{cross-checker} that is
constructed from another model or models (one of which is the one you
wish to check).  In order to implement the functionality required by
the concept, the cross-checker will use functions from the models upon
which it is built and perform tests for validity, etc.\ on them.  If
the tests succeed, the cross-checker returns the expected result.
Otherwise, the cross-checker can generate an assertion violation or a
warning, depending on the severity of the offense.

For example, if you have a version of an algorithm, traits class, or
kernel that you know works, you can easily use this as an oracle for
another version of the algorithm, traits class, or kernel that you
wish to test.  This is easily done because the code in \cgal\ is
highly templatized.  The cross-checker would simply plug in the two
different versions of, say, your traits class, as the relevant
template parameters for two different instantiations of a class, say,
and then compare the results from using the two different
instantiations.

\subsection*{An example: Traits class binary cross-checker}

As a more concrete example, assume that you have a traits class concept 
that requires a nested type \ccc{X_curve} and a function

\ccGlobalFunction{bool curve_is_vertical(const X_curve & cv) const;}
%\ccc{bool curve_is_vertical(const X_curve & cv) const;}

A binary cross-checker for this concept might look like 
%\begin{ccExampleCode} 
\begin{verbatim}
template <class Traits1,class Traits2,class Adapter>
class Binary_traits_checker{

Traits1 tr1;
Traits1 tr2;
Adapter P;

public:

typedef typename Traits1::X_curve X_curve;

Traits_binary_checker(Traits1 tr1_,Traits2 tr2_,Adapter P_) :
tr1(tr1_),tr2(tr2_),P(P_){};

bool curve_is_vertical(const X_curve & cv) const;

}
\end{verbatim}

and possibly be implemented as 

\begin{verbatim}
bool curve_is_vertical(const X_curve & cv) const
{
    CGAL_assertion(tr1.curve_is_vertical(cv)==tr2.curve_is_vertical(P(cv)));
    return tr1.curve_is_vertical(cv);
}
\end{verbatim}

Notice that the class \ccc{Binary_traits_checker} has template parameters 
named \ccc{Traits1} and \ccc{Traits2}, and a third parameter named 
\ccc{Adapter}. One of the traits classes is the one to be tested and the
other is (presumably) a traits class that always gives the right answer.
The \ccc{Adapter} is needed since the \ccc{X_curve} types 
for \ccc{Traits1} and \ccc{Traits2} might be different. 
This cross-checker does nothing other then asserting that the two traits 
classes return the same values by calling the 
the counterparts in the member traits classes 
(\ccc{tr1},\ccc{tr2}) and comparing the results.
\ccIndexMainItemEnd{cross-checker}
\ccIndexSubitemEnd{debugging}{with cross-checkers}


\section{Examining the values of variables\label{sec:debugging_handles_and_it}}
\ccIndexSubitemBegin{debugging}{interactive}
\ccIndexSubitemBegin{handles}{and debugging}

When using an interactive debugger, one often wishes to see the value of
a variable, such as the $y$-value of a segment's source point. Thus one
would naturally issue a command such as
\begin{verbatim}
    print segment.source().y()
\end{verbatim}
This most often produces disappointingly unrevealing results, \eg, an
error message saying the value cannot be evaluated because functions may
be inlined.  

We recommend the following approaches to work around (or avoid) this 
and similar problems:
\begin{itemize}
   \item Use the \ccc{Simple_cartesian} kernel 
         \ccIndexSubitem{kernel}{\ccFont Simple_cartesian}
         (Chapter~\ref{chap:kernels}), which does not do reference
         counting and uses no handles so data member values can be inspected
         directly.\index{debugging!with \ccc{Simple_cartesian}}
         \index{Simple_cartesian kernel@\ccc{Simple_cartesian} kernel}
   \item Print the values by following the pointers in the handles used to 
         represent objects.  For example, for the segment above, the statement
         \begin{verbatim}
            print s.ptr->start->ptr->e1
         \end{verbatim}
         is likely to work.  This technique can also work for non-kernel 
         handles, such as \ccc{Halfedge_handle} and \ccc{Vertex_handle}.
         One must know, of course, the right names for the data members,
         but this you can find out by printing the things that pointers
         point to.   For example,
         \begin{verbatim}
            print *s.ptr
         \end{verbatim}
         In the case of the planar map package, these handles are actually 
         polyhedron iterators. 
         If $h$ is a halfedge of a planar map and you want to know the curve 
         associated with it, then if
         \begin{verbatim}
          print h->curve()
         \end{verbatim}
         fails, try using 
         \begin{verbatim}
          print h.nt.node->cv
         \end{verbatim}
         instead.

         For a vertex $v$ of a planar map, if 
         \begin{verbatim}
          print v->point()
         \end{verbatim}
         fails, use  
         \begin{verbatim}
           print v.nt.node->p
         \end{verbatim}
         instead.
\end{itemize}

Note: You can also use watches to continuously examine such values during 
execution.
\ccIndexSubitemEnd{debugging}{interactive}
\ccIndexSubitemEnd{handles}{and debugging}
\ccIndexMainItemEnd{debugging}

\InternalOnly{

\section{Requirements and recommendations\label{sec:debugging_req_and_rec}}

\noindent
Requirements:
\begin{itemize}
   \item Test and debug your code \textbf{before} submitting it.
\end{itemize}

\noindent
Recommendations:
\begin{itemize}
   \item Don't write buggy code \texttt{:-)}.
\end{itemize}

}
