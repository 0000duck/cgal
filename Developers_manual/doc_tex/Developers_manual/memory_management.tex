% =============================================================================
% The CGAL Developers' Manual
% Chapter: Memory management
% -----------------------------------------------------------------------------
% file   : memory_management.tex
% authors: Michael Seel <seel@mpi-sb.mpg.de>
% -----------------------------------------------------------------------------
% $Id$
% $Date$
% =============================================================================

\chapter{Memory Management\label{chap:memory_management}}
\ccChapterRelease{Chapter Version: 1.0}
\ccChapterAuthor{Michael Seel ({\tt seel@mpi-sb.mpg.de})}

One of the design goals of \cgal\ (Section~\ref{sec:design_goals}) is 
efficiency,\ccIndexMainItem{efficiency}
and this means not only implementing efficient algorithms
but also implementing them efficiently.  One way to improve the efficiency
of an implementation is through efficient memory management.  Here we
describe one way to address this using the allocator interface.

\section{The \CC\ standard allocator interface\label{sec:allocator_interface}}
\ccIndexMainItemBegin{allocator}

We first give a short presentation of the memory allocator interface.
Objects of type \ccc{allocator<T>} can be used to obtain small, typed 
chunks of memory to be used, for example, as static members of a class. 
This is especially
interesting with classes of a constant size that are frequently
allocated and deallocated (geometric objects, etc.), since a memory
allocator can maintain the corresponding memory chunks in local blocks
and thus can answer allocation and deallocation calls much faster than
the corresponding system calls. We first recapitulate the interface of
an allocator:

% =============================================================================
% The CGAL Developers' Manual
% Chapter: Memory Management
% -----------------------------------------------------------------------------
% file   : Allocator.tex
% authors: Michael Seel <seel@mpi-sb.mpg.de>
% -----------------------------------------------------------------------------
% $Revision$
% $Date$
% =============================================================================

\ccHtmlNoClassLinks
\ccAutoIndexingOff
\begin{ccClassTemplate}{allocator<T>}
\ccCreationVariable{A}

\ccHeading{Class \ccc{CGAL::allocator<T>}}

\ccDefinition

An instance \ccc{A} of the data type \ccc{allocator<T>} is
a memory allocator according to the \CC standard. 

\ccSetOneOfTwoColumns{3.5cm}

\ccTypes

Local types are \ccc{size_type}, \ccc{difference_type}, \ccc{value_type},
\ccc{pointer}, \ccc{reference}, \ccc{const_pointer}, and \ccc{const_reference}. 


\ccNestedType{template <class T1> rebind}{allows the construction of a derived allocator:\\
     \ccc{allocator<T>::template rebind<T1>::other}\\ is the type
     \ccc{allocator<T1>}.  
}

\ccSetOneOfTwoColumns{3cm}

\ccCreation

\ccConstructor{allocator<T>()}{introduces a variable \ccc{A} of type \ccc{allocator<T>}.  
}

\ccSetTwoOfThreeColumns{2.5cm}{1cm}

\ccOperations

\ccMethod{pointer allocate(size_type n, const_pointer = 0) ;}{returns a pointer to a newly allocated memory range of size
        \ccc{n * sizeof(T)}. 
}

\ccMethod{void deallocate(pointer p, size_type n) ;}{deallocates a memory range of \ccc{n * sizeof(T)} starting
        at \ccc{p}. \ccPrecond the memory range was obtained via \ccc{allocate(n)}. 
}

\ccMethod{pointer address(reference r) ;}{returns \ccc{&r}. 
}

\ccMethod{const_pointer address(const_reference r) ;}{returns \ccc{&r}. 
}

\ccMethod{void construct(pointer p, const_reference r) ;}{copies the object referenced by \ccc{r} to \ccc{*p}.
(Technically this is achieved by an inplace new \ccc{new( (void*)p ) 
T(r)}). 
}

\ccMethod{void destroy(pointer p) ;}{destroys the object referenced via \ccc{p} by calling \lcTex{\ccc{p->\ccTilde T()}}\lcHtml{\ccc{p->~T()}}. 
}

\ccMethod{size_type max_size() ;}{the largest value \ccc{n} for which the call \ccc{allocate(n,0)} 
    might succeed. 
}

\end{ccClassTemplate}
\ccAutoIndexingOn

\ccIndexMainItemEnd{allocator}

\section{The allocator macro\label{sec:allocator_macro}}
\ccIndexSubitemBegin{allocator}{macro}

The macro \ccc{CGAL_ALLOCATOR}\ccIndexMainItem{\ccFont CGAL_ALLOCATOR}
is defined in the file \ccc{<CGAL/memory.h>} to be the standard allocator
from \ccc{<memory>}.
However, the user can redefine it, for example, if \leda\ is present,
he can define it (before including any CGAL header file) this way :

\begin{verbatim}
#include <LEDA/allocator.h>
#define CGAL_ALLOCATOR(t) leda_allocator<t>
\end{verbatim}
\ccIndexSubitemEnd{allocator}{macro}

\section{Using the allocator\label{sec:using_memory_allocator}}
\ccIndexSubitemBegin{allocator}{as template parameter}

How should a data structure use the allocator mechanism? Just make the
allocator one of the template arguments of the data structure. Then
use a static member object to allocate items on the heap that you
want to keep optimized regarding allocation and deallocation. We
show an example using a trivial list structure:

\begin{verbatim}
#include <CGAL/memory.h>

template <typename T> 
class dlink 
{ T some_member; };

template < typename T, typename Alloc = CGAL_ALLOCATOR(dlink<T>) >
class list 
{
public:
  typedef dlink<T>* dlink_ptr;
  typedef Alloc list_allocator;

  static list_allocator M;

list() {
  p = M.allocate(1);          // allocation of space for one dlink
  M.construct(p,dlink<T>());  // inplace construction of object
}

~list() {
  M.destroy(p);      // destroy object
  M.deallocate(p,1); // deallocate memory
}

private:
  dlink_ptr p;
};

// init static member allocator object:
template <typename T, typename Alloc>
typename list<T,Alloc>::list_allocator list<T,Alloc>::M =
             typename list<T,Alloc>::list_allocator();


int main()
{
  list<int> L;
  return 0;
}
\end{verbatim}
\ccIndexSubitemEnd{allocator}{as template parameter}

\InternalOnly{

\section{Requirements and recommendations\label{sec:memory_req_and_rec}}

%\noindent
%Requirements:
%\begin{itemize}
%\end{itemize}

\noindent
Recommendations:
\begin{itemize}
   \item Use an allocator template parameter (which defaults to
         \ccc{CGAL_ALLOCATOR}) for data structures for which an optimization
         with regard to allocation and deallocation is beneficial.
\end{itemize}

}
