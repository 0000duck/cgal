% =============================================================================
% The CGAL Developers' Manual
% Chapter: Memory management
% -----------------------------------------------------------------------------
% file   : memory_management.tex
% authors: Michael Seel <seel@mpi-sb.mpg.de>
% -----------------------------------------------------------------------------
% $Id$
% $Date$
% =============================================================================

\chapter{Memory Management\label{chap:memory_management}}
\ccChapterRelease{Chapter Version: 1.0}
\ccChapterAuthor{Michael Seel ({\tt seel@mpi-sb.mpg.de})\\
                 Efi Fogel ({\tt efif@post.tau.ac.il})}

One of the design goals of \cgal{} (Section~\ref{sec:design_goals}) is 
efficiency,\ccIndexMainItem{efficiency}
and this means not only implementing efficient algorithms
but also implementing them efficiently. One way to improve the efficiency
of an implementation is through efficient memory management. This can be
done by making the library data structures independent of the underlying
memory model. However, to avoid unacceptable efficiency degradations
complete abstraction of the memory model should be avoided. Here we
describe one way to address this using allocators. An allocator 
encapsulates the information about an allocation model.

\ccIndexMainItemBegin{allocator}

We adopted the definition of the Standard \CC{}
allocator~\cite{cgal:ansi-is14882-98}. The \ccc{std::allocator} is the
only predefined and required allocator imposed by [\CC] on all \CC{}
compiler implementations. The exact specification can also be found at
\path|http://en.wikipedia.org/wiki/Allocator_(C++)|.

Objects of type \ccc{std::allocator<T>} can be used to obtain small, typed 
chunks of memory to be used, for example, as static members of a class. 
This is especially interesting with classes of a constant size that are
frequently allocated and deallocated (e.g., points, lines, circles),
since a memory allocator can maintain the corresponding memory chunks
in local blocks and thus can answer allocation and deallocation calls
much faster than the corresponding system calls.

%% A class \ccc{A} capable of allocating memory for an
%% object of type \ccc{T} must provide the types \ccc{A::pointer},
%% \ccc{A::const_pointer}, \ccc{A::reference}, \ccc{A::const_reference},
%% and \ccc{A::value_type} for generically declaring objects and
%% references (or pointers) to objects of type \ccc{T}. It should also
%% provide type \ccc{A::size_type}, an unsigned type which can represent
%% the largest size for an object in the allocation model defined by
%% \ccc{A}, and similarly, a signed integral \ccc{A::difference_type}
%% that can represent the difference between any two pointers in the
%% allocation model.

%% Although you can assume that the allocator's \ccc{A::pointer} and
%% \ccc{A::const_pointer} are simply typedefs for \ccc{T*} and
%% \ccc{T const*}, if you intend to implement a model, you are encouraged
%% to support more general allocators.

%% An allocator, \ccc{A}, for objects of type \ccc{T} must have a
%% member function with the signature
%% \ccc{A::pointer A::allocate(size_type n, A<void>::const_pointer hint = 0)}.
%% This function returns a pointer to the first element of a newly allocated
%% array large enough to contain $n$ objects of type \ccc{T}; only the memory
%% is allocated, and the objects are not constructed. Moreover, an optional
%% pointer argument (that points to an object already allocated by \ccc{A})
%% can be used as a hint to the implementation about where the new memory
%% should be allocated in order to improve locality. However, the
%% implementation is free to ignore the argument.

%% The corresponding void
%% \ccc{A::deallocate(A::pointer p, A::size_type n)} member function
%% accepts any pointer that was returned from a previous invocation of the
%% \ccc{A::allocate} member function and the number of elements to
%% deallocate (but not destruct).

%% The \ccc{A::max_size()} member function returns the largest number of
%% objects of type \ccc{T} that could be expected to be successfully
%% allocated by an invocation of \ccc{A::allocate}; the value returned is
%% typically \ccc{A::size_type(-1) / sizeof(T)}. Also, the \ccc{A::address}
%% member function returns an \ccc{A::pointer} denoting the address of an
%% object, given an \ccc{A::reference} to it.

%% Object construction and destruction is performed separately from allocation
%% and deallocation. The allocator is required to have two member functions,
%% \ccc{A::construct} and \ccc{A::destroy}, which handles object construction
%% and destruction, respectively. The semantics of the functions should be
%% equivalent to the following:

%% \begin{verbatim}
%% template <typename T>
%% void A::construct(A::pointer p, A::const_reference t) { new ((void*) p) T(t); }
 
%% template <typename T>
%% void A::destroy(A::pointer p){ ((T*)p)->~T(); }
%% \end{verbatim}

%% The above code uses the placement new syntax, and calls the destructor
%% directly.

%% Allocators should be copy-constructible. An allocator for objects of type
%% \ccc{T} can be constructed from an allocator for objects of type \ccc{U}.
%% If an allocator, \ccc{A}, allocates a region of memory, $R$, then $R$ can
%% only be deallocated by an allocator that compares equal to \ccc{A}.

%% Allocators are required to supply a template class member template
%% \ccc{<typename U> struct A::rebind { typedef A<U> other; };},
%% which enables the possibility of obtaining a related allocator,
%% parameterized in terms of a different type. For example, given an allocator
%% type \ccc{IntAllocator} for objects of type \ccc{int}, a related allocator
%% type for objects of type long could be obtained using
%% \ccc{IntAllocator::rebind<long>::other}.
%% =============================================================================
% The CGAL Developers' Manual
% Chapter: Memory Management
% -----------------------------------------------------------------------------
% file   : Allocator.tex
% authors: Michael Seel <seel@mpi-sb.mpg.de>
% -----------------------------------------------------------------------------
% $Revision$
% $Date$
% =============================================================================

\ccHtmlNoClassLinks
\ccAutoIndexingOff
\begin{ccClassTemplate}{allocator<T>}
\ccCreationVariable{A}

\ccHeading{Class \ccc{CGAL::allocator<T>}}

\ccDefinition

An instance \ccc{A} of the data type \ccc{allocator<T>} is
a memory allocator according to the \CC standard. 

\ccSetOneOfTwoColumns{3.5cm}

\ccTypes

Local types are \ccc{size_type}, \ccc{difference_type}, \ccc{value_type},
\ccc{pointer}, \ccc{reference}, \ccc{const_pointer}, and \ccc{const_reference}. 


\ccNestedType{template <class T1> rebind}{allows the construction of a derived allocator:\\
     \ccc{allocator<T>::template rebind<T1>::other}\\ is the type
     \ccc{allocator<T1>}.  
}

\ccSetOneOfTwoColumns{3cm}

\ccCreation

\ccConstructor{allocator<T>()}{introduces a variable \ccc{A} of type \ccc{allocator<T>}.  
}

\ccSetTwoOfThreeColumns{2.5cm}{1cm}

\ccOperations

\ccMethod{pointer allocate(size_type n, const_pointer = 0) ;}{returns a pointer to a newly allocated memory range of size
        \ccc{n * sizeof(T)}. 
}

\ccMethod{void deallocate(pointer p, size_type n) ;}{deallocates a memory range of \ccc{n * sizeof(T)} starting
        at \ccc{p}. \ccPrecond the memory range was obtained via \ccc{allocate(n)}. 
}

\ccMethod{pointer address(reference r) ;}{returns \ccc{&r}. 
}

\ccMethod{const_pointer address(const_reference r) ;}{returns \ccc{&r}. 
}

\ccMethod{void construct(pointer p, const_reference r) ;}{copies the object referenced by \ccc{r} to \ccc{*p}.
(Technically this is achieved by an inplace new \ccc{new( (void*)p ) 
T(r)}). 
}

\ccMethod{void destroy(pointer p) ;}{destroys the object referenced via \ccc{p} by calling \lcTex{\ccc{p->\ccTilde T()}}\lcHtml{\ccc{p->~T()}}. 
}

\ccMethod{size_type max_size() ;}{the largest value \ccc{n} for which the call \ccc{allocate(n,0)} 
    might succeed. 
}

\end{ccClassTemplate}
\ccAutoIndexingOn

\ccIndexMainItemEnd{allocator}

\section{The allocator macro\label{sec:allocator_macro}}
\ccIndexSubitemBegin{allocator}{macro}

The macro \ccc{CGAL_ALLOCATOR(T)}\ccIndexMainItem{\ccFont CGAL_ALLOCATOR}
is defined as \ccc{std::allocator<T>} in the file \ccc{<CGAL/memory.h>}.
\ccc{CGAL_ALLOCATOR}\ccIndexMainItem{\ccFont CGAL_ALLOCATOR} is used
as the default allocator for all \cgal{} components. You can redefine it,
for example, if \leda{} is present, you can define it (before including
any \cgal{} header file) as follows:

\begin{verbatim}
#include <LEDA/allocator.h>
#define CGAL_ALLOCATOR(t) leda_allocator<t>
\end{verbatim}
\ccIndexSubitemEnd{allocator}{macro}

\section{Using the allocator\label{sec:using_memory_allocator}}
\ccIndexSubitemBegin{allocator}{as template parameter}

How should a data structure use the allocator mechanism? Just make the
allocator one of the template arguments of the data structure. Then
use a static member object to allocate items on the heap that you
want to keep optimized regarding allocation and deallocation. We
show an example using a trivial list structure:

\begin{verbatim}
#include <CGAL/memory.h>

template <typename T> 
class dlink 
{ T some_member; };

template < typename T, typename Alloc = CGAL_ALLOCATOR(dlink<T>) >
class list 
{
public:
  typedef dlink<T>* dlink_ptr;
  typedef Alloc list_allocator;

  static list_allocator M;

list() {
  p = M.allocate(1);          // allocation of space for one dlink
  M.construct(p,dlink<T>());  // inplace construction of object
}

~list() {
  M.destroy(p);      // destroy object
  M.deallocate(p,1); // deallocate memory
}

private:
  dlink_ptr p;
};

// init static member allocator object:
template <typename T, typename Alloc>
typename list<T,Alloc>::list_allocator list<T,Alloc>::M =
             typename list<T,Alloc>::list_allocator();


int main()
{
  list<int> L;
  return 0;
}
\end{verbatim}
\ccIndexSubitemEnd{allocator}{as template parameter}

\section{Requirements and recommendations\label{sec:memory_req_and_rec}}

\noindent
Recommendations:
\begin{itemize}
   \item Use an allocator template parameter (which defaults to
         \ccc{CGAL_ALLOCATOR}) for data structures for which an optimization
         with regard to allocation and deallocation is beneficial.
\end{itemize}
