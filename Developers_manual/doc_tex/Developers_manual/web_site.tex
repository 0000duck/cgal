\chapter{The \cgal\ Web Site\label{chap:web_site}}
\ccIndexMainItemBegin{web site}

The web site for \cgal\ is \path|http://www.cgal.org|.  It is
currently hosted by the Max-Planck-Institut f\"ur Informatik (MPI).

\section{SVN cgal-web Project and Maintenance}

Most files that are visible on the website are contained in a 
seperate SVN project called 
\index{SVN server!cgal-web project@\texttt{cgal-web} project}

\begin{quote}
    \textbf{cgal-web}
\end{quote}

Exceptions are large binaries and,
for example, the Manual, which is generated (and large) and is thus
not suitable to be placed in SVN. One can see the distinction in the
URL's. Everything starting with

\begin{quote}
    \textbf{pub/}\\
    \textbf{Members/pub/}\\
    \textbf{Manual/}
\end{quote}

is not in SVN. If you want something changed or added here, you have to
contact the \texttt{cgal-develop} mailing-list.
(For the manual an exception for the
above exception are some HTML files, for example the frameset,
that are actually available in the SVN package \texttt{Manual}.)

Several developers have write access to the new \texttt{cgal-web}
project, or might get it whenever need arises. Note that the SVN project

\begin{quote}
  \textit{\textbf{is not automatically synchronized}}
\end{quote}

with the actual web pages. It is a manual \texttt{svn up} call that
has to be initiated at MPI. This is also intended as a final quality
control before updates show up on our web pages. It allows us to have
the set of people with write access to the SVN project not to be too
restrictive.

How do you get your changes finally on the web pages of \cgal?

The SVN server actually informs the maintainer at MPI (see above)
automatically by email about committed changes in the
\texttt{cgal-web} project. So, either just commit small changes, the
maintainer verifies them and integrates them in the next update of the
web pages. If you plan something bigger, please let the maintainer
know, so that updates can be suspended until you are done.  Or, if you
plan to have the updates done with a particular deadline, inform the
maintainer well ahead, e.g., a week in advance, since:

\begin{itemize}
    \item
        The maintainer does not necessarily read email all the time
        (maybe not even daily ;-).
    \item
        The maintainer might not consider your request as important as
        other tasks, especially since
    \item
        almost no request for CGAL's web site has the importance that needs
        24/7 service, and most things can also wait a week.
\end{itemize}

If you plan updates that are not obviously consensus, you should discuss them
beforehand on the usual mailing lists. The maintainer might block an
update if its status is not clear.


\section{Regular Tasks}

Some aspects of the web pages need constant attention to keep them
up-to-date. If you can please do the necessary updates yourself,
notify the corresponding maintainer, or the maintainer mentioned above
if you know about updates for the following:

\begin{itemize}
   \item new\ccIndexMainItem{publications}
         \ccAnchor{http://www.cgal.org/bibliography.html}{publications}
         related to \cgal, maintained by
         \ccAnchor{mailto:Monique.Teillaud@inria.fr}{Monique Teillaud};
   \item updates for the\ccIndexMainItem{work in progress}
        \ccAnchor{http://www.cgal.org/in_progress.html}{work in progress}
         page, which includes work that is finished and thus no longer in
         progress and new work.
   \item new\ccIndexMainItem{demo programs}
         \ccAnchor{http://www.cgal.org/demos.html}{demo programs}
         to be included on the web site. (See Section~\ref{sec:demos_web_page}
         for the precise specification of how to submit these.)
   \item new\ccIndexMainItem{people}\ccIndexMainItem{developers}
         \ccAnchor{http://www.cgal.org/people.html}{people}
         involved in the project
   \item new\ccIndexMainItem{projects}\ccIndexMainItem{friends}
         \ccAnchor{http://www.cgal.org/projects.html}{projects}
         using \cgal\ by people other than developers.
\end{itemize}

And, of course, please tell the maintainer about any problems you have
with the web site.

\ccIndexMainItemEnd{web site}
