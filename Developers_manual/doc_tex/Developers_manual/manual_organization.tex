\section{Manual organization\label{sec:manual_org}}

This manual is meant to be a resource for developers who wish to contribute
to the \cgal\ library either by designing new packages or maintaining
or enhancing existing ones. The manual is organized roughly in the order in
which a developer will need the information in order to produce a package
for the library. We begin in this chapter with a description
of the design goals of \cgal\ and the overall design, which should be kept
in mind during all stages of development.  The remaining chapters describe
in more concrete terms the requirements and recommendations for documentation,
code writing, and testing that are derived from these goals.  We also describe
a number of tools that have been created to help in the development process
and give pointers to other sources of information.

A description of how the specification
for a package should be documented is provided along with information about
the tools available to help produce the documentation
in Chapter~\ref{chap:specification}.
Chapter~\ref{chap:svn} discusses the SVN server on which all \cgal\ source
code is kept.
Chapter~\ref{chap:directory_structure} describes the directory structure
required for a package and Chapter~\ref{chap:tools} describes a set of tools
that may be used to create or modify various files required within this
directory structure. Chapters~\ref{chap:code_format}
through~\ref{chap:robustness}
discuss issues related to the writing of code that is in keeping with
the goals of \cgal.  Chapter~\ref{chap:portability} describes issues
related to the configuration of \cgal\ and
discusses portability issues for various platforms. Chapter~\ref{chap:testing}
describes the requirements and behaviour of the test suite for a package and
Chapter~\ref{chap:debugging} consists of various hints for debugging.
Guidelines for the development of example and demo programs are given in
Chapter~\ref{chap:examples_and_demos}. Information
about how to submit a package's specification to the editorial board
for approval and the implementation (and documentation) for inclusion in
internal releases is presented in  Chapter~\ref{chap:submission}.
Chapter~\ref{chap:releases} gives information about the creation of the
internal, public, and bug-fix releases.
Chapters~\ref{chap:addresses} and Chapter~\ref{chap:info}
provide information about mailing lists and other information
resources developers might find useful.
