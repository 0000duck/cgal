%% =============================================================================
%% The CGAL Developers' Manual
%% Chapter: Portability Issues
%% -----------------------------------------------------------------------------
%% file   : portability.tex
%% authors: Michael Hoffmann <hoffmann@inf.ethz.ch> &
%%          Stefan Schirra <stschirr@mpi-sb.mpg.de>
%% -----------------------------------------------------------------------------
%% $Id$
%% $Date$
%% =============================================================================

\chapter{Portability Issues\label{chap:portability}}
\ccChapterRelease{Chapter Version: 1.0}
\ccChapterAuthor{Michael Hoffmann ({\tt hoffmann@inf.ethz.ch})\\
                 Stefan Schirra ({\tt stschirr@mpi-sb.mpg.de})\\
                 Sylvain Pion ({\tt Sylvain.Pion@sophia.inria.fr})}
\ccIndexMainItemBegin{portability}

This chapter gives an overview of issues related to the
configuration of \cgal\ that allow you to answer such questions as:
\begin{itemize}
\item Is \leda/\textsc{Gmp} there? (Section~\ref{sec:leda_gmp_support})
\item What version of \cgal\ am I running? (Section~\ref{sec:which_versions})
\item Which compiler is this? (Section~\ref{sec:which_compiler})
\item Does the compiler support Koenig lookup? (Section~\ref{sec:workaround_flags})
\end{itemize}

Also addressed here are issues related to writing code for 
non-standard-compliant compilers.  Compilers have made a lot of progress toward 
the \CC-standard recently.  But still they do not fully implement it. There 
are a few features you may assume; others you may not
assume. Especially you may assume that the compiler
\begin{itemize}
\item supports namespaces
\item supports member templates
\item support for \texttt{std::iterator\_traits}.
\end{itemize}
Still, there are many bugs (sometimes known as ``features'') left in the 
compilers.  Have a look at the list of (non-obsolete) workarounds in 
Section~\ref{sec:workaround_flags} to get an idea of which ``features'' are
still present.  

\ccIndexMainItemBegin{configuration}
\section{Checking for \leda\ or GMP support\label{sec:leda_gmp_support}}
\ccIndexSubsubitem{\leda}{support}{checking for}
%\index{leda support@\leda\ support!checking for}
\index{gmp support@GMP support!checking for}

In the makefiles included for the compilation of every \cgal\ program
(\ie, those to which the environment variable {\tt CGAL\_MAKEFILE} refers),
\ccIndexSubitem{\tt makefile}{\cgal}
\index{CGAL_MAKEFILE variable@{\tt CGAL\_MAKEFILE} variable}
we define command line switches that set the flags 
\begin{verbatim}
CGAL_USE_LEDA, CGAL_USE_GMP
\end{verbatim}
\index{CGAL_USE_LEDA flag@{\tt CGAL\_USE\_LEDA} flag}
\index{flag!for \leda}
\index{CGAL_USE_GMP flag@{\tt CGAL\_USE\_GMP} flag}
\index{flag!for GMP}
iff \cgal\ is configured with \leda\ or GMP support, respectively.

\section{Using Boost\label{sec:boost_support}}
\cgal\ code can rely on Boost libraries to some extent.

Boost was installed with \cgal\ Release 3.1, and is no longer
installed, as it is already distributed with Linux and cygwin.  


Since portability and backward compatibility are a concern in \cgal,
we have decided that the list of Boost libraries usable in \cgal\ will be
decided by the \cgal\ editorial board.  The requirements are higher
when it appears in the user visible interface than when Boost code
is used only internally.  Requirements are even lower for code that
is not released such as the test-suite.  Boost libraries already accepted
in the C++ Standard Library Technical Report will be the first easy
candidates (these are marked \texttt{[TR1]} in the list below).

Finally, the policy is that if a better alternative exists in Boost and is
allowed, then \cgal\ code must use it instead of a \cgal\ version (which
probably must be deprecated and phased out), trying not to break backward
compatibility too much.

Here follows a list of Boost libraries allowed for use in \cgal\ (those
with question marks are not decided yet) :
\begin{itemize}
\item Operators --- Templates ease arithmetic classes and iterators
\item ? Any --- Safe, generic container for single values of different value
      types (what's the relation with \texttt{variant})
\item Concept check --- Tools for generic programming
\item ? Bind \texttt{[TR1]} and  mem\_fn --- Generalized binders for
      function/object/pointers and member functions (overlaps with STL
      Extensions)
\item ? Graph --- Generic graph components and algorithms
\item ? Interval --- Extends the usual arithmetic functions to mathematical intervals (overlaps with CGAL::Interval\_nt)
\item Iterators --- Iterator construction framework, adaptors, concepts, and more
\item MPL --- Template metaprogramming framework of compile-time algorithms, sequences and metafunction classes
\item Optional --- Discriminated-union wrapper for optional values
\item Property map --- Concepts defining interfaces which map key objects to value objects
\item ? Random \texttt{[TR1]} --- A complete system for random number generation
      (overlaps with support library)
\item ? Rational --- A rational number class (overlaps with CGAL::Quotient)
\item ? Ref \texttt{[TR1]} --- A utility library for passing references to generic functions
%\item ? Result\_of \texttt{[TR1]}
\item ? Smart Pointers \texttt{[TR1]} --- Five smart pointer class templates
\item ? Static assertions --- Static assertions (compile time assertions)
\item Tuple \texttt{[TR1]} --- Ease definition of functions returning multiple values, and more
\item ? Type traits \texttt{[TR1]} --- Templates for fundamental properties of types
\item ? Variant --- Safe, generic, stack-based discriminated union container
      (how does it relate to \texttt{any} ?)
\end{itemize}



\section{Using the version-number and configuration macros and flags\label{sec:using_version_macros}}
\index{LEDA macro@{\tt \_\_LEDA\_\_} macro!using}
\index{CGAL_USE_LEDA flag@{\tt CGAL\_USE\_LEDA} flag!using}
\index{CGAL_USE_GMP flag@{\tt CGAL\_USE\_GMP} flag!using}

Here is a short example on how these macros can be used. Assume you have some
piece of code that depends on whether you have \leda-4.0 or later.
\begin{verbatim}
#ifdef CGAL_USE_LEDA
#include <LEDA/basic.h>
#endif

#if defined(CGAL_USE_LEDA) && __LEDA__ >= 400
... put your code for LEDA 4.0 or later ...
#else
... put your code for the other case ...
#endif 
\end{verbatim}

\section{Identifying compilers and architectures\label{sec:which_compiler}}
\ccIndexSubitem{compilers}{identifying}
\ccIndexSubitem{macros}{for compiler identification}
\ccIndexSubitem{flag}{for copmiler}

Every compiler defines some macros that allow you to identify it; see 
the following table.

\vspace{5mm}\fbox{\begin{tabular}{lll}
Borland 5.4     & \texttt{\_\_BORLANDC\_\_}         & 0x540\\
Borland 5.5     & \texttt{\_\_BORLANDC\_\_}         & 0x550\\
Borland 5.5.1   & \texttt{\_\_BORLANDC\_\_}         & 0x551\\
GNU 3.2.1       & \texttt{\_\_GNUC\_\_}             & 3\\
GNU 3.2.1       & \texttt{\_\_GNUC\_MINOR\_\_}      & 2\\
GNU 3.2.1       & \texttt{\_\_GNUC\_PATCHLEVEL\_\_} & 1\\
Microsoft VC7.1 & \texttt{\_MSC\_VER}               & 1310\\
Microsoft VC8.0 & \texttt{\_MSC\_VER}               & 1400\\
Intel 7.0       & \texttt{\_\_INTEL\_COMPILER}      & ???\\
SGI 7.3         & \texttt{\_COMPILER\_VERSION}      & 730\\
SUN 5.0         & \texttt{\_\_SUNPRO\_CC}           & 0x500\\
SUN 5.3         & \texttt{\_\_SUNPRO\_CC}           & 0x530\\
\end{tabular}}\vspace{5mm}

\noindent There are also flags to identify the architecture.
\ccIndexSubitem{architecture}{identifying}
\ccIndexSubitem{macros}{for architecture identification}
\ccIndexSubitem{flag}{for architecture}

\vspace{5mm}\fbox{\begin{tabular}{lll}
    SGI   & \texttt{\_\_sgi}\\
    SUN   & \texttt{\_\_sun}\\
    Linux & \texttt{\_\_linux}\\
\end{tabular}}\vspace{5mm}

\section{Known problems and workarounds\label{sec:problems_and_workarounds}}

For (good) reasons that will not be discussed here, it was decided to
use \CC\ for the development of \cgal. An international standard for
\CC\ has been sanctioned in 1998 \cite{cgal:ansi-is14882-98} and the
level of compliance varies widely between different
compilers\index{C++ standard@\CC\ standard}, let alone bugs.

\subsection{Workaround flags\label{sec:workaround_flags}}
\ccIndexMainItemBegin{workaround flags}

In order to provide a uniform development environment for \cgal\ that
looks more standard compliant than what the compilers provide, a number
of workaround flags and macros have been created.  Some of the
workaround macros are set in \ccAnchor{http://www.cgal.org/Manual/include/CGAL/config.h}{\texttt{<CGAL/config.h>}}
\ccIndexMainItem{\tt config.h}
using the macros 
listed in Section~\ref{sec:which_compiler} to identify the compiler. 
But most of them are set in the platform-specific configuration files
\ccIndexSubitem{configuration}{file}
\begin{center}
\texttt{<CGAL/config/}{\em os-compiler}\texttt{/CGAL/compiler\_config.h>}
\end{center}
where \textit{os-compiler} refers to a string describing your
operating system and compiler that is defined as follows.
\ccIndexSubitemBegin{flag}{for OS \& compiler}

\begin{center}
  \textit{$<$arch$>$\texttt{\_}$<$os$>$\texttt{-}$<$os-version$>$\texttt{\_}$<$comp$>${\tt
      -}$<$comp-version$>$}
\end{center}

\begin{description}
\item[$<$arch$>$] is the system architecture as defined by ``{\tt
    uname -p}'' or ``\texttt{uname -m}'',
\item[$<$os$>$] is the operating system as defined by ``\texttt{uname
    -s}'',
\item[$<$os-version$>$] is the operating system version as defined by
  ``\texttt{uname -r}'',
\item[$<$comp$>$] is the basename of the compiler executable (if it
  contains spaces, these are replaced by "-"), and
\item[$<$comp-version$>$] is the compiler's version number (which
  unfortunately can not be derived in a uniform manner, since it is
  quite compiler specific).
\end{description}

\noindent Examples are \texttt{mips\_IRIX64-6.5\_CC-n32-7.30} or {\tt
  sparc\_SunOS-5.6\_g++-2.95}. For more information, see the \cgal\ 
\ccAnchor{http://www.cgal.org/Manual/doc_html/installation/contents.html}{installation guide}.
\ccIndexSubitemEnd{flag}{for OS \& compiler}

This platform-specific configuration file is created during
\ccIndexSubsubitem{configuration}{file}{creation}
\ccIndexMainItem{installation}
installation by the script \texttt{install\_cgal}. The flags listed below
are set according to the results of test programs that are compiled and run.
These test programs reside in the directory
\begin{center}
\verb|$(CGAL_ROOT)/config/testfiles|
\end{center}
where \verb|$(CGAL_ROOT)| represents the installation directory for the library.
The names of all testfiles, which correspond to the names of the flags, 
\ccIndexSubitem{workaround flags}{names}
start with ``\texttt{CGAL\_CFG\_}'' followed by
\begin{itemize}
\item \textit{either} a description of a bug ending with
  ``\texttt{\_BUG}''
\item \textit{or} a description of a feature starting with
  ``\texttt{NO\_}''.
\end{itemize}
For any of these files a corresponding flag is set in the 
platform-specific configuration file, iff either compilation or execution
fails. The reasoning behind this sort of negative scheme is that on
standard-compliant platforms there should be no flags at all.

\InternalOnly{
Which compilers passed which tests can be determined by looking at the 
\ccAnchor{http://cgal.inria.fr/CGAL/Members/testsuite/}%
{test suite results page}\lcTex{ ( 
\nonlinkedpath|http://cgal.inria.fr/CGAL/Members/testsuite/|)}
for the package \texttt{Installation}.
}

\noindent Currently (CGAL-3.1-I-33), we have the following configuration
test files (and flags). The short descriptions that are given in the files are 
included here. In some cases, it is probably necessary to have a look at the
actual files to understand what the flag is for. This list is just to
give an overview.  See the section on
\ccAnchor{http://www.cgal.org/Manual/doc_html/installation/Chapter_installation.html#Section_17}{troubleshooting} in the installation guide
for more explanation of some of these problems and known workarounds. 
Be sure to have a look at \texttt{Installation/config/testfiles/} to have an
uptodate version of this list.

\begin{description}
\item[{\tt CGAL\_CFG\_CCTYPE\_MACRO\_BUG}]~\\
 \index{cctype functions@{\tt cctype} functions!as macros} %
 \ccIndexSubitem{compiler bugs}{macros} %
 This flag is set if a compiler defines the standard C library
 functions in {\tt cctype} ({\tt isdigit} etc.) as macros.  According
 to the standard they have to be functions.
 
\item[{\tt CGAL\_CFG\_LONGNAME\_BUG}]~\\
 \ccIndexMainItem{long-name problem} %
 \ccIndexSubitem{compiler bugs}{long symbol names} %
 This flag is set if a compiler (or assembler or linker) has problems 
 with long symbol names. 

\item[{\tt CGAL\_CFG\_MATCHING\_BUG\_3}]~\\
 \ccIndexSubitem{matching}{pointer type arguments} %
 \ccIndexSubitem{compiler bugs}{function template overloading} %
 This flag is set, if the compiler does not match function arguments
 of pointer type correctly, when the return type depends on the
 parameter's type (\eg, sun C++ 5.3).

\item[{\tt CGAL\_CFG\_MATCHING\_BUG\_4}]~\\
 \ccIndexSubitem{matching}{function template arguments} %
 \ccIndexSubitem{compiler bugs}{function template overloading} %
 This flag is set, if a compiler cannot distinguish the signature of
 overloaded function templates, which have arguments whose type
 depends on the template parameter.  This bug appears for example on
 Sunpro~5.3 and 5.4.

\item[{\tt CGAL\_CFG\_NET2003\_MATCHING\_BUG}]~\\
 \ccIndexSubitem{matching}{member functions} %
 \ccIndexSubitem{compiler bugs}{member functions} %
 This flag is set, if the compiler does not match a member definition
 to an existing declaration. This bug shows up on VC~7.1~Beta
 (\ccc{cl1310}).

\item[{\tt CGAL\_CFG\_NO\_BIG\_ENDIAN}]~\\
  \ccIndexMainItem{big-endian}
  \ccIndexMainItem{little-endian}
  The byte order of a machine architecture distinguishes into
  big-endian and little-endian machines.  This flag is
  set if it is a little-endian machine.
 
\item[{\tt CGAL\_CFG\_NO\_KOENIG\_LOOKUP}]~\\
  \ccIndexMainItem{Koenig lookup} %
  \ccIndexSubitem{compiler bugs}{name lookup} %
  This flag is set if the compiler does not support the operator
  Koenig lookup. That is, it does not search in the namespace of the
  arguments for the function.
 
\item[{\tt CGAL\_CFG\_NO\_LIMITS}]~\\
  \ccIndexMainItem{limits} %
  This flag is set if a compiler does not know the limits.

\item[{\tt CGAL\_CFG\_NO\_LOCALE}]~\\
  \ccIndexMainItem{locale} %
  This flag is set if a compiler does not know the locale classic. 

\item[{\tt CGAL\_CFG\_NO\_LONG\_LONG}]~\\
  \index{long long@{\tt long long}} %
  The \ccc{long long} built-in integral type is not part of the
  \textsc{Iso} C++ standard, but many compilers support it
  nevertheless, since it is part of the \textsc{Iso} C standard. This
  flag is set if it is supported.
  
% \item[{\tt CGAL\_CFG\_NO\_SFINAE}]~\\ 
%   \ccIndexSubitem{compiler bugs}{template instantiation} %
%   \ccIndexMainItem{sfinae} %
%   \ccIndexMainItem{substitution failure is not an error} %
%   This flag is set if the compiler doesn't support the SFINAE
%   principle (Substitution Failure Is Not An Error), which we
%   eventually plan to use for the next version of the kernel design.
 
\item[{\tt CGAL\_CFG\_NO\_TMPL\_IN\_TMPL\_DEPENDING\_FUNCTION\_PARAM}]~\\
  \ccIndexSubitem{compiler bugs}{template parameters} %
  \ccIndexSubitem{template}{template parameter} %
  This flag is set of a compiler does not support member functions
  that have parameter types that are dependent on the template
  parameter list of the class and are implemented outside of the class
  body (\eg, g++ 2.95.2).
  
\item[{\tt CGAL\_CFG\_NO\_TMPL\_IN\_TMPL\_PARAM}]~\\
  \ccIndexSubitem{compiler bugs}{template parameters} %
  \ccIndexSubitem{template}{template parameter} %
  Nested templates in template parameter, such as ``\texttt{template <
    template <class T> class A>}'' are not supported by any compiler.
  This flag is set if they are not supported.

\item[{\tt CGAL\_CFG\_OUTOFLINE\_TEMPLATE\_MEMBER\_DEFINITION\_BUG}]~\\
  \ccIndexSubitem{compiler bugs}{member definitions} %
  This flag is set, if a compiler does not support the definition of
  member templates out of line, \ie, outside class scope. The solution
  is to put the definition inside the class. This is a feature of
  SunPRO 5.5.

\end{description}
\ccIndexMainItemEnd{workaround flags}

\subsection{Macros connected to workarounds/compilers\label{sec:workaround_macros}}
\ccIndexSubitemBegin{macros}{for workarounds}

Some macros are defined according to certain workaround flags. This is
done to avoid some \texttt{\#ifdef}s in our actual code.

\begin{description}
\item[\texttt{CGAL\_LITTLE\_ENDIAN}] set, iff
 \index{cgal_little_endian macro@\texttt{CGAL\_LITTLE\_ENDIAN} macro}
  \ccIndexMainItem{little-endian}
  {\texttt{CGAL\_CFG\_NO\_BIG\_ENDIAN}} is set.
\item[\texttt{CGAL\_BIG\_ENDIAN}] set, iff
 \index{cgal_big_endian macro@\texttt{CGAL\_BIG\_ENDIAN} macro}
  \ccIndexMainItem{big-endian}
  {\texttt{CGAL\_CFG\_NO\_BIG\_ENDIAN}} is not set.

\item[\texttt{CGAL\_DEPRECATED}]
 \index{cgal_deprecated macro@\texttt{CGAL\_DEPRECATED} macro}
  \ccIndexMainItem{deprecated}
  used to declare a function as deprecated --- just add it before the
  function declaration:
\begin{verbatim}
CGAL_DEPRECATED void foo(int i)
{
 ...
}
\end{verbatim}
\end{description}
\ccIndexSubitemEnd{macros}{for workarounds}


\subsection{Various other problems and solutions\label{sec:various_problems}}

%\subsection{The Long-name Problem}
%\label{sec:long_name_problem}
%
%Because of the decisions to avoid using abbreviations in identifiers
%and to write generic code using template parameters, the names of things
%in \cgal\ can become a little long.  No, make that very long.
%While one can argue that 
%the genericty and readability of the code are good reasons to have such
%long names, some assemblers and compilers are not so easily convinced; 
%they have problems 
%with names that are beyond a certain length. In particular, the Solaris 
%assembler limit names to about ??? and VC++ truncates names used in
%debugging to about 250 bytes and, in general, to about 2000 bytes.  
%For Solaris, the solution is to install the GNU assemblers ({\tt gas}).
%However, since this is not always possible (and is not a solution for
%VC++), one should also define a file of abbreviations for those classes
%where the problems appear.  See the file 
%\ccAnchor{http://www.cgal.org/Manual/include/CGAL/Triangulation_short_names_2.h}{\texttt{<CGAL/Triangulation\_short\_names\_2.h>}}
%for an example of such a file.
%

\begin{description}
\item[\textbf{Templated member functions}]

For SunPRO \CC\, member function templates with dependent return type
must be defined in the body of the class. 

\item[\textbf{Function parameter matching}]

The function parameter matching capacities of Visual \CC are rather limited. 
Failures occur when your function \ccc{bar} is like
\begin{verbatim}
  bar(std::some_iterator<std::some_container<T>>....) ...
  ...
  bar(std::some_iterator<std::some_other_container<T>>....) ...
\end{verbatim}
VC++ fails to distinguish that these parameters have different types.
A workaround is to add some dummy parameters that are defaulted to
certain values, and this affects only the places where the functions
are defined, not the places where they are called.
This may not be true anymore for recent VC++ versions.

\item[\textbf{typedefs of derived classes}]
Microsoft VC++ does not like the following sorts of typedefs that are 
standard
\begin{verbatim}
class A : public B::C {
        typedef B::C C;
};
\end{verbatim}
It says that the typedef is "redefinition".  So such typedefs should be 
enclosed by
\begin{verbatim}
#ifndef _MSC_VER

#endif
\end{verbatim}
This may not be true anymore for recent VC++ versions.

\item[\textbf{parse error in constructions}]
\ccIndexSubitem{parse error}{construction}
The following program will produce a parse error with g++ 3.1.

\begin{verbatim}
#include <CGAL/Segment_circle_2.h> 

typedef CGAL::Segment_circle_2<double> Curve; 
typedef Curve::Segment                 Segment; 
typedef Curve::Point                   Point; 

int main()
{
  Segment s1(Point(0,0), Point(1,1));
  Curve curve(Segment(Point(0,0), Point(1,1))); // parse error
  
  // ...
  return 0;
}
\end{verbatim}

This is a well-known bug in the Gnu compiler 
(see \path|http://gcc.gnu.org/bugs.html#parsing|).
The workaround is to split :

\verb|   Curve curve(Segment(Point(0,0), Point(1,1)));| 

into, {\textit e.g.,} :
\begin{verbatim}
   Segment s (Point(0,0), Point(1,1));
   Curve rude_curve(s);
\end{verbatim}

\end{description}
\ccIndexMainItemEnd{configuration}


\InternalOnly{
\section{Requirements and recommendations\label{sec:portability_req_and_rec}}
}
\InternalOnly{
\noindent
Recommendations:
\begin{itemize}
\item Workarounds for a compiler bug or a missing feature should not
      be treated on a per-compiler basis. When you detect a deficiency,
      you should rather write a short test program that triggers the setting
      of a flag for this deficiency during configuration.
\item Avoid classes having friend functions and member functions with
      the same name. \texttt{g++ 2.95.*} does not like that. This holds also for
      operators, especially \ccc{operator-()}.
\end{itemize}
}

\ccIndexMainItemEnd{portability}

