\chapter{Example and Demo Programs\label{chap:examples_and_demos}}
\ccIndexMainItemBegin{example programs}
\ccIndexMainItemBegin{demo programs}

The best way to illustrate the functionality provided by the library
is through programs that users can compile, run, copy and modify to
their hearts' content.  Thus every package should contain some of
these programs.  In \cgal\ we distinguish between two types of
programs:  those that provided graphical output (demos) and those that
do not (examples).

In this chapter we provide guidelines for the development
of these programs and their inclusion in the documentation.  See
Sections~\ref{sec:examples_subdirectory} and~\ref{sec:demo_subdirectory}
for a description of the directory structure required for example and
demo programs, respectively. Note in particular that each directory
should contain a \texttt{README} file that explains what the programs do and
how one interacts with them.

\section{Coding conventions\label{sec:ex_and_demo_coding}}
\ccIndexSubitem{example programs}{coding conventions}
\ccIndexSubitem{demo programs}{coding conventions}

Remember that these programs are likely to be a user's first introduction
to the library, so you should be careful to follow our coding conventions 
(Chapter~\ref{chap:code_format}) and good programming practice in these 
programs.  In particular:

\begin{itemize}
   \item Do \textbf{not} use the statements 
         \begin{verbatim}
         using namespace CGAL;
         using namespace std;
         \end{verbatim}
         \ccIndexSubitem{\tt using}{in examples and demos}
         We discourage the use of these as they introduce more names than
         are necessary and may lead to more conflicts than are necessary.
   \item As of release 2.3, you can include only the kernel include file
         (\eg, \texttt{Cartesian.h} or \texttt{Homogeneous.h}) to get all
         kernel classes as well as the \texttt{basic.h} file.  All example
         and demo programs should do this.  For exmaple, you should have
         simply:
         \begin{verbatim}
         #include <CGAL/Cartesian.h>
         \end{verbatim}
         instead of:
         \begin{verbatim}
         #include <basic.h>
         #include <CGAL/Cartesian.h>
         #include <CGAL/Point_2.h>
         #include <CGAL/Triangle_2.h>
         #include <CGAL/Segment_3.h>
         // etc.
         \end{verbatim}
   \item Types should be declared using the following syntax:
         \begin{verbatim}
         typedef CGAL::Cartesian<double>     Kernel;
         typedef Kernel::Point_2             Point_2;
         typedef Kernel::Triangle_3          Triangle_3;
         \end{verbatim}
         instead of this syntax:
         \begin{verbatim}
         typedef CGAL::Cartesian<double>     Kernel;
         typedef Point_2<Kernel>             Point_2;
         typedef Triangle_3<Kernel>          Triangle_3;
         \end{verbatim}
         Although both will work, the former is to be preferred since
         it reflects that the types are actually part of the kernel and
         also reflects the new (as of release 2.3) kernel design that allows
         types to be easily exchanged and adapted.  
    
         Note also that the typedef used above is
         \begin{verbatim}
         typedef CGAL::Cartesian<double>     Kernel;
         \end{verbatim}
         instead of
         \begin{verbatim}
         typedef CGAL::Cartesian<double>     R; // for representation
         \end{verbatim}
         This also reflects the new design, where the kernel classes are
         kernels containing objects and predicates not just representation
         classes for the objects.
\end{itemize}

\section{The Programs\label{sec:ex_and_demo_content}}
\ccIndexSubitem{example programs}{content}
\ccIndexSubitem{demo programs}{content}

The following guidelines should be followed to the greatest extent possible
when writing the example and demo programs.
\begin{itemize}
   \item Provide simple programs with which a beginner can get started.
   \item Provide more involved programs that illustrate the power of the
         software.
   \item Provide programs that truly exercise the data structure.  Though you
         may have some canned programs that work on fixed data sets to 
         illustrate specific things, you should also have one (or more)
         programs that work on 
         \ccAnchor{http://www.cgal.org/Manual/doc_html/support_lib/Generator/Chapter_generators.html}{randomly generated} or user-generated data. 
         This illustrates confidence in the software (and can also build
         confidence by highlighting bugs).
   \item Take some care to design a good interface to the program; 
         the packaging of a product does make a difference.
\end{itemize}

\section{Including programs in documentation\label{sec:programs_in_doc}}
\ccIndexSubitem{example programs}{in manuals}

All programs included in the documentation should be included in either
the \texttt{examples} or the \texttt{demo} directory to ensure that:
\begin{itemize}
   \item[(a)] the programs will be included in the test suite and
   \item[(b)] the user can more easily use the example program
\end{itemize}
The macro to do this inclusion is \verb|\ccIncludeExampleCode|.
See Chapter~\ref{chap:specification} and the documentation of the
\ccAnchor{http://www.cgal.org/Members/Manual_tools}{Manual tools} for more 
details.

The reverse statement, that all programs in the \texttt{examples}
and \texttt{demo} directory should be included in the documentation,
is not necessarily true.  Ideally the programs included in the documentation
will be a proper (non-empty) subset of the programs included in the 
distribution.

Please make sure that your example programs do not have too lengthy lines
as they are copied verbatim in the manual, which can trigger overflows
in the PDF manual.

\section{Demo programs on the web\label{sec:demos_web_page}}
\ccIndexSubitem{demo programs}{on the web}

We maintain a \ccAnchor{http://www.cgal.org/demos.html}{page of demo 
programs} on the \cgal\ web site
\lcTex{ (\path|http://www.cgal.org/demos.html|)}.  For each demo on
the page, we have a screen shot of the demo in action, a link to the source
code, and a precompiled executable for the Windows platform.

If you have other demos you would like to add to the page (and the more 
here the better), you should do the following:
\begin{enumerate}
\item Compile the demo and create a screen shot (as a \texttt{.gif} or 
      \texttt{.jpg} file) and then create a smaller version of this 
      (approx $235 \times 280$) that can be used as a ``thumbnail'' 
      (admittedly for a rather large thumb) on the page.  These names of
      the image files should be as follows: 
      \centerline{\ccc{<program>_}\texttt{[big|small].[gif|jpg]}},
      where \textit{$<$prog\_name$>$} is the name of the program from which
      the screen shots were made (an nice menmonic name like 
      \ccc{alpha_shapes_2} or \ccc{hull_of_int_points_2}).

\item Create the \texttt{.exe} file for use under Windows (probably with the 
      compiler optimization flags turned on)
\item \texttt{gzip} the \texttt{.exe} file
\item Package these together with the source code using the following structure:
      \begin{verbatim}
          images/demo/<program>_big.[gif|jpg]
          demo/<program>.exe
          demo/<subdir>/<program>.cpp
      \end{verbatim}
      where \ccc{<subdir>} is the name of the subdirectory in which the program
      is (or will be) found in the distribution. \ccc{<program>} should be a 
      name that is self-explanatory (\eg, \texttt{demo.cpp} is a bad name here).

\item Send the package to \ccAnchor{mailto:ameyer@mpi-sb.mpg.de}{the web page
      maintainer} together with a one-paragraph description similar to the
      ones on the page already.
\end{enumerate}

\InternalOnly{

\section{Requirements and recommendations\label{sec:ex_and_demos_req_and_rec}}

\noindent
Requirements:
\begin{itemize}
   \item Follow the coding conventions outlined in 
         Section~\ref{sec:ex_and_demo_coding}.
   \item Include all example and demo programs from the documentation in
         the \texttt{examples} or \texttt{demo} directories.
\end{itemize}

\noindent
Recommendations:
\begin{itemize}
   \item Place a demo of your package on the web site.
\end{itemize}

}

\ccIndexMainItemEnd{example programs}
\ccIndexMainItemEnd{demo programs}
