% =============================================================================
% The CGAL Developers' Manual
% Chapter: Testing
% -----------------------------------------------------------------------------
% file   : testing.tex
% authors: Mathair Baeskin <baeskin@infsn2.informatik.uni-halle.de>
% -----------------------------------------------------------------------------
% $Id$
% $Date$
% =============================================================================

\chapter{Testing\label{chap:testing}}
\ccChapterRelease{Chapter Version: 1.2}
\ccChapterAuthor{Matthias B\"asken}
\ccChapterAuthor{Yves Brise ({\tt ybrise@inf.ethz.ch})}
\ccIndexMainItemBegin{test suite}

%\section{Introduction}
{\HUGE This page is no longer maintained. The content has been moved to

https://cgal.geometryfactory.com/CGAL/Members/wiki/Testing}

The \cgal\ test suite is a way to test the compilation and execution of \cgal\ 
programs automatically (\ie, without user interaction) on a number of 
different platforms.  Developers should, of course, thoroughly test their 
code on their own development platform(s) \textbf{before} submitting it.
The test suite serves as a way to test on additional platforms not available
to the developer.  

\section{What a test suite for a package should contain\label{sec:whats_in_test_suite}}

The test suite helps the developer(s) of a package to 
\begin{itemize}
\item detect compilation problems on the various platforms
\item detect runtime problems
\item check the correctness of the algorithms in the package
\end{itemize}

That does not mean that the test suite is a platform for initial testing of 
code.  New code should be tested on different platforms by the developer
before submission.

It is strongly recommended for a test suite of a package to
\begin {itemize}
\item Cover the complete code of the package; every (member) function
      should be called at least once. (See Section~\ref{sec:gcov} for
      a description of a tool you can use to test for code coverage.)
\item Use more than one instantiation of templated functions or classes.
\item A lot of classes in \cgal\ can be parametrized by traits classes, so 
      that they are usable with different kernels. In such cases more than one
      kernel should be used for testing.
\item Use pre- and postcondition checkers wherever it is possible in the
      main code.  In the test-suite code itself, the macro \ccc{assert}
      should be used in place of \ccc{CGAL_assertion} to check 
      all conditions, since \ccc{assert} is not disabled by \ccc{CGAL_NDEBUG}. 
\end {itemize}

\section{Using the code coverage tool \texttt{gcov}}
\label{sec:gcov}

The tool {\tt gcov} can be used together with the GNU C++ compiler to test 
for code coverage in your programs and may be helpful when you create your 
\cgal\ test suite programs.  You can find a complete guide to this tool 
in the GNU on-line documentation at
\path|http://gcc.gnu.org/onlinedocs/gcc-2.95.2/gcc_6.html|.
If you want to use the code coverage tool {\tt gcov}, you have to compile 
your programs with the options \texttt{-fprofile-arcs} and 
\texttt{-ftest-coverage}. This generates a file called
\texttt{your\_program.gcda}. Then you run the program, which generates a file
\texttt{your\_program.gcno}. Finally, you can run \texttt{gcov your\_program.cpp}.
This will generate a number of files with the ending \texttt{.gcov} which contain
human readable information. View it in a text editor.
Here is a simple example:\\
\begin{verbatim}
   #include<iostream>

   using namespace std;

   void fu(int val)
   {
    int w,v=0;
    if (val==0) {
     cout << "val == 0!\n";
     for(w=0;w<100;w++) v=v+w;
    }
    else {
     cout << "val != 0!\n";
     for(w=0;w<10;w++) v=v+w;  
    }
 
    cout << "v:" << v << "\n";
   }

   int main()
   {
     fu(0);
     return 0;
   }
\end{verbatim}
First you have to compile the example program \texttt{test.cpp} with the special 
options. Then you have to execute it, and, after this, \texttt{gcov} can be 
used.
\begin{verbatim}
   g++ -fprofile-arcs -ftest-coverage -o test test.cpp
   test
   gcov test.cpp  
\end{verbatim}
\texttt{gcov} will create a file \texttt{test.cpp.gcov} containing output 
from \texttt{gcov}:
\begin{verbatim}
                #include<iostream>
                
                using namespace std;
                
                void fu(int val)
           1    {
           1     int w,v=0;
           1     if (val==0) {
           1      cout << "val == 0!\n";
           1      for(w=0;w<100;w++) v=v+w;
                 }
      ######     else {
      ######      cout << "val != 0!\n";
      ######      for(w=0;w<10;w++) v=v+w;  
                }
                 
           1     cout << "v:" << v << "\n";
                }
                
                int main()
           1    {
           1      fu(0);
           1      return 0;
                }
\end{verbatim}
The lines that were not executed will be marked with \verb|######|,
so you will see what should be added in the (\cgal) test suite programs.

There are a few pitfalls when using CMake and \texttt{gcov} together.
First note, that with CMake you can add compiler flags using the \texttt{CGAL\_CXX\_FLAGS} variable. See the
\ccAnchor{http://www.cgal.org/Manual/latest/doc_html/installation_manual/Chapter_installation_manual.html#Subsection_8.2}{\cgal Installation Manual} to get more information on that. Typing \texttt{make your\_program} to compile your program is straightforward, as is running it. But then, the \texttt{.gcda}
and \texttt{.gcno} are created within the directory \texttt{CMakeFiles/your\_program.dir/} relative to your project path. You first have to copy them to the same folder where \texttt{your\_program.cpp} resides.
If you don't do that, running \texttt{gcov your\_program.cpp} will give an error
message.


\section{Test suite directory\label{sec:test_suite_directory}}

The test suite is located in the directory {\tt test} of the internal 
releases of \cgal. This directory is not part of external releases. The 
directory {\tt test}\index{test directory@{\tt test} directory!for test suite} 
contains:
\begin{itemize}
   \item a script {\tt run\_testsuite}
         \index{run_testsuite script@{\tt run\_testsuite} script}
         that is (not surprisingly) used to run the test suite.
   \item a subdirectory for every package included in the internal release.
         These subdirectories are created from the {\tt test} directories
         of the packages by copying the source, include, and input files from 
         these directories and adding makefiles and {\tt cgal\_test} scripts 
         where needed.  See Section~\ref{sec:test_subdirectory} for more
         information about the proper structure of the {\tt test} directory
         for a package.
         \index{test directory@{\tt test} directory!for packages}
   \item a subdirectory with a name that ends in {\tt \_Examples} for every 
         package that was submitted with an {\tt examples} directory 
         (Section~\ref{sec:examples_subdirectory})
         \index{examples directory@{\tt examples} directory}
         \ccIndexSubitem{test suite}{examples in}
         \ccIndexSubitem{example programs}{in test suite}
   \item a subdirectory with a name that ends in {\tt \_Demo} for every
         package that was submitted with a {\tt demo} directory
         (Section~\ref{sec:demo_subdirectory})
         \index{demo directory@{\tt demo} directory}
         \ccIndexSubitem{test suite}{demos in}
         \ccIndexSubitem{demo programs}{in test suite}
\end{itemize}

The test suite will attempt to compile all the programs in the
subdirectories of {\tt test} and to run all except
the demo programs (which usually require user interaction) by using the 
{\tt cgal\_test} scripts 
(Sections~\ref{sec:test_subdirectory} and~\ref{sec:create_cgal_test})
\index{cgal_test script@{\tt cgal\_test} script}
and will save the results in files in the package subdirectories
(Section~\ref{sec:test_suite_output}). 
Even if a program fails to compile or run, the test suite will continue.


\section{Test suite input\label{sec:test_suite_input}}
\ccIndexSubitemBegin{test suite}{input}

Input to programs in the test suite can be supplied in three different
ways:
\begin{description}
   \item[data files in the {\tt data} directory]
        \index{data directory@{\tt data} directory}
        \ccIndexSubsubitem{test suite}{input}{from files}
         As described in Section~\ref{sec:test_subdirectory}, a package's
         {\tt test} directory may contain a subdirectory {\tt data} that
         contains input files for the test programs.
   \item[{\tt *.cin} files]
        \index{test\ suite!input!from {\tt cin}}
         If a test program \texttt{program.cpp} requires input from standard 
         input (\ie, {\tt cin}), you should put a file called 
         \texttt{program.cin} in the test directory. 
         The test suite will then execute the program using the command
         \begin{verbatim}
         ./program < program.cin
         \end{verbatim}
   \item[command-line arguments supplied in the {\tt cgal\_test} script]
        \ccIndexSubsubitem{test suite}{input}{from command-line}
        {\em You are discouraged from using this option to give input values
        to your programs} since it requires you to edit and submit a 
        {\tt cgal\_test} script; see Section~\ref{sec:create_cgal_test}.

        \index{cgal_test script@{\tt cgal\_test} script}
         However, if a test program \texttt{program.cpp} absolutely requires 
         command-line parameters, you should do the following. Use
         \texttt{create\_cgal\_test} to create the script \texttt{cgal\_test}.
         This file contains an entry of the form 
         \begin{verbatim}
         compile_and_run program 
         \end{verbatim}
         Just put the command-line 
         parameters for \texttt{program} at the end of this line: 
         \begin{verbatim}
         compile_and_run program  arg1 arg2 ..
         \end{verbatim}
         The test suite will then execute the program using the command

         \begin{verbatim}
         ./program <arg1> <arg2> ...
         \end{verbatim}
\end{description}
\ccIndexSubitemEnd{test suite}{input}


\section{Running the test suite\label{sec:running_test_suite}}
\index{run_testsuite script@{\tt run\_testsuite} script|(}

The test suite is run using the {\tt run\_testsuite} script that is distributed
with every internal release in the {\tt test} directory.  There are several
ways you can customize this script to meet you needs:

\begin{itemize}
\item Add additional compiler and linker flags by setting the variables
      {\tt TESTSUITE\_CXXFLAGS}%
      \index{TESTSUITE_CXXFLAGS variable@{\tt TESTSUITE\_CXXFLAGS} variable} 
      and {\tt TESTSUITE\_LDFLAGS} 
      \index{TESTSUITE_LDFLAGS variable@{\tt TESTSUITE\_LDFLAGS} variable} 
      \ccIndexSubitem{header files}{overriding}
      at the top of the
      script.  These variables are prepended to {\tt CXX\_FLAGS} and
      {\tt LDFLAGS}, respectively, in the test suite makefiles.  So, for
      example, if you have a directory \verb|experimental/include/CGAL| 
      containing new or experimental \cgal\ files,  you can do the following:
      \begin{center}
      \verb|TESTSUITE_CXXFLAGS="-Iexperimental/include"|
      \end{center}
      and in this way test with your new files without overwriting
      the originals.
 
\item Export additional environment variables by adding lines to the 
      \ccIndexSubitem{environment variables}{test suite}
      \ccIndexSubitem{test suite}{environment variables}
      {\tt run\_testsuite} script. As an example, it will be demonstrated 
      how to export the {\tt LD\_LIBRARY\_PATH} by editing \texttt{run\_testsuite}.%
      \index{LD_LIBRARY_PATH variable@{\tt LD\_LIBRARY\_PATH} variable}
      \begin{enumerate}
      \item Add the line
            \begin{center}
               \verb|LD_LIBRARY_PATH=<your library path>|
            \end{center}
            to the script.
      \item Append {\tt LD\_LIBRARY\_PATH} to the line
            \begin{center}
            \verb|export PLATFORM CGAL_MAKEFILE TESTSUITE_CXXFLAGS TESTSUITE_LDFLAGS|
            \end{center}
            in the script.
      \end{enumerate}

      After this, the programs from the test suite will be run using the
      {\tt LD\_LIBRARY\_PATH} that was specified in step 1. 

\item Run the test suite on more than one platform by adding a line at the 
      bottom of the script of the form
      \begin{verbatim}
           run_testsuite <include makefile>
      \end{verbatim}
      for every platform that you wish to test. Just substitute for
      \verb|<include makefile>| the appropriate include makefiles that
      were generated during installation. (Don't forgot to use the
      full path name for the makefile!)  By default, the last line in the
      file is
      \begin{verbatim}
           run_testsuite $CGAL_MAKEFILE
      \end{verbatim}
      so you need not make any changes if you run the testsuite on only one
      platform and have set the {\tt CGAL\_MAKEFILE} environment variable
      properly.%
      \index{CGAL_MAKEFILE variable@{\tt CGAL\_MAKEFILE} variable!and test suite}
\end{itemize}

After these steps you are ready to run the test suite. It can be run in two
different ways:
\begin{verbatim}
./run_testsuite
\end{verbatim}

The test suite will run the tests from all test directories. This may take a
considerable amount of time.

\begin{verbatim}
./run_testsuite <dir1> <dir2> ...
\end{verbatim}

The test suite will run only the test programs in the test directories
\begin{verbatim}<dir1> <dir2> ... \end{verbatim}
\index{run_testsuite script@{\tt run\_testsuite} script|)}

To run an entire \cgal\ test suite automatically, including downloading of
an internal release, configuration, and installation of the library, you 
can use the {\tt autotest\_cgal} script described in 
Section~\ref{sec:autotest_cgal}.


\section{Files generated by the test suite\label{sec:test_suite_output}}
\ccIndexSubitemBegin{test suite}{output files}

The testsuite will generate the following output files:

\begin{itemize}
\item \verb|<testdir>/ErrorOutput_<platform>|
\index{ErrorOutput files@{\tt ErrorOutput} files}

This file contains two lines for every program that was tested on
platform \texttt{<platform>} in the test directory \texttt{<testdir>}. 
The first line
tells if the compilation was successful and the second line tells if
the execution was successful (\ie, the program returned the value 0).
(See Section~\ref{sec:test_subdirectory} for more details.)

\item \verb|<testdir>/ProgramOutput.<program>.<platform>|
\index{ProgramOutput files@{\tt ProgramOutput} files}

This file contains the console output from the test program
\texttt{<program.cpp>} run on platform \texttt{<platform>}.

\item \verb|<testdir>/CompilerOutput_<platform>|
\index{CompilerOutput files@{\tt CompilerOutput} files}

This file contains the compiler output from platform 
\texttt{<platform>} for all programs.

\item \verb|error.txt|
\index{error.txt@{\tt error.txt}}

This is just a concatenation of all the \texttt{ErrorOutput} files that were
generated during the last run of the test suite.
\end{itemize}
\ccIndexSubitemEnd{test suite}{output files}


\section{Test suite results\label{sec:test_suite_results}}
\ccIndexSubitemBegin{test suite}{results}
The results of test suites run on the various supported or soon-to-be-supported
platforms are posted on the 
\ccAnchor{http://cgal.geometryfactory.com/CGAL/Members/testsuite/}%
{test suite results} 
page 
\lcTex{(\path|http://cgal.geometryfactory.com/CGAL/Members/testsuite/|)}.
\ccIndexSubitemEnd{test suite}{results}

\section{Requirements and recommendations\label{sec:testing_req_and_rec}}

\noindent
Requirements:
\begin{itemize}
   \item Test your code thoroughly \textbf{before} submitting it.
   \item Obey the directory structure detailed in Section~\ref{sec:test_subdirectory}
   \item Check the test suite results for your package regularly. 
\end{itemize}

\noindent
Recommendations:
\begin{itemize}
   \item Write test suite programs that use more than one instantiation of
         templated functions and classes, call every member function at least
         once, and use more than one kernel. 
   \item Use pre- and postcondition checkers.
   \item Use \texttt{gcov} to test your code for coverage.
   \item Don't submit a makefile for your test suite unless you need to do
         something very special to compile or link your program.  If you find
         you want to do something very special in your makefile, think long
         and hard about whether it's really necessary or not.
   \item Don't submit the script \texttt{cgal\_test} with your package.
\end{itemize}


\ccIndexMainItemEnd{test suite}


