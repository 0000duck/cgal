% =============================================================================
% The CGAL Developers' Manual
% Chapter: Coding Conventions
% -----------------------------------------------------------------------------
% file   : coding.tex
% authors: Sven Schoenherr <sven@inf.ethz.ch>
% -----------------------------------------------------------------------------
% $Id$
% $Date$
% =============================================================================


\chapter{Coding Conventions\label{chap:code_format}}

\ccChapterRelease{Chapter Version: 1.1}
\ccChapterAuthor{Sven Sch{\"o}nherr}

We do not want to impose very strict coding rules on the developers.
What is most important is to follow the \cgal\ naming scheme
described in the next section. However, there are some programming conventions
(Section~\ref{sec:programming_conventions}) that should be adhered to,
rules for the code format
(Section~\ref{sec:code_format}), and a mandatory heading for each
source file (Section~\ref{sec:file_header})


% -----------------------------------------------------------------------------
\section{Naming scheme\label{sec:naming_scheme}}
\ccIndexMainItemBegin{naming scheme}

The \cgal\ naming scheme is intended to help the user
use the library and the developer develop the
library. The rules are simple and easy to remember. Where appropriate,
they aim for similarity with the names used in the \stl. Deviations from the
given rules should be avoided; however, exceptions are possible if
there are \emph{convincing} reasons.

\subsection*{General rules}

\begin{itemize}
\item Words in the names of everything except concepts should be separated by
      underscores. For example, one would use \ccc{function_name} and
       \ccc{Class_name} instead of \ccc{functionName} or \ccc{Classname}.
      \ccIndexSubitem{naming scheme}{word separators}
      \ccIndexSubitem{naming scheme}{underscores}
\item Words in the names of concepts (\textit{e.g.}, template parameters)
      \ccIndexSubitem{naming scheme}{concepts}
      \ccIndexSubitem{naming scheme}{capitalization}
      should be separated using capital letters.  The only use of underscores
      in concept names is before the dimension suffix.  For example, one
      should use a name such as ConvexHullTraits\_2 for the concept
      in contrast to \ccc{Convex_hull_traits_2} for the name of the class that
      is a model of this concept.  This different naming scheme for concepts
      and classes was adopted mainly for two reasons (a) it is consistent with
      the \stl\ (cf. InputIterator) and (b) it avoids name clashes between
      concepts and classes that would require one or the other to have a
      rather contrived name.
\item Abbreviations of words are to be avoided%
      \ccIndexSubitem{naming scheme}{abbreviations}
      ({\em e.g.}, use
      \ccc{Triangulation} instead of \ccc{Tri}).  The only exceptions
      might be standard geometric abbreviations (such as ``CH'' for ``convex
      hull'') and standard data structure abbreviations (such as ``DS'' for
      ``data structure'').  Unfortunately, the long names that result from
      the absence of abbreviations are known to cause problems with some
      compilers.\ccIndexMainItem{long-name problem}
%      See Section~\ref{sec:long_name_problem}
%      for further information about the so-called ``long-name problem.''
\ccIndexSubitemBegin{naming scheme}{capitalization}
\item Names of constants are uppercase ({\em e.g.}, \ccc{ORIGIN}).
      \ccModifierCrossRefOff
      \ccIndexSubitem{constants, global}{naming}
      \ccModifierCrossRefOn
\item The first word of a class or enumeration name should be capitalized
      ({\em e.g.}, \ccc{Quotient}, \ccc{Orientation}).
      \ccIndexSubitem{classes}{naming}
      \ccIndexSubitem{enumerations}{naming}
\item Function names are in lowercase
      ({\em e.g.}, \ccc{is_zero}).
      \ccIndexSubitem{functions}{naming}
\ccIndexSubitemEnd{naming scheme}{capitalization}
\item Boolean function names should begin with a verb.  For example, use
      \ccc{is_empty} instead of simply \ccc{empty} and
      \ccc{has_on_bounded_side} instead of \ccc{on_bounded_side}.%
      \ccIndexSubitem{functions}{naming}%
      \ccIndexSubitem{naming scheme}{boolean functions}%
      \ccIndexSubitem{naming scheme}{predicates}
\item Names of macros should begin with the prefix \ccc{CGAL_}.
      \ccIndexSubitem{macros}{naming}
\end{itemize}

\subsection*{Geometric objects}
\ccIndexSubitemBegin{naming scheme}{geometric objects}
\ccIndexSubitemBegin{naming scheme}{dimension number}
\ccIndexSubitemBegin{geometric objects}{naming}

\begin{itemize}
\item All geometric objects have the dimension as a suffix (\eg, \ccc{Vector_2}
      and \ccc{Plane_3}).
     \textbf{Exception:} For $d$-dimensional objects there may be a dynamic
     and a static version. The former has the suffix \ccc{_d}
     (\eg, \ccc{Point_d}), while the latter has the dimension as the first
     template parameter (\eg, \ccc{Point<d>}).
\end{itemize}
\ccIndexSubitemEnd{geometric objects}{naming}
\ccIndexSubitemEnd{naming scheme}{dimension number}
\ccIndexSubitemEnd{naming scheme}{geometric objects}

\subsection*{Geometric data structures and algorithms}
\ccIndexSubitemBegin{naming scheme}{data structures}
\ccIndexSubitemBegin{naming scheme}{algorithms}
\ccIndexSubitemBegin{data structures}{naming}
\begin{itemize}
\item Names for geometric data structures and algorithms should follow
      the ``spirit'' of the rules given so far, \eg~a data structure for
      triangulations in the plane is named \ccc{Triangulation_2}, and a
      convex hull algorithm in 3-space is named \ccc{convex_hull_3}.
\item Member functions realizing predicates should start with \ccc{is_} or
      \ccc{has_}, \eg~the data structure \ccc{Min_ellipse_2} has member
      functions \ccc{is_empty} and \ccc{has_on_bounded_side}.
      \ccIndexSubitem{naming scheme}{predicates}
\item Access to data structures is given via iterators and
      circulators (Chapter~\ref{chap:iterators_and_circulators}).
      For iterators and functions
      returning them we extend
      the \stl\ names with a prefix describing the items to be accessed.
      For example, the functions \ccc{vertices_begin} and \ccc{vertices_end}
      return a \ccc{Vertex_iterator}. (Note that we use the name of the items
      in singular for the iterator type name and in plural for the functions
      returning the iterator.) Names related to circulators are handled
      similarly, using the suffix \ccc{_circulator}. For example, the
      function \ccc{edges_circulator} returns an \ccc{Edge_circulator}.
      \ccIndexSubitem{naming scheme}{iterators}
      \ccIndexSubitem{naming scheme}{circulators}
      \ccIndexSubitem{naming scheme}{access functions}
      \ccIndexSubitem{iterators}{naming}
      \ccIndexSubitem{circulators}{naming}
      \ccIndexSubitem{access functions}{naming}
\end{itemize}
\ccIndexSubitemEnd{data structures}{naming}
\ccIndexSubitemEnd{naming scheme}{algorithms}
\ccIndexSubitemEnd{naming scheme}{data structures}


\subsection*{Kernel traits}
\ccIndexSubitemBegin{naming scheme}{kernel traits}
\ccIndexSubitemBegin{kernel traits}{naming scheme}
\ccIndexMainItemBegin{functors}

All types in the kernel concept are functor types. We distinguish the
following four categories:
\begin{enumerate}
\item\textbf{generalized predicates}, that is, standard predicates
  returning a Boolean value as well as functions such as
  \ccc{orientation()} that return an enumeration type (values from a
  finite discrete set).
\item\textbf{construction objects}, which replace constructors in the
  kernel,
\item\textbf{constructive procedures} and
\item\textbf{functors replacing operators}.
\end{enumerate}

\ccIndexSubitemBegin{functors}{naming} As detailed below, the names of
these functors reflect the actions performed by calls to
\ccc{operator()}.  This naming scheme was chosen instead of one in
which the computed object determines the name because this latter
naming scheme is more natural for functions that compute values where
the function call can be seen as a name for the object returned
instead of functors that simply maintain data associated with the
computation.  It was also noted that the naming of functions and
functors is not consistent, either in \cgal\ or in the \stl\ (In some
cases the action performed by a function determines its name (\eg,
\ccc{multiply()}) while in others the result of the action determines
the name (\eg, \ccc{product()}).), so achieving complete consistency
with an existing naming scheme is not possible.

Here are the naming rules:
\begin{itemize}
\item All names in the kernel traits have the dimension as a suffix.
      This is necessary because in some cases (\eg, the \ccc{Orientation_2}
      object) the absence of the suffix would cause a name conflict with
      an existing type or class (\eg, the enumeration type \ccc{Orientation}).
    \item The names of generalized predicates are determined by their
      results.  Furthermore, names are as much as possible consistent
      with the current kernel and the \stl. So, for example, we have
      objects like \ccc{Has_on_bounded_side_2}, \ccc{Is_degenerate_2},
      and \ccc{Is_horizontal_2}.  According to the current kernel we
      also have \ccc{Left_turn_2}.  For reasons of consistency with
      \stl, all ``less-than''-objects start with \ccc{Less_},
      \eg,~\ccc{Less_xy_2}.  Further examples are
      \ccc{Less_distance_to_point_2} and
      \ccc{Less_distance_to_line_2}. However, we have \ccc{Equal_2},
      whereas the corresponding \stl\ functor is called
      \ccc{equal_to}. Here, we give our dimension rule higher
      priority.
    \item The names of construction objects (category 2 above) follow
      the pattern \ccc{Construct_type_d}, where \texttt{type\_d}
      is the type constructed, \eg,~\ccc{Construct_point_2} and
      \ccc{Construct_line_2}.  The \ccc{operator()} of these functor
      classes is overloaded.  This reduces the number of names to
      remember drastically, while replacing one of the constructions
      gets more complicated, but is still possible.
    \item Functors in category 3 above can be further subdivided into
      two classes:
      \begin{itemize}
         \item constructive procedures that construct objects whose types are
               known {\em a priori}
         \item procedures that construct objects whose types are not known
               {\em a priori}
      \end{itemize}

      The names of functors in the first class also start with
      \ccc{Construct_} whenever a geometric object is constructed,
      otherwise they start with \ccc{Compute_}.  Here, real numbers
      are not considered to be 1-dimensional geometric objects.  For
      example, on the one hand we have
      \ccc{Construct_perpendicular_vector_2},
      \ccc{Construct_midpoint_3}, \ccc{Construct_circumcenter_d},
      \ccc{Construct_bisector_2}, and \ccc{Construct_point_on_3},
      while on the other hand there are \ccc{Compute_area_2} and
      \ccc{Compute_squared_length_3}.

      For the second class, the names of the objects describe the (generic)
      action, \eg~\ccc{Intersect_2}.
\item The equality operator (\ccc{operator==()}) is replaced by function
      objects with names of the form \ccc{Equal_k}, where {\tt k} is
      the dimension number (\ie, 2, 3, or \ccc{d}).
      For replacing arithmetic operators, we might also provide \ccc{Add_2},
      \ccc{Subtract_2}, \ccc{Multiply_2}, and \ccc{Divide_2}. (As mentioned
      above, the action determines the name, not the result.) We think
      that these objects are not really needed.  They are likely to be
      part of primitive operations that have a corresponding function
      object in the traits.
\end{itemize}
\ccIndexSubitemEnd{kernel traits}{naming scheme}
\ccIndexSubitemEnd{functors}{naming}
\ccIndexMainItemEnd{functors}

In addition, for each functor the kernel traits class has a member
function that returns an instance of this functor.  The name of this
function should be the (uncapitalized) name of the
functor followed by the suffix \ccc{_object}.%
%\index{object functions@{\ccFont \_object} functions}
For example, the function that returns an instance of the
\ccc{Less_xy_2} functor is called \ccc{less_xy_2_object}.
\ccIndexSubitemEnd{naming scheme}{kernel traits}

\subsection*{File names}
\ccIndexSubitemBegin{naming scheme}{source files}
\ccIndexSubitemBegin{source files}{naming scheme}

\begin{itemize}
\item File names should be chosen in the ``spirit'' of the naming rules given
      above.
\item If a single geometric object, data structure, or algorithm is provided
      in a single file, its name (and its capitalization) should be used for
      the file name. For
      example, the file \ccc{Triangulation_2.h} contains the data structure
      \ccc{Triangulation\_2}.
\item If a file does not contain a single class, its name should not begin
      with a capital letter.
\item No two files should have the same names even when capitalization is
      ignored.  This is to prevent overwriting of files on operating systems
      where file names are not case sensitive.  A package that contains
      file names that are the same as files already in the release will be
      rejected by the submission script.
\item The names of files should not contain any characters not allowed by
      all the platforms the library supports.  In particular, it should not
      contain the characters `:', `*', or `\ '.
\item Internal header files -- which are not documented to the user -- should
      have {\tt /internal/} as a directory higher up in their hierarchy.
      For example {\tt CGAL/internal/foo.h} or
      {\tt CGAL/Package/internal/predicates/my\_pred.h}.
\end{itemize}
\ccIndexSubitemEnd{source files}{naming scheme}
\ccIndexSubitemEnd{naming scheme}{source files}
\ccIndexMainItemEnd{naming scheme}


% -----------------------------------------------------------------------------
\section{Programming conventions\label{sec:programming_conventions}}
\ccIndexMainItemBegin{programming conventions}

The first list of items are meant as rules, \ie,~you should follow them.

\begin{itemize}
\item Give typedefs for all template arguments of a class that may be
      accessed later from outside the class.

     The template parameter is a concept and should follow the concept naming
     scheme outlines in the previous section.  As a general rule, the typedef
     should identify the template parameter with a type of the same name that
     follows the naming convention of types.  For example
     \begin{verbatim}
     template < class GeometricTraits_2 >
     class Something {
       public:
         typedef GeometricTraits_2 Geometric_traits_2;
     };
     \end{verbatim}
     For one-word template arguments, the template parameter name should be
     followed by an underscore.  (Note that using a preceding
     underscore is not allowed according to the \CC\ standard; all such names
     are reserved.)\index{C++ standard@\CC\ standard!underscores}%
     \begin{verbatim}
     template < class Arg_ >
     class Something {
       public:
         typedef  Arg_  Arg;
     };
     \end{verbatim}
  \ccIndexSubitem{underscores}{in names}%
  \ccIndexSubitem{naming scheme}{underscores}%
  \ccIndexSubitem{naming scheme}{template parameters}%
\item Use \ccc{const} when a call by reference argument is not
      modified, e.g.~\ccc{int f( const A& a)}%
     \index{call by reference!const and@{\ccFont const} and}.
\item Use \ccc{const} to indicate that a member function does not
      modify the object to which it is applied,
      \eg,~\ccStyle{class A \{ int f( void) const; \};}. This should also be
      done when it is only conceptually \ccc{const}%
      \index{conceptual const-ness@conceptual {\ccFont const}-ness|textbf}.
      This means that the member function \ccc{f()} is {\ccFont const} as seen from
      the outside, but internally it may modify some data members
      that are declared {\ccFont mutable}\ccIndexMainItem{\ccFont mutable}. An example
      is the caching of results from expensive computations.  For more
      information about conceptually {\ccFont const} functions and mutable data
      members see \cite{cgal:m-ec-97}.
\item Prefer \CC-style to C-style casts,%
      \index{casts!C++-style vs. C-style@\CC- style vs. C-style}
      \eg,~use \ccc{static_cast<double>( i)} instead of {\ccFont (double)i}.
\item Protect header files against multiple inclusion,
      \ccIndexSubitem{header files}{multiple inclusion of}
      \eg~the file {\tt This\_is\_an\_example.h} should begin/end with
      \begin{verbatim}
         #ifndef CGAL_THIS_IS_AN_EXAMPLE_H
         #define CGAL_THIS_IS_AN_EXAMPLE_H
         ...
         #endif // CGAL_THIS_IS_AN_EXAMPLE_H
     \end{verbatim}
    \item Support the
      \ccAnchor{http://www.boost.org/doc/libs/release/libs/utility/utility.htm#result_of}{result\_of}
      protocol whenever your functors have more than one return type
      otherwise provide a \ccStyle{result_type} member typedef.

      An example for this is a C++03 style \ccStyle{identity}
      functor:

      \begin{cprog}
        struct Identity {
          template<typename T>
          T& operator()(T& t) { return t; }
          template<typename T>
          const T& operator()(const T& t) { return t; }
          template<typename>
          struct result;
          template<typename F, typename T>
          struct result<F(T&)> {
            typedef T& type;
          };
          template<typename F, typename T>
          struct result<F(const T&)> {
            typedef const T& type;
          };
        };
      \end{cprog}
\end{itemize}

The following items can be seen as recommendations
in contrast to the rules of previous paragraph.

\begin{itemize}
\item Use \verb|#define| sparingly.
\item Do not rename the base types of \CC\ using {\ccFont typedef}.
\item When using an overloaded call, always give the exact match. Use
  explicit casting if necessary.
\item Do not call global functions unqualified, that is, always
  specify explicitly the namespace where the function is defined.
\item Do not give private types or typedefs a name likely to be used as
  a public interface by an other class, {\em e.g.} prefer \ccc{Point_3_}
  to \ccc{Point_3} (in general, add an underscore as suffix). The reason for this convention is that
  \ccAnchor{https://lists-sop.inria.fr/sympa/arc/cgal-develop/2010-10/msg00118.html}{SFINAE
  does not extend to access control} (meaning that the existence of a
  private type can break overloading for other classes).
\end{itemize}
\ccIndexMainItemEnd{programming conventions}


% -----------------------------------------------------------------------------
\section{Code format\label{sec:code_format}}
\ccIndexMainItemBegin{code format}

\begin{itemize}
\item Use indentation with at least two spaces extra per level.%
      \ccIndexSubitem{code format}{indentation}
\item Write only one statement per line.
\item Use \CC-style comments, \eg,~{\ccFont // some comment}.%
      \ccIndexSubitem{code format}{comments}
\end{itemize}
\ccIndexMainItemEnd{code format}


% -----------------------------------------------------------------------------
\section{File header\label{sec:file_header}}
\ccIndexSubitemBegin{source files}{headings for}

Each \cgal\ source file must start with a heading that allows for an easy
identification of the file.  The file header contains:
\begin{itemize}
\item a copyright notice, specifying all the years during which the file has
been written or modified, as well as the owner(s) (typically the institutions
employing the authors) of this work,
\item the corresponding license (at the moment, only LGPLv3+ and GPLv3+
are allowed in \cgal), and a pointer to the file containing its text in the
\cgal\ distribution,
\item a disclaimer notice,
\item then, there are 2 keywords, which are automatically expanded by SVN
(there are options in SVN to suppress these expansions if you need):
\begin{itemize}
% Note : the {} are there to prevent the expansion of the keywords here.
\item \${}URL: \$ : the name of the source file in the repository, it also
helps figuring out which package the file comes from,
\item \${}Id: \$ : the SVN revision number of the file, the date of this revision,
and the author of the last commit.
\end{itemize}
\item Then the authors of (non-negligible parts of) this file are listed, with
optional affiliation or e-mail address.
\end{itemize}

\ccIndexSubitem{example programs}{headings for}
\ccIndexSubitem{demo programs}{headings for}
For example and demo programs, the inclusion of the copyright notice is not
necessary as this will get in the way if the program is included in the
documentation.  However, these files should always contain the name of
the file relative to the \texttt{CGAL\_HOME} directory
(\eg, \nonlinkedpath|examples/Convex_hull_3/convex_hull_3.cpp|)
so the file can be located when seen out of context (\eg, in the documentation
or from the demos web page).

For the test-suite and the documentation source, these are not distributed at
the moment, so there is no policy for now.

\subsection*{GPL version}
\ccIndexSubsubitemBegin{source files}{headings for}{GPL version}

Here follows what this gives for a file under the GPL :

\begin{verbatim}
// Copyright (c) 1999,2000,2001,2002  INRIA Sophia-Antipolis (France).
// All rights reserved.
//
// This file is part of CGAL (www.cgal.org).
// You can redistribute it and/or modify it under the terms of the GNU
// General Public License as published by the Free Software Foundation,
// either version 3 of the License, or (at your option) any later version.
//
// Licensees holding a valid commercial license may use this file in
// accordance with the commercial license agreement provided with the software.
//
// This file is provided AS IS with NO WARRANTY OF ANY KIND, INCLUDING THE
// WARRANTY OF DESIGN, MERCHANTABILITY AND FITNESS FOR A PARTICULAR PURPOSE.
//
// $URL$
// $Id$
// 
//
// Author(s)     : Monique Teillaud <Monique.Teillaud@sophia.inria.fr>
//                 Sylvain Pion
\end{verbatim}
\ccIndexSubsubitemEnd{source files}{headings for}{GPL version}

\subsection*{LGPL version}
\ccIndexSubsubitemBegin{source files}{headings for}{LGPL version}

Here follows what this gives for a file under the LGPL :

\begin{verbatim}
// Copyright (c) 2000,2001,2002,2003  Utrecht University (The Netherlands),
// ETH Zurich (Switzerland),
// INRIA Sophia-Antipolis (France),
// Max-Planck-Institute Saarbruecken (Germany),
// and Tel-Aviv University (Israel).  All rights reserved.
//
// Licensees holding a valid commercial license may use this file in
// accordance with the commercial license agreement provided with the software.
//
// This file is part of CGAL (www.cgal.org); you can redistribute it and/or
// modify it under the terms of the GNU Lesser General Public License as
// published by the Free Software Foundation; either version 3 of the License,
// or (at your option) any later version.
//
// This file is provided AS IS with NO WARRANTY OF ANY KIND, INCLUDING THE
// WARRANTY OF DESIGN, MERCHANTABILITY AND FITNESS FOR A PARTICULAR PURPOSE.
//
// $URL$
// $Id$
// 
//
// Author(s)     : Herve Bronnimann, Sylvain Pion
\end{verbatim}
\ccIndexSubsubitemEnd{source files}{headings for}{LGPL version}
\ccIndexSubitemEnd{source files}{headings for}


\section{Requirements and recommendations\label{sec:code_format_doc_req_and_rec}}

\noindent
Requirements:
\begin{itemize}
   \item Follow the naming schemes outlined above.
   \item Provide typedefs of template arguments as necessary to make the
         template parameters accessible elsewhere.
   \item Label member function and parameters with \ccc{const} where 
         appropriate
   \item Use \CC-style type casts.
   \item Protect header files from multiple inclusions.
   \item Obey the code format rules outlined above.
   \item Provide a header for each submitted file in the proper format.
\end{itemize}


% ===== EOF ===================================================================
