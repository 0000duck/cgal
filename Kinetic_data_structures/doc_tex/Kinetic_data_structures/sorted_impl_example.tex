%%%%%%%%%%%%%%%%%%%%%%%%%%%%%%%%%%%%%%%%%%%%%%%%%%%%%%%%%%%%%%%%%%%%
\subsection{Implementing the Sort kinetic data structure}
\label{sec:sort_kds_overview}
%%%%%%%%%%%%%%%%%%%%%%%%%%%%%%%%%%%%%%%%%%%%%%%%%%%%%%%%%%%%%%%%%%%%
We will now explain how to implement the kinetic data structure used in the previous example. 

%%%%%%%%%%%%%%%%%%%%%%%%%%%%%%%%%%%%%%%%%%%%%%%%%%%%%%%%%%%%%%%%%%%%
\subsubsection{Overview}
%%%%%%%%%%%%%%%%%%%%%%%%%%%%%%%%%%%%%%%%%%%%%%%%%%%%%%%%%%%%%%%%%%%%


Figure~\ref{fig:example_program}, depicts a complete kinetic data
structure, \ccc{Kinetic::Sort<Traits>}, implemented using our
framework. The data structure being maintained is very simple: a list
of the geometric objects in the simulation sorted by their $x$
coordinate. However, it touches upon the most important parts of the
framework. For a simple kinetic data structure like this, much of the
code is shared with other kinetic data structures. 

Like most kinetic data structures the maintained data has
two parts (in this case stored separately)
\begin{itemize}
\item the combinatorial structure being maintained, in this case in the
list \ccc{objects_} declared at the end of the class.
\item the mapping between connections in the combinatorial structure and 
pending events. In this case the connections are pairs of adjacent
objects in the sorted list. The mapping is stored in the map
\ccc{certificates_} using the key of the first point in the pair.
When a pair is destroyed (because the objects are no longer adjacent),
the event key stored in the mapping is used to deschedule the
corresponding event.
\end{itemize}

As is characteristic of many kinetic data structures, \ccc{Sort}
defines a class \ccc{Event}, which stores the information for a single
event, and has six main methods. The methods are:
\begin{itemize}
\item \ccc{insert}: a point has been added to the simulation and
  must be added to the data structure.
\item \ccc{set}: an point has changed its trajectory and the
  two certificates involving it must be updated
\item \ccc{erase}: an point has been removed from the
  simulation. It must be removed from the data structure, the events
  involving it descheduled and a new event created.
\item \ccc{swap}: an event has occurred and two objects are about to
  become out of order in the list and so must be exchanged.
\item \ccc{rebuild_certificate}: for some reason, a predicate
  corresponding to a particular piece of the combinatorial structure
  is no longer valid or the action that was going to be taken in
  response to its failure is no longer correct. Update the predicate
  appropriately. This method is only called from within the kinetic
  data structure.
\item \ccc{audit}: check that the combinatorial structure is
  valid at the given time.
\end{itemize}

The first three methods are called in response to notifications from
the \ccc{ActiveObjectsTable}. The fourth method is called by
\ccc{Event} objects. The last method is called in response to a
notification from the \ccc{Simulator}.

%%%%%%%%%%%%%%%%%%%%%%%%%%%%%%%%%%%%%%%%%%%%%%%%%%%%%%%%%%%%%%%%%%%%
\subsubsection{\ccc{Sort} in detail.}
\label{sec:sorted_impl_example}
%%%%%%%%%%%%%%%%%%%%%%%%%%%%%%%%%%%%%%%%%%%%%%%%%%%%%%%%%%%%%%%%%%%%

On initialization the \ccc{Kinetic::Sort<Traits>} registers for
notifications with a \ccc{ActiveObjectsTable} and a \ccc{Simulator}. It
receives notifications through two proxy objects,
\ccc{mot_listener_} and \ccc{sim_listener_}, which implement the
notification interface and call functions on the kinetic data
structure when appropriate. We provide standard proxy objects,
\ccc{Active_objects_listener_helper} and
\ccc{Simulator_kds_listener_}, which are used, but implementers
of kinetic data structures are free to implement their own versions of
these simple classes. The former proxy calls the \ccc{insert},
\ccc{erase} and
\ccc{set} methods of the kinetic data structure when
appropriate. The latter proxy calls the \ccc{audit} method when there
is a rational time value at which verification can be performed. See
Section~\ref{sec:simulator} for an explanation of when this occurs.

The proxy objects store the (reference counted) pointers to the
\ccc{ActiveObjectsTable} and \ccc{Simulator} objects for later use. The
\ccc{Simulator} pointer is used by the kinetic data structure to
request the current time and schedule and deschedule events. The
\ccc{ActiveObjectsTable} pointer is used to access the actual
coordinates of the kinetic objects. The use of reference counted
pointers guarantees that there will be no dangling pointers in the
code.

 Once initialization is completed, the behavior of the kinetic data
structure is entirely event driven.

The first thing that will occur is the addition of a point to the
\ccc{ActiveObjectsTable} which results in the \ccc{insert} method being
called. This method is passed a \ccc{Key} which uniquely identifies a
point in the \ccc{ActiveObjectsTable}. The
\ccc{Kinetic::Sort<Traits>} makes use of the \ccc{InstantaneousKernel} to properly
handle the insertion by using a \ccc{InstantenousKernel}-provided
functor which compares the $x$ coordinates of two objects at the
current instant of time. This functor is then passed to the
STL library function
\ccc{upper_bound} which returns the location in the sorted list of the
point before which the new point should be inserted to maintain a
sorted order. The point is inserted and the new pairs created (the new
point and the objects before and after it) must have certificates
created for them and events scheduled. The \ccc{rebuild_certificate}
function is called to handle updating the certificates. The
\ccc{rebuild_certificate} function will also deschedule any previous
certificates when necessary.

Note that this implementation assumes that \ccc{insert} is only
called at instants when there is a rational time topologically
equivalent to the current root. The \ccc{Time current_time_nt()} call made
to the \ccc{Simulator} will fail otherwise--i.e.\ when two events
occur simultaneously, a degeneracy. The easiest way to handle this is
to postpone insertion until a non-degenerate rational time exists or
to only insert objects at rational times. We ignore that issue in the
example since handling it is somewhat situation dependent.

The \ccc{rebuild_certificate} function updates the certificate
associated with a passed pair to make sure it is correct. It first
checks if there is a previous event corresponding to the pair which
needs to be descheduled, and if so requests that the \ccc{Simulator}
deschedule it. Then a \ccc{RootStack} is requested from the
\ccc{Simulator}, passing in the certificate function created by the
\ccc{KineticKernel}'s \ccc{Less_x_2} predicate applied to the pair of objects in
question. Then an \ccc{Event} is created to exchange the two objects
and scheduled in the \ccc{Simulator} at for that time. Note that the
certificate function may not have any roots after the current time. In
that case, the root stack will return
\ccc{std::numeric_limits<Root>::infinity()}. The \ccc{Simulator}
detects this and will not schedule the associated event, but will
instead return a placeholder \ccc{Event_key}.

The \ccc{Event} is in charge of alerting the
\ccc{Kinetic::Sort<Traits>} that it needs to be updated when a
particular certificate failure occurs. Typically event classes are
very simple, effectively just storing a pointer to the kinetic data
structure and an identifier for the combinatorial piece which needs to
be updated in addition to the time when the update must occur. This
certificate also stores a copy of the \ccc{RootStack} for reasons
which will be discussed in the next paragraph. In order to be handled
by the \ccc{Simulator}, the \ccc{Event} class must have the following
method
\begin{itemize}
\item \ccc{process} which is called with the value true when the event occurs.
\end{itemize}
\noindent In addition, in order to ease debugging,
it must be able to be output to an \ccc{std::ostream}.

The \ccc{swap} method is the update method in the \ccc{Sort} kinetic
data structure. When a pair of objects is swapped, three old pairs of
points are destroyed and replaced by three new pairs. Calls to
\ccc{rebuild_certificate} handle the updating of the certificates
between a point of the swapped pair and its outside neighbors in the
list. The pair that has just been exchanged should be dealt with
differently for optimal efficiency. The predicate function
corresponding to the new ordering of the swapped pair is the negation
of that for the old ordering (i.e.\ $x_k(t)-x_j(t)$ as opposed to
$x_j(t)-x_k(t)$), and so has the same roots. As a result, the old
\ccc{RootStack} can be used to find the next root, saving a great
deal of time. In addition, the event which is currently being
processed does not need to be descheduled as it is deleted by the
\ccc{Simulator}. Notice that the update method does not make any
reference to time. This is necessary to properly support degeneracies,
since few or no exact calculations can be made without a topologically
equivalent rational time, which might not exist. The \ccc{insert}
method is mostly used for initialization and so can be assumed to
occur at a non-degenerate time, the same assumption is less easily
made about an event.

As described in Section~\ref{sec:simulator}, the \ccc{Simulator} can
periodically send out notifications that there is a rational time at
which all the kinetic data structures are non-degenerate and can be
easily verified. The \ccc{audit} method is called in response to such
a notification. Validation consists of using the
\ccc{InstantaneousKernel} to check that each pair in the list is
ordered correctly.

The remaining two methods, \ccc{set} and \ccc{erase} are only
necessary if the kinetic data structure wishes to support dynamic
trajectory changes and removals. These methods are called by the
\ccc{mot_listener_} helper when appropriate.

That is all it takes to implement a kinetic data structure which is
exact, supports dynamic insertions and deletions of objects, allows
points to change motions on the fly, and allows a variety of solvers
and motion types to be used without modifications.

\subsubsection{Code}
\label{fig:example_program}
An example kinetic data structure using the framework.  
\ccRefIdfierPage{Kinetic::Sort<Traits>}
