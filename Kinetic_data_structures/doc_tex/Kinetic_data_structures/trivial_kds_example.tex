
\subsection{Implementing a Trivial Kinetic Data Structure}
\label{sec:trivial_kds_example}

\subsection{Using the Pieces of the Package}

In this section we will explain in detail how a typical kinetic data
structure uses the various pieces of the framework. The kinetic data
structure of reference will, once again, be our implementation of
kinetic sorting.  A schematic of its relationship to the various
components is shown in the UML diagram in Figure~\ref{uml_usage}. In
this subsection we abuse, for reasons of simplicity of presentation,
the concept/model semantics: when we refer to concepts we actually
refer to an instance of a model of them.

As with most kinetic data structures, \ccc{Kinetic::Sort<Traits, Visitor>} maintains
some sort of combinatorial structure (in this case a sorted doubly linked list),
each element of which has a corresponding certificate in the event queue
maintained by the simulator. In the case of sorting, there is one
certificate maintained for each ``edge'' between two consecutive
elements in the list.

On creation, the data structure is passed a copy of the
Kinetic::SimulationTraits for this simulation, which it saves for
future use. It gets a handle to to the Kinetic::ActiveObjectsTable by
calling the
\ccc{Kinetic::SimulationTraits::active\_points\_table\_1\_handle()} method
and registers a proxy with the table in order to receive notifications
of changes to the point set.  The Kinetic::SimulationTraits method
returns a handle to, rather than a copy of, the
Kinetic::ActiveObjectsTable, since the table must be shared between
all the kinetic data structures using these points.  The handles are
reference counted pointers, thus saving the user from worrying about
cleaning things up properly.

When new points are added to the model of the
Kinetic::ActiveObjectsTable, the table calls the
\ccc{new_notification()} method on the proxy of the kinetic data
structure, which in turn calls the
\ccc{insert(Point_key)} method of the kinetic data structure. The
\ccc{Point_key} here is the key which uniquely identifies the
newly inserted point in the table. The data structure then requests an
instance of a model of the Kinetic::InstantaneousKernel from the
Kinetic::SimulationTraits. It sets the time on the instantaneous
kernel to the time value gotten from the
\ccc{Kinetic::Simulator::current_time_nt()} method. This method returns
a field number type that is between the previous and next event, as
discussed in the introduction. An instance of the
\ccc{Kinetic::InstantaneousKernel::Less_x_1} predicate and the STL
function \ccc{std::upper_bound()} are then used to insert the new
point in the sorted list. For each inserted object, the kinetic data
structure removes the no
longer relevant certificate from the event queue by calling the
\ccc{Kinetic::Simulator::delete_event(Key)} function and creates two new
certificates using a \ccc{Kinetic::Kernel::Is_less_x_1}
certificate functor. The new certificates are inserted in the event
queue by calling the \ccc{Kinetic::Simulator::new_event(Time, Event)}
method where Kinetic::Event is a proxy object which instructs the sort
kinetic data structure to swap two points when its \ccc{process()}
method is called.

Now that the kinetic data structure has been initialized, the
simulator is instructed to process all events. Each time an event
occurs, the simulator calls the \ccc{process()} method on the
corresponding proxy object. The proxy, in turn, tells the sort kinetic
data structure to swap the two points whose order has changed.

The Kinetic::Simulator can periodically instruct the kinetic data
structures to audit themselves.  As is explained in
Section~\ref{simulator}, a proxy object maps the notification on to an
\ccc{audit()} function call in the kinetic data structure. To audit
itself the kinetic data structure builds a list of all the current
points and uses \ccc{std::sort} to sort this list using a
comparison function gotten from the Kinetic::InstantaneousKernel.
This sorted list is compared to the maintained one to verify
correctness. This auditing could also have been done by evaluating the
Kinetic::InstantaneousKernel predicate for each sorted pair.  Since
auditing a kinetic data structure typically requires at least linear
time in the size of the combinatorial structure, the auditing
procedure in between events is deactivated by default.  The user can
however easily switch it on by defining the
\ccc{CGAL_CHECK_EXACTNESS} and \ccc{CGAL_CHECK_EXPENSIVE} CGAL
macros.

This general structure of the interaction between the kinetic data
structure and the framework is shared by all of the provided kinetic
data structures and has proved itself to go quite far.

To show how to implement such things, instead of presenting a full
kinetic data structure, we present a trivial one which maintains one
event in the queue which has a list of the coordinates of all the
known moving objects.  Two classes are defined, the
\ccc{Trivial_event}, and the
\ccc{Trivial_kds}. The event classes must be declared outside of the
kinetic data structure so that the \ccc{operator<<} can be defined for
them. 

The kinetic data structure maintains one event containing a list of
the trajectories of all objects in the simulation. This
event must updated whenever any objects change, in addition, it is
always created to fail one time unit in the future, so it must be
recreated when it fails. As a result, the kinetic data structure has
the main parts of a real one--it responds to changes in trajectories
of the objects and certificate failures (when the event expires).

 The public methods can be grouped into three sets which are shared
 with almost all other kinetic data structures:
\begin{itemize}
\item \ccc{has_certificates} and \ccc{set_has_certificates} which
  checks/sets whether the kinetic data structure is currently
  maintaining certificates.
\item \ccc{insert}, \ccc{set}, \ccc{erase} which are called by the
  \ccc{Kinetic::Active_objects_listener_helper} in response to the
  addition, modification, or deletion of an object to, in or from the
  simulation.
\item \ccc{audit} which is called periodically by the
  \ccc{Kinetic::Simulator_kds_listener} when kinetic data structures can
  easily audit themselves.
\end{itemize} 

In addition, it has one method which is called when a certificate
fails. The name/existence of such methods depend on the nature of the
kinetic data structure in question.

Like many kinetic data structures, it takes a Kinetic::SimulationTraits
as a template argument. This traits class defines the types needed for
the simulation and is responsible for instantiating them.

 \label{fig:trivial_usage_program}
\ccIncludeExampleCode{Kinetic_data_structures/trivial_kds.C}
%../../examples/Kinetic_data_structures/ % LocalWords: CGAL
