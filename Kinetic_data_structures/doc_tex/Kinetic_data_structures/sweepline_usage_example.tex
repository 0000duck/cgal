

\subsection{A Sweepline Algorithm}
\label{sec:sweepline_example}

Here we present a simple example that uses the kinetic data structures
framework to implement a sweepline to compute the arrangement of
x-monotone algebraic curves in the plane. The example builds on top of
the \ccc{Kinetic::Sort<Traits, Visitor>} kinetic data structure, using a visitor
to keep track of changes to the sorted order and newly inserted
points. To see an example using this kinetic data structure read the
example at examples/Kinetic\_data\_structures/sweepline.C.

First we define the visitor class. An object of this type is passed to
the \ccc{Kinetic::Sort} data structure and turns events into calls on
the arrangement structure. This class has to be handled externally
since the arrangement will inherit from the sorting structure.

\begin{ccExampleCode}
template <class Arrangement>
struct Arrangement_visitor: public CGAL::Kinetic::Sort_visitor_base
{
  Arrangement_visitor(Arrangement *a):p_(a){}
  template <class Vertex_handle>
  void remove_vertex(Vertex_handle a) {
    p_->erase(a);
  }
  template <class Vertex_handle>
  void create_vertex(Vertex_handle a) {
    p_->insert(a);
  }
  template <class Vertex_handle>
  void after_swap(Vertex_handle a, Vertex_handle b) {
    p_->swap(a, b);
  }
  Arrangement *p_;
};

\end{ccExampleCode}

Now we define the actual kinetic data structure. 

\begin{ccExampleCode}

template <class TraitsT> 
class Planar_arrangement: 
  public CGAL::Kinetic::Sort<TraitsT, 
			     Arrangement_visitor<Planar_arrangement<TraitsT> > > {
  typedef TraitsT Traits;
  typedef Planar_arrangement<TraitsT> This;
  typedef typename CGAL::Kinetic::Sort<TraitsT,
				       Arrangement_visitor<This> > Sort;
  typedef Arrangement_visitor<This> Visitor;
  typedef typename Traits::Active_objects_table::Key Key;

public:
  typedef CGAL::Exact_predicates_inexact_constructions_kernel::Point_2 Approximate_point;
  typedef std::pair<int,int> Edge;
  typedef typename Sort::Vertex_handle Vertex_handle; 

  // Register this KDS with the MovingObjectTable and the Simulator
  Planar_arrangement(Traits tr): Sort(tr, Visitor(this)) {}

  Approximate_point vertex(int i) const
  {
    return approx_coords_[i];
  }

  size_t vertices_size() const
  {
    return approx_coords_.size();
  }

  typedef std::vector<Edge >::const_iterator Edges_iterator;
  Edges_iterator edges_begin() const
  {
    return edges_.begin();
  }
  Edges_iterator edges_end() const
  {
    return edges_.end();
  }

  void insert(Vertex_handle k) {
    last_points_[*k]=new_point(*k);
  }

  void swap(Vertex_handle a, Vertex_handle b) {
    int swap_point= new_point(*a);
    edges_.push_back(Edge(swap_point, last_points_[*a]));
    edges_.push_back(Edge(swap_point, last_points_[*b]));
    last_points_[*a]= swap_point;
    last_points_[*b]= swap_point;
  }

  void erase(Vertex_handle a) {
    edges_.push_back(Edge(last_points_[*a], new_point(*a)));
  }

  int new_point(typename Traits::Active_objects_table::Key k) {
    double tv= CGAL::to_double(Sort::traits().simulator_pointer()->current_time());
    double dv= CGAL::to_double(Sort::traits().active_objects_table_pointer()->at(k).x()(tv));
    approx_coords_.push_back(Approximate_point(tv, dv));
    return approx_coords_.size()-1;
  }

  std::vector<Approximate_point > approx_coords_;
  std::map<Key, int> last_points_;
  std::vector<Edge> edges_;

};
\end{ccExampleCode}

%%% Local Variables: 
%%% mode: latex
%%% TeX-master: t
%%% End: 
