\subsection{Adding a new certificate type}
\label{sec:add_cert}

The following example shows how to add a new type of certificate to a simulation. 

First we code the actual certificate function generator. It must take
some sort (or sorts) of kinetic primitives, compute some function from
their coordinates.

\begin{ccExampleCode}
template <class KineticKernel>
struct Positive_x_f_2 {
  typedef typename KineticKernel::Certificate_function result_type;
  typedef typename KineticKernel::Point_2 argument_type;
  result_type operator()(const argument_type &p){
    return result_type(p.x()- result_type(0));
  }
};
\end{ccExampleCode}

Then we define a kinetic kernel which includes this predicate. To do
this we wrap the function generator generator in a
\ccc{Kinetic::Certificate_generator<Kernel, Generator>}.
This wrapper uses the generator to create the certificate function and
then the \ccc{Kinetic::FunctionKernel} to solve the certificate
function. The result is wrapped in a \ccc{Kinetic::Certificate}
object.

\begin{ccExampleCode}
template <class FunctionKernel> 
class My_kinetic_kernel:
  public CGAL::Kinetic::Cartesian_kinetic_kernel<FunctionKernel> {
  typedef CGAL::Kinetic::Cartesian_kinetic_kernel<FunctionKernel> P;
  typedef My_kinetic_kernel<FunctionKernel> This;
public:
  typedef CGAL::Kinetic::internal::Certificate_generator<This, Positive_x_f_2<This> > Positive_x_2;
  Positive_x_2 positive_x_2_object() const
  {
    return Positive_x_2(P::function_kernel_object());
  }
};
\end{ccExampleCode}

Now we have the unfortunately rather messy part of assembling a new
\ccc{Kinetic::SimulationTraits} model. This is done in two steps for convenience.

\begin{ccExampleCode}
struct My_st_types: public CGAL::Kinetic::Suggested_exact_simulation_traits_types {
  typedef CGAL::Kinetic::Suggested_exact_simulation_traits_types P;
  typedef My_kinetic_kernel<P::Function_kernel>::Point_2 Active_object;
  typedef CGAL::Kinetic::Active_objects_vector<Active_object> Active_objects_table;
  typedef CGAL::Kinetic::Cartesian_instantaneous_kernel< Active_objects_table,
							 Static_kernel> Instantaneous_kernel;
};

struct My_simulation_traits:
  public  CGAL::Kinetic::Simulation_traits<My_st_types::Static_kernel,
			    My_st_types::Kinetic_kernel,
			    My_st_types::Simulator>
{
  typedef  CGAL::Kinetic::Simulation_traits<My_st_types::Static_kernel,
					    My_st_types::Kinetic_kernel,
					    My_st_types::Simulator> P;
  My_simulation_traits(const P::Time &lb= P::Time(0),
		       const P::Time &ub=std::numeric_limits<P::Time>::infinity()): 
    P(lb,ub), 
    ap_(new Active_points_2_table()) {}

  typedef My_st_types::Active_objects_table Active_points_2_table;
  Active_points_2_table* active_points_2_table_handle() {
    return ap_.get();
  }
  const Active_points_2_table* active_points_2_table_handle() const {
    return ap_.get();
  }

  typedef My_st_types::Instantaneous_kernel Instantaneous_kernel;
  Instantaneous_kernel instantaneous_kernel_object() const
  {
    return Instantaneous_kernel(ap_, static_kernel_object());
  }
protected:
  Active_points_2_table::Handle ap_;
};
\end{ccExampleCode}

Now the simulation traits can be used by a kinetic data structure.
