%%%%%%%%%%%%%%%%%%%%%%%%%%%%%%%%%%%%%%%%%%%%%%%%%%%%%%%%%%%%%%%%%%%%
\section{Kinetic Data Structures and Sweep Algorithms\label{sec:kds_intro}}

%%%%%%%%%%%%%%%%%%%%%%%%%%%%%%%%%%%%%%%%%%%%%%%%%%%%%%%%%%%%%%%%%%%%


\subsection{Kinetic Data Structures}
Kinetic data structures were first introduced in by Basch et.\ al.\ in
1997~\cite{cgal:bgh-dsmd-97}.  The idea stems from the observation that
most, if not all, computational geometry structures are built using
{\em predicates} --- functions on quantities defining the geometric
input (e.g.\ point coordinates), which return a discrete set of
values. Many predicates reduce to determining the sign of a polynomial
on the defining parameters of the primitive objects. For example, to
test whether a point lies above or below a plane we compute the dot
product of the point with the normal of the plane and subtract the
plane's offset along the normal. If the result is positive, the point
is above the plane, zero on the plane, negative below. The validity of
many combinatorial structures built on top of geometric primitives can
be verified by checking a finite number of predicates of the geometric
primitives.  These predicates, which collectively certify the
correctness of the structure, are called {\em certificates}.  For a
Delaunay triangulation in three dimensions, for example, the
certificates are one \ccc{InCircle} test per facet of the
triangulation, plus a point plane orientation test for each facet or
edge of the convex hull.

The kinetic data structures approach is built on top of this view of
computational geometry. Let the geometric primitives move by replacing
each of their defining quantities with a function of time (generally a
polynomial). As time advances, the primitives trace out paths in
space called {\em trajectories}. The values of the polynomial
functions of the defining quantities used to evaluate the predicates now
also become functions of time. We call these functions
{\em certificate functions}. Typically, a geometric structure is valid when
all predicates have a specific non-zero sign. In the kinetic setting,
as long as the certificate functions maintain the correct sign as time varies,
the corresponding predicates do not change values,
and the original data structure remains correct. However, if
one of the certificate functions changes sign, the original structure
must be updated, as well as the set of certificate functions that
verify it.  We call such occurrences {\em events}.

Maintaining a kinetic data structure is then a matter of determining
which certificate function changes sign next, i.e.\ determining which
certificate function has the first real root that is greater than the
current time, and then updating the structure and the set of
certificate functions. In addition, the trajectories of primitives are
allowed to change at any time, although $C^0$-continuity of the
trajectories must be maintained.  When a trajectory update occurs for
a geometric primitive, all certificates involving that primitive must
be updated.  We call the collection of kinetic data structures,
primitives, event queue and other support structures a
{\em simulation}.

Sweepline algorithms for computing arrangements in $d$ dimensions
easily map on to kinetic data structures by taking one of the
coordinates of the ambient space as the time variable. The kinetic
data structure then maintains the arrangement of a set of objects
defined by the intersection of a hyperplane of dimension $d-1$ with
the objects whose arrangement is being computed. 


Time is one of the central concepts in a kinetic simulation.  Just as
static geometric data structures divide the continuous space of all
possible inputs (as defined by sets of coordinates) into a discrete
set of combinatorial structures, kinetic data structures divide the
continuous time domain into a set of disjoint intervals. In each
interval the combinatorial structure does not change, so, in terms of
the combinatorial structure, all times in the interval are equivalent.
We capitalize on this equivalence in the framework in order to
simplify computations. If the primitives move on polynomial
trajectories and the certificates are polynomials in the coordinates,
then events occur at real roots of polynomials of time. Real numbers,
which define the endpoints of the interval, are more expensive to
compute with than rational numbers, so performing computations at a
rational number inside the interval is preferable whenever
possible. See Section~\ref{sec:kds_instantaneous_kernel} for an example of
where this equivalence is exploited.

%%%%%%%%%%%%%%%%%%%%%%%%%%%%%%%%%%%%%%%%%%%%%%%%%%%%%%%%%%%%%%%%%%%%
\section{Terms Used\label{sec:kds_terms}}

%%%%%%%%%%%%%%%%%%%%%%%%%%%%%%%%%%%%%%%%%%%%%%%%%%%%%%%%%%%%%%%%%%%%

\begin{description}
\item[primitive] The basic geometric types--i.e.\ the points of a
  triangulation. A primitive has a set of {\em coordinates}.
\item[combinatorial structure] A structure built on top of the
  primitives. The structure does not depend directly on the
  coordinates of the primitives, only on relationships between them.
\item[trajectory] The path traced out by a primitive as time passes.
  In other words how the coordinates of a primitive change with time.
\item[snapshot] The position of all the primitives at a particular
  moment in time.
\item[static] Having to do with geometric data structures on
  non-moving primitives.
\item[predicate] A function which takes the coordinates of several
  primitives from a snapshot as input and produces one of a discrete
  set of outputs.
\item[certificate] One of a set of predicates which, when all having
  the correct values, ensure that the combinatorial structure is
  correct.
\item[certificate function] A function of time which is positive when
  the corresponding certificate has the correct value. When the
  certificate function changes sign, the combinatorial structure needs
  to be updated.
\item[event] When a certificate function changes sign and the
  combinatorial structure needs to be updated.
\end{description}
