\subsection{Two Dimensional Delaunay}
\label{sec:sort_example}

Using a kinetic data structure can be as simple as the following:
\label{fig:sort_program}
\ccIncludeExampleCode{Kinetic_data_structures/sort.C}

In the example, first the Kinetic::SimulationTraits object is chosen (in this
case one that supports exact computations). Then the kinetic data
structure is defined, using the chosen traits object and a visitor
class which logs changes to the sorted list.  Next, instances of the
two are created and a set of points is read from a file. Then, the
simulator is instructed to process all the events until the end of the
simulation.  Finally, a record of what happened is printed to the
terminal.

Several important things happen behind the scenes in this example.
First, the Kinetic::ActiveObjectsTable which holds the moving points
notifies the kinetic data structure that new points have been added to
the simulation. Second, the \ccc{Kinetic::Sort<Traits,Visitor>} kinetic data structure
registers its events with the Kinetic::Simulator by providing a time
and a proxy object. When a particular event occurs, the
Kinetic::Simulator calls a function on the proxy object which in turn
updates the kinetic data structure.

The example illustrates how to monitor the supplied data structures as
they evolve by using a Kinetic::SortVisitor object---a small class whose
methods are called whenever the kinetic data structure changes. Hooks
for such visitor concepts are provided for all of the shipped kinetic
data structures. In the case of kinetic sorting, the visitor's
methods are called every time a new point is inserted in the sorted
list, when one is removed, or when two points are swapped in the
sorted order. 


The visitor concept is quite powerful, allowing us, for example, to
implement a data structure for computing and storing two-dimensional
arrangements of $x$-monotone curves on top of the
\ccc{Kinetic::Sort<Traits, Visitor>} data structure using about 60
lines of code. This sweepline code is presented in
Section~\ref{sec:sweepline_example}.
