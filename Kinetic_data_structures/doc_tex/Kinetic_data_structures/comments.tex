\section{Comments}


The package is rather complex,


\subsection{User Manual}


define_exact_kds and define_numeric_kds are more than strange.
\textit{I agree. On the other hand, due to the complexity of things, I
think something along those lines is needed (otherwise there are 9
#includes and 10 typedefs which need to be added to every
file). Perhaps a header which defines a "Exact_simulator" or
"Numeric_simulation" class which simply defines the types as defined
in the headers.}






Section 1.3.1

The two ways to use KDS  "as fast as possible", and "Alternatively"
are hard to grasp. You might have two examples  insteas of the #ifdefs
If I understand it right they are very different in nature. The one
with GUI has a notion of a a regular beat of time, where snapshots
are taken and the snapshot gets visualized. Events can even happen
between the beats. The other mode of usage is only interested in
events.
\textit{Separate it into two examples, one explicitly showing the GUI}.


Also the example should do more than start the event loop  or
set current time to infinity.
\textit{Maybe}





The #include section could be shorter.
There is no need to include Delaunay_triangulation_{vertex,face}_base_2.h
\textit{I should remove some of the includes.}

You could regroup
#include <CGAL/Kinetic/Moving_object_table.h>
#include <CGAL/Kinetic/Moving_object_inserter.h>
in <CGAL/Kinetic/basic.h> and this one could
be included by Cartesain_kinetic_kernel.h
\textit{They aren't really conceptually related and can be used without one another, so I would rather not. Perhaps they can go into the ``Helper headers''. I could also use default arguments for \ccc{Kinetic::Delaunay}_triangulation and skip many of the headers. I had wanted to avoid that, but it could make it less intimidating.}


Are "Simulator", "Time", "FunctionKernel" defined in the
magic header file? Obscure.
And what about Cartesian_instantaneous_kernel? If it comes
from CGAL/Kinetic/Cartesian_kinetic_kernel.h, then this
header file might be called CGAL/Kinetic/Cartesian.h
with the two types in there.
\textit{It doesn't come from there, but maybe it could. Strictly speaking it is more of an adaptor, but I could simplify things. }


The template arguments of \ccc{Kinetic::Delaunay}_triangulation_2
do not correspond to the reference manual.
\textit{Indeed. This should be fixed.}

Would it be possible to have default template arguments which
make the example shorter?
\textit{I could, I was trying to avoid default arguments in the example in order to show everything, but I could change that.}

Concerning the trajectories, would it be possible to have CGAL::Vectors
where the direction of the vector is the direction and its length
the speed? This would be intuitive.
\textit{Hmmm, I could write constructors from a point and a vector.}

Also get rid of the random. It saves you one #include ==:>-
Instead take four points
p0(0,1) with Vector(0,1)
p1(1,0) with Vector(0,0) (does not move)
p2(1,3) with Vector(0,0)
p3(2,2) with Vector(-1,0)
\textit{Sure}

Then you can even make screenshots of all events.

Also I would start with a static setting. Why inserting them
one after the other. Set the stage, switch on the clock
to keep it simple.


p8: "five new events" why five?

p9: Simulator<..,Exact,..>  but the reference manual has no Exact

p11, paragraph3: MovingObjectTable -> Moving_object_table bwecause it is a model.


Reference Management  Don't we have that in the support library Handle/Rep

p12: "or other listening class" -> "or other listening objects"


p13: POLYNOMIAL -> Polynomial

p14: KDSMoving_object_table -> \ccc{Kinetic::Moving}_object_table
p14: passingh -> passing, possition -> position, being processs in order -> being processed in order

More Typos
simplist -> simplest
classeds -> classes
reference through -> referenced through
each of these is created -> each of these are created


I skipped the rest of the user manual for the next round of review ==:>-


__________________________
Reference Manual

The same example all over the place ??

_____
Listener<Interface>

you make define -> you ??? define
"inherits from the Listener interface??? Do you mean inherits from the template argument
\ccc{Interface> ?

listner -> listener
poxy -> proxy
An notifier -> A notifier
ref counted -> rfeference counted


_____
Moving_object_table<Object>

Object can easily be confused with CGAL::Obejct
true -> \ccc{true}
Object mot[Key key]   Do you return a copy or a reference

mot.editing() -> mot.is_editing()

I suggest to get rid of "_object" in the member function names.
STL container do not have that either.

_____
Multi_listener

Is this class obsolete? The fact that it is not in the right namespace
and not in the right directory seems to indicate that.


_____
Pointer_queue

Why is class Time in a comment?


_____
RootEnumerator

It looks very much like an iterator

\textit{But it is not quite an iterator. It is basically a container
or stack. We decided not to go for the iterator model since it is
comparatively heavy. In addition, allowing iterators into the
RootEnumerator disallows a number of optimizations. We have
RootContainers if you want iterators.

It is now called RootStack and has appropraitely named methods.}
