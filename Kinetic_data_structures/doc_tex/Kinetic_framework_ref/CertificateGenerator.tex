% +------------------------------------------------------------------------+
% | Reference manual page: Event.tex
% +------------------------------------------------------------------------+
% | 20.03.2005   Author
% | Package: Kinetic_data_structures
% | 
\RCSdef{\RCSEventRev}{$Id$}
\RCSdefDate{\RCSEventDate}{$Date$}
% |
%%RefPage: end of header, begin of main body
% +------------------------------------------------------------------------+

\ccDefGlobalScope{CGAL::}
\begin{ccRefConcept}[Kinetic::]{CertificateGenerator}

%% \ccHtmlCrossLink{}     %% add further rules for cross referencing links
%% \ccHtmlIndexC[concept]{} %% add further index entries

\ccDefinition
  
This functor allows you to create certificate objects of some type.
The models of this ``concept'' take some set of arguments which depend
on the certificate being computed (for example three points for a two
dimensional orientation) followed by either one or two instances of
the \ccc{Kinetic::Simulator::Time} concept. The functions either
return a \ccc{Certificate} or the corresponding value at the current
time (if only a time value rather than an interval is passed).

\ccCreationVariable{a}  %% choose variable name

%\ccTypes

%\ccNestedType{Certificate}{}

\ccOperations

\ccMethod{Certificate operator(Args, Time begin, Time end);}{Return a \ccc{Certifate} object for the corresponding certificate.}

\ccMethod{CGAL::Sign operator(Args, Time t);}{Compute the sign of the
  function at $\lim_{\delta\arrow 0} t+\delta$. This can be used to evaluate predicates at the current moment.}


\ccHasModels

All over the place.

\ccSeeAlso

\ccc{Kinetic::KineticKernel}

\ccExample

Here you see how to use both functions on an orientation predicate.

\begin{ccExampleCode}
KineticKernel::Point_2 a,b,c;
Simulator::Handle sh;
KineticKernek kk;

KineticKernel::Orientation_2 o2= kk.orientation_2_object();

KineticKernel::Certificate c= o2(a,b,c, sh->current_time(), sh->end_time());
if (c.will_fail()) {
  std::cout << "Certificate will fail" << std::endl;
}
// Compute the sign immediately following the current time
CGAL::Sign sn= o2(a,c,b, sh->current_time());
CGAL_postcondition(sn==CGAL::NEGATIVE);

\end{ccExampleCode}


\end{ccRefConcept}

% +------------------------------------------------------------------------+
%%RefPage: end of main body, begin of footer
% EOF
% +------------------------------------------------------------------------+


%%% Local Variables: 
%%% mode: latex
%%% TeX-master: t
%%% End: 
