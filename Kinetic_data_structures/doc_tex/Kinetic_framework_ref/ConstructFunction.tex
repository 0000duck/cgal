% +------------------------------------------------------------------------+
% | Reference manual page: Event.tex
% +------------------------------------------------------------------------+
% | 20.03.2005   Author
% | Package: Kinetic_data_structures
% | 
\RCSdef{\RCSEventRev}{$Id: Event.tex 28517 2006-02-14 23:14:42Z drussel $}
\RCSdefDate{\RCSEventDate}{$Date: 2006-02-14 15:14:42 -0800 (Tue, 14 Feb 2006) $}
% |
%%RefPage: end of header, begin of main body
% +------------------------------------------------------------------------+

\ccDefGlobalScope{CGAL::}
\begin{ccRefConcept}[Kinetic::FunctionKernel::]{ConstructFunction}

%% \ccHtmlCrossLink{}     %% add further rules for cross referencing links
%% \ccHtmlIndexC[concept]{} %% add further index entries

\ccDefinition
  
The concept \ccClassName\ is used to construct functions. 

\ccCreationVariable{a}  %% choose variable name

\ccOperations

\ccMethod{Function operator()(NT a, ...);} {This family of methods
  takes a list of coefficients and returns a function. There can be
  any number of coeffients passed as arguments (up to about 25 in the
  current implementations).}

\ccSeeAlso

\ccc{FunctionKernel}

\ccExample

\begin{ccExampleCode}
Function_kernel fk;
Function_kernel::Construct_function cf= fk.construct_function_object();
Function_kernel::Function f= cf(0,1,2,3,4,5);
\end{ccExampleCode}

\end{ccRefConcept}

% +------------------------------------------------------------------------+
%%RefPage: end of main body, begin of footer
% EOF
% +------------------------------------------------------------------------+


%%% Local Variables: 
%%% mode: latex
%%% TeX-master: t
%%% End: 
