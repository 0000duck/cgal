% +------------------------------------------------------------------------+
% | Reference manual page: SimulationTraits.tex
% +------------------------------------------------------------------------+
% | 20.03.2005   Author
% | Package: Kinetic_data_structures
% | 
% |
%%RefPage: end of header, begin of main body
% +------------------------------------------------------------------------+


\begin{ccRefConcept}{Kinetic::SimulationTraits}

%% \ccHtmlCrossLink{}     %% add further rules for cross referencing links
%% \ccHtmlIndexC[concept]{} %% add further index entries

\ccDefinition
  
This concept ties together the parts needed in order to run a kinetic
data structure. We provide several models of this concept:
\begin{itemize}
\item \ccc{Kinetic::Exact_simulation_traits_1}
\item \ccc{Kinetic::Exact_simulation_traits_2}
\item \ccc{Kinetic::Exact_simulation_traits_3}
\item \ccc{Kinetic::Inexact_simulation_traits_1}
\item \ccc{Kinetic::Inexact_simulation_traits_2}
\item \ccc{Kinetic::Inexact_simulation_traits_3}
\item \ccc{Kinetic::Exact_linear_simulation_traits_2}
\item \ccc{Kinetic::Exact_linear_simulation_traits_3}
\item \ccc{Kinetic::Inexact_linear_simulation_traits_2}
\item \ccc{Kinetic::Inexact_linear_simulation_traits_3}
\item \ccc{Kinetic::Regular_triangulation_exact_simulation_traits_3}
\item \ccc{Kinetic::Regular_triangulation_inexact_simulation_traits_3}
\end{itemize}

All support trajectories defined by polynomial coordinates. The
\ccc{Exact} vs \ccc{Inexect} picks whether the roots of the
certificate functions are compared exactly or approximated
numerically. The regular triangulation models have weighted points of
the appropriate dimension as the primitive used in the
\ccc{Kinetic::InstantaneousKernel} and the
\ccc{Kinetic::ActiveObjectsTable}.


\ccTypes

\ccNestedType{NT}{The number type used for representation.}

\ccNestedType{Instantaneous_kernel}{A model of
  \ccc{Kinetic::InstantaneousKernel} which can be used to apply static CGAL
  data structures to snapshots of moving data.}

\ccNestedType{Kinetic_kernel}{A model of \ccc{Kinetic::Kernel}.}

\ccNestedType{Function_kernel}{A model of \ccc{Kinetic::FunctionKernel}.}

\ccNestedType{Active_points_[123]_table}{A model of
  \ccc{Kinetic::ActiveObjectsTable} which holds the relevant kinetic
  primitives.}

\ccNestedType{Simulator}{A model of \ccc{Kinetic::Simulator} which will be
used by all the kinetic data structures.}

%\ccNestedType{Active_objects_table}{A model of \ccc{Kinetic::ActiveObjectsTable} which can be used to store moving points of an appropriate dimension. This is really optional and not needed if no kinetic data structures use points.}

\ccOperations
\ccCreationVariable{st}

\ccMethod{Instantaneous_kernel instantaneous_kernel_object();}{Get a new instantaneous kernel.}

\ccMethod{Kinetic_kernel kinetic_kernel_object();}{Get a new kinetic kernel.}

\ccMethod{Function_kernel function_kernel_object();}{Get a new function kernel.}

\ccMethod{Simulator::Handle simulator_handle();}{Return a pointer to the \ccc{Kinetic::Simulator} which is to be used in the simulation.}

\ccMethod{Active_points_[123]_table::Handle
  active_points_[123]_table_handle();}{Return a pointer to the
  \ccc{Kinetic::ActiveObjectsTable} which is to be used in the
  simulation.}

%\ccMethod{Active_objects_table* active_objects_table_pointer();}{Return a pointer to the table holding points which is to be used in the simulation.}

\ccHasModels

\ccc{Kinetic::Exact_simulation_traits_1},
\ccc{Kinetic::Exact_simulation_traits_2},
\ccc{Kinetic::Exact_simulation_traits_3},
\ccc{Kinetic::Inexact_simulation_traits_1},
\ccc{Kinetic::Inexact_simulation_traits_2},
\ccc{Kinetic::Inexact_simulation_traits_3},
\ccc{Kinetic::Exact_linear_simulation_traits_2},
\ccc{Kinetic::Exact_linear_simulation_traits_3},
\ccc{Kinetic::Inexact_linear_simulation_traits_2},
\ccc{Kinetic::Inexact_linear_simulation_traits_3},
\ccc{Kinetic::Regular_triangulation_exact_simulation_traits_3},
\ccc{Kinetic::Regular_triangulation_inexact_simulation_traits_3}

\end{ccRefConcept}

% +------------------------------------------------------------------------+
%%RefPage: end of main body, begin of footer
% EOF
% +------------------------------------------------------------------------+

