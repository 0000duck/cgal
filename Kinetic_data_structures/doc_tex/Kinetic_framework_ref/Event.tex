% +------------------------------------------------------------------------+
% | Reference manual page: Event.tex
% +------------------------------------------------------------------------+
% | 20.03.2005   Author
% | Package: Kinetic_data_structures
% | 
\RCSdef{\RCSEventRev}{$Id: Event.tex 29921 2006-04-03 12:53:56Z drussel $}
\RCSdefDate{\RCSEventDate}{$Date: 2006-04-03 14:53:56 +0200 (Mon, 03 Apr 2006) $}
% |
%%RefPage: end of header, begin of main body
% +------------------------------------------------------------------------+

\ccDefGlobalScope{CGAL::}
\begin{ccRefConcept}[Kinetic::Simulator::]{Event}

%% \ccHtmlCrossLink{}     %% add further rules for cross referencing links
%% \ccHtmlIndexC[concept]{} %% add further index entries

\ccDefinition
  
The concept \ccClassName\ represents a single event. Models of
\ccClassName\ should be passed to the \ccc{Kinetic::Simulator} when
scheduling events which will in turn pass them to the
\ccc{EventQueue}.

\ccCreationVariable{a}  %% choose variable name

\ccOperations

\ccMethod{void process();}{This method is called when the event
  occurs. This method will only be called once per time this event is
  scheduled and the event will be removed from the queue immediately
  afterwards.}

\ccGlobalFunction{std::ostream& operator<<(std::ostream&, Event);}{Write a text description of the event to a standard stream.}

\ccHasModels

All over the place. 

\ccSeeAlso

\ccc{Kinetic::EventQueue}

\ccExample

All of the kinetic data structures provided have models of
\ccRefName. Here is the code implementing a swap event from the
sorting kinetic data structure.

\begin{ccExampleCode}
template <class Certificate, class Id, class Root_enumerator> 
class Swap_event {
public:
  Swap_event(Id o, typename Sort::Handle sorter, 
	     const Certificate &s): left_object_(o), 
                                    sorter_(sorter), 
                                    s_(s){}
  void process(){
    sorter_->swap(left_object_, s_);
  }
  Id left_object_; 
  typename Sort::Handle sorter_; 
  Certificate s_;
};
\end{ccExampleCode}


\end{ccRefConcept}

% +------------------------------------------------------------------------+
%%RefPage: end of main body, begin of footer
% EOF
% +------------------------------------------------------------------------+

