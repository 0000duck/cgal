% +------------------------------------------------------------------------+
% | Reference manual page: Event.tex
% +------------------------------------------------------------------------+
% | 20.03.2005   Author
% | Package: Kinetic_data_structures
% | 
\RCSdef{\RCSEventRev}{$Id$}
\RCSdefDate{\RCSEventDate}{$Date$}
% |
%%RefPage: end of header, begin of main body
% +------------------------------------------------------------------------+

\ccDefGlobalScope{CGAL::}
\begin{ccRefConcept}[Kinetic::Simulator::]{Time}

%% \ccHtmlCrossLink{}     %% add further rules for cross referencing links
%% \ccHtmlIndexC[concept]{} %% add further index entries

\ccDefinition
  
The concept \ccClassName\ represents time in the simulator.

\ccCreationVariable{a}  %% choose variable name

\ccConstructor{Time(NT);}{Construct an instance of time from a number
type, where NT is the number type used in the simulation.}

\ccOperations

%\ccMethod{std::pair<NT, NT> isolating_interval();}{Return an interval (not necessarily a narrow one), containing the time represented by this object.}

\ccGlobalFunction{std::ostream& operator<<(std::ostream&, Time);}{Write it to a stream.}

\ccGlobalFunction{double to_double(Time);}{Return a double approximation of the time value.}

\ccGlobalFunction{std::pair<double, double> to_interval(Time);}{Return an interval containing the time value.}

Comparisons with other \ccc{Kinetic::Simulator::Time} objects are supported.

\ccHasModels

\ccc{double}, 
\ccc{Kinetic::FunctionKernel::Root}

\ccSeeAlso

\ccc{Kinetic::Simulator}


\end{ccRefConcept}

% +------------------------------------------------------------------------+
%%RefPage: end of main body, begin of footer
% EOF
% +------------------------------------------------------------------------+

