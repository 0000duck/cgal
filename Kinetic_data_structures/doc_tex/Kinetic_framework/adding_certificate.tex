\subsection{Adding a New Certificate Type\label{sec:kds_add_cert}}


The following example shows how to add a new type of certificate to a simulation. 

First we code the actual certificate function generator. It must take
some sort (or sorts) of kinetic primitives, compute some function from
their coordinates.

\begin{ccExampleCode}
template <class KineticKernel>
struct Positive_x_f_2 {
  typedef typename KineticKernel::Certificate_function result_type;
  typedef typename KineticKernel::Point_2 argument_type;
  result_type operator()(const argument_type &p){
    return result_type(p.x()- result_type(0));
  }
};
\end{ccExampleCode}

Then we define a kinetic kernel which includes this predicate. To do
this we wrap the function generator generator in a
\ccc{Kinetic::Certificate_generator<Kernel, Generator>}.
This wrapper uses the generator to create the certificate function and
then the \ccc{Kinetic::FunctionKernel} to solve the certificate
function. The result is wrapped in a \ccc{Kinetic::Certificate}
object.

\begin{ccExampleCode}
template <class FunctionKernel> 
class My_kinetic_kernel:
  public CGAL::Kinetic::Cartesian<FunctionKernel> {
  typedef CGAL::Kinetic::Cartesian<FunctionKernel> P;
  typedef My_kinetic_kernel<FunctionKernel> This;
public:
  typedef CGAL::Kinetic::internal::Certificate_generator<This, Positive_x_f_2<This> > Positive_x_2;
  Positive_x_2 positive_x_2_object() const
  {
    return Positive_x_2(P::function_kernel_object());
  }
};
\end{ccExampleCode}

Now we have the unfortunately rather messy part of assembling a new
\ccc{Kinetic::SimulationTraits} model. This is done in two steps for convenience.

\begin{ccExampleCode}
struct My_simulation_traits {
  typedef My_simulation_traits This;

  typedef CGAL::Exact_predicates_exact_constructions_kernel Static_kernel;
  //typedef CGAL::Regular_triangulation_euclidean_traits_3<Static_kernel_base> Static_kernel;
  typedef CGAL::POLYNOMIAL::Polynomial<Static_kernel::FT> Function;
  typedef CGAL::POLYNOMIAL::Sturm_root_stack_traits<Function> Root_stack_traits;
  typedef CGAL::POLYNOMIAL::Sturm_root_stack<Root_stack_traits> Root_stack;
  typedef CGAL::POLYNOMIAL::Kernel<Function, Root_stack> Function_kernel;

  typedef CGAL::Kinetic::Handle_degeneracy_function_kernel<Function_kernel, false>  Simulator_function_kernel_base;
  struct Simulator_function_kernel: public Simulator_function_kernel_base{};

  typedef My_kinetic_kernel<Simulator_function_kernel> Kinetic_kernel;
  typedef CGAL::Kinetic::Two_list_pointer_event_queue<Function_kernel> Event_queue;
  typedef CGAL::Kinetic::Default_simulator<Simulator_function_kernel, Event_queue > Simulator;

  typedef CGAL::Kinetic::Active_objects_vector<Kinetic_kernel::Point_1> Active_points_1_table;
  typedef CGAL::Kinetic::Active_objects_vector<Kinetic_kernel::Point_2> Active_points_2_table;
  typedef CGAL::Kinetic::Active_objects_vector<Kinetic_kernel::Point_3> Active_points_3_table;
  // typedef Active_objects_vector<Kinetic_kernel::Weighted_point_3> Active_weighted_points_3_table;
 
  typedef CGAL::Kinetic::Default_instantaneous_kernel<This> Instantaneous_kernel;

  Active_points_1_table* active_points_1_table_handle() const { return ap1_.get();}
  Active_points_2_table* active_points_2_table_handle() const {return ap2_.get();}
  Active_points_3_table* active_points_3_table_handle() const {return ap3_.get();}
  //Active_weighted_points_3_table* active_weighted_points_3_table_handle() const {return awp3_.get();}

  Simulator* simulator_handle() const { return sim_.get();}
  const Static_kernel& static_kernel_object() const {return k_;}
  const Kinetic_kernel& kinetic_kernel_object() const {return kk_;}
 
  Instantaneous_kernel instantaneous_kernel_object() const {
    return Instantaneous_kernel(*this);
  }

  My_simulation_traits(const Simulator::Time &lb,
		       const Simulator::Time &ub): sim_(new Simulator(lb, ub)),
						   ap1_(new Active_points_1_table()),
						   ap2_(new Active_points_2_table()),
						   ap3_(new Active_points_3_table())
  {}
 
  
  bool is_exact() const {
    return true;
  }
protected:
  Simulator::Handle sim_;
  Active_points_1_table::Handle ap1_;
  Active_points_2_table::Handle ap2_;
  Active_points_3_table::Handle ap3_;
  //Active_weighted_points_3_table::Handle awp3_;
  Static_kernel k_;
  Kinetic_kernel kk_;
  Function_kernel fk_;
};

\end{ccExampleCode}

Now the simulation traits can be used by a kinetic data structure.
Note that we define active point table for all dimensions. This is
needed by the \ccc{Kinetic::InstantaneousKernel}, even if they are not
used.

