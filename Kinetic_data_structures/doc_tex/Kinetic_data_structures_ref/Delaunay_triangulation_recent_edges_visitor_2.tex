% +------------------------------------------------------------------------+
% | Reference manual page: Event.tex
% +------------------------------------------------------------------------+
% | 20.03.2005   Author
% | Package: Kinetic_data_structures
% | 
\RCSdef{\RCSEventRev}{$Id: Delaunay_triangulation_recent_edges_visitor_2.tex 29475 2006-03-14 01:47:13Z drussel $}
\RCSdefDate{\RCSEventDate}{$Date: 2006-03-14 02:47:13 +0100 (Tue, 14 Mar 2006) $}
% |
%%RefPage: end of header, begin of main body
% +------------------------------------------------------------------------+


\begin{ccRefClass}{Kinetic::Delaunay_triangulation_recent_edges_visitor_2<Triangulation>}

%% \ccHtmlCrossLink{}     %% add further rules for cross referencing links
%% \ccHtmlIndexC[concept]{} %% add further index entries

\ccDefinition
  
The concept \ccRefName\ provides a model of
\ccc{Kinetic::DelaunayTriangulationVisitor2} which tracks which edges were created in
the most recent change.

\ccIsModel

\ccc{Kinetic::DelaunayTriangulationVisitor2}

\ccCreation
\ccCreationVariable{a}  %% choose variable name

\ccConstructor{Delaunay_triangulation_recent_edges_visitor_2();}{default constructor.}

\ccNestedType{iterator}{The iterator through the recently created edges.}

\ccOperations

\ccMethod{iterator begin() const;}{Begin iteration through the recent edges.}

\ccMethod{iterator end() const;}{End iteration through the recent edges.}

\ccMethod{bool contains(Triangulation::Edge) const;}{Returns true if this edge exists in the set.}

\ccSeeAlso

\ccc{Kinetic::Delaunay_triangulation_2<Traits, Triangulation, Visitor>}


\end{ccRefClass}

% +------------------------------------------------------------------------+
%%RefPage: end of main body, begin of footer
% EOF
% +------------------------------------------------------------------------+

