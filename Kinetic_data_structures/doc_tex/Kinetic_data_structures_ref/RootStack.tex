% +------------------------------------------------------------------------+
% | Reference manual page: RootEnumerator.tex
% +------------------------------------------------------------------------+
% | 20.03.2005   Author
% | Package: Kinetic_data_structures
% | 
\RCSdef{\RCSRootEnumeratorRev}{$Id$}
\RCSdefDate{\RCSRootEnumeratorDate}{$Date$}
% |
%%RefPage: end of header, begin of main body
% +------------------------------------------------------------------------+


\begin{ccRefConcept}{RootStack}

%% \ccHtmlCrossLink{}     %% add further rules for cross referencing links
%% \ccHtmlIndexC[concept]{} %% add further index entries

\ccDefinition
  
The concept \ccRefName\ enumerates through roots of a function. Most
of the root stacks maintain the invariant that
\ccc{pop()} can be called exactly once per root in the interval. However, certain of the \ccRefName s delay prooving that there are no remaining root, and so might return one root with the value \ccc{std::numeric_limits<Root>::infinity()} which does not correspond to an existing root. Comparing \ccc{top()} to the \ccc{Root} defining the end of the interval will allow you to make sure that it corresponds to an actual root of the polynomial (it is greater or equal then it is not in the interval). 

\ccTypes

\ccNestedType{Root}{The root of a function.}
\ccNestedType{Traits}{The traits class for this concept.}

\ccCreation
\ccCreationVariable{re}  %% choose variable name

\ccConstructor{RootStack();}{default constructor.}

\ccConstructor{RootStack(Function f, Root lb, Root ub, Traits tr);}{Construct a \ccRefName\ over the roots of \ccc{f} in the open interval \ccc{lb} to \ccc{ub}.}

\ccOperations

\ccMethod{void pop();}{Advance to the next root. As a precondition, empty() must be false.}

\ccMethod{Root top();}{Return the current root. As a precondition, empty() must be false. Note that the \ccc{Root} returned might not actually be in the interval (since the solver has not yet proved that there are no more roots).}

\ccMethod{bool empty();}{Return true if there are known to be no more roots left. There might not actually be any roots of the polynomial left in the interval, but the work necessary to prove this has been delayed.}

%\ccHasModels

%\ccc{Kinetic::Numeric_root_enumerator}.

\ccSeeAlso

FunctionKernel, Simulator.



\end{ccRefConcept}

% +------------------------------------------------------------------------+
%%RefPage: end of main body, begin of footer
% EOF
% +------------------------------------------------------------------------+

