% +------------------------------------------------------------------------+
% | Reference manual page: SimulationTraits.tex
% +------------------------------------------------------------------------+
% | 20.03.2005   Author
% | Package: Kinetic_data_structures
% | 
% |
%%RefPage: end of header, begin of main body
% +------------------------------------------------------------------------+


\begin{ccRefConcept}{Kinetic::SimulationTraits}

%% \ccHtmlCrossLink{}     %% add further rules for cross referencing links
%% \ccHtmlIndexC[concept]{} %% add further index entries

\ccDefinition
  
This concept ties together the parts needed in order to run a kinetic
data structure. 


\ccTypes

\ccNestedType{NT}{The number type used for representation.}

\ccNestedType{Static_kernel}{A CGAL kernel which can be used for static computations.}

\ccNestedType{Instantaneous_kernel}{A model of
  \ccc{Kinetic::InstantaneousKernel} which can be used to apply static CGAL
  data structures to snapshots of moving data.}

\ccNestedType{Kinetic_kernel}{A model of Kineti::Kernel.}

\ccNestedType{Simulator}{A model of \ccc{Kinetic::Simulator} which will be
used by all the kinetic data structures.}

%\ccNestedType{Active_objects_table}{A model of \ccc{Kinetic::ActiveObjectsTable} which can be used to store moving points of an appropriate dimension. This is really optional and not needed if no kinetic data structures use points.}

\ccOperations
\ccCreationVariable{st}

\ccMethod{Static_kernel static_kernel_object();}{Get a new static kernel.}

\ccMethod{Instantaneous_kernel instantaneous_kernel_object();}{Get a new instantaneous kernel.}

\ccMethod{Kinetic_kernel kinetic_kernel_object();}{Get a new kinetic kernel.}

\ccMethod{Simulator::Handle simulator_handle();}{Return a pointer to the \ccc{Simulator} which is to be used in the simulation.}

%\ccMethod{Active_objects_table* active_objects_table_pointer();}{Return a pointer to the table holding points which is to be used in the simulation.}

\ccHasModels

\ccc{Kinetic::Exact_simulation_traits_1},
\ccc{Kinetic::Exact_simulation_traits_2},
\ccc{Kinetic::Exact_simulation_traits_3},
\ccc{Kinetic::Inexact_simulation_traits_1},
\ccc{Kinetic::Inexact_simulation_traits_2},
\ccc{Kinetic::Inexact_simulation_traits_3},
\ccc{Kinetic::Exact_linear_simulation_traits_2},
\ccc{Kinetic::Exact_linear_simulation_traits_3},
\ccc{Kinetic::Inexact_linear_simulation_traits_2},
\ccc{Kinetic::Inexact_linear_simulation_traits_3}

\end{ccRefConcept}

% +------------------------------------------------------------------------+
%%RefPage: end of main body, begin of footer
% EOF
% +------------------------------------------------------------------------+

