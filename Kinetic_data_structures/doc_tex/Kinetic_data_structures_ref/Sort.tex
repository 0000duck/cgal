% +------------------------------------------------------------------------+
% | Reference manual page: Sort.tex
% +------------------------------------------------------------------------+
% | 27.03.2005   Author
% | Package: KDS
% | 
\RCSdef{\RCSSortRev}{$Id$}
\RCSdefDate{\RCSSortDate}{$Date$}
% |
%%RefPage: end of header, begin of main body
% +------------------------------------------------------------------------+

\ccDefGlobalScope{CGAL::}
\begin{ccRefClass}{Kinetic::Sort<Traits, Visitor>}  %% add template arg's if necessary

%% \ccHtmlCrossLink{}     %% add further rules for cross referencing links
%% \ccHtmlIndexC[class]{} %% add further index entries

\ccDefinition
  
The class \ccRefName\ maintains a sorted list of objects. It is the
simplest kinetic data structure provided and is a good place to start
when looking at the basics of implementing a kinetic data
structure. 

The \ccc{Kinetic::SortVisitor} can be used to monitor what is happening.

\ccInclude{CGAL/Kinetic/Sort.h}

\ccCreation
\ccCreationVariable{s}  %% choose variable name

\ccConstructor{Sort(Traits tr);}{The basic constructor.}

\ccTypes

\ccNestedType{Visitor}{The type of the visitor.}

\ccNestedType{Traits}{The traits type.}

\ccNestedType{Vertex_handle}{The handle used to refer to vertex in the sorted list. Derefernecing this returns a \ccc{Key} into the \ccc{ActiveObjectsTable}.}

\ccNestedType{Handle}{A reference counted pointer to be used for storing references to the object.}

\ccNestedType{Const_handle}{A reference counted pointer to be used for storing references to the object.}

\ccOperations

\ccMethod{Visitor& visitor();}{Access the visitor.}

\ccMethod{Traits& traits();}{Access the traits.}

\ccSeeAlso

\ccc{Kinetic::Ref_counted<T>}

\end{ccRefClass}

% +------------------------------------------------------------------------+
%%RefPage: end of main body, begin of footer
% EOF
% +------------------------------------------------------------------------+

