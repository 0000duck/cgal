% +------------------------------------------------------------------------+
% | Reference manual page: Event.tex
% +------------------------------------------------------------------------+
% | 20.03.2005   Author
% | Package: Kinetic_data_structures
% | 
\RCSdef{\RCSEventRev}{$Id$}
\RCSdefDate{\RCSEventDate}{$Date$}
% |
%%RefPage: end of header, begin of main body
% +------------------------------------------------------------------------+

\begin{ccRefConcept}{Kinetic::Event}

%% \ccHtmlCrossLink{}     %% add further rules for cross referencing links
%% \ccHtmlIndexC[concept]{} %% add further index entries

\ccDefinition
  
The concept \ccClassName\ represents a single event. Models of
\ccClassName\ should be passed to the \ccc{Kinetic::Simulator} when
scheduling events which will in turn pass them to the
\ccc{EventQueue}.

\ccCreationVariable{a}  %% choose variable name

\ccOperations

\ccMethod{void process();}{This method is called when the event occurs, and the time when it occurs is passed in as \ccc{t}. \ccc{Time} is the type defined by the \ccc{Simulator}. This method will only be called once per time this event is scheduled and the event will be removed from the queue immediately afterwards.}

\ccGlobalFunction{std::ostream& operator<<(std::ostream&, Event);}{Write a text description of the event to a standard stream.}

\ccHasModels

All over the place. 

\ccSeeAlso

Kinetic::EventQueue

\ccExample

All of the kinetic data structures provided have models of
\ccRefName. Here is the code implementing a swap event from the
sorting kinetic data structure.

\begin{ccExampleCode}
template <class Sort, class Id, class Root_enumerator> 
class Swap_event {
public:
  Swap_event(Id o, typename Sort::Pointer sorter, 
	     const Root_enumerator &s): left_object_(o), sorter_(sorter), s_(s){}
  void process(){
    sorter_->swap(left_object_, s_);
  }
  Id left_object_; typename Sort::Pointer sorter_; Solver s_;
};
\end{ccExampleCode}


\end{ccRefConcept}

% +------------------------------------------------------------------------+
%%RefPage: end of main body, begin of footer
% EOF
% +------------------------------------------------------------------------+

