% +------------------------------------------------------------------------+
% | Reference manual page: Qt_gui_2.tex
% +------------------------------------------------------------------------+
% | 20.03.2005   Author
% | Package: Kinetic_data_structures
% | 
\RCSdef{\RCSQtguiRev}{$Id: Qt_triangulation_2.tex 29411 2006-03-12 07:28:13Z drussel $}
\RCSdefDate{\RCSQtguiDate}{$Date: 2006-03-12 08:28:13 +0100 (Sun, 12 Mar 2006) $}
% |
%%RefPage: end of header, begin of main body
% +------------------------------------------------------------------------+


\begin{ccRefClass}{Kinetic::Qt_triangulation_2<KineticTriangulation_2, QtWidget_2, QtMovingPoints_2>}  %% add template arg's if necessary

%% \ccHtmlCrossLink{}     %% add further rules for cross referencing links
%% \ccHtmlIndexC[class]{} %% add further index entries

\ccDefinition
  
The class draws a triangulation into a \ccc{CGAL::Qt_widget_2}. This class is
very simple and a good one to look at if you want to see how to draw
your own two dimensional kinetic data structure.

See Section~\ref{sec:kds_delaunay_2_example} for an example using this class. 

\ccInclude{CGAL/Kinetic/IO/Qt_triangulation_2.h}

\ccCreation
\ccCreationVariable{a}  %% choose variable name

\ccConstructor{Qt_widget_2(KineticTriangulation_2::Handle,QtWidget_2::Handle,QtMovingPoints_2::Handle);}{Construct the object and make all the connections with the appropriate other objects.}

\ccSeeAlso

\ccc{Kinetic::Qt_widget_2<Simulator>}

\end{ccRefClass}

% +------------------------------------------------------------------------+
%%RefPage: end of main body, begin of footer
% EOF
% +------------------------------------------------------------------------+

