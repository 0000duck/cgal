% +------------------------------------------------------------------------+
% | Reference manual page: Qt_gui_2.tex
% +------------------------------------------------------------------------+
% | 20.03.2005   Author
% | Package: Kinetic_data_structures
% | 
\RCSdef{\RCSQtguiRev}{$Id$}
\RCSdefDate{\RCSQtguiDate}{$Date$}
% |
%%RefPage: end of header, begin of main body
% +------------------------------------------------------------------------+


\begin{ccRefClass}{Kinetic::Qt_widget_2<Simulator>}  %% add template arg's if necessary

%% \ccHtmlCrossLink{}     %% add further rules for cross referencing links
%% \ccHtmlIndexC[class]{} %% add further index entries

\ccDefinition
  
The class \ccRefName\ implements a graphical interface for 2D kinetic data structures.

\ccInclude{CGAL/Kinetic/IO/Qt_widget_2.h}

\ccTypes

\ccNestedType{Listener}{The listener base to listen for when to update the picture. This class includes an extra method \ccc{Qt_widget widget()} which returns the \ccc{Qt_widget} object which can be used for drawing.}

\ccCreation
\ccCreationVariable{a}  %% choose variable name

\ccConstructor{Qt_widget_2(int argc, char *argv[], Simulator::Handle);}{default constructor.}

\begin{ccExampleCode}
#include <CGAL/basic.h>
#include <CGAL/Kinetic/Exact_simulation_traits.h>
#include <CGAL/Kinetic/Delaunay_triangulation_2.h>
#include <CGAL/Kinetic/Delaunay_triangulation_vertex_base_2.h>
#include <CGAL/Kinetic/Delaunay_triangulation_recent_edges_visitor_2.h>
#include <CGAL/Kinetic/Enclosing_box_2.h>
#include <CGAL/Kinetic/IO/Qt_moving_points_2.h>
#include <CGAL/Kinetic/IO/Qt_triangulation_2.h>
#include <CGAL/Kinetic/IO/Qt_widget_2.h>
#include <CGAL/Kinetic/Insert_event.h>

int main(int argc, char *argv[])
{
  typedef CGAL::Kinetic::Exact_simulation_traits Traits;

 typedef CGAL::Triangulation_data_structure_2<
  CGAL::Kinetic::Delaunay_triangulation_vertex_base_2<Traits::Instantaneous_kernel>,
    CGAL::Kinetic::Delaunay_triangulation_face_base_2<Traits> > TDS;
  typedef CGAL::Delaunay_triangulation_2<Traits::Instantaneous_kernel, TDS > Del;
  // Color edges based on how recently they were created
  typedef CGAL::Kinetic::Delaunay_triangulation_recent_edges_visitor_2<Del> Visitor;
  typedef CGAL::Kinetic::Delaunay_triangulation_2<Traits, Visitor, Del> KDel;
  typedef CGAL::Kinetic::Qt_widget_2<Traits::Simulator> Qt_gui;
  typedef CGAL::Kinetic::Qt_moving_points_2<Traits, Qt_gui> Qt_mps;
  typedef CGAL::Kinetic::Qt_triangulation_2<KDel, Traits::Instantaneous_kernel, Qt_gui> Qt_triangulation;
  // Keeps the points confined
  typedef CGAL::Kinetic::Enclosing_box_2<Traits> Box;

  CGAL_SET_LOG_LEVEL(CGAL::Log::LOTS);

  // Run from time 0 until a really large value of time
  Traits tr(0, 10000000);
  Box::Handle box= new Box(tr);
  KDel::Handle kdel= new KDel(tr);

  Qt_gui::Handle qtsim= new Qt_gui(argc, argv, tr.simulator_handle());

  Qt_mps::Handle qtmps= new Qt_mps(qtsim, tr);
  Qt_triangulation::Handle qtdel= new Qt_triangulation(kdel, tr.instantaneous_kernel_object(), qtsim);
  Traits::Kinetic_kernel::Point_2 p;
  std::cin >> p;
  tr.active_points_2_table_handle()->insert(p);
  return qtsim->begin_event_loop();
}

\end{ccExampleCode}

\end{ccRefClass}

% +------------------------------------------------------------------------+
%%RefPage: end of main body, begin of footer
% EOF
% +------------------------------------------------------------------------+

