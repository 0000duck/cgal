% +------------------------------------------------------------------------+
% | Reference manual page: FunctionKernel.tex
% +------------------------------------------------------------------------+
% | 20.03.2005   Author
% | Package: Kinetic_data_structures
% | 
\RCSdef{\RCSFunctionKernelRev}{$Id$}
\RCSdefDate{\RCSFunctionKernelDate}{$Date$}
% |
%%RefPage: end of header, begin of main body
% +------------------------------------------------------------------------+


\begin{ccRefConcept}{Kinetic::FunctionKernel}

%% \ccHtmlCrossLink{}     %% add further rules for cross referencing links
%% \ccHtmlIndexC[concept]{} %% add further index entries

\ccDefinition
  
The concept \ccRefName\ encapsulates all the methods for representing
and handing functions. The set is kept deliberately small to easy use
of new \ccRefName s, but together these operations are sufficient to
allow the correct processing of events, handling of degeneracies,
usage of static data structures, run-time error checking as well as
run-time verification of the correctness of kinetic data structures.
The computation of a polynomial with the variable negated is used for
reversing time in kinetic data structures and can be omitted if that
capability is not needed.


\ccTypes

\ccNestedType{Function}{The type of function being handled.}

\ccNestedType{NT}{The basic representational number type.}

\ccNestedType{Root}{A type representing the roots of a \ccc{Function}.}

\ccNestedType{Root_stack}{A model of \ccc{RootStack}. These objects can be created by calling the \ccc{root_stack_object} method with a \ccc{Function} and two (optional) \ccc{Root} objects. The enumerator then enumerates all roots of the function in the open inverval defined by the two root arguments. They optional arguments default to positive and negative infinity. }

\ccNestedType{Root_enumerator_traits}{The traits for the \ccc{Root_enumerator} class.}

Each of the following types has a corresponding \ccc{type_object}
method (not explicitly documented) which takes a \ccc{Function} as an
argument.

\ccNestedType{Sign_at}{A functor which returns the sign of a
  \ccc{Function} at a \ccc{NT} or \ccc{Root}.}

\ccNestedType{Multiplicity}{A functor which returns the multiplicity of roots.}

\ccNestedType{Sign_above}{A functor which returns sign of a function
  immediately above a root.}

The following functor likewise have a \ccc{type_object} method, but
these take arguments other than a \ccc{Function}. The arguments are
given below.

\ccNestedType{Sign_between_roots}{This functor, creation of which
  requires two \ccc{Root}s, returns the sign of a passed function
  between the pair of roots.}

%\ccNestedType{Compare_isolated_roots_in_interval}{This functor, creation of which requires two functions, compares the roots of two functions in an isolating invterval. More specifically, it takes two \ccc{NT}s as an argument. These numbers must isolate a root for each of the functions the functor is constructed with. The functor then returns true if the isolated root of the first function is less than that of the second.}

%\ccNestedType{Quotient}{Compute the quotient of two \ccc{Functions}s.}

%\ccNestedType{Pseudo_quotient}{Compute the pseudo quotient of two \ccc{Functions}s.}

%\ccNestedType{Remainder}{Compute the remainder of one \ccc{Functions} divided by another.}

%\ccNestedType{Are_negations_object}{Return true of the two functions passed are negations of one another.}

%\ccNestedType{Sturm_sequence}{This object evaluates the Sturm sequence of two \ccc{Function}s at a \ccc{NT} value. Construction requires two \ccc{Function}s.}

\ccNestedType{Differentiate}{This functor computes the derivitive of a \ccc{Function}. Construction takes no arguments.}

%\ccNestedType{Sign_Sturm_sequence}{This object evaluates the Sturm sequence of two \ccc{Function}s at a \ccc{NT} value. Construction requires two \ccc{Function}s.}

The following methods do not require any arguments to get the functor and take one \ccc{Function} as a functor argument.

%\ccNestedType{Root_bound_evaluator}{This functor computes a root bound on a passed \ccc{Function}. }

%\ccNestedType{Invert_variable}{Map $f(x)$ to $x^d f(1/x)$.}

\ccNestedType{Negate_variable}{Map $f(x)$ to $f(-x)$.}

%\ccNestedType{Map_rational_interval_to_positive}{}

%\ccNestedType{Rational_translate_zero}{}
%\ccNestedType{Shift_power}{}

\ccNestedType{Construct_function}{The functor can be used to construct instances of \ccc{Function}.}


%\ccCreation
%\ccCreationVariable{fk}  %% choose variable name

%\ccConstructor{FunctionKernel(Root_enumerator_traits tr);}{}

%\ccOperations

%\ccMethod{void foo();}{some member functions}

\ccHasModels
\ccc{POLYNOMIAL::Kernel<RootStack>}, \ccc{POLYNOMIAL::Filtered_kernel<RootStack>}.

\ccSeeAlso

Kinetic::RootEnumerator

\ccExample

We provide several models of the concept, which are not documented
separately. The models of \ccc{Kinetic::SimulationTraits} all choose
appropriate models.  However, if
more control is desired, we here provide examples of how to create the
various supported \ccc{Kinetic::FunctionKernel}.

A Sturm sequence based kernel which supports exact comparisons of roots of polynomials (certificate failure times):
\begin{ccExampleCode}
typedef CGAL::POLYNOMIAL::Polynomial<CGAL::Gmpq> Function;
typedef CGAL::POLYNOMIAL::Sturm_root_stack_traits<Function> Root_stack_traits;
typedef CGAL::POLYNOMIAL::Sturm_root_stack<Root_stack_traits> Root_stack;
typedef CGAL::POLYNOMIAL::Kernel<Function, Root_stack> Function_kernel;
\end{ccExampleCode}

A wrapper for \ccc{CORE::Expr} which implements the necessary
operations:
\begin{ccExampleCode}
typedef CGAL::POLYNOMIAL::CORE_kernel Function_kernel;
\end{ccExampleCode}

A function kernel which computes approximations to the roots of the polynomials:
\begin{ccExampleCode}
typedef CGAL::POLYNOMIAL::Polynomial<double> Function;
typedef CGAL::POLYNOMIAL::Root_stack_default_traits<Function> Root_stack_traits;
typedef CGAL::POLYNOMIAL::Numeric_root_stack<Root_stack_traits> Root_stack;
typedef CGAL::POLYNOMIAL::Kernel<Function, Root_stack> Function_kernel;
\end{ccExampleCode}

When using the function kernel in kinetic data structures, especially
one that is in exact, it is useful to wrap the root stack. The wrapper
checks the sign of the certificate function being solved and uses that
to handle degenacies. This is done by, for the inexact solvers 
\begin{ccExampleCode}
typedef \ccc{Kinetic::Derivitive}_filter_function_kernel<Function_kernel> KDS_function_kernel;
\end{ccExampleCode}
and for exact solvers
\begin{ccExampleCode}
typedef \ccc{Kinetic::Handle}_degeneracy_function_kernel<Function_kernel> KDS_function_kernel;
\end{ccExampleCode}

For exact computations, the primary representation for roots is the
now standard choice of a polynomial with an associated isolating
interval (and interval containing exactly one distinct root of a
polynomial) along with whether the root has odd or even multiplicity
and, if needed, the Sturm sequence of the polynomial. Two intervals
can be compared by first seeing if the isolating intervals are
disjoint. If they are, then we know the ordering of the respective
roots. If not we can subdivide each of the intervals (using the
endpoints of the other interval) and repeat. In order to avoid
subdividing endlessly when comparing equal roots, once we subdivide a
constant number of times, we use the Sturm sequence of $p$ and $p'q$
(where $p$ and $q$ are the two polynomials and $p'$ is the derivative
of $p$) to evaluate the sign of the second at the root of the first
one directly (note that this Sturm sequence is applied to a common
isolating interval of the roots of interest of both polynomials).



\end{ccRefConcept}


% +------------------------------------------------------------------------+
%%RefPage: end of main body, begin of footer
% EOF
% +------------------------------------------------------------------------+

