% +------------------------------------------------------------------------+
% | Reference manual page: Regular_trianglation_3.tex
% +------------------------------------------------------------------------+
% | 20.03.2005   Author
% | Package: Kinetic_data_structures
% | 
\RCSdef{\RCSRegulartrianglationRev}{$Id$}
\RCSdefDate{\RCSRegulartrianglationDate}{$Date$}
% |
%%RefPage: end of header, begin of main body
% +------------------------------------------------------------------------+


\begin{ccRefClass}{KDS::Regular_triangulation_3<Traits, Visitor, Triangulation>}  %% add template arg's if necessary

%% \ccHtmlCrossLink{}     %% add further rules for cross referencing links
%% \ccHtmlIndexC[class]{} %% add further index entries

\ccDefinition
  
The class \ccRefName\ maintains a triangulation of set of moving weighted points. Its interface is the same as \ccc{KDS::Delaunay_triangulation_3}. 

The \ccc{Triangulation} template argument must have
\ccc{CGAL::KDS::Regular_triangulation_cell_base_3<Traits, Base>} as a
cell base and \ccc{CGAL::KDS::Regular_triangulation_vertex_base_3<Traits, Base>} as a vertex base.

\ccInclude{CGAL/KDS/Regular_triangulation_3.h}


\ccSeeAlso
\ccc{KDS::Delaunay_triangulation_3}.

\ccExample

\ccIncludeExampleCode{Kinetic_data_structures/regular_triangulation_3.C}| 


\end{ccRefClass}

% +------------------------------------------------------------------------+
%%RefPage: end of main body, begin of footer
% EOF
% +------------------------------------------------------------------------+

