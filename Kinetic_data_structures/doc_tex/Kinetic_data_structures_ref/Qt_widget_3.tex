% +------------------------------------------------------------------------+
% | Reference manual page: Qt_gui_3.tex
% +------------------------------------------------------------------------+
% | 20.03.2005   Author
% | Package: Kinetic_data_structures
% | 
\RCSdef{\RCSQtguiRev}{$Id$}
\RCSdefDate{\RCSQtguiDate}{$Date$}
% |
%%RefPage: end of header, begin of main body
% +------------------------------------------------------------------------+


\begin{ccRefClass}{KDS::Qt_widget_3<Simulator>}  %% add template arg's if necessary

%% \ccHtmlCrossLink{}     %% add further rules for cross referencing links
%% \ccHtmlIndexC[class]{} %% add further index entries

\ccDefinition
  
The class \ccRefName\ implements a graphical interface for running 3D
kinetic data structures. This class uses Coin or another Open Inventor
implementation to manage and display the geometry. 

\ccInclude{CGAL/KDS/IO/Qt_widget_3.h}

\ccTypes

\ccNestedType{Listener}{This listener provides \ccc{CURRENT_TIME} notifications when the structures should update their geometry. It also provides a method \ccc{SoSeparator* root()} which returns an place for geometry to be inserted into the scene graph.}

\ccCreation
\ccCreationVariable{a}  %% choose variable name

\ccConstructor{Qt_widget_3(int argc, char *arv[], Simulator::Pointer p);}{default constructor.}

\ccOperations

\ccMethod{int begin_event_loop();}{Start interactive processing.}

\ccMethod{Simulator* simulator();}{Return a pointer to the simulator.}

\ccMethod{double current_time();}{Return a the current time approximated as a double.}



\end{ccRefClass}


% +------------------------------------------------------------------------+
%%RefPage: end of main body, begin of footer
% EOF
% +------------------------------------------------------------------------+

