% +------------------------------------------------------------------------+
% | Reference manual page: Delaunay_triangulation_3.tex
% +------------------------------------------------------------------------+
% | 20.03.2005   Author Daniel Russel
% | Package: Kinetic_data_structures
% | 
\RCSdef{\RCSDelaunaytriangulationRev}{$Id$}
\RCSdefDate{\RCSDelaunaytriangulationDate}{$Date$}
% |
%%RefPage: end of header, begin of main body
% +------------------------------------------------------------------------+


\begin{ccRefClass}{Kinetic::Delaunay_triangulation_2<Traits, Visitor, Triangulation>}  %% add template arg's if necessary

%% \ccHtmlCrossLink{}     %% add further rules for cross referencing links
%% \ccHtmlIndexC[class]{} %% add further index entries

\ccDefinition
  
The class \ccRefName\ maintains a Delaunay triangulation on top of the
points contained in a \ccc{Kinetic::ActiveObjectsTable}. It has one
main method of interest, \ccc{triangulation()}, which returns the
triangulation it is maintaining.

The class \ccc{Kinetic::Qt_triangulation_2<KineticTriangulation_2,
  QtWidget_2, QtMovingPoints_2>} displays a kinetic Delaunay
triangulation using the Qt widget.

This class is a good example of a simple, but non-trivial, kinetic
data structure.

The \ccc{Triangulation} template parameter must be a model of
\ccc{CGAL::Delaunay_triangulation_2<Traits, Tds>} which uses
\ccc{Traits::Instantaneous_kernel} as its geometric traits and a
\ccc{Tds} whose face inherits from
\ccc{Kinetic::Delaunay_triangulation_face_base_2<Traits, Base>}.

The optional \ccc{Visitor} parameter takes a model of
\ccc{Kinetic::DelaunayTriangulationVisitor2}. Methods on this object will be called
whenever the triangulation changes.

\ccInclude{CGAL/Kinetic/Delaunay_triangulation_2.h}

\ccIsModel

\ccc{Ref_counted<T>}


\ccTypes

\ccNestedType{Triangulation}{The template argument triangulation.}

\ccNestedType{Visitor}{The template argument for the visitor.}

\ccCreation
\ccCreationVariable{dt}  %% choose variable name

\ccConstructor{Delaunay_triangulation_2(Traits tr);}{Maintain the
  Delaunay triangulation of the points in
  \ccc{tr.active_points_2_handle()}.}

\ccOperations

\ccMethod{const Triangulation& triangulation() const;}{Access the triangulation that is maintained. }

\ccMethod{Visitor& visitor();}{Access the visitor. }

\ccMethod{Vertex_handle insert(Point_key k);}{Insert the point. }

\ccMethod{void erase(Vertex_handle h);}{Erase the vertex.}

\ccSeeAlso

\ccc{Kinetic::DelaunayTriangulationVisitor2},
\ccc{Kinetic::Delaunay_triangulation_default_visitor_2},
\ccc{Kinetic::Delaunay_triangulation_recent_edges_visitor_2<Triangulation>},
\ccc{Kinetic::Delaunay_triangulation_event_log_visitor_2},
\ccc{Kinetic::Qt_Delaunay_triangulation_2}.


\end{ccRefClass}

% +------------------------------------------------------------------------+
%%RefPage: end of main body, begin of footer
% EOF
% +------------------------------------------------------------------------+

