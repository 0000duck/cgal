% +------------------------------------------------------------------------+
% | Reference manual page: Delaunay_triangulation_3.tex
% +------------------------------------------------------------------------+
% | 20.03.2005   Author Daniel Russel
% | Package: Kinetic_data_structures
% | 
\RCSdef{\RCSDelaunaytriangulationRev}{$Revision$}
\RCSdefDate{\RCSDelaunaytriangulationDate}{$Date$}
% |
%%RefPage: end of header, begin of main body
% +------------------------------------------------------------------------+


\begin{ccRefClass}{KDS::Delaunay_triangulation_2<Traits, Visitor, Triangulation>}  %% add template arg's if necessary

%% \ccHtmlCrossLink{}     %% add further rules for cross referencing links
%% \ccHtmlIndexC[class]{} %% add further index entries

\ccDefinition
  
The class \ccRefName\ maintains a Delaunay triangulation on top of the
points contained in a \ccc{ActiveObjectsTable}. It has one main method
of interest. \ccc{triangulation()} which returns the triangulation it
is maintaining. 

The class \ccc{KDS::Qt_Delaunay_triangulation_2} displays a kinetic
Delaunay triangulation using the Qt widget.

This class is a good example of a simple, but non-trivial, kinetic data structure. 

The Triangulation template parameter must use
\ccc{Traits::Instantaneous_kernel} as its geometric traits and the
\ccc{Triangulation::Face} must inherit from \ccc{CGAL::KDS::Delaunay_triangulation_face_base_2<Traits>}.

The optional \ccc{Visitor} parameter takes a model of
\ccc{DelaunayTriangulationVisitor_2}. Methods on this object will be called
whenever the triangulation changes.

\ccInclude{CGAL/KDS/Delaunay_triangulation_2.h}


\ccTypes

\ccNestedType{Triangulation}{The type of the Delaunay triangulation it maintains. This is a \ccc{CGAL::Delaunay_triangulation} built using the \ccc{InstantaneousKernel} as the kernel.}

\ccCreation
\ccCreationVariable{dt}  %% choose variable name

\ccConstructor{Delaunay_triangulation_2(Simulator::Pointer sp, ActiveObjectsTable::Pointer mot);}{Maintain the Delaunay triangulation of the points in \ccc{mot}.}

\ccOperations

\ccMethod{const Triangulation& triangulation();}{Access the triangulation that is maintained. }

\ccSeeAlso

DelaunayTriangulationVisitor2,
\ccc{KDS::Delaunay_triangulation_default_visitor_2},
\ccc{KDS::Delaunay_triangulation_recent_edges_visitor_2},
\ccc{KDS::Qt_Delaunay_triangulation_2}.

\ccExample

This example shows using the \ccRefName\ , the graphical interface and
a watcher class. The implementation code is simple.

\ccIncludeExampleCode{Kinetic_data_structures/Delaunay_triangulation_2.C}



\end{ccRefClass}

% +------------------------------------------------------------------------+
%%RefPage: end of main body, begin of footer
% EOF
% +------------------------------------------------------------------------+

