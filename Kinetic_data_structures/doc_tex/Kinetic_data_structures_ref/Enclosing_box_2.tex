% +------------------------------------------------------------------------+
% | Reference manual page: Delaunay_triangulation_3.tex
% +------------------------------------------------------------------------+
% | 20.03.2005   Author
% | Package: Kinetic_data_structures
% | 
\RCSdef{\RCSDelaunaytriangulationRev}{$Id$}
\RCSdefDate{\RCSDelaunaytriangulationDate}{$Date$}
% |
%%RefPage: end of header, begin of main body
% +------------------------------------------------------------------------+


\begin{ccRefClass}{KDS::Enclosing_box_2<Traits>}  %% add template arg's if necessary

%% \ccHtmlCrossLink{}     %% add further rules for cross referencing links
%% \ccHtmlIndexC[class]{} %% add further index entries

\ccDefinition
  
The class \ccRefName\ keeps the points in the simulation inside of a
box. Whenever the points come close to the wall of the box they bounce off of the wall.

Note that, in general, points hit the wall of the box at times which
are not easily represented by number types. In order to handle this,
the \ccRefName\ bounces the points at the nearest easily representable
time before the point would leave the box.

\ccInclude{CGAL/KDS/Enclosing_box_2.h}

\ccTypes

\ccNestedType{NT}{The number type used to represent the walls of the box and perform calculations. Generally this is \ccc{Traits::NT}.}

\ccCreation
\ccCreationVariable{eb}  %% choose variable name

\ccConstructor{Enclosing_box_2(Traits, NT xmin, NT xmax, NT ymin, NT ymax);}{This constructs a bounding box with the dimensions specified by the last 4 arguments. They are optional and will take the values $\pm$10 if omitted.}



\end{ccRefClass}

% +------------------------------------------------------------------------+
%%RefPage: end of main body, begin of footer
% EOF
% +------------------------------------------------------------------------+

