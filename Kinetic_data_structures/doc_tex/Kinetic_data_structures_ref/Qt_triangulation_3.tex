% +------------------------------------------------------------------------+
% | Reference manual page: Qt_gui_2.tex
% +------------------------------------------------------------------------+
% | 20.03.2005   Author
% | Package: Kinetic_data_structures
% | 
\RCSdef{\RCSQtguiRev}{$Id$}
\RCSdefDate{\RCSQtguiDate}{$Date$}
% |
%%RefPage: end of header, begin of main body
% +------------------------------------------------------------------------+


\begin{ccRefClass}{Kinetic::Qt_triangulation_3<KineticDelaunay_3, QtWidget_3, Qt_moving_points_3>}  %% add template arg's if necessary

%% \ccHtmlCrossLink{}     %% add further rules for cross referencing links
%% \ccHtmlIndexC[class]{} %% add further index entries

\ccDefinition
  
The class draws a triangulation into a \ccc{Qt_widget} in three
dimensions. In contrast to
\ccc{Kinetic::Qt_triangulation_2<KineticDelaunay_2, QtWidget_2,
Qt_moving_points_2>}, this class is quite complicated.

At runtime, the `h' key toggles display of the convex hull faces and
the `f' key toggles drawing of all the facets (the edges are always
displayed).

\ccInclude{CGAL/Kinetic/IO/Qt_triangulation_3.h}

\ccCreation
\ccCreationVariable{a}  %% choose variable name

\ccConstructor{Qt_widget_2(KineticDelaunay_3::Pointer,QtWidget_3::Pointer,Qt_moving_points_3::Pointer);}{Construct the object and make all the connections with the appropriate other objects.}


\end{ccRefClass}

% +------------------------------------------------------------------------+
%%RefPage: end of main body, begin of footer
% EOF
% +------------------------------------------------------------------------+

