% +------------------------------------------------------------------------+
% | Reference manual page: Qt_moving_points_2.tex
% +------------------------------------------------------------------------+
% | 20.03.2005   Author
% | Package: Kinetic_data_structures
% | 
\RCSdef{\RCSQtmovingpointsRev}{$Id$}
\RCSdefDate{\RCSQtmovingpointsDate}{$Date$}
% |
%%RefPage: end of header, begin of main body
% +------------------------------------------------------------------------+


\begin{ccRefClass}{Kinetic::Qt_moving_points_2<Traits, QtWidget_2>}  %% add template arg's if necessary

%% \ccHtmlCrossLink{}     %% add further rules for cross referencing links
%% \ccHtmlIndexC[class]{} %% add further index entries

\ccDefinition
  
The class \ccRefName\ displays a set of moving points in 2D.

See Section~\ref{sec:kds_delaunay_2_example} for an example using this class. 

\ccInclude{CGAL/Kinetic/IO/Qt_moving_points_2.h}




\ccCreation
\ccCreationVariable{a}  %% choose variable name

\ccConstructor{Qt_moving_points_2(QtGui::Handle,Traits::Active_points_2_table::Handle);}{default constructor.}


\ccSeeAlso

\ccc{Kinetic::Qt_widget_2<Simulator>}.

\end{ccRefClass}

% +------------------------------------------------------------------------+
%%RefPage: end of main body, begin of footer
% EOF
% +------------------------------------------------------------------------+

