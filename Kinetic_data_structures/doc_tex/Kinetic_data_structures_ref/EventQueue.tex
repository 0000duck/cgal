% +------------------------------------------------------------------------+
% | Reference manual page: EventQueue.tex
% +------------------------------------------------------------------------+
% | 20.03.2005   Author
% | Package: Kinetic_data_structures
% | 
\RCSdef{\RCSEventQueueRev}{$Id$}
\RCSdefDate{\RCSEventQueueDate}{$Date$}
% |
%%RefPage: end of header, begin of main body
% +------------------------------------------------------------------------+


\begin{ccRefConcept}{EventQueue}

%% \ccHtmlCrossLink{}     %% add further rules for cross referencing links
%% \ccHtmlIndexC[concept]{} %% add further index entries

\ccDefinition
  
The for priority queues used by the \ccc{Simulator}. The concept
basically defines a priority queue which supports deletions and
changes of items in the queue (but not their priorities). Items in the
queue must implement the \ccc{Event} concept.


\ccTypes

\ccNestedType{Key}{The type used to access items in the queue in order to change or delete them.}

\ccNestedType{Priority}{The priority type for items in the queue. This is typically the same as \ccc{Simulator::Time}}.

\ccCreation
\ccCreationVariable{q}  %% choose variable name

\ccConstructor{EventQueue(Priority start, Priority end, int size_hint);}{Construct a queue which will start at time start and run until time end.}

\ccOperations

\ccMethod{template <class Event> Key insert(Priority, Event);}{Insert
  an event into the event queue. A \ccc{Key} which can be used to
  manipulated the event is returned.}

\ccMethod{void erase(Key);}{Erase an event from the queue.}

\ccMethod{template <class Event> void set(Key, Event);}{Change the data in the event referred to by the key.}

\ccMethod{template <class Event> Event& get(Key) const;}{Access the event referred to by the passed key.}

\ccMethod{Priority priority(Key) const;}{Return the priority of the event.}

\ccMethod{bool empty();}{Return true if the queue is empty.}

\ccMethod{Priority next_priority() const;}{Return the priority of the next event in the queue.}

\ccMethod{void process_next();}{Process the next \ccc{Event} by calling its process method with its \ccc{Priority}.}

\ccMethod{void set_end_priority();}{Set the priority beyond which to ignore events.}

\ccHasModels

\ccc{Two_list_pointer_event_queue},
\ccc{Heap_pointer_event_queue}.


\end{ccRefConcept}

% +------------------------------------------------------------------------+
%%RefPage: end of main body, begin of footer
% EOF
% +------------------------------------------------------------------------+

