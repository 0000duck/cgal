% +------------------------------------------------------------------------+
% | Reference manual page: Delaunay_triangulation_3.tex
% +------------------------------------------------------------------------+
% | 20.03.2005   Author
% | Package: Kinetic_data_structures
% | 
\RCSdef{\RCSDelaunaytriangulationRev}{$Id$}
\RCSdefDate{\RCSDelaunaytriangulationDate}{$Date$}
% |
%%RefPage: end of header, begin of main body
% +------------------------------------------------------------------------+


\begin{ccRefClass}{Kinetic::Delaunay_triangulation_3<Traits, Visitor, Triangulation>}  %% add template arg's if necessary

%% \ccHtmlCrossLink{}     %% add further rules for cross referencing links
%% \ccHtmlIndexC[class]{} %% add further index entries

\ccDefinition
  
The class \ccRefName\ maintains a Delaunay triangulation on top of the
points contained in a \ccc{Kinetic::ActiveObjectsTable}. It has one main method
of interest. \ccc{triangulation()} which returns the triangulation it
is maintaining. In addition, as an optimisation, you can turn on and
off whether it is currently maintaining its certificates. This allows
a large number of changes to the underlying points to be made at one
time without recomputing the certificates each time a single point
changes.

The class \ccc{Kinetic::Qt_triangulation_3<Traits>}, included as part
of the demo code, displays a kinetic Delaunay triangulation in three
dimensions using the Coin library.

The optional \ccc{Visitor} template argument is a model of
\ccc{Kinetic::DelaunayTriangulationVisitor3} and can be used to monitor
changes in the kinetic data structure.

The optional \ccc{Triangulation} template argument must be a model of
a static Delaunay triangulation and have
\ccc{Kinetic::Delaunay_triangulation_cell_base_3<Traits, Base>} a the cell
type.

\ccInclude{CGAL/Kinetic/Delaunay_triangulation_3.h}


\ccTypes

\ccNestedType{Triangulation}{The template argument.}

\ccNestedType{Visitor}{The template argument.}

\ccCreation
\ccCreationVariable{dt}  %% choose variable name

\ccConstructor{Delaunay_triangulation_3(Traits tr);}{Maintain the Delaunay triangulation of the points in \ccc{tr.active_points_3_handle()}.}

\ccOperations

\ccMethod{const Triangulation* triangulation();}{Access the triangulation that is maintained. }

\ccMethod{bool has_certificates();}{This method returns true if the \ccRefName\ is currently maintaining certificates for a Delaunay triangulation. }

\ccMethod{void set_has_certificates(bool tf);}{This method allows you to control whether the triangulation is maintaining certificates.}

\ccMethod{Visitor& visitor();}{Access the visitor.}

\ccSeeAlso

\ccc{Kinetic::Regular_triangulation_3<Traits, Triangulation, Visitor>},
\ccc{Kinetic::Delaunay_triangulation_2<Traits, Triangulation, Visitor}, 
\ccc{Kinetic::Delaunay_triangulation_visitor_base_3},
\ccc{Kinetic::Delaunay_triangulation_event_log_visitor_3}.


\end{ccRefClass}

% +------------------------------------------------------------------------+
%%RefPage: end of main body, begin of footer
% EOF
% +------------------------------------------------------------------------+

