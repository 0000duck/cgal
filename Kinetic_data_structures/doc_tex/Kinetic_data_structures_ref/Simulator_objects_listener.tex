% +------------------------------------------------------------------------+
% | Reference manual page: Simulator_objects_listener.tex
% +------------------------------------------------------------------------+
% | 20.03.2005   Author
% | Package: Kinetic_data_structures
% | 
\RCSdef{\RCSSimulatorobjectslistenerRev}{$Id$}
\RCSdefDate{\RCSSimulatorobjectslistenerDate}{$Date$}
% |
%%RefPage: end of header, begin of main body
% +------------------------------------------------------------------------+


\begin{ccRefClass}{Kinetic::Simulator_objects_listener<Simulator_listener, Responder>}  %% add template arg's if necessary

%% \ccHtmlCrossLink{}     %% add further rules for cross referencing links
%% \ccHtmlIndexC[class]{} %% add further index entries

\ccDefinition
  
The class \ccRefName\ is a helper for classes which wish to react to
\ccc{Simulator::Listener::DIRECTION_OF_TIME} notifications. The helper
object translates such notifications \ccc{reverse_time} function calls
on the responder. See \ccc{Kinetic::Qt_moving_points_2} for a
simple example of using this helper function.

\ccInclude{CGAL/Kinetic/Simulator_objects_listener.h}


\ccCreation
\ccCreationVariable{a}  %% choose variable name

\ccConstructor{Simulator_objects_listener(Simulator::Pointer, Responder*);}{default constructor.}

\ccSeeAlso

\ccc{Kinetic::Listener},
\ccc{Kinetic::Qt_moving_points_2}.

\ccExample

See \ccc{Kinetic::Qt_moving_points_2} for an example of usage.

\end{ccRefClass}

% +------------------------------------------------------------------------+
%%RefPage: end of main body, begin of footer
% EOF
% +------------------------------------------------------------------------+

