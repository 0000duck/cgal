% +------------------------------------------------------------------------+
% | Reference manual page: Event.tex
% +------------------------------------------------------------------------+
% | 20.03.2005   Author
% | Package: Kinetic_data_structures
% | 
\RCSdef{\RCSEventRev}{$Id: Regular_triangulation_visitor_base_3.tex 29411 2006-03-12 07:28:13Z drussel $}
\RCSdefDate{\RCSEventDate}{$Date: 2006-03-12 08:28:13 +0100 (Sun, 12 Mar 2006) $}
% |
%%RefPage: end of header, begin of main body
% +------------------------------------------------------------------------+


\begin{ccRefClass}{Kinetic::Regular_triangulation_visitor_base_3}

%% \ccHtmlCrossLink{}     %% add further rules for cross referencing links
%% \ccHtmlIndexC[concept]{} %% add further index entries

\ccDefinition
  
The concept \ccRefName\ provides a model of
\ccc{Kinetic::RegularTriangulationVisitor3}. You can extend this class if you only
want to implement a few methods from \ccc{Kinetic::RegularTriangulationVisitor3}.

\ccIsModel

\ccc{Kinetic::RegularTriangulationVisitor3}

\ccCreation
\ccCreationVariable{a}  %% choose variable name

\ccConstructor{Regular_triangulation_default_visitor_3();}{default constructor.}

\ccSeeAlso

\ccc{Kinetic::Regular_triangulation_3<Traits, Triangulation, Visitor>}


\end{ccRefClass}

% +------------------------------------------------------------------------+
%%RefPage: end of main body, begin of footer
% EOF
% +------------------------------------------------------------------------+

