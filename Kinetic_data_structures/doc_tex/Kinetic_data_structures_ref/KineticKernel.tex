% +------------------------------------------------------------------------+
% | Reference manual page: KineticKernel.tex
% +------------------------------------------------------------------------+
% | 20.03.2005   Author
% | Package: Kinetic_data_structures
% | 
\RCSdef{\RCSKineticKernelRev}{$Id$}
\RCSdefDate{\RCSKineticKernelDate}{$Date$}
% |
%%RefPage: end of header, begin of main body
% +------------------------------------------------------------------------+


\begin{ccRefConcept}{Kinetic::Kernel}

%% \ccHtmlCrossLink{}     %% add further rules for cross referencing links
%% \ccHtmlIndexC[concept]{} %% add further index entries

\ccDefinition
  
The concept \ccRefName\ acts as the kinetic analog of a CGAL
kernel. It provides some set of primitives and predicats acting on
them.


\ccCreationVariable{kk}
\ccTypes

\ccNestedType{Motion_function}{The type which is used to represent coordinates of moving primitives. This is the analog of the CGAL kernel \ccc{RT}.}
\ccNestedType{Certificate}{The type representing the results of predicates. See Kinetic::Certificate.}
\ccNestedType{Function_kernel}{The type of the function kernel used. See Kinetic::FunctionKernel.}

\ccOperations

\ccMethod{Function_kernel function_kernel_object() const;}{Gets a copy of the function kernel.}

\ccCreationVariable{a}  %% choose variable name


\ccHasModels

\ccc{Kinetic::Cartesian_kinetic_kernel<FunctionKernel>}.




\end{ccRefConcept}


% +------------------------------------------------------------------------+
%%RefPage: end of main body, begin of footer
% EOF
% +------------------------------------------------------------------------+

