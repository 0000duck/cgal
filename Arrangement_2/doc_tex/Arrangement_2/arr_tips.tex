\section{How To Speed Up Your Computation\label{arr_sec:tips}}
%=======================================
Before the specific tips, we remind you that compiling programs with
debug flags disabled and with optimization flags enabled significantly
reduces the running time.

\begin{enumerate}
\item
When the curves to be inserted into an arrangement are $x$-monotone
and pairwise disjoint in their interior to start with, then it is more
efficient (in running time) and less demanding (in traits-class
functionality) to use the non-intersection insertion-functions instead
of the general ones; e.g., \ccc{insert_x_monotone_curve()}.

\item
When the curves to be inserted into an arrangement are segments that
are pairwise disjoint in their interior, it is more efficient to use
the traits class \ccc{Arr_non_caching_segment_traits_2} rather then
the default one (\ccc{Arr_segment_traits_2}).

If the segments may intersect each other, the default traits class
\ccc{Arr_segment_traits_2} can be safely used with the somehow limited
number type \ccc{Quotient<MP_float>}.

On rare occasions the traits class \ccc{Arr_non_caching_segment_traits_2}
exhibits slightly better performance than the default one
(\ccc{Arr_segment_traits_2}
even when the segments intersect each other, due to the small overhead
of the latter (optimized) traits class. (For example, when the so
called {\sc Leda} rational kernel is used).

\item
Prior knowledge of the combinatorial structure of the arrangement can
be used to accelerate operations that insert $x$-monotone curves,
whose interior is disjoint from existing edges and vertices of the
arrangement. The specialized insertion functions, i.e., 
\ccc{insert_in_face_interior()}, \ccc{insert_from_left_vertex()}, 
\ccc{insert_from_right_vertex()}, and \ccc{insert_at_vertices()}
can be used according to the available information. These functions
hardly involve any geometric operations, if at all. They accept
topologically related parameters, and use them to operate directly on
the \dcel\ records, thus saving algebraic operations, which are
especially expensive when high-degree curves are involved.

A polygon, represented by a list of segments along its boundary, can
be inserted into an empty arrangement as follows. First, one segment
is inserted using \ccc{insert_in_face_interior()} into the unbounded
face. Then, a segment with a common end point is inserted using either
\ccc{insert_from_left_vertex()} or \ccc{insert_from_right_vertex()},
and so on with the rest of the segments except for the last, which is
inserted using \ccc{insert_at_vertices()}, as both endpoints of which
are the mapping of known vertices.

\item
The main trade-off among point-location strategies, is between time
and storage. Using the naive or walk strategies, for example, takes
more query time but does not require preprocessing or maintenance of
auxiliary structures and saves storage space.

\item
If point-location queries are not performed frequently, but other
modifying functions, such as removing, splitting, or merging edges
are, then using a point-location strategy that does not require the
maintenance of auxiliary structures, such as the naive or walk
strategies, is preferable.

\item
There is a trade-off between two modes of the trapezoidal RIC strategy
that enables the user to choose whether preprocessing should be
performed or not. If preprocessing is not used, the creation of the
structure is faster. However, for some input sequences the structure
might be unbalanced and therefore queries and updates might take
longer, especially, if many removal and split operations are
performed.

\item
When the curves to be inserted into an arrangement are available in
advance (as opposed to supplied on-line), it is advised to use the
more efficient aggregate (sweep-based) insertion over the incremental
insertion; e.g., \ccc{insert_curves()}.


\item
The various traits classes should be instantiated with an exact number
type to ensure robustness, when the input of the operations to be
carried out might be degenerate, although inexact number types could
be used at the user's own risk.

\item
Maintaining short bit-lengths of coordinate representations may
drastically decrease the time consumption of arithmetic operations on
the coordinates. This can be achieved by caching certain information
or normalization (of rational numbers). However, both solutions should
be used cautiously, as the former may lead to an undue space
consumption, and indiscriminate normalization may considerably slow
down the overall process.

\item
Geometric functions (e.g., traits methods) dominate the time
consumption of most operations. Thus, calls to such function should be
avoided or at least their number should be decreased, perhaps at the
expense of increased combinatorial-function calls or increased space
consumption. For example, repetition of geometric-function calls could
be avoided by storing the results obtained by the first call, and
reusing them when needed.
\end{enumerate}
