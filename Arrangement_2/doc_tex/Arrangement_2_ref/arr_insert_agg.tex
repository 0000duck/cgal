\ccRefPageBegin

\begin{ccRefFunction}{insert_curves}

\ccDefinition

The function \ccRefName{} aggregately inserts curves or $x$-monotone
curves in a given range into a given arrangement, where no
restrictions are made on the nature of the inserted curves.  If the
curves are not $x$-monotone then the input curves are subdivided into
$x$-monotone subcurves (and perhaps isolated points), which are
inserted into the arrangement using the sweep-line algorithm.

%%%%

\ccInclude{CGAL/Arrangement_2.h}

\ccGlobalFunction{template<class Traits, class Dcel, class InputIterator>
            void insert_curves (Arrangement_2<Traits,Dcel>& arr,
                                InputIterator first, InputIterator last);}
Inserts the curves or $x$-monotone curves in the range
\ccc{[first,last)} into the arrangement \ccc{arr}.

%%%%

\ccInclude{CGAL/Arrangement_with_history_2.h}

\ccThree{Orientation~~}{}{\hspace*{8.5cm}}

\ccGlobalFunction{template<class Traits, class Dcel, InputIterator>
            void insert_curves (Arrangement_with_history_2<Traits,Dcel>& arr,
                         InputIterator first, InputIterator last);}
Inserts the curves in the range \ccc{[first,last)} into the
arrangement with history \ccc{arr}.

\ccRequirements
\begin{itemize}
\item If the curve is an $x$-monotone curve then The instantiated
  \ccc{Traits} class must model the \ccc{ArrangementXMonotoneTraits_2}
  concept. Otherwise, if the curve is not $x$-monotone, the
  instantiated \ccc{Traits} class must model the
  \ccc{ArrangementTraits_2} concept. 
\item \ccc{InputIterator::value_type} must be \ccc{Traits::Curve_2} or
  \ccc{Traits::X_monotone_curve_2}.
\end{itemize}

\end{ccRefFunction}

\ccRefPageEnd
