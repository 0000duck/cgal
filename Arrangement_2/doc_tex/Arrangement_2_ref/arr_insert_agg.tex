\ccRefPageBegin

\begin{ccRefFunction}{insert_curves}

\ccDefinition

Aggregately inserts the given range of curves \ccc{[first,last)} into
the arrangement \ccc{arr}, where no restrictions are made on the nature of
the inserted curves. The \ccc{Traits} parameter must be a model of the
\ccc{ArrangementTraits_2} concept. The input curves are subdivided into
$x$-monotone subcurves (and perhaps isolated points), which are inserted
into the arrangement using the sweep-line algorithm.

\ccInclude{CGAL/Arrangement_2.h}

\ccGlobalFunction{template<class Traits, class Dcel, InputIterator>
            void insert_curves (Arrangement_2<Traits,Dcel>& arr,
                                InputIterator first, InputIterator last);}
\ccPrecond{The value-type of \ccc{InputIterator} is 
           \ccc{Traits::Curve_2}.}

Aggregately inserts the given range of curves \ccc{[first,last)} into
the arrangement \ccc{arr}.

\ccInclude{CGAL/Arrangement_with_history_2.h}

\ccThree{Orientation~~}{}{\hspace*{8.5cm}}

\ccGlobalFunction{template<class Traits, class Dcel, InputIterator>
            void insert_curves (Arrangement_with_history_2<Traits,Dcel>& arr,
                         InputIterator first, InputIterator last);}
\ccPrecond{The value-type of \ccc{InputIterator} is \ccc{Traits::Curve_2}.}

\end{ccRefFunction}

\ccRefPageEnd
