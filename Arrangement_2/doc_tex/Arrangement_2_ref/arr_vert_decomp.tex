% +------------------------------------------------------------------------+
% | Reference manual page: arr_vert_decomp.tex
% +------------------------------------------------------------------------+
% | 
% | Package: Arrangement_2
% | 
% +------------------------------------------------------------------------+

\ccRefPageBegin

\begin{ccRefFunction}{decompose}
\ccThree{OutputIterator  }{}{\hspace*{12cm}}

\ccInclude{CGAL/Arr_vertical_decomposition.h}

\ccFunction{template<class Arrangement, class OutputIterator>
            OutputIterator locate (const Arrangement& arr,
                                   Vertical_map& vert_map,
                                   OutputIterator voi);}
   {Produces the symbolic vertical decomposition of the given arrangement
    \ccc{arr}. More precisely, it performs batched vertical ray-shooting
    query from all arrangement vertices, such that every vertex is associated
    with a pair of objects, one corresponding to the arrangement feature that
    lies below it, while the other corresponds to the feature that lies above
    it. The query results are given through the output \ccc{vert_map}. The
    type \ccc{Vertical_map} used in the function prototype stated above really
    stand for: \\
    \begin{tabbing}
    \ccc{Unique_hash_map<}\=\ccc{typename Arrangement::Vertex_const_handle,}\\
                      ~   \>\ccc{std::pair<CGAL::Object, CGAL::Object> >}
    \end{tabbing}
    Namely, it is a mapping of each arrangement vertex to a pair of \cgal\
    objects, the first representing the feature below the object and the second
    is the feature above it. Each \ccc{Object} may be: \\
   \begin{itemize}
   \item \ccc{Halfedge_const_handle}, if the vertex is located above (or below)
         an edge. The given halfedge is always directed from right to left.
         In case there is no concrete edge below (or above) the vertex, then
         the object returned is a {\em fictitious halfedge}.
   \item \ccc{Vertex_const_handle}, in case the vertex is located vertically
         above (or below) another arrangement vertex.
   \item An empty object, in case the vertex is the top end-vertex of
         a vertical edge, we define there is no feature below it. Similarly, if
         it is the bottom end-vertex of a vertical edge, we define that there
         is no feature above it.
   \end{itemize}
   The output of this function can be readily used for inserting vertical walls
   and physically decomposing the arrangement into pseudo-trapezoids. To do
   this, it is convenient to process the vertices in an ascending
   $xy$-lexicographic order. The function therefore returns of ordered vertices
   through the output iterator \ccc{voi}. Its returned value is a past-the-end
   iterator for this ordered sequence.
    \ccPrecond{The value-type of \ccc{OutputIterator} is 
               \ccc{Arrangement::Vertex_const_handle}.}}

\end{ccRefFunction}

\ccRefPageEnd
