% +------------------------------------------------------------------------+
% | Reference manual page: Arr_linear_traits.tex
% +------------------------------------------------------------------------+
% | 
% | Package: Arrangement_2
% | 
% +------------------------------------------------------------------------+

\ccRefPageBegin

\begin{ccRefClass}{Arr_linear_traits_2<Kernel>}
    
\ccDefinition 
%============

The traits class \ccRefName\ is a model of the \ccc{ArrangementTraits_2}
concept that allows for the construction and maintenance of arrangements of
linear objects which may be bounded (line segments) or unbounded (rays
and lines). The traits class is parameterized with a \cgal-kernel model;
see the reference page of \ccc{Arr_segment_traits_2<Kernel>}
(\ccRefPage{CGAL::Arr_segment_traits_2<Kernel>}) for further explanations
and recommendations on choosing a kernel.

\ccRefName\ defines \ccc{Kernel::Point_2} as its point type. The nested
\ccc{X_monotone_curve_2} and \ccc{Curve_2} types defined by the traits class
(as is the case with the various segment-traits classes, both types
refer to the same class, as {\sl every} linear object is (weakly) $x$-monotone),
are constructible from a point, a line segment, a ray and from a line
(objects of types \ccc{Kernel::Point_2}, \ccc{Kernel::Segment_2},
\ccc{Kernel::Ray_2} and \ccc{Kernel::Line_2}, respectively). On the other
hand, when we are given a curve we can find out its actual type and convert
it to the respective kernel object (say, to a \ccc{Kernel::Ray_2}).

\ccInclude{CGAL/Arr_linear_traits_2.h}
 
\ccIsModel
    \ccc{ArrangementTraits_2} \\
    \ccc{ArrangementLandmarkTraits_2}

\subsection*{Class 
        Arr\_linear\_traits\_2$<$Kernel$>$::Curve\_2}
%====================================================

The \ccc{Curve_2} (and the \ccc{X_monotone_curve_2}) class nested within
the linear-traits can represent all types of linear objects. The class
supports the copy and default constructor and the assignment operator.
In addition, the \ccc{operator<<} and \ccc{operator>>} for linear objects
are defined for standard output and input streams.

\begin{ccClass}{Arr_linear_traits_2<Kernel>::Curve_2}

\ccTypes
%-------

\ccTypedef{typedef typename Kernel::Point_2 Point_2;}{}
\ccGlue
\ccTypedef{typedef typename Kernel::Segment_2 Segment_2;}{}
\ccGlue
\ccTypedef{typedef typename Kernel::Ray_2 Ray_2;}{}
\ccGlue
\ccTypedef{typedef typename Kernel::Line_2 Line_2;}{}

\ccCreation
\ccCreationVariable{c}
%---------------------

\ccConstructor{Curve_2 (const Point_2& p);}
    {constructs an degenerate curve equivalent to the point \ccc{p}.}

\ccConstructor{Curve_2 (const Segment_2& seg);}
    {constructs an line segment.}

\ccConstructor{Curve_2 (const Ray_2& ray);}
    {constructs a ray.}

\ccConstructor{Curve_2 (const Line_2& ln);}
    {constructs a line.}

\ccAccessFunctions
%-----------------

\ccMethod{bool is_point() const;}
  {returns whether \ccVar\ is a point (a degenerate curve).}
\ccGlue
\ccMethod{Point_2 point() const;}
  {returns a point object equivalent to \ccVar.
   \ccPrecond{\ccVar\ is a point.}}

\ccMethod{bool is_segment() const;}
  {returns whether \ccVar\ is a segment (has two bounded endpoints).}
\ccGlue
\ccMethod{Segment_2 segment() const;}
  {returns a segment object equivalent to \ccVar.
   \ccPrecond{\ccVar\ is a segment.}}

\ccMethod{bool is_ray() const;}
  {returns whether \ccVar\ is a ray (has a bounded source point and its
   other end is unbounded).}
\ccGlue
\ccMethod{Ray_2 ray() const;}
  {returns a ray object equivalent to \ccVar.
   \ccPrecond{\ccVar\ is a ray.}}

\ccMethod{bool is_line() const;}
  {returns whether \ccVar\ is a line (has only unbounded ends).}
\ccGlue
\ccMethod{Line_2 line() const;}
  {returns a line object equivalent to \ccVar.
   \ccPrecond{\ccVar\ is a line.}}

\ccMethod{Line_2 supporting_line() const;}
  {returns the supporting line of \ccVar.
   \ccPrecond{\ccVar\ is not a point (a degenerate curve).}}

\ccMethod{Point_2 source() const;}
  {returns \ccVar's source point.
   \ccPrecond{\ccVar\ is a point, a segment or a ray.}}
\ccGlue
\ccMethod{Point_2 target() const;}
  {returns \ccVar's target point.
   \ccPrecond{\ccVar\ is a point or a segment.}}

\end{ccClass}

\end{ccRefClass}

\ccRefPageEnd
