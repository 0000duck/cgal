\ccRefPageBegin

\begin{ccRefFunction}{zone}

\ccDefinition

The function \ccRefName{} compute the zone of the given $x$-monotone
curve in the existing arrangement. Meaning, it output the
arrangement's vertices, edges and faces that the $x$-monotone curve 
intersects. The order of the objects is the order that they are
discovered when traversing the $x$-monotone curve from left to right.

A given point-location object is used for answering point-location queries
during the insertion process. By default, the function uses the
``walk along line'' point-location strategy --- namely an instance of the
class \ccc{Arr_walk_along_line_point_location<Arrangement_2<Traits,Dcel> >}.

%%%%

\ccInclude{CGAL/Arrangement_2.h}

\ccGlobalFunction{template <class Traits, class Dcel, 
                  class OutputIterator, class PointLocation>
		  OutputIterator zone (
		  Arrangement_2<Traits,Dcel>& arr, 
                  const typename Traits::X_monotone_curve_2& c,
                  OutputIterator oi,
                  const PointLocation& pl);}
Compute the zone of the given $x$-monotone curve \ccc{c} in the
arrangement \ccc{arr}.
\ccPrecond{If provided, \ccc{pl} must be attached to the given arrangement
\ccc{arr}.}

%%%%

\ccRequirements
\begin{itemize}
\item The instantiated \ccc{Traits} class must model the
  \ccc{ArrangementXMonotoneTraits_2} concept. 
\item The point-location object \ccc{pl}, must model the
  \ccc{ArrangementPointLocation_2} concept.
\end{itemize}

\end{ccRefFunction}

\ccRefPageEnd
