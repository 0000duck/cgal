\ccRefPageBegin

\begin{ccRefFunction}{remove_vertex}

\ccDefinition

The function \ccRefName{} attempts to removed a given vertex from a given
arrangement. The vertex can be removed if it is either an isolated vertex,
(and has no incident edge,) or if it is a {\em redundant} vertex. That
is, it has exactly two incident edges, whose associated curves can be
merged to form a single $x$-monotone curve.
The function returns a boolean value that indicates whether it succeeded
removing the vertex from the arrangement.

\ccInclude{CGAL/Arrangement_2.h}

\ccGlobalFunction{template <typename Traits, typename Dcel>
   bool remove_vertex (Arrangement_2<Traits,Dcel>& arr,
                       typename Arrangement_2<Traits,Dcel>::Vertex_handle v);}

\ccRequirements
\begin{itemize}
\item The instantiated \ccc{Traits} class must model the
  \ccc{ArrangementXMonotoneTraits_2} concept. Not all expressions listed 
  by this concept are required. In fact the traits class must model the 
  \ccc{ArrangementBasicTraits_2} concept and support the merging
  functionality.
\end{itemize}
	       
\end{ccRefFunction}

\ccRefPageEnd
