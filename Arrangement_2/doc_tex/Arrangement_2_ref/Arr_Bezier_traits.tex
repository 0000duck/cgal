% +------------------------------------------------------------------------+
% | Reference manual page: Arr_Bezier_traits.tex
% +------------------------------------------------------------------------+
% | 
% | Package: Arrangement_2
% | 
% +------------------------------------------------------------------------+

\ccRefPageBegin
\begin{ccRefClass}{Arr_Bezier_curve_traits_2<RatKernel,AlgKernel,NtTraits>}

\ccDefinition
%============

The traits class \ccRefName\ is a model of the \ccc{ArrangementTraits_2}
concept that handles planar B\'ezier curves. A planar {\em B\'ezier curve}
$B$ is a parametric curve defined by a sequence of {\em control points}
$p_0, \ldots, p_n$ as follows:
\begin{eqnarray*}
B(t) = \left(X(t), Y(t)\right)
  = \sum_{k=0}^{n}{p_k \cdot \frac{n!}{k! (n-k)!} \cdot
                   t^k (1-t)^{n-k}}\ .
\end{eqnarray*}
where $t \in [0, 1]$. The degree of the curve is therefore $n$ ---
namely, $X(t)$ and $Y(t)$ are polynomials of degree $n$. B\'ezier curves
have numerous applications in computer graphics and solid modelling. They
are used, for example, in free-form sketches and for defining the true-type
fonts.

In our representation, we assume that the coordinates of all control
points are rational numbers (namely they are given as objects of the
\ccc{RatKernel::Point_2} type), so both $X(t)$ and $Y(t)$ are polynomials
with rational coefficients. The intersection points between curves are
however algebraic numbers, and their exact computation is time-consuming.
The traits class therefore contains a layer of geometric filtering that
performs all computation in an approximate manner whenever possible, and
it resorts to exact computations only when the approximate computation
fails to produce an unambiguous result.

We therefore require separate representations of the control points and
the intersection points. The \ccc{NtTraits} should be instantiated with a class
that defines nested \ccc{Integer}, \ccc{Rational} and \ccc{Algebraic} number
types and supports various operations on them, yielding certified computation
results (for example, in can convert rational numbers to algebraic numbers
and can compute roots of polynomials with integer coefficients).
The other template parameters, \ccc{RatKernel} and \ccc{AlgKernel} should be
geometric kernels templated with the \ccc{NtTraits::Rational} and
\ccc{NtTraits::Algebraic} number types, repectively. It is recommended to
instantiate the \ccc{CORE_algebraic_number_traits} class as the \ccc{NtTraits}
parameter, with \ccc{Cartesian<NtTraits::Rational>} and
\ccc{Cartesian<NtTraits::Algebraic>} instantiating the two kernel types,
respectively. The number types in this case are provided by the {\sc Core}
library, with its ability to exactly represent simple algebraic numbers.

\ccInclude{CGAL/Arr_Bezier_curve_traits_2.h}

\ccIsModel
  \ccc{ArrangementTraits_2}

\ccTypes
%-------

\ccNestedType{Rational}{the \ccc{NtTraits::Rational} type
                        (and also the \ccc{RatKernel::FT} type).}

\ccNestedType{Algebraic}{the \ccc{NtTraits::Algebraic} type
                         (and also the \ccc{AlgKernel::FT} type).}

\subsection*{Class
         Arr\_Bezier\_curve\_traits\_2$<$RatKernel,AlgKernel,NtTraits$>$::Curve\_2}
%========================================================================

The \ccc{Curve_2} class nested within the B\'ezier traits class is used
to represent a B\'ezier curve of arbitrary degree, which is defined by a
sequence of rational control points. The only resriction we have is that
the curve does not contain any self-intersections. In addition to the
methods listed below, the
I/O operators \ccc{operator<<} and \ccc{operator>>} for standard output
streams are also supported. The copy constructor and assigment operator
are also supported.

\begin{ccClass}{Arr_Bezier_curve_traits_2<AlgKernel,NtTraits>::Curve_2}
%======================================================================
% 
\ccCreation
\ccCreationVariable{B}
%---------------------

\ccConstructor{Curve_2 ();}
  {default constructor.}

\ccConstructor{template <class InputIterator>
               Curve_2 (InputIterator pts_begin, InputIterator pts_end);}{%
  constructs a B\'ezier curve as defined by the given range of control
  points. The value-type of \ccc{InputIterator} is \ccc{RatKernel::Point_2}.
  \ccPrecond{The input range must contain at least two control points, and
    the polyline defined by these points must not be self-intersecting. That
    is, if we denote the control points by $p_0, \ldots, p_n$, then the $n$
    segements defined by $p_{i-1}p_i$ (with $1 \leq i \leq n$) must be
    pairwise disjoint in their interior.}
}

\ccAccessFunctions
%-----------------

\ccMethod{size_t number_of_control_point () const;}
  {returns the number of control points that define \ccVar.}

\ccMethod{typename RatKernel::Point_2 control_point (size_t k) const;}
  {returns the $k$th control point. Note that the first control point equals
   the curve source, while the last control point equals its target. The rest
   of the control points do not lie on the curve.
   \ccPrecond{$k$ is smaller than the number of control points.}}

\ccMethod{typename RatKernel::Point_2 operator() (const Rational& t) const;}
  {returns the point $B(t)$ on the curve that corresponds to the given
   rational parameter value.}

\ccMethod{typename AlgKernel::Point_2 operator() (const Algebraic& t) const;}
  {returns the point $B(t)$ on the curve that corresponds to the given
   algebraic parameter value.}

\end{ccClass}


\subsection*{Class
         Arr\_Bezier\_curve\_traits\_2$<$RatKernel,AlgKernel,NtTraits$>$::Point\_2}
%========================================================================

The \ccc{Point_2} class nested within the B\'ezier traits class is used
to represent: (i) an endpoint of a B\'ezier curve, (ii)~a vertical tangency
point of a curve, used to subdivide it into $x$-monotone subcurve, and
(iii)~an intersection point between two curves. While, points of type~(i) have
rational coordinates and are given as part of the input, points of the two
latter types have algebraic coordinates. However, to speed up the arrangement
construction, such point are not computed in an exact manner, and instead
are given in an approximate representation. Note that the exact coordinates
of a point may only be accessed if it is exactly computed.

In addition to the methods listed below, the copy constructor and assigment
operator for \ccc{Point_2} objects are also supported.

\begin{ccClass}{Arr_Bezier_curve_traits_2<AlgKernel,NtTraits>::Point_2}
%======================================================================

\ccCreation
\ccCreationVariable{p}
%---------------------

\ccConstructor{Point_2 ();}
  {default constructor.}

\ccConstructor{Point_2 (const Curve_2& B, const Algebraic& t_0);}
  {constructs the point $B(t_0)$ on the given curve. As $t_0$ is an
   algebraic number, the point has algebraic coordinates.}

\ccConstructor{Point_2 (const Curve_2& B, const Rational& t_0);}
  {constructs the point $B(t_0)$ on the given curve. As $t_0$ is a
   rational number, the point has rational coordinates.}

\ccAccessFunctions
%-----------------

\ccMethod{std::pair<double, double> approximate () const;}
  {returns the approximated coordinates of \ccVar.}

\ccMethod{bool is_exact () const;}
  {returns whether the coordinates of \ccVar\ are computed in an exact manner.}

\ccMethod{Algebraic x () const;}
  {returns the $x$-coordinate of \ccVar.
   \ccPrecond{\ccVar is exactly computed.}}
\ccGlue
\ccMethod{Algebraic y () const;}
  {returns the $y$-coordinate of \ccVar.
   \ccPrecond{\ccVar is exactly computed.}}

\ccMethod{bool is_rational () const;}
  {returns whether the coordinates of \ccVar\ are rational numbers.}

\ccMethod{operator typename RatKernel::Point_2 () const;}
  {casts \ccVar\ to a point with rational coordinates.
   \ccPrecond{\ccVar\ has rational coordinates.}}

\end{ccClass}


\subsection*{Class 
Arr\_Bezier\_curve\_traits\_2$<$RatKernel,AlgKernel,NtTraits$>$::X\_monotone\_curve\_2}
%==============================================================================

The \ccc{X_monotone_curve_2} class nested within the B\'ezier traits is
used to represent $x$-monotone subcurves of B\'ezier curves. The subcurve is
defined by a supporting B\'ezier curve $B(t)$ and a range of definition in
the parameter space $[t_1, t_2] \subseteq [0, 1]$, where $B(t_1)$ is the
subcurve source and $B(t_2)$ is its target. Note that as the point endpoints
may only be approxiamted, the parameter range definining the subcurve may
only be approximately known.

It is not possible to construct $x$-monotone subcurves directly. Instead,
use the \ccc{Make_x_monotone_2} functor supplied by the traits class to
subdivide a \ccc{Curve_2} object into $x$-monotone cubcurves.

\begin{ccClass}{Arr_Bezier_curve_traits_2<AlgKernel,NtTraits>::X_monotone_curve_2}
%===========================================================================

\ccCreationVariable{b}
\ccAccessFunctions
%-----------------

\ccMethod{Curve_2 supporting_curve () const;}
  {returns the supporting B\'ezier curve of \ccVar.}

\ccMethod{Point_2 source () const;}
  {returns the source point of \ccVar.}
\ccGlue
\ccMethod{Point_2 target () const;}
  {returns the target point of \ccVar.}

\ccMethod{Point_2 left () const;}
  {returns the left ($xy$-lexicographically smaller) endpoint of \ccVar.}
\ccGlue
\ccMethod{Point_2 right () const;}
  {returns the right ($xy$-lexicographically smaller) endpoint of \ccVar.}

\ccMethod{std::pair<double, double> parameter_range () const;}
  {return the approximate parameter range defining the subcurve \ccVar.}

\end{ccClass}

\end{ccRefClass}
\ccRefPageEnd
