% +------------------------------------------------------------------------+
% | Reference manual page: ArrangementLandmarkTraits.tex
% +------------------------------------------------------------------------+
% | 
% | Package: Arrangement_2
% | 
% +------------------------------------------------------------------------+

\ccRefPageBegin

\begin{ccRefConcept}{ArrangementLandmarkTraits_2}

\ccDefinition

The landmark-traits concept refines the basic traits concept by adding
operations needed for the landmarks point-location strategy, namely ---
approximating points and connecting points with a simple $x$-monotone
curve.

A model of this concept must define the \ccc{Approximate_number_type}, which
is used to approximate the coordinates of \ccc{Point_2} instances. It is
recommended to define the approximated number type as the built-in
\ccc{double} type. 

\ccRefines
\ccc{ArrangementBasicTraits_2}

\ccTypes
%=======

\ccNestedType{Approximate_number_type}{the number type used
                                       to approximate point coordinates.}

\ccHeading{Functor Types}
%========================

\ccThree{Compare_y_at_x_2}{}{\hspace*{14cm}}
\ccThreeToTwo

\ccNestedType{Approximate_2}
{provides the operator~:
 \begin{itemize}
 \item \ccc{Approximate_number_type operator() (Point_2 p, int i)} \\
 which returns an approximation of \ccc{p}'s $x$-coordinate (if \ccc{i == 0}),
 or of \ccc{p}'s $y$-coordinate (if \ccc{i == 1}).
 \end{itemize}}

\ccNestedType{Construct_x_monotone_curve_2}
{provides the operator~:
 \begin{itemize}
 \item \ccc{X_monotone_curve_2 operator() (Point_2 p1, Point_2 p2)} \\
 returns an $x$-monotone curve connecting \ccc{p1} and \ccc{p2} (i.e., the
 two input points are its endpoints).
 \end{itemize}}

\ccCreation
\ccCreationVariable{traits}
%==========================

\ccThree{Construct_x_monotone_curve_2~~~}{}{\hspace*{7cm}}
\ccThreeToTwo

\ccConstructor{ArrangementLandmarkTraits_2();}{default constructor.}
\ccGlue
\ccConstructor{ArrangementLandmarkTraits_2(ArrangementLandmarkTraits_2 other);}
{copy constructor.}
\ccGlue
\ccMethod{ArrangementLandmarkTraits_2  operator=(other);}{assignment operator.}

\ccHeading{Accessing Functor Objects}
%====================================

\ccMethod{Approximate_2 approximate_2_object() const;} {}
\ccGlue
\ccMethod{Construct_x_monotone_curve_2 
          construct_x_monotone_curve_2_object() const;}{}

\ccHasModels
%===========

\ccc{CGAL::Arr_non_caching_segment_basic_traits_2<Kernel>} \\
\ccc{CGAL::Arr_segment_traits_2<Kernel>} \\
\ccc{CGAL::Arr_polyline_traits_2<SegmentTraits>} \\
\ccc{CGAL::Arr_conic_traits_2<RatKernel,AlgKernel,NtTraits>} \\

\ccSeeAlso
%=========

\ccc{ArrangementBasicTraits_2}\lcTex{ 
      (\ccRefPage{ArrangementBasicTraits_2})}

\end{ccRefConcept}

\ccRefPageEnd
