% +------------------------------------------------------------------------+
% | Reference manual page: Arr_with_hist_in_formatter.tex
% +------------------------------------------------------------------------+
% | 
% | Package: Arrangement_2
% | 
% +------------------------------------------------------------------------+

\ccRefPageBegin

\begin{ccRefConcept}{ArrangementWithHistoryInputFormatter}

A model for the \ccRefName\ concept supports a set of functions that enable
reading an arrangement-with-history instance from an input stream using a
specific format.

\ccRefines
\ccc{ArrangementInputFormatter}

\ccTypes
%=======

\ccNestedType{Arr_with_history_2}{the type of arrangement to input.}

\ccTypedef{typedef typename Arrangement_2::Curve_2 Curve_2;}
    {the inducing curve type.}


\ccCreationVariable{inf}

\ccHeading{Formatted Input Functions}
%=====================================

\ccMethod{void read_curves_begin();}
    {reads a message indicating the beginning of the inducing curves.}
\ccGlue
\ccMethod{void read_curves_end();}
    {reads a message indicating the end of the inducing curves.}

\ccMethod{void read_curve_begin();}
    {reads a message indicating the beginning of a single curve record.}
\ccGlue
\ccMethod{void read_curve_end();}
    {reads a message indicating the end of a single curve record.}

\ccMethod{void read_curve (Curve_2& c);}
    {reads a curve.}

\ccMethod{void read_induced_edges_begin();}
    {reads a message indicating the beginning of the set of edges
     induced by the current curve.}
\ccGlue
\ccMethod{void read_induced_edges_end();}
    {reads a message indicating the end of the induced edges set.}

\ccHasModels
%===========

\ccc{Arr_with_history_text_formatter<ArrFormatter>}\lcTex{
        (\ccRefPage{CGAL::Arr_with_history_text_formatter<ArrFormatter>})}

\end{ccRefConcept}

\ccRefPageEnd

