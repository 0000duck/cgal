% +------------------------------------------------------------------------+
% | Reference manual page: Arr_face.tex
% +------------------------------------------------------------------------+
% | 
% | Package: Arrangement_2
% | 
% +------------------------------------------------------------------------+

\ccRefPageBegin

\begin{ccRefClass}[Arrangement_2<Traits,Dcel>::]{Face}
\ccRefLabel{arr_ref:face}

\ccDefinition
An object $e$ of the class \ccRefName\ represents an arrangement edge,
namely a $2$-dimensional arrangement cell. Every arrangement contains
exactly one \emph{unbounded} face, an a number of bounded face. Each
bounded face has a boundary comprised of a halfedge chain winding in
a counterclockwise orientation around it. A face may also contain holes,
which are defined by clockwise-oriented halfedge chains, and isolated
vertices.

\ccInheritsFrom
    \ccHtmlNoLinksFrom{\ccc{typename Dcel::Face}}

\ccCreation
\ccCreationVariable{f}
%=====================

\ccConstructor{Face();}
   {default constructor.}    	    
    
\ccAccessFunctions
%=================

All non-const methods listed below also have \ccc{const} counterparts
that return constant handles, iterators or circulators:

\ccMethod{bool is_fictitious () const;}
    {returns whether this is the fictitious face, which contain the entire
     arrangement (and has not outer CCB).}

\ccMethod{bool is_unbounded() const;} 
    {returns whether the face is unbounded.}    

\ccMethod{Ccb_halfedge_circulator outer_ccb();}
    {returns a circulator that allows going over the outer boundary of
     \ccVar{}. The edges along the CCB are traversed in a counterclockwise
     direction.
     \ccPrecond{\ccVar{} is not a fictitious face.}}
    
\ccMethod{Hole_iterator holes_begin();}
    {returns an iterator for traversing all the holes (inner CCBs) of
     \ccVar{}.}
\ccGlue
\ccMethod{Hole_iterator holes_end();}
    {returns a past-the-end iterator for the holes of \ccVar{}.}

\ccMethod{Isolated_vertex_iterator isolated_vertices_begin();}
    {returns an iterator for traversing all the isolated vertices
     contained in the interior of \ccVar{}.}
\ccGlue
\ccMethod{Isolated_vertex_iterator isolated_vertices_end();}
    {returns a past-the-end iterator for the isolated vertices 
     contained inside \ccVar{}.}

\end{ccRefClass}

\ccRefPageEnd