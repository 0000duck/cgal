% +------------------------------------------------------------------------+
% | Reference manual page: Arr_circle_segment_traits.tex
% +------------------------------------------------------------------------+
% | 
% | Package: Arrangement_2
% | 
% +------------------------------------------------------------------------+

\ccRefPageBegin
\begin{ccRefClass}{Arr_circle_segment_traits_2<Kernel>}

\ccDefinition
%============

The class \ccRefName\ is a model of the \ccc{ArrangementTraits_2} concept
and can be used to construct and maintain arrangements of circular arcs
and line segments.

The traits class must be instantiated with a geometric kernel, such that
the supporting circles of the circular arcs are of type \ccc{Kernel::Circle_2}
and the supporting lines of the line segments are of type \ccc{Kernel::Line_2}.
Thus, the coordinates of the center of supporting circles, and its squared
radius are of type \ccc{Kernel::FT}, which should be an exact rational
number-type; similarly, the coefficients of each supporting line
$ax + by + c = 0$ are also of type \ccc{Kernel::FT}. Note however that
the intersection point between two such arcs do not have rational coordinates
in general. For this reason, we do not require the endpoints of the input arcs
and segments to have rational coordinates.

The nested \ccc{Point_2} type defined by the traits class is therefore
{\em different} than the \ccc{Kernel::Point_2} type. Its coordinates are
of type \ccc{CoordNT}, and represent real numbers obtained from solving
quadratic equations with rational coordinates. A number of type
\ccc{CoordNT} can therefore be expressed as $\alpha + \beta\sqrt{\gamma}$,
where $\alpha$, $\beta$ and $\gamma$ are all rational numbers.
The definition of the curve and $x$-monotone curve types nested in the
traits class are detailed below.

\ccInclude{CGAL/Arr_circle_segment_traits_2.h}
 
\ccIsModel
    \ccc{ArrangementTraits_2}

\subsection*{Class 
Arr\_circle\_segment\_traits\_2$<$Kernel$>$::CoordNT}
%==============================================================================

The \ccc{CoordNT} number-type nested within the traits class represents
an algebraic number of degree 2; it can be represented as
$\alpha + \beta\sqrt{\gamma}$, where $\alpha$, $\beta$ and $\gamma$ are
all rational numbers of type \ccc{Kernel::FT}.

\begin{ccClass}{Arr_circle_segment_traits_2<Kernel>::CoordNT}

\ccTypes
%-------

\ccNestedType{Rational}{the \ccc{Kernel::FT} type.}

\ccCreation
\ccCreationVariable{x}
%---------------------

\ccConstructor{CoordNT ();}
    {creates a variable whose value is $0$.}

\ccConstructor{CoordNT (const Rational& alpha);}
    {creates a variable whose value is the rational number $\alpha$.}

\ccConstructor{CoordNT (const Rational& alpha,
                        const Rational& beta,
                        const Rational& gamma);}
    {creates a variable whose value is the algebraic number 
     $\alpha + \beta\sqrt{\gamma}$.}

\ccAccessFunctions
%-----------------

\ccMethod{bool is_rational () const;}
  {determines whether \ccVar\ is rational.}

\ccMethod{Rational alpha () const;}
  {returns $\alpha$.}
\ccGlue
\ccMethod{Rational beta () const;}
  {returns $\beta$ (0 if \ccVar\ is rational).}
\ccGlue
\ccMethod{Rational gamma () const;}
  {returns $\gamma$ (0 if \ccVar\ is rational).}

\ccHeading{Operators}
%--------------------

The \ccc{CoordNT} number-type supports all arithmetic operations with
rational numbers of type \ccc{Kernel::FT}. For example, it is possible to
compute \ccc{x + q}, \ccc{q - x}, \ccc{x += q}, etc. where \ccc{q} is rational.

The global functions \ccc{CGAL::sign(x)}, \ccc{CGAL::square(x)},
\ccc{CGAL::to_double(x)} and \ccc{CGAL::compare(x,y)}, where \ccc{x} and
\ccc{y} are of type \ccc{CoordNT}, are also supported.

\end{ccClass}

\subsection*{Class 
Arr\_circle\_segment\_traits\_2$<$Kernel$>$::Point\_2}
%==============================================================================

The \ccc{Point\_2} number-type nested within the traits class represents
a Cartesian point whose coordinates are algebraic numbers of type
\ccc{CoordNT}.

\begin{ccClass}{Arr_circle_segment_traits_2<Kernel>::Point_2}

\ccTypes
%-------

\ccNestedType{Rational}{the \ccc{Kernel::FT} type.}

\ccNestedType{CoordNT}{the algebraic number-type.}

\ccCreation
\ccCreationVariable{p}
%---------------------

\ccConstructor{Point_2 ();}
    {default constructor.}

\ccConstructor{Point_2 (const Rational& x, const Rational& y);}
    {creates the point $(x,y)$.}

\ccConstructor{Point_2 (const CoordNT& x, const CoordNT& y);}
    {creates the point $(x,y)$.}

\ccAccessFunctions
%-----------------

\ccMethod{CoordNT x () const;}
  {returns the $x$-coordinate.}
\ccGlue
\ccMethod{CoordNT y () const;}
  {returns the $y$-coordinate.}

\end{ccClass}

\subsection*{Class 
        Arr\_circle\_segment\_traits\_2$<$Kernel$>$::Curve\_2}
%=========================================================================

The \ccc{Curve_2} class nested within the traits class can represent
arbitrary circular arcs, full circles and line segments and support their
construction in various ways.
The copy and default constructor as well as the assignment operator are
provided. In addition, an \ccc{operator<<} for the curves is defined for
standard output streams.

\begin{ccClass}{Arr_circle_segment_traits_2<Kernel>::Curve_2}

\ccCreation
\ccCreationVariable{cv}
%----------------------

\ccConstructor{Curve_2 (const typename Kernel::Segment_2& seg);}
    {constructs an curve corresponding to the line segment \ccc{seg}.}

\ccConstructor{Curve_2 (const typename Kernel::Point_2& source,
                        const typename Kernel::Point_2& target);}
    {constructs an curve corresponding to the line segment directed
     from \ccc{source} to \ccc{target}, both having rational coordinates.}

\ccConstructor{Curve_2 (const typename Kernel::Line_2& line,
                        const Point_2& source,
                        const Point_2& target);}
    {constructs an curve corresponding to the line segment supported by
     the given line, directed from \ccc{source} to \ccc{target}.}

\ccConstructor{Curve_2 (const typename Kernel::Circle_2& circ);}
    {constructs an curve corresponding to the given circle. \ccc{circ}
     has a center point with rational coordinates and its {\em squared}
     radius is rational.}

\ccConstructor{Curve_2 (const typename Kernel::Point_2& c,
                        const typename Kernel::FT& r,
                        Orientation orient = COUNTERCLOCKWISE);}
    {constructs an curve corresponding to a circle centered at the rational
     point \ccc{c} whose radius \ccc{r} is rational.}

\ccConstructor{Curve_2 (const typename Kernel::Circle_2& circ,
                        const Point_2& source, const Point_2& target);}
    {constructs a circular arc supported by \ccc{circ}, which has a
     center point with rational coordinates and whose {\em squared}
     radius is rational, with the given endpoints. The orientation of the
     arc is the same as the orientation of \ccc{circ}.}

\ccConstructor{Curve_2 (const typename Kernel::Point_2& c,
                        const typename Kernel::FT& r,
                        Orientation orient,
                        const Point_2& source, const Point_2& target);}
    {constructs a circular arc supported by a circle centered at the rational
     point \ccc{c} whose radius \ccc{r} is rational, directed from
     \ccc{source} to \ccc{target} with the given orientation.}

\ccConstructor{Curve_2 (const typename Kernel::Point_2& source,
                        const typename Kernel::Point_2& mid,
                        const typename Kernel::Point_2& target);}
    {constructs an circular arc whose endpoints are \ccc{source} and
     \ccc{target} that passes through \ccc{mid}. All three points have
     rational coordinates.}

\ccAccessFunctions
%-----------------

\ccMethod{bool is_full() const;}
  {indicates whether the curve represents a full circle.}

\ccMethod{const Point_2& source() const;}
  {returns the source point.
   \ccPrecond{\ccVar{} is not a full circle.}}
\ccGlue
\ccMethod{const Point_2& target() const;}
  {returns the target point.
   \ccPrecond{\ccVar{} is not a full circle.}}

\ccMethod{Orientation orientation() const;}
  {returns the orientation of the curve (\ccc{COLLINEAR} in case of
   line segments).}

\ccMethod{bool is_linear () const;}
  {determines whether \ccVar{} is a line segment.}
\ccGlue
\ccMethod{bool is_circular () const;}
  {determines whether \ccVar{} is a circular arc.}

\ccMethod{typename Kernel::Line_2 supporting_line() const;}
  {returns the supporting line of \ccVar{}.
   \ccPrecond{\ccVar{} is a line segment.}}
\ccGlue
\ccMethod{typename Kernel::Circle_2 supporting_circle() const;}
  {returns the supporting circle of \ccVar{}.
   \ccPrecond{\ccVar{} is a circular arc.}}

\end{ccClass}

\subsection*{Class 
Arr\_circle\_segment\_traits\_2$<$Kernel$>$::X\_monotone\_curve\_2}
%==============================================================================

The \ccc{X_monotone_curve_2} class nested within the traits class can
represent $x$-monotone and line segments (which are always weakly $x$-monotone).
The copy and default constructor as well as the assignment operator are
provided. In addition, an \ccc{operator<<} for the curves is defined for
standard output streams.

\begin{ccClass}{Arr_circle_segment_traits_2<Kernel>::X_monotone_curve_2}

\ccCreation
\ccCreationVariable{xcv}
%-----------------------

\ccConstructor{X_monotone_curve_2 (const typename Kernel::Point_2& source,
                        const typename Kernel::Point_2& target);}
    {constructs an curve corresponding to the line segment directed
     from \ccc{source} to \ccc{target}, both having rational coordinates.}

\ccConstructor{X_monotone_curve_2 (const typename Kernel::Line_2& line,
                        const Point_2& source,
                        const Point_2& target);}
    {constructs an curve corresponding to the line segment supported by
     the given line, directed from \ccc{source} to \ccc{target}.}

\ccConstructor{X_monotone_curve_2 (const typename Kernel::Circle_2& circ,
                        const Point_2& source, const Point_2& target);}
    {constructs a circular arc supported by \ccc{circ}, which has a
     center point with rational coordinates and whose {\em squared}
     radius is rational, with the given endpoints. The orientation of the
     arc is the same as the orientation of \ccc{circ}.
     \ccPrecond{The circular arc is $x$-monotone.}}

\ccConstructor{X_monotone_curve_2 (const typename Kernel::Point_2& c,
                        const typename Kernel::FT& r,
                        Orientation orient,
                        const Point_2& source, const Point_2& target);}
    {constructs a circular arc supported by a circle centered at the rational
     point \ccc{c} whose radius \ccc{r} is rational, directed from
     \ccc{source} to \ccc{target} with the given orientation.
     \ccPrecond{The circular arc is $x$-monotone.}}

\ccAccessFunctions
%-----------------

\ccMethod{const Point_2& source() const;}
  {returns the source point of \ccVar{}.}
\ccGlue
\ccMethod{const Point_2& target() const;}
  {returns the target point of \ccVar{}.}

\ccMethod{const Point_2& left() const;}
  {returns the left (lexicographically smaller) endpoint of \ccVar{}.}
\ccGlue
\ccMethod{const Point_2& right() const;}
  {returns the right (lexicographically larger) endpoint of \ccVar{}.}

\ccMethod{Orientation orientation() const;}
  {returns the orientation of the curve (\ccc{COLLINEAR} in case of
   line segments).}

\ccMethod{bool is_linear () const;}
  {determines whether \ccVar{} is a line segment.}
\ccGlue
\ccMethod{bool is_circular () const;}
  {determines whether \ccVar{} is a circular arc.}

\ccMethod{typename Kernel::Line_2 supporting_line() const;}
  {returns the supporting line of \ccVar{}.
   \ccPrecond{\ccVar{} is a line segment.}}
\ccGlue
\ccMethod{typename Kernel::Circle_2 supporting_circle() const;}
  {returns the supporting circle of \ccVar{}.
   \ccPrecond{\ccVar{} is a circular arc.}}

\ccMethod{Bbox_2 bbox() const;}
  {returns a bounding box of the arc \ccVar{}.}

\end{ccClass}

\end{ccRefClass}

\ccRefPageEnd
