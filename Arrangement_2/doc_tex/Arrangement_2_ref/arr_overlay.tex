\ccRefPageBegin

\begin{ccRefFunction}{overlay}

\ccDefinition

The function \ccRefName\ computes the overlay of two input arrangement
objects, and returns the overlaid arrangement. All three arrangements
are instantiated using the same geometric traits class, but may be
represented using different \dcel\ classes. A given overlay-traits object
is used to properly construct the overlaid \dcel\ that represents the
resulting arrangement.

%%%%

\ccInclude{CGAL/Arr_overlay.h}

\ccGlobalFunction{template<typename Traits, typename Dcel1, typename Dcel2,
                           typename RedDcel, typename OverlayTraits>
            void overlay (const Arrangement_2<Traits,Dcel1>& arr1,
	                  const Arrangement_2<Traits,Dcel2>& arr2,
                          Arrangement_2<Traits,ResDcel>& res,
                          OverlayTraits ovl_tr);}
Computes the overlay of two arrangements \ccc{arr1} and \ccc{arr2}, and sets
the output arrangement \ccc{res} to represent the overlaid arrangement.
\ccPrecond{\ccc{res} does not refer to either \ccc{arr1} or \ccc{arr2}
(that is, ``self overlay'' is not supported).}

%%%%

\ccThree{Orientation~~}{}{\hspace*{8.5cm}}

\ccInclude{CGAL/Arrangement_with_history_2.h}

\ccGlobalFunction{template<typename Traits, typename Dcel1, typename Dcel2,
                           typename RedDcel, typename OverlayTraits>
           void overlay (const Arrangement_with_history_2<Traits,Dcel1>& arr1,
                         const Arrangement_with_history_2<Traits,Dcel2>& arr2,
                         Arrangement_2<Traits,ResDcel>& res,
                         OverlayTraits ovl_tr);}
Computes the overlay of two arrangements with history \ccc{arr1} and
\ccc{arr2}, and sets the output arrangement with history \ccc{res} to
represent the overlaid arrangement. The function also constructs a
consolidated set of curves that induce \ccc{res}.
\ccPrecond{\ccc{res} does not refer to either \ccc{arr1} or \ccc{arr2}
(that is, ``self overlay'' is not supported).}

\ccHeading{Requirements}
The overlay-traits object \ccc{ovl_tr} must model the \ccc{OverlayTraits}
concept, which is able to construct records of the \ccc{ResDcel} class on
the basis of the \ccc{Dcel1} and \ccc{Dcel2} records that induce them.

\ccSeeAlso
  \ccc{OverlayTraits}\lcTex{(\ccRefPage{OverlayTraits})}

\end{ccRefFunction}

\ccRefPageEnd
