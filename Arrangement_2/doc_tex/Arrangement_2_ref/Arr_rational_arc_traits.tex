% +------------------------------------------------------------------------+
% | Reference manual page: Arr_rational_arc_traits.tex
% +------------------------------------------------------------------------+
% | 
% | Package: Arrangement_2
% | 
% +------------------------------------------------------------------------+

\ccRefPageBegin
\begin{ccRefClass}{Arr_rational_arc_traits_2<AlgKernel,NtTraits>}

\ccDefinition
%============

The traits class \ccRefName\ is a model of the \ccc{ArrangementTraits_2}
concept. It handles bounded segments of rational functions, referred to as
{\sl rational arcs}, and enables the construction and maintenance of
arrangements of such arcs. A rational function $y = \frac{P(x)}{Q(x)}$
is defined by two polynomials $P$ and $Q$ of arbitrary degrees. In
particular, if $Q(x) = 1$ then the function is a simple polynomial
function. A bounded rational arc is defined by the graph of a rational
function over some internal $[x_{\rm min}, x_{\rm max}]$, where $Q$
does not have any real roots in this interval (thus the arc does not
contain any poles). Rational functions, and polynomial functions in
particular, are not only interesting in their own right, they are also
very useful for approximating or interpolating more complicated curves.

In our representation, all polynomial coefficients (the coefficients of $P$
and $Q$) must be rational numbers. This guarantees that the
$x$-coordinates of all arrangement vertices (in particular, those
representing instersection points) can be represneted as roots of
polynomials with integer coefficients --- namely, algebraic numbers.
The $y$-coordinates can be obtained by simple arithmetic operations on
the $x$-coordinates, hence they are also algebraic numbers.

We therefore require separate representations of the curve coefficients and
the point coordiantes. The \ccc{NtTraits} should be instantiated with a class
that defines nested \ccc{Integer}, \ccc{Rational} and \ccc{Algebraic} number
types and supports various operations on them, yielding certified computation
results (for example, in can convert rational numbers to algebraic numbers
and can compute roots of polynomials with integer coefficients).
The \ccc{AlgKernel} template-parameter should be a geometric kernel templated
with the \ccc{NtTraits::Algebraic} number-type. It is recommended to
instantiate the \ccc{CORE_algebraic_number_traits} class as the \ccc{NtTraits}
parameter, with \ccc{Cartesian<NtTraits::Algebraic>} instantiating the kernel.
The number types in this case are provided by the {\sc core} library, with its
ability to exactly represent simple algebraic numbers.

The traits class defined its point type to be \ccc{AlgKernel::Point_2},
and defines a curve type (and an identical $x$-monotone curve type, as 
a rational arc is always $x$-monotone by definition) as detailed below.

\ccInclude{CGAL/Arr_rational_arc_traits_2.h}

\ccIsModel
  \ccc{ArrangementTraits_2}

\subsection*{Class
         Arr\_rational\_arc\_traits\_2$<$AlgKernel,NtTraits$>$::Curve\_2}
%========================================================================

The \ccc{Curve_2} class nested within the rational-arc traits is used
to represent rational arcs and support their construction from a
single polynomial and the range where the arc is defined or a pair of
polynomials and a pair of corresponding ranges. The copy and default
constructor as well as the assignment operator are provided for
polyline curves. In addition, an \ccc{operator<<} for the curves is
defined for standard output streams.

\begin{ccClass}{Arr_rational_arc_traits_2<AlgKernel,NtTraits>::Curve_2}
%======================================================================

\ccTypes
%-------

\ccNestedType{Rat_vector}{A vector of rational numbers (equivalent to
                          \ccc{std::vector<typename NtTraits::Rational}).}

\ccNestedType{Polynomial}{the \ccc{NtTraits::Polynomial} type
                          (a polynomial with integer coefficients).}

\ccCreation
\ccCreationVariable{a}
%---------------------

\ccConstructor{Curve_2 ();}
  {default constructor.}

\ccConstructor{Curve_2 (const Rat_vector& p_coeffs,
                        const typename NtTraits::Algebraic& s_x,
                        const typename NtTraits::Algebraic& t_x);}
  {constructs an arc supported by the polynomial $y = P(x)$, defined over
   the interval \ccc{[x_s, x_t]}, given by the $x$-coordinates of the arc's
   source and target. The vector \ccc{p_coeffs} specifies
   the coefficients of $P(x)$, where the polynomial degree is
   \ccc{p_coeffs.size() - 1} and \ccc{p[k]} is the coefficient of $x^k$ in $P$.
   \ccPrecond{\ccc{s_x != t_x}.}}

\ccConstructor{Curve_2 (const Rat_vector& p_coeffs,
                        const Rat_vector& q_coeffs,
                        const typename NtTraits::Algebraic& s_x,
                        const typename NtTraits::Algebraic& t_x);}
  {constructs an arc supported by the rational function
   $y = \frac{P(x)}{Q(x)}$, defined over the internal \ccc{[s_x, t_x]},
   given by the $x$-coordinates of the arc's source and target.
   The vectors \ccc{p_coeffs} and \ccc{q_coeffs} specify the coefficients
   of $P(x)$ and $Q(x)$, respectively (see above).
   \ccPrecond{\ccc{x_min < x_max}.}
   \ccPrecond{For each \ccc{x_min} $\leq$ \ccc{x} $\leq$ \ccc{x_max},
              $Q(x) \neq 0$.}}

\ccAccessFunctions
%-----------------

\ccMethod{bool is_valid() const;}
  {indicates whether \ccVar{} is a valid rational arc. As the precondition 
   $Q(x) \neq 0$ to the constructor from two polynomials is quite complicated,
   its violation does not cause the program to abort. Instead, the constructed
   arc is invalid (a defaultly constructed arc is also invalid).
   It is however recommended to check that a constructed arc is valid before
   inserting it to an arrangement, as this operation {\sl will} cause the
   program to abort.}

\ccMethod{const Polynomial& numerator () const;}
  {returns a polynomial with integer coefficients equivalent to $P(x)$.}

\ccMethod{const Polynomial& denominator () const;}
  {returns a polynomial with integer coefficients equivalent to $Q(x)$.}

\ccMethod{const Point_2& source() const;}
  {returns the source point of the arc.
   \ccPrecond{\ccVar{} is not a full conic curve.}}
\ccGlue
\ccMethod{const Point_2& target() const;}
  {returns the target point of the arc.
   \ccPrecond{\ccVar{} is not a full conic curve.}}

\ccMethod{const Point_2& left() const;}
  {returns the left (lexicographically smaller) endpoint of \ccVar{}.}
\ccGlue
\ccMethod{const Point_2& right() const;}
  {returns the right (lexicographically larger) endpoint of \ccVar{}.}

\end{ccClass}

\end{ccRefClass}
\ccRefPageEnd
