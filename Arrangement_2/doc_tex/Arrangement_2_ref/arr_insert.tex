\ccRefPageBegin

\begin{ccRefFunction}{insert_curve}

\ccDefinition

The function \ccRefName{} inserts a given curve into a given arrangement,
where no restrictions are made on the nature of the inserted curve.
It subdivides the given curve into $x$-monotone subcurves. Each subcurve
is in turn inserted into the arrangement by locating its left endpoint
and computing its zone until reaching the right endpoint.

A given point-location object is used for answering point-location queries
during the insertion process. By default, the function uses the
``walk along line'' point-location strategy --- namely an instance of the
class \ccc{Arr_walk_along_line_point_location<Arrangement_2<Traits,Dcel> >}.

%%%%

\ccInclude{CGAL/Arrangement_2.h}

\ccGlobalFunction{template<typename Traits, typename Dcel,
                  typename PointLocation>
            void insert_curve (Arrangement_2<Traits,Dcel>& arr,
                               const typename Traits::Curve_2& c,
                               const PointLocation& pl = walk_pl);}
Inserts the given curve \ccc{c} into the arrangement \ccc{arr}.
\ccPrecond{If provided, \ccc{pl} must be attached to the given arrangement
\ccc{arr}.}

%%%%

\ccInclude{CGAL/Arrangement_with_history_2.h}

\ccThree{Orientation~~}{}{\hspace*{8.5cm}}

\ccGlobalFunction{template<typename Traits, typename Dcel,
                           typename PointLocation>
            typename Arrangement_with_history_2<Traits,Dcel>::Curve_handle
            insert_curve (Arrangement_with_history_2<Traits,Dcel>& arr,
                    const typename Traits::Curve_2& c,
                    const PointLocation& pl = walk_pl);}
Inserts the given curve \ccc{c} into the arrangement with history
\ccc{arr}, and returns a handle to the inserted curve.
\ccPrecond{If provided, \ccc{pl} is attached to the given arrangement
\ccc{arr}.}

\ccRequirements
\begin{itemize}
\item The instantiated \ccc{Traits} class must model the
  \ccc{ArrangementTraits_2} concept. That is, it should define the
  \ccc{Curve_2} type, and support its subdivision into $x$-monotone
  subcurves (and perhaps isolated points).
\item The point-location object \ccc{pl}, must model the
  \ccc{ArrangementPointLocation_2} concept.
\end{itemize}

\end{ccRefFunction}

\ccRefPageEnd
