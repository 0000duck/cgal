% +------------------------------------------------------------------------+
% | Reference manual page: ArrangementBasicTraits.tex
% +------------------------------------------------------------------------+
% | 
% | Package: Arrangement_2
% | 
% +------------------------------------------------------------------------+

\ccRefPageBegin

\begin{ccRefConcept}{ArrangementBasicTraits_2}

\ccDefinition

The basic arrangement-traits concept defines the minimal set of geometric
predicates needed for the constrcution and maintenance of instances of the
\ccc{Arrangement_2} class, as well as performing simple queries (such as
point-location queries) on such arrangements.

A model of this concept must define nested \ccc{Point_2} and
\ccc{X_monotone_curve_2} types, which represent planar points and
continuous $x$-monotone curves (a vertical segment is also considered to be
{\sl weakly} $x$-monotone), respectively. The $x$-monotone curves are assumed
to be pairwise disjoint in their interiors, so they do not intersect
except at their endpoints.

The $x$-monotone curves may be {\em unbounded}, namely they may have unbounded
ends that lie at infinity, or {\em bounded}, in which case their have finite
endpoints are representable as \ccc{Point_2} objects. An $x$-monotone curve
may also have one unbounded end and one finite endpoint (e.g. a ray).
If unbounded curves are supported, the requirements from the traits class
are extended a bit, as described below. In particular, the traits class needs
to support comparisons at infinity, where we interpret comparisons at infinity
as follows: Suppose we wish to compare the $y$-position of the hyperbolas
$y_1 = \frac{1}{x}$ and $y_2 = \frac{2}{x}$ at $x = \infty$; than since there
exists $x_0$ such that for each finite $x > x_0$ we have $y_2(x) > y_1(x)$
(in our case we can take $x_0 = 0$), we say that $y_2$ is {\em above} $y_1$ at
infinity. Similarly, when comparing the $x$-positions of the vertical line
$x = 0$ and the $xy$-minimal end of parabola $y = \frac{1}{x}$ for $x > 0$
(which has a vertical asymptote at $x = 0$), we define that the line lies to
the {\em left} of the hyperbola. Namely, comparing curves at their unbounded
ends should return the comparison result \ccc{EQUAL} only if the curves
overlap.

\ccTypes
%=======

\ccNestedType{Point_2}{represents a point on the plane.}

\ccNestedType{X_monotone_curve_2}{represents a planar (weakly) $x$-monotone
                                  curve.}

\ccHeading{Tags}
%===============

\ccNestedType{Has_left_category}{indicates whether the nested functor
                                 \ccc{Compare_at_x_left_2} is provided.}

\ccNestedType{Has_infinite_category}{indicates whether the traits class
                                     supports unbounded curves.}

\ccHeading{Functor Types}
%========================

\ccThree{Compare_y_at_x_2}{}{\hspace*{14cm}}
\ccThreeToTwo

\ccNestedType{Compare_x_2}
{provides the operator~:
 \begin{itemize}
 \item \ccc{Comparison_result operator() (Point_2 p1, Point_2 p2)} \\
 which returns \ccc{SMALLER, EQUAL} or \ccc{LARGER} according to the
 $x$-ordering of points \ccc{p1} and \ccc{p2}.
 \end{itemize}
 In case the traits class supports unbounded curves (the
 \ccc{Has_infinite_category} is defined as \ccc{Tag_true}), the functor
 should also provide the following operators~:
 \begin{itemize}
 \item \ccc{Comparison_result operator() (Point_2 p, X_monotone_curve_2 c, 
                                    Curve_end ind)} \\
 which returns \ccc{SMALLER, EQUAL} or \ccc{LARGER} according to the
 $x$-ordering of a vertical line passing through the point \ccc{p} and an
 unbounded end of the curve \ccc{c}, with the precondition that the curve end
 has a bounded $x$-coordinate and an unbounded $y$-coordinate (namely \ccc{c}
 is vertical or have a vertical asymptote). The relevant end is the minimal
 end if \ccc{ind} is \ccc{MIN_END} and the maximal end if \ccc{ind} is
 \ccc{MAX_END}.
 \item \ccc{Comparison_result operator() (X_monotone_curve_2 c1, Curve_end ind1, 
                                    X_monotone_curve_2 c2, Curve_end ind2)} \\
 which returns \ccc{SMALLER, EQUAL} or \ccc{LARGER} according to the
 $x$-ordering of the unbounded curve ends of \ccc{c1} and \ccc{c2}.
 \end{itemize}}

\ccNestedType{Compare_xy_2}
{provides the operator~:
 \begin{itemize}
 \item \ccc{Comparison_result operator() (Point_2 p1, Point_2 p2)} \\
 which returns \ccc{SMALLER}, \ccc{EQUAL}, or \ccc{LARGER} according to the
 lexicographic $xy$-order of the points \ccc{p1} and \ccc{p2}.
 \end{itemize}}

\ccNestedType{Infinite_in_x_2}
{provides the operator~:
 \begin{itemize}
 \item \ccc{Infinity_type operator() (X_monotone_curve_2 c, Curve_end ind)} \\
 which determines if the $x$-coordinate of the minimal end (or maximal end)
 of \ccc{c} is infinite. It returns \ccc{MINUS_INFINITY} or \ccc{PLUS_INFINITY}
 if the curve end is infinite in $x$, and \ccc{FINITE} if it has a finite
 $x$-coordinate.
 \end{itemize}
 This functor is required only if the traits class supports unbounded curves
 (the \ccc{Has_infinite_category} is defined as \ccc{Tag_true}).}

\ccNestedType{Infinite_in_y_2}
{provides the operator~:
 \begin{itemize}
 \item \ccc{Infinity_type operator() (X_monotone_curve_2 c, Curve_end ind)} \\
 which determines if the $y$-coordinate of the minimal end (or maximal end)
 of \ccc{c} is infinite. It returns \ccc{MINUS_INFINITY} or \ccc{PLUS_INFINITY}
 if the curve end is infinite in $y$, and \ccc{FINITE} if it has a finite
 $y$-coordinate.
 \end{itemize}
 This functor is required only if the traits class supports unbounded curves
 (the \ccc{Has_infinite_category} is defined as \ccc{Tag_true}).}

\ccNestedType{Construct_min_vertex_2}
{provides the operator~:
 \begin{itemize}
 \item \ccc{Point_2 operator() (X_monotone_curve_2 c)} \\
 which returns the lexicographically smaller (left) endpoint of \ccc{c},
 with the precondition that the left end of \ccc{c} is not infinite.
 \end{itemize}}

\ccNestedType{Construct_max_vertex_2}
{provides the operator~:
 \begin{itemize}
 \item \ccc{Point_2 operator() (X_monotone_curve_2 c)} \\
 which returns the lexicographically larger (right) endpoint of \ccc{c},
 with the precondition that the right end of \ccc{c} is not infinite.
 \end{itemize}}

\ccNestedType{Is_vertical_2}
{provides the operator~:
 \begin{itemize}
 \item \ccc{bool operator() (X_monotone_curve_2 c)} \\
 which determines whether \ccc{c} is a vertical segment.
 \end{itemize}}

\ccNestedType{Compare_y_at_x_2}
{provides the operator~:
 \begin{itemize}
 \item \ccc{Comparison_result operator() (Point_2 p, X_monotone_curve_2 c)} \\
 which compares the $y$-coordinates of \ccc{p} and the vertical projection
 of \ccc{p} on \ccc{c}, and returns \ccc{SMALLER, EQUAL} or \ccc{LARGER}
 according to the result.
 \end{itemize}
 In case the traits class supports unbounded curves (the
 \ccc{Has_infinite_category} is defined as \ccc{Tag_true}), the functor
 should also provide the following operator~:
 \begin{itemize}
 \item \ccc{Comparison_result operator() (X_monotone_curve_2 c1,
                                          X_monotone_curve_2 c2, 
                                          Curve_end ind)} \\
 which returns \ccc{SMALLER, EQUAL} or \ccc{LARGER} according to the
 $y$-ordering of the two curves \ccc{c1} and \ccc{c2} at $x = -\infty$
 (if \ccc{ind} is \ccc{MIN_END}) or at $x = \infty$ (if \ccc{ind} is 
 \ccc{MAX_END}), with the precondition that both curves have unbounded
 minimal (or maximal) ends that lie at $x = \pm\infty$.
 \end{itemize}}

\ccNestedType{Compare_y_at_x_left_2}
{provides the operator~:
 \begin{itemize}
 \item \ccc{Comparison_result operator() (X_monotone_curve_2 c1,
                                    X_monotone_curve_2 c2, Point_2 p)} \\
 which accepts two $x$-monotone curves \ccc{c1} and \ccc{c2} that have
 a common right endpoint \ccc{p}, and returns \ccc{SMALLER, EQUAL} or
 \ccc{LARGER} according to the relative position of the two curves
 immediately to the left of $p$. Note that in case one of the $x$-monotone
 curves is a vertical segment (emanating downward from \ccc{p}), it is always
 considered to be {\sl below} the other curve.
 \end{itemize}
 This functor is required only if the traits class defines the
 \ccc{Has_left_category} tag as \ccc{Tag_true}).}

\ccNestedType{Compare_y_at_x_right_2}
{provides the operator~:
 \begin{itemize}
 \item \ccc{Comparison_result operator() (X_monotone_curve_2 c1,
                                    X_monotone_curve_2 c2, Point_2 p)} \\
 which accepts two $x$-monotone curves \ccc{c1} and \ccc{c2} that have
 a common left endpoint \ccc{p}, and returns \ccc{SMALLER, EQUAL} or
 \ccc{LARGER} according to the relative position of the two curves
 immediately to the right of $p$. Note that in case one of the $x$-monotone
 curves is a vertical segment emanating upward from \ccc{p}, it is always
 considered to be {\sl above} the other curve.
 \end{itemize}}

\ccNestedType{Equal_2}
{provides the operators~:
 \begin{itemize}
 \item \ccc{bool operator() (Point_2 p1, Point_2 p2)} \\
 which determines whether \ccc{p1} and \ccc{p2} are geometrically
 equivalent.
 \item \ccc{bool operator() (X_monotone_curve_2 c1,
                             X_monotone_curve_2 c2)} \\
 which determines whether \ccc{c1} and \ccc{c2} are geometrically equivalent
 (have the same graph).
 \end{itemize}}

\ccCreation
\ccCreationVariable{traits}
%==========================

\ccThree{Construct_x_monotone_curve_2~~~}{}{\hspace*{7cm}}
\ccThreeToTwo

\ccConstructor{ArrangementBasicTraits_2();}{default constructor.}
\ccGlue
\ccConstructor{ArrangementBasicTraits_2(ArrangementBasicTraits_2 other);}
{copy constructor}
\ccGlue
\ccMethod{ArrangementBasicTraits_2  operator=(other);}{assignment operator.}

\ccHeading{Accessing Functor Objects}
%====================================

\ccMethod{Compare_x_2 compare_x_2_object() const;} {}
\ccGlue
\ccMethod{Compare_xy_2 compare_xy_2_object() const;}{}
\ccGlue
\ccMethod{Infinite_in_x_2 infinite_in_x_2_object() const;}{}
\ccGlue
\ccMethod{Infinite_in_y_2 infinite_in_y_2_object() const;}{}
\ccGlue
\ccMethod{Construct_min_vertex_2 construct_min_vertex_2_object() const;}{}
\ccGlue
\ccMethod{Construct_max_vertex_2 construct_max_vertex_2_object() const;}{}
\ccGlue
\ccMethod{Is_vertical_2 is_vertical_2_object() const;} {}
\ccGlue
\ccMethod{Compare_y_at_x_2 compare_y_at_x_2_object() const;} {}
\ccGlue
\ccMethod{Compare_y_at_x_left_2 compare_y_at_x_left_2_object() const;} {}
\ccGlue
\ccMethod{Compare_y_at_x_right_2 compare_y_at_x_right_2_object() const;} {}
\ccGlue
\ccMethod{Equal_2 equal_2_object() const;} {}

\ccHasModels
%===========

\ccc{CGAL::Arr_segment_traits_2<Kernel>}\\
\ccc{CGAL::Arr_non_caching_segment_basic_traits_2<Kernel>}\\
\ccc{CGAL::Arr_non_caching_segment_traits_2<Kernel>}\\
\ccc{CGAL::Arr_polyline_traits_2<SegmentTraits>}\\
\ccc{CGAL::Arr_circle_segment_traits_2<Kernel>} \\
\ccc{CGAL::Arr_conic_traits_2<RatKernel,AlgKernel,NtTraits>}\\
\ccc{CGAL::Arr_rational_arc_traits_2<AlgKernel,NtTraits>}\\
\ccc{CGAL::Arr_curve_data_traits_2<Tr,XData,Mrg,CData,Cnv>}\\
\ccc{CGAL::Arr_consolidated_curve_data_traits_2<Traits,Data>}

\end{ccRefConcept}

\ccRefPageEnd
