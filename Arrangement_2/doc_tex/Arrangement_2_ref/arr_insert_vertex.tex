\ccRefPageBegin

\begin{ccRefFunction}{insert_point}

\ccDefinition

Inserts the point \ccc{p} as an arrangement vertex into \ccc{arr}.
The function uses the point-location object \ccc{pl} (a model of the
\ccc{ArrangementPointLocation_2} concept) to locate \ccc{p} in \ccc{arr}.
If it conincides with an existing vertex, there is nothing left to
do; if it lies on an edge, the edge is split at \ccc{p};
otherwise, \ccc{p} is contained inside a face, and is inserted as
an isolated vertex inside this face.
At any case, the function returns a handle for the vertex associated
with \ccc{p}. The \ccc{Triats} parameter must model the refined concept
\ccc{ArrangementXMonotoneTraits_2}, which requires split operations.
By default, the function uses the ``walk along line'' point-location
strategy --- namely, an instance of the class
\ccc{Arr_walk_along_line_point_location<Arrangement_2<Traits,Dcel> >}.

\ccInclude{CGAL/Arrangement_2.h}

\ccGlobalFunction{template<typename Traits, typename Dcel,
                           typename PointLocation>
                       typename Arrangement_2<Traits,Dcel>::Vertex_handle
                       insert_point (Arrangement_2<Traits,Dcel>& arr,
                                         const typename Traits::Point_2& p,
                                         const PointLocation& pl = walk_pl);}
\ccPrecond{If provided, \ccc{pl} must be attached to the given arrangement
\ccc{arr}.}

% Inserts the point \ccc{p} as an arrangement vertex into \ccc{arr}.
% This function operated just like the function above, but it uses the
% default ``walk along line'' point-location strategy, so users need not
% provide a point-location object.
%
%\ccGlobalFunction{template<typename Traits, typename Dcel>
%            typename Arrangement_2<Traits,Dcel>::Vertex_handle
%            insert_point (Arrangement_2<Traits,Dcel>& arr,
%                           const typename Traits::Point_2& p);}

\end{ccRefFunction}

\ccRefPageEnd
