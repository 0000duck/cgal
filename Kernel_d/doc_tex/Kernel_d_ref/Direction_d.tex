\providecommand{\ccRequire}{\ccCommentHeading{Requirement}}

\begin{ccRefClass}{Direction_d<Kernel>}\ccCreationVariable{dir}

\ccDefinition


A \ccc{Direction_d} is a vector in the $d$-dimensional vector space
where we forget about its length. We represent directions in
$d$-dimensional space as a tuple $(h_0,\ldots,h_d)$ of variables of
type \ccc{RT} which we call the homogeneous coordinates of the
direction. The coordinate $h_d$ must be positive.  The Cartesian
coordinates of a direction are $c_i = h_i/h_d$ for $0 \le i < d$,
which are of type \ccc{FT}.  Two directions are equal if their
Cartesian coordinates are positive multiples of each other. Directions
are in one-to-one correspondence to points on the unit sphere.



\ccSetOneOfTwoColumns{4cm}

\ccTypes

% \ccNestedType{RT}{the ring type.}

% \ccNestedType{FT}{the field type.}

\ccNestedType{LA}{the linear algebra layer.}

\ccNestedType{Delta_const_iterator}{
  a read-only iterator for the deltas of \ccc{dir}.}

\ccNestedType{Base_direction}{construction tag.}

\ccSetOneOfTwoColumns{4cm}

\ccCreation

\ccConstructor{Direction_d<Kernel>()}{ introduces a variable \ccc{dir} of
  type \ccc{Direction_d<Kernel>}. }

\ccConstructor{Direction_d<Kernel>(Vector_d<Kernel> v)}{
  introduces a variable \ccc{dir} of type \ccc{Direction_d<Kernel>}
  initialized to the direction of \ccc{v}. }

\ccConstructor{template <class InputIterator>
Direction_d<Kernel>(int d, InputIterator first, InputIterator last)}{
  introduces a variable \ccc{dir} of type \ccc{Direction_d<Kernel>} in
  dimension \ccc{d} with representation tuple \ccc{set [first,last)}.
  \ccPrecond \ccc{d} is nonnegative, \ccc{[first,last)} has \ccc{d}
  elements. \ccRequire The value type of \ccc{InputIterator} is \ccc{RT}.}

\ccConstructor{Direction_d<Kernel>(int d, Base_direction, int i)}{ returns
  a variable \ccc{dir} of type \ccc{Direction_d<Kernel>} initialized to the
  direction of the $i$-th base vector of dimension $d$.  \ccPrecond $0
  \leq i < d$.}

\ccConstructor{Direction_d<Kernel>(RT x, RT y)}{
  introduces a variable \ccc{dir} of type \ccc{Direction_d<Kernel>} in 
  $2$-dimensional space.}

\ccConstructor{Direction_d<Kernel>(RT x, RT y, RT z)}{
  introduces a variable \ccc{dir} of type \ccc{Direction_d<Kernel>} in 
  $3$-dimensional space.}

\ccSetTwoOfThreeColumns{5cm}{3cm}

\ccOperations

\ccMethod{int dimension() ;}{
  returns the dimension of \ccc{dir}.}

\ccMethod{RT delta(int i) ;}{
  returns the $i$-th component of \ccc{dir}. 
   \ccPrecond $0 \leq i < d$.}

\ccMethod{RT operator[](int i) ;}{
  returns the $i$-th delta of \ccc{dir}. 
  \ccPrecond $0 \leq i < d$.}

\ccMethod{Delta_const_iterator deltas_begin() ;}{
  returns an iterator pointing to the first delta of \ccc{dir}.}

\ccMethod{Delta_const_iterator deltas_end() ;}{
  returns an iterator pointing beyond the last delta of \ccc{dir}.}

\ccMethod{Vector_d<Kernel> vector() ;}{
  returns a vector pointing in direction \ccc{dir}.}

\ccMethod{bool is_degenerate() ;}{
  returns true iff \ccc{dir.delta(i)==0} for all $0\leq i < d$.}

\ccMethod{Direction_d<Kernel> transform(const Aff_transformation_d<Kernel>& t) ;}{
  returns $t(p)$.}

\ccMethod{Direction_d<Kernel>  opposite() ;}{
  returns the direction opposite to \ccc{dir}.}

\ccMethod{Direction_d<Kernel> operator- () ;}{
  returns the direction opposite to \ccc{dir}.}

\ccHeading{Downward compatibility}
We provide the operations of the lower dimensional interface \ccc{dx()},
\ccc{dy()}, \ccc{dz()}. 

\ccImplementation

Directions are implemented by arrays of integers as an item type.  All
operations like creation, initialization, tests, inversion, input and
output on a direction $d$ take time $O(d.\mathit{dimension}())$.
\ccc{dimension()}, coordinate access and conversion take constant
time.  The space requirement is $O(d.\mathit{dimension}())$.

\end{ccRefClass}


